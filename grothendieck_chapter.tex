\newcommand{\Cat}{{\sf Cat}}
\newcommand{\D}{\mathcal{D}}
%
%
%The Grothendieck construction establishes an equivalence of categories between pseudofunctors from categories $\X$ into $\Cat$ and fibrations over $\X$.  There is a two-sided variation of this construction:
%
%
%\begin{theorem}{Benabou-Grothendieck}
%Take a lax normal functor $F:\mathcal{I}\to\Prof$ from a 1-category $\mathcal{I}$.
%
%Then the following pullback exists:
%$$
%\xymatrix{
%\int F \ar[d]_{} \ar[r]^{} & \Prof^*  \ar[d]\\
%\mathcal{I}^{\op} \ar[r] ^F          & \Prof
%}
%$$
%
%Where $\int F$ is a 1-category and $\int F \to \X$ is a functor.
%This extends to an isomorphism of categories:
%$$\int:\Cat_{l,n}(\X, \Prof)\cong \Cat/\X:\delta$$
%\end{theorem}
%
%$\Cat_{l,n}(\X, \Prof)$ is the category of lax normal functors from $\X$ to $\Prof$ and lax natural transformations. $\Cat/\X$ is the slice category over $\X$.
%
%
%Explicitly:
%\begin{lemma}
%$\int F$ has:
%\begin{description}
%\item[Objects:] Pairs $(Y \in \X_0, Y^\sharp \in F(Y))$
%\item[Morphisms:] 
%
%
%
%\item[Composition]
%\end{description}
%The functor $\int F\to \mathcal{I}^\op$ is the first projection from the pullback.
%
%\end{lemma}
%and conversely
%\begin{lemma}
%Given a functor $\delta: \mathcal{J}\to\mathcal{I}^\op$ $\delta \pi$ is the lax functor such that
%\begin{description}
%\item
%\item
%\itemend{description}
%\end{lemma}
%
%
%
%The Grothendieck construction has been extended to the monoidal case: establishing an equivalence between monoidal fibrations and monoidal pseudofunctors \cite[].  Similarly \cite[] prove the polycategorical Benabou-Grothendieck equivalence.  We adapt the work of both authors, by establishing a monoidal Grothendieck-Benabou equivalence:
%%
%%
%%
%%\begin{theorem}[Monoidal Grothendieck-Benabou construction.]
%%There is an equivalence of categories between the lax monoidal, lax normal functor category:
%%
%%\begin{description}
%%\item[Objects:]
%%\item[Maps:]
%%\item[Identity:]
%%\item[Composition:]
%%\end{description}
%%
%%and  the monoidal slice cateogry:
%%
%%\begin{description}
%%\item[Objects:]
%%\item[Maps:]
%%\item[Identity:]
%%\item[Composition:]
%%\end{description}
%%
%%\end{theorem}
%%\begin{proof}
%%Proof strategy: Start with a monoidal functor $\Y\to \X$ between strict monoidal categories.  Build a lax normal lax monoidal functor $\X\to \Prof$
%%
%%
%%Let $\X$ be a monoidal category, regarded as a monoidal bicategory with one object, and take a lax monoidal lax functor $F:\X\to\Prof$.  Then the following pullback exists in the category of monoidal bicategories, lax monoidal lax functors and lax natural transformations:
%%$$
%%\xymatrix{
%%\int F \ar[d]_{} \ar[r]^{} & \Prof^*  \ar[d]\\
%%\X \ar[r] ^F          & \Prof
%%}
%%$$
%%
%%$\int F$ has the same categorical structure as in the non-monoidal 
%%
%%
%%\end{proof}
%
%
%
%If $F$ is a normal frobenius monoidal lax normal functor
%with laxator $\ell$ monoidal laxator $\mu$ oplaxator $\nu$
%, then $\int F$ has an induced monoidal structure with:
%
% $F(X\otimes Y)\to F(X)\otimes F(Y) \to F(X')\otimes F(Y') \to F(X'\otimes Y')$
%
%\begin{description}
%\item[Tensor product:] On Objects:
%$$
%(Y,Y^\sharp) \otimes (Z,Z^\sharp)
%:=
%(Y\otimes Z, \nu(\mu(Y^\sharp, Z^\sharp)))
%$$
%
%On morphisms:
%$$
%(f,f^\sharp) \otimes (g,g^\sharp)
%:=
%(f\otimes g, \nu(\mu(f^\sharp, g^\sharp)) )
%$$
%
%\item[Tensor unit:]
%$$
%(I,* \in F(I))=\mathbb{1})
%$$
%\item[Unitors:]
%$$
%u_{(X,X^\sharp)}^R: (X,X^\sharp)\otimes (I,*)=(X\otimes I, \mu^{\otimes}(X^\sharp, *))  \to (X,X^\sharp)
%$$
%is given by
%$$
%(u_X^{R}:X\otimes I\to X,  f\in F(u_X^{R})(X^\sharp, \mu^{\otimes}(X^\sharp, *) )   )
%$$
%
%Where $f$ is the identity on $X$ along the isomorphism $ \mu^{\otimes}(X^\sharp, *) = $
%
%$
%(u_X^{R})(X^\sharp, X^\sharp)
%\cong
%(u_X^{R})(X^\sharp, \mu^{\otimes}(X^\sharp, *))
%$
%
%
%\item[Associator:]
%
%$$
%(\alpha_{X,Y,Z},  g):((X\otimes Y)\otimes Z, \mu^\otimes(\mu^{\otimes}(X^\sharp, Y^\sharp),Z^\sharp) \to 
%(X\otimes (Y\otimes Z),\mu^\otimes(X^\sharp, \mu^\otimes(Y^\sharp,Z^\sharp)))
%$$
%
%\end{description}
%
%
%Where the projection map $p:\int F\to\mathcal I$ moreover is strong monoidal:
%
%$$
%p ((X,X^\sharp )\times (Y,Y^\sharp )) = 
%$$
%
%
% Since we are regarding $\Prof$ as a quasistrict monoidal bicategory, if $\mathcal  I$ is strict monoidal, then so is $\int F$ so that the projection $\int F\to \mathcal I $ is strict monoidal.
%
%
%Conversely, suppose there is a strong monoidal functor $\pi:\mathcal{J}\to\mathcal{I}^\op$ between monoidal categories.
%Then $\delta \pi:\mathcal{I}^\op \to \Prof$ is a Frobenius monoidal, lax normal functor.  The monoidal laxators are given by:
%
%
%
%
%
%
%Given a monoidal category $p:\X\to \mathbb{1}$, the Frobenius monoidal lax structure of $\delta p : \mathbb{1}\to \Prof$ regarded as a lax monoidal functor is precisely the data of a representable lax special \dag-Frobenius algebra in $\Prof$:
%
%GIVE AXIOMS
%
%
%This is essentially the two-sided version of the coherence data of a monoidal category. 
%
%2-categorical spider theorem
%
%Since $\mathbb{1}$ is strict monoidal, $\int \delta p$ is as well.
%Therefore we can deduce that this is the strictification of $\X$.
%
%Slice category definition of grothendieck construction
%
%Recalling the string diagrams for pointed profunctors, we have that the strictification of $\X$ is generated by.
%
%
%The monoidal Benabou-Grothendieck construction is a very general construction for creating string diagrams for the strictification of monoidal functors.  Given some algebraic structure $F$ in  $\Prof$, the lax normal, lax monoidal structure can be regarded as the data for a normal form.  Then each object in $\int F$ contains the information need
%
%
%
%
%
%
%
%
%
%
%
%
%
%
%
%
%
%
%
%
%
%
%
%
%
%
%
%
%Monoidal categories $\X$ are in bijection with pseudomonoids in Cat.
%These are in bijection with extraspecial representable dagger frobenius algebras in Prof
%which are in bijection with lax seperable normal dagger frobenius monoidal lax functors $F_\X:\mathbb{1}\to\Prof$.
%
%Since $\mathbb 1$ is strict monoidal so is $\int F_\X$.
%Moreover, there is a $\dag$-Frobenius monoidal pseudo functor $\iota:\X\to \Prof^*$ making the diagram commute:
%
%$$
%\xymatrix{
%\X  \ar[drr]^\iota \ar[ddr]  & &\\
%       &  \int F_\X \ar[d]_{} \ar[r]^{} & \Prof^*  \ar[d]\\
%       &  \mathbb{1} \ar[r] ^F          & \Prof
%}
%$$
%
%Therefore, the universal map $G:\X\to F_\X$ is a Frobenius monoidal pseudofunctor.  It can also be shown to be strong monoidal, and moreover an equivalence of categories.  Therefore $\int F_\X$ is the monoidal strictification of $\X$.
%
%
%This extends to an equivalence of categories:
%
%Monoidal functors $\X\to \Y$ are in bijection with pseudomonoid homomorphisms in Cat.  These are in bijection with monoidal natural transformations 
%$F_\X\Rightarrow F_\Y$.
%These are in bijection with strict monoidal functors $\int F_\X\to \int F_\Y$. These are in 
%
%Surely intertwiners between pseudomonoid homorphisms correspond to strict monoidal natural transformations.
%
%
%
%
%
%
%
%
%
%
%
%First show
%
%\begin{description}
%\item[0-cells:] Frobenius monoidal functors \mathcal{I}\to\Prof$
%\item[1-cells:] Frobenius monoidal lax natural transformations.
%\item[2-cells:] Intertwiners
%\end{description}
%
%is 2-equivalent
%
%\begin{description}
%\item[0-cells:] \int F
%\item[1-cells:] strong monoidal functors $\int F \to \int G$ making the triangle commute.
%\item[2-cells:] monoidal natural transformations
%\end{description}
%
%
%
%
%
%
%
%
%
%
%
%Consider the 2-category of:
%
%\begin{description}
%\item[0-cells:] Monoidal categories
%\item[1-cells:] Monoidal functors.
%\item[2-cells:] Monoidal natural transformations
%\end{description}
%
%is 2-isomorphic 
%
%\begin{description}
%\item[0-cells:] Pseudomonoids in Cat
%\item[1-cells:] Pseudomonoid homomorphisms.
%\item[2-cells:] Intertwiners
%\end{description}
%
%is 2-isomorphic 
%
%\begin{description}
%\item[0-cells:] XXX Frobenius pseudomonoid 
%\item[1-cells:] Pseudomonoid homomorphisms.
%Composition by conjugation
%\item[2-cells:] Intertwiners
%\end{description}
%
%is 2-isomorphic 
%
%\begin{description}
%\item[0-cells:] Frobenius monoidal functors $\mathbb{1}\to\Prof$
%\item[1-cells:] Frobenius monoidal lax natural transformations.
%\item[2-cells:] Intertwiners
%\end{description}
%
%is 2-equivalent
%
%\begin{description}
%\item[0-cells:] \int F
%\item[1-cells:] \alpha:\int F\to \int G
%\item[2-cells:] Intertwiners
%\end{description}
%
%is 2-isomorphic 
%
%\begin{description}
%\item[0-cells:] Strict monoidal categories
%\item[1-cells:] monoidal 
%\item[2-cells:] Intertwiners
%\end{description}
%
%
%
%
%
%
%
%
%
%Scalable ZX-calculus.
%
%hierarchical string diagrams
%
%
%
%
%
%Take the lax Frobenius monoidal $F:\N \to\Prof$
%sending $n=\prod_i p_i^{a_i }\mapsto \prod \Span(\Mat_{\F_{p_i}})$  for where the tensor in $\N$ is multiplication.
%
%
%There is a faithful functor $\int F\to\FHilb$ picking out phase free ZX diagrams with arbitrary finite dimension. This is because for the full subcategory off prime prower dimension $\int F |_p \hookrightarrow \int F$, $\int F |_p\cong \Span(\Mat_{\F_{p_i}})$. Moreover, the maps given by the laxators are change of basis vectors.
%
%
%
%If instead we do the same trick but sending $p$ to odd prime dimensional qudit complete-ZX diagrams, then we regain the qufinite presentation of the ZX-calculus of \cite{wang???}.
%
%
%
%For stabilizers, we can do the same modulo scalars, but with $F:\N/\{2\} \to\Prof$ picking out odd prime dimensional stabilizer diagrams.
%
%











\begin{definition}
Given bicategories $\X$ and $\Y$, a lax normal functor $\X\to\Y$ is:

TODO
\end{definition}


\begin{definition}
Given monoidal bicategories $\X$ and $\Y$, a  Frobenius monoidal lax normal functor is a lax normal functor $\X\to\Y$, equipped with 2-cells $\mu$ and $\nu$ called the monoidal laxator and oplaxators, and coherences called the left and right frobeniusators interacting with the compositors todo 

TODO
\end{definition}



\begin{definition}
A morphism between Frobenius monoidal lax normal functors $F,G\X\to\Y$ is.  Given two monoidal bicategories, this notion induces the Frobenius monoidal lax normal functor category, denoted $[\X,\Y]_{fln}$
\end{definition}

\begin{lemma}
$[\X,\Y]_{fln}$ is a monoidal category with
\end{lemma}



\begin{definition}
A monoidal displayed category is a monoidal category $\D$ equipped with a  Frobenius monoidal lax functor $\mathcal{D}\to\Prof$.
\end{definition}

\begin{theorem}{Monoidal Grothendieck-Benabou construction}
Given a monoidal category $\X$, there is a monoidal equivalence between the Frobenius monoidal lax normal functor category $[\X,\Prof]_{fln}$ and the strict monoidal coslice category over $\X$.
\end{theorem}







\begin{proof}
Fix a monoidal category $\X$.

Take a strict monoidal functor $p:\Y\to\X$.


Given an object $X$ of $\X$, the indexed category of $p$ over $X$, $p^{-1}(X)$ has:

Objects in $Y \in \Y$ such that $p^{-1}(Y)=X$.

Morphisms $f:Y\to Y'$ such that $p^{-1}(f)=1_X$.

Composition and identities in $\Y$.


Given a morphism $f:X\to X'$ to in $\X$, the reindexing profunctor $p^{-1}(f):p^{-1}(X)^\op  \times p^{-1}(X')\to \Set$ sends:

objects: $p^{-1}(f)(Y,Y')= \{ g:Y\to Y' | p(g)=f \}$
maps: $p^{-1}(f)(h:Y\to Y', k:Z \to Z')=\lambda x \in p^{-1}(f)(Y,Y'). h;x;k$.


This has the structure of a lax normal functor $P^{-1}:\X\to\Prof$.


For objects $X,X',X'' \in \X$ the compositors at $X,X',X''$ at components $f:X\to X'$ and $g:X'\to X''$ are functions $\int^Z p^{-1}(f)(X,Z) \times p^{-1}(g)(Z,X'') \to  p^{-1}(f;g)(X,X'')$,
sending elements $(h,k)$ of the equivalence class to their composite $h;k$.  This is a function because $p(h;k) = p(h);p(k)=f;g$.


For each $X \in \X$, $p^{-1}(f)=1_X$ is the identity profunctor on $p^{-1}(X)$, so this functor is normal.
The desired commutative diagrams hold making this into a lax normal functor.

Furthermore, we now show that this lax normal functor preserves the monoidal structure lax-Frobeniusly.

 

%1_{p^{-1}(X)} \to 

%
%The components of the monoidal laxtator at $(f,g)$
%
%$$
%p^{-1}(f) \times p^{-1}(g) \Rightarrow p^{-1}(f \otimes g)
%$$
%
%are given by functions
%$$
%p^{-1}(f)\times p^{-1}(g) = \int^{Y} p^{-1}(f) (X,Y) \times p^{-1}(g) (W,Z)
%\Rightarrow 
%p^{-1}(f) (X,Y)\times p^{-1}(g) (W,Z)
%= \{h:X\to Y | p(h)=f \}\times  \{k: W\to Z | p(k)=g \}
%\Rightarrow
%\{\ell :X\otimes W \to Z \otimes Y | p(\ell)= f\otimes g  \}
%$$
%
%taking $(h,k) \mapsto h\otimes k$
%
%and oplaxator:
%
%$$
% p^{-1}(f \otimes g) \Rightarrow p^{-1}(f) \times p^{-1}(g)
%$$
%
%by functions:
%
%$$
% p^{-1}(f \otimes g)
%=
% \{h:X\to Y | p(h)=f\otimes g \}
%$$
%




The components of the monoidal laxtator at $(f:X\to X',g:X''\to X''')$

$$
p^{-1}(f) \times p^{-1}(g) \Rightarrow p^{-1}(f \otimes g)
$$

are given by functions 
sending elements $(h,k)\in p^{-1}(f)\times p^{-1}(g)$ to $h\otimes k \in p^{-1}(f \otimes g)$.  This is a function because $p(h\otimes k) = p(h)\otimes p(k)= h\otimes k$.


Unitor



For the oplaxator at $(f:X\to X',g:X''\to X''')$


$$
p^{-1}(f \otimes g) \Rightarrow p^{-1}(f) \times p^{-1}(g)  
$$
are given by functions 
sending elements $h \in p^{-1}(f\otimes g)$


% $h;k$.  This is a function because $p(h;k) = p(h);p(k)=f;g$.


\end{proof}






