

%
%%
%%%The Grothendieck construction establishes an equivalence of categories between pseudofunctors from categories $\X$ into $\Cat$ and fibrations over $\X$.  There is a two-sided variation of this construction:
%%%
%%%
%%%\begin{theorem}{Benabou-Grothendieck}
%%%Take a lax normal functor $F:\mathcal{I}\to\Prof$ from a 1-category $\mathcal{I}$.
%%%
%%%Then the following pullback exists:
%%%$$
%%%\xymatrix{
%%%\int F \ar[d]_{} \ar[r]^{} & \Prof^*  \ar[d]\\
%%%\mathcal{I}^{\op} \ar[r] ^F          & \Prof
%%%}
%%%$$
%%%
%%%Where $\int F$ is a 1-category and $\int F \to \X$ is a functor.
%%%This extends to an isomorphism of categories:
%%%$$\int:\Cat_{l,n}(\X, \Prof)\cong \Cat/\X:\delta$$
%%%\end{theorem}
%%%
%%%$\Cat_{l,n}(\X, \Prof)$ is the category of lax normal functors from $\X$ to $\Prof$ and lax natural transformations. $\Cat/\X$ is the slice category over $\X$.
%%%
%%%
%%%Explicitly:
%%%\begin{lemma}
%%%$\int F$ has:
%%%\begin{description}
%%%\item[Objects:] Pairs $(Y \in \X_0, Y^\sharp \in F(Y))$
%%%\item[Morphisms:] 
%%%
%%%
%%%
%%%\item[Composition]
%%%\end{description}
%%%The functor $\int F\to \mathcal{I}^\op$ is the first projection from the pullback.
%%%
%%%\end{lemma}
%%%and conversely
%%%\begin{lemma}
%%%Given a functor $\delta: \mathcal{J}\to\mathcal{I}^\op$ $\delta \pi$ is the lax functor such that
%%%\begin{description}
%%%\item
%%%\item
%%%\itemend{description}
%%%\end{lemma}
%%%
%%%
%%%
%%%The Grothendieck construction has been extended to the monoidal case: establishing an equivalence between monoidal fibrations and monoidal pseudofunctors \cite[].  Similarly \cite[] prove the polycategorical Benabou-Grothendieck equivalence.  We adapt the work of both authors, by establishing a monoidal Grothendieck-Benabou equivalence:
%%%%
%%%%
%%%%
%%%%\begin{theorem}[Monoidal Grothendieck-Benabou construction.]
%%%%There is an equivalence of categories between the lax monoidal, lax normal functor category:
%%%%
%%%%\begin{description}
%%%%\item[Objects:]
%%%%\item[Maps:]
%%%%\item[Identity:]
%%%%\item[Composition:]
%%%%\end{description}
%%%%
%%%%and  the monoidal slice cateogry:
%%%%
%%%%\begin{description}
%%%%\item[Objects:]
%%%%\item[Maps:]
%%%%\item[Identity:]
%%%%\item[Composition:]
%%%%\end{description}
%%%%
%%%%\end{theorem}
%%%%\begin{proof}
%%%%Proof strategy: Start with a monoidal functor $\Y\to \X$ between strict monoidal categories.  Build a lax normal lax monoidal functor $\X\to \Prof$
%%%%
%%%%
%%%%Let $\X$ be a monoidal category, regarded as a monoidal bicategory with one object, and take a lax monoidal lax functor $F:\X\to\Prof$.  Then the following pullback exists in the category of monoidal bicategories, lax monoidal lax functors and lax natural transformations:
%%%%$$
%%%%\xymatrix{
%%%%\int F \ar[d]_{} \ar[r]^{} & \Prof^*  \ar[d]\\
%%%%\X \ar[r] ^F          & \Prof
%%%%}
%%%%$$
%%%%
%%%%$\int F$ has the same categorical structure as in the non-monoidal 
%%%%
%%%%
%%%%\end{proof}
%%%
%%%
%%%
%%%If $F$ is a normal frobenius monoidal lax normal functor
%%%with laxator $\ell$ monoidal laxator $\mu$ oplaxator $\nu$
%%%, then $\int F$ has an induced monoidal structure with:
%%%
%%% $F(X\otimes Y)\to F(X)\otimes F(Y) \to F(X')\otimes F(Y') \to F(X'\otimes Y')$
%%%
%%%\begin{description}
%%%\item[Tensor product:] On Objects:
%%%$$
%%%(Y,Y^\sharp) \otimes (Z,Z^\sharp)
%%%:=
%%%(Y\otimes Z, \nu(\mu(Y^\sharp, Z^\sharp)))
%%%$$
%%%
%%%On morphisms:
%%%$$
%%%(f,f^\sharp) \otimes (g,g^\sharp)
%%%:=
%%%(f\otimes g, \nu(\mu(f^\sharp, g^\sharp)) )
%%%$$
%%%
%%%\item[Tensor unit:]
%%%$$
%%%(I,* \in F(I))=\mathbb{1})
%%%$$
%%%\item[Unitors:]
%%%$$
%%%u_{(X,X^\sharp)}^R: (X,X^\sharp)\otimes (I,*)=(X\otimes I, \mu^{\otimes}(X^\sharp, *))  \to (X,X^\sharp)
%%%$$
%%%is given by
%%%$$
%%%(u_X^{R}:X\otimes I\to X,  f\in F(u_X^{R})(X^\sharp, \mu^{\otimes}(X^\sharp, *) )   )
%%%$$
%%%
%%%Where $f$ is the identity on $X$ along the isomorphism $ \mu^{\otimes}(X^\sharp, *) = $
%%%
%%%$
%%%(u_X^{R})(X^\sharp, X^\sharp)
%%%\cong
%%%(u_X^{R})(X^\sharp, \mu^{\otimes}(X^\sharp, *))
%%%$
%%%
%%%
%%%\item[Associator:]
%%%
%%%$$
%%%(\alpha_{X,Y,Z},  g):((X\otimes Y)\otimes Z, \mu^\otimes(\mu^{\otimes}(X^\sharp, Y^\sharp),Z^\sharp) \to 
%%%(X\otimes (Y\otimes Z),\mu^\otimes(X^\sharp, \mu^\otimes(Y^\sharp,Z^\sharp)))
%%%$$
%%%
%%%\end{description}
%%%
%%%
%%%Where the projection map $p:\int F\to\mathcal I$ moreover is strong monoidal:
%%%
%%%$$
%%%p ((X,X^\sharp )\times (Y,Y^\sharp )) = 
%%%$$
%%%
%%%
%%% Since we are regarding $\Prof$ as a quasistrict monoidal bicategory, if $\mathcal  I$ is strict monoidal, then so is $\int F$ so that the projection $\int F\to \mathcal I $ is strict monoidal.
%%%
%%%
%%%Conversely, suppose there is a strong monoidal functor $\pi:\mathcal{J}\to\mathcal{I}^\op$ between monoidal categories.
%%%Then $\delta \pi:\mathcal{I}^\op \to \Prof$ is a Frobenius monoidal, lax normal functor.  The monoidal laxators are given by:
%%%
%%%
%%%
%%%
%%%
%%%
%%%Given a monoidal category $p:\X\to \mathbb{1}$, the Frobenius monoidal lax structure of $\delta p : \mathbb{1}\to \Prof$ regarded as a lax monoidal functor is precisely the data of a representable lax special \dag-Frobenius algebra in $\Prof$:
%%%
%%%GIVE AXIOMS
%%%
%%%
%%%This is essentially the two-sided version of the coherence data of a monoidal category. 
%%%
%%%2-categorical spider theorem
%%%
%%%Since $\mathbb{1}$ is strict monoidal, $\int \delta p$ is as well.
%%%Therefore we can deduce that this is the strictification of $\X$.
%%%
%%%Slice category definition of grothendieck construction
%%%
%%%Recalling the string diagrams for pointed profunctors, we have that the strictification of $\X$ is generated by.
%%%
%%%
%%%The monoidal Benabou-Grothendieck construction is a very general construction for creating string diagrams for the strictification of monoidal functors.  Given some algebraic structure $F$ in  $\Prof$, the lax normal, lax monoidal structure can be regarded as the data for a normal form.  Then each object in $\int F$ contains the information need
%%%
%%%
%%%
%%%
%%%
%%%
%%%
%%%
%%%
%%%
%%%
%%%
%%%
%%%
%%%
%%%
%%%
%%%
%%%
%%%
%%%
%%%
%%%
%%%
%%%
%%%
%%%
%%%
%%%Monoidal categories $\X$ are in bijection with pseudomonoids in Cat.
%%%These are in bijection with extraspecial representable dagger frobenius algebras in Prof
%%%which are in bijection with lax seperable normal dagger frobenius monoidal lax functors $F_\X:\mathbb{1}\to\Prof$.
%%%
%%%Since $\mathbb 1$ is strict monoidal so is $\int F_\X$.
%%%Moreover, there is a $\dag$-Frobenius monoidal pseudo functor $\iota:\X\to \Prof^*$ making the diagram commute:
%%%
%%%$$
%%%\xymatrix{
%%%\X  \ar[drr]^\iota \ar[ddr]  & &\\
%%%       &  \int F_\X \ar[d]_{} \ar[r]^{} & \Prof^*  \ar[d]\\
%%%       &  \mathbb{1} \ar[r] ^F          & \Prof
%%%}
%%%$$
%%%
%%%Therefore, the universal map $G:\X\to F_\X$ is a Frobenius monoidal pseudofunctor.  It can also be shown to be strong monoidal, and moreover an equivalence of categories.  Therefore $\int F_\X$ is the monoidal strictification of $\X$.
%%%
%%%
%%%This extends to an equivalence of categories:
%%%
%%%Monoidal functors $\X\to \Y$ are in bijection with pseudomonoid homomorphisms in Cat.  These are in bijection with monoidal natural transformations 
%%%$F_\X\Rightarrow F_\Y$.
%%%These are in bijection with strict monoidal functors $\int F_\X\to \int F_\Y$. These are in 
%%%
%%%Surely intertwiners between pseudomonoid homorphisms correspond to strict monoidal natural transformations.
%%%
%%%
%%%
%%%
%%%
%%%
%%%
%%%
%%%
%%%
%%%
%%%First show
%%%
%%%\begin{description}
%%%\item[0-cells:] Frobenius monoidal functors \mathcal{I}\to\Prof$
%%%\item[1-cells:] Frobenius monoidal lax natural transformations.
%%%\item[2-cells:] Intertwiners
%%%\end{description}
%%%
%%%is 2-equivalent
%%%
%%%\begin{description}
%%%\item[0-cells:] \int F
%%%\item[1-cells:] strong monoidal functors $\int F \to \int G$ making the triangle commute.
%%%\item[2-cells:] monoidal natural transformations
%%%\end{description}
%%%
%%%
%%%
%%%
%%%
%%%
%%%
%%%
%%%
%%%
%%%
%%%Consider the 2-category of:
%%%
%%%\begin{description}
%%%\item[0-cells:] Monoidal categories
%%%\item[1-cells:] Monoidal functors.
%%%\item[2-cells:] Monoidal natural transformations
%%%\end{description}
%%%
%%%is 2-isomorphic 
%%%
%%%\begin{description}
%%%\item[0-cells:] Pseudomonoids in Cat
%%%\item[1-cells:] Pseudomonoid homomorphisms.
%%%\item[2-cells:] Intertwiners
%%%\end{description}
%%%
%%%is 2-isomorphic 
%%%
%%%\begin{description}
%%%\item[0-cells:] XXX Frobenius pseudomonoid 
%%%\item[1-cells:] Pseudomonoid homomorphisms.
%%%Composition by conjugation
%%%\item[2-cells:] Intertwiners
%%%\end{description}
%%%
%%%is 2-isomorphic 
%%%
%%%\begin{description}
%%%\item[0-cells:] Frobenius monoidal functors $\mathbb{1}\to\Prof$
%%%\item[1-cells:] Frobenius monoidal lax natural transformations.
%%%\item[2-cells:] Intertwiners
%%%\end{description}
%%%
%%%is 2-equivalent
%%%
%%%\begin{description}
%%%\item[0-cells:] \int F
%%%\item[1-cells:] \alpha:\int F\to \int G
%%%\item[2-cells:] Intertwiners
%%%\end{description}
%%%
%%%is 2-isomorphic 
%%%
%%%\begin{description}
%%%\item[0-cells:] Strict monoidal categories
%%%\item[1-cells:] monoidal 
%%%\item[2-cells:] Intertwiners
%%%\end{description}
%%%
%%%
%%%
%%%
%%%
%%%
%%%
%%%
%%%
%%%Scalable ZX-calculus.
%%%
%%%hierarchical string diagrams
%%%
%%%
%%%
%%%
%%%
%%%Take the lax Frobenius monoidal $F:\N \to\Prof$
%%%sending $n=\prod_i p_i^{a_i }\mapsto \prod \Span(\Mat_{\F_{p_i}})$  for where the tensor in $\N$ is multiplication.
%%%
%%%
%%%There is a faithful functor $\int F\to\FHilb$ picking out phase free ZX diagrams with arbitrary finite dimension. This is because for the full subcategory off prime prower dimension $\int F |_p \hookrightarrow \int F$, $\int F |_p\cong \Span(\Mat_{\F_{p_i}})$. Moreover, the maps given by the laxators are change of basis vectors.
%%%
%%%
%%%
%%%If instead we do the same trick but sending $p$ to odd prime dimensional qudit complete-ZX diagrams, then we regain the qufinite presentation of the ZX-calculus of \cite{wang???}.
%%%
%%%
%%%
%%%For stabilizers, we can do the same modulo scalars, but with $F:\N/\{2\} \to\Prof$ picking out odd prime dimensional stabilizer diagrams.
%%%
%%%
%%
%%
%%
%%
%%
%%
%%
%%
%%
%%
%%
%%\begin{definition}
%%Given bicategories $\X$ and $\Y$, a lax normal functor $\X\to\Y$ is:
%%
%%TODO
%%\end{definition}
%%
%%
%%\begin{definition}
%%Given monoidal bicategories $\X$ and $\Y$, a  Frobenius monoidal lax normal functor is a lax normal functor $\X\to\Y$, equipped with 2-cells $\mu$ and $\nu$ called the monoidal laxator and oplaxators, and coherences called the left and right frobeniusators interacting with the compositors todo 
%%
%%TODO
%%\end{definition}
%%
%%
%%
%%\begin{definition}
%%A morphism between Frobenius monoidal lax normal functors $F,G\X\to\Y$ is.  Given two monoidal bicategories, this notion induces the Frobenius monoidal lax normal functor category, denoted $[\X,\Y]_{fln}$
%%\end{definition}
%%
%%\begin{lemma}
%%$[\X,\Y]_{fln}$ is a monoidal category with
%%\end{lemma}
%%
%%
%%
%%\begin{definition}
%%A monoidal displayed category is a monoidal category $\D$ equipped with a  Frobenius monoidal lax functor $\mathcal{D}\to\Prof$.
%%\end{definition}
%%
%%\begin{theorem}{Monoidal Grothendieck-Benabou construction}
%%Given a monoidal category $\X$, there is a monoidal equivalence between the Frobenius monoidal lax normal functor category $[\X,\Prof]_{fln}$ and the strict monoidal coslice category over $\X$.
%%\end{theorem}
%%
%%
%%
%%
%%
%%
%%
%%\begin{proof}
%%Fix a monoidal category $\X$.
%%
%%Take a strict monoidal functor $p:\Y\to\X$.
%%
%%
%%Given an object $X$ of $\X$, the indexed category of $p$ over $X$, $p^{-1}(X)$ has:
%%
%%Objects in $Y \in \Y$ such that $p^{-1}(Y)=X$.
%%
%%Morphisms $f:Y\to Y'$ such that $p^{-1}(f)=1_X$.
%%
%%Composition and identities in $\Y$.
%%
%%
%%Given a morphism $f:X\to X'$ to in $\X$, the reindexing profunctor $p^{-1}(f):p^{-1}(X)^\op  \times p^{-1}(X')\to \Set$ sends:
%%
%%objects: $p^{-1}(f)(Y,Y')= \{ g:Y\to Y' | p(g)=f \}$
%%maps: $p^{-1}(f)(h:Y\to Y', k:Z \to Z')=\lambda x \in p^{-1}(f)(Y,Y'). h;x;k$.
%%
%%
%%This has the structure of a lax normal functor $P^{-1}:\X\to\Prof$.
%%
%%
%%For objects $X,X',X'' \in \X$ the compositors at $X,X',X''$ at components $f:X\to X'$ and $g:X'\to X''$ are functions $\int^Z p^{-1}(f)(X,Z) \times p^{-1}(g)(Z,X'') \to  p^{-1}(f;g)(X,X'')$,
%%sending elements $(h,k)$ of the equivalence class to their composite $h;k$.  This is a function because $p(h;k) = p(h);p(k)=f;g$.
%%
%%
%%For each $X \in \X$, $p^{-1}(f)=1_X$ is the identity profunctor on $p^{-1}(X)$, so this functor is normal.
%%The desired commutative diagrams hold making this into a lax normal functor.
%%
%%Furthermore, we now show that this lax normal functor preserves the monoidal structure lax-Frobeniusly.
%%
%% 
%%
%%%1_{p^{-1}(X)} \to 
%%
%%%
%%%The components of the monoidal laxtator at $(f,g)$
%%%
%%%$$
%%%p^{-1}(f) \times p^{-1}(g) \Rightarrow p^{-1}(f \otimes g)
%%%$$
%%%
%%%are given by functions
%%%$$
%%%p^{-1}(f)\times p^{-1}(g) = \int^{Y} p^{-1}(f) (X,Y) \times p^{-1}(g) (W,Z)
%%%\Rightarrow 
%%%p^{-1}(f) (X,Y)\times p^{-1}(g) (W,Z)
%%%= \{h:X\to Y | p(h)=f \}\times  \{k: W\to Z | p(k)=g \}
%%%\Rightarrow
%%%\{\ell :X\otimes W \to Z \otimes Y | p(\ell)= f\otimes g  \}
%%%$$
%%%
%%%taking $(h,k) \mapsto h\otimes k$
%%%
%%%and oplaxator:
%%%
%%%$$
%%% p^{-1}(f \otimes g) \Rightarrow p^{-1}(f) \times p^{-1}(g)
%%%$$
%%%
%%%by functions:
%%%
%%%$$
%%% p^{-1}(f \otimes g)
%%%=
%%% \{h:X\to Y | p(h)=f\otimes g \}
%%%$$
%%%
%%
%%
%%
%%
%%The components of the monoidal laxtator at $(f:X\to X',g:X''\to X''')$
%%
%%$$
%%p^{-1}(f) \times p^{-1}(g) \Rightarrow p^{-1}(f \otimes g)
%%$$
%%
%%are given by functions 
%%sending elements $(h,k)\in p^{-1}(f)\times p^{-1}(g)$ to $h\otimes k \in p^{-1}(f \otimes g)$.  This is a function because $p(h\otimes k) = p(h)\otimes p(k)= h\otimes k$.
%%
%%
%%Unitor
%%
%%
%%
%%For the oplaxator at $(f:X\to X',g:X''\to X''')$
%%
%%
%%$$
%%p^{-1}(f \otimes g) \Rightarrow p^{-1}(f) \times p^{-1}(g)  
%%$$
%%are given by functions 
%%sending elements $h \in p^{-1}(f\otimes g)$
%%
%%
%%% $h;k$.  This is a function because $p(h;k) = p(h);p(k)=f;g$.
%%\end{proof}
%
%
%
%\newcommand{\Cat}{{\sf Cat}}
%
%In this Section, we will propose how to categorify categorical quantum mechanics by regarding monoidal categories themselves as a certain kind of special-commutative \dag-Frobenius algebra.  This will give a categorical account of proof nets for monoidal categories which we reviewed in the first Chapter.  This chapter is much more exploratory than the others, and indeed there is much more work that should be worked out in the future.
%
%
%We first need the following definition to motivate where we are going:
%
%\begin{definition}
%The symmetric monoidal 2-category of {\bf pointed Categories} is the coslice category $\Cat^*:=1/\Cat$.  Explicitly, this has:
%
%\begin{description}
%\item[0-cells:] A pointed category is a pair consisting of a category along with a chosen object of that category: 
%
%$$(\X, X\in\X_0)$$
%
%\item[1-cells:] A pointed functor between pointed categories is a pair consisting of a functor between the underlying categories and a morphism that preserves the points, as follows:
%
%$$(F:\X\to\Y, f\in \Y(F(X),Y) ):(\X, X\in\X_0)\to (\Y, Y\in\Y_0)$$
%
%\item[2-cells:] Given two parallel pointed functors, 
%
%$$
% (F:\X\to\Y, f\in \Y(F(X),Y) ),  (G:\X\to\Y, g\in \Y(G(X),Y) ):(\X, X\in\X_0)\to $(\Y, Y\in\Y_0)$
%$$
%
%a pointed natural transformation is natural transformation  $\phi:F\to G$ that preserves the distinguished map, so that $\phi_X:g=f$.
%\end{description}
%
%
%Composition of the 1-cells and 2-cells is given pointwise; and the monoidal structure is given by the Cartesian product.
%
%\end{definition}
%
%There is a graphical calculus for pointed categories.  If $\X$ is a monoidal category, then for every map $f:X\otimes Y\to Z$, there is a pointed functor:
%
%
%$$(\_\otimes \_ :\X^2\to\X, f\in \Y(F(X),Y) ):(\X^2, (X,Y)\in\X^2_0)\to (\X, Z\in\X_0)$$
%
%Drawn as follows:
%
%TODO DRAW
%
%
%Moreover, for every state $f:I\to X$, there is a pointed functor
%
%
%$$(I:1\to\X, f\in \Y(I,X) ):(1,* \in 1)\to (\X, I\in\X_0)$$
%
%
%
%Drawn as follows:
%
%DRAW
%
%
%
%Where $1$ is the strict monoidal category with one object and one morphism and $I:1\to \X$ is the functor which picks out the tensor unit and its identity in $\X$.
%
%
%Now, the pentagon equation means that the following diagram commutes:
%
%DRAW ASSOCIATOR WITH TREES AND ELEMENTS
%
%And the left and right unit equations mean that the following diagrams commute:
%
%DRAW LEFT AND RIGHT UNITORS WITH ELEMENTS 
%
%
%Instead of defining a monoidal category as we did in the first chapter, we could have instead defined a monoidal category as a pseudomonoid in $\Cat$.  Then the tensor product would become the multiplication and the tensor unit would become the unit:
%
%\begin{definition}
%Draw pseudomonoid coherences
%
%
%
%
%
%\end{definition}
%
%However, this does not generalize to other algenraic structures in $\Cat$.  For this we will need the following definition:
%
%
%\begin{definition}
%
%A {\bf pseudofunctor }
%A {\bf lax monoidal pseudofunctor}
%
%
%\end{definition}
%
%With this in mind, we have a concise definition of a monoidal category:
%
%
%\begin{lemma}
%A monoidal category is the data of a  lax monoidal pseudofunctor $:1\to \Cat$, where $1$ is the terminal monoidal category with one object and one morphism.
%\end{lemma}
%
%
%
%This definition may seem terse, as it moves around the coherence data into a different place; however, this is much more amenable to generalization.
%Indeed,  a {\bf monoidal indexed category} is a monoidal category $\X$ equipped with a lax monoidal functor $F:\X\to \Cat$.  
%
%
%\begin{definition}
%To every monoidal indexed category $F:\X\to \Cat$, the {\bf monoidal Grothendieck category} $\int F$ is a monoidal category with:
%
%
%\begin{description}
%
%\item[Objects: ]
%
%\item[Maps: ]
%
%Where the composition is:
%
%And the identity is:
%
%
%\item[Monoidal structure:]
%
%\end{description}
%
%\end{definition}
%
%
%Note the the projection  $\int F\to \X$ is strict monoidal.
%
%
%This projection is actually more than just a strict monoidal functor, it is a fibration!  In \cite{???}, they show that given a fixed monoidal category $\X$, the slice category of strict monoidal fibrations is equivalent to the lax monoidal pseudofunctor category from $\X\to \Cat$.  We wont discuss this further, because it is not immediately useful for our purposes.
%
%
%If we return to our working example $F_\X:1\to \Cat$ picking out a monoidal category, we find that there is a strict monoidal isomorhism between $\int F_\X$ and $\X$:
%
%TODO Draw composition and tensor product in terms of tubes.
%
%
%
%However, we could also take a different monoidal pseudofunctor into $\Cat$.
%
%\begin{definition}
%
%Given a monoidal category $\X$, define a strong monoidal pseudofunctor: $\Delta_\X:\Delta\to \Cat$
%
%sending $n\mapsto \X^n$, where the laxator left associates trees and eliminates units.
%
%\end{definition}
%
%Explicitly:
%
%\begin{lemma}
%$\int \Delta_\X$ is the strict monoidal category with:
%\begin{description}
%\item[Objects:] Finite lists of objects in $\X$.
%
%\item[Maps:]  Maps are binary forests, which are left associated where the units are elminated whenever possible.
%
%\item[Tensor:]  The tensor product is given by the cartesian product in pointed functors.
%
%\item[Composition:] The composition is given by nesting and then reducing the all connected components to the same normal form.
%
%
%\end{description}
%\end{lemma}
%
%
%However, this is a very one-sided construction, as the string diagrams in $\int \Delta_\X$ has the shapes of trees. 
%
%Draw laxator:
%
%
%One way to get around this is look at $\X$ as an object in $\Prof$, rather than $\Cat$.  There are two yoneda embeddings $y^* \Cat^\co \to \Prof$ and $y_*:\Cat^\op \to \Prof$.  Therefore, we can regard a monoidal category as both a pseudomonoid in $\Prof$ or a pseudocomonoid in $\Prof$. 
%
%
%
%\begin{definition}
%The monoidal 2-category of pointed profunctors, $\Prof^*$, has
%
%
%\begin{description}
%\item[0-cells:]
%
%\item[1-cells:]
%
%\item[2-cells:]
%\end{description}
%\end{definition}
%
%
%Pointed profunctors has a very similar graphical calculus to pointed categories.  Given a monoidal category one can regard not only factorizations of the domains of maps $f:X\otimes Y\to Z$, and states $g:I\to X$ as pointed profunctors, but also factorizations of the codomains of maps   $g:X \to Y\times Z$, and effects $h:X\to I$:
%
%
%TODO
%
%
%Now, it is known that autonomous monoidal categories correspond to pseudofrobenius algebras in prof which interact with the units and counits of the adjunctions between the the different components of the yoneda embedding, see \cite{???,???}.  However, the literature is quite terse and lacking; indeed, in the literature, many people miss this extra coherence equation governing the interaction of the pseudofrobenius algebra with the adjoints.  Moreover, a general monoidal category would not induce a pseudo frobenius algebra, but a Lax frobenius algebra. 
%
%The lax frobeniusators are given by the following 2-cells; which are hardly ever going to be invertible:
%
%TODO
%
%Given a monoidal category $\X$, it is therefore and open question as to what notion of functor is needed to regard the monoidal structure of  $\X$ as some sort of monoidal functor $1\to\Prof$.  There are lots of coherence equations to work out, but I conjecture that it is a lax normal functor that is lax monoidal and oplax monoidal such that the lax and oplax structures interact to form a lax frobenius algebra where the monoidal lax structure  is XXXX adjoint to the oplax structure. 
%
%
%As much as it would be nice to complete the picture, we don't really need this.  
%
%\begin{definition}
%Given a category $\X$ and a 2-category $\Y$, a lax normal functor is a pseudofunctor where the compositors are no longer required to be isomorphisms.
%A {\bf displayed category } is a category $\X$ equipped with a lax normal functor $\X\to \Prof$.
%\end{definition}
%
%\begin{lemma}
%Given a displayed category $F:\X\to \Prof$, the Grothendieck-Benabou category $\Pi \X$ has:
%
%\begin{description}
%\item[Objects:]
%\item[Maps:]
%\item[Composition:]
%\item[Identities:]
%\end{description}
%
%
%\end{lemma}
%
%
%Moreover, the projection $\Pi \X\to \X$ is a functor.  Given some fixed category $\X$, the correspondence extends between an equivalence of categories between the lax normal functor category from $\X$ to $\Prof$ and the slice category over $\X$.  This correspondence exists in several places in the literature and is thought to be due to Benabou.  It has recently started being called the Grothendieck-Benabou construction.
%
%
%\begin{lemma}
%Take a displayed category $F:\X\to \Prof$ such that $\X$ is monoidal and $F$ has the structure of a strong monoidal 2-functor.
%
%Then $\Pi \X$ has a monoidal structure given by:
%
%\end{lemma}
%
%
%%%%%This is where the important stuff is
%
%\begin{defintion}
%Given a monoidal category $\X$, there is a strong monoidal lax functor $G_\X:frob\to \Prof$ with
%
%\begin{description}
%\item[Objects:] $n\mapsto \X^n$
%\item[Compositor:] 
%\item[Unitor:] 
%\item[Tensorator:] 
%\item[Tensor-unitor:] 
%\end{description}
%
%
%\end{defintion}
%
%
%\begin{lemma}
%The Grothendieck benabou category $\Pi_\X$ is monoidal with:
%
%
%\begin{description}
%\item[Objects:] Finite lists of objects in $\X$.
%
%\item[Maps:] Two-sided forests in spider normal form, alongside elements:
%
%\item[Composition:] Composition is given by spider fusion of shapes.
%
%\item[Tensor:] The tensor product is given by the cartesian product in pointed profunctors.
%
%
%
%\end{description}
%\end{lemma}
%
%Proof nets live within the subcategory whose shapes are connected.  To actually get things to be connected, we have to force things to be connected:
%
%
%\begin{lemma}
%
%we can define a different tensor product on this category that squeezed to top and bottom wires together:
%
%\begin{definition}
%Given an object $X=[X_0,\cdots, X_{n-1}]$, define the projector on $X$ to be the map which inserts identities into the following diagram:
%
%Take the full subcategory of the Karoubi envelope, $\Lambda \X$, consisting of these projectors.
%
%This is the same as pr
%\end{definition}
%
%
%
%
%
%
%
%
%
%
%
%
%
%
%
%
%
%
%
%
%
%
%
%
%
%
%
%
%
%
%
%
%
%
%
%
%
%
%
%
%
%
%
%
%
%
%
%
%
%
%
%
%
%


In this Chapter, we attempt a high-level construction of  proof nets by categorifying Frobenius algebras and the spider theorem. We have left the categorification of the spider theorem as a conjecture; however, we  hope that the highly conceptual level of the construction opens up avenues for finding string diagrams for algebraic structures other than pseudomonoids.


Throughout this thesis, we have been drawing string diagrams in monoidal, or symmetric monoidal categories in which our familiar notions of monoids and Frobenius algebras and so on live.  However, if we move to the higher dimensional setting of monoidal 2-categories, all of the equalities defining the axioms for these structures become natural transformations, satisfying more subtle coherence laws.   For this section, I will assume that the reader has some familiarity with monoidal 2-categories. First, we categorify a monoid in a monoidal category:

\begin{definition}
A {\bf pseudomonoid} in a monoidal 2-category is an object $\mathcal C$ equipped with two 1-cells $\mu:\mathcal{C} \otimes \mathcal{C} \to \mathcal{C}$ and $\eta:I\to \mathcal C$ drawn as follows
$$
\begin{tikzpicture}
	\begin{pgfonlayer}{nodelayer}
		\node [style=none] (0) at (1, 0) {};
		\node [style=none] (1) at (1, 1) {};
		\node [style=none] (2) at (0.5, -1) {};
		\node [style=none] (3) at (1.5, -1) {};
		\node [style=dot] (4) at (2.5, 0) {};
		\node [style=none] (5) at (2.5, 1) {};
		\node [style=none] (6) at (2, -1) {};
		\node [style=none] (7) at (3, -1) {};
		\node [style=none] (8) at (1.75, 0) {$=:$};
		\node [style=box] (9) at (4, 0) {$I$};
		\node [style=none] (10) at (4, 1) {};
		\node [style=dot] (11) at (5.5, 0) {};
		\node [style=none] (12) at (5.5, 1) {};
		\node [style=none] (13) at (4.75, 0) {$=:$};
		\node [style=none] (14) at (3.25, 0) {$,$};
		\node [style=dot, fill=white] (15) at (1, 0) {$\otimes$};
	\end{pgfonlayer}
	\begin{pgfonlayer}{edgelayer}
		\draw [in=-135, out=90] (2.center) to (0.center);
		\draw (0.center) to (1.center);
		\draw [in=-45, out=90] (3.center) to (0.center);
		\draw [in=-135, out=90] (6.center) to (4);
		\draw (4) to (5.center);
		\draw [in=-45, out=90] (7.center) to (4);
		\draw (9) to (10.center);
		\draw (11) to (12.center);
	\end{pgfonlayer}
\end{tikzpicture}
$$





 as well as two 2-cells, the associator, left and right unitors:

$$
\begin{tikzpicture}
	\begin{pgfonlayer}{nodelayer}
		\node [style=dot,fill=white] (0) at (4, 0) {};
		\node [style=dot,fill=white] (1) at (4.5, 1) {};
		\node [style=none] (2) at (3.5, -1) {};
		\node [style=none] (3) at (4.5, -1) {};
		\node [style=none] (4) at (5.5, -1) {};
		\node [style=none] (5) at (4.5, 2) {};
		\node [style=dot,fill=white] (6) at (7.5, 0) {};
		\node [style=dot,fill=white] (7) at (7, 1) {};
		\node [style=none] (8) at (8, -1) {};
		\node [style=none] (9) at (7, -1) {};
		\node [style=none] (10) at (6, -1) {};
		\node [style=none] (11) at (7, 2) {};
		\node [style=none] (12) at (5.75, 1) {$\xRightarrow{\alpha}$};
		\node [style=dot,fill=white] (13) at (9.5, 1) {};
		\node [style=none] (15) at (10, 0) {};
		\node [style=none] (16) at (9, 0) {};
		\node [style=none] (18) at (9.5, 1.75) {};
		\node [style=dot,fill=white] (19) at (9, 0) {};
		\node [style=none] (20) at (10.25, 1) {$\xRightarrow{u^L}$};
		\node [style=dot,fill=white] (21) at (12.5, 1) {};
		\node [style=none] (22) at (12, 0) {};
		\node [style=none] (23) at (13, 0) {};
		\node [style=none] (24) at (12.5, 1.75) {};
		\node [style=dot,fill=white] (25) at (13, 0) {};
		\node [style=none] (26) at (11.75, 1) {$\xLeftarrow{u^R}$};
		\node [style=none] (27) at (11, 1.75) {};
		\node [style=none] (28) at (11, 0) {};
		\node [style=none] (29) at (8.25, 0.75) {,};
	\end{pgfonlayer}
	\begin{pgfonlayer}{edgelayer}
		\draw [in=90, out=-135] (0) to (2.center);
		\draw [in=-45, out=90] (3.center) to (0);
		\draw [in=-135, out=90] (0) to (1);
		\draw (1) to (5.center);
		\draw [in=-45, out=90] (4.center) to (1);
		\draw [in=90, out=-45] (6) to (8.center);
		\draw [in=-135, out=90] (9.center) to (6);
		\draw [in=-45, out=90] (6) to (7);
		\draw (7) to (11.center);
		\draw [in=-135, out=90] (10.center) to (7);
		\draw [in=90, out=-45] (13) to (15.center);
		\draw [in=-135, out=90] (16.center) to (13);
		\draw (13) to (18.center);
		\draw [in=90, out=-135] (21) to (22.center);
		\draw [in=-45, out=90] (23.center) to (21);
		\draw (21) to (24.center);
		\draw (28.center) to (27.center);
	\end{pgfonlayer}
\end{tikzpicture}
$$

Satisfying the maclane pentagon coherence equation (where the red box indicates where the nonidentity natural transformation is being applied):

$$
\begin{tikzpicture}
	\begin{pgfonlayer}{nodelayer}
		\node [style=none] (0) at (3.5, -4) {};
		\node [style=none] (1) at (4, -4) {};
		\node [style=none] (2) at (4.5, -4) {};
		\node [style=none] (3) at (5, -4) {};
		\node [style=none] (4) at (4.25, -1.5) {};
		\node [style=dot] (5) at (4.25, -3.25) {};
		\node [style=dot] (6) at (3.75, -2.75) {};
		\node [style=dot] (7) at (4.25, -2.25) {};
		\node [style=none] (8) at (2.75, -2.5) {$\xRightarrow{\alpha}$};
		\node [style=none] (9) at (5.75, -4) {};
		\node [style=none] (10) at (6.25, -4) {};
		\node [style=none] (11) at (6.75, -4) {};
		\node [style=none] (12) at (7.25, -4) {};
		\node [style=none] (13) at (6.5, -1.5) {};
		\node [style=dot] (14) at (6.5, -3.25) {};
		\node [style=dot] (15) at (6.5, -2.25) {};
		\node [style=none] (16) at (5.25, -2.5) {$\xRightarrow{\alpha}$};
		\node [style=dot] (17) at (7, -2.75) {};
		\node [style=none] (18) at (8, -2.5) {$\xRightarrow{\alpha}$};
		\node [style=none] (19) at (2.75, 0.75) {$\xRightarrow{\alpha}$};
		\node [style=none] (20) at (5.5, 0.75) {$\cong$};
		\node [style=none] (21) at (8, 0.75) {$\xRightarrow{\alpha}$};
		\node [style=none] (22) at (1.75, -3.5) {};
		\node [style=none] (23) at (1.75, -2.5) {};
		\node [style=none] (24) at (0.5, -2.5) {};
		\node [style=none] (25) at (0.5, -3.5) {};
		\node [style=none] (26) at (4.5, -3) {};
		\node [style=none] (27) at (4.5, -2) {};
		\node [style=none] (28) at (3.5, -2) {};
		\node [style=none] (29) at (3.5, -3) {};
		\node [style=none] (30) at (7.5, -3.5) {};
		\node [style=none] (31) at (7.5, -2.5) {};
		\node [style=none] (32) at (6.25, -2.5) {};
		\node [style=none] (33) at (6.25, -3.5) {};
		\node [style=none] (34) at (3.25, -0.75) {};
		\node [style=none] (35) at (3.75, -0.75) {};
		\node [style=none] (36) at (4.25, -0.75) {};
		\node [style=none] (37) at (4.75, -0.75) {};
		\node [style=dot] (38) at (3.5, 0) {};
		\node [style=dot] (39) at (4.5, 0.5) {};
		\node [style=dot] (40) at (4, 1) {};
		\node [style=none] (41) at (4, 1.75) {};
		\node [style=none] (42) at (5, -0.25) {};
		\node [style=none] (43) at (5, 0.75) {};
		\node [style=none] (44) at (3.25, 0.75) {};
		\node [style=none] (45) at (3.25, -0.25) {};
		\node [style=none] (46) at (7, 0.25) {};
		\node [style=none] (47) at (7, 1.25) {};
		\node [style=none] (48) at (6, 1.25) {};
		\node [style=none] (49) at (6, 0.25) {};
		\node [style=none] (50) at (2.25, 0.25) {};
		\node [style=none] (51) at (2.25, 1.25) {};
		\node [style=none] (52) at (1, 1.25) {};
		\node [style=none] (53) at (1, 0.25) {};
		\node [style=none] (54) at (6, -0.75) {};
		\node [style=none] (55) at (6.5, -0.75) {};
		\node [style=none] (56) at (7, -0.75) {};
		\node [style=none] (57) at (7.5, -0.75) {};
		\node [style=dot] (58) at (6.25, 0.5) {};
		\node [style=dot] (59) at (7.25, 0) {};
		\node [style=dot] (60) at (6.75, 1) {};
		\node [style=none] (61) at (6.75, 1.75) {};
		\node [style=none] (62) at (10, -0.5) {};
		\node [style=none] (63) at (9.5, -0.5) {};
		\node [style=none] (64) at (9, -0.5) {};
		\node [style=none] (65) at (8.5, -0.5) {};
		\node [style=dot] (66) at (9.75, 0.25) {};
		\node [style=dot] (67) at (9.25, 0.75) {};
		\node [style=dot] (68) at (8.75, 1.25) {};
		\node [style=none] (69) at (8.75, 2) {};
		\node [style=none] (70) at (1.5, -1.25) {$\shortparallel$};
		\node [style=none] (71) at (9.25, -1.25) {$\shortparallel$};
		\node [style=none] (72) at (10, -4) {};
		\node [style=none] (73) at (9.5, -4) {};
		\node [style=none] (74) at (9, -4) {};
		\node [style=none] (75) at (8.5, -4) {};
		\node [style=dot] (76) at (9.75, -3.25) {};
		\node [style=dot] (77) at (9.25, -2.75) {};
		\node [style=dot] (78) at (8.75, -2.25) {};
		\node [style=none] (79) at (8.75, -1.5) {};
		\node [style=none] (80) at (0.5, -4) {};
		\node [style=none] (81) at (1, -4) {};
		\node [style=none] (82) at (1.5, -4) {};
		\node [style=none] (83) at (2, -4) {};
		\node [style=dot] (84) at (0.75, -3.25) {};
		\node [style=dot] (85) at (1.25, -2.75) {};
		\node [style=dot] (86) at (1.75, -2.25) {};
		\node [style=none] (87) at (1.75, -1.5) {};
		\node [style=none] (88) at (0.5, -0.75) {};
		\node [style=none] (89) at (1, -0.75) {};
		\node [style=none] (90) at (1.5, -0.75) {};
		\node [style=none] (91) at (2, -0.75) {};
		\node [style=dot] (92) at (0.75, 0) {};
		\node [style=dot] (93) at (1.25, 0.5) {};
		\node [style=dot] (94) at (1.75, 1) {};
		\node [style=none] (95) at (1.75, 1.75) {};
	\end{pgfonlayer}
	\begin{pgfonlayer}{edgelayer}
		\draw [in=90, out=-60] (5) to (2.center);
		\draw [in=-120, out=90] (1.center) to (5);
		\draw (5) to (6);
		\draw [in=-165, out=90] (6) to (7);
		\draw (7) to (4.center);
		\draw [in=90, out=-45, looseness=0.75] (7) to (3.center);
		\draw [in=90, out=-120] (6) to (0.center);
		\draw [in=90, out=-60] (14) to (11.center);
		\draw [in=-120, out=90] (10.center) to (14);
		\draw (15) to (13.center);
		\draw [in=90, out=-165] (17) to (14);
		\draw [in=-45, out=90, looseness=0.75] (12.center) to (17);
		\draw (17) to (15);
		\draw [in=90, out=-150] (15) to (9.center);
		\draw [style=red,dashed] (22.center) to (23.center);
		\draw [style=red,dashed] (23.center) to (24.center);
		\draw [style=red,dashed] (24.center) to (25.center);
		\draw [style=red,dashed] (25.center) to (22.center);
		\draw [style=red,dashed] (26.center) to (27.center);
		\draw [style=red,dashed] (27.center) to (28.center);
		\draw [style=red,dashed] (28.center) to (29.center);
		\draw [style=red,dashed] (29.center) to (26.center);
		\draw [style=red,dashed] (30.center) to (31.center);
		\draw [style=red,dashed] (31.center) to (32.center);
		\draw [style=red, dashed] (32.center) to (33.center);
		\draw [style=red,dashed] (33.center) to (30.center);
		\draw [in=90, out=-60] (38) to (35.center);
		\draw [in=90, out=-120] (38) to (34.center);
		\draw (41.center) to (40);
		\draw [in=90, out=-15] (40) to (39);
		\draw [in=90, out=-60] (39) to (37.center);
		\draw [in=-120, out=90] (36.center) to (39);
		\draw [in=90, out=-135] (40) to (38);
		\draw [style=red,dashed] (42.center) to (43.center);
		\draw [style=red,dashed] (43.center) to (44.center);
		\draw [style=red,dashed] (44.center) to (45.center);
		\draw [style=red,dashed] (45.center) to (42.center);
		\draw [style=red,dashed] (46.center) to (47.center);
		\draw [style=red,dashed] (47.center) to (48.center);
		\draw [style=red,dashed] (48.center) to (49.center);
		\draw [style=red,dashed] (49.center) to (46.center);
		\draw [style=red,dashed] (50.center) to (51.center);
		\draw [style=red,dashed] (51.center) to (52.center);
		\draw [style=red,dashed] (52.center) to (53.center);
		\draw [style=red,dashed] (53.center) to (50.center);
		\draw [in=90, out=-60] (58) to (55.center);
		\draw [in=90, out=-120] (58) to (54.center);
		\draw (61.center) to (60);
		\draw [in=90, out=-45] (60) to (59);
		\draw [in=90, out=-60] (59) to (57.center);
		\draw [in=-120, out=90] (56.center) to (59);
		\draw [in=90, out=-165] (60) to (58);
		\draw [in=90, out=-120] (66) to (63.center);
		\draw [in=90, out=-60] (66) to (62.center);
		\draw (69.center) to (68);
		\draw [in=90, out=-15] (68) to (67);
		\draw [in=90, out=-15] (67) to (66);
		\draw [in=90, out=-120] (67) to (64.center);
		\draw [in=90, out=-120] (68) to (65.center);
		\draw [in=90, out=-120] (76) to (73.center);
		\draw [in=90, out=-60] (76) to (72.center);
		\draw (79.center) to (78);
		\draw [in=90, out=-15] (78) to (77);
		\draw [in=90, out=-15] (77) to (76);
		\draw [in=90, out=-120] (77) to (74.center);
		\draw [in=90, out=-120] (78) to (75.center);
		\draw [in=90, out=-60] (84) to (81.center);
		\draw [in=90, out=-120] (84) to (80.center);
		\draw (87.center) to (86);
		\draw [in=90, out=-165] (86) to (85);
		\draw [in=90, out=-165] (85) to (84);
		\draw [in=90, out=-60] (85) to (82.center);
		\draw [in=90, out=-60] (86) to (83.center);
		\draw [in=90, out=-60] (92) to (89.center);
		\draw [in=90, out=-120] (92) to (88.center);
		\draw (95.center) to (94);
		\draw [in=90, out=-165] (94) to (93);
		\draw [in=90, out=-165] (93) to (92);
		\draw [in=90, out=-60] (93) to (90.center);
		\draw [in=90, out=-60] (94) to (91.center);
	\end{pgfonlayer}
\end{tikzpicture}
$$


As well as the unit coherences:

$$
\begin{tikzpicture}
	\begin{pgfonlayer}{nodelayer}
		\node [style=none] (47) at (2.75, 0.75) {$\xRightarrow{\alpha}$};
		\node [style=none] (100) at (1.5, -1.25) {$\shortparallel$};
		\node [style=none] (120) at (0.75, -0.75) {};
		\node [style=none] (121) at (2.25, -0.75) {};
		\node [style=dot] (122) at (1.5, -0.25) {};
		\node [style=dot] (123) at (1.25, 0.5) {};
		\node [style=dot] (124) at (1.75, 1.25) {};
		\node [style=none] (125) at (1.75, 1.75) {};
		\node [style=none] (126) at (5, -0.5) {};
		\node [style=none] (127) at (5, 0.75) {};
		\node [style=none] (128) at (3.75, 0.75) {};
		\node [style=none] (129) at (3.75, -0.5) {};
		\node [style=none] (130) at (4.75, -0.75) {};
		\node [style=none] (131) at (3.25, -0.75) {};
		\node [style=dot] (132) at (4, -0.25) {};
		\node [style=dot] (133) at (4.25, 0.5) {};
		\node [style=dot] (134) at (3.75, 1.25) {};
		\node [style=none] (135) at (3.75, 1.75) {};
		\node [style=none] (136) at (2.25, 0.25) {};
		\node [style=none] (137) at (2.25, 1.5) {};
		\node [style=none] (138) at (1, 1.5) {};
		\node [style=none] (139) at (1, 0.25) {};
		\node [style=dot] (144) at (7.25, 0.75) {};
		\node [style=none] (145) at (7.25, 1.75) {};
		\node [style=none] (146) at (6.75, -0.75) {};
		\node [style=none] (147) at (7.75, -0.75) {};
		\node [style=none] (148) at (6, 0.75) {$\xRightarrow{ u^L}$};
		\node [style=none] (149) at (7.25, -1.25) {$\shortparallel$};
		\node [style=none] (150) at (0.5, -4.25) {};
		\node [style=none] (151) at (2, -4.25) {};
		\node [style=dot] (152) at (1.25, -3.75) {};
		\node [style=dot] (153) at (1, -3) {};
		\node [style=dot] (154) at (1.5, -2.25) {};
		\node [style=none] (155) at (1.5, -1.75) {};
		\node [style=none] (156) at (1.5, -4) {};
		\node [style=none] (157) at (1.5, -2.75) {};
		\node [style=none] (158) at (0.25, -2.75) {};
		\node [style=none] (159) at (0.25, -4) {};
		\node [style=none] (160) at (4.25, -3.25) {$\xRightarrow{u^R}$};
		\node [style=dot] (161) at (7.25, -3) {};
		\node [style=none] (162) at (7.25, -1.75) {};
		\node [style=none] (163) at (6.75, -4.25) {};
		\node [style=none] (164) at (7.75, -4.25) {};
	\end{pgfonlayer}
	\begin{pgfonlayer}{edgelayer}
		\draw (125.center) to (124);
		\draw [in=90, out=-165] (124) to (123);
		\draw [in=90, out=-45] (123) to (122);
		\draw [in=90, out=-135] (123) to (120.center);
		\draw [in=90, out=-45, looseness=0.75] (124) to (121.center);
		\draw [style=red,dashed] (126.center) to (127.center);
		\draw [style=red,dashed] (127.center) to (128.center);
		\draw [style=red,dashed] (128.center) to (129.center);
		\draw [style=red,dashed] (129.center) to (126.center);
		\draw (135.center) to (134);
		\draw [in=90, out=-15] (134) to (133);
		\draw [in=90, out=-135] (133) to (132);
		\draw [in=90, out=-45] (133) to (130.center);
		\draw [in=90, out=-135, looseness=0.75] (134) to (131.center);
		\draw [style=red,dashed] (136.center) to (137.center);
		\draw [style=red,dashed] (137.center) to (138.center);
		\draw [style=red,dashed] (138.center) to (139.center);
		\draw [style=red,dashed] (139.center) to (136.center);
		\draw (145.center) to (144);
		\draw [in=90, out=-45] (144) to (147.center);
		\draw [in=-135, out=90] (146.center) to (144);
		\draw (155.center) to (154);
		\draw [in=90, out=-165] (154) to (153);
		\draw [in=90, out=-45] (153) to (152);
		\draw [in=90, out=-135] (153) to (150.center);
		\draw [in=90, out=-45, looseness=0.75] (154) to (151.center);
		\draw [style=red,dashed] (156.center) to (157.center);
		\draw [style=red,dashed] (157.center) to (158.center);
		\draw [style=red,dashed] (158.center) to (159.center);
		\draw [style=red,dashed] (159.center) to (156.center);
		\draw (162.center) to (161);
		\draw [in=90, out=-45] (161) to (164.center);
		\draw [in=-135, out=90] (163.center) to (161);
	\end{pgfonlayer}
\end{tikzpicture}
$$

A pseudomonoid is {\bf strict} when the associator and unitors are idenities.

\end{definition}

Regard $\Cat$ as a symmetric monoidal 2-category where the tensor product is given by the product.

\begin{lemma}
Monoidal categories are pseudomonoids in $\Cat$ and strict monoidal categories are strict pseudomonoids in $\Cat$.
\end{lemma}

By looking at the points inside of $\Cat$, this way of viewing a monoidal category hints at some connection to proof nets:

\begin{definition}
The symmetric monoidal 2-category of {\bf pointed categories} is the coslice category $\Cat^*:=1/\Cat$.  Explicitly, this has:

\begin{description}
\item[0-cells:] A pointed category is a pair consisting of a category along with a chosen object of that category: 

$$(\X, X\in\X_0)$$

\item[1-cells:] A pointed functor between pointed categories is a pair consisting of a functor between the underlying categories and a morphism that preserves the point:

$$(F:\X\to\Y, f\in \Y(F(X),Y) ):(\X, X\in\X_0)\to (\Y, Y\in\Y_0)$$

\item[2-cells:] Given two parallel pointed functors $(F,f),  (G,g):(\X,X)\to (\Y,X)$,
a pointed natural transformation is natural transformation  $\phi:F\to G$ that preserves the distinguished map, so that $\phi_X:g=f$.
\end{description}


Composition of the 1-cells and 2-cells is given pointwise; and the monoidal structure is given by the Cartesian product.

\end{definition}

There is a graphical calculus for pointed categories.  Although to my knowledge, it is not fully worked out in the literature, it is alluded to in the work of \cite{vicary} (from which I will be using the source code for typesetting the diagrams) in the setting of vect-enriched profunctors; as well as later in the work of \cite{mario} for set-enriched profunctors.

Given a category $\X$, and map $f:X\to Y$ in $\X$; draw the pointed functor $(1_\X, f \in \X(1_{\X}(X),Y))$ as follows:


$$
\begin{tikzpicture}[scale=2]
    \node[Cyl, bot, top, height scale=1.0] (A) at (0,0) {};
    \begin{scope}[internal string scope]
        \node (i) at  ($(A.bot)+(0,-.3)$)  {\tiny $X$};
        \node (j) at ($(A.top)+(0,.3)$)  {\tiny $Y$};
        \node [tiny label] (g) at (A.center) {$f$};
        \draw (A.bot) to (A.top);
    \end{scope}
\end{tikzpicture}
$$

The maps inside the category live within the convex space contained withing the tubes; wherein one can apply rewrite rules coming from the equational theory of the category.  

Functors are drawn as membranes between tubes; functoriality means that things can pass up through the membrane:


$$
\begin{tikzpicture}[scale=2]
    \node[Cyl, bot, top,xscale=1.5,yscale=1.1] (A) at (0,0) {};
    \node[Cyl, bot, anchor=top,xscale=1.5,yscale=1.1] (B) at (A.bot) {};
    \node (g) at ($(A.bot)+(-.5,0)$) {\tiny $F$};
    \begin{scope}[internal string scope]
        \node (i) at  ($(B.bot)+(0,-.3)$)  {\tiny $X$};
        \node (j) at ($(A.top)+(0,.3)$)  {\tiny $F(Y)$};
        \node [tiny label] (g) at (B.center) {$f$};
        \draw (B.bot) to (A.top);
    \end{scope}
\end{tikzpicture}
=
\begin{tikzpicture}[scale=2]
    \node[Cyl, bot, top,xscale=1.5,yscale=1.1] (A) at (0,0) {};
    \node[Cyl, bot, anchor=top,xscale=1.5,yscale=1.1] (B) at (A.bot) {};
    \node (g) at ($(A.bot)+(-.5,0)$) {\tiny $F$};
    \begin{scope}[internal string scope]
        \node (i) at  ($(B.bot)+(0,-.3)$)  {\tiny $X$};
        \node (j) at ($(A.top)+(0,.3)$)  {\tiny $F(Y)$};
        \node [tiny label] (g) at (A.center) {$F(f)$};
        \draw (B.bot) to (A.top);
    \end{scope}
\end{tikzpicture}
$$

 If $\X$ is a monoidal category, then for every map with a binary tensor factorization of the domain  $f:X\otimes Y\to Z$ there is a pointed functor:


$$
(\_\otimes \_ :\X^2\to\X, f\in \Y(X\otimes Y ,Z) ):(\X^2, (X,Y)\in\X^2_0)\to (\X, Z\in\X_0)$$

Drawn as follows:

$$
\begin{tikzpicture}[scale=2]
    \node[Pants, bot, top, height scale=1.0] (A) at (0,0) {};
    \begin{scope}[internal string scope]
        \node (i) at (A.belt) [above] {\tiny $Z$};
        \node (j) at (A.leftleg) [below] {\tiny $X$};
        \node (k) at (A.rightleg) [below] { \tiny $Y$};
        \node [tiny label] (g) at (0,0.02\cobheight) {$f$};
        \draw (i.south)
            to (g.center)
            to [out=-140, in=up] (j.north);
        \draw (g.center)
            to [out=-40, in=up] (k.north);
    \end{scope}
\end{tikzpicture}
$$


And for every state $f:I\to X$, there is a pointed functor


$$(I:1\to\X, f\in \Y(I,X) ):(1,* \in 1)\to (\X, I\in\X_0)$$



Drawn as follows:

$$
\begin{tikzpicture}[scale=2]
\setlength\cupheight{1.5\cupheight}
\node (i) at (0,0) [Cup, top] {};
\node (j) at (0,-\cobheight) [Bot3D, invisible] {};
\node (g) [tiny label] at (0,-0.4\cobheight) {$f$};
\begin{scope}[internal string scope]
\node (di) at (i.center) [above] {\tiny $A$};
\draw (i.center) to (g.center) {};
\end{scope}
\end{tikzpicture}
$$

Inside of the cobordisms, draw the string diagrams for a strictification of the monoidal category.
Remeniscent of proof nets, this gives us a way to decompose the inputs of string diagrams into tensor factors.  
From now on, we will generally omit the points for the objects.
Consider the action of the associator and unitor on the points:



$$
\begin{tikzpicture}[scale=2]
    \node[Pants, bot, top] (B) at (0,0) {};
    \node[Pants, bot, anchor=belt] (A) at (B.leftleg) {};
    \node[SwishL, bot, anchor=top] (C) at (B.rightleg) {};
    \begin{scope}[internal string scope]
        \node (i) at (B.belt) {};
        \node (j) at (A.leftleg){};
        \node (k) at (A.rightleg) {};
        \node (l) at (C.bot) {} ;
        \node [tiny label] (f) at (B.center) {$f$};
        \node [tiny label] (g) at (A.center) {$g$};
        \node [tiny label] (h) at (C.center) {$h$};
        \draw (f.center) to (i.center);
        \draw (j.center) to [out=up, in=-135] (g.center);
        \draw (k.center) to [out=up, in=-35] (g.center);
        \draw (g.center) to [out=90, in=-135] node [left=-4pt] {} (f.center);
        \draw (l.center)
            to [out=90, in=-80] (h.center) 
            to [out=up, in=down, out looseness=0.7]
                (B.rightleg)
            to [out=up, in=-45]
                node [right=-4pt, pos=0.11] {}
 (f.center);
    \end{scope}
    \end{tikzpicture}
    \xRightarrow{\alpha}
\begin{tikzpicture}[scale=2]
    \node[Pants, bot, top] (B) at (0,0) {};
    \node[Pants, bot, anchor=belt] (A) at (B.rightleg) {};
    \node[SwishR, bot, anchor=top] (C) at (B.leftleg) {};
    \begin{scope}[internal string scope]
        \node (i) at (B.belt) {};
        \node (j) at (C.bot)  {};
        \node (k) at (A.leftleg) {};
        \node (l) at (A.rightleg)  {};
        \node [tiny label] (f) at (0.05\cobwidth,0.18\cobheight) {$f$};
        \node [tiny label] (g)  at (-0.4\cobwidth,-0.1\cobheight) {$g$};
        \draw (j.center)
            to [out=up, in=-130] (g.center);
        \draw (k.center)
            to [out=up, in=down] (B-rightleg.in-leftthird)
            to [out=up, in=-40] (g.center);
        \draw (l.center) to [out=up, in=down] (B-rightleg.in-rightthird)
            to [out=up, in=-60] (f.center);
        \draw (f.center) to [out=90, in=-90, looseness=2] (i.center);
        \draw (f.center) to [in=90, out=-120] (g.center);
        \node [tiny label] (h) at (0.8\cobwidth,-0.25\cobheight) {$h$};
    \end{scope}
    \end{tikzpicture}
$$
$$
\begin{tikzpicture}[scale=2]
    \node[Pants, top, bot] (A) at (0,0) {};
    \node[Cup] (B) at (A.leftleg) {};
    \node[Cyl, bot, anchor=top] (C) at (A.rightleg) {};
    \begin{scope}[internal string scope]
        \node [tiny label] (f) at (0,0) {$f$};
        \node [tiny label] (g) at ([yshift=-0.3\cobheight] B) {$g$};
        \node [tiny label] (h) at (C) {$h$};
        \node (i) at (A.belt) [above] {};
        \node (i2) at (C.bottom) [below] {};
        \draw (f.center) to [out=-140, in=90] node [left=-3pt] {} (g.center);
        \draw (f.center) to (i);
        \draw (f.center)
            to [out=-40, in=90, looseness=0.9]
                node [right=-2pt, pos=0.4] {} (h.center);
        \draw (h.center) to (i2);
    \end{scope}
\end{tikzpicture}
\xRightarrow{u^L}
\begin{tikzpicture}[scale=2]
    \node[Cyl, tall, bot, top] (A) at (0,0) {};
    \begin{scope}[internal string scope]
        \node [tiny label] (f) at (0,0.4\cobheight) {$f$};
        \node [tiny label] (g) at (-0.20\cobwidth, -0.1\cobheight) {$g$};
        \node [tiny label] (h) at (0.20\cobwidth,-0.5\cobheight) {$h$};
        \node (i) at (A.top) [above] {};
        \node (i2) at (A.bot) [below] {};
        \draw (f.center) to [out=-120, in=90] (g.center);
        \draw (f.center) to (i);
        \draw (f.center)
            to [out=-60, in=90, looseness=0.9] (h.center);
        \draw (h.center) to [out=-90, in=up, looseness=1.2] (i2.north);
    \end{scope}
\end{tikzpicture}\ ,
\hspace*{.5cm}
\begin{tikzpicture}[scale=2]
    \node[Pants, top, bot] (A) at (0,0) {};
    \node[Cup] (B) at (A.rightleg) {};
    \node[Cyl, bot, anchor=top] (C) at (A.leftleg) {};
    \begin{scope}[internal string scope]
        \node [tiny label] (f) at (0,0) {$f$};
        \node [tiny label] (g) at ([yshift=-0.3\cobheight] B) {$g$};
        \node [tiny label] (h) at (C) {$h$};
        \node (i) at (A.belt) [above] {};
        \node (i2) at (C.bottom) [below] {};
        \draw (f.center) to [out=-40, in=90] node [right=-1pt] {} (g.center);
        \draw (f.center) to (i);
        \draw (f.center)
            to [out=-140, in=90, looseness=0.9]
                %node [left=-2pt, pos=0.4] {$Y$}
                (h.center);
        \draw (h.center) to (i2);
    \end{scope}
\end{tikzpicture}
\xRightarrow{u^R}
\begin{tikzpicture}[scale=2]
    \node[Cyl, tall, bot, top] (A) at (0,0) {};
    \begin{scope}[internal string scope]
        \node [tiny label] (f) at (0,0.4\cobheight) {$f$};
        \node [tiny label] (g) at (0.2\cobwidth, -0.1\cobheight) {$g$};
        \node [tiny label] (h) at (-0.2\cobwidth,-0.5\cobheight) {$h$};
        \node (i) at (A.top) [above] {};
        \node (i2) at (A.bot) [below] {};
        \draw (f.center) to [out=-60, in=90] (g.center);
        \draw (f.center) to (i);
        \draw (f.center)
            to [out=-120, in=90, looseness=0.9] (h.center);
        \draw (h.center) to [out=-90, in=90, looseness=1.2] (i2);
    \end{scope}
\end{tikzpicture}
$$

This is unsatisfying, because of the one-sided nature of the way we are able to factorize the tensor product of maps.
Recall the category of internal profunctors described in Section \ref{???}, which is the 2-sided version of $\Cat$.  In this section it will be easier to work with profunctors {\em enriched} in $\Sets$, as opposed to profunctors internal to set.  This is more general as well, because it allows us to work with locally small categories, rather than merely small categories:


\begin{description}
\item[0-cells:] Categories
\item[1-cells:]  The morphisms are {\em profunctors} given by the following correspondence.
$$
\dfrac
{F:\X^\op\times\Y\to\Sets \quad \in\ \Cat}
{F:\X\proarrow \Y\quad \in\ \Prof}
$$

The composition of profunctors $P:\X\proarrow \Y$ and $Q:\Y\proarrow \Z$ is given by the coend:
$$
P;Q := \int^{Y \in \Y } P(-,Y) \times Q(Y,=):\X \proarrow \Z
$$
Where the coend of a functor $F:\X^\op\times\X\to \Sets$ is given by the coequalizer diagram (in $\Cat$):
$$
\coprod_{X_1,X_2 \in \X} {\X}(X_1,X_2) \times F(X_1,X_2) \rightrightarrows \prod_{X \in \X} F(X,X) \to \int^{X \in \X} F(X,X)
$$


The identity profunctor on $\X$ is given by the hom functor
$$\X(-,=): \X^\op\times\X\to\Sets$$


Intutitively, this is a categorification of trace of a matrix where the natural numbers are replaced with categories, and the commutative ring is replaced with $\Sets$.  Thus, the composition of profunctors categorifies matrix multiplication: the trace of the product of matrices.


\item[2-cells:]  2-cells between parallel profunctors $P,Q:\X\proarrow \Y$ are natural transformations between the underlying profunctors $P,Q:\X^\op\times \Y \to \Sets$.

\item[Compact closed structure:] The symmetric monoidal structure of $\Prof$ is given by extending the Cartesian structure in $\Cat$.   The units and counits of the compact closed structure are given by the hom functor.

\end{description}

There are two classes of profunctors which will be of interest to us:

\begin{definition}


A profunctor $\X\proarrow \Y$ is {\bf representable} when it is naturally isomorphic to the profunctor $F_*:=\Y(F-,=)$ for a functor $F: \X\to \Y$.


Dually, a profunctor $\Y\proarrow \X$ is {\bf corepresentable} when it is naturally isomorphic to the profunctor $F^*:=\Y(-,F=)$ for a functor $F: \X\to \mathbb{D}$.


\end{definition}

There are two embeddings of $\Cat$ into $\Prof$ that preserve the monoidal structure (formally, they are strong monoidal pseudofunctors):

\begin{definition}
The representable embedding $(\_)_*:\Cat^\co \to \Prof$ is the identity on objects,  covariant on 1-cells and contravariant on 2-cells.  It takes functors $F:\X\to \Y$ to profunctors  $F_*:\X\proarrow \Y$.


Dually, the corepresentable embedding $(\_)^*:\Cat^\op \to \Prof$  is the identity on objects, contravariant on 1-cells and covariant on 2-cells.It takes functors $F:\X\to \Y$ to profunctors  $F^*:\Y\proarrow \X$.

\end{definition}

This two embeddings interact nicely:

\begin{lemma}
Given any functor $F:\X\to \Y$, there is an adjunction $F_* \dashv F^*$ with unit $\eta^F$ and counit $\epsilon^F$.
\end{lemma}

For example, if we draw the two embeddings of the pseudomonoid as follows:
$$
\begin{tikzpicture}
	\begin{pgfonlayer}{nodelayer}
		\node [style=box] (0) at (1, 0) {$\otimes_*$};
		\node [style=none] (1) at (1, 1) {};
		\node [style=none] (2) at (0.5, -1) {};
		\node [style=none] (3) at (1.5, -1) {};
		\node [style=dot] (4) at (2.5, 0) {};
		\node [style=none] (5) at (2.5, 1) {};
		\node [style=none] (6) at (2, -1) {};
		\node [style=none] (7) at (3, -1) {};
		\node [style=none] (8) at (1.75, 0) {$=:$};
		\node [style=box] (9) at (4, 0) {$I_*$};
		\node [style=none] (10) at (4, 1) {};
		\node [style=dot] (11) at (5.5, 0) {};
		\node [style=none] (12) at (5.5, 1) {};
		\node [style=none] (13) at (4.75, 0) {$=:$};
		\node [style=none] (14) at (3.25, 0) {$,$};
		\node [style=box] (15) at (7.25, 0) {$\otimes^*$};
		\node [style=none] (16) at (7.25, -1) {};
		\node [style=none] (17) at (6.75, 1) {};
		\node [style=none] (18) at (7.75, 1) {};
		\node [style=dot] (19) at (8.75, 0) {};
		\node [style=none] (20) at (8.75, -1) {};
		\node [style=none] (21) at (8.25, 1) {};
		\node [style=none] (22) at (9.25, 1) {};
		\node [style=none] (23) at (8, 0) {$=:$};
		\node [style=box] (24) at (10.25, 0) {$I^*$};
		\node [style=none] (25) at (10.25, -1) {};
		\node [style=dot] (26) at (11.75, 0) {};
		\node [style=none] (27) at (11.75, -1) {};
		\node [style=none] (28) at (11, 0) {$=:$};
		\node [style=none] (29) at (9.5, 0) {$,$};
		\node [style=none] (30) at (6.25, 0) {$,$};
	\end{pgfonlayer}
	\begin{pgfonlayer}{edgelayer}
		\draw [in=-135, out=90] (2.center) to (0);
		\draw (0) to (1.center);
		\draw [in=-45, out=90] (3.center) to (0);
		\draw [in=-135, out=90] (6.center) to (4);
		\draw (4) to (5.center);
		\draw [in=-45, out=90] (7.center) to (4);
		\draw (9) to (10.center);
		\draw (11) to (12.center);
		\draw [in=135, out=-90] (17.center) to (15);
		\draw [in=90, out=-90] (15) to (16.center);
		\draw [in=45, out=-90] (18.center) to (15);
		\draw [in=135, out=-90] (21.center) to (19);
		\draw (19) to (20.center);
		\draw [in=45, out=-90] (22.center) to (19);
		\draw (24) to (25.center);
		\draw (26) to (27.center);
	\end{pgfonlayer}
\end{tikzpicture}
$$

Then we have the following 2-cells:


$$
\begin{tikzpicture}
	\begin{pgfonlayer}{nodelayer}
		\node [style=dot] (4) at (2.5, 0) {};
		\node [style=none] (5) at (2.5, 0.75) {};
		\node [style=dot] (6) at (2.5, -1) {};
		\node [style=none] (7) at (2.5, -1.75) {};
		\node [style=none] (9) at (3.75, 0.75) {};
		\node [style=none] (11) at (3.75, -1.75) {};
		\node [style=none] (12) at (3.25, -0.5) {$\xRightarrow{\eta^\otimes}$};
		\node [style=none] (13) at (4.25, -0.5) {,};
		\node [style=none] (14) at (5.25, 0.75) {};
		\node [style=none] (15) at (5.25, -1.75) {};
		\node [style=none] (16) at (6, 0.75) {};
		\node [style=none] (17) at (6, -1.75) {};
		\node [style=dot] (18) at (7.25, -1) {};
		\node [style=dot] (19) at (7.25, 0) {};
		\node [style=none] (20) at (6.75, 0.75) {};
		\node [style=none] (21) at (7.75, 0.75) {};
		\node [style=none] (22) at (6.75, -1.75) {};
		\node [style=none] (23) at (7.75, -1.75) {};
		\node [style=none] (24) at (6.5, -0.5) {$\xRightarrow{\epsilon^\otimes}$};
		\node [style=none] (25) at (8.75, -0.5) {,};
		\node [style=none] (26) at (11.25, 0.75) {};
		\node [style=none] (27) at (11.25, -1.75) {};
		\node [style=dot] (30) at (9.75, -1) {};
		\node [style=dot] (31) at (9.75, 0) {};
		\node [style=none] (36) at (10.5, -0.5) {$\xRightarrow{\epsilon^I}$};
		\node [style=none] (37) at (12, -0.5) {,};
		\node [style=dot] (40) at (14.5, -1) {};
		\node [style=dot] (41) at (14.5, 0) {};
		\node [style=none] (42) at (13.75, -0.5) {$\xRightarrow{\epsilon^I}$};
		\node [style=none] (43) at (12.5, 0) {};
		\node [style=none] (44) at (13.25, 0) {};
		\node [style=none] (45) at (13.25, -1) {};
		\node [style=none] (46) at (12.5, -1) {};
	\end{pgfonlayer}
	\begin{pgfonlayer}{edgelayer}
		\draw (4) to (5.center);
		\draw (6) to (7.center);
		\draw [bend right=45, looseness=1.25] (4) to (6);
		\draw [bend left=45, looseness=1.25] (4) to (6);
		\draw (11.center) to (9.center);
		\draw (15.center) to (14.center);
		\draw (17.center) to (16.center);
		\draw [in=-30, out=90] (23.center) to (18);
		\draw [in=90, out=-150] (18) to (22.center);
		\draw (18) to (19);
		\draw [in=-90, out=30] (19) to (21.center);
		\draw [in=-90, out=150] (19) to (20.center);
		\draw (27.center) to (26.center);
		\draw (30) to (31);
		\draw (40) to (41);
		\draw [dashed] (45.center) to (44.center);
		\draw [dashed] (44.center) to (43.center);
		\draw [dashed] (43.center) to (46.center);
		\draw [dashed] (46.center) to (45.center);
	\end{pgfonlayer}
\end{tikzpicture}
$$

Indeed for any monoidal category, there are lax-Frobeniusators:

$$
\phi^L:=
\left(
\begin{tikzpicture}
	\begin{pgfonlayer}{nodelayer}
		\node [style=dot,fill=white] (96) at (13.25, 1) {};
		\node [style=dot,fill=white] (97) at (14.25, 1.75) {};
		\node [style=none] (98) at (14.25, 2.5) {};
		\node [style=none] (99) at (13, 2.5) {};
		\node [style=none] (100) at (13.25, -0.5) {};
		\node [style=none] (101) at (14.5, -0.5) {};
		\node [style=none] (102) at (14.75, 0) {};
		\node [style=none] (103) at (14.75, 0.5) {};
		\node [style=none] (104) at (13, 0.5) {};
		\node [style=none] (105) at (13, 0) {};
		\node [style=dot,fill=white] (106) at (16.5, 1.5) {};
		\node [style=dot,fill=white] (107) at (17.5, 2.25) {};
		\node [style=none] (108) at (17.5, 2.75) {};
		\node [style=none] (109) at (16.25, 2.75) {};
		\node [style=none] (110) at (17, 0.75) {};
		\node [style=none] (111) at (17, 0.75) {};
		\node [style=none] (112) at (16.5, -1) {};
		\node [style=none] (113) at (17.5, -1) {};
		\node [style=dot,fill=white] (114) at (17, 0.75) {};
		\node [style=dot,fill=white] (115) at (17, -0.25) {};
		\node [style=none] (116) at (17.75, 0.25) {};
		\node [style=none] (117) at (17.75, 1.75) {};
		\node [style=none] (118) at (16, 1.75) {};
		\node [style=none] (119) at (16, 0.25) {};
		\node [style=dot,fill=white] (120) at (20.25, 1.5) {};
		\node [style=none] (122) at (20.25, 2.75) {};
		\node [style=none] (123) at (19, 2.75) {};
		\node [style=none] (124) at (19.75, 0.75) {};
		\node [style=none] (125) at (19.75, 0.75) {};
		\node [style=none] (126) at (19.25, -1) {};
		\node [style=none] (127) at (20.25, -1) {};
		\node [style=dot,fill=white] (128) at (19.75, 0.75) {};
		\node [style=dot,fill=white] (129) at (19.75, -0.25) {};
		\node [style=dot,fill=white] (134) at (20.25, 2.25) {};
		\node [style=none] (135) at (20.75, 1.25) {};
		\node [style=none] (136) at (20.75, 2.5) {};
		\node [style=none] (137) at (19.75, 2.5) {};
		\node [style=none] (138) at (19.75, 1.25) {};
		\node [style=none] (144) at (22, -0.5) {};
		\node [style=none] (145) at (23, -0.5) {};
		\node [style=dot,fill=white] (147) at (22.5, 0.25) {};
		\node [style=none] (148) at (22, 2) {};
		\node [style=none] (149) at (23, 2) {};
		\node [style=dot,fill=white] (150) at (22.5, 1.25) {};
		\node [style=none] (151) at (15.5, 1) {$\xRightarrow{\epsilon^\otimes}$};
		\node [style=none] (152) at (18.5, 1) {$\xRightarrow{\alpha_*^{-1}}$};
		\node [style=none] (153) at (21.5, 1) {$\xRightarrow{\eta^\otimes}$};
	\end{pgfonlayer}
	\begin{pgfonlayer}{edgelayer}
		\draw [in=-75, out=90] (101.center) to (97);
		\draw [in=15, out=-165, looseness=0.75] (97) to (96);
		\draw (96) to (100.center);
		\draw [in=270, out=105] (96) to (99.center);
		\draw (98.center) to (97);
		\draw [style=red, dashed] (102.center) to (103.center);
		\draw [style=red, dashed] (103.center) to (104.center);
		\draw [style=red, dashed] (104.center) to (105.center);
		\draw [style=red, dashed] (105.center) to (102.center);
		\draw [in=-75, out=45, looseness=0.75] (111.center) to (107);
		\draw [in=15, out=-165, looseness=0.75] (107) to (106);
		\draw [in=135, out=-90] (106) to (110.center);
		\draw [in=270, out=105] (106) to (109.center);
		\draw (108.center) to (107);
		\draw [in=90, out=-150] (115) to (112.center);
		\draw (115) to (114);
		\draw [in=90, out=-30] (115) to (113.center);
		\draw [style=red, dashed] (116.center) to (117.center);
		\draw [style=red, dashed] (117.center) to (118.center);
		\draw [style=red, dashed] (118.center) to (119.center);
		\draw [style=red, dashed] (119.center) to (116.center);
		\draw [in=45, out=-90] (120) to (124.center);
		\draw [in=90, out=-150] (129) to (126.center);
		\draw (129) to (128);
		\draw [in=90, out=-30] (129) to (127.center);
		\draw [bend left=45, looseness=1.25] (134) to (120);
		\draw [bend left=45, looseness=1.25] (120) to (134);
		\draw (122.center) to (134);
		\draw [in=150, out=-90, looseness=0.75] (123.center) to (128);
		\draw [style=red, dashed] (135.center) to (136.center);
		\draw [style=red, dashed] (136.center) to (137.center);
		\draw [style=red, dashed] (137.center) to (138.center);
		\draw [style=red, dashed] (138.center) to (135.center);
		\draw [in=90, out=-150] (147) to (144.center);
		\draw [in=90, out=-30] (147) to (145.center);
		\draw [in=-90, out=150] (150) to (148.center);
		\draw [in=-90, out=30] (150) to (149.center);
		\draw (147) to (150);
	\end{pgfonlayer}
\end{tikzpicture}
\right)
$$

$$
\phi^R:=
\left(
\begin{tikzpicture}
	\begin{pgfonlayer}{nodelayer}
		\node [style=dot, fill=white] (0) at (14.5, 1) {};
		\node [style=dot, fill=white] (1) at (13.5, 1.75) {};
		\node [style=none] (2) at (13.5, 2.5) {};
		\node [style=none] (3) at (14.75, 2.5) {};
		\node [style=none] (4) at (14.5, -0.5) {};
		\node [style=none] (5) at (13.25, -0.5) {};
		\node [style=none] (6) at (13, 0) {};
		\node [style=none] (7) at (13, 0.5) {};
		\node [style=none] (8) at (14.75, 0.5) {};
		\node [style=none] (9) at (14.75, 0) {};
		\node [style=dot, fill=white] (10) at (17.25, 1.5) {};
		\node [style=dot, fill=white] (11) at (16.25, 2.25) {};
		\node [style=none] (12) at (16.25, 2.75) {};
		\node [style=none] (13) at (17.5, 2.75) {};
		\node [style=none] (14) at (16.75, 0.75) {};
		\node [style=none] (15) at (16.75, 0.75) {};
		\node [style=none] (16) at (17.25, -1) {};
		\node [style=none] (17) at (16.25, -1) {};
		\node [style=dot, fill=white] (18) at (16.75, 0.75) {};
		\node [style=dot, fill=white] (19) at (16.75, -0.25) {};
		\node [style=none] (20) at (16, 0.25) {};
		\node [style=none] (21) at (16, 1.75) {};
		\node [style=none] (22) at (17.75, 1.75) {};
		\node [style=none] (23) at (17.75, 0.25) {};
		\node [style=dot, fill=white] (24) at (19.5, 1.5) {};
		\node [style=none] (25) at (19.5, 2.75) {};
		\node [style=none] (26) at (20.75, 2.75) {};
		\node [style=none] (27) at (20, 0.75) {};
		\node [style=none] (28) at (20, 0.75) {};
		\node [style=none] (29) at (20.5, -1) {};
		\node [style=none] (30) at (19.5, -1) {};
		\node [style=dot, fill=white] (31) at (20, 0.75) {};
		\node [style=dot, fill=white] (32) at (20, -0.25) {};
		\node [style=dot, fill=white] (33) at (19.5, 2.25) {};
		\node [style=none] (34) at (19, 1.25) {};
		\node [style=none] (35) at (19, 2.5) {};
		\node [style=none] (36) at (20, 2.5) {};
		\node [style=none] (37) at (20, 1.25) {};
		\node [style=none] (38) at (23, -0.5) {};
		\node [style=none] (39) at (22, -0.5) {};
		\node [style=dot, fill=white] (40) at (22.5, 0.25) {};
		\node [style=none] (41) at (23, 2) {};
		\node [style=none] (42) at (22, 2) {};
		\node [style=dot, fill=white] (43) at (22.5, 1.25) {};
		\node [style=none] (44) at (15.5, 1) {$\xRightarrow{\epsilon^\otimes}$};
		\node [style=none] (45) at (18.5, 1) {$\xRightarrow{\alpha_*}$};
		\node [style=none] (46) at (21.5, 1) {$\xRightarrow{\eta^\otimes}$};
	\end{pgfonlayer}
	\begin{pgfonlayer}{edgelayer}
		\draw [in=-105, out=90] (5.center) to (1);
		\draw [in=165, out=-15, looseness=0.75] (1) to (0);
		\draw (0) to (4.center);
		\draw [in=-90, out=75] (0) to (3.center);
		\draw (2.center) to (1);
		\draw [style=red, dashed] (6.center) to (7.center);
		\draw [style=red, dashed] (7.center) to (8.center);
		\draw [style=red, dashed] (8.center) to (9.center);
		\draw [style=red, dashed] (9.center) to (6.center);
		\draw [in=-105, out=135, looseness=0.75] (15.center) to (11);
		\draw [in=165, out=-15, looseness=0.75] (11) to (10);
		\draw [in=45, out=-90] (10) to (14.center);
		\draw [in=-90, out=75] (10) to (13.center);
		\draw (12.center) to (11);
		\draw [in=90, out=-30] (19) to (16.center);
		\draw (19) to (18);
		\draw [in=90, out=-150] (19) to (17.center);
		\draw [style=red, dashed] (20.center) to (21.center);
		\draw [style=red, dashed] (21.center) to (22.center);
		\draw [style=red, dashed] (22.center) to (23.center);
		\draw [style=red, dashed] (23.center) to (20.center);
		\draw [in=135, out=-90] (24) to (27.center);
		\draw [in=90, out=-30] (32) to (29.center);
		\draw (32) to (31);
		\draw [in=90, out=-150] (32) to (30.center);
		\draw [bend right=45, looseness=1.25] (33) to (24);
		\draw [bend right=45, looseness=1.25] (24) to (33);
		\draw (25.center) to (33);
		\draw [in=30, out=-90, looseness=0.75] (26.center) to (31);
		\draw [style=red, dashed] (34.center) to (35.center);
		\draw [style=red, dashed] (35.center) to (36.center);
		\draw [style=red, dashed] (36.center) to (37.center);
		\draw [style=red, dashed] (37.center) to (34.center);
		\draw [in=90, out=-30] (40) to (38.center);
		\draw [in=90, out=-150] (40) to (39.center);
		\draw [in=-90, out=30] (43) to (41.center);
		\draw [in=-90, out=150] (43) to (42.center);
		\draw (40) to (43);
	\end{pgfonlayer}
\end{tikzpicture}
\right)
$$

These Frobeniusators interact with the (co)associators and (co)unitors of the (co)omonoid to satisfy several coherences to form a {\bf lax-Frobenius algebra} algebra in $\Prof$.  The coherences for a strong pseudofrobenius algebra (where the frobeniusators are isomorphisms) are stated explicitly in \cite{???}.  The lax version is presumably the same situation.  Monoidal categories are not just any old Frobenius algebras in Prof, as the pseudomonoid and pseudocomonoid are adjoint to each other making them categorified $\dag$-frobenius algebras.  Moreover, the unit and counits of these adjunctions are coherent with respec to the lax-Frobenius structure.  


This leads to the following conjecture. 

\begin{definition}
Fix a monoidal category $\X$. Say that a connected diagram in prof composed of the generators of the corresponding pseudomonoid is in {\bf spider normal form} when is any of the four following types of diagrams


$$
\begin{tikzpicture}
	\begin{pgfonlayer}{nodelayer}
		\node [style=dot] (0) at (1.25, 3) {};
		\node [style=dot] (1) at (0.5, 4) {};
		\node [style=dot] (2) at (1.25, 2.25) {};
		\node [style=dot] (3) at (0.5, 1.25) {};
		\node [style=none] (4) at (1.5, 4) {};
		\node [style=none] (5) at (1.5, 1.25) {};
		\node [style=none] (6) at (0.25, 0.5) {};
		\node [style=none] (7) at (1.5, 4.75) {};
		\node [style=none] (8) at (1.5, 0.5) {};
		\node [style=none] (9) at (0.75, 4.75) {};
		\node [style=none] (10) at (0.25, 4.75) {};
		\node [style=none] (11) at (0.75, 0.5) {};
		\node [style=none] (12) at (1, 3.25) {};
		\node [style=none] (13) at (0.5, 3.75) {};
		\node [style=none] (14) at (0.5, 1.5) {};
		\node [style=none] (15) at (1, 2) {};
		\node [style=none] (16) at (0.75, 3.5) {$\ddots$};
		\node [style=none] (17) at (0.75, 1.75) {$\reflectbox{$\ddots$}$};
		\node [style=none] (18) at (1.2, 0.5) {$\cdots$};
		\node [style=none] (19) at (1.2, 4.75) {$\cdots$};
	\end{pgfonlayer}
	\begin{pgfonlayer}{edgelayer}
		\draw (7.center) to (4.center);
		\draw [in=105, out=-90] (10.center) to (1);
		\draw [in=60, out=-90, looseness=0.75] (4.center) to (0);
		\draw [in=-90, out=75] (1) to (9.center);
		\draw [in=300, out=90] (5.center) to (2);
		\draw [in=90, out=-120] (3) to (6.center);
		\draw [in=90, out=-60] (3) to (11.center);
		\draw (8.center) to (5.center);
		\draw (0) to (2);
		\draw (3) to (14.center);
		\draw (15.center) to (2);
		\draw (13.center) to (1);
		\draw (0) to (12.center);
	\end{pgfonlayer}
\end{tikzpicture}
=:
\begin{tikzpicture}
	\begin{pgfonlayer}{nodelayer}
		\node [style=none] (0) at (1.5, 1.75) {};
		\node [style=none] (1) at (2.75, 1.75) {};
		\node [style=none] (2) at (2, 1.75) {};
		\node [style=none] (3) at (2.45, 1.75) {$\cdots$};
		\node [style=none] (4) at (2.75, 3.25) {};
		\node [style=none] (5) at (2, 3.25) {};
		\node [style=none] (6) at (1.5, 3.25) {};
		\node [style=none] (7) at (2.45, 3.25) {$\cdots$};
		\node [style=dot] (8) at (2, 2.5) {};
	\end{pgfonlayer}
	\begin{pgfonlayer}{edgelayer}
		\draw [in=-90, out=45] (8) to (4.center);
		\draw (8) to (5.center);
		\draw [in=135, out=-90] (6.center) to (8);
		\draw [in=90, out=-150] (8) to (0.center);
		\draw (2.center) to (8);
		\draw [in=90, out=-30] (8) to (1.center);
	\end{pgfonlayer}
\end{tikzpicture}\ ,
\hspace*{.5cm}
\begin{tikzpicture}
	\begin{pgfonlayer}{nodelayer}
		\node [style=dot] (0) at (1.5, 3) {};
		\node [style=dot] (2) at (1.5, 2.25) {};
	\end{pgfonlayer}
	\begin{pgfonlayer}{edgelayer}
		\draw (0) to (2);
	\end{pgfonlayer}
\end{tikzpicture}
\ ,
\hspace*{.5cm}
\begin{tikzpicture}
	\begin{pgfonlayer}{nodelayer}
		\node [style=dot] (3) at (2.5, 3) {};
		\node [style=none] (4) at (2.5, 2.25) {};
	\end{pgfonlayer}
	\begin{pgfonlayer}{edgelayer}
		\draw (4.center) to (3);
	\end{pgfonlayer}
\end{tikzpicture}
\ ,
\hspace*{.5cm}
\begin{tikzpicture}
	\begin{pgfonlayer}{nodelayer}
		\node [style=dot] (5) at (3.5, 2.25) {};
		\node [style=none] (6) at (3.5, 3) {};
	\end{pgfonlayer}
	\begin{pgfonlayer}{edgelayer}
		\draw (6.center) to (5);
	\end{pgfonlayer}
\end{tikzpicture}
$$


Say that a not necessarily connected diagram is in {\bf stratified spider normal form} when it can be composed into a strictly progressive sequence of spiders:

$$
\begin{tikzpicture}
	\begin{pgfonlayer}{nodelayer}
		\node [style=none] (0) at (1.5, 1.75) {};
		\node [style=none] (1) at (2.75, 1.75) {};
		\node [style=none] (2) at (2, 1.75) {};
		\node [style=none] (3) at (2.45, 1.75) {$\cdots$};
		\node [style=none] (4) at (2.75, 3.25) {};
		\node [style=none] (5) at (2, 3.25) {};
		\node [style=none] (6) at (1.5, 3.25) {};
		\node [style=none] (7) at (2.45, 6.25) {$\cdots$};
		\node [style=dot] (8) at (2, 2.5) {};
		\node [style=none] (9) at (3.5, 3.25) {};
		\node [style=none] (10) at (4.75, 3.25) {};
		\node [style=none] (11) at (4, 3.25) {};
		\node [style=none] (12) at (4.45, 1.75) {$\cdots$};
		\node [style=none] (13) at (4.75, 4.75) {};
		\node [style=none] (14) at (4, 4.75) {};
		\node [style=none] (15) at (3.5, 4.75) {};
		\node [style=none] (16) at (4.45, 6.25) {$\cdots$};
		\node [style=dot] (17) at (4, 4) {};
		\node [style=none] (18) at (6, 4.75) {};
		\node [style=none] (19) at (7.25, 4.75) {};
		\node [style=none] (20) at (6.5, 4.75) {};
		\node [style=none] (21) at (6.95, 1.75) {$\cdots$};
		\node [style=none] (22) at (7.25, 6.25) {};
		\node [style=none] (23) at (6.5, 6.25) {};
		\node [style=none] (24) at (6, 6.25) {};
		\node [style=none] (25) at (6.95, 6.25) {$\cdots$};
		\node [style=dot] (26) at (6.5, 5.5) {};
		\node [style=none] (27) at (2, 6.25) {};
		\node [style=none] (28) at (1.5, 6.25) {};
		\node [style=none] (29) at (2.75, 6.25) {};
		\node [style=none] (30) at (4, 6.25) {};
		\node [style=none] (31) at (3.5, 6.25) {};
		\node [style=none] (32) at (4.75, 6.25) {};
		\node [style=none] (33) at (4, 1.75) {};
		\node [style=none] (34) at (3.5, 1.75) {};
		\node [style=none] (35) at (4.75, 1.75) {};
		\node [style=none] (36) at (6.5, 1.75) {};
		\node [style=none] (37) at (6, 1.75) {};
		\node [style=none] (38) at (7.25, 1.75) {};
		\node [style=none] (39) at (5.4, 4.45) {\reflectbox{$\ddots$}};
	\end{pgfonlayer}
	\begin{pgfonlayer}{edgelayer}
		\draw [in=-90, out=45] (8) to (4.center);
		\draw (8) to (5.center);
		\draw [in=135, out=-90] (6.center) to (8);
		\draw [in=90, out=-150] (8) to (0.center);
		\draw (2.center) to (8);
		\draw [in=90, out=-30] (8) to (1.center);
		\draw [in=-90, out=45] (17) to (13.center);
		\draw (17) to (14.center);
		\draw [in=135, out=-90] (15.center) to (17);
		\draw [in=90, out=-150] (17) to (9.center);
		\draw (11.center) to (17);
		\draw [in=90, out=-30] (17) to (10.center);
		\draw [in=-90, out=45] (26) to (22.center);
		\draw (26) to (23.center);
		\draw [in=135, out=-90] (24.center) to (26);
		\draw [in=90, out=-150] (26) to (18.center);
		\draw (20.center) to (26);
		\draw [in=90, out=-30] (26) to (19.center);
		\draw (28.center) to (6.center);
		\draw (5.center) to (27.center);
		\draw (4.center) to (29.center);
		\draw (15.center) to (31.center);
		\draw (30.center) to (14.center);
		\draw (13.center) to (32.center);
		\draw (34.center) to (9.center);
		\draw (11.center) to (33.center);
		\draw (35.center) to (10.center);
		\draw (18.center) to (37.center);
		\draw (36.center) to (20.center);
		\draw (19.center) to (38.center);
	\end{pgfonlayer}
\end{tikzpicture}
$$

\end{definition}

\begin{lemma}
Given a monoidal category $\X$, and a connected diagram in $\Prof$ generated by the $1$-cells of the pseudo-Frobenius structure induced by $\X$ one can always reduce the diagram to spider normal form by repeated application of

$$\phi^L,\phi^R,\eta^\otimes,\alpha_*,\alpha^*,(u^L)_*,(u^L)^*,(u^R)_*,(u^R)^*$$
Furthermore, given any {\em not-necessarily connected} diagram can be reduced to stratified spider normal form.
\end{lemma}


This lemma is an obvious corollary  of spider theorem for special Frobenius algebras; however, the following is not  so immediate:


\begin{conjecture}
Stratified spider normal form is strictly confluent so that any $2$-cells witnessing the reduction to the spider normal form are equal.
\end{conjecture}


 In some sense, this should be  restatement of the coherence theorem for monoidal categories; however, the author is unable to prove it.
The rest of this chapter relies on this categorification of the spider theorem being true.  


To see what is going on here, we can do the same thing we did with pointed categories, and look inside the profunctors:

\begin{definition}
The symmetric monoidal 2-category of {\bf pointed Profunctors}, $\Prof^*$ has:

\begin{description}
\item[0-cells:] Pointed categories:

$$(\X, X\in\X_0)$$

\item[1-cells:] A pointed functor  between pointed categories is a pair consisting of a profunctor between the underlying categories and a morphism that preserves the point:

$$(F:\X\proarrow\Y, f\in F(X,Y)):(\X, X\in\X_0)\to (\Y, Y\in\Y_0)$$

\item[2-cells:] Given two parallel pointed functors $(F, f),  (G, g):(\X, X)\to (\Y,Y)$,
a pointed 2-cell is 2-cell $\phi:F\Rightarrow G$ of profunctors that preserves the distinguished map, so that $\phi_X:g=f$.
\end{description}

\end{definition}


The graphical calculus for pointed profunctors is essentially the same as for pointed categories.  Except now, due to the two different yoneda embeddings, we can factorize the domain and codomain of maps.  For example, consider the action unit and counit for the tensor product:


$$
  \begin{tikzpicture}[baseline={([yshift=-.5ex]current bounding box.center)},scale=2]
    \node[Pants, top] (pants) {};
    \node[Copants, bot, lowercob, anchor=leftleg] (copants) at (pants.leftleg) {};
   % \node[Top3D] at (copants.rightleg) {};
    \node[Bot3D] at (pants.rightleg) {};
    \node[Bot3D] at (pants.leftleg) {};
   \begin{scope}[internal string scope]
     \node[sq tiny label] (f) at (pants.center) {$f$};
     \node[sq tiny label] (g) at (copants.center) {$g$};
     \draw (f.center) to (pants.belt);
     \draw[bend right] (f.center) to (pants.leftleg);
     \draw[bend left] (f.center) to (pants.rightleg);
     \draw (g.center) to (copants.belt);
     \draw[bend left] (g.center) to (copants.leftleg);
     \draw[bend right] (g.center) to (copants.rightleg);
   \end{scope}
  \end{tikzpicture}
\xRightarrow{\epsilon^\otimes}
  \begin{tikzpicture}[baseline={([yshift=-.5ex]current bounding box.center)},scale=2]
    \node[Cyl,xscale=1.2,top,anchor=bot] (tube) {};
    \node[Cyl,xscale=1.2,bot,anchor=top] (tube1) at (tube.bot) {};
    \begin{scope}[internal string scope]
     \node[sq tiny label] (f) at (tube.center) {$f$};
     \node[sq tiny label] (g) at (tube1.center) {$g$};
     \draw (tube1.bot) to (g.center);
     \draw (f.center) to (tube.top);
     \draw[bend left] (f.center) to (g.center);
     \draw[bend right] (f.center) to (g.center);
    \end{scope}
  \end{tikzpicture}\ ,
\hspace*{.5cm}
  \begin{tikzpicture}[baseline={([yshift=-.5ex]current bounding box.center)},scale=2]
    \node[Cyl,top,anchor=bot] (tube) {};
    \node[Cyl,bot,anchor=top] (tube1) at (tube.bot) {};
    \begin{scope}[internal string scope]
     \node[sq tiny label] (f) at (tube.bot) {$f$};
     \draw (f.center) to (tube1.bot);
     \draw (f.center) to (tube.top);
    \end{scope}
  \end{tikzpicture}\
  \begin{tikzpicture}[baseline={([yshift=-.5ex]current bounding box.center)},scale=2]
    \node[Cyl,top,anchor=bot] (tube) {};
    \node[Cyl,bot,anchor=top] (tube1) at (tube.bot) {};
    \begin{scope}[internal string scope]
     \node[sq tiny label] (g) at (tube.bot) {$g$};
     \draw (g.center) to (tube1.bot);
     \draw (g.center) to (tube.top);
    \end{scope}
  \end{tikzpicture}
\xRightarrow{\eta^\otimes}
  \begin{tikzpicture}[baseline={([yshift=-.5ex]current bounding box.center)},scale=2]
    \node[Pants,xscale=1.5,bot] (pants) {};
    \node[Copants,xscale=1.5, top, anchor=belt] (copants) at (pants.belt) {};
   % \node[Top3D] at (copants.rightleg) {};
   % \node[Bot3D] at (pants.rightleg) {};
    \node[Bot3D,xscale=1.5] at (pants.belt) {};
    \begin{scope}[internal string scope]
    \node[sq tiny label] (f) at ($(pants.belt)+(-0.22,.2)$) {$f$};
     \node[sq tiny label] (g) at ($(pants.belt)+(0.22,.2)$) {$g$};
     \draw[in=-90, out=90, looseness=1.3]  (f.center) to ($(copants.leftleg)+(0,0)$);
     \draw[in=-90, out=90, looseness=1.3]  (g.center) to ($(copants.rightleg)+(0,0)$);
     \draw[in=90, out=-90, looseness=1.3]  (f.center) to ($(pants.leftleg)+(0,0)$);
     \draw[in=90, out=-90, looseness=1.3]  (g.center) to ($(pants.rightleg)+(0,0)$);
    \end{scope}
  \end{tikzpicture}
$$
And similarly for the tensor unit:
$$
  \begin{tikzpicture}[baseline={([yshift=-.5ex]current bounding box.center)},scale=2]
    \node[Cup, top,scale=1.2] (cup) at (0,1.5) {};
    \node[Cap, bot,scale=1.2] (cap) at (0,0) {};
   % \node[Top3D] at (copants.rightleg) {};
    \begin{scope}[internal string scope]
    \node[sq tiny label] (f) at ($(cap.center)+(0,.2)$) {$g$};
    \node[sq tiny label] (g) at ($(cup.center)+(0,-.27)$)  {$f$};
     \draw (g.center) to (cup.center);
     \draw (f.center) to (cap.center);
    \end{scope}
  \end{tikzpicture}
\xRightarrow{\epsilon^I}
  \begin{tikzpicture}[baseline={([yshift=-.5ex]current bounding box.center)},scale=2]
    \node[Cyl,xscale=1.2,top,anchor=bot] (tube) {};
    \node[Cyl,xscale=1.2,bot,anchor=top] (tube1) at (tube.bot) {};
    \begin{scope}[internal string scope]
     \node[sq tiny label] (f) at (tube.center) {$f$};
     \node[sq tiny label] (g) at (tube1.center) {$g$};
     \draw (tube1.bot) to (g.center);
     \draw (f.center) to (tube.top);
    \end{scope}
  \end{tikzpicture}\ ,
\hspace*{.5cm}
  \begin{tikzpicture}[baseline={([yshift=-.5ex]current bounding box.center)},scale=2]
	\begin{pgfonlayer}{nodelayer}
		\node [style=none] (0) at (-1, 1) {};
		\node [style=none] (1) at (-1, 0) {};
		\node [style=none] (2) at (0, 0) {};
		\node [style=none] (3) at (0, 1) {};
	\end{pgfonlayer}
	\begin{pgfonlayer}{edgelayer}
		\draw[style=dashed] (2.center) to (3.center);
		\draw[style=dashed] (3.center) to (0.center);
		\draw[style=dashed] (0.center) to (1.center);
		\draw[style=dashed] (1.center) to (2.center);
	\end{pgfonlayer}
\end{tikzpicture}
\xRightarrow{\eta^I}
  \begin{tikzpicture}[baseline={([yshift=-.5ex]current bounding box.center)},scale=2]
    \node[Cup] (cup) {};
    \node[Cap, bot]  at (cup) {};
   % \node[Top3D] at (copants.rightleg) {};
    \begin{scope}[internal string scope]
    \end{scope}
  \end{tikzpicture}
$$



These diagrams are very closed to proof nets.  For example, by applying, $\phi^L$ followed by two applications of $\eta^\otimes$  to the diagram on the left, we obtain the pointed profunctor 
$$
(\X(-,=), \alpha_{X,Y,Z}):
(\X(-,=), (X\otimes Y)\otimes Z)\proarrow (\X(-,=),X\otimes(Y\otimes Z))
$$


$$
\begin{tikzpicture}[scale=2]
    \node[Pants, bot, top] (B) at (0,0) {};
    \node[Pants, bot, anchor=belt] (A) at (B.rightleg) {};
    \node[SwishR, bot, anchor=top] (C) at (B.leftleg) {};
    \node[Copants, bot, anchor=rightleg] (D) at (A.leftleg) {};
    \node[SwishR, bot, anchor=top] (E) at (A.rightleg) {};
    \node[Copants, bot, anchor=rightleg] (F) at (E.bot) {};
    \begin{scope}[internal string scope]
    \draw [in=270, out=90] ($(F.belt)+(-.125,0)$) to ($(D.belt)+(-.1,0)$) to (C.bot) to (B.leftleg) to ($(B.belt)+(-.125,0)$);
    \draw [in=270, out=90] (F.belt) to ($(D.belt)+(.1,0)$) to (A.leftleg) to ($(B.rightleg)+(-.125,0)$) to (B.belt);
    \draw [in=270, out=90] ($(F.belt)+(.125,0)$)  to (E.bot) to (A.rightleg) to ($(B.rightleg)+(.125,0)$) to ($(B.belt)+(.125,0)$);
    \node (i) at (B.belt) [above] {\tiny $X\otimes(Y\otimes Z)$};
    \node (i) at (F.belt) [below] {\tiny $(X\otimes Y)\otimes Z$};
    \end{scope}
\end{tikzpicture}
\Rightarrow
\begin{tikzpicture}[scale=2]
    \node[Pants, bot, top] (A) at (0,0) {};
    \node[Copants, bot, anchor=leftleg] (B) at (A.leftleg) {};
    \node[Pants, bot, anchor=belt] (C) at (B.belt) {};
    \node[Copants, bot, anchor=leftleg] (D) at (C.leftleg) {};
    \begin{scope}[internal string scope]
    \draw [in=270, out=90] ($(D.belt)+(-.125,0)$) to ($(C.leftleg)+(-.1,0)$) to ($(B.belt)+(-.125,0)$) to ($(B.leftleg)+(0,0)$) to  ($(A.belt)+(-.125,0)$);
    \draw [in=270, out=90] ($(D.belt)+(0,0)$) to ($(C.leftleg)+(.1,0)$) to ($(B.belt)+(0,0)$) to ($(B.rightleg)+(-.1,0)$) to  ($(A.belt)+(0,0)$);
    \draw [in=270, out=90] ($(D.belt)+(.125,0)$) to ($(C.rightleg)+(0,0)$) to ($(B.belt)+(.125,0)$) to ($(B.rightleg)+(.1,0)$) to  ($(A.belt)+(.125,0)$);
    \node (i) at (A.belt) [above] {\tiny $X\otimes(Y\otimes Z)$};
    \node (i) at (D.belt) [below] {\tiny $(X\otimes Y)\otimes Z$};
    \end{scope}
\end{tikzpicture}
\Rightarrow
\begin{tikzpicture}[scale=2]
    \node[Cyl, top] (A) at (0,0) {};
    \node[Cyl, anchor=top] (B) at (A.bot) {};
    \node[Cyl, anchor=top] (C) at (B.bot) {};
    \node[Cyl, bot, anchor=top] (D) at (C.bot) {};
    \begin{scope}[internal string scope]
    \draw ($(A.top)+(.125,0)$) to ($(D.bot)+(.125,0)$) ;
    \draw ($(A.top)+(0,0)$) to ($(D.bot)+(0,0)$);
    \draw ($(A.top)+(-.125,0)$) to ($(D.bot)+(-.125,0)$) ;
    \node (i) at (A.top) [above] {\tiny $X\otimes(Y\otimes Z)$};
    \node (i) at (D.bot) [below] {\tiny $(X\otimes Y)\otimes Z$};
    \end{scope}
\end{tikzpicture}
$$


We can do a similar thing for the left unitor:

$$
\begin{tikzpicture}[scale=2]
    \node[Copants, bot] (A) at (0,0) {};
    \node[Cyl, top, bot, anchor=bot] (B) at (A.rightleg) {};
    \node[Cap,bot] (C) at (A.leftleg) {};
    \begin{scope}[internal string scope]
    \draw [in=270, out=90] (A.belt) to (A.rightleg) to (B.top);
    \node (i) at (A.belt) [below] {\tiny $ I\otimes A$};
    \node (i) at (B.top) [above] {\tiny $ A$};
    \end{scope}
\end{tikzpicture}
\Rightarrow
\begin{tikzpicture}[scale=2]
    \node[Cyl, top] (A) at (0,0) {};
    \node[Cyl, bot, anchor=top] (B) at (A.bot) {};
    \begin{scope}[internal string scope]
    \draw [in=270, out=90] ($(A.top)+(0,-.1)$) to ($(B.bot)+(0,.1)$);
    \node (i) at (B.bot) [below] {\tiny $ I\otimes A$};
    \node (i) at (A.top) [above] {\tiny $ A$};
    \end{scope}
\end{tikzpicture}
$$

%
%$$
%\begin{tikzpicture}[scale=2]
%    \node[Pants, bot, top] (B) at (0,0) {};
%    \node[Pants, bot, anchor=belt] (A) at (B.leftleg) {};
%    \node[SwishL, bot, anchor=top] (C) at (B.rightleg) {};
%    \node[Copants, bot, anchor=leftleg] (D) at (A.rightleg) {};
%    \node[SwishL, bot, anchor=top] (E) at (A.leftleg) {};
%    \node[Copants, bot, anchor=leftleg] (F) at (E.bot) {};
%    \begin{scope}[internal string scope]
%    \end{scope}
%    \end{tikzpicture}
%$$





In the setting of proof nets, the tensor is inverse to the cotensor and the unit introduction is inverse to the unit removal.  However, in $\Prof^*$ these 2-cells are merely adjoint to each other.  In order to relate these two notions, we want to consider the 2-cells in $\Prof^*$ which normalize the shapes of diagrams, as the maps themselves.  We need the following definition to this end:

\begin{definition}
A {\bf displayed category} is an ordinary category $\D$ equipped with a lax normal functor $F:\D\to \Prof$.
That is to say, $F$ has the data of:

\begin{itemize}
\item A function $F:\D_0\to \Prof_0$ taking objects of $\D$ to categories.
\item For every pair of objects $X,Y \in \D_0$, a function $F_{X,Y}:\D(X,Y)\to \Prof(F(X),F(Y))$  such that $1_{F(X)}=F_{X,X}(1_X)$.
\item For every triple of objects $X,Y,Z \in \D_0$, a $2$-cell, with components at $f:X\to Y,gY\to Z$
$$F_{X,Y,Z}(f,g):F_{X,Y}(f);F_{Y,Z}(g) \Rightarrow F_{X,Z}(f;g)$$
\end{itemize}

Such that for any diagram $W\xrightarrow{f} X \xrightarrow{g} Y \xrightarrow{h} Z$ in $\X$ the following diagram commutes:


$$
\xymatrix{
(F_{W,X}(f);F_{X,Y}(g));F_{Y,Z} \ar[d]_{F_{W,X,Y}(f;g);1_{F_{Y,Z}(h)}} \ar[rr]^{(\alpha_;)_{F_{W,X}(f),F_{X,Y}(g),F_{Y,Z}(h)}}
  & & F_{W,X}(f);(F_{X,Y}(g);F_{Y,Z}(h)) \ar[d]^{1_{F_{W,X}(f)};F_{X,Y,Z}(g,h)}\\
F_{W,Y}(f;g);F_{Y,Z}(h) \ar[dr]_{F_{W,Y,Z}(f;g,h) \ \ }
  && F_{W,X}(f);F_{X,Z}(g;h) \ar[dl]^{\ \ F_{W,X,Z}(f,g;h)}\\
  & F(W,Z)(f;g;h)
}
$$

Where $(\alpha_;)$ is the associator for composition in $\Prof$.
\end{definition}

We can recast the spider theorem in this light:

\begin{definition}
Given any monoidal category $\X$, there is an indexed category

$F_\X:\esfa\to \Prof$ such that  
\begin{itemize}
\item $(F_\X)(n) = \X^n$
\item $(F_\X){n,m}$ takes diagrams in $\esfa$ to (stratified) spiders in $\Prof$ composed of the monoidal structure of $\X$; moreover, sending scalars to the identity on $1$.
\item The natural transformation $(F_\X)_{n,m,k}$ performs (stratified) spider fusion.
\end{itemize}


\end{definition}

This way of framing the categorified spider theorem lends itself to the following canonical construction, usually attributed to Benabou (this is a variation of the eponymous construction of Grothendieck for pseudofunctors from ordinary categories into $\Cat$):



\begin{theorem}{Benabou-Grothendieck construction}

Given a displayed category $F:\D\to \Prof$, the Benabou-Grothendieck category,  $\Pi F$ is given by the pullback:

$$
\xymatrix{
\Pi  F \ar[r]^{\pi_1} \ar[d]_{\pi_0} & \Prof^* \ar[d] \\
\D \ar[r]_F & \Prof
}
$$

Where $\Prof^* \to \Prof$ is the canonical projection.


Concretely $\Pi F$ has:


\begin{description}
\item[Objects:] Pairs $(X\in \D_0, X^\sharp \in (F(X))_0)$
\item[Maps:] The maps are  pointed profunctors:
$$(f, f^\sharp):(X,X^\sharp )\to (Y,Y^\sharp)$$
 where $f \in \D(X,Y)$ and $f^\sharp \in F_{X,Y}(f)(Y^\sharp,X^\sharp)$
\item[Identity:] $1_{(X,X^\sharp)} := (1_X, 1_{X^\sharp})$
\item[Composition:] Given a composable pair:
$$(X,X^\sharp)\xrightarrow{(f, f^\sharp)} (Y,Y^\sharp)\xrightarrow{(g, g^\sharp)} (Z,Z^\sharp)$$
The composite is defined as follows:
$$(f, f^\sharp);(g, g^\sharp):= (F_{X,Y,Z}(f,g))(f^\sharp, g^\sharp)$$
\end{description}


Moreover, the first projection map $\pi_0:\Pi F\to \D$  is a (strict) functor.


We can actually go the other direction.  Given some fixed $\D$, this extends to an equivalence of categories between the slice category $\Cat/\D$ and the lax normal functor category $[\D,\Prof]$.  We won't restate this equivalence of categories, because it is not directly useful for us, and it takes considerable effort to expose.
\end{theorem}

Now, let us crank the handle for our example and see what we get out:

\begin{lemma}
The indexed category $\Pi F_{\X}$ has a concrete presentation:

\begin{description}
\item[Objects:] Finite lists of objects in $\X$.
%\item[Maps:] Given two finite lists $X=[X_0,\ldots, X_{n-1}]$ and $Y=[Y_0,\ldots, Y_{m-1}]$ of objects in $\X$, a map from $X\to Y$ is a pair $(f,f^\sharp)$ where $f:n\to m$ is a map in $\sfa$ equipped with an element $f^\sharp \in  (F_{\X})_{n,m}( F_{\X})$.
%
%
%$f^\sharp$ can be described concretely by induction on the connected components on $n$. If $f$ is a single connected component, then $f^\sharp$ is a map with domains and codomains left-factorized as follows
%$$\otimes_{i=0}^{n-1} X_i \to \otimes_{i=0}^{m-1} Y_i$$
%
%Moreover, if $f$ is multiple connected components, then $f^\sharp$ is a finite list of maps whose domains and codomains factorized in this way.

\item[Maps:] Given two finite lists $X=[X_0,\ldots, X_{n-1}]$ and $Y=[Y_0,\ldots, Y_{m-1}]$ of objects in $\X$, a map from $X\to Y$ is a pointed profunctor

$$
(P,f): (\X^n,X)) \proarrow (\X^n,X)
$$

Generated by the (co)tensor and (co)unit of the monoidal structure of $\X$.


Modulo the equivalence relation that two parallel maps

$$ (P,f),(Q,g): (\X^n,X) \proarrow (\X^n,X)$$

are equivalent when $\nu(P,f) = \nu(Q,g)$.

\item[Identity:]  The identity on $(\X^n,X)$ is the identity in poitned profunctors

$$
1_{(\X^n,X)} = (\X(-,=)^n,1_X)
$$


%Given a list of objects $X=[X_0,\ldots, X_{n-1}]$  in $\X$, $$1_X:=([1_{X_0},\ldots, 1_{X_{n-1}}], \X(-,=)^n)$$
%vizualized as $n$ tubes occupied by identities:
%
%$$
%\begin{tikzpicture}[scale=2]
%    \node[Cyl, top,bot] (A) at (0,0) {};
%    \begin{scope}[internal string scope]
%    \draw [in=270, out=90] ($(A.top)+(0,-.01)$) to ($(A.bot)+(0,.01)$);
%    \node (i) at (A.bot) [below] {\tiny $X_0$};
%    \node (i) at (A.top) [above] {\tiny $ X_0$};
%    \end{scope}
%\end{tikzpicture}\
%\begin{tikzpicture}[scale=2]
%    \node[Cyl, top,bot] (A) at (0,0) {};
%    \begin{scope}[internal string scope]
%    \draw [in=270, out=90] ($(A.top)+(0,-.01)$) to ($(A.bot)+(0,.01)$);
%    \node (i) at (A.bot) [below] {\tiny $X_1$};
%    \node (i) at (A.top) [above] {\tiny $ X_1$};
%    \end{scope}
%\end{tikzpicture}\ \
%\cdots
%\begin{tikzpicture}[scale=2]
%    \node[Cyl, top,bot] (A) at (0,0) {};
%    \begin{scope}[internal string scope]
%    \draw [in=270, out=90] ($(A.top)+(0,-.01)$) to ($(A.bot)+(0,.01)$);
%    \node (i) at (A.bot) [below] {\tiny $X_{n-1}$};
%    \node (i) at (A.top) [above] {\tiny $ X_{n-1}$};
%    \end{scope}
%\end{tikzpicture}
%$$


\item[Composition:] The composition is the composition of pointed profunctors.

\end{description}

This is moreover a strict monoidal category.  The tensor product is given by the tensor product in $\Prof^*$.


\end{lemma}


In $\Pi F_\X$, the components of the unitors and associators that we drew before as 2-cells are now honest maps.  Moreover, the different ways of associating and introducing/removing units are now genuine inverses because their composite reduces to the identity.




However, this is still not proof nets.  We can only normalize connected components.  For example, the following equation does not hold because the profunctors in which the string diagrams are drawn are not connected:

$$
\begin{tikzpicture}[scale=2]
    \node[Copants, top,bot] (A) at (0,0) {};
    \node[Pants,bot, anchor=belt] (B) at (A.belt) {};
    \begin{scope}[internal string scope]
    \draw [in=270, out=90] (B.leftleg) to ($(B.belt)+(-.1,0)$) to (A.leftleg);
    \draw [in=270, out=90] (B.rightleg) to ($(B.belt)+(.1,0)$) to (A.rightleg);
    \end{scope}
\end{tikzpicture}
\neq\ \
\begin{tikzpicture}[scale=2]
    \node[Cyl, top] (A) at (0,0) {};
    \node[Cyl, bot, anchor=top] (B) at (A.bot) {};
    \node[Cyl, top] (AA) at (.75,0) {};
    \node[Cyl, bot, anchor=top] (BB) at (AA.bot) {};
    \begin{scope}[internal string scope]
    \draw [in=270, out=90] ($(A.top)+(0,-.1)$) to ($(B.bot)+(0,.1)$);
    \draw [in=270, out=90] ($(AA.top)+(0,-.1)$) to ($(BB.bot)+(0,.1)$);
    \end{scope}
\end{tikzpicture}
$$

However, if in context $\Gamma$ we knew that the profunctors were connected then we could apply this rewrite rule:


$$
\begin{tikzpicture}[scale=2]
    \node[Copants,bot] (A) at (0,0) {};
    \node[Pants, anchor=belt] (B) at (A.belt) {};
    \node (x1) at ($(A.rightleg)+(.5,0)$) {};
    \node (x2) at ($(A.rightleg)+(.5,.4)$) {};
    \node (x3) at ($(A.leftleg)+(-1.5,.4)$) {};
    \node (x4) at ($(B.leftleg)+(-1.5,-.4)$) {};
    \node (x5) at ($(B.rightleg)+(.5,-.4)$) {};
    \node (x6) at ($(B.rightleg)+(.5,0)$) {};
    \node (x7) at ($(B.leftleg)+(-.5,0)$) {};
    \node (x8) at ($(A.leftleg)+(-.5,0)$) {};
    \node[Cyl,top, anchor=bot] (C) at ($(x3.center)+(.75,0)$) {};
    \node[Cyl,top, anchor=bot] (D) at ($(x2.center)+(-.75,0)$) {};
    \node[Cyl,bot, anchor=top] (E) at ($(x4.center)+(.75,0)$) {};
    \node[Cyl,bot, anchor=top] (F) at ($(x5.center)+(-.75,0)$) {};
    \node(G) at($(x3.center)+(1.5,.5)$)  {$\cdots$};
    \node(H) at($(x4.center)+(1.5,-.5)$)  {$\cdots$};
    \node(K) at($(A.belt)+(-1.55,0)$)  {$\Gamma$};
    \begin{scope}[internal string scope]
    \draw [in=270, out=90] (B.leftleg) to ($(B.belt)+(-.1,0)$) to (A.leftleg);
    \draw [in=270, out=90] (B.rightleg) to ($(B.belt)+(.1,0)$) to (A.rightleg);
    \draw (C.bot) to (C.top);
    \draw (D.bot) to (D.top);
    \draw (E.bot) to (E.top);
    \draw (F.bot) to (F.top);
    \end{scope}
    \draw [color=black, fill=blue, fill opacity=.4] (x1.center) to  (x1.center) to [bend left=15]  (x2.center) to (x3.center) to  [bend left=-15] (x4.center) to (x5.center) to [bend left=15]  (x6.center) to (x7.center) to  [bend left=15] (x8.center) to cycle;
\end{tikzpicture}
=
\begin{tikzpicture}[scale=2]
    \node[Cyl] (A0) at (0,0) {};
    \node[Cyl,anchor=center] (A1) at ($(A.center)+(1,0)$) {};
    \node[Cyl,anchor=top] (B0) at (A0.bot) {};
    \node[Cyl,anchor=top] (B1) at (A1.bot) {};
    \node (x1) at ($(A1.top)+(.5,0)$) {};
    \node (x2) at ($(A1.top)+(.5,.4)$) {};
    \node (x3) at ($(A0.top)+(-1.5,.4)$) {};
    \node (x4) at ($(B0.bot)+(-1.5,-.4)$) {};
    \node (x5) at ($(B1.bot)+(.5,-.4)$) {};
    \node (x6) at ($(B1.bot)+(.5,0)$) {};
    \node (x7) at ($(B0.bot)+(-.5,0)$) {};
    \node (x8) at ($(A0.top)+(-.5,0)$) {};
    \node[Cyl,top, anchor=bot] (C) at ($(x3.center)+(.75,0)$) {};
    \node[Cyl,top, anchor=bot] (D) at ($(x2.center)+(-.75,0)$) {};
    \node[Cyl,bot, anchor=top] (E) at ($(x4.center)+(.75,0)$) {};
    \node[Cyl,bot, anchor=top] (F) at ($(x5.center)+(-.75,0)$) {};
    \node(G) at($(x3.center)+(1.5,.5)$)  {$\cdots$};
    \node(H) at($(x4.center)+(1.5,-.5)$)  {$\cdots$};
    \node(K) at($(A0.bot)+(-1.1,0)$)  {$\Gamma$};
    \begin{scope}[internal string scope]
    \draw (C.bot) to (C.top);
    \draw (D.bot) to (D.top);
    \draw (E.bot) to (E.top);
    \draw (F.bot) to (F.top);
    \draw (A0.top) to (B0.bot);
    \draw (A1.top) to (B1.bot);
    \end{scope}
    \draw [color=black, fill=blue, fill opacity=.4] (x1.center) to  (x1.center) to [bend left=15]  (x2.center) to (x3.center) to  [bend left=-15] (x4.center) to (x5.center) to [bend left=15]  (x6.center) to (x7.center) to  [bend left=15] (x8.center) to cycle;
\end{tikzpicture}
$$



To this end, we want to force all components to be connected. To do this, first remark that for any  finite list of objects $X=[X_0,\ldots, X_{n-1}]$ in $\X$ there is an idempotent $(s_X, s_X^\sharp)=e_X:X\to X$ in   $\Pi F_{\X}$ where $s_X$ is the fully connected spider from $\X^n\to\X^n$ and $s_X^\sharp$ is the tensor factorized identity $s_X^\sharp = \otimes_{i=0}^{n-1}1 _{X_i}$ regarded as an element of the spider.
For example 

$$
e_{[X_0,X_1,X_2]}
=
\begin{tikzpicture}[scale=2]
    \node[Copants,top,bot] (A) at (0,0) {};
    \node[Copants, bot,anchor=leftleg] (B) at (A.belt) {};
    \node[Pants, bot ,anchor=belt] (C) at (B.belt) {};
    \node[Pants, bot ,anchor=belt] (D) at (C.leftleg) {};
    \node[SwishR, bot,top, anchor=bot] (E) at (B.rightleg) {};
    \node[SwishL, bot, anchor=top] (F) at (C.rightleg) {};
    \begin{scope}[internal string scope]
    \draw [in=270, out=90] ($(D.leftleg)+(0,0)$) to ($(C.leftleg)+(-.1,0)$) to ($(B.belt)+(-.125,0)$) to ($(A.belt)+(-.1,0)$) to ($(A.leftleg)+(0,0)$);
    \draw [in=270, out=90] ($(D.rightleg)+(0,0)$) to ($(C.leftleg)+(.1,0)$) to ($(B.belt)+(0,0)$) to ($(A.belt)+(.1,0)$) to ($(A.rightleg)+(0,0)$);
    \draw [in=270, out=90] ($(F.bot)+(0,0)$) to ($(C.rightleg)+(0,0)$) to ($(B.belt)+(.125,0)$) to (E.bot) to (E.top);
    \node (i) at (A.leftleg) [above] {\tiny $ X_0$};
    \node (i) at (A.rightleg) [above] {\tiny $ X_1$};
    \node (i) at (E.top) [above] {\tiny $ X_2$};
    \node (i) at (D.leftleg) [below] {\tiny $ X_0$};
    \node (i) at (D.rightleg) [below] {\tiny $ X_1$};
    \node (i) at (F.bot) [below] {\tiny $ X_3$};
    \end{scope}
\end{tikzpicture}
$$




\begin{definition}
Take $N\X:=K_{\{e_X\ | \ X \in [\X_0]\}}(\Pi F_{\X})$; that is, the full subcategory of the  the Karoubi envelope of $\Pi F_{\X}$ with objects $(X,e_X)$. Concretely $N\X$ has;

\begin{description}
\item[Objects:] Lists of objects in $\X$.

\item[Maps:] 
$$\dfrac{(P,f);X\to Y \hspace*{.5cm} \in \Pi F_\X\ \text{ where $P$ is fully connected} }{(P,f);X\to Y \hspace*{.5cm} \in \N\X}$$

\item[Composition:] Same as in $\Pi F_\X$.

\item[Identity:] Same as in $\Pi F_\X$.

\item[Monoidal structure:] Given two $(P,f);W\to X$ and $(Q,g);Y\to Z$

$$
(P,f)\otimes(Q,g) := e_{W,Y};(P\times Q, (f,g)); e_{X,Z}
$$

\end{description}

\end{definition}

\begin{theorem}
$N\X$ is the strict monoidal category of proof nets in $\X$.
\end{theorem}
\begin{proof}

\end{proof}


%Connecting idempotents
%Subcategory of Karoubi envelope has another monoidal product







Obviously there is a lot more to be done here; a proof of the uniqueness of the stratified spider normal form being the most important.  However, this gives a highly conceptual approach to proof nets for monoidal categories. The (co)tensor and unit introduction/removal in some can be regarded as the "abstract shapes" of string diagrams in monoidal categories, coming from the monoidal structure of the original category.  Moreover, the idempotents can be regarded as the {\em property} of such an abstract shape being connected.

This is highly amenable to different notions of algebraic structure in $\Prof$.One could change the displayed category to pick out an extra special lax commutative Frobenius algebra in $\Prof$; corresponding to categorification of the spider theorem for special commutative Frobenius algebras to get a semantics for proof nets for monoidal closed categories.

Similarly, by asking that the frobeniusators be pseudo, one would get a semantics for proof nets for monoidal categories with duals; by asking for commutativity and pseudo-ness of the Frobeniusators, this would give a semantics for proof nets for compact closed categories.



These graphical calculi already exist, however, this approach to coherence begs the question of string diagrams for more exotic structures.  For example a bi-actegory is essentially a monoidal category with compatible left and right strengths that commute with the tensor product. However, regarded as extra structure on a lax Frobenius algebra in Prof, this is a categorification of phased spiders. A categorified phased-spider normal form would apparently give string diagrams for monoidal actegories (presumably which would be a special case of the proof nets for linear actegories found in \cite{}).  More generally, one could ask for the categorification of the various other normal forms we have discussed throughout this thesis... does the bialgebra law have some sort of categorified analogue?  From a quantum perspective, the question I pose is if the structures of the ZX-calculus can categorified to structures in Prof; and hopefuly they yield string diagrams with useful semantics.





