% !TEX TS-program = pdflatex
% !TEX encoding = UTF-8 Unicode

% This is a simple template for a LaTeX document using the "article" class.
% See "book", "report", "letter" for other types of document.
\errorcontextlines=500

\documentclass[12pt]{ociamthesis}  % default square logo 

%\documentclass[a4paper,11pt,oneside]{bo222ok}
%\usepackage[DIV=14,BCOR=2mm,headinclude=true,footinclude=false]{typearea}


\usepackage{hhline}
\usepackage{bookmark}
\usepackage[table]{xcolor}% http://ctan.org/pkg/xcolor
\usepackage[all,cmtip]{xy} 
\usepackage{float}

\usepackage{tikzit}
\input{thesis.tikzstyles}
\input{thesis.tikzdefs}

\usepackage{comment}


\usepackage{mdframed}
\usepackage{arydshln}
\usepackage{multicol}
\renewcommand{\tilde}{\widetilde}
\usepackage{everypage}
\usepackage{lipsum}
\usepackage{amsthm}
\usepackage[inline]{enumitem}   
\usepackage{scalerel,stackengine}
\stackMath
\renewcommand\hat[1]{%
\savestack{\tmpbox}{\stretchto{%
  \scaleto{%
    \scalerel*[\widthof{\ensuremath{#1}}]{\kern-.6pt\bigwedge\kern-.6pt}%
    {\rule[-\textheight/2]{1ex}{\textheight}}%WIDTH-LIMITED BIG WEDGE
  }{\textheight}% 
}{0.5ex}}%
\stackon[1pt]{#1}{\tmpbox}%
}
\parskip 1ex



\newcommand{\bcell}{\cellcolor{black!10}}

\makeatletter
% This command ignores the optional argument for itemize and enumerate lists
\newcommand{\inlineitem}[1][]{%
\ifnum\enit@type=\tw@
    {\descriptionlabel{#1}}
  \hspace{\labelsep}%
\else
  \ifnum\enit@type=\z@
       \refstepcounter{\@listctr}\fi
    \quad\@itemlabel\hspace{\labelsep}%
\fi}
\makeatother
\parindent=0pt
 

%\usepackage{extpfeil}
%\newextarrow{\xleftarrowtail}{500(40)}{\leftarrow\relbar<}
%\newextarrow{\xrightarrowtail}{500(40)}{>\relbar\rightarrow}



\makeatletter
\def\proarrowfill@#1#2#3#4#5{%
  $\m@th\thickmuskip0mu\medmuskip\thickmuskip\thinmuskip\thickmuskip
   \relax#5#1\mkern-7mu%
   \cleaders\hbox{$#5\mkern-2mu#2\mkern-2mu$}\hfill
   \mathclap{#3}\mathclap{#2}%
   \cleaders\hbox{$#5\mkern-2mu#2\mkern-2mu$}\hfill
   \mkern-7mu#4$%
}
\def\rightproarrowfill@{%
  \proarrowfill@\relbar\relbar\mapstochar\rightarrow}
\newcommand\xproarrow[2][]{%
  \ext@arrow 0055{\rightproarrowfill@}{#1}{#2}}
\makeatother

\newcommand{\proarrow}{\xproarrow{}}



%\newcommand\xrightarrowtail[2][]{\ensurestackMath{\mathrel{%
%  \stackengine{1pt}{%
 %   \stackengine{0pt}{\rightarrowtail}{\scriptstyle#2}{O}{c}{F}{F}{S}%
%  }{\scriptstyle#1}{U}{c}{F}{F}{S}%
%}}}



%\newcommand\xleftarrowtail[2][]{\ensurestackMath{\mathrel{%
%  \stackengine{1pt}{%
%    \stackengine{0pt}{\leftarrowtail}{\scriptstyle#2}{O}{c}{F}{F}{S}%
%  }{\scriptstyle#1}{U}{c}{F}{F}{S}%
%}}}


%\newcommand{\xrightarrowtail}[1]{\!\!\stackrel{#1}{\xymatrix@C=0.78em{\ar@{>->}[r]&}}\!\!\!}
%\newcommand{\xleftarrowtail}[1]{\!\!\!\stackrel{#1}{\xymatrix@C=0.78em{&\ar@{>->}[l]}}\!\!}



\newcommand{\alr}{{\sf alr}}
\newcommand{\lr}{{\sf lr}}
\newcommand{\rel}{{\sf r}}
\newcommand{\aih}{{\sf aih}}
\newcommand{\ih}{{\sf ih}}


\newcommand{\xrightarrowtail}[1]{\!\!{\xymatrix@C=1em{\ar@{>->}[r]^{#1}&}}\!\!\!}
\newcommand{\xleftarrowtail}[1]{\!\!\!{\xymatrix@C=1em{&\ar@{>->}[l]_{#1}}}\!\!}


\newcommand{\xrightarrowiso}[1]{\!\!{\xymatrix@C=1em{\ar@{->}[r]^{#1}_\cong&}}\!\!\!}
\newcommand{\xleftarrowiso}[1]{\!\!\!{\xymatrix@C=1em{&\ar@{->}[l]_{#1}^\cong}}\!\!}




%\theoremstyle{theorem} 
  \newtheorem{theorem}{Theorem}[section]
  \newtheorem{corollary}[theorem]{Corollary}
  \newtheorem{lemma}[theorem]{Lemma}
  \newtheorem{proposition}[theorem]{Proposition}
  
%\theoremstyle{definition}    conjecture
  \newtheorem{definition}[theorem]{Definition}
  \newtheorem{example}[theorem]{Example}
  \newtheorem{conjecture}[theorem]{Conjecture}
  \newtheorem{remark}[theorem]{Remark}
  
  
\newcommand{\Mat}{\mathsf{Mat}}


\newcommand{\dom}{{\sf dom}}
\newcommand{\cod}{{\sf cod}}

\newcommand{\cnot}{\mathsf{cnot}}
\newcommand{\tof}{\mathsf{tof}}
\newcommand{\Not}{\mathsf{not}}
\newcommand{\zeroin}{|0\rangle}
\newcommand{\zeroout}{\langle 0|}
\newcommand{\CNOT}{\mathsf{CNOT}}
\newcommand{\Sets}{\mathsf{Set}}
\newcommand{\FSets}{\mathsf{FinOrd}}
\newcommand{\FinOrd}{\mathsf{FinOrd}}
\newcommand{\Fin}{\mathsf{Fd}}
\newcommand{\TOF}{\mathsf{TOF}}
\newcommand{\Span}{\mathsf{Span}}
\newcommand{\dec}{\mathsf{dec}}
\newcommand{\Rel}{\mathsf{Rel}}
\newcommand{\op}{\mathsf{op}}
\newcommand{\co}{\mathsf{co}}
\newcommand{\Hilb}{\mathsf{Hilb}}
\newcommand{\FdHilb}{\mathsf{FHilb}}
\newcommand{\FHilb}{\mathsf{FHilb}}
\newcommand{\CPM}{\mathsf{CPM}}
\newcommand{\CP}{\mathsf{CP}}
\newcommand{\FPinj}{\mathsf{FPinj}}
\newcommand{\FPar}{\mathsf{FPar}}
\newcommand{\FSpan}{\mathsf{FSpan}}
\newcommand{\Pinj}{\mathsf{Pinj}}
\newcommand{\Par}{\mathsf{Par}}
\newcommand{\Aff}{\mathsf{Aff}}
\newcommand{\ParIso}{\mathsf{ParIso}}

\newcommand{\Total}{\mathsf{Total}}
\newcommand{\CFrob}{\mathsf{CFrob}}
\newcommand{\tr}{\mathsf{Tr}}
\newcommand{\ox}{\otimes}
\newcommand{\Csp}{{\sf Cospan}}
\newcommand{\Corel}{{\sf Corel}}
\newcommand{\Bool}{\mathbb{B}}
\newcommand{\Iso}{{\sf Iso}}
\renewcommand{\P}{{\sf p}}
\newcommand{\pmul}{{\sf pmul}}

\newcommand{\Prof}{{\sf Prof}}
\newcommand{\Mod}{{\sf Mod}}

\newcommand{\unit}{{\sf unit}}
\newcommand{\comm}{{\sf comm}}
\newcommand{\assoc}{{\sf assoc}}
\newcommand{\inj}{{\sf Inj}}
\newcommand{\surj}{{\sf Surj}}
\newcommand{\PSurj}{{\sf PSurj}}

\newcommand{\pre}{{\sf pre}}
\newcommand{\poly}{{\sf poly}}
\newcommand{\sub}{{\sf sub}}

\newcommand{\C}{\mathbb{C}}

\newcommand{\ch}{{\sf ch}}
\newcommand{\m}{{\sf m}}
\newcommand{\cm}{{\sf cm}}
\newcommand{\cb}{{\sf cb}}
\newcommand{\pcm}{{\sf pcm}}
\renewcommand{\r}{{\sf r}}
\newcommand{\scfrob}{{\sf scfrob}}

\newcommand{\bi}{{\sf b1}}
\newcommand{\bii}{{\sf b2}}
\newcommand{\biii}{{\sf b3}}
\newcommand{\biv}{{\sf b4}}

\newcommand{\Kl}{{\sf Kl}}
\newcommand{\Mon}{{\sf Mon}}

\newcommand{\ev}{{\sf ev}}

\renewcommand{\P}{{\sf p}}
\newcommand{\f}{\mathsf{f}}

\newcommand{\F}{\mathbb{F}}
\newcommand{\X}{\mathbb{X}}
\newcommand{\Y}{\mathbb{Y}}
\newcommand{\Z}{\mathbb{Z}}
\newcommand{\N}{\mathbb{N}}
\newcommand{\T}{\mathbb{T}}
\newcommand{\s}{\mathbb{S}}
\newcommand{\U}{\mathbb{U}}

\newcommand{\IH}{\mathbb{IH}}


\newcommand{\M}{\mathcal{M}}
\newcommand{\E}{\mathcal{E}}



\renewcommand{\sp}{\mathsf{sp}}
\newcommand{\pr}{\mathsf{p}}
\newcommand{\iso}{\mathsf{i}}







\newcounter{eq}

\makeatletter
\newcommand{\ltxlabel}{\ltx@label}
\makeatother

\newcommand{\eqzxa}[1]{%
\refstepcounter{eq}%
\ltxlabel{#1}%
\eqstack{#1}%
}




\newcommand{\eqstack}[1]{%
\stackrel{\scalebox{.6}{(\ref{#1})}}{=}%
}

\newcommand{\eq}[1]{\stackrel{\scalebox{.6}{#1}}{=}}

\newcommand{\defeq}[1]{\stackrel{\scalebox{.6}{#1}}{:=}}


\newcommand{\eref}{\eqstack}

\newcommand{\erefop}[1]{%
\stackrel{\scalebox{.6}{(\ref{#1})${}^\op$}}{=}%
}

\newcommand{\ZXA}{\mathsf{ZX}\textit{\&}}


\newcommand{\Vect}{\mathsf{Vect}}
\newcommand{\Lag}{\mathsf{Lag}}
\newcommand{\im}{\mathsf{im}}
\newcommand{\ZX}{\mathsf{ZX}}
\newcommand{\ZH}{\mathsf{ZH}}
\DeclareMathSymbol{\bot}{\mathord}{symbols}{"3F}


\newcommand{\pullbackcorner}[1][dl]{\save*!/#1-1pc/#1:(-1,1)@^{|-}\restore}

\renewcommand{\epsilon}{\varepsilon}
\renewcommand{\bar}[1]{\overline{#1}\hspace*{.01cm}}

\newcommand{\Stab}{{\sf Stab}}
\newcommand{\LinRel}{\sf LinRel}


\newcommand{\Isot}{{\sf Isot}}
\newcommand{\Co}{{\sf Co}}

\newcommand{\STOCH}{\mathsf{STOCH}}

%\renewcommand\floatpagefraction{.9}
%\renewcommand\topfraction{.9}
%\renewcommand\bottomfraction{.9}
%\renewcommand\textfraction{.1}   
%\Setcounter{totalnumber}{50}
%\Setcounter{topnumber}{50}
%\Setcounter{bottomnumber}{50}


\usepackage{amsmath}
\usepackage{pict2e}

\newcommand{\lbparen}{\{
}

\newcommand{\rbparen}{ \}
}










\newdir{|>}{-<5pt,0pt>{
\begin{tikzpicture}[scale=.7]
	\begin{pgfonlayer}{nodelayer}
		\node [style=none] (0) at (0, 0) {};
		\node [style=none] (1) at (1, 0) {};
		\node [style=none] (2) at (-1, -0.25) {};
	\end{pgfonlayer}
	\begin{pgfonlayer}{edgelayer}
		\draw (2.center) to (0.center);
		\draw (0.center) to (1.center);
	\end{pgfonlayer}
\end{tikzpicture}
}}
\newdir{|<}{-<5pt,0pt>{
\begin{tikzpicture}[scale=.9]
	\begin{pgfonlayer}{nodelayer}
		\node [style=none] (0) at (0, -0.25) {};
		\node [style=none] (1) at (-1, -0.25) {};
		\node [style=none] (2) at (1, 0) {};
	\end{pgfonlayer}
	\begin{pgfonlayer}{edgelayer}
		\draw (2.center) to (0.center);
		\draw (0.center) to (1.center);
	\end{pgfonlayer}
\end{tikzpicture}
}}


\newcommand{\skewpullbackcorner}[1][dl]{\save*!/#1-1.1pc/#1:(-.5,1)@^{|>}\restore}
\newcommand{\skewpushoutcorner}[1][dl]{\save*!/#1-1pc/#1:(-1,1)@^{|<}\restore}


\DeclareFontFamily{U}{mathx}{\hyphenchar\font45}
\DeclareFontShape{U}{mathx}{m}{n}{
      <5> <6> <7> <8> <9> <10>
      <10.95> <12> <14.4> <17.28> <20.74> <24.88>
      mathx10
      }{}
\DeclareSymbolFont{mathx}{U}{mathx}{m}{n}
\DeclareFontSubstitution{U}{mathx}{m}{n}
\DeclareMathAccent{\widecheck}{0}{mathx}{"71}
\DeclareMathAccent{\wideparen}{0}{mathx}{"75}

\def\cs#1{\texttt{\char`\\#1}}


\usepackage{amsmath}


\usepackage{hyperref}

\newcommand\numeq[2]%
  {\label{#2}\stackrel{\scriptscriptstyle(\mkern-1.5mu#1\mkern-1.5mu)}{=}}

\newcommand{\cubetopbl}{A}
\newcommand{\cubetopbr}{B}
\newcommand{\cubetopfl}{C}
\newcommand{\cubetopfr}{D}
\newcommand{\cubebotbl}{E}
\newcommand{\cubebotbr}{F}
\newcommand{\cubebotfl}{G}
\newcommand{\cubebotfr}{H}

\xymatrixrowsep{.5cm}
\xymatrixcolsep{.65cm}

\usepackage[utf8]{inputenc} % set input encoding (not needed with XeLaTeX)


\title{Relational Semantics for Quantum Protocols}
\author{Cole Comfort}
\college{New College}  %your college
\degree{Doctor of Philosophy in Computer Science} 
\degreedate{????? 2023}    

%\renewcommand{\submittedtext}{change the default text here if needed}
%\date{} % Activate to display a given date or no date (if empty),
         % otherwise the current date is printed 





%introduction
%  motivation behind relational interpretations
%basic category theory
%  Monoidal categories and compact closed categories
%    Definitions
%    Strictification
%      string diagrams for strict monoidal categories
%      Proof nets for non strict monoidal categories
%    Process theoretic interpretation
%      Space and time
%    Monoidal theories
%      Finite sets
%  Spans and relations
%     Spans and cospans of finite sets and frobenius algebras
%     Linear relations, affine relations and their monoidal theories
%     Cartesian restriction categories
%     Discrete inverse categories
%       Cartesian completion and Para construction
%     Cartesian bicategories
%Internal categories
%  Monads
%  Internal categories
%  Distributive laws
%    Distributive laws of monoidal categories
%  Bimodules of monads (internal profunctors)
%    Distributive laws of symmetric monoidal categories
%Categorical quantum mechanics
%  Finite dimensional hilbert spaces and their strictification to complex matrices
%  Dagger compact closed structure
%  Isometries, unitary evolution and quantum observables
%  CPM construction and splitting dagger idempotents as measurement
%  Z and X observables
%  Stabilizers as +1 eigenvectors
%  Joint stabilizers and the stabilizer formalism
%  ZX-calculus
%    Phase-free ZX calculus
%       Phase-free ZX-calculus and linear relations
%    Z and X fragments
%       Affine relations
%    Stabilizer ZX-calculus
%      Hint at Symplectic algebra
%  ZH-calculus
%    Hint at ZX& construction
%  Discard construction
%ZX&-calculus
%Graphical symplectic algebra
%Deconstructing things using distributive laws
%Conclusion
% Hint at combining ZX& and stabilizer ZX-calculus to get nonlinear processing

%TODO
%Standardize tensor product for linear relations (tensor, direct sum, addition on objects)
%Give better exposition of the symplectic formalism in the introduction
%  Talk about spekkens toy model more and the divergence between the stabilizer formalism in odd dimensions in the symplectic algebra chapter
%Give examples of error correcting codes and protocols with affine and non-affine classical processing
%Talk about ZX& calculus as being stabilized by nonlinear set of X operators.
%Talk about combining stabilizer codes with classical processing
%L2 function in weil representation of ZXA and stabilizers
%Change words white and grey to inline tikz circles

%give a nontechnical overview at the beginning of each section (for example, string diagrams from a graphical perspective) Put these resumes in special boxes

%Also put axioms in special boxes

\usepackage{showframe}


\setenumerate{align=left}

\begin{document}
\maketitle

\sloppy

\begin{abstract}
In this thesis we exhibit nondeterministic semantics for fragments of quantum circuits.  First, we show that the class of circuits generated by Toffoli gates as well as state prepration and postselection in the Z and X basis is characterized in terms of spans of finite sets. With this semantics in mind, we discuss the connection to partial and reversible computation. We also exhibit an isomorphism between the prop of affine coisotropic relations over odd prime fields and odd-prime-dimensional mixed stabilizer circuits.   We show how this is closely related to previously known semantics for idealized classes of electrical circuits.  
\end{abstract}


\tableofcontents


\chapter{Introduction}
\label{chap:intro}


The traditional paradigm for quantum computing decomposes a quantum computation into three separate stages. First, a quantum state is prepared in the lab; second the state is evolved by applying unitary operations; and finally, the quantum state is measured according to the Born rule.  This is a relatively rigid model, where the three stages are all modelled by different kinds of mathematical objects.  Moreover, the born rule entails that the measurements depend {\em probabilistically} on the choice state preparation and unitary evolution.

In this thesis, we model quantum computation in the more flexible quantum circuit model, where the state preparation, evolution and measurement all live on similar footing.  More precisely, departing from most treatments of quantum circuits, we regard all three stages of a quantum computation as certain subspaces, where subspaces are composed by tracing out the common elements in the intersection of the subspaces.  Two types of subspaces appear in this thesis, namely subsets of finite sets and affine subspaces of affine vector spaces over finite fields.  We associate the Hilbert space of square-integrable functions to these spaces; and the subspaces correspond to the {\em nondeterministic} evolution of the system.  That is to say, by regarding these classes of quantum circuits as subspaces, the inputs and outputs are associated to each other with respect to which output is {\em possible} from which input.

In Chapter \ref{chap:background}, we review the mathematical background which is needed to understand this thesis.  In Section \ref{sec:cat}, we give a brief review of category theory.  We first review the theory of monoidal categories and string diagrams which gives allows us to regard circuits as abstract mathematical objects.  Next we review categories of spans and relations, which gives our mathematical semantics for circuits with nondeterministic evolution.  Finally in Section \ref{sec:cqm}, we review some of the basic results in categorical quantum mechanics: which relates monoidal categories and string diagrams to quantum computing. This entails giving an overview of the language for quantum circuits known as the {\em ZX-calculus}; as well as reviewing the mathematical machinery needed to model mixed states and measurement within this framework.

In Chapter \ref{chap:zxa} we analyze the class of quantum circuits generated by the Toffoli gate as well as state preparation and post selection in the $Z$ and $X$ basis.  We give a complete presentation for this category and interpret it in terms of spans of finite sets.  We show that this class of circuits has very close to a nondeterministic semantics, except where outcomes can happen multiple times.  By imposing an additional equation, we depart from the interpretation into Hilbert spaces, and obtain a semantics with a proper {\em nondeterministic} semantics in terms of relations between finite sets. We then decompose this presentation into small fragments; composing them incrementally via distributive law and pushout.
We exhibit substructural features of these various decompositions and discuss how by allowing only some of the generators, we obtain semantics which are partial, partially invertible and so on; associating classes of generators to different semantic paradigms of computation.


Finally in Chapter \ref{chap:stab}  we analyze the structure of odd-prime dimensional stabilizer circuits.
We expose the relational interpretation of these circuits, by adding generators: at each point, obtaining a richer semantics.  We first recall that the phase-free fragment of the ZX-calculus modulo scalars, for prime qudit dimension $p$ is isomorphic to the prop of linear relations over $\F_p$: ie where the maps are linear subspaces over $\F_p$.  By doubling the phase-free picture using the CPM construction, we obtain a semantics for Weyl-free odd-prime dimensional qudit stabilizer circuits: Lagrangian relations over $\F_p$.  The Weyl operators are introduced to this picture by adding affine shifts to obtain a prop of affine Lagrangian relations.  Finally, to add quantum discarding, we show that one doesn't need to take the CPM construction again, but it suffices to add the discard {\em relation} to obtain the prop of affine coisotropic relations.  By splitting idempotents, we recover measurement; which has a nice relational interpretation.  Using this relational interpretation of mixed stabilizer circuits, we show how stabilizer error correction protocols can be implemented.


In Chapter \ref{chap:grothendieck}, we regard monoidal categories as certain categorified Frobenius algebras in profunctors; the 2-category profunctors being itself a categorification of relations.  We conjecture that there is a confluent normal form for these structures, categorifying the spider theorem for special Frobnenius algebras.  By regarding this spider theorem as a monoidal displayed category, we compute the Grothendieck Benabou construction, and split idempotents, to obtain a strict monoidal category which is very close to proof nets for monoidal categories. We discuss the relation to the scalable ZX-calculus as well as avenues for future work.

In Chapter \ref{chap:conclusion} we discuss future work and the limitations of this thesis.


%
%There is a conceptual advantage to modelling these fragments in terms of categories of spans or relations:  this perspective illuminates some of the most elegant symmetries of the ZX and ZH.
%
%For example, \dag-Frobenius algebras in $\FHilb$, which are used to express orthonormal bases are shown to arise from a colimit in such a fragment.  Moreover, the Euler decomposition of the Fourier transform is shown to arise from the hopf law.
%
%The CPM construction, which is traditionally used to model density matrices using monoidal categories, also surprisingly plays various roles in our presentations of these fragments of quantum circuits.  Not only does it allow one to mix, quantum circuits, a very closely related construction adds the plus state to partial, reversible boolean circuits.  Perhaps more surprisingly, the CPM construction also captures the notion of the Fourier transform, and Euler decomposition; when it is applied to phase-free circuits: ie, circuits which correspond to linear subspaces of $\F_p$-vector spaces.
%
%Finally, we conclude the thesis by regarding {\em quantum protocols themselves} as a generalized category of relations.  The machinery involved in this case is the 2-category of profunctors: which can be thought of as bimodules of categories.  This gives a structural account of the scalable ZX-calculus, which is used to parametrize families of circuits using string diagrams.  In particular, this is a limit over profunctors: an instance of the so called Benabou-Grothendieck construction.  Because of the generality at which this is proven, we suggest applications in other domains where string diagrams are used. 

\

{\bf \large  Declaration of authorship}

\

Chapter \ref{chap:background} is a literature review.  Chapter \ref{chap:zxa} contains work written solely by the author of this thesis with Sections \ref{sec:szx:intro}-\ref{sec:ZXA} of Chapter \ref{chap:zxa} being adapted from conference proceedings \cite{zxa}.  Chapter \ref{chap:stab} contains work jointly coauthored with my supervisor Aleks Kissinger, published in conference proceedings \cite{lagrel}.  The final part of Chapter  \ref{chap:stab} after and including Section \ref{sec:coisotrel} contains original work of my own. Chapter \ref{chap:grothendieck} contains work solely by the author of this thesis.



\chapter{Background}


\label{chap:background}
In order to understand this thesis, we will assume only basic knowledge of category theory and quantum computing.  I intend for most of this thesis to be understandable by an average theoretical computer science; except for some parts to be understood by category theorists.


Although, we formally state the various categorical constructions which are used throughout this thesis, in almost all cases, the accompanying string diagrams also help give intuition to the reader.  The exception to this rule is the somewhat more technical  material on internal category theory and distributive laws of monoidal theories reviewed in Subsection \ref{subsec:internal} and used in Section \ref{sec:dist}.  For this, we assume some basic understanding of 2-categories.  The most category theory heavy material is contained in Chapter \ref{grothendieck_chapter} and assumes understanding of monoidal 2-categories.  However, both of these parts stand on their own and are not needed to understand the rest of this thesis.

As far as quantum theory is involved, in Section \ref{sec:cqm} we will introduce everything from scratch using the traditional algebraic paradigm, followed by a translation into string diagrams.




\section{Category theory}





We assume that the reader has a basic understanding of monoidal bicategories.  For reference, refer to ?????JAMIE AND CHRIS HEUNEN


\subsection{Internal category theory}

%\begin{definition}
%\label{def:monad}
%%monad
%\end{definition}
%
%
%
%\begin{definition}
%\label{def:span}
%
%%2-category of spans, cospans
%\end{definition}
%
%
%\begin{definition}
%\label{def:rel}
%
%%2-category of relations, corelations
%\end{definition}

\begin{definition}
Given a finitely complete category $\X$, let $\Span(\X)$ denote the 2-category of spans in $\X$.  Let $\Span^\sim(\X)$ denote the 1-category of spans given by quotienting 1-cells by isomorphism.

Similarly, given a finitely cocomplete category $\X$, let $\Csp(\X)$ denote the 2-category of cospans in $\X$.  Let $\Csp^\sim(\X)$ denote the 1-category of cospans given by quotienting 1-cells by isomorphism.

Given a regular category $\X$ let $\Rel(\X)$ denote the category of relations in $\X$.
\end{definition}



\begin{definition}
\label{def:internalcat}

%Internal category
Given a category $\mathcal V$ with finite pullbacks $\mathcal V$, a {\it category internal to} $\mathcal V$ is a monad in $\Span(\mathcal V)$.
\end{definition}

Internal categories are indeed categories.  The collection of objects is given by the feet of the span, the set of morphisms by the apex, the domain and codomain by the left and right legs respectively.  The components of the unit of the comonad give the identity morphisms and the multiplication of the monad gives the composition.

\begin{lemma}
\label{lem:internalcat}

Monads internal to $\Set$ are in bijection with small categories.
\end{lemma}


It is not the case that monad maps correspond to functors between internal categories.  A canonical way to obtain such a notion requires the machinery of double categories, which is outside of the purview of this thesis.  However, (globular) 2-categories suffice to construct internal profunctors which is what we need the machinery for.




\begin{definition}
\label{def:monoid}
Let $\Mon$ denote the category with set-monoids as objects and monoid homorphisms as morphisms.
\end{definition}



\begin{lemma}
\label{def:internalmonoidalcat}

Monads internal to $\Mon$ are in bijection with small monoidal categories.
\end{lemma}

String diagrams are the canonical graphical calculi for {\em strict monoidal categories}. Objects are drawn as wires, morphisms are drawn as boxes; the tensor product is giving by connecting the wires together and the tensor product is given by monoidal pasting.  The coherence for strict monoidal categories is equivalent to planar isotopy of these diagrams. As a matter of convention, we will draw the order of composition from bottom to top and the tensor from left to right.


GIVE EXAMPLE

String diagrams for monoidal categories can be augmented to describe morphisms in non-strict monoidal categories by adding four connectives and equations:

\begin{definition}
Given a (non-strict) monoidal category the monoidal category of proof nets in $\X$ is generated by the string diagrams for $\X$ addition to the following generators for all objects $X,Y$

modulo the equations
\end{definition}


\begin{lemma}
There is a fully faithful monoidal functor from $\C$ to proof nets over $\C$ given by:

Draw action


give coherence rules
\end{lemma}

Although this has long been known, the idea of proof nets for monoidal categories has recently been rediscovered \cite{wilson}, where the coauthors exhibit proof nets as the residue of a novel algebraic proof of MacLane's coherence theorem for monoidal categories.In the ZX-calculus literature, proof nets for strict monoidal categories have also been rediscovered as the scalable ZX calculus.  In the scalable ZX-calculus, the nets for the units have been ommited, and they use the proof nets to index wires when specifying quantum protocols diagrammatically.

 we give a novel conceptual proof in Section \ref{??} which constructs proof nets from string diagrams in a canonical way.  This way of viewing things can be generalized to other settings that monoidal categories.


\begin{definition}
\label{def:pro}
A {\bf pro} is a strict monoidal category generated by one object under the tensor product, and a {\bf prop} is a  strict {\em symmetric} monoidal category generated by one object under the tensor product.  A {\bf multicoloured pro/prop} is a strict (symmetric) monoidal category generated by some specified class of object under the tensor product.
\end{definition}


\begin{definition}
\label{def:monoidaltheory}

A {\bf monoidal theory} is a pair $(\Sigma,E)$ of {\bf generators} $\Sigma$ and {\bf equations} $E$.
Each generator $f \in \Sigma$ has a chosen domain $\dom (f) \in \N$  and codomain $\cod (f) \in \N$, so that $f$ can be seen as a map from $\dom(f)$ to $\cod (f)$.

The free pro with signature $\Sigma$ has maps in $\Sigma^*$ obtained by inductively  tensoring all the generators and composing all appropriately typed generators in $\Sigma$,
The equations in $E$ are pairs of parallel maps in $\Sigma^*$.
Any monoidal theory $(\Sigma,E)$  generates a pro $\bar{(\Sigma,E)}$ given by the free pro with signature $\Sigma$ modulo the equations in $E$.

TODO GIVE EQUALIZER

A {\bf symmetric monoidal theory} is the symmetric version of a monoidal theory, which generates a prop.  Here the set $\Sigma^*$ is obtained by composing and tensoring maps with symmetries, and then quotienting by the axioms of a prop.

\end{definition}


\begin{lemma}
Given two (symmetric) monoidal theories $(\Sigma_1,E_1)$  and $(\Sigma_2,E_2)$  the coproduct of pro(p)s  $\bar{(\Sigma_1,E_1)}+\bar{(\Sigma_2,E_2)}$ is generated by the (symmetric) monoidal theory $(\Sigma_1+\Sigma_2,E_1+E_2)$.
\end{lemma}


\begin{lemma}
Given three  (symmetric) monoidal theories $(\Sigma_1,E_1)$, $(\Sigma_2,E_2)$ and $(\Sigma_3,E_3)$ where $\bar{(\Sigma_3,E_3)}$ is a sub-pro(p) of both $\bar{(\Sigma_1,E_1)}$ and $\bar{(\Sigma_2,E_2)}$.  The pushout of the diagram of pro(p)s
$$
\bar{(\Sigma_1,E_1)} \leftarrow \bar{(\Sigma_3,E_3)}\rightarrow \bar{(\Sigma_2,E_2)}
$$
is generated by the (symmetric) monoidal theory $(\Sigma_1^* +_{\Sigma_3} \Sigma_2^*, E_1 + E_2)$.
\end{lemma}


%Talk about Lawvere theories


\begin{definition}
\label{def:walkingcmonoid}
Let $\cm$ be the pro generated by the commutative free monoid.
\end{definition}


\begin{lemma}
\label{lem:setpres}
$\cm$ is monoidally equivalent to $\FSets$ as a monoidal category with respect to the coproduct.
\end{lemma}

\begin{definition}
\label{def:distmonad}
%distributive law of monads, induced monad
\end{definition}


\begin{lemma}
\label{lem:distmonoidaltheory}


Suppose there three  (symmetric) monoidal theories $(\Sigma_1,E_1)$, $(\Sigma_2,E_2)$ and $(\Sigma_3,E_3)$ where $\bar{(\Sigma_3,E_3)}$ is a sub-pro(p) of both $\bar{(\Sigma_1,E_1)}$ and $\bar{(\Sigma_2,E_2)}$. A {\bf distributive law of pro(p)s} is a distributive law $\lambda:\bar{(\Sigma_2,E_2)} \otimes_{\bar{(\Sigma_3,E_3)}} \bar{ (\Sigma_1,E_1)}$   in $\Mon$-$\Prof$.  Informally, this is a way to push all the maps in $\Sigma_1^*$ past those of  $\Sigma_2^*$ modulo $\Sigma_3$ and the equations $E_1+E_2$ and the axioms of a pro(p).
\end{lemma}



In \cite{lack} it is required that $\bar{(\Sigma_3,E_3)}$ is a groupoid.

%Ex distributive law of lawvere theories.
%  Ex a(b+c) = ab+ac




\begin{definition}
\label{def:fa}
%Pro for frobenius algebras
\end{definition}


\begin{definition}
\label{def:ba}
%Pro for bialgebras
\end{definition}



\begin{lemma}
\label{lemma:spanpres}
%Span and cospan of finset are bialgebras and frobenius algebras
\end{lemma}



\begin{definition}
\label{def:bimod}
%2-category of bimodules
\end{definition}



\begin{definition}
\label{def:internalprof}
Given a category $\mathcal V$ with finite pullbacks, the 2-category of profunctors internal to $\mathcal V$, $\mathcal{V}-\Prof$ is $\Mod(\Mnd(\Span(\mathcal{V})))$.
\end{definition}

Bimodules allow us to consider distributive laws between two monads with shared structure, identified by the module actions.  For example, a distributive law of monoidal categories should identify the action of permuting wires on both categories. 



\begin{definition}
Consider the following two distributive laws: 
\begin{align*}
\cm^\op  \otimes_\P \cm;&
  \begin{tikzpicture}
	\begin{pgfonlayer}{nodelayer}
		\node [style=X] (0) at (-3.75, -1) {};
		\node [style=none] (1) at (-4, -1.75) {};
		\node [style=none] (2) at (-3.5, -1.75) {};
		\node [style=Z] (3) at (-3.75, -0.25) {};
		\node [style=none] (4) at (-4, 0.5) {};
		\node [style=none] (5) at (-3.5, 0.5) {};
	\end{pgfonlayer}
	\begin{pgfonlayer}{edgelayer}
		\draw [in=90, out=-60, looseness=1.00] (0) to (2.center);
		\draw [in=-120, out=90, looseness=1.00] (1.center) to (0);
		\draw (0) to (3);
		\draw [in=60, out=-90, looseness=1.00] (5.center) to (3);
		\draw [in=-90, out=120, looseness=1.00] (3) to (4.center);
	\end{pgfonlayer}
  \end{tikzpicture}
  \eqzxa{bi.one}
  \begin{tikzpicture}
	\begin{pgfonlayer}{nodelayer}
		\node [style=X] (0) at (-4, 0.5) {};
		\node [style=Z] (1) at (-4, -0.25) {};
		\node [style=X] (2) at (-4.5, 0.5) {};
		\node [style=Z] (3) at (-4.5, -0.25) {};
		\node [style=none] (4) at (-4, -1) {};
		\node [style=none] (5) at (-4.5, -1) {};
		\node [style=none] (6) at (-4.5, 1.25) {};
		\node [style=none] (7) at (-4, 1.25) {};
	\end{pgfonlayer}
	\begin{pgfonlayer}{edgelayer}
		\draw [bend left, looseness=1.25] (0) to (1);
		\draw [bend right, looseness=1.25] (2) to (3);
		\draw (1) to (2);
		\draw (3) to (0);
		\draw (0) to (7.center);
		\draw (6.center) to (2);
		\draw (3) to (5.center);
		\draw (4.center) to (1);
	\end{pgfonlayer}
\end{tikzpicture},
\hspace*{.5cm}
  \begin{tikzpicture}
	\begin{pgfonlayer}{nodelayer}
		\node [style=Z] (0) at (-4, -0) {};
		\node [style=X] (1) at (-4, -0.75) {};
		\node [style=none] (2) at (-4.25, -1.5) {};
		\node [style=none] (3) at (-3.75, -1.5) {};
	\end{pgfonlayer}
	\begin{pgfonlayer}{edgelayer}
		\draw [in=-60, out=90, looseness=1.00] (3.center) to (1);
		\draw (1) to (0);
		\draw [in=90, out=-120, looseness=1.00] (1) to (2.center);
	\end{pgfonlayer}
  \end{tikzpicture}
  \eqzxa{bi.two}
  \begin{tikzpicture}
	\begin{pgfonlayer}{nodelayer}
		\node [style=Z] (0) at (-4.25, -0.75) {};
		\node [style=none] (1) at (-4.25, -1.5) {};
		\node [style=none] (2) at (-3.5, -1.5) {};
		\node [style=Z] (3) at (-3.5, -0.75) {};
	\end{pgfonlayer}
	\begin{pgfonlayer}{edgelayer}
		\draw (2.center) to (3);
		\draw (0) to (1.center);
	\end{pgfonlayer}
  \end{tikzpicture},
  \hspace*{.5cm}
   \begin{tikzpicture}[yscale=-1]
	\begin{pgfonlayer}{nodelayer}
		\node [style=X] (0) at (-4, -0) {};
		\node [style=Z] (1) at (-4, -0.75) {};
		\node [style=none] (2) at (-4.25, -1.5) {};
		\node [style=none] (3) at (-3.75, -1.5) {};
	\end{pgfonlayer}
	\begin{pgfonlayer}{edgelayer}
		\draw [in=-60, out=90, looseness=1.00] (3.center) to (1);
		\draw (1) to (0);
		\draw [in=90, out=-120, looseness=1.00] (1) to (2.center);
	\end{pgfonlayer}
  \end{tikzpicture}
  \erefop{bi.two}
   \begin{tikzpicture}[yscale=-1]
	\begin{pgfonlayer}{nodelayer}
		\node [style=X] (0) at (-4.25, -0.75) {};
		\node [style=none] (1) at (-4.25, -1.5) {};
		\node [style=none] (2) at (-3.5, -1.5) {};
		\node [style=X] (3) at (-3.5, -0.75) {};
	\end{pgfonlayer}
	\begin{pgfonlayer}{edgelayer}
		\draw (2.center) to (3);
		\draw (0) to (1.center);
	\end{pgfonlayer}
  \end{tikzpicture},
\hspace*{.5cm}
  \begin{tikzpicture}[rotate=90]
	\begin{pgfonlayer}{nodelayer}
		\node [style=Z] (0) at (-8.25, -0) {};
		\node [style=X] (1) at (-9.25, -0) {};
	\end{pgfonlayer}
	\begin{pgfonlayer}{edgelayer}
		\draw (0) to (1);
	\end{pgfonlayer}
\end{tikzpicture}
  \eqzxa{extra}
\\
 \cm \otimes_\P \cm^\op;&
    \begin{tikzpicture}[rotate=90]
	\begin{pgfonlayer}{nodelayer}
		\node [style=X] (0) at (-6.25, 0.25) {};
		\node [style=none] (1) at (-7, 0.25) {};
		\node [style=none] (2) at (-4.75, 0.25) {};
		\node [style=X] (3) at (-5.5, 0.25) {};
	\end{pgfonlayer}
	\begin{pgfonlayer}{edgelayer}
		\draw (0) to (1.center);
		\draw (3) to (2.center);
		\draw [bend right, looseness=1.25] (3) to (0);
		\draw [bend right, looseness=1.25] (0) to (3);
	\end{pgfonlayer}
  \end{tikzpicture}
  \eqzxa{special}
  \begin{tikzpicture}[rotate=90]
	\begin{pgfonlayer}{nodelayer}
		\node [style=none] (0) at (-7, 0.25) {};
		\node [style=none] (1) at (-6, 0.25) {};
	\end{pgfonlayer}
	\begin{pgfonlayer}{edgelayer}
		\draw (1.center) to (0.center);
	\end{pgfonlayer}
  \end{tikzpicture},
  \hspace*{.5cm}
  \begin{tikzpicture}[rotate=90]
	\begin{pgfonlayer}{nodelayer}
		\node [style=X] (0) at (-7, -0) {};
		\node [style=X] (1) at (-6.25, 0.5) {};
		\node [style=none] (2) at (-7, 0.75) {};
		\node [style=none] (3) at (-7.75, 0.75) {};
		\node [style=none] (4) at (-7.75, -0) {};
		\node [style=none] (5) at (-6.25, -0.25) {};
		\node [style=none] (6) at (-5.5, -0.25) {};
		\node [style=none] (7) at (-5.5, 0.5) {};
	\end{pgfonlayer}
	\begin{pgfonlayer}{edgelayer}
		\draw (6.center) to (5.center);
		\draw [in=-30, out=180, looseness=1.00] (5.center) to (0);
		\draw (1) to (0);
		\draw [in=0, out=150, looseness=1.00] (1) to (2.center);
		\draw (2.center) to (3.center);
		\draw (0) to (4.center);
		\draw (1) to (7.center);
	\end{pgfonlayer}
  \end{tikzpicture}
 =
  \begin{tikzpicture}[rotate=90,xscale=-1]
	\begin{pgfonlayer}{nodelayer}
		\node [style=X] (0) at (-7, -0) {};
		\node [style=X] (1) at (-6.25, 0.5) {};
		\node [style=none] (2) at (-7, 0.75) {};
		\node [style=none] (3) at (-7.75, 0.75) {};
		\node [style=none] (4) at (-7.75, -0) {};
		\node [style=none] (5) at (-6.25, -0.25) {};
		\node [style=none] (6) at (-5.5, -0.25) {};
		\node [style=none] (7) at (-5.5, 0.5) {};
	\end{pgfonlayer}
	\begin{pgfonlayer}{edgelayer}
		\draw (6.center) to (5.center);
		\draw [in=-30, out=180, looseness=1.00] (5.center) to (0);
		\draw (1) to (0);
		\draw [in=0, out=150, looseness=1.00] (1) to (2.center);
		\draw (2.center) to (3.center);
		\draw (0) to (4.center);
		\draw (1) to (7.center);
	\end{pgfonlayer}
  \end{tikzpicture}
  \eqzxa{frob}
  \begin{tikzpicture}[rotate=90]
	\begin{pgfonlayer}{nodelayer}
		\node [style=none] (0) at (-4.75, -0.25) {};
		\node [style=X] (1) at (-5.5, -0) {};
		\node [style=none] (2) at (-7, -0.25) {};
		\node [style=X] (3) at (-6.25, 0) {};
		\node [style=none] (4) at (-4.75, 0.25) {};
		\node [style=none] (5) at (-7, 0.25) {};
	\end{pgfonlayer}
	\begin{pgfonlayer}{edgelayer}
		\draw [in=-30, out=180, looseness=1.25] (0.center) to (1);
		\draw (3) to (1);
		\draw [in=180, out=30, looseness=1.25] (1) to (4.center);
		\draw [in=0, out=-150, looseness=1.25] (3) to (2.center);
		\draw [in=0, out=150, looseness=1.25] (3) to (5.center);
	\end{pgfonlayer}
\end{tikzpicture}
  \end{align*}

The former yields, {\sf cb}, the prop for the free {\bf bicommutative bialgebra} and the latter yields, {\sf scfa}, the prop for the free {\bf special commutative Frobenius algebra}.

\end{definition}


\begin{lemma} \cite[\S 5.3, 5.4]{lack}
{\sf cb} is a presentation for $(\Span(\FSets),+)$ and {\sf scfa} is a presentation for $(\Csp(\FSets),+)$.

\end{lemma}


\begin{lemma}
{\sf cb} is a presentation for $\Mat_\N$ under the direct sum.
\end{lemma}

Give examples.

\begin{definition}
A {\bf Hopf algebra} is a bialgebra with an antipode map $s:1\to1$ satisfying the following equations:

$$
\begin{tikzpicture}
	\begin{pgfonlayer}{nodelayer}
		\node [style=none] (0) at (2.5, 5) {};
		\node [style=X] (1) at (2.5, 4.25) {};
		\node [style=Z] (2) at (2.5, 2.75) {};
		\node [style=none] (3) at (2.5, 2) {};
		\node [style=map] (4) at (2, 3.5) {$s$};
	\end{pgfonlayer}
	\begin{pgfonlayer}{edgelayer}
		\draw [in=150, out=-90] (4) to (2);
		\draw [bend right=60, looseness=1.25] (2) to (1);
		\draw [in=90, out=-150] (1) to (4);
		\draw (2) to (3.center);
		\draw (1) to (0.center);
	\end{pgfonlayer}
\end{tikzpicture}
=
\begin{tikzpicture}
	\begin{pgfonlayer}{nodelayer}
		\node [style=none] (0) at (2.5, 2) {};
		\node [style=Z] (1) at (2.5, 2.75) {};
		\node [style=X] (2) at (2.5, 4.25) {};
		\node [style=none] (3) at (2.5, 5) {};
	\end{pgfonlayer}
	\begin{pgfonlayer}{edgelayer}
		\draw (2) to (3.center);
		\draw (1) to (0.center);
	\end{pgfonlayer}
\end{tikzpicture}
=
\begin{tikzpicture}
	\begin{pgfonlayer}{nodelayer}
		\node [style=none] (0) at (2, 5) {};
		\node [style=X] (1) at (2, 4.25) {};
		\node [style=Z] (2) at (2, 2.75) {};
		\node [style=none] (3) at (2, 2) {};
		\node [style=map] (4) at (2.5, 3.5) {$s$};
	\end{pgfonlayer}
	\begin{pgfonlayer}{edgelayer}
		\draw [in=30, out=-90] (4) to (2);
		\draw [bend left=60, looseness=1.25] (2) to (1);
		\draw [in=90, out=-30] (1) to (4);
		\draw (2) to (3.center);
		\draw (1) to (0.center);
	\end{pgfonlayer}
\end{tikzpicture}
$$



Let $\ch$ denote the prop for the free commutative hopf algebra.
\end{definition}

\begin{lemma}
$\ch$ is a presentation for $\Mat_\Z$ under the direct sum.
\end{lemma}


\begin{lemma}
\label{lem:distfact}
%Distributive laws and factorization theorems
\end{lemma}

%Give normal form for commutative frobenius algebras, bialgebras.


\begin{lemma}[Spider theorem]
Given parallel string diagrams generated by the components of a Frobenius algebra, then they are equal if and only if they have the same connectivity.  A connected component with $n$ inputs and $m$ outputs has the normal form where the $n$ inputs are left associated, and plugged into the left coassociated $m$ outputs.


Graphically, the connected components are normalized to the following shape which we contract using the spider notation:

$$
\begin{tikzpicture}
	\begin{pgfonlayer}{nodelayer}
		\node [style=Z] (0) at (1.25, 3) {};
		\node [style=Z] (1) at (0.5, 4) {};
		\node [style=Z] (2) at (1.25, 2.25) {};
		\node [style=Z] (3) at (0.5, 1.25) {};
		\node [style=none] (4) at (1.5, 4) {};
		\node [style=none] (5) at (1.5, 1.25) {};
		\node [style=none] (6) at (0.25, 0.5) {};
		\node [style=none] (7) at (1.5, 4.75) {};
		\node [style=none] (8) at (1.5, 0.5) {};
		\node [style=none] (9) at (0.75, 4.75) {};
		\node [style=none] (10) at (0.25, 4.75) {};
		\node [style=none] (11) at (0.75, 0.5) {};
		\node [style=none] (12) at (1, 3.25) {};
		\node [style=none] (13) at (0.5, 3.75) {};
		\node [style=none] (14) at (0.5, 1.5) {};
		\node [style=none] (15) at (1, 2) {};
		\node [style=none] (16) at (0.75, 3.5) {$\ddots$};
		\node [style=none] (17) at (0.75, 1.75) {$\reflectbox{$\ddots$}$};
		\node [style=none] (18) at (1.2, 0.5) {$\cdots$};
		\node [style=none] (19) at (1.2, 4.75) {$\cdots$};
	\end{pgfonlayer}
	\begin{pgfonlayer}{edgelayer}
		\draw (7.center) to (4.center);
		\draw [in=105, out=-90] (10.center) to (1);
		\draw [in=60, out=-90, looseness=0.75] (4.center) to (0);
		\draw [in=-90, out=75] (1) to (9.center);
		\draw [in=300, out=90] (5.center) to (2);
		\draw [in=90, out=-120] (3) to (6.center);
		\draw [in=90, out=-60] (3) to (11.center);
		\draw (8.center) to (5.center);
		\draw (0) to (2);
		\draw (3) to (14.center);
		\draw (15.center) to (2);
		\draw (13.center) to (1);
		\draw (0) to (12.center);
	\end{pgfonlayer}
\end{tikzpicture}
=:
\begin{tikzpicture}
	\begin{pgfonlayer}{nodelayer}
		\node [style=none] (0) at (1.5, 1.75) {};
		\node [style=none] (1) at (2.75, 1.75) {};
		\node [style=none] (2) at (2, 1.75) {};
		\node [style=none] (3) at (2.45, 1.75) {$\cdots$};
		\node [style=none] (4) at (2.75, 3.25) {};
		\node [style=none] (5) at (2, 3.25) {};
		\node [style=none] (6) at (1.5, 3.25) {};
		\node [style=none] (7) at (2.45, 3.25) {$\cdots$};
		\node [style=Z] (8) at (2, 2.5) {};
	\end{pgfonlayer}
	\begin{pgfonlayer}{edgelayer}
		\draw [in=-90, out=45] (8) to (4.center);
		\draw (8) to (5.center);
		\draw [in=135, out=-90] (6.center) to (8);
		\draw [in=90, out=-150] (8) to (0.center);
		\draw (2.center) to (8);
		\draw [in=90, out=-30] (8) to (1.center);
	\end{pgfonlayer}
\end{tikzpicture}
$$

\end{lemma}



\subsection{Enriched category theory}
In this chapter, we develop the theory of enriched profunctors.  Recall in Definition \ref{def:internalprof}, in order to define distributive laws of props, we briefly described internal profunctors.  This imposes size restrictions which are sometimes undesirable. The cousin of internal category theory is enriched category theory:  in the $\mathcal V$-enriched setting, there is only the requirement that between two objects there is a $\mathcal V$-category.  


The following data gives enough structure to develop the theory of enriched categories: 
\begin{definition}
A {\bf Benabou cosmos } is a complete, cocomplete symmetric monoidal closed category.
\end{definition}



\begin{definition}
Given a Benabou cosmos ${\mathcal V}$,  a $\mathcal V$-{\bf profunctor} $\X \proarrow \Y$ is a functor  $\X^\op \times \Y \to \mathcal V$.
\end{definition}


\begin{definition}
Given an endo $\mathcal V$-{\bf profunctor} $P:\X \proarrow \X$, the  {\bf coend} $\int^{X} P(X,X) $ is given by the coequalizer:

$
  \xymatrix{
      \coprod_{X\to X'} P(X',X)  \ar@<-0.5ex>[r]\ar@<0.5ex>[r] &
      \coprod_{X \in \X_0} P(X,X) \ar[r] &
  	\int^{X} P(X,X) \\
  }
$

\end{definition}



\begin{lemma}
Monoidal bicategory
\end{lemma}


\begin{definition}
The monoidal bicategory $\Prof$ has:

%embeddings and adjoints
\end{definition}



\begin{definition}
The monoidal bicategory $\Prof^*$ has:


\end{definition}


\begin{theorem}
Quasitrictification theorem for monoidal 2-category
\end{theorem}

\begin{corollary}
Graphical calculus for quasitrict monoidal 2-category
\end{corollary}


\begin{lemma}
Graphical calculus for pointed profunctors

%
%monoidal functors
\end{lemma}



\subsection{Restriction and inverse category theory}


\label{sec:rest}

Restriction and inverse categories provide a categorical semantics for partial computing and reversible computing, respectively.  We review how weakened products can be constructed in both settings; relating one to the other.

\begin{definition}\cite[\S 2.1.1]{cockett}
A {\bf restriction category} is a category along with a restriction operator:

\hfil
$
(A \xrightarrow{f} B )\mapsto (A \xrightarrow{\bar f} A)
$\\
such that:\footnote{Using diagrammatic composition.}


\begin{center}
%\begin{mdframed}
\begin{multicols}{4}
\begin{enumerate}[label={\bf [R.\arabic*]}, ref={\bf [R.\arabic*]}]
\item $\bar f f  = f$
\label{R.1}
\item $\bar f \bar g = \bar g \bar f$
\label{R.2}
\item $\bar f \bar g = \bar{\bar f g}$
\label{R.3}
\item $f \bar g = \bar{fg} f$
\label{R.4}
\end{enumerate}
\end{multicols}
%\end{mdframed}
\end{center}


Maps of the form $\bar f$ are called restriction idempotents.
The canonical example of a restriction category is $\Par$, sets and partial maps.  The restriction in this case, just restricts partial functions to their domain of definition.


Restriction categories have a partial order on homsets given by $f \leq g \iff \bar f g = f$.


A map $f$ in a restriction category is called a {\bf partial isomorphism}, in case there exists a map $g$ called the partial inverse of $f$ so that $fg=\bar f$ and $gf = \bar g$.  Similarly, a map $f$ in a restriction category is {\bf total} if $\bar f =1$.  Denote the subcategories of partial isomorphisms and total maps of a restriction category $\X$, respectively by $\ParIso(\X)$ and $\Total(\X)$.



%A {\bf split restriction category} is a restriction category in which all restriction idempotents split.
\end{definition}



\begin{example} \cite[p. 101]{pcat} \cite[\S 5]{restiii}
A {\bf counital copy category} (or a p-category with a one element object) is a monoidal category with a family of commutative comonoids on every object compatible with the monoidal structure, with a natural comultiplication.  This gives a restriction via copying and then discarding:
$$
\begin{tikzpicture}
	\begin{pgfonlayer}{nodelayer}
		\node [style=none] (0) at (0.75, -2.5) {};
		\node [style=none] (1) at (0.75, -0.5) {};
		\node [style=map] (2) at (0.75, -1.5) {$\bar f$};
	\end{pgfonlayer}
	\begin{pgfonlayer}{edgelayer}
		\draw [style=simple] (0.center) to (2);
		\draw [style=simple] (2) to (1.center);
	\end{pgfonlayer}
\end{tikzpicture}
:=
\begin{tikzpicture}
	\begin{pgfonlayer}{nodelayer}
		\node [style=map] (0) at (0, 2.5) {$f$};
		\node [style=X] (1) at (0, 3.5) {};
		\node [style=X] (2) at (0.5, 1.5) {};
		\node [style=none] (3) at (1, 3.5) {};
		\node [style=none] (4) at (0.5, 0.5) {};
	\end{pgfonlayer}
	\begin{pgfonlayer}{edgelayer}
		\draw [style=simple] (1) to (0);
		\draw [style=simple, in=117, out=-90] (0) to (2);
		\draw [style=simple] (2) to (4.center);
		\draw [style=simple, in=-90, out=60] (2) to (3.center);
	\end{pgfonlayer}
\end{tikzpicture}
$$
\end{example}


\begin{definition}\cite[\S 3.1]{cockett}
A {\bf stable system of monics} $\M$ of $\X$ is a collection of monics in $\X$ containing all isomorphisms; where for any cospan $ X\xrightarrow{f} Z \xleftarrowtail{m} Y$  in $\X$, where $m'$ is in $\M$, the following pullback exists:

%\hfil$
%\xymatrixrowsep{.005in}
%\xymatrixcolsep{.13in}
%  \xymatrix{
%    W \ar[r]^{f'} \ar@{>->}[d]_{m'} & Y  \ar@{>->}[d]^m \\
%    X \ar[r]_{f} & Z
%  }
%$\\

\hfil$
\xymatrixrowsep{.005in}
\xymatrixcolsep{.13in}
  \xymatrix{
  	& W \ar@{>->}[dl]_{m'} \ar[dr]^{f'}\\
  	X \ar[dr]_f &  & Y \ar@{>->}[dl]^m\\
  	& Z
  }
$

Where $m'$ is in $\M$.

\end{definition}

Stable systems of monics allow one to represent the domains of definition of a partial functions as a subobjects:

\begin{definition}\cite[\S 3.1]{cockett}
Given a stable system of monics $\M$ in a category $\X$, the {\bf partial map category} $\Par(\X,\M)$ is given by the same objects as in $\X$ where morphisms $X\to Y$, given by isomorphism classes of spans $X\xleftarrowtail{m} Z \xrightarrow{f} Y$ where $f$ is a map in $\X$ and $m$ is a map in $\M$.  Composition is given by pullback and the identity is given by the trivial span.


Partial map categories have a restriction structure given by:  $(X\xleftarrowtail{m} Z \xrightarrow{f} Y) \mapsto (X\xleftarrowtail{m} Z \xrightarrowtail{m} X)$.  Moreover, a partial isomorphism is a span $X\xleftarrowtail{e} Z \xrightarrowtail{m} Y$ where $e,m \in \M$; the partial inverse given by  $Y\xleftarrowtail{m} Z \xrightarrowtail{e} X$.
\end{definition}


$\Par$ is equivalently the partial map category $\Par(\Sets,\M)$ where $\M$ is all monics in $\Sets$.




%\begin{lemma} \cite[Prop. 3.1]{cockett}
%Partial map categories are split restriction categories.
%\end{lemma}


%If restriction idempotents split then X is a cartesian restriction category if and only if Tot(X) is a cartesian category



If  $\X$ is finitely complete, then $\Span^\sim(\X)$ exists, and thus, there is a faithful functor $\Par(\X,\M)\to \Span^\sim(\X)$.


\begin{definition}\cite[\S 2.3.2]{cockett}
An {\bf inverse category} is a restriction category in which all maps are partial isomorphisms.  The subcategory of partial isomorphisms of $\Par$ is called $\Pinj$.
\end{definition}

Inverse categories can be presented with a dagger functor taking maps to their partial inverses:

\begin{theorem}\cite[Thm. 2.20]{cockett}
A restriction category $\X$ is an inverse category if and only if there is a dagger functor $(\_)^\circ:\X^\op\to\X$ such that for all $X\xleftarrow{f} Z \xrightarrow{g} Y$:
\begin{center}
\begin{tabular}{cc}
 $f f^\circ f = f$ & 
 $f f ^\circ gg^\circ = gg^\circ f f ^\circ $
\end{tabular}
\end{center}
\end{theorem}

Since restriction categories  and inverse categories give a categorical semantics for partial computing  and reversible computing, respectively, it is natural to ask when these categories have copying.


In the case of restriction categories, one must weaken the notion of the product to lax products using the partial order enrichment:


\begin{definition}\cite{restiii}
A restriction category has {\bf binary restriction products}, when for all objects  $X,Y$, there exists an object $X\times Y$ and total maps $X \xleftarrow{\pi_0}  X\times Y \xrightarrow{\pi_1} Y$, so that for all objects $Z$ and all maps $X \xleftarrow{f} Z \xrightarrow{g} Y$, the following diagram commutes there exists a unique $Z\xrightarrow{\langle f,g \rangle} X\times Y$ making the diagram commute:
\hfil
$
\xymatrixrowsep{0.2cm}
\xymatrixcolsep{0.4cm}
\xymatrix{
&& Z\ar@{..>}[dd]|-{\langle f, g\rangle} \ar@/_/[ddll]_f \ar@/^/[ddrr]^g &&\\
& \ar@{}[dr]|-{\geq} && \ar@{}[dl] |-{\leq} &\\
X &&  X\times Y \ar[rr]_{\pi_1} \ar[ll]^{\pi_0}  && Y
}
$

so that $\bar{\langle f, g\rangle \pi_0} f = \langle f, g\rangle \pi_0$ and $\bar{\langle f, g\rangle \pi_1} g = \langle f, g\rangle \pi_1$;
where additionally $\bar{\langle f, g\rangle} =  \bar f \bar g$.

%%DRAW DIAGRAM
%\begin{center}
%\begin{tabular}{ccc}
%  $\langle f, g\rangle \pi_0 \leq f$ &
%  $\langle f, g\rangle \pi_1 \leq g$ &
%  $\bar{\langle f, g\rangle} =  \bar f \bar g$
%\end{tabular}
%\end{center}

A restriction category has a {\bf restriction terminal object} $\top$ when for all objects $X$, there exists a unique total map $!_X:X\to\top$ such that $f !_Y = \bar f !_X$.

A restriction category with a restriction terminal object and binary restriction products is a {\bf Cartesian restriction category}.


An object $A$ in a restriction category with restriction products is {\bf discrete} when the diagonal map $\Delta_X:=\langle 1_X, 1_X\rangle$ is a partial isomorphism. A restriction category is discrete when all objects are discrete.  Discrete Cartesian restriction categories are said to have restriction products.
\end{definition}




\begin{theorem}\cite[Thm. 5.2]{restiii}
The structure of a  counital copy category structure is precisely that of a Cartesian restriction category.
\end{theorem}

In particular, the restriction operator is defined as follows, where the components of the restriction products are drawn in grey:

\begin{remark}
\label{cor:copy}

$$
\begin{tikzpicture}
	\begin{pgfonlayer}{nodelayer}
		\node [style=map] (14) at (0, 0) {$f$};
		\node [style=none] (15) at (0, -0.75) {};
		\node [style=X] (16) at (0, 0.75) {};
	\end{pgfonlayer}
	\begin{pgfonlayer}{edgelayer}
		\draw (16) to (14);
		\draw (14) to (15.center);
	\end{pgfonlayer}
\end{tikzpicture}
=
\begin{tikzpicture}
	\begin{pgfonlayer}{nodelayer}
		\node [style=map] (15) at (0, 0.75) {$f$};
		\node [style=X] (16) at (0, 1.5) {};
		\node [style=none] (17) at (0.5, -0.75) {};
		\node [style=X] (18) at (0.5, 0) {};
		\node [style=X] (19) at (1, 0.75) {};
	\end{pgfonlayer}
	\begin{pgfonlayer}{edgelayer}
		\draw (16) to (15);
		\draw [in=30, out=-90] (19) to (18);
		\draw [in=-90, out=150] (18) to (15);
		\draw (18) to (17.center);
	\end{pgfonlayer}
\end{tikzpicture}
=
\begin{tikzpicture}
	\begin{pgfonlayer}{nodelayer}
		\node [style=map] (16) at (0, 0) {$\bar f$};
		\node [style=X] (17) at (0, 0.75) {};
		\node [style=none] (18) at (0, -0.75) {};
	\end{pgfonlayer}
	\begin{pgfonlayer}{edgelayer}
		\draw (17) to (16);
		\draw (16) to (18.center);
	\end{pgfonlayer}
\end{tikzpicture}
$$
\end{remark}


\begin{proposition} \cite[\S 5.1]{restiii}
\label{prop:cartesian}

If $\X$ is a discrete Cartesian restriction category, then $\Total(\X)$ is Cartesian.
\end{proposition}



$\Par$ is a canonical example of a discrete Cartesian restriction category; the restriction product is given by the Cartesian product on underlying sets and the terminal object is  the singleton set.




The weakened notion of products in restriction categories is not satisfying for inverse categories because it does not impose enough equations governing the interaction between the diagonal map and its partial inverse.

\begin{definition}\cite[Def. 4.3.1]{giles}
A symmetric monoidal inverse category $\X$ is a {\bf discrete inverse category} when there is a natural, special commutative $\dag$-semi-Frobenius algebra\footnote{The ``semi'' adjective on Frobenius just means that the a semigroup and cosemigroup are interacting instead of a monoid and comonoid.} on every object (where the components of the semi-Frobenius algebra are drawn in grey)  compatible with the tensor product:

$$
\begin{tikzpicture}
	\begin{pgfonlayer}{nodelayer}
		\node [style=none] (0) at (0, 2.5) {};
		\node [style=none] (1) at (1, 2.5) {};
		\node [style=X] (2) at (0.5, 1.5) {};
		\node [style=none] (3) at (0.5, 0.5) {};
	\end{pgfonlayer}
	\begin{pgfonlayer}{edgelayer}
		\draw [style=simple] (3.center) to (2);
		\draw [style=simple, in=-90, out=117] (2) to (0.center);
		\draw [style=simple, in=63, out=-90] (1.center) to (2);
	\end{pgfonlayer}
\end{tikzpicture}
=
\begin{tikzpicture}
	\begin{pgfonlayer}{nodelayer}
		\node [style=X] (0) at (0, 2.5) {};
		\node [style=X] (1) at (1, 2.5) {};
		\node [style=none] (2) at (0.5, 1.5) {};
		\node [style=none] (3) at (0.5, 0.5) {};
		\node [style=none] (4) at (0, 3.5) {};
		\node [style=none] (5) at (1, 3.5) {};
		\node [style=none] (6) at (0, 4.5) {};
		\node [style=none] (7) at (1, 4.5) {};
		\node [style=otimes] (8) at (0.5, 1.5) {};
		\node [style=otimes] (9) at (1, 3.5) {};
		\node [style=otimes] (10) at (0, 3.5) {};
	\end{pgfonlayer}
	\begin{pgfonlayer}{edgelayer}
		\draw [style=simple] (3.center) to (2.center);
		\draw [style=simple, in=-90, out=135] (2.center) to (0);
		\draw [style=simple] (0) to (5.center);
		\draw [style=simple, in=120, out=-120, looseness=1.25] (4.center) to (0);
		\draw [style=simple, in=-60, out=60, looseness=1.25] (1) to (5.center);
		\draw [style=simple] (1) to (4.center);
		\draw [style=simple, in=45, out=-90] (1) to (2.center);
		\draw [style=simple] (4.center) to (6.center);
		\draw [style=simple] (5.center) to (7.center);
	\end{pgfonlayer}
\end{tikzpicture}
\hspace*{1cm}
\begin{tikzpicture}
	\begin{pgfonlayer}{nodelayer}
		\node [style=X] (0) at (0, 1.5) {};
		\node [style=none] (1) at (-0.5, 2.5) {};
		\node [style=none] (2) at (0.5, 2.5) {};
		\node [style=none] (3) at (0, 0.5) {};
	\end{pgfonlayer}
	\begin{pgfonlayer}{edgelayer}
		\draw [style=dashed] (3.center) to (0);
		\draw [style=dashed, in=-90, out=117] (0) to (1.center);
		\draw [style=dashed, in=63, out=-90] (2.center) to (0);
	\end{pgfonlayer}
\end{tikzpicture}
=
\begin{tikzpicture}
	\begin{pgfonlayer}{nodelayer}
		\node [style=none] (0) at (0, 1.5) {};
		\node [style=none] (1) at (-0.5, 2.5) {};
		\node [style=none] (2) at (0.5, 2.5) {};
		\node [style=none] (3) at (0, 0.5) {};
		\node [style=otimes] (4) at (0, 1.5) {};
	\end{pgfonlayer}
	\begin{pgfonlayer}{edgelayer}
		\draw [style=dashed] (3.center) to (0.center);
		\draw [style=dashed, in=-90, out=117] (0.center) to (1.center);
		\draw [style=dashed, in=63, out=-90] (2.center) to (0.center);
	\end{pgfonlayer}
\end{tikzpicture}
$$


Where the tensor product is also required to preserve restriction in both components.
\end{definition}

In a discrete inverse category, restriction idempotents are prephases for the Frobenius algebra, so that:
$$
\begin{tikzpicture}
	\begin{pgfonlayer}{nodelayer}
		\node [style=X] (0) at (3, 1.75) {};
		\node [style=map] (1) at (3, 1) {$\bar f$};
		\node [style=none] (2) at (3, 0.5) {};
		\node [style=none] (3) at (2.5, 2.5) {};
		\node [style=none] (4) at (3.5, 2.5) {};
	\end{pgfonlayer}
	\begin{pgfonlayer}{edgelayer}
		\draw [style=simple, in=63, out=-90] (4.center) to (0);
		\draw [style=simple, in=-90, out=117] (0) to (3.center);
		\draw [style=simple] (1) to (0);
		\draw [style=simple] (1) to (2.center);
	\end{pgfonlayer}
\end{tikzpicture}
=
\begin{tikzpicture}
	\begin{pgfonlayer}{nodelayer}
		\node [style=X] (0) at (3, 2) {};
		\node [style=none] (1) at (3, 1.5) {};
		\node [style=none] (2) at (2.5, 3) {};
		\node [style=none] (3) at (3.5, 3) {};
		\node [style=map] (4) at (2.5, 3) {$\bar f$};
		\node [style=none] (5) at (3.5, 3.5) {};
		\node [style=none] (6) at (2.5, 3.5) {};
	\end{pgfonlayer}
	\begin{pgfonlayer}{edgelayer}
		\draw [style=simple, in=63, out=-90] (3.center) to (0);
		\draw [style=simple, in=-90, out=117] (0) to (2.center);
		\draw [style=simple] (6.center) to (2.center);
		\draw [style=simple] (5.center) to (3.center);
		\draw [style=simple] (0) to (1.center);
	\end{pgfonlayer}
\end{tikzpicture}
=
\begin{tikzpicture}
	\begin{pgfonlayer}{nodelayer}
		\node [style=X] (0) at (3, 2) {};
		\node [style=none] (1) at (3, 1.5) {};
		\node [style=none] (2) at (3.5, 3) {};
		\node [style=none] (3) at (2.5, 3) {};
		\node [style=map] (4) at (3.5, 3) {$\bar f$};
		\node [style=none] (5) at (2.5, 3.5) {};
		\node [style=none] (6) at (3.5, 3.5) {};
	\end{pgfonlayer}
	\begin{pgfonlayer}{edgelayer}
		\draw [style=simple, in=117, out=-90] (3.center) to (0);
		\draw [style=simple, in=-90, out=63] (0) to (2.center);
		\draw [style=simple] (6.center) to (2.center);
		\draw [style=simple] (5.center) to (3.center);
		\draw [style=simple] (0) to (1.center);
	\end{pgfonlayer}
\end{tikzpicture}
\hspace*{.6cm}
\begin{tikzpicture}
	\begin{pgfonlayer}{nodelayer}
		\node [style=X] (0) at (3, 3) {};
		\node [style=none] (1) at (3, 3.5) {};
		\node [style=none] (2) at (3.5, 2) {};
		\node [style=none] (3) at (2.5, 2) {};
		\node [style=map] (4) at (3.5, 2) {$\bar f$};
		\node [style=none] (5) at (2.5, 1.5) {};
		\node [style=none] (6) at (3.5, 1.5) {};
	\end{pgfonlayer}
	\begin{pgfonlayer}{edgelayer}
		\draw [style=simple, in=-117, out=90] (3.center) to (0);
		\draw [style=simple, in=90, out=-63] (0) to (2.center);
		\draw [style=simple] (6.center) to (2.center);
		\draw [style=simple] (5.center) to (3.center);
		\draw [style=simple] (0) to (1.center);
	\end{pgfonlayer}
\end{tikzpicture}
=
\begin{tikzpicture}
	\begin{pgfonlayer}{nodelayer}
		\node [style=X] (0) at (3, 3) {};
		\node [style=none] (1) at (3, 3.5) {};
		\node [style=none] (2) at (2.5, 2) {};
		\node [style=none] (3) at (3.5, 2) {};
		\node [style=map] (4) at (2.5, 2) {$\bar f$};
		\node [style=none] (5) at (3.5, 1.5) {};
		\node [style=none] (6) at (2.5, 1.5) {};
	\end{pgfonlayer}
	\begin{pgfonlayer}{edgelayer}
		\draw [style=simple, in=-63, out=90] (3.center) to (0);
		\draw [style=simple, in=90, out=-117] (0) to (2.center);
		\draw [style=simple] (6.center) to (2.center);
		\draw [style=simple] (5.center) to (3.center);
		\draw [style=simple] (0) to (1.center);
	\end{pgfonlayer}
\end{tikzpicture}
=
\begin{tikzpicture}
	\begin{pgfonlayer}{nodelayer}
		\node [style=X] (0) at (3, 1.25) {};
		\node [style=map] (1) at (3, 2) {$\bar f$};
		\node [style=none] (2) at (3, 2.5) {};
		\node [style=none] (3) at (2.5, 0.5) {};
		\node [style=none] (4) at (3.5, 0.5) {};
	\end{pgfonlayer}
	\begin{pgfonlayer}{edgelayer}
		\draw [style=simple, in=-63, out=90] (4.center) to (0);
		\draw [style=simple, in=90, out=-117] (0) to (3.center);
		\draw [style=simple] (1) to (0);
		\draw [style=simple] (1) to (2.center);
	\end{pgfonlayer}
\end{tikzpicture}
$$


Discrete inverse categories are the ``right'' notion of weakened products for monoidal inverse categories:

\begin{theorem}\cite[Thm. 5.2.6]{giles}
There is an equivalence of categories between the category of discrete inverse categories and the category of discrete Cartesian categories.
\end{theorem}

To go from  discrete Cartesian restriction categories to discrete inverse categories, one takes the subcategory of partial isomorphisms.
The other direction is less trivial; in particular, this involves adding a restriction terminal object via the following construction which ``adds a history'' to a partial isomorphism:

\begin{definition}\cite[Def. 5.1.1]{giles}
Given a discrete inverse category $\X$, define its {\bf Cartesian completion} $\tilde \X$ as the category with:

\begin{description}
\item[Objects:] The same objects as $\X$.
\item[Maps:]
\hfil
$
\dfrac{ X\xrightarrow{f} Y \otimes S \in \X}{ X\xrightarrow{(f,S)} Y \in \tilde \X}
$



Where two parallel maps $X\xrightarrow{(f,S), (g,T)} Y $ are equivalent when either (both conditions are equivalent):
$$
\begin{tikzpicture}
	\begin{pgfonlayer}{nodelayer}
		\node [style=map] (0) at (0, 1.5) {$f$};
		\node [style=none] (1) at (0, 0.5) {};
		\node [style=map] (2) at (0, 3) {$f^\circ$};
		\node [style=map] (3) at (0, 4) {$g$};
		\node [style=X] (4) at (-0.5, 2.25) {};
		\node [style=X] (5) at (-0.5, 5) {};
		\node [style=none] (6) at (-0.5, 6) {};
		\node [style=none] (7) at (0.25, 6) {};
	\end{pgfonlayer}
	\begin{pgfonlayer}{edgelayer}
		\draw (0) to (1.center);
		\draw [in=75, out=-90] (7.center) to (3);
		\draw (6.center) to (5);
		\draw [in=120, out=-120] (5) to (4);
		\draw (4) to (2);
		\draw [in=60, out=-60, looseness=1.25] (2) to (0);
		\draw (0) to (4);
		\draw (3) to (2);
		\draw (3) to (5);
	\end{pgfonlayer}
\end{tikzpicture}
=
\begin{tikzpicture}
	\begin{pgfonlayer}{nodelayer}
		\node [style=map] (0) at (0, 1.5) {$g$};
		\node [style=none] (1) at (-0.5, 2.5) {};
		\node [style=none] (2) at (0.5, 2.5) {};
		\node [style=none] (3) at (0, 0.5) {};
	\end{pgfonlayer}
	\begin{pgfonlayer}{edgelayer}
		\draw [in=117, out=-90] (1.center) to (0);
		\draw [in=-90, out=63] (0) to (2.center);
		\draw (0) to (3.center);
	\end{pgfonlayer}
\end{tikzpicture}
\hspace*{.3cm}
or
\hspace*{.3cm}
\begin{tikzpicture}
	\begin{pgfonlayer}{nodelayer}
		\node [style=map] (0) at (0, 1.5) {$g$};
		\node [style=none] (1) at (0, 0.5) {};
		\node [style=map] (2) at (0, 3) {$g^\circ$};
		\node [style=map] (3) at (0, 4) {$f$};
		\node [style=X] (4) at (-0.5, 2.25) {};
		\node [style=X] (5) at (-0.5, 5) {};
		\node [style=none] (6) at (-0.5, 6) {};
		\node [style=none] (7) at (0.25, 6) {};
	\end{pgfonlayer}
	\begin{pgfonlayer}{edgelayer}
		\draw (0) to (1.center);
		\draw [in=75, out=-90] (7.center) to (3);
		\draw (6.center) to (5);
		\draw [in=120, out=-120] (5) to (4);
		\draw (4) to (2);
		\draw [in=60, out=-60, looseness=1.25] (2) to (0);
		\draw (0) to (4);
		\draw (3) to (2);
		\draw (3) to (5);
	\end{pgfonlayer}
\end{tikzpicture}
=
\begin{tikzpicture}
	\begin{pgfonlayer}{nodelayer}
		\node [style=map] (0) at (0, 1.5) {$f$};
		\node [style=none] (1) at (-0.5, 2.5) {};
		\node [style=none] (2) at (0.5, 2.5) {};
		\node [style=none] (3) at (0, 0.5) {};
	\end{pgfonlayer}
	\begin{pgfonlayer}{edgelayer}
		\draw [in=117, out=-90] (1.center) to (0);
		\draw [in=-90, out=63] (0) to (2.center);
		\draw (0) to (3.center);
	\end{pgfonlayer}
\end{tikzpicture}
$$

\item[Composition:]
\hfil
$
\begin{tikzpicture}
	\begin{pgfonlayer}{nodelayer}
		\node [style=map] (0) at (0, 1.5) {$f$};
		\node [style=none] (1) at (-0.5, 2.5) {};
		\node [style=none] (2) at (0.5, 2.5) {};
		\node [style=none] (3) at (0, 0.5) {};
	\end{pgfonlayer}
	\begin{pgfonlayer}{edgelayer}
		\draw [in=117, out=-90] (1.center) to (0);
		\draw [in=-90, out=63] (0) to (2.center);
		\draw (0) to (3.center);
	\end{pgfonlayer}
\end{tikzpicture}
;
\begin{tikzpicture}
	\begin{pgfonlayer}{nodelayer}
		\node [style=map] (0) at (0, 1.5) {$g$};
		\node [style=none] (1) at (-0.5, 2.5) {};
		\node [style=none] (2) at (0.5, 2.5) {};
		\node [style=none] (3) at (0, 0.5) {};
	\end{pgfonlayer}
	\begin{pgfonlayer}{edgelayer}
		\draw [in=117, out=-90] (1.center) to (0);
		\draw [in=-90, out=63] (0) to (2.center);
		\draw (0) to (3.center);
	\end{pgfonlayer}
\end{tikzpicture}
:=
\begin{tikzpicture}
	\begin{pgfonlayer}{nodelayer}
		\node [style=map] (0) at (0, 1.5) {$f$};
		\node [style=none] (1) at (0.5, 2.5) {};
		\node [style=none] (2) at (0, 0.5) {};
		\node [style=map] (3) at (-0.5, 2.5) {$g$};
		\node [style=none] (4) at (-1, 3.5) {};
		\node [style=otimes] (5) at (0, 3.5) {};
		\node [style=none] (6) at (-0.5, 2.5) {};
		\node [style=none] (7) at (-1, 4.5) {};
		\node [style=none] (8) at (0, 4.5) {};
	\end{pgfonlayer}
	\begin{pgfonlayer}{edgelayer}
		\draw [in=-90, out=63] (0) to (1.center);
		\draw (0) to (2.center);
		\draw [in=117, out=-90] (4.center) to (3);
		\draw (3) to (5);
		\draw [in=117, out=-90] (6.center) to (0);
		\draw [in=-63, out=90] (1.center) to (5);
		\draw (5) to (8.center);
		\draw (4.center) to (7.center);
	\end{pgfonlayer}
\end{tikzpicture}
$



\item[Identity:]
\hfil
$
\begin{tikzpicture}
	\begin{pgfonlayer}{nodelayer}
		\node [style=none] (0) at (0, 0.5) {};
		\node [style=none] (1) at (0, 2) {};
		\node [style=none] (2) at (0.75, 2) {};
		\node [style=none] (3) at (0, 1.25) {};
	\end{pgfonlayer}
	\begin{pgfonlayer}{edgelayer}
		\draw [style=dashed, in=-90, out=15] (3.center) to (2.center);
		\draw [style=simple] (0.center) to (1.center);
	\end{pgfonlayer}
\end{tikzpicture}
$

\item[Restriction: ] %Restriction is given by the underlying restriction of $\X$, so that:
\hfil
$
\bar{\left(
\begin{tikzpicture}
	\begin{pgfonlayer}{nodelayer}
		\node [style=map] (0) at (0, 1.5) {$f$};
		\node [style=none] (1) at (0, 0.5) {};
		\node [style=none] (2) at (-0.5, 2.5) {};
		\node [style=none] (3) at (0.5, 2.5) {};
	\end{pgfonlayer}
	\begin{pgfonlayer}{edgelayer}
		\draw [style=simple] (1.center) to (0);
		\draw [style=simple, in=117, out=-90] (2.center) to (0);
		\draw [style=simple, in=63, out=-90] (3.center) to (0);
	\end{pgfonlayer}
\end{tikzpicture}
\right)}
:=
\begin{tikzpicture}
	\begin{pgfonlayer}{nodelayer}
		\node [style=map] (0) at (0, 1.5) {$\bar f$};
		\node [style=none] (1) at (0, 0.5) {};
		\node [style=none] (2) at (0, 2.5) {};
		\node [style=none] (3) at (0, 2) {};
		\node [style=none] (4) at (0.5, 2.5) {};
	\end{pgfonlayer}
	\begin{pgfonlayer}{edgelayer}
		\draw [style=simple] (1.center) to (0);
		\draw [style=simple] (2.center) to (0);
		\draw [style=dashed, in=-90, out=15] (3.center) to (4.center);
	\end{pgfonlayer}
\end{tikzpicture}
$

\item[Restriction product:]
\hfil
$
\langle f,g \rangle:=
\begin{tikzpicture}
	\begin{pgfonlayer}{nodelayer}
		\node [style=map] (0) at (-0.25, 2.5) {$f$};
		\node [style=none] (1) at (-0.25, 3.5) {};
		\node [style=none] (2) at (0.75, 3.5) {};
		\node [style=none] (3) at (-0.25, 3.5) {};
		\node [style=map] (4) at (0.75, 2.5) {$g$};
		\node [style=none] (5) at (0.75, 3.5) {};
		\node [style=otimes] (6) at (0.75, 3.5) {};
		\node [style=otimes] (7) at (-0.25, 3.5) {};
		\node [style=X] (8) at (0.25, 1.5) {};
		\node [style=none] (9) at (-0.25, 4.5) {};
		\node [style=none] (10) at (0.75, 4.5) {};
		\node [style=none] (11) at (0.25, 0.5) {};
	\end{pgfonlayer}
	\begin{pgfonlayer}{edgelayer}
		\draw [style=simple, in=117, out=-120] (1.center) to (0);
		\draw [style=simple] (2.center) to (0);
		\draw [style=simple] (3.center) to (4);
		\draw [style=simple, in=63, out=-60] (5.center) to (4);
		\draw [style=simple, in=56, out=-90] (4) to (8);
		\draw [style=simple, in=-90, out=124] (8) to (0);
		\draw [style=simple] (9.center) to (1.center);
		\draw [style=simple] (2.center) to (10.center);
		\draw [style=simple] (8) to (11.center);
	\end{pgfonlayer}
\end{tikzpicture}
$

\item[Restriction terminal map:]
\hfil
$
\begin{tikzpicture}[xscale=-1]
	\begin{pgfonlayer}{nodelayer}
		\node [style=none] (0) at (0, 0.5) {};
		\node [style=none] (1) at (0, 2) {};
		\node [style=none] (2) at (0.75, 2) {};
		\node [style=none] (3) at (0, 1.25) {};
	\end{pgfonlayer}
	\begin{pgfonlayer}{edgelayer}
		\draw [style=dashed, in=-90, out=15] (3.center) to (2.center);
		\draw [style=simple] (0.center) to (1.center);
	\end{pgfonlayer}
\end{tikzpicture}
$

%%%%%TODO ROTATE DIAGRAMS FROM HERE
\item[Tensor product:]
\hfil
$
\begin{tikzpicture}
	\begin{pgfonlayer}{nodelayer}
		\node [style=map] (0) at (0, 1.5) {$f$};
		\node [style=none] (1) at (-0.5, 2.5) {};
		\node [style=none] (2) at (0.5, 2.5) {};
		\node [style=none] (3) at (0, 0.5) {};
	\end{pgfonlayer}
	\begin{pgfonlayer}{edgelayer}
		\draw [in=117, out=-90] (1.center) to (0);
		\draw [in=-90, out=63] (0) to (2.center);
		\draw (0) to (3.center);
	\end{pgfonlayer}
\end{tikzpicture}
\otimes
\begin{tikzpicture}
	\begin{pgfonlayer}{nodelayer}
		\node [style=map] (0) at (0, 1.5) {$g$};
		\node [style=none] (1) at (-0.5, 2.5) {};
		\node [style=none] (2) at (0.5, 2.5) {};
		\node [style=none] (3) at (0, 0.5) {};
	\end{pgfonlayer}
	\begin{pgfonlayer}{edgelayer}
		\draw [in=117, out=-90] (1.center) to (0);
		\draw [in=-90, out=63] (0) to (2.center);
		\draw (0) to (3.center);
	\end{pgfonlayer}
\end{tikzpicture}
:=
\begin{tikzpicture}
	\begin{pgfonlayer}{nodelayer}
		\node [style=map] (9) at (3.5, 1.5) {$f$};
		\node [style=map] (13) at (4.5, 1.5) {$g$};
		\node [style=otimes] (17) at (4.5, 2.5) {};
		\node [style=otimes] (18) at (3.5, 2.5) {};
		\node [style=otimes] (190) at (4, 0.75) {};
		\node  (19) at (4, 0.75) {};
		\node [style=none] (20) at (3.5, 3) {};
		\node [style=none] (21) at (4.5, 3) {};
		\node [style=none] (22) at (4, 0.25) {};
	\end{pgfonlayer}
	\begin{pgfonlayer}{edgelayer}
		\draw (13) to (18);
		\draw [bend right] (18) to (9);
		\draw (9) to (17);
		\draw [bend left] (17) to (13);
		\draw [in=45, out=-90] (13) to (19);
		\draw [in=-90, out=135] (19) to (9);
		\draw (21.center) to (17);
		\draw (22.center) to (19);
		\draw (18) to (20.center);
	\end{pgfonlayer}
\end{tikzpicture}
$

\item[Tensor unit:]  The same as in $\X$.
\end{description}

\end{definition}


\begin{example}\cite[Ex. 5.3.3]{giles}
$\tilde \Pinj$ is $\Par$.
\end{example}
\begin{proof}
For a partial function $f:X\to Y$, $\{(x,(y,x)) | (x,y) \in f \}/\sim$ is a partial isomorphism.
\end{proof}



\begin{lemma}
\label{lemma:xtildefaithful}
The canonical functor $\iota:\X\to \tilde \X$ is faithful.
\end{lemma}

\begin{proof}
Suppose that $\iota(f)\sim\iota(g)$, Then:

\begin{align*}
\begin{tikzpicture}
	\begin{pgfonlayer}{nodelayer}
		\node [style=map] (0) at (-0.5, 1.5) {$g$};
		\node [style=none] (1) at (-0.5, 2.5) {};
		\node [style=none] (2) at (-0.5, 0.5) {};
	\end{pgfonlayer}
	\begin{pgfonlayer}{edgelayer}
		\draw (1.center) to (0);
		\draw (0) to (2.center);
	\end{pgfonlayer}
\end{tikzpicture}
=
\begin{tikzpicture}
	\begin{pgfonlayer}{nodelayer}
		\node [style=map] (0) at (-0.5, 1.25) {$f$};
		\node [style=none] (1) at (-0.5, 0.5) {};
		\node [style=map] (2) at (-0.25, 3) {$f^\circ$};
		\node [style=map] (3) at (-0.25, 3.75) {$g$};
		\node [style=X] (4) at (-0.5, 2) {};
		\node [style=X] (5) at (-0.5, 4.75) {};
		\node [style=none] (6) at (-0.5, 5.75) {};
	\end{pgfonlayer}
	\begin{pgfonlayer}{edgelayer}
		\draw (0) to (1.center);
		\draw (6.center) to (5);
		\draw [in=120, out=-120, looseness=0.75] (5) to (4);
		\draw [in=-90, out=56] (4) to (2);
		\draw (0) to (4);
		\draw (3) to (2);
		\draw [in=-63, out=90] (3) to (5);
	\end{pgfonlayer}
\end{tikzpicture}
=
\begin{tikzpicture}
	\begin{pgfonlayer}{nodelayer}
		\node [style=none] (0) at (-0.5, 0.5) {};
		\node [style=map] (1) at (-0.25, 4.5) {$g$};
		\node [style=X] (2) at (-0.5, 2.75) {};
		\node [style=X] (3) at (-0.5, 5.5) {};
		\node [style=none] (4) at (-0.5, 6.5) {};
		\node [style=map] (5) at (-0.25, 3.75) {$f^\circ$};
		\node [style=map] (6) at (-0.5, 1.25) {$f$};
		\node [style=map] (7) at (-0.5, 2) {$f^\circ f$};
	\end{pgfonlayer}
	\begin{pgfonlayer}{edgelayer}
		\draw (4.center) to (3);
		\draw [in=120, out=-120, looseness=0.75] (3) to (2);
		\draw [in=-63, out=90] (1) to (3);
		\draw [in=-90, out=56] (2) to (5);
		\draw (1) to (5);
		\draw [style=simple] (2) to (7);
		\draw [style=simple] (7) to (6);
		\draw [style=simple] (6) to (0.center);
	\end{pgfonlayer}
\end{tikzpicture}
=
\begin{tikzpicture}
	\begin{pgfonlayer}{nodelayer}
		\node [style=none] (0) at (-0.5, 0.5) {};
		\node [style=map] (1) at (0, 4) {$g$};
		\node [style=X] (2) at (-0.5, 2.25) {};
		\node [style=X] (3) at (-0.5, 5) {};
		\node [style=none] (4) at (-0.5, 6) {};
		\node [style=map] (5) at (0, 3.25) {$f^\circ$};
		\node [style=map] (6) at (-0.5, 1.25) {$f$};
		\node [style=map] (7) at (-1, 3.25) {$f^\circ f$};
		\node [style=none] (8) at (-1, 4) {};
	\end{pgfonlayer}
	\begin{pgfonlayer}{edgelayer}
		\draw (4.center) to (3);
		\draw [in=-60, out=90] (1) to (3);
		\draw [in=-90, out=56] (2) to (5);
		\draw (1) to (5);
		\draw [style=simple] (6) to (0.center);
		\draw [style=simple, in=-90, out=120] (2) to (7);
		\draw [style=simple] (2) to (6);
		\draw [style=simple, in=90, out=-120] (3) to (8.center);
		\draw [style=simple] (8.center) to (7);
	\end{pgfonlayer}
\end{tikzpicture}
=
\begin{tikzpicture}
	\begin{pgfonlayer}{nodelayer}
		\node [style=none] (0) at (-0.5, 0.5) {};
		\node [style=map] (1) at (0, 3.25) {$g$};
		\node [style=X] (2) at (-0.5, 2.25) {};
		\node [style=X] (3) at (-0.5, 4.25) {};
		\node [style=none] (4) at (-0.5, 5.25) {};
		\node [style=map] (5) at (-0.5, 1.25) {$ff^\circ$};
		\node [style=map] (6) at (-1, 3.25) {$f$};
	\end{pgfonlayer}
	\begin{pgfonlayer}{edgelayer}
		\draw (4.center) to (3);
		\draw [in=-60, out=90] (1) to (3);
		\draw [style=simple] (5) to (0.center);
		\draw [style=simple, in=-90, out=120] (2) to (6);
		\draw [style=simple] (2) to (5);
		\draw [style=simple, in=60, out=-90] (1) to (2);
		\draw [style=simple, in=90, out=-120] (3) to (6);
	\end{pgfonlayer}
\end{tikzpicture}
=
\begin{tikzpicture}
	\begin{pgfonlayer}{nodelayer}
		\node [style=none] (0) at (-0.5, 0.5) {};
		\node [style=map] (1) at (0, 3.25) {$g$};
		\node [style=X] (2) at (-0.5, 1.5) {};
		\node [style=X] (3) at (-0.5, 4.25) {};
		\node [style=none] (4) at (-0.5, 5.25) {};
		\node [style=map] (5) at (-1, 3.25) {$f$};
		\node [style=map] (6) at (-1, 2.5) {$ff^\circ$};
		\node [style=none] (7) at (0, 2.5) {};
	\end{pgfonlayer}
	\begin{pgfonlayer}{edgelayer}
		\draw (4.center) to (3);
		\draw [in=-60, out=90] (1) to (3);
		\draw [style=simple, in=90, out=-120] (3) to (5);
		\draw (1) to (7.center);
		\draw [in=60, out=-90] (7.center) to (2);
		\draw (2) to (0.center);
		\draw [in=-90, out=120] (2) to (6);
		\draw (6) to (5);
	\end{pgfonlayer}
\end{tikzpicture}
=
\begin{tikzpicture}
	\begin{pgfonlayer}{nodelayer}
		\node [style=none] (0) at (-0.5, 0.5) {};
		\node [style=map] (1) at (0, 2.5) {$g$};
		\node [style=X] (2) at (-0.5, 1.5) {};
		\node [style=X] (3) at (-0.5, 3.5) {};
		\node [style=none] (4) at (-0.5, 4.5) {};
		\node [style=map] (5) at (-1, 2.5) {$f$};
	\end{pgfonlayer}
	\begin{pgfonlayer}{edgelayer}
		\draw (4.center) to (3);
		\draw [in=-60, out=90] (1) to (3);
		\draw [style=simple, in=90, out=-120] (3) to (5);
		\draw (2) to (0.center);
		\draw [in=60, out=-90] (1) to (2);
		\draw [in=-90, out=120] (2) to (5);
	\end{pgfonlayer}
\end{tikzpicture}
=
\begin{tikzpicture}
	\begin{pgfonlayer}{nodelayer}
		\node [style=none] (0) at (-0.5, 0.5) {};
		\node [style=map] (1) at (-0.2, 2.5) {$g$};
		\node [style=X] (2) at (-0.5, 1.5) {};
		\node [style=X] (3) at (-0.5, 3.5) {};
		\node [style=none] (4) at (-0.5, 4.5) {};
		\node [style=map] (5) at (-0.8, 2.5) {$f$};
	\end{pgfonlayer}
	\begin{pgfonlayer}{edgelayer}
		\draw (4.center) to (3);
		\draw [in=-120, out=90, looseness=1.25] (1) to (3);
		\draw [style=simple, in=90, out=-60, looseness=1.25] (3) to (5);
		\draw (2) to (0.center);
		\draw [in=120, out=-90, looseness=1.25] (1) to (2);
		\draw [in=-90, out=60, looseness=1.25] (2) to (5);
	\end{pgfonlayer}
\end{tikzpicture}
=
\begin{tikzpicture}
	\begin{pgfonlayer}{nodelayer}
		\node [style=none] (0) at (-0.5, 0.5) {};
		\node [style=map] (1) at (0, 2.5) {$f$};
		\node [style=X] (2) at (-0.5, 1.5) {};
		\node [style=X] (3) at (-0.5, 3.5) {};
		\node [style=none] (4) at (-0.5, 4.5) {};
		\node [style=map] (5) at (-1, 2.5) {$g$};
	\end{pgfonlayer}
	\begin{pgfonlayer}{edgelayer}
		\draw (4.center) to (3);
		\draw [in=-60, out=90] (1) to (3);
		\draw [style=simple, in=90, out=-120] (3) to (5);
		\draw (2) to (0.center);
		\draw [in=60, out=-90] (1) to (2);
		\draw [in=-90, out=120] (2) to (5);
	\end{pgfonlayer}
\end{tikzpicture}
=
\begin{tikzpicture}
	\begin{pgfonlayer}{nodelayer}
		\node [style=map] (0) at (-0.5, 1.25) {$g$};
		\node [style=none] (1) at (-0.5, 0.5) {};
		\node [style=map] (2) at (-0.25, 3) {$g^\circ$};
		\node [style=map] (3) at (-0.25, 3.75) {$f$};
		\node [style=X] (4) at (-0.5, 2) {};
		\node [style=X] (5) at (-0.5, 4.75) {};
		\node [style=none] (6) at (-0.5, 5.75) {};
	\end{pgfonlayer}
	\begin{pgfonlayer}{edgelayer}
		\draw (0) to (1.center);
		\draw (6.center) to (5);
		\draw [in=120, out=-120, looseness=0.75] (5) to (4);
		\draw [in=-90, out=56] (4) to (2);
		\draw (0) to (4);
		\draw (3) to (2);
		\draw [in=-63, out=90] (3) to (5);
	\end{pgfonlayer}
\end{tikzpicture}
=
\begin{tikzpicture}
	\begin{pgfonlayer}{nodelayer}
		\node [style=map] (0) at (-0.5, 1.5) {$f$};
		\node [style=none] (1) at (-0.5, 2.5) {};
		\node [style=none] (2) at (-0.5, 0.5) {};
	\end{pgfonlayer}
	\begin{pgfonlayer}{edgelayer}
		\draw (1.center) to (0);
		\draw (0) to (2.center);
	\end{pgfonlayer}
\end{tikzpicture}
\end{align*}


\end{proof}



\begin{lemma}
The induced Frobenius algebra structure in $\tilde \X$ is counital.
\end{lemma}
\begin{proof}
For all $X$, the map $X \to (X\otimes X) \otimes I$ in $\tilde\X$ induced by the Frobenius algebra in $\X$ has a counit given by the  unitor $X\to I\otimes X$ since, in $\X$:
$$
\begin{tikzpicture}
	\begin{pgfonlayer}{nodelayer}
		\node [style=X] (0) at (0, 3.75) {};
		\node [style=none] (1) at (0, 3) {};
		\node [style=X] (2) at (-0.25, 4.75) {};
		\node [style=X] (3) at (0, 5.75) {};
		\node [style=X] (4) at (-0.25, 7.25) {};
		\node [style=none] (5) at (-0.25, 8) {};
		\node [style=none] (6) at (0.25, 8) {};
		\node [style=none] (7) at (0, 6.5) {};
	\end{pgfonlayer}
	\begin{pgfonlayer}{edgelayer}
		\draw (0) to (1.center);
		\draw [in=-90, out=120] (0) to (2);
		\draw (2) to (3);
		\draw (5.center) to (4);
		\draw [in=120, out=-120, looseness=0.75] (4) to (2);
		\draw [in=60, out=-60] (3) to (0);
		\draw (3) to (7.center);
		\draw [in=-45, out=90, looseness=0.75] (7.center) to (4);
		\draw [style=dashed, in=-90, out=75] (7.center) to (6.center);
	\end{pgfonlayer}
\end{tikzpicture}
=
\begin{tikzpicture}
	\begin{pgfonlayer}{nodelayer}
		\node [style=none] (0) at (0, 3) {};
		\node [style=none] (1) at (0, 4.5) {};
		\node [style=none] (2) at (0.5, 4.5) {};
		\node [style=none] (3) at (0, 3.75) {};
	\end{pgfonlayer}
	\begin{pgfonlayer}{edgelayer}
		\draw [style=dashed, in=-90, out=15] (3.center) to (2.center);
		\draw [style=simple] (3.center) to (0.center);
		\draw [style=simple] (3.center) to (1.center);
	\end{pgfonlayer}
\end{tikzpicture}
$$
\end{proof}

\section{Categorical quantum mechanics}
\begin{definition}
\label{def:dag}
A {\bf dagger category} is a category $\X$ equipped with an identity-on-objects involution $(\_)^\dag:\X^\op\to\X$ called the dagger.

A isomorphism $f$ in a dagger category is {\bf unitary} when $f^{-1}=f^\dag$.


A {\bf dagger monoidal  category} is a monoidal dagger category where all the coherence isomorphisms of the monoidal structure are unitary. And for all morphisms $f,g$ $(f \otimes g)^\dag = f^\dag \otimes g^\dag$.

A {\bf dagger symmetric monoidal  category} is a monoidal dagger category whose underlying monoidal structure is addionally symmetric monoidal, where the symmetry maps are unitary.

A {\bf Strongly compact closed category } is a compact closed category equipped with a dagger functor whose underlying symmetric monoidal closed structure forms a symmetric monoidal dagger category, and where for all $X$ $\epsilon^\dagger = \eta;c$.
\end{definition}

\begin{example}
$\Hilb$ is a symmetric monoidal dagger category with respect to the Hermetian adjoint functor.

Moreover, $\FHilb$ is a strongly compact closed category.  Transported along the equivalence $\FHilb\cong \Mat(\C)$ this dagger structure corresponds to complex conjugation.
\end{example}


\begin{example}
Given a finitely complete category $\X$, $\Span^\sim (\X)$ is strongly compact closed with respect to the cartesian product and the transpose.

Similarly, given a regular category $\X$, $\Rel(\X)$ is a strongly compact closed.

With a chosen stabile system of monics $\mathcal M$, this restricts to a dagger symmetric monoidal category $\ParIso(\C,{\mathcal M})$. Therefore $\Pinj$ and $\FPinj$ are dagger symmetric monoidal categories.
\end{example}



\begin{definition}
\label{def:specialdagfa}
%special dag-Frobenius algebras
A {\bf dagger Frobenius algebra}  is a frobenius algebra in a dagger category whose monoid structure is the dagger of the comonoid structure.

\end{definition}

\begin{lemma}
\label{lem:specialdagfa}
%special dag-Frobenius algebras in FHilb are orthonomal bases/quantum observables

Special commutative dagger Frobenius algebras in $\FHilb$ are in bijection with orthonormal bases.
\end{lemma}


\begin{definition}
\label{def:complementary}
%Interacting Hopf-Frobenius algebras/ strongly complementary observables

Two bases in $\FHilb$ are {\bf strongly complementary} when their corresponding Frobenius algebras interact to form a Hopf algebra whose antipode is equivalently any of the following maps:
$$
\begin{tikzpicture}
	\begin{pgfonlayer}{nodelayer}
		\node [style=Z] (0) at (0.5, 0) {};
		\node [style=X] (1) at (1, 0.5) {};
		\node [style=none] (2) at (0, 1) {};
		\node [style=none] (3) at (1.5, -0.5) {};
	\end{pgfonlayer}
	\begin{pgfonlayer}{edgelayer}
		\draw [in=-90, out=135] (0) to (2.center);
		\draw (0) to (1);
		\draw [in=90, out=-45] (1) to (3.center);
	\end{pgfonlayer}
\end{tikzpicture}=
\begin{tikzpicture}
	\begin{pgfonlayer}{nodelayer}
		\node [style=X] (0) at (0.5, 0) {};
		\node [style=Z] (1) at (1, 0.5) {};
		\node [style=none] (2) at (0, 1) {};
		\node [style=none] (3) at (1.5, -0.5) {};
	\end{pgfonlayer}
	\begin{pgfonlayer}{edgelayer}
		\draw [in=-90, out=135] (0) to (2.center);
		\draw (0) to (1);
		\draw [in=90, out=-45] (1) to (3.center);
	\end{pgfonlayer}
\end{tikzpicture}=
\begin{tikzpicture}
	\begin{pgfonlayer}{nodelayer}
		\node [style=Z] (0) at (1, 0) {};
		\node [style=X] (1) at (0.5, 0.5) {};
		\node [style=none] (2) at (1.5, 1) {};
		\node [style=none] (3) at (0, -0.5) {};
	\end{pgfonlayer}
	\begin{pgfonlayer}{edgelayer}
		\draw [in=-90, out=45] (0) to (2.center);
		\draw (0) to (1);
		\draw [in=90, out=-135] (1) to (3.center);
	\end{pgfonlayer}
\end{tikzpicture}=
\begin{tikzpicture}
	\begin{pgfonlayer}{nodelayer}
		\node [style=X] (0) at (1, 0) {};
		\node [style=Z] (1) at (0.5, 0.5) {};
		\node [style=none] (2) at (1.5, 1) {};
		\node [style=none] (3) at (0, -0.5) {};
	\end{pgfonlayer}
	\begin{pgfonlayer}{edgelayer}
		\draw [in=-90, out=45] (0) to (2.center);
		\draw (0) to (1);
		\draw [in=90, out=-135] (1) to (3.center);
	\end{pgfonlayer}
\end{tikzpicture}
$$

\end{definition}
%example: this can be used to construct the CNOT gate


\begin{example}
Given some fixed orthonormal basis in $\FHilb$ the Fourier transform of the basis is strongly complementary.
\end{example}

\begin{definition}
\label{def:phases}
Given a $\dag$-Frobenius algebra on an object $X$, a {\bf phase} for the Frobenius algebra is a unitary endomorphism on $X$ which commutes with the multiplication and comultiplication.

Phases for Frobenius algebras are preserved by composition; and they form a group called the {\bf phase group} for the Frobenius algebra.
\end{definition}

\begin{example}
Given an orthonormal basis in $\Hilb$, the phases are generated by unitaries $\sum_{i} e^{ \theta \pi i}|  i \rangle\langle i|$.  This generates the unit circle in the complex plane; thus the phase group is isomorphic to the circle (hence the name).
\end{example}

\begin{lemma}{Phased spider theorem}
There is a normal form for the string diagrams generated by the components of the Frobenius algebra and its phase group.
\end{lemma}

The normal form is the same as the vanilla spider theorem, except decorated with a single element of the phase group in the middle, between where the monoid and comonoid meet:

$$
\begin{tikzpicture}
	\begin{pgfonlayer}{nodelayer}
		\node [style=Z] (9) at (4.75, 3) {};
		\node [style=Z] (10) at (4, 4) {};
		\node [style=Z] (11) at (4.75, 2) {};
		\node [style=Z] (12) at (4, 1) {};
		\node [style=none] (13) at (5, 4) {};
		\node [style=none] (14) at (5, 1) {};
		\node [style=none] (15) at (3.75, 0.25) {};
		\node [style=none] (16) at (5, 4.75) {};
		\node [style=none] (17) at (5, 0.25) {};
		\node [style=none] (18) at (4.25, 4.75) {};
		\node [style=none] (19) at (3.75, 4.75) {};
		\node [style=none] (20) at (4.25, 0.25) {};
		\node [style=none] (21) at (4.5, 3.25) {};
		\node [style=none] (22) at (4, 3.75) {};
		\node [style=none] (23) at (4, 1.25) {};
		\node [style=none] (24) at (4.5, 1.75) {};
		\node [style=none] (25) at (4.25, 3.5) {$\ddots$};
		\node [style=none] (26) at (4.25, 1.5) {$\reflectbox{$\ddots$}$};
		\node [style=none] (27) at (4.7, 0.25) {$\cdots$};
		\node [style=none] (28) at (4.7, 4.75) {$\cdots$};
		\node [style=Z] (29) at (4.75, 2.5) {$g$};
	\end{pgfonlayer}
	\begin{pgfonlayer}{edgelayer}
		\draw (16.center) to (13.center);
		\draw [in=105, out=-90] (19.center) to (10);
		\draw [in=60, out=-90, looseness=0.75] (13.center) to (9);
		\draw [in=-90, out=75] (10) to (18.center);
		\draw [in=300, out=90] (14.center) to (11);
		\draw [in=90, out=-120] (12) to (15.center);
		\draw [in=90, out=-60] (12) to (20.center);
		\draw (17.center) to (14.center);
		\draw (9) to (11);
		\draw (12) to (23.center);
		\draw (24.center) to (11);
		\draw (22.center) to (10);
		\draw (9) to (21.center);
	\end{pgfonlayer}
\end{tikzpicture}
=:
\begin{tikzpicture}
	\begin{pgfonlayer}{nodelayer}
		\node [style=none] (0) at (1.5, 1.75) {};
		\node [style=none] (1) at (2.75, 1.75) {};
		\node [style=none] (2) at (2, 1.75) {};
		\node [style=none] (3) at (2.45, 1.75) {$\cdots$};
		\node [style=none] (4) at (2.75, 3.25) {};
		\node [style=none] (5) at (2, 3.25) {};
		\node [style=none] (6) at (1.5, 3.25) {};
		\node [style=none] (7) at (2.45, 3.25) {$\cdots$};
		\node [style=Z] (8) at (2, 2.5) {$g$};
	\end{pgfonlayer}
	\begin{pgfonlayer}{edgelayer}
		\draw [in=-90, out=45] (8) to (4.center);
		\draw (8) to (5.center);
		\draw [in=135, out=-90] (6.center) to (8);
		\draw [in=90, out=-150] (8) to (0.center);
		\draw (2.center) to (8);
		\draw [in=90, out=-30] (8) to (1.center);
	\end{pgfonlayer}
\end{tikzpicture}
$$


The normal form induces a phased spider fusion:

\begin{definition}
\label{def:zx}
Given some fixed dimension $d$, the qudit ZX-calculus is a collection of related graphical calculi with faihtful interpretations into $\FHilb$ generated by the phased Frobenius algebras for the standard basis and Fourier bases.

A prop is a {\bf fragment of the ZX-calculus} when it is a symmetric monoidal subtheory of the $\ZX$-calculus with a faithful interpretation in $\FHilb$.
\end{definition}


\begin{definition}
Given some prime $d$, the {\bf phase-free} qudit $\ZX$-calculus
is the fragment of the ZX-calculus generated by both Frobenius algebras as well as their trivial phase groups.
\end{definition}

\begin{definition}
Given a principal ideal domain $k$, the strongly compact closed category of {\bf linear relations} is $\LinRel_k := \Rel(\Mat_k)$.
\end{definition}

\begin{lemma}
Given a prime $p$ is an isomorphism of props between the phase-free qudit $\ZX$-calculus and $\LinRel_{\F_p}$.
\end{lemma}

%Talk about the alternative scaling that we use to simplify things so that phase free ZX is presented by spans of Hopf algebras


%density matrices

Now that we have developed the theory of bases within the language of monoidal categories, we can now talk about classical collapse and measurement.

\begin{definition}
\label{def:cpm}

%Dagger category... equivalent to ioo compact closed conjugation 

Given a $\dag$-compact closed category $\X$, there is a $\dag$-compact closed category $\CPM(\X)$ has:

\begin{description}
\item[Objects:] Same as in $\X$.

\item[Maps:]  
$
\dfrac{ X\xrightarrow{f} Y \otimes S \in \X           }
         { X\xrightarrow{(f,S)} Y \in \tilde \CPM(\X) }
$
\hspace*{.5cm}
modulo:
\hspace*{.5cm}
$
(f,S) \sim (g,T) \iff 
\begin{tikzpicture}
	\begin{pgfonlayer}{nodelayer}
		\node [style=none] (2) at (0.75, 11.75) {};
		\node [style=none] (4) at (0.75, 10.75) {};
		\node [style=none] (5) at (0.75, 10) {};
		\node [style=map] (6) at (0.75, 10.75) {$f$};
		\node [style=none] (7) at (1.75, 11.75) {};
		\node [style=none] (8) at (1.75, 10.75) {};
		\node [style=none] (9) at (1.75, 10) {};
		\node [style=map] (10) at (1.75, 10.75) {$f_*$};
	\end{pgfonlayer}
	\begin{pgfonlayer}{edgelayer}
		\draw (9.center) to (10);
		\draw (10) to (7.center);
		\draw (6) to (2.center);
		\draw (5.center) to (6);
		\draw [bend left=60, looseness=1.50] (6) to (10);
	\end{pgfonlayer}
\end{tikzpicture}
=
\begin{tikzpicture}
	\begin{pgfonlayer}{nodelayer}
		\node [style=none] (2) at (0.75, 11.75) {};
		\node [style=none] (4) at (0.75, 10.75) {};
		\node [style=none] (5) at (0.75, 10) {};
		\node [style=map] (6) at (0.75, 10.75) {$g$};
		\node [style=none] (7) at (1.75, 11.75) {};
		\node [style=none] (8) at (1.75, 10.75) {};
		\node [style=none] (9) at (1.75, 10) {};
		\node [style=map] (10) at (1.75, 10.75) {$g_*$};
	\end{pgfonlayer}
	\begin{pgfonlayer}{edgelayer}
		\draw (9.center) to (10);
		\draw (10) to (7.center);
		\draw (6) to (2.center);
		\draw (5.center) to (6);
		\draw [bend left=60, looseness=1.50] (6) to (10);
	\end{pgfonlayer}
\end{tikzpicture}
$

\item[Composition]:  
$
\dfrac{X\xrightarrow{(f,S)} Y , \hspace*{.5cm} Y\xrightarrow{(g,T)} Z }
         {(f,S);(g;T) := (f;(g\otimes 1_S);\alpha^{-1}_{Z,S,T} ,S\otimes T) } 
$

Or using proof net notation:
\hspace*{.5cm}
$
\begin{tikzpicture}
	\begin{pgfonlayer}{nodelayer}
		\node [style=map] (0) at (0, 1.5) {$f$};
		\node [style=none] (1) at (-0.5, 2.5) {};
		\node [style=none] (2) at (0.5, 2.5) {};
		\node [style=none] (3) at (0, 0.5) {};
	\end{pgfonlayer}
	\begin{pgfonlayer}{edgelayer}
		\draw [in=117, out=-90] (1.center) to (0);
		\draw [in=-90, out=63] (0) to (2.center);
		\draw (0) to (3.center);
	\end{pgfonlayer}
\end{tikzpicture}
;
\begin{tikzpicture}
	\begin{pgfonlayer}{nodelayer}
		\node [style=map] (0) at (0, 1.5) {$g$};
		\node [style=none] (1) at (-0.5, 2.5) {};
		\node [style=none] (2) at (0.5, 2.5) {};
		\node [style=none] (3) at (0, 0.5) {};
	\end{pgfonlayer}
	\begin{pgfonlayer}{edgelayer}
		\draw [in=117, out=-90] (1.center) to (0);
		\draw [in=-90, out=63] (0) to (2.center);
		\draw (0) to (3.center);
	\end{pgfonlayer}
\end{tikzpicture}
:=
\begin{tikzpicture}
	\begin{pgfonlayer}{nodelayer}
		\node [style=map] (0) at (0, 1.5) {$f$};
		\node [style=none] (1) at (0.5, 2.5) {};
		\node [style=none] (2) at (0, 0.5) {};
		\node [style=map] (3) at (-0.5, 2.5) {$g$};
		\node [style=none] (4) at (-1, 3.5) {};
		\node [style=otimes] (5) at (0, 3.5) {};
		\node [style=none] (6) at (-0.5, 2.5) {};
		\node [style=none] (7) at (-1, 4.5) {};
		\node [style=none] (8) at (0, 4.5) {};
	\end{pgfonlayer}
	\begin{pgfonlayer}{edgelayer}
		\draw [in=-90, out=63] (0) to (1.center);
		\draw (0) to (2.center);
		\draw [in=117, out=-90] (4.center) to (3);
		\draw (3) to (5);
		\draw [in=117, out=-90] (6.center) to (0);
		\draw [in=-63, out=90] (1.center) to (5);
		\draw (5) to (8.center);
		\draw (4.center) to (7.center);
	\end{pgfonlayer}
\end{tikzpicture}
$

\item[Identity:]
$
\dfrac{ 1_X \in \tilde \CPM(\X)}{(u^R_A)^{-1} \in \X}
$

\item[Tensor product:]

$$
\dfrac{X\xrightarrow{(f,S)} Y, \hspace*{.5cm} Z\xrightarrow{(g,T)} W}
{(f,S)\otimes (g;T) := ((f\otimes g);(1_{X\otimes S} \otimes c_{W,T});\alpha_{X,S,T\otimes W};(1_X\otimes \alpha_{S,T,W}^{-1};(c_{S,T}));\alpha_{Y,W,S\otimes T}^{-1} ,S\otimes T)} 
$$

Or in proof net notation:
$
\begin{tikzpicture}
	\begin{pgfonlayer}{nodelayer}
		\node [style=map] (0) at (0, 1.5) {$f$};
		\node [style=none] (1) at (-0.5, 2.5) {};
		\node [style=none] (2) at (0.5, 2.5) {};
		\node [style=none] (3) at (0, 0.5) {};
	\end{pgfonlayer}
	\begin{pgfonlayer}{edgelayer}
		\draw [in=117, out=-90] (1.center) to (0);
		\draw [in=-90, out=63] (0) to (2.center);
		\draw (0) to (3.center);
	\end{pgfonlayer}
\end{tikzpicture}
\otimes
\begin{tikzpicture}
	\begin{pgfonlayer}{nodelayer}
		\node [style=map] (0) at (0, 1.5) {$g$};
		\node [style=none] (1) at (-0.5, 2.5) {};
		\node [style=none] (2) at (0.5, 2.5) {};
		\node [style=none] (3) at (0, 0.5) {};
	\end{pgfonlayer}
	\begin{pgfonlayer}{edgelayer}
		\draw [in=117, out=-90] (1.center) to (0);
		\draw [in=-90, out=63] (0) to (2.center);
		\draw (0) to (3.center);
	\end{pgfonlayer}
\end{tikzpicture}
:=
\begin{tikzpicture}
	\begin{pgfonlayer}{nodelayer}
		\node [style=map] (9) at (3.5, 1.5) {$f$};
		\node [style=map] (13) at (4.5, 1.5) {$g$};
		\node [style=otimes] (17) at (4.5, 2.5) {};
		\node [style=otimes] (18) at (3.5, 2.5) {};
		\node [style=otimes] (190) at (4, 0.75) {};
		\node  (19) at (4, 0.75) {};
		\node [style=none] (20) at (3.5, 3) {};
		\node [style=none] (21) at (4.5, 3) {};
		\node [style=none] (22) at (4, 0.25) {};
	\end{pgfonlayer}
	\begin{pgfonlayer}{edgelayer}
		\draw (13) to (18);
		\draw [bend right] (18) to (9);
		\draw (9) to (17);
		\draw [bend left] (17) to (13);
		\draw [in=45, out=-90] (13) to (19);
		\draw [in=-90, out=135] (19) to (9);
		\draw (21.center) to (17);
		\draw (22.center) to (19);
		\draw (18) to (20.center);
	\end{pgfonlayer}
\end{tikzpicture}
$

\item[Dagger compact closed structure:] Inherited pointwise from $\X$.

\end{description}
\end{definition}

\begin{example}
$\CPM(\FHilb)$ is the strongly compact closed category of density matrices between finite dimensional Hilbert spaces.

Density matrices model mixed quantum circuits.  The circuits in the image of the doubling functor are interpreted as the rays of pure quantum processes, unexposed to a classical system.

The canonical effect $X\to I$ in $\CPM(\FHilb)$ given by the counit of the compact closed structure is interpreted as the quantum discard map.
\end{example}


\begin{definition}
Define environment structure TODO
\end{definition}




\begin{definition}
Given a orthonormal basis in $\FHilb$, the projector onto this basis, is first given by copying and then by discarding, in the doubled picture:

DRAW MAP
\end{definition}



\begin{definition}
Define bastard spider
\end{definition}



\begin{lemma}
Bastard spider theorem
\end{lemma}

\begin{definition}
Splitting dagger Frobenius algebras
\end{definition}

\begin{lemma}
In $\FHilb$, the canonical maps to and from ???? are interpreted as preparing and measuring with respect to an orthonormal basis
\end{lemma}


\begin{remark}
Meauring and preparing strongly complementary observables preserves no information (actually we only need the hopf and not the bialgebra).
\end{remark}


\begin{example}
Splitting the projector onto the standard basis in phase-free qudit ZX calculus, this allows us to perform the standard quantum teleportation algorithm:


TODO
\end{example}
%Give example of qudit quantum teleportation

\begin{definition}
CPM infinity construction
\end{definition}

\begin{lemma}
CPM infinity of Hilb is infinite dimensional density matrices.
\end{lemma}



%ZH calculus 




\chapter{Boolean circuits as spans of finite sets}
\label{chap:zxa}


In this chapter we provide a complete set of identities for quantum circuits generated by $Z$ and $X$-spiders, the not gate and the {\sf and} gate. We show that this is a universal and complete presentation of $2^n \times 2^m$ dimensional matrices over $\N$; equivalently the subcategory of spans of finite sets where the objects are powers of two element sets. We show that this is the ``natural number phase'' Fragment of the ZH-calculus.


%The identities which are given by this two way translation are {\em almost} the union of the complete identities for Boolean functions \cite[Thm. 10]{lafont} (functions of type $\F_2^n \to \F_2$) and the identities for $\Span^\sim(\Mat(\F_2))$ \cite[Def. 5.1]{ihpub}.  These classes of circuits, and these identities for that matter, are nothing new; however, we provide a completeness result, as well as a structural account of how the full classical qubit fragment of $\FHilb$ can be obtained from adding discarding and codiscarding to the full classically reversible Boolean fragment.  In fact, some of these identities are presented in \cite[Chap. 5]{herrmann},  and they are used in the $\ZH$-calculus \cite{zh,zhpi}, as well as in some presentations of the $\ZX$-calculus with the triangle generator as a primitive \cite{munson2019note,ringZX}.  This is particularity unsurprising for the latter, \cite{ringZX}, where the author proves completeness of the $\ZX$-calculus over arbitrary semirings, which subsumes the completeness result herein.  Albeit, the presentation given here is substantially simpler.  It worth mentioning that $\ZXA$ is not a $\ZX*$-calculus in the sense of \cite{zxstar}, because the {\sf and} gate is not a spider.  $\ZXA$ should be instead though of as the ``classical fragment'' of the phase-free $\ZH$-calculus: retaining the monoid for ``and'' without $H$-boxes.   From this presentation only natural-number H-boxes can be derived.

The proof of the completeness theorem involves computing the Cartesian completion and ``cocartesian completion'' of the discrete inverse category $\TOF$ (generated by the Toffoli gate, the $\mathcal X$ gate,  $|0\rangle$ and $\langle 0|$) and then glueing them together.

To this end, in Section \ref{sec:cpm} we first reformulate the Cartesian completion of a discrete inverse category in terms of generators and relations.  We show that this can be presented by freely adjoining a counits to the inverse products of the base inverse category.   In the Cartesian completion of $\TOF$, this is interpreted as adjoining the generator $\sqrt 2 | + \rangle=(1,1)^T$.

In Section \ref{sec:ZXA}, we take the pushout of the unit and counit completion of $\TOF$, interpreted as adding the generators $2|+ \rangle$ and $2\langle + |$.  This yields the completeness result as well as a presentation which can be regarded as a fragment of the ZH-calculus.





\section{Cartesian completion as counit completion}
\label{sec:cpm}

In this section we prove that the  Cartesian completion of a discrete inverse category can be presented in terms of freely adding a counit to the inverse products of the base monoidal category.  At the end of this section we state this construction in more quantum terms.

\begin{lemma}
\label{lem:latching}

Given two parallel maps $X\xrightarrow{f,g} Y\otimes Z$ in a discrete inverse category:

$$
\begin{tikzpicture}
	\begin{pgfonlayer}{nodelayer}
		\node [style=map] (0) at (0, 1.25) {$f$};
		\node [style=none] (1) at (0, 0.5) {};
		\node [style=none] (2) at (-0.25, 2) {};
		\node [style=none] (3) at (0.25, 2) {};
	\end{pgfonlayer}
	\begin{pgfonlayer}{edgelayer}
		\draw (1.center) to (0);
		\draw [in=60, out=-90] (3.center) to (0);
		\draw [in=-90, out=120] (0) to (2.center);
	\end{pgfonlayer}
\end{tikzpicture}
=
\begin{tikzpicture}
	\begin{pgfonlayer}{nodelayer}
		\node [style=map] (0) at (0, 1.25) {$g$};
		\node [style=none] (1) at (0, 0.5) {};
		\node [style=none] (2) at (-0.25, 2) {};
		\node [style=none] (3) at (0.25, 2) {};
	\end{pgfonlayer}
	\begin{pgfonlayer}{edgelayer}
		\draw (1.center) to (0);
		\draw [in=60, out=-90] (3.center) to (0);
		\draw [in=-90, out=120] (0) to (2.center);
	\end{pgfonlayer}
\end{tikzpicture}
\iff
\begin{tikzpicture}
	\begin{pgfonlayer}{nodelayer}
		\node [style=map] (0) at (0, 1.25) {$f$};
		\node [style=Z] (1) at (-0.5, 2.25) {};
		\node [style=none] (2) at (-0.5, 3) {};
		\node [style=none] (3) at (0.25, 3) {};
		\node [style=none] (4) at (-0.75, 0.5) {};
		\node [style=none] (5) at (0, 0.5) {};
	\end{pgfonlayer}
	\begin{pgfonlayer}{edgelayer}
		\draw (5.center) to (0);
		\draw (0) to (1);
		\draw (1) to (2.center);
		\draw [in=75, out=-90] (3.center) to (0);
		\draw [in=90, out=-105] (1) to (4.center);
	\end{pgfonlayer}
\end{tikzpicture}
=
\begin{tikzpicture}
	\begin{pgfonlayer}{nodelayer}
		\node [style=map] (0) at (0, 1.25) {$g$};
		\node [style=Z] (1) at (-0.5, 2.25) {};
		\node [style=none] (2) at (-0.5, 3) {};
		\node [style=none] (3) at (0.25, 3) {};
		\node [style=none] (4) at (-0.75, 0.5) {};
		\node [style=none] (5) at (0, 0.5) {};
	\end{pgfonlayer}
	\begin{pgfonlayer}{edgelayer}
		\draw (5.center) to (0);
		\draw (0) to (1);
		\draw (1) to (2.center);
		\draw [in=75, out=-90] (3.center) to (0);
		\draw [in=90, out=-105] (1) to (4.center);
	\end{pgfonlayer}
\end{tikzpicture}
$$
\end{lemma}

\begin{proof}
The forward direction is trivial. For the converse, assume the right hand side holds.  Then:

\begin{align*}
\begin{tikzpicture}
	\begin{pgfonlayer}{nodelayer}
		\node [style=map] (0) at (0, 1.25) {$f$};
		\node [style=none] (1) at (0, 0.5) {};
		\node [style=none] (2) at (-0.25, 2) {};
		\node [style=none] (3) at (0.25, 2) {};
	\end{pgfonlayer}
	\begin{pgfonlayer}{edgelayer}
		\draw (1.center) to (0);
		\draw [in=60, out=-90] (3.center) to (0);
		\draw [in=-90, out=120] (0) to (2.center);
	\end{pgfonlayer}
\end{tikzpicture}
=
\begin{tikzpicture}
	\begin{pgfonlayer}{nodelayer}
		\node [style=map] (0) at (0, 1.25) {$f$};
		\node [style=none] (1) at (0, 0.5) {};
		\node [style=Z] (2) at (-0.25, 2) {};
		\node [style=Z] (3) at (0.25, 2) {};
		\node [style=Z] (4) at (0.25, 2.75) {};
		\node [style=Z] (5) at (-0.25, 2.75) {};
		\node [style=none] (6) at (-0.25, 3.25) {};
		\node [style=none] (7) at (0.25, 3.25) {};
	\end{pgfonlayer}
	\begin{pgfonlayer}{edgelayer}
		\draw (1.center) to (0);
		\draw [in=60, out=-90] (3) to (0);
		\draw [in=-90, out=120] (0) to (2);
		\draw (6.center) to (5);
		\draw (4) to (7.center);
		\draw [in=-60, out=60] (3) to (4);
		\draw [in=120, out=-120] (5) to (2);
		\draw [in=60, out=-60] (5) to (2);
		\draw [in=-120, out=120] (3) to (4);
	\end{pgfonlayer}
\end{tikzpicture}
=
\begin{tikzpicture}
	\begin{pgfonlayer}{nodelayer}
		\node [style=Z] (0) at (0, 1) {};
		\node [style=Z] (1) at (0.5, 2.75) {};
		\node [style=none] (2) at (-0.5, 3.25) {};
		\node [style=none] (3) at (0.5, 3.25) {};
		\node [style=none] (4) at (0, 0.5) {};
		\node [style=Z] (5) at (-0.5, 2.75) {};
		\node [style=map] (6) at (0.5, 1.75) {$f$};
		\node [style=map] (7) at (-0.5, 1.75) {$f$};
	\end{pgfonlayer}
	\begin{pgfonlayer}{edgelayer}
		\draw (2.center) to (5);
		\draw (1) to (3.center);
		\draw (4.center) to (0);
		\draw [in=-90, out=135] (0) to (7);
		\draw [in=45, out=-90] (6) to (0);
		\draw (7) to (1);
		\draw [in=120, out=-120] (5) to (7);
		\draw (6) to (5);
		\draw [in=60, out=-60] (1) to (6);
	\end{pgfonlayer}
\end{tikzpicture}
=
\begin{tikzpicture}
	\begin{pgfonlayer}{nodelayer}
		\node [style=Z] (0) at (1.25, 0.5) {};
		\node [style=Z] (1) at (1.5, 3.25) {};
		\node [style=none] (2) at (0.5, 3.75) {};
		\node [style=none] (3) at (1.5, 3.75) {};
		\node [style=none] (4) at (1.25, 0) {};
		\node [style=Z] (5) at (0.5, 3.25) {};
		\node [style=map] (6) at (1.5, 1.25) {$f$};
		\node [style=map] (7) at (1, 2.5) {$f$};
		\node [style=none] (8) at (0.5, 2.25) {};
	\end{pgfonlayer}
	\begin{pgfonlayer}{edgelayer}
		\draw (2.center) to (5);
		\draw (1) to (3.center);
		\draw (4.center) to (0);
		\draw [in=-90, out=135] (0) to (7);
		\draw [in=45, out=-90] (6) to (0);
		\draw [in=-124, out=60] (7) to (1);
		\draw [in=120, out=-60] (5) to (7);
		\draw [in=60, out=-60] (1) to (6);
		\draw [in=-90, out=120, looseness=0.75] (6) to (8.center);
		\draw [in=-90, out=90] (8.center) to (5);
	\end{pgfonlayer}
\end{tikzpicture}
=
\begin{tikzpicture}
	\begin{pgfonlayer}{nodelayer}
		\node [style=Z] (0) at (1.25, 0.5) {};
		\node [style=Z] (1) at (1.5, 3.25) {};
		\node [style=none] (2) at (0.5, 3.75) {};
		\node [style=none] (3) at (1.5, 3.75) {};
		\node [style=none] (4) at (1.25, 0) {};
		\node [style=Z] (5) at (0.5, 3.25) {};
		\node [style=map] (6) at (1.5, 1.25) {$f$};
		\node [style=map] (7) at (1, 2.5) {$g$};
		\node [style=none] (8) at (0.5, 2.25) {};
	\end{pgfonlayer}
	\begin{pgfonlayer}{edgelayer}
		\draw (2.center) to (5);
		\draw (1) to (3.center);
		\draw (4.center) to (0);
		\draw [in=-90, out=135] (0) to (7);
		\draw [in=45, out=-90] (6) to (0);
		\draw [in=-124, out=60] (7) to (1);
		\draw [in=120, out=-60] (5) to (7);
		\draw [in=60, out=-60] (1) to (6);
		\draw [in=-90, out=120, looseness=0.75] (6) to (8.center);
		\draw [in=-90, out=90] (8.center) to (5);
	\end{pgfonlayer}
\end{tikzpicture}
=
\begin{tikzpicture}
	\begin{pgfonlayer}{nodelayer}
		\node [style=Z] (0) at (0, 1) {};
		\node [style=Z] (1) at (0.5, 2.75) {};
		\node [style=none] (2) at (-0.5, 3.25) {};
		\node [style=none] (3) at (0.5, 3.25) {};
		\node [style=none] (4) at (0, 0.5) {};
		\node [style=Z] (5) at (-0.5, 2.75) {};
		\node [style=map] (6) at (0.5, 1.75) {$g$};
		\node [style=map] (7) at (-0.5, 1.75) {$g$};
	\end{pgfonlayer}
	\begin{pgfonlayer}{edgelayer}
		\draw (2.center) to (5);
		\draw (1) to (3.center);
		\draw (4.center) to (0);
		\draw [in=-90, out=135] (0) to (7);
		\draw [in=45, out=-90] (6) to (0);
		\draw (7) to (1);
		\draw [in=120, out=-120] (5) to (7);
		\draw (6) to (5);
		\draw [in=60, out=-60] (1) to (6);
	\end{pgfonlayer}
\end{tikzpicture}
=
\begin{tikzpicture}
	\begin{pgfonlayer}{nodelayer}
		\node [style=map] (0) at (0, 1.25) {$g$};
		\node [style=none] (1) at (0, 0.5) {};
		\node [style=none] (2) at (-0.25, 2) {};
		\node [style=none] (3) at (0.25, 2) {};
	\end{pgfonlayer}
	\begin{pgfonlayer}{edgelayer}
		\draw (1.center) to (0);
		\draw [in=60, out=-90] (3.center) to (0);
		\draw [in=-90, out=120] (0) to (2.center);
	\end{pgfonlayer}
\end{tikzpicture}
\end{align*}
\end{proof}





\begin{lemma}
\label{theorem:cpstartheorem}
Given two maps $X \xrightarrow{f} Y\otimes S$ and $X \xrightarrow{g} Y\otimes T$, in a discrete inverse category:

\begin{align*}
\begin{tikzpicture}
	\begin{pgfonlayer}{nodelayer}
		\node [style=map] (0) at (0.5, 1.75) {$g$};
		\node [style=none] (1) at (0.5, 1) {};
		\node [style=map] (2) at (0.5, 3.25) {$g^\circ$};
		\node [style=map] (3) at (0.5, 4) {$f$};
		\node [style=Z] (4) at (0.25, 2.5) {};
		\node [style=Z] (5) at (0.25, 4.75) {};
		\node [style=none] (6) at (0.25, 5.25) {};
		\node [style=none] (7) at (0.75, 5.25) {};
	\end{pgfonlayer}
	\begin{pgfonlayer}{edgelayer}
		\draw (1.center) to (0);
		\draw [in=-90, out=124] (0) to (4);
		\draw (4) to (2);
		\draw [in=60, out=-60] (2) to (0);
		\draw [in=-120, out=120] (4) to (5);
		\draw (5) to (6.center);
		\draw [in=60, out=-90] (7.center) to (3);
		\draw (3) to (5);
		\draw (3) to (2);
	\end{pgfonlayer}
\end{tikzpicture}
=
\begin{tikzpicture}
	\begin{pgfonlayer}{nodelayer}
		\node [style=map] (0) at (0.5, 1.75) {$f$};
		\node [style=none] (1) at (0.5, 1) {};
		\node [style=none] (2) at (0.25, 2.5) {};
		\node [style=none] (3) at (0.75, 2.5) {};
	\end{pgfonlayer}
	\begin{pgfonlayer}{edgelayer}
		\draw (1.center) to (0);
		\draw [in=-90, out=60] (0) to (3.center);
		\draw [in=120, out=-90] (2.center) to (0);
	\end{pgfonlayer}
\end{tikzpicture}
\iff
\begin{tikzpicture}
	\begin{pgfonlayer}{nodelayer}
		\node [style=Z] (0) at (0, 2.25) {};
		\node [style=Z] (1) at (0, 3) {};
		\node [style=map] (2) at (0.5, 1.75) {$f$};
		\node [style=map] (3) at (0.5, 3.5) {$f^\circ$};
		\node [style=none] (4) at (-0.25, 1) {};
		\node [style=none] (5) at (0.5, 1) {};
		\node [style=none] (6) at (-0.25, 4.25) {};
		\node [style=none] (7) at (0.5, 4.25) {};
	\end{pgfonlayer}
	\begin{pgfonlayer}{edgelayer}
		\draw (5.center) to (2);
		\draw (2) to (0);
		\draw [in=90, out=-101] (0) to (4.center);
		\draw (0) to (1);
		\draw [in=-90, out=101] (1) to (6.center);
		\draw (7.center) to (3);
		\draw (3) to (1);
		\draw [in=-75, out=75] (2) to (3);
	\end{pgfonlayer}
\end{tikzpicture}
=
\begin{tikzpicture}
	\begin{pgfonlayer}{nodelayer}
		\node [style=Z] (0) at (0, 2.25) {};
		\node [style=Z] (1) at (0, 3) {};
		\node [style=map] (2) at (0.5, 1.75) {$g$};
		\node [style=map] (3) at (0.5, 3.5) {$g^\circ$};
		\node [style=none] (4) at (-0.25, 1) {};
		\node [style=none] (5) at (0.5, 1) {};
		\node [style=none] (6) at (-0.25, 4.25) {};
		\node [style=none] (7) at (0.5, 4.25) {};
	\end{pgfonlayer}
	\begin{pgfonlayer}{edgelayer}
		\draw (5.center) to (2);
		\draw (2) to (0);
		\draw [in=90, out=-101] (0) to (4.center);
		\draw (0) to (1);
		\draw [in=-90, out=101] (1) to (6.center);
		\draw (7.center) to (3);
		\draw (3) to (1);
		\draw [in=-75, out=75] (2) to (3);
	\end{pgfonlayer}
\end{tikzpicture}
\iff
\begin{tikzpicture}
	\begin{pgfonlayer}{nodelayer}
		\node [style=Z] (0) at (0, 2.25) {};
		\node [style=Z] (1) at (0, 3) {};
		\node [style=map] (2) at (0.5, 1.75) {$f$};
		\node [style=map] (3) at (0.5, 3.5) {$f^\circ$};
		\node [style=none] (4) at (-0.25, 0.75) {};
		\node [style=none] (5) at (-0.25, 4.5) {};
		\node [style=Z] (6) at (1, 1.25) {};
		\node [style=Z] (7) at (1, 4) {};
		\node [style=none] (8) at (1, 4.5) {};
		\node [style=none] (9) at (1, 0.75) {};
	\end{pgfonlayer}
	\begin{pgfonlayer}{edgelayer}
		\draw (2) to (0);
		\draw [in=90, out=-101] (0) to (4.center);
		\draw (0) to (1);
		\draw [in=-90, out=101] (1) to (5.center);
		\draw (3) to (1);
		\draw [in=-75, out=75] (2) to (3);
		\draw (6) to (2);
		\draw [bend right=15] (6) to (7);
		\draw (7) to (3);
		\draw (7) to (8.center);
		\draw (6) to (9.center);
	\end{pgfonlayer}
\end{tikzpicture}
=
\begin{tikzpicture}
	\begin{pgfonlayer}{nodelayer}
		\node [style=Z] (0) at (0, 2.25) {};
		\node [style=Z] (1) at (0, 3) {};
		\node [style=map] (2) at (0.5, 1.75) {$g$};
		\node [style=map] (3) at (0.5, 3.5) {$g^\circ$};
		\node [style=none] (4) at (-0.25, 0.75) {};
		\node [style=none] (5) at (-0.25, 4.5) {};
		\node [style=Z] (6) at (1, 1.25) {};
		\node [style=Z] (7) at (1, 4) {};
		\node [style=none] (8) at (1, 4.5) {};
		\node [style=none] (9) at (1, 0.75) {};
	\end{pgfonlayer}
	\begin{pgfonlayer}{edgelayer}
		\draw (2) to (0);
		\draw [in=90, out=-101] (0) to (4.center);
		\draw (0) to (1);
		\draw [in=-90, out=101] (1) to (5.center);
		\draw (3) to (1);
		\draw [in=-75, out=75] (2) to (3);
		\draw (6) to (2);
		\draw [bend right=15] (6) to (7);
		\draw (7) to (3);
		\draw (7) to (8.center);
		\draw (6) to (9.center);
	\end{pgfonlayer}
\end{tikzpicture}
\end{align*}
\end{lemma}


\begin{proof}
First note:

\begin{align*}
\begin{tikzpicture}
	\begin{pgfonlayer}{nodelayer}
		\node [style=Z] (0) at (0, 2.25) {};
		\node [style=Z] (1) at (0, 3) {};
		\node [style=map] (2) at (0.5, 1.75) {$f$};
		\node [style=map] (3) at (0.5, 3.5) {$f^\circ$};
		\node [style=none] (4) at (-0.25, 1.25) {};
		\node [style=none] (5) at (-0.25, 4) {};
		\node [style=none] (6) at (0.5, 4) {};
		\node [style=none] (7) at (0.5, 1.25) {};
	\end{pgfonlayer}
	\begin{pgfonlayer}{edgelayer}
		\draw (2) to (0);
		\draw [in=90, out=-101] (0) to (4.center);
		\draw (0) to (1);
		\draw [in=-90, out=101] (1) to (5.center);
		\draw (3) to (1);
		\draw [in=-75, out=75] (2) to (3);
		\draw (2) to (7.center);
		\draw (3) to (6.center);
	\end{pgfonlayer}
\end{tikzpicture}
=
\begin{tikzpicture}
	\begin{pgfonlayer}{nodelayer}
		\node [style=Z] (0) at (0, 2.25) {};
		\node [style=Z] (1) at (0, 3) {};
		\node [style=map] (2) at (0.5, 1) {$f$};
		\node [style=map] (3) at (0.5, 4.25) {$f^\circ$};
		\node [style=none] (4) at (-0.25, 0.5) {};
		\node [style=none] (5) at (-0.25, 4.75) {};
		\node [style=none] (6) at (0.5, 4.75) {};
		\node [style=none] (7) at (0.5, 0.5) {};
		\node [style=Z] (8) at (0.25, 1.75) {};
		\node [style=Z] (9) at (0.25, 3.5) {};
	\end{pgfonlayer}
	\begin{pgfonlayer}{edgelayer}
		\draw [in=90, out=-101] (0) to (4.center);
		\draw (0) to (1);
		\draw [in=-90, out=101] (1) to (5.center);
		\draw [in=-75, out=75] (2) to (3);
		\draw (2) to (7.center);
		\draw (3) to (6.center);
		\draw (2) to (8);
		\draw (8) to (0);
		\draw [bend right=15] (8) to (9);
		\draw (9) to (3);
		\draw (9) to (1);
	\end{pgfonlayer}
\end{tikzpicture}
=
\begin{tikzpicture}
	\begin{pgfonlayer}{nodelayer}
		\node [style=Z] (0) at (0, 2.25) {};
		\node [style=Z] (1) at (0, 3) {};
		\node [style=map] (2) at (0.5, 1) {$f$};
		\node [style=map] (3) at (0.5, 4.25) {$f^\circ$};
		\node [style=none] (4) at (-0.25, 0.5) {};
		\node [style=none] (5) at (-0.25, 4.75) {};
		\node [style=none] (6) at (0.5, 4.75) {};
		\node [style=none] (7) at (0.5, 0.5) {};
		\node [style=Z] (8) at (0.25, 1.75) {};
		\node [style=Z] (9) at (0.25, 3.5) {};
		\node [style=Z] (10) at (0.75, 1.75) {};
		\node [style=Z] (11) at (0.75, 3.5) {};
	\end{pgfonlayer}
	\begin{pgfonlayer}{edgelayer}
		\draw [in=90, out=-101] (0) to (4.center);
		\draw (0) to (1);
		\draw [in=-90, out=101] (1) to (5.center);
		\draw (2) to (7.center);
		\draw (3) to (6.center);
		\draw (2) to (8);
		\draw (8) to (0);
		\draw [bend right=15] (8) to (9);
		\draw (9) to (3);
		\draw (9) to (1);
		\draw (2) to (10);
		\draw [in=-135, out=135, looseness=1.25] (10) to (11);
		\draw (11) to (3);
		\draw [in=75, out=-75] (11) to (10);
	\end{pgfonlayer}
\end{tikzpicture}
=
\begin{tikzpicture}
	\begin{pgfonlayer}{nodelayer}
		\node [style=Z] (0) at (1.5, 2.25) {};
		\node [style=none] (1) at (1.25, 0.5) {};
		\node [style=none] (2) at (2.5, 0.5) {};
		\node [style=none] (3) at (1.25, 4.75) {};
		\node [style=Z] (4) at (1.5, 3) {};
		\node [style=none] (5) at (2.5, 4.75) {};
		\node [style=map] (6) at (2, 1.75) {$f$};
		\node [style=map] (7) at (2.75, 1.75) {$f$};
		\node [style=map] (8) at (2, 3.5) {$f^\circ$};
		\node [style=map] (9) at (2.75, 3.5) {$f^\circ$};
		\node [style=Z] (10) at (2.5, 1) {};
		\node [style=Z] (11) at (2.5, 4.25) {};
	\end{pgfonlayer}
	\begin{pgfonlayer}{edgelayer}
		\draw [in=90, out=-101] (0) to (1.center);
		\draw (0) to (4);
		\draw [in=-90, out=101] (4) to (3.center);
		\draw (2.center) to (10);
		\draw (10) to (6);
		\draw (6) to (0);
		\draw (8) to (4);
		\draw (8) to (11);
		\draw (11) to (9);
		\draw [in=120, out=-120] (9) to (7);
		\draw (7) to (10);
		\draw [in=-60, out=60] (7) to (9);
		\draw [in=75, out=-75] (8) to (6);
		\draw (5.center) to (11);
	\end{pgfonlayer}
\end{tikzpicture}
=
\begin{tikzpicture}
	\begin{pgfonlayer}{nodelayer}
		\node [style=Z] (0) at (1.5, 3.75) {};
		\node [style=none] (1) at (1.25, 0.5) {};
		\node [style=none] (2) at (2.5, 0.5) {};
		\node [style=none] (3) at (1.25, 6.25) {};
		\node [style=Z] (4) at (1.5, 4.5) {};
		\node [style=none] (5) at (2.5, 6.25) {};
		\node [style=map] (6) at (2, 3.25) {$f$};
		\node [style=map] (7) at (2, 5) {$f^\circ$};
		\node [style=Z] (8) at (2.5, 1) {};
		\node [style=Z] (9) at (2.5, 5.75) {};
		\node [style=map] (10) at (2, 2.5) {$f^\circ$};
		\node [style=map] (11) at (2, 1.75) {$f$};
	\end{pgfonlayer}
	\begin{pgfonlayer}{edgelayer}
		\draw [in=90, out=-101] (0) to (1.center);
		\draw (0) to (4);
		\draw [in=-90, out=101] (4) to (3.center);
		\draw (2.center) to (8);
		\draw (6) to (0);
		\draw (7) to (4);
		\draw (7) to (9);
		\draw [in=75, out=-75] (7) to (6);
		\draw [in=120, out=-120] (10) to (11);
		\draw [in=-60, out=60] (11) to (10);
		\draw (8) to (11);
		\draw [in=-75, out=75, looseness=0.75] (8) to (9);
		\draw (9) to (5.center);
		\draw (6) to (10);
	\end{pgfonlayer}
\end{tikzpicture}
=
\begin{tikzpicture}
	\begin{pgfonlayer}{nodelayer}
		\node [style=Z] (0) at (1.5, 2.25) {};
		\node [style=none] (1) at (1.25, 0.5) {};
		\node [style=none] (2) at (2.5, 0.5) {};
		\node [style=none] (3) at (1.25, 4.75) {};
		\node [style=Z] (4) at (1.5, 3) {};
		\node [style=none] (5) at (2.5, 4.75) {};
		\node [style=map] (6) at (2, 1.75) {$f$};
		\node [style=map] (7) at (2, 3.5) {$f^\circ$};
		\node [style=Z] (8) at (2.5, 1) {};
		\node [style=Z] (9) at (2.5, 4.25) {};
	\end{pgfonlayer}
	\begin{pgfonlayer}{edgelayer}
		\draw [in=90, out=-101] (0) to (1.center);
		\draw (0) to (4);
		\draw [in=-90, out=101] (4) to (3.center);
		\draw (2.center) to (8);
		\draw (6) to (0);
		\draw (7) to (4);
		\draw (7) to (9);
		\draw [in=75, out=-75] (7) to (6);
		\draw [in=-75, out=75, looseness=0.75] (8) to (9);
		\draw (9) to (5.center);
		\draw (8) to (6);
	\end{pgfonlayer}
\end{tikzpicture}
\end{align*}

So that we only have to prove the first biconditional.
Suppose that the left hand side holds, then:

\begin{align*}
\begin{tikzpicture}
	\begin{pgfonlayer}{nodelayer}
		\node [style=Z] (0) at (0, 2.25) {};
		\node [style=Z] (1) at (0, 3) {};
		\node [style=map] (2) at (0.5, 1.75) {$f$};
		\node [style=map] (3) at (0.5, 3.5) {$f^\circ$};
		\node [style=none] (4) at (-0.25, 1) {};
		\node [style=none] (5) at (0.5, 1) {};
		\node [style=none] (6) at (-0.25, 4.25) {};
		\node [style=none] (7) at (0.5, 4.25) {};
	\end{pgfonlayer}
	\begin{pgfonlayer}{edgelayer}
		\draw (5.center) to (2);
		\draw (2) to (0);
		\draw [in=90, out=-101] (0) to (4.center);
		\draw (0) to (1);
		\draw [in=-90, out=101] (1) to (6.center);
		\draw (7.center) to (3);
		\draw (3) to (1);
		\draw [in=-75, out=75] (2) to (3);
	\end{pgfonlayer}
\end{tikzpicture}
=
\begin{tikzpicture}
	\begin{pgfonlayer}{nodelayer}
		\node [style=map] (0) at (0.5, 1.75) {$g$};
		\node [style=none] (1) at (0.5, 1) {};
		\node [style=map] (2) at (0.5, 3.25) {$g^\circ$};
		\node [style=map] (3) at (0.5, 4) {$f$};
		\node [style=Z] (4) at (0.25, 2.5) {};
		\node [style=Z] (5) at (0.25, 4.75) {};
		\node [style=map] (6) at (0.5, 9.75) {$g^\circ$};
		\node [style=map] (7) at (0.5, 8.25) {$g$};
		\node [style=map] (8) at (0.5, 7.5) {$f^\circ$};
		\node [style=Z] (9) at (0.25, 9) {};
		\node [style=Z] (10) at (0.25, 6.75) {};
		\node [style=none] (11) at (0.5, 10.5) {};
		\node [style=Z] (12) at (0, 5.5) {};
		\node [style=Z] (13) at (0, 6) {};
		\node [style=none] (14) at (-0.25, 1) {};
		\node [style=none] (15) at (-0.25, 10.5) {};
		\node [style=none] (16) at (-0.25, 4.75) {};
		\node [style=none] (17) at (-0.25, 6.75) {};
	\end{pgfonlayer}
	\begin{pgfonlayer}{edgelayer}
		\draw (1.center) to (0);
		\draw [in=-90, out=124] (0) to (4);
		\draw (4) to (2);
		\draw [in=60, out=-60] (2) to (0);
		\draw [in=-120, out=120] (4) to (5);
		\draw (3) to (5);
		\draw (3) to (2);
		\draw (11.center) to (6);
		\draw [in=90, out=-124] (6) to (9);
		\draw (9) to (7);
		\draw [in=-60, out=60] (7) to (6);
		\draw [in=120, out=-120] (9) to (10);
		\draw (8) to (10);
		\draw (8) to (7);
		\draw [in=-75, out=75, looseness=0.75] (3) to (8);
		\draw [in=72, out=-90] (10) to (13);
		\draw (13) to (12);
		\draw [in=90, out=-72] (12) to (5);
		\draw [in=90, out=-108] (12) to (16.center);
		\draw (16.center) to (14.center);
		\draw [in=-90, out=108] (13) to (17.center);
		\draw (17.center) to (15.center);
	\end{pgfonlayer}
\end{tikzpicture}
=
\begin{tikzpicture}
	\begin{pgfonlayer}{nodelayer}
		\node [style=map] (0) at (2.75, 4.25) {$f$};
		\node [style=none] (1) at (2.25, 10.25) {};
		\node [style=none] (2) at (2.25, 0.5) {};
		\node [style=none] (3) at (2.75, 0.5) {};
		\node [style=Z] (4) at (2.5, 7.25) {};
		\node [style=Z] (5) at (2.5, 5) {};
		\node [style=none] (6) at (2.75, 10.25) {};
		\node [style=Z] (7) at (2.5, 2.75) {};
		\node [style=map] (8) at (2.75, 5.75) {$f^\circ$};
		\node [style=map] (9) at (2.75, 3.5) {$g^\circ$};
		\node [style=map] (10) at (2.75, 6.5) {$g$};
		\node [style=map] (11) at (2.75, 1.25) {$g$};
		\node [style=map] (12) at (2.75, 9.5) {$g^\circ$};
		\node [style=Z] (13) at (2.5, 8.75) {};
		\node [style=Z] (14) at (2.5, 2) {};
		\node [style=Z] (15) at (2.5, 8) {};
	\end{pgfonlayer}
	\begin{pgfonlayer}{edgelayer}
		\draw (3.center) to (11);
		\draw (7) to (9);
		\draw [in=60, out=-60] (9) to (11);
		\draw (0) to (9);
		\draw (6.center) to (12);
		\draw (4) to (10);
		\draw [in=-60, out=60] (10) to (12);
		\draw [in=120, out=-120] (4) to (5);
		\draw (8) to (5);
		\draw (8) to (10);
		\draw [in=-60, out=60] (0) to (8);
		\draw [in=-99, out=90] (2.center) to (14);
		\draw (14) to (7);
		\draw (14) to (11);
		\draw (4) to (13);
		\draw (13) to (12);
		\draw [in=-90, out=99] (13) to (1.center);
		\draw [in=-90, out=114] (0) to (5);
		\draw [bend left=45, looseness=0.50] (7) to (15);
	\end{pgfonlayer}
\end{tikzpicture}
=
\begin{tikzpicture}
	\begin{pgfonlayer}{nodelayer}
		\node [style=Z] (0) at (-0.5, 5.75) {};
		\node [style=none] (1) at (-0.75, 7.25) {};
		\node [style=map] (2) at (-0.25, 4.25) {$g$};
		\node [style=map] (3) at (-0.25, 1.25) {$g$};
		\node [style=none] (4) at (-0.75, 0.5) {};
		\node [style=none] (5) at (-0.25, 0.5) {};
		\node [style=none] (6) at (-0.25, 7.25) {};
		\node [style=map] (7) at (-0.25, 3.5) {$g^\circ$};
		\node [style=Z] (8) at (-0.5, 2) {};
		\node [style=map] (9) at (-0.25, 6.5) {$g^\circ$};
		\node [style=Z] (10) at (-0.5, 5) {};
		\node [style=Z] (11) at (-0.5, 2.75) {};
	\end{pgfonlayer}
	\begin{pgfonlayer}{edgelayer}
		\draw (5.center) to (3);
		\draw (11) to (7);
		\draw [in=60, out=-60] (7) to (3);
		\draw (6.center) to (9);
		\draw [in=-60, out=60] (2) to (9);
		\draw [in=-99, out=90] (4.center) to (8);
		\draw (8) to (11);
		\draw (8) to (3);
		\draw (0) to (9);
		\draw [in=-90, out=99] (0) to (1.center);
		\draw (10) to (2);
		\draw (10) to (0);
		\draw (2) to (7);
		\draw [in=-120, out=120] (11) to (10);
	\end{pgfonlayer}
\end{tikzpicture}
=
\begin{tikzpicture}
	\begin{pgfonlayer}{nodelayer}
		\node [style=map] (0) at (2.75, 8) {$g^\circ$};
		\node [style=none] (1) at (2.75, 0.5) {};
		\node [style=Z] (2) at (2.25, 7.25) {};
		\node [style=Z] (3) at (2.25, 2) {};
		\node [style=none] (4) at (2, 0.5) {};
		\node [style=none] (5) at (2, 8.75) {};
		\node [style=map] (6) at (2.75, 1.25) {$g$};
		\node [style=Z] (7) at (2.25, 2.75) {};
		\node [style=none] (8) at (2.75, 8.75) {};
		\node [style=Z] (9) at (2.25, 6.5) {};
		\node [style=map] (10) at (3, 5) {$g$};
		\node [style=map] (11) at (3, 4.25) {$g^\circ$};
		\node [style=Z] (12) at (2.5, 3.25) {};
		\node [style=Z] (13) at (3, 3.25) {};
		\node [style=Z] (14) at (2.5, 6) {};
		\node [style=Z] (15) at (3, 6) {};
	\end{pgfonlayer}
	\begin{pgfonlayer}{edgelayer}
		\draw (1.center) to (6);
		\draw (8.center) to (0);
		\draw [in=-99, out=90] (4.center) to (3);
		\draw (3) to (7);
		\draw (3) to (6);
		\draw (2) to (0);
		\draw [in=-90, out=99] (2) to (5.center);
		\draw (9) to (2);
		\draw [in=-120, out=120] (7) to (9);
		\draw (10) to (11);
		\draw [in=-90, out=76] (6) to (13);
		\draw (7) to (12);
		\draw [bend left] (12) to (14);
		\draw [bend right, looseness=1.25] (15) to (13);
		\draw (14) to (9);
		\draw [in=-76, out=90] (15) to (0);
		\draw (12) to (11);
		\draw (13) to (11);
		\draw (10) to (14);
		\draw [in=90, out=-90] (15) to (10);
	\end{pgfonlayer}
\end{tikzpicture}
=
\begin{tikzpicture}
	\begin{pgfonlayer}{nodelayer}
		\node [style=map] (0) at (3, 7) {$g^\circ$};
		\node [style=none] (1) at (3, 0.5) {};
		\node [style=Z] (2) at (2.5, 6.25) {};
		\node [style=Z] (3) at (2.5, 2) {};
		\node [style=none] (4) at (2.25, 0.5) {};
		\node [style=none] (5) at (2.25, 7.75) {};
		\node [style=map] (6) at (3, 1.25) {$g$};
		\node [style=none] (7) at (3, 7.75) {};
		\node [style=map] (8) at (3, 4.5) {$g$};
		\node [style=map] (9) at (3, 3.75) {$g^\circ$};
		\node [style=Z] (10) at (2.5, 2.75) {};
		\node [style=Z] (11) at (3, 2.75) {};
		\node [style=Z] (12) at (2.5, 5.5) {};
		\node [style=Z] (13) at (3, 5.5) {};
	\end{pgfonlayer}
	\begin{pgfonlayer}{edgelayer}
		\draw (1.center) to (6);
		\draw (7.center) to (0);
		\draw [in=-99, out=90] (4.center) to (3);
		\draw (3) to (6);
		\draw (2) to (0);
		\draw [in=-90, out=99] (2) to (5.center);
		\draw (8) to (9);
		\draw (6) to (11);
		\draw [bend left] (10) to (12);
		\draw [bend right, looseness=1.25] (13) to (11);
		\draw (13) to (0);
		\draw (10) to (9);
		\draw (11) to (9);
		\draw (8) to (12);
		\draw [in=90, out=-90] (13) to (8);
		\draw (3) to (10);
		\draw (12) to (2);
	\end{pgfonlayer}
\end{tikzpicture}
=
\begin{tikzpicture}
	\begin{pgfonlayer}{nodelayer}
		\node [style=map] (0) at (0, 7.25) {$g^\circ$};
		\node [style=map] (1) at (0.25, 4) {$g$};
		\node [style=Z] (2) at (-0.5, 6.5) {};
		\node [style=none] (3) at (-0.75, 8) {};
		\node [style=map] (4) at (0, 1.25) {$g$};
		\node [style=none] (5) at (-0.75, 5) {};
		\node [style=Z] (6) at (-0.25, 5) {};
		\node [style=Z] (7) at (-0.25, 2.25) {};
		\node [style=none] (8) at (0, 8) {};
		\node [style=Z] (9) at (-0.5, 5.75) {};
		\node [style=Z] (10) at (0.25, 5) {};
		\node [style=none] (11) at (0, 0.5) {};
		\node [style=map] (12) at (0.25, 3.25) {$g^\circ$};
		\node [style=Z] (13) at (0.25, 2.25) {};
		\node [style=none] (14) at (-0.75, 0.5) {};
	\end{pgfonlayer}
	\begin{pgfonlayer}{edgelayer}
		\draw (11.center) to (4);
		\draw (8.center) to (0);
		\draw (2) to (0);
		\draw [in=-90, out=99] (2) to (3.center);
		\draw (9) to (2);
		\draw (1) to (12);
		\draw [in=-90, out=76] (4) to (13);
		\draw [bend left] (7) to (6);
		\draw [bend right, looseness=1.25] (10) to (13);
		\draw [in=-60, out=90] (6) to (9);
		\draw [in=-76, out=90] (10) to (0);
		\draw (7) to (12);
		\draw (13) to (12);
		\draw (1) to (6);
		\draw [in=90, out=-90] (10) to (1);
		\draw [in=104, out=-90] (7) to (4);
		\draw [in=90, out=-120] (9) to (5.center);
		\draw (14.center) to (5.center);
	\end{pgfonlayer}
\end{tikzpicture}
=
\begin{tikzpicture}
	\begin{pgfonlayer}{nodelayer}
		\node [style=map] (0) at (0, 5.25) {$g^\circ$};
		\node [style=Z] (1) at (-0.5, 4.5) {};
		\node [style=none] (2) at (-0.75, 6) {};
		\node [style=map] (3) at (0, 1.5) {$g$};
		\node [style=none] (4) at (-0.75, 3) {};
		\node [style=none] (5) at (0, 6) {};
		\node [style=Z] (6) at (-0.5, 3.75) {};
		\node [style=none] (7) at (0, 1) {};
		\node [style=none] (8) at (-0.75, 1) {};
		\node [style=map] (9) at (0, 2.25) {$g^\circ$};
		\node [style=map] (10) at (0, 3) {$g$};
	\end{pgfonlayer}
	\begin{pgfonlayer}{edgelayer}
		\draw (7.center) to (3);
		\draw (5.center) to (0);
		\draw (1) to (0);
		\draw [in=-90, out=99] (1) to (2.center);
		\draw (6) to (1);
		\draw [in=90, out=-120] (6) to (4.center);
		\draw (8.center) to (4.center);
		\draw [in=-120, out=120] (3) to (9);
		\draw [in=60, out=-60] (9) to (3);
		\draw (9) to (10);
		\draw (10) to (6);
		\draw [in=75, out=-75] (0) to (10);
	\end{pgfonlayer}
\end{tikzpicture}
=
\begin{tikzpicture}
	\begin{pgfonlayer}{nodelayer}
		\node [style=Z] (0) at (0, 2.25) {};
		\node [style=Z] (1) at (0, 3) {};
		\node [style=map] (2) at (0.5, 1.75) {$g$};
		\node [style=map] (3) at (0.5, 3.5) {$g^\circ$};
		\node [style=none] (4) at (-0.25, 1) {};
		\node [style=none] (5) at (0.5, 1) {};
		\node [style=none] (6) at (-0.25, 4.25) {};
		\node [style=none] (7) at (0.5, 4.25) {};
	\end{pgfonlayer}
	\begin{pgfonlayer}{edgelayer}
		\draw (5.center) to (2);
		\draw (2) to (0);
		\draw [in=90, out=-101] (0) to (4.center);
		\draw (0) to (1);
		\draw [in=-90, out=101] (1) to (6.center);
		\draw (7.center) to (3);
		\draw (3) to (1);
		\draw [in=-75, out=75] (2) to (3);
	\end{pgfonlayer}
\end{tikzpicture}
\end{align*}

Conversely, suppose that the right hand side holds.  Then:

\begin{align*}
\begin{tikzpicture}
	\begin{pgfonlayer}{nodelayer}
		\node [style=map] (0) at (0.5, 1.75) {$g$};
		\node [style=none] (1) at (0.5, 1) {};
		\node [style=map] (2) at (0.5, 3.25) {$g^\circ$};
		\node [style=map] (3) at (0.5, 4) {$f$};
		\node [style=Z] (4) at (0.25, 2.5) {};
		\node [style=none] (5) at (0.75, 5.75) {};
		\node [style=none] (6) at (0.25, 5.75) {};
		\node [style=Z] (7) at (0.25, 4.75) {};
		\node [style=Z] (8) at (0.25, 5.25) {};
		\node [style=none] (9) at (-0.25, 4.25) {};
		\node [style=none] (10) at (-0.25, 1) {};
	\end{pgfonlayer}
	\begin{pgfonlayer}{edgelayer}
		\draw (1.center) to (0);
		\draw [in=-90, out=124] (0) to (4);
		\draw (4) to (2);
		\draw [in=60, out=-60] (2) to (0);
		\draw [in=60, out=-90] (5.center) to (3);
		\draw (3) to (2);
		\draw [in=-120, out=120] (4) to (7);
		\draw (3) to (7);
		\draw (10.center) to (9.center);
		\draw [in=-135, out=90] (9.center) to (8);
		\draw (8) to (6.center);
		\draw (8) to (7);
	\end{pgfonlayer}
\end{tikzpicture}
=
\begin{tikzpicture}
	\begin{pgfonlayer}{nodelayer}
		\node [style=map] (0) at (0.5, 1.25) {$g$};
		\node [style=none] (1) at (0.5, 0.5) {};
		\node [style=map] (2) at (0.5, 3.25) {$g^\circ$};
		\node [style=map] (3) at (0.5, 4) {$f$};
		\node [style=Z] (4) at (0.25, 2.5) {};
		\node [style=none] (5) at (0.75, 5.25) {};
		\node [style=none] (6) at (0.25, 5.25) {};
		\node [style=Z] (7) at (0.25, 4.75) {};
		\node [style=none] (8) at (0, 0.5) {};
		\node [style=Z] (9) at (0.25, 2) {};
	\end{pgfonlayer}
	\begin{pgfonlayer}{edgelayer}
		\draw (1.center) to (0);
		\draw (4) to (2);
		\draw [in=60, out=-60] (2) to (0);
		\draw [in=60, out=-90] (5.center) to (3);
		\draw (3) to (2);
		\draw [in=-120, out=120] (4) to (7);
		\draw (3) to (7);
		\draw (9) to (0);
		\draw (9) to (4);
		\draw [in=90, out=-120] (9) to (8.center);
		\draw (7) to (6.center);
	\end{pgfonlayer}
\end{tikzpicture}
=
\begin{tikzpicture}
	\begin{pgfonlayer}{nodelayer}
		\node [style=map] (0) at (0.5, 1.25) {$f$};
		\node [style=none] (1) at (0.5, 0.5) {};
		\node [style=map] (2) at (0.5, 3.25) {$f^\circ$};
		\node [style=map] (3) at (0.5, 4) {$f$};
		\node [style=Z] (4) at (0.25, 2.5) {};
		\node [style=none] (5) at (0.75, 5.25) {};
		\node [style=none] (6) at (0.25, 5.25) {};
		\node [style=Z] (7) at (0.25, 4.75) {};
		\node [style=none] (8) at (0, 0.5) {};
		\node [style=Z] (9) at (0.25, 2) {};
	\end{pgfonlayer}
	\begin{pgfonlayer}{edgelayer}
		\draw (1.center) to (0);
		\draw (4) to (2);
		\draw [in=60, out=-60] (2) to (0);
		\draw [in=60, out=-90] (5.center) to (3);
		\draw (3) to (2);
		\draw [in=-120, out=120] (4) to (7);
		\draw (3) to (7);
		\draw (9) to (0);
		\draw (9) to (4);
		\draw [in=90, out=-120] (9) to (8.center);
		\draw (7) to (6.center);
	\end{pgfonlayer}
\end{tikzpicture}
=
\begin{tikzpicture}
	\begin{pgfonlayer}{nodelayer}
		\node [style=map] (0) at (2.5, 1.25) {$f$};
		\node [style=Z] (1) at (2, 6.25) {};
		\node [style=none] (2) at (1.75, 0.5) {};
		\node [style=none] (3) at (2.75, 6.75) {};
		\node [style=none] (4) at (2.5, 0.5) {};
		\node [style=Z] (5) at (2, 2.5) {};
		\node [style=Z] (6) at (2, 2) {};
		\node [style=none] (7) at (2, 6.75) {};
		\node [style=map] (8) at (2.75, 4.75) {$f$};
		\node [style=map] (9) at (2.75, 4) {$f^\circ$};
		\node [style=Z] (10) at (2.25, 5.75) {};
		\node [style=Z] (11) at (2.25, 3) {};
		\node [style=Z] (12) at (2.75, 5.75) {};
		\node [style=Z] (13) at (2.75, 3) {};
	\end{pgfonlayer}
	\begin{pgfonlayer}{edgelayer}
		\draw (4.center) to (0);
		\draw [in=-120, out=120] (5) to (1);
		\draw (6) to (0);
		\draw (6) to (5);
		\draw [in=90, out=-120] (6) to (2.center);
		\draw (1) to (7.center);
		\draw (8) to (9);
		\draw (11) to (5);
		\draw [in=74, out=-90] (13) to (0);
		\draw [in=-120, out=120] (11) to (10);
		\draw (10) to (1);
		\draw (3.center) to (12);
		\draw [in=120, out=-120, looseness=1.25] (12) to (13);
		\draw (9) to (11);
		\draw (8) to (10);
		\draw (9) to (13);
		\draw (8) to (12);
	\end{pgfonlayer}
\end{tikzpicture}
=
\begin{tikzpicture}
	\begin{pgfonlayer}{nodelayer}
		\node [style=map] (0) at (2.75, 1.25) {$f$};
		\node [style=none] (1) at (2, 0.5) {};
		\node [style=none] (2) at (2.75, 6.25) {};
		\node [style=none] (3) at (2.75, 0.5) {};
		\node [style=Z] (4) at (2.25, 2) {};
		\node [style=none] (5) at (2.25, 6.25) {};
		\node [style=map] (6) at (2.75, 4.5) {$f$};
		\node [style=map] (7) at (2.75, 3.75) {$f^\circ$};
		\node [style=Z] (8) at (2.25, 5.5) {};
		\node [style=Z] (9) at (2.25, 2.75) {};
		\node [style=Z] (10) at (2.75, 5.5) {};
		\node [style=Z] (11) at (2.75, 2.75) {};
	\end{pgfonlayer}
	\begin{pgfonlayer}{edgelayer}
		\draw (3.center) to (0);
		\draw (4) to (0);
		\draw [in=90, out=-120] (4) to (1.center);
		\draw (6) to (7);
		\draw (11) to (0);
		\draw [in=-120, out=120] (9) to (8);
		\draw (2.center) to (10);
		\draw [in=120, out=-120, looseness=1.25] (10) to (11);
		\draw (7) to (9);
		\draw (6) to (8);
		\draw (7) to (11);
		\draw (6) to (10);
		\draw (9) to (4);
		\draw (8) to (5.center);
	\end{pgfonlayer}
\end{tikzpicture}
=
\begin{tikzpicture}
	\begin{pgfonlayer}{nodelayer}
		\node [style=map] (0) at (2.5, 2) {$f$};
		\node [style=none] (1) at (1.75, 1.25) {};
		\node [style=none] (2) at (2.75, 6.75) {};
		\node [style=none] (3) at (2.5, 1.25) {};
		\node [style=none] (4) at (2.25, 6.75) {};
		\node [style=map] (5) at (2.75, 4.5) {$f$};
		\node [style=map] (6) at (2.75, 3.75) {$f^\circ$};
		\node [style=Z] (7) at (2.25, 5.5) {};
		\node [style=Z] (8) at (2.25, 2.75) {};
		\node [style=Z] (9) at (2.75, 5.5) {};
		\node [style=Z] (10) at (2.75, 2.75) {};
		\node [style=none] (11) at (1.75, 5.5) {};
		\node [style=Z] (12) at (2.25, 6.25) {};
	\end{pgfonlayer}
	\begin{pgfonlayer}{edgelayer}
		\draw (3.center) to (0);
		\draw (5) to (6);
		\draw [in=60, out=-90] (10) to (0);
		\draw [in=-120, out=120] (8) to (7);
		\draw (2.center) to (9);
		\draw [in=120, out=-120, looseness=1.25] (9) to (10);
		\draw (6) to (8);
		\draw (5) to (7);
		\draw (6) to (10);
		\draw (5) to (9);
		\draw (7) to (4.center);
		\draw [in=90, out=-135] (12) to (11.center);
		\draw (11.center) to (1.center);
		\draw [in=120, out=-90] (8) to (0);
	\end{pgfonlayer}
\end{tikzpicture}
=
\begin{tikzpicture}
	\begin{pgfonlayer}{nodelayer}
		\node [style=map] (0) at (2.5, 2) {$f$};
		\node [style=none] (1) at (2, 1.25) {};
		\node [style=none] (2) at (2.75, 4.75) {};
		\node [style=none] (3) at (2.5, 1.25) {};
		\node [style=none] (4) at (2.25, 4.75) {};
		\node [style=map] (5) at (2.5, 3.5) {$f$};
		\node [style=map] (6) at (2.5, 2.75) {$f^\circ$};
		\node [style=none] (7) at (2, 3.5) {};
		\node [style=Z] (8) at (2.25, 4.25) {};
	\end{pgfonlayer}
	\begin{pgfonlayer}{edgelayer}
		\draw (3.center) to (0);
		\draw (5) to (6);
		\draw [in=90, out=-135] (8) to (7.center);
		\draw (7.center) to (1.center);
		\draw [bend left, looseness=1.25] (0) to (6);
		\draw [bend left, looseness=1.25] (6) to (0);
		\draw (5) to (8);
		\draw (8) to (4.center);
		\draw [in=60, out=-90] (2.center) to (5);
	\end{pgfonlayer}
\end{tikzpicture}
=
\begin{tikzpicture}
	\begin{pgfonlayer}{nodelayer}
		\node [style=none] (0) at (2, 2.75) {};
		\node [style=none] (1) at (2.75, 4.75) {};
		\node [style=none] (2) at (2.5, 2.75) {};
		\node [style=none] (3) at (2.25, 4.75) {};
		\node [style=map] (4) at (2.5, 3.5) {$f$};
		\node [style=none] (5) at (2, 3.5) {};
		\node [style=Z] (6) at (2.25, 4.25) {};
	\end{pgfonlayer}
	\begin{pgfonlayer}{edgelayer}
		\draw [in=90, out=-135, looseness=0.75] (6) to (5.center);
		\draw (5.center) to (0.center);
		\draw (4) to (6);
		\draw (6) to (3.center);
		\draw [in=60, out=-90] (1.center) to (4);
		\draw (4) to (2.center);
	\end{pgfonlayer}
\end{tikzpicture}
\end{align*}

Thus, by Lemma \ref{lem:latching}:
$$
\begin{tikzpicture}
	\begin{pgfonlayer}{nodelayer}
		\node [style=map] (0) at (0.5, 1.75) {$g$};
		\node [style=none] (1) at (0.5, 1) {};
		\node [style=map] (2) at (0.5, 3.25) {$g^\circ$};
		\node [style=map] (3) at (0.5, 4) {$f$};
		\node [style=Z] (4) at (0.25, 2.5) {};
		\node [style=Z] (5) at (0.25, 4.75) {};
		\node [style=none] (6) at (0.25, 5.25) {};
		\node [style=none] (7) at (0.75, 5.25) {};
	\end{pgfonlayer}
	\begin{pgfonlayer}{edgelayer}
		\draw (1.center) to (0);
		\draw [in=-90, out=124] (0) to (4);
		\draw (4) to (2);
		\draw [in=60, out=-60] (2) to (0);
		\draw [in=-120, out=120] (4) to (5);
		\draw (5) to (6.center);
		\draw [in=60, out=-90] (7.center) to (3);
		\draw (3) to (5);
		\draw (3) to (2);
	\end{pgfonlayer}
\end{tikzpicture}
=
\begin{tikzpicture}
	\begin{pgfonlayer}{nodelayer}
		\node [style=map] (0) at (0.5, 1.75) {$f$};
		\node [style=none] (1) at (0.5, 1) {};
		\node [style=none] (2) at (0.25, 2.5) {};
		\node [style=none] (3) at (0.75, 2.5) {};
	\end{pgfonlayer}
	\begin{pgfonlayer}{edgelayer}
		\draw (1.center) to (0);
		\draw [in=-90, out=60] (0) to (3.center);
		\draw [in=120, out=-90] (2.center) to (0);
	\end{pgfonlayer}
\end{tikzpicture}
$$


\end{proof}


%\subsection{Environment structures}
\label{sec:env}



\begin{definition}
Given a discrete inverse category $\X$  define the counital completion of $\X$, $c(\X)$, to be the quotient of ${\CoPara}(\X)$ freely making  the discard maps $((u_X^L)^{-1},X)$ into the counit of the cosemigroup of the inverse product on $\X$.
\end{definition}

Unrolling the definition, this construction just freely adds counits to the diagonal maps coming from the inverse products of $\X$. 

\begin{lemma}
$c(\X)$ is a discrete Cartesian restriction category.
\end{lemma}
\begin{proof}
This is clearly a counital copy category, with a restriction terminal object given by the tensor unit.  Moreover, because the Frobenius structure is special, it is also discrete.
\end{proof}



\begin{lemma}
\label{lemma:envstruct}
\label{cor:copy}
Given a discrete inverse category $\X$, its counital completion $c(\X)$ and  Cartesian completion $\tilde \X$ are isomorphic as discrete Cartesian restriction categories.
\end{lemma}

\begin{proof}
First, observe that the restriction category structure of $\tilde \X$ and  $c(\X)$ both agree, as:

$$
\begin{tikzpicture}
	\begin{pgfonlayer}{nodelayer}
		\node [style=Z] (0) at (8.75, -1) {};
		\node [style=map] (1) at (8.25, -0.25) {$f$};
		\node [style=Z] (2) at (8.25, 0.5) {};
		\node [style=none] (3) at (9.25, -0.25) {};
		\node [style=none] (4) at (9.25, 0.5) {};
		\node [style=none] (5) at (9.25, 1) {};
		\node [style=none] (6) at (8.75, -1.75) {};
	\end{pgfonlayer}
	\begin{pgfonlayer}{edgelayer}
		\draw (6.center) to (0);
		\draw [in=-90, out=150] (0) to (1);
		\draw (1) to (2);
		\draw [in=-90, out=30] (0) to (3.center);
		\draw (3.center) to (4.center);
		\draw (4.center) to (5.center);
	\end{pgfonlayer}
\end{tikzpicture}
=
\begin{tikzpicture}
	\begin{pgfonlayer}{nodelayer}
		\node [style=Z] (25) at (10.5, -1) {};
		\node [style=map] (26) at (10, -0.25) {$f;f^\circ; f$};
		\node [style=Z] (27) at (10, 0.5) {};
		\node [style=none] (28) at (11, -0.25) {};
		\node [style=none] (29) at (11, 0.5) {};
		\node [style=none] (30) at (11, 1) {};
		\node [style=none] (31) at (10.5, -1.75) {};
	\end{pgfonlayer}
	\begin{pgfonlayer}{edgelayer}
		\draw (31.center) to (25);
		\draw [in=-90, out=150] (25) to (26);
		\draw (26) to (27);
		\draw [in=-90, out=30] (25) to (28.center);
		\draw (28.center) to (29.center);
		\draw (29.center) to (30.center);
	\end{pgfonlayer}
\end{tikzpicture}
=
\begin{tikzpicture}
	\begin{pgfonlayer}{nodelayer}
		\node [style=Z] (32) at (12.5, -0.75) {};
		\node [style=map] (33) at (12, 0) {$f$};
		\node [style=Z] (34) at (12, 0.75) {};
		\node [style=none] (35) at (13, 0) {};
		\node [style=none] (36) at (13, 0.75) {};
		\node [style=none] (37) at (13, 1.25) {};
		\node [style=none] (38) at (12.5, -2.25) {};
		\node [style=map] (39) at (12.5, -1.5) {$f;f^\circ$};
	\end{pgfonlayer}
	\begin{pgfonlayer}{edgelayer}
		\draw (38.center) to (32);
		\draw [in=-90, out=150] (32) to (33);
		\draw (33) to (34);
		\draw [in=-90, out=30] (32) to (35.center);
		\draw (35.center) to (36.center);
		\draw (36.center) to (37.center);
	\end{pgfonlayer}
\end{tikzpicture}
=
\begin{tikzpicture}
	\begin{pgfonlayer}{nodelayer}
		\node [style=Z] (40) at (14.5, -1) {};
		\node [style=map] (41) at (14, -0.25) {$f^\circ;f$};
		\node [style=Z] (42) at (14, 0.5) {};
		\node [style=none] (43) at (15, -0.25) {};
		\node [style=none] (44) at (15, 0.5) {};
		\node [style=none] (45) at (15, 1) {};
		\node [style=none] (46) at (14.5, -2.5) {};
		\node [style=map] (47) at (14.5, -1.75) {$f$};
		\node [style=map] (48) at (15, -0.25) {$f^\circ$};
	\end{pgfonlayer}
	\begin{pgfonlayer}{edgelayer}
		\draw (46.center) to (40);
		\draw [in=-90, out=150] (40) to (41);
		\draw (41) to (42);
		\draw [in=-90, out=30] (40) to (43.center);
		\draw (43.center) to (44.center);
		\draw (44.center) to (45.center);
	\end{pgfonlayer}
\end{tikzpicture}
=
\begin{tikzpicture}
	\begin{pgfonlayer}{nodelayer}
		\node [style=Z] (49) at (17.25, -1.25) {};
		\node [style=Z] (51) at (16.75, -0.5) {};
		\node [style=none] (52) at (17.75, -0.5) {};
		\node [style=none] (53) at (17.75, 0.25) {};
		\node [style=none] (55) at (17.25, -2.75) {};
		\node [style=map] (57) at (17.75, -0.5) {$f^\circ$};
		\node [style=map] (58) at (17.25, -2) {$f;f^\circ;f$};
	\end{pgfonlayer}
	\begin{pgfonlayer}{edgelayer}
		\draw [in=-90, out=30] (49) to (52.center);
		\draw (52.center) to (53.center);
		\draw [in=150, out=-90] (51) to (49);
		\draw (49) to (58);
		\draw (58) to (55.center);
	\end{pgfonlayer}
\end{tikzpicture}
=
\begin{tikzpicture}
	\begin{pgfonlayer}{nodelayer}
		\node [style=none] (61) at (19.25, -0.5) {};
		\node [style=none] (62) at (19.25, 0.25) {};
		\node [style=none] (63) at (19.25, -2.25) {};
		\node [style=map] (64) at (19.25, -0.5) {$f^\circ$};
		\node [style=map] (65) at (19.25, -1.5) {$f;f^\circ;f$};
	\end{pgfonlayer}
	\begin{pgfonlayer}{edgelayer}
		\draw (61.center) to (62.center);
		\draw (65) to (63.center);
		\draw (65) to (64);
	\end{pgfonlayer}
\end{tikzpicture}
=
\begin{tikzpicture}
	\begin{pgfonlayer}{nodelayer}
		\node [style=none] (66) at (21, -1) {};
		\node [style=none] (67) at (21, 0.25) {};
		\node [style=none] (68) at (21, -2.25) {};
		\node [style=map] (70) at (21, -1) {$f;f^\circ$};
	\end{pgfonlayer}
	\begin{pgfonlayer}{edgelayer}
		\draw (66.center) to (67.center);
		\draw (70) to (68.center);
	\end{pgfonlayer}
\end{tikzpicture}
$$


Since both $\tilde \X$ and $c(\X)$ are quotients of $\CoPara(\X)$, there are identity on objects mappings $F:c(\X)\to \tilde \X$ and $g:\tilde \X\to c(\X)$ which are inverse to each other.  It remains to show that these are both functors.


The functoriality of $F$ is immediate because $\tilde \X$ is a Cartesian restriction category and thus the diagonal maps all have counits.


%%%%%%%%%%%%%%%%%%%%%%%%%%%%%


To prove that $G$ is a functor, take some $(f,S)\sim (g,T)$ in $\tilde \X$. Since the Frobenius structure in $\tilde \X$ is counital:
$$
\begin{tikzpicture}
	\begin{pgfonlayer}{nodelayer}
		\node [style=map] (0) at (1.5, 3.25) {$f^\circ$};
		\node [style=Z] (1) at (1, 2.25) {};
		\node [style=map] (2) at (1.5, 1.25) {$f$};
		\node [style=none] (3) at (1.5, 4.25) {};
		\node [style=none] (4) at (0.5, 0.5) {};
		\node [style=none] (5) at (1.5, 0.5) {};
		\node [style=none] (7) at (2.25, 4.25) {};
		\node [style=unit] (8) at (2.25, 3.5) {};
	\end{pgfonlayer}
	\begin{pgfonlayer}{edgelayer}
		\draw [style=simple] (0) to (3.center);
		\draw [style=simple] (1) to (2);
		\draw [style=simple, in=90, out=-104] (1) to (4.center);
		\draw [style=simple] (5.center) to (2);
		\draw [style=simple, in=60, out=-60, looseness=0.75] (0) to (2);
		\draw [in=-117, out=90] (1) to (0);
		\draw (8.center) to (7.center);
	\end{pgfonlayer}
\end{tikzpicture}
\sim
\begin{tikzpicture}
	\begin{pgfonlayer}{nodelayer}
		\node [style=map] (0) at (0, 4.25) {$f^\circ$};
		\node [style=Z] (1) at (-0.5, 3.25) {};
		\node [style=Z] (2) at (-0.5, 2.5) {};
		\node [style=map] (3) at (0, 1.5) {$f$};
		\node [style=none] (4) at (0, 5.25) {};
		\node [style=none] (5) at (-1, 0.5) {};
		\node [style=none] (6) at (0, 0.5) {};
		\node [style=none] (7) at (0.5, 5.25) {};
		\node [style=none] (8) at (-0.75, 4.5) {};
	\end{pgfonlayer}
	\begin{pgfonlayer}{edgelayer}
		\draw [style=simple] (1) to (0);
		\draw [style=simple] (0) to (4.center);
		\draw [style=simple] (1) to (2);
		\draw [style=simple] (2) to (3);
		\draw [style=simple, in=90, out=-104] (2) to (5.center);
		\draw [style=simple] (6.center) to (3);
		\draw [style=simple, in=60, out=-60, looseness=0.75] (0) to (3);
		\draw [style=simple, in=90, out=-90, looseness=0.75] (7.center) to (8.center);
		\draw [style=simple, in=-90, out=101] (1) to (8.center);
	\end{pgfonlayer}
\end{tikzpicture}
=
\begin{tikzpicture}
	\begin{pgfonlayer}{nodelayer}
		\node [style=map] (0) at (0, 4.25) {$g^\circ$};
		\node [style=Z] (1) at (-0.5, 3.25) {};
		\node [style=Z] (2) at (-0.5, 2.5) {};
		\node [style=map] (3) at (0, 1.5) {$g$};
		\node [style=none] (4) at (0, 5.25) {};
		\node [style=none] (5) at (-1, 0.5) {};
		\node [style=none] (6) at (0, 0.5) {};
		\node [style=none] (7) at (0.5, 5.25) {};
		\node [style=none] (8) at (-0.75, 4.5) {};
	\end{pgfonlayer}
	\begin{pgfonlayer}{edgelayer}
		\draw [style=simple] (1) to (0);
		\draw [style=simple] (0) to (4.center);
		\draw [style=simple] (1) to (2);
		\draw [style=simple] (2) to (3);
		\draw [style=simple, in=90, out=-104] (2) to (5.center);
		\draw [style=simple] (6.center) to (3);
		\draw [style=simple, in=60, out=-60, looseness=0.75] (0) to (3);
		\draw [style=simple, in=90, out=-90, looseness=0.75] (7.center) to (8.center);
		\draw [style=simple, in=-90, out=101] (1) to (8.center);
	\end{pgfonlayer}
\end{tikzpicture}
\sim
\begin{tikzpicture}
	\begin{pgfonlayer}{nodelayer}
		\node [style=map] (0) at (1.5, 3.25) {$g^\circ$};
		\node [style=Z] (1) at (1, 2.25) {};
		\node [style=map] (2) at (1.5, 1.25) {$g$};
		\node [style=none] (3) at (1.5, 4.25) {};
		\node [style=none] (4) at (0.5, 0.5) {};
		\node [style=none] (5) at (1.5, 0.5) {};
		\node [style=none] (7) at (2.25, 4.25) {};
		\node [style=unit] (8) at (2.25, 3.5) {};
	\end{pgfonlayer}
	\begin{pgfonlayer}{edgelayer}
		\draw [style=simple] (0) to (3.center);
		\draw [style=simple] (1) to (2);
		\draw [style=simple, in=90, out=-104] (1) to (4.center);
		\draw [style=simple] (5.center) to (2);
		\draw [style=simple, in=60, out=-60, looseness=0.75] (0) to (2);
		\draw [in=-117, out=90] (1) to (0);
		\draw (8.center) to (7.center);
	\end{pgfonlayer}
\end{tikzpicture}
$$


However, since the functor $\X\to \tilde \X $ is faithful by Lemma \ref{lemma:xtildefaithful}, using the alternate equivalence relation of $\tilde \X$ by  Lemma \ref{theorem:cpstartheorem}, we have that in $\X$:

$$
\begin{tikzpicture}
	\begin{pgfonlayer}{nodelayer}
		\node [style=map] (0) at (0, 3.25) {$f^\circ$};
		\node [style=Z] (1) at (-0.5, 2.25) {};
		\node [style=map] (2) at (0, 1.25) {$f$};
		\node [style=none] (3) at (0, 4.25) {};
		\node [style=none] (4) at (-1, 0.5) {};
		\node [style=none] (5) at (0, 0.5) {};
	\end{pgfonlayer}
	\begin{pgfonlayer}{edgelayer}
		\draw [style=simple] (0) to (3.center);
		\draw [style=simple] (1) to (2);
		\draw [style=simple, in=90, out=-104] (1) to (4.center);
		\draw [style=simple] (5.center) to (2);
		\draw [style=simple, in=60, out=-60, looseness=0.75] (0) to (2);
		\draw [in=-117, out=90] (1) to (0);
	\end{pgfonlayer}
\end{tikzpicture}
=
\begin{tikzpicture}
	\begin{pgfonlayer}{nodelayer}
		\node [style=map] (0) at (0, 3.5) {$g^\circ$};
		\node [style=Z] (1) at (-0.5, 2.5) {};
		\node [style=map] (2) at (0, 1.5) {$g$};
		\node [style=none] (3) at (0, 4.5) {};
		\node [style=none] (4) at (-1, 0.5) {};
		\node [style=none] (5) at (0, 0.5) {};
	\end{pgfonlayer}
	\begin{pgfonlayer}{edgelayer}
		\draw [style=simple] (0) to (3.center);
		\draw [style=simple] (1) to (2);
		\draw [style=simple, in=90, out=-104] (1) to (4.center);
		\draw [style=simple] (5.center) to (2);
		\draw [style=simple, in=60, out=-60, looseness=0.75] (0) to (2);
		\draw [in=-117, out=90] (1) to (0);
	\end{pgfonlayer}
\end{tikzpicture}
\hspace*{.2cm}\text{and thus}\hspace*{.1cm}
\begin{tikzpicture}
	\begin{pgfonlayer}{nodelayer}
		\node [style=map] (0) at (0, 1.5) {$f$};
		\node [style=Z] (1) at (-0.5, 2.5) {};
		\node [style=map] (2) at (0, 3.5) {$f^\circ$};
		\node [style=none] (3) at (0, 0.5) {};
		\node [style=none] (4) at (-1, 4.5) {};
		\node [style=none] (5) at (0, 4.5) {};
	\end{pgfonlayer}
	\begin{pgfonlayer}{edgelayer}
		\draw [style=simple] (0) to (3.center);
		\draw [style=simple] (1) to (2);
		\draw [style=simple, in=-90, out=104] (1) to (4.center);
		\draw [style=simple] (5.center) to (2);
		\draw [style=simple, in=-60, out=60, looseness=0.75] (0) to (2);
		\draw [in=117, out=-90] (1) to (0);
	\end{pgfonlayer}
\end{tikzpicture}
=
\begin{tikzpicture}
	\begin{pgfonlayer}{nodelayer}
		\node [style=map] (0) at (0, 1.5) {$g$};
		\node [style=Z] (1) at (-0.5, 2.5) {};
		\node [style=map] (2) at (0, 3.5) {$g^\circ$};
		\node [style=none] (3) at (0, 0.5) {};
		\node [style=none] (4) at (-1, 4.25) {};
		\node [style=none] (5) at (0, 4.25) {};
	\end{pgfonlayer}
	\begin{pgfonlayer}{edgelayer}
		\draw [style=simple] (0) to (3.center);
		\draw [style=simple] (1) to (2);
		\draw [style=simple, in=-90, out=104] (1) to (4.center);
		\draw [style=simple] (5.center) to (2);
		\draw [style=simple, in=-60, out=60, looseness=0.75] (0) to (2);
		\draw [in=117, out=-90] (1) to (0);
	\end{pgfonlayer}
\end{tikzpicture}
$$


Therefore in $c(\X)$:

\begin{align*}
\begin{tikzpicture}
	\begin{pgfonlayer}{nodelayer}
		\node [style=map] (0) at (0, 1.5) {$f$};
		\node [style=Z] (1) at (-0.5, 2.5) {};
		\node [style=map] (2) at (0, 3.5) {$f^\circ$};
		\node [style=none] (3) at (0, 0.5) {};
		\node [style=none] (4) at (-1, 4.5) {};
		\node [style=none] (5) at (0, 4.25) {};
		\node [style=Z] (6) at (0, 4.25) {};
	\end{pgfonlayer}
	\begin{pgfonlayer}{edgelayer}
		\draw [style=simple] (0) to (3.center);
		\draw [style=simple] (1) to (2);
		\draw [style=simple, in=-90, out=104] (1) to (4.center);
		\draw [style=simple] (5.center) to (2);
		\draw [style=simple, in=-60, out=60, looseness=0.75] (0) to (2);
		\draw [in=117, out=-90] (1) to (0);
	\end{pgfonlayer}
\end{tikzpicture}
=
\begin{tikzpicture}
	\begin{pgfonlayer}{nodelayer}
		\node [style=map] (0) at (0, 1.5) {$f$};
		\node [style=Z] (1) at (-0.5, 2.5) {};
		\node [style=map] (2) at (0, 3.5) {$\bar {f^\circ}$};
		\node [style=none] (3) at (0, 0.5) {};
		\node [style=none] (4) at (-1, 4.5) {};
		\node [style=Z] (5) at (-0.25, 4.5) {};
		\node [style=Z] (6) at (0.25, 4.5) {};
	\end{pgfonlayer}
	\begin{pgfonlayer}{edgelayer}
		\draw [style=simple] (0) to (3.center);
		\draw [style=simple] (1) to (2);
		\draw [style=simple, in=-90, out=104] (1) to (4.center);
		\draw [style=simple, in=-60, out=60, looseness=0.75] (0) to (2);
		\draw [in=117, out=-90] (1) to (0);
		\draw [in=104, out=-90] (5) to (2);
		\draw [in=-90, out=76] (2) to (6);
	\end{pgfonlayer}
\end{tikzpicture}
=
\begin{tikzpicture}
	\begin{pgfonlayer}{nodelayer}
		\node [style=map] (0) at (1.5, 1.5) {$f$};
		\node [style=Z] (1) at (1, 2.5) {};
		\node [style=none] (2) at (1.5, 0.5) {};
		\node [style=none] (3) at (0.5, 4.5) {};
		\node [style=Z] (4) at (1.25, 4.5) {};
		\node [style=Z] (5) at (1.75, 4.5) {};
		\node [style=map] (6) at (2, 3.75) {$\bar {f^\circ}$};
		\node [style=Z] (7) at (1.25, 5.25) {};
		\node [style=Z] (8) at (1.75, 5.25) {};
		\node [style=Z] (9) at (1.25, 3) {};
		\node [style=Z] (10) at (1.75, 3) {};
	\end{pgfonlayer}
	\begin{pgfonlayer}{edgelayer}
		\draw [style=simple] (0) to (2.center);
		\draw [style=simple, in=-90, out=104] (1) to (3.center);
		\draw [in=117, out=-90] (1) to (0);
		\draw [in=-120, out=120, looseness=1.25] (9) to (4);
		\draw [in=-120, out=120, looseness=1.25] (10) to (5);
		\draw [in=-75, out=72] (10) to (6);
		\draw [in=-120, out=45] (9) to (6);
		\draw [in=-45, out=120] (6) to (4);
		\draw [in=75, out=-72] (5) to (6);
		\draw (5) to (8);
		\draw (7) to (4);
		\draw (9) to (1);
		\draw [in=60, out=-90] (10) to (0);
	\end{pgfonlayer}
\end{tikzpicture}
=
\begin{tikzpicture}
	\begin{pgfonlayer}{nodelayer}
		\node [style=map] (0) at (0, 1.5) {$f$};
		\node [style=none] (1) at (0, 1) {};
		\node [style=none] (2) at (-1, 5.75) {};
		\node [style=Z] (3) at (-0.25, 4) {};
		\node [style=Z] (4) at (0.25, 4) {};
		\node [style=map] (5) at (0.5, 3.25) {$\bar {f^\circ}$};
		\node [style=Z] (6) at (-0.25, 5.5) {};
		\node [style=Z] (7) at (0.25, 5.5) {};
		\node [style=Z] (8) at (-0.25, 2.5) {};
		\node [style=Z] (9) at (0.25, 2.5) {};
		\node [style=Z] (10) at (-0.25, 4.75) {};
	\end{pgfonlayer}
	\begin{pgfonlayer}{edgelayer}
		\draw [style=simple] (0) to (1.center);
		\draw [in=-120, out=120, looseness=1.25] (8) to (3);
		\draw [in=-120, out=120, looseness=1.25] (9) to (4);
		\draw [in=-75, out=72] (9) to (5);
		\draw [in=-120, out=45] (8) to (5);
		\draw [in=-45, out=120] (5) to (3);
		\draw [in=75, out=-72] (4) to (5);
		\draw (4) to (7);
		\draw (6) to (3);
		\draw [in=60, out=-90] (9) to (0);
		\draw [in=127, out=-90] (2.center) to (10);
		\draw [in=120, out=-90] (8) to (0);
	\end{pgfonlayer}
\end{tikzpicture}
=
\begin{tikzpicture}
	\begin{pgfonlayer}{nodelayer}
		\node [style=map] (0) at (0, 1.5) {$f$};
		\node [style=none] (1) at (0, 1) {};
		\node [style=none] (2) at (-1, 4) {};
		\node [style=map] (3) at (0, 2.25) {$\bar {f^\circ}$};
		\node [style=Z] (4) at (-0.25, 3.75) {};
		\node [style=Z] (5) at (0.25, 3.75) {};
		\node [style=Z] (6) at (-0.25, 3) {};
	\end{pgfonlayer}
	\begin{pgfonlayer}{edgelayer}
		\draw [style=simple] (0) to (1.center);
		\draw [in=127, out=-90] (2.center) to (6);
		\draw [in=120, out=-120, looseness=1.25] (3) to (0);
		\draw [in=-75, out=75, looseness=1.25] (0) to (3);
		\draw [in=-90, out=120] (3) to (6);
		\draw [in=-90, out=75] (3) to (5);
		\draw (4) to (6);
	\end{pgfonlayer}
\end{tikzpicture}
=
\begin{tikzpicture}
	\begin{pgfonlayer}{nodelayer}
		\node [style=none] (0) at (0, 1.5) {};
		\node [style=none] (1) at (-1, 4) {};
		\node [style=Z] (2) at (0.25, 3.75) {};
		\node [style=Z] (3) at (-0.25, 3) {};
		\node [style=map] (4) at (0, 2) {$f$};
		\node [style=Z] (5) at (-0.25, 3.75) {};
	\end{pgfonlayer}
	\begin{pgfonlayer}{edgelayer}
		\draw [in=127, out=-90] (1.center) to (3);
		\draw [style=simple] (4) to (0.center);
		\draw (5) to (3);
		\draw [in=75, out=-90] (2) to (4);
		\draw [in=-90, out=120] (4) to (3);
	\end{pgfonlayer}
\end{tikzpicture}
=
\begin{tikzpicture}
	\begin{pgfonlayer}{nodelayer}
		\node [style=none] (0) at (0, 1.5) {};
		\node [style=Z] (1) at (0.25, 3) {};
		\node [style=map] (2) at (0, 2) {$f$};
		\node [style=none] (3) at (-0.25, 3) {};
		\node [style=none] (4) at (-0.25, 3.5) {};
	\end{pgfonlayer}
	\begin{pgfonlayer}{edgelayer}
		\draw [style=simple] (2) to (0.center);
		\draw [in=75, out=-90] (1) to (2);
		\draw (4.center) to (3.center);
		\draw [in=104, out=-90] (3.center) to (2);
	\end{pgfonlayer}
\end{tikzpicture}
\end{align*}

So that combining the previous two equations:

\begin{align*}
\begin{tikzpicture}
	\begin{pgfonlayer}{nodelayer}
		\node [style=none] (0) at (0, 1.5) {};
		\node [style=Z] (1) at (0.25, 3) {};
		\node [style=map] (2) at (0, 2) {$f$};
		\node [style=none] (3) at (-0.25, 3) {};
		\node [style=none] (4) at (-0.25, 3.5) {};
	\end{pgfonlayer}
	\begin{pgfonlayer}{edgelayer}
		\draw [style=simple] (2) to (0.center);
		\draw [in=75, out=-90] (1) to (2);
		\draw (4.center) to (3.center);
		\draw [in=104, out=-90] (3.center) to (2);
	\end{pgfonlayer}
\end{tikzpicture}
=
\begin{tikzpicture}
	\begin{pgfonlayer}{nodelayer}
		\node [style=map] (0) at (0, 1.5) {$f$};
		\node [style=Z] (1) at (-0.5, 2.5) {};
		\node [style=map] (2) at (0, 3.5) {$f^\circ$};
		\node [style=none] (3) at (0, 0.5) {};
		\node [style=none] (4) at (-1, 4.5) {};
		\node [style=none] (5) at (0, 4.25) {};
		\node [style=Z] (6) at (0, 4.25) {};
	\end{pgfonlayer}
	\begin{pgfonlayer}{edgelayer}
		\draw [style=simple] (0) to (3.center);
		\draw [style=simple] (1) to (2);
		\draw [style=simple, in=-90, out=104] (1) to (4.center);
		\draw [style=simple] (5.center) to (2);
		\draw [style=simple, in=-60, out=60, looseness=0.75] (0) to (2);
		\draw [in=117, out=-90] (1) to (0);
	\end{pgfonlayer}
\end{tikzpicture}
=
\begin{tikzpicture}
	\begin{pgfonlayer}{nodelayer}
		\node [style=map] (0) at (0, 1.5) {$g$};
		\node [style=Z] (1) at (-0.5, 2.5) {};
		\node [style=map] (2) at (0, 3.5) {$g^\circ$};
		\node [style=none] (3) at (0, 0.5) {};
		\node [style=none] (4) at (-1, 4.5) {};
		\node [style=none] (5) at (0, 4.25) {};
		\node [style=Z] (6) at (0, 4.25) {};
	\end{pgfonlayer}
	\begin{pgfonlayer}{edgelayer}
		\draw [style=simple] (0) to (3.center);
		\draw [style=simple] (1) to (2);
		\draw [style=simple, in=-90, out=104] (1) to (4.center);
		\draw [style=simple] (5.center) to (2);
		\draw [style=simple, in=-60, out=60, looseness=0.75] (0) to (2);
		\draw [in=117, out=-90] (1) to (0);
	\end{pgfonlayer}
\end{tikzpicture}
=
\begin{tikzpicture}
	\begin{pgfonlayer}{nodelayer}
		\node [style=none] (0) at (0, 1.5) {};
		\node [style=Z] (1) at (0.25, 3) {};
		\node [style=map] (2) at (0, 2) {$g$};
		\node [style=none] (3) at (-0.25, 3) {};
		\node [style=none] (4) at (-0.25, 3.5) {};
	\end{pgfonlayer}
	\begin{pgfonlayer}{edgelayer}
		\draw [style=simple] (2) to (0.center);
		\draw [in=75, out=-90] (1) to (2);
		\draw (4.center) to (3.center);
		\draw [in=104, out=-90] (3.center) to (2);
	\end{pgfonlayer}
\end{tikzpicture}
\end{align*}


\end{proof}


There is another equivalent way of viewing this construction, which follows immediately:

\begin{corollary}
$c(\X)$ is isomorphic to ${\sf Split}_{\{(\Delta_X,X)\ | \ X \in \X_0 \}}(\CoPara(\X))$.
\end{corollary}


Therefore, we can regard the Cartesian completion in some sense as being analagous to looking at the stochastic channels of a quantum system.

\section{A graphical calculus for Boolean multirelations}
\label{sec:ZXA}

In this section, we give a complete presentation, $\ZXA$, for the full monoidal subcategory of spans of finite sets where the objects are powers of the two element set.  This is performed by freely adding a counit and unit to the semi-Frobenius algebra structure of the category $\TOF$, and then performing a two way translation between this prop and $\ZXA$ which we  prove is an  isomorphism. 






%\subsection{The category \texorpdfstring{$\TOF$}{TOF}}
%\label{sec:tof}

\begin{definition}
\label{def:tof}
$\TOF$ \cite{tof} is the prop, generated by the 1 ancillary bits $| 1\rangle$ and $\langle 1|$ as well as the Toffoli gate, satisfying the identities given in Figure \ref{fig:TOF}.
		

\begin{figure}[H]
\noindent
\scalebox{1.0}{%
\vbox{%
\begin{mdframed}
\begin{multicols}{2}
\begin{enumerate}[label={\bf [TOF.\arabic*]}, ref={\bf [TOF.\arabic*]}, wide = 0pt, leftmargin = 2em]
\item
\label{TOF.1}
{\hfil
$
\begin{tabular}{c}
\begin{tikzpicture}
	\begin{pgfonlayer}{nodelayer}
		\node [style=nothing] (2) at (0, 1.5) {};
		\node [style=nothing] (3) at (-0.5, 1.5) {};
		\node [style=oplus] (4) at (0, 2) {};
		\node [style=dot] (5) at (-0.5, 2) {};
		\node [style=dot] (6) at (-1, 2) {};
		\node [style=onein] (7) at (-1, 1.5) {};
		\node [style=nothing] (8) at (-1, 2.5) {};
		\node [style=nothing] (9) at (-0.5, 2.5) {};
		\node [style=nothing] (10) at (0, 2.5) {};
	\end{pgfonlayer}
	\begin{pgfonlayer}{edgelayer}
		\draw (7) to (6);
		\draw (6) to (8);
		\draw (9) to (5);
		\draw (3) to (5);
		\draw (2) to (4);
		\draw (4) to (10);
		\draw (4) to (5);
		\draw (5) to (6);
	\end{pgfonlayer}
\end{tikzpicture}
=
\begin{tikzpicture}
	\begin{pgfonlayer}{nodelayer}
		\node [style=nothing] (3) at (0, 1.5) {};
		\node [style=nothing] (4) at (-0.5, 1.5) {};
		\node [style=oplus] (5) at (0, 2) {};
		\node [style=dot] (6) at (-0.5, 2) {};
		\node [style=onein] (7) at (-1, 2) {};
		\node [style=nothing] (8) at (-1, 2.5) {};
		\node [style=nothing] (9) at (-0.5, 2.5) {};
		\node [style=nothing] (10) at (0, 2.5) {};
	\end{pgfonlayer}
	\begin{pgfonlayer}{edgelayer}
		\draw (4) to (6);
		\draw (3) to (5);
		\draw (5) to (6);
		\draw (9) to (6);
		\draw (7) to (8);
		\draw (5) to (10);
	\end{pgfonlayer}
\end{tikzpicture}
\\
{}\\
\begin{tikzpicture}
	\begin{pgfonlayer}{nodelayer}
		\node [style=nothing] (0) at (0, 1.5) {};
		\node [style=nothing] (1) at (-0.5, 1.5) {};
		\node [style=oplus] (2) at (0, 1) {};
		\node [style=dot] (3) at (-0.5, 1) {};
		\node [style=dot] (4) at (-1, 1) {};
		\node [style=oneout] (5) at (-1, 1.5) {};
		\node [style=nothing] (6) at (-1, 0.5) {};
		\node [style=nothing] (7) at (-0.5, 0.5) {};
		\node [style=nothing] (8) at (0, 0.5) {};
	\end{pgfonlayer}
	\begin{pgfonlayer}{edgelayer}
		\draw (5) to (4);
		\draw (4) to (6);
		\draw (7) to (3);
		\draw (1) to (3);
		\draw (0) to (2);
		\draw (2) to (8);
		\draw (2) to (3);
		\draw (3) to (4);
	\end{pgfonlayer}
\end{tikzpicture}
=
\begin{tikzpicture}
	\begin{pgfonlayer}{nodelayer}
		\node [style=nothing] (1) at (0, 1) {};
		\node [style=nothing] (2) at (-0.5, 1) {};
		\node [style=oplus] (3) at (0, 0.5) {};
		\node [style=dot] (4) at (-0.5, 0.5) {};
		\node [style=oneout] (5) at (-1, 0.5) {};
		\node [style=nothing] (6) at (-1, 0) {};
		\node [style=nothing] (7) at (-0.5, 0) {};
		\node [style=nothing] (8) at (0, 0) {};
	\end{pgfonlayer}
	\begin{pgfonlayer}{edgelayer}
		\draw (2) to (4);
		\draw (1) to (3);
		\draw (3) to (4);
		\draw (7) to (4);
		\draw (5) to (6);
		\draw (3) to (8);
	\end{pgfonlayer}
\end{tikzpicture}
\end{tabular}
$}


\item
\label{TOF.2}
{\hfil
$
\begin{tabular}{c}
\begin{tikzpicture}
	\begin{pgfonlayer}{nodelayer}
		\node [style=nothing] (2) at (-1.25, 0) {};
		\node [style=nothing] (3) at (-0.75, 0) {};
		\node [style=nothing] (4) at (-1.75, 2) {};
		\node [style=nothing] (5) at (-1.25, 2) {};
		\node [style=nothing] (6) at (-0.75, 2) {};
		\node [style=dot] (7) at (-1.75, 1) {};
		\node [style=dot] (8) at (-1.25, 1) {};
		\node [style=oplus] (9) at (-0.75, 1) {};
		\node [style=zeroin] (10) at (-1.75, 0) {};
	\end{pgfonlayer}
	\begin{pgfonlayer}{edgelayer}
		\draw (7) to (4);
		\draw (5) to (8);
		\draw (8) to (2);
		\draw (3) to (9);
		\draw (9) to (6);
		\draw (9) to (8);
		\draw (8) to (7);
		\draw (10) to (7);
	\end{pgfonlayer}
\end{tikzpicture}
=
\begin{tikzpicture}
	\begin{pgfonlayer}{nodelayer}
		\node [style=nothing] (3) at (-1.25, 0) {};
		\node [style=nothing] (4) at (-0.75, 0) {};
		\node [style=nothing] (5) at (-1.75, 1.5) {};
		\node [style=nothing] (6) at (-1.25, 1.5) {};
		\node [style=nothing] (7) at (-0.75, 1.5) {};
		\node [style=zeroin] (8) at (-1.75, 0) {};
	\end{pgfonlayer}
	\begin{pgfonlayer}{edgelayer}
		\draw (8) to (5);
		\draw (3) to (6);
		\draw (4) to (7);
	\end{pgfonlayer}
\end{tikzpicture}\\
{}\\
\begin{tikzpicture}
	\begin{pgfonlayer}{nodelayer}
		\node [style=nothing] (0) at (1, 0) {};
		\node [style=nothing] (1) at (1.5, 0) {};
		\node [style=nothing] (2) at (0.5, 2) {};
		\node [style=nothing] (3) at (1, 2) {};
		\node [style=nothing] (4) at (1.5, 2) {};
		\node [style=dot] (5) at (0.5, 1) {};
		\node [style=dot] (6) at (1, 1) {};
		\node [style=oplus] (7) at (1.5, 1) {};
		\node [style=zeroin] (8) at (0.5, 0) {};
	\end{pgfonlayer}
	\begin{pgfonlayer}{edgelayer}
		\draw (5) to (2);
		\draw (3) to (6);
		\draw (6) to (0);
		\draw (1) to (7);
		\draw (7) to (4);
		\draw (7) to (6);
		\draw (6) to (5);
		\draw (8) to (5);
	\end{pgfonlayer}
\end{tikzpicture}
=
\begin{tikzpicture}
	\begin{pgfonlayer}{nodelayer}
		\node [style=nothing] (0) at (-1.25, 2) {};
		\node [style=nothing] (1) at (-0.75, 2) {};
		\node [style=zeroin] (2) at (-1.75, 0.5) {};
		\node [style=nothing] (3) at (-1.25, 0.5) {};
		\node [style=nothing] (4) at (-0.75, 0.5) {};
		\node [style=nothing] (5) at (-1.75, 2) {};
	\end{pgfonlayer}
	\begin{pgfonlayer}{edgelayer}
		\draw (5) to (2);
		\draw (0) to (3);
		\draw (1) to (4);
	\end{pgfonlayer}
\end{tikzpicture}
\end{tabular}
$}

\item
\label{TOF.3}
{\hfil
$
\begin{tikzpicture}
	\begin{pgfonlayer}{nodelayer}
		\node [style=nothing] (0) at (-0.5, 0.5) {};
		\node [style=nothing] (1) at (0, 0.5) {};
		\node [style=nothing] (2) at (-1, 0.5) {};
		\node [style=nothing] (3) at (-1.5, 0.5) {};
		\node [style=nothing] (4) at (-2, 0.5) {};
		\node [style=dot] (5) at (-1.5, 1) {};
		\node [style=oplus] (6) at (-1, 1) {};
		\node [style=oplus] (7) at (-1, 1.5) {};
		\node [style=dot] (8) at (-0.5, 1.5) {};
		\node [style=dot] (9) at (-2, 1) {};
		\node [style=dot] (10) at (0, 1.5) {};
		\node [style=nothing] (11) at (-0.5, 2) {};
		\node [style=nothing] (12) at (-1.5, 2) {};
		\node [style=nothing] (13) at (-2, 2) {};
		\node [style=nothing] (14) at (0, 2) {};
		\node [style=nothing] (15) at (-1, 2) {};
	\end{pgfonlayer}
	\begin{pgfonlayer}{edgelayer}
		\draw (4) to (9);
		\draw (9) to (13);
		\draw (3) to (5);
		\draw (5) to (12);
		\draw (2) to (6);
		\draw (6) to (7);
		\draw (7) to (15);
		\draw (0) to (8);
		\draw (8) to (11);
		\draw (1) to (10);
		\draw (10) to (14);
		\draw (10) to (8);
		\draw (8) to (7);
		\draw (6) to (5);
		\draw (5) to (9);
	\end{pgfonlayer}
\end{tikzpicture}
=
\begin{tikzpicture}
	\begin{pgfonlayer}{nodelayer}
		\node [style=nothing] (1) at (-0.5, 0) {};
		\node [style=nothing] (2) at (0, 0) {};
		\node [style=nothing] (3) at (-1, 0) {};
		\node [style=nothing] (4) at (-1.5, 0) {};
		\node [style=nothing] (5) at (-2, 0) {};
		\node [style=dot] (6) at (-1.5, 1) {};
		\node [style=dot] (7) at (-0.5, 0.5) {};
		\node [style=dot] (8) at (-2, 1) {};
		\node [style=dot] (9) at (0, 0.5) {};
		\node [style=nothing] (10) at (-0.5, 1.5) {};
		\node [style=nothing] (11) at (-1.5, 1.5) {};
		\node [style=nothing] (12) at (-2, 1.5) {};
		\node [style=nothing] (13) at (0, 1.5) {};
		\node [style=nothing] (14) at (-1, 1.5) {};
		\node [style=oplus] (15) at (-1, 1) {};
		\node [style=oplus] (16) at (-1, 0.5) {};
	\end{pgfonlayer}
	\begin{pgfonlayer}{edgelayer}
		\draw (5) to (8);
		\draw (8) to (12);
		\draw (4) to (6);
		\draw (6) to (11);
		\draw (1) to (7);
		\draw (7) to (10);
		\draw (2) to (9);
		\draw (9) to (13);
		\draw (9) to (7);
		\draw (6) to (8);
		\draw (3) to (16);
		\draw (16) to (15);
		\draw (15) to (14);
		\draw (15) to (6);
		\draw (7) to (16);
	\end{pgfonlayer}
\end{tikzpicture}
$}


\item
\label{TOF.4}
{\hfil
$
\begin{tikzpicture}
	\begin{pgfonlayer}{nodelayer}
		\node [style=nothing] (2) at (-0.5, 0) {};
		\node [style=nothing] (3) at (0, 0) {};
		\node [style=nothing] (4) at (-1, 0) {};
		\node [style=nothing] (5) at (-1.5, 0) {};
		\node [style=nothing] (6) at (-2, 0) {};
		\node [style=dot] (7) at (-1.5, 0.5) {};
		\node [style=dot] (8) at (-1, 0.5) {};
		\node [style=dot] (9) at (-1, 1) {};
		\node [style=dot] (10) at (-0.5, 1) {};
		\node [style=oplus] (11) at (-2, 0.5) {};
		\node [style=oplus] (12) at (0, 1) {};
		\node [style=nothing] (13) at (-0.5, 1.5) {};
		\node [style=nothing] (14) at (-1.5, 1.5) {};
		\node [style=nothing] (15) at (-2, 1.5) {};
		\node [style=nothing] (16) at (0, 1.5) {};
		\node [style=nothing] (17) at (-1, 1.5) {};
	\end{pgfonlayer}
	\begin{pgfonlayer}{edgelayer}
		\draw (6) to (11);
		\draw (11) to (15);
		\draw (5) to (7);
		\draw (7) to (14);
		\draw (4) to (8);
		\draw (8) to (9);
		\draw (9) to (17);
		\draw (2) to (10);
		\draw (10) to (13);
		\draw (3) to (12);
		\draw (12) to (16);
		\draw (12) to (10);
		\draw (10) to (9);
		\draw (8) to (7);
		\draw (7) to (11);
	\end{pgfonlayer}
\end{tikzpicture}
=
\begin{tikzpicture}
	\begin{pgfonlayer}{nodelayer}
		\node [style=nothing] (3) at (-0.5, 0) {};
		\node [style=nothing] (4) at (0, 0) {};
		\node [style=nothing] (5) at (-1, 0) {};
		\node [style=nothing] (6) at (-1.5, 0) {};
		\node [style=nothing] (7) at (-2, 0) {};
		\node [style=dot] (8) at (-1.5, 1) {};
		\node [style=dot] (9) at (-0.5, 0.5) {};
		\node [style=oplus] (10) at (-2, 1) {};
		\node [style=oplus] (11) at (0, 0.5) {};
		\node [style=nothing] (12) at (-0.5, 1.5) {};
		\node [style=nothing] (13) at (-1.5, 1.5) {};
		\node [style=nothing] (14) at (-2, 1.5) {};
		\node [style=nothing] (15) at (0, 1.5) {};
		\node [style=nothing] (16) at (-1, 1.5) {};
		\node [style=dot] (17) at (-1, 1) {};
		\node [style=dot] (18) at (-1, 0.5) {};
	\end{pgfonlayer}
	\begin{pgfonlayer}{edgelayer}
		\draw (7) to (10);
		\draw (10) to (14);
		\draw (6) to (8);
		\draw (8) to (13);
		\draw (3) to (9);
		\draw (9) to (12);
		\draw (4) to (11);
		\draw (11) to (15);
		\draw (11) to (9);
		\draw (8) to (10);
		\draw (5) to (18);
		\draw (18) to (17);
		\draw (17) to (16);
		\draw (17) to (8);
		\draw (9) to (18);
	\end{pgfonlayer}
\end{tikzpicture}
$}

\item
\label{TOF.5}
{\hfil
$
\begin{tikzpicture}
	\begin{pgfonlayer}{nodelayer}
		\node [style=nothing] (4) at (-1, 2) {};
		\node [style=nothing] (5) at (-0.5, 2) {};
		\node [style=nothing] (6) at (-1.5, 2) {};
		\node [style=nothing] (7) at (-2, 2) {};
		\node [style=nothing] (8) at (-1, 3.5) {};
		\node [style=nothing] (9) at (-1.5, 3.5) {};
		\node [style=nothing] (10) at (-2, 3.5) {};
		\node [style=nothing] (11) at (-0.5, 3.5) {};
		\node [style=oplus] (12) at (-2, 2.5) {};
		\node [style=oplus] (13) at (-0.5, 3) {};
		\node [style=dot] (14) at (-1.5, 2.5) {};
		\node [style=dot] (15) at (-1, 2.5) {};
		\node [style=dot] (16) at (-1.5, 3) {};
		\node [style=dot] (17) at (-1, 3) {};
	\end{pgfonlayer}
	\begin{pgfonlayer}{edgelayer}
		\draw (7) to (12);
		\draw (12) to (10);
		\draw (9) to (16);
		\draw (16) to (14);
		\draw (14) to (6);
		\draw (4) to (15);
		\draw (15) to (17);
		\draw (17) to (8);
		\draw (11) to (13);
		\draw (13) to (5);
		\draw (14) to (15);
		\draw (14) to (12);
		\draw (16) to (17);
		\draw (17) to (13);
	\end{pgfonlayer}
\end{tikzpicture}
=
\begin{tikzpicture}
	\begin{pgfonlayer}{nodelayer}
		\node [style=nothing] (5) at (-1, 2) {};
		\node [style=nothing] (6) at (-0.5, 2) {};
		\node [style=nothing] (7) at (-1.5, 2) {};
		\node [style=nothing] (8) at (-2, 2) {};
		\node [style=nothing] (9) at (-1, 3.5) {};
		\node [style=nothing] (10) at (-1.5, 3.5) {};
		\node [style=nothing] (11) at (-2, 3.5) {};
		\node [style=nothing] (12) at (-0.5, 3.5) {};
		\node [style=oplus] (13) at (-2, 3) {};
		\node [style=dot] (14) at (-1.5, 3) {};
		\node [style=dot] (15) at (-1, 3) {};
		\node [style=oplus] (16) at (-0.5, 2.5) {};
		\node [style=dot] (17) at (-1, 2.5) {};
		\node [style=dot] (18) at (-1.5, 2.5) {};
	\end{pgfonlayer}
	\begin{pgfonlayer}{edgelayer}
		\draw (14) to (15);
		\draw (14) to (13);
		\draw (18) to (17);
		\draw (17) to (16);
		\draw (8) to (13);
		\draw (13) to (11);
		\draw (10) to (14);
		\draw (14) to (18);
		\draw (18) to (7);
		\draw (5) to (17);
		\draw (17) to (15);
		\draw (15) to (9);
		\draw (12) to (16);
		\draw (16) to (6);
	\end{pgfonlayer}
\end{tikzpicture}
$}


\item
\label{TOF.6}
{\hfil
$
\begin{tikzpicture}
	\begin{pgfonlayer}{nodelayer}
		\node [style=nothing] (6) at (-1, 2) {};
		\node [style=nothing] (7) at (-1.5, 2) {};
		\node [style=nothing] (8) at (-2, 2) {};
		\node [style=nothing] (9) at (-1, 3.5) {};
		\node [style=nothing] (10) at (-1.5, 3.5) {};
		\node [style=nothing] (11) at (-2, 3.5) {};
		\node [style=nothing] (12) at (-0.5, 3.5) {};
		\node [style=oplus] (13) at (-0.5, 2.5) {};
		\node [style=dot] (14) at (-1.5, 3) {};
		\node [style=dot] (15) at (-1, 3) {};
		\node [style=dot] (16) at (-1, 2.5) {};
		\node [style=oplus] (17) at (-0.5, 3) {};
		\node [style=nothing] (18) at (-0.5, 2) {};
		\node [style=dot] (19) at (-2, 2.5) {};
	\end{pgfonlayer}
	\begin{pgfonlayer}{edgelayer}
		\draw (14) to (7);
		\draw (6) to (15);
		\draw (15) to (16);
		\draw (16) to (9);
		\draw (12) to (13);
		\draw (14) to (15);
		\draw (16) to (13);
		\draw (18) to (17);
		\draw (17) to (13);
		\draw (14) to (10);
		\draw (15) to (17);
		\draw (16) to (19);
		\draw (19) to (11);
		\draw (19) to (8);
	\end{pgfonlayer}
\end{tikzpicture}
=
\begin{tikzpicture}
	\begin{pgfonlayer}{nodelayer}
		\node [style=nothing] (7) at (-1, 2) {};
		\node [style=nothing] (8) at (-1.5, 2) {};
		\node [style=nothing] (9) at (-2, 2) {};
		\node [style=nothing] (10) at (-1, 3.5) {};
		\node [style=nothing] (11) at (-1.5, 3.5) {};
		\node [style=nothing] (12) at (-2, 3.5) {};
		\node [style=nothing] (13) at (-0.5, 3.5) {};
		\node [style=oplus] (14) at (-0.5, 3) {};
		\node [style=dot] (15) at (-1.5, 2.5) {};
		\node [style=dot] (16) at (-1, 2.5) {};
		\node [style=dot] (17) at (-1, 3) {};
		\node [style=oplus] (18) at (-0.5, 2.5) {};
		\node [style=nothing] (19) at (-0.5, 2) {};
		\node [style=dot] (20) at (-2, 3) {};
	\end{pgfonlayer}
	\begin{pgfonlayer}{edgelayer}
		\draw (15) to (8);
		\draw (7) to (16);
		\draw (16) to (17);
		\draw (17) to (10);
		\draw (13) to (14);
		\draw (15) to (16);
		\draw (17) to (14);
		\draw (19) to (18);
		\draw (18) to (14);
		\draw (15) to (11);
		\draw (16) to (18);
		\draw (17) to (20);
		\draw (20) to (12);
		\draw (20) to (9);
	\end{pgfonlayer}
\end{tikzpicture}
$}

\item
\label{TOF.7}
{\hfil
$
\begin{tikzpicture}
	\begin{pgfonlayer}{nodelayer}
		\node [style=nothing] (8) at (0, 2) {};
		\node [style=nothing] (9) at (-0.5, 2) {};
		\node [style=nothing] (10) at (-0.5, 4.5) {};
		\node [style=nothing] (11) at (0, 4.5) {};
		\node [style=zeroout] (12) at (0.5, 4.5) {};
		\node [style=oplus] (13) at (0.5, 4) {};
		\node [style=dot] (14) at (0, 4) {};
		\node [style=dot] (15) at (-0.5, 2.5) {};
		\node [style=oplus] (16) at (0.5, 2.5) {};
		\node [style=zeroout] (17) at (0.5, 3) {};
		\node [style=onein] (18) at (0.5, 2) {};
		\node [style=onein] (19) at (0.5, 3.5) {};
	\end{pgfonlayer}
	\begin{pgfonlayer}{edgelayer}
		\draw (9) to (15);
		\draw (15) to (10);
		\draw (11) to (14);
		\draw (14) to (8);
		\draw (16) to (17);
		\draw (16) to (15);
		\draw (13) to (12);
		\draw (13) to (14);
		\draw (18) to (16);
		\draw (19) to (13);
	\end{pgfonlayer}
\end{tikzpicture}
=
\begin{tikzpicture}
	\begin{pgfonlayer}{nodelayer}
		\node [style=nothing] (9) at (0, 2) {};
		\node [style=nothing] (10) at (-0.5, 2) {};
		\node [style=nothing] (11) at (-0.5, 3) {};
		\node [style=nothing] (12) at (0, 3) {};
		\node [style=dot] (13) at (-0.5, 2.5) {};
		\node [style=dot] (14) at (0, 2.5) {};
		\node [style=onein] (15) at (0.5, 2) {};
		\node [style=zeroout] (16) at (0.5, 3) {};
		\node [style=oplus] (17) at (0.5, 2.5) {};
	\end{pgfonlayer}
	\begin{pgfonlayer}{edgelayer}
		\draw (10) to (13);
		\draw (13) to (11);
		\draw (12) to (14);
		\draw (14) to (9);
		\draw (15) to (17);
		\draw (17) to (16);
		\draw (17) to (14);
		\draw (14) to (13);
	\end{pgfonlayer}
\end{tikzpicture}
$}
%
%\item
%\label{TOF.7}
%{\hfil
%$
%\begin{tikzpicture}
%	\begin{pgfonlayer}{nodelayer}
%		\node [style=nothing] (0) at (0, -0) {};
%		\node [style=nothing] (1) at (1.5, -0) {};
%		\node [style=onein] (2) at (0.4, 0.5) {};
%		\node [style=zeroout] (3) at (1.1, 0.5) {};
%	\end{pgfonlayer}
%	\begin{pgfonlayer}{edgelayer}
%		\draw (2) to (3);
%		\draw (0) to (1);
%	\end{pgfonlayer}
%\end{tikzpicture}
%=
%\begin{tikzpicture}
%	\begin{pgfonlayer}{nodelayer}
%		\node [style=nothing] (0) at (0, -0) {};
%		\node [style=nothing] (1) at (1.5, -0) {};
%		\node [style=onein] (2) at (0.4, 0.5) {};
%		\node [style=zeroout] (3) at (1.1, 0.5) {};
%		\node [style=onein] (4) at (1, -0) {};
%		\node [style=oneout] (5) at (0.5000002, -0) {};
%	\end{pgfonlayer}
%	\begin{pgfonlayer}{edgelayer}
%		\draw (2) to (3);
%		\draw (5) to (0);
%		\draw (4) to (1);
%	\end{pgfonlayer}
%\end{tikzpicture}
%$}

\item
\label{TOF.8}
{\hfil
$
\begin{tikzpicture}
	\begin{pgfonlayer}{nodelayer}
		\node [style=onein] (10) at (0, 2) {};
		\node [style=oneout] (11) at (0, 3) {};
	\end{pgfonlayer}
	\begin{pgfonlayer}{edgelayer}
		\draw (10) to (11);
	\end{pgfonlayer}
\end{tikzpicture}
=
\begin{tikzpicture}
	\begin{pgfonlayer}{nodelayer}
		\node [style=rn] (11) at (0, 2) {};
		\node [style=rn] (12) at (0, 3) {};
	\end{pgfonlayer}
\end{tikzpicture}
%\hspace*{-.8cm}
%\begin{tikzpicture}[scale=.5]
%\begin{pgfonlayer}{nodelayer}
%\begin{tikzpicture}
%\node[cloud, cloud puffs=15.7,minimum width=3cm, draw,] (cloud) at (0,0) {$1_0$};
%\end{tikzpicture}
%\end{pgfonlayer}
%\begin{pgfonlayer}{edgelayer}
%\end{pgfonlayer}
%\end{tikzpicture}
$}

\item
\label{TOF.9}
{\hfil
$
\begin{tikzpicture}
	\begin{pgfonlayer}{nodelayer}
		\node [style=nothing] (12) at (-1.75, 2) {};
		\node [style=nothing] (13) at (-1.25, 2) {};
		\node [style=nothing] (14) at (-0.75, 2) {};
		\node [style=dot] (15) at (-1.75, 2.5) {};
		\node [style=dot] (16) at (-1.25, 2.5) {};
		\node [style=oplus] (17) at (-0.75, 2.5) {};
		\node [style=dot] (18) at (-1.75, 3) {};
		\node [style=oplus] (19) at (-0.75, 3) {};
		\node [style=dot] (20) at (-1.25, 3) {};
		\node [style=nothing] (21) at (-1.25, 3.5) {};
		\node [style=nothing] (22) at (-0.75, 3.5) {};
		\node [style=nothing] (23) at (-1.75, 3.5) {};
	\end{pgfonlayer}
	\begin{pgfonlayer}{edgelayer}
		\draw (12) to (15);
		\draw (13) to (16);
		\draw (14) to (17);
		\draw (15) to (16);
		\draw (16) to (17);
		\draw (18) to (20);
		\draw (20) to (19);
		\draw (15) to (18);
		\draw (18) to (23);
		\draw (16) to (20);
		\draw (20) to (21);
		\draw (17) to (19);
		\draw (19) to (22);
	\end{pgfonlayer}
\end{tikzpicture}
=
\begin{tikzpicture}
	\begin{pgfonlayer}{nodelayer}
		\node [style=nothing] (13) at (-1.75, 2) {};
		\node [style=nothing] (14) at (-1.25, 2) {};
		\node [style=nothing] (15) at (-0.75, 2) {};
		\node [style=nothing] (16) at (-1.25, 3.5) {};
		\node [style=nothing] (17) at (-0.75, 3.5) {};
		\node [style=nothing] (18) at (-1.75, 3.5) {};
	\end{pgfonlayer}
	\begin{pgfonlayer}{edgelayer}
		\draw (13) to (18);
		\draw (14) to (16);
		\draw (15) to (17);
	\end{pgfonlayer}
\end{tikzpicture}
$}

\item
\label{TOF.10}
{\hfil
$
\begin{tikzpicture}
	\begin{pgfonlayer}{nodelayer}
		\node [style=nothing] (14) at (0, 2) {};
		\node [style=nothing] (15) at (-0.5, 2) {};
		\node [style=nothing] (16) at (-1, 2) {};
		\node [style=nothing] (17) at (-1.5, 2) {};
		\node [style=dot] (18) at (-1, 2.5) {};
		\node [style=dot] (19) at (-0.5, 2.5) {};
		\node [style=oplus] (20) at (0, 2.5) {};
		\node [style=dot] (21) at (-1.5, 3) {};
		\node [style=oplus] (22) at (-0.5, 3) {};
		\node [style=dot] (23) at (-1, 3) {};
		\node [style=dot] (24) at (-1, 3.5) {};
		\node [style=oplus] (25) at (0, 3.5) {};
		\node [style=dot] (26) at (-0.5, 3.5) {};
		\node [style=nothing] (27) at (-1.5, 4) {};
		\node [style=nothing] (28) at (-0.5, 4) {};
		\node [style=nothing] (29) at (-1, 4) {};
		\node [style=nothing] (30) at (0, 4) {};
	\end{pgfonlayer}
	\begin{pgfonlayer}{edgelayer}
		\draw (18) to (19);
		\draw (19) to (20);
		\draw (21) to (23);
		\draw (23) to (22);
		\draw (24) to (26);
		\draw (26) to (25);
		\draw (17) to (21);
		\draw (21) to (27);
		\draw (29) to (24);
		\draw (24) to (23);
		\draw (23) to (18);
		\draw (18) to (16);
		\draw (15) to (19);
		\draw (19) to (22);
		\draw (22) to (26);
		\draw (26) to (28);
		\draw (30) to (25);
		\draw (25) to (20);
		\draw (20) to (14);
	\end{pgfonlayer}
\end{tikzpicture}
=
\begin{tikzpicture}
	\begin{pgfonlayer}{nodelayer}
		\node [style=nothing] (15) at (0, 2) {};
		\node [style=nothing] (16) at (-0.5, 2) {};
		\node [style=nothing] (17) at (-1, 2) {};
		\node [style=nothing] (18) at (-1.5, 2) {};
		\node [style=nothing] (19) at (-1.5, 3.5) {};
		\node [style=nothing] (20) at (-0.5, 3.5) {};
		\node [style=nothing] (21) at (-1, 3.5) {};
		\node [style=nothing] (22) at (0, 3.5) {};
		\node [style=dot] (23) at (-1.5, 2.5) {};
		\node [style=dot] (24) at (-1, 2.5) {};
		\node [style=dot] (25) at (-1.5, 3) {};
		\node [style=dot] (26) at (-1, 3) {};
		\node [style=oplus] (27) at (-0.5, 3) {};
		\node [style=oplus] (28) at (0, 2.5) {};
	\end{pgfonlayer}
	\begin{pgfonlayer}{edgelayer}
		\draw (18) to (23);
		\draw (23) to (25);
		\draw (25) to (19);
		\draw (21) to (26);
		\draw (26) to (24);
		\draw (24) to (17);
		\draw (16) to (27);
		\draw (27) to (20);
		\draw (22) to (28);
		\draw (28) to (15);
		\draw (28) to (24);
		\draw (24) to (23);
		\draw (25) to (26);
		\draw (26) to (27);
	\end{pgfonlayer}
\end{tikzpicture}
$}

\item
\label{TOF.11}
{\hfil
$
\begin{tikzpicture}
	\begin{pgfonlayer}{nodelayer}
		\node [style=nothing] (16) at (0, 2) {};
		\node [style=nothing] (17) at (-0.5, 2) {};
		\node [style=nothing] (18) at (-1, 2) {};
		\node [style=nothing] (19) at (-1.5, 2) {};
		\node [style=nothing] (20) at (-0.5, 4) {};
		\node [style=nothing] (21) at (0, 4) {};
		\node [style=dot] (22) at (-1.5, 2.5) {};
		\node [style=dot] (23) at (-1, 3) {};
		\node [style=dot] (24) at (-0.5, 3) {};
		\node [style=oplus] (25) at (-1, 2.5) {};
		\node [style=oplus] (26) at (0, 3) {};
		\node [style=nothing] (27) at (-1.5, 4) {};
		\node [style=nothing] (28) at (-1, 4) {};
		\node [style=oplus] (29) at (-1, 3.5) {};
		\node [style=dot] (30) at (-1.5, 3.5) {};
	\end{pgfonlayer}
	\begin{pgfonlayer}{edgelayer}
		\draw (22) to (25);
		\draw (23) to (24);
		\draw (24) to (26);
		\draw (16) to (26);
		\draw (26) to (21);
		\draw (20) to (24);
		\draw (24) to (17);
		\draw (18) to (25);
		\draw (25) to (23);
		\draw (22) to (19);
		\draw (22) to (30);
		\draw (30) to (27);
		\draw (28) to (29);
		\draw (29) to (23);
		\draw (29) to (30);
	\end{pgfonlayer}
\end{tikzpicture}
=
\begin{tikzpicture}
	\begin{pgfonlayer}{nodelayer}
		\node [style=nothing] (17) at (0, 2) {};
		\node [style=nothing] (18) at (-1, 2) {};
		\node [style=nothing] (19) at (-0.5, 2) {};
		\node [style=nothing] (20) at (-1.5, 2) {};
		\node [style=dot] (21) at (-1.5, 2.5) {};
		\node [style=dot] (22) at (-0.5, 2.5) {};
		\node [style=oplus] (23) at (0, 2.5) {};
		\node [style=nothing] (24) at (-0.5, 3.5) {};
		\node [style=nothing] (25) at (-1, 3.5) {};
		\node [style=nothing] (26) at (-1.5, 3.5) {};
		\node [style=nothing] (27) at (0, 3.5) {};
		\node [style=dot] (28) at (-1, 3) {};
		\node [style=dot] (29) at (-0.5, 3) {};
		\node [style=oplus] (30) at (0, 3) {};
	\end{pgfonlayer}
	\begin{pgfonlayer}{edgelayer}
		\draw (20) to (21);
		\draw (19) to (22);
		\draw (23) to (17);
		\draw (23) to (22);
		\draw (22) to (21);
		\draw (28) to (18);
		\draw (22) to (29);
		\draw (29) to (24);
		\draw (27) to (30);
		\draw (30) to (23);
		\draw (30) to (29);
		\draw (29) to (28);
		\draw (21) to (26);
		\draw (28) to (25);
	\end{pgfonlayer}
\end{tikzpicture}
$}

\item
\label{TOF.12}
{\hfil
$
\begin{tikzpicture}
	\begin{pgfonlayer}{nodelayer}
		\node [style=nothing] (18) at (-0.5, 2) {};
		\node [style=nothing] (19) at (0, 2) {};
		\node [style=nothing] (20) at (-1, 2) {};
		\node [style=nothing] (21) at (-1.5, 2) {};
		\node [style=nothing] (22) at (-0.5, 4) {};
		\node [style=nothing] (23) at (-1.5, 4) {};
		\node [style=nothing] (24) at (0, 4) {};
		\node [style=nothing] (25) at (-1, 4) {};
		\node [style=dot] (26) at (-1.5, 2.5) {};
		\node [style=dot] (27) at (-1, 2.5) {};
		\node [style=oplus] (28) at (-0.5, 2.5) {};
		\node [style=oplus] (29) at (0, 3) {};
		\node [style=dot] (30) at (-1, 3) {};
		\node [style=dot] (31) at (-0.5, 3) {};
		\node [style=oplus] (32) at (-0.5, 3.5) {};
		\node [style=dot] (33) at (-1.5, 3.5) {};
		\node [style=dot] (34) at (-1, 3.5) {};
	\end{pgfonlayer}
	\begin{pgfonlayer}{edgelayer}
		\draw (26) to (27);
		\draw (27) to (28);
		\draw (30) to (31);
		\draw (31) to (29);
		\draw (33) to (34);
		\draw (34) to (32);
		\draw (21) to (26);
		\draw (26) to (33);
		\draw (33) to (23);
		\draw (25) to (34);
		\draw (34) to (30);
		\draw (30) to (27);
		\draw (27) to (20);
		\draw (18) to (28);
		\draw (28) to (31);
		\draw (31) to (32);
		\draw (32) to (22);
		\draw (24) to (29);
		\draw (29) to (19);
	\end{pgfonlayer}
\end{tikzpicture}
=
\begin{tikzpicture}
	\begin{pgfonlayer}{nodelayer}
		\node [style=nothing] (19) at (-0.5, 3.5) {};
		\node [style=nothing] (20) at (0, 3.5) {};
		\node [style=nothing] (21) at (-1, 3.5) {};
		\node [style=nothing] (22) at (-1.5, 3.5) {};
		\node [style=nothing] (23) at (-0.5, 5) {};
		\node [style=nothing] (24) at (-1.5, 5) {};
		\node [style=nothing] (25) at (0, 5) {};
		\node [style=nothing] (26) at (-1, 5) {};
		\node [style=dot] (27) at (-1, 4.5) {};
		\node [style=dot] (28) at (-0.5, 4.5) {};
		\node [style=dot] (29) at (-1.5, 4) {};
		\node [style=dot] (30) at (-1, 4) {};
		\node [style=oplus] (31) at (0, 4) {};
		\node [style=oplus] (32) at (0, 4.5) {};
	\end{pgfonlayer}
	\begin{pgfonlayer}{edgelayer}
		\draw (27) to (28);
		\draw (22) to (29);
		\draw (29) to (24);
		\draw (21) to (30);
		\draw (30) to (27);
		\draw (27) to (26);
		\draw (19) to (28);
		\draw (28) to (23);
		\draw (20) to (31);
		\draw (31) to (32);
		\draw (32) to (25);
		\draw (32) to (28);
		\draw (31) to (30);
		\draw (30) to (29);
	\end{pgfonlayer}
\end{tikzpicture}
$}

\item
\label{TOF.13}
{\hfil
$
\begin{tikzpicture}
	\begin{pgfonlayer}{nodelayer}
		\node [style=nothing] (20) at (0, 3.5) {};
		\node [style=nothing] (21) at (-1, 3.5) {};
		\node [style=nothing] (22) at (-0.5, 3.5) {};
		\node [style=nothing] (23) at (-1.5, 3.5) {};
		\node [style=nothing] (24) at (0, 5.5) {};
		\node [style=dot] (25) at (-1.5, 4) {};
		\node [style=dot] (26) at (-1, 4) {};
		\node [style=dot] (27) at (-0.5, 4.5) {};
		\node [style=oplus] (28) at (-0.5, 4) {};
		\node [style=oplus] (29) at (0, 4.5) {};
		\node [style=nothing] (30) at (-0.5, 5.5) {};
		\node [style=nothing] (31) at (-1.5, 5.5) {};
		\node [style=nothing] (32) at (-1, 5.5) {};
		\node [style=oplus] (33) at (-0.5, 5) {};
		\node [style=dot] (34) at (-1, 5) {};
		\node [style=dot] (35) at (-1.5, 5) {};
	\end{pgfonlayer}
	\begin{pgfonlayer}{edgelayer}
		\draw (25) to (23);
		\draw (26) to (21);
		\draw (22) to (28);
		\draw (28) to (27);
		\draw (24) to (29);
		\draw (29) to (20);
		\draw (28) to (26);
		\draw (26) to (25);
		\draw (29) to (27);
		\draw (25) to (35);
		\draw (35) to (31);
		\draw (32) to (34);
		\draw (34) to (26);
		\draw (27) to (33);
		\draw (33) to (30);
		\draw (33) to (34);
		\draw (34) to (35);
	\end{pgfonlayer}
\end{tikzpicture}
=
\begin{tikzpicture}
	\begin{pgfonlayer}{nodelayer}
		\node [style=nothing] (21) at (0, 3.5) {};
		\node [style=nothing] (22) at (-1, 3.5) {};
		\node [style=nothing] (23) at (-0.5, 3.5) {};
		\node [style=nothing] (24) at (-1.5, 3.5) {};
		\node [style=dot] (25) at (-1.5, 4) {};
		\node [style=dot] (26) at (-1, 4) {};
		\node [style=oplus] (27) at (0, 4) {};
		\node [style=nothing] (28) at (-0.5, 5) {};
		\node [style=nothing] (29) at (-1, 5) {};
		\node [style=nothing] (30) at (0, 5) {};
		\node [style=nothing] (31) at (-1.5, 5) {};
		\node [style=dot] (32) at (-0.5, 4.5) {};
		\node [style=oplus] (33) at (0, 4.5) {};
	\end{pgfonlayer}
	\begin{pgfonlayer}{edgelayer}
		\draw (21) to (27);
		\draw (22) to (26);
		\draw (25) to (24);
		\draw (25) to (26);
		\draw (26) to (27);
		\draw (32) to (33);
		\draw (33) to (30);
		\draw (33) to (27);
		\draw (23) to (32);
		\draw (25) to (31);
		\draw (29) to (26);
		\draw (32) to (28);
	\end{pgfonlayer}
\end{tikzpicture}
$}

\item
\label{TOF.14}
{\hfil
$
\begin{tikzpicture}
	\begin{pgfonlayer}{nodelayer}
		\node [style=nothing] (22) at (0, 3.5) {};
		\node [style=nothing] (23) at (-0.5, 3.5) {};
		\node [style=nothing] (24) at (-0.5, 5.5) {};
		\node [style=nothing] (25) at (0, 5.5) {};
		\node [style=oplus] (26) at (0, 4) {};
		\node [style=oplus] (27) at (0, 5) {};
		\node [style=oplus] (28) at (-0.5, 4.5) {};
		\node [style=dot] (29) at (-0.5, 5) {};
		\node [style=dot] (30) at (0, 4.5) {};
		\node [style=dot] (31) at (-0.5, 4) {};
	\end{pgfonlayer}
	\begin{pgfonlayer}{edgelayer}
		\draw (23) to (31);
		\draw (31) to (28);
		\draw (28) to (29);
		\draw (29) to (24);
		\draw (25) to (27);
		\draw (27) to (30);
		\draw (30) to (26);
		\draw (26) to (22);
		\draw (26) to (31);
		\draw (30) to (28);
		\draw (27) to (29);
	\end{pgfonlayer}
\end{tikzpicture}
=
\begin{tikzpicture}
	\begin{pgfonlayer}{nodelayer}
		\node [style=nothing] (23) at (0, 3.5) {};
		\node [style=nothing] (24) at (-0.5, 3.5) {};
		\node [style=nothing] (25) at (-0.5, 4.5) {};
		\node [style=nothing] (26) at (0, 4.5) {};
	\end{pgfonlayer}
	\begin{pgfonlayer}{edgelayer}
		\draw [in=-90, out=90, looseness=1.25] (24) to (26);
		\draw [in=-90, out=90, looseness=1.25] (23) to (25);
	\end{pgfonlayer}
\end{tikzpicture}
$}

\item
\label{TOF.15}
{\hfil
$
\begin{tikzpicture}
	\begin{pgfonlayer}{nodelayer}
		\node [style=nothing] (24) at (-1.75, 3.5) {};
		\node [style=nothing] (25) at (-1.25, 3.5) {};
		\node [style=nothing] (26) at (-0.75, 3.5) {};
		\node [style=nothing] (27) at (-1.75, 5.5) {};
		\node [style=nothing] (28) at (-1.25, 5.5) {};
		\node [style=nothing] (29) at (-0.75, 5.5) {};
		\node [style=dot] (30) at (-1.75, 4.5) {};
		\node [style=dot] (31) at (-1.25, 4.5) {};
		\node [style=oplus] (32) at (-0.75, 4.5) {};
	\end{pgfonlayer}
	\begin{pgfonlayer}{edgelayer}
		\draw (24) to (30);
		\draw (30) to (27);
		\draw (28) to (31);
		\draw (31) to (25);
		\draw (26) to (32);
		\draw (32) to (29);
		\draw (32) to (31);
		\draw (31) to (30);
	\end{pgfonlayer}
\end{tikzpicture}
=
\begin{tikzpicture}
	\begin{pgfonlayer}{nodelayer}
		\node [style=nothing] (25) at (-1.75, 3.5) {};
		\node [style=nothing] (26) at (-1.25, 3.5) {};
		\node [style=nothing] (27) at (-0.75, 3.5) {};
		\node [style=dot] (28) at (-1.75, 4.5) {};
		\node [style=dot] (29) at (-1.25, 4.5) {};
		\node [style=oplus] (30) at (-0.75, 4.5) {};
		\node [style=nothing] (31) at (-1.75, 5.5) {};
		\node [style=nothing] (32) at (-1.25, 5.5) {};
		\node [style=nothing] (33) at (-0.75, 5.5) {};
	\end{pgfonlayer}
	\begin{pgfonlayer}{edgelayer}
		\draw [in=-90, out=90, looseness=1.25] (25) to (29);
		\draw [in=-90, out=90, looseness=1.25] (29) to (31);
		\draw [in=-90, out=90, looseness=1.25] (28) to (32);
		\draw [in=90, out=-90, looseness=1.25] (28) to (26);
		\draw (27) to (30);
		\draw (30) to (33);
		\draw (28) to (29);
		\draw (29) to (30);
	\end{pgfonlayer}
\end{tikzpicture}
$}

\item
\label{TOF.16}
{\hfil
$
\begin{tikzpicture}
	\begin{pgfonlayer}{nodelayer}
		\node [style=nothing] (26) at (0, 3.5) {};
		\node [style=nothing] (27) at (-0.5, 3.5) {};
		\node [style=nothing] (28) at (-1.5, 3.5) {};
		\node [style=nothing] (29) at (-2, 3.5) {};
		\node [style=zeroin] (30) at (-1, 3.5) {};
		\node [style=oplus] (31) at (-1, 4) {};
		\node [style=oplus] (32) at (-1, 5) {};
		\node [style=dot] (33) at (-1, 4.5) {};
		\node [style=dot] (34) at (-0.5, 4.5) {};
		\node [style=dot] (35) at (-1.5, 4) {};
		\node [style=dot] (36) at (-2, 4) {};
		\node [style=dot] (37) at (-1.5, 5) {};
		\node [style=dot] (38) at (-2, 5) {};
		\node [style=oplus] (39) at (0, 4.5) {};
		\node [style=zeroout] (40) at (-1, 5.5) {};
		\node [style=nothing] (41) at (0, 5.5) {};
		\node [style=nothing] (42) at (-2, 5.5) {};
		\node [style=nothing] (43) at (-0.5, 5.5) {};
		\node [style=nothing] (44) at (-1.5, 5.5) {};
	\end{pgfonlayer}
	\begin{pgfonlayer}{edgelayer}
		\draw (29) to (36);
		\draw (36) to (38);
		\draw (38) to (42);
		\draw (37) to (35);
		\draw (41) to (39);
		\draw (39) to (26);
		\draw (39) to (34);
		\draw (34) to (33);
		\draw (35) to (31);
		\draw (35) to (36);
		\draw (38) to (37);
		\draw (32) to (37);
		\draw (30) to (31);
		\draw (31) to (33);
		\draw (33) to (32);
		\draw (40) to (32);
		\draw [style=simple] (43) to (34);
		\draw [style=simple] (34) to (27);
		\draw [style=simple] (28) to (35);
		\draw [style=simple] (37) to (44);
	\end{pgfonlayer}
\end{tikzpicture}
=
\begin{tikzpicture}
	\begin{pgfonlayer}{nodelayer}
		\node [style=nothing] (27) at (0, 3.5) {};
		\node [style=nothing] (28) at (-0.5, 3.5) {};
		\node [style=nothing] (29) at (-1.5, 3.5) {};
		\node [style=nothing] (30) at (-2, 3.5) {};
		\node [style=zeroin] (31) at (-1, 4.25) {};
		\node [style=oplus] (32) at (-1, 4.75) {};
		\node [style=oplus] (33) at (-1, 5.75) {};
		\node [style=dot] (34) at (-1, 5.25) {};
		\node [style=dot] (35) at (-0.5, 5.25) {};
		\node [style=dot] (36) at (-1.5, 4.75) {};
		\node [style=dot] (37) at (-2, 4.75) {};
		\node [style=dot] (38) at (-1.5, 5.75) {};
		\node [style=dot] (39) at (-2, 5.75) {};
		\node [style=oplus] (40) at (0, 5.25) {};
		\node [style=zeroout] (41) at (-1, 6.25) {};
		\node [style=nothing] (42) at (0, 7) {};
		\node [style=nothing] (43) at (-2, 7) {};
		\node [style=nothing] (44) at (-0.5, 7) {};
		\node [style=nothing] (45) at (-1.5, 7) {};
		\node [style=none] (46) at (-1.5, 6.5) {};
		\node [style=none] (47) at (-0.5, 6.5) {};
		\node [style=none] (48) at (-0.5, 4) {};
		\node [style=none] (49) at (-1.5, 4) {};
	\end{pgfonlayer}
	\begin{pgfonlayer}{edgelayer}
		\draw (30) to (37);
		\draw (37) to (39);
		\draw (39) to (43);
		\draw (38) to (36);
		\draw (42) to (40);
		\draw (40) to (27);
		\draw (40) to (35);
		\draw (35) to (34);
		\draw (36) to (32);
		\draw (36) to (37);
		\draw (39) to (38);
		\draw (33) to (38);
		\draw (31) to (32);
		\draw (32) to (34);
		\draw (34) to (33);
		\draw (33) to (41);
		\draw [in=90, out=-90, looseness=0.50] (44) to (46.center);
		\draw [in=90, out=-90, looseness=0.75] (45) to (47.center);
		\draw (47.center) to (35);
		\draw (35) to (48.center);
		\draw [in=90, out=-105, looseness=0.50] (48.center) to (29);
		\draw [in=-90, out=90, looseness=0.50] (28) to (49.center);
		\draw (49.center) to (36);
		\draw (38) to (46.center);
	\end{pgfonlayer}
\end{tikzpicture}
$}
\end{enumerate}
\end{multicols}
\
\end{mdframed}
}}
\caption{The identities of \texorpdfstring{$\TOF$}{TOF}}
\label{fig:TOF}
\end{figure}


The Toffoli gate, $\tof$ as well as the ancillae are interpreted as follows:

$$
\left\llbracket\
\begin{tikzpicture}
	\begin{pgfonlayer}{nodelayer}
		\node [style=nothing] (29) at (0, 3.5) {};
		\node [style=nothing] (30) at (-0.5, 3.5) {};
		\node [style=nothing] (31) at (-0.5, 4.5) {};
		\node [style=nothing] (32) at (0, 4.5) {};
		\node [style=oplus] (33) at (0, 4) {};
		\node [style=dot] (34) at (-0.5, 4) {};
		\node [style=nothing] (35) at (-1, 3.5) {};
		\node [style=nothing] (36) at (-1, 4.5) {};
		\node [style=dot] (37) at (-1, 4) {};
	\end{pgfonlayer}
	\begin{pgfonlayer}{edgelayer}
		\draw (30) to (34);
		\draw (34) to (31);
		\draw (32) to (33);
		\draw (33) to (29);
		\draw (33) to (34);
		\draw (35) to (37);
		\draw (37) to (36);
		\draw (37) to (34);
	\end{pgfonlayer}
\end{tikzpicture}
\ \right\rrbracket
=\sum_{x_0,x_1,x_2=0}^{1} |x_0,x_1, x_0\cdot x_1+x_2 \rangle\langle x_0, x_1,x_2|
\ ,\hspace*{.5cm}
\left\llbracket\
\begin{tikzpicture}
	\begin{pgfonlayer}{nodelayer}
		\node [style=onein] (15) at (8.75, -2.5) {};
		\node [style=none] (19) at (8.75, -1.75) {};
	\end{pgfonlayer}
	\begin{pgfonlayer}{edgelayer}
		\draw (19.center) to (15);
	\end{pgfonlayer}
\end{tikzpicture}
\ \right\rrbracket
=|0 \rangle
 \ ,\hspace*{.5cm}
\left\llbracket\
\begin{tikzpicture}
	\begin{pgfonlayer}{nodelayer}
		\node [style=none] (15) at (8.75, -2.5) {};
		\node [style=oneout] (19) at (8.75, -1.75) {};
	\end{pgfonlayer}
	\begin{pgfonlayer}{edgelayer}
		\draw (19.center) to (15);
	\end{pgfonlayer}
\end{tikzpicture}
\ \right\rrbracket
=\langle 0 |
$$

The generators $\tof$, $|0\rangle$ and $\langle 0|$ allow the $\cnot$, $\Not$ gates to be defined as well as $\sqrt{2}|+\rangle$ and $\sqrt{2}\langle +|$:




\[ \begin{array}{cccc}
\begin{tikzpicture}
	\begin{pgfonlayer}{nodelayer}
		\node [style=nothing] (28) at (0, 3.5) {};
		\node [style=nothing] (29) at (-0.5, 3.5) {};
		\node [style=nothing] (30) at (-0.5, 4.5) {};
		\node [style=nothing] (31) at (0, 4.5) {};
		\node [style=oplus] (32) at (0, 4) {};
		\node [style=dot] (33) at (-0.5, 4) {};
	\end{pgfonlayer}
	\begin{pgfonlayer}{edgelayer}
		\draw (29) to (33);
		\draw (33) to (30);
		\draw (31) to (32);
		\draw (32) to (28);
		\draw (32) to (33);
	\end{pgfonlayer}
\end{tikzpicture}
:=
\begin{tikzpicture}
	\begin{pgfonlayer}{nodelayer}
		\node [style=nothing] (29) at (0, 3.5) {};
		\node [style=nothing] (30) at (-0.5, 3.5) {};
		\node [style=nothing] (31) at (-0.5, 4.5) {};
		\node [style=nothing] (32) at (0, 4.5) {};
		\node [style=oplus] (33) at (0, 4) {};
		\node [style=dot] (34) at (-0.5, 4) {};
		\node [style=onein] (35) at (-1, 3.5) {};
		\node [style=oneout] (36) at (-1, 4.5) {};
		\node [style=dot] (37) at (-1, 4) {};
	\end{pgfonlayer}
	\begin{pgfonlayer}{edgelayer}
		\draw (30) to (34);
		\draw (34) to (31);
		\draw (32) to (33);
		\draw (33) to (29);
		\draw (33) to (34);
		\draw (35) to (37);
		\draw (37) to (36);
		\draw (37) to (34);
	\end{pgfonlayer}
\end{tikzpicture}\
,
&
\begin{tikzpicture}
	\begin{pgfonlayer}{nodelayer}
		\node [style=nothing] (0) at (0, 0.5) {};
		\node [style=nothing] (1) at (0, 1.5) {};
		\node [style=oplus] (2) at (0, 1) {};
	\end{pgfonlayer}
	\begin{pgfonlayer}{edgelayer}
		\draw (1) to (2);
		\draw (2) to (0);
	\end{pgfonlayer}
\end{tikzpicture}
:=
\begin{tikzpicture}
	\begin{pgfonlayer}{nodelayer}
		\node [style=nothing] (1) at (0, 0) {};
		\node [style=nothing] (2) at (0, 1) {};
		\node [style=oplus] (3) at (0, 0.5) {};
		\node [style=dot] (4) at (-0.5, 0.5) {};
		\node [style=onein] (5) at (-0.5, 0) {};
		\node [style=oneout] (6) at (-0.5, 1) {};
	\end{pgfonlayer}
	\begin{pgfonlayer}{edgelayer}
		\draw (2) to (3);
		\draw (3) to (1);
		\draw (3) to (4);
		\draw (5) to (4);
		\draw (4) to (6);
	\end{pgfonlayer}
\end{tikzpicture}\
,
&
\begin{tikzpicture}
	\begin{pgfonlayer}{nodelayer}
		\node [style=zeroin] (2) at (0, 0) {};
		\node [style=nothing] (3) at (0, 1) {};
	\end{pgfonlayer}
	\begin{pgfonlayer}{edgelayer}
		\draw (2) to (3);
	\end{pgfonlayer}
\end{tikzpicture}
:=
\begin{tikzpicture}
	\begin{pgfonlayer}{nodelayer}
		\node [style=nothing] (3) at (0, 1) {};
		\node [style=onein] (4) at (0, 0) {};
		\node [style=oplus] (5) at (0, 0.5) {};
	\end{pgfonlayer}
	\begin{pgfonlayer}{edgelayer}
		\draw (3) to (5);
		\draw (5) to (4);
	\end{pgfonlayer}
\end{tikzpicture}\ ,
&
\begin{tikzpicture}
	\begin{pgfonlayer}{nodelayer}
		\node [style=nothing] (4) at (0, 0) {};
		\node [style=zeroout] (5) at (0, 1) {};
	\end{pgfonlayer}
	\begin{pgfonlayer}{edgelayer}
		\draw (4) to (5);
	\end{pgfonlayer}
\end{tikzpicture}
:=
\begin{tikzpicture}
	\begin{pgfonlayer}{nodelayer}
		\node [style=nothing] (5) at (0, 0) {};
		\node [style=oneout] (6) at (0, 1) {};
		\node [style=oplus] (7) at (0, 0.5) {};
	\end{pgfonlayer}
	\begin{pgfonlayer}{edgelayer}
		\draw (5) to (7);
		\draw (7) to (6);
	\end{pgfonlayer}
\end{tikzpicture}
\end{array}  \]



As well as the flipped $\tof$ gate and flipped $\cnot$ gate can defined in this setting:

\[
\begin{array}{cc}
\begin{tikzpicture}
	\begin{pgfonlayer}{nodelayer}
		\node [style=nothing] (6) at (0, 0) {};
		\node [style=nothing] (7) at (-0.5, 0) {};
		\node [style=nothing] (8) at (-1, 0) {};
		\node [style=nothing] (9) at (-1, 1.5) {};
		\node [style=nothing] (10) at (-0.5, 1.5) {};
		\node [style=nothing] (11) at (0, 1.5) {};
		\node [style=dot] (12) at (0, 0.75) {};
		\node [style=dot] (13) at (-0.5, 0.75) {};
		\node [style=oplus] (14) at (-1, 0.75) {};
	\end{pgfonlayer}
	\begin{pgfonlayer}{edgelayer}
		\draw (8) to (14);
		\draw (14) to (9);
		\draw (10) to (13);
		\draw (13) to (7);
		\draw (6) to (12);
		\draw (12) to (11);
		\draw (14) to (13);
		\draw (12) to (13);
	\end{pgfonlayer}
\end{tikzpicture}
:=
\begin{tikzpicture}
	\begin{pgfonlayer}{nodelayer}
		\node [style=nothing] (7) at (0, 0) {};
		\node [style=nothing] (8) at (-0.5, 0) {};
		\node [style=nothing] (9) at (-1, 0) {};
		\node [style=nothing] (10) at (-1, 2) {};
		\node [style=nothing] (11) at (-0.5, 2) {};
		\node [style=nothing] (12) at (0, 2) {};
		\node [style=dot] (13) at (-1, 1) {};
		\node [style=dot] (14) at (-0.5, 1) {};
		\node [style=oplus] (15) at (0, 1) {};
	\end{pgfonlayer}
	\begin{pgfonlayer}{edgelayer}
		\draw [in=-90, out=90, looseness=1.25] (9) to (15);
		\draw [in=-90, out=90, looseness=1.25] (15) to (10);
		\draw (8) to (14);
		\draw (14) to (11);
		\draw [in=90, out=-90, looseness=1.25] (12) to (13);
		\draw [in=90, out=-90, looseness=1.25] (13) to (7);
		\draw (13) to (14);
		\draw (14) to (15);
	\end{pgfonlayer}
\end{tikzpicture}
\ ,
&
\begin{tikzpicture}
	\begin{pgfonlayer}{nodelayer}
		\node [style=nothing] (8) at (0, 0) {};
		\node [style=nothing] (9) at (-0.5, 0) {};
		\node [style=nothing] (10) at (-0.5, 1) {};
		\node [style=nothing] (11) at (0, 1) {};
		\node [style=dot] (12) at (0, 0.5) {};
		\node [style=oplus] (13) at (-0.5, 0.5) {};
	\end{pgfonlayer}
	\begin{pgfonlayer}{edgelayer}
		\draw (9) to (13);
		\draw (13) to (10);
		\draw (11) to (12);
		\draw (12) to (8);
		\draw (12) to (13);
	\end{pgfonlayer}
\end{tikzpicture}
:=
\begin{tikzpicture}
	\begin{pgfonlayer}{nodelayer}
		\node [style=nothing] (9) at (0, 0) {};
		\node [style=nothing] (10) at (-0.5, 0) {};
		\node [style=nothing] (11) at (-0.5, 1.5) {};
		\node [style=nothing] (12) at (0, 1.5) {};
		\node [style=dot] (13) at (-0.5, 0.75) {};
		\node [style=oplus] (14) at (0, 0.75) {};
	\end{pgfonlayer}
	\begin{pgfonlayer}{edgelayer}
		\draw [in=-90, out=90, looseness=1.25] (10) to (14);
		\draw [in=-90, out=90, looseness=1.25] (9) to (13);
		\draw [in=-90, out=90, looseness=1.25] (13) to (12);
		\draw [in=-90, out=90, looseness=1.25] (14) to (11);
		\draw (13) to (14);
	\end{pgfonlayer}
\end{tikzpicture}
\end{array}
\]
\end{definition}


\begin{lemma}[{\cite[Proposition 6.2]{tof}}]
\label{lem:tof.frob}

$\TOF$ is a discrete inverse category, where the partial inverse sends:

$$
\tof\mapsto \tof \ , \hspace*{.2cm} |0\rangle \mapsto \langle 0|\ , \hspace*{.2cm} \langle 0| \mapsto |0\rangle
$$

The diagonal map on one wire is defined as follows:
$$
\begin{tikzpicture}
	\begin{pgfonlayer}{nodelayer}
		\node [style=fanout] (10) at (0, 1) {};
		\node [style=none] (11) at (-0.5, 1.75) {};
		\node [style=none] (12) at (0.5, 1.75) {};
		\node [style=none] (13) at (0, 0.25) {};
	\end{pgfonlayer}
	\begin{pgfonlayer}{edgelayer}
		\draw (13.center) to (10);
		\draw [in=-90, out=124] (10) to (11.center);
		\draw [in=56, out=-90] (12.center) to (10);
	\end{pgfonlayer}
\end{tikzpicture}
:=
\begin{tikzpicture}
	\begin{pgfonlayer}{nodelayer}
		\node [style=zeroin] (11) at (0, 1) {};
		\node [style=oplus] (12) at (0, 1.5) {};
		\node [style=dot] (13) at (-0.5, 1.5) {};
		\node [style=none] (14) at (-0.5, 0.75) {};
		\node [style=none] (15) at (-0.5, 2) {};
		\node [style=none] (16) at (0, 2) {};
	\end{pgfonlayer}
	\begin{pgfonlayer}{edgelayer}
		\draw (16.center) to (11);
		\draw (12) to (13);
		\draw (15.center) to (14.center);
	\end{pgfonlayer}
\end{tikzpicture}
$$

\end{lemma}


\begin{theorem}[{\cite[Theorem 10.6]{tof}}]
$\TOF$ is isomorphic to $\FPinj_2$ as a discrete inverse category.
\end{theorem}

\begin{definition}
Let $\FPinj_2$, $\FPar_2$ and $\FSpan_2$ denote, repsectively, the full subcategories of $\ParIso(\Par(\FinOrd))$ , $\Par(\FinOrd)$, $\Span^\sim (\FinOrd)$ where the objects are powers of two.
\end{definition}

The interpretation of the generators we gave into $\Mat_\C$ can therefore be restated in terms of the $\ell^2$ functor:

\begin{corollary}
$\TOF$ embeds in $\FHilb$ via the $\ell^2$ functor so that:
$$\TOF \cong \FPinj\ \xrightarrowtail{\ell^2} \FHilb$$


\end{corollary}




\subsection{Adding a unit and counit to  \texorpdfstring{$\TOF$}{TOF}}

\begin{definition}
Define $\hat\TOF$ to be the pushout of the following diagram of props:
$$c(\TOF)^\op \leftarrow \TOF \rightarrow c(\TOF)$$
\end{definition}

By adding a unit and counit, we obtain a full subcategory of spans of sets and finite ordinals:

\begin{lemma}
\label{lemma:unitcounit}

$\hat \TOF \cong \Span_2$


\end{lemma}



\begin{proof}
Recall that $\TOF$ is presented by the subcategory $\FPinj_2$ of $(\Span^\sim (\FinOrd),\times)$ with morphisms of the form $ 2^n \xleftarrowtail{e}\ k\ \xrightarrowtail{e'} 2^m$ for arbitrary natural numbers $n,m,k$ and monics $e$ and $e'$.



Similarly, $\tilde \TOF$ is presented by the subcategory $\FPar_2$ of  $(\Span^\sim (\FinOrd),\times)$ with morphisms of the form $2^\ell \xleftarrow{f}  2^n  \xleftarrowtail{e}\ k \ \xrightarrowtail{e'}  2^m$ for arbitrary natural numbers $\ell, n,m,k$ and monics $e$ and $e'$ and function $f$.
Let $\FSpan_2$ denote the full subcategory of $(\Span^\sim(\FinOrd),\times)$ generated by powers of two.
Consider the pushout $\X$ of the following diagram of props:

$$\FPar_2^\op \xleftarrowtail{}  \ \FPinj_2 \ \xrightarrowtail{} \FPar_2$$ 


Consider the functor $F:\X\to\FSpan_2$ induced by the universal property of the pushout.  We show that this functor is an isomorphism.
This functor is clearly the identity on objects.

For fullness consider some span $2^n \xleftarrow{f} k \xrightarrow{g} 2^m$. We can construct a function $f':2^{\lceil \log_2 k \rceil} \rightarrow 2^n$ and monic $e_f: k \ \xrightarrowtail{} 2^{\lceil \log_2 k \rceil}$ so that $f=e;f'$.  Similarly, we can construct some  $g':2^{\lceil \log_2 k \rceil} \rightarrow 2^n$ and monic $e_g: k\ \xrightarrowtail{} 2^{\lceil \log_2 k \rceil}$ so that $g=e_g;g'$.  Therefore:


{\xymatrixcolsep{.2cm}
\begin{align*}
F&\left(
\xymatrix{
         & 2^{\lceil \log_2 k \rceil} \ar[dl]_{f'} \ar@{=}[dr]\\
2^n &                                                                                 &2^{\lceil \log_2 k \rceil}
};
\xymatrix{
         & k \ar@{>->}[dl]_{e_f} \ar@{>->}[dr]^{e_m}\\
2^{\lceil \log_2 k \rceil} &                                                                                 & 2^{\lceil \log_2 k \rceil}
};
\xymatrix{
                                       & 2^{\lceil \log_2 k \rceil} \ar[dr]^{g'} \ar@{=}[dl]\\
2^{\lceil \log_2 k \rceil} &                                                                                 & 2^m
}
\right)\\
&=
\xymatrix{
         &                                                                               &                                             &  k \ar@{=}[dl] \ar@{=}[dr] \ar@/_2.0pc/[dddlll]_{f} \ar@/^2.0pc/[dddrrr]^{g} \pullbackcorner[d] \\
         &                                                                               &k \ar@{=}[dr] \ar@{>->}[dl]_{e_f} \pullbackcorner[d]&                                                & k \ar@{=}[dl] \ar@{>->}[dr]^{e_g}\pullbackcorner[d]\\
         & 2^{\lceil \log_2 k \rceil}\ar[dl]^{f'}\ar@{=}[dr]   &                                             & k \ar@{>->}[dl]_{e_f} \ar@{>->}[dr]^{e_g} &                                       & 2^{\lceil \log_2 k \rceil}\ar[dr]_{g'}\ar@{=}[dl] \\
2^n  &                                                                                & 2^{\lceil \log_2 k \rceil}     &                                                & 2^{\lceil \log_2 k \rceil} &                   & 2^m
}
\end{align*}}

So $F$ is full.


For faithfulness suppose we have  two isomorphic spans in $F(\X)$, which we can factorize as before:

$$
\xymatrix{
                 &                                       & k \ar@{>->}[dl]_{e_1}\ar@{->}[dddd]_\cong^{\alpha} \ar@{>->}[dr]^{e_2} \\ 
                 & 2^{n_2} \ar[dl]_{f_1}   &                                                                              & 2^{n_3} \ar[dr]^{f_2}\\ 
2^{n_1}   &                                       &                                                                              &                 & 2^{n_4}\\
                 & 2^{n_2'} \ar[ul]^{f_1'} &                                                                              & 2^{n_3'} \ar[ur]_{f_2'}\\ 
                 &                                       & k \ar@{>->}[ul]^{e_1'} \ar@{>->}[ur]_{e_2'} \\ 
}
$$



In $\X$, we have:

\begin{align*}
\xymatrix{
                & 2^{n_2} \ar[dl]_{f_1} \ar@{=}[dr] \\
2^{n_1} &                                                             & 2^{n_2}
};&
\xymatrix{
               & k \ar@{>->}[dl]_{e_1} \ar@{>->}[dr]^{e_2}\\
2^{n_2} &                                               & 2^{n_3}
};
\xymatrix{
                & 2^{n_3} \ar@{=}[dl] \ar[dr]^{f_2} \\
2^{n_3} &                                                             & 2^{n_4}
}\\
&=
\xymatrix{
                 &                                                           & k \ar@{>->}[dl]_{e_1} \ar@{=}[dr]  \ar@/_2.0pc/[ddll]_{\alpha e_1' f_1'} \pullbackcorner[d] \\
                & 2^{n_2} \ar[dl]_{f_1} \ar@{=}[dr]   &                         & k \ar@{>->}[dl]_{e_1 } \ar@{>->}[dr]^{e_2}  \\
2^{n_1} &                                                             & 2^{n_2}          &                                                 & 2^{n_3}
};
\xymatrix{
                & 2^{n_3} \ar@{=}[dl] \ar[dr]^{f_2} \\
2^{n_3} &                                                             & 2^{n_4}
}\\
&=
\xymatrix{
                 &                                                           & k \ar@{>->}[dl]_{\alpha e_1'} \ar@{>->}[ddrr]^{e_2} \\
                & 2^{n_2'} \ar[dl]_{f_1'}                        &                         &  \\
2^{n_1} &                                                             &           &                                                 & 2^{n_3}
};
\xymatrix{
                & 2^{n_3} \ar@{=}[dl] \ar[dr]^{f_2} \\
2^{n_3} &                                                             & 2^{n_4}
}\\
&=
\xymatrix{
                & 2^{n_2'} \ar[dl]_{f_1'} \ar@{=}[dr] \\
2^{n_1} &                                                             & 2^{n_2'}
};
\xymatrix{
               & k \ar@{>->}[dl]_{\alpha e_1'} \ar@{>->}[dr]^{e_2}\\
2^{n_2'} &                                               & 2^{n_3}
};
\xymatrix{
                & 2^{n_3} \ar@{=}[dl] \ar[dr]^{f_2} \\
2^{n_3} &                                                             & 2^{n_4}
}\\
&=
\xymatrix{
                & 2^{n_2'} \ar[dl]_{f_1'} \ar@{=}[dr] \\
2^{n_1} &                                                             & 2^{n_2'}
};
\xymatrix{
                                       & k \ar@{>->}[dl]_{\alpha e_1'} \ar@{>->}[dr]^{\alpha e_2'} \ar[dd]_\cong^\alpha\\
2^{n_2'} &                                                                         & 2^{n_3'} \\
                                       & k  \ar@{>->}[ul]^{e_1'} \ar@{>->}[ur]_{e_2'}
};
\xymatrix{
                & 2^{n_3'} \ar@{=}[dl] \ar[dr]^{f_2} \\
2^{n_3'} &                                                             & 2^{n_4}
}\\
&=
\xymatrix{
                & 2^{n_2'} \ar[dl]_{f_1'} \ar@{=}[dr] \\
2^{n_1} &                                                             & 2^{n_2'}
};
\xymatrix{
               & k \ar@{>->}[dl]_{ e_1'} \ar@{>->}[dr]^{ e_2'}\\
2^{n_2'} &                                               & 2^{n_3'}
};
\xymatrix{
                & 2^{n_3'} \ar@{=}[dl] \ar[dr]^{f_2} \\
2^{n_3'} &                                                             & 2^{n_4}
}
\end{align*}

Therefore $\FSpan_2 \cong \X$.


To show that $\hat \TOF \cong \FSpan_2$, consider the following diagram where each horizontal face is a pushout:



$$
\xymatrixrowsep{10mm}\xymatrixcolsep{10mm}
\xymatrix{
                                       & {(\FPinj_2,\times)} \ar[dl] \ar@/^.5pc/[rr] \ar@{=}[d]  &                                                  & (\FPar_2,\times) \ar[d]^{\cong} \ar[dl] \\
 (\FPar_2,\times)^\op \ar@/_1pc/[rr]  \ar[d]_{\cong}           &                   {(\FPinj_2,\times)}\ar[dl] \ar@/^.5pc/[rr]    \ar[d]^\cong                                                                       & (\FSpan_2,\times)   \skewpullbackcorner[ul]    \ar@{-->}[d]^(.35){\cong}    & \tilde{(\FPinj_2,\times)} \ar[dl]       \ar[d]^\cong       \\
\tilde{(\FPinj_2,\times)}^\op \ar@/_1pc/[rr]            \ar[d]_{\cong}                               &      \TOF \ar[dl] \ar@/^.5pc/[rr]  \ar@{=}[d]       &            \skewpullbackcorner[ul]                         \ar@{-->}[d]^(.35){\cong}             & \tilde \TOF  \ar[d]_{\cong} \ar[dl]\\
\tilde{\TOF}^\op \ar@/_1pc/[rr]   \ar[d]_{\cong}   &                  \TOF \ar[dl] \ar@/^.5pc/[rr]                                                                      &  \skewpullbackcorner[ul]    \ar@{-->}[d]^(.35){\cong}  & c(\TOF)  \ar[dl]\\
c(\TOF)^\op        \ar@/_1pc/[rr]                          &                                                                                             &          \hat\TOF  \skewpullbackcorner[ul]    &                        &            \\
}
$$


All of the rear and left faces commute. Moreover, the vertical maps are isomorphisms, therefore the maps induced by universal property of the pushout are isomorphisms.







\end{proof}



%If $f$ is a partial isomorphism between finite sets, then the white spiders correspond to the classical structure for the chosen computational basis.  For the interpretation into $\FHilb$ via the $\ell^2$ functor, this means that in the  qubit case, the unit and counit correspond to $\sqrt{2}|+\rangle$ and $\sqrt{2}\langle +|$. 


We give a more elegant presentation of this category in terms of interacting monoids and %\linebreak[4]
 comonoids:

\begin{definition}
The  \dag-compact closed prop $\ZXA$ is presented by phase-free $Z$-spiders, $\Z/2\Z$-phased  $X$-spiders and the $\AND$-gate.  Where these generators are interpreted in $\Mat_\C$ as follows:

\begin{align*}
\left\llbracket\ 
\begin{tikzpicture}
	\begin{pgfonlayer}{nodelayer}
		\node [style=none] (0) at (4, -0.5) {};
		\node [style=none] (1) at (3, -0.5) {};
		\node [style=none] (2) at (3.5, -0.75) {$\cdots$};
		\node [style=Z] (4) at (3.5, -1.25) {$ $};
		\node [style=none] (6) at (3.5, -1.75) {$\cdots$};
		\node [style=none] (7) at (3, -2) {};
		\node [style=none] (8) at (3.5, -1.25) {};
		\node [style=none] (9) at (4, -2) {};
		\node [style=none] (10) at (3.5, -2) {$n$};
		\node [style=none] (11) at (3.5, -0.5) {$m$};
	\end{pgfonlayer}
	\begin{pgfonlayer}{edgelayer}
		\draw [in=-90, out=56] (4) to (0.center);
		\draw [in=124, out=-90] (1.center) to (4);
		\draw [in=-124, out=90] (7.center) to (8);
		\draw [in=90, out=-56] (8) to (9.center);
	\end{pgfonlayer}
\end{tikzpicture}
\ \right\rrbracket
&=
\sum_{j=0}^{1}  | j, \ldots, j\rangle \langle j,\ldots, j|\\
\left\llbracket\ 
\begin{tikzpicture}
	\begin{pgfonlayer}{nodelayer}
		\node [style=none] (0) at (4, -0.5) {};
		\node [style=none] (1) at (3, -0.5) {};
		\node [style=none] (2) at (3.5, -0.75) {$\cdots$};
		\node [style=X] (4) at (3.5, -1.25) {$\ \psi\ $};
		\node [style=none] (6) at (3.5, -1.75) {$\cdots$};
		\node [style=none] (7) at (3, -2) {};
		\node [style=none] (8) at (3.5, -1.25) {};
		\node [style=none] (9) at (4, -2) {};
		\node [style=none] (10) at (3.5, -2) {$n$};
		\node [style=none] (11) at (3.5, -0.5) {$m$};
	\end{pgfonlayer}
	\begin{pgfonlayer}{edgelayer}
		\draw [in=-90, out=56] (4) to (0.center);
		\draw [in=124, out=-90] (1.center) to (4);
		\draw [in=-124, out=90] (7.center) to (8);
		\draw [in=90, out=-56] (8) to (9.center);
	\end{pgfonlayer}
\end{tikzpicture}
\ \right\rrbracket
&=
\sqrt{2}
\sum_{j=0}^{1} e^{2\cdot \pi \cdot i \cdot j\cdot \psi /2} \mathcal{F} | j, \ldots, j\rangle \langle j,\ldots, j| \mathcal{F}^\dag\\
&=
\sum_{\sum  x_i = \sum y _j +\phi \mod 2} | y_1 ,\ldots, y_n \rangle \langle  x_1,\ldots, x_n|\\
\left\llbracket\ 
\begin{tikzpicture}
	\begin{pgfonlayer}{nodelayer}
		\node [style=none] (167) at (79.5, 3.5) {};
		\node [style=andin] (168) at (79.5, 3.5) {};
		\node [style=none] (169) at (80, 2.75) {};
		\node [style=none] (170) at (79, 2.75) {};
		\node [style=none] (171) at (79.5, 3) {$\cdots$};
		\node [style=none] (172) at (79.5, 4.5) {};
		\node [style=none] (173) at (79.5, 2.75) {$n$};
	\end{pgfonlayer}
	\begin{pgfonlayer}{edgelayer}
		\draw [in=90, out=-30] (167.center) to (169.center);
		\draw [in=-150, out=90] (170.center) to (167.center);
		\draw (167.center) to (172.center);
	\end{pgfonlayer}
\end{tikzpicture}
\ \right\rrbracket
&=
|a_0\cdot\ldots\cdot a_{n-1}\rangle\langle a_0,\cdots, a_{n-1}|
\end{align*}

Subject to the identities in Figure \ref{fig:ZXA}:


\begin{figure}[H]
	\noindent
	\scalebox{1.0}{%
		\vbox{%
			\begin{mdframed}
				\begin{multicols}{2}
					\begin{enumerate}[label={\bf [ZX{\it \&}.\arabic*]}, ref={\bf [ZX{\it \&}.\arabic*]}, wide = 0pt, leftmargin = 2em]
						\item
						\label{ZXA.1}
						{\hfil
							$
\begin{tikzpicture}
	\begin{pgfonlayer}{nodelayer}
		\node [style=none] (0) at (0, 0.5) {};
		\node [style=none] (1) at (1, 0.5) {};
		\node [style=none] (2) at (0, 2.75) {};
		\node [style=none] (3) at (1, 2.75) {};
		\node [style=X] (4) at (0.5, 1.25) {$\alpha$};
		\node [style=X] (5) at (0.5, 2) {$\beta$};
		\node [style=none] (6) at (0.5, 2.5) {$\cdots$};
		\node [style=none] (7) at (0.5, 0.75) {$\cdots$};
	\end{pgfonlayer}
	\begin{pgfonlayer}{edgelayer}
		\draw [style=simple, in=-56, out=90] (1.center) to (4);
		\draw [style=simple, in=90, out=-124] (4) to (0.center);
		\draw [style=simple, in=-90, out=124] (5) to (2.center);
		\draw [style=simple, in=-90, out=56] (5) to (3.center);
		\draw [style=simple] (5) to (4);
	\end{pgfonlayer}
\end{tikzpicture}
=
\begin{tikzpicture}
	\begin{pgfonlayer}{nodelayer}
		\node [style=none] (0) at (0, 0.5) {};
		\node [style=none] (1) at (1, 0.5) {};
		\node [style=none] (2) at (0.5, 0.5) {$\cdots$};
		\node [style=none] (3) at (0.5, 2) {$\cdots$};
		\node [style=none] (4) at (1, 2) {};
		\node [style=X] (5) at (0.5, 1.25) {$\alpha+\beta$};
		\node [style=none] (6) at (0, 2) {};
	\end{pgfonlayer}
	\begin{pgfonlayer}{edgelayer}
		\draw [style=simple, in=56, out=-90] (4.center) to (5);
		\draw [style=simple, in=-90, out=124] (5) to (6.center);
		\draw [style=simple, in=90, out=-124] (5) to (0.center);
		\draw [style=simple, in=-56, out=90] (1.center) to (5);
	\end{pgfonlayer}
\end{tikzpicture}
							$
						}

						\item
						\label{ZXA.2}
						{\hfil
							$
\begin{tikzpicture}
	\begin{pgfonlayer}{nodelayer}
		\node [style=none] (0) at (0, 2.25) {};
		\node [style=none] (1) at (1, 2.25) {};
		\node [style=X] (2) at (0.5, 1.5) {$\alpha$};
		\node [style=none] (3) at (0.5, 2) {$\cdots$};
		\node [style=none] (4) at (0.25, 0.5) {};
		\node [style=none] (5) at (0.75, 0.5) {};
		\node [style=none] (6) at (0.75, 1) {};
		\node [style=none] (7) at (0.25, 1) {};
	\end{pgfonlayer}
	\begin{pgfonlayer}{edgelayer}
		\draw [style=simple, in=56, out=-90] (1.center) to (2);
		\draw [style=simple, in=-90, out=124] (2) to (0.center);
		\draw [style=simple, in=90, out=-63] (2) to (6.center);
		\draw [style=simple, in=90, out=-90] (6.center) to (4.center);
		\draw [style=simple, in=-90, out=90] (5.center) to (7.center);
		\draw [style=simple, in=-117, out=90] (7.center) to (2);
	\end{pgfonlayer}
\end{tikzpicture}
=
\begin{tikzpicture}
	\begin{pgfonlayer}{nodelayer}
		\node [style=none] (0) at (0, 2) {};
		\node [style=none] (1) at (1, 2) {};
		\node [style=X] (2) at (0.5, 1.25) {$\alpha$};
		\node [style=none] (3) at (0.5, 1.75) {$\cdots$};
		\node [style=none] (4) at (0.25, 0.5) {};
		\node [style=none] (5) at (0.75, 0.5) {};
	\end{pgfonlayer}
	\begin{pgfonlayer}{edgelayer}
		\draw [style=simple, in=56, out=-90] (1.center) to (2);
		\draw [style=simple, in=-90, out=124] (2) to (0.center);
		\draw [style=simple, in=-56, out=90] (5.center) to (2);
		\draw [style=simple, in=-124, out=90] (4.center) to (2);
	\end{pgfonlayer}
\end{tikzpicture}
							$
						}

						\item
						\label{ZXA.3}
						{\hfil
							$
\begin{tikzpicture}
	\begin{pgfonlayer}{nodelayer}
		\node [style=none] (0) at (0, 0.5) {};
		\node [style=none] (1) at (1, 0.5) {};
		\node [style=none] (2) at (0, 2.75) {};
		\node [style=none] (3) at (1, 2.75) {};
		\node [style=Z] (4) at (0.5, 1.25) {};
		\node [style=Z] (5) at (0.5, 2) {};
		\node [style=none] (6) at (0.5, 2.5) {$\cdots$};
		\node [style=none] (7) at (0.5, 0.75) {$\cdots$};
		\node [style=none] (8) at (0.45, 1.625) {\scalebox{.8}{$\cdots$}};
	\end{pgfonlayer}
	\begin{pgfonlayer}{edgelayer}
		\draw [style=simple, in=-56, out=90] (1.center) to (4);
		\draw [style=simple, in=90, out=-124] (4) to (0.center);
		\draw [style=simple, in=-135, out=135, looseness=1.25] (4) to (5);
		\draw [style=simple, in=45, out=-45, looseness=1.25] (5) to (4);
		\draw [style=simple, in=-90, out=124] (5) to (2.center);
		\draw [style=simple, in=-90, out=56] (5) to (3.center);
	\end{pgfonlayer}
\end{tikzpicture}
=
\begin{tikzpicture}
	\begin{pgfonlayer}{nodelayer}
		\node [style=none] (0) at (0, 0.5) {};
		\node [style=none] (1) at (1, 0.5) {};
		\node [style=Z] (2) at (0.5, 1.25) {};
		\node [style=none] (3) at (0.5, 0.75) {$\cdots$};
		\node [style=none] (4) at (0.5, 1.75) {$\cdots$};
		\node [style=none] (5) at (1, 2) {};
		\node [style=Z] (6) at (0.5, 1.25) {};
		\node [style=none] (7) at (0, 2) {};
	\end{pgfonlayer}
	\begin{pgfonlayer}{edgelayer}
		\draw [style=simple, in=-56, out=90] (1.center) to (2);
		\draw [style=simple, in=90, out=-124] (2) to (0.center);
		\draw [style=simple, in=56, out=-90] (5.center) to (6);
		\draw [style=simple, in=-90, out=124] (6) to (7.center);
	\end{pgfonlayer}
\end{tikzpicture}
							$
						}



						\item
						\label{ZXA.4}
						{\hfil
							$
\begin{tikzpicture}
	\begin{pgfonlayer}{nodelayer}
		\node [style=none] (0) at (0, 0.5) {};
		\node [style=none] (1) at (1, 0.5) {};
		\node [style=Z] (2) at (0.5, 1.25) {};
		\node [style=none] (3) at (0.5, 0.75) {$\cdots$};
		\node [style=none] (4) at (0.25, 2.25) {};
		\node [style=none] (5) at (0.75, 2.25) {};
		\node [style=none] (6) at (0.75, 1.75) {};
		\node [style=none] (7) at (0.25, 1.75) {};
	\end{pgfonlayer}
	\begin{pgfonlayer}{edgelayer}
		\draw [style=simple, in=-56, out=90] (1.center) to (2);
		\draw [style=simple, in=90, out=-124] (2) to (0.center);
		\draw [style=simple, in=-90, out=63] (2) to (6.center);
		\draw [style=simple, in=-90, out=90] (6.center) to (4.center);
		\draw [style=simple, in=90, out=-90] (5.center) to (7.center);
		\draw [style=simple, in=117, out=-90] (7.center) to (2);
	\end{pgfonlayer}
\end{tikzpicture}
=
\begin{tikzpicture}
	\begin{pgfonlayer}{nodelayer}
		\node [style=none] (0) at (0, 0.5) {};
		\node [style=none] (1) at (1, 0.5) {};
		\node [style=Z] (2) at (0.5, 1.25) {};
		\node [style=none] (3) at (0.5, 0.75) {$\cdots$};
		\node [style=none] (4) at (0.25, 2) {};
		\node [style=none] (5) at (0.75, 2) {};
	\end{pgfonlayer}
	\begin{pgfonlayer}{edgelayer}
		\draw [style=simple, in=-56, out=90] (1.center) to (2);
		\draw [style=simple, in=90, out=-124] (2) to (0.center);
		\draw [style=simple, in=56, out=-90] (5.center) to (2);
		\draw [style=simple, in=124, out=-90] (4.center) to (2);
	\end{pgfonlayer}
\end{tikzpicture}
							$
						}
						
						
						\item
						\label{ZXA.5}
						{\hfil
							$
\begin{tikzpicture}
	\begin{pgfonlayer}{nodelayer}
		\node [style=X] (0) at (-1, 1) {};
		\node [style=none] (1) at (-1.25, 0.5) {};
		\node [style=none] (2) at (-0.75, 0.5) {};
		\node [style=Z] (3) at (-1, 1.75) {};
		\node [style=none] (4) at (-1.25, 2.25) {};
		\node [style=none] (5) at (-0.75, 2.25) {};
	\end{pgfonlayer}
	\begin{pgfonlayer}{edgelayer}
		\draw [in=63, out=-90] (5.center) to (3);
		\draw (3) to (0);
		\draw [in=90, out=-117] (0) to (1.center);
		\draw [in=-63, out=90] (2.center) to (0);
		\draw [in=-90, out=117] (3) to (4.center);
	\end{pgfonlayer}
\end{tikzpicture}
=
\begin{tikzpicture}
	\begin{pgfonlayer}{nodelayer}
		\node [style=Z] (0) at (-1, 1) {};
		\node [style=Z] (1) at (-0.25, 1) {};
		\node [style=X] (2) at (-0.25, 1.75) {};
		\node [style=X] (3) at (-1, 1.75) {};
		\node [style=none] (4) at (-1, 2.25) {};
		\node [style=none] (5) at (-0.25, 2.25) {};
		\node [style=none] (6) at (-1, 0.5) {};
		\node [style=none] (7) at (-0.25, 0.5) {};
	\end{pgfonlayer}
	\begin{pgfonlayer}{edgelayer}
		\draw (7.center) to (1);
		\draw (1) to (3);
		\draw [in=120, out=-120, looseness=1.25] (3) to (0);
		\draw (0) to (2);
		\draw (2) to (5.center);
		\draw [in=60, out=-60, looseness=1.25] (2) to (1);
		\draw (0) to (6.center);
		\draw (3) to (4.center);
	\end{pgfonlayer}
\end{tikzpicture}
							$
						}
						
											
\item
	\label{ZXA.6}

	{\hfil\hspace*{.5cm}
							$
\begin{tikzpicture}
	\begin{pgfonlayer}{nodelayer}
		\node [style=none] (0) at (-0.25, 2) {};
		\node [style=Z] (1) at (0, 1.25) {};
		\node [style=X] (2) at (0, 0.5) {};
		\node [style=none] (3) at (0.25, 2) {};
	\end{pgfonlayer}
	\begin{pgfonlayer}{edgelayer}
		\draw [style=simple, in=-90, out=124] (1) to (0.center);
		\draw [style=simple, in=60, out=-90] (3.center) to (1);
		\draw [style=simple] (1) to (2);
	\end{pgfonlayer}
\end{tikzpicture}
=
\begin{tikzpicture}
	\begin{pgfonlayer}{nodelayer}
		\node [style=none] (0) at (-0.25, 1) {};
		\node [style=X] (1) at (-0.25, 0.5) {};
		\node [style=none] (2) at (0.25, 1) {};
		\node [style=X] (3) at (0.25, 0.5) {};
	\end{pgfonlayer}
	\begin{pgfonlayer}{edgelayer}
		\draw [style=simple] (3) to (2.center);
		\draw [style=simple] (1) to (0.center);
	\end{pgfonlayer}
\end{tikzpicture}
							$
						}
						
						\item
						\label{ZXA.7}
						{\hfil
							$
\begin{tikzpicture}
	\begin{pgfonlayer}{nodelayer}
		\node [style=X] (0) at (0, 4.5) {};
		\node [style=Z] (1) at (0, 5.25) {};
	\end{pgfonlayer}
	\begin{pgfonlayer}{edgelayer}
		\draw (1) to (0);
	\end{pgfonlayer}
\end{tikzpicture}
=
\begin{tikzpicture}
	\begin{pgfonlayer}{nodelayer}
	\end{pgfonlayer}
\end{tikzpicture}
							$
						}
						
						
							\item
						\label{ZXA.8}
						{\hfil
							$
\begin{tikzpicture}
	\begin{pgfonlayer}{nodelayer}
		\node [style=X] (0) at (-1, 3) {};
		\node [style=Z] (1) at (-1, 2.25) {};
		\node [style=none] (2) at (-1, 3.5) {};
		\node [style=none] (3) at (-1, 1.75) {};
	\end{pgfonlayer}
	\begin{pgfonlayer}{edgelayer}
		\draw (2.center) to (0);
		\draw (1) to (3.center);
	\end{pgfonlayer}
\end{tikzpicture}
=
\begin{tikzpicture}
	\begin{pgfonlayer}{nodelayer}
		\node [style=X] (0) at (-1, 3) {};
		\node [style=Z] (1) at (-1, 2.25) {};
		\node [style=none] (2) at (-1, 3.5) {};
		\node [style=none] (3) at (-1, 1.75) {};
	\end{pgfonlayer}
	\begin{pgfonlayer}{edgelayer}
		\draw (2.center) to (0);
		\draw [in=120, out=-120, looseness=1.25] (0) to (1);
		\draw [in=-60, out=60, looseness=1.25] (1) to (0);
		\draw (1) to (3.center);
	\end{pgfonlayer}
\end{tikzpicture}
							$
						}

\item
						\label{ZXA.9}
						{\hfil
							$
\begin{tikzpicture}
	\begin{pgfonlayer}{nodelayer}
		\node [style=none] (0) at (0, 3) {};
		\node [style=none] (1) at (0, 3.5) {};
		\node [style=none] (2) at (0, 2.25) {};
		\node [style=none] (3) at (-0.25, 1.5) {};
		\node [style=none] (4) at (0.25, 1.5) {};
		\node [style=none] (5) at (0.5, 1.5) {};
		\node [style=none] (6) at (1, 1.5) {};
		\node [style=none] (7) at (-0.5, 1.5) {};
		\node [style=none] (8) at (-1, 1.5) {};
		\node [style=none] (9) at (-0.8, 1.7) {$\cdots$};
		\node [style=none] (10) at (-0.055, 1.7) {$\cdots$};
		\node [style=none] (11) at (0.68, 1.7) {$\cdots$};
		\node [style=andin] (12) at (0, 2.25) {};
		\node [style=andin] (13) at (0, 3) {};
	\end{pgfonlayer}
	\begin{pgfonlayer}{edgelayer}
		\draw [style=simple, in=-90, out=90] (0.center) to (1.center);
		\draw [style=simple, in=-90, out=90] (2.center) to (0.center);
		\draw [style=simple, in=-63, out=90] (4.center) to (2.center);
		\draw [style=simple, in=90, out=-117] (2.center) to (3.center);
		\draw [style=simple, in=-120, out=90] (7.center) to (0.center);
		\draw [style=simple, in=90, out=-135] (0.center) to (8.center);
		\draw [style=simple, in=-60, out=90] (5.center) to (0.center);
		\draw [style=simple, in=-45, out=90] (6.center) to (0.center);
	\end{pgfonlayer}
\end{tikzpicture}
=
\begin{tikzpicture}
	\begin{pgfonlayer}{nodelayer}
		\node [style=andin] (0) at (0, 3) {};
		\node [style=none] (1) at (1, 1.5) {};
		\node [style=none] (2) at (0.25, 1.5) {};
		\node [style=none] (3) at (-0.5, 1.5) {};
		\node [style=none] (4) at (-1, 1.5) {};
		\node [style=none] (5) at (0.5, 1.5) {};
		\node [style=none] (6) at (-0.8, 1.7) {$\cdots$};
		\node [style=none] (7) at (0, 3) {};
		\node [style=none] (8) at (0.68, 1.7) {$\cdots$};
		\node [style=none] (9) at (-0.055, 1.7) {$\cdots$};
		\node [style=none] (10) at (-0.25, 1.5) {};
		\node [style=none] (11) at (0, 3.5) {};
		\node [style=none] (12) at (0, 3) {};
	\end{pgfonlayer}
	\begin{pgfonlayer}{edgelayer}
		\draw [style=simple, in=-45, out=90] (1.center) to (7.center);
		\draw [style=simple, in=90, out=-135] (7.center) to (4.center);
		\draw [style=simple, in=90, out=-105] (12.center) to (10.center);
		\draw [style=simple, in=-120, out=90] (3.center) to (7.center);
		\draw [style=simple, in=-60, out=90] (5.center) to (7.center);
		\draw [style=simple, in=-75, out=90] (2.center) to (12.center);
		\draw [style=simple, in=-90, out=90] (7.center) to (11.center);
		\draw [style=simple, in=-90, out=90] (12.center) to (7.center);
	\end{pgfonlayer}
\end{tikzpicture}
							$
						}



						\item
						\label{ZXA.10}
						{\hfil
							$
\begin{tikzpicture}
	\begin{pgfonlayer}{nodelayer}
		\node [style=none] (0) at (-1, 2) {};
		\node [style=andin] (10) at (-1, 2) {};
		\node [style=none] (1) at (-1, 2.5) {};
		\node [style=none] (2) at (-0.75, 1.25) {};
		\node [style=X] (3) at (-1.25, 1.25) {$1$};
	\end{pgfonlayer}
	\begin{pgfonlayer}{edgelayer}
		\draw (0) to (1.center);
		\draw [in=90, out=-108] (0) to (3);
		\draw [in=-72, out=90] (2.center) to (0);
	\end{pgfonlayer}
\end{tikzpicture}
=
\begin{tikzpicture}
	\begin{pgfonlayer}{nodelayer}
		\node [style=none] (0) at (-1, 2.5) {};
		\node [style=none] (1) at (-1, 1.75) {};
	\end{pgfonlayer}
	\begin{pgfonlayer}{edgelayer}
		\draw (0.center) to (1.center);
	\end{pgfonlayer}
\end{tikzpicture}
							$
						}

						\item
						\label{ZXA.11}
						{\hfil
							$
\begin{tikzpicture}
	\begin{pgfonlayer}{nodelayer}
		\node [style=andin] (10) at (0, 2.5) {};
		\node [style=none] (0) at (0, 2.5) {};
		\node [style=none] (1) at (-0.25, 2) {};
		\node [style=none] (2) at (0.25, 2) {};
		\node [style=none] (3) at (0, 3) {};
		\node [style=none] (4) at (-0.25, 1.5) {};
		\node [style=none] (5) at (0.25, 1.5) {};
	\end{pgfonlayer}
	\begin{pgfonlayer}{edgelayer}
		\draw [in=-63, out=90] (2.center) to (0);
		\draw [in=90, out=-117, looseness=1.25] (0) to (1.center);
		\draw (3.center) to (0);
		\draw [in=-90, out=90, looseness=1.25] (5.center) to (1.center);
		\draw [in=90, out=-90, looseness=1.25] (2.center) to (4.center);
	\end{pgfonlayer}
\end{tikzpicture}
=
\begin{tikzpicture}
	\begin{pgfonlayer}{nodelayer}
		\node [style=andin] (10) at (0, 2.5) {};
		\node [style=none] (0) at (0, 2.5) {};
		\node [style=none] (1) at (-0.25, 2) {};
		\node [style=none] (2) at (0.25, 2) {};
		\node [style=none] (3) at (0, 3) {};
	\end{pgfonlayer}
	\begin{pgfonlayer}{edgelayer}
		\draw [in=-63, out=90] (2.center) to (0);
		\draw [in=90, out=-117, looseness=1.25] (0) to (1.center);
		\draw (3.center) to (0);
	\end{pgfonlayer}
\end{tikzpicture}
							$
						}

						\item
						\label{ZXA.12}
						{\hfil
							$
\begin{tikzpicture}
	\begin{pgfonlayer}{nodelayer}
		\node [style=andin] (10) at (-1, 1) {};
		\node [style=none] (0) at (-1, 1) {};
		\node [style=none] (1) at (-1.25, 0.5) {};
		\node [style=none] (2) at (-0.75, 0.5) {};
		\node [style=Z] (3) at (-1, 1.75) {};
		\node [style=none] (4) at (-1.25, 2.25) {};
		\node [style=none] (5) at (-0.75, 2.25) {};
	\end{pgfonlayer}
	\begin{pgfonlayer}{edgelayer}
		\draw [in=63, out=-90] (5.center) to (3);
		\draw (3) to (0);
		\draw [in=90, out=-117] (0) to (1.center);
		\draw [in=-63, out=90] (2.center) to (0);
		\draw [in=-90, out=117] (3) to (4.center);
	\end{pgfonlayer}
\end{tikzpicture}
=
\begin{tikzpicture}
	\begin{pgfonlayer}{nodelayer}
		\node [style=Z] (0) at (-1, 1) {};
		\node [style=Z] (1) at (-0.25, 1) {};
		\node [style=andin] (2) at (-0.25, 2) {};
		\node [style=andin] (3) at (-1, 2) {};
		\node [style=none] (4) at (-1, 2.5) {};
		\node [style=none] (5) at (-0.25, 2.5) {};
		\node [style=none] (6) at (-1, 0.5) {};
		\node [style=none] (7) at (-0.25, 0.5) {};
	\end{pgfonlayer}
	\begin{pgfonlayer}{edgelayer}
		\draw (7.center) to (1);
		\draw [in=-60, out=127] (1) to (3.center);
		\draw [in=120, out=-120, looseness=1.25] (3.center) to (0);
		\draw [in=-120, out=53] (0) to (2.center);
		\draw (2.center) to (5.center);
		\draw [in=60, out=-60, looseness=1.25] (2.center) to (1);
		\draw (0) to (6.center);
		\draw (3.center) to (4.center);
	\end{pgfonlayer}
\end{tikzpicture}
							$
						}


					\item
					\label{ZXA.13}
						{\hfil
							$
\begin{tikzpicture}
	\begin{pgfonlayer}{nodelayer}
		\node [style=none] (0) at (-0.5, 4.25) {};
		\node [style=none] (1) at (0, 3.5) {};
		\node [style=none] (2) at (-1, 3.5) {};
		\node [style=Z] (3) at (-0.5, 5) {};
		\node [style=andin] (4) at (-0.5, 4.25) {};
	\end{pgfonlayer}
	\begin{pgfonlayer}{edgelayer}
		\draw [in=90, out=-135] (0.center) to (2.center);
		\draw [in=-41, out=90] (1.center) to (0.center);
		\draw (3) to (0.center);
	\end{pgfonlayer}
\end{tikzpicture}
		=
\begin{tikzpicture}
	\begin{pgfonlayer}{nodelayer}
		\node [style=none] (0) at (-0.5, 3.5) {};
		\node [style=none] (1) at (-1, 3.5) {};
		\node [style=Z] (2) at (-1, 4.25) {};
		\node [style=Z] (3) at (-0.5, 4.25) {};
	\end{pgfonlayer}
	\begin{pgfonlayer}{edgelayer}
		\draw (3) to (0.center);
		\draw (2) to (1.center);
	\end{pgfonlayer}
\end{tikzpicture}
$
						}

						\item
						\label{ZXA.14}
						{\hfil
							$
\begin{tikzpicture}
	\begin{pgfonlayer}{nodelayer}
		\node [style=none] (0) at (-0.25, 2) {};
		\node [style=Z] (1) at (0, 1.25) {};
		\node [style=X] (2) at (0, 0.5) {$1$};
		\node [style=none] (3) at (0.25, 2) {};
	\end{pgfonlayer}
	\begin{pgfonlayer}{edgelayer}
		\draw [style=simple, in=-90, out=124] (1) to (0.center);
		\draw [style=simple, in=60, out=-90] (3.center) to (1);
		\draw [style=simple] (1) to (2);
	\end{pgfonlayer}
\end{tikzpicture}
=
\begin{tikzpicture}
	\begin{pgfonlayer}{nodelayer}
		\node [style=none] (0) at (-0.25, 1) {};
		\node [style=X] (1) at (-0.25, 0.5) {$1$};
		\node [style=none] (2) at (0.25, 1) {};
		\node [style=X] (3) at (0.25, 0.5) {$1$};
	\end{pgfonlayer}
	\begin{pgfonlayer}{edgelayer}
		\draw [style=simple] (3) to (2.center);
		\draw [style=simple] (1) to (0.center);
	\end{pgfonlayer}
\end{tikzpicture}
							$
						}
						
						
						

					

						\item
						\label{ZXA.15}
						{\hfil
							$
\begin{tikzpicture}
	\begin{pgfonlayer}{nodelayer}
		\node [style=none] (0) at (-1, 2) {};
		\node [style=none] (1) at (-1, 1) {};
	\end{pgfonlayer}
	\begin{pgfonlayer}{edgelayer}
		\draw (0.center) to (1.center);
	\end{pgfonlayer}
\end{tikzpicture}
=
\begin{tikzpicture}
	\begin{pgfonlayer}{nodelayer}
		\node [style=Z] (0) at (-1, 1.5) {};
		\node [style=andin] (1) at (-1, 2.5) {};
		\node [style=none] (2) at (-1, 3) {};
		\node [style=none] (3) at (-1, 1) {};
	\end{pgfonlayer}
	\begin{pgfonlayer}{edgelayer}
		\draw (2.center) to (1.center);
		\draw [in=120, out=-120, looseness=1.25] (1.center) to (0);
		\draw [in=-60, out=60, looseness=1.25] (0) to (1.center);
		\draw (0) to (3.center);
	\end{pgfonlayer}
\end{tikzpicture}
							$
						}


						\item
						\label{ZXA.16}
						{\hfil
							$
\begin{tikzpicture}
	\begin{pgfonlayer}{nodelayer}
		\node [style=andin] (1) at (-1, 2) {};
		\node [style=X] (2) at (-1, 2.75) {$1$};
		\node [style=none] (3) at (-1.25, 1.25) {};
		\node [style=none] (4) at (-0.75, 1.25) {};
	\end{pgfonlayer}
	\begin{pgfonlayer}{edgelayer}
		\draw [in=90, out=-108] (1.center) to (3.center);
		\draw [in=-72, out=90] (4.center) to (1.center);
		\draw (1.center) to (2);
	\end{pgfonlayer}
\end{tikzpicture}
=
\begin{tikzpicture}
	\begin{pgfonlayer}{nodelayer}
		\node [style=X] (2) at (-1.25, 2) {$1$};
		\node [style=none] (3) at (-1.25, 1.25) {};
		\node [style=none] (4) at (-0.75, 1.25) {};
		\node [style=X] (5) at (-0.75, 2) {$1$};
	\end{pgfonlayer}
	\begin{pgfonlayer}{edgelayer}
		\draw (5) to (4.center);
		\draw (3.center) to (2);
	\end{pgfonlayer}
\end{tikzpicture}
							$
						}

						\item
						\label{ZXA.17}
						{\hfil
							$
\begin{tikzpicture}
	\begin{pgfonlayer}{nodelayer}
		\node [style=X] (3) at (0, 3) {};
		\node [style=andin] (4) at (-0.25, 3.75) {};
		\node [style=none] (5) at (-0.5, 3) {};
		\node [style=none] (6) at (-0.25, 2.5) {};
		\node [style=none] (7) at (0.25, 2.5) {};
		\node [style=none] (8) at (-0.5, 2.5) {};
		\node [style=none] (9) at (-0.25, 4.25) {};
	\end{pgfonlayer}
	\begin{pgfonlayer}{edgelayer}
		\draw [in=-72, out=90] (3) to (4.center);
		\draw (4.center) to (9.center);
		\draw [in=90, out=-108] (4.center) to (5.center);
		\draw (5.center) to (8.center);
		\draw [in=90, out=-117] (3) to (6.center);
		\draw [in=90, out=-63] (3) to (7.center);
	\end{pgfonlayer}
\end{tikzpicture}
=
\begin{tikzpicture}
	\begin{pgfonlayer}{nodelayer}
		\node [style=none] (4) at (0.25, 3) {};
		\node [style=andin] (5) at (-0.35, 3.75) {};
		\node [style=none] (6) at (-0.25, 3) {};
		\node [style=andin] (7) at (0.35, 3.75) {};
		\node [style=none] (8) at (-0.25, 3) {};
		\node [style=Z] (9) at (-0.25, 3) {};
		\node [style=X] (10) at (0, 4.5) {};
		\node [style=none] (11) at (0, 5) {};
		\node [style=none] (12) at (-0.25, 2.5) {};
		\node [style=none] (13) at (0.5, 2.5) {};
		\node [style=none] (14) at (0.25, 2.5) {};
	\end{pgfonlayer}
	\begin{pgfonlayer}{edgelayer}
		\draw [in=-72, out=90] (4.center) to (5.center);
		\draw [in=120, out=-108] (5.center) to (6.center);
		\draw [in=45, out=-108] (7.center) to (8.center);
		\draw (4.center) to (14.center);
		\draw (12.center) to (6.center);
		\draw [in=-117, out=90] (5.center) to (10);
		\draw (10) to (11.center);
		\draw [in=90, out=-63] (10) to (7.center);
		\draw [in=-75, out=90, looseness=1.25] (13.center) to (7.center);
	\end{pgfonlayer}
\end{tikzpicture}
							$
						}

						

						
						
	


						
					\end{enumerate}
				\end{multicols}
				\
			\end{mdframed}
	}}
	\caption{The identities of \texorpdfstring{$\ZXA$}{ZX\&}}
	\label{fig:ZXA}
\end{figure}

\end{definition}


Before, we prove there is a functor from $\ZXA$ to $\hat \TOF$ we recall some basic properties of $\TOF$.
In $\TOF$, one can construct controlled-not gates with arbitrarily many control wires using ladders of Toffoli gates:

\begin{definition}
A {\bf generalized controlled-not gate} on $n$ wires is denoted by $\lbparen x,X \rbparen$, where $X$ indexes a subset of the $n$ wires, and $x$ is an index for precisely one wire such that $x \notin X$.

We define draw a generalized controlled-not gates $\lbparen x,X \rbparen$ on $n$ wires where $x$ is the last wire and $|X|=n-1$ as follows:

$$
\begin{tikzpicture}
	\begin{pgfonlayer}{nodelayer}
		\node [style=nothing] (79) at (18.5, -0.5) {};
		\node [style=nothing] (80) at (18, -0.5) {};
		\node [style=nothing] (81) at (18, 0.5) {};
		\node [style=nothing] (82) at (18.5, 0.5) {};
		\node [style=oplus] (83) at (18.5, 0) {};
		\node [style=dot] (84) at (18, 0) {};
		\node [style=nothing] (85) at (16.25, -0.5) {};
		\node [style=nothing] (86) at (16.25, 0.5) {};
		\node [style=dot] (87) at (16.25, 0) {};
		\node [style=none] (88) at (16.75, 0) {};
		\node [style=none] (89) at (17.5, 0) {};
		\node [style=none] (93) at (17.125, 0.275) {$n-1$};
	\end{pgfonlayer}
	\begin{pgfonlayer}{edgelayer}
		\draw (80) to (84);
		\draw (84) to (81);
		\draw (82) to (83);
		\draw (83) to (79);
		\draw (83) to (84);
		\draw (85) to (87);
		\draw (87) to (86);
		\draw (89.center) to (84);
		\draw (88.center) to (87);
		\draw [style=dotted] (89.center) to (88.center);
	\end{pgfonlayer}
\end{tikzpicture}
$$

These gates are defined by induction on the number of wires.  For the base cases:

\begin{itemize}
\item For $n=1$ it is $\Not$.
\item For $n=2$ it is  $\cnot$.
\item For $n=3$ it is $\tof$.
\end{itemize}


For $n\geq 3$, then generalized controlled-not gate on $n+1$ wires is defined as follows:

$$
\begin{tikzpicture}
	\begin{pgfonlayer}{nodelayer}
		\node [style=oplus] (50) at (24.25, -1) {};
		\node [style=dot] (51) at (23.75, -1) {};
		\node [style=dot] (52) at (23.25, -1) {};
		\node [style=oplus] (53) at (24.25, -2) {};
		\node [style=dot] (54) at (23.75, -2) {};
		\node [style=dot] (55) at (23.25, -2) {};
		\node [style=dot] (56) at (24.25, -1.5) {};
		\node [style=none] (59) at (24.75, -1.5) {};
		\node [style=none] (60) at (25.5, -1.5) {};
		\node [style=none] (61) at (25.125, -1.225) {$n-1$};
		\node [style=oplus] (62) at (26.5, -1.5) {};
		\node [style=dot] (63) at (26, -1.5) {};
		\node [style=zeroout] (64) at (24.25, -0.5) {};
		\node [style=zeroin] (65) at (24.25, -2.5) {};
		\node [style=nothing] (66) at (23.75, -0.25) {};
		\node [style=nothing] (67) at (23.25, -0.25) {};
		\node [style=nothing] (68) at (23.75, -2.75) {};
		\node [style=nothing] (69) at (23.25, -2.75) {};
		\node [style=nothing] (70) at (26, -0.25) {};
		\node [style=nothing] (71) at (26.5, -0.25) {};
		\node [style=nothing] (72) at (26, -2.75) {};
		\node [style=nothing] (73) at (26.5, -2.75) {};
	\end{pgfonlayer}
	\begin{pgfonlayer}{edgelayer}
		\draw (51) to (52);
		\draw (51) to (50);
		\draw (54) to (55);
		\draw (54) to (53);
		\draw [style=dotted] (60.center) to (59.center);
		\draw (56) to (59.center);
		\draw (63) to (60.center);
		\draw (63) to (62);
		\draw (69) to (55);
		\draw (55) to (52);
		\draw (52) to (67);
		\draw (66) to (51);
		\draw (51) to (54);
		\draw (54) to (68);
		\draw (65) to (53);
		\draw (53) to (56);
		\draw (56) to (50);
		\draw (50) to (64);
		\draw (70) to (63);
		\draw (63) to (72);
		\draw (73) to (62);
		\draw (62) to (71);
	\end{pgfonlayer}
\end{tikzpicture}
$$



\end{definition}




We can partially commute generalized controlled-not gates:

\begin{lemma}[{\cite[\S 3 (3)]{iwama}}]
\label{lemma:Iwama}
Let $\lbparen x,X \rbparen$ and $\lbparen y,Y \rbparen$ be generalized controlled-not gates in $\TOF$ where $x\notin Y$.  By the completeness of $\TOF$, we can commute them past each other with a trailing generalized controlled-not gate as a side effect:
$$
\lbparen x,X\rbparen;\lbparen y,\{x\}\sqcup Y\rbparen = \lbparen y,{X\cup Y}\rbparen ; \lbparen y,{Y\sqcup\{x\}}\rbparen ;\lbparen x,X \rbparen
$$
\end{lemma}



The following equations hold in $\TOF$:

\begin{lemma}[{\cite[Lemma 4.1]{cnot}}]
\label{CNOT.2}

$$
\begin{tikzpicture}
	\begin{pgfonlayer}{nodelayer}
		\node [style=nothing] (109) at (22.25, -1.25) {};
		\node [style=nothing] (110) at (22.75, -1.25) {};
		\node [style=nothing] (111) at (22.25, -2.75) {};
		\node [style=nothing] (112) at (22.75, -2.75) {};
		\node [style=oplus] (113) at (22.75, -1.75) {};
		\node [style=dot] (114) at (22.25, -1.75) {};
		\node [style=oplus] (115) at (22.75, -2.25) {};
		\node [style=dot] (116) at (22.25, -2.25) {};
	\end{pgfonlayer}
	\begin{pgfonlayer}{edgelayer}
		\draw (114) to (113);
		\draw (116) to (115);
		\draw (111) to (116);
		\draw (116) to (114);
		\draw (110) to (113);
		\draw (113) to (115);
		\draw (115) to (112);
		\draw (114) to (109);
	\end{pgfonlayer}
\end{tikzpicture}
=
\begin{tikzpicture}
	\begin{pgfonlayer}{nodelayer}
		\node [style=nothing] (117) at (23.75, -1.25) {};
		\node [style=nothing] (118) at (24.25, -1.25) {};
		\node [style=nothing] (119) at (23.75, -2.75) {};
		\node [style=nothing] (120) at (24.25, -2.75) {};
	\end{pgfonlayer}
	\begin{pgfonlayer}{edgelayer}
		\draw (119) to (117);
		\draw (120) to (118);
	\end{pgfonlayer}
\end{tikzpicture}
$$
\end{lemma}

\begin{lemma}[{\cite[Lemma 4.1]{cnot}}]
\label{CNOT.5}

$$
\begin{tikzpicture}
	\begin{pgfonlayer}{nodelayer}
		\node [style=nothing] (109) at (22.25, -1.25) {};
		\node [style=nothing] (110) at (22.75, -1.25) {};
		\node [style=nothing] (111) at (22.25, -2.75) {};
		\node [style=nothing] (112) at (22.75, -2.75) {};
		\node [style=oplus] (113) at (22.75, -1.75) {};
		\node [style=dot] (114) at (22.25, -1.75) {};
		\node [style=nothing] (117) at (23.25, -1.25) {};
		\node [style=nothing] (125) at (23.25, -2.75) {};
		\node [style=oplus] (126) at (22.75, -2.25) {};
		\node [style=dot] (127) at (23.25, -2.25) {};
	\end{pgfonlayer}
	\begin{pgfonlayer}{edgelayer}
		\draw (114) to (113);
		\draw (110) to (113);
		\draw (114) to (109);
		\draw (127) to (126);
		\draw (125) to (127);
		\draw (127) to (117);
		\draw (113) to (126);
		\draw (114) to (111);
		\draw (112) to (126);
	\end{pgfonlayer}
\end{tikzpicture}
=
\begin{tikzpicture}
	\begin{pgfonlayer}{nodelayer}
		\node [style=nothing] (128) at (24.25, -2.75) {};
		\node [style=nothing] (129) at (24.75, -2.75) {};
		\node [style=nothing] (130) at (24.25, -1.25) {};
		\node [style=nothing] (131) at (24.75, -1.25) {};
		\node [style=oplus] (132) at (24.75, -2.25) {};
		\node [style=dot] (133) at (24.25, -2.25) {};
		\node [style=nothing] (134) at (25.25, -2.75) {};
		\node [style=nothing] (135) at (25.25, -1.25) {};
		\node [style=oplus] (136) at (24.75, -1.75) {};
		\node [style=dot] (137) at (25.25, -1.75) {};
	\end{pgfonlayer}
	\begin{pgfonlayer}{edgelayer}
		\draw (133) to (132);
		\draw (129) to (132);
		\draw (133) to (128);
		\draw (137) to (136);
		\draw (135) to (137);
		\draw (137) to (134);
		\draw (132) to (136);
		\draw (133) to (130);
		\draw (131) to (136);
	\end{pgfonlayer}
\end{tikzpicture}
$$
\end{lemma}

The diagonal map is natural on target qubits:
\begin{lemma}[{\cite[Lem. B.0.2 (iii)]{cole}}]
\label{lemma:natoplus}
$$
\begin{tikzpicture}
	\begin{pgfonlayer}{nodelayer}
		\node [style=fanout] (12) at (0, 2.25) {};
		\node [style=oplus] (13) at (0, 1.75) {};
		\node [style=dot] (14) at (-0.75, 1.75) {};
		\node [style=none] (15) at (-0.75, 1.25) {};
		\node [style=none] (16) at (0, 1.25) {};
		\node [style=none] (17) at (0.25, 3) {};
		\node [style=none] (18) at (-0.25, 3) {};
		\node [style=none] (19) at (-0.75, 3) {};
	\end{pgfonlayer}
	\begin{pgfonlayer}{edgelayer}
		\draw (15.center) to (14);
		\draw (14) to (19.center);
		\draw (13) to (14);
		\draw (16.center) to (13);
		\draw (13) to (12);
		\draw [in=-90, out=108] (12) to (18.center);
		\draw [in=72, out=-90] (17.center) to (12);
	\end{pgfonlayer}
\end{tikzpicture}
=
\begin{tikzpicture}
	\begin{pgfonlayer}{nodelayer}
		\node [style=fanout] (13) at (0, 0.75) {};
		\node [style=none] (14) at (-0.75, 0.25) {};
		\node [style=none] (15) at (0, 0.25) {};
		\node [style=none] (16) at (0.25, 1.5) {};
		\node [style=none] (17) at (-0.25, 1.5) {};
		\node [style=none] (18) at (-0.75, 2.5) {};
		\node [style=dot] (19) at (-0.75, 1.5) {};
		\node [style=oplus] (20) at (-0.25, 1.5) {};
		\node [style=none] (21) at (0.25, 2) {};
		\node [style=dot] (22) at (-0.75, 2) {};
		\node [style=oplus] (23) at (0.25, 2) {};
		\node [style=none] (24) at (0.25, 2.5) {};
		\node [style=none] (25) at (-0.25, 2.5) {};
	\end{pgfonlayer}
	\begin{pgfonlayer}{edgelayer}
		\draw [in=-90, out=108] (13) to (17.center);
		\draw [in=72, out=-90] (16.center) to (13);
		\draw (20) to (19);
		\draw (18.center) to (14.center);
		\draw (23) to (22);
		\draw (24.center) to (21.center);
		\draw (21.center) to (16.center);
		\draw (13) to (15.center);
		\draw (17.center) to (25.center);
	\end{pgfonlayer}
\end{tikzpicture}
$$
\end{lemma}


We also establish some basic properties of $\hat \TOF$.
First, the $\cnot$ gate is its own mate on the second wire:
\begin{lemma}
\label{prop:twist}
$$
\begin{tikzpicture}
	\begin{pgfonlayer}{nodelayer}
		\node [style=dot] (17) at (0, 3) {};
		\node [style=oplus] (18) at (1, 3) {};
		\node [style=none] (19) at (0, 2) {};
		\node [style=none] (20) at (0, 4) {};
		\node [style=none] (21) at (0.5, 2.75) {};
		\node [style=none] (22) at (1, 2.75) {};
		\node [style=none] (23) at (1.5, 3.25) {};
		\node [style=none] (24) at (1, 3.25) {};
		\node [style=none] (25) at (0.5, 3.25) {};
		\node [style=none] (26) at (1.5, 2.75) {};
		\node [style=none] (27) at (1, 2) {};
		\node [style=none] (28) at (1, 4) {};
	\end{pgfonlayer}
	\begin{pgfonlayer}{edgelayer}
		\draw (19.center) to (17);
		\draw (17) to (20.center);
		\draw (18) to (17);
		\draw [in=-90, out=-90, looseness=1.50] (22.center) to (21.center);
		\draw [in=90, out=90, looseness=1.50] (23.center) to (24.center);
		\draw (24.center) to (18);
		\draw (22.center) to (18);
		\draw (23.center) to (26.center);
		\draw (21.center) to (25.center);
		\draw [in=-90, out=90] (27.center) to (26.center);
		\draw [in=90, out=-90, looseness=1.25] (28.center) to (25.center);
	\end{pgfonlayer}
\end{tikzpicture}
=
\begin{tikzpicture}
	\begin{pgfonlayer}{nodelayer}
		\node [style=dot] (18) at (0, 3) {};
		\node [style=oplus] (19) at (0.75, 3) {};
		\node [style=none] (20) at (0, 2) {};
		\node [style=none] (21) at (0, 4) {};
		\node [style=none] (22) at (0.75, 4) {};
		\node [style=none] (23) at (0.75, 2) {};
	\end{pgfonlayer}
	\begin{pgfonlayer}{edgelayer}
		\draw (20.center) to (18);
		\draw (18) to (21.center);
		\draw (19) to (18);
		\draw (23.center) to (19);
		\draw (19) to (22.center);
	\end{pgfonlayer}
\end{tikzpicture}
$$
\end{lemma}

\begin{proof}
\begin{align*}
\begin{tikzpicture}
	\begin{pgfonlayer}{nodelayer}
		\node [style=oplus] (0) at (1, 3) {};
		\node [style=none] (1) at (0, 2) {};
		\node [style=none] (2) at (0, 4) {};
		\node [style=none] (3) at (0.5, 2.75) {};
		\node [style=none] (4) at (1, 2.75) {};
		\node [style=none] (5) at (1.5, 3.25) {};
		\node [style=none] (6) at (1, 3.25) {};
		\node [style=none] (7) at (0.5, 3.25) {};
		\node [style=none] (8) at (1.5, 2.75) {};
		\node [style=none] (9) at (1, 2) {};
		\node [style=none] (10) at (1, 4) {};
		\node [style=dot] (11) at (0, 3) {};
	\end{pgfonlayer}
	\begin{pgfonlayer}{edgelayer}
		\draw [in=-90, out=-90, looseness=1.50] (4.center) to (3.center);
		\draw [in=90, out=90, looseness=1.50] (5.center) to (6.center);
		\draw (6.center) to (0);
		\draw (4.center) to (0);
		\draw (5.center) to (8.center);
		\draw (3.center) to (7.center);
		\draw [in=-90, out=90] (9.center) to (8.center);
		\draw [in=90, out=-90, looseness=1.25] (10.center) to (7.center);
		\draw (11) to (0);
		\draw (11) to (2.center);
		\draw (11) to (1.center);
	\end{pgfonlayer}
\end{tikzpicture}
&=
\begin{tikzpicture}
	\begin{pgfonlayer}{nodelayer}
		\node [style=dot] (1) at (0, 3.25) {};
		\node [style=oplus] (2) at (1, 3.25) {};
		\node [style=none] (3) at (0, 2) {};
		\node [style=none] (4) at (0, 4.5) {};
		\node [style=none] (5) at (0.5, 3.5) {};
		\node [style=none] (6) at (1.5, 3) {};
		\node [style=none] (7) at (1.25, 2) {};
		\node [style=none] (8) at (0.75, 4.5) {};
		\node [style=fanout] (9) at (0.75, 2.75) {};
		\node [style=fanin] (10) at (1.25, 3.75) {};
		\node [style=Z] (11) at (0.75, 2) {};
		\node [style=Z] (12) at (1.25, 4.5) {};
	\end{pgfonlayer}
	\begin{pgfonlayer}{edgelayer}
		\draw (3.center) to (1);
		\draw (1) to (4.center);
		\draw (2) to (1);
		\draw [in=-90, out=90] (7.center) to (6.center);
		\draw [in=90, out=-90, looseness=1.25] (8.center) to (5.center);
		\draw [in=-72, out=90] (6.center) to (10);
		\draw [in=90, out=-117, looseness=1.25] (10) to (2);
		\draw [in=63, out=-90, looseness=1.25] (2) to (9);
		\draw [in=-90, out=108] (9) to (5.center);
		\draw (11) to (9);
		\draw (10) to (12);
	\end{pgfonlayer}
\end{tikzpicture}
\eq{Lem. \ref{CNOT.2}}
\begin{tikzpicture}
	\begin{pgfonlayer}{nodelayer}
		\node [style=dot] (2) at (0, 3.25) {};
		\node [style=oplus] (3) at (1, 3.25) {};
		\node [style=none] (4) at (0, 2) {};
		\node [style=none] (5) at (0, 5) {};
		\node [style=none] (6) at (1.5, 3) {};
		\node [style=none] (7) at (1.25, 2) {};
		\node [style=none] (8) at (0.5, 5) {};
		\node [style=fanout] (9) at (0.75, 2.75) {};
		\node [style=fanin] (10) at (1.25, 3.75) {};
		\node [style=Z] (11) at (0.75, 2) {};
		\node [style=Z] (12) at (1.25, 5) {};
		\node [style=dot] (13) at (0, 3.75) {};
		\node [style=oplus] (14) at (0.5, 3.75) {};
		\node [style=oplus] (15) at (0.5, 4.5) {};
		\node [style=dot] (16) at (0, 4.5) {};
	\end{pgfonlayer}
	\begin{pgfonlayer}{edgelayer}
		\draw (4.center) to (2);
		\draw (2) to (5.center);
		\draw (3) to (2);
		\draw [in=-90, out=90] (7.center) to (6.center);
		\draw [in=-72, out=90] (6.center) to (10);
		\draw [in=90, out=-117, looseness=1.25] (10) to (3);
		\draw [in=63, out=-90, looseness=1.25] (3) to (9);
		\draw (11) to (9);
		\draw (10) to (12);
		\draw (14) to (13);
		\draw (15) to (16);
		\draw (8.center) to (15);
		\draw (15) to (14);
		\draw [in=-90, out=104] (9) to (14);
	\end{pgfonlayer}
\end{tikzpicture}
\eq{Lem. \ref{lemma:natoplus}}
\begin{tikzpicture}
	\begin{pgfonlayer}{nodelayer}
		\node [style=none] (3) at (0, 1.5) {};
		\node [style=none] (4) at (0, 4.5) {};
		\node [style=none] (5) at (1.5, 3) {};
		\node [style=none] (6) at (1.25, 1.5) {};
		\node [style=none] (7) at (0.5, 4.5) {};
		\node [style=fanout] (8) at (0.75, 2.75) {};
		\node [style=fanin] (9) at (1.25, 3.75) {};
		\node [style=Z] (10) at (0.75, 1.5) {};
		\node [style=Z] (11) at (1.25, 4.5) {};
		\node [style=oplus] (12) at (0.75, 2.25) {};
		\node [style=dot] (13) at (0, 2.25) {};
		\node [style=dot] (14) at (0, 3.5) {};
		\node [style=oplus] (15) at (0.5, 3.5) {};
	\end{pgfonlayer}
	\begin{pgfonlayer}{edgelayer}
		\draw [in=-90, out=90] (6.center) to (5.center);
		\draw [in=-72, out=90] (5.center) to (9);
		\draw (9) to (11);
		\draw (12) to (13);
		\draw (15) to (14);
		\draw (4.center) to (14);
		\draw (15) to (7.center);
		\draw [in=108, out=-90] (15) to (8);
		\draw (8) to (9);
		\draw (8) to (12);
		\draw (12) to (10);
		\draw (3.center) to (13);
		\draw (13) to (14);
	\end{pgfonlayer}
\end{tikzpicture}
\eq{Lem. \ref{lem:tof.frob}}
\begin{tikzpicture}
	\begin{pgfonlayer}{nodelayer}
		\node [style=none] (4) at (0, 1.25) {};
		\node [style=none] (5) at (0, 4.75) {};
		\node [style=none] (6) at (1, 2) {};
		\node [style=none] (7) at (1, 1.25) {};
		\node [style=none] (8) at (0.5, 4.75) {};
		\node [style=Z] (9) at (0.5, 1.5) {};
		\node [style=Z] (10) at (1, 4.25) {};
		\node [style=oplus] (11) at (0.5, 2) {};
		\node [style=dot] (12) at (0, 2) {};
		\node [style=dot] (13) at (0, 4.25) {};
		\node [style=oplus] (14) at (0.5, 4.25) {};
		\node [style=fanin] (15) at (0.75, 2.75) {};
		\node [style=fanout] (16) at (0.75, 3.5) {};
	\end{pgfonlayer}
	\begin{pgfonlayer}{edgelayer}
		\draw [in=-90, out=90] (7.center) to (6.center);
		\draw (11) to (12);
		\draw (14) to (13);
		\draw (5.center) to (13);
		\draw (14) to (8.center);
		\draw (11) to (9);
		\draw (4.center) to (12);
		\draw (12) to (13);
		\draw [in=63, out=-90, looseness=1.25] (10) to (16);
		\draw [in=-90, out=117, looseness=1.25] (16) to (14);
		\draw (16) to (15);
		\draw [in=90, out=-108] (15) to (11);
		\draw [in=-72, out=90] (6.center) to (15);
	\end{pgfonlayer}
\end{tikzpicture}\\
&\eq{unit}
\begin{tikzpicture}
	\begin{pgfonlayer}{nodelayer}
		\node [style=none] (5) at (0, 1.25) {};
		\node [style=none] (6) at (0, 4.75) {};
		\node [style=none] (7) at (1, 2) {};
		\node [style=none] (8) at (1, 1.25) {};
		\node [style=none] (9) at (0.75, 4.75) {};
		\node [style=Z] (10) at (0.5, 1.25) {};
		\node [style=oplus] (11) at (0.5, 2) {};
		\node [style=dot] (12) at (0, 2) {};
		\node [style=dot] (13) at (0, 4) {};
		\node [style=oplus] (14) at (0.75, 4) {};
		\node [style=fanin] (15) at (0.75, 2.75) {};
	\end{pgfonlayer}
	\begin{pgfonlayer}{edgelayer}
		\draw [in=-90, out=90] (8.center) to (7.center);
		\draw (11) to (12);
		\draw (14) to (13);
		\draw (6.center) to (13);
		\draw (14) to (9.center);
		\draw (11) to (10);
		\draw (5.center) to (12);
		\draw (12) to (13);
		\draw [in=90, out=-108] (15) to (11);
		\draw [in=-72, out=90] (7.center) to (15);
		\draw (14) to (15);
	\end{pgfonlayer}
\end{tikzpicture}
\eq{Lem. \ref{lemma:natoplus}}
\begin{tikzpicture}
	\begin{pgfonlayer}{nodelayer}
		\node [style=none] (6) at (0, 4) {};
		\node [style=none] (7) at (1, 2.5) {};
		\node [style=none] (8) at (1, 1) {};
		\node [style=none] (9) at (0.75, 4) {};
		\node [style=oplus] (10) at (0.5, 2.5) {};
		\node [style=dot] (11) at (0, 2.5) {};
		\node [style=fanin] (12) at (0.75, 3.25) {};
		\node [style=dot] (13) at (0, 2) {};
		\node [style=oplus] (14) at (0.5, 2) {};
		\node [style=none] (15) at (0, 1) {};
		\node [style=Z] (16) at (0.5, 1) {};
		\node [style=dot] (17) at (0, 1.5) {};
		\node [style=oplus] (18) at (1, 1.5) {};
	\end{pgfonlayer}
	\begin{pgfonlayer}{edgelayer}
		\draw [in=-90, out=90] (8.center) to (7.center);
		\draw (10) to (11);
		\draw [in=90, out=-108] (12) to (10);
		\draw [in=-72, out=90] (7.center) to (12);
		\draw (14) to (13);
		\draw (18) to (17);
		\draw (16) to (14);
		\draw (14) to (10);
		\draw (6.center) to (11);
		\draw (11) to (17);
		\draw (17) to (13);
		\draw (13) to (15.center);
		\draw (9.center) to (12);
	\end{pgfonlayer}
\end{tikzpicture}
\eq{\ref{CNOT.2}}
\begin{tikzpicture}
	\begin{pgfonlayer}{nodelayer}
		\node [style=none] (7) at (0, 3.75) {};
		\node [style=none] (8) at (1, 2) {};
		\node [style=none] (9) at (0.75, 3.75) {};
		\node [style=fanin] (10) at (0.75, 3.25) {};
		\node [style=none] (11) at (0, 2) {};
		\node [style=Z] (12) at (0.5, 2) {};
		\node [style=dot] (13) at (0, 2.5) {};
		\node [style=oplus] (14) at (1, 2.5) {};
	\end{pgfonlayer}
	\begin{pgfonlayer}{edgelayer}
		\draw (14) to (13);
		\draw (9.center) to (10);
		\draw [in=90, out=-105] (10) to (12);
		\draw (11.center) to (7.center);
		\draw [in=90, out=-72] (10) to (14);
		\draw (14) to (8.center);
	\end{pgfonlayer}
\end{tikzpicture}
\eq{unit}
\begin{tikzpicture}
	\begin{pgfonlayer}{nodelayer}
		\node [style=none] (8) at (0, 6.25) {};
		\node [style=none] (9) at (0.5, 5.25) {};
		\node [style=none] (10) at (0.5, 6.25) {};
		\node [style=none] (11) at (0, 5.25) {};
		\node [style=dot] (12) at (0, 5.75) {};
		\node [style=oplus] (13) at (0.5, 5.75) {};
	\end{pgfonlayer}
	\begin{pgfonlayer}{edgelayer}
		\draw (13) to (12);
		\draw (11.center) to (8.center);
		\draw (13) to (9.center);
		\draw (10.center) to (13);
	\end{pgfonlayer}
\end{tikzpicture}
\end{align*}
\end{proof}


Therefore, 

\begin{lemma}
\label{lemma:cnotslide}
$$
\begin{tikzpicture}
	\begin{pgfonlayer}{nodelayer}
		\node [style=oplus] (9) at (1, 5.75) {};
		\node [style=none] (10) at (0.5, 6.25) {};
		\node [style=none] (11) at (0.5, 5.25) {};
		\node [style=none] (12) at (1, 6.25) {};
		\node [style=none] (13) at (1.5, 5.5) {};
		\node [style=none] (14) at (1, 5.5) {};
		\node [style=none] (15) at (1.5, 6.25) {};
		\node [style=dot] (16) at (0.5, 5.75) {};
	\end{pgfonlayer}
	\begin{pgfonlayer}{edgelayer}
		\draw [in=-90, out=-90, looseness=1.50] (13.center) to (14.center);
		\draw (14.center) to (9);
		\draw (16) to (9);
		\draw (16) to (11.center);
		\draw (16) to (10.center);
		\draw (15.center) to (13.center);
		\draw (9) to (12.center);
	\end{pgfonlayer}
\end{tikzpicture}
\eq{Prop. \ref{prop:twist}}
\begin{tikzpicture}
	\begin{pgfonlayer}{nodelayer}
		\node [style=dot] (10) at (-0.25, 6.25) {};
		\node [style=oplus] (11) at (0.75, 6.25) {};
		\node [style=none] (12) at (-0.25, 5.25) {};
		\node [style=none] (13) at (-0.25, 7.25) {};
		\node [style=none] (14) at (0.25, 6) {};
		\node [style=none] (15) at (0.75, 6) {};
		\node [style=none] (16) at (1.25, 6.5) {};
		\node [style=none] (17) at (0.75, 6.5) {};
		\node [style=none] (18) at (0.25, 6.5) {};
		\node [style=none] (19) at (0.25, 7.25) {};
		\node [style=none] (20) at (1.25, 5.75) {};
		\node [style=none] (21) at (1.75, 5.75) {};
		\node [style=none] (22) at (1.75, 7.25) {};
	\end{pgfonlayer}
	\begin{pgfonlayer}{edgelayer}
		\draw (12.center) to (10);
		\draw (10) to (13.center);
		\draw (11) to (10);
		\draw [in=-90, out=-90, looseness=1.50] (15.center) to (14.center);
		\draw [in=90, out=90, looseness=1.50] (16.center) to (17.center);
		\draw (17.center) to (11);
		\draw (15.center) to (11);
		\draw (14.center) to (18.center);
		\draw [in=90, out=-90, looseness=1.25] (19.center) to (18.center);
		\draw [in=-90, out=-90, looseness=1.50] (21.center) to (20.center);
		\draw (22.center) to (21.center);
		\draw (20.center) to (16.center);
	\end{pgfonlayer}
\end{tikzpicture}
\eq{yanking}
\begin{tikzpicture}
	\begin{pgfonlayer}{nodelayer}
		\node [style=dot] (11) at (-0.25, 5.75) {};
		\node [style=oplus] (12) at (0.75, 5.75) {};
		\node [style=none] (13) at (-0.25, 5.25) {};
		\node [style=none] (14) at (-0.25, 6.25) {};
		\node [style=none] (15) at (0.25, 5.5) {};
		\node [style=none] (16) at (0.75, 5.5) {};
		\node [style=none] (17) at (0.75, 6.25) {};
		\node [style=none] (18) at (0.25, 6.25) {};
	\end{pgfonlayer}
	\begin{pgfonlayer}{edgelayer}
		\draw (13.center) to (11);
		\draw (11) to (14.center);
		\draw (12) to (11);
		\draw [in=-90, out=-90, looseness=1.50] (16.center) to (15.center);
		\draw (17.center) to (12);
		\draw (16.center) to (12);
		\draw (15.center) to (18.center);
	\end{pgfonlayer}
\end{tikzpicture}
$$
\end{lemma}


Thus

\begin{lemma}
\label{lemma:whiteunit}

$$
\begin{tikzpicture}
	\begin{pgfonlayer}{nodelayer}
		\node [style=dot] (12) at (-0.25, 5.75) {};
		\node [style=oplus] (13) at (0.25, 5.75) {};
		\node [style=none] (14) at (-0.25, 5.25) {};
		\node [style=none] (15) at (-0.25, 6.25) {};
		\node [style=none] (16) at (0.25, 6.25) {};
		\node [style=Z] (17) at (0.25, 5.25) {};
	\end{pgfonlayer}
	\begin{pgfonlayer}{edgelayer}
		\draw (14.center) to (12);
		\draw (12) to (15.center);
		\draw (13) to (12);
		\draw (16.center) to (13);
		\draw (13) to (17);
	\end{pgfonlayer}
\end{tikzpicture}
=
\begin{tikzpicture}
	\begin{pgfonlayer}{nodelayer}
		\node [style=none] (13) at (-0.25, 5.25) {};
		\node [style=none] (14) at (-0.25, 6) {};
		\node [style=none] (15) at (0.25, 6) {};
		\node [style=Z] (16) at (0.25, 5.25) {};
	\end{pgfonlayer}
	\begin{pgfonlayer}{edgelayer}
		\draw (15.center) to (16);
		\draw (13.center) to (14.center);
	\end{pgfonlayer}
\end{tikzpicture}
$$
\end{lemma}

\begin{proof}
\begin{align*}
\begin{tikzpicture}
	\begin{pgfonlayer}{nodelayer}
		\node [style=dot] (14) at (-0.25, 5.75) {};
		\node [style=oplus] (15) at (0.25, 5.75) {};
		\node [style=none] (16) at (-0.25, 5.25) {};
		\node [style=none] (17) at (-0.25, 6.25) {};
		\node [style=none] (18) at (0.25, 6.25) {};
		\node [style=Z] (19) at (0.25, 5.25) {};
	\end{pgfonlayer}
	\begin{pgfonlayer}{edgelayer}
		\draw (16.center) to (14);
		\draw (14) to (17.center);
		\draw (15) to (14);
		\draw (18.center) to (15);
		\draw (15) to (19);
	\end{pgfonlayer}
\end{tikzpicture}
&\eq{unit}
\begin{tikzpicture}
	\begin{pgfonlayer}{nodelayer}
		\node [style=dot] (15) at (-0.25, 6.5) {};
		\node [style=oplus] (16) at (0.25, 6.5) {};
		\node [style=none] (17) at (-0.25, 5.25) {};
		\node [style=none] (18) at (-0.25, 7) {};
		\node [style=none] (19) at (0.25, 7) {};
		\node [style=Z] (20) at (0.5, 5.25) {};
		\node [style=fanout] (21) at (0.5, 5.75) {};
		\node [style=Z] (22) at (0.75, 6.5) {};
	\end{pgfonlayer}
	\begin{pgfonlayer}{edgelayer}
		\draw (17.center) to (15);
		\draw (15) to (18.center);
		\draw (16) to (15);
		\draw (19.center) to (16);
		\draw [in=117, out=-90] (16) to (21);
		\draw (21) to (20);
		\draw [in=63, out=-90] (22) to (21);
	\end{pgfonlayer}
\end{tikzpicture}
=
\begin{tikzpicture}
	\begin{pgfonlayer}{nodelayer}
		\node [style=dot] (16) at (-0.25, 5.75) {};
		\node [style=oplus] (17) at (0.25, 5.75) {};
		\node [style=none] (18) at (-0.25, 5.25) {};
		\node [style=none] (19) at (-0.25, 6.25) {};
		\node [style=none] (20) at (0.25, 6.25) {};
		\node [style=Z] (21) at (0.75, 5.75) {};
	\end{pgfonlayer}
	\begin{pgfonlayer}{edgelayer}
		\draw (18.center) to (16);
		\draw (16) to (19.center);
		\draw (17) to (16);
		\draw (20.center) to (17);
		\draw [in=-90, out=-90, looseness=2.25] (17) to (21);
	\end{pgfonlayer}
\end{tikzpicture}\
\eq{Lem. \ref{lemma:cnotslide}}
\begin{tikzpicture}
	\begin{pgfonlayer}{nodelayer}
		\node [style=dot] (17) at (0, 5.75) {};
		\node [style=oplus] (18) at (0.75, 5.75) {};
		\node [style=none] (19) at (0, 5.25) {};
		\node [style=none] (20) at (0, 6.25) {};
		\node [style=none] (21) at (0.25, 6.25) {};
		\node [style=Z] (22) at (0.75, 6.25) {};
		\node [style=none] (23) at (0.25, 5.5) {};
		\node [style=none] (24) at (0.75, 5.5) {};
	\end{pgfonlayer}
	\begin{pgfonlayer}{edgelayer}
		\draw (19.center) to (17);
		\draw (17) to (20.center);
		\draw (18) to (17);
		\draw (22) to (18);
		\draw (18) to (24.center);
		\draw (23.center) to (21.center);
		\draw [in=-90, out=-90, looseness=1.25] (23.center) to (24.center);
	\end{pgfonlayer}
\end{tikzpicture}\
=
\begin{tikzpicture}
	\begin{pgfonlayer}{nodelayer}
		\node [style=dot] (18) at (0, 6.25) {};
		\node [style=oplus] (19) at (0.75, 6.25) {};
		\node [style=none] (20) at (0, 5.25) {};
		\node [style=none] (21) at (0, 6.75) {};
		\node [style=none] (22) at (0.25, 6.25) {};
		\node [style=Z] (23) at (0.75, 6.75) {};
		\node [style=fanout] (24) at (0.5, 5.75) {};
		\node [style=Z] (25) at (0.5, 5.25) {};
		\node [style=none] (26) at (0.25, 6.75) {};
	\end{pgfonlayer}
	\begin{pgfonlayer}{edgelayer}
		\draw (20.center) to (18);
		\draw (18) to (21.center);
		\draw (19) to (18);
		\draw (23) to (19);
		\draw [in=63, out=-90] (19) to (24);
		\draw [in=-90, out=117] (24) to (22.center);
		\draw (24) to (25);
		\draw (26.center) to (22.center);
	\end{pgfonlayer}
\end{tikzpicture}
\eq{Lem. \ref{CNOT.2}}
\begin{tikzpicture}
	\begin{pgfonlayer}{nodelayer}
		\node [style=dot] (19) at (-0.25, 6.25) {};
		\node [style=oplus] (20) at (0.75, 6.25) {};
		\node [style=none] (21) at (-0.25, 5.25) {};
		\node [style=none] (22) at (0.25, 6.25) {};
		\node [style=Z] (23) at (0.75, 6.75) {};
		\node [style=fanout] (24) at (0.5, 5.75) {};
		\node [style=Z] (25) at (0.5, 5.25) {};
		\node [style=oplus] (26) at (0.25, 7.25) {};
		\node [style=dot] (27) at (-0.25, 7.25) {};
		\node [style=none] (28) at (0.25, 7.75) {};
		\node [style=none] (29) at (-0.25, 7.75) {};
		\node [style=dot] (30) at (-0.25, 6.75) {};
		\node [style=oplus] (31) at (0.25, 6.75) {};
	\end{pgfonlayer}
	\begin{pgfonlayer}{edgelayer}
		\draw (21.center) to (19);
		\draw (20) to (19);
		\draw (23) to (20);
		\draw [in=63, out=-90] (20) to (24);
		\draw [in=-90, out=117] (24) to (22.center);
		\draw (24) to (25);
		\draw (26) to (27);
		\draw (31) to (30);
		\draw (28.center) to (26);
		\draw (26) to (31);
		\draw (31) to (22.center);
		\draw (19) to (30);
		\draw (30) to (27);
		\draw (27) to (29.center);
	\end{pgfonlayer}
\end{tikzpicture}\
\eq{Lem. \ref{lemma:natoplus}}
\begin{tikzpicture}
	\begin{pgfonlayer}{nodelayer}
		\node [style=Z] (20) at (0.75, 7.25) {};
		\node [style=fanout] (21) at (0.5, 6.25) {};
		\node [style=none] (22) at (0.25, 8.25) {};
		\node [style=none] (23) at (-0.25, 8.25) {};
		\node [style=dot] (24) at (-0.25, 6.75) {};
		\node [style=oplus] (25) at (0.25, 6.75) {};
		\node [style=Z] (26) at (0.5, 5.25) {};
		\node [style=dot] (27) at (-0.25, 5.75) {};
		\node [style=oplus] (28) at (0.5, 5.75) {};
		\node [style=none] (29) at (-0.25, 5.25) {};
		\node [style=none] (30) at (0.75, 6.75) {};
	\end{pgfonlayer}
	\begin{pgfonlayer}{edgelayer}
		\draw (25) to (24);
		\draw (28) to (27);
		\draw (27) to (29.center);
		\draw (26) to (28);
		\draw (28) to (21);
		\draw [in=-90, out=63] (21) to (30.center);
		\draw (30.center) to (20);
		\draw (25) to (22.center);
		\draw (23.center) to (24);
		\draw [in=117, out=-90] (25) to (21);
		\draw (27) to (24);
	\end{pgfonlayer}
\end{tikzpicture}\\
&\eq{unit}
\begin{tikzpicture}
	\begin{pgfonlayer}{nodelayer}
		\node [style=none] (21) at (0.25, 6.75) {};
		\node [style=none] (22) at (-0.25, 6.75) {};
		\node [style=dot] (23) at (-0.25, 6.25) {};
		\node [style=oplus] (24) at (0.25, 6.25) {};
		\node [style=Z] (25) at (0.25, 5.25) {};
		\node [style=dot] (26) at (-0.25, 5.75) {};
		\node [style=oplus] (27) at (0.25, 5.75) {};
		\node [style=none] (28) at (-0.25, 5.25) {};
	\end{pgfonlayer}
	\begin{pgfonlayer}{edgelayer}
		\draw (24) to (23);
		\draw (27) to (26);
		\draw (26) to (28.center);
		\draw (25) to (27);
		\draw (24) to (21.center);
		\draw (22.center) to (23);
		\draw (26) to (23);
		\draw (24) to (27);
	\end{pgfonlayer}
\end{tikzpicture}
\eq{Lem. \ref{CNOT.2}}
\begin{tikzpicture}
	\begin{pgfonlayer}{nodelayer}
		\node [style=none] (22) at (0.25, 6) {};
		\node [style=none] (23) at (-0.25, 6) {};
		\node [style=Z] (24) at (0.25, 5.25) {};
		\node [style=none] (25) at (-0.25, 5.25) {};
	\end{pgfonlayer}
	\begin{pgfonlayer}{edgelayer}
		\draw (22.center) to (24);
		\draw (25.center) to (23.center);
	\end{pgfonlayer}
\end{tikzpicture}
\end{align*}
\end{proof}



\begin{proposition}
\label{prop:TOFZXA}
Consider the interpretation $\llbracket\_\rrbracket_{\ZXA}:\ZXA\to\hat \TOF$ taking:

$$
\begin{tikzpicture}
	\begin{pgfonlayer}{nodelayer}
		\node [style=none] (7) at (0.25, 2) {};
		\node [style=none] (8) at (0.75, 2) {};
		\node [style=X] (9) at (0.5, 1.25) {};
		\node [style=none] (10) at (0.5, 0.5) {};
	\end{pgfonlayer}
	\begin{pgfonlayer}{edgelayer}
		\draw [style=simple, in=-90, out=124] (9) to (7.center);
		\draw [style=simple, in=-90, out=56] (9) to (8.center);
		\draw [style=simple] (9) to (10.center);
	\end{pgfonlayer}
\end{tikzpicture}
\mapsto
\begin{tikzpicture}
	\begin{pgfonlayer}{nodelayer}
		\node [style=none] (8) at (0, 2.75) {};
		\node [style=none] (9) at (0, 1.25) {};
		\node [style=dot] (10) at (0.5, 2) {};
		\node [style=oplus] (11) at (0, 2) {};
		\node [style=Z] (12) at (0.5, 1.25) {};
		\node [style=none] (13) at (0.5, 2.75) {};
	\end{pgfonlayer}
	\begin{pgfonlayer}{edgelayer}
		\draw [style=simple] (13.center) to (10);
		\draw [style=simple] (10) to (12);
		\draw [style=simple] (10) to (11);
		\draw [style=simple] (11) to (9.center);
		\draw [style=simple] (11) to (8.center);
	\end{pgfonlayer}
\end{tikzpicture}
\hspace*{.5cm}
\begin{tikzpicture}
	\begin{pgfonlayer}{nodelayer}
		\node [style=none] (9) at (0.25, 1.25) {};
		\node [style=none] (10) at (0.75, 1.25) {};
		\node [style=X] (11) at (0.5, 2) {};
		\node [style=none] (12) at (0.5, 2.75) {};
	\end{pgfonlayer}
	\begin{pgfonlayer}{edgelayer}
		\draw [style=simple, in=90, out=-124] (11) to (9.center);
		\draw [style=simple, in=90, out=-56] (11) to (10.center);
		\draw [style=simple] (11) to (12.center);
	\end{pgfonlayer}
\end{tikzpicture}
\mapsto
\begin{tikzpicture}
	\begin{pgfonlayer}{nodelayer}
		\node [style=none] (15) at (0, 1.25) {};
		\node [style=none] (16) at (0, 2.75) {};
		\node [style=dot] (17) at (0.5, 2) {};
		\node [style=oplus] (18) at (0, 2) {};
		\node [style=Z] (19) at (0.5, 2.75) {};
		\node [style=none] (20) at (0.5, 1.25) {};
	\end{pgfonlayer}
	\begin{pgfonlayer}{edgelayer}
		\draw [style=simple] (20.center) to (17);
		\draw [style=simple] (17) to (19);
		\draw [style=simple] (17) to (18);
		\draw [style=simple] (18) to (16.center);
		\draw [style=simple] (18) to (15.center);
	\end{pgfonlayer}
\end{tikzpicture}
\hspace*{.5cm}
\begin{tikzpicture}
	\begin{pgfonlayer}{nodelayer}
		\node [style=X] (0) at (0, 1) {};
		\node [style=none] (1) at (0, 2) {};
	\end{pgfonlayer}
	\begin{pgfonlayer}{edgelayer}
		\draw [style=simple] (1.center) to (0);
	\end{pgfonlayer}
\end{tikzpicture}
\mapsto
\begin{tikzpicture}
	\begin{pgfonlayer}{nodelayer}
		\node [style=zeroin] (0) at (0, 1) {};
		\node [style=none] (1) at (0, 2) {};
	\end{pgfonlayer}
	\begin{pgfonlayer}{edgelayer}
		\draw [style=simple] (1.center) to (0);
	\end{pgfonlayer}
\end{tikzpicture}
\hspace*{.5cm}
\begin{tikzpicture}
	\begin{pgfonlayer}{nodelayer}
		\node [style=X] (1) at (0.5, 2) {};
		\node [style=none] (2) at (0.5, 1) {};
	\end{pgfonlayer}
	\begin{pgfonlayer}{edgelayer}
		\draw [style=simple] (2.center) to (1);
	\end{pgfonlayer}
\end{tikzpicture}
\mapsto
\begin{tikzpicture}
	\begin{pgfonlayer}{nodelayer}
		\node [style=zeroout] (1) at (0, 2) {};
		\node [style=none] (2) at (0, 1) {};
	\end{pgfonlayer}
	\begin{pgfonlayer}{edgelayer}
		\draw [style=simple] (2.center) to (1);
	\end{pgfonlayer}
\end{tikzpicture}
\hspace*{.5cm}
\begin{tikzpicture}
	\begin{pgfonlayer}{nodelayer}
		\node [style=none] (2) at (0.25, 2) {};
		\node [style=none] (3) at (0.75, 2) {};
		\node [style=Z] (4) at (0.5, 1.25) {};
		\node [style=none] (5) at (0.5, 0.5) {};
	\end{pgfonlayer}
	\begin{pgfonlayer}{edgelayer}
		\draw [style=simple, in=-90, out=124] (4) to (2.center);
		\draw [style=simple, in=-90, out=56] (4) to (3.center);
		\draw [style=simple] (4) to (5.center);
	\end{pgfonlayer}
\end{tikzpicture}
\mapsto
\begin{tikzpicture}
	\begin{pgfonlayer}{nodelayer}
		\node [style=none] (3) at (0, 2) {};
		\node [style=none] (4) at (0, 0.5) {};
		\node [style=oplus] (5) at (0.5, 1.25) {};
		\node [style=dot] (6) at (0, 1.25) {};
		\node [style=zeroin] (7) at (0.5, 0.5) {};
		\node [style=none] (8) at (0.5, 2) {};
	\end{pgfonlayer}
	\begin{pgfonlayer}{edgelayer}
		\draw [style=simple] (8.center) to (5);
		\draw [style=simple] (5) to (7);
		\draw [style=simple] (5) to (6);
		\draw [style=simple] (6) to (4.center);
		\draw [style=simple] (6) to (3.center);
	\end{pgfonlayer}
\end{tikzpicture}
\hspace*{.5cm}
\begin{tikzpicture}
	\begin{pgfonlayer}{nodelayer}
		\node [style=none] (4) at (0.25, 2.25) {};
		\node [style=none] (5) at (0.75, 2.25) {};
		\node [style=Z] (6) at (0.5, 3) {};
		\node [style=none] (7) at (0.5, 3.75) {};
	\end{pgfonlayer}
	\begin{pgfonlayer}{edgelayer}
		\draw [style=simple, in=90, out=-124] (6) to (4.center);
		\draw [style=simple, in=90, out=-56] (6) to (5.center);
		\draw [style=simple] (6) to (7.center);
	\end{pgfonlayer}
\end{tikzpicture}
\mapsto
\begin{tikzpicture}
	\begin{pgfonlayer}{nodelayer}
		\node [style=none] (0) at (6.5, -7.75) {};
		\node [style=none] (1) at (6.5, -6.25) {};
		\node [style=oplus] (2) at (7, -7) {};
		\node [style=dot] (3) at (6.5, -7) {};
		\node [style=zeroout] (4) at (7, -6.25) {};
		\node [style=none] (5) at (7, -7.75) {};
	\end{pgfonlayer}
	\begin{pgfonlayer}{edgelayer}
		\draw [style=simple] (5.center) to (2);
		\draw [style=simple] (2) to (4);
		\draw [style=simple] (2) to (3);
		\draw [style=simple] (3) to (1.center);
		\draw [style=simple] (3) to (0.center);
	\end{pgfonlayer}
\end{tikzpicture}
\hspace*{.5cm}
\begin{tikzpicture}
	\begin{pgfonlayer}{nodelayer}
		\node [style=Z] (1) at (0, 1) {};
		\node [style=none] (2) at (0, 2) {};
	\end{pgfonlayer}
	\begin{pgfonlayer}{edgelayer}
		\draw [style=simple] (2.center) to (1);
	\end{pgfonlayer}
\end{tikzpicture}
\mapsto
\begin{tikzpicture}
	\begin{pgfonlayer}{nodelayer}
		\node [style=Z] (2) at (0, 13.25) {};
		\node [style=none] (3) at (0, 14.25) {};
	\end{pgfonlayer}
	\begin{pgfonlayer}{edgelayer}
		\draw [style=simple] (3.center) to (2);
	\end{pgfonlayer}
\end{tikzpicture}
$$
$$
\begin{tikzpicture}
	\begin{pgfonlayer}{nodelayer}
		\node [style=Z] (0) at (0, 2) {};
		\node [style=none] (1) at (0, 1) {};
	\end{pgfonlayer}
	\begin{pgfonlayer}{edgelayer}
		\draw [style=simple] (1.center) to (0);
	\end{pgfonlayer}
\end{tikzpicture}
\mapsto
\begin{tikzpicture}
	\begin{pgfonlayer}{nodelayer}
		\node [style=Z] (0) at (0, 2) {};
		\node [style=none] (1) at (0, 1) {};
	\end{pgfonlayer}
	\begin{pgfonlayer}{edgelayer}
		\draw [style=simple] (1.center) to (0);
	\end{pgfonlayer}
\end{tikzpicture}
\hspace*{.5cm}
\begin{tikzpicture}
	\begin{pgfonlayer}{nodelayer}
		\node [style=X] (0) at (0, 1.5) {$1$};
		\node [style=none] (1) at (0, 2.5) {};
		\node [style=none] (2) at (0, 0.5) {};
	\end{pgfonlayer}
	\begin{pgfonlayer}{edgelayer}
		\draw [style=simple] (1.center) to (0);
		\draw [style=simple] (0) to (2.center);
	\end{pgfonlayer}
\end{tikzpicture}
\mapsto
\begin{tikzpicture}
	\begin{pgfonlayer}{nodelayer}
		\node [style=oplus] (1) at (0, 1) {};
		\node [style=none] (2) at (0, 2) {};
		\node [style=none] (3) at (0, 0) {};
	\end{pgfonlayer}
	\begin{pgfonlayer}{edgelayer}
		\draw [style=simple] (2.center) to (1);
		\draw [style=simple] (1) to (3.center);
	\end{pgfonlayer}
\end{tikzpicture}
\hspace*{.5cm}
\begin{tikzpicture}
	\begin{pgfonlayer}{nodelayer}
		\node [style=none] (2) at (0, 0) {};
		\node [style=none] (3) at (0.5, 0.75) {};
		\node [style=andin] (30) at (0.5, 0.75) {};
		\node [style=none] (4) at (0.5, 1.5) {};
		\node [style=none] (5) at (1, 0) {};
	\end{pgfonlayer}
	\begin{pgfonlayer}{edgelayer}
		\draw [style=simple] (4.center) to (3);
		\draw [style=simple, in=90, out=-124] (3) to (2.center);
		\draw [style=simple, in=90, out=-56] (3) to (5.center);
	\end{pgfonlayer}
\end{tikzpicture}
\mapsto
\begin{tikzpicture}
	\begin{pgfonlayer}{nodelayer}
		\node [style=dot] (0) at (0, 1.5) {};
		\node [style=dot] (1) at (0.5, 1.5) {};
		\node [style=oplus] (2) at (1, 1.5) {};
		\node [style=Z] (3) at (0, 2.25) {};
		\node [style=Z] (4) at (0.5, 2.25) {};
		\node [style=none] (5) at (1, 2.5) {};
		\node [style=zeroin] (6) at (1, 0.75) {};
		\node [style=none] (7) at (0, 0.5) {};
		\node [style=none] (8) at (0.5, 0.5) {};
	\end{pgfonlayer}
	\begin{pgfonlayer}{edgelayer}
		\draw [style=simple] (5.center) to (2);
		\draw [style=simple] (2) to (6);
		\draw [style=simple] (2) to (1);
		\draw [style=simple] (1) to (0);
		\draw [style=simple] (3) to (0);
		\draw [style=simple] (0) to (7.center);
		\draw [style=simple] (8.center) to (1);
		\draw [style=simple] (1) to (4);
	\end{pgfonlayer}
\end{tikzpicture}
\hspace*{.5cm}
\begin{tikzpicture}
	\begin{pgfonlayer}{nodelayer}
		\node [style=none] (0) at (0, 2) {};
		\node [style=andout] (1) at (0.5, 1.25) {};
		\node [style=none] (2) at (0.5, 0.5) {};
		\node [style=none] (3) at (1, 2) {};
	\end{pgfonlayer}
	\begin{pgfonlayer}{edgelayer}
		\draw [style=simple] (2.center) to (1.center);
		\draw [style=simple, in=-90, out=124] (1.center) to (0.center);
		\draw [style=simple, in=-90, out=56] (1.center) to (3.center);
	\end{pgfonlayer}
\end{tikzpicture}
\mapsto
\begin{tikzpicture}
	\begin{pgfonlayer}{nodelayer}
		\node [style=dot] (1) at (0, 1) {};
		\node [style=dot] (2) at (0.5, 1) {};
		\node [style=oplus] (3) at (1, 1) {};
		\node [style=Z] (4) at (0, 0.25) {};
		\node [style=Z] (5) at (0.5, 0.25) {};
		\node [style=none] (6) at (1, 0) {};
		\node [style=zeroout] (7) at (1, 1.75) {};
		\node [style=none] (8) at (0, 2) {};
		\node [style=none] (9) at (0.5, 2) {};
	\end{pgfonlayer}
	\begin{pgfonlayer}{edgelayer}
		\draw [style=simple] (6.center) to (3);
		\draw [style=simple] (3) to (7);
		\draw [style=simple] (3) to (2);
		\draw [style=simple] (2) to (1);
		\draw [style=simple] (4) to (1);
		\draw [style=simple] (1) to (8.center);
		\draw [style=simple] (9.center) to (2);
		\draw [style=simple] (2) to (5);
	\end{pgfonlayer}
\end{tikzpicture}
$$

This interpretation is a strict symmetric \dag-monoidal functor.
\end{proposition}


\begin{proof}
We prove that all of the axioms of $\ZXA$ hold in $\hat \TOF$:
\begin{enumerate}
\item[\ref{ZXA.1}:]
\begin{description}
\item[Unitality:] By Lemma \ref{lemma:whiteunit}:

\begin{align*}
\left\llbracket
\begin{tikzpicture}
	\begin{pgfonlayer}{nodelayer}
		\node [style=none] (0) at (0, 2) {};
		\node [style=none] (1) at (1, 2) {};
		\node [style=X] (2) at (0.5, 1.25) {};
		\node [style=none] (3) at (0.5, 0.5) {};
		\node [style=X] (4) at (0, 2) {};
	\end{pgfonlayer}
	\begin{pgfonlayer}{edgelayer}
		\draw [style=simple, in=-90, out=124] (2) to (0.center);
		\draw [style=simple, in=-90, out=56] (2) to (1.center);
		\draw [style=simple] (2) to (3.center);
	\end{pgfonlayer}
\end{tikzpicture}
\right\rrbracket_{\ZXA}
&=
\begin{tikzpicture}
	\begin{pgfonlayer}{nodelayer}
		\node [style=none] (0) at (0, 0.5) {};
		\node [style=dot] (1) at (0.75, 1.25) {};
		\node [style=oplus] (2) at (0, 1.25) {};
		\node [style=Z] (3) at (0.75, 0.5) {};
		\node [style=none] (4) at (0.75, 2) {};
		\node [style=zeroout] (5) at (0, 2) {};
	\end{pgfonlayer}
	\begin{pgfonlayer}{edgelayer}
		\draw [style=simple] (4.center) to (1);
		\draw [style=simple] (1) to (3);
		\draw [style=simple] (1) to (2);
		\draw [style=simple] (2) to (0.center);
		\draw [style=simple] (5) to (2);
	\end{pgfonlayer}
\end{tikzpicture}
\eq{Lem. \ref{lem:tof.frob}}
\begin{tikzpicture}
	\begin{pgfonlayer}{nodelayer}
		\node [style=none] (1) at (0.75, 0.5) {};
		\node [style=dot] (2) at (0.75, 1.25) {};
		\node [style=oplus] (3) at (0, 1.25) {};
		\node [style=Z] (4) at (0, 0.5) {};
		\node [style=none] (5) at (0.75, 2) {};
		\node [style=zeroout] (6) at (0, 2) {};
	\end{pgfonlayer}
	\begin{pgfonlayer}{edgelayer}
		\draw [style=simple] (5.center) to (2);
		\draw [style=simple] (2) to (3);
		\draw [style=simple] (6) to (3);
		\draw [style=simple] (1.center) to (2);
		\draw [style=simple] (3) to (4);
	\end{pgfonlayer}
\end{tikzpicture}
\eq{unit}
\begin{tikzpicture}
	\begin{pgfonlayer}{nodelayer}
		\node [style=none] (2) at (0.75, 0.5) {};
		\node [style=Z] (3) at (0, 0.5) {};
		\node [style=none] (4) at (0.75, 2) {};
		\node [style=zeroout] (5) at (0, 2) {};
	\end{pgfonlayer}
	\begin{pgfonlayer}{edgelayer}
		\draw [style=simple] (4.center) to (2.center);
		\draw [style=simple] (3) to (5);
	\end{pgfonlayer}
\end{tikzpicture}\
\eq{Lem. \ref{cor:copy}}
\begin{tikzpicture}
	\begin{pgfonlayer}{nodelayer}
		\node [style=none] (3) at (0.5, 3) {};
		\node [style=none] (4) at (0.5, 2) {};
	\end{pgfonlayer}
	\begin{pgfonlayer}{edgelayer}
		\draw [style=simple] (3.center) to (4.center);
	\end{pgfonlayer}
\end{tikzpicture}
=
\left\llbracket
\begin{tikzpicture}
	\begin{pgfonlayer}{nodelayer}
		\node [style=none] (4) at (0.5, 3) {};
		\node [style=none] (5) at (0.5, 2) {};
	\end{pgfonlayer}
	\begin{pgfonlayer}{edgelayer}
		\draw [style=simple] (4.center) to (5.center);
	\end{pgfonlayer}
\end{tikzpicture}
\right\rrbracket_{\ZXA}
\end{align*}

\item[Associativity:]
\begin{align*}
\left\llbracket
\begin{tikzpicture}
	\begin{pgfonlayer}{nodelayer}
		\node [style=none] (5) at (0, 3.5) {};
		\node [style=none] (6) at (1, 3.5) {};
		\node [style=X] (7) at (0.5, 2.75) {};
		\node [style=none] (8) at (0.5, 2) {};
		\node [style=X] (9) at (0, 3.5) {};
		\node [style=none] (10) at (0.5, 4.25) {};
		\node [style=none] (11) at (1, 4.25) {};
		\node [style=none] (12) at (-0.5, 4.25) {};
	\end{pgfonlayer}
	\begin{pgfonlayer}{edgelayer}
		\draw [style=simple, in=-90, out=124] (7) to (5.center);
		\draw [style=simple, in=-90, out=56] (7) to (6.center);
		\draw [style=simple] (7) to (8.center);
		\draw [style=simple, in=56, out=-90] (10.center) to (5.center);
		\draw [style=simple, in=-90, out=124] (5.center) to (12.center);
		\draw [style=simple] (11.center) to (6.center);
	\end{pgfonlayer}
\end{tikzpicture}
\right\rrbracket_{\ZXA}
=
\begin{tikzpicture}
	\begin{pgfonlayer}{nodelayer}
		\node [style=none] (6) at (0, 2) {};
		\node [style=dot] (7) at (1.25, 2.75) {};
		\node [style=oplus] (8) at (0, 2.75) {};
		\node [style=Z] (9) at (1.25, 2) {};
		\node [style=none] (10) at (1.25, 5) {};
		\node [style=dot] (11) at (0.75, 4.25) {};
		\node [style=Z] (12) at (0.75, 3.5) {};
		\node [style=oplus] (13) at (0, 4.25) {};
		\node [style=none] (14) at (0.75, 5) {};
		\node [style=none] (15) at (0, 5) {};
	\end{pgfonlayer}
	\begin{pgfonlayer}{edgelayer}
		\draw [style=simple] (10.center) to (7);
		\draw [style=simple] (7) to (9);
		\draw [style=simple] (7) to (8);
		\draw [style=simple] (8) to (6.center);
		\draw [style=simple] (14.center) to (11);
		\draw [style=simple] (11) to (12);
		\draw [style=simple] (11) to (13);
		\draw [style=simple] (13) to (15.center);
		\draw [style=simple] (13) to (8);
	\end{pgfonlayer}
\end{tikzpicture}
\eq{Lem. \ref{lemma:Iwama}}
\begin{tikzpicture}
	\begin{pgfonlayer}{nodelayer}
		\node [style=none] (7) at (0, 2) {};
		\node [style=Z] (8) at (1.5, 2) {};
		\node [style=none] (9) at (1.5, 5) {};
		\node [style=dot] (10) at (0.75, 3.5) {};
		\node [style=Z] (11) at (0.75, 2) {};
		\node [style=oplus] (12) at (0, 3.5) {};
		\node [style=none] (13) at (0.75, 5) {};
		\node [style=none] (14) at (0, 5) {};
		\node [style=oplus] (15) at (0.75, 4.25) {};
		\node [style=dot] (16) at (1.5, 4.25) {};
		\node [style=oplus] (17) at (0.75, 2.75) {};
		\node [style=dot] (18) at (1.5, 2.75) {};
	\end{pgfonlayer}
	\begin{pgfonlayer}{edgelayer}
		\draw [style=simple] (13.center) to (10);
		\draw [style=simple] (10) to (11);
		\draw [style=simple] (10) to (12);
		\draw [style=simple] (12) to (14.center);
		\draw [style=simple] (16) to (15);
		\draw [style=simple] (18) to (17);
		\draw [style=simple] (9.center) to (16);
		\draw [style=simple] (16) to (18);
		\draw [style=simple] (18) to (8);
		\draw [style=simple] (7.center) to (12);
	\end{pgfonlayer}
\end{tikzpicture}
\eq{Lem. \ref{cor:copy}}
\begin{tikzpicture}
	\begin{pgfonlayer}{nodelayer}
		\node [style=none] (8) at (0, 3.25) {};
		\node [style=dot] (9) at (0.75, 4) {};
		\node [style=oplus] (10) at (0, 4) {};
		\node [style=Z] (11) at (0.75, 3.25) {};
		\node [style=Z] (12) at (1.5, 4) {};
		\node [style=none] (13) at (0.75, 5.5) {};
		\node [style=none] (14) at (1.5, 5.5) {};
		\node [style=oplus] (15) at (0.75, 4.75) {};
		\node [style=dot] (16) at (1.5, 4.75) {};
		\node [style=none] (17) at (0, 5.5) {};
	\end{pgfonlayer}
	\begin{pgfonlayer}{edgelayer}
		\draw [style=simple] (9) to (11);
		\draw [style=simple] (9) to (10);
		\draw [style=simple] (10) to (8.center);
		\draw [style=simple] (14.center) to (16);
		\draw [style=simple] (16) to (12);
		\draw [style=simple] (16) to (15);
		\draw [style=simple] (15) to (13.center);
		\draw [style=simple] (9) to (15);
		\draw [style=simple] (17.center) to (10);
	\end{pgfonlayer}
\end{tikzpicture}
=
\left\llbracket
\begin{tikzpicture}
	\begin{pgfonlayer}{nodelayer}
		\node [style=none] (9) at (0.5, 4.75) {};
		\node [style=none] (10) at (-0.5, 4.75) {};
		\node [style=X] (11) at (0, 4) {};
		\node [style=none] (12) at (0, 3.25) {};
		\node [style=X] (13) at (0.5, 4.75) {};
		\node [style=none] (14) at (0, 5.5) {};
		\node [style=none] (15) at (-0.5, 5.5) {};
		\node [style=none] (16) at (1, 5.5) {};
	\end{pgfonlayer}
	\begin{pgfonlayer}{edgelayer}
		\draw [style=simple, in=-90, out=56] (11) to (9.center);
		\draw [style=simple, in=-90, out=124] (11) to (10.center);
		\draw [style=simple] (11) to (12.center);
		\draw [style=simple, in=124, out=-90] (14.center) to (9.center);
		\draw [style=simple, in=-90, out=56] (9.center) to (16.center);
		\draw [style=simple] (15.center) to (10.center);
	\end{pgfonlayer}
\end{tikzpicture}
\right\rrbracket_{\ZXA}
\end{align*}

\item[Frobenius:]
\begin{align*}
\left\llbracket
\begin{tikzpicture}
	\begin{pgfonlayer}{nodelayer}
		\node [style=none] (10) at (1, 4.75) {};
		\node [style=X] (11) at (0.5, 4) {};
		\node [style=none] (12) at (0.5, 3.25) {};
		\node [style=none] (13) at (0, 4.75) {};
		\node [style=X] (14) at (0, 4.75) {};
		\node [style=none] (15) at (0.5, 4) {};
		\node [style=none] (16) at (-0.5, 4) {};
		\node [style=none] (17) at (1, 5.5) {};
		\node [style=none] (18) at (0, 5.5) {};
		\node [style=none] (19) at (-0.5, 3.25) {};
	\end{pgfonlayer}
	\begin{pgfonlayer}{edgelayer}
		\draw [style=simple, in=-90, out=56] (11) to (10.center);
		\draw [style=simple] (11) to (12.center);
		\draw [style=simple, in=-56, out=90] (15.center) to (13.center);
		\draw [style=simple, in=90, out=-124] (13.center) to (16.center);
		\draw [style=simple] (17.center) to (10.center);
		\draw [style=simple] (13.center) to (18.center);
		\draw [style=simple] (16.center) to (19.center);
	\end{pgfonlayer}
\end{tikzpicture}
\right\rrbracket_{\ZXA}
=&
\begin{tikzpicture}
	\begin{pgfonlayer}{nodelayer}
		\node [style=none] (11) at (0, 3.25) {};
		\node [style=dot] (12) at (0.75, 4) {};
		\node [style=oplus] (13) at (0, 4) {};
		\node [style=Z] (14) at (0.75, 3.25) {};
		\node [style=none] (15) at (0, 5.5) {};
		\node [style=oplus] (16) at (0.75, 4.75) {};
		\node [style=Z] (17) at (1.5, 5.5) {};
		\node [style=dot] (18) at (1.5, 4.75) {};
		\node [style=none] (19) at (0.75, 5.5) {};
		\node [style=none] (20) at (1.5, 3.25) {};
	\end{pgfonlayer}
	\begin{pgfonlayer}{edgelayer}
		\draw [style=simple] (12) to (14);
		\draw [style=simple] (12) to (13);
		\draw [style=simple] (13) to (11.center);
		\draw [style=simple] (15.center) to (13);
		\draw [style=simple] (18) to (17);
		\draw [style=simple] (18) to (16);
		\draw [style=simple] (18) to (20.center);
		\draw [style=simple] (12) to (16);
		\draw [style=simple] (16) to (19.center);
	\end{pgfonlayer}
\end{tikzpicture}
\eq{Lem. \ref{lemma:Iwama}}
\begin{tikzpicture}
	\begin{pgfonlayer}{nodelayer}
		\node [style=dot] (12) at (0.75, 4.75) {};
		\node [style=oplus] (13) at (0, 4.75) {};
		\node [style=dot] (14) at (1.5, 5.5) {};
		\node [style=Z] (15) at (1.5, 6.25) {};
		\node [style=none] (16) at (0, 6.25) {};
		\node [style=none] (17) at (0.75, 6.25) {};
		\node [style=none] (18) at (0, 3.25) {};
		\node [style=Z] (19) at (0.75, 3.25) {};
		\node [style=none] (20) at (1.5, 3.25) {};
		\node [style=oplus] (21) at (0, 5.5) {};
		\node [style=oplus] (22) at (0.75, 4) {};
		\node [style=dot] (23) at (1.5, 4) {};
	\end{pgfonlayer}
	\begin{pgfonlayer}{edgelayer}
		\draw [style=simple] (12) to (13);
		\draw [style=simple] (15) to (14);
		\draw [style=simple] (20.center) to (23);
		\draw [style=simple] (23) to (14);
		\draw [style=simple] (14) to (21);
		\draw [style=simple] (17.center) to (12);
		\draw [style=simple] (12) to (22);
		\draw [style=simple] (22) to (19);
		\draw [style=simple] (18.center) to (13);
		\draw [style=simple] (13) to (21);
		\draw [style=simple] (21) to (16.center);
		\draw [style=simple] (23) to (22);
	\end{pgfonlayer}
\end{tikzpicture}
\eq{Lem. \ref{lemma:whiteunit}}
\begin{tikzpicture}
	\begin{pgfonlayer}{nodelayer}
		\node [style=dot] (13) at (0.75, 4.75) {};
		\node [style=oplus] (14) at (0, 4.75) {};
		\node [style=dot] (15) at (1.5, 5.5) {};
		\node [style=Z] (16) at (1.5, 6.25) {};
		\node [style=none] (17) at (0, 6.25) {};
		\node [style=none] (18) at (0.75, 6.25) {};
		\node [style=none] (19) at (0, 3.25) {};
		\node [style=Z] (20) at (0.75, 3.25) {};
		\node [style=none] (21) at (1.5, 3.25) {};
		\node [style=oplus] (22) at (0, 5.5) {};
	\end{pgfonlayer}
	\begin{pgfonlayer}{edgelayer}
		\draw [style=simple] (13) to (14);
		\draw [style=simple] (16) to (15);
		\draw [style=simple] (15) to (22);
		\draw [style=simple] (18.center) to (13);
		\draw [style=simple] (19.center) to (14);
		\draw [style=simple] (14) to (22);
		\draw [style=simple] (22) to (17.center);
		\draw [style=simple] (15) to (21.center);
		\draw [style=simple] (20) to (13);
	\end{pgfonlayer}
\end{tikzpicture}\\
\eq{Lem. \ref{CNOT.5}}&
\begin{tikzpicture}
	\begin{pgfonlayer}{nodelayer}
		\node [style=dot] (14) at (0.75, 6.25) {};
		\node [style=oplus] (15) at (0, 6.25) {};
		\node [style=dot] (16) at (0.75, 4) {};
		\node [style=Z] (17) at (0.75, 4.75) {};
		\node [style=none] (18) at (0, 7) {};
		\node [style=none] (19) at (0.75, 7) {};
		\node [style=none] (20) at (0, 3.25) {};
		\node [style=Z] (21) at (0.75, 5.5) {};
		\node [style=none] (22) at (0.75, 3.25) {};
		\node [style=oplus] (23) at (0, 4) {};
	\end{pgfonlayer}
	\begin{pgfonlayer}{edgelayer}
		\draw [style=simple] (14) to (15);
		\draw [style=simple] (17) to (16);
		\draw [style=simple] (16) to (23);
		\draw [style=simple] (19.center) to (14);
		\draw [style=simple] (20.center) to (15);
		\draw [style=simple] (23) to (18.center);
		\draw [style=simple] (16) to (22.center);
		\draw [style=simple] (21) to (14);
	\end{pgfonlayer}
\end{tikzpicture}
=
\left\llbracket
\begin{tikzpicture}
	\begin{pgfonlayer}{nodelayer}
		\node [style=none] (15) at (1, 5.75) {};
		\node [style=X] (16) at (0.5, 5) {};
		\node [style=none] (17) at (0, 5.75) {};
		\node [style=none] (18) at (0.5, 5) {};
		\node [style=none] (19) at (1, 3.25) {};
		\node [style=none] (20) at (0.5, 4) {};
		\node [style=X] (21) at (0.5, 4) {};
		\node [style=none] (22) at (0, 3.25) {};
	\end{pgfonlayer}
	\begin{pgfonlayer}{edgelayer}
		\draw [style=simple, in=-90, out=60] (16) to (15.center);
		\draw [style=simple, in=-90, out=120] (18.center) to (17.center);
		\draw [style=simple, in=90, out=-60] (21) to (19.center);
		\draw [style=simple, in=90, out=-120] (20.center) to (22.center);
		\draw [style=simple] (16) to (20.center);
	\end{pgfonlayer}
\end{tikzpicture}
\right\rrbracket_{\ZXA}
\end{align*}


\item[Phase amalgamation:]

\begin{align*}
\left\llbracket
\begin{tikzpicture}
	\begin{pgfonlayer}{nodelayer}
		\node [style=none] (16) at (0.75, 3.25) {};
		\node [style=X] (17) at (0.75, 4.25) {$1$};
		\node [style=X] (18) at (0.75, 5.25) {$1$};
		\node [style=none] (19) at (0.75, 6.25) {};
	\end{pgfonlayer}
	\begin{pgfonlayer}{edgelayer}
		\draw [style=simple] (19.center) to (18);
		\draw [style=simple] (18) to (17);
		\draw [style=simple] (17) to (16.center);
	\end{pgfonlayer}
\end{tikzpicture}
\right\rrbracket_{\ZXA}
&=
\begin{tikzpicture}
	\begin{pgfonlayer}{nodelayer}
		\node [style=none] (17) at (0.75, 3.25) {};
		\node [style=none] (18) at (0.75, 6.25) {};
		\node [style=oplus] (19) at (0.75, 4.25) {};
		\node [style=oplus] (20) at (0.75, 5.25) {};
	\end{pgfonlayer}
	\begin{pgfonlayer}{edgelayer}
		\draw [style=simple] (18.center) to (20);
		\draw [style=simple] (20) to (19);
		\draw [style=simple] (19) to (17.center);
	\end{pgfonlayer}
\end{tikzpicture}
=
\begin{tikzpicture}
	\begin{pgfonlayer}{nodelayer}
		\node [style=none] (18) at (0.75, 3.25) {};
		\node [style=none] (19) at (0.75, 4.25) {};
	\end{pgfonlayer}
	\begin{pgfonlayer}{edgelayer}
		\draw [style=simple] (19.center) to (18.center);
	\end{pgfonlayer}
\end{tikzpicture}
=
\left\llbracket
\begin{tikzpicture}
	\begin{pgfonlayer}{nodelayer}
		\node [style=none] (19) at (0.75, 3.25) {};
		\node [style=none] (20) at (0.75, 4.25) {};
	\end{pgfonlayer}
	\begin{pgfonlayer}{edgelayer}
		\draw [style=simple] (20.center) to (19.center);
	\end{pgfonlayer}
\end{tikzpicture}
\right\rrbracket_{\ZXA}
\end{align*}



\end{description}


\item[\ref{ZXA.2}:]
\begin{align*}
\left\llbracket
\begin{tikzpicture}
	\begin{pgfonlayer}{nodelayer}
		\node [style=none] (20) at (0, 4.25) {};
		\node [style=none] (21) at (0, 3.25) {};
		\node [style=X] (22) at (0, 4.25) {};
		\node [style=none] (23) at (0.5, 5) {};
		\node [style=none] (24) at (-0.5, 5) {};
		\node [style=none] (25) at (0.5, 5.75) {};
		\node [style=none] (26) at (-0.5, 5.75) {};
	\end{pgfonlayer}
	\begin{pgfonlayer}{edgelayer}
		\draw [style=simple, in=56, out=-90] (23.center) to (20.center);
		\draw [style=simple, in=-90, out=124] (20.center) to (24.center);
		\draw [style=simple] (21.center) to (20.center);
		\draw [style=simple, in=90, out=-90] (25.center) to (24.center);
		\draw [style=simple, in=90, out=-90] (26.center) to (23.center);
	\end{pgfonlayer}
\end{tikzpicture}
\right\rrbracket_{\ZXA}
=
\begin{tikzpicture}
	\begin{pgfonlayer}{nodelayer}
		\node [style=oplus] (21) at (0, 4.25) {};
		\node [style=dot] (22) at (0.75, 4.25) {};
		\node [style=Z] (23) at (0.75, 3.5) {};
		\node [style=none] (24) at (0, 3.25) {};
		\node [style=none] (25) at (0, 5) {};
		\node [style=none] (26) at (0.75, 5) {};
		\node [style=none] (27) at (0, 6) {};
		\node [style=none] (28) at (0.75, 6) {};
	\end{pgfonlayer}
	\begin{pgfonlayer}{edgelayer}
		\draw [style=simple] (26.center) to (22);
		\draw [style=simple] (23) to (22);
		\draw [style=simple] (22) to (21);
		\draw [style=simple] (21) to (24.center);
		\draw [style=simple] (21) to (25.center);
		\draw [style=simple, in=90, out=-90] (28.center) to (25.center);
		\draw [style=simple, in=90, out=-90] (27.center) to (26.center);
	\end{pgfonlayer}
\end{tikzpicture}
\eq{\ref{TOF.14}}
\begin{tikzpicture}
	\begin{pgfonlayer}{nodelayer}
		\node [style=oplus] (22) at (0, 4.25) {};
		\node [style=dot] (23) at (0.75, 4.25) {};
		\node [style=Z] (24) at (0.75, 3.5) {};
		\node [style=none] (25) at (0, 3.25) {};
		\node [style=none] (26) at (0, 6.5) {};
		\node [style=none] (27) at (0.75, 6.5) {};
		\node [style=oplus] (28) at (0, 4.75) {};
		\node [style=dot] (29) at (0.75, 5.75) {};
		\node [style=dot] (30) at (0.75, 4.75) {};
		\node [style=oplus] (31) at (0.75, 5.25) {};
		\node [style=oplus] (32) at (0, 5.75) {};
		\node [style=dot] (33) at (0, 5.25) {};
	\end{pgfonlayer}
	\begin{pgfonlayer}{edgelayer}
		\draw [style=simple] (24) to (23);
		\draw [style=simple] (23) to (22);
		\draw [style=simple] (22) to (25.center);
		\draw [style=simple] (30) to (28);
		\draw [style=simple] (33) to (31);
		\draw [style=simple] (29) to (32);
		\draw [style=simple] (32) to (33);
		\draw [style=simple] (33) to (28);
		\draw [style=simple] (30) to (31);
		\draw [style=simple] (31) to (29);
		\draw [style=simple] (27.center) to (29);
		\draw [style=simple] (30) to (23);
		\draw [style=simple] (22) to (28);
		\draw [style=simple] (32) to (26.center);
	\end{pgfonlayer}
\end{tikzpicture}
\eq{\ref{CNOT.2}}
\begin{tikzpicture}
	\begin{pgfonlayer}{nodelayer}
		\node [style=Z] (2) at (1.25, 4.5) {};
		\node [style=none] (3) at (0.5, 4.25) {};
		\node [style=none] (4) at (0.5, 6.5) {};
		\node [style=none] (5) at (1.25, 6.5) {};
		\node [style=dot] (7) at (1.25, 5.75) {};
		\node [style=oplus] (9) at (1.25, 5.25) {};
		\node [style=oplus] (10) at (0.5, 5.75) {};
		\node [style=dot] (11) at (0.5, 5.25) {};
	\end{pgfonlayer}
	\begin{pgfonlayer}{edgelayer}
		\draw [style=simple] (11) to (9);
		\draw [style=simple] (7) to (10);
		\draw [style=simple] (10) to (11);
		\draw [style=simple] (9) to (7);
		\draw [style=simple] (5.center) to (7);
		\draw [style=simple] (10) to (4.center);
		\draw (2) to (9);
		\draw (3.center) to (11);
	\end{pgfonlayer}
\end{tikzpicture}
\eq{Lem. \ref{lemma:whiteunit}}
\begin{tikzpicture}
	\begin{pgfonlayer}{nodelayer}
		\node [style=oplus] (0) at (0, 4) {};
		\node [style=dot] (1) at (0.75, 4) {};
		\node [style=Z] (2) at (0.75, 3.25) {};
		\node [style=none] (3) at (0, 3) {};
		\node [style=none] (4) at (0, 4.75) {};
		\node [style=none] (5) at (0.75, 4.75) {};
	\end{pgfonlayer}
	\begin{pgfonlayer}{edgelayer}
		\draw [style=simple] (5.center) to (1);
		\draw [style=simple] (2) to (1);
		\draw [style=simple] (1) to (0);
		\draw [style=simple] (0) to (3.center);
		\draw [style=simple] (0) to (4.center);
	\end{pgfonlayer}
\end{tikzpicture}
=
\left\llbracket
\begin{tikzpicture}
	\begin{pgfonlayer}{nodelayer}
		\node [style=none] (0) at (0, 2) {};
		\node [style=none] (1) at (0, 1) {};
		\node [style=X] (2) at (0, 2) {};
		\node [style=none] (3) at (0.5, 2.75) {};
		\node [style=none] (4) at (-0.5, 2.75) {};
	\end{pgfonlayer}
	\begin{pgfonlayer}{edgelayer}
		\draw [style=simple, in=56, out=-90] (3.center) to (0.center);
		\draw [style=simple, in=-90, out=124] (0.center) to (4.center);
		\draw [style=simple] (1.center) to (0.center);
	\end{pgfonlayer}
\end{tikzpicture}
\right\rrbracket_{\ZXA}
\end{align*}


\item[\ref{ZXA.3}:]
This is immediate.

\item[\ref{ZXA.4}:]
This is immediate.


\item[\ref{ZXA.5}:]

\begin{align*}
\left\llbracket
\begin{tikzpicture}
	\begin{pgfonlayer}{nodelayer}
		\node [style=Z] (0) at (-1, 1) {};
		\node [style=Z] (1) at (-0.25, 1) {};
		\node [style=X] (2) at (-0.25, 1.75) {};
		\node [style=X] (3) at (-1, 1.75) {};
		\node [style=none] (4) at (-1, 2.25) {};
		\node [style=none] (5) at (-0.25, 2.25) {};
		\node [style=none] (6) at (-1, 0.5) {};
		\node [style=none] (7) at (-0.25, 0.5) {};
	\end{pgfonlayer}
	\begin{pgfonlayer}{edgelayer}
		\draw (7.center) to (1);
		\draw (1) to (3);
		\draw [in=120, out=-120, looseness=1.25] (3) to (0);
		\draw (0) to (2);
		\draw (2) to (5.center);
		\draw [in=60, out=-60, looseness=1.25] (2) to (1);
		\draw (0) to (6.center);
		\draw (3) to (4.center);
	\end{pgfonlayer}
\end{tikzpicture}
\right\rrbracket_{\ZXA}
&=
\begin{tikzpicture}
	\begin{pgfonlayer}{nodelayer}
		\node [style=none] (0) at (-1, 0.5) {};
		\node [style=none] (1) at (0.5, 0.5) {};
		\node [style=none] (2) at (-1, 3) {};
		\node [style=none] (3) at (0.5, 3) {};
		\node [style=zeroin] (4) at (-0.5, 0.75) {};
		\node [style=zeroin] (5) at (0, 0.75) {};
		\node [style=oplus] (6) at (-0.5, 1.25) {};
		\node [style=oplus] (7) at (0, 1.25) {};
		\node [style=dot] (8) at (-1, 1.25) {};
		\node [style=dot] (9) at (0.5, 1.25) {};
		\node [style=dot] (10) at (0, 1.75) {};
		\node [style=dot] (11) at (-0.5, 2.25) {};
		\node [style=oplus] (12) at (-1, 1.75) {};
		\node [style=oplus] (13) at (0.5, 2.25) {};
		\node [style=Z] (14) at (-0.5, 2.75) {};
		\node [style=Z] (15) at (0, 2.75) {};
	\end{pgfonlayer}
	\begin{pgfonlayer}{edgelayer}
		\draw (1.center) to (9);
		\draw (9) to (13);
		\draw (13) to (3.center);
		\draw (15) to (10);
		\draw (10) to (7);
		\draw (7) to (5);
		\draw (4) to (6);
		\draw (6) to (11);
		\draw (11) to (14);
		\draw (2.center) to (12);
		\draw (12) to (8);
		\draw (8) to (0.center);
		\draw (8) to (6);
		\draw (7) to (9);
		\draw (10) to (12);
		\draw (11) to (13);
	\end{pgfonlayer}
\end{tikzpicture}
\eq{Lem. \ref{lemma:Iwama}}
\begin{tikzpicture}
	\begin{pgfonlayer}{nodelayer}
		\node [style=none] (0) at (-1, 0.5) {};
		\node [style=none] (1) at (0.5, 0.5) {};
		\node [style=none] (2) at (-1, 4) {};
		\node [style=none] (3) at (0.5, 4) {};
		\node [style=zeroin] (4) at (-0.5, 0.75) {};
		\node [style=zeroin] (5) at (0, 0.75) {};
		\node [style=oplus] (6) at (-0.5, 1.25) {};
		\node [style=oplus] (7) at (0, 2.75) {};
		\node [style=dot] (8) at (-1, 1.25) {};
		\node [style=dot] (9) at (0.5, 2.75) {};
		\node [style=dot] (10) at (0, 2.25) {};
		\node [style=dot] (11) at (-0.5, 3.25) {};
		\node [style=oplus] (12) at (-1, 2.25) {};
		\node [style=oplus] (13) at (0.5, 3.25) {};
		\node [style=Z] (14) at (-0.5, 3.75) {};
		\node [style=Z] (15) at (0, 3.75) {};
		\node [style=oplus] (16) at (-1, 1.75) {};
		\node [style=dot] (17) at (0.5, 1.75) {};
	\end{pgfonlayer}
	\begin{pgfonlayer}{edgelayer}
		\draw (1.center) to (9);
		\draw (9) to (13);
		\draw (13) to (3.center);
		\draw (15) to (10);
		\draw (10) to (7);
		\draw (7) to (5);
		\draw (4) to (6);
		\draw (6) to (11);
		\draw (11) to (14);
		\draw (2.center) to (12);
		\draw (12) to (8);
		\draw (8) to (0.center);
		\draw (8) to (6);
		\draw (7) to (9);
		\draw (10) to (12);
		\draw (11) to (13);
		\draw (17) to (16);
	\end{pgfonlayer}
\end{tikzpicture}
\eq{\ref{TOF.2}}
\begin{tikzpicture}
	\begin{pgfonlayer}{nodelayer}
		\node [style=none] (0) at (-1, 0.5) {};
		\node [style=none] (1) at (0.5, 0.5) {};
		\node [style=none] (2) at (-1, 3.5) {};
		\node [style=none] (3) at (0.5, 3.5) {};
		\node [style=zeroin] (4) at (-0.5, 0.75) {};
		\node [style=zeroin] (5) at (0, 0.75) {};
		\node [style=oplus] (6) at (-0.5, 1.25) {};
		\node [style=oplus] (7) at (0, 2.25) {};
		\node [style=dot] (8) at (-1, 1.25) {};
		\node [style=dot] (9) at (0.5, 2.25) {};
		\node [style=dot] (10) at (-0.5, 2.75) {};
		\node [style=oplus] (11) at (0.5, 2.75) {};
		\node [style=Z] (12) at (-0.5, 3.25) {};
		\node [style=Z] (13) at (0, 3.25) {};
		\node [style=oplus] (14) at (-1, 1.75) {};
		\node [style=dot] (15) at (0.5, 1.75) {};
	\end{pgfonlayer}
	\begin{pgfonlayer}{edgelayer}
		\draw (1.center) to (9);
		\draw (9) to (11);
		\draw (11) to (3.center);
		\draw (7) to (5);
		\draw (4) to (6);
		\draw (6) to (10);
		\draw (10) to (12);
		\draw (8) to (0.center);
		\draw (8) to (6);
		\draw (7) to (9);
		\draw (10) to (11);
		\draw (15) to (14);
		\draw (2.center) to (14);
		\draw (14) to (8);
		\draw (7) to (13);
	\end{pgfonlayer}
\end{tikzpicture}
\eq{unit}
\begin{tikzpicture}
	\begin{pgfonlayer}{nodelayer}
		\node [style=none] (0) at (-1, 0.5) {};
		\node [style=none] (1) at (0, 0.5) {};
		\node [style=none] (2) at (-1, 3) {};
		\node [style=none] (3) at (0, 3) {};
		\node [style=zeroin] (4) at (-0.5, 0.75) {};
		\node [style=oplus] (5) at (-0.5, 1.25) {};
		\node [style=dot] (6) at (-1, 1.25) {};
		\node [style=dot] (7) at (-0.5, 2.25) {};
		\node [style=oplus] (8) at (0, 2.25) {};
		\node [style=Z] (9) at (-0.5, 2.75) {};
		\node [style=oplus] (10) at (-1, 1.75) {};
		\node [style=dot] (11) at (0, 1.75) {};
	\end{pgfonlayer}
	\begin{pgfonlayer}{edgelayer}
		\draw (8) to (3.center);
		\draw (4) to (5);
		\draw (5) to (7);
		\draw (7) to (9);
		\draw (6) to (0.center);
		\draw (6) to (5);
		\draw (7) to (8);
		\draw (11) to (10);
		\draw (2.center) to (10);
		\draw (10) to (6);
		\draw (8) to (11);
		\draw (11) to (1.center);
	\end{pgfonlayer}
\end{tikzpicture}\\
&=
\begin{tikzpicture}
	\begin{pgfonlayer}{nodelayer}
		\node [style=none] (0) at (-0.75, 0.5) {};
		\node [style=none] (1) at (0, 0.5) {};
		\node [style=none] (2) at (-1, 3) {};
		\node [style=none] (3) at (0, 3) {};
		\node [style=dot] (4) at (-0.5, 2.25) {};
		\node [style=oplus] (5) at (0, 2.25) {};
		\node [style=Z] (6) at (-0.5, 2.75) {};
		\node [style=oplus] (7) at (-1, 1.75) {};
		\node [style=dot] (8) at (0, 1.75) {};
		\node [style=fanout] (9) at (-0.75, 1) {};
		\node [style=none] (10) at (-0.5, 1.75) {};
	\end{pgfonlayer}
	\begin{pgfonlayer}{edgelayer}
		\draw (5) to (3.center);
		\draw (4) to (6);
		\draw (4) to (5);
		\draw (8) to (7);
		\draw (2.center) to (7);
		\draw (5) to (8);
		\draw (8) to (1.center);
		\draw (0.center) to (9);
		\draw [in=-90, out=108] (9) to (7);
		\draw (4) to (10.center);
		\draw [in=72, out=-90] (10.center) to (9);
	\end{pgfonlayer}
\end{tikzpicture}
\eq{\ref{CNOT.2}}
\begin{tikzpicture}
	\begin{pgfonlayer}{nodelayer}
		\node [style=none] (0) at (-0.75, 0.5) {};
		\node [style=none] (1) at (0, 0.5) {};
		\node [style=none] (2) at (-1, 4) {};
		\node [style=none] (3) at (0, 4) {};
		\node [style=dot] (4) at (-0.5, 3.25) {};
		\node [style=oplus] (5) at (0, 3.25) {};
		\node [style=Z] (6) at (-0.5, 3.75) {};
		\node [style=oplus] (7) at (-1, 2.75) {};
		\node [style=dot] (8) at (0, 2.75) {};
		\node [style=fanout] (9) at (-0.75, 1) {};
		\node [style=oplus] (10) at (-0.5, 2.25) {};
		\node [style=dot] (11) at (0, 2.25) {};
		\node [style=oplus] (12) at (-0.5, 1.75) {};
		\node [style=dot] (13) at (0, 1.75) {};
		\node [style=none] (14) at (-1, 1.75) {};
	\end{pgfonlayer}
	\begin{pgfonlayer}{edgelayer}
		\draw (5) to (3.center);
		\draw (4) to (6);
		\draw (4) to (5);
		\draw (8) to (7);
		\draw (2.center) to (7);
		\draw (5) to (8);
		\draw (8) to (1.center);
		\draw (0.center) to (9);
		\draw (11) to (10);
		\draw (13) to (12);
		\draw (4) to (10);
		\draw (10) to (12);
		\draw [in=60, out=-90] (12) to (9);
		\draw (7) to (14.center);
		\draw [in=120, out=-90] (14.center) to (9);
	\end{pgfonlayer}
\end{tikzpicture}\\
&\eq{Lem. \ref{lemma:natoplus}}
\begin{tikzpicture}
	\begin{pgfonlayer}{nodelayer}
		\node [style=none] (0) at (-0.75, 0.5) {};
		\node [style=none] (1) at (0, 0.5) {};
		\node [style=none] (2) at (-1, 3.75) {};
		\node [style=none] (3) at (0, 3.75) {};
		\node [style=dot] (4) at (-0.5, 3) {};
		\node [style=oplus] (5) at (0, 3) {};
		\node [style=Z] (6) at (-0.5, 3.5) {};
		\node [style=fanout] (7) at (-0.75, 1.75) {};
		\node [style=oplus] (8) at (-0.5, 2.5) {};
		\node [style=dot] (9) at (0, 2.5) {};
		\node [style=none] (10) at (-1, 2.5) {};
		\node [style=oplus] (11) at (-0.75, 1) {};
		\node [style=dot] (12) at (0, 1) {};
	\end{pgfonlayer}
	\begin{pgfonlayer}{edgelayer}
		\draw (5) to (3.center);
		\draw (4) to (6);
		\draw (4) to (5);
		\draw (0.center) to (7);
		\draw (9) to (8);
		\draw (4) to (8);
		\draw [in=120, out=-90] (10.center) to (7);
		\draw (12) to (11);
		\draw [in=-90, out=60] (7) to (8);
		\draw (9) to (12);
		\draw (12) to (1.center);
		\draw (9) to (5);
		\draw (2.center) to (10.center);
	\end{pgfonlayer}
\end{tikzpicture}
\eq{Lem. \ref{lemma:whiteunit}}
\begin{tikzpicture}
	\begin{pgfonlayer}{nodelayer}
		\node [style=none] (0) at (-0.75, 0.5) {};
		\node [style=none] (1) at (0, 0.5) {};
		\node [style=none] (2) at (-1, 4.25) {};
		\node [style=none] (3) at (0, 4.25) {};
		\node [style=dot] (4) at (-0.5, 3) {};
		\node [style=oplus] (5) at (0, 3) {};
		\node [style=Z] (6) at (-0.5, 4) {};
		\node [style=fanout] (7) at (-0.75, 1.75) {};
		\node [style=oplus] (8) at (-0.5, 2.5) {};
		\node [style=dot] (9) at (0, 2.5) {};
		\node [style=none] (10) at (-1, 2.5) {};
		\node [style=oplus] (11) at (-0.75, 1) {};
		\node [style=dot] (12) at (0, 1) {};
		\node [style=oplus] (13) at (-0.5, 3.5) {};
		\node [style=dot] (14) at (0, 3.5) {};
	\end{pgfonlayer}
	\begin{pgfonlayer}{edgelayer}
		\draw (5) to (3.center);
		\draw (4) to (6);
		\draw (4) to (5);
		\draw (0.center) to (7);
		\draw (9) to (8);
		\draw (4) to (8);
		\draw [in=120, out=-90] (10.center) to (7);
		\draw (12) to (11);
		\draw [in=-90, out=60] (7) to (8);
		\draw (9) to (12);
		\draw (12) to (1.center);
		\draw (9) to (5);
		\draw (2.center) to (10.center);
		\draw (14) to (13);
	\end{pgfonlayer}
\end{tikzpicture}
\eq{\ref{TOF.14}}
\begin{tikzpicture}
	\begin{pgfonlayer}{nodelayer}
		\node [style=none] (0) at (-0.75, 0.5) {};
		\node [style=none] (1) at (0, 0.5) {};
		\node [style=none] (2) at (-1, 3.75) {};
		\node [style=none] (3) at (0, 3.75) {};
		\node [style=Z] (4) at (-0.5, 3.5) {};
		\node [style=fanout] (5) at (-0.75, 1.75) {};
		\node [style=none] (6) at (-1, 2.5) {};
		\node [style=oplus] (7) at (-0.75, 1) {};
		\node [style=dot] (8) at (0, 1) {};
		\node [style=none] (9) at (0, 3.5) {};
		\node [style=none] (10) at (0, 2.5) {};
		\node [style=none] (11) at (-0.5, 2.5) {};
	\end{pgfonlayer}
	\begin{pgfonlayer}{edgelayer}
		\draw (0.center) to (5);
		\draw [in=120, out=-90] (6.center) to (5);
		\draw (8) to (7);
		\draw (8) to (1.center);
		\draw (2.center) to (6.center);
		\draw [in=90, out=-90] (9.center) to (11.center);
		\draw [in=-90, out=90] (10.center) to (4);
		\draw (3.center) to (9.center);
		\draw (10.center) to (8);
		\draw [in=60, out=-90] (11.center) to (5);
	\end{pgfonlayer}
\end{tikzpicture}
=
\begin{tikzpicture}
	\begin{pgfonlayer}{nodelayer}
		\node [style=none] (0) at (-0.75, 0.5) {};
		\node [style=none] (1) at (-0.25, 0.5) {};
		\node [style=Z] (2) at (-0.25, 1.5) {};
		\node [style=fanout] (3) at (-0.75, 1.75) {};
		\node [style=none] (4) at (-1, 2.5) {};
		\node [style=oplus] (5) at (-0.75, 1) {};
		\node [style=dot] (6) at (-0.25, 1) {};
		\node [style=none] (7) at (-0.5, 2.5) {};
	\end{pgfonlayer}
	\begin{pgfonlayer}{edgelayer}
		\draw (0.center) to (3);
		\draw [in=120, out=-90] (4.center) to (3);
		\draw (6) to (5);
		\draw (6) to (1.center);
		\draw [in=60, out=-90] (7.center) to (3);
		\draw (2) to (6);
	\end{pgfonlayer}
\end{tikzpicture}
=
\left\llbracket
\begin{tikzpicture}
	\begin{pgfonlayer}{nodelayer}
		\node [style=X] (0) at (-1, 1) {};
		\node [style=none] (1) at (-1.25, 0.5) {};
		\node [style=none] (2) at (-0.75, 0.5) {};
		\node [style=Z] (3) at (-1, 1.75) {};
		\node [style=none] (4) at (-1.25, 2.25) {};
		\node [style=none] (5) at (-0.75, 2.25) {};
	\end{pgfonlayer}
	\begin{pgfonlayer}{edgelayer}
		\draw [in=63, out=-90] (5.center) to (3);
		\draw (3) to (0);
		\draw [in=90, out=-117] (0) to (1.center);
		\draw [in=-63, out=90] (2.center) to (0);
		\draw [in=-90, out=117] (3) to (4.center);
	\end{pgfonlayer}
\end{tikzpicture}
\right\rrbracket_{\ZXA}
\end{align*}



\item[\ref{ZXA.6}:]

$$
\left\llbracket
\begin{tikzpicture}
	\begin{pgfonlayer}{nodelayer}
		\node [style=none] (0) at (-0.25, 2) {};
		\node [style=Z] (1) at (0, 1.25) {};
		\node [style=X] (2) at (0, 0.5) {};
		\node [style=none] (3) at (0.25, 2) {};
	\end{pgfonlayer}
	\begin{pgfonlayer}{edgelayer}
		\draw [style=simple, in=-90, out=124] (1) to (0.center);
		\draw [style=simple, in=60, out=-90] (3.center) to (1);
		\draw [style=simple] (1) to (2);
	\end{pgfonlayer}
\end{tikzpicture}
\right\rrbracket_{\ZXA}
=
\begin{tikzpicture}
	\begin{pgfonlayer}{nodelayer}
		\node [style=dot] (0) at (0, 1) {};
		\node [style=zeroin] (1) at (0, 0.5) {};
		\node [style=zeroin] (2) at (0.75, 0.5) {};
		\node [style=none] (3) at (0.75, 1.5) {};
		\node [style=none] (4) at (0, 1.5) {};
		\node [style=oplus] (5) at (0.75, 1) {};
	\end{pgfonlayer}
	\begin{pgfonlayer}{edgelayer}
		\draw [style=simple] (0) to (4.center);
		\draw [style=simple] (0) to (1);
		\draw [style=simple] (2) to (5);
		\draw [style=simple] (5) to (3.center);
		\draw [style=simple] (5) to (0);
	\end{pgfonlayer}
\end{tikzpicture}
\eq{\ref{TOF.2}}
\begin{tikzpicture}
	\begin{pgfonlayer}{nodelayer}
		\node [style=zeroin] (0) at (0, 0.5) {};
		\node [style=zeroin] (1) at (0.75, 0.5) {};
		\node [style=none] (2) at (0.75, 1.5) {};
		\node [style=none] (3) at (0, 1.5) {};
	\end{pgfonlayer}
	\begin{pgfonlayer}{edgelayer}
		\draw [style=simple] (2.center) to (1);
		\draw [style=simple] (0) to (3.center);
	\end{pgfonlayer}
\end{tikzpicture}
=
\left\llbracket
\begin{tikzpicture}
	\begin{pgfonlayer}{nodelayer}
		\node [style=none] (0) at (-0.25, 1) {};
		\node [style=X] (1) at (-0.25, 0.5) {};
		\node [style=none] (2) at (0.25, 1) {};
		\node [style=X] (3) at (0.25, 0.5) {};
	\end{pgfonlayer}
	\begin{pgfonlayer}{edgelayer}
		\draw [style=simple] (3) to (2.center);
		\draw [style=simple] (1) to (0.center);
	\end{pgfonlayer}
\end{tikzpicture}
\right\rrbracket_{\ZXA}
$$



\item[\ref{ZXA.7}:]
This is immediate.

%
%
%\item[\ref{ZXA.13old}:]
%\begin{align*}
%\left\llbracket
%\begin{tikzpicture}
%	\begin{pgfonlayer}{nodelayer}
%		\node [style=none] (0) at (4.25, 0.5) {};
%		\node [style=none] (1) at (3.5, -0) {};
%		\node [style=X] (2) at (3.5, 1) {};
%		\node [style=none] (3) at (5, 0.5) {};
%		\node [style=none] (4) at (3, -0) {};
%		\node [style=andin] (5) at (4.25, 0.5) {};
%	\end{pgfonlayer}
%	\begin{pgfonlayer}{edgelayer}
%		\draw [in=0, out=135, looseness=1.00] (0.center) to (2);
%		\draw [in=-131, out=0, looseness=1.00] (1.center) to (0.center);
%		\draw (3.center) to (0.center);
%		\draw (1.center) to (4.center);
%	\end{pgfonlayer}
%\end{tikzpicture}
%\right\rrbracket_{\ZXA}
%&=
%\begin{tikzpicture}
%	\begin{pgfonlayer}{nodelayer}
%		\node [style=dot] (0) at (4, -1) {};
%		\node [style=dot] (1) at (4, -1.5) {};
%		\node [style=oplus] (2) at (4, -2) {};
%		\node [style=Z] (3) at (4.5, -1.5) {};
%		\node [style=Z] (4) at (4.5, -1) {};
%		\node [style=zeroin] (5) at (3.5, -2) {};
%		\node [style=none] (6) at (4.75, -2) {};
%		\node [style=none] (7) at (3.25, -1.5) {};
%		\node [style=zeroin] (8) at (3.5, -1) {};
%	\end{pgfonlayer}
%	\begin{pgfonlayer}{edgelayer}
%		\draw (4) to (0);
%		\draw (0) to (8);
%		\draw (7.center) to (3);
%		\draw (0) to (2);
%		\draw (5) to (6.center);
%	\end{pgfonlayer}
%\end{tikzpicture}
%\eq{\ref{TOF.2}}
%\begin{tikzpicture}
%	\begin{pgfonlayer}{nodelayer}
%		\node [style=Z] (0) at (4.5, -1.5) {};
%		\node [style=Z] (1) at (4.5, -1) {};
%		\node [style=zeroin] (2) at (3.5, -2) {};
%		\node [style=none] (3) at (4.75, -2) {};
%		\node [style=none] (4) at (3.25, -1.5) {};
%		\node [style=zeroin] (5) at (3.5, -1) {};
%	\end{pgfonlayer}
%	\begin{pgfonlayer}{edgelayer}
%		\draw (4.center) to (0);
%		\draw (2) to (3.center);
%		\draw (1) to (5);
%	\end{pgfonlayer}
%\end{tikzpicture}
%\eq{Lem. \ref{cor:copy}}
%\begin{tikzpicture}
%	\begin{pgfonlayer}{nodelayer}
%		\node [style=Z] (0) at (4.5, -1.5) {};
%		\node [style=zeroin] (1) at (5.25, -1.5) {};
%		\node [style=none] (2) at (6, -1.5) {};
%		\node [style=none] (3) at (3.75, -1.5) {};
%	\end{pgfonlayer}
%	\begin{pgfonlayer}{edgelayer}
%		\draw (3.center) to (0);
%		\draw (1) to (2.center);
%	\end{pgfonlayer}
%\end{tikzpicture}
%=
%\left\llbracket
%\begin{tikzpicture}
%	\begin{pgfonlayer}{nodelayer}
%		\node [style=X] (0) at (4.75, -0) {};
%		\node [style=none] (1) at (5.5, -0) {};
%		\node [style=none] (2) at (3.25, -0) {};
%		\node [style=Z] (3) at (4, -0) {};
%	\end{pgfonlayer}
%	\begin{pgfonlayer}{edgelayer}
%		\draw (1.center) to (0);
%		\draw (3) to (2.center);
%	\end{pgfonlayer}
%\end{tikzpicture}
%\right\rrbracket_{\ZXA}
%\end{align*}



\item[\ref{ZXA.8}:]
\begin{align*}
\left\llbracket
\begin{tikzpicture}
	\begin{pgfonlayer}{nodelayer}
		\node [style=X] (0) at (-1, 3) {};
		\node [style=Z] (1) at (-1, 2.25) {};
		\node [style=none] (2) at (-1, 3.5) {};
		\node [style=none] (3) at (-1, 1.75) {};
	\end{pgfonlayer}
	\begin{pgfonlayer}{edgelayer}
		\draw (2.center) to (0);
		\draw [in=120, out=-120, looseness=1.25] (0) to (1);
		\draw [in=-60, out=60, looseness=1.25] (1) to (0);
		\draw (1) to (3.center);
	\end{pgfonlayer}
\end{tikzpicture}
\right\rrbracket_{\ZXA}
=
\begin{tikzpicture}
	\begin{pgfonlayer}{nodelayer}
		\node [style=none] (0) at (-1, 3) {};
		\node [style=dot] (1) at (-1, 1.25) {};
		\node [style=dot] (2) at (-0.5, 2.25) {};
		\node [style=oplus] (3) at (-1, 2.25) {};
		\node [style=oplus] (4) at (-0.5, 1.25) {};
		\node [style=Z] (5) at (-0.5, 2.75) {};
		\node [style=zeroin] (6) at (-0.5, 0.75) {};
		\node [style=none] (7) at (-1, 0.5) {};
	\end{pgfonlayer}
	\begin{pgfonlayer}{edgelayer}
		\draw (5) to (2);
		\draw (2) to (3);
		\draw (3) to (0.center);
		\draw (3) to (1);
		\draw (1) to (4);
		\draw (4) to (6);
		\draw (4) to (2);
		\draw (1) to (7.center);
	\end{pgfonlayer}
\end{tikzpicture}
\eq{Lem. \ref{lemma:whiteunit}}
\begin{tikzpicture}
	\begin{pgfonlayer}{nodelayer}
		\node [style=dot] (0) at (-1, 1.25) {};
		\node [style=dot] (1) at (-0.5, 1.75) {};
		\node [style=oplus] (2) at (-1, 1.75) {};
		\node [style=oplus] (3) at (-0.5, 1.25) {};
		\node [style=zeroin] (4) at (-0.5, 0.75) {};
		\node [style=none] (5) at (-1, 0.5) {};
		\node [style=none] (6) at (-1, 3) {};
		\node [style=Z] (7) at (-0.5, 2.75) {};
		\node [style=dot] (8) at (-1, 2.25) {};
		\node [style=oplus] (9) at (-0.5, 2.25) {};
	\end{pgfonlayer}
	\begin{pgfonlayer}{edgelayer}
		\draw (1) to (2);
		\draw (2) to (0);
		\draw (0) to (3);
		\draw (3) to (4);
		\draw (3) to (1);
		\draw (0) to (5.center);
		\draw (8) to (9);
		\draw (7) to (9);
		\draw (9) to (1);
		\draw (2) to (8);
		\draw (8) to (6.center);
	\end{pgfonlayer}
\end{tikzpicture}
\eq{\ref{TOF.14}}
\begin{tikzpicture}
	\begin{pgfonlayer}{nodelayer}
		\node [style=zeroin] (0) at (-0.5, 0.75) {};
		\node [style=none] (1) at (-1, 0.5) {};
		\node [style=none] (2) at (-1, 2.5) {};
		\node [style=Z] (3) at (-0.5, 2.25) {};
		\node [style=none] (4) at (-1, 0.75) {};
		\node [style=none] (5) at (-1, 2.25) {};
	\end{pgfonlayer}
	\begin{pgfonlayer}{edgelayer}
		\draw [in=90, out=-90] (3) to (4.center);
		\draw [in=-90, out=90] (0) to (5.center);
		\draw (5.center) to (2.center);
		\draw (4.center) to (1.center);
	\end{pgfonlayer}
\end{tikzpicture}
=
\begin{tikzpicture}
	\begin{pgfonlayer}{nodelayer}
		\node [style=zeroin] (0) at (-1, 2) {};
		\node [style=Z] (1) at (-1, 1.25) {};
		\node [style=none] (2) at (-1, 0.5) {};
		\node [style=none] (3) at (-1, 2.75) {};
	\end{pgfonlayer}
	\begin{pgfonlayer}{edgelayer}
		\draw (1) to (2.center);
		\draw (0) to (3.center);
	\end{pgfonlayer}
\end{tikzpicture}
=
\left\llbracket
\begin{tikzpicture}
	\begin{pgfonlayer}{nodelayer}
		\node [style=X] (0) at (-1, 3) {};
		\node [style=Z] (1) at (-1, 2.25) {};
		\node [style=none] (2) at (-1, 3.5) {};
		\node [style=none] (3) at (-1, 1.75) {};
	\end{pgfonlayer}
	\begin{pgfonlayer}{edgelayer}
		\draw (2.center) to (0);
		\draw (1) to (3.center);
	\end{pgfonlayer}
\end{tikzpicture}
\right\rrbracket_{\ZXA}
\end{align*}


\item[\ref{ZXA.9}:]
\begin{align*}
\left\llbracket
\begin{tikzpicture}
	\begin{pgfonlayer}{nodelayer}
		\node [style=andin] (1) at (0, 3) {};
		\node [style=andin] (2) at (0.5, 4) {};
		\node [style=none] (3) at (0.5, 4.75) {};
		\node [style=none] (4) at (0.75, 3) {};
		\node [style=none] (5) at (-0.5, 2) {};
		\node [style=none] (6) at (0.25, 2) {};
		\node [style=none] (7) at (0.75, 2) {};
	\end{pgfonlayer}
	\begin{pgfonlayer}{edgelayer}
		\draw [style=simple] (3.center) to (2.center);
		\draw [style=simple, in=90, out=-117] (2.center) to (1.center);
		\draw [style=simple, in=90, out=-117] (1.center) to (5.center);
		\draw [style=simple, in=90, out=-76] (1.center) to (6.center);
		\draw [style=simple] (7.center) to (4.center);
		\draw [style=simple, in=-63, out=90] (4.center) to (2.center);
	\end{pgfonlayer}
\end{tikzpicture}
\right\rrbracket_{\ZXA}
&=
\begin{tikzpicture}
	\begin{pgfonlayer}{nodelayer}
		\node [style=dot] (2) at (0, 1) {};
		\node [style=dot] (3) at (0.5, 1) {};
		\node [style=oplus] (4) at (1, 1) {};
		\node [style=Z] (5) at (0, 1.75) {};
		\node [style=Z] (6) at (0.5, 1.75) {};
		\node [style=none] (7) at (1, 2) {};
		\node [style=zeroin] (8) at (1, 0.25) {};
		\node [style=none] (9) at (0.5, -0.5) {};
		\node [style=Z] (10) at (-0.5, 1.25) {};
		\node [style=none] (11) at (-1, -0.5) {};
		\node [style=dot] (12) at (-0.5, 0.5) {};
		\node [style=dot] (13) at (-1, 0.5) {};
		\node [style=oplus] (14) at (0, 0.5) {};
		\node [style=zeroin] (15) at (0, -0.25) {};
		\node [style=Z] (16) at (-1, 1.25) {};
		\node [style=none] (17) at (-0.5, -0.5) {};
	\end{pgfonlayer}
	\begin{pgfonlayer}{edgelayer}
		\draw [style=simple] (7.center) to (4);
		\draw [style=simple] (4) to (8);
		\draw [style=simple] (4) to (3);
		\draw [style=simple] (3) to (2);
		\draw [style=simple] (5) to (2);
		\draw [style=simple] (9.center) to (3);
		\draw [style=simple] (3) to (6);
		\draw [style=simple] (14) to (15);
		\draw [style=simple] (14) to (12);
		\draw [style=simple] (12) to (13);
		\draw [style=simple] (16) to (13);
		\draw [style=simple] (13) to (11.center);
		\draw [style=simple] (17.center) to (12);
		\draw [style=simple] (12) to (10);
		\draw [style=simple] (14) to (2);
	\end{pgfonlayer}
\end{tikzpicture}
\eq{Lem. \ref{lemma:Iwama}}
\begin{tikzpicture}
	\begin{pgfonlayer}{nodelayer}
		\node [style=dot] (3) at (0, 1.5) {};
		\node [style=dot] (4) at (0.5, 1.5) {};
		\node [style=oplus] (5) at (1, 1.5) {};
		\node [style=none] (6) at (1, 3) {};
		\node [style=zeroin] (7) at (1, 0.25) {};
		\node [style=none] (8) at (0.5, 0) {};
		\node [style=none] (9) at (-1, 0) {};
		\node [style=zeroin] (10) at (0, 0.25) {};
		\node [style=none] (11) at (-0.5, 0) {};
		\node [style=Z] (12) at (-0.5, 2.75) {};
		\node [style=Z] (13) at (-1, 2.75) {};
		\node [style=Z] (14) at (0, 2.75) {};
		\node [style=Z] (15) at (0.5, 2.75) {};
		\node [style=dot] (16) at (-1, 2) {};
		\node [style=dot] (17) at (-0.5, 2) {};
		\node [style=oplus] (18) at (0, 2) {};
		\node [style=dot] (19) at (-1, 0.75) {};
		\node [style=dot] (20) at (-0.5, 0.75) {};
		\node [style=oplus] (21) at (1, 0.75) {};
		\node [style=dot] (22) at (0.5, 0.75) {};
	\end{pgfonlayer}
	\begin{pgfonlayer}{edgelayer}
		\draw [style=simple] (6.center) to (5);
		\draw [style=simple] (5) to (4);
		\draw [style=simple] (4) to (3);
		\draw [style=simple] (18) to (17);
		\draw [style=simple] (17) to (16);
		\draw [style=simple] (21) to (20);
		\draw [style=simple] (20) to (19);
		\draw [style=simple] (5) to (21);
		\draw [style=simple] (8.center) to (22);
		\draw [style=simple] (21) to (7);
		\draw [style=simple] (4) to (22);
		\draw [style=simple] (15) to (4);
		\draw [style=simple] (14) to (18);
		\draw [style=simple] (18) to (3);
		\draw [style=simple] (10) to (3);
		\draw [style=simple] (12) to (17);
		\draw [style=simple] (17) to (20);
		\draw [style=simple] (20) to (11.center);
		\draw [style=simple] (9.center) to (19);
		\draw [style=simple] (19) to (16);
		\draw [style=simple] (13) to (16);
	\end{pgfonlayer}
\end{tikzpicture}
\eq{Lem. \ref{cor:copy}}
\begin{tikzpicture}
	\begin{pgfonlayer}{nodelayer}
		\node [style=dot] (4) at (0, 1.5) {};
		\node [style=dot] (5) at (0.5, 1.5) {};
		\node [style=oplus] (6) at (1, 1.5) {};
		\node [style=none] (7) at (1, 2.5) {};
		\node [style=zeroin] (8) at (1, 0.25) {};
		\node [style=none] (9) at (0.5, 0) {};
		\node [style=none] (10) at (-1, 0) {};
		\node [style=zeroin] (11) at (0, 0.25) {};
		\node [style=none] (12) at (-0.5, 0) {};
		\node [style=Z] (13) at (-0.5, 2.25) {};
		\node [style=Z] (14) at (-1, 2.25) {};
		\node [style=Z] (15) at (0, 2.25) {};
		\node [style=Z] (16) at (0.5, 2.25) {};
		\node [style=dot] (17) at (-1, 0.75) {};
		\node [style=dot] (18) at (-0.5, 0.75) {};
		\node [style=oplus] (19) at (1, 0.75) {};
		\node [style=dot] (20) at (0.5, 0.75) {};
	\end{pgfonlayer}
	\begin{pgfonlayer}{edgelayer}
		\draw [style=simple] (7.center) to (6);
		\draw [style=simple] (6) to (5);
		\draw [style=simple] (5) to (4);
		\draw [style=simple] (19) to (18);
		\draw [style=simple] (18) to (17);
		\draw [style=simple] (6) to (19);
		\draw [style=simple] (9.center) to (20);
		\draw [style=simple] (19) to (8);
		\draw [style=simple] (5) to (20);
		\draw [style=simple] (16) to (5);
		\draw [style=simple] (11) to (4);
		\draw [style=simple] (18) to (12.center);
		\draw [style=simple] (10.center) to (17);
		\draw [style=simple] (15) to (4);
		\draw [style=simple] (18) to (13);
		\draw [style=simple] (14) to (17);
	\end{pgfonlayer}
\end{tikzpicture}
\eq{\ref{TOF.2}}
\begin{tikzpicture}
	\begin{pgfonlayer}{nodelayer}
		\node [style=none] (5) at (1, 2.25) {};
		\node [style=zeroin] (6) at (1, 0.25) {};
		\node [style=none] (7) at (0.5, 0) {};
		\node [style=none] (8) at (-1, 0) {};
		\node [style=zeroin] (9) at (0, 1.25) {};
		\node [style=none] (10) at (-0.5, 0) {};
		\node [style=Z] (11) at (-0.5, 2.25) {};
		\node [style=Z] (12) at (-1, 2.25) {};
		\node [style=Z] (13) at (0, 2.25) {};
		\node [style=Z] (14) at (0.5, 2.25) {};
		\node [style=dot] (15) at (-1, 0.75) {};
		\node [style=dot] (16) at (-0.5, 0.75) {};
		\node [style=oplus] (17) at (1, 0.75) {};
		\node [style=dot] (18) at (0.5, 0.75) {};
	\end{pgfonlayer}
	\begin{pgfonlayer}{edgelayer}
		\draw [style=simple] (17) to (16);
		\draw [style=simple] (16) to (15);
		\draw [style=simple] (7.center) to (18);
		\draw [style=simple] (17) to (6);
		\draw [style=simple] (16) to (10.center);
		\draw [style=simple] (8.center) to (15);
		\draw [style=simple] (16) to (11);
		\draw [style=simple] (12) to (15);
		\draw [style=simple] (5.center) to (17);
		\draw [style=simple] (9) to (13);
		\draw [style=simple] (14) to (18);
	\end{pgfonlayer}
\end{tikzpicture}\\
&\eq{\ref{TOF.2}}
\begin{tikzpicture}
	\begin{pgfonlayer}{nodelayer}
		\node [style=Z] (6) at (0.5, 4) {};
		\node [style=zeroin] (7) at (1, 1.25) {};
		\node [style=Z] (8) at (0, 4) {};
		\node [style=none] (9) at (-0.5, 1) {};
		\node [style=dot] (10) at (0.5, 3.25) {};
		\node [style=none] (11) at (1, 4.25) {};
		\node [style=Z] (12) at (-1, 4) {};
		\node [style=Z] (13) at (-0.5, 4) {};
		\node [style=dot] (14) at (-1, 3.25) {};
		\node [style=none] (15) at (0, 1) {};
		\node [style=oplus] (16) at (1, 3.25) {};
		\node [style=zeroin] (17) at (0.5, 1.25) {};
		\node [style=none] (18) at (-1, 1) {};
		\node [style=dot] (19) at (-1, 2.5) {};
		\node [style=dot] (20) at (-0.5, 2.5) {};
		\node [style=dot] (21) at (0, 2.5) {};
		\node [style=oplus] (22) at (1, 2.5) {};
	\end{pgfonlayer}
	\begin{pgfonlayer}{edgelayer}
		\draw [style=simple] (11.center) to (16);
		\draw [style=simple] (16) to (7);
		\draw [style=simple] (16) to (10);
		\draw [style=simple] (10) to (14);
		\draw [style=simple] (12) to (14);
		\draw [style=simple] (10) to (6);
		\draw (14) to (18.center);
		\draw (22) to (21);
		\draw (21) to (20);
		\draw (19) to (20);
		\draw (21) to (15.center);
		\draw (9.center) to (20);
		\draw (10) to (17);
		\draw (6) to (10);
		\draw (8) to (21);
		\draw (20) to (13);
	\end{pgfonlayer}
\end{tikzpicture}
\eq{Lem. \ref{lemma:Iwama}}
\begin{tikzpicture}
	\begin{pgfonlayer}{nodelayer}
		\node [style=Z] (7) at (0.5, 5.5) {};
		\node [style=zeroin] (8) at (1, 1.25) {};
		\node [style=Z] (9) at (0, 5.5) {};
		\node [style=none] (10) at (-0.5, 1) {};
		\node [style=dot] (11) at (0.5, 3.25) {};
		\node [style=none] (12) at (1, 4.25) {};
		\node [style=Z] (13) at (-1, 5.5) {};
		\node [style=Z] (14) at (-0.5, 5.5) {};
		\node [style=dot] (15) at (-1, 3.25) {};
		\node [style=none] (16) at (0, 1) {};
		\node [style=oplus] (17) at (1, 3.25) {};
		\node [style=zeroin] (18) at (0.5, 1.25) {};
		\node [style=none] (19) at (-1, 1) {};
		\node [style=dot] (20) at (0, 4.75) {};
		\node [style=dot] (21) at (-0.5, 4.75) {};
		\node [style=dot] (22) at (-1, 2.5) {};
		\node [style=dot] (23) at (-0.5, 2.5) {};
		\node [style=dot] (24) at (0, 2.5) {};
		\node [style=oplus] (25) at (0.5, 4.75) {};
		\node [style=oplus] (26) at (1, 2.5) {};
	\end{pgfonlayer}
	\begin{pgfonlayer}{edgelayer}
		\draw [style=simple] (12.center) to (17);
		\draw [style=simple] (17) to (8);
		\draw [style=simple] (17) to (11);
		\draw [style=simple] (11) to (15);
		\draw [style=simple] (13) to (15);
		\draw [style=simple] (11) to (7);
		\draw (15) to (19.center);
		\draw [style=simple] (20) to (21);
		\draw (26) to (24);
		\draw (24) to (23);
		\draw (22) to (23);
		\draw (23) to (21);
		\draw (21) to (14);
		\draw (9) to (20);
		\draw (20) to (24);
		\draw (24) to (16.center);
		\draw (10.center) to (23);
		\draw (25) to (20);
		\draw (11) to (18);
	\end{pgfonlayer}
\end{tikzpicture}
\eq{Lem. \ref{cor:copy}}
\begin{tikzpicture}
	\begin{pgfonlayer}{nodelayer}
		\node [style=Z] (7) at (4.175, 5.5) {};
		\node [style=zeroin] (8) at (4.65, 4) {};
		\node [style=dot] (9) at (3.225, 3.5) {};
		\node [style=Z] (10) at (3.7, 4.25) {};
		\node [style=dot] (11) at (3.7, 3.5) {};
		\node [style=none] (12) at (3.225, 2.5) {};
		\node [style=dot] (13) at (4.175, 4.75) {};
		\node [style=none] (14) at (4.65, 5.75) {};
		\node [style=Z] (15) at (2.75, 5.5) {};
		\node [style=Z] (16) at (3.225, 4.25) {};
		\node [style=dot] (17) at (2.75, 4.75) {};
		\node [style=none] (18) at (3.7, 2.5) {};
		\node [style=oplus] (19) at (4.65, 4.75) {};
		\node [style=oplus] (20) at (4.175, 3.5) {};
		\node [style=zeroin] (21) at (4.175, 2.75) {};
		\node [style=none] (22) at (2.75, 2.5) {};
	\end{pgfonlayer}
	\begin{pgfonlayer}{edgelayer}
		\draw [style=simple] (14.center) to (19);
		\draw [style=simple] (19) to (8);
		\draw [style=simple] (19) to (13);
		\draw [style=simple] (13) to (17);
		\draw [style=simple] (15) to (17);
		\draw [style=simple] (13) to (7);
		\draw [style=simple] (11) to (9);
		\draw [style=simple] (16) to (9);
		\draw [style=simple] (9) to (12.center);
		\draw [style=simple] (18.center) to (11);
		\draw [style=simple] (11) to (10);
		\draw [style=simple] (20) to (21);
		\draw (17) to (22.center);
		\draw (20) to (13);
		\draw (11) to (20);
	\end{pgfonlayer}
\end{tikzpicture}
=
\left\llbracket
\begin{tikzpicture}
	\begin{pgfonlayer}{nodelayer}
		\node [style=andin] (9) at (0.25, 3) {};
		\node [style=andin] (10) at (-0.25, 4) {};
		\node [style=none] (11) at (-0.25, 4.75) {};
		\node [style=none] (12) at (-0.5, 3) {};
		\node [style=none] (13) at (0.75, 2) {};
		\node [style=none] (14) at (0, 2) {};
		\node [style=none] (15) at (-0.5, 2) {};
	\end{pgfonlayer}
	\begin{pgfonlayer}{edgelayer}
		\draw [style=simple] (11.center) to (10.center);
		\draw [style=simple, in=90, out=-63] (10.center) to (9.center);
		\draw [style=simple, in=90, out=-63] (9.center) to (13.center);
		\draw [style=simple, in=90, out=-104] (9.center) to (14.center);
		\draw [style=simple] (15.center) to (12.center);
		\draw [style=simple, in=-117, out=90] (12.center) to (10.center);
	\end{pgfonlayer}
\end{tikzpicture}
\right\rrbracket_{\ZXA}
\end{align*}

\item[\ref{ZXA.10}:]
\begin{align*}
\left\llbracket
\begin{tikzpicture}
	\begin{pgfonlayer}{nodelayer}
		\node [style=andin] (100) at (-1, 5.5) {};
		\node [style=none] (10) at (-1, 5.5) {};
		\node [style=none] (11) at (-1, 6) {};
		\node [style=none] (12) at (-0.75, 4.75) {};
		\node [style=X] (13) at (-1.25, 4.75) {$1$};
	\end{pgfonlayer}
	\begin{pgfonlayer}{edgelayer}
		\draw (10) to (11.center);
		\draw [in=90, out=-108] (10) to (13);
		\draw [in=-72, out=90] (12.center) to (10);
	\end{pgfonlayer}
\end{tikzpicture}
\right\rrbracket_{\ZXA}
=
\begin{tikzpicture}
	\begin{pgfonlayer}{nodelayer}
		\node [style=Z] (11) at (0, 7) {};
		\node [style=Z] (12) at (-0.5, 7) {};
		\node [style=none] (13) at (0, 4.75) {};
		\node [style=zeroin] (14) at (0.5, 5.5) {};
		\node [style=oplus] (15) at (0.5, 6.25) {};
		\node [style=dot] (16) at (0, 6.25) {};
		\node [style=dot] (17) at (-0.5, 6.25) {};
		\node [style=none] (18) at (0.5, 7.25) {};
		\node [style=zeroin] (19) at (-0.5, 5) {};
		\node [style=oplus] (20) at (-0.5, 5.75) {};
	\end{pgfonlayer}
	\begin{pgfonlayer}{edgelayer}
		\draw [style=simple] (16) to (17);
		\draw [style=simple] (16) to (11);
		\draw [style=simple] (15) to (14);
		\draw [style=simple] (13.center) to (16);
		\draw [style=simple] (12) to (17);
		\draw (16) to (15);
		\draw (20) to (19);
		\draw (20) to (17);
		\draw (18.center) to (15);
	\end{pgfonlayer}
\end{tikzpicture}
=
\begin{tikzpicture}
	\begin{pgfonlayer}{nodelayer}
		\node [style=Z] (12) at (0, 7) {};
		\node [style=Z] (13) at (-0.5, 7) {};
		\node [style=none] (14) at (0, 4.75) {};
		\node [style=zeroin] (15) at (0.5, 5.5) {};
		\node [style=oplus] (16) at (0.5, 6.25) {};
		\node [style=dot] (17) at (0, 6.25) {};
		\node [style=dot] (18) at (-0.5, 6.25) {};
		\node [style=none] (19) at (0.5, 7.25) {};
		\node [style=onein] (20) at (-0.5, 5.5) {};
	\end{pgfonlayer}
	\begin{pgfonlayer}{edgelayer}
		\draw [style=simple] (17) to (18);
		\draw [style=simple] (17) to (12);
		\draw [style=simple] (16) to (15);
		\draw [style=simple] (14.center) to (17);
		\draw [style=simple] (13) to (18);
		\draw (17) to (16);
		\draw (19.center) to (16);
		\draw (18) to (20);
	\end{pgfonlayer}
\end{tikzpicture}
\eq{\ref{TOF.1}}
\begin{tikzpicture}
	\begin{pgfonlayer}{nodelayer}
		\node [style=Z] (13) at (0, 7) {};
		\node [style=Z] (14) at (-0.5, 7) {};
		\node [style=none] (15) at (0, 4.75) {};
		\node [style=zeroin] (16) at (0.5, 5.5) {};
		\node [style=none] (17) at (0.5, 7.25) {};
		\node [style=onein] (18) at (-0.5, 5.5) {};
		\node [style=oplus] (19) at (0.5, 6.25) {};
		\node [style=dot] (20) at (0, 6.25) {};
	\end{pgfonlayer}
	\begin{pgfonlayer}{edgelayer}
		\draw [style=simple] (19) to (16);
		\draw (17.center) to (19);
		\draw [style=simple] (15.center) to (20);
		\draw (20) to (19);
		\draw [style=simple] (20) to (13);
		\draw (14) to (18);
	\end{pgfonlayer}
\end{tikzpicture}
\eq{Lem. \ref{cor:copy}}
\begin{tikzpicture}
	\begin{pgfonlayer}{nodelayer}
		\node [style=Z] (14) at (0, 8) {};
		\node [style=none] (15) at (0, 5.75) {};
		\node [style=zeroin] (16) at (0.5, 6.5) {};
		\node [style=none] (17) at (0.5, 8.25) {};
		\node [style=oplus] (18) at (0.5, 7.25) {};
		\node [style=dot] (19) at (0, 7.25) {};
	\end{pgfonlayer}
	\begin{pgfonlayer}{edgelayer}
		\draw [style=simple] (18) to (16);
		\draw (17.center) to (18);
		\draw [style=simple] (15.center) to (19);
		\draw (19) to (18);
		\draw [style=simple] (19) to (14);
	\end{pgfonlayer}
\end{tikzpicture}
\eq{Lem. \ref{lemma:whiteunit}}
\begin{tikzpicture}
	\begin{pgfonlayer}{nodelayer}
		\node [style=none] (15) at (0, 5.75) {};
		\node [style=none] (16) at (0, 6.75) {};
	\end{pgfonlayer}
	\begin{pgfonlayer}{edgelayer}
		\draw (16.center) to (15.center);
	\end{pgfonlayer}
\end{tikzpicture}=
\left\llbracket
\begin{tikzpicture}
	\begin{pgfonlayer}{nodelayer}
		\node [style=none] (16) at (0, 5.75) {};
		\node [style=none] (17) at (0, 6.75) {};
	\end{pgfonlayer}
	\begin{pgfonlayer}{edgelayer}
		\draw (17.center) to (16.center);
	\end{pgfonlayer}
\end{tikzpicture}
\right\rrbracket_{\ZXA}
\end{align*}

\item[\ref{ZXA.11}:]

\begin{align*}
\left\llbracket
\begin{tikzpicture}
	\begin{pgfonlayer}{nodelayer}
		\node [style=andin] (170) at (0, 6.75) {};
		\node [style=none] (17) at (0, 6.75) {};
		\node [style=none] (18) at (-0.25, 6.25) {};
		\node [style=none] (19) at (0.25, 6.25) {};
		\node [style=none] (20) at (0, 7.25) {};
		\node [style=none] (21) at (-0.25, 5.75) {};
		\node [style=none] (22) at (0.25, 5.75) {};
	\end{pgfonlayer}
	\begin{pgfonlayer}{edgelayer}
		\draw [in=-63, out=90] (19.center) to (17);
		\draw [in=90, out=-117, looseness=1.25] (17) to (18.center);
		\draw (20.center) to (17);
		\draw [in=-90, out=90, looseness=1.25] (22.center) to (18.center);
		\draw [in=90, out=-90, looseness=1.25] (19.center) to (21.center);
	\end{pgfonlayer}
\end{tikzpicture}
\right\rrbracket_{\ZXA}
=
\begin{tikzpicture}
	\begin{pgfonlayer}{nodelayer}
		\node [style=none] (18) at (0.5, 5.75) {};
		\node [style=dot] (19) at (0, 6.75) {};
		\node [style=dot] (20) at (0.5, 6.75) {};
		\node [style=oplus] (21) at (1, 6.75) {};
		\node [style=Z] (22) at (0, 7.5) {};
		\node [style=Z] (23) at (0.5, 7.5) {};
		\node [style=none] (24) at (1, 7.75) {};
		\node [style=zeroin] (25) at (1, 6) {};
		\node [style=none] (26) at (0, 5.75) {};
	\end{pgfonlayer}
	\begin{pgfonlayer}{edgelayer}
		\draw (25) to (21);
		\draw (20) to (23);
		\draw (24.center) to (21);
		\draw (21) to (20);
		\draw (20) to (19);
		\draw (19) to (22);
		\draw [in=90, out=-90] (19) to (18.center);
		\draw [in=90, out=-90] (20) to (26.center);
	\end{pgfonlayer}
\end{tikzpicture}
\eq{\ref{TOF.15}}
\begin{tikzpicture}
	\begin{pgfonlayer}{nodelayer}
		\node [style=none] (19) at (0.5, 6.75) {};
		\node [style=dot] (20) at (0, 7.75) {};
		\node [style=dot] (21) at (0.5, 7.75) {};
		\node [style=oplus] (22) at (1, 7.75) {};
		\node [style=Z] (23) at (0.5, 8.75) {};
		\node [style=Z] (24) at (0, 8.75) {};
		\node [style=none] (25) at (1, 8.75) {};
		\node [style=zeroin] (26) at (1, 7) {};
		\node [style=none] (27) at (0, 6.75) {};
		\node [style=none] (28) at (0.5, 5.75) {};
		\node [style=none] (29) at (0, 5.75) {};
	\end{pgfonlayer}
	\begin{pgfonlayer}{edgelayer}
		\draw (26) to (22);
		\draw [in=-90, out=90] (21) to (24);
		\draw (25.center) to (22);
		\draw (22) to (21);
		\draw (21) to (20);
		\draw [in=-90, out=90] (20) to (23);
		\draw [in=90, out=-90] (20) to (19.center);
		\draw [in=90, out=-90] (21) to (27.center);
		\draw [in=90, out=-90] (19.center) to (29.center);
		\draw [in=-90, out=90] (28.center) to (27.center);
	\end{pgfonlayer}
\end{tikzpicture}
=
\begin{tikzpicture}
	\begin{pgfonlayer}{nodelayer}
		\node [style=none] (20) at (0, 5.75) {};
		\node [style=dot] (21) at (0, 6.75) {};
		\node [style=dot] (22) at (0.5, 6.75) {};
		\node [style=oplus] (23) at (1, 6.75) {};
		\node [style=Z] (24) at (0, 7.5) {};
		\node [style=Z] (25) at (0.5, 7.5) {};
		\node [style=none] (26) at (1, 7.75) {};
		\node [style=zeroin] (27) at (1, 6) {};
		\node [style=none] (28) at (0.5, 5.75) {};
	\end{pgfonlayer}
	\begin{pgfonlayer}{edgelayer}
		\draw (27) to (23);
		\draw (22) to (25);
		\draw (26.center) to (23);
		\draw (23) to (22);
		\draw (22) to (21);
		\draw (21) to (24);
		\draw (21) to (20.center);
		\draw (22) to (28.center);
	\end{pgfonlayer}
\end{tikzpicture}
=
\left\llbracket
\begin{tikzpicture}
	\begin{pgfonlayer}{nodelayer}
		\node [style=andin] (210) at (0, 6.25) {};
		\node [style=none] (21) at (0, 6.25) {};
		\node [style=none] (22) at (-0.25, 5.75) {};
		\node [style=none] (23) at (0.25, 5.75) {};
		\node [style=none] (24) at (0, 6.75) {};
	\end{pgfonlayer}
	\begin{pgfonlayer}{edgelayer}
		\draw [in=-63, out=90] (23.center) to (21);
		\draw [in=90, out=-117, looseness=1.25] (21) to (22.center);
		\draw (24.center) to (21);
	\end{pgfonlayer}
\end{tikzpicture}
\right\rrbracket_{\ZXA}
\end{align*}

\item[\ref{ZXA.12}:]

\begin{align*}
\left\llbracket
\begin{tikzpicture}
	\begin{pgfonlayer}{nodelayer}
		\node [style=Z] (22) at (-1, 6.25) {};
		\node [style=Z] (23) at (-0.25, 6.25) {};
		\node [style=andin] (240) at (-0.25, 7.25) {};
		\node [style=andin] (250) at (-1, 7.25) {};
		\node [style=none] (24) at (-0.25, 7.25) {};
		\node [style=none] (25) at (-1, 7.25) {};
		\node [style=none] (26) at (-1, 7.75) {};
		\node [style=none] (27) at (-0.25, 7.75) {};
		\node [style=none] (28) at (-1, 5.75) {};
		\node [style=none] (29) at (-0.25, 5.75) {};
	\end{pgfonlayer}
	\begin{pgfonlayer}{edgelayer}
		\draw (29.center) to (23);
		\draw [in=-60, out=127] (23) to (25);
		\draw [in=120, out=-120, looseness=1.25] (25) to (22);
		\draw [in=-120, out=53] (22) to (24);
		\draw (24) to (27.center);
		\draw [in=60, out=-60, looseness=1.25] (24) to (23);
		\draw (22) to (28.center);
		\draw (25) to (26.center);
	\end{pgfonlayer}
\end{tikzpicture}
\right\rrbracket_{\ZXA}
&=
\begin{tikzpicture}
	\begin{pgfonlayer}{nodelayer}
		\node [style=dot] (23) at (-0.5, 7.5) {};
		\node [style=dot] (24) at (-1, 7.5) {};
		\node [style=oplus] (25) at (-1.5, 7.5) {};
		\node [style=zeroin] (26) at (-1.5, 7) {};
		\node [style=Z] (27) at (-0.5, 8) {};
		\node [style=Z] (28) at (-1, 8) {};
		\node [style=none] (29) at (-1.5, 8.25) {};
		\node [style=dot] (30) at (0.5, 7.5) {};
		\node [style=oplus] (31) at (1, 7.5) {};
		\node [style=none] (32) at (1, 8.25) {};
		\node [style=dot] (33) at (0, 7.5) {};
		\node [style=Z] (34) at (0.5, 8) {};
		\node [style=Z] (35) at (0, 8) {};
		\node [style=zeroin] (36) at (1, 7) {};
		\node [style=fanout] (37) at (-0.75, 6.5) {};
		\node [style=fanout] (38) at (0.25, 6.5) {};
		\node [style=none] (39) at (0.25, 5.75) {};
		\node [style=none] (40) at (-0.75, 5.75) {};
	\end{pgfonlayer}
	\begin{pgfonlayer}{edgelayer}
		\draw (27) to (23);
		\draw (23) to (24);
		\draw (24) to (28);
		\draw (29.center) to (25);
		\draw (25) to (26);
		\draw (25) to (24);
		\draw (35) to (33);
		\draw (33) to (30);
		\draw (30) to (34);
		\draw (32.center) to (31);
		\draw (31) to (36);
		\draw (31) to (30);
		\draw [in=99, out=-90] (24) to (37);
		\draw [in=-90, out=63] (37) to (33);
		\draw [in=117, out=-90] (23) to (38);
		\draw [in=-90, out=81] (38) to (30);
		\draw (37) to (40.center);
		\draw (39.center) to (38);
	\end{pgfonlayer}
\end{tikzpicture}
\eq{\ref{TOF.4}}
\begin{tikzpicture}
	\begin{pgfonlayer}{nodelayer}
		\node [style=Z] (26) at (-0.5, 8.5) {};
		\node [style=Z] (27) at (-1, 8.5) {};
		\node [style=none] (28) at (-1.5, 8.75) {};
		\node [style=dot] (29) at (0.5, 8) {};
		\node [style=oplus] (30) at (1, 8) {};
		\node [style=none] (31) at (1, 8.75) {};
		\node [style=dot] (32) at (0, 8) {};
		\node [style=Z] (33) at (0.5, 8.5) {};
		\node [style=Z] (34) at (0, 8.5) {};
		\node [style=zeroin] (35) at (1, 7.5) {};
		\node [style=fanout] (36) at (-0.5, 7) {};
		\node [style=fanout] (37) at (0, 7) {};
		\node [style=none] (38) at (0, 5.75) {};
		\node [style=none] (39) at (-0.5, 5.75) {};
		\node [style=dot] (40) at (0, 6.5) {};
		\node [style=zeroin] (41) at (-1.5, 5.75) {};
		\node [style=oplus] (42) at (-1.5, 6.5) {};
		\node [style=dot] (43) at (-0.5, 6.5) {};
	\end{pgfonlayer}
	\begin{pgfonlayer}{edgelayer}
		\draw (34) to (32);
		\draw (32) to (29);
		\draw (29) to (33);
		\draw (31.center) to (30);
		\draw (30) to (35);
		\draw (30) to (29);
		\draw [in=-90, out=63] (36) to (32);
		\draw [in=-90, out=60] (37) to (29);
		\draw (36) to (39.center);
		\draw (38.center) to (37);
		\draw (40) to (43);
		\draw (42) to (41);
		\draw (42) to (43);
		\draw [in=-90, out=117] (37) to (26);
		\draw [in=120, out=-90] (27) to (36);
		\draw (42) to (28.center);
	\end{pgfonlayer}
\end{tikzpicture}
\eq{unit}
\begin{tikzpicture}
	\begin{pgfonlayer}{nodelayer}
		\node [style=none] (27) at (-1, 8.25) {};
		\node [style=dot] (28) at (0, 7.25) {};
		\node [style=oplus] (29) at (0.5, 7.25) {};
		\node [style=none] (30) at (0.5, 8.25) {};
		\node [style=dot] (31) at (-0.5, 7.25) {};
		\node [style=Z] (32) at (0, 7.75) {};
		\node [style=Z] (33) at (-0.5, 7.75) {};
		\node [style=zeroin] (34) at (0.5, 6.75) {};
		\node [style=none] (35) at (0, 5.75) {};
		\node [style=none] (36) at (-0.5, 5.75) {};
		\node [style=dot] (37) at (0, 6.75) {};
		\node [style=zeroin] (38) at (-1, 6) {};
		\node [style=oplus] (39) at (-1, 6.75) {};
		\node [style=dot] (40) at (-0.5, 6.75) {};
	\end{pgfonlayer}
	\begin{pgfonlayer}{edgelayer}
		\draw (33) to (31);
		\draw (31) to (28);
		\draw (28) to (32);
		\draw (30.center) to (29);
		\draw (29) to (34);
		\draw (29) to (28);
		\draw (37) to (40);
		\draw (39) to (38);
		\draw (39) to (40);
		\draw (39) to (27.center);
		\draw (31) to (40);
		\draw (37) to (28);
		\draw (37) to (35.center);
		\draw (36.center) to (40);
	\end{pgfonlayer}
\end{tikzpicture}
=
\begin{tikzpicture}
	\begin{pgfonlayer}{nodelayer}
		\node [style=none] (28) at (-1, 8.25) {};
		\node [style=dot] (29) at (-1.5, 7.25) {};
		\node [style=oplus] (30) at (-0.5, 7.25) {};
		\node [style=none] (31) at (-0.5, 8.25) {};
		\node [style=dot] (32) at (-2, 7.25) {};
		\node [style=Z] (33) at (-1.5, 7.75) {};
		\node [style=Z] (34) at (-2, 7.75) {};
		\node [style=zeroin] (35) at (-0.5, 6.75) {};
		\node [style=none] (36) at (-1.5, 5.75) {};
		\node [style=none] (37) at (-2, 5.75) {};
		\node [style=dot] (38) at (-1.5, 6.75) {};
		\node [style=zeroin] (39) at (-1, 6) {};
		\node [style=oplus] (40) at (-1, 6.75) {};
		\node [style=dot] (41) at (-2, 6.75) {};
	\end{pgfonlayer}
	\begin{pgfonlayer}{edgelayer}
		\draw (34) to (32);
		\draw (32) to (29);
		\draw (29) to (33);
		\draw (31.center) to (30);
		\draw (30) to (35);
		\draw (30) to (29);
		\draw (38) to (41);
		\draw (40) to (39);
		\draw (40) to (41);
		\draw (40) to (28.center);
		\draw (32) to (41);
		\draw (38) to (29);
		\draw (38) to (36.center);
		\draw (37.center) to (41);
	\end{pgfonlayer}
\end{tikzpicture}\\
&\eq{\ref{TOF.2}}
\begin{tikzpicture}
	\begin{pgfonlayer}{nodelayer}
		\node [style=none] (29) at (-1, 8.75) {};
		\node [style=dot] (30) at (-1.5, 7.75) {};
		\node [style=oplus] (31) at (-0.5, 7.75) {};
		\node [style=none] (32) at (-0.5, 8.75) {};
		\node [style=dot] (33) at (-2, 7.75) {};
		\node [style=Z] (34) at (-1.5, 8.25) {};
		\node [style=Z] (35) at (-2, 8.25) {};
		\node [style=zeroin] (36) at (-0.5, 6) {};
		\node [style=none] (37) at (-1.5, 5.75) {};
		\node [style=none] (38) at (-2, 5.75) {};
		\node [style=dot] (39) at (-1.5, 7.25) {};
		\node [style=zeroin] (40) at (-1, 6) {};
		\node [style=oplus] (41) at (-1, 7.25) {};
		\node [style=dot] (42) at (-2, 7.25) {};
		\node [style=dot] (43) at (-1, 6.75) {};
		\node [style=oplus] (44) at (-0.5, 6.75) {};
	\end{pgfonlayer}
	\begin{pgfonlayer}{edgelayer}
		\draw (35) to (33);
		\draw (33) to (30);
		\draw (30) to (34);
		\draw (32.center) to (31);
		\draw (31) to (36);
		\draw (31) to (30);
		\draw (39) to (42);
		\draw (41) to (40);
		\draw (41) to (42);
		\draw (41) to (29.center);
		\draw (33) to (42);
		\draw (39) to (30);
		\draw (39) to (37.center);
		\draw (38.center) to (42);
		\draw (44) to (43);
	\end{pgfonlayer}
\end{tikzpicture}
\eq{Lem. \ref{lemma:Iwama}}
\begin{tikzpicture}
	\begin{pgfonlayer}{nodelayer}
		\node [style=dot] (30) at (0.25, 6.75) {};
		\node [style=zeroin] (31) at (1.25, 6) {};
		\node [style=none] (32) at (1.75, 7.75) {};
		\node [style=oplus] (33) at (1.25, 6.75) {};
		\node [style=none] (34) at (1.25, 7.75) {};
		\node [style=Z] (35) at (0.25, 7.25) {};
		\node [style=Z] (36) at (0.75, 7.25) {};
		\node [style=zeroin] (37) at (1.75, 6) {};
		\node [style=none] (38) at (0.25, 5.75) {};
		\node [style=none] (39) at (0.75, 5.75) {};
		\node [style=dot] (40) at (0.75, 6.75) {};
		\node [style=dot] (41) at (1.25, 7.25) {};
		\node [style=oplus] (42) at (1.75, 7.25) {};
	\end{pgfonlayer}
	\begin{pgfonlayer}{edgelayer}
		\draw (40) to (30);
		\draw (33) to (31);
		\draw (33) to (30);
		\draw (33) to (34.center);
		\draw (40) to (39.center);
		\draw (38.center) to (30);
		\draw (42) to (41);
		\draw (32.center) to (42);
		\draw (42) to (37);
		\draw (36) to (40);
		\draw (30) to (35);
	\end{pgfonlayer}
\end{tikzpicture}=
\left\llbracket
\begin{tikzpicture}
	\begin{pgfonlayer}{nodelayer}
		\node [style=andin] (310) at (-1, 6.25) {};
		\node [style=none] (31) at (-1, 6.25) {};
		\node [style=none] (32) at (-1.25, 5.75) {};
		\node [style=none] (33) at (-0.75, 5.75) {};
		\node [style=Z] (34) at (-1, 7) {};
		\node [style=none] (35) at (-1.25, 7.5) {};
		\node [style=none] (36) at (-0.75, 7.5) {};
	\end{pgfonlayer}
	\begin{pgfonlayer}{edgelayer}
		\draw [in=63, out=-90] (36.center) to (34);
		\draw (34) to (31);
		\draw [in=90, out=-117] (31) to (32.center);
		\draw [in=-63, out=90] (33.center) to (31);
		\draw [in=-90, out=117] (34) to (35.center);
	\end{pgfonlayer}
\end{tikzpicture}
\right\rrbracket_{\ZXA}
\end{align*}



\item[\ref{ZXA.13}:]
\begin{align*}
\left\llbracket
\begin{tikzpicture}
	\begin{pgfonlayer}{nodelayer}
		\node [style=none] (32) at (-0.5, 6.5) {};
		\node [style=none] (33) at (0, 5.75) {};
		\node [style=none] (34) at (-1, 5.75) {};
		\node [style=Z] (35) at (-0.5, 7.25) {};
		\node [style=andin] (36) at (-0.5, 6.5) {};
	\end{pgfonlayer}
	\begin{pgfonlayer}{edgelayer}
		\draw [in=90, out=-135] (32.center) to (34.center);
		\draw [in=-41, out=90] (33.center) to (32.center);
		\draw (35) to (32.center);
	\end{pgfonlayer}
\end{tikzpicture}
\right\rrbracket_{\ZXA}
		&=
\begin{tikzpicture}
	\begin{pgfonlayer}{nodelayer}
		\node [style=none] (33) at (-0.5, 5.75) {};
		\node [style=none] (34) at (-1, 5.75) {};
		\node [style=dot] (35) at (-1, 6.5) {};
		\node [style=dot] (36) at (-0.5, 6.5) {};
		\node [style=oplus] (37) at (0, 6.5) {};
		\node [style=zeroin] (38) at (0, 6) {};
		\node [style=Z] (39) at (0, 7) {};
		\node [style=Z] (40) at (-0.5, 7) {};
		\node [style=Z] (41) at (-1, 7) {};
	\end{pgfonlayer}
	\begin{pgfonlayer}{edgelayer}
		\draw (37) to (35);
		\draw (41) to (34.center);
		\draw (33.center) to (40);
		\draw (39) to (38);
	\end{pgfonlayer}
\end{tikzpicture}
\eq{\ref{TOF.2}}
\begin{tikzpicture}
	\begin{pgfonlayer}{nodelayer}
		\node [style=none] (34) at (-0.5, 5.75) {};
		\node [style=none] (35) at (-1, 5.75) {};
		\node [style=zeroin] (36) at (0, 6) {};
		\node [style=Z] (37) at (0, 6.75) {};
		\node [style=Z] (38) at (-0.5, 6.75) {};
		\node [style=Z] (39) at (-1, 6.75) {};
	\end{pgfonlayer}
	\begin{pgfonlayer}{edgelayer}
		\draw (39) to (35.center);
		\draw (34.center) to (38);
		\draw (37) to (36);
	\end{pgfonlayer}
\end{tikzpicture}
\eq{Lem. \ref{cor:copy}}
\begin{tikzpicture}
	\begin{pgfonlayer}{nodelayer}
		\node [style=none] (35) at (-0.5, 5.75) {};
		\node [style=none] (36) at (-1, 5.75) {};
		\node [style=Z] (37) at (-0.5, 6.75) {};
		\node [style=Z] (38) at (-1, 6.75) {};
	\end{pgfonlayer}
	\begin{pgfonlayer}{edgelayer}
		\draw (38) to (36.center);
		\draw (35.center) to (37);
	\end{pgfonlayer}
\end{tikzpicture}
=
\left\llbracket
\begin{tikzpicture}
	\begin{pgfonlayer}{nodelayer}
		\node [style=none] (36) at (-0.5, 5.75) {};
		\node [style=none] (37) at (-1, 5.75) {};
		\node [style=Z] (38) at (-1, 6.5) {};
		\node [style=Z] (39) at (-0.5, 6.5) {};
	\end{pgfonlayer}
	\begin{pgfonlayer}{edgelayer}
		\draw (39) to (36.center);
		\draw (38) to (37.center);
	\end{pgfonlayer}
\end{tikzpicture}
\right\rrbracket_{\ZXA}
\end{align*}


\item[\ref{ZXA.14}:]

$$
\left\llbracket
\begin{tikzpicture}
	\begin{pgfonlayer}{nodelayer}
		\node [style=none] (37) at (-0.25, 7.25) {};
		\node [style=Z] (38) at (0, 6.5) {};
		\node [style=X] (39) at (0, 5.75) {$1$};
		\node [style=none] (40) at (0.25, 7.25) {};
	\end{pgfonlayer}
	\begin{pgfonlayer}{edgelayer}
		\draw [style=simple, in=-90, out=124] (38) to (37.center);
		\draw [style=simple, in=60, out=-90] (40.center) to (38);
		\draw [style=simple] (38) to (39);
	\end{pgfonlayer}
\end{tikzpicture}
\right\rrbracket_{\ZXA}
=
\begin{tikzpicture}
	\begin{pgfonlayer}{nodelayer}
		\node [style=dot] (38) at (1, 6.25) {};
		\node [style=oplus] (39) at (1.75, 6.25) {};
		\node [style=onein] (40) at (1, 5.75) {};
		\node [style=zeroin] (41) at (1.75, 5.75) {};
		\node [style=none] (42) at (1, 6.75) {};
		\node [style=none] (43) at (1.75, 6.75) {};
	\end{pgfonlayer}
	\begin{pgfonlayer}{edgelayer}
		\draw (43.center) to (39);
		\draw (39) to (41);
		\draw (39) to (38);
		\draw (38) to (42.center);
		\draw (38) to (40);
	\end{pgfonlayer}
\end{tikzpicture}
\eq{\ref{TOF.1}}
\begin{tikzpicture}
	\begin{pgfonlayer}{nodelayer}
		\node [style=onein] (39) at (1, 5.75) {};
		\node [style=none] (40) at (1, 6.75) {};
		\node [style=none] (41) at (1.75, 6.75) {};
		\node [style=onein] (42) at (1.75, 5.75) {};
	\end{pgfonlayer}
	\begin{pgfonlayer}{edgelayer}
		\draw (39) to (40.center);
		\draw (41.center) to (42);
	\end{pgfonlayer}
\end{tikzpicture}
=
\left\llbracket
\begin{tikzpicture}
	\begin{pgfonlayer}{nodelayer}
		\node [style=none] (40) at (-0.25, 6.25) {};
		\node [style=X] (41) at (-0.25, 5.75) {$1$};
		\node [style=none] (42) at (0.25, 6.25) {};
		\node [style=X] (43) at (0.25, 5.75) {$1$};
	\end{pgfonlayer}
	\begin{pgfonlayer}{edgelayer}
		\draw [style=simple] (43) to (42.center);
		\draw [style=simple] (41) to (40.center);
	\end{pgfonlayer}
\end{tikzpicture}
\right\rrbracket_{\ZXA}
$$
%
%$$
%\left\llbracket
%\begin{tikzpicture}
%	\begin{pgfonlayer}{nodelayer}
%		\node [style=Z] (0) at (3, -0.25) {};
%		\node [style=none] (1) at (2, -0) {};
%		\node [style=none] (2) at (2, -0.5) {};
%		\node [style=X] (3) at (3.75, -0.25) {$1$};
%		\node [style=none] (4) at (4.5, -0.25) {};
%	\end{pgfonlayer}
%	\begin{pgfonlayer}{edgelayer}
%		\draw (3) to (0);
%		\draw [in=0, out=166, looseness=1.00] (0) to (1.center);
%		\draw [in=-166, out=0, looseness=1.00] (2.center) to (0);
%		\draw (4.center) to (3);
%	\end{pgfonlayer}
%\end{tikzpicture}
%\right\rrbracket_{\ZXA}
%=
%\begin{tikzpicture}
%	\begin{pgfonlayer}{nodelayer}
%		\node [style=fanin] (0) at (2, -0) {};
%		\node [style=oplus] (1) at (2.5, -0) {};
%		\node [style=none] (2) at (3, -0) {};
%		\node [style=none] (3) at (1.25, 0.25) {};
%		\node [style=none] (4) at (1.25, -0.25) {};
%	\end{pgfonlayer}
%	\begin{pgfonlayer}{edgelayer}
%		\draw (2.center) to (1);
%		\draw (1) to (0);
%		\draw [in=0, out=162, looseness=1.00] (0) to (3.center);
%		\draw [in=-162, out=0, looseness=1.00] (4.center) to (0);
%	\end{pgfonlayer}
%\end{tikzpicture}
%\eq{nat.}
%\begin{tikzpicture}
%	\begin{pgfonlayer}{nodelayer}
%		\node [style=fanin] (0) at (2, -0) {};
%		\node [style=none] (1) at (2.5, -0) {};
%		\node [style=none] (2) at (1.25, 0.25) {};
%		\node [style=none] (3) at (1.25, -0.25) {};
%		\node [style=oplus] (4) at (1.25, 0.25) {};
%		\node [style=oplus] (5) at (1.25, -0.25) {};
%		\node [style=none] (6) at (0.5, 0.25) {};
%		\node [style=none] (7) at (0.5, -0.25) {};
%	\end{pgfonlayer}
%	\begin{pgfonlayer}{edgelayer}
%		\draw [in=0, out=162, looseness=1.00] (0) to (2.center);
%		\draw [in=-162, out=0, looseness=1.00] (3.center) to (0);
%		\draw (1.center) to (0);
%		\draw (2.center) to (6.center);
%		\draw (7.center) to (3.center);
%	\end{pgfonlayer}
%\end{tikzpicture}
%=
%\left\llbracket
%\begin{tikzpicture}
%	\begin{pgfonlayer}{nodelayer}
%		\node [style=none] (0) at (2, -0) {};
%		\node [style=none] (1) at (2, -0.5) {};
%		\node [style=X] (2) at (2.75, -0) {$1$};
%		\node [style=X] (3) at (2.75, -0.5) {$1$};
%		\node [style=Z] (4) at (3.5, -0.25) {};
%		\node [style=none] (5) at (4.25, -0.25) {};
%	\end{pgfonlayer}
%	\begin{pgfonlayer}{edgelayer}
%		\draw (3) to (1.center);
%		\draw (2) to (0.center);
%		\draw [in=162, out=0, looseness=1.00] (2) to (4);
%		\draw (4) to (5.center);
%		\draw [in=0, out=-162, looseness=1.00] (4) to (3);
%	\end{pgfonlayer}
%\end{tikzpicture}
%\right\rrbracket_{\ZXA}
%$$
%
%
%$$
%\left\llbracket
%\begin{tikzpicture}
%	\begin{pgfonlayer}{nodelayer}
%		\node [style=X] (0) at (3, -0.25) {};
%		\node [style=none] (1) at (2, -0) {};
%		\node [style=none] (2) at (2, -0.5) {};
%		\node [style=Z] (3) at (3.75, -0.25) {};
%	\end{pgfonlayer}
%	\begin{pgfonlayer}{edgelayer}
%		\draw (3) to (0);
%		\draw [in=0, out=166, looseness=1.00] (0) to (1.center);
%		\draw [in=-166, out=0, looseness=1.00] (2.center) to (0);
%	\end{pgfonlayer}
%\end{tikzpicture}
%\right\rrbracket_{\ZXA}
%=
%\begin{tikzpicture}
%	\begin{pgfonlayer}{nodelayer}
%		\node [style=dot] (0) at (1, -0) {};
%		\node [style=oplus] (1) at (1, -0.5) {};
%		\node [style=Z] (2) at (1.5, -0) {};
%		\node [style=Z] (3) at (1.5, -0.5) {};
%		\node [style=none] (4) at (0.5, -0.5) {};
%		\node [style=none] (5) at (0.5, -0) {};
%	\end{pgfonlayer}
%	\begin{pgfonlayer}{edgelayer}
%		\draw (0) to (1);
%		\draw (3) to (1);
%		\draw (1) to (4.center);
%		\draw (5.center) to (0);
%		\draw (0) to (2);
%	\end{pgfonlayer}
%\end{tikzpicture}
%\eq{Lem. \ref{lemma:whiteunit}}
%\begin{tikzpicture}
%	\begin{pgfonlayer}{nodelayer}
%		\node [style=Z] (0) at (1.5, -0) {};
%		\node [style=Z] (1) at (1.5, -0.5) {};
%		\node [style=none] (2) at (0.5, -0.5) {};
%		\node [style=none] (3) at (0.5, -0) {};
%	\end{pgfonlayer}
%	\begin{pgfonlayer}{edgelayer}
%		\draw (1) to (2.center);
%		\draw (3.center) to (0);
%	\end{pgfonlayer}
%\end{tikzpicture}
%=
%\left\llbracket
%\begin{tikzpicture}
%	\begin{pgfonlayer}{nodelayer}
%		\node [style=none] (0) at (2, -0) {};
%		\node [style=none] (1) at (2, -0.5) {};
%		\node [style=Z] (2) at (2.75, -0) {};
%		\node [style=Z] (3) at (2.75, -0.5) {};
%	\end{pgfonlayer}
%	\begin{pgfonlayer}{edgelayer}
%		\draw (3) to (1.center);
%		\draw (2) to (0.center);
%	\end{pgfonlayer}
%\end{tikzpicture}
%\right\rrbracket_{\ZXA}
%$$
%
%
%$$
%\left\llbracket
%\begin{tikzpicture}
%	\begin{pgfonlayer}{nodelayer}
%		\node [style=Z] (0) at (3, -0.25) {};
%		\node [style=none] (1) at (2, -0) {};
%		\node [style=none] (2) at (2, -0.5) {};
%		\node [style=X] (3) at (3.75, -0.25) {};
%	\end{pgfonlayer}
%	\begin{pgfonlayer}{edgelayer}
%		\draw (3) to (0);
%		\draw [in=0, out=166, looseness=1.00] (0) to (1.center);
%		\draw [in=-166, out=0, looseness=1.00] (2.center) to (0);
%	\end{pgfonlayer}
%\end{tikzpicture}
%\right\rrbracket_{\ZXA}
%=
%\begin{tikzpicture}
%	\begin{pgfonlayer}{nodelayer}
%		\node [style=oplus] (0) at (3, -1.75) {};
%		\node [style=dot] (1) at (3, -1) {};
%		\node [style=zeroout] (2) at (3.5, -1) {};
%		\node [style=zeroout] (3) at (3.5, -1.75) {};
%		\node [style=none] (4) at (2.5, -1) {};
%		\node [style=none] (5) at (2.5, -1.75) {};
%	\end{pgfonlayer}
%	\begin{pgfonlayer}{edgelayer}
%		\draw (3) to (0);
%		\draw (0) to (5.center);
%		\draw (4.center) to (1);
%		\draw (1) to (2);
%		\draw (1) to (0);
%	\end{pgfonlayer}
%\end{tikzpicture}
%\eq{\ref{TOF.2}}
%\left\llbracket
%\begin{tikzpicture}
%	\begin{pgfonlayer}{nodelayer}
%		\node [style=none] (0) at (2, -0) {};
%		\node [style=none] (1) at (2, -0.5) {};
%		\node [style=X] (2) at (2.75, -0) {};
%		\node [style=X] (3) at (2.75, -0.5) {};
%	\end{pgfonlayer}
%	\begin{pgfonlayer}{edgelayer}
%		\draw (3) to (1.center);
%		\draw (2) to (0.center);
%	\end{pgfonlayer}
%\end{tikzpicture}
%\right\rrbracket_{\ZXA}
%$$



\item[\ref{ZXA.15}:]
\begin{align*}
\left\llbracket
\begin{tikzpicture}
	\begin{pgfonlayer}{nodelayer}
		\node [style=Z] (41) at (-1, 6.25) {};
		\node [style=none] (42) at (-1, 7.25) {};
		\node [style=andin] (420) at (-1, 7.25) {};
		\node [style=none] (43) at (-1, 7.75) {};
		\node [style=none] (44) at (-1, 5.75) {};
	\end{pgfonlayer}
	\begin{pgfonlayer}{edgelayer}
		\draw (43.center) to (42);
		\draw [in=120, out=-120, looseness=1.25] (42) to (41);
		\draw [in=-60, out=60, looseness=1.25] (41) to (42);
		\draw (41) to (44.center);
	\end{pgfonlayer}
\end{tikzpicture}
\right\rrbracket_{\ZXA}
&=
\begin{tikzpicture}
	\begin{pgfonlayer}{nodelayer}
		\node [style=dot] (42) at (0, 7) {};
		\node [style=dot] (43) at (0.5, 7) {};
		\node [style=oplus] (44) at (1, 7) {};
		\node [style=Z] (45) at (0, 7.5) {};
		\node [style=Z] (46) at (0.5, 7.5) {};
		\node [style=zeroin] (47) at (1, 6.5) {};
		\node [style=none] (48) at (1, 7.75) {};
		\node [style=dot] (49) at (0, 6.5) {};
		\node [style=oplus] (50) at (0.5, 6.5) {};
		\node [style=zeroin] (51) at (0.5, 6) {};
		\node [style=none] (52) at (0, 5.75) {};
	\end{pgfonlayer}
	\begin{pgfonlayer}{edgelayer}
		\draw (48.center) to (44);
		\draw (44) to (47);
		\draw (44) to (43);
		\draw (43) to (46);
		\draw (45) to (42);
		\draw (42) to (43);
		\draw (43) to (50);
		\draw (50) to (49);
		\draw (49) to (42);
		\draw (50) to (51);
		\draw (49) to (52.center);
	\end{pgfonlayer}
\end{tikzpicture}
\eq{Lem. \ref{lemma:Iwama}}
\begin{tikzpicture}
	\begin{pgfonlayer}{nodelayer}
		\node [style=dot] (43) at (0, 7) {};
		\node [style=dot] (44) at (0.5, 7) {};
		\node [style=oplus] (45) at (1, 7) {};
		\node [style=Z] (46) at (0, 8) {};
		\node [style=Z] (47) at (0.5, 8) {};
		\node [style=zeroin] (48) at (1, 6) {};
		\node [style=none] (49) at (1, 8.25) {};
		\node [style=zeroin] (50) at (0.5, 6) {};
		\node [style=none] (51) at (0, 5.75) {};
		\node [style=dot] (52) at (0, 7.5) {};
		\node [style=oplus] (53) at (0.5, 7.5) {};
		\node [style=dot] (54) at (0, 6.5) {};
		\node [style=oplus] (55) at (1, 6.5) {};
	\end{pgfonlayer}
	\begin{pgfonlayer}{edgelayer}
		\draw (49.center) to (45);
		\draw (45) to (48);
		\draw (45) to (44);
		\draw (44) to (47);
		\draw (46) to (43);
		\draw (43) to (44);
		\draw (53) to (52);
		\draw (55) to (54);
		\draw (44) to (50);
		\draw (51.center) to (54);
		\draw (54) to (43);
	\end{pgfonlayer}
\end{tikzpicture}
\eq{Lem. \ref{cor:copy}}
\begin{tikzpicture}
	\begin{pgfonlayer}{nodelayer}
		\node [style=dot] (44) at (0, 7) {};
		\node [style=dot] (45) at (0.5, 7) {};
		\node [style=oplus] (46) at (1, 7) {};
		\node [style=Z] (47) at (0, 7.5) {};
		\node [style=Z] (48) at (0.5, 7.5) {};
		\node [style=zeroin] (49) at (1, 6) {};
		\node [style=none] (50) at (1, 7.75) {};
		\node [style=zeroin] (51) at (0.5, 6) {};
		\node [style=none] (52) at (0, 5.75) {};
		\node [style=dot] (53) at (0, 6.5) {};
		\node [style=oplus] (54) at (1, 6.5) {};
	\end{pgfonlayer}
	\begin{pgfonlayer}{edgelayer}
		\draw (50.center) to (46);
		\draw (46) to (49);
		\draw (46) to (45);
		\draw (45) to (48);
		\draw (47) to (44);
		\draw (44) to (45);
		\draw (54) to (53);
		\draw (45) to (51);
		\draw (52.center) to (53);
		\draw (53) to (44);
	\end{pgfonlayer}
\end{tikzpicture}
\eq{\ref{TOF.2}}
\begin{tikzpicture}
	\begin{pgfonlayer}{nodelayer}
		\node [style=Z] (45) at (0, 7) {};
		\node [style=Z] (46) at (0.5, 7) {};
		\node [style=zeroin] (47) at (1, 6) {};
		\node [style=none] (48) at (1, 7.25) {};
		\node [style=zeroin] (49) at (0.5, 6) {};
		\node [style=none] (50) at (0, 5.75) {};
		\node [style=dot] (51) at (0, 6.5) {};
		\node [style=oplus] (52) at (1, 6.5) {};
	\end{pgfonlayer}
	\begin{pgfonlayer}{edgelayer}
		\draw (52) to (51);
		\draw (50.center) to (51);
		\draw (48.center) to (52);
		\draw (52) to (47);
		\draw (49) to (46);
		\draw (45) to (51);
	\end{pgfonlayer}
\end{tikzpicture}\\
&\eq{Lem. \ref{cor:copy}}
\begin{tikzpicture}
	\begin{pgfonlayer}{nodelayer}
		\node [style=Z] (46) at (0, 7) {};
		\node [style=zeroin] (47) at (0.5, 6) {};
		\node [style=none] (48) at (0.5, 7.25) {};
		\node [style=none] (49) at (0, 5.75) {};
		\node [style=dot] (50) at (0, 6.5) {};
		\node [style=oplus] (51) at (0.5, 6.5) {};
	\end{pgfonlayer}
	\begin{pgfonlayer}{edgelayer}
		\draw (51) to (50);
		\draw (49.center) to (50);
		\draw (48.center) to (51);
		\draw (51) to (47);
		\draw (46) to (50);
	\end{pgfonlayer}
\end{tikzpicture}
\eq{Lem. \ref{lemma:whiteunit}}
\begin{tikzpicture}
	\begin{pgfonlayer}{nodelayer}
		\node [style=none] (47) at (-1, 6.75) {};
		\node [style=none] (48) at (-1, 5.75) {};
	\end{pgfonlayer}
	\begin{pgfonlayer}{edgelayer}
		\draw (47.center) to (48.center);
	\end{pgfonlayer}
\end{tikzpicture}
=
\left\llbracket
\begin{tikzpicture}
	\begin{pgfonlayer}{nodelayer}
		\node [style=none] (47) at (-1, 6.75) {};
		\node [style=none] (48) at (-1, 5.75) {};
	\end{pgfonlayer}
	\begin{pgfonlayer}{edgelayer}
		\draw (47.center) to (48.center);
	\end{pgfonlayer}
\end{tikzpicture}
\right\rrbracket_{\ZXA}
\end{align*}


\item[\ref{ZXA.16}:]
This is precisely \ref{TOF.7}.

\item[\ref{ZXA.17}:]
\begin{align*}
\left\llbracket
\begin{tikzpicture}
	\begin{pgfonlayer}{nodelayer}
		\node [style=X] (48) at (0, 6.25) {};
		\node [style=andin] (49) at (-0.25, 7) {};
		\node [style=none] (50) at (-0.5, 6.25) {};
		\node [style=none] (51) at (-0.25, 5.75) {};
		\node [style=none] (52) at (0.25, 5.75) {};
		\node [style=none] (53) at (-0.5, 5.75) {};
		\node [style=none] (54) at (-0.25, 7.5) {};
	\end{pgfonlayer}
	\begin{pgfonlayer}{edgelayer}
		\draw [in=-72, out=90] (48) to (49.center);
		\draw (49.center) to (54.center);
		\draw [in=90, out=-108] (49.center) to (50.center);
		\draw (50.center) to (53.center);
		\draw [in=90, out=-117] (48) to (51.center);
		\draw [in=90, out=-63] (48) to (52.center);
	\end{pgfonlayer}
\end{tikzpicture}
\right\rrbracket_{\ZXA}
&=
\begin{tikzpicture}
	\begin{pgfonlayer}{nodelayer}
		\node [style=Z] (49) at (1.5, 6.75) {};
		\node [style=oplus] (50) at (1.5, 8) {};
		\node [style=zeroin] (51) at (1.5, 7.5) {};
		\node [style=dot] (52) at (1, 8) {};
		\node [style=dot] (53) at (0.5, 8) {};
		\node [style=Z] (54) at (1, 8.5) {};
		\node [style=Z] (55) at (0.5, 8.5) {};
		\node [style=dot] (56) at (1.5, 6.25) {};
		\node [style=oplus] (57) at (1, 6.25) {};
		\node [style=none] (58) at (1.5, 5.75) {};
		\node [style=none] (59) at (1, 5.75) {};
		\node [style=none] (60) at (0.5, 5.75) {};
		\node [style=none] (61) at (1.5, 8.75) {};
	\end{pgfonlayer}
	\begin{pgfonlayer}{edgelayer}
		\draw (56) to (57);
		\draw (57) to (59.center);
		\draw (58.center) to (56);
		\draw (61.center) to (50);
		\draw (50) to (51);
		\draw (50) to (52);
		\draw (52) to (54);
		\draw (52) to (53);
		\draw (53) to (55);
		\draw (53) to (60.center);
		\draw (57) to (52);
		\draw (49) to (56);
	\end{pgfonlayer}
\end{tikzpicture}
=
\begin{tikzpicture}
	\begin{pgfonlayer}{nodelayer}
		\node [style=Z] (50) at (2, 7.5) {};
		\node [style=oplus] (51) at (1.5, 7) {};
		\node [style=zeroin] (52) at (1.5, 6) {};
		\node [style=dot] (53) at (1, 7) {};
		\node [style=dot] (54) at (0.5, 7) {};
		\node [style=Z] (55) at (1, 7.5) {};
		\node [style=Z] (56) at (0.5, 7.5) {};
		\node [style=dot] (57) at (2, 6.5) {};
		\node [style=oplus] (58) at (1, 6.5) {};
		\node [style=none] (59) at (2, 5.75) {};
		\node [style=none] (60) at (1, 5.75) {};
		\node [style=none] (61) at (0.5, 5.75) {};
		\node [style=none] (62) at (1.5, 7.75) {};
	\end{pgfonlayer}
	\begin{pgfonlayer}{edgelayer}
		\draw (57) to (58);
		\draw (58) to (60.center);
		\draw (59.center) to (57);
		\draw (62.center) to (51);
		\draw (51) to (52);
		\draw (51) to (53);
		\draw (53) to (55);
		\draw (53) to (54);
		\draw (54) to (56);
		\draw (54) to (61.center);
		\draw (58) to (53);
		\draw (50) to (57);
	\end{pgfonlayer}
\end{tikzpicture}\
\eq{Lem. \ref{lemma:Iwama}}
\begin{tikzpicture}
	\begin{pgfonlayer}{nodelayer}
		\node [style=Z] (51) at (2, 8) {};
		\node [style=oplus] (52) at (1.5, 7) {};
		\node [style=zeroin] (53) at (1.5, 6) {};
		\node [style=dot] (54) at (1, 7) {};
		\node [style=dot] (55) at (0.5, 7) {};
		\node [style=Z] (56) at (1, 8) {};
		\node [style=Z] (57) at (0.5, 8) {};
		\node [style=dot] (58) at (2, 7.5) {};
		\node [style=oplus] (59) at (1, 7.5) {};
		\node [style=none] (60) at (2, 5.75) {};
		\node [style=none] (61) at (1, 5.75) {};
		\node [style=none] (62) at (0.5, 5.75) {};
		\node [style=none] (63) at (1.5, 8.25) {};
		\node [style=oplus] (64) at (1.5, 6.5) {};
		\node [style=dot] (65) at (2, 6.5) {};
		\node [style=dot] (66) at (0.5, 6.5) {};
	\end{pgfonlayer}
	\begin{pgfonlayer}{edgelayer}
		\draw (58) to (59);
		\draw (59) to (61.center);
		\draw (60.center) to (58);
		\draw (63.center) to (52);
		\draw (52) to (53);
		\draw (52) to (54);
		\draw (54) to (56);
		\draw (54) to (55);
		\draw (55) to (57);
		\draw (55) to (62.center);
		\draw (59) to (54);
		\draw (51) to (58);
		\draw (65) to (64);
		\draw (64) to (66);
	\end{pgfonlayer}
\end{tikzpicture}
\eq{Lem. \ref{cor:copy}}
\begin{tikzpicture}
	\begin{pgfonlayer}{nodelayer}
		\node [style=Z] (52) at (2, 7.5) {};
		\node [style=oplus] (53) at (1.5, 7) {};
		\node [style=zeroin] (54) at (1.5, 6) {};
		\node [style=dot] (55) at (1, 7) {};
		\node [style=dot] (56) at (0.5, 7) {};
		\node [style=Z] (57) at (1, 7.5) {};
		\node [style=Z] (58) at (0.5, 7.5) {};
		\node [style=none] (59) at (2, 5.75) {};
		\node [style=none] (60) at (1, 5.75) {};
		\node [style=none] (61) at (0.5, 5.75) {};
		\node [style=none] (62) at (1.5, 7.75) {};
		\node [style=oplus] (63) at (1.5, 6.5) {};
		\node [style=dot] (64) at (2, 6.5) {};
		\node [style=dot] (65) at (0.5, 6.5) {};
	\end{pgfonlayer}
	\begin{pgfonlayer}{edgelayer}
		\draw (62.center) to (53);
		\draw (53) to (54);
		\draw (53) to (55);
		\draw (55) to (57);
		\draw (55) to (56);
		\draw (56) to (58);
		\draw (56) to (61.center);
		\draw (64) to (63);
		\draw (63) to (65);
		\draw (52) to (64);
		\draw (64) to (59.center);
		\draw (60.center) to (55);
	\end{pgfonlayer}
\end{tikzpicture}\\
&\eq{Lem. \ref{cor:copy}}
\begin{tikzpicture}
	\begin{pgfonlayer}{nodelayer}
		\node [style=Z] (53) at (2, 8) {};
		\node [style=oplus] (54) at (1, 7) {};
		\node [style=zeroin] (55) at (1, 6) {};
		\node [style=dot] (56) at (0.5, 7) {};
		\node [style=dot] (57) at (0, 7) {};
		\node [style=Z] (58) at (0.5, 8) {};
		\node [style=Z] (59) at (0, 8) {};
		\node [style=none] (60) at (2, 5.75) {};
		\node [style=none] (61) at (0.5, 5.75) {};
		\node [style=none] (62) at (0, 5.75) {};
		\node [style=none] (63) at (1, 8.25) {};
		\node [style=oplus] (64) at (1, 6.5) {};
		\node [style=dot] (65) at (2, 6.5) {};
		\node [style=dot] (66) at (0, 6.5) {};
		\node [style=zeroin] (67) at (1.5, 6) {};
		\node [style=Z] (68) at (1.5, 8) {};
		\node [style=oplus] (69) at (1.5, 7.5) {};
		\node [style=dot] (70) at (2, 7.5) {};
		\node [style=dot] (71) at (0, 7.5) {};
	\end{pgfonlayer}
	\begin{pgfonlayer}{edgelayer}
		\draw (63.center) to (54);
		\draw (54) to (55);
		\draw (54) to (56);
		\draw (56) to (58);
		\draw (56) to (57);
		\draw (57) to (59);
		\draw (57) to (62.center);
		\draw (65) to (64);
		\draw (64) to (66);
		\draw (53) to (65);
		\draw (65) to (60.center);
		\draw (61.center) to (56);
		\draw (67) to (68);
		\draw (70) to (69);
		\draw (69) to (71);
	\end{pgfonlayer}
\end{tikzpicture}
\eq{\ref{TOF.2}}
\begin{tikzpicture}
	\begin{pgfonlayer}{nodelayer}
		\node [style=Z] (54) at (2, 8.5) {};
		\node [style=oplus] (55) at (1, 7) {};
		\node [style=zeroin] (56) at (1, 6) {};
		\node [style=dot] (57) at (0.5, 7) {};
		\node [style=dot] (58) at (0, 7) {};
		\node [style=Z] (59) at (0.5, 8.5) {};
		\node [style=Z] (60) at (0, 8.5) {};
		\node [style=none] (61) at (2, 5.75) {};
		\node [style=none] (62) at (0.5, 5.75) {};
		\node [style=none] (63) at (0, 5.75) {};
		\node [style=none] (64) at (1, 8.75) {};
		\node [style=oplus] (65) at (1, 6.5) {};
		\node [style=dot] (66) at (2, 6.5) {};
		\node [style=dot] (67) at (0, 6.5) {};
		\node [style=zeroin] (68) at (1.5, 6) {};
		\node [style=Z] (69) at (1.5, 8.5) {};
		\node [style=oplus] (70) at (1.5, 8) {};
		\node [style=dot] (71) at (2, 8) {};
		\node [style=dot] (72) at (0, 8) {};
		\node [style=oplus] (73) at (1, 7.5) {};
		\node [style=dot] (74) at (1.5, 7.5) {};
	\end{pgfonlayer}
	\begin{pgfonlayer}{edgelayer}
		\draw (64.center) to (55);
		\draw (55) to (56);
		\draw (55) to (57);
		\draw (57) to (59);
		\draw (57) to (58);
		\draw (58) to (60);
		\draw (58) to (63.center);
		\draw (66) to (65);
		\draw (65) to (67);
		\draw (54) to (66);
		\draw (66) to (61.center);
		\draw (62.center) to (57);
		\draw (68) to (69);
		\draw (71) to (70);
		\draw (70) to (72);
		\draw (74) to (73);
	\end{pgfonlayer}
\end{tikzpicture}
\eq{Lem. \ref{lemma:Iwama}}
\begin{tikzpicture}
	\begin{pgfonlayer}{nodelayer}
		\node [style=Z] (55) at (2, 9) {};
		\node [style=oplus] (56) at (1, 7) {};
		\node [style=zeroin] (57) at (1, 6) {};
		\node [style=dot] (58) at (0.5, 7) {};
		\node [style=dot] (59) at (0, 7) {};
		\node [style=Z] (60) at (0.5, 9) {};
		\node [style=Z] (61) at (0, 9) {};
		\node [style=none] (62) at (2, 5.75) {};
		\node [style=none] (63) at (0.5, 5.75) {};
		\node [style=none] (64) at (0, 5.75) {};
		\node [style=none] (65) at (1, 9.25) {};
		\node [style=zeroin] (66) at (1.5, 6) {};
		\node [style=Z] (67) at (1.5, 9) {};
		\node [style=oplus] (68) at (1.5, 7.5) {};
		\node [style=dot] (69) at (2, 7.5) {};
		\node [style=dot] (70) at (0, 7.5) {};
		\node [style=oplus] (71) at (1, 8) {};
		\node [style=dot] (72) at (1.5, 8) {};
		\node [style=oplus] (73) at (1, 8.5) {};
		\node [style=dot] (74) at (0, 8.5) {};
		\node [style=dot] (75) at (2, 8.5) {};
		\node [style=oplus] (76) at (1, 6.5) {};
		\node [style=dot] (77) at (2, 6.5) {};
		\node [style=dot] (78) at (0, 6.5) {};
	\end{pgfonlayer}
	\begin{pgfonlayer}{edgelayer}
		\draw (65.center) to (56);
		\draw (56) to (57);
		\draw (56) to (58);
		\draw (58) to (60);
		\draw (58) to (59);
		\draw (59) to (61);
		\draw (59) to (64.center);
		\draw (63.center) to (58);
		\draw (66) to (67);
		\draw (69) to (68);
		\draw (68) to (70);
		\draw (72) to (71);
		\draw (75) to (73);
		\draw (73) to (74);
		\draw (76) to (78);
		\draw (77) to (76);
		\draw (55) to (77);
		\draw (77) to (62.center);
	\end{pgfonlayer}
\end{tikzpicture}\\
&\eq{\ref{TOF.9}}
\begin{tikzpicture}
	\begin{pgfonlayer}{nodelayer}
		\node [style=Z] (56) at (2, 8) {};
		\node [style=oplus] (57) at (1, 6.5) {};
		\node [style=zeroin] (58) at (1, 6) {};
		\node [style=dot] (59) at (0.5, 6.5) {};
		\node [style=dot] (60) at (0, 6.5) {};
		\node [style=Z] (61) at (0.5, 8) {};
		\node [style=Z] (62) at (0, 8) {};
		\node [style=none] (63) at (2, 5.75) {};
		\node [style=none] (64) at (0.5, 5.75) {};
		\node [style=none] (65) at (0, 5.75) {};
		\node [style=none] (66) at (1, 8.25) {};
		\node [style=zeroin] (67) at (1.5, 6) {};
		\node [style=Z] (68) at (1.5, 8) {};
		\node [style=oplus] (69) at (1.5, 7) {};
		\node [style=dot] (70) at (2, 7) {};
		\node [style=dot] (71) at (0, 7) {};
		\node [style=oplus] (72) at (1, 7.5) {};
		\node [style=dot] (73) at (1.5, 7.5) {};
	\end{pgfonlayer}
	\begin{pgfonlayer}{edgelayer}
		\draw (66.center) to (57);
		\draw (57) to (58);
		\draw (57) to (59);
		\draw (59) to (61);
		\draw (59) to (60);
		\draw (60) to (62);
		\draw (60) to (65.center);
		\draw (64.center) to (59);
		\draw (67) to (68);
		\draw (70) to (69);
		\draw (69) to (71);
		\draw (73) to (72);
		\draw (63.center) to (56);
	\end{pgfonlayer}
\end{tikzpicture}
\eq{Lem. \ref{cor:copy}}
\begin{tikzpicture}
	\begin{pgfonlayer}{nodelayer}
		\node [style=dot] (57) at (0.5, 6.75) {};
		\node [style=Z] (58) at (1, 8.25) {};
		\node [style=Z] (59) at (3, 8.25) {};
		\node [style=none] (60) at (1, 5.75) {};
		\node [style=dot] (61) at (3, 6.75) {};
		\node [style=dot] (62) at (1, 7.25) {};
		\node [style=none] (63) at (3, 5.75) {};
		\node [style=oplus] (64) at (1.5, 7.25) {};
		\node [style=none] (65) at (1.5, 8.5) {};
		\node [style=none] (66) at (0.5, 5.75) {};
		\node [style=Z] (67) at (2, 8.25) {};
		\node [style=oplus] (68) at (2, 6.75) {};
		\node [style=Z] (69) at (0.5, 8.25) {};
		\node [style=dot] (70) at (0.5, 7.25) {};
		\node [style=zeroin] (71) at (2, 6) {};
		\node [style=zeroin] (72) at (1.5, 6) {};
		\node [style=oplus] (73) at (1.5, 7.75) {};
		\node [style=dot] (74) at (2, 7.75) {};
		\node [style=zeroin] (75) at (2.5, 6) {};
		\node [style=Z] (76) at (2.5, 8.25) {};
	\end{pgfonlayer}
	\begin{pgfonlayer}{edgelayer}
		\draw (65.center) to (64);
		\draw (64) to (72);
		\draw (64) to (62);
		\draw (62) to (58);
		\draw (60.center) to (62);
		\draw (62) to (70);
		\draw (70) to (69);
		\draw (70) to (66.center);
		\draw (67) to (71);
		\draw (68) to (57);
		\draw (61) to (68);
		\draw (74) to (73);
		\draw (59) to (61);
		\draw (61) to (63.center);
		\draw (76) to (75);
	\end{pgfonlayer}
\end{tikzpicture}
\eq{Lem. \ref{cor:copy}}
\begin{tikzpicture}
	\begin{pgfonlayer}{nodelayer}
		\node [style=dot] (58) at (0.5, 6.75) {};
		\node [style=Z] (59) at (1, 8.75) {};
		\node [style=Z] (60) at (3, 8.75) {};
		\node [style=none] (61) at (1, 5.75) {};
		\node [style=dot] (62) at (3, 6.75) {};
		\node [style=dot] (63) at (1, 7.75) {};
		\node [style=none] (64) at (3, 5.75) {};
		\node [style=oplus] (65) at (1.5, 7.75) {};
		\node [style=none] (66) at (1.5, 9) {};
		\node [style=none] (67) at (0.5, 5.75) {};
		\node [style=Z] (68) at (2, 8.75) {};
		\node [style=oplus] (69) at (2, 6.75) {};
		\node [style=Z] (70) at (0.5, 8.75) {};
		\node [style=dot] (71) at (0.5, 7.75) {};
		\node [style=zeroin] (72) at (2, 6) {};
		\node [style=zeroin] (73) at (1.5, 6) {};
		\node [style=oplus] (74) at (1.5, 8.25) {};
		\node [style=dot] (75) at (2, 8.25) {};
		\node [style=zeroin] (76) at (2.5, 6) {};
		\node [style=Z] (77) at (2.5, 8.75) {};
		\node [style=dot] (78) at (0.5, 7.25) {};
		\node [style=oplus] (79) at (2.5, 7.25) {};
	\end{pgfonlayer}
	\begin{pgfonlayer}{edgelayer}
		\draw (66.center) to (65);
		\draw (65) to (73);
		\draw (65) to (63);
		\draw (63) to (59);
		\draw (61.center) to (63);
		\draw (63) to (71);
		\draw (71) to (70);
		\draw (71) to (67.center);
		\draw (68) to (72);
		\draw (69) to (58);
		\draw (62) to (69);
		\draw (75) to (74);
		\draw (60) to (62);
		\draw (62) to (64.center);
		\draw (77) to (76);
		\draw (79) to (78);
	\end{pgfonlayer}
\end{tikzpicture}\\
&\eq{\ref{TOF.2}}
\begin{tikzpicture}
	\begin{pgfonlayer}{nodelayer}
		\node [style=dot] (59) at (0.5, 6.75) {};
		\node [style=Z] (60) at (1, 9.25) {};
		\node [style=Z] (61) at (3, 9.25) {};
		\node [style=none] (62) at (1, 5.75) {};
		\node [style=dot] (63) at (3, 6.75) {};
		\node [style=dot] (64) at (1, 8.25) {};
		\node [style=none] (65) at (3, 5.75) {};
		\node [style=oplus] (66) at (1.5, 8.25) {};
		\node [style=none] (67) at (1.5, 9.5) {};
		\node [style=none] (68) at (0.5, 5.75) {};
		\node [style=Z] (69) at (2, 9.25) {};
		\node [style=oplus] (70) at (2, 6.75) {};
		\node [style=Z] (71) at (0.5, 9.25) {};
		\node [style=dot] (72) at (0.5, 8.25) {};
		\node [style=zeroin] (73) at (2, 6) {};
		\node [style=zeroin] (74) at (1.5, 6) {};
		\node [style=oplus] (75) at (1.5, 8.75) {};
		\node [style=dot] (76) at (2, 8.75) {};
		\node [style=zeroin] (77) at (2.5, 6) {};
		\node [style=Z] (78) at (2.5, 9.25) {};
		\node [style=oplus] (79) at (2, 7.25) {};
		\node [style=dot] (80) at (2.5, 7.25) {};
		\node [style=dot] (81) at (3, 7.25) {};
		\node [style=dot] (82) at (0.5, 7.75) {};
		\node [style=oplus] (83) at (2.5, 7.75) {};
	\end{pgfonlayer}
	\begin{pgfonlayer}{edgelayer}
		\draw (67.center) to (66);
		\draw (66) to (74);
		\draw (66) to (64);
		\draw (64) to (60);
		\draw (62.center) to (64);
		\draw (64) to (72);
		\draw (72) to (71);
		\draw (72) to (68.center);
		\draw (69) to (73);
		\draw (70) to (59);
		\draw (63) to (70);
		\draw (76) to (75);
		\draw (61) to (63);
		\draw (63) to (65.center);
		\draw (78) to (77);
		\draw (81) to (80);
		\draw (80) to (79);
		\draw (83) to (82);
	\end{pgfonlayer}
\end{tikzpicture}
\eq{\ref{TOF.2}}
\begin{tikzpicture}
	\begin{pgfonlayer}{nodelayer}
		\node [style=dot] (59) at (0.5, 6.75) {};
		\node [style=Z] (60) at (1, 9.25) {};
		\node [style=Z] (61) at (3, 9.25) {};
		\node [style=none] (62) at (1, 5.75) {};
		\node [style=dot] (63) at (3, 6.75) {};
		\node [style=dot] (64) at (1, 8.25) {};
		\node [style=none] (65) at (3, 5.75) {};
		\node [style=oplus] (66) at (1.5, 8.25) {};
		\node [style=none] (67) at (1.5, 9.5) {};
		\node [style=none] (68) at (0.5, 5.75) {};
		\node [style=Z] (69) at (2, 9.25) {};
		\node [style=oplus] (70) at (2, 6.75) {};
		\node [style=Z] (71) at (0.5, 9.25) {};
		\node [style=dot] (72) at (0.5, 8.25) {};
		\node [style=zeroin] (73) at (2, 6) {};
		\node [style=zeroin] (74) at (1.5, 6) {};
		\node [style=oplus] (75) at (1.5, 8.75) {};
		\node [style=dot] (76) at (2, 8.75) {};
		\node [style=zeroin] (77) at (2.5, 6) {};
		\node [style=Z] (78) at (2.5, 9.25) {};
		\node [style=oplus] (79) at (2, 7.25) {};
		\node [style=dot] (80) at (2.5, 7.25) {};
		\node [style=dot] (81) at (3, 7.25) {};
		\node [style=dot] (82) at (0.5, 7.75) {};
		\node [style=oplus] (83) at (2.5, 7.75) {};
	\end{pgfonlayer}
	\begin{pgfonlayer}{edgelayer}
		\draw (67.center) to (66);
		\draw (66) to (74);
		\draw (66) to (64);
		\draw (64) to (60);
		\draw (62.center) to (64);
		\draw (64) to (72);
		\draw (72) to (71);
		\draw (72) to (68.center);
		\draw (69) to (73);
		\draw (70) to (59);
		\draw (63) to (70);
		\draw (76) to (75);
		\draw (61) to (63);
		\draw (63) to (65.center);
		\draw (78) to (77);
		\draw (81) to (80);
		\draw (80) to (79);
		\draw (83) to (82);
	\end{pgfonlayer}
\end{tikzpicture}
\eq{\ref{ZXA.11}}
\begin{tikzpicture}
	\begin{pgfonlayer}{nodelayer}
		\node [style=dot] (60) at (-0.5, 7) {};
		\node [style=dot] (61) at (0, 7) {};
		\node [style=dot] (62) at (1, 8) {};
		\node [style=oplus] (63) at (0.5, 8) {};
		\node [style=oplus] (64) at (0.5, 7) {};
		\node [style=dot] (65) at (2, 7) {};
		\node [style=dot] (66) at (1.5, 7) {};
		\node [style=oplus] (67) at (1, 7) {};
		\node [style=oplus] (68) at (1.5, 7.5) {};
		\node [style=dot] (69) at (0, 7.5) {};
		\node [style=Z] (70) at (0, 8) {};
		\node [style=Z] (71) at (-0.5, 8) {};
		\node [style=Z] (72) at (2, 8) {};
		\node [style=Z] (73) at (1.5, 8) {};
		\node [style=zeroin] (74) at (0.5, 6) {};
		\node [style=zeroin] (75) at (1, 6) {};
		\node [style=zeroin] (76) at (1.5, 6) {};
		\node [style=none] (77) at (0.5, 8.75) {};
		\node [style=none] (78) at (2, 5.75) {};
		\node [style=none] (79) at (0, 5.75) {};
		\node [style=none] (80) at (-0.5, 5.75) {};
		\node [style=Z] (81) at (1, 8.5) {};
		\node [style=dot] (82) at (2, 6.5) {};
		\node [style=dot] (83) at (0, 6.5) {};
		\node [style=oplus] (84) at (1, 6.5) {};
	\end{pgfonlayer}
	\begin{pgfonlayer}{edgelayer}
		\draw (70) to (79.center);
		\draw (80.center) to (71);
		\draw (77.center) to (74);
		\draw (72) to (78.center);
		\draw (76) to (73);
		\draw (65) to (67);
		\draw (62) to (63);
		\draw (64) to (60);
		\draw (69) to (68);
		\draw (81) to (62);
		\draw (62) to (67);
		\draw (67) to (75);
		\draw (82) to (84);
		\draw (84) to (83);
	\end{pgfonlayer}
\end{tikzpicture}\\
&\eq{Lem. \ref{lemma:Iwama}}
\begin{tikzpicture}
	\begin{pgfonlayer}{nodelayer}
		\node [style=dot] (61) at (-0.5, 7.5) {};
		\node [style=dot] (62) at (0, 7.5) {};
		\node [style=dot] (63) at (1, 8) {};
		\node [style=oplus] (64) at (0.5, 8) {};
		\node [style=oplus] (65) at (0.5, 7.5) {};
		\node [style=dot] (66) at (2, 7.5) {};
		\node [style=dot] (67) at (1.5, 7.5) {};
		\node [style=oplus] (68) at (1, 7.5) {};
		\node [style=oplus] (69) at (1.5, 6.5) {};
		\node [style=dot] (70) at (0, 6.5) {};
		\node [style=Z] (71) at (0, 8) {};
		\node [style=Z] (72) at (-0.5, 8) {};
		\node [style=Z] (73) at (2, 8) {};
		\node [style=Z] (74) at (1.5, 8) {};
		\node [style=zeroin] (75) at (0.5, 7) {};
		\node [style=zeroin] (76) at (1, 7) {};
		\node [style=zeroin] (77) at (1.5, 6) {};
		\node [style=none] (78) at (0.5, 8.75) {};
		\node [style=none] (79) at (2, 5.75) {};
		\node [style=none] (80) at (0, 5.75) {};
		\node [style=none] (81) at (-0.5, 5.75) {};
		\node [style=Z] (82) at (1, 8.5) {};
	\end{pgfonlayer}
	\begin{pgfonlayer}{edgelayer}
		\draw (71) to (80.center);
		\draw (81.center) to (72);
		\draw (78.center) to (75);
		\draw (73) to (79.center);
		\draw (77) to (74);
		\draw (66) to (68);
		\draw (63) to (64);
		\draw (65) to (61);
		\draw (70) to (69);
		\draw (82) to (63);
		\draw (63) to (68);
		\draw (68) to (76);
	\end{pgfonlayer}
\end{tikzpicture}
=
\left\llbracket
\begin{tikzpicture}
	\begin{pgfonlayer}{nodelayer}
		\node [style=none] (62) at (0.25, 6.25) {};
		\node [style=andin] (63) at (-0.35, 7) {};
		\node [style=none] (64) at (-0.25, 6.25) {};
		\node [style=andin] (65) at (0.35, 7) {};
		\node [style=none] (66) at (-0.25, 6.25) {};
		\node [style=Z] (67) at (-0.25, 6.25) {};
		\node [style=X] (68) at (0, 7.75) {};
		\node [style=none] (69) at (0, 8.25) {};
		\node [style=none] (70) at (-0.25, 5.75) {};
		\node [style=none] (71) at (0.5, 5.75) {};
		\node [style=none] (72) at (0.25, 5.75) {};
	\end{pgfonlayer}
	\begin{pgfonlayer}{edgelayer}
		\draw [in=-72, out=90] (62.center) to (63.center);
		\draw [in=120, out=-108] (63.center) to (64.center);
		\draw [in=45, out=-108] (65.center) to (66.center);
		\draw (62.center) to (72.center);
		\draw (70.center) to (64.center);
		\draw [in=-117, out=90] (63.center) to (68);
		\draw (68) to (69.center);
		\draw [in=90, out=-63] (68) to (65.center);
		\draw [in=-75, out=90, looseness=1.25] (71.center) to (65.center);
	\end{pgfonlayer}
\end{tikzpicture}
\right\rrbracket_{\ZXA}
\end{align*}


\end{enumerate}

\end{proof}



To prove functoriality in the other direction, we prove some basic properties of $\ZXA$.

\begin{lemma}
\label{lem:blackdot}
$$
\begin{tikzpicture}
	\begin{pgfonlayer}{nodelayer}
		\node [style=X] (23) at (0, 5.25) {};
	\end{pgfonlayer}
\end{tikzpicture}
=
\begin{tikzpicture}
	\begin{pgfonlayer}{nodelayer}
		\node [style=none] (0) at (2, 0) {};
		\node [style=none] (1) at (2, -1) {};
		\node [style=none] (2) at (3, -1) {};
		\node [style=none] (3) at (3, 0) {};
	\end{pgfonlayer}
	\begin{pgfonlayer}{edgelayer}
		\draw[style=dashed] (3.center) to (0.center) to (1.center) to (2.center) to cycle;
	\end{pgfonlayer}
\end{tikzpicture}
$$
\end{lemma}
\begin{proof}
\begin{align*}
\begin{tikzpicture}
	\begin{pgfonlayer}{nodelayer}
		\node [style=X] (0) at (0, -0) {};
	\end{pgfonlayer}
\end{tikzpicture}
\eq{\ref{ZXA.1}}
\begin{tikzpicture}
	\begin{pgfonlayer}{nodelayer}
		\node [style=X] (1) at (-0.5, -0.75) {};
		\node [style=X] (2) at (0, -0.75) {};
	\end{pgfonlayer}
	\begin{pgfonlayer}{edgelayer}
		\draw [in=90, out=90, looseness=2.25] (2) to (1);
	\end{pgfonlayer}
\end{tikzpicture}
\eq{\ref{ZXA.3}}
\begin{tikzpicture}
	\begin{pgfonlayer}{nodelayer}
		\node [style=Z] (2) at (0, 0) {};
		\node [style=X] (3) at (-0.25, -0.75) {};
		\node [style=X] (4) at (0.25, -0.75) {};
		\node [style=Z] (5) at (0, 0.75) {};
	\end{pgfonlayer}
	\begin{pgfonlayer}{edgelayer}
		\draw (5) to (2);
		\draw [in=90, out=-124] (2) to (3);
		\draw [in=-56, out=90] (4) to (2);
	\end{pgfonlayer}
\end{tikzpicture}
\eq{\ref{ZXA.6}}
\begin{tikzpicture}
	\begin{pgfonlayer}{nodelayer}
		\node [style=Z] (3) at (0, 0.75) {};
		\node [style=Z] (4) at (0, 0) {};
		\node [style=X] (5) at (0, -1.75) {};
		\node [style=Z] (6) at (0, -1) {};
	\end{pgfonlayer}
	\begin{pgfonlayer}{edgelayer}
		\draw (3) to (4);
		\draw (5) to (6);
		\draw [bend left, looseness=1.25] (6) to (4);
		\draw [bend left, looseness=1.25] (4) to (6);
	\end{pgfonlayer}
\end{tikzpicture}
\eq{\ref{ZXA.3}}
\begin{tikzpicture}
	\begin{pgfonlayer}{nodelayer}
		\node [style=Z] (4) at (0, 6.75) {};
		\node [style=X] (5) at (0, 5.75) {};
	\end{pgfonlayer}
	\begin{pgfonlayer}{edgelayer}
		\draw (5) to (4);
	\end{pgfonlayer}
\end{tikzpicture}
\eq{\ref{ZXA.7}}
\end{align*}
\end{proof}




\begin{lemma}
The phase fusion of the black spider in $\ZXA$, 
$$
\begin{tikzpicture}
	\begin{pgfonlayer}{nodelayer}
		\node [style=X] (5) at (-0.5, 5.75) {$1$};
		\node [style=X] (6) at (0, 5.75) {$1$};
		\node [style=X] (7) at (-0.25, 6.5) {};
		\node [style=none] (8) at (-0.25, 7) {};
	\end{pgfonlayer}
	\begin{pgfonlayer}{edgelayer}
		\draw [in=-108, out=90] (5) to (7);
		\draw [in=90, out=-72] (7) to (6);
		\draw (7) to (8.center);
	\end{pgfonlayer}
\end{tikzpicture}
=
\begin{tikzpicture}
	\begin{pgfonlayer}{nodelayer}
		\node [style=X] (6) at (-0.25, 5.75) {};
		\node [style=none] (7) at (-0.25, 6.25) {};
	\end{pgfonlayer}
	\begin{pgfonlayer}{edgelayer}
		\draw (6) to (7.center);
	\end{pgfonlayer}
\end{tikzpicture}
$$
in the presence of the other axioms is equivalent to asserting:
$$
\begin{tikzpicture}
	\begin{pgfonlayer}{nodelayer}
		\node [style=X] (7) at (-0.5, 5.75) {$1$};
		\node [style=Z] (8) at (-0.5, 6.5) {};
	\end{pgfonlayer}
	\begin{pgfonlayer}{edgelayer}
		\draw (8) to (7);
	\end{pgfonlayer}
\end{tikzpicture}
=
\begin{tikzpicture}
	\begin{pgfonlayer}{nodelayer}
		\node [style=none] (0) at (2, 0) {};
		\node [style=none] (1) at (2, -1) {};
		\node [style=none] (2) at (3, -1) {};
		\node [style=none] (3) at (3, 0) {};
	\end{pgfonlayer}
	\begin{pgfonlayer}{edgelayer}
		\draw[style=dashed] (3.center) to (0.center) to (1.center) to (2.center) to cycle;
	\end{pgfonlayer}
\end{tikzpicture}
$$
Or in other terms, the phase fusion of the black spider is equivalent to the interaction of the unit for and and the counit for copying as a bialgebra.
\end{lemma}

\begin{proof}
For the one direction, suppose that phase fusion holds:


\begin{align*}
\begin{tikzpicture}
	\begin{pgfonlayer}{nodelayer}
		\node [style=X] (9) at (-0.5, 6) {$1$};
		\node [style=Z] (10) at (-0.5, 6.75) {};
	\end{pgfonlayer}
	\begin{pgfonlayer}{edgelayer}
		\draw (10) to (9);
	\end{pgfonlayer}
\end{tikzpicture}
\eq{\ref{ZXA.3}}
\begin{tikzpicture}
	\begin{pgfonlayer}{nodelayer}
		\node [style=X] (10) at (-0.5, 6) {$1$};
		\node [style=Z] (11) at (-0.5, 6.75) {};
	\end{pgfonlayer}
	\begin{pgfonlayer}{edgelayer}
		\draw (11) to (10);
		\draw [in=135, out=45, loop] (11) to ();
	\end{pgfonlayer}
\end{tikzpicture}
\eq{\ref{ZXA.1}}
\begin{tikzpicture}
	\begin{pgfonlayer}{nodelayer}
		\node [style=X] (11) at (-0.5, 6) {$1$};
		\node [style=Z] (12) at (-0.5, 6.75) {};
		\node [style=X] (13) at (-0.5, 7.75) {};
		\node [style=X] (14) at (-0.5, 8.5) {};
	\end{pgfonlayer}
	\begin{pgfonlayer}{edgelayer}
		\draw (12) to (11);
		\draw [bend left=45, looseness=1.25] (12) to (13);
		\draw [bend left, looseness=1.25] (13) to (12);
		\draw (13) to (14);
	\end{pgfonlayer}
\end{tikzpicture}
\eq{\ref{ZXA.8}}
\begin{tikzpicture}
	\begin{pgfonlayer}{nodelayer}
		\node [style=X] (12) at (-1, 6) {$1$};
		\node [style=X] (13) at (-0.5, 7) {};
		\node [style=X] (14) at (-0.5, 7.75) {};
		\node [style=X] (15) at (0, 6) {$1$};
	\end{pgfonlayer}
	\begin{pgfonlayer}{edgelayer}
		\draw (13) to (14);
		\draw [in=90, out=-117] (13) to (12);
		\draw [in=-63, out=90] (15) to (13);
	\end{pgfonlayer}
\end{tikzpicture}
=
\begin{tikzpicture}
	\begin{pgfonlayer}{nodelayer}
		\node [style=X] (13) at (-0.5, 6) {};
		\node [style=X] (14) at (-0.5, 6.75) {};
	\end{pgfonlayer}
	\begin{pgfonlayer}{edgelayer}
		\draw (13) to (14);
	\end{pgfonlayer}
\end{tikzpicture}
\eq{\ref{ZXA.7}, Lem. \ref{lem:blackdot}}
\begin{tikzpicture}
	\begin{pgfonlayer}{nodelayer}
		\node [style=none] (0) at (2, 0) {};
		\node [style=none] (1) at (2, -1) {};
		\node [style=none] (2) at (3, -1) {};
		\node [style=none] (3) at (3, 0) {};
	\end{pgfonlayer}
	\begin{pgfonlayer}{edgelayer}
		\draw[style=dashed] (3.center) to (0.center) to (1.center) to (2.center) to cycle;
	\end{pgfonlayer}
\end{tikzpicture}
\end{align*}


Conversely if the unit part of the bialgebra rule holds:

\begin{align*}
\begin{tikzpicture}
	\begin{pgfonlayer}{nodelayer}
		\node [style=X] (15) at (-0.5, 6) {$1$};
		\node [style=X] (16) at (0, 6) {$1$};
		\node [style=X] (17) at (-0.25, 6.75) {};
		\node [style=none] (18) at (-0.25, 7.25) {};
	\end{pgfonlayer}
	\begin{pgfonlayer}{edgelayer}
		\draw [in=-108, out=90] (15) to (17);
		\draw [in=90, out=-72] (17) to (16);
		\draw (17) to (18.center);
	\end{pgfonlayer}
\end{tikzpicture}
\eq{\ref{ZXA.14}}
\begin{tikzpicture}
	\begin{pgfonlayer}{nodelayer}
		\node [style=X] (16) at (-0.25, 7.5) {};
		\node [style=none] (17) at (-0.25, 8) {};
		\node [style=X] (18) at (-0.25, 6) {$1$};
		\node [style=Z] (19) at (-0.25, 6.75) {};
	\end{pgfonlayer}
	\begin{pgfonlayer}{edgelayer}
		\draw (16) to (17.center);
		\draw [in=120, out=-120, looseness=1.25] (16) to (19);
		\draw (19) to (18);
		\draw [in=-60, out=60, looseness=1.25] (19) to (16);
	\end{pgfonlayer}
\end{tikzpicture}
\eq{\ref{ZXA.8}}
\begin{tikzpicture}
	\begin{pgfonlayer}{nodelayer}
		\node [style=X] (17) at (-0.25, 7.5) {};
		\node [style=none] (18) at (-0.25, 8) {};
		\node [style=X] (19) at (-0.25, 6) {$1$};
		\node [style=Z] (20) at (-0.25, 6.75) {};
	\end{pgfonlayer}
	\begin{pgfonlayer}{edgelayer}
		\draw (17) to (18.center);
		\draw (20) to (19);
	\end{pgfonlayer}
\end{tikzpicture}
=
\begin{tikzpicture}
	\begin{pgfonlayer}{nodelayer}
		\node [style=X] (18) at (-0.25, 6) {};
		\node [style=none] (19) at (-0.25, 6.5) {};
	\end{pgfonlayer}
	\begin{pgfonlayer}{edgelayer}
		\draw (18) to (19.center);
	\end{pgfonlayer}
\end{tikzpicture}
\end{align*}


\end{proof}


\begin{lemma}
\label{lem:oldaxiom}
$$
\begin{tikzpicture}
	\begin{pgfonlayer}{nodelayer}
		\node [style=X] (19) at (0, 6.25) {};
		\node [style=none] (20) at (0.5, 7.25) {};
		\node [style=andin] (21) at (0.5, 7.25) {};
		\node [style=none] (22) at (0.5, 8) {};
		\node [style=none] (23) at (1, 6.25) {};
		\node [style=none] (24) at (1, 6) {};
	\end{pgfonlayer}
	\begin{pgfonlayer}{edgelayer}
		\draw [in=90, out=-63] (20.center) to (23.center);
		\draw (22.center) to (20.center);
		\draw [in=90, out=-117] (20.center) to (19);
		\draw (24.center) to (23.center);
	\end{pgfonlayer}
\end{tikzpicture}
=
\begin{tikzpicture}
	\begin{pgfonlayer}{nodelayer}
		\node [style=X] (20) at (0.5, 7.5) {};
		\node [style=none] (21) at (0.5, 8.25) {};
		\node [style=none] (22) at (0.5, 6) {};
		\node [style=Z] (23) at (0.5, 6.75) {};
	\end{pgfonlayer}
	\begin{pgfonlayer}{edgelayer}
		\draw (21.center) to (20);
		\draw (22.center) to (23);
	\end{pgfonlayer}
\end{tikzpicture}
$$
\end{lemma}

\begin{proof}
$$
\begin{tikzpicture}
	\begin{pgfonlayer}{nodelayer}
		\node [style=X] (21) at (0, 6.25) {};
		\node [style=none] (22) at (0.5, 7.25) {};
		\node [style=andin] (23) at (0.5, 7.25) {};
		\node [style=none] (24) at (0.5, 8) {};
		\node [style=none] (25) at (1, 6.25) {};
		\node [style=none] (26) at (1, 6) {};
	\end{pgfonlayer}
	\begin{pgfonlayer}{edgelayer}
		\draw [in=90, out=-63] (22.center) to (25.center);
		\draw (24.center) to (22.center);
		\draw [in=90, out=-117] (22.center) to (21);
		\draw (26.center) to (25.center);
	\end{pgfonlayer}
\end{tikzpicture}
\eq{\ref{ZXA.1}}
\begin{tikzpicture}
	\begin{pgfonlayer}{nodelayer}
		\node [style=X] (22) at (0, 7) {};
		\node [style=none] (23) at (0.5, 8.75) {};
		\node [style=none] (24) at (1, 6) {};
		\node [style=none] (25) at (1, 7) {};
		\node [style=andin] (26) at (0.5, 8) {};
		\node [style=none] (27) at (0.5, 8) {};
		\node [style=X] (28) at (-0.25, 6.25) {$1$};
		\node [style=X] (29) at (0.25, 6.25) {$1$};
	\end{pgfonlayer}
	\begin{pgfonlayer}{edgelayer}
		\draw (24.center) to (25.center);
		\draw (23.center) to (27.center);
		\draw [in=90, out=-63] (27.center) to (25.center);
		\draw [in=90, out=-117] (27.center) to (22);
		\draw [in=90, out=-108] (22) to (28);
		\draw [in=-72, out=90] (29) to (22);
	\end{pgfonlayer}
\end{tikzpicture}
\eq{\ref{ZXA.17}}
\begin{tikzpicture}
	\begin{pgfonlayer}{nodelayer}
		\node [style=none] (23) at (-2.25, 7.5) {};
		\node [style=none] (24) at (-0.75, 7.5) {};
		\node [style=Z] (25) at (-0.75, 6.5) {};
		\node [style=none] (26) at (-1.5, 9) {};
		\node [style=none] (27) at (-0.75, 6) {};
		\node [style=X] (28) at (-1.5, 6.25) {$1$};
		\node [style=X] (29) at (-2.25, 6.25) {$1$};
		\node [style=X] (30) at (-1.5, 8.25) {};
		\node [style=andin] (31) at (-2.25, 7.5) {};
		\node [style=andin] (32) at (-0.75, 7.5) {};
	\end{pgfonlayer}
	\begin{pgfonlayer}{edgelayer}
		\draw (26.center) to (30);
		\draw [in=90, out=-150] (30) to (23.center);
		\draw (23.center) to (29);
		\draw [in=90, out=-121] (24.center) to (28);
		\draw [in=-30, out=90] (24.center) to (30);
		\draw [in=146, out=-45] (23.center) to (25);
		\draw (25) to (24.center);
		\draw (25) to (27.center);
	\end{pgfonlayer}
\end{tikzpicture}
\eq{\ref{ZXA.10}}
\begin{tikzpicture}
	\begin{pgfonlayer}{nodelayer}
		\node [style=none] (24) at (1, 6) {};
		\node [style=Z] (25) at (1, 6.75) {};
		\node [style=none] (26) at (1, 8.5) {};
		\node [style=X] (27) at (1, 7.75) {};
	\end{pgfonlayer}
	\begin{pgfonlayer}{edgelayer}
		\draw (24.center) to (25);
		\draw (27) to (26.center);
		\draw [in=120, out=-120, looseness=1.25] (27) to (25);
		\draw [in=-60, out=60, looseness=1.25] (25) to (27);
	\end{pgfonlayer}
\end{tikzpicture}
\eq{\ref{ZXA.8}}
\begin{tikzpicture}
	\begin{pgfonlayer}{nodelayer}
		\node [style=X] (25) at (0.5, 7.5) {};
		\node [style=none] (26) at (0.5, 8.25) {};
		\node [style=none] (27) at (0.5, 6) {};
		\node [style=Z] (28) at (0.5, 6.75) {};
	\end{pgfonlayer}
	\begin{pgfonlayer}{edgelayer}
		\draw (26.center) to (25);
		\draw (27.center) to (28);
	\end{pgfonlayer}
\end{tikzpicture}
$$
\end{proof}


\begin{proposition}
\label{prop:ZXATOF}
Consider the interpretation $\llbracket\_\rrbracket_{\hat \TOF}:\hat \TOF\to \ZXA$ taking:

\begin{center}
\begin{tabular}{c}
$
\begin{tikzpicture}
	\begin{pgfonlayer}{nodelayer}
		\node [style=dot] (63) at (0, 6.5) {};
		\node [style=oplus] (64) at (0.5, 6.5) {};
		\node [style=dot] (65) at (-0.5, 6.5) {};
		\node [style=none] (66) at (0.5, 7.25) {};
		\node [style=none] (67) at (0, 7.25) {};
		\node [style=none] (68) at (-0.5, 7.25) {};
		\node [style=none] (69) at (-0.5, 5.75) {};
		\node [style=none] (70) at (0, 5.75) {};
		\node [style=none] (71) at (0.5, 5.75) {};
	\end{pgfonlayer}
	\begin{pgfonlayer}{edgelayer}
		\draw [style=simple] (66.center) to (64);
		\draw [style=simple] (64) to (63);
		\draw [style=simple] (63) to (65);
		\draw [style=simple] (65) to (68.center);
		\draw [style=simple] (67.center) to (63);
		\draw [style=simple] (63) to (70.center);
		\draw [style=simple] (69.center) to (65);
		\draw [style=simple] (64) to (71.center);
	\end{pgfonlayer}
\end{tikzpicture}
\mapsto
\begin{tikzpicture}
	\begin{pgfonlayer}{nodelayer}
		\node [style=none] (64) at (0, 5.75) {};
		\node [style=none] (65) at (1, 5.75) {};
		\node [style=none] (66) at (1.5, 5.75) {};
		\node [style=X] (67) at (1.5, 7.75) {};
		\node [style=Z] (68) at (1, 6.25) {};
		\node [style=Z] (69) at (0, 6.25) {};
		\node [style=andin] (70) at (0.5, 7.25) {};
		\node [style=none] (71) at (0, 8.5) {};
		\node [style=none] (72) at (1.5, 8.5) {};
		\node [style=none] (73) at (1, 8.5) {};
	\end{pgfonlayer}
	\begin{pgfonlayer}{edgelayer}
		\draw [style=simple, in=90, out=180, looseness=0.75] (67) to (70.center);
		\draw [style=simple, in=45, out=-120] (70.center) to (69);
		\draw [style=simple] (69) to (64.center);
		\draw [style=simple] (65.center) to (68);
		\draw [style=simple] (67) to (66.center);
		\draw [style=simple] (73.center) to (68);
		\draw [style=simple] (72.center) to (67);
		\draw [style=simple, in=-60, out=135] (68) to (70.center);
		\draw [style=simple] (71.center) to (69);
	\end{pgfonlayer}
\end{tikzpicture}
\hspace*{.5cm}
\begin{tikzpicture}
	\begin{pgfonlayer}{nodelayer}
		\node [style=onein] (67) at (0, 5.75) {};
		\node [style=none] (68) at (0, 6.75) {};
	\end{pgfonlayer}
	\begin{pgfonlayer}{edgelayer}
		\draw [style=simple] (68.center) to (67);
	\end{pgfonlayer}
\end{tikzpicture}
\mapsto
\begin{tikzpicture}
	\begin{pgfonlayer}{nodelayer}
		\node [style=X] (68) at (0, 5.75) {$1$};
		\node [style=none] (69) at (0, 6.75) {};
	\end{pgfonlayer}
	\begin{pgfonlayer}{edgelayer}
		\draw [style=simple] (69.center) to (68);
	\end{pgfonlayer}
\end{tikzpicture}
\hspace*{.5cm}
\begin{tikzpicture}
	\begin{pgfonlayer}{nodelayer}
		\node [style=oneout] (70) at (0, 6.75) {};
		\node [style=none] (71) at (0, 5.75) {};
	\end{pgfonlayer}
	\begin{pgfonlayer}{edgelayer}
		\draw [style=simple] (71.center) to (70);
	\end{pgfonlayer}
\end{tikzpicture}
\mapsto
\begin{tikzpicture}
	\begin{pgfonlayer}{nodelayer}
		\node [style=X] (0) at (0, 1.5) {$1$};
		\node [style=none] (1) at (0, 0.5) {};
	\end{pgfonlayer}
	\begin{pgfonlayer}{edgelayer}
		\draw [style=simple] (1.center) to (0);
	\end{pgfonlayer}
\end{tikzpicture}
\hspace*{.5cm}
\begin{tikzpicture}
	\begin{pgfonlayer}{nodelayer}
		\node [style=Z] (1) at (0, 0) {};
		\node [style=none] (2) at (0, 1) {};
	\end{pgfonlayer}
	\begin{pgfonlayer}{edgelayer}
		\draw [style=simple] (2.center) to (1);
	\end{pgfonlayer}
\end{tikzpicture}
\mapsto
\begin{tikzpicture}
	\begin{pgfonlayer}{nodelayer}
		\node [style=Z] (2) at (0, 0) {};
		\node [style=none] (3) at (0, 1) {};
	\end{pgfonlayer}
	\begin{pgfonlayer}{edgelayer}
		\draw [style=simple] (3.center) to (2);
	\end{pgfonlayer}
\end{tikzpicture}
\hspace*{.5cm}
\begin{tikzpicture}
	\begin{pgfonlayer}{nodelayer}
		\node [style=Z] (4) at (0, 1) {};
		\node [style=none] (5) at (0, 0) {};
	\end{pgfonlayer}
	\begin{pgfonlayer}{edgelayer}
		\draw [style=simple] (5.center) to (4);
	\end{pgfonlayer}
\end{tikzpicture}
\mapsto
\begin{tikzpicture}
	\begin{pgfonlayer}{nodelayer}
		\node [style=Z] (5) at (0, 1) {};
		\node [style=none] (6) at (0, 0) {};
	\end{pgfonlayer}
	\begin{pgfonlayer}{edgelayer}
		\draw [style=simple] (6.center) to (5);
	\end{pgfonlayer}
\end{tikzpicture}
$
\end{tabular}
\end{center}

This interepretation is a strict symmetric \dag-monoidal functor.
\end{proposition}

\begin{proof}
First, observe:
\begin{align*}
\left\llbracket
\begin{tikzpicture}
	\begin{pgfonlayer}{nodelayer}
		\node [style=dot] (6) at (0, 7) {};
		\node [style=oplus] (7) at (0.5, 7) {};
		\node [style=none] (8) at (0.5, 7.5) {};
		\node [style=none] (9) at (0, 7.5) {};
		\node [style=none] (10) at (0, 6.5) {};
		\node [style=none] (11) at (0.5, 6.5) {};
	\end{pgfonlayer}
	\begin{pgfonlayer}{edgelayer}
		\draw (7) to (8.center);
		\draw (7) to (11.center);
		\draw (7) to (6);
		\draw (6) to (9.center);
		\draw (6) to (10.center);
	\end{pgfonlayer}
\end{tikzpicture}
\right\rrbracket_{\hat{\TOF}}
&=
\begin{tikzpicture}
	\begin{pgfonlayer}{nodelayer}
		\node [style=none] (7) at (1, -0.5) {};
		\node [style=none] (8) at (1.5, -0.5) {};
		\node [style=X] (9) at (1.5, 2) {};
		\node [style=Z] (10) at (1, 0.5) {};
		\node [style=Z] (11) at (0, 0.5) {};
		\node [style=none] (12) at (0.5, 1.5) {};
		\node [style=none] (13) at (1.5, 2.75) {};
		\node [style=none] (14) at (1, 2.75) {};
		\node [style=X] (15) at (0, 1.25) {$1$};
		\node [style=X] (16) at (0, -0.25) {$1$};
		\node [style=andin] (17) at (0.5, 1.5) {};
	\end{pgfonlayer}
	\begin{pgfonlayer}{edgelayer}
		\draw [style=simple, in=90, out=180, looseness=0.75] (9) to (12.center);
		\draw [style=simple, in=45, out=-120] (12.center) to (11);
		\draw [style=simple] (7.center) to (10);
		\draw [style=simple] (9) to (8.center);
		\draw [style=simple] (14.center) to (10);
		\draw [style=simple] (13.center) to (9);
		\draw [style=simple, in=-60, out=135] (10) to (12.center);
		\draw (15) to (11);
		\draw (11) to (16);
	\end{pgfonlayer}
\end{tikzpicture}
\eq{\ref{ZXA.14}}
\begin{tikzpicture}
	\begin{pgfonlayer}{nodelayer}
		\node [style=none] (8) at (1.5, 8.25) {};
		\node [style=X] (9) at (0.25, 4.5) {$1$};
		\node [style=none] (10) at (1, 4.25) {};
		\node [style=X] (11) at (1.5, 7.5) {};
		\node [style=Z] (12) at (1, 6) {};
		\node [style=none] (13) at (0.5, 7) {};
		\node [style=none] (14) at (1, 8.25) {};
		\node [style=X] (15) at (0.25, 5.25) {$1$};
		\node [style=none] (16) at (1.5, 4.25) {};
		\node [style=X] (17) at (0, 6) {$1$};
		\node [style=andin] (18) at (0.5, 7) {};
	\end{pgfonlayer}
	\begin{pgfonlayer}{edgelayer}
		\draw [style=simple, in=90, out=180, looseness=0.75] (11) to (13.center);
		\draw [style=simple] (10.center) to (12);
		\draw [style=simple] (11) to (16.center);
		\draw [style=simple] (14.center) to (12);
		\draw [style=simple] (8.center) to (11);
		\draw [style=simple, in=-60, out=135] (12) to (13.center);
		\draw [in=90, out=-135] (13.center) to (17);
		\draw (15) to (9);
	\end{pgfonlayer}
\end{tikzpicture}
\eq{\ref{ZXA.1}}
\begin{tikzpicture}
	\begin{pgfonlayer}{nodelayer}
		\node [style=none] (9) at (1.5, 8.25) {};
		\node [style=X] (10) at (0.5, 5.25) {};
		\node [style=none] (11) at (1, 4.75) {};
		\node [style=X] (12) at (1.5, 7.5) {};
		\node [style=Z] (13) at (1, 6) {};
		\node [style=none] (14) at (0.5, 7) {};
		\node [style=none] (15) at (1, 8.25) {};
		\node [style=none] (16) at (1.5, 4.75) {};
		\node [style=X] (17) at (0, 6) {$1$};
		\node [style=andin] (18) at (0.5, 7) {};
	\end{pgfonlayer}
	\begin{pgfonlayer}{edgelayer}
		\draw [style=simple, in=90, out=180, looseness=0.75] (12) to (14.center);
		\draw [style=simple] (11.center) to (13);
		\draw [style=simple] (12) to (16.center);
		\draw [style=simple] (15.center) to (13);
		\draw [style=simple] (9.center) to (12);
		\draw [style=simple, in=-60, out=135] (13) to (14.center);
		\draw [in=90, out=-135] (14.center) to (17);
	\end{pgfonlayer}
\end{tikzpicture}\\
&
\eq{Lem. \ref{lem:blackdot}, \ref{ZXA.7}}
\begin{tikzpicture}
	\begin{pgfonlayer}{nodelayer}
		\node [style=none] (10) at (1.5, 8.25) {};
		\node [style=none] (11) at (1, 5.25) {};
		\node [style=X] (12) at (1.5, 7.5) {};
		\node [style=Z] (13) at (1, 6) {};
		\node [style=none] (14) at (0.5, 7) {};
		\node [style=none] (15) at (1, 8.25) {};
		\node [style=none] (16) at (1.5, 5.25) {};
		\node [style=X] (17) at (0, 6) {$1$};
		\node [style=andin] (18) at (0.5, 7) {};
	\end{pgfonlayer}
	\begin{pgfonlayer}{edgelayer}
		\draw [style=simple, in=90, out=180, looseness=0.75] (12) to (14.center);
		\draw [style=simple] (11.center) to (13);
		\draw [style=simple] (12) to (16.center);
		\draw [style=simple] (15.center) to (13);
		\draw [style=simple] (10.center) to (12);
		\draw [style=simple, in=-60, out=135] (13) to (14.center);
		\draw [in=90, out=-135] (14.center) to (17);
	\end{pgfonlayer}
\end{tikzpicture}
\eq{\ref{ZXA.10}}
\begin{tikzpicture}
	\begin{pgfonlayer}{nodelayer}
		\node [style=none] (11) at (1.5, 8.25) {};
		\node [style=none] (12) at (1, 5.25) {};
		\node [style=X] (13) at (1.5, 7.5) {};
		\node [style=Z] (14) at (1, 6) {};
		\node [style=none] (15) at (1, 8.25) {};
		\node [style=none] (16) at (1.5, 5.25) {};
	\end{pgfonlayer}
	\begin{pgfonlayer}{edgelayer}
		\draw [style=simple] (12.center) to (14);
		\draw [style=simple] (13) to (16.center);
		\draw [style=simple] (15.center) to (14);
		\draw [style=simple] (11.center) to (13);
		\draw [in=120, out=-135, looseness=1.25] (13) to (14);
	\end{pgfonlayer}
\end{tikzpicture}
\eq{\ref{ZXA.4}}
\begin{tikzpicture}
	\begin{pgfonlayer}{nodelayer}
		\node [style=none] (12) at (1.5, 8.25) {};
		\node [style=none] (13) at (1, 6.75) {};
		\node [style=X] (14) at (1.5, 7.5) {};
		\node [style=Z] (15) at (1, 7.5) {};
		\node [style=none] (16) at (1, 8.25) {};
		\node [style=none] (17) at (1.5, 6.75) {};
	\end{pgfonlayer}
	\begin{pgfonlayer}{edgelayer}
		\draw [style=simple] (13.center) to (15);
		\draw [style=simple] (14) to (17.center);
		\draw [style=simple, in=90, out=-90] (16.center) to (15);
		\draw [style=simple] (12.center) to (14);
		\draw (14) to (15);
	\end{pgfonlayer}
\end{tikzpicture}
\end{align*}

Thus:
\begin{align*}
\left\llbracket
\begin{tikzpicture}
	\begin{pgfonlayer}{nodelayer}
		\node [style=oplus] (13) at (0.5, 7) {};
		\node [style=none] (14) at (0.5, 7.5) {};
		\node [style=none] (15) at (0.5, 6.5) {};
	\end{pgfonlayer}
	\begin{pgfonlayer}{edgelayer}
		\draw (13) to (14.center);
		\draw (13) to (15.center);
	\end{pgfonlayer}
\end{tikzpicture}
\right\rrbracket_{\hat{\TOF}}
&=
\begin{tikzpicture}
	\begin{pgfonlayer}{nodelayer}
		\node [style=none] (14) at (1.5, 1.5) {};
		\node [style=X] (15) at (1.5, 4) {};
		\node [style=Z] (16) at (1, 2.5) {};
		\node [style=Z] (17) at (0, 2.5) {};
		\node [style=none] (18) at (0.5, 3.5) {};
		\node [style=none] (19) at (1.5, 4.75) {};
		\node [style=X] (20) at (0, 3.25) {$1$};
		\node [style=X] (21) at (0, 1.75) {$1$};
		\node [style=X] (22) at (1, 1.75) {$1$};
		\node [style=X] (23) at (1, 3.25) {$1$};
		\node [style=andin] (24) at (0.5, 3.5) {};
	\end{pgfonlayer}
	\begin{pgfonlayer}{edgelayer}
		\draw [style=simple, in=90, out=180, looseness=0.75] (15) to (18.center);
		\draw [style=simple, in=45, out=-120] (18.center) to (17);
		\draw [style=simple] (15) to (14.center);
		\draw [style=simple] (19.center) to (15);
		\draw [style=simple, in=-60, out=135] (16) to (18.center);
		\draw (20) to (17);
		\draw (17) to (21);
		\draw (23) to (16);
		\draw (16) to (22);
	\end{pgfonlayer}
\end{tikzpicture}
=
\begin{tikzpicture}
	\begin{pgfonlayer}{nodelayer}
		\node [style=none] (15) at (1.5, 1.5) {};
		\node [style=X] (16) at (1.5, 2.25) {};
		\node [style=Z] (17) at (1, 2.25) {};
		\node [style=none] (18) at (1.5, 3) {};
		\node [style=X] (19) at (1, 1.75) {$1$};
		\node [style=X] (20) at (1, 2.75) {$1$};
	\end{pgfonlayer}
	\begin{pgfonlayer}{edgelayer}
		\draw [style=simple] (16) to (15.center);
		\draw [style=simple] (18.center) to (16);
		\draw (20) to (17);
		\draw (17) to (19);
		\draw (16) to (17);
	\end{pgfonlayer}
\end{tikzpicture}
\eq{\ref{ZXA.14}}
\begin{tikzpicture}
	\begin{pgfonlayer}{nodelayer}
		\node [style=none] (16) at (1.5, 1.5) {};
		\node [style=X] (17) at (1.5, 2.5) {};
		\node [style=none] (18) at (1.5, 4) {};
		\node [style=X] (19) at (1, 3) {$1$};
		\node [style=X] (20) at (1, 3.75) {$1$};
		\node [style=X] (21) at (1, 2) {$1$};
	\end{pgfonlayer}
	\begin{pgfonlayer}{edgelayer}
		\draw [style=simple] (17) to (16.center);
		\draw [style=simple] (18.center) to (17);
		\draw (20) to (19);
		\draw [in=90, out=180, looseness=1.25] (17) to (21);
	\end{pgfonlayer}
\end{tikzpicture}
\eq{\ref{ZXA.1}}
\begin{tikzpicture}
	\begin{pgfonlayer}{nodelayer}
		\node [style=none] (17) at (1.5, 1.5) {};
		\node [style=X] (18) at (1.5, 2) {$1$};
		\node [style=none] (19) at (1.5, 2.5) {};
		\node [style=X] (20) at (1, 2) {};
	\end{pgfonlayer}
	\begin{pgfonlayer}{edgelayer}
		\draw [style=simple] (18) to (17.center);
		\draw [style=simple] (19.center) to (18);
	\end{pgfonlayer}
\end{tikzpicture}
\eq{Lem. \ref{lem:blackdot}, \ref{ZXA.7}}
\begin{tikzpicture}
	\begin{pgfonlayer}{nodelayer}
		\node [style=none] (18) at (1.5, 2) {};
		\node [style=X] (19) at (1.5, 2.5) {$1$};
		\node [style=none] (20) at (1.5, 3) {};
	\end{pgfonlayer}
	\begin{pgfonlayer}{edgelayer}
		\draw [style=simple] (19) to (18.center);
		\draw [style=simple] (20.center) to (19);
	\end{pgfonlayer}
\end{tikzpicture}
\end{align*}

Thus:
\begin{align*}
\left\llbracket
\begin{tikzpicture}
	\begin{pgfonlayer}{nodelayer}
		\node [style=none] (19) at (1.5, 5.25) {};
		\node [style=none] (20) at (1.5, 8.25) {};
		\node [style=X] (21) at (1.5, 7.5) {};
		\node [style=Z] (22) at (1, 6) {};
		\node [style=X] (23) at (0, 6) {$1$};
		\node [style=andin] (24) at (0.5, 7) {};
		\node [style=none] (25) at (0.5, 7) {};
		\node [style=none] (26) at (1, 5.25) {};
		\node [style=none] (27) at (1, 8.25) {};
	\end{pgfonlayer}
	\begin{pgfonlayer}{edgelayer}
		\draw [style=simple, in=90, out=180, looseness=0.75] (21) to (25.center);
		\draw [style=simple] (26.center) to (22);
		\draw [style=simple] (21) to (19.center);
		\draw [style=simple] (27.center) to (22);
		\draw [style=simple] (20.center) to (21);
		\draw [style=simple, in=-60, out=135] (22) to (25.center);
		\draw [in=90, out=-135] (25.center) to (23);
	\end{pgfonlayer}
\end{tikzpicture}
\right\rrbracket_{\hat{\TOF}}
=
\begin{tikzpicture}
	\begin{pgfonlayer}{nodelayer}
		\node [style=X] (20) at (1.5, 4.25) {};
		\node [style=Z] (21) at (1, 2.75) {};
		\node [style=Z] (22) at (0, 2.75) {};
		\node [style=none] (23) at (0.5, 3.75) {};
		\node [style=none] (24) at (1.5, 5) {};
		\node [style=X] (25) at (0, 3.5) {$1$};
		\node [style=X] (26) at (0, 2) {$1$};
		\node [style=X] (27) at (1, 2) {$1$};
		\node [style=X] (28) at (1, 3.5) {$1$};
		\node [style=andin] (29) at (0.5, 3.75) {};
		\node [style=X] (30) at (1.5, 2) {$1$};
	\end{pgfonlayer}
	\begin{pgfonlayer}{edgelayer}
		\draw [style=simple, in=90, out=180, looseness=0.75] (20) to (23.center);
		\draw [style=simple, in=45, out=-120] (23.center) to (22);
		\draw [style=simple] (24.center) to (20);
		\draw [style=simple, in=-60, out=135] (21) to (23.center);
		\draw (25) to (22);
		\draw (22) to (26);
		\draw (28) to (21);
		\draw (21) to (27);
		\draw (20) to (30);
	\end{pgfonlayer}
\end{tikzpicture}
=
\begin{tikzpicture}
	\begin{pgfonlayer}{nodelayer}
		\node [style=X] (21) at (1.5, 2.75) {$1$};
		\node [style=none] (22) at (1.5, 3.5) {};
		\node [style=X] (23) at (1.5, 2) {$1$};
	\end{pgfonlayer}
	\begin{pgfonlayer}{edgelayer}
		\draw [style=simple] (22.center) to (21);
		\draw (21) to (23);
	\end{pgfonlayer}
\end{tikzpicture}
\eq{\ref{ZXA.1}}
\begin{tikzpicture}
	\begin{pgfonlayer}{nodelayer}
		\node [style=X] (22) at (1.5, 2) {};
		\node [style=none] (23) at (1.5, 2.75) {};
	\end{pgfonlayer}
	\begin{pgfonlayer}{edgelayer}
		\draw [style=simple] (23.center) to (22);
	\end{pgfonlayer}
\end{tikzpicture}
\end{align*}


We prove that all of the axioms of $\hat \TOF$ hold in $\ZXA$ :
\begin{enumerate}
\item[\ref{TOF.1}:]
\begin{align*}
\left\llbracket
\begin{tikzpicture}
	\begin{pgfonlayer}{nodelayer}
		\node [style=nothing] (23) at (0, 2) {};
		\node [style=nothing] (24) at (-0.5, 2) {};
		\node [style=oplus] (25) at (0, 2.5) {};
		\node [style=dot] (26) at (-0.5, 2.5) {};
		\node [style=dot] (27) at (-1, 2.5) {};
		\node [style=onein] (28) at (-1, 2) {};
		\node [style=nothing] (29) at (-1, 3) {};
		\node [style=nothing] (30) at (-0.5, 3) {};
		\node [style=nothing] (31) at (0, 3) {};
	\end{pgfonlayer}
	\begin{pgfonlayer}{edgelayer}
		\draw (28) to (27);
		\draw (27) to (29);
		\draw (30) to (26);
		\draw (24) to (26);
		\draw (23) to (25);
		\draw (25) to (31);
		\draw (25) to (26);
		\draw (26) to (27);
	\end{pgfonlayer}
\end{tikzpicture}
\right\rrbracket_{\hat{\TOF}}
=&
\begin{tikzpicture}
	\begin{pgfonlayer}{nodelayer}
		\node [style=andin] (24) at (-0.5, 4.25) {};
		\node [style=Z] (25) at (-1, 3.5) {};
		\node [style=Z] (26) at (0, 3.5) {};
		\node [style=X] (27) at (0.5, 5) {};
		\node [style=none] (28) at (0.5, 2.5) {};
		\node [style=none] (29) at (0.5, 5.75) {};
		\node [style=none] (30) at (0, 5.75) {};
		\node [style=none] (31) at (-1, 5.75) {};
		\node [style=X] (32) at (-1, 2.75) {$1$};
		\node [style=none] (33) at (0, 2.5) {};
	\end{pgfonlayer}
	\begin{pgfonlayer}{edgelayer}
		\draw (29.center) to (27);
		\draw (27) to (28.center);
		\draw (33.center) to (26);
		\draw [in=90, out=180, looseness=0.75] (27) to (24.center);
		\draw (24.center) to (25);
		\draw (24.center) to (26);
		\draw (25) to (32);
		\draw (25) to (31.center);
		\draw (30.center) to (26);
	\end{pgfonlayer}
\end{tikzpicture}
\eq{\ref{ZXA.14}}
\begin{tikzpicture}
	\begin{pgfonlayer}{nodelayer}
		\node [style=andin] (25) at (-0.5, 4.25) {};
		\node [style=Z] (26) at (0, 3.5) {};
		\node [style=X] (27) at (0.5, 5) {};
		\node [style=none] (28) at (0.5, 2.5) {};
		\node [style=none] (29) at (0.5, 5.75) {};
		\node [style=none] (30) at (0, 5.75) {};
		\node [style=none] (31) at (-1, 5.75) {};
		\node [style=X] (32) at (-1, 3.5) {$1$};
		\node [style=none] (33) at (0, 2.5) {};
		\node [style=X] (34) at (-1, 5) {$1$};
	\end{pgfonlayer}
	\begin{pgfonlayer}{edgelayer}
		\draw (29.center) to (27);
		\draw (27) to (28.center);
		\draw (33.center) to (26);
		\draw [in=90, out=180, looseness=0.75] (27) to (25.center);
		\draw (25.center) to (26);
		\draw (30.center) to (26);
		\draw (31.center) to (34);
		\draw [in=-124, out=90] (32) to (25.center);
	\end{pgfonlayer}
\end{tikzpicture}
\eq{\ref{ZXA.10}}
\begin{tikzpicture}
	\begin{pgfonlayer}{nodelayer}
		\node [style=Z] (26) at (0, 4) {};
		\node [style=X] (27) at (0.5, 5) {};
		\node [style=none] (28) at (0.5, 3.5) {};
		\node [style=none] (29) at (0.5, 5.5) {};
		\node [style=none] (30) at (0, 5.5) {};
		\node [style=none] (31) at (-0.5, 5.5) {};
		\node [style=none] (32) at (0, 3.5) {};
		\node [style=X] (33) at (-0.5, 5) {$1$};
	\end{pgfonlayer}
	\begin{pgfonlayer}{edgelayer}
		\draw (29.center) to (27);
		\draw (27) to (28.center);
		\draw (32.center) to (26);
		\draw (30.center) to (26);
		\draw (31.center) to (33);
		\draw [in=-108, out=120, looseness=1.25] (26) to (27);
	\end{pgfonlayer}
\end{tikzpicture}\\
\eq{\ref{ZXA.3}}&
\begin{tikzpicture}
	\begin{pgfonlayer}{nodelayer}
		\node [style=Z] (27) at (0, 5) {};
		\node [style=X] (28) at (0.5, 5) {};
		\node [style=none] (29) at (0.5, 4.5) {};
		\node [style=none] (30) at (0.5, 5.5) {};
		\node [style=none] (31) at (0, 5.5) {};
		\node [style=none] (32) at (-0.5, 5.5) {};
		\node [style=none] (33) at (0, 4.5) {};
		\node [style=X] (34) at (-0.5, 5) {$1$};
	\end{pgfonlayer}
	\begin{pgfonlayer}{edgelayer}
		\draw (30.center) to (28);
		\draw (28) to (29.center);
		\draw (33.center) to (27);
		\draw (31.center) to (27);
		\draw (32.center) to (34);
		\draw (27) to (28);
	\end{pgfonlayer}
\end{tikzpicture}
=
\left\llbracket
\begin{tikzpicture}
	\begin{pgfonlayer}{nodelayer}
		\node [style=nothing] (28) at (0, 2) {};
		\node [style=nothing] (29) at (-0.5, 2) {};
		\node [style=oplus] (30) at (0, 2.5) {};
		\node [style=dot] (31) at (-0.5, 2.5) {};
		\node [style=onein] (32) at (-1, 2.5) {};
		\node [style=nothing] (33) at (-1, 3) {};
		\node [style=nothing] (34) at (-0.5, 3) {};
		\node [style=nothing] (35) at (0, 3) {};
	\end{pgfonlayer}
	\begin{pgfonlayer}{edgelayer}
		\draw (29) to (31);
		\draw (28) to (30);
		\draw (30) to (31);
		\draw (34) to (31);
		\draw (32) to (33);
		\draw (30) to (35);
	\end{pgfonlayer}
\end{tikzpicture}
\right\rrbracket_{\hat{\TOF}}
\end{align*}

\item[\ref{TOF.2}:]
\begin{align*}
\left\llbracket
\begin{tikzpicture}
	\begin{pgfonlayer}{nodelayer}
		\node [style=nothing] (29) at (-1.25, 2) {};
		\node [style=nothing] (30) at (-0.75, 2) {};
		\node [style=nothing] (31) at (-1.75, 4) {};
		\node [style=nothing] (32) at (-1.25, 4) {};
		\node [style=nothing] (33) at (-0.75, 4) {};
		\node [style=dot] (34) at (-1.75, 3) {};
		\node [style=dot] (35) at (-1.25, 3) {};
		\node [style=oplus] (36) at (-0.75, 3) {};
		\node [style=zeroin] (37) at (-1.75, 2) {};
	\end{pgfonlayer}
	\begin{pgfonlayer}{edgelayer}
		\draw (34) to (31);
		\draw (32) to (35);
		\draw (35) to (29);
		\draw (30) to (36);
		\draw (36) to (33);
		\draw (36) to (35);
		\draw (35) to (34);
		\draw (37) to (34);
	\end{pgfonlayer}
\end{tikzpicture}
\right\rrbracket_{\hat{\TOF}}
&=
\begin{tikzpicture}
	\begin{pgfonlayer}{nodelayer}
		\node [style=andin] (30) at (-0.5, 4.25) {};
		\node [style=Z] (31) at (-1, 3.5) {};
		\node [style=Z] (32) at (0, 3.5) {};
		\node [style=X] (33) at (0.5, 5) {};
		\node [style=none] (34) at (0.5, 2.5) {};
		\node [style=none] (35) at (0.5, 5.75) {};
		\node [style=none] (36) at (0, 5.75) {};
		\node [style=none] (37) at (-1, 5.75) {};
		\node [style=none] (38) at (0, 2.5) {};
		\node [style=X] (39) at (-1, 2.75) {};
	\end{pgfonlayer}
	\begin{pgfonlayer}{edgelayer}
		\draw (35.center) to (33);
		\draw (33) to (34.center);
		\draw (38.center) to (32);
		\draw [in=90, out=180, looseness=0.75] (33) to (30.center);
		\draw (30.center) to (31);
		\draw (30.center) to (32);
		\draw (31) to (37.center);
		\draw (36.center) to (32);
		\draw (31) to (39);
	\end{pgfonlayer}
\end{tikzpicture}
\eq{\ref{ZXA.6}}
\begin{tikzpicture}
	\begin{pgfonlayer}{nodelayer}
		\node [style=andin] (31) at (-0.5, 4.25) {};
		\node [style=Z] (32) at (0, 3.5) {};
		\node [style=X] (33) at (0.5, 5) {};
		\node [style=none] (34) at (0.5, 3) {};
		\node [style=none] (35) at (0.5, 5.5) {};
		\node [style=none] (36) at (0, 5.5) {};
		\node [style=none] (37) at (-1, 5.5) {};
		\node [style=none] (38) at (0, 3) {};
		\node [style=X] (39) at (-1, 3.5) {};
		\node [style=X] (40) at (-1, 5) {};
	\end{pgfonlayer}
	\begin{pgfonlayer}{edgelayer}
		\draw (35.center) to (33);
		\draw (33) to (34.center);
		\draw (38.center) to (32);
		\draw [in=90, out=180, looseness=0.75] (33) to (31.center);
		\draw (31.center) to (32);
		\draw (36.center) to (32);
		\draw [in=90, out=-124] (31.center) to (39);
		\draw (37.center) to (40);
	\end{pgfonlayer}
\end{tikzpicture}
\eq{Lem. \ref{lem:oldaxiom}}
\begin{tikzpicture}
	\begin{pgfonlayer}{nodelayer}
		\node [style=Z] (32) at (0, 4.25) {};
		\node [style=X] (33) at (0.5, 6.25) {};
		\node [style=none] (34) at (0.5, 3.75) {};
		\node [style=none] (35) at (0.5, 6.75) {};
		\node [style=none] (36) at (0, 6.75) {};
		\node [style=none] (37) at (-0.5, 6.75) {};
		\node [style=none] (38) at (0, 3.75) {};
		\node [style=X] (39) at (-0.5, 6.25) {};
		\node [style=X] (40) at (-0.5, 5.5) {};
		\node [style=Z] (41) at (-0.5, 5) {};
	\end{pgfonlayer}
	\begin{pgfonlayer}{edgelayer}
		\draw (35.center) to (33);
		\draw (33) to (34.center);
		\draw (38.center) to (32);
		\draw (36.center) to (32);
		\draw (37.center) to (39);
		\draw [in=90, out=-143, looseness=0.75] (33) to (40);
		\draw [in=-90, out=124] (32) to (41);
	\end{pgfonlayer}
\end{tikzpicture}
\eq{\ref{ZXA.1}}
\begin{tikzpicture}
	\begin{pgfonlayer}{nodelayer}
		\node [style=Z] (33) at (0, 4.25) {};
		\node [style=none] (34) at (0.5, 3.75) {};
		\node [style=none] (35) at (0.5, 6) {};
		\node [style=none] (36) at (0, 6) {};
		\node [style=none] (37) at (-0.5, 6) {};
		\node [style=none] (38) at (0, 3.75) {};
		\node [style=X] (39) at (-0.5, 5.5) {};
		\node [style=Z] (40) at (-0.5, 5) {};
	\end{pgfonlayer}
	\begin{pgfonlayer}{edgelayer}
		\draw (38.center) to (33);
		\draw (36.center) to (33);
		\draw (37.center) to (39);
		\draw [in=-90, out=124] (33) to (40);
		\draw (35.center) to (34.center);
	\end{pgfonlayer}
\end{tikzpicture}\\
&\eq{\ref{ZXA.3}}
\begin{tikzpicture}
	\begin{pgfonlayer}{nodelayer}
		\node [style=none] (34) at (0.5, 3.75) {};
		\node [style=none] (35) at (0.5, 4.75) {};
		\node [style=none] (36) at (0, 4.75) {};
		\node [style=none] (37) at (-0.5, 4.75) {};
		\node [style=none] (38) at (0, 3.75) {};
		\node [style=X] (39) at (-0.5, 4.25) {};
	\end{pgfonlayer}
	\begin{pgfonlayer}{edgelayer}
		\draw (37.center) to (39);
		\draw (35.center) to (34.center);
		\draw (36.center) to (38.center);
	\end{pgfonlayer}
\end{tikzpicture}
=
\left\llbracket
\begin{tikzpicture}
	\begin{pgfonlayer}{nodelayer}
		\node [style=nothing] (35) at (-1.25, 3.75) {};
		\node [style=nothing] (36) at (-0.75, 3.75) {};
		\node [style=nothing] (37) at (-1.75, 5.25) {};
		\node [style=nothing] (38) at (-1.25, 5.25) {};
		\node [style=nothing] (39) at (-0.75, 5.25) {};
		\node [style=zeroin] (40) at (-1.75, 3.75) {};
	\end{pgfonlayer}
	\begin{pgfonlayer}{edgelayer}
		\draw (40) to (37);
		\draw (35) to (38);
		\draw (36) to (39);
	\end{pgfonlayer}
\end{tikzpicture}
\right\rrbracket_{\hat{\TOF}}
\end{align*}

\item[\ref{TOF.3}:]
This follows from the spider law.

\item[\ref{TOF.4}:]
This follows from the spider law.

\item[\ref{TOF.5}:]
This follows from the spider law.

\item[\ref{TOF.6}:]
This follows from the spider law.

\item[\ref{TOF.7}:]
\begin{align*}
\left\llbracket
\begin{tikzpicture}
	\begin{pgfonlayer}{nodelayer}
		\node [style=nothing] (36) at (0, 3.75) {};
		\node [style=nothing] (37) at (-0.5, 3.75) {};
		\node [style=nothing] (38) at (-0.5, 6.25) {};
		\node [style=nothing] (39) at (0, 6.25) {};
		\node [style=zeroout] (40) at (0.5, 6.25) {};
		\node [style=oplus] (41) at (0.5, 5.75) {};
		\node [style=dot] (42) at (0, 5.75) {};
		\node [style=dot] (43) at (-0.5, 4.25) {};
		\node [style=oplus] (44) at (0.5, 4.25) {};
		\node [style=zeroout] (45) at (0.5, 4.75) {};
		\node [style=onein] (46) at (0.5, 3.75) {};
		\node [style=onein] (47) at (0.5, 5.25) {};
	\end{pgfonlayer}
	\begin{pgfonlayer}{edgelayer}
		\draw (37) to (43);
		\draw (43) to (38);
		\draw (39) to (42);
		\draw (42) to (36);
		\draw (44) to (45);
		\draw (44) to (43);
		\draw (41) to (40);
		\draw (41) to (42);
		\draw (46) to (44);
		\draw (47) to (41);
	\end{pgfonlayer}
\end{tikzpicture}
\right\rrbracket_{\hat{\TOF}}
=&
\begin{tikzpicture}
	\begin{pgfonlayer}{nodelayer}
		\node [style=Z] (37) at (-0.5, 4.5) {};
		\node [style=Z] (38) at (0, 6.25) {};
		\node [style=X] (39) at (0.5, 6.25) {};
		\node [style=X] (40) at (0.5, 4.5) {};
		\node [style=X] (41) at (0.5, 5) {};
		\node [style=X] (42) at (0.5, 6.75) {};
		\node [style=none] (43) at (0, 3.75) {};
		\node [style=none] (44) at (-0.5, 3.75) {};
		\node [style=none] (45) at (0, 7) {};
		\node [style=none] (46) at (-0.5, 7) {};
		\node [style=X] (47) at (0.5, 4) {$1$};
		\node [style=X] (48) at (0.5, 5.75) {$1$};
	\end{pgfonlayer}
	\begin{pgfonlayer}{edgelayer}
		\draw (46.center) to (44.center);
		\draw (43.center) to (45.center);
		\draw (42) to (48);
		\draw (41) to (47);
		\draw (40) to (37);
		\draw (39) to (38);
	\end{pgfonlayer}
\end{tikzpicture}
\eq{\ref{ZXA.1}}
\begin{tikzpicture}
	\begin{pgfonlayer}{nodelayer}
		\node [style=Z] (38) at (-0.5, 4.5) {};
		\node [style=Z] (39) at (0, 5.25) {};
		\node [style=none] (40) at (0, 3.75) {};
		\node [style=none] (41) at (-0.5, 3.75) {};
		\node [style=none] (42) at (0, 6) {};
		\node [style=none] (43) at (-0.5, 6) {};
		\node [style=X] (44) at (0.5, 4.5) {$1$};
		\node [style=X] (45) at (0.5, 5.25) {$1$};
	\end{pgfonlayer}
	\begin{pgfonlayer}{edgelayer}
		\draw (43.center) to (41.center);
		\draw (40.center) to (42.center);
		\draw (44) to (38);
		\draw (45) to (39);
	\end{pgfonlayer}
\end{tikzpicture}
\eq{\ref{ZXA.16}}
\begin{tikzpicture}
	\begin{pgfonlayer}{nodelayer}
		\node [style=Z] (39) at (-0.5, 4.25) {};
		\node [style=Z] (40) at (0, 5) {};
		\node [style=none] (41) at (0, 3.75) {};
		\node [style=none] (42) at (-0.5, 3.75) {};
		\node [style=none] (43) at (0, 6.75) {};
		\node [style=none] (44) at (-0.5, 6.75) {};
		\node [style=X] (45) at (0.5, 6.5) {$1$};
		\node [style=andin] (46) at (0.5, 5.75) {};
		\node [style=none] (47) at (1, 5.25) {};
		\node [style=none] (48) at (1, 4.75) {};
	\end{pgfonlayer}
	\begin{pgfonlayer}{edgelayer}
		\draw (44.center) to (42.center);
		\draw (41.center) to (43.center);
		\draw (45) to (46);
		\draw [in=30, out=-124] (46) to (40);
		\draw [in=90, out=-45] (46) to (47.center);
		\draw (47.center) to (48.center);
		\draw [in=0, out=-90, looseness=0.50] (48.center) to (39);
	\end{pgfonlayer}
\end{tikzpicture}\\
\eq{\ref{ZXA.1}}&
\begin{tikzpicture}
	\begin{pgfonlayer}{nodelayer}
		\node [style=Z] (40) at (-0.75, 4.25) {};
		\node [style=Z] (41) at (0.25, 4.25) {};
		\node [style=none] (42) at (0.25, 3.75) {};
		\node [style=none] (43) at (-0.75, 3.75) {};
		\node [style=none] (44) at (0.25, 6.75) {};
		\node [style=none] (45) at (-0.75, 6.75) {};
		\node [style=andin] (46) at (-0.25, 5.25) {};
		\node [style=X] (47) at (0.75, 5.5) {$1$};
		\node [style=X] (48) at (0.75, 6.5) {};
		\node [style=X] (49) at (0.75, 6) {};
	\end{pgfonlayer}
	\begin{pgfonlayer}{edgelayer}
		\draw (45.center) to (43.center);
		\draw (42.center) to (44.center);
		\draw (46.center) to (41);
		\draw (48) to (49);
		\draw (49) to (47);
		\draw [in=90, out=180, looseness=0.75] (49) to (46.center);
		\draw [in=63, out=-117] (46.center) to (40);
	\end{pgfonlayer}
\end{tikzpicture}
=
\left\llbracket
\begin{tikzpicture}
	\begin{pgfonlayer}{nodelayer}
		\node [style=nothing] (41) at (0, 3.75) {};
		\node [style=nothing] (42) at (-0.5, 3.75) {};
		\node [style=nothing] (43) at (-0.5, 4.75) {};
		\node [style=nothing] (44) at (0, 4.75) {};
		\node [style=dot] (45) at (-0.5, 4.25) {};
		\node [style=dot] (46) at (0, 4.25) {};
		\node [style=onein] (47) at (0.5, 3.75) {};
		\node [style=zeroout] (48) at (0.5, 4.75) {};
		\node [style=oplus] (49) at (0.5, 4.25) {};
	\end{pgfonlayer}
	\begin{pgfonlayer}{edgelayer}
		\draw (42) to (45);
		\draw (45) to (43);
		\draw (44) to (46);
		\draw (46) to (41);
		\draw (47) to (49);
		\draw (49) to (48);
		\draw (49) to (46);
		\draw (46) to (45);
	\end{pgfonlayer}
\end{tikzpicture}
\right\rrbracket_{\hat{\TOF}}
\end{align*}


\item[\ref{TOF.8}:]
This follows immediately from Lemma \ref{lem:blackdot} and \ref{ZXA.7}.

\item[\ref{TOF.9}:]

\begin{align*}
\left\llbracket
\begin{tikzpicture}
	\begin{pgfonlayer}{nodelayer}
		\node [style=nothing] (42) at (-1.75, 3.75) {};
		\node [style=nothing] (43) at (-1.25, 3.75) {};
		\node [style=nothing] (44) at (-0.75, 3.75) {};
		\node [style=dot] (45) at (-1.75, 4.25) {};
		\node [style=dot] (46) at (-1.25, 4.25) {};
		\node [style=oplus] (47) at (-0.75, 4.25) {};
		\node [style=dot] (48) at (-1.75, 4.75) {};
		\node [style=oplus] (49) at (-0.75, 4.75) {};
		\node [style=dot] (50) at (-1.25, 4.75) {};
		\node [style=nothing] (51) at (-1.25, 5.25) {};
		\node [style=nothing] (52) at (-0.75, 5.25) {};
		\node [style=nothing] (53) at (-1.75, 5.25) {};
	\end{pgfonlayer}
	\begin{pgfonlayer}{edgelayer}
		\draw (42) to (45);
		\draw (43) to (46);
		\draw (44) to (47);
		\draw (45) to (46);
		\draw (46) to (47);
		\draw (48) to (50);
		\draw (50) to (49);
		\draw (45) to (48);
		\draw (48) to (53);
		\draw (46) to (50);
		\draw (50) to (51);
		\draw (47) to (49);
		\draw (49) to (52);
	\end{pgfonlayer}
\end{tikzpicture}
\right\rrbracket_{\hat{\TOF}}
&=
\begin{tikzpicture}
	\begin{pgfonlayer}{nodelayer}
		\node [style=Z] (43) at (-0.75, 6.25) {};
		\node [style=Z] (44) at (0.25, 6.25) {};
		\node [style=none] (45) at (0.25, 3.75) {};
		\node [style=none] (46) at (-0.75, 3.75) {};
		\node [style=none] (47) at (0.25, 8.5) {};
		\node [style=none] (48) at (-0.75, 8.5) {};
		\node [style=andin] (49) at (-0.25, 7.25) {};
		\node [style=X] (50) at (0.75, 8) {};
		\node [style=none] (51) at (0.75, 8.5) {};
		\node [style=none] (52) at (0.75, 3.75) {};
		\node [style=Z] (53) at (0.25, 4.25) {};
		\node [style=andin] (54) at (-0.25, 5.25) {};
		\node [style=X] (55) at (0.75, 6) {};
		\node [style=Z] (56) at (-0.75, 4.25) {};
	\end{pgfonlayer}
	\begin{pgfonlayer}{edgelayer}
		\draw (48.center) to (46.center);
		\draw (45.center) to (47.center);
		\draw (49.center) to (44);
		\draw [in=90, out=180, looseness=0.75] (50) to (49.center);
		\draw [in=63, out=-117] (49.center) to (43);
		\draw (54.center) to (53);
		\draw [in=90, out=180, looseness=0.75] (55) to (54.center);
		\draw [in=63, out=-117] (54.center) to (56);
		\draw (51.center) to (52.center);
	\end{pgfonlayer}
\end{tikzpicture}
\eq{\ref{ZXA.3}}
\begin{tikzpicture}
	\begin{pgfonlayer}{nodelayer}
		\node [style=Z] (1) at (-0.75, 8.25) {};
		\node [style=Z] (2) at (0.25, 8.25) {};
		\node [style=none] (3) at (0.75, 6.5) {};
		\node [style=none] (4) at (-1.25, 6.5) {};
		\node [style=none] (5) at (0.75, 10) {};
		\node [style=none] (6) at (-1.25, 10) {};
		\node [style=andin] (7) at (-0.25, 9.25) {};
		\node [style=X] (8) at (1.25, 8.25) {};
		\node [style=none] (9) at (1.25, 10) {};
		\node [style=none] (10) at (1.25, 6.5) {};
		\node [style=andout] (11) at (-0.25, 7.25) {};
		\node [style=Z] (12) at (-1.25, 8.25) {};
		\node [style=Z] (13) at (0.75, 8.25) {};
	\end{pgfonlayer}
	\begin{pgfonlayer}{edgelayer}
		\draw (7.center) to (2);
		\draw [in=90, out=105, looseness=1.50] (8) to (7.center);
		\draw [in=63, out=-117] (7.center) to (1);
		\draw (9.center) to (10.center);
		\draw (2) to (11.center);
		\draw (11.center) to (1);
		\draw [in=-105, out=-90, looseness=1.75] (11.center) to (8);
		\draw (5.center) to (13);
		\draw (13) to (3.center);
		\draw (13) to (2);
		\draw (1) to (12);
		\draw (12) to (6.center);
		\draw (12) to (4.center);
	\end{pgfonlayer}
\end{tikzpicture}
=
\begin{tikzpicture}
	\begin{pgfonlayer}{nodelayer}
		\node [style=Z] (2) at (-1, 9.25) {};
		\node [style=Z] (3) at (-0.25, 9.25) {};
		\node [style=none] (4) at (0.25, 6.5) {};
		\node [style=none] (5) at (-1.5, 6.5) {};
		\node [style=none] (6) at (0.25, 10.5) {};
		\node [style=none] (7) at (-1.5, 10.5) {};
		\node [style=none] (8) at (-1, 8.25) {};
		\node [style=X] (9) at (0.75, 7) {};
		\node [style=none] (10) at (0.75, 10.5) {};
		\node [style=none] (11) at (0.75, 6.5) {};
		\node [style=none] (12) at (-0.25, 8.25) {};
		\node [style=Z] (13) at (-1.5, 9.75) {};
		\node [style=Z] (14) at (0.25, 9.75) {};
		\node [style=none] (15) at (-1, 7.5) {};
		\node [style=none] (16) at (-0.25, 7.75) {};
		\node [style=andout] (17) at (-1, 8.25) {};
		\node [style=andout] (18) at (-0.25, 8.25) {};
	\end{pgfonlayer}
	\begin{pgfonlayer}{edgelayer}
		\draw (8.center) to (3);
		\draw [in=-120, out=120, looseness=1.25] (8.center) to (2);
		\draw (10.center) to (11.center);
		\draw [in=60, out=-60, looseness=1.25] (3) to (12.center);
		\draw (12.center) to (2);
		\draw (6.center) to (14);
		\draw (14) to (4.center);
		\draw [in=90, out=180, looseness=1.75] (14) to (3);
		\draw [in=0, out=90, looseness=1.75] (2) to (13);
		\draw (13) to (7.center);
		\draw (13) to (5.center);
		\draw [in=-90, out=153, looseness=0.75] (9) to (16.center);
		\draw [in=-90, out=180] (9) to (15.center);
		\draw (15.center) to (8.center);
		\draw (16.center) to (12.center);
	\end{pgfonlayer}
\end{tikzpicture}
\eq{\ref{ZXA.12}}
\begin{tikzpicture}
	\begin{pgfonlayer}{nodelayer}
		\node [style=none] (3) at (0, 6.5) {};
		\node [style=none] (4) at (-1, 6.5) {};
		\node [style=none] (5) at (0, 9) {};
		\node [style=none] (6) at (-1, 9) {};
		\node [style=X] (7) at (0.5, 7.25) {};
		\node [style=none] (8) at (0.5, 9) {};
		\node [style=none] (9) at (0.5, 6.5) {};
		\node [style=Z] (10) at (-1, 8.5) {};
		\node [style=Z] (11) at (0, 8.5) {};
		\node [style=Z] (12) at (-0.5, 7.25) {};
		\node [style=none] (13) at (-0.5, 8) {};
		\node [style=andout] (14) at (-0.5, 8) {};
	\end{pgfonlayer}
	\begin{pgfonlayer}{edgelayer}
		\draw (8.center) to (9.center);
		\draw (5.center) to (11);
		\draw (11) to (3.center);
		\draw (10) to (6.center);
		\draw (10) to (4.center);
		\draw [bend right] (7) to (12);
		\draw [bend right] (12) to (7);
		\draw (11) to (13.center);
		\draw (13.center) to (10);
		\draw (13.center) to (12);
	\end{pgfonlayer}
\end{tikzpicture}\\
&\eq{\ref{ZXA.8}}
\begin{tikzpicture}
	\begin{pgfonlayer}{nodelayer}
		\node [style=none] (4) at (0, 6.5) {};
		\node [style=none] (5) at (-1, 6.5) {};
		\node [style=none] (6) at (0, 9) {};
		\node [style=none] (7) at (-1, 9) {};
		\node [style=X] (8) at (0.5, 7.25) {};
		\node [style=none] (9) at (0.5, 9) {};
		\node [style=none] (10) at (0.5, 6.5) {};
		\node [style=Z] (11) at (-1, 8.5) {};
		\node [style=Z] (12) at (0, 8.5) {};
		\node [style=Z] (13) at (-0.5, 7.25) {};
		\node [style=none] (14) at (-0.5, 8) {};
		\node [style=andout] (15) at (-0.5, 8) {};
	\end{pgfonlayer}
	\begin{pgfonlayer}{edgelayer}
		\draw (9.center) to (10.center);
		\draw (6.center) to (12);
		\draw (12) to (4.center);
		\draw (11) to (7.center);
		\draw (11) to (5.center);
		\draw (12) to (14.center);
		\draw (14.center) to (11);
		\draw (14.center) to (13);
	\end{pgfonlayer}
\end{tikzpicture}
\eq{\ref{ZXA.1}}
\begin{tikzpicture}
	\begin{pgfonlayer}{nodelayer}
		\node [style=none] (0) at (0, 3) {};
		\node [style=none] (1) at (-1, 3) {};
		\node [style=none] (2) at (0, 5.5) {};
		\node [style=none] (3) at (-1, 5.5) {};
		\node [style=none] (4) at (0.5, 5.5) {};
		\node [style=none] (5) at (0.5, 3) {};
		\node [style=Z] (6) at (-1, 5) {};
		\node [style=Z] (7) at (0, 5) {};
		\node [style=Z] (8) at (-0.5, 3.75) {};
		\node [style=none] (9) at (-0.5, 4.5) {};
		\node [style=andout] (10) at (-0.5, 4.5) {};
	\end{pgfonlayer}
	\begin{pgfonlayer}{edgelayer}
		\draw (4.center) to (5.center);
		\draw (2.center) to (7);
		\draw (7) to (0.center);
		\draw (6) to (3.center);
		\draw (6) to (1.center);
		\draw (7) to (9.center);
		\draw (9.center) to (6);
		\draw (9.center) to (8);
	\end{pgfonlayer}
\end{tikzpicture}\\
&\eq{\ref{ZXA.13}}
\begin{tikzpicture}
	\begin{pgfonlayer}{nodelayer}
		\node [style=none] (1) at (0, 3.75) {};
		\node [style=none] (2) at (-1.5, 3.75) {};
		\node [style=none] (3) at (0, 5.5) {};
		\node [style=none] (4) at (-1.5, 5.5) {};
		\node [style=none] (5) at (0.5, 5.5) {};
		\node [style=none] (6) at (0.5, 3.75) {};
		\node [style=Z] (7) at (-1.5, 5) {};
		\node [style=Z] (8) at (0, 5) {};
		\node [style=Z] (9) at (-1, 4.25) {};
		\node [style=Z] (10) at (-0.5, 4.25) {};
	\end{pgfonlayer}
	\begin{pgfonlayer}{edgelayer}
		\draw (5.center) to (6.center);
		\draw (3.center) to (8);
		\draw (8) to (1.center);
		\draw (7) to (4.center);
		\draw (7) to (2.center);
		\draw [in=90, out=-124] (8) to (10);
		\draw [in=-56, out=90] (9) to (7);
	\end{pgfonlayer}
\end{tikzpicture}
\eq{\ref{ZXA.3}}
\begin{tikzpicture}
	\begin{pgfonlayer}{nodelayer}
		\node [style=nothing] (2) at (-1.75, 0) {};
		\node [style=nothing] (3) at (-1.25, 0) {};
		\node [style=nothing] (4) at (-0.75, 0) {};
		\node [style=nothing] (5) at (-1.25, 1.5) {};
		\node [style=nothing] (6) at (-0.75, 1.5) {};
		\node [style=nothing] (7) at (-1.75, 1.5) {};
	\end{pgfonlayer}
	\begin{pgfonlayer}{edgelayer}
		\draw (2) to (7);
		\draw (3) to (5);
		\draw (4) to (6);
	\end{pgfonlayer}
\end{tikzpicture}
=
\left\llbracket
\begin{tikzpicture}
	\begin{pgfonlayer}{nodelayer}
		\node [style=nothing] (3) at (-1.75, 0) {};
		\node [style=nothing] (4) at (-1.25, 0) {};
		\node [style=nothing] (5) at (-0.75, 0) {};
		\node [style=nothing] (6) at (-1.25, 1.5) {};
		\node [style=nothing] (7) at (-0.75, 1.5) {};
		\node [style=nothing] (8) at (-1.75, 1.5) {};
	\end{pgfonlayer}
	\begin{pgfonlayer}{edgelayer}
		\draw (3) to (8);
		\draw (4) to (6);
		\draw (5) to (7);
	\end{pgfonlayer}
\end{tikzpicture}
\right\rrbracket_{\hat{\TOF}}
\end{align*}


\item[\ref{TOF.10}:]  It is easier to prove that $\ref{TOF.10}$ is redundant.  Given \ref{TOF.9},  \ref{TOF.6} and \ref{TOF.12}, \ref{TOF.10} is equivalent to the following:
$$
\begin{tikzpicture}
	\begin{pgfonlayer}{nodelayer}
		\node [style=dot] (4) at (0, 3) {};
		\node [style=dot] (5) at (0.5, 3) {};
		\node [style=dot] (6) at (-0.5, 3.5) {};
		\node [style=dot] (7) at (0, 3.5) {};
		\node [style=dot] (8) at (0, 4) {};
		\node [style=dot] (9) at (0.5, 4) {};
		\node [style=dot] (10) at (-0.5, 4.5) {};
		\node [style=dot] (11) at (0, 4.5) {};
		\node [style=oplus] (12) at (1, 3) {};
		\node [style=oplus] (13) at (0.5, 3.5) {};
		\node [style=oplus] (14) at (1, 4) {};
		\node [style=oplus] (15) at (0.5, 4.5) {};
		\node [style=none] (16) at (1, 2.5) {};
		\node [style=none] (17) at (0.5, 2.5) {};
		\node [style=none] (18) at (0, 2.5) {};
		\node [style=none] (19) at (-0.5, 2.5) {};
		\node [style=none] (20) at (-0.5, 5) {};
		\node [style=none] (21) at (0, 5) {};
		\node [style=none] (22) at (0.5, 5) {};
		\node [style=none] (23) at (1, 5) {};
	\end{pgfonlayer}
	\begin{pgfonlayer}{edgelayer}
		\draw (16.center) to (23.center);
		\draw (22.center) to (17.center);
		\draw (18.center) to (21.center);
		\draw (20.center) to (19.center);
		\draw (4) to (12);
		\draw (13) to (6);
		\draw (8) to (14);
		\draw (15) to (10);
	\end{pgfonlayer}
\end{tikzpicture}
\eq{\ref{TOF.10}}
\begin{tikzpicture}
	\begin{pgfonlayer}{nodelayer}
		\node [style=dot] (5) at (-0.5, 3.5) {};
		\node [style=dot] (6) at (0, 3.5) {};
		\node [style=dot] (7) at (-0.5, 4) {};
		\node [style=dot] (8) at (0, 4) {};
		\node [style=dot] (9) at (-0.5, 4.5) {};
		\node [style=dot] (10) at (0, 4.5) {};
		\node [style=oplus] (11) at (1, 3.5) {};
		\node [style=oplus] (12) at (0.5, 4) {};
		\node [style=oplus] (13) at (0.5, 4.5) {};
		\node [style=none] (14) at (1, 3) {};
		\node [style=none] (15) at (0.5, 3) {};
		\node [style=none] (16) at (0, 3) {};
		\node [style=none] (17) at (-0.5, 3) {};
		\node [style=none] (18) at (-0.5, 5) {};
		\node [style=none] (19) at (0, 5) {};
		\node [style=none] (20) at (0.5, 5) {};
		\node [style=none] (21) at (1, 5) {};
	\end{pgfonlayer}
	\begin{pgfonlayer}{edgelayer}
		\draw (14.center) to (21.center);
		\draw (20.center) to (15.center);
		\draw (16.center) to (19.center);
		\draw (18.center) to (17.center);
		\draw (11) to (5);
		\draw (7) to (12);
		\draw (13) to (9);
	\end{pgfonlayer}
\end{tikzpicture}
\eq{\ref{TOF.9}}
\begin{tikzpicture}
	\begin{pgfonlayer}{nodelayer}
		\node [style=dot] (6) at (-0.5, 3) {};
		\node [style=dot] (7) at (0, 3) {};
		\node [style=oplus] (8) at (1, 3) {};
		\node [style=none] (9) at (1, 2.5) {};
		\node [style=none] (10) at (0.5, 2.5) {};
		\node [style=none] (11) at (0, 2.5) {};
		\node [style=none] (12) at (-0.5, 2.5) {};
		\node [style=none] (13) at (-0.5, 3.5) {};
		\node [style=none] (14) at (0, 3.5) {};
		\node [style=none] (15) at (0.5, 3.5) {};
		\node [style=none] (16) at (1, 3.5) {};
	\end{pgfonlayer}
	\begin{pgfonlayer}{edgelayer}
		\draw (9.center) to (16.center);
		\draw (15.center) to (10.center);
		\draw (11.center) to (14.center);
		\draw (13.center) to (12.center);
		\draw (6) to (8);
	\end{pgfonlayer}
\end{tikzpicture}
$$

However
$$
\begin{tikzpicture}
	\begin{pgfonlayer}{nodelayer}
		\node [style=dot] (7) at (0, 3) {};
		\node [style=dot] (8) at (0.5, 3) {};
		\node [style=dot] (9) at (-0.5, 3.5) {};
		\node [style=dot] (10) at (0, 3.5) {};
		\node [style=dot] (11) at (0, 4) {};
		\node [style=dot] (12) at (0.5, 4) {};
		\node [style=dot] (13) at (-0.5, 4.5) {};
		\node [style=dot] (14) at (0, 4.5) {};
		\node [style=oplus] (15) at (1, 3) {};
		\node [style=oplus] (16) at (0.5, 3.5) {};
		\node [style=oplus] (17) at (1, 4) {};
		\node [style=oplus] (18) at (0.5, 4.5) {};
		\node [style=none] (19) at (1, 2.5) {};
		\node [style=none] (20) at (0.5, 2.5) {};
		\node [style=none] (21) at (0, 2.5) {};
		\node [style=none] (22) at (-0.5, 2.5) {};
		\node [style=none] (23) at (-0.5, 5) {};
		\node [style=none] (24) at (0, 5) {};
		\node [style=none] (25) at (0.5, 5) {};
		\node [style=none] (26) at (1, 5) {};
	\end{pgfonlayer}
	\begin{pgfonlayer}{edgelayer}
		\draw (19.center) to (26.center);
		\draw (25.center) to (20.center);
		\draw (21.center) to (24.center);
		\draw (23.center) to (22.center);
		\draw (7) to (15);
		\draw (16) to (9);
		\draw (11) to (17);
		\draw (18) to (13);
	\end{pgfonlayer}
\end{tikzpicture}
\eq{\ref{TOF.12}}
\begin{tikzpicture}
	\begin{pgfonlayer}{nodelayer}
		\node [style=dot] (8) at (0, 3) {};
		\node [style=dot] (9) at (0.5, 3) {};
		\node [style=dot] (10) at (-0.5, 3.5) {};
		\node [style=dot] (11) at (0, 3.5) {};
		\node [style=dot] (12) at (0, 4) {};
		\node [style=dot] (13) at (0.5, 4) {};
		\node [style=oplus] (14) at (1, 3) {};
		\node [style=oplus] (15) at (1, 3.5) {};
		\node [style=oplus] (16) at (1, 4) {};
		\node [style=none] (17) at (1, 2.5) {};
		\node [style=none] (18) at (0.5, 2.5) {};
		\node [style=none] (19) at (0, 2.5) {};
		\node [style=none] (20) at (-0.5, 2.5) {};
		\node [style=none] (21) at (-0.5, 4.5) {};
		\node [style=none] (22) at (0, 4.5) {};
		\node [style=none] (23) at (0.5, 4.5) {};
		\node [style=none] (24) at (1, 4.5) {};
	\end{pgfonlayer}
	\begin{pgfonlayer}{edgelayer}
		\draw (17.center) to (24.center);
		\draw (23.center) to (18.center);
		\draw (19.center) to (22.center);
		\draw (21.center) to (20.center);
		\draw (8) to (14);
		\draw (15) to (10);
		\draw (12) to (16);
	\end{pgfonlayer}
\end{tikzpicture}
\eq{\ref{TOF.6}}
\begin{tikzpicture}
	\begin{pgfonlayer}{nodelayer}
		\node [style=dot] (9) at (0, 3) {};
		\node [style=dot] (10) at (0.5, 3) {};
		\node [style=dot] (11) at (-0.5, 3.5) {};
		\node [style=dot] (12) at (0, 3.5) {};
		\node [style=dot] (13) at (0, 4) {};
		\node [style=dot] (14) at (0.5, 4) {};
		\node [style=oplus] (15) at (1, 3) {};
		\node [style=oplus] (16) at (1, 3.5) {};
		\node [style=oplus] (17) at (1, 4) {};
		\node [style=none] (18) at (1, 2.5) {};
		\node [style=none] (19) at (0.5, 2.5) {};
		\node [style=none] (20) at (0, 2.5) {};
		\node [style=none] (21) at (-0.5, 2.5) {};
		\node [style=none] (22) at (-0.5, 4.5) {};
		\node [style=none] (23) at (0, 4.5) {};
		\node [style=none] (24) at (0.5, 4.5) {};
		\node [style=none] (25) at (1, 4.5) {};
	\end{pgfonlayer}
	\begin{pgfonlayer}{edgelayer}
		\draw (18.center) to (25.center);
		\draw (24.center) to (19.center);
		\draw (20.center) to (23.center);
		\draw (22.center) to (21.center);
		\draw (9) to (15);
		\draw (16) to (11);
		\draw (13) to (17);
	\end{pgfonlayer}
\end{tikzpicture}
\eq{\ref{TOF.9}}
\begin{tikzpicture}
	\begin{pgfonlayer}{nodelayer}
		\node [style=dot] (10) at (-0.5, 4) {};
		\node [style=dot] (11) at (0, 4) {};
		\node [style=oplus] (12) at (1, 4) {};
		\node [style=none] (13) at (1, 3.5) {};
		\node [style=none] (14) at (0.5, 3.5) {};
		\node [style=none] (15) at (0, 3.5) {};
		\node [style=none] (16) at (-0.5, 3.5) {};
		\node [style=none] (17) at (-0.5, 4.5) {};
		\node [style=none] (18) at (0, 4.5) {};
		\node [style=none] (19) at (0.5, 4.5) {};
		\node [style=none] (20) at (1, 4.5) {};
	\end{pgfonlayer}
	\begin{pgfonlayer}{edgelayer}
		\draw (13.center) to (20.center);
		\draw (19.center) to (14.center);
		\draw (15.center) to (18.center);
		\draw (17.center) to (16.center);
		\draw (12) to (10);
	\end{pgfonlayer}
\end{tikzpicture}
$$

\item[\ref{TOF.11}:]
\begin{align*}
\left\llbracket
\begin{tikzpicture}
	\begin{pgfonlayer}{nodelayer}
		\node [style=nothing] (11) at (0, 2.5) {};
		\node [style=nothing] (12) at (-0.5, 2.5) {};
		\node [style=nothing] (13) at (-1, 2.5) {};
		\node [style=nothing] (14) at (-1.5, 2.5) {};
		\node [style=nothing] (15) at (-0.5, 4.5) {};
		\node [style=nothing] (16) at (0, 4.5) {};
		\node [style=dot] (17) at (-1.5, 3) {};
		\node [style=dot] (18) at (-1, 3.5) {};
		\node [style=dot] (19) at (-0.5, 3.5) {};
		\node [style=oplus] (20) at (-1, 3) {};
		\node [style=oplus] (21) at (0, 3.5) {};
		\node [style=nothing] (22) at (-1.5, 4.5) {};
		\node [style=nothing] (23) at (-1, 4.5) {};
		\node [style=oplus] (24) at (-1, 4) {};
		\node [style=dot] (25) at (-1.5, 4) {};
	\end{pgfonlayer}
	\begin{pgfonlayer}{edgelayer}
		\draw (17) to (20);
		\draw (18) to (19);
		\draw (19) to (21);
		\draw (11) to (21);
		\draw (21) to (16);
		\draw (15) to (19);
		\draw (19) to (12);
		\draw (13) to (20);
		\draw (20) to (18);
		\draw (17) to (14);
		\draw (17) to (25);
		\draw (25) to (22);
		\draw (23) to (24);
		\draw (24) to (18);
		\draw (24) to (25);
	\end{pgfonlayer}
\end{tikzpicture}
\right\rrbracket_{\hat{\TOF}}
=&
\begin{tikzpicture}
	\begin{pgfonlayer}{nodelayer}
		\node [style=none] (12) at (-0.75, 4.25) {};
		\node [style=Z] (13) at (-1.25, 3.5) {};
		\node [style=Z] (14) at (-0.25, 3.5) {};
		\node [style=X] (15) at (0.25, 4.75) {};
		\node [style=X] (16) at (-1.25, 3) {};
		\node [style=X] (17) at (-1.25, 4.75) {};
		\node [style=Z] (18) at (-1.75, 3) {};
		\node [style=Z] (19) at (-1.75, 4.75) {};
		\node [style=none] (20) at (0.25, 5.25) {};
		\node [style=none] (21) at (-0.25, 5.25) {};
		\node [style=none] (22) at (-1.75, 2.5) {};
		\node [style=none] (23) at (-1.25, 2.5) {};
		\node [style=none] (24) at (-0.25, 2.5) {};
		\node [style=none] (25) at (0.25, 2.5) {};
		\node [style=none] (26) at (-1.75, 5.25) {};
		\node [style=none] (27) at (-1.25, 5.25) {};
		\node [style=andin] (28) at (-0.75, 4.25) {};
	\end{pgfonlayer}
	\begin{pgfonlayer}{edgelayer}
		\draw (20.center) to (15);
		\draw [in=90, out=180] (15) to (12.center);
		\draw (12.center) to (14);
		\draw (14) to (24.center);
		\draw (25.center) to (15);
		\draw (12.center) to (13);
		\draw (13) to (17);
		\draw (13) to (16);
		\draw (16) to (18);
		\draw (19) to (17);
		\draw (18) to (22.center);
		\draw (23.center) to (16);
		\draw (27.center) to (17);
		\draw (19) to (26.center);
		\draw (19) to (18);
		\draw (21.center) to (14);
	\end{pgfonlayer}
\end{tikzpicture}
\eq{\ref{ZXA.3}}
\begin{tikzpicture}
	\begin{pgfonlayer}{nodelayer}
		\node [style=none] (13) at (-0.25, 4.5) {};
		\node [style=Z] (14) at (-0.75, 3.75) {};
		\node [style=Z] (15) at (0.25, 3.75) {};
		\node [style=X] (16) at (0.75, 5) {};
		\node [style=X] (17) at (-1.25, 3) {};
		\node [style=X] (18) at (-1.25, 4.5) {};
		\node [style=Z] (19) at (-1.75, 3.75) {};
		\node [style=none] (20) at (0.75, 5.5) {};
		\node [style=none] (21) at (0.25, 5.5) {};
		\node [style=none] (22) at (-2.25, 2.5) {};
		\node [style=none] (23) at (-1.25, 2.5) {};
		\node [style=none] (24) at (0.25, 2.5) {};
		\node [style=none] (25) at (0.75, 2.5) {};
		\node [style=none] (26) at (-2.25, 5.5) {};
		\node [style=none] (27) at (-1.25, 5.5) {};
		\node [style=andin] (28) at (-0.25, 4.5) {};
		\node [style=Z] (29) at (-2.25, 3.75) {};
	\end{pgfonlayer}
	\begin{pgfonlayer}{edgelayer}
		\draw (20.center) to (16);
		\draw [in=90, out=180] (16) to (13.center);
		\draw (13.center) to (15);
		\draw (15) to (24.center);
		\draw (25.center) to (16);
		\draw (13.center) to (14);
		\draw (17) to (19);
		\draw (23.center) to (17);
		\draw (27.center) to (18);
		\draw (21.center) to (15);
		\draw (18) to (19);
		\draw (19) to (29);
		\draw (29) to (26.center);
		\draw (29) to (22.center);
		\draw (14) to (18);
		\draw (14) to (17);
	\end{pgfonlayer}
\end{tikzpicture}
=
\begin{tikzpicture}
	\begin{pgfonlayer}{nodelayer}
		\node [style=none] (14) at (-0.25, 4.5) {};
		\node [style=Z] (15) at (-0.75, 3.75) {};
		\node [style=Z] (16) at (0.25, 3.75) {};
		\node [style=X] (17) at (0.75, 5) {};
		\node [style=X] (18) at (-1.25, 3) {};
		\node [style=X] (19) at (-1.75, 3.75) {};
		\node [style=Z] (20) at (-1.25, 4.5) {};
		\node [style=none] (21) at (0.75, 5.5) {};
		\node [style=none] (22) at (0.25, 5.5) {};
		\node [style=none] (23) at (-2.25, 2.5) {};
		\node [style=none] (24) at (-1.25, 2.5) {};
		\node [style=none] (25) at (0.25, 2.5) {};
		\node [style=none] (26) at (0.75, 2.5) {};
		\node [style=none] (27) at (-2.25, 5.5) {};
		\node [style=none] (28) at (-1.75, 5.5) {};
		\node [style=andin] (29) at (-0.25, 4.5) {};
		\node [style=Z] (30) at (-2.25, 5) {};
	\end{pgfonlayer}
	\begin{pgfonlayer}{edgelayer}
		\draw (21.center) to (17);
		\draw [in=90, out=180] (17) to (14.center);
		\draw (14.center) to (16);
		\draw (16) to (25.center);
		\draw (26.center) to (17);
		\draw (14.center) to (15);
		\draw (18) to (20);
		\draw (24.center) to (18);
		\draw [in=90, out=-90] (28.center) to (19);
		\draw (22.center) to (16);
		\draw (19) to (20);
		\draw (20) to (30);
		\draw (30) to (27.center);
		\draw (30) to (23.center);
		\draw (15) to (19);
		\draw (15) to (18);
	\end{pgfonlayer}
\end{tikzpicture}\\
\eq{\ref{ZXA.5}}&
\begin{tikzpicture}
	\begin{pgfonlayer}{nodelayer}
		\node [style=Z] (15) at (-0.75, 8.75) {};
		\node [style=none] (16) at (-2.75, 10.5) {};
		\node [style=none] (17) at (-0.25, 10.5) {};
		\node [style=none] (18) at (-2.75, 7.5) {};
		\node [style=none] (19) at (-3.25, 7.5) {};
		\node [style=none] (20) at (-0.75, 7.5) {};
		\node [style=none] (21) at (-0.75, 10.5) {};
		\node [style=X] (22) at (-0.25, 10) {};
		\node [style=none] (23) at (-1.25, 9.5) {};
		\node [style=Z] (24) at (-3.25, 10) {};
		\node [style=none] (25) at (-0.25, 7.5) {};
		\node [style=none] (26) at (-3.25, 10.5) {};
		\node [style=X] (27) at (-2, 9) {};
		\node [style=Z] (28) at (-2.75, 8.25) {};
		\node [style=andin] (29) at (-1.25, 9.5) {};
	\end{pgfonlayer}
	\begin{pgfonlayer}{edgelayer}
		\draw (17.center) to (22);
		\draw [in=90, out=180] (22) to (23.center);
		\draw (23.center) to (15);
		\draw (15) to (20.center);
		\draw (25.center) to (22);
		\draw (21.center) to (15);
		\draw (24) to (26.center);
		\draw (24) to (19.center);
		\draw (27) to (28);
		\draw (27) to (24);
		\draw (28) to (18.center);
		\draw (28) to (16.center);
		\draw (23.center) to (27);
	\end{pgfonlayer}
\end{tikzpicture}
\eq{\ref{ZXA.17}}
\begin{tikzpicture}
	\begin{pgfonlayer}{nodelayer}
		\node [style=none] (0) at (-1.25, 10.75) {};
		\node [style=none] (1) at (-3.5, 10.75) {};
		\node [style=Z] (2) at (-1.25, 11.5) {};
		\node [style=none] (3) at (-4, 10.75) {};
		\node [style=Z] (4) at (-4, 11.5) {};
		\node [style=none] (5) at (-1.25, 15) {};
		\node [style=none] (6) at (-4, 15) {};
		\node [style=none] (7) at (-3.5, 15) {};
		\node [style=none] (8) at (-0.75, 10.75) {};
		\node [style=none] (9) at (-0.75, 15) {};
		\node [style=X] (10) at (-0.75, 14.5) {};
		\node [style=Z] (11) at (-3.5, 12.5) {};
		\node [style=none] (12) at (-1.75, 13.25) {};
		\node [style=X] (13) at (-2.25, 14) {};
		\node [style=Z] (14) at (-1.75, 12.25) {};
		\node [style=andin] (15) at (-2.75, 13.25) {};
		\node [style=none] (16) at (-2.75, 13.25) {};
		\node [style=andin] (17) at (-1.75, 13.25) {};
	\end{pgfonlayer}
	\begin{pgfonlayer}{edgelayer}
		\draw (9.center) to (10);
		\draw (2) to (0.center);
		\draw (8.center) to (10);
		\draw (5.center) to (2);
		\draw (4) to (6.center);
		\draw (4) to (3.center);
		\draw (11) to (1.center);
		\draw (11) to (7.center);
		\draw [in=-90, out=124] (2) to (14);
		\draw (14) to (12.center);
		\draw (12.center) to (13);
		\draw (12.center) to (4);
		\draw (13) to (16.center);
		\draw (16.center) to (11);
		\draw (14) to (16.center);
		\draw [in=90, out=180] (10) to (13);
	\end{pgfonlayer}
\end{tikzpicture}\\
\eq{\ref{ZXA.1},\ref{ZXA.3}}&
\begin{tikzpicture}
	\begin{pgfonlayer}{nodelayer}
		\node [style=none] (1) at (-1.25, 10.75) {};
		\node [style=none] (2) at (-3.5, 10.75) {};
		\node [style=Z] (3) at (-1.25, 11.25) {};
		\node [style=none] (4) at (-4, 10.75) {};
		\node [style=Z] (5) at (-4, 11.25) {};
		\node [style=none] (6) at (-1.25, 15) {};
		\node [style=none] (7) at (-4, 15) {};
		\node [style=none] (8) at (-3.5, 15) {};
		\node [style=none] (9) at (-0.75, 10.75) {};
		\node [style=none] (10) at (-0.75, 15) {};
		\node [style=X] (11) at (-0.75, 14.5) {};
		\node [style=Z] (12) at (-3.5, 12.5) {};
		\node [style=none] (13) at (-1.75, 13.25) {};
		\node [style=andin] (14) at (-1.75, 13.25) {};
		\node [style=none] (15) at (-2.75, 13.25) {};
		\node [style=X] (16) at (-0.75, 14) {};
		\node [style=Z] (17) at (-1.25, 12.5) {};
		\node [style=andin] (18) at (-2.75, 13.25) {};
	\end{pgfonlayer}
	\begin{pgfonlayer}{edgelayer}
		\draw (10.center) to (11);
		\draw (3) to (1.center);
		\draw (9.center) to (11);
		\draw (6.center) to (3);
		\draw (5) to (7.center);
		\draw (5) to (4.center);
		\draw (12) to (2.center);
		\draw (12) to (8.center);
		\draw (13.center) to (5);
		\draw (15.center) to (12);
		\draw [in=90, out=180] (16) to (13.center);
		\draw [in=180, out=90, looseness=0.75] (15.center) to (11);
		\draw (3) to (15.center);
		\draw (13.center) to (17);
	\end{pgfonlayer}
\end{tikzpicture}
=
\left\llbracket
\begin{tikzpicture}
	\begin{pgfonlayer}{nodelayer}
		\node [style=nothing] (2) at (0, 0) {};
		\node [style=nothing] (3) at (-1, 0) {};
		\node [style=nothing] (4) at (-0.5, 0) {};
		\node [style=nothing] (5) at (-1.5, 0) {};
		\node [style=dot] (6) at (-1.5, 0.5) {};
		\node [style=dot] (7) at (-0.5, 0.5) {};
		\node [style=oplus] (8) at (0, 0.5) {};
		\node [style=nothing] (9) at (-0.5, 1.5) {};
		\node [style=nothing] (10) at (-1, 1.5) {};
		\node [style=nothing] (11) at (-1.5, 1.5) {};
		\node [style=nothing] (12) at (0, 1.5) {};
		\node [style=dot] (13) at (-1, 1) {};
		\node [style=dot] (14) at (-0.5, 1) {};
		\node [style=oplus] (15) at (0, 1) {};
	\end{pgfonlayer}
	\begin{pgfonlayer}{edgelayer}
		\draw (5) to (6);
		\draw (4) to (7);
		\draw (8) to (2);
		\draw (8) to (7);
		\draw (7) to (6);
		\draw (13) to (3);
		\draw (7) to (14);
		\draw (14) to (9);
		\draw (12) to (15);
		\draw (15) to (8);
		\draw (15) to (14);
		\draw (14) to (13);
		\draw (6) to (11);
		\draw (13) to (10);
	\end{pgfonlayer}
\end{tikzpicture}
\right\rrbracket_{\hat{\TOF}}
\end{align*}


\item[\ref{TOF.12}:]
\begingroup
\allowdisplaybreaks
\begin{align*}
&\left\llbracket
\begin{tikzpicture}
	\begin{pgfonlayer}{nodelayer}
		\node [style=nothing] (3) at (-0.5, 0) {};
		\node [style=nothing] (4) at (0, 0) {};
		\node [style=nothing] (5) at (-1, 0) {};
		\node [style=nothing] (6) at (-1.5, 0) {};
		\node [style=nothing] (7) at (-0.5, 2) {};
		\node [style=nothing] (8) at (-1.5, 2) {};
		\node [style=nothing] (9) at (0, 2) {};
		\node [style=nothing] (10) at (-1, 2) {};
		\node [style=dot] (11) at (-1.5, 0.5) {};
		\node [style=dot] (12) at (-1, 0.5) {};
		\node [style=oplus] (13) at (-0.5, 0.5) {};
		\node [style=oplus] (14) at (0, 1) {};
		\node [style=dot] (15) at (-1, 1) {};
		\node [style=dot] (16) at (-0.5, 1) {};
		\node [style=oplus] (17) at (-0.5, 1.5) {};
		\node [style=dot] (18) at (-1.5, 1.5) {};
		\node [style=dot] (19) at (-1, 1.5) {};
	\end{pgfonlayer}
	\begin{pgfonlayer}{edgelayer}
		\draw (11) to (12);
		\draw (12) to (13);
		\draw (15) to (16);
		\draw (16) to (14);
		\draw (18) to (19);
		\draw (19) to (17);
		\draw (6) to (11);
		\draw (11) to (18);
		\draw (18) to (8);
		\draw (10) to (19);
		\draw (19) to (15);
		\draw (15) to (12);
		\draw (12) to (5);
		\draw (3) to (13);
		\draw (13) to (16);
		\draw (16) to (17);
		\draw (17) to (7);
		\draw (9) to (14);
		\draw (14) to (4);
	\end{pgfonlayer}
\end{tikzpicture}
\right\rrbracket_{\hat{\TOF}}
=
\begin{tikzpicture}
	\begin{pgfonlayer}{nodelayer}
		\node [style=none] (4) at (-1.5, 11.25) {};
		\node [style=Z] (5) at (-2, 10.5) {};
		\node [style=Z] (6) at (-1, 10.5) {};
		\node [style=X] (7) at (0, 12) {};
		\node [style=andin] (8) at (-1.5, 11.25) {};
		\node [style=Z] (9) at (-2, 13) {};
		\node [style=none] (10) at (-1.5, 13.75) {};
		\node [style=Z] (11) at (-1, 13) {};
		\node [style=X] (12) at (0, 14.5) {};
		\node [style=andin] (13) at (-1.5, 13.75) {};
		\node [style=Z] (14) at (-1, 12.5) {};
		\node [style=none] (15) at (-0.5, 13.25) {};
		\node [style=Z] (16) at (0, 12.5) {};
		\node [style=X] (17) at (0.5, 14) {};
		\node [style=andin] (18) at (-0.5, 13.25) {};
		\node [style=none] (19) at (-2, 10) {};
		\node [style=none] (20) at (-1, 10) {};
		\node [style=none] (21) at (0, 10) {};
		\node [style=none] (22) at (0.5, 10) {};
		\node [style=none] (23) at (0.5, 15) {};
		\node [style=none] (24) at (0, 15) {};
		\node [style=none] (25) at (-1, 15) {};
		\node [style=none] (26) at (-2, 15) {};
	\end{pgfonlayer}
	\begin{pgfonlayer}{edgelayer}
		\draw [in=90, out=180] (7) to (4.center);
		\draw (4.center) to (5);
		\draw (4.center) to (6);
		\draw [in=90, out=180] (12) to (10.center);
		\draw (10.center) to (9);
		\draw (10.center) to (11);
		\draw [in=90, out=180] (17) to (15.center);
		\draw (15.center) to (14);
		\draw (15.center) to (16);
		\draw (19.center) to (26.center);
		\draw (25.center) to (20.center);
		\draw (24.center) to (21.center);
		\draw (22.center) to (23.center);
	\end{pgfonlayer}
\end{tikzpicture}
\eq{\ref{ZXA.3}}
\begin{tikzpicture}
	\begin{pgfonlayer}{nodelayer}
		\node [style=none] (5) at (-2, 11.25) {};
		\node [style=Z] (6) at (-3, 12) {};
		\node [style=Z] (7) at (-1.5, 12) {};
		\node [style=X] (8) at (0, 10.5) {};
		\node [style=andout] (9) at (-2, 11.25) {};
		\node [style=Z] (10) at (-2.5, 12) {};
		\node [style=none] (11) at (-2, 12.75) {};
		\node [style=Z] (12) at (-1, 12) {};
		\node [style=X] (13) at (0, 14.5) {};
		\node [style=andin] (14) at (-2, 12.75) {};
		\node [style=Z] (15) at (-1, 12.75) {};
		\node [style=none] (16) at (-0.5, 13.5) {};
		\node [style=Z] (17) at (0, 12.75) {};
		\node [style=X] (18) at (0.5, 14.25) {};
		\node [style=andin] (19) at (-0.5, 13.5) {};
		\node [style=none] (20) at (-3, 10) {};
		\node [style=none] (21) at (-1, 10) {};
		\node [style=none] (22) at (0, 10) {};
		\node [style=none] (23) at (0.5, 10) {};
		\node [style=none] (24) at (0.5, 15) {};
		\node [style=none] (25) at (0, 15) {};
		\node [style=none] (26) at (-1, 15) {};
		\node [style=none] (27) at (-3, 15) {};
	\end{pgfonlayer}
	\begin{pgfonlayer}{edgelayer}
		\draw [in=-90, out=180] (8) to (5.center);
		\draw (5.center) to (7);
		\draw [in=90, out=180] (13) to (11.center);
		\draw (11.center) to (10);
		\draw [in=90, out=180] (18) to (16.center);
		\draw (16.center) to (15);
		\draw (16.center) to (17);
		\draw (20.center) to (27.center);
		\draw (26.center) to (21.center);
		\draw (25.center) to (22.center);
		\draw (23.center) to (24.center);
		\draw (7) to (11.center);
		\draw (7) to (12);
		\draw (5.center) to (10);
		\draw (10) to (6);
	\end{pgfonlayer}
\end{tikzpicture}
=
\begin{tikzpicture}
	\begin{pgfonlayer}{nodelayer}
		\node [style=none] (6) at (-1.5, 10.75) {};
		\node [style=Z] (7) at (-3.5, 12.75) {};
		\node [style=Z] (8) at (-1.5, 12) {};
		\node [style=X] (9) at (0, 10) {};
		\node [style=andout] (10) at (-1.5, 10.75) {};
		\node [style=Z] (11) at (-2.5, 12) {};
		\node [style=none] (12) at (-2.5, 10.75) {};
		\node [style=Z] (13) at (-1, 12.75) {};
		\node [style=X] (14) at (0, 15.25) {};
		\node [style=Z] (15) at (-1, 13.25) {};
		\node [style=none] (16) at (-0.5, 14) {};
		\node [style=Z] (17) at (0, 13.25) {};
		\node [style=X] (18) at (0.5, 14.75) {};
		\node [style=andin] (19) at (-0.5, 14) {};
		\node [style=none] (20) at (-3.5, 9.5) {};
		\node [style=none] (21) at (-1, 9.5) {};
		\node [style=none] (22) at (0, 9.5) {};
		\node [style=none] (23) at (0.5, 9.5) {};
		\node [style=none] (24) at (0.5, 15.75) {};
		\node [style=none] (25) at (0, 15.75) {};
		\node [style=none] (26) at (-1, 15.75) {};
		\node [style=none] (27) at (-3.5, 15.75) {};
		\node [style=none] (28) at (-3, 10.75) {};
		\node [style=none] (29) at (-3, 14.75) {};
		\node [style=andout] (30) at (-2.5, 10.75) {};
	\end{pgfonlayer}
	\begin{pgfonlayer}{edgelayer}
		\draw [in=-90, out=180] (9) to (6.center);
		\draw [in=-60, out=60] (6.center) to (8);
		\draw [in=-120, out=120] (12.center) to (11);
		\draw [in=90, out=180] (18) to (16.center);
		\draw (16.center) to (15);
		\draw (16.center) to (17);
		\draw (20.center) to (27.center);
		\draw (26.center) to (21.center);
		\draw (25.center) to (22.center);
		\draw (23.center) to (24.center);
		\draw (8) to (12.center);
		\draw [in=180, out=90] (8) to (13);
		\draw (6.center) to (11);
		\draw [in=0, out=90, looseness=1.25] (11) to (7);
		\draw [in=-90, out=-90, looseness=3.50] (12.center) to (28.center);
		\draw (28.center) to (29.center);
		\draw [in=90, out=180, looseness=0.50] (14) to (29.center);
	\end{pgfonlayer}
\end{tikzpicture}\\
&\eq{\ref{ZXA.12}}
\begin{tikzpicture}
	\begin{pgfonlayer}{nodelayer}
		\node [style=none] (7) at (-0.25, 4.25) {};
		\node [style=none] (8) at (-0.75, 0.75) {};
		\node [style=Z] (9) at (-0.75, 3) {};
		\node [style=none] (10) at (-2.75, 6) {};
		\node [style=none] (11) at (-2.25, 1.75) {};
		\node [style=none] (12) at (-2.25, 5) {};
		\node [style=none] (13) at (-2.75, 0.75) {};
		\node [style=none] (14) at (0.75, 0.75) {};
		\node [style=none] (15) at (0.25, 0.75) {};
		\node [style=Z] (16) at (-0.75, 3.5) {};
		\node [style=andin] (17) at (-0.25, 4.25) {};
		\node [style=none] (18) at (-0.75, 6) {};
		\node [style=X] (19) at (0.75, 5) {};
		\node [style=none] (20) at (0.25, 6) {};
		\node [style=X] (21) at (0.25, 5.5) {};
		\node [style=X] (22) at (0.25, 1.25) {};
		\node [style=Z] (23) at (-2.75, 3.5) {};
		\node [style=Z] (24) at (0.25, 3.5) {};
		\node [style=none] (25) at (0.75, 6) {};
		\node [style=none] (26) at (-1.5, 2.5) {};
		\node [style=Z] (27) at (-1.5, 1.75) {};
		\node [style=andout] (28) at (-1.5, 2.5) {};
	\end{pgfonlayer}
	\begin{pgfonlayer}{edgelayer}
		\draw [in=90, out=180] (19) to (7.center);
		\draw (7.center) to (16);
		\draw (7.center) to (24);
		\draw (13.center) to (10.center);
		\draw (18.center) to (8.center);
		\draw (20.center) to (15.center);
		\draw (14.center) to (25.center);
		\draw (11.center) to (12.center);
		\draw [in=90, out=180, looseness=0.50] (21) to (12.center);
		\draw (26.center) to (9);
		\draw (26.center) to (23);
		\draw [in=180, out=-60] (27) to (22);
		\draw [in=-90, out=-135, looseness=1.25] (27) to (11.center);
		\draw (26.center) to (27);
	\end{pgfonlayer}
\end{tikzpicture}
\eq{\ref{ZXA.5}}
\begin{tikzpicture}
	\begin{pgfonlayer}{nodelayer}
		\node [style=X] (8) at (1, 13.25) {};
		\node [style=none] (9) at (-2.75, 8.5) {};
		\node [style=none] (10) at (0.5, 8.5) {};
		\node [style=none] (11) at (-1.5, 10.75) {};
		\node [style=Z] (12) at (-2.75, 11.75) {};
		\node [style=Z] (13) at (-0.75, 11.75) {};
		\node [style=none] (14) at (-0.75, 8.5) {};
		\node [style=Z] (15) at (-0.75, 11.25) {};
		\node [style=none] (16) at (1, 14.25) {};
		\node [style=none] (17) at (-2.25, 13.25) {};
		\node [style=X] (18) at (0.5, 13.75) {};
		\node [style=andin] (19) at (-0.25, 12.5) {};
		\node [style=none] (20) at (1, 8.5) {};
		\node [style=andout] (21) at (-1.5, 10.75) {};
		\node [style=Z] (22) at (-1.5, 9.5) {};
		\node [style=none] (23) at (-0.25, 12.5) {};
		\node [style=none] (24) at (-2.25, 9.5) {};
		\node [style=none] (25) at (0.5, 14.25) {};
		\node [style=none] (26) at (-2.75, 14.25) {};
		\node [style=none] (27) at (-0.75, 14.25) {};
		\node [style=Z] (28) at (0, 9.75) {};
		\node [style=Z] (29) at (0.5, 9.75) {};
		\node [style=X] (30) at (0.5, 10.5) {};
		\node [style=X] (31) at (0, 10.5) {};
	\end{pgfonlayer}
	\begin{pgfonlayer}{edgelayer}
		\draw [in=90, out=180] (8) to (23.center);
		\draw (23.center) to (13);
		\draw (9.center) to (26.center);
		\draw (27.center) to (14.center);
		\draw (20.center) to (16.center);
		\draw (24.center) to (17.center);
		\draw [in=90, out=180, looseness=0.50] (18) to (17.center);
		\draw (11.center) to (15);
		\draw (11.center) to (12);
		\draw [in=-90, out=-135, looseness=1.25] (22) to (24.center);
		\draw (11.center) to (22);
		\draw [in=-60, out=-90, looseness=1.25] (28) to (22);
		\draw (29) to (10.center);
		\draw (29) to (31);
		\draw (30) to (28);
		\draw [bend right, looseness=1.25] (31) to (28);
		\draw [bend right, looseness=1.25] (29) to (30);
		\draw (31) to (23.center);
		\draw (30) to (18);
		\draw (25.center) to (18);
	\end{pgfonlayer}
\end{tikzpicture}
\eq{\ref{ZXA.1},\ref{ZXA.2}}
\begin{tikzpicture}
	\begin{pgfonlayer}{nodelayer}
		\node [style=X] (9) at (1, 12.25) {};
		\node [style=none] (10) at (-2.75, 8.75) {};
		\node [style=none] (11) at (0.5, 8.75) {};
		\node [style=none] (12) at (-1.5, 10) {};
		\node [style=Z] (13) at (-2.75, 11) {};
		\node [style=Z] (14) at (-0.75, 11) {};
		\node [style=none] (15) at (-0.75, 8.75) {};
		\node [style=Z] (16) at (-0.75, 10.5) {};
		\node [style=none] (17) at (1, 12.75) {};
		\node [style=andin] (18) at (-0.25, 11.5) {};
		\node [style=none] (19) at (1, 8.75) {};
		\node [style=andout] (20) at (-1.5, 10) {};
		\node [style=none] (21) at (-0.25, 11.5) {};
		\node [style=none] (22) at (0.5, 12.75) {};
		\node [style=none] (23) at (-2.75, 12.75) {};
		\node [style=none] (24) at (-0.75, 12.75) {};
		\node [style=Z] (25) at (0, 9.75) {};
		\node [style=Z] (26) at (0.5, 9.75) {};
		\node [style=X] (27) at (0.5, 10.5) {};
		\node [style=X] (28) at (0, 10.5) {};
		\node [style=none] (29) at (-1.5, 9.75) {};
	\end{pgfonlayer}
	\begin{pgfonlayer}{edgelayer}
		\draw [in=90, out=180] (9) to (21.center);
		\draw (21.center) to (14);
		\draw (10.center) to (23.center);
		\draw (24.center) to (15.center);
		\draw (19.center) to (17.center);
		\draw (12.center) to (16);
		\draw (12.center) to (13);
		\draw (26) to (11.center);
		\draw (26) to (28);
		\draw [bend right=15, looseness=1.25] (27) to (25);
		\draw [bend right, looseness=1.25] (28) to (25);
		\draw [bend right, looseness=1.25] (26) to (27);
		\draw (28) to (21.center);
		\draw (22.center) to (27);
		\draw [bend right=15] (25) to (27);
		\draw [in=-90, out=-90, looseness=1.25] (25) to (29.center);
		\draw (29.center) to (12.center);
	\end{pgfonlayer}
\end{tikzpicture}\\
&\eq{\ref{ZXA.8}}
\begin{tikzpicture}
	\begin{pgfonlayer}{nodelayer}
		\node [style=X] (10) at (1, 12.75) {};
		\node [style=none] (11) at (-2.25, 8.75) {};
		\node [style=none] (12) at (0.5, 8.75) {};
		\node [style=none] (13) at (-1.5, 10.5) {};
		\node [style=Z] (14) at (-2.25, 11) {};
		\node [style=Z] (15) at (-0.75, 11.5) {};
		\node [style=none] (16) at (-0.75, 8.75) {};
		\node [style=Z] (17) at (-0.75, 11) {};
		\node [style=none] (18) at (1, 13.5) {};
		\node [style=andin] (19) at (-0.25, 12.25) {};
		\node [style=none] (20) at (1, 8.75) {};
		\node [style=andout] (21) at (-1.5, 10.5) {};
		\node [style=none] (22) at (-0.25, 12.25) {};
		\node [style=none] (23) at (0.5, 13.5) {};
		\node [style=none] (24) at (-2.25, 13.5) {};
		\node [style=none] (25) at (-0.75, 13.5) {};
		\node [style=Z] (26) at (0.5, 9.5) {};
		\node [style=X] (27) at (0, 10.25) {};
		\node [style=none] (28) at (-1.5, 10.25) {};
	\end{pgfonlayer}
	\begin{pgfonlayer}{edgelayer}
		\draw [in=90, out=180] (10) to (22.center);
		\draw (22.center) to (15);
		\draw (11.center) to (24.center);
		\draw (25.center) to (16.center);
		\draw (20.center) to (18.center);
		\draw (13.center) to (17);
		\draw (13.center) to (14);
		\draw (26) to (12.center);
		\draw (26) to (27);
		\draw [in=-60, out=90, looseness=0.75] (27) to (22.center);
		\draw (28.center) to (13.center);
		\draw (23.center) to (26);
		\draw [in=-90, out=-120] (27) to (28.center);
	\end{pgfonlayer}
\end{tikzpicture}
\eq{\ref{ZXA.17}}
\begin{tikzpicture}
	\begin{pgfonlayer}{nodelayer}
		\node [style=andin] (11) at (0, 5.5) {};
		\node [style=none] (12) at (-1.5, 1.5) {};
		\node [style=none] (13) at (1, 7.25) {};
		\node [style=andout] (14) at (-2.25, 2.75) {};
		\node [style=none] (15) at (-2.25, 2.5) {};
		\node [style=Z] (16) at (-1.5, 3.5) {};
		\node [style=Z] (17) at (-1.5, 4) {};
		\node [style=Z] (18) at (-3, 3.5) {};
		\node [style=none] (19) at (-1.5, 7.25) {};
		\node [style=none] (20) at (-3, 1.5) {};
		\node [style=none] (21) at (-2.25, 2.75) {};
		\node [style=none] (22) at (-3, 7.25) {};
		\node [style=none] (23) at (1, 1.5) {};
		\node [style=Z] (24) at (0.5, 4.5) {};
		\node [style=none] (25) at (0.5, 1.5) {};
		\node [style=X] (26) at (1, 6.75) {};
		\node [style=none] (27) at (0.5, 7.25) {};
		\node [style=none] (28) at (-1, 5.5) {};
		\node [style=none] (29) at (0, 5.5) {};
		\node [style=X] (30) at (-0.5, 6.25) {};
		\node [style=Z] (31) at (-1, 4.5) {};
		\node [style=none] (32) at (0, 4) {};
		\node [style=none] (33) at (0, 2.5) {};
		\node [style=andin] (34) at (-1, 5.5) {};
	\end{pgfonlayer}
	\begin{pgfonlayer}{edgelayer}
		\draw (20.center) to (22.center);
		\draw (19.center) to (12.center);
		\draw (23.center) to (13.center);
		\draw (21.center) to (16);
		\draw (21.center) to (18);
		\draw (24) to (25.center);
		\draw (15.center) to (21.center);
		\draw (27.center) to (24);
		\draw [in=90, out=-124] (30) to (28.center);
		\draw [in=90, out=-56] (30) to (29.center);
		\draw [in=37, out=-120] (29.center) to (31);
		\draw (31) to (28.center);
		\draw (31) to (17);
		\draw (29.center) to (24);
		\draw [in=90, out=-45, looseness=1.25] (28.center) to (32.center);
		\draw (32.center) to (33.center);
		\draw [in=-90, out=-90] (33.center) to (15.center);
		\draw [in=90, out=180, looseness=0.75] (26) to (30);
	\end{pgfonlayer}
\end{tikzpicture}
=
\begin{tikzpicture}
	\begin{pgfonlayer}{nodelayer}
		\node [style=andin] (12) at (0, 5.25) {};
		\node [style=none] (13) at (-1.5, 2) {};
		\node [style=none] (14) at (1, 7.25) {};
		\node [style=none] (15) at (-0.75, 4.5) {};
		\node [style=Z] (16) at (-1.5, 2.5) {};
		\node [style=Z] (17) at (-2.5, 2.5) {};
		\node [style=none] (18) at (-1.5, 7.25) {};
		\node [style=none] (19) at (-2.5, 2) {};
		\node [style=none] (20) at (-2, 3.25) {};
		\node [style=none] (21) at (-2.5, 7.25) {};
		\node [style=none] (22) at (1, 2) {};
		\node [style=Z] (23) at (0.5, 4.25) {};
		\node [style=none] (24) at (0.5, 2) {};
		\node [style=X] (25) at (1, 6.75) {};
		\node [style=none] (26) at (0.5, 7.25) {};
		\node [style=none] (27) at (-1, 5.25) {};
		\node [style=none] (28) at (0, 5.25) {};
		\node [style=X] (29) at (-0.5, 6) {};
		\node [style=Z] (30) at (-1.5, 4.25) {};
		\node [style=none] (31) at (-0.75, 4.5) {};
		\node [style=andin] (32) at (-1, 5.25) {};
		\node [style=andin] (33) at (-2, 3.25) {};
	\end{pgfonlayer}
	\begin{pgfonlayer}{edgelayer}
		\draw (19.center) to (21.center);
		\draw (18.center) to (13.center);
		\draw (22.center) to (14.center);
		\draw (20.center) to (16);
		\draw (20.center) to (17);
		\draw (23) to (24.center);
		\draw [in=90, out=-90] (15.center) to (20.center);
		\draw (26.center) to (23);
		\draw [in=90, out=-124] (29) to (27.center);
		\draw [in=90, out=-56] (29) to (28.center);
		\draw [in=37, out=-120] (28.center) to (30);
		\draw (30) to (27.center);
		\draw (28.center) to (23);
		\draw [in=90, out=-45, looseness=1.25] (27.center) to (31.center);
		\draw [in=90, out=180] (25) to (29);
	\end{pgfonlayer}
\end{tikzpicture}\\
&\eq{\ref{ZXA.11}}
\begin{tikzpicture}
	\begin{pgfonlayer}{nodelayer}
		\node [style=andin] (13) at (0, 5.5) {};
		\node [style=none] (14) at (-1.5, 1.25) {};
		\node [style=none] (15) at (1, 7.5) {};
		\node [style=none] (16) at (-0.5, 4.5) {};
		\node [style=none] (17) at (-1.5, 7.5) {};
		\node [style=none] (18) at (-2.5, 1.25) {};
		\node [style=none] (19) at (-2, 3.25) {};
		\node [style=none] (20) at (-2.5, 7.5) {};
		\node [style=none] (21) at (1, 1.25) {};
		\node [style=Z] (22) at (0.5, 4.5) {};
		\node [style=none] (23) at (0.5, 1.25) {};
		\node [style=X] (24) at (1, 7) {};
		\node [style=none] (25) at (0.5, 7.5) {};
		\node [style=none] (26) at (-1, 5.5) {};
		\node [style=none] (27) at (0, 5.5) {};
		\node [style=X] (28) at (-0.5, 6.25) {};
		\node [style=Z] (29) at (-1.5, 4.5) {};
		\node [style=none] (30) at (-0.5, 4.5) {};
		\node [style=andin] (31) at (-1, 5.5) {};
		\node [style=andin] (32) at (-2, 3.25) {};
		\node [style=Z] (33) at (-2.5, 1.75) {};
		\node [style=Z] (34) at (-1.5, 1.75) {};
		\node [style=none] (35) at (-2.25, 2.5) {};
		\node [style=none] (36) at (-1.75, 2.5) {};
	\end{pgfonlayer}
	\begin{pgfonlayer}{edgelayer}
		\draw (18.center) to (20.center);
		\draw (17.center) to (14.center);
		\draw (21.center) to (15.center);
		\draw (22) to (23.center);
		\draw [in=90, out=-90, looseness=1.25] (16.center) to (19.center);
		\draw (25.center) to (22);
		\draw [in=90, out=-124] (28) to (26.center);
		\draw [in=90, out=-56] (28) to (27.center);
		\draw [in=37, out=-120] (27.center) to (29);
		\draw (29) to (26.center);
		\draw (27.center) to (22);
		\draw [in=90, out=-45, looseness=1.25] (26.center) to (30.center);
		\draw [in=90, out=180] (24) to (28);
		\draw [in=90, out=-108] (19.center) to (35.center);
		\draw [in=135, out=-90] (35.center) to (34);
		\draw [in=-72, out=90] (36.center) to (19.center);
		\draw [in=45, out=-90] (36.center) to (33);
	\end{pgfonlayer}
\end{tikzpicture}
\eq{\ref{ZXA.9}}
\begin{tikzpicture}
	\begin{pgfonlayer}{nodelayer}
		\node [style=andin] (14) at (0, 6.25) {};
		\node [style=none] (15) at (-2, 2.5) {};
		\node [style=none] (16) at (1, 8.25) {};
		\node [style=none] (17) at (-0.5, 4.25) {};
		\node [style=Z] (18) at (-2.5, 3.25) {};
		\node [style=Z] (19) at (-2, 3.25) {};
		\node [style=none] (20) at (-2, 8.25) {};
		\node [style=none] (21) at (-2.5, 2.5) {};
		\node [style=none] (22) at (-2.5, 8.25) {};
		\node [style=none] (23) at (1, 2.5) {};
		\node [style=Z] (24) at (0.5, 4.75) {};
		\node [style=none] (25) at (0.5, 2.5) {};
		\node [style=X] (26) at (1, 7.75) {};
		\node [style=none] (27) at (0.5, 8.25) {};
		\node [style=none] (28) at (-1, 6.25) {};
		\node [style=none] (29) at (0, 6.25) {};
		\node [style=X] (30) at (-0.5, 7) {};
		\node [style=Z] (31) at (-2, 4.25) {};
		\node [style=none] (32) at (-0.5, 4.25) {};
		\node [style=none] (33) at (-1.5, 5.5) {};
		\node [style=andin] (34) at (-1, 6.25) {};
		\node [style=andin] (35) at (-1.5, 5.5) {};
		\node [style=none] (36) at (-1.25, 4.25) {};
	\end{pgfonlayer}
	\begin{pgfonlayer}{edgelayer}
		\draw (21.center) to (22.center);
		\draw (20.center) to (15.center);
		\draw (23.center) to (16.center);
		\draw (24) to (25.center);
		\draw (27.center) to (24);
		\draw [in=90, out=-124] (30) to (28.center);
		\draw [in=90, out=-56] (30) to (29.center);
		\draw [in=37, out=-120] (29.center) to (31);
		\draw (29.center) to (24);
		\draw [in=90, out=-45] (28.center) to (32.center);
		\draw [in=90, out=180] (26) to (30);
		\draw [in=90, out=-124] (28.center) to (33.center);
		\draw (33.center) to (31);
		\draw [in=53, out=-90, looseness=0.75] (17.center) to (18);
		\draw [in=90, out=-60] (33.center) to (36.center);
		\draw [in=49, out=-90, looseness=0.75] (36.center) to (19);
	\end{pgfonlayer}
\end{tikzpicture}
\eq{\ref{ZXA.3}}
\begin{tikzpicture}
	\begin{pgfonlayer}{nodelayer}
		\node [style=andin] (15) at (0, 5.75) {};
		\node [style=none] (16) at (-2, 2.5) {};
		\node [style=none] (17) at (1, 7.5) {};
		\node [style=none] (18) at (-0.5, 4.25) {};
		\node [style=Z] (19) at (-2.5, 3.25) {};
		\node [style=Z] (20) at (-2, 4.25) {};
		\node [style=none] (21) at (-2, 7.5) {};
		\node [style=none] (22) at (-2.5, 2.5) {};
		\node [style=none] (23) at (-2.5, 7.5) {};
		\node [style=none] (24) at (1, 2.5) {};
		\node [style=Z] (25) at (0.5, 4.75) {};
		\node [style=none] (26) at (0.5, 2.5) {};
		\node [style=X] (27) at (1, 7) {};
		\node [style=none] (28) at (0.5, 7.5) {};
		\node [style=none] (29) at (-1, 5.75) {};
		\node [style=none] (30) at (0, 5.75) {};
		\node [style=X] (31) at (-0.5, 6.5) {};
		\node [style=Z] (32) at (-2, 4.25) {};
		\node [style=none] (33) at (-0.5, 4.25) {};
		\node [style=none] (34) at (-1.5, 5) {};
		\node [style=andin] (35) at (-1, 5.75) {};
		\node [style=andin] (36) at (-1.5, 5) {};
	\end{pgfonlayer}
	\begin{pgfonlayer}{edgelayer}
		\draw (22.center) to (23.center);
		\draw (21.center) to (16.center);
		\draw (24.center) to (17.center);
		\draw (25) to (26.center);
		\draw (28.center) to (25);
		\draw [in=90, out=-124] (31) to (29.center);
		\draw [in=90, out=-56] (31) to (30.center);
		\draw [in=0, out=-120, looseness=0.75] (30.center) to (32);
		\draw (30.center) to (25);
		\draw [in=90, out=-45] (29.center) to (33.center);
		\draw [in=90, out=180, looseness=0.75] (27) to (31);
		\draw [in=90, out=-124] (29.center) to (34.center);
		\draw [in=60, out=-142, looseness=0.75] (34.center) to (32);
		\draw [in=53, out=-90, looseness=0.75] (18.center) to (19);
		\draw [bend left=45] (34.center) to (20);
	\end{pgfonlayer}
\end{tikzpicture}\\
&\eq{\ref{ZXA.15}}
\begin{tikzpicture}
	\begin{pgfonlayer}{nodelayer}
		\node [style=andin] (23) at (7.6, 6.25) {};
		\node [style=none] (24) at (6.25, 2.5) {};
		\node [style=none] (25) at (8.55, 8.25) {};
		\node [style=none] (26) at (7.15, 4.25) {};
		\node [style=Z] (27) at (5.75, 3.25) {};
		\node [style=Z] (28) at (6.25, 4.25) {};
		\node [style=none] (29) at (6.25, 8.25) {};
		\node [style=none] (30) at (5.75, 2.5) {};
		\node [style=none] (31) at (5.75, 8.25) {};
		\node [style=none] (32) at (8.55, 2.5) {};
		\node [style=Z] (33) at (8.05, 4.75) {};
		\node [style=none] (34) at (8.05, 2.5) {};
		\node [style=X] (35) at (8.55, 7.75) {};
		\node [style=none] (36) at (8.05, 8.25) {};
		\node [style=none] (37) at (6.7, 6.25) {};
		\node [style=none] (38) at (7.6, 6.25) {};
		\node [style=X] (39) at (7.15, 7) {};
		\node [style=Z] (40) at (6.25, 4.25) {};
		\node [style=none] (41) at (7.15, 4.25) {};
		\node [style=andin] (42) at (6.7, 6.25) {};
	\end{pgfonlayer}
	\begin{pgfonlayer}{edgelayer}
		\draw (30.center) to (31.center);
		\draw (29.center) to (24.center);
		\draw (32.center) to (25.center);
		\draw (33) to (34.center);
		\draw (36.center) to (33);
		\draw [in=90, out=-124] (39) to (37.center);
		\draw [in=90, out=-56] (39) to (38.center);
		\draw [in=0, out=-120, looseness=0.75] (38.center) to (40);
		\draw (38.center) to (33);
		\draw [in=90, out=-45] (37.center) to (41.center);
		\draw [in=90, out=180] (35) to (39);
		\draw [in=53, out=-90, looseness=0.75] (26.center) to (27);
		\draw [in=75, out=-105] (37.center) to (28);
	\end{pgfonlayer}
\end{tikzpicture}
\eq{\ref{ZXA.3}}
\begin{tikzpicture}
	\begin{pgfonlayer}{nodelayer}
		\node [style=none] (104) at (19.1, 3.5) {};
		\node [style=none] (105) at (21.525, 8.25) {};
		\node [style=none] (106) at (20, 6.25) {};
		\node [style=Z] (107) at (18.6, 5.75) {};
		\node [style=Z] (108) at (19.1, 5) {};
		\node [style=none] (109) at (19.1, 8.25) {};
		\node [style=none] (110) at (18.6, 3.5) {};
		\node [style=none] (111) at (18.6, 8.25) {};
		\node [style=none] (112) at (21.525, 3.5) {};
		\node [style=Z] (113) at (21.025, 5.25) {};
		\node [style=none] (114) at (21.025, 3.5) {};
		\node [style=X] (115) at (21.525, 7.75) {};
		\node [style=none] (116) at (21.025, 8.25) {};
		\node [style=none] (117) at (19.6, 7) {};
		\node [style=none] (118) at (20.525, 6.25) {};
		\node [style=Z] (119) at (19.1, 4.25) {};
		\node [style=none] (120) at (20, 6.25) {};
		\node [style=andin] (121) at (19.6, 7) {};
		\node [style=X] (122) at (21.525, 7) {};
		\node [style=andin] (123) at (20.5, 6.25) {};
	\end{pgfonlayer}
	\begin{pgfonlayer}{edgelayer}
		\draw (110.center) to (111.center);
		\draw (109.center) to (104.center);
		\draw (112.center) to (105.center);
		\draw (113) to (114.center);
		\draw (116.center) to (113);
		\draw [in=0, out=-120, looseness=0.75] (118.center) to (119);
		\draw (118.center) to (113);
		\draw [in=90, out=-45] (117.center) to (120.center);
		\draw [in=53, out=-90, looseness=0.75] (106.center) to (107);
		\draw [in=60, out=-120, looseness=1.25] (117.center) to (108);
		\draw [in=90, out=180, looseness=0.75] (115) to (117.center);
		\draw [in=90, out=180] (122) to (118.center);
	\end{pgfonlayer}
\end{tikzpicture}
\eq{\ref{ZXA.11}}
\begin{tikzpicture}
	\begin{pgfonlayer}{nodelayer}
		\node [style=none] (18) at (-0.25, 4.75) {};
		\node [style=none] (19) at (1.25, 8.25) {};
		\node [style=Z] (20) at (-0.25, 6) {};
		\node [style=Z] (21) at (-1.25, 6) {};
		\node [style=none] (22) at (-0.25, 8.25) {};
		\node [style=none] (23) at (-1.25, 4.75) {};
		\node [style=none] (24) at (-1.25, 8.25) {};
		\node [style=none] (25) at (1.25, 4.75) {};
		\node [style=Z] (26) at (0.75, 5.25) {};
		\node [style=none] (27) at (0.75, 4.75) {};
		\node [style=X] (28) at (1.25, 7.75) {};
		\node [style=none] (29) at (0.75, 8.25) {};
		\node [style=none] (30) at (-0.75, 7) {};
		\node [style=none] (31) at (0.25, 6.25) {};
		\node [style=Z] (32) at (-0.25, 5.25) {};
		\node [style=andin] (33) at (-0.75, 7) {};
		\node [style=X] (34) at (1.25, 7) {};
		\node [style=andin] (35) at (0.25, 6.25) {};
	\end{pgfonlayer}
	\begin{pgfonlayer}{edgelayer}
		\draw (23.center) to (24.center);
		\draw (22.center) to (18.center);
		\draw (25.center) to (19.center);
		\draw (26) to (27.center);
		\draw (29.center) to (26);
		\draw (31.center) to (32);
		\draw (31.center) to (26);
		\draw [in=90, out=180, looseness=0.75] (28) to (30.center);
		\draw [in=90, out=180] (34) to (31.center);
		\draw (30.center) to (21);
		\draw (20) to (30.center);
	\end{pgfonlayer}
\end{tikzpicture}
=
\left\llbracket
\begin{tikzpicture}
	\begin{pgfonlayer}{nodelayer}
		\node [style=nothing] (19) at (-0.5, 0.5) {};
		\node [style=nothing] (20) at (0, 0.5) {};
		\node [style=nothing] (21) at (-1, 0.5) {};
		\node [style=nothing] (22) at (-1.5, 0.5) {};
		\node [style=nothing] (23) at (-0.5, 2) {};
		\node [style=nothing] (24) at (-1.5, 2) {};
		\node [style=nothing] (25) at (0, 2) {};
		\node [style=nothing] (26) at (-1, 2) {};
		\node [style=dot] (27) at (-1, 1.5) {};
		\node [style=dot] (28) at (-0.5, 1.5) {};
		\node [style=dot] (29) at (-1.5, 1) {};
		\node [style=dot] (30) at (-1, 1) {};
		\node [style=oplus] (31) at (0, 1) {};
		\node [style=oplus] (32) at (0, 1.5) {};
	\end{pgfonlayer}
	\begin{pgfonlayer}{edgelayer}
		\draw (27) to (28);
		\draw (22) to (29);
		\draw (29) to (24);
		\draw (21) to (30);
		\draw (30) to (27);
		\draw (27) to (26);
		\draw (19) to (28);
		\draw (28) to (23);
		\draw (20) to (31);
		\draw (31) to (32);
		\draw (32) to (25);
		\draw (32) to (28);
		\draw (31) to (30);
		\draw (30) to (29);
	\end{pgfonlayer}
\end{tikzpicture}
\right\rrbracket_{\hat{\TOF}}
\end{align*}
\endgroup


\item[\ref{TOF.13}:]
\begingroup
\allowdisplaybreaks
\begin{align*}
&\left\llbracket 
\begin{tikzpicture}
	\begin{pgfonlayer}{nodelayer}
		\node [style=nothing] (21) at (0, 0.5) {};
		\node [style=nothing] (22) at (-1, 0.5) {};
		\node [style=nothing] (23) at (-0.5, 0.5) {};
		\node [style=nothing] (24) at (-1.5, 0.5) {};
		\node [style=nothing] (25) at (0, 2.5) {};
		\node [style=dot] (26) at (-1.5, 1) {};
		\node [style=dot] (27) at (-1, 1) {};
		\node [style=dot] (28) at (-0.5, 1.5) {};
		\node [style=oplus] (29) at (-0.5, 1) {};
		\node [style=oplus] (30) at (0, 1.5) {};
		\node [style=nothing] (31) at (-0.5, 2.5) {};
		\node [style=nothing] (32) at (-1.5, 2.5) {};
		\node [style=nothing] (33) at (-1, 2.5) {};
		\node [style=oplus] (34) at (-0.5, 2) {};
		\node [style=dot] (35) at (-1, 2) {};
		\node [style=dot] (36) at (-1.5, 2) {};
	\end{pgfonlayer}
	\begin{pgfonlayer}{edgelayer}
		\draw (26) to (24);
		\draw (27) to (22);
		\draw (23) to (29);
		\draw (29) to (28);
		\draw (25) to (30);
		\draw (30) to (21);
		\draw (29) to (27);
		\draw (27) to (26);
		\draw (30) to (28);
		\draw (26) to (36);
		\draw (36) to (32);
		\draw (33) to (35);
		\draw (35) to (27);
		\draw (28) to (34);
		\draw (34) to (31);
		\draw (34) to (35);
		\draw (35) to (36);
	\end{pgfonlayer}
\end{tikzpicture}
\right\rrbracket_{\hat{\TOF}}
=
\begin{tikzpicture}
	\begin{pgfonlayer}{nodelayer}
		\node [style=none] (22) at (-2, 7) {};
		\node [style=none] (23) at (-1, 7) {};
		\node [style=none] (24) at (-2, 12.25) {};
		\node [style=none] (25) at (-1, 12.25) {};
		\node [style=Z] (26) at (-2, 8) {};
		\node [style=Z] (27) at (-1, 8) {};
		\node [style=none] (28) at (-1.5, 9) {};
		\node [style=X] (29) at (-0.5, 9.5) {};
		\node [style=andin] (30) at (-1.5, 9) {};
		\node [style=andin] (31) at (-1.5, 11) {};
		\node [style=Z] (32) at (-1, 10) {};
		\node [style=X] (33) at (-0.5, 11.5) {};
		\node [style=Z] (34) at (-2, 10) {};
		\node [style=none] (35) at (-1.5, 11) {};
		\node [style=Z] (36) at (-0.5, 10.5) {};
		\node [style=X] (37) at (0, 10.5) {};
		\node [style=none] (38) at (-0.5, 12.25) {};
		\node [style=none] (39) at (-0.5, 7) {};
		\node [style=none] (40) at (0, 12.25) {};
		\node [style=none] (41) at (0, 7) {};
	\end{pgfonlayer}
	\begin{pgfonlayer}{edgelayer}
		\draw [in=90, out=180] (29) to (28.center);
		\draw (28.center) to (26);
		\draw (27) to (28.center);
		\draw [in=90, out=180] (33) to (35.center);
		\draw (35.center) to (34);
		\draw (32) to (35.center);
		\draw (37) to (36);
		\draw (40.center) to (41.center);
		\draw (39.center) to (38.center);
		\draw (25.center) to (23.center);
		\draw (22.center) to (24.center);
	\end{pgfonlayer}
\end{tikzpicture}
\eq{\ref{ZXA.3}}
\begin{tikzpicture}
	\begin{pgfonlayer}{nodelayer}
		\node [style=none] (23) at (-2, 8) {};
		\node [style=none] (24) at (-1, 8) {};
		\node [style=none] (25) at (-2, 12) {};
		\node [style=none] (26) at (-1, 12) {};
		\node [style=andin] (27) at (-1.5, 11) {};
		\node [style=Z] (28) at (-1, 10) {};
		\node [style=X] (29) at (-0.5, 11.5) {};
		\node [style=Z] (30) at (-2, 10) {};
		\node [style=none] (31) at (-1.5, 11) {};
		\node [style=Z] (32) at (-0.5, 10) {};
		\node [style=X] (33) at (0, 10) {};
		\node [style=none] (34) at (-0.5, 12) {};
		\node [style=none] (35) at (-0.5, 8) {};
		\node [style=none] (36) at (0, 12) {};
		\node [style=none] (37) at (0, 8) {};
		\node [style=andout] (38) at (-1.5, 9) {};
		\node [style=Z] (39) at (-1, 10) {};
		\node [style=X] (40) at (-0.5, 8.5) {};
		\node [style=Z] (41) at (-2, 10) {};
		\node [style=none] (42) at (-1.5, 9) {};
	\end{pgfonlayer}
	\begin{pgfonlayer}{edgelayer}
		\draw [in=90, out=180] (29) to (31.center);
		\draw (31.center) to (30);
		\draw (28) to (31.center);
		\draw (33) to (32);
		\draw (36.center) to (37.center);
		\draw (35.center) to (34.center);
		\draw (26.center) to (24.center);
		\draw (23.center) to (25.center);
		\draw [in=-90, out=180] (40) to (42.center);
		\draw (42.center) to (41);
		\draw (39) to (42.center);
	\end{pgfonlayer}
\end{tikzpicture}
\eq{\ref{ZXA.3}}
\begin{tikzpicture}
	\begin{pgfonlayer}{nodelayer}
		\node [style=none] (24) at (-3, 9.5) {};
		\node [style=none] (25) at (-1, 9.5) {};
		\node [style=none] (26) at (-3, 13.5) {};
		\node [style=none] (27) at (-1, 13.5) {};
		\node [style=andin] (28) at (-2, 12.5) {};
		\node [style=Z] (29) at (-1.5, 11.5) {};
		\node [style=X] (30) at (-0.5, 13) {};
		\node [style=Z] (31) at (-2.5, 11.5) {};
		\node [style=none] (32) at (-2, 12.5) {};
		\node [style=Z] (33) at (-0.5, 11.5) {};
		\node [style=X] (34) at (0, 11.5) {};
		\node [style=none] (35) at (-0.5, 13.5) {};
		\node [style=none] (36) at (-0.5, 9.5) {};
		\node [style=none] (37) at (0, 13.5) {};
		\node [style=none] (38) at (0, 9.5) {};
		\node [style=Z] (39) at (-1, 11.5) {};
		\node [style=X] (40) at (-0.5, 10) {};
		\node [style=Z] (41) at (-3, 11.5) {};
		\node [style=none] (42) at (-2, 10.5) {};
		\node [style=andout] (43) at (-2, 10.5) {};
	\end{pgfonlayer}
	\begin{pgfonlayer}{edgelayer}
		\draw [in=90, out=180] (30) to (32.center);
		\draw (32.center) to (31);
		\draw (29) to (32.center);
		\draw (34) to (33);
		\draw (37.center) to (38.center);
		\draw (36.center) to (35.center);
		\draw (27.center) to (25.center);
		\draw (24.center) to (26.center);
		\draw [in=-90, out=180] (40) to (42.center);
		\draw (31) to (41);
		\draw (31) to (42.center);
		\draw (29) to (39);
		\draw (29) to (42.center);
	\end{pgfonlayer}
\end{tikzpicture}\\
&=
\begin{tikzpicture}
	\begin{pgfonlayer}{nodelayer}
		\node [style=none] (25) at (-3.5, 9.5) {};
		\node [style=none] (26) at (-1, 9.5) {};
		\node [style=none] (27) at (-3.5, 13.5) {};
		\node [style=none] (28) at (-1, 13.5) {};
		\node [style=Z] (29) at (-1.75, 11.5) {};
		\node [style=none] (30) at (-2.5, 9.75) {};
		\node [style=Z] (31) at (-2.5, 11.5) {};
		\node [style=none] (32) at (-2.5, 10.5) {};
		\node [style=Z] (33) at (-0.5, 11.5) {};
		\node [style=X] (34) at (0, 11.5) {};
		\node [style=none] (35) at (-0.5, 13.5) {};
		\node [style=none] (36) at (-0.5, 9.5) {};
		\node [style=none] (37) at (0, 13.5) {};
		\node [style=none] (38) at (0, 9.5) {};
		\node [style=Z] (39) at (-1, 11.5) {};
		\node [style=X] (40) at (-0.5, 10) {};
		\node [style=Z] (41) at (-3.5, 12.5) {};
		\node [style=none] (42) at (-1.75, 10.5) {};
		\node [style=andout] (43) at (-1.75, 10.5) {};
		\node [style=andout] (44) at (-2.5, 10.5) {};
		\node [style=none] (45) at (-3, 9.75) {};
		\node [style=none] (46) at (-3, 12.25) {};
		\node [style=X] (47) at (-0.5, 12.75) {};
	\end{pgfonlayer}
	\begin{pgfonlayer}{edgelayer}
		\draw (30.center) to (32.center);
		\draw [in=-120, out=120, looseness=1.25] (32.center) to (31);
		\draw (29) to (32.center);
		\draw (34) to (33);
		\draw (37.center) to (38.center);
		\draw (36.center) to (35.center);
		\draw (28.center) to (26.center);
		\draw (25.center) to (27.center);
		\draw [in=-90, out=180] (40) to (42.center);
		\draw [in=-63, out=90] (31) to (41);
		\draw (31) to (42.center);
		\draw (29) to (39);
		\draw [in=60, out=-60, looseness=1.25] (29) to (42.center);
		\draw [in=90, out=-174, looseness=0.50] (47) to (46.center);
		\draw (46.center) to (45.center);
		\draw [in=-90, out=-90, looseness=1.50] (45.center) to (30.center);
	\end{pgfonlayer}
\end{tikzpicture}
\eq{\ref{ZXA.12}}
\begin{tikzpicture}
	\begin{pgfonlayer}{nodelayer}
		\node [style=none] (26) at (-3.5, 9.5) {};
		\node [style=none] (27) at (-1, 9.5) {};
		\node [style=none] (28) at (-3.5, 14.25) {};
		\node [style=none] (29) at (-1, 14.25) {};
		\node [style=Z] (30) at (-0.5, 12.25) {};
		\node [style=X] (31) at (0, 12.25) {};
		\node [style=none] (32) at (-0.5, 14.25) {};
		\node [style=none] (33) at (-0.5, 9.5) {};
		\node [style=none] (34) at (0, 14.25) {};
		\node [style=none] (35) at (0, 9.5) {};
		\node [style=Z] (36) at (-1, 12.25) {};
		\node [style=X] (37) at (-0.5, 10) {};
		\node [style=Z] (38) at (-3.5, 13.25) {};
		\node [style=none] (39) at (-2.75, 10.5) {};
		\node [style=none] (40) at (-2.75, 13) {};
		\node [style=X] (41) at (-0.5, 13.5) {};
		\node [style=none] (42) at (-2, 11.5) {};
		\node [style=andout] (43) at (-2, 11.5) {};
		\node [style=Z] (44) at (-2, 10.5) {};
	\end{pgfonlayer}
	\begin{pgfonlayer}{edgelayer}
		\draw (31) to (30);
		\draw (34.center) to (35.center);
		\draw (33.center) to (32.center);
		\draw (29.center) to (27.center);
		\draw (26.center) to (28.center);
		\draw [in=90, out=-174, looseness=0.50] (41) to (40.center);
		\draw (40.center) to (39.center);
		\draw [in=60, out=-143] (36) to (42.center);
		\draw [in=-49, out=120] (42.center) to (38);
		\draw (42.center) to (44);
		\draw [in=180, out=-60, looseness=1.25] (44) to (37);
		\draw [in=-90, out=-120, looseness=2.00] (44) to (39.center);
	\end{pgfonlayer}
\end{tikzpicture}
\eq{\ref{ZXA.5}}
\begin{tikzpicture}
	\begin{pgfonlayer}{nodelayer}
		\node [style=X] (27) at (0, 13) {};
		\node [style=none] (28) at (-2, 11.5) {};
		\node [style=X] (29) at (0.5, 12.25) {};
		\node [style=none] (30) at (0.5, 13.75) {};
		\node [style=Z] (31) at (-1, 12.25) {};
		\node [style=none] (32) at (-3.25, 13.75) {};
		\node [style=none] (33) at (-1, 13.75) {};
		\node [style=none] (34) at (0, 13.75) {};
		\node [style=none] (35) at (-2.75, 12.5) {};
		\node [style=Z] (36) at (-3.25, 13) {};
		\node [style=Z] (37) at (-2, 10.5) {};
		\node [style=none] (38) at (-1, 9.5) {};
		\node [style=andout] (39) at (-2, 11.5) {};
		\node [style=none] (40) at (0.5, 9.5) {};
		\node [style=none] (41) at (-3.25, 9.5) {};
		\node [style=none] (42) at (-2.75, 10.5) {};
		\node [style=none] (43) at (0, 9.5) {};
		\node [style=Z] (44) at (-0.5, 10.5) {};
		\node [style=Z] (45) at (0, 10.5) {};
		\node [style=X] (46) at (0, 11.5) {};
		\node [style=X] (47) at (-0.5, 11.5) {};
	\end{pgfonlayer}
	\begin{pgfonlayer}{edgelayer}
		\draw (30.center) to (40.center);
		\draw (33.center) to (38.center);
		\draw (41.center) to (32.center);
		\draw [in=90, out=-174, looseness=0.50] (27) to (35.center);
		\draw (35.center) to (42.center);
		\draw [in=60, out=-143] (31) to (28.center);
		\draw [in=-49, out=120] (28.center) to (36);
		\draw (28.center) to (37);
		\draw [in=-90, out=-120, looseness=2.00] (37) to (42.center);
		\draw (46) to (44);
		\draw [in=-120, out=120, looseness=1.25] (44) to (47);
		\draw (47) to (45);
		\draw [in=-60, out=60, looseness=1.25] (45) to (46);
		\draw [in=-75, out=-90, looseness=1.25] (44) to (37);
		\draw (45) to (43.center);
		\draw [in=-124, out=90] (46) to (29);
		\draw [in=90, out=-90] (27) to (47);
		\draw (27) to (34.center);
	\end{pgfonlayer}
\end{tikzpicture}\\
&\eq{\ref{ZXA.1},\ref{ZXA.3}}
\begin{tikzpicture}
	\begin{pgfonlayer}{nodelayer}
		\node [style=none] (28) at (-1.5, 10.5) {};
		\node [style=X] (29) at (0.5, 12.25) {};
		\node [style=none] (30) at (0.5, 14.25) {};
		\node [style=Z] (31) at (-1, 12.25) {};
		\node [style=none] (32) at (-2, 14.25) {};
		\node [style=none] (33) at (-1, 14.25) {};
		\node [style=none] (34) at (0, 14.25) {};
		\node [style=Z] (35) at (-2, 12.25) {};
		\node [style=none] (36) at (-1, 9.5) {};
		\node [style=andout] (37) at (-1.5, 10.5) {};
		\node [style=none] (38) at (0.5, 9.5) {};
		\node [style=none] (39) at (-2, 9.5) {};
		\node [style=none] (40) at (0, 9.5) {};
		\node [style=Z] (41) at (-0.5, 10.5) {};
		\node [style=Z] (42) at (0, 10.5) {};
		\node [style=X] (43) at (0, 11.5) {};
		\node [style=X] (44) at (-0.5, 11.5) {};
	\end{pgfonlayer}
	\begin{pgfonlayer}{edgelayer}
		\draw (30.center) to (38.center);
		\draw (33.center) to (36.center);
		\draw (39.center) to (32.center);
		\draw (31) to (28.center);
		\draw (28.center) to (35);
		\draw (43) to (41);
		\draw [in=-120, out=120, looseness=1.25] (41) to (44);
		\draw (44) to (42);
		\draw [in=-60, out=60, looseness=1.25] (42) to (43);
		\draw (42) to (40.center);
		\draw [in=-124, out=90] (43) to (29);
		\draw (44) to (41);
		\draw [in=-90, out=-90] (41) to (28.center);
		\draw [in=90, out=-90] (34.center) to (44);
	\end{pgfonlayer}
\end{tikzpicture}
\eq{\ref{ZXA.8}}
\begin{tikzpicture}
	\begin{pgfonlayer}{nodelayer}
		\node [style=none] (29) at (-1.5, 10) {};
		\node [style=X] (30) at (0.5, 11.75) {};
		\node [style=none] (31) at (0.5, 12.25) {};
		\node [style=Z] (32) at (-1, 11) {};
		\node [style=none] (33) at (-2, 12.25) {};
		\node [style=none] (34) at (-1, 12.25) {};
		\node [style=none] (35) at (0, 12.25) {};
		\node [style=Z] (36) at (-2, 11) {};
		\node [style=none] (37) at (-1, 9.5) {};
		\node [style=andout] (38) at (-1.5, 10) {};
		\node [style=none] (39) at (0.5, 9.5) {};
		\node [style=none] (40) at (-2, 9.5) {};
		\node [style=none] (41) at (0, 9.5) {};
		\node [style=none] (42) at (-0.5, 10) {};
		\node [style=Z] (43) at (0, 10) {};
		\node [style=X] (44) at (0, 11) {};
		\node [style=none] (45) at (-0.5, 11) {};
	\end{pgfonlayer}
	\begin{pgfonlayer}{edgelayer}
		\draw (31.center) to (39.center);
		\draw (34.center) to (37.center);
		\draw (40.center) to (33.center);
		\draw (32) to (29.center);
		\draw (29.center) to (36);
		\draw [in=90, out=-117] (44) to (42.center);
		\draw [in=117, out=-90] (45.center) to (43);
		\draw [in=-60, out=60, looseness=1.25] (43) to (44);
		\draw (43) to (41.center);
		\draw [in=-124, out=90] (44) to (30);
		\draw [in=-90, out=-90] (42.center) to (29.center);
		\draw [in=90, out=-90] (35.center) to (45.center);
	\end{pgfonlayer}
\end{tikzpicture}
\eq{\ref{ZXA.1}}
\begin{tikzpicture}
	\begin{pgfonlayer}{nodelayer}
		\node [style=none] (30) at (-1.5, 10.75) {};
		\node [style=X] (31) at (0, 10.25) {};
		\node [style=none] (32) at (0, 12.5) {};
		\node [style=Z] (33) at (-1, 11.75) {};
		\node [style=none] (34) at (-2, 12.5) {};
		\node [style=none] (35) at (-1, 12.5) {};
		\node [style=none] (36) at (-0.5, 12.5) {};
		\node [style=Z] (37) at (-2, 11.75) {};
		\node [style=none] (38) at (-1, 9.5) {};
		\node [style=andout] (39) at (-1.5, 10.75) {};
		\node [style=none] (40) at (0, 9.5) {};
		\node [style=none] (41) at (-2, 9.5) {};
		\node [style=none] (42) at (-0.5, 9.5) {};
		\node [style=Z] (43) at (-0.5, 11.75) {};
		\node [style=X] (44) at (0, 11.75) {};
	\end{pgfonlayer}
	\begin{pgfonlayer}{edgelayer}
		\draw (32.center) to (40.center);
		\draw (35.center) to (38.center);
		\draw (41.center) to (34.center);
		\draw (33) to (30.center);
		\draw (30.center) to (37);
		\draw (43) to (44);
		\draw (43) to (42.center);
		\draw [in=-90, out=180, looseness=0.75] (31) to (30.center);
		\draw (36.center) to (43);
	\end{pgfonlayer}
\end{tikzpicture}
=
\left\llbracket
\begin{tikzpicture}
	\begin{pgfonlayer}{nodelayer}
		\node [style=nothing] (31) at (0, 9.5) {};
		\node [style=nothing] (32) at (-1, 9.5) {};
		\node [style=nothing] (33) at (-0.5, 9.5) {};
		\node [style=nothing] (34) at (-1.5, 9.5) {};
		\node [style=dot] (35) at (-1.5, 10) {};
		\node [style=dot] (36) at (-1, 10) {};
		\node [style=oplus] (37) at (0, 10) {};
		\node [style=nothing] (38) at (-0.5, 11) {};
		\node [style=nothing] (39) at (-1, 11) {};
		\node [style=nothing] (40) at (0, 11) {};
		\node [style=nothing] (41) at (-1.5, 11) {};
		\node [style=dot] (42) at (-0.5, 10.5) {};
		\node [style=oplus] (43) at (0, 10.5) {};
	\end{pgfonlayer}
	\begin{pgfonlayer}{edgelayer}
		\draw (31) to (37);
		\draw (32) to (36);
		\draw (35) to (34);
		\draw (35) to (36);
		\draw (36) to (37);
		\draw (42) to (43);
		\draw (43) to (40);
		\draw (43) to (37);
		\draw (33) to (42);
		\draw (35) to (41);
		\draw (39) to (36);
		\draw (42) to (38);
	\end{pgfonlayer}
\end{tikzpicture}
\right\rrbracket_{\hat{\TOF}}
\end{align*}
\endgroup

\item[\ref{TOF.14}:]
\begin{align*}
\left\llbracket
\begin{tikzpicture}
	\begin{pgfonlayer}{nodelayer}
		\node [style=nothing] (32) at (0, 9.5) {};
		\node [style=nothing] (33) at (-0.5, 9.5) {};
		\node [style=nothing] (34) at (-0.5, 11.5) {};
		\node [style=nothing] (35) at (0, 11.5) {};
		\node [style=oplus] (36) at (0, 10) {};
		\node [style=oplus] (37) at (0, 11) {};
		\node [style=oplus] (38) at (-0.5, 10.5) {};
		\node [style=dot] (39) at (-0.5, 11) {};
		\node [style=dot] (40) at (0, 10.5) {};
		\node [style=dot] (41) at (-0.5, 10) {};
	\end{pgfonlayer}
	\begin{pgfonlayer}{edgelayer}
		\draw (33) to (41);
		\draw (41) to (38);
		\draw (38) to (39);
		\draw (39) to (34);
		\draw (35) to (37);
		\draw (37) to (40);
		\draw (40) to (36);
		\draw (36) to (32);
		\draw (36) to (41);
		\draw (40) to (38);
		\draw (37) to (39);
	\end{pgfonlayer}
\end{tikzpicture}
\right\rrbracket_{\hat{\TOF}}
&=
\begin{tikzpicture}
	\begin{pgfonlayer}{nodelayer}
		\node [style=Z] (33) at (-3.25, 10) {};
		\node [style=X] (34) at (-2.75, 10) {};
		\node [style=X] (35) at (-3.25, 10.5) {};
		\node [style=X] (36) at (-2.75, 11) {};
		\node [style=Z] (37) at (-3.25, 11) {};
		\node [style=none] (38) at (-3.25, 11.5) {};
		\node [style=none] (39) at (-2.75, 11.5) {};
		\node [style=none] (40) at (-3.25, 9.5) {};
		\node [style=none] (41) at (-2.75, 9.5) {};
		\node [style=Z] (42) at (-2.75, 10.5) {};
	\end{pgfonlayer}
	\begin{pgfonlayer}{edgelayer}
		\draw (39.center) to (36);
		\draw (36) to (42);
		\draw (42) to (34);
		\draw (34) to (41.center);
		\draw (40.center) to (33);
		\draw (33) to (34);
		\draw (42) to (35);
		\draw (35) to (33);
		\draw (35) to (37);
		\draw (37) to (36);
		\draw (37) to (38.center);
	\end{pgfonlayer}
\end{tikzpicture}
=
\begin{tikzpicture}
	\begin{pgfonlayer}{nodelayer}
		\node [style=Z] (34) at (-3.25, 10) {};
		\node [style=X] (35) at (-2.75, 10) {};
		\node [style=X] (36) at (-2.75, 11) {};
		\node [style=X] (37) at (-2.75, 12) {};
		\node [style=Z] (38) at (-3.25, 12) {};
		\node [style=none] (39) at (-3.25, 12.5) {};
		\node [style=none] (40) at (-2.75, 12.5) {};
		\node [style=none] (41) at (-3.25, 9.5) {};
		\node [style=none] (42) at (-2.75, 9.5) {};
		\node [style=Z] (43) at (-3.25, 11) {};
	\end{pgfonlayer}
	\begin{pgfonlayer}{edgelayer}
		\draw (40.center) to (37);
		\draw [in=90, out=-90] (37) to (43);
		\draw [in=90, out=-90] (43) to (35);
		\draw (35) to (42.center);
		\draw (41.center) to (34);
		\draw (34) to (35);
		\draw (43) to (36);
		\draw [in=90, out=-90] (36) to (34);
		\draw [in=-90, out=90] (36) to (38);
		\draw (38) to (37);
		\draw (38) to (39.center);
	\end{pgfonlayer}
\end{tikzpicture}
\eq{\ref{ZXA.5}}
\begin{tikzpicture}
	\begin{pgfonlayer}{nodelayer}
		\node [style=X] (35) at (-2.75, 11.25) {};
		\node [style=Z] (36) at (-3.25, 11.25) {};
		\node [style=none] (37) at (-3.25, 11.75) {};
		\node [style=none] (38) at (-2.75, 11.75) {};
		\node [style=none] (39) at (-3.25, 9.5) {};
		\node [style=none] (40) at (-2.75, 9.5) {};
		\node [style=X] (41) at (-3.25, 10) {};
		\node [style=Z] (42) at (-2.75, 10) {};
	\end{pgfonlayer}
	\begin{pgfonlayer}{edgelayer}
		\draw (38.center) to (35);
		\draw (36) to (35);
		\draw (36) to (37.center);
		\draw (42) to (41);
		\draw [in=-90, out=90] (41) to (35);
		\draw [in=-90, out=90] (42) to (36);
		\draw (41) to (39.center);
		\draw (40.center) to (42);
	\end{pgfonlayer}
\end{tikzpicture}
=
\begin{tikzpicture}
	\begin{pgfonlayer}{nodelayer}
		\node [style=X] (36) at (-3.25, 10.75) {};
		\node [style=Z] (37) at (-2.75, 10.75) {};
		\node [style=none] (38) at (-3.25, 11.75) {};
		\node [style=none] (39) at (-2.75, 11.75) {};
		\node [style=none] (40) at (-3.25, 9.5) {};
		\node [style=none] (41) at (-2.75, 9.5) {};
		\node [style=X] (42) at (-3.25, 10) {};
		\node [style=Z] (43) at (-2.75, 10) {};
	\end{pgfonlayer}
	\begin{pgfonlayer}{edgelayer}
		\draw [in=90, out=-90] (39.center) to (36);
		\draw (37) to (36);
		\draw [in=-90, out=90] (37) to (38.center);
		\draw (43) to (42);
		\draw (42) to (36);
		\draw (43) to (37);
		\draw (42) to (40.center);
		\draw (41.center) to (43);
	\end{pgfonlayer}
\end{tikzpicture}
\eq{\ref{ZXA.1},\ref{ZXA.3},\ref{ZXA.15}}
\begin{tikzpicture}
	\begin{pgfonlayer}{nodelayer}
		\node [style=nothing] (37) at (0, 9.5) {};
		\node [style=nothing] (38) at (-0.5, 9.5) {};
		\node [style=nothing] (39) at (-0.5, 10.5) {};
		\node [style=nothing] (40) at (0, 10.5) {};
	\end{pgfonlayer}
	\begin{pgfonlayer}{edgelayer}
		\draw [in=-90, out=90, looseness=1.25] (38) to (40);
		\draw [in=-90, out=90, looseness=1.25] (37) to (39);
	\end{pgfonlayer}
\end{tikzpicture}
=
\left\llbracket
\begin{tikzpicture}
	\begin{pgfonlayer}{nodelayer}
		\node [style=nothing] (38) at (0, 9.5) {};
		\node [style=nothing] (39) at (-0.5, 9.5) {};
		\node [style=nothing] (40) at (-0.5, 10.5) {};
		\node [style=nothing] (41) at (0, 10.5) {};
	\end{pgfonlayer}
	\begin{pgfonlayer}{edgelayer}
		\draw [in=-90, out=90, looseness=1.25] (39) to (41);
		\draw [in=-90, out=90, looseness=1.25] (38) to (40);
	\end{pgfonlayer}
\end{tikzpicture}
\right\rrbracket_{\hat{\TOF}}
\end{align*}

\item[\ref{TOF.15}:]
\begin{align*}
\left\llbracket
\begin{tikzpicture}
	\begin{pgfonlayer}{nodelayer}
		\node [style=nothing] (39) at (-1.75, 9.5) {};
		\node [style=nothing] (40) at (-1.25, 9.5) {};
		\node [style=nothing] (41) at (-0.75, 9.5) {};
		\node [style=nothing] (42) at (-1.75, 11.5) {};
		\node [style=nothing] (43) at (-1.25, 11.5) {};
		\node [style=nothing] (44) at (-0.75, 11.5) {};
		\node [style=dot] (45) at (-1.75, 10.5) {};
		\node [style=dot] (46) at (-1.25, 10.5) {};
		\node [style=oplus] (47) at (-0.75, 10.5) {};
	\end{pgfonlayer}
	\begin{pgfonlayer}{edgelayer}
		\draw (39) to (45);
		\draw (45) to (42);
		\draw (43) to (46);
		\draw (46) to (40);
		\draw (41) to (47);
		\draw (47) to (44);
		\draw (47) to (46);
		\draw (46) to (45);
	\end{pgfonlayer}
\end{tikzpicture}
\right\rrbracket_{\hat{\TOF}}
&=
\begin{tikzpicture}
	\begin{pgfonlayer}{nodelayer}
		\node [style=none] (40) at (-2, 9.5) {};
		\node [style=none] (41) at (-1, 9.5) {};
		\node [style=none] (42) at (-0.5, 9.5) {};
		\node [style=Z] (43) at (-2, 10) {};
		\node [style=Z] (44) at (-1, 10) {};
		\node [style=andin] (45) at (-1.5, 10.75) {};
		\node [style=X] (46) at (-0.5, 11.5) {};
		\node [style=none] (47) at (-0.5, 12) {};
		\node [style=none] (48) at (-2, 12) {};
		\node [style=none] (49) at (-1, 12) {};
		\node [style=none] (50) at (-1.5, 10.75) {};
	\end{pgfonlayer}
	\begin{pgfonlayer}{edgelayer}
		\draw [in=90, out=180] (46) to (50.center);
		\draw (50.center) to (43);
		\draw (43) to (40.center);
		\draw (43) to (48.center);
		\draw (49.center) to (44);
		\draw (44) to (41.center);
		\draw (42.center) to (46);
		\draw (46) to (47.center);
		\draw (50.center) to (44);
	\end{pgfonlayer}
\end{tikzpicture}
\eq{\ref{ZXA.11}}
\begin{tikzpicture}
	\begin{pgfonlayer}{nodelayer}
		\node [style=none] (41) at (-2, 9.5) {};
		\node [style=none] (42) at (-1, 9.5) {};
		\node [style=none] (43) at (-0.5, 9.5) {};
		\node [style=Z] (44) at (-2, 10) {};
		\node [style=Z] (45) at (-1, 10) {};
		\node [style=andin] (46) at (-1.5, 11.25) {};
		\node [style=X] (47) at (-0.5, 12) {};
		\node [style=none] (48) at (-0.5, 12.5) {};
		\node [style=none] (49) at (-2, 12.5) {};
		\node [style=none] (50) at (-1, 12.5) {};
		\node [style=none] (51) at (-1.5, 11.25) {};
		\node [style=none] (52) at (-1.75, 10.5) {};
		\node [style=none] (53) at (-1.25, 10.5) {};
	\end{pgfonlayer}
	\begin{pgfonlayer}{edgelayer}
		\draw [in=90, out=180] (47) to (51.center);
		\draw (44) to (41.center);
		\draw (44) to (49.center);
		\draw (50.center) to (45);
		\draw (45) to (42.center);
		\draw (43.center) to (47);
		\draw (47) to (48.center);
		\draw [in=90, out=-108] (51.center) to (52.center);
		\draw [in=146, out=-90] (52.center) to (45);
		\draw [in=34, out=-90, looseness=0.75] (53.center) to (44);
		\draw [in=-72, out=90] (53.center) to (51.center);
	\end{pgfonlayer}
\end{tikzpicture}
=
\begin{tikzpicture}
	\begin{pgfonlayer}{nodelayer}
		\node [style=none] (42) at (-2, 9.5) {};
		\node [style=none] (43) at (-1, 9.5) {};
		\node [style=none] (44) at (-0.5, 9.5) {};
		\node [style=Z] (45) at (-1, 10.75) {};
		\node [style=Z] (46) at (-2, 10.75) {};
		\node [style=andin] (47) at (-1.5, 11.5) {};
		\node [style=X] (48) at (-0.5, 12.5) {};
		\node [style=none] (49) at (-0.5, 13) {};
		\node [style=none] (50) at (-2, 13) {};
		\node [style=none] (51) at (-1, 13) {};
		\node [style=none] (52) at (-1.5, 11.5) {};
		\node [style=none] (53) at (-2, 12) {};
		\node [style=none] (54) at (-1, 12) {};
	\end{pgfonlayer}
	\begin{pgfonlayer}{edgelayer}
		\draw [in=90, out=180] (48) to (52.center);
		\draw [in=90, out=-90] (45) to (42.center);
		\draw [in=90, out=-90] (46) to (43.center);
		\draw (44.center) to (48);
		\draw (48) to (49.center);
		\draw [in=90, out=-90] (51.center) to (53.center);
		\draw [in=-90, out=90] (54.center) to (50.center);
		\draw (54.center) to (45);
		\draw (46) to (52.center);
		\draw (52.center) to (45);
		\draw (46) to (53.center);
	\end{pgfonlayer}
\end{tikzpicture}
=
\left\llbracket
\begin{tikzpicture}
	\begin{pgfonlayer}{nodelayer}
		\node [style=nothing] (43) at (-1.75, 9.5) {};
		\node [style=nothing] (44) at (-1.25, 9.5) {};
		\node [style=nothing] (45) at (-0.75, 9.5) {};
		\node [style=dot] (46) at (-1.75, 10.5) {};
		\node [style=dot] (47) at (-1.25, 10.5) {};
		\node [style=oplus] (48) at (-0.75, 10.5) {};
		\node [style=nothing] (49) at (-1.75, 11.5) {};
		\node [style=nothing] (50) at (-1.25, 11.5) {};
		\node [style=nothing] (51) at (-0.75, 11.5) {};
	\end{pgfonlayer}
	\begin{pgfonlayer}{edgelayer}
		\draw [in=-90, out=90, looseness=1.25] (43) to (47);
		\draw [in=-90, out=90, looseness=1.25] (47) to (49);
		\draw [in=-90, out=90, looseness=1.25] (46) to (50);
		\draw [in=90, out=-90, looseness=1.25] (46) to (44);
		\draw (45) to (48);
		\draw (48) to (51);
		\draw (46) to (47);
		\draw (47) to (48);
	\end{pgfonlayer}
\end{tikzpicture}
\right\rrbracket_{\hat{\TOF}}
\end{align*}


\item[\ref{TOF.16}:]

\begingroup
\allowdisplaybreaks
\begin{align*}
&\left\llbracket
\begin{tikzpicture}
	\begin{pgfonlayer}{nodelayer}
		\node [style=nothing] (44) at (0, 9.5) {};
		\node [style=nothing] (45) at (-0.5, 9.5) {};
		\node [style=nothing] (46) at (-1.5, 9.5) {};
		\node [style=nothing] (47) at (-2, 9.5) {};
		\node [style=zeroin] (48) at (-1, 9.5) {};
		\node [style=oplus] (49) at (-1, 10) {};
		\node [style=oplus] (50) at (-1, 11) {};
		\node [style=dot] (51) at (-1, 10.5) {};
		\node [style=dot] (52) at (-0.5, 10.5) {};
		\node [style=dot] (53) at (-1.5, 10) {};
		\node [style=dot] (54) at (-2, 10) {};
		\node [style=dot] (55) at (-1.5, 11) {};
		\node [style=dot] (56) at (-2, 11) {};
		\node [style=oplus] (57) at (0, 10.5) {};
		\node [style=zeroout] (58) at (-1, 11.5) {};
		\node [style=nothing] (59) at (0, 11.5) {};
		\node [style=nothing] (60) at (-2, 11.5) {};
		\node [style=nothing] (61) at (-0.5, 11.5) {};
		\node [style=nothing] (62) at (-1.5, 11.5) {};
	\end{pgfonlayer}
	\begin{pgfonlayer}{edgelayer}
		\draw (47) to (54);
		\draw (54) to (56);
		\draw (56) to (60);
		\draw (55) to (53);
		\draw (59) to (57);
		\draw (57) to (44);
		\draw (57) to (52);
		\draw (52) to (51);
		\draw (53) to (49);
		\draw (53) to (54);
		\draw (56) to (55);
		\draw (50) to (55);
		\draw (48) to (49);
		\draw (49) to (51);
		\draw (51) to (50);
		\draw (58) to (50);
		\draw [style=simple] (61) to (52);
		\draw [style=simple] (52) to (45);
		\draw [style=simple] (46) to (53);
		\draw [style=simple] (55) to (62);
	\end{pgfonlayer}
\end{tikzpicture}
\right\rrbracket_{\hat{\TOF}}
=
\begin{tikzpicture}
	\begin{pgfonlayer}{nodelayer}
		\node [style=Z] (0) at (0.55, 10) {};
		\node [style=Z] (1) at (1.45, 10) {};
		\node [style=none] (2) at (1, 10.75) {};
		\node [style=X] (3) at (1.95, 11.5) {};
		\node [style=X] (4) at (1.95, 10.75) {};
		\node [style=X] (5) at (1.95, 14) {};
		\node [style=Z] (6) at (1.45, 12) {};
		\node [style=X] (7) at (1.95, 13.5) {};
		\node [style=Z] (8) at (0.55, 12) {};
		\node [style=none] (9) at (1, 12.75) {};
		\node [style=Z] (10) at (2.85, 12) {};
		\node [style=X] (11) at (3.3, 13.5) {};
		\node [style=Z] (12) at (1.95, 12) {};
		\node [style=none] (13) at (2.4, 12.75) {};
		\node [style=none] (14) at (0.55, 9.5) {};
		\node [style=none] (15) at (1.45, 9.5) {};
		\node [style=none] (16) at (2.85, 9.5) {};
		\node [style=none] (17) at (3.3, 9.5) {};
		\node [style=none] (18) at (1.45, 14.5) {};
		\node [style=none] (19) at (0.55, 14.5) {};
		\node [style=none] (20) at (3.3, 14.5) {};
		\node [style=none] (21) at (2.85, 14.5) {};
		\node [style=andin] (22) at (1, 10.75) {};
		\node [style=andin] (23) at (1, 12.75) {};
		\node [style=andin] (24) at (2.4, 12.75) {};
	\end{pgfonlayer}
	\begin{pgfonlayer}{edgelayer}
		\draw (1) to (2.center);
		\draw (2.center) to (0);
		\draw (3) to (4);
		\draw [in=90, out=180] (3) to (2.center);
		\draw (6) to (9.center);
		\draw (9.center) to (8);
		\draw (7) to (5);
		\draw [in=90, out=180] (7) to (9.center);
		\draw (10) to (13.center);
		\draw (13.center) to (12);
		\draw [in=90, out=180] (11) to (13.center);
		\draw (20.center) to (17.center);
		\draw (16.center) to (21.center);
		\draw (7) to (12);
		\draw (12) to (3);
		\draw (18.center) to (15.center);
		\draw (14.center) to (19.center);
	\end{pgfonlayer}
\end{tikzpicture}
\eq{\ref{ZXA.1}}
\begin{tikzpicture}
	\begin{pgfonlayer}{nodelayer}
		\node [style=Z] (25) at (5.125, 11.75) {};
		\node [style=Z] (26) at (5.975, 11.75) {};
		\node [style=none] (27) at (5.55, 11) {};
		\node [style=Z] (28) at (5.975, 11.75) {};
		\node [style=Z] (29) at (5.125, 11.75) {};
		\node [style=none] (30) at (5.55, 12.5) {};
		\node [style=Z] (31) at (7.375, 11.75) {};
		\node [style=X] (32) at (7.825, 13.25) {};
		\node [style=Z] (33) at (6.475, 11.75) {};
		\node [style=none] (34) at (6.925, 12.5) {};
		\node [style=none] (35) at (5.125, 9.75) {};
		\node [style=none] (36) at (5.975, 9.75) {};
		\node [style=none] (37) at (7.375, 9.75) {};
		\node [style=none] (38) at (7.825, 9.75) {};
		\node [style=none] (39) at (5.975, 13.75) {};
		\node [style=none] (40) at (5.125, 13.75) {};
		\node [style=none] (41) at (7.825, 13.75) {};
		\node [style=none] (42) at (7.375, 13.75) {};
		\node [style=andout] (43) at (5.55, 11) {};
		\node [style=andin] (44) at (5.55, 12.5) {};
		\node [style=andin] (45) at (6.925, 12.5) {};
		\node [style=none] (46) at (5.55, 13) {};
		\node [style=none] (47) at (6.475, 13) {};
		\node [style=none] (48) at (5.55, 10.5) {};
		\node [style=none] (49) at (6.475, 10.5) {};
	\end{pgfonlayer}
	\begin{pgfonlayer}{edgelayer}
		\draw (26) to (27.center);
		\draw (27.center) to (25);
		\draw (28) to (30.center);
		\draw (30.center) to (29);
		\draw (31) to (34.center);
		\draw (34.center) to (33);
		\draw [in=90, out=180] (32) to (34.center);
		\draw (41.center) to (38.center);
		\draw (37.center) to (42.center);
		\draw (39.center) to (36.center);
		\draw (35.center) to (40.center);
		\draw [in=90, out=90, looseness=1.25] (47.center) to (46.center);
		\draw (46.center) to (30.center);
		\draw (27.center) to (48.center);
		\draw [in=-90, out=-90, looseness=1.25] (48.center) to (49.center);
		\draw (49.center) to (33);
		\draw (33) to (47.center);
	\end{pgfonlayer}
\end{tikzpicture}
\eq{\ref{ZXA.3}}
\begin{tikzpicture}
	\begin{pgfonlayer}{nodelayer}
		\node [style=Z] (77) at (14.4, 11.5) {};
		\node [style=Z] (78) at (15.25, 11.5) {};
		\node [style=none] (79) at (14.825, 10.75) {};
		\node [style=Z] (80) at (15.25, 11.5) {};
		\node [style=Z] (81) at (14.4, 11.5) {};
		\node [style=none] (82) at (14.825, 12.25) {};
		\node [style=Z] (83) at (17.15, 11.5) {};
		\node [style=X] (84) at (17.6, 13) {};
		\node [style=Z] (85) at (16.25, 11.5) {};
		\node [style=none] (86) at (16.7, 12.25) {};
		\node [style=none] (87) at (13.9, 9.5) {};
		\node [style=none] (88) at (15.75, 9.5) {};
		\node [style=none] (89) at (17.15, 9.5) {};
		\node [style=none] (90) at (17.6, 9.5) {};
		\node [style=none] (91) at (15.75, 13.5) {};
		\node [style=none] (92) at (13.9, 13.5) {};
		\node [style=none] (93) at (17.6, 13.5) {};
		\node [style=none] (94) at (17.15, 13.5) {};
		\node [style=andout] (95) at (14.825, 10.75) {};
		\node [style=andin] (96) at (14.825, 12.25) {};
		\node [style=andin] (97) at (16.7, 12.25) {};
		\node [style=none] (98) at (14.825, 12.75) {};
		\node [style=none] (99) at (16.25, 12.75) {};
		\node [style=none] (100) at (14.825, 10.25) {};
		\node [style=none] (101) at (16.25, 10.25) {};
		\node [style=Z] (102) at (15.75, 11.5) {};
		\node [style=Z] (103) at (13.9, 11.5) {};
	\end{pgfonlayer}
	\begin{pgfonlayer}{edgelayer}
		\draw (78) to (79.center);
		\draw (79.center) to (77);
		\draw (80) to (82.center);
		\draw (82.center) to (81);
		\draw (83) to (86.center);
		\draw (86.center) to (85);
		\draw [in=90, out=180] (84) to (86.center);
		\draw (93.center) to (90.center);
		\draw (89.center) to (94.center);
		\draw (91.center) to (88.center);
		\draw (87.center) to (92.center);
		\draw [in=90, out=90, looseness=1.25] (99.center) to (98.center);
		\draw (98.center) to (82.center);
		\draw (79.center) to (100.center);
		\draw [in=-90, out=-90, looseness=1.25] (100.center) to (101.center);
		\draw (101.center) to (85);
		\draw (85) to (99.center);
		\draw (102) to (78);
		\draw (77) to (103);
	\end{pgfonlayer}
\end{tikzpicture}\\
&=
\begin{tikzpicture}
	\begin{pgfonlayer}{nodelayer}
		\node [style=Z] (151) at (30.15, 11.5) {};
		\node [style=none] (152) at (31.05, 9.5) {};
		\node [style=andin] (153) at (30.6, 12.25) {};
		\node [style=none] (154) at (30.15, 12.75) {};
		\node [style=none] (155) at (31.5, 9.5) {};
		\node [style=none] (156) at (29.225, 10.75) {};
		\node [style=Z] (157) at (31.05, 11.5) {};
		\node [style=none] (158) at (31.5, 13.5) {};
		\node [style=none] (159) at (29.725, 13.5) {};
		\node [style=none] (160) at (27.525, 9.75) {};
		\node [style=none] (161) at (27.975, 12.75) {};
		\node [style=Z] (162) at (29.725, 12.5) {};
		\node [style=X] (163) at (31.5, 13) {};
		\node [style=none] (164) at (30.6, 12.25) {};
		\node [style=none] (165) at (30.15, 10.25) {};
		\node [style=none] (166) at (29.725, 9.5) {};
		\node [style=none] (167) at (29.225, 10.25) {};
		\node [style=none] (168) at (27.525, 13.5) {};
		\node [style=Z] (169) at (27.525, 12.5) {};
		\node [style=none] (170) at (31.05, 13.5) {};
		\node [style=andout] (171) at (28.475, 10.75) {};
		\node [style=none] (172) at (28.475, 10.25) {};
		\node [style=none] (173) at (28.475, 10.75) {};
		\node [style=Z] (174) at (28.475, 11.75) {};
		\node [style=Z] (175) at (29.225, 11.75) {};
		\node [style=andout] (176) at (29.225, 10.75) {};
		\node [style=none] (177) at (27.975, 10.25) {};
	\end{pgfonlayer}
	\begin{pgfonlayer}{edgelayer}
		\draw (157) to (164.center);
		\draw (164.center) to (151);
		\draw [in=90, out=180] (163) to (164.center);
		\draw (158.center) to (155.center);
		\draw (152.center) to (170.center);
		\draw (159.center) to (166.center);
		\draw (160.center) to (168.center);
		\draw [in=90, out=90, looseness=0.50] (154.center) to (161.center);
		\draw (156.center) to (167.center);
		\draw [in=-90, out=-90, looseness=1.25] (167.center) to (165.center);
		\draw (165.center) to (151);
		\draw (151) to (154.center);
		\draw (173.center) to (172.center);
		\draw (175) to (173.center);
		\draw [in=-120, out=120, looseness=1.25] (173.center) to (174);
		\draw [in=60, out=-60, looseness=1.25] (175) to (156.center);
		\draw (156.center) to (174);
		\draw [in=0, out=90] (174) to (169);
		\draw (161.center) to (177.center);
		\draw [bend right=90, looseness=1.50] (177.center) to (172.center);
		\draw [in=90, out=180] (162) to (175);
	\end{pgfonlayer}
\end{tikzpicture}
\eq{\ref{ZXA.12}}
\begin{tikzpicture}
	\begin{pgfonlayer}{nodelayer}
		\node [style=none] (178) at (32.5, 9.75) {};
		\node [style=Z] (179) at (33.925, 12.5) {};
		\node [style=X] (180) at (35.775, 13) {};
		\node [style=none] (181) at (35.775, 9.5) {};
		\node [style=none] (182) at (35.325, 13.5) {};
		\node [style=none] (183) at (32.925, 12.75) {};
		\node [style=andout] (184) at (33.425, 11.75) {};
		\node [style=Z] (185) at (33.425, 10.75) {};
		\node [style=none] (186) at (34.825, 12.25) {};
		\node [style=none] (187) at (33.925, 13.5) {};
		\node [style=none] (188) at (35.775, 13.5) {};
		\node [style=andin] (189) at (34.825, 12.25) {};
		\node [style=none] (190) at (32.925, 10.75) {};
		\node [style=none] (191) at (33.425, 11.75) {};
		\node [style=none] (192) at (34.325, 10.75) {};
		\node [style=none] (193) at (32.5, 13.5) {};
		\node [style=none] (194) at (35.325, 9.5) {};
		\node [style=Z] (195) at (34.325, 11.5) {};
		\node [style=none] (196) at (33.925, 9.5) {};
		\node [style=Z] (197) at (35.325, 11.5) {};
		\node [style=none] (198) at (34.325, 12.75) {};
		\node [style=Z] (199) at (32.5, 12.5) {};
	\end{pgfonlayer}
	\begin{pgfonlayer}{edgelayer}
		\draw (197) to (186.center);
		\draw (186.center) to (195);
		\draw [in=90, out=180] (180) to (186.center);
		\draw (188.center) to (181.center);
		\draw (194.center) to (182.center);
		\draw (187.center) to (196.center);
		\draw (178.center) to (193.center);
		\draw [in=90, out=90, looseness=0.50] (198.center) to (183.center);
		\draw (192.center) to (195);
		\draw (195) to (198.center);
		\draw (183.center) to (190.center);
		\draw [in=-60, out=-90] (192.center) to (185);
		\draw [in=-90, out=-105, looseness=1.75] (185) to (190.center);
		\draw (191.center) to (185);
		\draw (191.center) to (199);
		\draw (191.center) to (179);
	\end{pgfonlayer}
\end{tikzpicture}
\eq{\ref{ZXA.3}}
\begin{tikzpicture}
	\begin{pgfonlayer}{nodelayer}
		\node [style=none] (49) at (0.25, 9.5) {};
		\node [style=andin] (50) at (-0.25, 11.25) {};
		\node [style=none] (51) at (0.75, 9.5) {};
		\node [style=Z] (52) at (0.25, 10.25) {};
		\node [style=none] (53) at (0.75, 12.75) {};
		\node [style=none] (54) at (-0.75, 12.75) {};
		\node [style=none] (55) at (-1.75, 9.5) {};
		\node [style=Z] (56) at (-0.75, 12) {};
		\node [style=X] (57) at (0.75, 12) {};
		\node [style=none] (58) at (-0.25, 11.25) {};
		\node [style=none] (59) at (-0.75, 9.5) {};
		\node [style=none] (60) at (-1.75, 12.75) {};
		\node [style=Z] (61) at (-1.75, 12) {};
		\node [style=none] (62) at (0.25, 12.75) {};
		\node [style=none] (63) at (-1.25, 11.25) {};
		\node [style=andout] (64) at (-1.25, 11.25) {};
	\end{pgfonlayer}
	\begin{pgfonlayer}{edgelayer}
		\draw (52) to (58.center);
		\draw [in=90, out=180] (57) to (58.center);
		\draw (53.center) to (51.center);
		\draw (49.center) to (62.center);
		\draw (54.center) to (59.center);
		\draw (55.center) to (60.center);
		\draw (63.center) to (61);
		\draw (63.center) to (56);
		\draw [in=-90, out=-120, looseness=2.00] (58.center) to (63.center);
	\end{pgfonlayer}
\end{tikzpicture}
=
\begin{tikzpicture}
	\begin{pgfonlayer}{nodelayer}
		\node [style=none] (50) at (0.25, 9.5) {};
		\node [style=none] (51) at (0.75, 9.5) {};
		\node [style=Z] (52) at (0.25, 11.25) {};
		\node [style=none] (53) at (0.75, 13.5) {};
		\node [style=none] (54) at (-0.75, 13.5) {};
		\node [style=none] (55) at (-1.75, 9.5) {};
		\node [style=Z] (56) at (-0.75, 10.25) {};
		\node [style=X] (57) at (0.75, 13) {};
		\node [style=none] (58) at (-0.75, 9.5) {};
		\node [style=none] (59) at (-1.75, 13.5) {};
		\node [style=Z] (60) at (-1.75, 10.25) {};
		\node [style=none] (61) at (0.25, 13.5) {};
		\node [style=none] (62) at (-1.25, 11.25) {};
		\node [style=andin] (63) at (-1.25, 11.25) {};
		\node [style=none] (64) at (-0.25, 12.25) {};
		\node [style=andin] (65) at (-0.25, 12.25) {};
	\end{pgfonlayer}
	\begin{pgfonlayer}{edgelayer}
		\draw (53.center) to (51.center);
		\draw (50.center) to (61.center);
		\draw (54.center) to (58.center);
		\draw (55.center) to (59.center);
		\draw [in=90, out=180, looseness=1.50] (57) to (64.center);
		\draw (64.center) to (52);
		\draw [in=-124, out=90] (62.center) to (64.center);
		\draw (62.center) to (56);
		\draw (60) to (62.center);
	\end{pgfonlayer}
\end{tikzpicture}\\
&
\eq{\ref{ZXA.11}}
\begin{tikzpicture}
	\begin{pgfonlayer}{nodelayer}
		\node [style=none] (51) at (0.25, 9.5) {};
		\node [style=none] (52) at (0.75, 9.5) {};
		\node [style=Z] (53) at (0.25, 11.75) {};
		\node [style=none] (54) at (0.75, 14.25) {};
		\node [style=none] (55) at (-0.75, 14.25) {};
		\node [style=none] (56) at (-1.75, 9.5) {};
		\node [style=Z] (57) at (-0.75, 10.25) {};
		\node [style=X] (58) at (0.75, 13.75) {};
		\node [style=none] (59) at (-0.75, 9.5) {};
		\node [style=none] (60) at (-1.75, 14.25) {};
		\node [style=Z] (61) at (-1.75, 10.25) {};
		\node [style=none] (62) at (0.25, 14.25) {};
		\node [style=none] (63) at (-1.25, 11.75) {};
		\node [style=andin] (64) at (-1.25, 11.75) {};
		\node [style=none] (65) at (-0.25, 13) {};
		\node [style=andin] (66) at (-0.25, 13) {};
		\node [style=none] (67) at (-1.5, 11) {};
		\node [style=none] (68) at (-1, 11) {};
	\end{pgfonlayer}
	\begin{pgfonlayer}{edgelayer}
		\draw (54.center) to (52.center);
		\draw (51.center) to (62.center);
		\draw (55.center) to (59.center);
		\draw (56.center) to (60.center);
		\draw [in=90, out=180, looseness=1.50] (58) to (65.center);
		\draw [in=90, out=-120, looseness=1.25] (63.center) to (67.center);
		\draw [in=-60, out=90, looseness=1.25] (68.center) to (63.center);
		\draw [in=45, out=-90] (68.center) to (61);
		\draw [in=135, out=-90] (67.center) to (57);
		\draw (65.center) to (53);
		\draw [in=90, out=-129] (65.center) to (63.center);
	\end{pgfonlayer}
\end{tikzpicture}
\eq{\ref{ZXA.9}}
\begin{tikzpicture}
	\begin{pgfonlayer}{nodelayer}
		\node [style=none] (52) at (-0.25, 9.5) {};
		\node [style=none] (53) at (0.25, 9.5) {};
		\node [style=Z] (54) at (-0.25, 11.5) {};
		\node [style=none] (55) at (0.25, 14.25) {};
		\node [style=none] (56) at (-0.75, 14.25) {};
		\node [style=none] (57) at (-1.75, 9.5) {};
		\node [style=Z] (58) at (-0.75, 10.25) {};
		\node [style=X] (59) at (0.25, 13.75) {};
		\node [style=none] (60) at (-0.75, 9.5) {};
		\node [style=none] (61) at (-1.75, 14.25) {};
		\node [style=Z] (62) at (-1.75, 10.25) {};
		\node [style=none] (63) at (-0.25, 14.25) {};
		\node [style=none] (64) at (-1.5, 13) {};
		\node [style=andin] (65) at (-1.5, 13) {};
		\node [style=none] (66) at (-1.5, 11) {};
		\node [style=none] (67) at (-1, 11) {};
		\node [style=andin] (68) at (-1, 12.25) {};
		\node [style=none] (69) at (-1, 12.25) {};
	\end{pgfonlayer}
	\begin{pgfonlayer}{edgelayer}
		\draw (55.center) to (53.center);
		\draw (52.center) to (63.center);
		\draw (56.center) to (60.center);
		\draw (57.center) to (61.center);
		\draw [in=90, out=180, looseness=1.25] (59) to (64.center);
		\draw [in=45, out=-90] (67.center) to (62);
		\draw [in=135, out=-90] (66.center) to (58);
		\draw (69.center) to (54);
		\draw (69.center) to (64.center);
		\draw (64.center) to (66.center);
		\draw [in=90, out=-120, looseness=1.25] (69.center) to (67.center);
	\end{pgfonlayer}
\end{tikzpicture}
\eq{\ref{ZXA.11}}
\begin{tikzpicture}
	\begin{pgfonlayer}{nodelayer}
		\node [style=none] (53) at (0, 9.5) {};
		\node [style=none] (54) at (0.5, 9.5) {};
		\node [style=Z] (55) at (0, 10.75) {};
		\node [style=none] (56) at (0.5, 14.5) {};
		\node [style=none] (57) at (-0.5, 14.5) {};
		\node [style=none] (58) at (-2, 9.5) {};
		\node [style=Z] (59) at (-0.5, 10.25) {};
		\node [style=X] (60) at (0.5, 14) {};
		\node [style=none] (61) at (-0.5, 9.5) {};
		\node [style=none] (62) at (-2, 14.5) {};
		\node [style=Z] (63) at (-2, 10.25) {};
		\node [style=none] (64) at (0, 14.5) {};
		\node [style=none] (65) at (-1.5, 13.25) {};
		\node [style=andin] (66) at (-1.5, 13.25) {};
		\node [style=none] (67) at (-1.75, 11.75) {};
		\node [style=andin] (68) at (-1, 12.5) {};
		\node [style=none] (69) at (-1, 12.5) {};
		\node [style=none] (70) at (-1.25, 11.75) {};
		\node [style=none] (71) at (-0.75, 11.75) {};
	\end{pgfonlayer}
	\begin{pgfonlayer}{edgelayer}
		\draw (56.center) to (54.center);
		\draw (53.center) to (64.center);
		\draw (57.center) to (61.center);
		\draw (58.center) to (62.center);
		\draw [in=90, out=180, looseness=1.25] (60) to (65.center);
		\draw [in=135, out=-90] (67.center) to (59);
		\draw (69.center) to (65.center);
		\draw [in=90, out=-99] (65.center) to (67.center);
		\draw [in=90, out=-72] (69.center) to (71.center);
		\draw [in=-108, out=90] (70.center) to (69.center);
		\draw [in=149, out=-90] (70.center) to (55);
		\draw [in=56, out=-90] (71.center) to (63);
	\end{pgfonlayer}
\end{tikzpicture}
\eq{\ref{ZXA.9}}
\begin{tikzpicture}
	\begin{pgfonlayer}{nodelayer}
		\node [style=none] (54) at (0, 9.5) {};
		\node [style=none] (55) at (0.5, 9.5) {};
		\node [style=Z] (56) at (0, 10.75) {};
		\node [style=none] (57) at (0.5, 14.5) {};
		\node [style=none] (58) at (-0.5, 14.5) {};
		\node [style=none] (59) at (-2, 9.5) {};
		\node [style=Z] (60) at (-0.5, 10.25) {};
		\node [style=X] (61) at (0.5, 14) {};
		\node [style=none] (62) at (-0.5, 9.5) {};
		\node [style=none] (63) at (-2, 14.5) {};
		\node [style=Z] (64) at (-2, 10.25) {};
		\node [style=none] (65) at (0, 14.5) {};
		\node [style=none] (66) at (-1, 13.25) {};
		\node [style=andin] (67) at (-1, 13.25) {};
		\node [style=none] (68) at (-1.75, 11.75) {};
		\node [style=none] (69) at (-1.25, 11.75) {};
		\node [style=none] (70) at (-0.75, 11.75) {};
		\node [style=none] (71) at (-1.5, 12.5) {};
		\node [style=andin] (72) at (-1.5, 12.5) {};
	\end{pgfonlayer}
	\begin{pgfonlayer}{edgelayer}
		\draw (57.center) to (55.center);
		\draw (54.center) to (65.center);
		\draw (58.center) to (62.center);
		\draw (59.center) to (63.center);
		\draw [in=90, out=180, looseness=1.25] (61) to (66.center);
		\draw [in=135, out=-90] (68.center) to (60);
		\draw [in=149, out=-90] (69.center) to (56);
		\draw [in=56, out=-90] (70.center) to (64);
		\draw [in=90, out=-81] (66.center) to (70.center);
		\draw [in=90, out=-124] (66.center) to (71.center);
		\draw [in=90, out=-72] (71.center) to (69.center);
		\draw [in=90, out=-108] (71.center) to (68.center);
	\end{pgfonlayer}
\end{tikzpicture}\\
&\eq{\ref{ZXA.11}}
\begin{tikzpicture}
	\begin{pgfonlayer}{nodelayer}
		\node [style=none] (55) at (0, 10.5) {};
		\node [style=none] (56) at (0.5, 10.5) {};
		\node [style=Z] (57) at (0, 11.75) {};
		\node [style=none] (58) at (0.5, 16.25) {};
		\node [style=none] (59) at (-0.5, 16.25) {};
		\node [style=none] (60) at (-2, 10.5) {};
		\node [style=Z] (61) at (-0.5, 11.25) {};
		\node [style=X] (62) at (0.5, 15.75) {};
		\node [style=none] (63) at (-0.5, 10.5) {};
		\node [style=none] (64) at (-2, 16.25) {};
		\node [style=Z] (65) at (-2, 11.25) {};
		\node [style=none] (66) at (0, 16.25) {};
		\node [style=none] (67) at (-1, 15) {};
		\node [style=andin] (68) at (-1, 15) {};
		\node [style=none] (69) at (-1.75, 12.75) {};
		\node [style=none] (70) at (-1.25, 12.75) {};
		\node [style=none] (71) at (-0.75, 12.75) {};
		\node [style=none] (72) at (-1.5, 14.25) {};
		\node [style=andin] (73) at (-1.5, 14.25) {};
		\node [style=none] (74) at (-1.75, 13.5) {};
		\node [style=none] (75) at (-1.25, 13.5) {};
	\end{pgfonlayer}
	\begin{pgfonlayer}{edgelayer}
		\draw (58.center) to (56.center);
		\draw (55.center) to (66.center);
		\draw (59.center) to (63.center);
		\draw (60.center) to (64.center);
		\draw [in=90, out=180, looseness=1.25] (62) to (67.center);
		\draw [in=135, out=-90] (69.center) to (61);
		\draw [in=149, out=-90] (70.center) to (57);
		\draw [in=56, out=-90] (71.center) to (65);
		\draw [in=90, out=-81] (67.center) to (71.center);
		\draw [in=90, out=-124] (67.center) to (72.center);
		\draw [in=90, out=-72] (72.center) to (75.center);
		\draw [in=90, out=-108] (72.center) to (74.center);
		\draw [in=90, out=-90] (74.center) to (70.center);
		\draw [in=90, out=-90] (75.center) to (69.center);
	\end{pgfonlayer}
\end{tikzpicture}
=
\begin{tikzpicture}
	\begin{pgfonlayer}{nodelayer}
		\node [style=none] (56) at (-0.75, 13.75) {};
		\node [style=X] (57) at (-0.25, 15.25) {};
		\node [style=andin] (58) at (-1.75, 13.5) {};
		\node [style=Z] (59) at (-0.75, 12) {};
		\node [style=none] (60) at (-1.25, 15.75) {};
		\node [style=none] (61) at (-2.75, 10.5) {};
		\node [style=Z] (62) at (-2.75, 12) {};
		\node [style=none] (63) at (-1.75, 13.5) {};
		\node [style=none] (64) at (-0.25, 10.5) {};
		\node [style=andin] (65) at (-2.25, 12.75) {};
		\node [style=none] (66) at (-1.25, 10.5) {};
		\node [style=none] (67) at (-0.75, 10.5) {};
		\node [style=Z] (68) at (-1.25, 12) {};
		\node [style=none] (69) at (-2.25, 12.75) {};
		\node [style=none] (70) at (-2.75, 14) {};
		\node [style=none] (71) at (-0.25, 15.75) {};
		\node [style=none] (72) at (-2.75, 15) {};
		\node [style=none] (73) at (-0.75, 15.25) {};
		\node [style=none] (74) at (-2.75, 15.75) {};
		\node [style=none] (75) at (-0.75, 15.75) {};
	\end{pgfonlayer}
	\begin{pgfonlayer}{edgelayer}
		\draw (71.center) to (64.center);
		\draw (60.center) to (66.center);
		\draw [in=90, out=180, looseness=1.25] (57) to (63.center);
		\draw [in=90, out=-124, looseness=1.25] (63.center) to (69.center);
		\draw [in=90, out=-90] (62) to (67.center);
		\draw [in=-90, out=90] (61.center) to (59);
		\draw (69.center) to (62);
		\draw (68) to (69.center);
		\draw [in=105, out=-60] (63.center) to (59);
		\draw [in=90, out=-90] (72.center) to (56.center);
		\draw [in=90, out=-90] (73.center) to (70.center);
		\draw (75.center) to (73.center);
		\draw (56.center) to (59);
		\draw (62) to (70.center);
		\draw (72.center) to (74.center);
	\end{pgfonlayer}
\end{tikzpicture}
=
\begin{tikzpicture}
	\begin{pgfonlayer}{nodelayer}
		\node [style=Z] (57) at (-2, 11.75) {};
		\node [style=Z] (58) at (-1, 11.75) {};
		\node [style=none] (59) at (-1.5, 12.5) {};
		\node [style=X] (60) at (-0.5, 13.25) {};
		\node [style=X] (61) at (-0.5, 12.5) {};
		\node [style=X] (62) at (-0.5, 15.75) {};
		\node [style=Z] (63) at (-1, 13.75) {};
		\node [style=X] (64) at (-0.5, 15.25) {};
		\node [style=Z] (65) at (-2, 13.75) {};
		\node [style=none] (66) at (-1.5, 14.5) {};
		\node [style=Z] (67) at (0.5, 13.75) {};
		\node [style=X] (68) at (1, 15.25) {};
		\node [style=Z] (69) at (-0.5, 13.75) {};
		\node [style=none] (70) at (0, 14.5) {};
		\node [style=none] (71) at (-2, 10.5) {};
		\node [style=none] (72) at (-1, 11.75) {};
		\node [style=none] (73) at (0.5, 11.75) {};
		\node [style=none] (74) at (1, 10.5) {};
		\node [style=none] (75) at (-1, 15.75) {};
		\node [style=none] (76) at (-2, 17) {};
		\node [style=none] (77) at (1, 17) {};
		\node [style=none] (78) at (0.5, 15.75) {};
		\node [style=andin] (79) at (-1.5, 12.5) {};
		\node [style=andin] (80) at (-1.5, 14.5) {};
		\node [style=andin] (81) at (0, 14.5) {};
		\node [style=none] (82) at (-1, 17) {};
		\node [style=none] (83) at (0.5, 17) {};
		\node [style=none] (84) at (0.5, 10.5) {};
		\node [style=none] (85) at (-1, 10.5) {};
	\end{pgfonlayer}
	\begin{pgfonlayer}{edgelayer}
		\draw (58) to (59.center);
		\draw (59.center) to (57);
		\draw (60) to (61);
		\draw [in=90, out=180] (60) to (59.center);
		\draw (63) to (66.center);
		\draw (66.center) to (65);
		\draw (64) to (62);
		\draw [in=90, out=180] (64) to (66.center);
		\draw (67) to (70.center);
		\draw (70.center) to (69);
		\draw [in=90, out=180] (68) to (70.center);
		\draw (77.center) to (74.center);
		\draw (73.center) to (78.center);
		\draw (64) to (69);
		\draw (69) to (60);
		\draw (71.center) to (76.center);
		\draw [in=90, out=-90] (82.center) to (78.center);
		\draw [in=-90, out=90] (75.center) to (83.center);
		\draw [in=90, out=-90] (72.center) to (84.center);
		\draw [in=-90, out=90] (85.center) to (73.center);
		\draw (75.center) to (58);
	\end{pgfonlayer}
\end{tikzpicture}
=
\left\llbracket
\begin{tikzpicture}
	\begin{pgfonlayer}{nodelayer}
		\node [style=nothing] (58) at (0, 10.5) {};
		\node [style=nothing] (59) at (-0.5, 10.5) {};
		\node [style=nothing] (60) at (-1.5, 10.5) {};
		\node [style=nothing] (61) at (-2, 10.5) {};
		\node [style=zeroin] (62) at (-1, 11.25) {};
		\node [style=oplus] (63) at (-1, 11.75) {};
		\node [style=oplus] (64) at (-1, 12.75) {};
		\node [style=dot] (65) at (-1, 12.25) {};
		\node [style=dot] (66) at (-0.5, 12.25) {};
		\node [style=dot] (67) at (-1.5, 11.75) {};
		\node [style=dot] (68) at (-2, 11.75) {};
		\node [style=dot] (69) at (-1.5, 12.75) {};
		\node [style=dot] (70) at (-2, 12.75) {};
		\node [style=oplus] (71) at (0, 12.25) {};
		\node [style=zeroout] (72) at (-1, 13.25) {};
		\node [style=nothing] (73) at (0, 14) {};
		\node [style=nothing] (74) at (-2, 14) {};
		\node [style=nothing] (75) at (-0.5, 14) {};
		\node [style=nothing] (76) at (-1.5, 14) {};
		\node [style=none] (77) at (-1.5, 13.5) {};
		\node [style=none] (78) at (-0.5, 13.5) {};
		\node [style=none] (79) at (-0.5, 11) {};
		\node [style=none] (80) at (-1.5, 11) {};
	\end{pgfonlayer}
	\begin{pgfonlayer}{edgelayer}
		\draw (61) to (68);
		\draw (68) to (70);
		\draw (70) to (74);
		\draw (69) to (67);
		\draw (73) to (71);
		\draw (71) to (58);
		\draw (71) to (66);
		\draw (66) to (65);
		\draw (67) to (63);
		\draw (67) to (68);
		\draw (70) to (69);
		\draw (64) to (69);
		\draw (62) to (63);
		\draw (63) to (65);
		\draw (65) to (64);
		\draw (64) to (72);
		\draw [in=90, out=-90, looseness=0.50] (75) to (77.center);
		\draw [in=90, out=-90, looseness=0.75] (76) to (78.center);
		\draw (78.center) to (66);
		\draw (66) to (79.center);
		\draw [in=90, out=-105, looseness=0.50] (79.center) to (60);
		\draw [in=-90, out=90, looseness=0.50] (59) to (80.center);
		\draw (80.center) to (67);
		\draw (69) to (77.center);
	\end{pgfonlayer}
\end{tikzpicture}
\right\rrbracket_{\hat{\TOF}}
\end{align*}
\endgroup



\end{enumerate}
Where unitality and counitality follow from the fact that the white spiders are Frobenius algebras.  Also, we must also note that both Frobenius algebras induce the same compact closed structure, as is implied by the spider law;  this is immediate.

\end{proof}

\begin{theorem}
\label{theorem:TOFZXAiso}
The interpretation functors $\llbracket\_\rrbracket_{\ZXA}$ and $\llbracket\_\rrbracket_{\hat \TOF}$ are inverses, so that $\hat \TOF$ and $\ZXA$ are isomorphic as strongly compact closed props.
\end{theorem}

\begin{proof}
First we show that $\llbracket\llbracket\_\rrbracket_{\ZXA}\rrbracket_{\hat \TOF}=1$:
\begin{description}
\item[For the white spider:]
The case for the unit and counit is trivial.  For the (co)multiplication we have:
\begin{align*}
\left\llbracket\left\llbracket
\begin{tikzpicture}
	\begin{pgfonlayer}{nodelayer}
		\node [style=Z] (59) at (0, 11.5) {};
		\node [style=none] (60) at (-0.5, 12.5) {};
		\node [style=none] (61) at (0.5, 12.5) {};
		\node [style=none] (62) at (0, 10.5) {};
	\end{pgfonlayer}
	\begin{pgfonlayer}{edgelayer}
		\draw [in=63, out=-90] (61.center) to (59);
		\draw [in=-90, out=117] (59) to (60.center);
		\draw (59) to (62.center);
	\end{pgfonlayer}
\end{tikzpicture}
\right\rrbracket_{\ZXA}\right\rrbracket_{\hat \TOF}
&=
\left\llbracket
\begin{tikzpicture}
	\begin{pgfonlayer}{nodelayer}
		\node [style=none] (60) at (-0.5, 11.75) {};
		\node [style=none] (61) at (0, 11.75) {};
		\node [style=none] (62) at (-0.5, 10.5) {};
		\node [style=dot] (63) at (-0.5, 11.25) {};
		\node [style=oplus] (64) at (0, 11.25) {};
		\node [style=zeroin] (65) at (0, 10.75) {};
	\end{pgfonlayer}
	\begin{pgfonlayer}{edgelayer}
		\draw (61.center) to (64);
		\draw (64) to (65);
		\draw (64) to (63);
		\draw (63) to (60.center);
		\draw (63) to (62.center);
	\end{pgfonlayer}
\end{tikzpicture}
\right\rrbracket_{\hat \TOF}
=
\begin{tikzpicture}
	\begin{pgfonlayer}{nodelayer}
		\node [style=andin] (61) at (-1.5, 12.5) {};
		\node [style=Z] (62) at (-2, 11.5) {};
		\node [style=Z] (63) at (-1, 11.5) {};
		\node [style=X] (64) at (-0.5, 13) {};
		\node [style=none] (65) at (-1.5, 12.5) {};
		\node [style=X] (66) at (-2, 10.5) {$1$};
		\node [style=X] (67) at (-2, 12.5) {$1$};
		\node [style=none] (68) at (-0.5, 13.75) {};
		\node [style=none] (69) at (-1, 13.75) {};
		\node [style=none] (70) at (-1, 10.5) {};
		\node [style=X] (71) at (-0.5, 11.5) {};
	\end{pgfonlayer}
	\begin{pgfonlayer}{edgelayer}
		\draw [in=90, out=180] (64) to (65.center);
		\draw (65.center) to (62);
		\draw (63) to (65.center);
		\draw (67) to (62);
		\draw (62) to (66);
		\draw (69.center) to (63);
		\draw (63) to (70.center);
		\draw (68.center) to (64);
		\draw (64) to (71);
	\end{pgfonlayer}
\end{tikzpicture}
=
\begin{tikzpicture}
	\begin{pgfonlayer}{nodelayer}
		\node [style=Z] (62) at (-1, 11.5) {};
		\node [style=X] (63) at (-0.5, 11.5) {};
		\node [style=none] (64) at (-0.5, 12.25) {};
		\node [style=none] (65) at (-1, 12.25) {};
		\node [style=none] (66) at (-1, 10.5) {};
		\node [style=X] (67) at (-0.5, 10.75) {};
	\end{pgfonlayer}
	\begin{pgfonlayer}{edgelayer}
		\draw (65.center) to (62);
		\draw (62) to (66.center);
		\draw (64.center) to (63);
		\draw (63) to (67);
		\draw (63) to (62);
	\end{pgfonlayer}
\end{tikzpicture}
=
\begin{tikzpicture}
	\begin{pgfonlayer}{nodelayer}
		\node [style=Z] (63) at (0, 11.25) {};
		\node [style=none] (64) at (-0.5, 12.25) {};
		\node [style=none] (65) at (0.5, 12.25) {};
		\node [style=none] (66) at (0, 10.5) {};
	\end{pgfonlayer}
	\begin{pgfonlayer}{edgelayer}
		\draw [in=63, out=-90] (65.center) to (63);
		\draw [in=-90, out=117] (63) to (64.center);
		\draw (63) to (66.center);
	\end{pgfonlayer}
\end{tikzpicture}
\end{align*}

\item[For the grey spider:]
The cases for the unit, counit and $1$ phase are trivial.  For the (co)multiplication we have:

\begin{align*}
\left\llbracket\left\llbracket
\begin{tikzpicture}
	\begin{pgfonlayer}{nodelayer}
		\node [style=X] (64) at (0, 11) {};
		\node [style=none] (65) at (-0.5, 11.75) {};
		\node [style=none] (66) at (0.5, 11.75) {};
		\node [style=none] (67) at (0, 10.5) {};
	\end{pgfonlayer}
	\begin{pgfonlayer}{edgelayer}
		\draw [in=63, out=-90] (66.center) to (64);
		\draw [in=-90, out=117] (64) to (65.center);
		\draw (64) to (67.center);
	\end{pgfonlayer}
\end{tikzpicture}
\right\rrbracket_{\ZXA}\right\rrbracket_{\hat \TOF}
&=
\left\llbracket
\begin{tikzpicture}
	\begin{pgfonlayer}{nodelayer}
		\node [style=none] (65) at (-0.5, 11.75) {};
		\node [style=none] (66) at (0, 11.75) {};
		\node [style=none] (67) at (-0.5, 10.5) {};
		\node [style=dot] (68) at (0, 11.25) {};
		\node [style=oplus] (69) at (-0.5, 11.25) {};
		\node [style=Z] (70) at (0, 10.75) {};
	\end{pgfonlayer}
	\begin{pgfonlayer}{edgelayer}
		\draw (69) to (68);
		\draw (68) to (66.center);
		\draw (70) to (68);
		\draw (69) to (65.center);
		\draw (69) to (67.center);
	\end{pgfonlayer}
\end{tikzpicture}
\right\rrbracket_{\hat \TOF}
=
\begin{tikzpicture}
	\begin{pgfonlayer}{nodelayer}
		\node [style=Z] (66) at (-2, 11.75) {};
		\node [style=Z] (67) at (-1, 11.75) {};
		\node [style=none] (68) at (-1.5, 12.5) {};
		\node [style=X] (69) at (-2.5, 13.25) {};
		\node [style=X] (70) at (-1, 10.75) {$1$};
		\node [style=X] (71) at (-1, 13.25) {$1$};
		\node [style=Z] (72) at (-2, 10.75) {};
		\node [style=none] (73) at (-2.5, 10.5) {};
		\node [style=none] (74) at (-2.5, 13.75) {};
		\node [style=none] (75) at (-2, 13.75) {};
		\node [style=andin] (76) at (-1.5, 12.5) {};
	\end{pgfonlayer}
	\begin{pgfonlayer}{edgelayer}
		\draw (70) to (67);
		\draw (67) to (68.center);
		\draw (71) to (67);
		\draw (68.center) to (66);
		\draw (68.center) to (69);
		\draw (75.center) to (66);
		\draw (66) to (72);
		\draw (73.center) to (69);
		\draw (69) to (74.center);
	\end{pgfonlayer}
\end{tikzpicture}
=
\begin{tikzpicture}
	\begin{pgfonlayer}{nodelayer}
		\node [style=Z] (67) at (-2, 11.25) {};
		\node [style=X] (68) at (-2.5, 11.25) {};
		\node [style=Z] (69) at (-2, 10.75) {};
		\node [style=none] (70) at (-2.5, 10.5) {};
		\node [style=none] (71) at (-2.5, 11.75) {};
		\node [style=none] (72) at (-2, 11.75) {};
	\end{pgfonlayer}
	\begin{pgfonlayer}{edgelayer}
		\draw (72.center) to (67);
		\draw (67) to (69);
		\draw (70.center) to (68);
		\draw (68) to (71.center);
		\draw (67) to (68);
	\end{pgfonlayer}
\end{tikzpicture}
=
\begin{tikzpicture}
	\begin{pgfonlayer}{nodelayer}
		\node [style=X] (68) at (0, 11) {};
		\node [style=none] (69) at (-0.5, 11.75) {};
		\node [style=none] (70) at (0.5, 11.75) {};
		\node [style=none] (71) at (0, 10.5) {};
	\end{pgfonlayer}
	\begin{pgfonlayer}{edgelayer}
		\draw [in=63, out=-90] (70.center) to (68);
		\draw [in=-90, out=117] (68) to (69.center);
		\draw (68) to (71.center);
	\end{pgfonlayer}
\end{tikzpicture}
\end{align*}


\item[For the {\sf and} gate:]
\begin{align*}
\left\llbracket\left\llbracket
\begin{tikzpicture}
	\begin{pgfonlayer}{nodelayer}
		\node [style=none] (69) at (0, 11.5) {};
		\node [style=none] (70) at (-0.5, 10.5) {};
		\node [style=none] (71) at (0.5, 10.5) {};
		\node [style=none] (72) at (0, 12.5) {};
		\node [style=andin] (73) at (0, 11.5) {};
	\end{pgfonlayer}
	\begin{pgfonlayer}{edgelayer}
		\draw [in=-63, out=90] (71.center) to (69.center);
		\draw [in=90, out=-117] (69.center) to (70.center);
		\draw (69.center) to (72.center);
	\end{pgfonlayer}
\end{tikzpicture}
\right\rrbracket_{\ZXA}\right\rrbracket_{\hat \TOF}
&=
\left\llbracket
\begin{tikzpicture}
	\begin{pgfonlayer}{nodelayer}
		\node [style=dot] (70) at (-2, 11.25) {};
		\node [style=dot] (71) at (-1.5, 11.25) {};
		\node [style=oplus] (72) at (-1, 11.25) {};
		\node [style=zeroin] (73) at (-1, 10.75) {};
		\node [style=Z] (74) at (-2, 11.75) {};
		\node [style=Z] (75) at (-1.5, 11.75) {};
		\node [style=none] (76) at (-1, 12) {};
		\node [style=none] (77) at (-2, 10.5) {};
		\node [style=none] (78) at (-1.5, 10.5) {};
	\end{pgfonlayer}
	\begin{pgfonlayer}{edgelayer}
		\draw (75) to (71);
		\draw (70) to (71);
		\draw (71) to (72);
		\draw (72) to (73);
		\draw (72) to (76.center);
		\draw (74) to (70);
		\draw (70) to (77.center);
		\draw (78.center) to (71);
	\end{pgfonlayer}
\end{tikzpicture}
\right\rrbracket_{\hat \TOF}
=
\begin{tikzpicture}
	\begin{pgfonlayer}{nodelayer}
		\node [style=Z] (71) at (-2, 11) {};
		\node [style=Z] (72) at (-1, 11) {};
		\node [style=none] (73) at (-1.5, 11.75) {};
		\node [style=X] (74) at (-0.5, 12.5) {};
		\node [style=X] (75) at (-0.5, 11.5) {};
		\node [style=Z] (76) at (-1, 13) {};
		\node [style=Z] (77) at (-2, 13) {};
		\node [style=none] (78) at (-0.5, 13.25) {};
		\node [style=none] (79) at (-2, 10.5) {};
		\node [style=none] (80) at (-1, 10.5) {};
		\node [style=andin] (81) at (-1.5, 11.75) {};
	\end{pgfonlayer}
	\begin{pgfonlayer}{edgelayer}
		\draw (73.center) to (71);
		\draw (72) to (73.center);
		\draw (74) to (75);
		\draw [in=90, out=180] (74) to (73.center);
		\draw (76) to (72);
		\draw (72) to (80.center);
		\draw (79.center) to (71);
		\draw (71) to (77);
		\draw (78.center) to (74);
	\end{pgfonlayer}
\end{tikzpicture}
=
\begin{tikzpicture}
	\begin{pgfonlayer}{nodelayer}
		\node [style=none] (72) at (0, 11.5) {};
		\node [style=none] (73) at (-0.5, 10.5) {};
		\node [style=none] (74) at (0.5, 10.5) {};
		\node [style=none] (75) at (0, 12.5) {};
		\node [style=andin] (76) at (0, 11.5) {};
	\end{pgfonlayer}
	\begin{pgfonlayer}{edgelayer}
		\draw [in=-63, out=90] (74.center) to (72.center);
		\draw [in=90, out=-117] (72.center) to (73.center);
		\draw (72.center) to (75.center);
	\end{pgfonlayer}
\end{tikzpicture}
\end{align*}
\end{description}

Next, we show that $\llbracket\llbracket\_\rrbracket_{\hat \TOF}\rrbracket_{\ZXA}=1$:
The ancillae are trivial.  For the Toffoli gate:
\begin{align*}
\left\llbracket\left\llbracket
\begin{tikzpicture}
	\begin{pgfonlayer}{nodelayer}
		\node [style=dot] (73) at (-2, 11.25) {};
		\node [style=dot] (74) at (-1.5, 11.25) {};
		\node [style=oplus] (75) at (-1, 11.25) {};
		\node [style=none] (76) at (-1, 12) {};
		\node [style=none] (77) at (-1, 10.5) {};
		\node [style=none] (78) at (-2, 10.5) {};
		\node [style=none] (79) at (-1.5, 10.5) {};
		\node [style=none] (80) at (-1.5, 12) {};
		\node [style=none] (81) at (-2, 12) {};
	\end{pgfonlayer}
	\begin{pgfonlayer}{edgelayer}
		\draw (75) to (74);
		\draw (74) to (73);
		\draw (81.center) to (73);
		\draw (73) to (78.center);
		\draw (79.center) to (74);
		\draw (74) to (80.center);
		\draw (76.center) to (75);
		\draw (75) to (77.center);
	\end{pgfonlayer}
\end{tikzpicture}
\right\rrbracket_{\hat \TOF}\right\rrbracket_{\ZXA}
&=
\left\llbracket
\begin{tikzpicture}
	\begin{pgfonlayer}{nodelayer}
		\node [style=Z] (74) at (-2, 11) {};
		\node [style=Z] (75) at (-1, 11) {};
		\node [style=none] (76) at (-1.5, 12) {};
		\node [style=X] (77) at (-0.5, 13) {};
		\node [style=none] (78) at (-0.5, 10.5) {};
		\node [style=none] (79) at (-0.5, 13.75) {};
		\node [style=none] (80) at (-1, 13.75) {};
		\node [style=none] (81) at (-2, 13.75) {};
		\node [style=none] (82) at (-2, 10.5) {};
		\node [style=none] (83) at (-1, 10.5) {};
		\node [style=andin] (84) at (-1.5, 12) {};
	\end{pgfonlayer}
	\begin{pgfonlayer}{edgelayer}
		\draw (77) to (79.center);
		\draw (77) to (78.center);
		\draw (83.center) to (75);
		\draw (75) to (80.center);
		\draw (81.center) to (74);
		\draw (74) to (82.center);
		\draw (74) to (76.center);
		\draw [in=180, out=90] (76.center) to (77);
		\draw (76.center) to (75);
	\end{pgfonlayer}
\end{tikzpicture}
\right\rrbracket_{\ZXA}
=
\begin{tikzpicture}
	\begin{pgfonlayer}{nodelayer}
		\node [style=dot] (75) at (-2, 12) {};
		\node [style=dot] (76) at (-1, 12) {};
		\node [style=oplus] (77) at (-0.25, 12) {};
		\node [style=Z] (78) at (-2, 12.5) {};
		\node [style=Z] (79) at (-1, 12.5) {};
		\node [style=oplus] (80) at (-0.25, 12.5) {};
		\node [style=dot] (81) at (0.25, 12.5) {};
		\node [style=Z] (82) at (0.25, 12) {};
		\node [style=none] (83) at (0.25, 13) {};
		\node [style=dot] (84) at (-2.5, 11.5) {};
		\node [style=oplus] (85) at (-2, 11.5) {};
		\node [style=zeroin] (86) at (-2, 11) {};
		\node [style=oplus] (87) at (-1, 11.5) {};
		\node [style=dot] (88) at (-1.5, 11.5) {};
		\node [style=zeroin] (89) at (-1, 11) {};
		\node [style=none] (90) at (-1.5, 13) {};
		\node [style=none] (91) at (-2.5, 13) {};
		\node [style=none] (92) at (-1.5, 10.5) {};
		\node [style=none] (93) at (-2.5, 10.5) {};
		\node [style=none] (94) at (-0.25, 10.5) {};
		\node [style=zeroout] (95) at (-0.25, 13) {};
	\end{pgfonlayer}
	\begin{pgfonlayer}{edgelayer}
		\draw (83.center) to (81);
		\draw (81) to (80);
		\draw (80) to (77);
		\draw (82) to (81);
		\draw (77) to (76);
		\draw (76) to (75);
		\draw (75) to (78);
		\draw (79) to (76);
		\draw (85) to (86);
		\draw (85) to (84);
		\draw (87) to (89);
		\draw (87) to (88);
		\draw (76) to (87);
		\draw (85) to (75);
		\draw (91.center) to (84);
		\draw (84) to (93.center);
		\draw (92.center) to (88);
		\draw (88) to (90.center);
		\draw (95) to (80);
		\draw (77) to (94.center);
	\end{pgfonlayer}
\end{tikzpicture}
\eq{unit}
\begin{tikzpicture}
	\begin{pgfonlayer}{nodelayer}
		\node [style=dot] (76) at (-2, 12) {};
		\node [style=dot] (77) at (-1, 12) {};
		\node [style=oplus] (78) at (-0.25, 12) {};
		\node [style=Z] (79) at (-2, 12.5) {};
		\node [style=Z] (80) at (-1, 12.5) {};
		\node [style=none] (81) at (-0.25, 13) {};
		\node [style=dot] (82) at (-2.5, 11.5) {};
		\node [style=oplus] (83) at (-2, 11.5) {};
		\node [style=zeroin] (84) at (-2, 11) {};
		\node [style=oplus] (85) at (-1, 11.5) {};
		\node [style=dot] (86) at (-1.5, 11.5) {};
		\node [style=zeroin] (87) at (-1, 11) {};
		\node [style=none] (88) at (-1.5, 13) {};
		\node [style=none] (89) at (-2.5, 13) {};
		\node [style=none] (90) at (-1.5, 10.5) {};
		\node [style=none] (91) at (-2.5, 10.5) {};
		\node [style=none] (92) at (-0.25, 10.5) {};
	\end{pgfonlayer}
	\begin{pgfonlayer}{edgelayer}
		\draw (78) to (77);
		\draw (77) to (76);
		\draw (76) to (79);
		\draw (80) to (77);
		\draw (83) to (84);
		\draw (83) to (82);
		\draw (85) to (87);
		\draw (85) to (86);
		\draw (77) to (85);
		\draw (83) to (76);
		\draw (89.center) to (82);
		\draw (82) to (91.center);
		\draw (90.center) to (86);
		\draw (86) to (88.center);
		\draw (78) to (92.center);
		\draw (81.center) to (78);
	\end{pgfonlayer}
\end{tikzpicture}\\
&\eq{Lem. \ref{lemma:Iwama}}
\begin{tikzpicture}
	\begin{pgfonlayer}{nodelayer}
		\node [style=dot] (77) at (-2, 12.5) {};
		\node [style=dot] (78) at (-1, 12.5) {};
		\node [style=oplus] (79) at (-0.25, 12.5) {};
		\node [style=Z] (80) at (-2, 13.5) {};
		\node [style=Z] (81) at (-1, 13.5) {};
		\node [style=none] (82) at (-0.25, 14) {};
		\node [style=dot] (83) at (-2.5, 13) {};
		\node [style=oplus] (84) at (-2, 13) {};
		\node [style=zeroin] (85) at (-2, 11) {};
		\node [style=oplus] (86) at (-1, 11.5) {};
		\node [style=dot] (87) at (-1.5, 11.5) {};
		\node [style=zeroin] (88) at (-1, 11) {};
		\node [style=none] (89) at (-1.5, 14) {};
		\node [style=none] (90) at (-2.5, 14) {};
		\node [style=none] (91) at (-1.5, 10.5) {};
		\node [style=none] (92) at (-2.5, 10.5) {};
		\node [style=none] (93) at (-0.25, 10.5) {};
		\node [style=dot] (94) at (-2.5, 12) {};
		\node [style=dot] (95) at (-1, 12) {};
		\node [style=oplus] (96) at (-0.25, 12) {};
	\end{pgfonlayer}
	\begin{pgfonlayer}{edgelayer}
		\draw (79) to (78);
		\draw (78) to (77);
		\draw (77) to (80);
		\draw (81) to (78);
		\draw (84) to (85);
		\draw (84) to (83);
		\draw (86) to (88);
		\draw (86) to (87);
		\draw (78) to (86);
		\draw (84) to (77);
		\draw (90.center) to (83);
		\draw (83) to (92.center);
		\draw (91.center) to (87);
		\draw (87) to (89.center);
		\draw (79) to (93.center);
		\draw (82.center) to (79);
		\draw (96) to (95);
		\draw (95) to (94);
	\end{pgfonlayer}
\end{tikzpicture}
\eq{\ref{TOF.2}}
\begin{tikzpicture}
	\begin{pgfonlayer}{nodelayer}
		\node [style=Z] (78) at (-2, 13) {};
		\node [style=Z] (79) at (-1, 13) {};
		\node [style=none] (80) at (-0.25, 13.5) {};
		\node [style=dot] (81) at (-2.5, 12.5) {};
		\node [style=oplus] (82) at (-2, 12.5) {};
		\node [style=zeroin] (83) at (-2, 11) {};
		\node [style=oplus] (84) at (-1, 11.5) {};
		\node [style=dot] (85) at (-1.5, 11.5) {};
		\node [style=zeroin] (86) at (-1, 11) {};
		\node [style=none] (87) at (-1.5, 13.5) {};
		\node [style=none] (88) at (-2.5, 13.5) {};
		\node [style=none] (89) at (-1.5, 10.5) {};
		\node [style=none] (90) at (-2.5, 10.5) {};
		\node [style=none] (91) at (-0.25, 10.5) {};
		\node [style=dot] (92) at (-2.5, 12) {};
		\node [style=dot] (93) at (-1, 12) {};
		\node [style=oplus] (94) at (-0.25, 12) {};
	\end{pgfonlayer}
	\begin{pgfonlayer}{edgelayer}
		\draw (82) to (83);
		\draw (82) to (81);
		\draw (84) to (86);
		\draw (84) to (85);
		\draw (88.center) to (81);
		\draw (81) to (90.center);
		\draw (89.center) to (85);
		\draw (85) to (87.center);
		\draw (94) to (93);
		\draw (93) to (92);
		\draw (80.center) to (91.center);
		\draw (84) to (79);
		\draw (78) to (82);
	\end{pgfonlayer}
\end{tikzpicture}
\eq{unit}
\begin{tikzpicture}
	\begin{pgfonlayer}{nodelayer}
		\node [style=Z] (79) at (-1, 13) {};
		\node [style=none] (80) at (-0.25, 13.5) {};
		\node [style=oplus] (81) at (-1, 11.5) {};
		\node [style=dot] (82) at (-1.5, 11.5) {};
		\node [style=zeroin] (83) at (-1, 11) {};
		\node [style=none] (84) at (-1.5, 13.5) {};
		\node [style=none] (85) at (-2, 13.5) {};
		\node [style=none] (86) at (-1.5, 10.5) {};
		\node [style=none] (87) at (-2, 10.5) {};
		\node [style=none] (88) at (-0.25, 10.5) {};
		\node [style=dot] (89) at (-2, 12) {};
		\node [style=dot] (90) at (-1, 12) {};
		\node [style=oplus] (91) at (-0.25, 12) {};
	\end{pgfonlayer}
	\begin{pgfonlayer}{edgelayer}
		\draw (81) to (83);
		\draw (81) to (82);
		\draw (86.center) to (82);
		\draw (82) to (84.center);
		\draw (91) to (90);
		\draw (90) to (89);
		\draw (80.center) to (88.center);
		\draw (81) to (79);
		\draw (85.center) to (87.center);
	\end{pgfonlayer}
\end{tikzpicture}\\
&\eq{Lem.  \ref{lemma:Iwama}}
\begin{tikzpicture}
	\begin{pgfonlayer}{nodelayer}
		\node [style=Z] (80) at (-1, 13) {};
		\node [style=none] (81) at (-0.25, 13.5) {};
		\node [style=oplus] (82) at (-1, 12.5) {};
		\node [style=dot] (83) at (-1.5, 12.5) {};
		\node [style=zeroin] (84) at (-1, 11) {};
		\node [style=none] (85) at (-1.5, 13.5) {};
		\node [style=none] (86) at (-2, 13.5) {};
		\node [style=none] (87) at (-1.5, 10.5) {};
		\node [style=none] (88) at (-2, 10.5) {};
		\node [style=none] (89) at (-0.25, 10.5) {};
		\node [style=dot] (90) at (-2, 12) {};
		\node [style=dot] (91) at (-1, 12) {};
		\node [style=oplus] (92) at (-0.25, 12) {};
		\node [style=dot] (93) at (-2, 11.5) {};
		\node [style=dot] (94) at (-1.5, 11.5) {};
		\node [style=oplus] (95) at (-0.25, 11.5) {};
	\end{pgfonlayer}
	\begin{pgfonlayer}{edgelayer}
		\draw (82) to (84);
		\draw (82) to (83);
		\draw (87.center) to (83);
		\draw (83) to (85.center);
		\draw (92) to (91);
		\draw (91) to (90);
		\draw (81.center) to (89.center);
		\draw (82) to (80);
		\draw (86.center) to (88.center);
		\draw (95) to (93);
	\end{pgfonlayer}
\end{tikzpicture}
\eq{\ref{TOF.2}}
\begin{tikzpicture}
	\begin{pgfonlayer}{nodelayer}
		\node [style=Z] (81) at (-1, 12.5) {};
		\node [style=none] (82) at (-0.25, 13) {};
		\node [style=oplus] (83) at (-1, 12) {};
		\node [style=dot] (84) at (-1.5, 12) {};
		\node [style=zeroin] (85) at (-1, 11) {};
		\node [style=none] (86) at (-1.5, 13) {};
		\node [style=none] (87) at (-2, 13) {};
		\node [style=none] (88) at (-1.5, 10.5) {};
		\node [style=none] (89) at (-2, 10.5) {};
		\node [style=none] (90) at (-0.25, 10.5) {};
		\node [style=dot] (91) at (-2, 11.5) {};
		\node [style=dot] (92) at (-1.5, 11.5) {};
		\node [style=oplus] (93) at (-0.25, 11.5) {};
	\end{pgfonlayer}
	\begin{pgfonlayer}{edgelayer}
		\draw (83) to (85);
		\draw (83) to (84);
		\draw (88.center) to (84);
		\draw (84) to (86.center);
		\draw (82.center) to (90.center);
		\draw (83) to (81);
		\draw (87.center) to (89.center);
		\draw (93) to (91);
	\end{pgfonlayer}
\end{tikzpicture}
\eq{unit}
\begin{tikzpicture}
	\begin{pgfonlayer}{nodelayer}
		\node [style=dot] (82) at (-2, 11.25) {};
		\node [style=dot] (83) at (-1.5, 11.25) {};
		\node [style=oplus] (84) at (-1, 11.25) {};
		\node [style=none] (85) at (-1, 12) {};
		\node [style=none] (86) at (-1, 10.5) {};
		\node [style=none] (87) at (-2, 10.5) {};
		\node [style=none] (88) at (-1.5, 10.5) {};
		\node [style=none] (89) at (-1.5, 12) {};
		\node [style=none] (90) at (-2, 12) {};
	\end{pgfonlayer}
	\begin{pgfonlayer}{edgelayer}
		\draw (84) to (83);
		\draw (83) to (82);
		\draw (90.center) to (82);
		\draw (82) to (87.center);
		\draw (88.center) to (83);
		\draw (83) to (89.center);
		\draw (85.center) to (84);
		\draw (84) to (86.center);
	\end{pgfonlayer}
\end{tikzpicture}
\end{align*}
\end{proof}

We can represent the natural number labelled $H$-boxes in $\ZXA$:
\begin{remark}
The triangle gate and $H$-boxes have the following interpretation in $\ZXA$, for $n \in \N$:

$$
\begin{tikzpicture}
	\begin{pgfonlayer}{nodelayer}
		\node [style=none] (83) at (0.75, 11.25) {};
		\node [style=none] (84) at (0.75, 10.5) {};
		\node [style=none] (85) at (0.75, 12) {};
		\node [style=triflip] (86) at (0.75, 11.25) {};
	\end{pgfonlayer}
	\begin{pgfonlayer}{edgelayer}
		\draw (85.center) to (83.center);
		\draw (83.center) to (84.center);
	\end{pgfonlayer}
\end{tikzpicture}
:=
\begin{tikzpicture}
	\begin{pgfonlayer}{nodelayer}
		\node [style=none] (131) at (24.75, 11.5) {};
		\node [style=none] (132) at (25.25, 11.5) {};
		\node [style=X] (133) at (24.75, 12.25) {};
		\node [style=none] (134) at (24.75, 10) {};
		\node [style=andin] (135) at (24.75, 11.5) {};
		\node [style=none] (136) at (25.25, 13) {};
		\node [style=X] (137) at (24.75, 10.75) {$1$};
	\end{pgfonlayer}
	\begin{pgfonlayer}{edgelayer}
		\draw [style=simple] (133) to (131.center);
		\draw [style=simple, in=-75, out=-90, looseness=2.75] (132.center) to (131.center);
		\draw [style=simple] (134.center) to (131.center);
		\draw (136.center) to (132.center);
	\end{pgfonlayer}
\end{tikzpicture}
\hspace*{1cm}
\begin{tikzpicture}
	\begin{pgfonlayer}{nodelayer}
		\node [style=H] (85) at (1, 11.25) {$n$};
		\node [style=none] (86) at (1, 10.5) {};
		\node [style=none] (87) at (1, 12) {};
	\end{pgfonlayer}
	\begin{pgfonlayer}{edgelayer}
		\draw (87.center) to (85);
		\draw (85) to (86.center);
	\end{pgfonlayer}
\end{tikzpicture}
:=
\begin{tikzpicture}
	\begin{pgfonlayer}{nodelayer}
		\node [style=none] (59) at (10.25, -0.5) {};
		\node [style=none] (60) at (10.75, -0.5) {};
		\node [style=X] (61) at (10.25, 2) {};
		\node [style=none] (62) at (10.25, -1.5) {};
		\node [style=andin] (63) at (10.25, -0.5) {};
		\node [style=none] (64) at (10.75, 2.5) {};
		\node [style=triflip] (65) at (10.25, 1.25) {};
		\node [style=triflip] (66) at (10.25, 0.25) {};
		\node [style=none] (67) at (10, 0.75) {$n$};
	\end{pgfonlayer}
	\begin{pgfonlayer}{edgelayer}
		\draw [style=simple, in=-75, out=-90, looseness=2.75] (60.center) to (59.center);
		\draw [style=simple] (62.center) to (59.center);
		\draw [style=dotted] (65) to (66);
		\draw (65) to (61);
		\draw (66) to (59.center);
		\draw [style=simple] (60.center) to (64.center);
	\end{pgfonlayer}
\end{tikzpicture}
\hspace*{1cm}
\begin{tikzpicture}
	\begin{pgfonlayer}{nodelayer}
		\node [style=none] (1) at (0.5, -0.75) {};
		\node [style=none] (2) at (0.5, 0.75) {};
		\node [style=none] (3) at (-0.5, -0.75) {};
		\node [style=none] (4) at (-0.5, 0.75) {};
		\node [style=none] (5) at (0, 0.75) {$\cdots$};
		\node [style=none] (6) at (0, -0.75) {$\cdots$};
		\node [style=none] (14) at (0, 0) {};
		\node [style=H] (15) at (0, 0) {$n$};
	\end{pgfonlayer}
	\begin{pgfonlayer}{edgelayer}
		\draw [in=-90, out=120] (14.center) to (4.center);
		\draw [in=-90, out=60] (14.center) to (2.center);
		\draw [in=90, out=-120] (14.center) to (3.center);
		\draw [in=90, out=-60] (14.center) to (1.center);
	\end{pgfonlayer}
\end{tikzpicture}
:=
\begin{tikzpicture}
	\begin{pgfonlayer}{nodelayer}
		\node [style=H] (0) at (1, 11.25) {$n$};
		\node [style=none] (2) at (1, 12) {};
		\node [style=andout] (3) at (1, 12) {};
		\node [style=andin] (4) at (1, 10.5) {};
		\node [style=none] (5) at (0.5, 12.75) {};
		\node [style=none] (6) at (1.5, 12.75) {};
		\node [style=none] (7) at (1, 10.5) {};
		\node [style=none] (8) at (0.5, 9.75) {};
		\node [style=none] (9) at (1.5, 9.75) {};
		\node [style=none] (10) at (1, 12.75) {$\cdots$};
		\node [style=none] (11) at (1, 9.75) {$\cdots$};
	\end{pgfonlayer}
	\begin{pgfonlayer}{edgelayer}
		\draw (2.center) to (0);
		\draw [in=-90, out=30] (2.center) to (6.center);
		\draw [in=-90, out=150] (2.center) to (5.center);
		\draw [in=90, out=-30] (7.center) to (9.center);
		\draw [in=90, out=-150] (7.center) to (8.center);
		\draw (7.center) to (0);
	\end{pgfonlayer}
\end{tikzpicture}
$$

Where the triangle gate is interpreted as:

$$
\left\llbracket\
\begin{tikzpicture}
	\begin{pgfonlayer}{nodelayer}
		\node [style=none] (83) at (0.75, 11.25) {};
		\node [style=none] (84) at (0.75, 10.5) {};
		\node [style=none] (85) at (0.75, 12) {};
		\node [style=triflip] (86) at (0.75, 11.25) {};
	\end{pgfonlayer}
	\begin{pgfonlayer}{edgelayer}
		\draw (85.center) to (83.center);
		\draw (83.center) to (84.center);
	\end{pgfonlayer}
\end{tikzpicture}
\ \right\rrbracket
=
\begin{bmatrix}
1 & 1\\
0 & 1
\end{bmatrix}
$$

So that:

$$
\left\llbracket\
\begin{tikzpicture}
	\begin{pgfonlayer}{nodelayer}
		\node [style=none] (138) at (26.75, 11.25) {};
		\node [style=none] (139) at (26.75, 10.5) {};
		\node [style=triflip] (141) at (26.75, 11.25) {};
		\node [style=none] (142) at (26.75, 12) {};
		\node [style=none] (143) at (26.75, 12.75) {};
		\node [style=triflip] (144) at (26.75, 12) {};
		\node [style=none] (145) at (26.5, 11.625) {$n$};
	\end{pgfonlayer}
	\begin{pgfonlayer}{edgelayer}
		\draw (138.center) to (139.center);
		\draw (142.center) to (143.center);
		\draw [style=dotted] (141) to (144);
	\end{pgfonlayer}
\end{tikzpicture}
\ \right\rrbracket
=
\begin{bmatrix}
1 & n\\
0 & 1
\end{bmatrix}
$$

\end{remark}

Therefore, by imposing the quotient that $2=1$, we get a presentation for the prop of ``qubit relations:''

\begin{corollary}
$\ZXA$ modulo the following quotient is a presentation for the prop of $2^n\times 2^m$ matrices over the Boolean semiring:

$$
\begin{tikzpicture}
	\begin{pgfonlayer}{nodelayer}
		\node [style=none] (0) at (0.75, 11.25) {};
		\node [style=none] (2) at (0.75, 12) {};
		\node [style=triflip] (3) at (0.75, 11.25) {};
		\node [style=none] (4) at (0.75, 10.25) {};
		\node [style=none] (5) at (0.75, 9.5) {};
		\node [style=triflip] (7) at (0.75, 10.25) {};
	\end{pgfonlayer}
	\begin{pgfonlayer}{edgelayer}
		\draw (2.center) to (0.center);
		\draw (4.center) to (5.center);
		\draw (7) to (3);
	\end{pgfonlayer}
\end{tikzpicture}
=
\begin{tikzpicture}
	\begin{pgfonlayer}{nodelayer}
		\node [style=none] (2) at (0.75, 12) {};
		\node [style=none] (4) at (0.75, 10.75) {};
		\node [style=none] (5) at (0.75, 9.5) {};
		\node [style=triflip] (7) at (0.75, 10.75) {};
	\end{pgfonlayer}
	\begin{pgfonlayer}{edgelayer}
		\draw (4.center) to (5.center);
		\draw (7) to (2.center);
	\end{pgfonlayer}
\end{tikzpicture}
\hspace*{.5cm}
\text{or equivalently}
\hspace*{.5cm}
\begin{tikzpicture}
	\begin{pgfonlayer}{nodelayer}
		\node [style=none] (1) at (0.5, -0.75) {};
		\node [style=none] (2) at (0.5, 0.75) {};
		\node [style=none] (3) at (-0.5, -0.75) {};
		\node [style=none] (4) at (-0.5, 0.75) {};
		\node [style=none] (5) at (0, 0.75) {$\cdots$};
		\node [style=none] (6) at (0, -0.75) {$\cdots$};
		\node [style=none] (14) at (0, 0) {};
		\node [style=H] (15) at (0, 0) {$2$};
	\end{pgfonlayer}
	\begin{pgfonlayer}{edgelayer}
		\draw [in=-90, out=120] (14.center) to (4.center);
		\draw [in=-90, out=60] (14.center) to (2.center);
		\draw [in=90, out=-120] (14.center) to (3.center);
		\draw [in=90, out=-60] (14.center) to (1.center);
	\end{pgfonlayer}
\end{tikzpicture}
=
\begin{tikzpicture}
	\begin{pgfonlayer}{nodelayer}
		\node [style=none] (1) at (0.5, -0.75) {};
		\node [style=none] (2) at (0.5, 0.75) {};
		\node [style=none] (3) at (-0.5, -0.75) {};
		\node [style=none] (4) at (-0.5, 0.75) {};
		\node [style=none] (5) at (0, 0.75) {$\cdots$};
		\node [style=none] (6) at (0, -0.75) {$\cdots$};
		\node [style=none] (14) at (0, 0) {};
		\node [style=H] (15) at (0, 0) {$1$};
	\end{pgfonlayer}
	\begin{pgfonlayer}{edgelayer}
		\draw [in=-90, out=120] (14.center) to (4.center);
		\draw [in=-90, out=60] (14.center) to (2.center);
		\draw [in=90, out=-120] (14.center) to (3.center);
		\draw [in=90, out=-60] (14.center) to (1.center);
	\end{pgfonlayer}
\end{tikzpicture}
$$
\end{corollary}


The fact that we can construct natural number parametrized $H$-boxes means that $\ZXA$ can be regarded as a fragment of the ZH-calculus (modulo scalars/how one defines a fragment of the ZH-calculus). 

If we could moreover, express the $H$-box labelled by $-1$ as well as the scalar $1/\sqrt 2$, as mentioned in the previous chapter, this would give us the phase-free ZH-calculus.  These obviously don't live in $\ZXA$; however, if we allowed the added the un-normalized minus state  to our semantics

$$
\left\llbracket\
\begin{tikzpicture}
	\begin{pgfonlayer}{nodelayer}
		\node [style=Z] (0) at (0, -2) {$1$};
		\node [style=none] (1) at (0, -1.25) {};
	\end{pgfonlayer}
	\begin{pgfonlayer}{edgelayer}
		\draw (1.center) to (0);
	\end{pgfonlayer}
\end{tikzpicture}
\ \right\rrbracket
=
\sqrt{2}{\cal F}|1\rangle = | -\rangle = (1,-1)^T$$

so that the $Z$-spiders are also phased by $\alpha,\beta\in \Z/2\Z$:

$$
\begin{tikzpicture}
	\begin{pgfonlayer}{nodelayer}
		\node [style=none] (17) at (2.5, 0.5) {};
		\node [style=none] (18) at (3.5, 0.5) {};
		\node [style=Z] (19) at (3, 1.25) {$0$};
		\node [style=none] (20) at (3, 0.75) {$\cdots$};
		\node [style=none] (21) at (3, 1.75) {$\cdots$};
		\node [style=none] (22) at (3.5, 2) {};
		\node [style=none] (23) at (3, 1.25) {};
		\node [style=none] (24) at (2.5, 2) {};
	\end{pgfonlayer}
	\begin{pgfonlayer}{edgelayer}
		\draw [style=simple, in=-56, out=90] (18.center) to (19);
		\draw [style=simple, in=90, out=-124] (19) to (17.center);
		\draw [style=simple, in=56, out=-90] (22.center) to (23.center);
		\draw [style=simple, in=-90, out=124] (23.center) to (24.center);
	\end{pgfonlayer}
\end{tikzpicture}
:=
\begin{tikzpicture}
	\begin{pgfonlayer}{nodelayer}
		\node [style=none] (25) at (4.5, 0.5) {};
		\node [style=none] (26) at (5.5, 0.5) {};
		\node [style=Z] (27) at (5, 1.25) {};
		\node [style=none] (28) at (5, 0.75) {$\cdots$};
		\node [style=none] (29) at (5, 1.75) {$\cdots$};
		\node [style=none] (30) at (5.5, 2) {};
		\node [style=none] (31) at (5, 1.25) {};
		\node [style=none] (32) at (4.5, 2) {};
	\end{pgfonlayer}
	\begin{pgfonlayer}{edgelayer}
		\draw [style=simple, in=-56, out=90] (26.center) to (27);
		\draw [style=simple, in=90, out=-124] (27) to (25.center);
		\draw [style=simple, in=56, out=-90] (30.center) to (31.center);
		\draw [style=simple, in=-90, out=124] (31.center) to (32.center);
	\end{pgfonlayer}
\end{tikzpicture}
,\ \hspace*{.2cm}
\begin{tikzpicture}
	\begin{pgfonlayer}{nodelayer}
		\node [style=none] (0) at (-3.75, 0.5) {};
		\node [style=none] (1) at (-2.75, 0.5) {};
		\node [style=Z] (2) at (-3.25, 1.25) {$1$};
		\node [style=none] (3) at (-3.25, 0.75) {$\cdots$};
		\node [style=none] (4) at (-3.25, 1.75) {$\cdots$};
		\node [style=none] (5) at (-2.75, 2) {};
		\node [style=none] (6) at (-3.25, 1.25) {};
		\node [style=none] (7) at (-3.75, 2) {};
	\end{pgfonlayer}
	\begin{pgfonlayer}{edgelayer}
		\draw [style=simple, in=-56, out=90] (1.center) to (2);
		\draw [style=simple, in=90, out=-124] (2) to (0.center);
		\draw [style=simple, in=56, out=-90] (5.center) to (6.center);
		\draw [style=simple, in=-90, out=124] (6.center) to (7.center);
	\end{pgfonlayer}
\end{tikzpicture}
:=
\begin{tikzpicture}
	\begin{pgfonlayer}{nodelayer}
		\node [style=none] (45) at (14.25, 2) {};
		\node [style=none] (46) at (15.25, 2) {};
		\node [style=Z] (47) at (14.75, 1.25) {};
		\node [style=none] (48) at (14.75, 1.75) {$\cdots$};
		\node [style=none] (49) at (14.75, 0.75) {$\cdots$};
		\node [style=none] (50) at (15.25, 0.5) {};
		\node [style=none] (51) at (14.75, 1.25) {};
		\node [style=none] (52) at (14.25, 0.5) {};
		\node [style=Z] (53) at (15.75, 0.75) {$1$};
	\end{pgfonlayer}
	\begin{pgfonlayer}{edgelayer}
		\draw [style=simple, in=56, out=-90] (46.center) to (47);
		\draw [style=simple, in=-90, out=124] (47) to (45.center);
		\draw [style=simple, in=-56, out=90] (50.center) to (51.center);
		\draw [style=simple, in=90, out=-124] (51.center) to (52.center);
		\draw [in=90, out=0, looseness=0.75] (51.center) to (53);
	\end{pgfonlayer}
\end{tikzpicture}
,\ \hspace*{.2cm}
\begin{tikzpicture}
	\begin{pgfonlayer}{nodelayer}
		\node [style=none] (0) at (0, 0.5) {};
		\node [style=none] (1) at (1, 0.5) {};
		\node [style=none] (2) at (0, 2.75) {};
		\node [style=none] (3) at (1, 2.75) {};
		\node [style=Z] (4) at (0.5, 1.25) {$\alpha$};
		\node [style=Z] (5) at (0.5, 2) {$\beta$};
		\node [style=none] (6) at (0.5, 2.5) {$\cdots$};
		\node [style=none] (7) at (0.5, 0.75) {$\cdots$};
		\node [style=none] (8) at (0.45, 1.625) {\scalebox{.8}{$\cdots$}};
	\end{pgfonlayer}
	\begin{pgfonlayer}{edgelayer}
		\draw [style=simple, in=-56, out=90] (1.center) to (4);
		\draw [style=simple, in=90, out=-124] (4) to (0.center);
		\draw [style=simple, in=-135, out=135, looseness=1.25] (4) to (5);
		\draw [style=simple, in=45, out=-45, looseness=1.25] (5) to (4);
		\draw [style=simple, in=-90, out=124] (5) to (2.center);
		\draw [style=simple, in=-90, out=56] (5) to (3.center);
	\end{pgfonlayer}
\end{tikzpicture}
=
\begin{tikzpicture}
	\begin{pgfonlayer}{nodelayer}
		\node [style=none] (0) at (0, 0.5) {};
		\node [style=none] (1) at (1, 0.5) {};
		\node [style=Z] (2) at (0.5, 1.25) {$\alpha+\beta$};
		\node [style=none] (3) at (0.5, 0.55) {$\cdots$};
		\node [style=none] (4) at (0.5, 1.9) {$\cdots$};
		\node [style=none] (5) at (1, 2) {};
		\node [style=none] (6) at (0.5, 1.25) {};
		\node [style=none] (7) at (0, 2) {};
	\end{pgfonlayer}
	\begin{pgfonlayer}{edgelayer}
		\draw [style=simple, in=-56, out=90] (1.center) to (2);
		\draw [style=simple, in=90, out=-124] (2) to (0.center);
		\draw [style=simple, in=56, out=-90] (5.center) to (6);
		\draw [style=simple, in=-90, out=124] (6) to (7.center);
	\end{pgfonlayer}
\end{tikzpicture}
$$

Then we can construct the inverse of the triangle gate by conjugation with $\mathcal Z$ gates:
$$
\begin{tikzpicture}
	\begin{pgfonlayer}{nodelayer}
		\node [style=none] (92) at (18.5, 9.25) {};
		\node [style=none] (93) at (18.5, 8.5) {};
		\node [style=none] (94) at (18.5, 10) {};
		\node [style=triflip] (95) at (18.5, 9.25) {};
		\node [style=none] (96) at (18.85, 9.5) {$-1$};
	\end{pgfonlayer}
	\begin{pgfonlayer}{edgelayer}
		\draw (94.center) to (92.center);
		\draw (92.center) to (93.center);
	\end{pgfonlayer}
\end{tikzpicture}
:=
\begin{tikzpicture}
	\begin{pgfonlayer}{nodelayer}
		\node [style=none] (97) at (19.75, 9.25) {};
		\node [style=none] (98) at (19.75, 8) {};
		\node [style=none] (99) at (19.75, 10.5) {};
		\node [style=triflip] (100) at (19.75, 9.25) {};
		\node [style=Z] (101) at (19.75, 10) {$1$};
		\node [style=Z] (102) at (19.75, 8.5) {$1$};
	\end{pgfonlayer}
	\begin{pgfonlayer}{edgelayer}
		\draw [in=90, out=-90] (99.center) to (97.center);
		\draw (97.center) to (98.center);
	\end{pgfonlayer}
\end{tikzpicture}
\hspace*{.5cm}\text{where}\hspace*{.5cm}
\left\llbracket\
\begin{tikzpicture}
	\begin{pgfonlayer}{nodelayer}
		\node [style=none] (92) at (18.5, 9.25) {};
		\node [style=none] (93) at (18.5, 8.5) {};
		\node [style=none] (94) at (18.5, 10) {};
		\node [style=triflip] (95) at (18.5, 9.25) {};
		\node [style=none] (96) at (18.85, 9.5) {$-1$};
	\end{pgfonlayer}
	\begin{pgfonlayer}{edgelayer}
		\draw (94.center) to (92.center);
		\draw (92.center) to (93.center);
	\end{pgfonlayer}
\end{tikzpicture}
 \right\rrbracket
=
\begin{bmatrix}
1 & -1\\
0 & 1
\end{bmatrix}
$$

So that now for $n \in \Z$:
$$
\begin{tikzpicture}
	\begin{pgfonlayer}{nodelayer}
		\node [style=H] (85) at (1, 11.25) {$-n$};
		\node [style=none] (86) at (1, 10.5) {};
		\node [style=none] (87) at (1, 12) {};
	\end{pgfonlayer}
	\begin{pgfonlayer}{edgelayer}
		\draw (87.center) to (85);
		\draw (85) to (86.center);
	\end{pgfonlayer}
\end{tikzpicture}
:=
\begin{tikzpicture}
	\begin{pgfonlayer}{nodelayer}
		\node [style=none] (103) at (21, -0.5) {};
		\node [style=none] (104) at (21.75, -0.5) {};
		\node [style=X] (105) at (21, 2) {};
		\node [style=none] (106) at (21, -1.5) {};
		\node [style=andin] (107) at (21, -0.5) {};
		\node [style=none] (108) at (21.75, 2.5) {};
		\node [style=triflip] (109) at (21, 1.25) {};
		\node [style=triflip] (110) at (21, 0.25) {};
		\node [style=none] (111) at (20.75, 0.75) {$n$};
		\node [style=none] (112) at (21.35, 1.5) {$-1$};
		\node [style=none] (113) at (21.35, 0.5) {$-1$};
	\end{pgfonlayer}
	\begin{pgfonlayer}{edgelayer}
		\draw [style=simple, in=-75, out=-90, looseness=2.75] (104.center) to (103.center);
		\draw [style=simple] (106.center) to (103.center);
		\draw [style=dotted] (109) to (110);
		\draw (109) to (105);
		\draw (110) to (103.center);
		\draw [style=simple] (104.center) to (108.center);
	\end{pgfonlayer}
\end{tikzpicture}\ ,
\hspace*{1cm}
\begin{tikzpicture}
	\begin{pgfonlayer}{nodelayer}
		\node [style=none] (0) at (1, 11.25) {};
		\node [style=H] (10) at (1, 11.25) {$-n$};
		\node [style=none] (1) at (1.5, 10.5) {};
		\node [style=none] (2) at (1.5, 12) {};
		\node [style=none] (3) at (0.5, 10.5) {};
		\node [style=none] (4) at (0.5, 12) {};
		\node [style=none] (5) at (1, 12) {$\cdots$};
		\node [style=none] (6) at (1, 10.5) {$\cdots$};
	\end{pgfonlayer}
	\begin{pgfonlayer}{edgelayer}
		\draw [in=30, out=-90] (2.center) to (0);
		\draw [in=90, out=-30] (0) to (1.center);
		\draw [in=-150, out=90] (3.center) to (0);
		\draw [in=-90, out=150] (0) to (4.center);
	\end{pgfonlayer}
\end{tikzpicture}
:=
\begin{tikzpicture}
	\begin{pgfonlayer}{nodelayer}
		\node [style=H] (0) at (1, 11.25) {$-n$};
		\node [style=none] (2) at (1, 12) {};
		\node [style=andout] (3) at (1, 12) {};
		\node [style=andin] (4) at (1, 10.5) {};
		\node [style=none] (5) at (0.5, 12.75) {};
		\node [style=none] (6) at (1.5, 12.75) {};
		\node [style=none] (7) at (1, 10.5) {};
		\node [style=none] (8) at (0.5, 9.75) {};
		\node [style=none] (9) at (1.5, 9.75) {};
		\node [style=none] (10) at (1, 12.75) {$\cdots$};
		\node [style=none] (11) at (1, 9.75) {$\cdots$};
	\end{pgfonlayer}
	\begin{pgfonlayer}{edgelayer}
		\draw (2.center) to (0);
		\draw [in=-90, out=30] (2.center) to (6.center);
		\draw [in=-90, out=150] (2.center) to (5.center);
		\draw [in=90, out=-30] (7.center) to (9.center);
		\draw [in=90, out=-150] (7.center) to (8.center);
		\draw (7.center) to (0);
	\end{pgfonlayer}
\end{tikzpicture}
\ ,
\hspace*{1cm}
\begin{tikzpicture}
	\begin{pgfonlayer}{nodelayer}
		\node [style=none] (0) at (1, 11.25) {};
		\node [style=H] (10) at (1, 11.25) {};
		\node [style=none] (1) at (1.5, 10.5) {};
		\node [style=none] (2) at (1.5, 12) {};
		\node [style=none] (3) at (0.5, 10.5) {};
		\node [style=none] (4) at (0.5, 12) {};
		\node [style=none] (5) at (1, 12) {$\cdots$};
		\node [style=none] (6) at (1, 10.5) {$\cdots$};
	\end{pgfonlayer}
	\begin{pgfonlayer}{edgelayer}
		\draw [in=30, out=-90] (2.center) to (0);
		\draw [in=90, out=-30] (0) to (1.center);
		\draw [in=-150, out=90] (3.center) to (0);
		\draw [in=-90, out=150] (0) to (4.center);
	\end{pgfonlayer}
\end{tikzpicture}
:=
\begin{tikzpicture}
	\begin{pgfonlayer}{nodelayer}
		\node [style=none] (0) at (1, 11.25) {};
		\node [style=H] (10) at (1, 11.25) {$-1$};
		\node [style=none] (1) at (1.5, 10.5) {};
		\node [style=none] (2) at (1.5, 12) {};
		\node [style=none] (3) at (0.5, 10.5) {};
		\node [style=none] (4) at (0.5, 12) {};
		\node [style=none] (5) at (1, 12) {$\cdots$};
		\node [style=none] (6) at (1, 10.5) {$\cdots$};
	\end{pgfonlayer}
	\begin{pgfonlayer}{edgelayer}
		\draw [in=30, out=-90] (2.center) to (0);
		\draw [in=90, out=-30] (0) to (1.center);
		\draw [in=-150, out=90] (3.center) to (0);
		\draw [in=-90, out=150] (0) to (4.center);
	\end{pgfonlayer}
\end{tikzpicture}
$$



Therefore, by translating presentation of the phase-free ZH-calculus found in \cite{zhpi}, we have immediately:

\begin{corollary}
The prop presented by the generators and relations of $\ZXA$ in addition to the two extra generators interpreted in $\Mat_{\Z[1/\sqrt{2}]}$ as:

$$
\left\llbracket\ 
\begin{tikzpicture}
	\begin{pgfonlayer}{nodelayer}
		\node [style=none] (5) at (0.75, 0.5) {};
		\node [style=Z] (16) at (0.75, -0.25) {$1$};
	\end{pgfonlayer}
	\begin{pgfonlayer}{edgelayer}
		\draw (5.center) to (16);
	\end{pgfonlayer}
\end{tikzpicture}
\ \right\rrbracket
=
(1,-1)^T \ ,\hspace*{.5cm}
\left\llbracket \
\begin{tikzpicture}
	\begin{pgfonlayer}{nodelayer}
		\node [shape=star,fill=black,scale=.5] (16) at (0.75, 0.25) {};
	\end{pgfonlayer}
	\begin{pgfonlayer}{edgelayer}
	\end{pgfonlayer}
\end{tikzpicture}
\ \right\rrbracket
=1/\sqrt 2
$$

Modulo the equations:

$$
\begin{tikzpicture}
	\begin{pgfonlayer}{nodelayer}
		\node [shape=star, fill=black,scale=.5] (33) at (-0.25, -2) {};
		\node [shape=star, fill=black,scale=.5] (34) at (0.25, -2) {};
		\node [style=Z] (35) at (0.75, -2) {};
	\end{pgfonlayer}
\end{tikzpicture}
=
\begin{tikzpicture}
	\begin{pgfonlayer}{nodelayer}
		\node [style=none] (0) at (2, 0) {};
		\node [style=none] (1) at (2, -1) {};
		\node [style=none] (2) at (3, -1) {};
		\node [style=none] (3) at (3, 0) {};
	\end{pgfonlayer}
	\begin{pgfonlayer}{edgelayer}
		\draw[style=dashed] (3.center) to (0.center) to (1.center) to (2.center) to cycle;
	\end{pgfonlayer}
\end{tikzpicture}\ ,
\hspace*{.2cm}
\begin{tikzpicture}
	\begin{pgfonlayer}{nodelayer}
		\node [style=none] (25) at (4.5, 0.5) {};
		\node [style=none] (26) at (5.5, 0.5) {};
		\node [style=Z] (27) at (5, 1.25) {};
		\node [style=none] (28) at (5, 0.75) {$\cdots$};
		\node [style=none] (29) at (5, 1.75) {$\cdots$};
		\node [style=none] (30) at (5.5, 2) {};
		\node [style=none] (31) at (5, 1.25) {};
		\node [style=none] (32) at (4.5, 2) {};
		\node [shape=star, fill=black,scale=.5] (33) at (3.5, 1.25) {};
		\node [shape=star, fill=black,scale=.5] (34) at (4, 1.25) {};
		\node [style=H] (36) at (4.5, 2) {};
		\node [style=H] (37) at (5.5, 2) {};
		\node [style=H] (38) at (5.5, 0.5) {};
		\node [style=H] (39) at (4.5, 0.5) {};
		\node [style=none] (40) at (4.5, 2.5) {};
		\node [style=none] (41) at (5.5, 2.5) {};
		\node [style=none] (42) at (5.5, 0) {};
		\node [style=none] (43) at (4.5, 0) {};
	\end{pgfonlayer}
	\begin{pgfonlayer}{edgelayer}
		\draw [style=simple, in=-56, out=90] (26.center) to (27);
		\draw [style=simple, in=90, out=-124] (27) to (25.center);
		\draw [style=simple, in=56, out=-90] (30.center) to (31.center);
		\draw [style=simple, in=-90, out=124] (31.center) to (32.center);
		\draw (40.center) to (36);
		\draw (37) to (41.center);
		\draw (43.center) to (39);
		\draw (42.center) to (38);
	\end{pgfonlayer}
\end{tikzpicture}
=
\begin{tikzpicture}
	\begin{pgfonlayer}{nodelayer}
		\node [style=none] (44) at (6.5, 0.75) {};
		\node [style=none] (45) at (7.5, 0.75) {};
		\node [style=X] (46) at (7, 1.5) {};
		\node [style=none] (47) at (7, 1) {$\cdots$};
		\node [style=none] (48) at (7, 2) {$\cdots$};
		\node [style=none] (49) at (7.5, 2.25) {};
		\node [style=none] (50) at (7, 1.5) {};
		\node [style=none] (51) at (6.5, 2.25) {};
	\end{pgfonlayer}
	\begin{pgfonlayer}{edgelayer}
		\draw [style=simple, in=-56, out=90] (45.center) to (46);
		\draw [style=simple, in=90, out=-124] (46) to (44.center);
		\draw [style=simple, in=56, out=-90] (49.center) to (50.center);
		\draw [style=simple, in=-90, out=124] (50.center) to (51.center);
	\end{pgfonlayer}
\end{tikzpicture}
$$
is isomorphic to the phase-free ZH-calculus.
\end{corollary}

These two equations express the fact that $1/\sqrt 2$ is invertible and that the two spiders are related via Fourier transform.  The phase free ZH-calculus is already known to be approximately universal for qubits, so it is quite remarkable that our classical presentation of $\ZXA$ needs so little to be so quantum.  Because of the way we proved completeness for natural number qubit matrices; perhaps there is a more direct way to get around adding the scalar $1/\sqrt 2$ to get a presentation for qubit matrices over $\Z$, however we leave this for future work.



%
%\nocite{coecke2008classical}
%\nocite{cnot}
%\nocite{tof}
%\nocite{Cole}
%\nocite{elltwo}
%%\nocite{sam}
%\nocite{coecke2017two}
%\nocite{carboni}
%\nocite{butz}
%\nocite{pqp}
%\nocite{lack2004composing}


\section{Decomposing Boolean circuits via distributive law}
\label{sec:dist}
In this section, we modularly build up to the presentation of $\ZXA$ by taking distributive laws and pushouts of smaller symmetric monoidal theories.  Along the way, we obtain various fragments of quantum circuits with interesting semantics.


%%We first review some basic theory involving the presentation of props.  These results are mostly folklore, however, I will refer the reader to \cite[\S 2]{ih} for a more comprehensive introduction.
%
%\begin{definition}
%A {\bf pro} is a strict monoidal category generated by one object under the tensor product, and a {\bf prop} is a  strict {\em symmetric} monoidal category generated by one object under the tensor product
%
%\end{definition}

%
%\begin{definition}
%A {\bf monoidal theory} is a pair $(\Sigma,E)$ of {\bf generators} $\Sigma$ and {\bf equations} $E$.
%Each generator $f \in \Sigma$ has a chosen domain $\dom (f) \in \N$  and codomain $\cod (f) \in \N$, so that $f$ can be seen as a map from $\dom(f)$ to $\cod (f)$.
%
%The free pro with signature $\Sigma$ has maps in $\Sigma^*$ obtained by inductively  tensoring all the generators and composing all appropriately typed generators in $\Sigma$,
%The equations in $E$ are pairs of parallel maps in $\Sigma^*$.
%Any monoidal theory $(\Sigma,E)$  generates a pro $\bar{(\Sigma,E)}$ given by the free pro with signature $\Sigma$ modulo the equations in $E$.
%
%A {\bf symmetric monoidal theory} is the symmetric version of a monoidal theory, which generates a prop.  Here the set $\Sigma^*$ is obtained by composing and tensoring maps with symmetries, and then quotienting by the axioms of a prop.
%\end{definition}
%
%\begin{lemma}
%Given two (symmetric) monoidal theories $(\Sigma_1,E_1)$  and $(\Sigma_2,E_2)$  the coproduct of pro(p)s  $\bar{(\Sigma_1,E_1)}+\bar{(\Sigma_2,E_2)}$ is generated by the (symmetric) monoidal theory $(\Sigma_1+\Sigma_2,E_1+E_2)$.
%\end{lemma}
%
%
%\begin{lemma}
%Given three  (symmetric) monoidal theories $(\Sigma_1,E_1)$, $(\Sigma_2,E_2)$ and $(\Sigma_3,E_3)$ where $\bar{(\Sigma_3,E_3)}$ is a sub-pro(p) of both $\bar{(\Sigma_1,E_1)}$ and $\bar{(\Sigma_2,E_2)}$.  The pushout of the diagram of pro(p)s
%$$
%\bar{(\Sigma_1,E_1)} \leftarrow \bar{(\Sigma_3,E_3)}\rightarrow \bar{(\Sigma_2,E_2)}
%$$
%is generated by the (symmetric) monoidal theory $(\Sigma_1^* +_{\Sigma_3} \Sigma_2^*, E_1 + E_2)$.
%\end{lemma}

%
%\begin{definition}
%Informally, given two small categories $\X,\Y$ with the same objects; a distributive law on $\Y \otimes \X$ is a way to turn formal composites of maps in $\Y$ followed by $\X$ into a category.  That is to say, a quotient which gives us a way to turn composites of the form $\xrightarrow{f \in \X} \xrightarrow{g \in \Y}$  into ones of the form  $\xrightarrow{g' \in \Y} \xrightarrow{f' \in \X}$.
%
%If $\X$ and $\Y$ have some shared structure witnessed by a subcategory $\Z$ with the same objects, a relaxed distributive law $\Y \otimes_{\Z} \X$ is like a distributive law $\Y \otimes \X$, except where the maps in $\Y$ can be identified as either being in $\X$ or in $\Y$.
%\footnote{These are called distributive laws because small categories are monads in the bicategory $\Span(\Sets)$; where distributive laws in the first sense are distributive laws of the corresponding monads in $\Span(\Sets)$.  In the latter case, these are given by distributive laws of bimodules of spans of sets (ie. distributive laws in small profunctors).}
%\end{definition}
%
%This second notion of distributive law will be needed when $\X$ and $\Y$ are props; because they (minimally) share a subcategory $\P$ of permutations.

%We recall the novel way to compose pro(p), first described in \cite{lack}:
%\begin{definition}
%Suppose there three  (symmetric) monoidal theories $(\Sigma_1,E_1)$, $(\Sigma_2,E_2)$ and $(\Sigma_3,E_3)$ where $\bar{(\Sigma_3,E_3)}$ is a sub-pro(p) of both $\bar{(\Sigma_1,E_1)}$ and $\bar{(\Sigma_2,E_2)}$. A {\bf distributive law of pro(p)s} is a distributive law $\lambda:\bar{(\Sigma_2,E_2)} \otimes_{\bar{(\Sigma_3,E_3)}} \bar{ (\Sigma_1,E_1)}$   in $\Mon$-$\Prof$.  Informally, this is a way to push all the maps in $\Sigma_1^*$ past those of  $\Sigma_2^*$ modulo $\Sigma_3$ and the equations $E_1+E_2$ and the axioms of a pro(p).
%\end{definition}
%
%In \cite{lack} it is required that $\bar{(\Sigma_3,E_3)}$ is a groupoid; however, we must loosen this requirement (note that when this is not a groupoid, there is no correspondence to factorization systems as in \cite{rosebrugh}).  

%\begin{lemma}
%Suppose that we have three (symmetric) monoidal theories and a distribtuive law $\lambda:\bar{(\Sigma_2,E_2)} \otimes_{\bar{(\Sigma_3,E_3)}} \bar{ (\Sigma_1,E_1)}$ as above.
%
%Then the induced pro(p) $\bar{(\Sigma_2,E_2)} \otimes_{\bar{(\Sigma_3,E_3)}} \bar{ (\Sigma_1,E_1)}$ is presented by the monoidal theory\\ ${(\Sigma_1^* +_{\Sigma_3} \Sigma_2^*, E_1 + E_2+E_\lambda)}$, where $E_\lambda$ are all the equations needed to push elements of $\Sigma_1^*$ past those of $ \Sigma_2^*$ up to  $\Sigma_3^*$, dictated by $\lambda$.
%\end{lemma}






\section{The phase-free fragment}
\label{sec:one}

In this section we build up to giving a presentation for $(\Span(\Mat(\F_2)),+)$ in a modular way. This category is shown to be the same as the phase-free Hadamard free fragment of the ZX-calculus. Although this presentation of linear spans has already been discussed in great detail  for arbitrary PIDs \cite{ih}, our particular method of exposition is necessary to motivate the affine and full cases. 


\begin{definition}
Consider the prop $\Iso(\cb_2)$ generated by the controlled not gate:
\begin{align*}
&\left\llbracket
\begin{tikzpicture}
	\begin{pgfonlayer}{nodelayer}
		\node [style=oplus] (5) at (0.5, 2.75) {};
		\node [style=dot] (6) at (0, 2.75) {};
		\node [style=none] (7) at (0.5, 3.5) {};
		\node [style=none] (8) at (0.5, 2) {};
		\node [style=none] (9) at (0, 2) {};
		\node [style=none] (10) at (0, 3.5) {};
	\end{pgfonlayer}
	\begin{pgfonlayer}{edgelayer}
		\draw (8.center) to (5);
		\draw (5) to (7.center);
		\draw (10.center) to (6);
		\draw (6) to (5);
		\draw (6) to (9.center);
	\end{pgfonlayer}
\end{tikzpicture}
\right\rrbracket
=
\begin{tikzpicture}
	\begin{pgfonlayer}{nodelayer}
		\node [style=X] (0) at (-0.25, -1) {};
		\node [style=Z] (1) at (-0.75, -1.75) {};
		\node [style=none] (2) at (0, -2.25) {};
		\node [style=none] (3) at (-0.25, -0.5) {};
		\node [style=none] (4) at (-1, -0.5) {};
		\node [style=none] (5) at (-0.75, -2.25) {};
	\end{pgfonlayer}
	\begin{pgfonlayer}{edgelayer}
		\draw (3.center) to (0);
		\draw [in=90, out=-75] (0) to (2.center);
		\draw (5.center) to (1);
		\draw (1) to (0);
		\draw [in=-90, out=105] (1) to (4.center);
	\end{pgfonlayer}
\end{tikzpicture} 
\hspace*{1cm}
\text{modulo the following relations:}\\
\begin{tikzpicture}
	\begin{pgfonlayer}{nodelayer}
		\node [style=dot] (0) at (7, 0) {};
		\node [style=oplus] (1) at (7.5, 0) {};
		\node [style=oplus] (2) at (7.5, 0.5) {};
		\node [style=dot] (3) at (7, 0.5) {};
		\node [style=none] (4) at (7.5, 1) {};
		\node [style=none] (5) at (7, 1) {};
		\node [style=none] (6) at (7, -0.5) {};
		\node [style=none] (7) at (7.5, -0.5) {};
	\end{pgfonlayer}
	\begin{pgfonlayer}{edgelayer}
		\draw (5.center) to (3);
		\draw (3) to (0);
		\draw (0) to (6.center);
		\draw (7.center) to (1);
		\draw (1) to (2);
		\draw (2) to (4.center);
		\draw (2) to (3);
		\draw (0) to (1);
	\end{pgfonlayer}
\end{tikzpicture}
\eqzxa{cnot.one}
\begin{tikzpicture}
	\begin{pgfonlayer}{nodelayer}
		\node [style=none] (4) at (7.5, 1) {};
		\node [style=none] (5) at (7, 1) {};
		\node [style=none] (6) at (7, -0.5) {};
		\node [style=none] (7) at (7.5, -0.5) {};
	\end{pgfonlayer}
	\begin{pgfonlayer}{edgelayer}
		\draw (7.center) to (4.center);
		\draw (5.center) to (6.center);
	\end{pgfonlayer}
\end{tikzpicture}
&\hspace*{.5cm}
\begin{tikzpicture}
	\begin{pgfonlayer}{nodelayer}
		\node [style=none] (4) at (7.5, 1) {};
		\node [style=none] (5) at (7, 1) {};
		\node [style=none] (6) at (7, -1) {};
		\node [style=none] (7) at (7.5, -1) {};
		\node [style=dot] (8) at (7.5, 0.5) {};
		\node [style=dot] (9) at (7.5, -0.5) {};
		\node [style=dot] (10) at (7, 0) {};
		\node [style=oplus] (11) at (7.5, 0) {};
		\node [style=oplus] (12) at (7, 0.5) {};
		\node [style=oplus] (13) at (7, -0.5) {};
	\end{pgfonlayer}
	\begin{pgfonlayer}{edgelayer}
		\draw (7.center) to (4.center);
		\draw (5.center) to (6.center);
		\draw (8) to (12);
		\draw (10) to (11);
		\draw (9) to (13);
	\end{pgfonlayer}
\end{tikzpicture}
\eqzxa{cnot.two}
\begin{tikzpicture}
	\begin{pgfonlayer}{nodelayer}
		\node [style=none] (4) at (7, 1) {};
		\node [style=none] (5) at (7.5, 1) {};
		\node [style=none] (6) at (7, -1) {};
		\node [style=none] (7) at (7.5, -1) {};
	\end{pgfonlayer}
	\begin{pgfonlayer}{edgelayer}
		\draw [in=270, out=90] (7.center) to (4.center);
		\draw [in=90, out=-90] (5.center) to (6.center);
	\end{pgfonlayer}
\end{tikzpicture}
\hspace*{.5cm}
\begin{tikzpicture}
	\begin{pgfonlayer}{nodelayer}
		\node [style=dot] (0) at (8, 0) {};
		\node [style=dot] (1) at (8.5, 0.5) {};
		\node [style=dot] (2) at (8.5, -0.5) {};
		\node [style=oplus] (3) at (8.5, 0) {};
		\node [style=oplus] (4) at (9, 0.5) {};
		\node [style=oplus] (5) at (9, -0.5) {};
		\node [style=none] (6) at (9, -1) {};
		\node [style=none] (7) at (8.5, -1) {};
		\node [style=none] (8) at (8, -1) {};
		\node [style=none] (9) at (8, 1) {};
		\node [style=none] (10) at (8.5, 1) {};
		\node [style=none] (11) at (9, 1) {};
	\end{pgfonlayer}
	\begin{pgfonlayer}{edgelayer}
		\draw (9.center) to (8.center);
		\draw (7.center) to (10.center);
		\draw (11.center) to (6.center);
		\draw (5) to (2);
		\draw (3) to (0);
		\draw (1) to (4);
	\end{pgfonlayer}
\end{tikzpicture}
\eqzxa{cnot.three}
\begin{tikzpicture}
	\begin{pgfonlayer}{nodelayer}
		\node [style=dot] (0) at (8, -0.25) {};
		\node [style=dot] (1) at (8, 0.25) {};
		\node [style=oplus] (3) at (8.5, -0.25) {};
		\node [style=oplus] (4) at (9, 0.25) {};
		\node [style=none] (6) at (9, -1) {};
		\node [style=none] (7) at (8.5, -1) {};
		\node [style=none] (8) at (8, -1) {};
		\node [style=none] (9) at (8, 1) {};
		\node [style=none] (10) at (8.5, 1) {};
		\node [style=none] (11) at (9, 1) {};
	\end{pgfonlayer}
	\begin{pgfonlayer}{edgelayer}
		\draw (9.center) to (8.center);
		\draw (7.center) to (10.center);
		\draw (11.center) to (6.center);
		\draw (3) to (0);
		\draw (1) to (4);
	\end{pgfonlayer}
\end{tikzpicture}
\hspace*{.5cm}
\begin{tikzpicture}
	\begin{pgfonlayer}{nodelayer}
		\node [style=dot] (0) at (2, 1.5) {};
		\node [style=oplus] (1) at (2.5, 1.5) {};
		\node [style=none] (3) at (2, 1) {};
		\node [style=none] (4) at (2.5, 1) {};
		\node [style=none] (5) at (2, 2.5) {};
		\node [style=none] (6) at (2.5, 2.5) {};
		\node [style=none] (7) at (3, 2.5) {};
		\node [style=none] (8) at (3, 1) {};
		\node [style=dot] (9) at (3, 2) {};
		\node [style=oplus] (10) at (2.5, 2) {};
	\end{pgfonlayer}
	\begin{pgfonlayer}{edgelayer}
		\draw (0) to (5.center);
		\draw (6.center) to (1);
		\draw (1) to (0);
		\draw (4.center) to (1);
		\draw (3.center) to (0);
		\draw (10) to (9);
		\draw (8.center) to (7.center);
	\end{pgfonlayer}
\end{tikzpicture}
\eqzxa{cnot.four}
\begin{tikzpicture}
	\begin{pgfonlayer}{nodelayer}
		\node [style=dot] (0) at (2, 2) {};
		\node [style=oplus] (1) at (2.5, 2) {};
		\node [style=none] (3) at (2, 2.5) {};
		\node [style=none] (4) at (2.5, 2.5) {};
		\node [style=none] (5) at (2, 1) {};
		\node [style=none] (6) at (2.5, 1) {};
		\node [style=none] (7) at (3, 1) {};
		\node [style=none] (8) at (3, 2.5) {};
		\node [style=dot] (9) at (3, 1.5) {};
		\node [style=oplus] (10) at (2.5, 1.5) {};
	\end{pgfonlayer}
	\begin{pgfonlayer}{edgelayer}
		\draw (0) to (5.center);
		\draw (6.center) to (1);
		\draw (1) to (0);
		\draw (4.center) to (1);
		\draw (3.center) to (0);
		\draw (10) to (9);
		\draw (8.center) to (7.center);
	\end{pgfonlayer}
\end{tikzpicture}
\hspace*{.5cm}
\begin{tikzpicture}
	\begin{pgfonlayer}{nodelayer}
		\node [style=oplus] (0) at (2, 1.5) {};
		\node [style=dot] (1) at (2.5, 1.5) {};
		\node [style=none] (3) at (2, 1) {};
		\node [style=none] (4) at (2.5, 1) {};
		\node [style=none] (5) at (2, 2.5) {};
		\node [style=none] (6) at (2.5, 2.5) {};
		\node [style=none] (7) at (3, 2.5) {};
		\node [style=none] (8) at (3, 1) {};
		\node [style=oplus] (9) at (3, 2) {};
		\node [style=dot] (10) at (2.5, 2) {};
	\end{pgfonlayer}
	\begin{pgfonlayer}{edgelayer}
		\draw (0) to (5.center);
		\draw (6.center) to (1);
		\draw (1) to (0);
		\draw (4.center) to (1);
		\draw (3.center) to (0);
		\draw (10) to (9);
		\draw (8.center) to (7.center);
	\end{pgfonlayer}
\end{tikzpicture}
\eqzxa{cnot.five}
\begin{tikzpicture}
	\begin{pgfonlayer}{nodelayer}
		\node [style=oplus] (0) at (2, 2) {};
		\node [style=dot] (1) at (2.5, 2) {};
		\node [style=none] (3) at (2, 2.5) {};
		\node [style=none] (4) at (2.5, 2.5) {};
		\node [style=none] (5) at (2, 1) {};
		\node [style=none] (6) at (2.5, 1) {};
		\node [style=none] (7) at (3, 1) {};
		\node [style=none] (8) at (3, 2.5) {};
		\node [style=oplus] (9) at (3, 1.5) {};
		\node [style=dot] (10) at (2.5, 1.5) {};
	\end{pgfonlayer}
	\begin{pgfonlayer}{edgelayer}
		\draw (0) to (5.center);
		\draw (6.center) to (1);
		\draw (1) to (0);
		\draw (4.center) to (1);
		\draw (3.center) to (0);
		\draw (10) to (9);
		\draw (8.center) to (7.center);
	\end{pgfonlayer}
\end{tikzpicture}
\end{align*}
\end{definition}




\begin{lemma} \cite[Thm. 6]{lafont}
$\Iso(\cb_2)$ is a presentation for the prop $(\Iso(\Mat(\F_2),+))$
\end{lemma}



\begin{definition}
Consider the prop $\inj(\cb_2)$ generated by the coproduct of props $\Iso(\cb_2)+\inj$ modulo the equation:
\hspace*{1cm}
$
\begin{tikzpicture}
	\begin{pgfonlayer}{nodelayer}
		\node [style=X] (0) at (0, 0) {};
		\node [style=dot] (1) at (0, 0.5) {};
		\node [style=oplus] (2) at (0.5, 0.5) {};
		\node [style=none] (3) at (0.5, -0.25) {};
		\node [style=none] (4) at (0.5, 1) {};
		\node [style=none] (5) at (0, 1) {};
	\end{pgfonlayer}
	\begin{pgfonlayer}{edgelayer}
		\draw (0) to (1);
		\draw (1) to (5.center);
		\draw (1) to (2);
		\draw (2) to (4.center);
		\draw (3.center) to (2);
	\end{pgfonlayer}
\end{tikzpicture}
\eqzxa{cnot.six}
\begin{tikzpicture}
	\begin{pgfonlayer}{nodelayer}
		\node [style=X] (0) at (0, 0) {};
		\node [style=none] (3) at (0.5, -0.25) {};
		\node [style=none] (4) at (0.5, 0.5) {};
		\node [style=none] (5) at (0, 0.5) {};
	\end{pgfonlayer}
	\begin{pgfonlayer}{edgelayer}
		\draw (0) to (5.center);
		\draw (3.center) to (4.center);
	\end{pgfonlayer}
\end{tikzpicture}
$

\end{definition}


\begin{lemma} \cite[Thm. 7]{lafont}
$\inj(\cb_2)$ is a presentation for the prop $(\inj(\Mat(\F_2)),+)$
\end{lemma}


The white comultiplication can be derived in this fragment:
\hspace*{1cm}$
\left\llbracket
\begin{tikzpicture}
	\begin{pgfonlayer}{nodelayer}
		\node [style=oplus] (0) at (1, 2.75) {};
		\node [style=dot] (1) at (0.5, 2.75) {};
		\node [style=none] (2) at (1, 3.5) {};
		\node [style=none] (3) at (1, 2.25) {};
		\node [style=none] (4) at (0.5, 2) {};
		\node [style=none] (5) at (0.5, 3.5) {};
		\node [style=X] (6) at (1, 2.25) {};
	\end{pgfonlayer}
	\begin{pgfonlayer}{edgelayer}
		\draw (3.center) to (0);
		\draw (0) to (2.center);
		\draw (5.center) to (1);
		\draw (1) to (0);
		\draw (1) to (4.center);
	\end{pgfonlayer}
\end{tikzpicture}
\right\rrbracket
=
\begin{tikzpicture}
	\begin{pgfonlayer}{nodelayer}
		\node [style=X] (0) at (1.25, -1) {};
		\node [style=Z] (1) at (0.75, -1.75) {};
		\node [style=none] (2) at (1.5, -2) {};
		\node [style=none] (3) at (1.25, -0.5) {};
		\node [style=none] (4) at (0.5, -0.5) {};
		\node [style=none] (5) at (0.75, -2.25) {};
		\node [style=X] (6) at (1.5, -2) {};
	\end{pgfonlayer}
	\begin{pgfonlayer}{edgelayer}
		\draw (3.center) to (0);
		\draw [in=90, out=-75] (0) to (2.center);
		\draw (5.center) to (1);
		\draw (1) to (0);
		\draw [in=-90, out=105] (1) to (4.center);
	\end{pgfonlayer}
\end{tikzpicture}
=
\begin{tikzpicture}
	\begin{pgfonlayer}{nodelayer}
		\node [style=Z] (1) at (0.75, -1.75) {};
		\node [style=none] (3) at (1, -1) {};
		\node [style=none] (4) at (0.5, -1) {};
		\node [style=none] (5) at (0.75, -2.25) {};
	\end{pgfonlayer}
	\begin{pgfonlayer}{edgelayer}
		\draw (5.center) to (1);
		\draw [in=-90, out=120] (1) to (4.center);
		\draw [in=-90, out=60] (1) to (3.center);
	\end{pgfonlayer}
\end{tikzpicture}
$

%This is to be expected because there is a faithful ``'graph functor' monoidal functor $(\inj(\FinSet),+) \to (\Inj(\Span(\FinSet)),+)$.




As a matter of notation, given a category $\X$ with finite limits, we refer to the subcategory of $\Span(\X)$ where the left leg is monic as $\Par(\X)$, and the subcategory of spans where all legs are monic by $\Par\Iso(\X)$.  These two categories, respectively, give semantics for partial maps and partially invertible maps in $\X$ (see \cite{cockett} for more details).


\begin{definition}
\label{def:pariso:cb}
Consider the prop $\ParIso(\cb_2)$ generated by the distributive law of props:
$$
\inj(\cb_2)^\op \otimes_{\Iso(\cb_2)} \inj(\cb_2);
\begin{tikzpicture}
	\begin{pgfonlayer}{nodelayer}
		\node [style=X] (0) at (0, 0) {};
		\node [style=X] (1) at (0, 0.75) {};
	\end{pgfonlayer}
	\begin{pgfonlayer}{edgelayer}
		\draw (0) to (1);
	\end{pgfonlayer}
\end{tikzpicture}
\eref{extra}
$$
\end{definition}


\begin{remark}
\label{rem:pariso:cb}
This is actually a distributive law because the only only seemingly nontrivial situation arises when controlled not gates are sandwiched by black units/counits on their target wires.  However the following identity holds by induction on the number of controlled not gates.   For the base case of $n=0$, this follows from the bone law which we added to the distributive law.  For $n>1$, we have the following situation:
$$
\begin{tikzpicture}
	\begin{pgfonlayer}{nodelayer}
		\node [style=X] (0) at (1.5, -0.25) {};
		\node [style=oplus] (1) at (1.5, -0.75) {};
		\node [style=oplus] (2) at (1.5, -1.5) {};
		\node [style=dot] (3) at (1, -0.75) {};
		\node [style=dot] (4) at (0.5, -1.5) {};
		\node [style=none] (5) at (1, 0) {};
		\node [style=none] (6) at (0.5, 0) {};
		\node [style=none] (7) at (0.75, -1) {$\iddots$};
		\node [style=none] (8) at (1.5, -1) {$\vdots$};
		\node [style=X] (9) at (1.5, -2.5) {};
		\node [style=none] (10) at (0.5, -2.75) {};
		\node [style=none] (11) at (1, -2.75) {};
		\node [style=none] (12) at (0.75, -0.25) {$\cdots$};
		\node [style=oplus] (13) at (1.5, -2) {};
		\node [style=dot] (14) at (0, -2) {};
		\node [style=none] (15) at (0, -2.75) {};
		\node [style=none] (16) at (0, 0) {};
		\node [style=none] (18) at (1.25, -0.75) {};
	\end{pgfonlayer}
	\begin{pgfonlayer}{edgelayer}
		\draw (0) to (1);
		\draw (1) to (3);
		\draw (5.center) to (3);
		\draw (6.center) to (4);
		\draw (4) to (2);
		\draw (4) to (10.center);
		\draw (11.center) to (3);
		\draw (9) to (2);
		\draw (14) to (13);
		\draw (15.center) to (14);
		\draw (14) to (16.center);
	\end{pgfonlayer}
\end{tikzpicture}
=
\begin{tikzpicture}
	\begin{pgfonlayer}{nodelayer}
		\node [style=X] (19) at (3.5, -1.25) {};
		\node [style=none] (24) at (3.5, 0.5) {};
		\node [style=none] (25) at (3, 0.5) {};
		\node [style=X] (28) at (3.5, -1.75) {};
		\node [style=none] (29) at (3, -3.5) {};
		\node [style=none] (30) at (3.5, -3.5) {};
		\node [style=none] (34) at (2.5, -3.5) {};
		\node [style=none] (35) at (2.5, 0.5) {};
		\node [style=oplus] (36) at (3.5, -0.75) {};
		\node [style=oplus] (37) at (3.5, 0) {};
		\node [style=oplus] (38) at (3.5, -3) {};
		\node [style=oplus] (39) at (3.5, -2.25) {};
		\node [style=dot] (40) at (2.5, -3) {};
		\node [style=dot] (41) at (3, -2.25) {};
		\node [style=dot] (42) at (3, -0.75) {};
		\node [style=dot] (43) at (2.5, 0) {};
		\node [style=none] (44) at (3.5, -0.25) {$\vdots$};
		\node [style=none] (45) at (3.5, -2.5) {$\vdots$};
		\node [style=none] (46) at (2.75, -1.5) {$\cdots$};
		\node [style=none] (47) at (2.75, -2.5) {$\iddots$};
		\node [style=none] (48) at (2.75, -0.25) {$\ddots$};
	\end{pgfonlayer}
	\begin{pgfonlayer}{edgelayer}
		\draw (24.center) to (37);
		\draw (36) to (19);
		\draw (28) to (39);
		\draw (38) to (30.center);
		\draw (29.center) to (41);
		\draw (41) to (42);
		\draw (42) to (25.center);
		\draw (35.center) to (43);
		\draw (43) to (40);
		\draw (40) to (38);
		\draw (37) to (43);
		\draw (42) to (36);
		\draw (39) to (41);
	\end{pgfonlayer}
\end{tikzpicture}
$$
For $n=1$, that is:
$$
\begin{tikzpicture}
	\begin{pgfonlayer}{nodelayer}
		\node [style=X] (0) at (4, 2) {};
		\node [style=X] (1) at (4, 1) {};
		\node [style=oplus] (2) at (4, 1.5) {};
		\node [style=dot] (3) at (3.5, 1.5) {};
		\node [style=none] (4) at (3.5, 2.25) {};
		\node [style=none] (5) at (3.5, 0.75) {};
	\end{pgfonlayer}
	\begin{pgfonlayer}{edgelayer}
		\draw (1) to (0);
		\draw (4.center) to (5.center);
		\draw (3) to (2);
	\end{pgfonlayer}
\end{tikzpicture}
=
\begin{tikzpicture}
	\begin{pgfonlayer}{nodelayer}
		\node [style=X] (0) at (4, 2.5) {};
		\node [style=X] (1) at (4, 0.5) {};
		\node [style=oplus] (2) at (4, 1.5) {};
		\node [style=dot] (3) at (3.5, 1.5) {};
		\node [style=none] (4) at (3.5, 2.75) {};
		\node [style=none] (5) at (3.5, 0.25) {};
		\node [style=oplus] (6) at (3.5, 2) {};
		\node [style=dot] (7) at (4, 2) {};
		\node [style=oplus] (8) at (3.5, 1) {};
		\node [style=dot] (9) at (4, 1) {};
	\end{pgfonlayer}
	\begin{pgfonlayer}{edgelayer}
		\draw (1) to (0);
		\draw (4.center) to (5.center);
		\draw (3) to (2);
		\draw (7) to (6);
		\draw (9) to (8);
	\end{pgfonlayer}
\end{tikzpicture}
=
\begin{tikzpicture}
	\begin{pgfonlayer}{nodelayer}
		\node [style=X] (0) at (4, 2.5) {};
		\node [style=X] (1) at (4, 1.5) {};
		\node [style=none] (4) at (3.5, 2.75) {};
		\node [style=none] (5) at (3.5, 1.25) {};
	\end{pgfonlayer}
	\begin{pgfonlayer}{edgelayer}
		\draw [in=-90, out=90] (1) to (4.center);
		\draw [in=90, out=-90] (0) to (5.center);
	\end{pgfonlayer}
\end{tikzpicture}
=
\begin{tikzpicture}
	\begin{pgfonlayer}{nodelayer}
		\node [style=X] (0) at (3.5, 1.75) {};
		\node [style=X] (1) at (3.5, 2.25) {};
		\node [style=none] (4) at (3.5, 2.75) {};
		\node [style=none] (5) at (3.5, 1.25) {};
	\end{pgfonlayer}
	\begin{pgfonlayer}{edgelayer}
		\draw [in=-90, out=90] (1) to (4.center);
		\draw [in=90, out=-90] (0) to (5.center);
	\end{pgfonlayer}
\end{tikzpicture}
$$
And for the base case for $n=2$:
$$
\begin{tikzpicture}
	\begin{pgfonlayer}{nodelayer}
		\node [style=X] (0) at (4.5, 2) {};
		\node [style=X] (1) at (4.5, 0.5) {};
		\node [style=oplus] (2) at (4.5, 1) {};
		\node [style=dot] (3) at (3.5, 1) {};
		\node [style=none] (4) at (3.5, 2.25) {};
		\node [style=none] (5) at (3.5, 0.25) {};
		\node [style=oplus] (6) at (4.5, 1.5) {};
		\node [style=dot] (7) at (4, 1.5) {};
		\node [style=none] (8) at (4, 2.25) {};
		\node [style=none] (9) at (4, 0.25) {};
	\end{pgfonlayer}
	\begin{pgfonlayer}{edgelayer}
		\draw (1) to (0);
		\draw (4.center) to (5.center);
		\draw (3) to (2);
		\draw (8.center) to (9.center);
		\draw (7) to (6);
	\end{pgfonlayer}
\end{tikzpicture}
=
\begin{tikzpicture}
	\begin{pgfonlayer}{nodelayer}
		\node [style=X] (0) at (4.5, 2.5) {};
		\node [style=X] (1) at (4.5, 0.5) {};
		\node [style=oplus] (2) at (4.5, 1) {};
		\node [style=dot] (3) at (3.5, 1) {};
		\node [style=none] (4) at (3.5, 2.75) {};
		\node [style=none] (5) at (3.5, 0.25) {};
		\node [style=oplus] (6) at (4.5, 1.5) {};
		\node [style=dot] (7) at (4, 1.5) {};
		\node [style=none] (8) at (4, 2.75) {};
		\node [style=none] (9) at (4, 0.25) {};
		\node [style=oplus] (10) at (4, 2) {};
		\node [style=dot] (11) at (4.5, 2) {};
	\end{pgfonlayer}
	\begin{pgfonlayer}{edgelayer}
		\draw (1) to (0);
		\draw (4.center) to (5.center);
		\draw (3) to (2);
		\draw (8.center) to (9.center);
		\draw (7) to (6);
		\draw (11) to (10);
	\end{pgfonlayer}
\end{tikzpicture}
=
\begin{tikzpicture}
	\begin{pgfonlayer}{nodelayer}
		\node [style=X] (0) at (4.5, 3.5) {};
		\node [style=X] (1) at (4.5, 0.5) {};
		\node [style=oplus] (2) at (4.5, 1) {};
		\node [style=dot] (3) at (3.5, 1) {};
		\node [style=none] (4) at (3.5, 3.75) {};
		\node [style=none] (5) at (3.5, 0.25) {};
		\node [style=oplus] (6) at (4.5, 2.5) {};
		\node [style=dot] (7) at (4, 2.5) {};
		\node [style=none] (8) at (4, 3.75) {};
		\node [style=none] (9) at (4, 0.25) {};
		\node [style=oplus] (10) at (4, 3) {};
		\node [style=dot] (11) at (4.5, 3) {};
		\node [style=oplus] (12) at (4, 2) {};
		\node [style=dot] (13) at (4.5, 2) {};
		\node [style=oplus] (14) at (4, 1.5) {};
		\node [style=dot] (15) at (4.5, 1.5) {};
	\end{pgfonlayer}
	\begin{pgfonlayer}{edgelayer}
		\draw (1) to (0);
		\draw (4.center) to (5.center);
		\draw (3) to (2);
		\draw (8.center) to (9.center);
		\draw (7) to (6);
		\draw (11) to (10);
		\draw (13) to (12);
		\draw (15) to (14);
	\end{pgfonlayer}
\end{tikzpicture}
=
\begin{tikzpicture}
	\begin{pgfonlayer}{nodelayer}
		\node [style=X] (0) at (4.5, 2.5) {};
		\node [style=X] (1) at (4.5, 0.5) {};
		\node [style=oplus] (2) at (4.5, 1) {};
		\node [style=dot] (3) at (3.5, 1) {};
		\node [style=none] (4) at (3.5, 2.75) {};
		\node [style=none] (5) at (3.5, 0.25) {};
		\node [style=none] (8) at (4, 2.75) {};
		\node [style=none] (9) at (4, 0.25) {};
		\node [style=oplus] (14) at (4, 1.5) {};
		\node [style=dot] (15) at (4.5, 1.5) {};
		\node [style=none] (16) at (4, 2.5) {};
	\end{pgfonlayer}
	\begin{pgfonlayer}{edgelayer}
		\draw (4.center) to (5.center);
		\draw (3) to (2);
		\draw (15) to (14);
		\draw (9.center) to (14);
		\draw (1) to (15);
		\draw [in=-90, out=90] (15) to (16.center);
		\draw [in=-90, out=90] (14) to (0);
		\draw (8.center) to (16.center);
	\end{pgfonlayer}
\end{tikzpicture}
=
\begin{tikzpicture}
	\begin{pgfonlayer}{nodelayer}
		\node [style=X] (0) at (4, 2) {};
		\node [style=X] (1) at (4.5, 0.5) {};
		\node [style=oplus] (2) at (4.5, 1) {};
		\node [style=dot] (3) at (3.5, 1) {};
		\node [style=none] (4) at (3.5, 2.75) {};
		\node [style=none] (5) at (3.5, 0.25) {};
		\node [style=none] (8) at (4, 2.75) {};
		\node [style=none] (9) at (4, 0.25) {};
		\node [style=oplus] (14) at (4, 1.5) {};
		\node [style=dot] (15) at (4.5, 1.5) {};
		\node [style=none] (16) at (4.5, 2) {};
	\end{pgfonlayer}
	\begin{pgfonlayer}{edgelayer}
		\draw (4.center) to (5.center);
		\draw (3) to (2);
		\draw (15) to (14);
		\draw (9.center) to (14);
		\draw (1) to (15);
		\draw [in=-90, out=90] (14) to (0);
		\draw (15) to (16.center);
		\draw [in=-90, out=90, looseness=0.75] (16.center) to (8.center);
	\end{pgfonlayer}
\end{tikzpicture}
=
\begin{tikzpicture}
	\begin{pgfonlayer}{nodelayer}
		\node [style=X] (17) at (6, 2.5) {};
		\node [style=X] (18) at (6.5, 1) {};
		\node [style=oplus] (19) at (6.5, 2) {};
		\node [style=dot] (20) at (5.5, 2) {};
		\node [style=none] (21) at (5.5, 3.25) {};
		\node [style=none] (22) at (5.5, 0.25) {};
		\node [style=none] (23) at (6, 3.25) {};
		\node [style=none] (24) at (6, 0.25) {};
		\node [style=oplus] (25) at (6, 1.5) {};
		\node [style=dot] (26) at (6.5, 1.5) {};
		\node [style=none] (27) at (6.5, 2.5) {};
		\node [style=oplus] (28) at (6, 1) {};
		\node [style=dot] (29) at (5.5, 1) {};
	\end{pgfonlayer}
	\begin{pgfonlayer}{edgelayer}
		\draw (21.center) to (22.center);
		\draw (20) to (19);
		\draw (26) to (25);
		\draw (24.center) to (25);
		\draw (18) to (26);
		\draw [in=-90, out=90] (25) to (17);
		\draw (26) to (27.center);
		\draw [in=-90, out=90, looseness=0.75] (27.center) to (23.center);
		\draw (29) to (28);
	\end{pgfonlayer}
\end{tikzpicture}
=
\begin{tikzpicture}
	\begin{pgfonlayer}{nodelayer}
		\node [style=X] (17) at (6, 2) {};
		\node [style=X] (18) at (6.5, 1) {};
		\node [style=oplus] (19) at (6.5, 1.5) {};
		\node [style=dot] (20) at (5.5, 1.5) {};
		\node [style=none] (21) at (5.5, 2.75) {};
		\node [style=none] (22) at (5.5, 0.25) {};
		\node [style=none] (23) at (6, 2.75) {};
		\node [style=none] (24) at (6, 0.25) {};
		\node [style=none] (27) at (6.5, 2) {};
		\node [style=oplus] (28) at (6, 1) {};
		\node [style=dot] (29) at (5.5, 1) {};
	\end{pgfonlayer}
	\begin{pgfonlayer}{edgelayer}
		\draw (21.center) to (22.center);
		\draw (20) to (19);
		\draw [in=-90, out=90, looseness=0.75] (27.center) to (23.center);
		\draw (29) to (28);
		\draw (24.center) to (28);
		\draw (28) to (17);
		\draw (27.center) to (18);
	\end{pgfonlayer}
\end{tikzpicture}
=
\begin{tikzpicture}
	\begin{pgfonlayer}{nodelayer}
		\node [style=X] (17) at (6, 1.5) {};
		\node [style=X] (18) at (6, 2) {};
		\node [style=oplus] (19) at (6, 2.5) {};
		\node [style=dot] (20) at (5.5, 2.5) {};
		\node [style=none] (21) at (5.5, 3) {};
		\node [style=none] (22) at (5.5, 0.5) {};
		\node [style=none] (24) at (6, 0.5) {};
		\node [style=none] (27) at (6, 3) {};
		\node [style=oplus] (28) at (6, 1) {};
		\node [style=dot] (29) at (5.5, 1) {};
	\end{pgfonlayer}
	\begin{pgfonlayer}{edgelayer}
		\draw (21.center) to (22.center);
		\draw (20) to (19);
		\draw (29) to (28);
		\draw (24.center) to (28);
		\draw (28) to (17);
		\draw (27.center) to (18);
	\end{pgfonlayer}
\end{tikzpicture}
$$


The inductive case is essentially the same as the base case for 2.
%
%Note that the distributive law is actually a partially reversible formulation of the bialgebra law in disguise since:
%
%
%$$
%\left\llbracket
%\begin{tikzpicture}
%	\begin{pgfonlayer}{nodelayer}
%		\node [style=X] (0) at (1, -0.25) {};
%		\node [style=oplus] (1) at (1, -0.75) {};
%		\node [style=oplus] (2) at (1, -1.5) {};
%		\node [style=dot] (3) at (0.5, -0.75) {};
%		\node [style=dot] (4) at (0, -1.5) {};
%		\node [style=none] (5) at (0.5, 0) {};
%		\node [style=none] (6) at (0, 0) {};
%		\node [style=none] (7) at (0.25, -1) {$\iddots$};
%		\node [style=none] (8) at (1, -1) {$\vdots$};
%		\node [style=X] (9) at (1, -2) {};
%		\node [style=none] (18) at (0, -2.25) {};
%		\node [style=none] (19) at (0.5, -2.25) {};
%		\node [style=none] (22) at (0.25, -0.25) {$\cdots$};
%	\end{pgfonlayer}
%	\begin{pgfonlayer}{edgelayer}
%		\draw (0) to (1);
%		\draw (1) to (3);
%		\draw (5.center) to (3);
%		\draw (6.center) to (4);
%		\draw (4) to (2);
%		\draw (4) to (18.center);
%		\draw (19.center) to (3);
%		\draw (9) to (2);
%	\end{pgfonlayer}
%\end{tikzpicture}
%\right\rrbracket
%=
%\begin{tikzpicture}
%	\begin{pgfonlayer}{nodelayer}
%		\node [style=none] (0) at (9.25, 0.5) {};
%		\node [style=none] (1) at (8.75, 0.5) {};
%		\node [style=none] (2) at (8.75, -1.75) {};
%		\node [style=none] (3) at (9.25, -1.75) {};
%		\node [style=Z] (4) at (9.25, 0) {};
%		\node [style=Z] (5) at (8.75, -0.75) {};
%		\node [style=none] (6) at (9.5, -1.25) {};
%		\node [style=none] (7) at (10.5, -1.25) {};
%		\node [style=none] (8) at (10.5, 0.25) {};
%		\node [style=none] (9) at (9.5, 0.25) {};
%		\node [style=none] (10) at (10.25, -1) {$!$};
%		\node [style=X] (11) at (10, 0) {};
%		\node [style=X] (12) at (10, -0.75) {};
%	\end{pgfonlayer}
%	\begin{pgfonlayer}{edgelayer}
%		\draw (1.center) to (5);
%		\draw (5) to (2.center);
%		\draw (3.center) to (4);
%		\draw (4) to (0.center);
%		\draw [style=dotted] (6.center) to (7.center);
%		\draw [style=dotted] (7.center) to (8.center);
%		\draw [style=dotted] (8.center) to (9.center);
%		\draw [style=dotted] (9.center) to (6.center);
%		\draw (5) to (12);
%		\draw (12) to (11);
%		\draw (11) to (4);
%	\end{pgfonlayer}
%\end{tikzpicture}
%=
%\begin{tikzpicture}
%	\begin{pgfonlayer}{nodelayer}
%		\node [style=none] (12) at (9.25, 1) {};
%		\node [style=none] (13) at (8.75, 1) {};
%		\node [style=none] (15) at (8.75, -1.75) {};
%		\node [style=none] (16) at (9.25, -1.75) {};
%		\node [style=Z] (18) at (8.75, -1.25) {};
%		\node [style=none] (19) at (9.5, -1.25) {};
%		\node [style=none] (20) at (10.5, -1.25) {};
%		\node [style=none] (21) at (10.5, 0.75) {};
%		\node [style=none] (22) at (9.5, 0.75) {};
%		\node [style=none] (23) at (10.25, -1) {$!$};
%		\node [style=X] (25) at (10, -0.75) {};
%		\node [style=Z] (28) at (10, 0.25) {};
%		\node [style=X] (29) at (9.25, 0.5) {};
%		\node [style=X] (30) at (9.25, -0.25) {};
%	\end{pgfonlayer}
%	\begin{pgfonlayer}{edgelayer}
%		\draw (13.center) to (18);
%		\draw (18) to (15.center);
%		\draw [style=dotted] (19.center) to (20.center);
%		\draw [style=dotted] (20.center) to (21.center);
%		\draw [style=dotted] (21.center) to (22.center);
%		\draw [style=dotted] (22.center) to (19.center);
%		\draw (18) to (25);
%		\draw (12.center) to (29);
%		\draw (29) to (28);
%		\draw (28) to (30);
%		\draw (25) to (28);
%		\draw (16.center) to (30);
%	\end{pgfonlayer}
%\end{tikzpicture}
%=
%\begin{tikzpicture}
%	\begin{pgfonlayer}{nodelayer}
%		\node [style=none] (0) at (9.25, 1.25) {};
%		\node [style=none] (1) at (8.75, 1.25) {};
%		\node [style=none] (2) at (8.75, -1.75) {};
%		\node [style=none] (3) at (9.25, -1.75) {};
%		\node [style=none] (5) at (9.5, -1.25) {};
%		\node [style=none] (6) at (10.5, -1.25) {};
%		\node [style=none] (7) at (10.5, 1) {};
%		\node [style=none] (8) at (9.5, 1) {};
%		\node [style=none] (9) at (10.25, -1) {$!$};
%		\node [style=Z] (11) at (10, 0.5) {};
%		\node [style=X] (12) at (9.25, 0.75) {};
%		\node [style=X] (13) at (9.25, 0) {};
%		\node [style=Z] (14) at (10, -0.75) {};
%		\node [style=X] (15) at (8.75, -0.25) {};
%		\node [style=X] (16) at (8.75, -1.25) {};
%	\end{pgfonlayer}
%	\begin{pgfonlayer}{edgelayer}
%		\draw [style=dotted] (5.center) to (6.center);
%		\draw [style=dotted] (6.center) to (7.center);
%		\draw [style=dotted] (7.center) to (8.center);
%		\draw [style=dotted] (8.center) to (5.center);
%		\draw (0.center) to (12);
%		\draw (12) to (11);
%		\draw (11) to (13);
%		\draw (3.center) to (13);
%		\draw (16) to (14);
%		\draw (14) to (11);
%		\draw (14) to (15);
%		\draw (15) to (1.center);
%		\draw (16) to (2.center);
%	\end{pgfonlayer}
%\end{tikzpicture}
%=
%\begin{tikzpicture}
%	\begin{pgfonlayer}{nodelayer}
%		\node [style=none] (0) at (9.25, 0.75) {};
%		\node [style=none] (1) at (8.75, 0.75) {};
%		\node [style=none] (2) at (8.75, -2) {};
%		\node [style=none] (3) at (9.25, -2) {};
%		\node [style=none] (4) at (9.5, -1.75) {};
%		\node [style=none] (5) at (10.5, -1.75) {};
%		\node [style=none] (6) at (10.5, 0.5) {};
%		\node [style=none] (7) at (9.5, 0.5) {};
%		\node [style=none] (8) at (10.25, -1.5) {$!$};
%		\node [style=Z] (9) at (10, 0.25) {};
%		\node [style=X] (10) at (9.25, 0.25) {};
%		\node [style=X] (11) at (9.25, -0.25) {};
%		\node [style=Z] (12) at (10, -0.75) {};
%		\node [style=X] (13) at (8.75, -0.75) {};
%		\node [style=X] (14) at (8.75, -1.25) {};
%		\node [style=Z] (15) at (10, -0.25) {};
%		\node [style=Z] (16) at (10, -1.25) {};
%	\end{pgfonlayer}
%	\begin{pgfonlayer}{edgelayer}
%		\draw [style=dotted] (4.center) to (5.center);
%		\draw [style=dotted] (5.center) to (6.center);
%		\draw [style=dotted] (6.center) to (7.center);
%		\draw [style=dotted] (7.center) to (4.center);
%		\draw (0.center) to (10);
%		\draw (10) to (9);
%		\draw (3.center) to (11);
%		\draw (13) to (12);
%		\draw (13) to (1.center);
%		\draw (14) to (2.center);
%		\draw (9) to (12);
%		\draw (16) to (14);
%		\draw (11) to (15);
%		\draw (16) to (12);
%	\end{pgfonlayer}
%\end{tikzpicture}
%=
%\left\llbracket
%\begin{tikzpicture}
%	\begin{pgfonlayer}{nodelayer}
%		\node [style=none] (0) at (7, 2.25) {};
%		\node [style=none] (1) at (6.5, 2.25) {};
%		\node [style=none] (2) at (6.75, -.2) {$\iddots$};
%		\node [style=none] (3) at (6.5, -3) {};
%		\node [style=none] (4) at (7, -3) {};
%		\node [style=oplus] (6) at (6.5, -2.25) {};
%		\node [style=X] (7) at (6.5, -1.75) {};
%		\node [style=X] (8) at (6.5, -1.25) {};
%		\node [style=dot] (9) at (7.5, -2.25) {};
%		\node [style=dot] (10) at (7.5, -0.75) {};
%		\node [style=Z] (12) at (7.5, -2.75) {};
%		\node [style=oplus] (13) at (7, 0) {};
%		\node [style=X] (14) at (7, 0.5) {};
%		\node [style=X] (15) at (7, 1) {};
%		\node [style=dot] (16) at (7.5, 0) {};
%		\node [style=dot] (17) at (7.5, 1.5) {};
%		\node [style=none] (18) at (7.5, -0.25) {$\vdots$};
%		\node [style=Z] (19) at (7.5, 2) {};
%		\node [style=oplus] (20) at (7, 1.5) {};
%		\node [style=oplus] (21) at (6.5, -0.75) {};
%	\end{pgfonlayer}
%	\begin{pgfonlayer}{edgelayer}
%		\draw (3.center) to (6);
%		\draw (6) to (7);
%		\draw (6) to (9);
%		\draw (12) to (9);
%		\draw (9) to (10);
%		\draw (13) to (14);
%		\draw (13) to (16);
%		\draw (16) to (17);
%		\draw (4.center) to (13);
%		\draw (19) to (17);
%		\draw (17) to (20);
%		\draw (20) to (15);
%		\draw (0.center) to (20);
%		\draw (10) to (21);
%		\draw (21) to (8);
%		\draw (21) to (1.center);
%	\end{pgfonlayer}
%\end{tikzpicture}
%\right\rrbracket
%=
%\left\llbracket
%\begin{tikzpicture}
%	\begin{pgfonlayer}{nodelayer}
%		\node [style=none] (18) at (7.5, 1.25) {};
%		\node [style=none] (19) at (6.5, 0.5) {};
%		\node [style=none] (20) at (6.7, -0.25) {$\iddots$};
%		\node [style=none] (23) at (7, -2) {};
%		\node [style=none] (24) at (7, -1.25) {};
%		\node [style=X] (30) at (6.5, -1.25) {};
%		\node [style=dot] (32) at (7.5, -0.75) {};
%		\node [style=oplus] (35) at (7, 0) {};
%		\node [style=X] (36) at (7, 0.5) {};
%		\node [style=dot] (38) at (7.5, 0) {};
%		\node [style=none] (41) at (7.5, -0.25) {$\vdots$};
%		\node [style=none] (42) at (7.5, -1.25) {};
%		\node [style=oplus] (43) at (6.5, -0.75) {};
%		\node [style=none] (44) at (7.5, -2) {};
%		\node [style=none] (45) at (7, 1.25) {};
%	\end{pgfonlayer}
%	\begin{pgfonlayer}{edgelayer}
%		\draw (35) to (36);
%		\draw (35) to (38);
%		\draw [in=270, out=90] (24.center) to (35);
%		\draw (38) to (18.center);
%		\draw [in=-90, out=90, looseness=0.75] (23.center) to (42.center);
%		\draw (42.center) to (32);
%		\draw (32) to (43);
%		\draw (43) to (19.center);
%		\draw (43) to (30);
%		\draw [in=-90, out=90] (44.center) to (24.center);
%		\draw [in=-90, out=90] (19.center) to (45.center);
%	\end{pgfonlayer}
%\end{tikzpicture}
%\right\rrbracket
%=
%\left\llbracket
%\begin{tikzpicture}
%	\begin{pgfonlayer}{nodelayer}
%		\node [style=none] (15) at (9.5, 1.5) {};
%		\node [style=none] (19) at (10, -1) {};
%		\node [style=X] (20) at (9, 0.5) {};
%		\node [style=dot] (21) at (9.5, 1) {};
%		\node [style=oplus] (22) at (10, -0.5) {};
%		\node [style=X] (23) at (10, 0) {};
%		\node [style=dot] (24) at (9.5, -0.5) {};
%		\node [style=none] (26) at (9.5, -1) {};
%		\node [style=oplus] (27) at (9, 1) {};
%		\node [style=none] (29) at (9, 1.5) {};
%		\node [style=none] (30) at (9.5, 0.25) {$\ddots$};
%	\end{pgfonlayer}
%	\begin{pgfonlayer}{edgelayer}
%		\draw (22) to (23);
%		\draw (22) to (24);
%		\draw [in=270, out=90] (19.center) to (22);
%		\draw (21) to (27);
%		\draw (27) to (20);
%		\draw (21) to (15.center);
%		\draw (24) to (26.center);
%		\draw (29.center) to (27);
%		\draw (24) to (21);
%	\end{pgfonlayer}
%\end{tikzpicture}
%\right\rrbracket
%$$
%\end{definition}

%Notice that the choice of which wires to straighten out the zig-zag is arbitrary.

\end{remark}


\begin{lemma}
\label{lem:parisocb}
$\Par\Iso(\cb_2)$ is a presentation for the prop $(\Par\Iso(\Mat(\F_2),+))$.
\end{lemma}
%This follows from \cite[??]{ih}.


We can get partial maps by freely adding a counit to the nonunital, noncounital special commutative Frobenius algebra:

\begin{definition}

Let $\Par(\cb_2)$ denote the pushout of the diagram of props:
$$
\Par\Iso(\cb_2)  \leftarrow  \surj^\op \rightarrow   \cm^\op
$$

\end{definition}



%The following Lemma follows after meticulous calculation and the application of \cite[Lem. 3.5]{zxa}:

\begin{lemma}
\label{lem:parcb}

$\Par(\cb_2)$ is a presentation for the prop $(\Par(\Mat(\F_2),+))$.
\end{lemma}



\begin{proof}
One must show that the following diagram commutes:

%
%\renewcommand{\cubetopbl}{$\inj(\cb_2)$}
%\renewcommand{\cubetopbr}{$\inj(\cb_2)^\op \otimes_{\Iso(\cb_2)} \inj(\cb_2)$}
%\renewcommand{\cubetopfl}{$\inj(\cb_2)\otimes_{\Iso(\cb_2)} \surj(\cb_2)^\op$}
%\renewcommand{\cubetopfr}{$\Par(\cb_2)$}\ParIso(\cb_2)
%\renewcommand{\cubebotbl}{$(\inj(\Mat(\F_2)), +)$}
%\renewcommand{\cubebotbr}{$(\Par\Iso(\Mat(\F_2)), +)$}
%\renewcommand{\cubebotfl}{$(\Par\surj(\Mat(\F_2)), +)$}
%\renewcommand{\cubebotfr}{}
%
%$$
%\xymatrixrowsep{3mm}\xymatrixcolsep{1mm}
%\xymatrix{
%                                       & \mbox{\cubetopbl} \ar[rr] \ar[dl] \ar[dd]^(.7){\cong}      &                                                  & \mbox{\cubetopbr}  \ar[dd]^{\cong} \ar[dl] \\
%\mbox{\cubetopfl} \ar[rr]  \ar[dd]_{\cong}           &                                                                                              &\mbox{\cubetopfr} \ar@{-->}[dd]^(.35){\cong}   \skewpullbackcorner[ul]              \\
%                                       &  \mbox{\cubebotbl} \ar[dl] \ar[rr]                    &                                                  & \mbox{\cubebotbr} \ar@/^1pc/[ddl] \ar[dl] \\
%\mbox{\cubebotfl} \ar@/_1pc/[drr] \ar[rr]  &                                                                                             & \mbox{\cubebotfr} \skewpullbackcorner[ul]    \ar@{-->}[d]^{\cong}  \\
%                                                   &                                                                                             & (\Par(\Mat(\F_2)),+)
%}
%$$
%

\renewcommand{\cubetopbl}{$\surj^\op$}
\renewcommand{\cubetopbr}{$\cm^\op$}
\renewcommand{\cubetopfl}{$\ParIso(\cb_2)$}
\renewcommand{\cubetopfr}{$\Par(\cb_2)$}
\renewcommand{\cubebotbl}{$\surj^\op$ }
\renewcommand{\cubebotbr}{$\cm^\op$ }
\renewcommand{\cubebotfl}{$\ParIso(\Mat(\F_2)),+)$ }
\renewcommand{\cubebotfr}{}

$$
\xymatrixrowsep{2mm}\xymatrixcolsep{2mm}
\xymatrix{
                                       & \mbox{\cubetopbl} \ar[rr] \ar[dl] \ar@{=}[dd]     &                                                  & \mbox{\cubetopbr} \ar@{=}[dd] \ar[dl] \\
\mbox{\cubetopfl} \ar[rr]  \ar[dd]_{\cong}           &                                                                                              &\mbox{\cubetopfr} \ar@{-->}[dd]^(.35){\cong}   \skewpullbackcorner[ul]              \\
                                       &  \mbox{\cubebotbl} \ar[dl] \ar[rr]                    &                                                  & \mbox{\cubebotbr} \ar@/^1pc/[ddl] \ar[dl] \\
\mbox{\cubebotfl} \ar@/_1pc/[drr] \ar[rr]  &                                                                                             & \mbox{\cubebotfr} \skewpullbackcorner[ul]    \ar@{-->}[d]^{\cong}  \\
                                                   &                                                                                             & (\Par(\Mat(\F_2)),+)
}
$$

It doesn't take to much work to show that $\ParIso(\cb_2)\cong\ParIso(\Mat(\F_2))$ is a discrete inverse category (defined in \cite[\S 4.3]{giles}).
We know that the counital completion of a discrete inverse category is the same as its Cartesian completion from \cite[Lem. 3.5]{zxa}; moreover, the Cartesian completion of  $\ParIso(\Mat(\F_2))$ is $\Par(\Mat(\F_2))$.  So this diagram commutes as a consequence.

\end{proof}



%\begin{comment}
This props has a particularly elegant presentation which is given in \S \ref{subsubsec:presentations:one:par}.
%\end{comment}



%
%\begin{corollary}
%Equivalently, $(\Par(\Mat(\F_2),+))$ is presented by the coproduct
%$ T_{\eta_X} + \cb_2 $
%modulo:
%
%?????
%
%\end{corollary}


%
%For $R$ a PID, $(\Span(\Mat(R)),+)$ TODO
%
%
%
%
%For $R$ a PID, $(\Rel(\Mat(R)),+)$ TODO
%
%
%  Kernel, Image, orthogonal complement
%  Pushout of TODO
%
%
%Because are self-dual with respect to the transpose functor, we omit the discussion of cospans and corelations.
%


%I will omit the discussion of spans and relations; however, the details are contained in \cite{ih}.




\begin{definition}
Let $\Span(\cb_2)$ denote the pushout of the diagram of props:
$$
\Par(\cb_2)^\op \leftarrow  \ParIso(\cb_2) \rightarrow \Par(\cb_2)
$$
\end{definition}

The following lemma holds because of \cite[Lem. 4.3]{zxa}:


\begin{lemma}
\label{lem:spancb}

$\Span(\cb_2)$ is a presentation for the prop $(\Span(\Mat(\F_2)), +)$.
\end{lemma}


\begin{proof}


\renewcommand{\cubetopbl}{$\inj(\cb_2)^\op \otimes_{\Iso(\cb_2)} \inj(\cb_2)$}
\renewcommand{\cubetopbr}{$\Par(\cb_2)$}
\renewcommand{\cubetopfl}{$\Par(\cb_2)^\op$}
\renewcommand{\cubetopfr}{$\Span(\cb_2)$}
\renewcommand{\cubebotbl}{$(\Par\Iso(\Mat(\F_2)),+)$ }
\renewcommand{\cubebotbr}{$(\Par(\Mat(\F_2)),+)$ }
\renewcommand{\cubebotfl}{$(\Par(\Mat(\F_2)),+)^\op$ }
\renewcommand{\cubebotfr}{}

$$
\xymatrixrowsep{2mm}\xymatrixcolsep{1mm}
\xymatrix{
                                       & \mbox{\cubetopbl} \ar[rr] \ar[dl] \ar[dd]^(.7){\cong}      &                                                  & \mbox{\cubetopbr}  \ar[dd]^{\cong} \ar[dl] \\
\mbox{\cubetopfl} \ar[rr]  \ar[dd]_{\cong}           &                                                                                              &\mbox{\cubetopfr} \ar@{-->}[dd]^(.35){\cong}   \skewpullbackcorner[ul]              \\
                                       &  \mbox{\cubebotbl} \ar[dl] \ar[rr]                    &                                                  & \mbox{\cubebotbr} \ar@/^1pc/[ddl] \ar[dl] \\
\mbox{\cubebotfl} \ar@/_1pc/[drr] \ar[rr]  &                                                                                             & \mbox{\cubebotfr} \skewpullbackcorner[ul]    \ar@{-->}[d]^{\cong}_F \\
                                                   &                                                                                             & (\Span(\Mat(\F_2)),+)
}
$$


The cube easily commutes.  What remains to be shown is that the universal map $F$ is an isomorphism of props.  It is clearly the identity on objects, so we just need to show it is full and faithful.

It is clearly full as any span $ n \xleftarrow{ f}  k \xrightarrow{g } m$, we have:
$$
F\left( (n \xleftarrow{f} k = k);(k = k \xrightarrow{g} m) \right)=n \xleftarrow{ f}  k \xrightarrow{g } m
$$ 
For faithfulness, we must observe given for any two isomorphic maps in $\Span(\Mat(\F_2))$:
$$
\xymatrixrowsep{2mm}\xymatrixcolsep{6mm}
\xymatrix{
          & k \ar[dl]_{f'} \ar[dd]_{\cong}^{h} \ar[dr]^{g'}\\
n  &                                                                                                    & m\\
         & k \ar[ul]^{f} \ar[ur]_{g}\\
}
$$
Then in the domain of $F$, we have:
{
\xymatrixrowsep{0mm}\xymatrixcolsep{1.7mm}
\begin{align*}
&
\xymatrix{
   & k \ar[dl]_f \ar@{=}[dr]\\
n &                                      &k
};
\xymatrix{
   & k \ar[dr]^g \ar@{=}[dl]\\
k &                                      &m
}
%&
 =
\xymatrix{
   & k \ar[dl]_f \ar@{=}[dr]\\
n &                                      &k
};
\xymatrix{
   & k \ar@{=}[dl] \ar@{=}[dr]\\
k &                                             & k\\
   & k \ar[ul]^h \ar[ur]_h \ar[uu]^\cong_h
};
\xymatrix{
   & k \ar[dr]^g \ar@{=}[dl]\\
k &                                      &m
}\\
 &=
\xymatrix{
   & k \ar[dl]_f \ar@{=}[dr]\\
n &                                      &k
};
\xymatrix{
   & k \ar[dl]_h \ar@{=}[dr]\\
k &                                         & k
};
\xymatrix{
   & k \ar[dr]^h \ar@{=}[dl]\\
k &                                         & k
};
\xymatrix{
   & k \ar[dr]^g \ar@{=}[dl]\\
k &                                      &m
}
%\\&
=
\xymatrix{
            &                                                        &k \ar[dl]_{h} \ar@{=}[dr] \ar@/_1.2pc/[ddll]_{f'}\\
            & k \ar@{=}[dr] \ar[dl]^{f}&                                                          & k \ar@{=}[dr] \ar[dl]_{h}\\
n &                                                         & k                                             &                                                         &k
};
\xymatrix{
            &                                                        & k \ar[dr]^{h} \ar@{=}[dl] \ar@/^1.2pc/[ddrr]^{g'}  \\
            & k \ar[dr]^{h}   \ar@{=}[dl] &                                                          & k \ar@{=}[dl] \ar[dr]_{g}\\
k &                                                         & k                                             &                                                         &m
}
\end{align*}
}
 
\end{proof}


Given a PID $k$, the prop $(\Span(\Mat(k)), +)$ is already known to have a presentation given in terms of ``interacting Hopf algebras" \cite[Definition 3.13]{ih}.  This is also the way in which the phase-free fragment of the ZX-calculus would be presented, in terms of two Frobenius algebras  corresponding to the $Z$ and $X$ observables, interacting to form Hopf algebras in addition to a few more equations.
%\begin{comment}
 We have included this presentation in \S \ref{subsubsec:presentations:one:span}.
%\end{comment}



%
%\begin{corollary}
%Because  prop $(\Mat(\F_2),+)$ is self dual, the prop $\Csp(\cb_2)$ obtained by swapping colours is a presentation for the prop $(\Csp(\Mat(\F_2)),+)$
%\end{corollary}
%
%
%
%
%\begin{definition} 
%
%Consider the prop $\Rel(\cb_2)$ given by the following pushout of the following diagram of props:
%
%$$
%\Csp(\cb_2) \leftarrow \inj(\cb_2)^\op + \inj(\cb_2)  \rightarrow \ParIso(\cb_2)
%$$
%
%\end{definition}
%
%This particular construction of the following presentation is due to \cite[Thm. 3.6]{universal}; however, it was constructed in a slightly different manner before in \cite[Thm. 3.49]{ih}:
%
%\begin{lemma}
%$\Rel(\cb_2)$ is a presentation for the prop $(\Rel(\Mat(\F_2)), +)$
%\end{lemma}
%
%
%
%\begin{proof}
%
%
%
%\renewcommand{\cubetopbl}{$\inj(\cb_2)^\op + \inj(\cb_2)$}
%\renewcommand{\cubetopbr}{$\inj(\cb_2)\otimes_{\Iso(\cb_2)} \inj(\cb_2)^\op$}
%\renewcommand{\cubetopfl}{$\cb_2^\op \otimes_{\Iso(\cb_2)}  \cb_2 $}
%\renewcommand{\cubetopfr}{$\Rel(\cb_2)$}
%\renewcommand{\cubebotbl}{$(\inj(\Mat(\F_2)),+)^\op+(\inj(\Mat(\F_2)),+)$ }
%\renewcommand{\cubebotbr}{$(\Par\Iso(\Mat(\F_2)),+)$ }
%\renewcommand{\cubebotfl}{$(\Csp(\Mat(\F_2)),+)$ }
%\renewcommand{\cubebotfr}{$$}
%
%$$
%\xymatrixrowsep{3mm}\xymatrixcolsep{1mm}
%\xymatrix{
%                                       & \mbox{\cubetopbl} \ar[rr] \ar[dl] \ar[dd]^(.7){\cong}      &                                                  & \mbox{\cubetopbr}  \ar[dd]^{\cong} \ar[dl] \\
%\mbox{\cubetopfl} \ar[rr]  \ar[dd]_{\cong}           &                                                                                              &\mbox{\cubetopfr} \ar@{-->}[dd]^(.35){\cong}   \skewpullbackcorner[ul]              \\
%                                       &  \mbox{\cubebotbl} \ar[dl] \ar[rr]                    &                                                  & \mbox{\cubebotbr} \ar@/^1pc/[ddl] \ar[dl] \\
%\mbox{\cubebotfl} \ar@/_1pc/[drr] \ar[rr]  &                                                                                             & \mbox{\cubebotfr} \skewpullbackcorner[ul]    \ar@{-->}[d]^{\cong}  \\
%                                                   &                                                                                             & (\Rel(\Mat(\F_2)),+)
%}
%$$
%
%
%\end{proof}
%
%


\section{Additive affine models}
\label{sec:two}


\begin{definition}
Consider the prop  $\Aff\cb_2$ given by adjoining the following generator to $\cb_2$
\hfil
$
\begin{tikzpicture}
	\begin{pgfonlayer}{nodelayer}
		\node [style=none] (0) at (-3.75, -0.25) {};
		\node [style=X] (1) at (-3.75, -1) {$1$};
	\end{pgfonlayer}
	\begin{pgfonlayer}{edgelayer}
		\draw (0.center) to (1);
	\end{pgfonlayer}
\end{tikzpicture}
$

modulo the equations:\hspace*{2.2cm}
$
\begin{tikzpicture}
	\begin{pgfonlayer}{nodelayer}
		\node [style=X] (0) at (0.75, 0.25) {$1$};
		\node [style=Z] (1) at (0.75, 0.75) {};
		\node [style=none] (2) at (0.5, 1.5) {};
		\node [style=none] (3) at (1, 1.5) {};
	\end{pgfonlayer}
	\begin{pgfonlayer}{edgelayer}
		\draw (1) to (0);
		\draw [in=-90, out=60] (1) to (3.center);
		\draw [in=120, out=-90] (2.center) to (1);
	\end{pgfonlayer}
\end{tikzpicture}
\erefop{bi.two}
\begin{tikzpicture}
	\begin{pgfonlayer}{nodelayer}
		\node [style=X] (0) at (0.5, 0.5) {$1$};
		\node [style=none] (1) at (0.5, 1.75) {};
		\node [style=none] (2) at (1, 1.75) {};
		\node [style=X] (3) at (1, 0.5) {$1$};
	\end{pgfonlayer}
	\begin{pgfonlayer}{edgelayer}
		\draw (0) to (1.center);
		\draw (2.center) to (3);
	\end{pgfonlayer}
\end{tikzpicture}
\hspace*{1cm}
\begin{tikzpicture}
	\begin{pgfonlayer}{nodelayer}
		\node [style=X] (0) at (0, 0) {$1$};
		\node [style=Z] (1) at (0, 0.75) {};
	\end{pgfonlayer}
	\begin{pgfonlayer}{edgelayer}
		\draw (1) to (0);
	\end{pgfonlayer}
\end{tikzpicture}
\eref{extra}
$



\end{definition}



\begin{lemma} \cite[\S 4]{lafont}
 $\Aff\cb_2$ is a presentation for the prop $(\Aff\Mat(\F_2),+)$.
\end{lemma}


Note that this assumes that affine matrices are non-empty, as this is a prop.  This will become a problem later, when we wish to pull back affine spaces.






\begin{definition}
Consider the prop $\Iso(\Aff\cb_2)$ generated by the controlled not gate, and the not gate (interpreted as matrices):
\hspace*{.2cm}
$
\left\llbracket
\begin{tikzpicture}
	\begin{pgfonlayer}{nodelayer}
		\node [style=oplus] (5) at (0.5, 2.75) {};
		\node [style=dot] (6) at (0, 2.75) {};
		\node [style=none] (7) at (0.5, 3.5) {};
		\node [style=none] (8) at (0.5, 2) {};
		\node [style=none] (9) at (0, 2) {};
		\node [style=none] (10) at (0, 3.5) {};
	\end{pgfonlayer}
	\begin{pgfonlayer}{edgelayer}
		\draw (8.center) to (5);
		\draw (5) to (7.center);
		\draw (10.center) to (6);
		\draw (6) to (5);
		\draw (6) to (9.center);
	\end{pgfonlayer}
\end{tikzpicture}
\right\rrbracket
=
\begin{tikzpicture}
	\begin{pgfonlayer}{nodelayer}
		\node [style=X] (0) at (-0.25, -1) {};
		\node [style=Z] (1) at (-0.75, -1.75) {};
		\node [style=none] (2) at (0, -2.25) {};
		\node [style=none] (3) at (-0.25, -0.5) {};
		\node [style=none] (4) at (-1, -0.5) {};
		\node [style=none] (5) at (-0.75, -2.25) {};
	\end{pgfonlayer}
	\begin{pgfonlayer}{edgelayer}
		\draw (3.center) to (0);
		\draw [in=90, out=-75] (0) to (2.center);
		\draw (5.center) to (1);
		\draw (1) to (0);
		\draw [in=-90, out=105] (1) to (4.center);
	\end{pgfonlayer}
\end{tikzpicture}
\hspace*{1cm}
\left\llbracket
\begin{tikzpicture}
	\begin{pgfonlayer}{nodelayer}
		\node [style=none] (0) at (0, -0.5) {};
		\node [style=none] (1) at (0, -1.5) {};
		\node [style=oplus] (2) at (0, -1) {};
	\end{pgfonlayer}
	\begin{pgfonlayer}{edgelayer}
		\draw (1.center) to (2);
		\draw (2) to (0.center);
	\end{pgfonlayer}
\end{tikzpicture}
\right\rrbracket
=
\begin{tikzpicture}
	\begin{pgfonlayer}{nodelayer}
		\node [style=none] (0) at (1, 1) {};
		\node [style=none] (1) at (0.75, 2) {};
		\node [style=X] (2) at (0.75, 1.5) {};
		\node [style=X] (3) at (0.5, 1) {$1$};
		\node [style=none] (4) at (1, 0.5) {};
	\end{pgfonlayer}
	\begin{pgfonlayer}{edgelayer}
		\draw (1.center) to (2);
		\draw [in=90, out=-45, looseness=0.75] (2) to (0.center);
		\draw [in=90, out=-135, looseness=0.75] (2) to (3);
		\draw (0.center) to (4.center);
	\end{pgfonlayer}
\end{tikzpicture}
$

Modulo the relations of $\Iso(\cb_2)$ as well as the additional relations:
$$
\begin{tikzpicture}
	\begin{pgfonlayer}{nodelayer}
		\node [style=oplus] (0) at (0, 0) {};
		\node [style=oplus] (1) at (0, 0.5) {};
		\node [style=none] (2) at (0, 1) {};
		\node [style=none] (3) at (0, -0.5) {};
	\end{pgfonlayer}
	\begin{pgfonlayer}{edgelayer}
		\draw (2.center) to (3.center);
	\end{pgfonlayer}
\end{tikzpicture}
\eqzxa{cnot.seven}
\begin{tikzpicture}
	\begin{pgfonlayer}{nodelayer}
		\node [style=none] (2) at (0, 1) {};
		\node [style=none] (3) at (0, -0.5) {};
	\end{pgfonlayer}
	\begin{pgfonlayer}{edgelayer}
		\draw (2.center) to (3.center);
	\end{pgfonlayer}
\end{tikzpicture}
\hspace*{.5cm}
\begin{tikzpicture}
	\begin{pgfonlayer}{nodelayer}
		\node [style=dot] (0) at (2, 2) {};
		\node [style=oplus] (1) at (2.5, 2) {};
		\node [style=oplus] (2) at (2, 1.5) {};
		\node [style=none] (3) at (2, 1) {};
		\node [style=none] (4) at (2.5, 1) {};
		\node [style=none] (5) at (2, 2.5) {};
		\node [style=none] (6) at (2.5, 2.5) {};
	\end{pgfonlayer}
	\begin{pgfonlayer}{edgelayer}
		\draw (3.center) to (2);
		\draw (2) to (0);
		\draw (0) to (5.center);
		\draw (6.center) to (1);
		\draw (1) to (0);
		\draw (4.center) to (1);
	\end{pgfonlayer}
\end{tikzpicture}
\eqzxa{cnot.eight}
\begin{tikzpicture}
	\begin{pgfonlayer}{nodelayer}
		\node [style=dot] (0) at (2, 1.5) {};
		\node [style=oplus] (1) at (2.5, 1.5) {};
		\node [style=none] (3) at (2, 1) {};
		\node [style=none] (4) at (2.5, 1) {};
		\node [style=none] (5) at (2, 2.5) {};
		\node [style=none] (6) at (2.5, 2.5) {};
		\node [style=oplus] (7) at (2, 2) {};
		\node [style=oplus] (8) at (2.5, 2) {};
	\end{pgfonlayer}
	\begin{pgfonlayer}{edgelayer}
		\draw (0) to (5.center);
		\draw (6.center) to (1);
		\draw (1) to (0);
		\draw (4.center) to (1);
		\draw (3.center) to (0);
	\end{pgfonlayer}
\end{tikzpicture}
\hspace*{.5cm}
\begin{tikzpicture}
	\begin{pgfonlayer}{nodelayer}
		\node [style=dot] (0) at (2, 2) {};
		\node [style=oplus] (1) at (2.5, 2) {};
		\node [style=none] (3) at (2, 1) {};
		\node [style=none] (4) at (2.5, 1) {};
		\node [style=none] (5) at (2, 2.5) {};
		\node [style=none] (6) at (2.5, 2.5) {};
		\node [style=oplus] (8) at (2.5, 1.5) {};
	\end{pgfonlayer}
	\begin{pgfonlayer}{edgelayer}
		\draw (0) to (5.center);
		\draw (6.center) to (1);
		\draw (1) to (0);
		\draw (4.center) to (1);
		\draw (3.center) to (0);
	\end{pgfonlayer}
\end{tikzpicture}
\eqzxa{cnot.nine}
\begin{tikzpicture}
	\begin{pgfonlayer}{nodelayer}
		\node [style=dot] (0) at (2, 1.5) {};
		\node [style=oplus] (1) at (2.5, 1.5) {};
		\node [style=none] (3) at (2, 1) {};
		\node [style=none] (4) at (2.5, 1) {};
		\node [style=none] (5) at (2, 2.5) {};
		\node [style=none] (6) at (2.5, 2.5) {};
		\node [style=oplus] (8) at (2.5, 2) {};
	\end{pgfonlayer}
	\begin{pgfonlayer}{edgelayer}
		\draw (0) to (5.center);
		\draw (6.center) to (1);
		\draw (1) to (0);
		\draw (4.center) to (1);
		\draw (3.center) to (0);
	\end{pgfonlayer}
\end{tikzpicture}
$$

\end{definition}


\begin{lemma}\cite[Thm. 11]{lafont}
$\Iso(\Aff\cb_2)$ is a presentation for the prop $(\Iso(\Aff\Mat(\F_2),+))$.
\end{lemma}






\begin{definition}
Let $\inj(\Aff\cb_2)$ denote the pushout of the diagram of props:
$$
 \inj(\cb_2) \leftarrow  \Iso(\cb_2)\rightarrow  \Iso(\Aff\cb_2)
$$
\end{definition}




\begin{lemma}
\label{lem:injaffcb}
$\inj(\Aff\cb_2)$ is a presentation for the prop $(\inj(\Aff\Mat(\F_2)),+)$.
\end{lemma}


\begin{proof}
Consider the following diagram:


\renewcommand{\cubetopbl}{$\Iso(\cb_2)$}
\renewcommand{\cubetopbr}{$\Iso(\Aff\cb_2)$}
\renewcommand{\cubetopfl}{$\inj(\cb_2)$}
\renewcommand{\cubetopfr}{$\inj(\Aff\cb_2)$}
\renewcommand{\cubebotbl}{$(\Iso(\Mat(\F_2)),+)$ }
\renewcommand{\cubebotbr}{$(\Iso(\Aff\Mat(\F_2)),+)$ }
\renewcommand{\cubebotfl}{$(\inj(\Mat(\F_2)),+)$ }
\renewcommand{\cubebotfr}{}

$$
\xymatrixrowsep{2mm}\xymatrixcolsep{1.5mm}
\xymatrix{
                                       & \mbox{\cubetopbl} \ar[rr] \ar[dl] \ar[dd]^(.7){\cong}      &                                                  & \mbox{\cubetopbr}  \ar[dd]^{\cong} \ar[dl] \\
\mbox{\cubetopfl} \ar[rr]  \ar[dd]_{\cong}           &                                                                                              &\mbox{\cubetopfr} \ar@{-->}[dd]^(.35){\cong}   \skewpullbackcorner[ul]              \\
                                       &  \mbox{\cubebotbl} \ar[dl] \ar[rr]                    &                                                  & \mbox{\cubebotbr} \ar@/^1pc/[ddl] \ar[dl] \\
\mbox{\cubebotfl} \ar@/_1pc/[drr] \ar[rr]  &                                                                                             & \mbox{\cubebotfr} \skewpullbackcorner[ul]    \ar@{-->}[d]^{\cong}_F  \\
                                                   &                                                                                             & (\inj(\Aff\Mat(\F_2)),+)
}
$$

 The rear and left faces of the cube commute and the vertical maps are all isomorphisms. Therefore, the whole cube commutes via universal property of the pushout, with the upper universal map being an isomorphism.

We seek to show that the lower universal map  $F$ is also an isomorphism.  It is clearly the identity on objects, so we just have to show fullness and faithfulness.

For fullness, consider any map $n\xrightarrowtail{(A,x)} m$ in $(\inj(\Aff\Mat(\F_2)),+)$.  Note that this can be factored into:
$$
n\xrightarrowtail{(A,0)} m \xrightarrowiso{(1,x)}  m
$$
Which lies in the image of $F$ as $m \xrightarrowiso{(1,x)} m$ is an isomorphism.

For faithfulness, we show that every map in $(\Iso(\Aff\Mat(\F_2)),+)+_{(\Iso(\Mat(\F_2)),+)} (\inj(\Mat(\F_2)),+)$ can be factored uniquely in this way. 
There are two cases:
$$
\left( n \xrightarrowtail{ A} m ; m \xrightarrowiso{(B, x)} m \right)
= \left( n \xrightarrowtail{ A} m ; m \xrightarrowiso{(B, 0)} m; m \xrightarrowiso{(1, x)}  m \right)
= \left( n \xrightarrowtail{ A;B}  m\xrightarrowiso{(1, x)}  m \right)
$$
$$
\left(n \xrightarrowtail{ (A,x)} m ; m \xrightarrowiso{B} m \right)
= \left( n \xrightarrowtail{ (A,0)}m; m \xrightarrowiso{(1,x)} m ; m \xrightarrowiso{B} m \right)
= \left( n \xrightarrowtail{ A }m; m \xrightarrowiso{(B,B(x))} m  \right)
= \left( n \xrightarrowtail{ A;B }m; m \xrightarrowiso{(1,B(x))} m  \right)
$$
So every map in this pushout has the correct form, which is unique by construction.
\end{proof}


To define partial isomorphisms, we must add a point to the constituent props of the desired distributive law, because the empty set can arise as a subobject by pullback (where the empty set is not properly an object in the prop).

%\begin{definition}
%Given a prop $\X$, let $\X!$ denote the prop generated by adding a scalar $0$,  quotiented by the equation, for all parallel $f,g$: 
%$
%f \otimes 0  =  g \otimes 0
%$ 
%\end{definition}
%
%
%
%That is to say, $\X!$ is the prop with zero maps formally added.  In affine matrices, there is no proper zero object: the one element space is the terminal object and the empty set is the initial object.  By taking spans of affine matrices, the initial object becomes a zero object; however, seeing as we are working with props, the empty set can not be represented using this formalism.  Thus we just add the zero object as a subjobject.





\begin{definition}
\label{def:isoaffcbzero}
Let $\Iso(\Aff\cb_2)^{+1}$ denote the prop obtained by adjoining the following generator to $\Iso(\Aff\cb_2)$ 
$
\begin{tikzpicture}
	\begin{pgfonlayer}{nodelayer}
		\node [style=X] (0) at (0, 0) {$1$};
	\end{pgfonlayer}
\end{tikzpicture}
$
modulo the equations:
$$
\begin{tikzpicture}
	\begin{pgfonlayer}{nodelayer}
		\node [style=X] (0) at (0, 0) {$1$};
		\node [style=X] (3) at (0.5, 0) {$1$};
	\end{pgfonlayer}
\end{tikzpicture}
\eqzxa{zero.one}
\begin{tikzpicture}
	\begin{pgfonlayer}{nodelayer}
		\node [style=X] (0) at (0, 0) {$1$};
	\end{pgfonlayer}
\end{tikzpicture},
\hspace*{.5cm}
\begin{tikzpicture}
	\begin{pgfonlayer}{nodelayer}
		\node [style=X] (0) at (0, 1) {$1$};
		\node [style=none] (1) at (0.5, 0.5) {};
		\node [style=none] (2) at (0.5, 1.5) {};
		\node [style=none] (3) at (1, 1.5) {};
		\node [style=none] (4) at (1, 0.5) {};
		\node [style=dot] (5) at (0.5, 1) {};
		\node [style=oplus] (6) at (1, 1) {};
	\end{pgfonlayer}
	\begin{pgfonlayer}{edgelayer}
		\draw [in=90, out=-90] (2.center) to (1.center);
		\draw [in=-90, out=90] (4.center) to (3.center);
		\draw (6) to (5);
	\end{pgfonlayer}
\end{tikzpicture}
\eqzxa{zero.two}
\begin{tikzpicture}
	\begin{pgfonlayer}{nodelayer}
		\node [style=X] (0) at (0, 1) {$1$};
		\node [style=none] (1) at (0.5, 0.5) {};
		\node [style=none] (2) at (0.5, 1.5) {};
		\node [style=none] (3) at (1, 1.5) {};
		\node [style=none] (4) at (1, 0.5) {};
	\end{pgfonlayer}
	\begin{pgfonlayer}{edgelayer}
		\draw [in=90, out=-90] (2.center) to (1.center);
		\draw [in=-90, out=90] (4.center) to (3.center);
	\end{pgfonlayer}
\end{tikzpicture},
\hspace*{.5cm}
\begin{tikzpicture}
	\begin{pgfonlayer}{nodelayer}
		\node [style=X] (0) at (0, 1) {$1$};
		\node [style=none] (1) at (0.5, 0.5) {};
		\node [style=none] (2) at (1, 1.5) {};
		\node [style=none] (3) at (0.5, 1.5) {};
		\node [style=none] (4) at (1, 0.5) {};
	\end{pgfonlayer}
	\begin{pgfonlayer}{edgelayer}
		\draw [in=90, out=-90] (2.center) to (1.center);
		\draw [in=-90, out=90] (4.center) to (3.center);
	\end{pgfonlayer}
\end{tikzpicture}
\eqzxa{zero.three}
\begin{tikzpicture}
	\begin{pgfonlayer}{nodelayer}
		\node [style=X] (0) at (0, 1) {$1$};
		\node [style=none] (1) at (0.5, 0.5) {};
		\node [style=none] (2) at (0.5, 1.5) {};
		\node [style=none] (3) at (1, 1.5) {};
		\node [style=none] (4) at (1, 0.5) {};
	\end{pgfonlayer}
	\begin{pgfonlayer}{edgelayer}
		\draw [in=90, out=-90] (2.center) to (1.center);
		\draw [in=-90, out=90] (4.center) to (3.center);
	\end{pgfonlayer}
\end{tikzpicture},
\hspace*{.5cm}
\begin{tikzpicture}
	\begin{pgfonlayer}{nodelayer}
		\node [style=X] (0) at (0, 1) {$1$};
		\node [style=none] (1) at (0.5, 0.5) {};
		\node [style=none] (2) at (0.5, 1.5) {};
		\node [style=oplus] (3) at (0.5, 1) {};
	\end{pgfonlayer}
	\begin{pgfonlayer}{edgelayer}
		\draw (2.center) to (1.center);
	\end{pgfonlayer}
\end{tikzpicture}
\eqzxa{zero.four}
\begin{tikzpicture}
	\begin{pgfonlayer}{nodelayer}
		\node [style=X] (0) at (0, 1) {$1$};
		\node [style=none] (1) at (0.5, 0.5) {};
		\node [style=none] (2) at (0.5, 1.5) {};
	\end{pgfonlayer}
	\begin{pgfonlayer}{edgelayer}
		\draw (2.center) to (1.center);
	\end{pgfonlayer}
\end{tikzpicture}
$$
\end{definition}


\begin{lemma}
$\Iso(\Aff\cb_2)^{+1}$ is a presentation for the subcategory of $(\Span(\Aff\Fin\Vect(\F_2)), +)$ generated by spans $\F_2^n = \F_2^n \xrightarrow[\cong]{f} \F_2^n$ and $\F_2^n \xleftarrowtail {?} \emptyset \xrightarrowtail{?}  \F_2^n$, for all $n \in \N$ and isomorphisms $f$. 
\end{lemma}


%
%\begin{lemma}
%
%The props $\Iso(\Aff\cb_2)^{+1}$ and $\Iso(\Aff\cb_2)!$ are isomorphic.
%
%\end{lemma}

\begin{proof}
Identify this new generator with the span $\F_2^0 \leftarrow \emptyset \rightarrow \F_2^0$.  If there is a factor of 
$
\begin{tikzpicture}
	\begin{pgfonlayer}{nodelayer}
		\node [style=X] (0) at (0, 0) {$1$};
	\end{pgfonlayer}
\end{tikzpicture}
$,   repeatedly apply these identities from left to right until the diagram corresponding to the identity tensored by $
\begin{tikzpicture}
	\begin{pgfonlayer}{nodelayer}
		\node [style=X] (0) at (0, 0) {$1$};
	\end{pgfonlayer}
\end{tikzpicture}
$ is obtained, which is as a normal form.
\end{proof}

\begin{definition}
Let $\inj(\Aff\cb_2)^{+1}$ denote the pushout of the diagram of props:


$$
\inj(\Aff\cb_2) \leftarrow \Iso(\Aff\cb_2) \rightarrow \Iso(\Aff\cb_2)^{+1}
$$


\end{definition}

\begin{lemma}
$\inj(\Aff\cb_2)^{+1}$ is a presentation for the subcategory of $(\Span(\Aff\Fin\Vect(\F_2)), +)$ generated by spans $\F_2^n = \F_2^n \xrightarrowtail{e} \F_2^m$ and $\F_2^n \xleftarrowtail{?} \emptyset \xrightarrowtail{?}  \F_2^n$, for all $n,m \in \N$ and monics $e$. 
\end{lemma}
%
%\begin{lemma}
%The props $\inj(\Aff\cb_2)^{+1}$ and $\inj(\Aff\cb_2)!$ are isomorphic.
%\end{lemma}

The proof of this lemma is essentially the same for $\Iso(\Aff\cb_2)^{+1}$, although diagrams with a factor of
$
\begin{tikzpicture}
	\begin{pgfonlayer}{nodelayer}
		\node [style=X] (0) at (0, 0) {$1$};
	\end{pgfonlayer}
\end{tikzpicture}
$ are reduced to the following normal form:
\hspace*{3cm}
$
\begin{tikzpicture}
	\begin{pgfonlayer}{nodelayer}
		\node [style=X] (0) at (0, 1.25) {$1$};
		\node [style=none] (1) at (0.5, 0.5) {};
		\node [style=none] (2) at (0.5, 1.75) {};
		\node [style=none] (3) at (1, 0.5) {};
		\node [style=none] (4) at (1, 1.75) {};
		\node [style=X] (5) at (1.5, 0.75) {};
		\node [style=X] (6) at (2, 0.75) {};
		\node [style=none] (7) at (1.5, 1.75) {};
		\node [style=none] (8) at (2, 1.75) {};
		\node [style=none] (9) at (0.75, 1.5) {$n$};
		\node [style=none] (10) at (1.75, 1.5) {$m$};
		\node [style=none] (11) at (1.77, 1.25) {$\cdots$};
		\node [style=none] (12) at (0.77, 1.25) {$\cdots$};
	\end{pgfonlayer}
	\begin{pgfonlayer}{edgelayer}
		\draw (2.center) to (1.center);
		\draw (4.center) to (3.center);
		\draw (5) to (7.center);
		\draw (8.center) to (6);
	\end{pgfonlayer}
\end{tikzpicture}
$

Unlike in the linear case, now we must consider a distributive law over a prop which is not a groupoid: we add a single idempotent corresponding to the empty set to the isomorphisms.  To satisfy the requirement that this prop is a sub-prop of the left and right components of the  distributive law, we also add this idempotent to the injections and the co-injections:


\begin{definition}
\label{def:parisoaffcb}
Consider the prop $\pr\iso\Aff\cb_2$ generated by the distributive law of props:

$$
 (\inj(\Aff\cb_2)^{+1})^\op \otimes_{\Iso(\Aff\cb_2)^{+1}}  \inj(\Aff\cb_2)^{+1}
$$
Given by the equations of $\pr\iso\Aff\cb_2$ as well as:
\!
$
\begin{tikzpicture}
	\begin{pgfonlayer}{nodelayer}
		\node [style=X] (0) at (0.5, 0.75) {$1$};
		\node [style=X] (1) at (0.5, 0) {};
	\end{pgfonlayer}
	\begin{pgfonlayer}{edgelayer}
		\draw (0) to (1);
	\end{pgfonlayer}
\end{tikzpicture}
=
\begin{tikzpicture}
	\begin{pgfonlayer}{nodelayer}
		\node [style=X] (0) at (0, 0) {$1$};
		\node [style=X] (1) at (0, 0.75) {};
	\end{pgfonlayer}
	\begin{pgfonlayer}{edgelayer}
		\draw (0) to (1);
	\end{pgfonlayer}
\end{tikzpicture}
\eqzxa{zero.five}
\begin{tikzpicture}
	\begin{pgfonlayer}{nodelayer}
		\node [style=X] (0) at (0, 0) {$1$};
	\end{pgfonlayer}
	\begin{pgfonlayer}{edgelayer}
	\end{pgfonlayer}
\end{tikzpicture}
$

\end{definition}


\begin{remark}
\label{rem:parisoaffcb}
$ (\inj(\Aff\cb_2)^{+1})^\op \otimes_{\Iso(\Aff\cb_2)^{+1}}  \inj(\Aff\cb_2)^{+1}$ is actually a distributive law because the only only nontrivial situation arises when controlled-not gates are sandwiched between black, or black $1$ units/counits on their target wires.  The case where there are no controlled not gates in between is resolved by the new axiom we have added.  When there are more controlled-not gates, they can be pushed past each other as follows:
$$
\begin{tikzpicture}
	\begin{pgfonlayer}{nodelayer}
		\node [style=X] (0) at (2.5, 0.25) {};
		\node [style=oplus] (1) at (2.5, -0.75) {};
		\node [style=oplus] (2) at (2.5, -1.5) {};
		\node [style=dot] (3) at (1.5, -0.75) {};
		\node [style=dot] (4) at (1, -1.5) {};
		\node [style=none] (5) at (1.5, 0.5) {};
		\node [style=none] (6) at (1, 0.5) {};
		\node [style=none] (7) at (1.25, -1) {$\iddots$};
		\node [style=none] (8) at (2.5, -1) {$\vdots$};
		\node [style=X] (9) at (2.5, -2) {$1$};
		\node [style=none] (10) at (1, -2.25) {};
		\node [style=none] (11) at (1.5, -2.25) {};
		\node [style=none] (12) at (1.25, -0.25) {$\cdots$};
		\node [style=none] (17) at (2, -2.25) {};
		\node [style=none] (18) at (2, 0.5) {};
		\node [style=oplus] (19) at (2.5, -0.25) {};
		\node [style=dot] (20) at (2, -0.25) {};
	\end{pgfonlayer}
	\begin{pgfonlayer}{edgelayer}
		\draw (0) to (1);
		\draw (1) to (3);
		\draw (5.center) to (3);
		\draw (6.center) to (4);
		\draw (4) to (2);
		\draw (4) to (10.center);
		\draw (11.center) to (3);
		\draw (9) to (2);
		\draw (17.center) to (20);
		\draw (20) to (18.center);
		\draw (19) to (20);
	\end{pgfonlayer}
\end{tikzpicture}
=
\begin{tikzpicture}
	\begin{pgfonlayer}{nodelayer}
		\node [style=X] (0) at (2.5, 0.25) {};
		\node [style=oplus] (1) at (2.5, -0.75) {};
		\node [style=oplus] (2) at (2.5, -1.5) {};
		\node [style=dot] (3) at (1.5, -0.75) {};
		\node [style=dot] (4) at (1, -1.5) {};
		\node [style=none] (5) at (1.5, 0.5) {};
		\node [style=none] (6) at (1, 0.5) {};
		\node [style=none] (7) at (1.25, -1) {$\iddots$};
		\node [style=none] (8) at (2.5, -1) {$\vdots$};
		\node [style=X] (9) at (2.5, -2.5) {};
		\node [style=none] (10) at (1, -2.75) {};
		\node [style=none] (11) at (1.5, -2.75) {};
		\node [style=none] (12) at (1.25, -0.25) {$\cdots$};
		\node [style=none] (17) at (2, -2.75) {};
		\node [style=none] (18) at (2, 0.5) {};
		\node [style=oplus] (19) at (2.5, -0.25) {};
		\node [style=dot] (20) at (2, -0.25) {};
		\node [style=oplus] (21) at (2.5, -2) {};
	\end{pgfonlayer}
	\begin{pgfonlayer}{edgelayer}
		\draw (0) to (1);
		\draw (1) to (3);
		\draw (5.center) to (3);
		\draw (6.center) to (4);
		\draw (4) to (2);
		\draw (4) to (10.center);
		\draw (11.center) to (3);
		\draw (9) to (2);
		\draw (17.center) to (20);
		\draw (20) to (18.center);
		\draw (19) to (20);
	\end{pgfonlayer}
\end{tikzpicture}
=
\begin{tikzpicture}
	\begin{pgfonlayer}{nodelayer}
		\node [style=X] (0) at (2.5, 0.75) {};
		\node [style=oplus] (1) at (2.5, -0.75) {};
		\node [style=oplus] (2) at (2.5, -1.5) {};
		\node [style=dot] (3) at (1.5, -0.75) {};
		\node [style=dot] (4) at (1, -1.5) {};
		\node [style=none] (5) at (1.5, 1.25) {};
		\node [style=none] (6) at (1, 1.25) {};
		\node [style=none] (7) at (1.25, -1) {$\iddots$};
		\node [style=none] (8) at (2.5, -1) {$\vdots$};
		\node [style=X] (9) at (2.5, -2) {};
		\node [style=none] (10) at (1, -2.25) {};
		\node [style=none] (11) at (1.5, -2.25) {};
		\node [style=none] (12) at (1.25, 0.25) {$\cdots$};
		\node [style=none] (17) at (2, -2.25) {};
		\node [style=none] (18) at (2, 1.25) {};
		\node [style=oplus] (19) at (2.5, 0.25) {};
		\node [style=dot] (20) at (2, 0.25) {};
		\node [style=oplus] (21) at (2, -0.25) {};
		\node [style=oplus] (22) at (2, 0.75) {};
	\end{pgfonlayer}
	\begin{pgfonlayer}{edgelayer}
		\draw (0) to (1);
		\draw (1) to (3);
		\draw (5.center) to (3);
		\draw (6.center) to (4);
		\draw (4) to (2);
		\draw (4) to (10.center);
		\draw (11.center) to (3);
		\draw (9) to (2);
		\draw (17.center) to (20);
		\draw (20) to (18.center);
		\draw (19) to (20);
	\end{pgfonlayer}
\end{tikzpicture}
=
\begin{tikzpicture}
	\begin{pgfonlayer}{nodelayer}
		\node [style=X] (0) at (2.5, 0.25) {};
		\node [style=oplus] (1) at (2.5, -0.75) {};
		\node [style=oplus] (2) at (2.5, -1.5) {};
		\node [style=dot] (3) at (1.5, -0.75) {};
		\node [style=dot] (4) at (1, -1.5) {};
		\node [style=none] (5) at (1.5, 1) {};
		\node [style=none] (6) at (1, 1) {};
		\node [style=none] (7) at (1.25, -1) {$\iddots$};
		\node [style=none] (8) at (2.5, -1) {$\vdots$};
		\node [style=X] (9) at (2.5, -2) {};
		\node [style=none] (10) at (1, -2.75) {};
		\node [style=none] (11) at (1.5, -2.75) {};
		\node [style=none] (12) at (1.25, -0.25) {$\cdots$};
		\node [style=none] (17) at (2, -2.75) {};
		\node [style=none] (18) at (2, 1) {};
		\node [style=oplus] (19) at (2.5, -0.25) {};
		\node [style=dot] (20) at (2, -0.25) {};
		\node [style=oplus] (21) at (2, -2.25) {};
		\node [style=oplus] (22) at (2, 0.5) {};
	\end{pgfonlayer}
	\begin{pgfonlayer}{edgelayer}
		\draw (0) to (1);
		\draw (1) to (3);
		\draw (5.center) to (3);
		\draw (6.center) to (4);
		\draw (4) to (2);
		\draw (4) to (10.center);
		\draw (11.center) to (3);
		\draw (9) to (2);
		\draw (17.center) to (20);
		\draw (20) to (18.center);
		\draw (19) to (20);
	\end{pgfonlayer}
\end{tikzpicture}
=
\begin{tikzpicture}
	\begin{pgfonlayer}{nodelayer}
		\node [style=none] (24) at (4.5, 1.75) {};
		\node [style=none] (25) at (4, 1.75) {};
		\node [style=X] (28) at (5, -1) {};
		\node [style=none] (29) at (4, -3.25) {};
		\node [style=none] (30) at (4.5, -3.25) {};
		\node [style=none] (32) at (5, -3.25) {};
		\node [style=none] (33) at (5, 1.75) {};
		\node [style=oplus] (37) at (5, 1.25) {};
		\node [style=oplus] (38) at (5, -1.5) {};
		\node [style=dot] (39) at (4.5, -1.5) {};
		\node [style=oplus] (40) at (5, -2.25) {};
		\node [style=dot] (41) at (4, -2.25) {};
		\node [style=oplus] (42) at (5, 0) {};
		\node [style=dot] (43) at (4.5, 0) {};
		\node [style=oplus] (44) at (5, 0.75) {};
		\node [style=dot] (45) at (4, 0.75) {};
		\node [style=oplus] (46) at (5, -2.75) {};
		\node [style=X] (47) at (5, -0.5) {};
		\node [style=none] (48) at (4.25, -0.75) {$\cdots$};
		\node [style=none] (49) at (4.25, -1.75) {$\iddots$};
		\node [style=none] (50) at (4.25, 0.5) {$\ddots$};
		\node [style=none] (51) at (5, 0.5) {$\vdots$};
		\node [style=none] (52) at (5, -1.75) {$\vdots$};
	\end{pgfonlayer}
	\begin{pgfonlayer}{edgelayer}
		\draw (38) to (39);
		\draw (40) to (41);
		\draw (42) to (43);
		\draw (44) to (45);
		\draw (29.center) to (25.center);
		\draw (24.center) to (30.center);
		\draw (38) to (28);
		\draw (40) to (32.center);
		\draw (47) to (42);
		\draw (44) to (33.center);
	\end{pgfonlayer}
\end{tikzpicture}
=
\begin{tikzpicture}
	\begin{pgfonlayer}{nodelayer}
		\node [style=none] (24) at (4.5, 1.25) {};
		\node [style=none] (25) at (4, 1.25) {};
		\node [style=X] (28) at (5, -1) {$1$};
		\node [style=none] (29) at (4, -2.75) {};
		\node [style=none] (30) at (4.5, -2.75) {};
		\node [style=none] (32) at (5, -2.75) {};
		\node [style=none] (33) at (5, 1.25) {};
		\node [style=oplus] (38) at (5, -1.5) {};
		\node [style=dot] (39) at (4.5, -1.5) {};
		\node [style=oplus] (40) at (5, -2.25) {};
		\node [style=dot] (41) at (4, -2.25) {};
		\node [style=oplus] (42) at (5, 0) {};
		\node [style=dot] (43) at (4.5, 0) {};
		\node [style=oplus] (44) at (5, 0.75) {};
		\node [style=dot] (45) at (4, 0.75) {};
		\node [style=X] (47) at (5, -0.5) {$1$};
		\node [style=none] (48) at (4.25, -0.75) {$\cdots$};
		\node [style=none] (49) at (4.25, -1.75) {$\iddots$};
		\node [style=none] (50) at (4.25, 0.5) {$\ddots$};
		\node [style=none] (51) at (5, 0.5) {$\vdots$};
		\node [style=none] (52) at (5, -1.75) {$\vdots$};
	\end{pgfonlayer}
	\begin{pgfonlayer}{edgelayer}
		\draw (38) to (39);
		\draw (40) to (41);
		\draw (42) to (43);
		\draw (44) to (45);
		\draw (29.center) to (25.center);
		\draw (24.center) to (30.center);
		\draw (38) to (28);
		\draw (40) to (32.center);
		\draw (47) to (42);
		\draw (44) to (33.center);
	\end{pgfonlayer}
\end{tikzpicture}
$$
%Notice that the choice of which wires to straighten out the zig-zag is arbitrary.
\end{remark}







\begin{lemma}
\label{lem:parisoaffcb}
$\ParIso(\Aff\cb_2)$ is a presentation for the full subcategory $\Par\Iso(\Aff\Fin\Vect(\F_2))^*$ of $\Par\Iso(\Aff\Fin\Vect(\F_2))$ where the objects are nonempty affine vector spaces.
\end{lemma}


\begin{proof}
The obvious functor $\ParIso(\Aff\cb_2)\to \Par\Iso(\Aff\Fin\Vect(\F_2))^*$ is clearly full,  as well as an isomorphism on objects.
It remains to show it is faihful.  It is faithful on maps which are taken to spans with nonempty apex by the same argument as Lemma \ref{lem:parisocb}. For empty case, there is exactly one diagram of each type with a factor of $0$; and similarly, there is exactly one span with an empty apex.
\end{proof}

By \cite{cnot}  in this the identities of Definition \ref{def:isoaffcbzero}
 can be replaced by the following identity, while maintaining completeness:
\hspace*{.4cm}
$
\begin{tikzpicture}
	\begin{pgfonlayer}{nodelayer}
		\node [style=X] (0) at (0, 5) {$1$};
		\node [style=none] (1) at (0.5, 5.75) {};
		\node [style=none] (2) at (0.5, 4.25) {};
	\end{pgfonlayer}
	\begin{pgfonlayer}{edgelayer}
		\draw (2.center) to (1.center);
	\end{pgfonlayer}
\end{tikzpicture}
\eqzxa{zero.six}
\begin{tikzpicture}
	\begin{pgfonlayer}{nodelayer}
		\node [style=X] (0) at (0, 5) {$1$};
		\node [style=none] (1) at (0.5, 5.75) {};
		\node [style=none] (2) at (0.5, 4.25) {};
		\node [style=X] (3) at (0.5, 5.25) {$1$};
		\node [style=X] (4) at (0.5, 4.75) {$1$};
	\end{pgfonlayer}
	\begin{pgfonlayer}{edgelayer}
		\draw (3) to (1.center);
		\draw (4) to (2.center);
	\end{pgfonlayer}
\end{tikzpicture}
$

%Using the identities presented in \cite{cnot}, this has a more compact presentation given in Appendix \ref{subsubsec:presentations:three:pinj}.
%This alternative form is much more in the aesthetic vein of the ZX-calculus,

%\begin{corollary}
%The prop $\ParIso(\Aff\cb_2)$ has a finite presentation with axioms where the axioms are the union of the axioms for $\inj(\Aff\cb_2),\inj(\Aff\cb_2)^\op$, the law $\eta_X\eta_X^\op=1_I$ as well as the $1$-law, replacing the universally quantified law for the 0 scalar:
%$$
%\begin{tikzpicture}
%	\begin{pgfonlayer}{nodelayer}
%		\node [style=none] (0) at (-1, -7) {};
%		\node [style=X] (1) at (-0.5, -7.75) {$1$};
%		\node [style=none] (2) at (-1, -8.5) {};
%	\end{pgfonlayer}
%	\begin{pgfonlayer}{edgelayer}
%		\draw (0.center) to (2.center);
%	\end{pgfonlayer}
%\end{tikzpicture}
%=
%\begin{tikzpicture}
%	\begin{pgfonlayer}{nodelayer}
%		\node [style=none] (0) at (-1, -7) {};
%		\node [style=X] (1) at (-0.5, -7.75) {$1$};
%		\node [style=X] (2) at (-1, -7.5) {};
%		\node [style=X] (3) at (-1, -8) {};
%		\node [style=none] (4) at (-1, -8.5) {};
%	\end{pgfonlayer}
%	\begin{pgfonlayer}{edgelayer}
%		\draw (0.center) to (2);
%		\draw (3) to (4.center);
%	\end{pgfonlayer}
%\end{tikzpicture}
%$$
% 
%
%\end{corollary}


\begin{definition}

Let $\pr\Aff\cb_2$ denote the pushout of the diagram of props:
$$
\pr\iso\Aff\cb_2 \leftarrow \surj^\op \rightarrow \cm^\op
$$

\end{definition}





\begin{lemma}
\label{lem:paraffcb}

$\pr\Aff\cb_2$ is a presentation for the prop $(\Par(\Aff\Fin\Vect(\F_2))^*,+)$.
\end{lemma}

\begin{proof}
%\renewcommand{\cubetopbl}{$\inj(\Aff\cb_2)$}
%\renewcommand{\cubetopbr}{$\inj(\Aff\cb_2)+1\otimes_{\Iso(\Aff\cb_2)+1} \inj(\Aff\cb_2)+1^\op$}
%\renewcommand{\cubetopfl}{$\inj(\Aff\cb_2)\otimes_{\Iso(\Aff\cb_2)} \surj(\Aff\cb_2)^\op$}
%\renewcommand{\cubetopfr}{$\Par(\Aff\cb_2)$}
%\renewcommand{\cubebotbl}{$(\inj(\Aff\Mat(\F_2)),+)$ }
%\renewcommand{\cubebotbr}{$(\ParIso(\Aff\Vect(\F_2))^*,+)$ }
%\renewcommand{\cubebotfl}{$(\Par\surj(\Aff\Mat(\F_2)),+)^\op$ }
%\renewcommand{\cubebotfr}{}
%
%$$
%\xymatrixrowsep{3mm}\xymatrixcolsep{-10mm}
%\xymatrix{
%                                       & \mbox{\cubetopbl} \ar[rr] \ar[dl] \ar[dd]^(.7){\cong}      &                                                  & \mbox{\cubetopbr}  \ar[dd]^{\cong} \ar[dl] \\
%\mbox{\cubetopfl} \ar[rr]  \ar[dd]_{\cong}           &                                                                                              &\mbox{\cubetopfr} \ar@{-->}[dd]^(.35){\cong}   \skewpullbackcorner[ul]              \\
%                                       &  \mbox{\cubebotbl} \ar[dl] \ar[rr]                    &                                                  & \mbox{\cubebotbr} \ar@/^1pc/[ddl] \ar[dl] \\
%\mbox{\cubebotfl} \ar@/_1pc/[drr] \ar[rr]  &                                                                                             & \mbox{\cubebotfr} \skewpullbackcorner[ul]    \ar@{-->}[d]^{\cong}  \\
%                                                   &                                                                                             & (\Par(\Aff\Mat(\F_2)),+)
%}
%$$



\renewcommand{\cubetopbl}{$\surj^\op$}
\renewcommand{\cubetopbr}{$\cm^\op$}
\renewcommand{\cubetopfl}{$\pr\iso\Aff\cb_2$}
\renewcommand{\cubetopfr}{$\pr\Aff\cb_2$}
\renewcommand{\cubebotbl}{$\surj^\op$ }
\renewcommand{\cubebotbr}{$\cm^\op$ }
\renewcommand{\cubebotfl}{$(\ParIso(\Aff\Vect(\F_2))^*,+)$ }
\renewcommand{\cubebotfr}{}

$$
\xymatrixrowsep{2mm}\xymatrixcolsep{1mm}
\xymatrix{
                                       & \mbox{\cubetopbl} \ar[rr] \ar[dl] \ar@{=}[dd]     &                                                  & \mbox{\cubetopbr} \ar@{=}[dd] \ar[dl] \\
\mbox{\cubetopfl} \ar[rr]  \ar[dd]_{\cong}           &                                                                                              &\mbox{\cubetopfr} \ar@{-->}[dd]^(.35){\cong}   \skewpullbackcorner[ul]              \\
                                       &  \mbox{\cubebotbl} \ar[dl] \ar[rr]                    &                                                  & \mbox{\cubebotbr} \ar@/^1pc/[ddl] \ar[dl] \\
\mbox{\cubebotfl} \ar@/_1pc/[drr] \ar[rr]  &                                                                                             & \mbox{\cubebotfr} \skewpullbackcorner[ul]    \ar@{-->}[d]_F^{\cong}  \\
                                                   &                                                                                             & (\Par(\Aff\Vect(\F_2))^*,+) 
}
$$


We know that $\pr\iso\Aff\cb_2\cong \ParIso(\Aff\Vect(\F_2))^*,+)$ is a discrete inverse category by \cite[Prop. 3.4]{cnot}.

The cube commutes by the universal property of the pushout, as before.

We just have to show that the universal map $F$ is an isomorphism.  It is clearly the identity on objects, so we just have to show it is full and faithful.
This follows from essentially the same argument as in the linear case.


\end{proof}


%\begin{comment}
$\pr\Aff\cb_2$ has a particularly elegant presentation given in \S \ref{subsubsec:presentations:two:par}, which is much more in the spirit of the ZX-calculus.
%\end{comment}


\begin{definition}
Let $\sp\Aff\cb_2$ denote the pushout of the diagram of props:
$$
 \pr\Aff\cb_2^\op \leftarrow \pr\iso\Aff\cb_2 \rightarrow \pr\Aff\cb_2
$$

\end{definition}



\begin{lemma}
\label{lem:spanaffcb}
$\sp\Aff\cb_2$ is a presentation for the prop $(\Span(\Aff\Fin\Vect(\F_2))^*,+)$.
\end{lemma}

\begin{proof}
\renewcommand{\cubetopbl}{$\pr\iso\Aff\cb_2$}
\renewcommand{\cubetopbr}{$\pr\Aff\cb_2$}
\renewcommand{\cubetopfl}{$\pr\Aff\cb_2^\op$}
\renewcommand{\cubetopfr}{$\sp\Aff\cb_2$}
\renewcommand{\cubebotbl}{$(\ParIso(\Aff\Vect(\F_2))^*,+)$ }
\renewcommand{\cubebotbr}{$(\Par(\Aff\Vect(\F_2))^*,+)$ }
\renewcommand{\cubebotfl}{$(\Par(\Aff\Vect(\F_2))^*,+)^\op$ }
\renewcommand{\cubebotfr}{}

$$
\hspace*{-1cm}
\xymatrixrowsep{2mm}\xymatrixcolsep{.5mm}
\xymatrix{
                                       & \mbox{\cubetopbl} \ar[rr] \ar[dl] \ar[dd]^(.7){\cong}      &                                                  & \mbox{\cubetopbr}  \ar[dd]^{\cong} \ar[dl] \\
\mbox{\cubetopfl} \ar[rr]  \ar[dd]_{\cong}           &                                                                                              &\mbox{\cubetopfr} \ar@{-->}[dd]^(.35){\cong}   \skewpullbackcorner[ul]              \\
                                       &  \mbox{\cubebotbl} \ar[dl] \ar[rr]                    &                                                  & \mbox{\cubebotbr} \ar@/^1pc/[ddl] \ar[dl] \\
\mbox{\cubebotfl} \ar@/_1pc/[drr] \ar[rr]  &                                                                                             & \mbox{\cubebotfr} \skewpullbackcorner[ul]    \ar@{-->}[d]_F^{\cong}  \\
                                                   &                                                                                             & (\Span(\Aff\Vect(\F_2))^*,+)
}
$$

 The rear and left faces of the cube commute and the vertical maps are all isomorphisms. Therefore, the whole cube commutes by the universal property of the pushout, with the upper universal map being an isomorphism.

We seek to show that the lower universal map  $F$ is also an isomorphism.  It is clearly the identity on objects, so we just have to show fullness and faithfulness.

For fullness, let us first consider the nonempty case; that is a map $\F_2^n \xleftarrow{(A,x)} \F_2^k \xrightarrow{(B,y)}\F^m$ in $(\Span(\Aff\Vect(\F_2))^*,+)$.  This is in the image of the following diagram under $F$:
$$
(\F_2^n \xleftarrow{(A,x)} \F_2^k  = \F_2^k); (\F_2^k = \F_2^k  \xrightarrow{(B,y)}\F^m)
$$ 
Otherwise, consider a map of the form  $\F_2^n \xleftarrow{?} \emptyset  \xrightarrow{?}\F^m$.  This the image of the following diagram:
$$
(\F_2^n \xleftarrow{?} \emptyset \xrightarrow {?} \F_2^0  );(\F_2^0 \xleftarrow{?} \emptyset  \xrightarrow{?}\F^m)
$$
For faithfulness, again, we separate the proof into two cases.  The functor is faithful on diagrams in $(\Span(\Aff\Vect(\F_2))^*,+)$ with nonempty apex by the same argument as in Lemma \ref{lem:spancb}.
%$$
%\xymatrix{
%          & \F_2^k \ar[dl]_{(A',x')} \ar[dd]_{\cong}^{(C,z)} \ar[dr]^{(B',y')}\\
%\F_2^n  &                                                                                                    & \F_2^m\\
%         & \F_2^k \ar[ul]^{(A,x)} \ar[ur]_{(B,y)}\\
%}
%$$
%We have the following equation in $\Span(\Aff\cb_2)$:
%{
%\xymatrixrowsep{1mm}\xymatrixcolsep{3.5mm}
%\begin{align*}
%\xymatrix{
%          & \F_2^k \ar[dl]_{(A,x)}  \ar@{=}[dr]\\
%\F_2^n  &                                                                                                    & \F_2^k\\
%};
%\xymatrix{
%          & \F_2^k \ar[dr]^{(B,y)}  \ar@{=}[dl]\\
%\F_2^k  &                                                                                                    & \F_2^m\\
%} &=
%\xymatrix{
%          & \F_2^k \ar[dl]_{(A,x)}  \ar@{=}[dr]\\
%\F_2^n  &                                                                                                    & \F_2^k\\
%};
%\xymatrix{
%          & \F_2^k \ar@{=}[dr] \ar@{=}[dl] \\
%\F_2^k  &                                                                                                    & \F_2^k\\
%         & \F_2^k \ar[ul]^{(C,z)} \ar[ur] _{(C,z)} \ar[uu]^{\cong}_{(C,z)}
%};
%\xymatrix{
%          & \F_2^k \ar[dr]^{(B,y)}  \ar@{=}[dl]\\
%\F_2^k  &                                                                                                    & \F_2^m\\
%}\\
%&=
%\xymatrix{
%          & \F_2^k \ar[dl]_{(A,x)}  \ar@{=}[dr]\\
%\F_2^n  &                                                                                                    & \F_2^k\\
%};
%\xymatrix{
%        & \F_2^k \ar[dl]_{(C,z)} \ar@{=}[dr]\\
%\F_2^k  &                                                     & \F_2^k
%};
%\xymatrix{
%        & \F_2^k \ar[dr]^{(C,z)} \ar@{=}[dl]\\
%\F_2^k  &                                                     & \F_2^k
%};
%\xymatrix{
%          & \F_2^k \ar[dr]^{(B,y)}  \ar@{=}[dl]\\
%\F_2^k  &                                                                                                    & \F_2^m\\
%}\\
%&=
%\xymatrix{
%            &                                                        & \F_2^k \ar[dl]_{(C,z)} \ar@{=}[dr] \ar@/_2.0pc/[ddll]_{(A',x')}\\
%            & \F_2^k \ar@{=}[dr] \ar[dl]^{(A,x)}&                                                          & \F_2^k \ar@{=}[dr] \ar[dl]_{(C,z)}\\
%\F_2^k &                                                         & \F_2^k                                             &                                                         &\F_2^k
%};
%\xymatrix{
%            &                                                        & \F_2^k \ar[dr]^{(C,z)} \ar@{=}[dl] \ar@/^2.0pc/[ddrr]^{(B',y')}  \\
%            & \F_2^k \ar[dr]^{(C,z)}   \ar@{=}[dl] &                                                          & \F_2^k \ar@{=}[dl] \ar[dr]_{(B,y)}\\
%\F_2^k &                                                         & \F_2^k                                             &                                                         &\F_2^k
%}
%\end{align*}
%}
The case for spans with empty apex follows immediately as the only endomorphism on the empty set is the identity; thus,  isomorphic spans must be equal on the nose.

\end{proof}

%\begin{comment}
There is a particularly elegant equivalent presentation given in \S \ref{subsubsec:presentations:two:span}.
%\end{comment}
This is almost equivalent to the presentation given in \cite{affine} which gives a presentation for the full subcategory of relations of finite dimensional affine vector spaces where the objects are given by the nonempty vector spaces, and is much more in the spirit of the ZX-calculus.


\section{The and gate}
\label{sec:three}

Recall that unlike when the tensor product is the coproduct; when the tensor product is induced by the multiplication, to obtain a prop, one must consider the subcategory generated by tensoring a fixed object with itsef.
%
%Because $\sum_n 1=n$ grows linearly in $n$ and $\prod_n k = k^n$ grows exponentially, giving presentations for multiplicative models will be much more involved because there are more points to deal with, and in particular, more subobjects arise by pullback.


\begin{definition}
Let $L_{\F_2^\times}$ be the prop generated by quotienting $\cb$ by the equation:

$$
\begin{tikzpicture}
	\begin{pgfonlayer}{nodelayer}
		\node [style=none] (0) at (-7, 1) {};
		\node [style=none] (1) at (-7, 0.5) {};
		\node [style=Z] (2) at (-7, -0.25) {};
		\node [style=none] (3) at (-7, -0.75) {};
		\node [style=andin] (4) at (-7, 0.5) {};
	\end{pgfonlayer}
	\begin{pgfonlayer}{edgelayer}
		\draw (3.center) to (2.center);
		\draw [in=-60, out=60, looseness=1.25] (2.center) to (1);
		\draw [in=120, out=-120, looseness=1.25] (1) to (2.center);
		\draw (1) to (0.center);
	\end{pgfonlayer}
\end{tikzpicture}
\eqzxa{antispecial}
\begin{tikzpicture}
	\begin{pgfonlayer}{nodelayer}
		\node [style=none] (0) at (-7, 1) {};
		\node [style=none] (1) at (-7, -0.75) {};
	\end{pgfonlayer}
	\begin{pgfonlayer}{edgelayer}
		\draw (1.center) to (0.center);
	\end{pgfonlayer}
\end{tikzpicture}
$$

Where the components of the  monoid are relabled as follows:
\hspace*{.5cm}
$
\left(
\begin{tikzpicture}
	\begin{pgfonlayer}{nodelayer}
		\node [style=none] (0) at (-3.75, 0.5) {};
		\node [style=none] (1) at (-3.75, -0.25) {};
		\node [style=andin] (2) at (-3.75, -0.25) {};
		\node [style=none] (3) at (-4, -1) {};
		\node [style=none] (4) at (-3.5, -1) {};
	\end{pgfonlayer}
	\begin{pgfonlayer}{edgelayer}
		\draw (0.center) to (1.center);
		\draw [in=-60, out=90, looseness=1.00] (4.center) to (1.center);
		\draw [in=90, out=-120, looseness=1.00] (1.center) to (3.center);
	\end{pgfonlayer}
\end{tikzpicture},
\begin{tikzpicture}
	\begin{pgfonlayer}{nodelayer}
		\node [style=none] (0) at (-3.75, -0.25) {};
		\node [style=X] (1) at (-3.75, -1) {$1$};
	\end{pgfonlayer}
	\begin{pgfonlayer}{edgelayer}
		\draw (0.center) to (1);
	\end{pgfonlayer}
\end{tikzpicture}
\right)
$


\end{definition}


\begin{lemma}
$L_{\F_2}^\times$ is a presentation for the Lawvere theory for the group of units of the field $\F_2$.
\end{lemma}

\begin{definition}
Consider the prop $\f_2$, generated by the distributive law:
$$
L_{\F_2^\times} \otimes_{\cm^\op} \cb_2;
\begin{tikzpicture}
	\begin{pgfonlayer}{nodelayer}
		\node [style=andin] (4) at (1.25, 0.5) {};
		\node [style=X] (5) at (0.75, -0.5) {};
		\node [style=none] (6) at (0.5, -1) {};
		\node [style=none] (7) at (1, -1) {};
		\node [style=none] (8) at (1.75, -1) {};
		\node [style=none] (9) at (1.25, 0.5) {};
		\node [style=none] (10) at (1.25, 1.5) {};
	\end{pgfonlayer}
	\begin{pgfonlayer}{edgelayer}
		\draw [in=-30, out=90] (8.center) to (9.center);
		\draw [in=90, out=-150] (9.center) to (5);
		\draw [in=90, out=-45] (5) to (7.center);
		\draw [in=-135, out=90] (6.center) to (5);
		\draw (9.center) to (10.center);
	\end{pgfonlayer}
\end{tikzpicture}
\eqzxa{ring.mul}
\begin{tikzpicture}
	\begin{pgfonlayer}{nodelayer}
		\node [style=none] (0) at (1, 0) {};
		\node [style=none] (1) at (0.5, -1.25) {};
		\node [style=none] (2) at (1.75, -0.75) {};
		\node [style=none] (3) at (1.33, 0.75) {};
		\node [style=andin] (4) at (1, 0) {};
		\node [style=none] (5) at (1.75, 0) {};
		\node [style=none] (6) at (1, -1.25) {};
		\node [style=none] (7) at (1.75, -0.75) {};
		\node [style=none] (8) at (1.33, 0.75) {};
		\node [style=andin] (9) at (1.75, 0) {};
		\node [style=X] (10) at (1.33, 0.75) {};
		\node [style=none] (11) at (1.33, 1.25) {};
		\node [style=none] (12) at (1.75, -1.25) {};
		\node [style=Z] (13) at (1.75, -0.75) {};
	\end{pgfonlayer}
	\begin{pgfonlayer}{edgelayer}
		\draw [in=-135, out=90] (0.center) to (3.center);
		\draw [in=165, out=-30, looseness=1.25] (0.center) to (2.center);
		\draw [in=-45, out=90] (5.center) to (8.center);
		\draw [in=45, out=-45, looseness=1.25] (5.center) to (7.center);
		\draw (10) to (11.center);
		\draw [in=90, out=-150] (4) to (1.center);
		\draw [in=-150, out=90] (6.center) to (9);
		\draw (12.center) to (13);
	\end{pgfonlayer}
\end{tikzpicture},
\hspace*{.5cm}
\begin{tikzpicture}
	\begin{pgfonlayer}{nodelayer}
		\node [style=none] (0) at (2, 0) {};
		\node [style=none] (1) at (1.75, -0.75) {};
		\node [style=none] (2) at (2.25, -0.75) {};
		\node [style=none] (3) at (2, 0.5) {};
		\node [style=none] (4) at (2.25, -1) {};
		\node [style=X] (5) at (1.75, -0.75) {};
		\node [style=andin] (6) at (2, 0) {};
	\end{pgfonlayer}
	\begin{pgfonlayer}{edgelayer}
		\draw (0.center) to (3.center);
		\draw [in=90, out=-45] (0.center) to (2.center);
		\draw (4.center) to (2.center);
		\draw [in=-135, out=90] (1.center) to (0.center);
	\end{pgfonlayer}
\end{tikzpicture}
\eqzxa{ring.unit}
\begin{tikzpicture}
	\begin{pgfonlayer}{nodelayer}
		\node [style=none] (12) at (2, 0.5) {};
		\node [style=none] (14) at (2, -1) {};
		\node [style=X] (15) at (2, 0) {};
		\node [style=Z] (16) at (2, -0.5) {};
	\end{pgfonlayer}
	\begin{pgfonlayer}{edgelayer}
		\draw (15) to (12.center);
		\draw (16) to (14.center);
	\end{pgfonlayer}
\end{tikzpicture}
$$

\end{definition}


\begin{lemma} \cite[Thm. 10]{lafont}
$\f_2$ is a presentation for the prop $(\FSets_2,\times)$.
\end{lemma}


Therefore in some sense, we are justified in thinking of this prop $(\FSets_2,\times)$ as a sort of categorification of boolean polynomials.

%\begin{proof}
%The generators have the following interpretations in $\Sets_2$;  $X$ corresponds to addition:
%
%$$
%|a,b\rangle \xmapsto{\llbracket \mu_X \rrbracket} | a+b\rangle
%\hspace*{.5cm}
%* \xmapsto{\llbracket \mu_X \rrbracket} | 0 \rangle
%$$
%
%$\&$ corresponds to multiplication:
%
%$$
%|a,b\rangle \xmapsto{\llbracket \mu_\& \rrbracket} | a\cdot b\rangle
%\hspace*{.5cm}
%* \xmapsto{\llbracket\mu_X \rrbracket} | 1 \rangle
%$$
%
%
%The two bicommutative bialgebra laws correspond to the commutation of multiplication and addition with copying; and the monad map corresponds to the fact that multiplication distributes over addition.  Therefore, this is just the presentation of the ring $\Z_2$ as a Lawvere theory. 
%\end{proof}


To find larger fragments, it will be useful to first identify the isomorphisms and the monics of $\f_2$.


\begin{definition}
Given a map $f$ in  $\f_2$, the {\bf oracle} for $f$, ${\mathcal O}_f$ is defined as follows:
$$
\begin{tikzpicture}
	\begin{pgfonlayer}{nodelayer}
		\node [style=Z] (0) at (0.75, 0.25) {};
		\node [style=X] (1) at (1.5, 2.25) {};
		\node [style=map] (2) at (1, 1.5) {$f$};
		\node [style=none] (3) at (0.5, 2.75) {};
		\node [style=none] (4) at (1.5, 2.75) {};
		\node [style=none] (5) at (1.5, -0.25) {};
		\node [style=none] (6) at (0.75, -0.25) {};
		\node [style=Z] (7) at (-0.25, 0.25) {};
		\node [style=none] (8) at (-0.5, 2.75) {};
		\node [style=none] (9) at (-0.25, -0.25) {};
		\node [style=none] (10) at (0.25, 0) {$\cdots$};
		\node [style=none] (11) at (0, 2.5) {$\cdots$};
	\end{pgfonlayer}
	\begin{pgfonlayer}{edgelayer}
		\draw (6.center) to (0);
		\draw [in=-60, out=60] (0) to (2);
		\draw [in=-120, out=90] (2) to (1);
		\draw (1) to (4.center);
		\draw [in=90, out=-60] (1) to (5.center);
		\draw [in=-90, out=120] (0) to (3.center);
		\draw (9.center) to (7);
		\draw [in=-90, out=120] (7) to (8.center);
		\draw [in=45, out=-120] (2) to (7);
	\end{pgfonlayer}
\end{tikzpicture}
$$

\end{definition}

\begin{lemma}
The oracles in $f_2$ are generated by the generalized controlled-not gates:
$$
\begin{tikzpicture}
	\begin{pgfonlayer}{nodelayer}
		\node [style=none] (0) at (1, -0.75) {};
		\node [style=X] (1) at (0.5, -0.75) {$1$};
		\node [style=none] (2) at (0.75, 0.75) {};
		\node [style=X] (3) at (0.75, 0) {};
		\node [style=none] (4) at (1, -1.25) {};
	\end{pgfonlayer}
	\begin{pgfonlayer}{edgelayer}
		\draw [in=-45, out=90, looseness=0.75] (0.center) to (3);
		\draw [in=90, out=-135, looseness=0.75] (3) to (1);
		\draw (3) to (2.center);
		\draw (4.center) to (0.center);
	\end{pgfonlayer}
\end{tikzpicture},
\hspace*{.5cm}
\begin{tikzpicture}[xscale=-1]
	\begin{pgfonlayer}{nodelayer}
		\node [style=X] (0) at (0.75, 0) {};
		\node [style=Z] (1) at (1.25, -0.5) {};
		\node [style=none] (2) at (0.5, -1) {};
		\node [style=none] (3) at (1.25, -1) {};
		\node [style=none] (4) at (1.5, 0.5) {};
		\node [style=none] (5) at (0.75, 0.5) {};
	\end{pgfonlayer}
	\begin{pgfonlayer}{edgelayer}
		\draw (5.center) to (0);
		\draw [in=150, out=-30] (0) to (1);
		\draw [in=-90, out=60, looseness=0.75] (1) to (4.center);
		\draw (1) to (3.center);
		\draw [in=90, out=-120, looseness=0.75] (0) to (2.center);
	\end{pgfonlayer}
\end{tikzpicture},
\hspace*{.5cm}
\begin{tikzpicture}
	\begin{pgfonlayer}{nodelayer}
		\node [style=Z] (0) at (-10.25, 0.25) {};
		\node [style=Z] (1) at (-11.25, 0.25) {};
		\node [style=none] (2) at (-10.75, 1) {};
		\node [style=X] (3) at (-9.75, 1.75) {};
		\node [style=none] (4) at (-11.25, -0.5) {};
		\node [style=none] (5) at (-10.25, -0.5) {};
		\node [style=none] (6) at (-9.75, -0.5) {};
		\node [style=none] (7) at (-9.75, 2.25) {};
		\node [style=none] (8) at (-10.25, 2.25) {};
		\node [style=none] (9) at (-11.25, 2.25) {};
		\node [style=andin] (10) at (-10.75, 1) {};
		\node [style=none] (11) at (-10.75, 2.25) {$n$};
		\node [style=none] (12) at (-10.75, 0.25) {$n$};
		\node [style=none] (13) at (-10.75, 2) {$\cdots$};
		\node [style=none] (14) at (-10.75, 0.5) {$\cdots$};
	\end{pgfonlayer}
	\begin{pgfonlayer}{edgelayer}
		\draw (4.center) to (1);
		\draw (1) to (2.center);
		\draw (2.center) to (0);
		\draw (0) to (5.center);
		\draw (6.center) to (3);
		\draw [in=90, out=-146, looseness=1.50] (3) to (2.center);
		\draw [in=-90, out=120, looseness=1.00] (1) to (9.center);
		\draw [in=-90, out=60, looseness=0.75] (0) to (8.center);
		\draw (3) to (7.center);
	\end{pgfonlayer}
\end{tikzpicture}
$$




%and the equations of Iwama et al, where $[n,X]$ denotes an $|X|$-controlled not gate controlled by the wires indexed by the set X, and targetting the wire $n \notin X$  \cite{iwama} generalized b

%https://web.eecs.umich.edu/~imarkov/pubs/jour/tcad03-iwls.pdf

%\begin{description}
%\item $[x,X][x,X]=1$
%\item  When the target wire are the same $[x,X][x,Y] = [x,Y] [x,X]$
%\item  When $x \not\in Y$ and $y \not\in X$ then $[x,X][y,Y] = [y,Y] [x,X]$
%\item  $[x,X] [y,{\{ x\} \sqcup Y}] = [y,{X \cup Y}][y,{\{ x\} \sqcup Y}] [x,X]$
%\item For $x \neq y$, $[x,] $
%\end{description}
%
%
%TODO


\end{lemma}


\begin{proof}
Any  Boolean function of $n$ arguments can be represented by a polynomial in\\
 $\F_2[x_1,\ldots, x_n]/\langle x_1^2-x_1,\ldots x_1^2-x_1\rangle$.  Every polynomial in this quotient ring has a unique normal form given by sums of products (which is not true for arbitrary finite fields).  Each product corresponds to a generalized controlled-not gate, and the sum corresponds to composing these generalized controlled-not gates in sequence.
\end{proof}

In the quantum circuit notation the generalized controlled-not gates are drawn as follows (the first being the not gate, and the second being the controlled-not gate):
$$
\begin{tikzpicture}
	\begin{pgfonlayer}{nodelayer}
		\node [style=oplus] (0) at (0, 1.5) {};
		\node [style=none] (1) at (0, 2) {};
		\node [style=none] (2) at (0, 1) {};
	\end{pgfonlayer}
	\begin{pgfonlayer}{edgelayer}
		\draw (0) to (1.center);
		\draw (0) to (2.center);
	\end{pgfonlayer}
\end{tikzpicture}
,
\hspace*{.5cm}
\begin{tikzpicture}
	\begin{pgfonlayer}{nodelayer}
		\node [style=oplus] (0) at (0, 1.5) {};
		\node [style=none] (1) at (0, 2) {};
		\node [style=none] (2) at (0, 1) {};
		\node [style=none] (4) at (-0.5, 2) {};
		\node [style=none] (5) at (-0.5, 1) {};
		\node [style=dot] (6) at (-0.5, 1.5) {};
	\end{pgfonlayer}
	\begin{pgfonlayer}{edgelayer}
		\draw (0) to (1.center);
		\draw (0) to (2.center);
		\draw (0) to (6);
		\draw (6) to (4.center);
		\draw (6) to (5.center);
	\end{pgfonlayer}
\end{tikzpicture}
,
\hspace*{.5cm}
\begin{tikzpicture}
	\begin{pgfonlayer}{nodelayer}
		\node [style=oplus] (0) at (-0.25, 1.5) {};
		\node [style=none] (1) at (-0.25, 2) {};
		\node [style=none] (2) at (-0.25, 1) {};
		\node [style=none] (4) at (-0.75, 2) {};
		\node [style=none] (5) at (-0.75, 1) {};
		\node [style=dot] (6) at (-0.75, 1.5) {};
		\node [style=none] (7) at (-1.75, 2) {};
		\node [style=none] (8) at (-1.75, 1) {};
		\node [style=dot] (9) at (-1.75, 1.5) {};
		\node [style=none] (10) at (-1, 1.5) {};
		\node [style=none] (11) at (-1.5, 1.5) {};
		\node [style=none] (12) at (-1.25, 1.5) {$\cdots$};
		\node [style=none] (13) at (-1.25, 1.75) {$n$};
	\end{pgfonlayer}
	\begin{pgfonlayer}{edgelayer}
		\draw (0) to (1.center);
		\draw (0) to (2.center);
		\draw (0) to (6);
		\draw (6) to (4.center);
		\draw (6) to (5.center);
		\draw (9) to (7.center);
		\draw (9) to (8.center);
		\draw (11.center) to (9);
		\draw (10.center) to (6);
	\end{pgfonlayer}
\end{tikzpicture}
$$


%We will also allow generalized controlled not gates controlled and targetting arbitrary wires, possibly with gaps in the middle.




\begin{lemma}\cite[Thm. 5.1]{toffolireversible}
The prop generated by the oracles in $\f_2$ generate $\Iso(\f_2)$.
\end{lemma}

%This actually follows from https://arxiv.org/pdf/quant-ph/0207001.pdf
%Scratch space is used to construct n-bit cnot gate
%http://theory.caltech.edu/~preskill/ph229/notes/chap6.pdf

Denote a generalized controlled not gate controlled by wires indexed by $X$, operating on $x$ by $\lbparen X,x\rbparen$


%
%
%Iwama et al,  give a set of identities which are complete for oracles, not general isomorphisms \cite{iwama}.
%Shende restated these identities using the commutator \cite{shende}.






Iwama et al \cite{iwama} originally gave a complete set of identities for circuits generated by generalized controlled not gates where the value of all-but-one output wires are fixed.  It is worth mentioning that Shende et al. later used the commutator to generalize some of these identities \cite[Cor. 26]{shende}.  We conjecture that a very similar set of identities is complete for Boolean isomorphisms: 
%Recall that the {\bf commutator} of two group elements $f,g$ is the element $[f,g]:=fgf^{-1}g^{-1}$; therefore $fg=[f,g]gf$.


%Shende gives a way to compute commutators of controlled isomorphsisms in $\Iso(\FSets_2)$.  In particular, take isomorphisms $f,g$  which only change bits in sets indexed by $X,Y$, respectively.  Then if we control these gates by $Z$ and $W$ respectively, denoted by $f^Z,g^W$, we have :
%$$
%[f,g] = [f^{Y * Z}, g^{X*W}]^{\bar{(X\#Y)} * (W\#Z) }
%$$
%Where $X\# Y $ is the defined as the pointwise xor, and $X * Y$ is defined as the pointwise and.

\begin{conjecture}
%Denote a generalized controlled-not gate controlled from wires indexed by $X$ and operating on the wire $x$ by $\lbparen X,x \rbparen$.

Let  $\Iso(\FSets_2)$ denote the prop generated by all generalized controlled-not gates modulo the following identities:

\begin{itemize}
\item $\lbparen X,x\rbparen \lbparen X,x \rbparen= 1$ 
%So the first identity is that  in this case $\lbparen X,x\rbparen\lbparen Y,y\rbparen=\lbparen Y,y\rbparen\lbparen X,x\rbparen$


\item
If  $x \notin Y $ and $ y \notin X$ then $\lbparen X,x\rbparen \lbparen Y,y\rbparen =\lbparen Y,y\rbparen \lbparen X,x\rbparen $.


%
%So the second identity is that in this case:
%$$
%\lbparen  X,x\rbparen  \lbparen  Y ,y\rbparen = \lbparen \lbparen  X \cup Y -y,x\rbparen \rbparen  \lbparen  Y ,y\rbparen  \lbparen  X,x\rbparen  
%$$

\item
If $x \notin Y$, then $\lbparen X,x\rbparen \lbparen \{x\} \sqcup Y, y\rbparen = \lbparen X\cup Y,y\rbparen  \lbparen \{x\} \sqcup Y, y\rbparen  \lbparen X,x\rbparen $.

\item
If $x \notin Y$, then $ \lbparen \{x\} \sqcup Y, y\rbparen \lbparen X,x\rbparen = \lbparen X,x\rbparen   \lbparen \{x\} \sqcup Y, y\rbparen  \lbparen X\cup Y,y\rbparen $.


%So the third identity is that $\lbparen  X,x\rbparen ^2= 1 $.

\item
If $x \in Y$ and $y \in X$, then
$
\lbparen  X,x \rbparen  \lbparen  Y,y \rbparen   \lbparen  X,x \rbparen 
=
\lbparen  Y,y \rbparen  \lbparen  X,x \rbparen   \lbparen  Y,y \rbparen 
$.



\end{itemize}

\end{conjecture}



Note that the symmetry is derived in this fragment by composing 3 controlled not gates, as in Definition \label{def:isoaff}.  The axioms of a prop are derived, so we are justified in calling $\Iso(\f_2)$ a prop.


Although we aren't sure if these identities are complete, it doesn't matter in the end.  With each generator we add, we add new enough identities to give a complete presentation, given that there is a complete presentation for $\Iso(\f_2)$.  However, eventually once we add enough generators and identities, we get a finite, complete presentation.

\begin{definition}
Let $\inj(\f_2)$ be the prop given by adjoining the black unit to $\Iso(\f_2)$ modulo:

$$
\begin{tikzpicture}
	\begin{pgfonlayer}{nodelayer}
		\node [style=oplus] (0) at (2, 1.5) {};
		\node [style=none] (1) at (2, 2.25) {};
		\node [style=none] (2) at (2, 0.75) {};
		\node [style=none] (3) at (1.5, 2.25) {};
		\node [style=none] (4) at (1.5, 0.75) {};
		\node [style=dot] (5) at (1.5, 1.5) {};
		\node [style=none] (6) at (0.5, 2.25) {};
		\node [style=dot] (7) at (0.5, 1.5) {};
		\node [style=none] (8) at (1.25, 1.5) {};
		\node [style=none] (9) at (0.75, 1.5) {};
		\node [style=none] (10) at (1, 1.5) {$\cdots$};
		\node [style=none] (11) at (1, 1.75) {$n$};
		\node [style=none] (13) at (0, 2.25) {};
		\node [style=dot] (14) at (0, 1.5) {};
		\node [style=X] (15) at (0, 1) {};
		\node [style=none] (16) at (0.5, 0.75) {};
	\end{pgfonlayer}
	\begin{pgfonlayer}{edgelayer}
		\draw (0) to (1.center);
		\draw (0) to (2.center);
		\draw (0) to (5);
		\draw (5) to (3.center);
		\draw (5) to (4.center);
		\draw (7) to (6.center);
		\draw (9.center) to (7);
		\draw (8.center) to (5);
		\draw (14) to (13.center);
		\draw (15) to (14);
		\draw (16.center) to (7);
		\draw (7) to (14);
	\end{pgfonlayer}
\end{tikzpicture}
\eqzxa{mono.ftwo}
\begin{tikzpicture}
	\begin{pgfonlayer}{nodelayer}
		\node [style=none] (1) at (2, 2.25) {};
		\node [style=none] (2) at (2, 0.75) {};
		\node [style=none] (3) at (1.5, 2.25) {};
		\node [style=none] (4) at (1.5, 0.75) {};
		\node [style=none] (6) at (0.5, 2.25) {};
		\node [style=none] (10) at (1, 1.5) {$\cdots$};
		\node [style=none] (11) at (1, 1.75) {$n$};
		\node [style=none] (13) at (0, 2.25) {};
		\node [style=X] (15) at (0, 1) {};
		\node [style=none] (16) at (0.5, 0.75) {};
	\end{pgfonlayer}
	\begin{pgfonlayer}{edgelayer}
		\draw (16.center) to (6.center);
		\draw (13.center) to (15);
		\draw (4.center) to (3.center);
		\draw (1.center) to (2.center);
	\end{pgfonlayer}
\end{tikzpicture}
$$
\end{definition}

\begin{lemma}
\label{lem:injand}
$\inj(\f_2)$ is a presentation for the prop $(\inj(\FSets_2),\times)$.
\end{lemma}

%Note, however, that the monoidal theory for $\inj(\f_2)$ is dependant on that of $\iso(\f_2)$ the identities of which are conjectured, even though we only need one extra identity to get injections.

The pushout of a diagram of sets and functions $2^n \xleftarrowtail{} 2^k \xrightarrowtail{} 2^m$ is not always a power of 2.  Therefore, one should not expect to construct categories of partial isomorphisms via a distributive law of on $\inj(\f_2)\otimes_{\Iso(\f_2)} \inj(\f_2)^\op$. Instead one must add all of the nontrivial subobjects to the constituent props forming the distributive law; as opposed to the affine case, there are more than one such subobjects which arise in this way.

\begin{definition}
Consider the pro $\sub_2$ generated by endomorphisms such that for any $n$, $\sub_2(n,n)$ is the set described by all $n$-variable polynomials over $\F_2$.  Denote such a generator by a box with $n$ inputs and $n$ outputs labelled by the corresponding polynomial.

We require that the following equations hold so that
$$\forall n,m \in \N, p,r \in \F_2[x_1,\ldots, x_n],  q \in \F_2[x_{n+1},\ldots, x_{n+m}]:\hspace*{1cm}
$$
$$
\begin{tikzpicture}
	\begin{pgfonlayer}{nodelayer}
		\node [style=none] (0) at (3, 2.25) {};
		\node [style=none] (1) at (3, 3.25) {};
		\node [style=map] (2) at (3, 2.75) {$1$};
		\node [style=none] (6) at (3, 3.5) {$n$};
		\node [style=none] (8) at (3, 2) {$n$};
	\end{pgfonlayer}
	\begin{pgfonlayer}{edgelayer}
		\draw (0.center) to (1.center);
	\end{pgfonlayer}
\end{tikzpicture}
\eqzxa{sub.one}
\begin{tikzpicture}
	\begin{pgfonlayer}{nodelayer}
		\node [style=none] (0) at (3, 2.25) {};
		\node [style=none] (1) at (3, 3.25) {};
		\node [style=none] (6) at (3, 3.5) {$n$};
		\node [style=none] (8) at (3, 2) {$n$};
	\end{pgfonlayer}
	\begin{pgfonlayer}{edgelayer}
		\draw (0.center) to (1.center);
	\end{pgfonlayer}
\end{tikzpicture}
\hspace*{,5cm}
\begin{tikzpicture}
	\begin{pgfonlayer}{nodelayer}
		\node [style=none] (0) at (5, 2) {};
		\node [style=none] (1) at (5, 4) {};
		\node [style=map] (2) at (5, 2.5) {$r$};
		\node [style=map] (3) at (5, 3.5) {$p$};
		\node [style=none] (4) at (5, 4.25) {$n$};
		\node [style=none] (5) at (5, 1.75) {$n$};
	\end{pgfonlayer}
	\begin{pgfonlayer}{edgelayer}
		\draw (0.center) to (1.center);
	\end{pgfonlayer}
\end{tikzpicture}
\eqzxa{sub.two}
\begin{tikzpicture}
	\begin{pgfonlayer}{nodelayer}
		\node [style=none] (0) at (5, 2.25) {};
		\node [style=none] (1) at (5, 4.25) {};
		\node [style=map] (2) at (5, 3.25) {$p+r+pr$};
		\node [style=none] (4) at (5, 4.5) {$n$};
		\node [style=none] (5) at (5, 2) {$n$};
	\end{pgfonlayer}
	\begin{pgfonlayer}{edgelayer}
		\draw (0.center) to (1.center);
	\end{pgfonlayer}
\end{tikzpicture}
\hspace*{.5cm}
\begin{tikzpicture}
	\begin{pgfonlayer}{nodelayer}
		\node [style=none] (0) at (2.4, 2.25) {};
		\node [style=none] (1) at (2.4, 3.25) {};
		\node [style=map] (2) at (2.4, 2.75) {$p$};
		\node [style=none] (3) at (3, 3.25) {};
		\node [style=none] (4) at (3, 2.25) {};
		\node [style=map] (5) at (3, 2.75) {$q$};
		\node [style=none] (6) at (2.4, 3.5) {$n$};
		\node [style=none] (7) at (3, 3.5) {$m$};
		\node [style=none] (8) at (2.4, 2) {$n$};
		\node [style=none] (9) at (3, 2) {$m$};
	\end{pgfonlayer}
	\begin{pgfonlayer}{edgelayer}
		\draw (0.center) to (1.center);
		\draw (3.center) to (4.center);
	\end{pgfonlayer}
\end{tikzpicture}
\eqzxa{sub.three}
\begin{tikzpicture}
	\begin{pgfonlayer}{nodelayer}
		\node [style=none] (0) at (2.5, 2.25) {};
		\node [style=none] (1) at (2.5, 3.25) {};
		\node [style=map] (2) at (2.75, 2.75) {$p\cdot q$};
		\node [style=none] (3) at (3, 3.25) {};
		\node [style=none] (4) at (3, 2.25) {};
		\node [style=none] (6) at (2.5, 3.5) {$n$};
		\node [style=none] (7) at (3, 3.5) {$m$};
		\node [style=none] (8) at (2.5, 2) {$n$};
		\node [style=none] (9) at (3, 2) {$m$};
	\end{pgfonlayer}
	\begin{pgfonlayer}{edgelayer}
		\draw (0.center) to (1.center);
		\draw (3.center) to (4.center);
	\end{pgfonlayer}
\end{tikzpicture}
$$
As well as, for all $n$, the equations of the quotient rings  $\F_2[x_1,\ldots, x_n]/\langle x_1^2-x_1,\ldots, x_n^2-x_n \rangle$.

\end{definition}


\begin{lemma}
\label{lem:sub}
$\sub_2$ is a presentation for the pro of symmetric spans of monic functions, ie spans of the following form $2^n \xleftarrow{e} k \xrightarrow{e}2^n$, for all $n,k \in \N$ and monics $e$.
\end{lemma}



\begin{proof}
Each polynomial  $p \in \F_2[x_1,\ldots, x_n]/\langle x_1^2-x_1,\ldots, x_n^2-x_n \rangle$ corresponds to a function $\ev_p:\Z_2^n \to \Z_2$ given by evaluation.  Let $k = |\ev^{-1}(1)|$, then there chose a function $f_p:k \rightarrowtail 2^n$ picking out all the solutions which evaluate to $1$. The functor from $\sub_2$ to this subcategory spans takes polynomials $p \mapsto (2^n \xleftarrowtail {f_p} k \xrightarrowtail {f_p} 2^n)$.  Any two two spans induced by the same polynomial are isomorphic, so this is actually well defined.  It is clearly an isomorphism on objects, and it can easily be shown to be a monoidal functor.

The fullness is easy and the faithfulness comes from the fact that we can reduce every map to a polynomial and then reduce the polynomial to algebraic normal form.

\end{proof}


%
%The first axiom enforces that the trivial polynomial is the identity.
%The second axiom allows one to perform elementary row operations on polynomials.
%The third axiom allows one to multiply all polynomials.
%Because every map $f$ in $\sub_2(n,n)$ has precisely one representative polynomial, and polynomials have a normal form via their reduction, and row reduction is confluent, completeness is immediate.




\begin{definition}
Let $\sub\Iso\f_2$ be the prop generated by a distributive law of pros:
$$
\sub_2 \otimes \Iso(\f_2);
$$
$$
 \forall n,m,k \in \N, \forall p \in \F_2[x_1,\ldots, x_{n+2+m}],
q \in \F_2[x_1,\ldots,x_{n+m+1+k}],
r \in \F_2[x_1,\ldots, x_n]:
$$
$$
\begin{tikzpicture}
	\begin{pgfonlayer}{nodelayer}
		\node [style=map] (0) at (5, 3.5) {$p(x_1,\ldots, x_n, x_{n+1}, x_{n+2}, x_{n+3},\ldots, x_{n+2+m})$};
		\node [style=none] (1) at (4.5, 4.25) {};
		\node [style=none] (2) at (5.5, 4.25) {};
		\node [style=none] (3) at (4.75, 4.25) {};
		\node [style=none] (4) at (5.25, 4.25) {};
		\node [style=none] (5) at (4.5, 2.25) {};
		\node [style=none] (6) at (5.5, 2.25) {};
		\node [style=none] (7) at (4.75, 2.75) {};
		\node [style=none] (8) at (5.25, 2.75) {};
		\node [style=none] (9) at (4.5, 4.5) {$n$};
		\node [style=none] (10) at (4.5, 2) {$n$};
		\node [style=none] (11) at (5.5, 4.5) {$m$};
		\node [style=none] (12) at (5.5, 2) {$m$};
		\node [style=none] (13) at (5.25, 2.25) {};
		\node [style=none] (14) at (4.75, 2.25) {};
	\end{pgfonlayer}
	\begin{pgfonlayer}{edgelayer}
		\draw [in=120, out=-90] (1.center) to (0);
		\draw [in=-90, out=60] (0) to (2.center);
		\draw [in=75, out=-90, looseness=0.75] (4.center) to (0);
		\draw [in=-90, out=105, looseness=0.75] (0) to (3.center);
		\draw [in=300, out=90] (6.center) to (0);
		\draw [in=90, out=-75, looseness=0.75] (0) to (8.center);
		\draw [in=255, out=90, looseness=0.75] (7.center) to (0);
		\draw [in=90, out=-120] (0) to (5.center);
		\draw [in=270, out=90] (13.center) to (7.center);
		\draw [in=270, out=90] (14.center) to (8.center);
	\end{pgfonlayer}
\end{tikzpicture}
\eqzxa{subiso.one}
\begin{tikzpicture}
	\begin{pgfonlayer}{nodelayer}
		\node [style=map] (0) at (5, 3.25) {$p(x_1,\ldots, x_n, x_{n+2}, x_{n+1}, x_{n+3},\ldots, x_{n+2+m})$};
		\node [style=none] (1) at (4.5, 2.5) {};
		\node [style=none] (2) at (5.5, 2.5) {};
		\node [style=none] (3) at (4.75, 2.5) {};
		\node [style=none] (4) at (5.25, 2.5) {};
		\node [style=none] (5) at (4.5, 4.5) {};
		\node [style=none] (6) at (5.5, 4.5) {};
		\node [style=none] (7) at (4.75, 4) {};
		\node [style=none] (8) at (5.25, 4) {};
		\node [style=none] (9) at (4.5, 2.25) {$n$};
		\node [style=none] (10) at (4.5, 4.75) {$n$};
		\node [style=none] (11) at (5.5, 2.25) {$m$};
		\node [style=none] (12) at (5.5, 4.75) {$m$};
		\node [style=none] (13) at (5.25, 4.5) {};
		\node [style=none] (14) at (4.75, 4.5) {};
	\end{pgfonlayer}
	\begin{pgfonlayer}{edgelayer}
		\draw [in=-120, out=90] (1.center) to (0);
		\draw [in=90, out=-60] (0) to (2.center);
		\draw [in=-75, out=90, looseness=0.75] (4.center) to (0);
		\draw [in=90, out=-105, looseness=0.75] (0) to (3.center);
		\draw [in=-300, out=-90] (6.center) to (0);
		\draw [in=-90, out=75, looseness=0.75] (0) to (8.center);
		\draw [in=-255, out=-90, looseness=0.75] (7.center) to (0);
		\draw [in=-90, out=120] (0) to (5.center);
		\draw [in=-270, out=-90] (13.center) to (7.center);
		\draw [in=-270, out=-90] (14.center) to (8.center);
	\end{pgfonlayer}
\end{tikzpicture}
$$
$$
\begin{tikzpicture}
	\begin{pgfonlayer}{nodelayer}
		\node [style=map] (0) at (5, 3.5) {$q(x_1,\ldots, x_{n+m+1+k})$};
		\node [style=none] (1) at (3.75, 4) {};
		\node [style=none] (2) at (6.25, 4) {};
		\node [style=none] (3) at (4.25, 4) {};
		\node [style=none] (4) at (5.75, 4) {};
		\node [style=none] (5) at (3.75, 3) {};
		\node [style=none] (6) at (6.25, 3) {};
		\node [style=none] (7) at (4.25, 3) {};
		\node [style=none] (8) at (5.75, 3) {};
		\node [style=none] (9) at (3.75, 2.25) {$n$};
		\node [style=none] (10) at (4.25, 2.5) {};
		\node [style=none] (11) at (5.75, 2.5) {};
		\node [style=dot] (12) at (4.25, 2.75) {};
		\node [style=oplus] (13) at (5.75, 2.75) {};
		\node [style=none] (14) at (3.75, 2.5) {};
		\node [style=none] (15) at (6.25, 2.5) {};
		\node [style=none] (16) at (5.25, 4) {};
		\node [style=dot] (17) at (5.25, 2.75) {};
		\node [style=none] (18) at (5.25, 2.5) {};
		\node [style=none] (19) at (6.25, 2.25) {$k$};
		\node [style=none] (20) at (4.75, 2.5) {$m$};
		\node [style=none] (21) at (3.75, 5) {$n$};
		\node [style=none] (22) at (6.25, 5) {$k$};
		\node [style=none] (23) at (4.75, 4) {$m$};
		\node [style=none] (24) at (4.25, 4.5) {};
		\node [style=none] (25) at (5.75, 4.5) {};
		\node [style=none] (26) at (3.75, 4.5) {};
		\node [style=none] (27) at (6.25, 4.5) {};
		\node [style=none] (28) at (5.25, 4.5) {};
	\end{pgfonlayer}
	\begin{pgfonlayer}{edgelayer}
		\draw [in=270, out=90] (10.center) to (7.center);
		\draw [in=270, out=90] (11.center) to (8.center);
		\draw (15.center) to (6.center);
		\draw (14.center) to (5.center);
		\draw (13) to (17);
		\draw [style=dotted] (17) to (12);
		\draw (18.center) to (17);
		\draw (1.center) to (26.center);
		\draw (3.center) to (24.center);
		\draw (16.center) to (28.center);
		\draw (4.center) to (25.center);
		\draw (2.center) to (27.center);
		\draw (1.center) to (5.center);
		\draw (12) to (3.center);
		\draw (16.center) to (17);
		\draw (13) to (4.center);
		\draw (2.center) to (6.center);
	\end{pgfonlayer}
\end{tikzpicture}
\eqzxa{subiso.two}
\begin{tikzpicture}
	\begin{pgfonlayer}{nodelayer}
		\node [style=map] (0) at (5, 3) {$q(x_1,\ldots, x_{n+m}, (x_{n+1}\ldots x_{n+m-1})+x_{n+m+1}, x_{n+m+2}, \ldots, x_{n+m+1+k})$};
		\node [style=none] (1) at (3.75, 2.5) {};
		\node [style=none] (2) at (6.25, 2.5) {};
		\node [style=none] (3) at (4.25, 2.5) {};
		\node [style=none] (4) at (5.75, 2.5) {};
		\node [style=none] (5) at (3.75, 3.5) {};
		\node [style=none] (6) at (6.25, 3.5) {};
		\node [style=none] (7) at (4.25, 3.5) {};
		\node [style=none] (8) at (5.75, 3.5) {};
		\node [style=none] (9) at (3.75, 4.25) {$n$};
		\node [style=none] (10) at (4.25, 4) {};
		\node [style=none] (11) at (5.75, 4) {};
		\node [style=dot] (12) at (4.25, 3.75) {};
		\node [style=oplus] (13) at (5.75, 3.75) {};
		\node [style=none] (14) at (3.75, 4) {};
		\node [style=none] (15) at (6.25, 4) {};
		\node [style=none] (16) at (5.25, 2.5) {};
		\node [style=dot] (17) at (5.25, 3.75) {};
		\node [style=none] (18) at (5.25, 4) {};
		\node [style=none] (19) at (6.25, 4.25) {$k$};
		\node [style=none] (20) at (4.75, 4) {$m$};
		\node [style=none] (21) at (3.75, 1.75) {$n$};
		\node [style=none] (22) at (6.25, 1.75) {$k$};
		\node [style=none] (23) at (4.75, 2.5) {$m$};
		\node [style=none] (24) at (4.25, 2) {};
		\node [style=none] (25) at (5.75, 2) {};
		\node [style=none] (26) at (3.75, 2) {};
		\node [style=none] (27) at (6.25, 2) {};
		\node [style=none] (28) at (5.25, 2) {};
	\end{pgfonlayer}
	\begin{pgfonlayer}{edgelayer}
		\draw [in=-270, out=-90] (10.center) to (7.center);
		\draw [in=-270, out=-90] (11.center) to (8.center);
		\draw (15.center) to (6.center);
		\draw (14.center) to (5.center);
		\draw (13) to (17);
		\draw [style=dotted] (17) to (12);
		\draw (18.center) to (17);
		\draw (26.center) to (1.center);
		\draw (24.center) to (3.center);
		\draw (28.center) to (16.center);
		\draw (25.center) to (4.center);
		\draw (27.center) to (2.center);
		\draw (1.center) to (5.center);
		\draw (3.center) to (12);
		\draw (16.center) to (17);
		\draw (4.center) to (13);
		\draw (2.center) to (6.center);
	\end{pgfonlayer}
\end{tikzpicture}\hspace*{.5cm}
\begin{tikzpicture}
	\begin{pgfonlayer}{nodelayer}
		\node [style=none] (5) at (3.25, 4.25) {};
		\node [style=none] (6) at (3.25, 3.25) {};
		\node [style=map] (9) at (3.25, 3.75) {$r$};
		\node [style=none] (10) at (3.25, 2.25) {};
		\node [style=none] (11) at (3.75, 4.25) {};
		\node [style=none] (12) at (3.75, 2.25) {};
		\node [style=none] (13) at (3.75, 3.25) {};
	\end{pgfonlayer}
	\begin{pgfonlayer}{edgelayer}
		\draw (5.center) to (6.center);
		\draw [in=90, out=-90] (13.center) to (10.center);
		\draw [in=-90, out=90] (12.center) to (6.center);
		\draw (13.center) to (11.center);
	\end{pgfonlayer}
\end{tikzpicture}
\eqzxa{subiso.three}
\begin{tikzpicture}
	\begin{pgfonlayer}{nodelayer}
		\node [style=none] (5) at (3.75, 2.25) {};
		\node [style=none] (6) at (3.75, 3.25) {};
		\node [style=map] (9) at (3.75, 2.75) {$r$};
		\node [style=none] (10) at (3.75, 4.25) {};
		\node [style=none] (11) at (3.25, 2.25) {};
		\node [style=none] (12) at (3.25, 4.25) {};
		\node [style=none] (13) at (3.25, 3.25) {};
	\end{pgfonlayer}
	\begin{pgfonlayer}{edgelayer}
		\draw (5.center) to (6.center);
		\draw [in=270, out=90] (13.center) to (10.center);
		\draw [in=90, out=-90] (12.center) to (6.center);
		\draw (13.center) to (11.center);
	\end{pgfonlayer}
\end{tikzpicture}
$$


\end{definition}

\begin{lemma}
\label{lem:subiso}
$\sub\Iso\f_2$ is a presentation for the subcategory of $(\Span(\FSets),\times)$ generated by spans of the form $2^n \xleftarrowtail{e} k \xrightarrowtail {e} 2^m \xrightarrow[\cong]{f} 2^m$, for all $n,m k \in \N$ and all isomorphisms $f$ and monics $e$.
\end{lemma}


\begin{proof}
The obvious functor is clearly monoidal. Moreover, it is full by construction.
For the faithfulness, take two maps $f$ and $g$ in $\sub\Iso\f_2$.  Then one can just push everything to the end and then use the decidability of equality on both factors of the distributive law to show that they are equal.
\end{proof}

% $2^n \xleftarrow{e} k \xrightarrow{e} 2^n$, for all monics $e$ and isomorphisms of the form $2^n = 2^n \xrightarrow{\sim} 2^n$. 




\begin{definition}
Consider the prop $\sub\inj\f_2$ generated by a distributive law of props:
$$
 \sub\Iso\f_2 \otimes_{\Iso(\f_2)} \inj(\f_2);
 \forall n,m \in \N, p \in \F_2[x_1,\ldots, x_{n+1+m}]:
$$
$$
\begin{tikzpicture}
	\begin{pgfonlayer}{nodelayer}
		\node [style=none] (0) at (1.5, 3.5) {};
		\node [style=map] (1) at (2.5, 2.75) {$p(x_1,\ldots,x_{n+1+m})$};
		\node [style=none] (2) at (1.5, 3.75) {$n$};
		\node [style=none] (3) at (3.5, 3.5) {};
		\node [style=none] (4) at (3.5, 3.75) {$m$};
		\node [style=none] (5) at (1.5, 1.75) {};
		\node [style=none] (6) at (3.5, 1.75) {};
		\node [style=none] (7) at (1.5, 1.5) {$n$};
		\node [style=none] (8) at (3.5, 1.5) {$m$};
		\node [style=X] (9) at (2.5, 2) {};
		\node [style=none] (10) at (2.5, 3.5) {};
	\end{pgfonlayer}
	\begin{pgfonlayer}{edgelayer}
		\draw (0.center) to (5.center);
		\draw (3.center) to (6.center);
		\draw (10.center) to (9);
	\end{pgfonlayer}
\end{tikzpicture}
\eqzxa{subinj}
\begin{tikzpicture}
	\begin{pgfonlayer}{nodelayer}
		\node [style=none] (0) at (1.5, 3.75) {};
		\node [style=map] (1) at (2.5, 2.75) {$p(x_1,\ldots,x_n,0,x_{n+2},\ldots,x_{n+1+m})$};
		\node [style=none] (2) at (1.5, 4) {$n$};
		\node [style=none] (3) at (3.5, 3.75) {};
		\node [style=none] (4) at (3.5, 4) {$m$};
		\node [style=none] (5) at (1.5, 2) {};
		\node [style=none] (6) at (3.5, 2) {};
		\node [style=none] (7) at (1.5, 1.75) {$n$};
		\node [style=none] (8) at (3.5, 1.75) {$m$};
		\node [style=X] (9) at (2.5, 3.4) {};
		\node [style=none] (10) at (2.5, 3.75) {};
	\end{pgfonlayer}
	\begin{pgfonlayer}{edgelayer}
		\draw (0.center) to (5.center);
		\draw (3.center) to (6.center);
		\draw (10.center) to (9);
	\end{pgfonlayer}
\end{tikzpicture}
$$
\end{definition}



\begin{lemma}
\label{lem:subinj}
$\sub\inj\f_2$ is a presentation for the subcategory of $(\Span(\FSets),\times)$ generated by spans of the form $2^n \xleftarrowtail{e} k \xrightarrowtail{e} 2^n \xrightarrowtail{e'} 2^{m}$ for all $n,m,k \in \N$ and all monics $e,e'$.
\end{lemma}

The proof is completely analogous to as in the case of $\sub\iso\f_2$.

Any $n$ variable polynomial $p$ can be interpreted as a span of monics via the oracle $\mathcal{O}_p$, where the value of the target wire is restricted to have the value $0$.  Each such polynomial corresponds to a subobject, which complicates the matter further than in the affine case.


\begin{definition}
Consider the prop $\pr\iso\f_2$ given by the distributive law of props:

$$
\sub\inj\f_2^\op \otimes_{\sub\Iso\f_2} \sub\inj\f_2;
\begin{tikzpicture}
	\begin{pgfonlayer}{nodelayer}
		\node [style=map] (0) at (0, 0) {$\mathcal{O}_p$};
		\node [style=none] (1) at (-0.25, -0.75) {};
		\node [style=none] (2) at (-0.25, 0.75) {};
		\node [style=none] (3) at (-0.25, 1) {};
		\node [style=none] (4) at (-0.25, -1) {};
		\node [style=X] (5) at (0.25, 0.75) {};
		\node [style=X] (6) at (0.25, -0.75) {};
	\end{pgfonlayer}
	\begin{pgfonlayer}{edgelayer}
		\draw (3.center) to (2.center);
		\draw (4.center) to (1.center);
		\draw [in=-60, out=90] (6) to (0);
		\draw [in=-90, out=60] (0) to (5);
		\draw [in=120, out=-90] (2.center) to (0);
		\draw [in=90, out=-120, looseness=1.25] (0) to (1.center);
	\end{pgfonlayer}
\end{tikzpicture}
\eqzxa{oracle}
\begin{tikzpicture}
	\begin{pgfonlayer}{nodelayer}
		\node [style=map] (0) at (-0.25, 0) {$p$};
		\node [style=none] (1) at (-0.25, -0.75) {};
		\node [style=none] (2) at (-0.25, 0.75) {};
		\node [style=none] (3) at (-0.25, 1) {};
		\node [style=none] (4) at (-0.25, -1) {};
	\end{pgfonlayer}
	\begin{pgfonlayer}{edgelayer}
		\draw (3.center) to (2.center);
		\draw (4.center) to (1.center);
		\draw (2.center) to (0);
		\draw (0) to (1.center);
	\end{pgfonlayer}
\end{tikzpicture}
$$

\end{definition}

%This is actually a distributive law, because it need only be witnessed by pushing $\eta_X$ past $\eta_X^\op$.  The only time this can't be done is when there is the target  of a generalized controlled-not not gate--or those of several generalized controlled-not gates is in between both of these generators.  In which case, the obstructing generalized controlled-not gates form an oracle for which we can apply this equation.  This is computing the apex of the span when performing a pullback.

%Note that $\pr\iso\f_2$ is not actually partial isomorphisms over $\f_2$, per se, but rather a full subcategory of partial isomorphisms of sets and functions, which is not iself a prop with respect to the Cartesian product.  It is because of such a complication that the aforementioned distributive law isn't quite a pullback.

\begin{lemma}
\label{lem:parisof}
$\pr\iso \f_2$ is a presentation for the full subcategory $(\FPinj_2,\times)$ of $(\ParIso(\FSets),\times)$ with objects powers of two.
\end{lemma}

Unlike the previous lemmas, this is not dependant on a complete presentation for the isomorphisms.
The proof is a consequence of \cite[Thm 7.6.14]{cole} where they give a finite, complete presentation for this category.  The identities up to this point are equivalent to this finite presentation, whether or not the conjectured presentation for the isomorphisms is complete.

%\begin{comment}
$\pr\iso \f_2$ can be presented in terms of finitely many generators and relations.  The identities are contained in \S \ref{subsubsec:presentations:three:pinj}.
%\end{comment}

\begin{definition}

Consider the prop $\pr\f_2$ given by the  pushout of the following diagram of props, given by adding a counit to the diagonal map:
$$\pr\iso\f_2 \leftarrow \surj^\op \rightarrow \cm^\op$$

\end{definition}


\begin{lemma}
\label{lem:parand}
$\pr\f_2$ is a presentation for the the full subcategory $(\FPar_2,\times)$ of $(\Par(\FSets),\times)$ with objects powers of two.
\end{lemma}

\begin{proof}
One has to show that the following diagram commutes:

\renewcommand{\cubetopbl}{$\surj^\op$}
\renewcommand{\cubetopbr}{$\cm^\op$}
\renewcommand{\cubetopfl}{$\pr\iso\f_2$}
\renewcommand{\cubetopfr}{$\pr\f_2$}
\renewcommand{\cubebotbl}{$\surj^\op$ }
\renewcommand{\cubebotbr}{$\cm^\op$ }
\renewcommand{\cubebotfl}{$(\FPinj_2,\times)$ }
\renewcommand{\cubebotfr}{}

$$
\xymatrixrowsep{2mm}\xymatrixcolsep{2mm}
\xymatrix{
                                       & \mbox{\cubetopbl} \ar[rr] \ar[dl] \ar@{=}[dd]     &                                                  & \mbox{\cubetopbr} \ar@{=}[dd] \ar[dl] \\
\mbox{\cubetopfl} \ar[rr]  \ar[dd]_{\cong}           &                                                                                              &\mbox{\cubetopfr} \ar@{-->}[dd]^(.35){\cong}   \skewpullbackcorner[ul]              \\
                                       &  \mbox{\cubebotbl} \ar[dl] \ar[rr]                    &                                                  & \mbox{\cubebotbr} \ar@/^1pc/[ddl] \ar[dl] \\
\mbox{\cubebotfl} \ar@/_1pc/[drr] \ar[rr]  &                                                                                             & \mbox{\cubebotfr} \skewpullbackcorner[ul]    \ar@{-->}[d]^{\cong}  \\
                                                   &                                                                                             & (\FPar_2,\times)
}
$$

Again, the proof is essentially the same as for the linear and affine cases; the only difference being that the Cartesian completion of $\FPinj_2$ is $\FPar_2$.


\end{proof}


%\begin{comment}
There is a particularly elegant finite presentation  contained in \S \ref{subsubsec:presentations:three:par}, which  is much more ZX-flavoured.
%\end{comment}


%
%\begin{corollary}
%Give easier chacterization TODO
%\end{corollary}


\begin{definition}
Let $\sp\f_2$ denote the pushout of the diagram of props:
$$
\pr\f_2^\op\leftarrow \pr\iso\f_2 \rightarrow \pr\f_2
$$
\end{definition}

\begin{lemma}\cite{zxa}
\label{lem:spanand}
$\sp\f_2$ is a presentation for  the full subcategory $(\FSpan_2,\times)$ of $(\Span(\FSets),\times)$ with objects powers of two.
\end{lemma}

\begin{proof}
One has to show that the following diagram commutes:

\renewcommand{\cubetopbl}{$\pr\iso\f_2$}
\renewcommand{\cubetopbr}{$ \pr\f_2$}
\renewcommand{\cubetopfl}{$\pr\f_2^\op$}
\renewcommand{\cubetopfr}{$\sp\f_2$}
\renewcommand{\cubebotbl}{$(\FPinj_2,\times)$ }
\renewcommand{\cubebotbr}{$(\FPar_2,\times)$ }
\renewcommand{\cubebotfl}{$(\FPar_2,\times)^\op$ }
\renewcommand{\cubebotfr}{}

$$
\xymatrixrowsep{2mm}\xymatrixcolsep{0mm}
\xymatrix{
                                       & \mbox{\cubetopbl} \ar[rr] \ar[dl] \ar[dd]^(.7){\cong}      &                                                  & \mbox{\cubetopbr}  \ar[dd]^{\cong} \ar[dl] \\
\mbox{\cubetopfl} \ar[rr]  \ar[dd]_{\cong}           &                                                                                              &\mbox{\cubetopfr} \ar@{-->}[dd]^(.35){\cong}   \skewpullbackcorner[ul]              \\
                                       &  \mbox{\cubebotbl} \ar[dl] \ar[rr]                    &                                                  & \mbox{\cubebotbr} \ar@/^1pc/[ddl] \ar[dl] \\
\mbox{\cubebotfl} \ar@/_1pc/[drr] \ar[rr]  &                                                                                             & \mbox{\cubebotfr} \skewpullbackcorner[ul]    \ar@{-->}[d]^{\cong}  \\
                                                   &                                                                                             &( \FSpan_2,\times)
}
$$


This follows from \cite[Lem. 4.3]{zxa}.

\end{proof}


\begin{remark}
$\sp\f_2$ is isomorphic to $\ZXA$.
\end{remark}






%
%
%That is to say $\Span(\f_2)$ is a presentation for the prop of ``qubit matrices'' over $\N$.  There is an alternative presentation of this category due to \cite{zxa}:
%\begin{corollary}
%$\Span(\f_2)$ can equivalently can be presented in terms of the coproduct of the monoids $X,Z,\&$ and comonoids $Z^\dag,X^\dag$
%\begin{itemize}
%\item An extra-Frobenius algebra between $X^\op$ and $Y$.
%
%\item A special-Frobenius algebra between $Z^\op$ and $Z$.
%
%\item The Lawvere theory for $\F_2^+$ represented by $X$ and $Z^\op$.
%
%\item The Lawvere theory for $\F_2^\times$  represented by  $\&$ and $Z^\op$.
%
%\item A distributive law $L_{X} \otimes_L L_{\&}  \Rightarrow  L_{\&} \otimes_L L_{X}$  in $\Kl(\T_{\Mon-\Prof}^\times)$, where $Z^\op$ is identified with prop for the diagonal monoid.
%
%\item The naturality of $\eta_{\&}$ with respect to $\mu_{\&}^\op$.
%
%\end{itemize}
%
%
%\end{corollary}


%
%The proof follows from realizing that this equation makes the triangle gate idempotent, which allows one to reduce the value of non-scalar positive natural number H-boxes to $1$, alike to the quotient described in \cite{niel} (H-boxes are first described in the paper \cite{zh}).    The other law forces all nonzero scalar H-boxes, and thus all nonzero scalars to be 0. So this is complete for qubit boolean matrices, and thus, qubit relations.
%


\nocite{ih}
\nocite{coecke2008interacting}
\nocite{zh}
\nocite{tof}

%\appendix 



%
%\begin{lemma}
%
%The phase-free fragment of the ZH calculus is presented by the pushout of the following diagram of props:
%
%
%TODO
%
%modulo the quotient:
%
%\end{lemma}



%\begin{comment}
%






%
%\section{Alternative presentations}
%\label{sec:presentations}
%
%In this section, we give alternative presentations of the props presented in the main body of this paper.  With the exception of the alternative presentation of $(\FPinj_2,\times)$, these are presented in terms of a bunch of (co)monoids modulo equations.  This is more in the aesthetic tradition of the ZX-calculus, for example.  
%
%\subsection{Section \ref{sec:one}}
%\label{subsec:presentations:one}
%
%\subsubsection{$(\Par(\Mat(\F_2)),+)$}
%\label{subsubsec:presentations:one:par}
%
%
%$(\Par(\Mat(\F_2)),+)$ is presented by the symmetric monoidal theory with the following generators:
%$$
%\begin{tikzpicture}
%	\begin{pgfonlayer}{nodelayer}
%		\node [style=Z] (0) at (0.75, 5) {};
%		\node [style=none] (1) at (0.5, 4.5) {};
%		\node [style=none] (2) at (1, 4.5) {};
%		\node [style=none] (3) at (0.75, 5.5) {};
%	\end{pgfonlayer}
%	\begin{pgfonlayer}{edgelayer}
%		\draw [in=-135, out=90] (1.center) to (0);
%		\draw [in=90, out=-45] (0) to (2.center);
%		\draw (0) to (3.center);
%	\end{pgfonlayer}
%\end{tikzpicture}
%\hspace*{.5cm}
%\begin{tikzpicture}
%	\begin{pgfonlayer}{nodelayer}
%		\node [style=Z] (0) at (0, 5) {};
%		\node [style=none] (1) at (-0.25, 5.5) {};
%		\node [style=none] (2) at (0.25, 5.5) {};
%		\node [style=none] (3) at (0, 4.5) {};
%	\end{pgfonlayer}
%	\begin{pgfonlayer}{edgelayer}
%		\draw [in=135, out=-90] (1.center) to (0);
%		\draw [in=-90, out=45] (0) to (2.center);
%		\draw (0) to (3.center);
%	\end{pgfonlayer}
%\end{tikzpicture}
%\hspace*{.5cm}
%\begin{tikzpicture}
%	\begin{pgfonlayer}{nodelayer}
%		\node [style=Z] (0) at (0, 5) {};
%		\node [style=none] (3) at (0, 4.5) {};
%	\end{pgfonlayer}
%	\begin{pgfonlayer}{edgelayer}
%		\draw (0) to (3.center);
%	\end{pgfonlayer}
%\end{tikzpicture}
%\hspace*{.5cm}
%\begin{tikzpicture}
%	\begin{pgfonlayer}{nodelayer}
%		\node [style=X] (0) at (0.75, 5) {};
%		\node [style=none] (1) at (0.5, 4.5) {};
%		\node [style=none] (2) at (1, 4.5) {};
%		\node [style=none] (3) at (0.75, 5.5) {};
%	\end{pgfonlayer}
%	\begin{pgfonlayer}{edgelayer}
%		\draw [in=-135, out=90] (1.center) to (0);
%		\draw [in=90, out=-45] (0) to (2.center);
%		\draw (0) to (3.center);
%	\end{pgfonlayer}
%\end{tikzpicture}
%\hspace*{.5cm}
%\begin{tikzpicture}
%	\begin{pgfonlayer}{nodelayer}
%		\node [style=X] (0) at (0, 4.5) {};
%		\node [style=none] (3) at (0, 5) {};
%	\end{pgfonlayer}
%	\begin{pgfonlayer}{edgelayer}
%		\draw (0) to (3.center);
%	\end{pgfonlayer}
%\end{tikzpicture}
%\hspace*{,5cm}
%\begin{tikzpicture}
%	\begin{pgfonlayer}{nodelayer}
%		\node [style=X] (0) at (0.5, 5) {};
%		\node [style=none] (1) at (0.5, 4.5) {};
%	\end{pgfonlayer}
%	\begin{pgfonlayer}{edgelayer}
%		\draw (0) to (1.center);
%	\end{pgfonlayer}
%\end{tikzpicture}
%$$
%Modulo the all the equations of $\cb_2$ in addition to the identity and its transpose:
%$$
%\begin{tikzpicture}
%	\begin{pgfonlayer}{nodelayer}
%		\node [style=X] (0) at (0.5, 5) {};
%		\node [style=none] (1) at (0.5, 4.5) {};
%		\node [style=Z] (2) at (0.5, 4.5) {};
%		\node [style=none] (3) at (0.25, 4) {};
%		\node [style=none] (4) at (0.75, 4) {};
%	\end{pgfonlayer}
%	\begin{pgfonlayer}{edgelayer}
%		\draw (0) to (1.center);
%		\draw [in=-135, out=90] (3.center) to (2);
%		\draw [in=-45, out=90] (4.center) to (2);
%	\end{pgfonlayer}
%\end{tikzpicture}
%  \eref{bi.two}
%\begin{tikzpicture}
%	\begin{pgfonlayer}{nodelayer}
%		\node [style=X] (0) at (0.25, 5) {};
%		\node [style=none] (3) at (0.25, 4) {};
%		\node [style=none] (4) at (0.75, 4) {};
%		\node [style=X] (5) at (0.75, 5) {};
%	\end{pgfonlayer}
%	\begin{pgfonlayer}{edgelayer}
%		\draw (4.center) to (5);
%		\draw (0) to (3.center);
%	\end{pgfonlayer}
%\end{tikzpicture}
%$$
%
%
%\subsubsection{$(\Span(\Mat(\F_2)),+)$}
%\label{subsubsec:presentations:one:span}
%
%$(\Span(\Mat(\F_2)),+)$ is presented by the symmetric monoidal theory with the same generators and equations as \S \ref{subsubsec:presentations:one:par} and their transposes as well as the following equations making the white monoid/comnoid pair into a special commutative Frobenius algebra and the black monoid/comonoid pair into a commutative Frobenius algebra.
%
%
%
%\subsection{Section \ref{sec:two}}
%\label{subsec:presentations:two}
%
%
%\subsubsection{$(\Par(\Aff\Fin\Vect(\F_2))^*,+)$}
%\label{subsubsec:presentations:two:par}
%
%$(\Par(\Aff\Fin\Vect(\F_2))^*,+)$ is presented by the symmetric monoidal theory with the same generators and equations as in \S \ref{subsubsec:presentations:one:par} in addition to the following generator:
%$$
%\begin{tikzpicture}
%	\begin{pgfonlayer}{nodelayer}
%		\node [style=X] (0) at (0, 4) {$1$};
%		\node [style=none] (1) at (0, 4.5) {};
%	\end{pgfonlayer}
%	\begin{pgfonlayer}{edgelayer}
%		\draw (0) to (1.center);
%	\end{pgfonlayer}
%\end{tikzpicture}
%$$
%and the following equations:
%$$
%\begin{tikzpicture}
%	\begin{pgfonlayer}{nodelayer}
%		\node [style=X] (0) at (-4, 0.75) {$1$};
%		\node [style=none] (1) at (-3.5, 0) {};
%		\node [style=none] (2) at (-3.5, 1.5) {};
%	\end{pgfonlayer}
%	\begin{pgfonlayer}{edgelayer}
%		\draw (1.center) to (2.center);
%	\end{pgfonlayer}
%\end{tikzpicture}
%\eqzxa{zero.new}
%\begin{tikzpicture}
%	\begin{pgfonlayer}{nodelayer}
%		\node [style=X] (0) at (-4, 0.75) {$1$};
%		\node [style=none] (1) at (-3.5, 0) {};
%		\node [style=none] (2) at (-3.5, 1.5) {};
%		\node [style=Z] (3) at (-3.5, 0.5) {};
%		\node [style=X] (4) at (-3.5, 1) {};
%	\end{pgfonlayer}
%	\begin{pgfonlayer}{edgelayer}
%		\draw (2.center) to (4);
%		\draw (3) to (1.center);
%	\end{pgfonlayer}
%\end{tikzpicture},
%\hspace*{,5cm}
%\begin{tikzpicture}
%	\begin{pgfonlayer}{nodelayer}
%		\node [style=X] (0) at (0.75, 4) {$1$};
%		\node [style=none] (1) at (0.75, 4.5) {};
%		\node [style=Z] (2) at (0.75, 4.5) {};
%		\node [style=none] (3) at (0.5, 5) {};
%		\node [style=none] (4) at (1, 5) {};
%	\end{pgfonlayer}
%	\begin{pgfonlayer}{edgelayer}
%		\draw (0) to (1.center);
%		\draw [in=135, out=-90] (3.center) to (2);
%		\draw [in=45, out=-90] (4.center) to (2);
%	\end{pgfonlayer}
%\end{tikzpicture}
%  \erefop{bi.two}
%\begin{tikzpicture}
%	\begin{pgfonlayer}{nodelayer}
%		\node [style=X] (0) at (0.5, 4) {$1$};
%		\node [style=none] (1) at (0.5, 5) {};
%		\node [style=none] (2) at (1, 5) {};
%		\node [style=X] (3) at (1, 4) {$1$};
%	\end{pgfonlayer}
%	\begin{pgfonlayer}{edgelayer}
%		\draw (2.center) to (3);
%		\draw (0) to (1.center);
%	\end{pgfonlayer}
%\end{tikzpicture}
%\hspace*{,5cm}
%\begin{tikzpicture}
%	\begin{pgfonlayer}{nodelayer}
%		\node [style=X] (0) at (0.75, 4) {$1$};
%		\node [style=none] (1) at (0.75, 4.5) {};
%		\node [style=Z] (2) at (0.75, 4.5) {};
%	\end{pgfonlayer}
%	\begin{pgfonlayer}{edgelayer}
%		\draw (0) to (1.center);
%	\end{pgfonlayer}
%\end{tikzpicture}
%  \eref{extra}
%$$
%
%
%\subsubsection{$(\Span(\Aff\Fin\Vect(\F_2))^*,+)$}
%\label{subsubsec:presentations:two:span}
%
%$(\Span(\Aff\Fin\Vect(\F_2))^*,+)$ is presented by the generators and identities of  \ref{subsubsec:presentations:two:par} as well as as well as the generator 
%$\begin{tikzpicture}
%	\begin{pgfonlayer}{nodelayer}
%		\node [style=none] (0) at (0.75, 0.5) {};
%		\node [style=none] (1) at (0.75, -0.25) {};
%		\node [style=Z] (2) at (0.75, -0.25) {};
%	\end{pgfonlayer}
%	\begin{pgfonlayer}{edgelayer}
%		\draw (0.center) to (1.center);
%	\end{pgfonlayer}
%\end{tikzpicture}$ 
%and the equation making the codiagonal map counital:
%$$
%  \begin{tikzpicture}[rotate=90,yscale=-1]
%	\begin{pgfonlayer}{nodelayer}
%		\node [style=Z] (0) at (-9, -0) {};
%		\node [style=none] (1) at (-8.25, -0) {};
%		\node [style=Z] (2) at (-9.75, 0.25) {};
%		\node [style=none] (3) at (-10, -0.25) {};
%	\end{pgfonlayer}
%	\begin{pgfonlayer}{edgelayer}
%		\draw [in=-150, out=0, looseness=1.00] (3.center) to (0);
%		\draw [in=150, out=0, looseness=1.00] (2.center) to (0);
%		\draw (0) to (1.center);
%	\end{pgfonlayer}
%  \end{tikzpicture}
%  \eref{unit}
%  \begin{tikzpicture}[rotate=90]
%	\begin{pgfonlayer}{nodelayer}
%		\node [style=none] (0) at (-9, 0.25) {};
%		\node [style=none] (1) at (-9.75, 0.25) {};
%	\end{pgfonlayer}
%	\begin{pgfonlayer}{edgelayer}
%		\draw (1) to (0.center);
%	\end{pgfonlayer}
%  \end{tikzpicture}
%$$
%
%\subsection{Section \ref{sec:three}}
%\label{subsec:presentations:three}
%
%
%
%%\subsubsection{$(\FPinj_2,\times)$}
%%\label{subsubsec:presentations:three:pinj}
%
%%By \cite[\S 7]{cole} $(\FPinj_2,\times)$ is presented by the prop generated by the Toffoli gate (the triple-controlled-not gate) as well as unit for the and gate and its transpose modulo the following equations:
%%
%%\begin{multicols}{2}
%%\begin{enumerate}[label={\bf [TOF.\arabic*]}, ref={\bf [TOF.\arabic*]}, wide = 0pt, leftmargin = 2em]
%%\item
%%\label{TOF.1}
%%{\hfil
%%$
%%\begin{tabular}{cc}
%%\begin{tikzpicture}
%%	\begin{pgfonlayer}{nodelayer}
%%		\node [style=nothing] (0) at (1.5, 0) {};
%%		\node [style=nothing] (1) at (1, 0) {};
%%		\node [style=oplus] (2) at (1.5, 1) {};
%%		\node [style=dot] (3) at (1, 1) {};
%%		\node [style=dot] (4) at (0.5, 1) {};
%%		\node [style=X] (5) at (0.5, 0.5) {$1$};
%%		\node [style=nothing] (6) at (0.5, 1.5) {};
%%		\node [style=nothing] (7) at (1, 1.5) {};
%%		\node [style=nothing] (8) at (1.5, 1.5) {};
%%	\end{pgfonlayer}
%%	\begin{pgfonlayer}{edgelayer}
%%		\draw (5) to (4);
%%		\draw (4) to (6);
%%		\draw (7) to (3);
%%		\draw (1) to (3);
%%		\draw (0) to (2);
%%		\draw (2) to (8);
%%		\draw (2) to (3);
%%		\draw (3) to (4);
%%	\end{pgfonlayer}
%%\end{tikzpicture}
%%=
%%\begin{tikzpicture}
%%	\begin{pgfonlayer}{nodelayer}
%%		\node [style=nothing] (0) at (1.5, 0) {};
%%		\node [style=nothing] (1) at (1, 0) {};
%%		\node [style=oplus] (2) at (1.5, 1) {};
%%		\node [style=dot] (3) at (1, 1) {};
%%		\node [style=X] (4) at (0.5, 1) {$1$};
%%		\node [style=nothing] (5) at (0.5, 1.5) {};
%%		\node [style=nothing] (6) at (1, 1.5) {};
%%		\node [style=nothing] (7) at (1.5, 1.5) {};
%%	\end{pgfonlayer}
%%	\begin{pgfonlayer}{edgelayer}
%%		\draw (1) to (3);
%%		\draw (0) to (2);
%%		\draw (2) to (3);
%%		\draw (6) to (3);
%%		\draw (4) to (5);
%%		\draw (2) to (7);
%%	\end{pgfonlayer}
%%\end{tikzpicture} &
%%\begin{tikzpicture}
%%	\begin{pgfonlayer}{nodelayer}
%%		\node [style=nothing] (0) at (1.5, 2) {};
%%		\node [style=nothing] (1) at (1, 2) {};
%%		\node [style=oplus] (2) at (1.5, 1) {};
%%		\node [style=dot] (3) at (1, 1) {};
%%		\node [style=dot] (4) at (0.5, 1) {};
%%		\node [style=X] (5) at (0.5, 1.5) {$1$};
%%		\node [style=nothing] (6) at (0.5, 0.5) {};
%%		\node [style=nothing] (7) at (1, 0.5) {};
%%		\node [style=nothing] (8) at (1.5, 0.5) {};
%%	\end{pgfonlayer}
%%	\begin{pgfonlayer}{edgelayer}
%%		\draw (5) to (4);
%%		\draw (4) to (6);
%%		\draw (7) to (3);
%%		\draw (1) to (3);
%%		\draw (0) to (2);
%%		\draw (2) to (8);
%%		\draw (2) to (3);
%%		\draw (3) to (4);
%%	\end{pgfonlayer}
%%\end{tikzpicture}
%%=
%%\begin{tikzpicture}
%%	\begin{pgfonlayer}{nodelayer}
%%		\node [style=nothing] (0) at (1.5, 2) {};
%%		\node [style=nothing] (1) at (1, 2) {};
%%		\node [style=oplus] (2) at (1.5, 1) {};
%%		\node [style=dot] (3) at (1, 1) {};
%%		\node [style=X] (4) at (0.5, 1) {$1$};
%%		\node [style=nothing] (5) at (0.5, 0.5) {};
%%		\node [style=nothing] (6) at (1, 0.5) {};
%%		\node [style=nothing] (7) at (1.5, 0.5) {};
%%	\end{pgfonlayer}
%%	\begin{pgfonlayer}{edgelayer}
%%		\draw (1) to (3);
%%		\draw (0) to (2);
%%		\draw (2) to (3);
%%		\draw (6) to (3);
%%		\draw (4) to (5);
%%		\draw (2) to (7);
%%	\end{pgfonlayer}
%%\end{tikzpicture}
%%\end{tabular}
%%$}
%%
%%
%%\item
%%\label{TOF.2}
%%{\hfil
%%$
%%\begin{tabular}{cc}
%%\begin{tikzpicture}
%%	\begin{pgfonlayer}{nodelayer}
%%		\node [style=nothing] (0) at (1, 0.5) {};
%%		\node [style=nothing] (1) at (1.5, 0.5) {};
%%		\node [style=nothing] (2) at (0.5, 2) {};
%%		\node [style=nothing] (3) at (1, 2) {};
%%		\node [style=nothing] (4) at (1.5, 2) {};
%%		\node [style=dot] (5) at (0.5, 1.5) {};
%%		\node [style=dot] (6) at (1, 1.5) {};
%%		\node [style=oplus] (7) at (1.5, 1.5) {};
%%		\node [style=X] (8) at (0.5, 1) {};
%%	\end{pgfonlayer}
%%	\begin{pgfonlayer}{edgelayer}
%%		\draw (5) to (2);
%%		\draw (3) to (6);
%%		\draw (6) to (0);
%%		\draw (1) to (7);
%%		\draw (7) to (4);
%%		\draw (7) to (6);
%%		\draw (6) to (5);
%%		\draw (8) to (5);
%%	\end{pgfonlayer}
%%\end{tikzpicture}
%%=
%%\begin{tikzpicture}
%%	\begin{pgfonlayer}{nodelayer}
%%		\node [style=nothing] (0) at (1, 0.75) {};
%%		\node [style=nothing] (1) at (1.5, 0.75) {};
%%		\node [style=nothing] (2) at (0.5, 2) {};
%%		\node [style=nothing] (3) at (1, 2) {};
%%		\node [style=nothing] (4) at (1.5, 2) {};
%%		\node [style=X] (5) at (0.5, 1.25) {};
%%	\end{pgfonlayer}
%%	\begin{pgfonlayer}{edgelayer}
%%		\draw (5) to (2);
%%		\draw (0) to (3);
%%		\draw (1) to (4);
%%	\end{pgfonlayer}
%%\end{tikzpicture} & 
%%\begin{tikzpicture}
%%	\begin{pgfonlayer}{nodelayer}
%%		\node [style=nothing] (0) at (1, 2.5) {};
%%		\node [style=nothing] (1) at (1.5, 2.5) {};
%%		\node [style=nothing] (2) at (0.5, 1) {};
%%		\node [style=nothing] (3) at (1, 1) {};
%%		\node [style=nothing] (4) at (1.5, 1) {};
%%		\node [style=dot] (5) at (0.5, 1.5) {};
%%		\node [style=dot] (6) at (1, 1.5) {};
%%		\node [style=oplus] (7) at (1.5, 1.5) {};
%%		\node [style=X] (8) at (0.5, 2) {};
%%	\end{pgfonlayer}
%%	\begin{pgfonlayer}{edgelayer}
%%		\draw (5) to (2);
%%		\draw (3) to (6);
%%		\draw (6) to (0);
%%		\draw (1) to (7);
%%		\draw (7) to (4);
%%		\draw (7) to (6);
%%		\draw (6) to (5);
%%		\draw (8) to (5);
%%	\end{pgfonlayer}
%%\end{tikzpicture}
%%=
%%\begin{tikzpicture}
%%	\begin{pgfonlayer}{nodelayer}
%%		\node [style=nothing] (0) at (1, 2.5) {};
%%		\node [style=nothing] (1) at (1.5, 2.5) {};
%%		\node [style=nothing] (2) at (0.5, 1.25) {};
%%		\node [style=nothing] (3) at (1, 1.25) {};
%%		\node [style=nothing] (4) at (1.5, 1.25) {};
%%		\node [style=X] (5) at (0.5, 2) {};
%%	\end{pgfonlayer}
%%	\begin{pgfonlayer}{edgelayer}
%%		\draw (5) to (2);
%%		\draw (0) to (3);
%%		\draw (1) to (4);
%%	\end{pgfonlayer}
%%\end{tikzpicture}
%%\end{tabular}
%%$}
%%
%%\item
%%\label{TOF.3}
%%{\hfil
%%$
%%\begin{tikzpicture}
%%	\begin{pgfonlayer}{nodelayer}
%%		\node [style=nothing] (0) at (-0.5, 0.5) {};
%%		\node [style=nothing] (1) at (0, 0.5) {};
%%		\node [style=nothing] (2) at (-1, 0.5) {};
%%		\node [style=nothing] (3) at (-1.5, 0.5) {};
%%		\node [style=nothing] (4) at (-2, 0.5) {};
%%		\node [style=dot] (5) at (-1.5, 1) {};
%%		\node [style=oplus] (6) at (-1, 1) {};
%%		\node [style=oplus] (7) at (-1, 1.5) {};
%%		\node [style=dot] (8) at (-0.5, 1.5) {};
%%		\node [style=dot] (9) at (-2, 1) {};
%%		\node [style=dot] (10) at (0, 1.5) {};
%%		\node [style=nothing] (11) at (-0.5, 2) {};
%%		\node [style=nothing] (12) at (-1.5, 2) {};
%%		\node [style=nothing] (13) at (-2, 2) {};
%%		\node [style=nothing] (14) at (0, 2) {};
%%		\node [style=nothing] (15) at (-1, 2) {};
%%	\end{pgfonlayer}
%%	\begin{pgfonlayer}{edgelayer}
%%		\draw (4) to (9);
%%		\draw (9) to (13);
%%		\draw (3) to (5);
%%		\draw (5) to (12);
%%		\draw (2) to (6);
%%		\draw (6) to (7);
%%		\draw (7) to (15);
%%		\draw (0) to (8);
%%		\draw (8) to (11);
%%		\draw (1) to (10);
%%		\draw (10) to (14);
%%		\draw (10) to (8);
%%		\draw (8) to (7);
%%		\draw (6) to (5);
%%		\draw (5) to (9);
%%	\end{pgfonlayer}
%%\end{tikzpicture}
%%=
%%\begin{tikzpicture}
%%	\begin{pgfonlayer}{nodelayer}
%%		\node [style=nothing] (0) at (-0.5, 0.5) {};
%%		\node [style=nothing] (1) at (0, 0.5) {};
%%		\node [style=nothing] (2) at (-1, 0.5) {};
%%		\node [style=nothing] (3) at (-1.5, 0.5) {};
%%		\node [style=nothing] (4) at (-2, 0.5) {};
%%		\node [style=dot] (5) at (-1.5, 1.5) {};
%%		\node [style=dot] (6) at (-0.5, 1) {};
%%		\node [style=dot] (7) at (-2, 1.5) {};
%%		\node [style=dot] (8) at (0, 1) {};
%%		\node [style=nothing] (9) at (-0.5, 2) {};
%%		\node [style=nothing] (10) at (-1.5, 2) {};
%%		\node [style=nothing] (11) at (-2, 2) {};
%%		\node [style=nothing] (12) at (0, 2) {};
%%		\node [style=nothing] (13) at (-1, 2) {};
%%		\node [style=oplus] (14) at (-1, 1.5) {};
%%		\node [style=oplus] (15) at (-1, 1) {};
%%	\end{pgfonlayer}
%%	\begin{pgfonlayer}{edgelayer}
%%		\draw (4) to (7);
%%		\draw (7) to (11);
%%		\draw (3) to (5);
%%		\draw (5) to (10);
%%		\draw (0) to (6);
%%		\draw (6) to (9);
%%		\draw (1) to (8);
%%		\draw (8) to (12);
%%		\draw (8) to (6);
%%		\draw (5) to (7);
%%		\draw (2) to (15);
%%		\draw (15) to (14);
%%		\draw (14) to (13);
%%		\draw (14) to (5);
%%		\draw (6) to (15);
%%	\end{pgfonlayer}
%%\end{tikzpicture}
%%$}
%%
%%
%%\item
%%\label{TOF.4}
%%{\hfil
%%$
%%\begin{tikzpicture}
%%	\begin{pgfonlayer}{nodelayer}
%%		\node [style=nothing] (0) at (-0.5, 0.5) {};
%%		\node [style=nothing] (1) at (0, 0.5) {};
%%		\node [style=nothing] (2) at (-1, 0.5) {};
%%		\node [style=nothing] (3) at (-1.5, 0.5) {};
%%		\node [style=nothing] (4) at (-2, 0.5) {};
%%		\node [style=dot] (5) at (-1.5, 1) {};
%%		\node [style=dot] (6) at (-1, 1) {};
%%		\node [style=dot] (7) at (-1, 1.5) {};
%%		\node [style=dot] (8) at (-0.5, 1.5) {};
%%		\node [style=oplus] (9) at (-2, 1) {};
%%		\node [style=oplus] (10) at (0, 1.5) {};
%%		\node [style=nothing] (11) at (-0.5, 2) {};
%%		\node [style=nothing] (12) at (-1.5, 2) {};
%%		\node [style=nothing] (13) at (-2, 2) {};
%%		\node [style=nothing] (14) at (0, 2) {};
%%		\node [style=nothing] (15) at (-1, 2) {};
%%	\end{pgfonlayer}
%%	\begin{pgfonlayer}{edgelayer}
%%		\draw (4) to (9);
%%		\draw (9) to (13);
%%		\draw (3) to (5);
%%		\draw (5) to (12);
%%		\draw (2) to (6);
%%		\draw (6) to (7);
%%		\draw (7) to (15);
%%		\draw (0) to (8);
%%		\draw (8) to (11);
%%		\draw (1) to (10);
%%		\draw (10) to (14);
%%		\draw (10) to (8);
%%		\draw (8) to (7);
%%		\draw (6) to (5);
%%		\draw (5) to (9);
%%	\end{pgfonlayer}
%%\end{tikzpicture}
%%=
%%\begin{tikzpicture}
%%	\begin{pgfonlayer}{nodelayer}
%%		\node [style=nothing] (0) at (-0.5, 0.5) {};
%%		\node [style=nothing] (1) at (0, 0.5) {};
%%		\node [style=nothing] (2) at (-1, 0.5) {};
%%		\node [style=nothing] (3) at (-1.5, 0.5) {};
%%		\node [style=nothing] (4) at (-2, 0.5) {};
%%		\node [style=dot] (5) at (-1.5, 1.5) {};
%%		\node [style=dot] (6) at (-0.5, 1) {};
%%		\node [style=oplus] (7) at (-2, 1.5) {};
%%		\node [style=oplus] (8) at (0, 1) {};
%%		\node [style=nothing] (9) at (-0.5, 2) {};
%%		\node [style=nothing] (10) at (-1.5, 2) {};
%%		\node [style=nothing] (11) at (-2, 2) {};
%%		\node [style=nothing] (12) at (0, 2) {};
%%		\node [style=nothing] (13) at (-1, 2) {};
%%		\node [style=dot] (14) at (-1, 1.5) {};
%%		\node [style=dot] (15) at (-1, 1) {};
%%	\end{pgfonlayer}
%%	\begin{pgfonlayer}{edgelayer}
%%		\draw (4) to (7);
%%		\draw (7) to (11);
%%		\draw (3) to (5);
%%		\draw (5) to (10);
%%		\draw (0) to (6);
%%		\draw (6) to (9);
%%		\draw (1) to (8);
%%		\draw (8) to (12);
%%		\draw (8) to (6);
%%		\draw (5) to (7);
%%		\draw (2) to (15);
%%		\draw (15) to (14);
%%		\draw (14) to (13);
%%		\draw (14) to (5);
%%		\draw (6) to (15);
%%	\end{pgfonlayer}
%%\end{tikzpicture}
%%$}
%%
%%\item
%%\label{TOF.5}
%%{\hfil
%%$
%%\begin{tikzpicture}
%%	\begin{pgfonlayer}{nodelayer}
%%		\node [style=nothing] (0) at (-1, 0.5) {};
%%		\node [style=nothing] (1) at (-0.5, 0.5) {};
%%		\node [style=nothing] (2) at (-1.5, 0.5) {};
%%		\node [style=nothing] (3) at (-2, 0.5) {};
%%		\node [style=nothing] (4) at (-1, 2) {};
%%		\node [style=nothing] (5) at (-1.5, 2) {};
%%		\node [style=nothing] (6) at (-2, 2) {};
%%		\node [style=nothing] (7) at (-0.5, 2) {};
%%		\node [style=oplus] (8) at (-2, 1) {};
%%		\node [style=oplus] (9) at (-0.5, 1.5) {};
%%		\node [style=dot] (10) at (-1.5, 1) {};
%%		\node [style=dot] (11) at (-1, 1) {};
%%		\node [style=dot] (12) at (-1.5, 1.5) {};
%%		\node [style=dot] (13) at (-1, 1.5) {};
%%	\end{pgfonlayer}
%%	\begin{pgfonlayer}{edgelayer}
%%		\draw (3) to (8);
%%		\draw (8) to (6);
%%		\draw (5) to (12);
%%		\draw (12) to (10);
%%		\draw (10) to (2);
%%		\draw (0) to (11);
%%		\draw (11) to (13);
%%		\draw (13) to (4);
%%		\draw (7) to (9);
%%		\draw (9) to (1);
%%		\draw (10) to (11);
%%		\draw (10) to (8);
%%		\draw (12) to (13);
%%		\draw (13) to (9);
%%	\end{pgfonlayer}
%%\end{tikzpicture}
%%=
%%\begin{tikzpicture}
%%	\begin{pgfonlayer}{nodelayer}
%%		\node [style=nothing] (0) at (-1, 0.5) {};
%%		\node [style=nothing] (1) at (-0.5, 0.5) {};
%%		\node [style=nothing] (2) at (-1.5, 0.5) {};
%%		\node [style=nothing] (3) at (-2, 0.5) {};
%%		\node [style=nothing] (4) at (-1, 2) {};
%%		\node [style=nothing] (5) at (-1.5, 2) {};
%%		\node [style=nothing] (6) at (-2, 2) {};
%%		\node [style=nothing] (7) at (-0.5, 2) {};
%%		\node [style=oplus] (8) at (-2, 1.5) {};
%%		\node [style=dot] (9) at (-1.5, 1.5) {};
%%		\node [style=dot] (10) at (-1, 1.5) {};
%%		\node [style=oplus] (11) at (-0.5, 1) {};
%%		\node [style=dot] (12) at (-1, 1) {};
%%		\node [style=dot] (13) at (-1.5, 1) {};
%%	\end{pgfonlayer}
%%	\begin{pgfonlayer}{edgelayer}
%%		\draw (9) to (10);
%%		\draw (9) to (8);
%%		\draw (13) to (12);
%%		\draw (12) to (11);
%%		\draw (3) to (8);
%%		\draw (8) to (6);
%%		\draw (5) to (9);
%%		\draw (9) to (13);
%%		\draw (13) to (2);
%%		\draw (0) to (12);
%%		\draw (12) to (10);
%%		\draw (10) to (4);
%%		\draw (7) to (11);
%%		\draw (11) to (1);
%%	\end{pgfonlayer}
%%\end{tikzpicture}
%%$}
%%
%%
%%\item
%%\label{TOF.6}
%%{\hfil
%%$
%%\begin{tikzpicture}
%%	\begin{pgfonlayer}{nodelayer}
%%		\node [style=nothing] (0) at (-1, 0.5) {};
%%		\node [style=nothing] (1) at (-1.5, 0.5) {};
%%		\node [style=nothing] (2) at (-2, 0.5) {};
%%		\node [style=nothing] (3) at (-1, 2) {};
%%		\node [style=nothing] (4) at (-1.5, 2) {};
%%		\node [style=nothing] (5) at (-2, 2) {};
%%		\node [style=nothing] (6) at (-0.5, 2) {};
%%		\node [style=oplus] (7) at (-0.5, 1) {};
%%		\node [style=dot] (8) at (-1.5, 1.5) {};
%%		\node [style=dot] (9) at (-1, 1.5) {};
%%		\node [style=dot] (10) at (-1, 1) {};
%%		\node [style=oplus] (11) at (-0.5, 1.5) {};
%%		\node [style=nothing] (12) at (-0.5, 0.5) {};
%%		\node [style=dot] (13) at (-2, 1) {};
%%	\end{pgfonlayer}
%%	\begin{pgfonlayer}{edgelayer}
%%		\draw (8) to (1);
%%		\draw (0) to (9);
%%		\draw (9) to (10);
%%		\draw (10) to (3);
%%		\draw (6) to (7);
%%		\draw (8) to (9);
%%		\draw (10) to (7);
%%		\draw (12) to (11);
%%		\draw (11) to (7);
%%		\draw (8) to (4);
%%		\draw (9) to (11);
%%		\draw (10) to (13);
%%		\draw (13) to (5);
%%		\draw (13) to (2);
%%	\end{pgfonlayer}
%%\end{tikzpicture}
%%=
%%\begin{tikzpicture}
%%	\begin{pgfonlayer}{nodelayer}
%%		\node [style=nothing] (0) at (-1, 0.5) {};
%%		\node [style=nothing] (1) at (-1.5, 0.5) {};
%%		\node [style=nothing] (2) at (-2, 0.5) {};
%%		\node [style=nothing] (3) at (-1, 2) {};
%%		\node [style=nothing] (4) at (-1.5, 2) {};
%%		\node [style=nothing] (5) at (-2, 2) {};
%%		\node [style=nothing] (6) at (-0.5, 2) {};
%%		\node [style=oplus] (7) at (-0.5, 1.5) {};
%%		\node [style=dot] (8) at (-1.5, 1) {};
%%		\node [style=dot] (9) at (-1, 1) {};
%%		\node [style=dot] (10) at (-1, 1.5) {};
%%		\node [style=oplus] (11) at (-0.5, 1) {};
%%		\node [style=nothing] (12) at (-0.5, 0.5) {};
%%		\node [style=dot] (13) at (-2, 1.5) {};
%%	\end{pgfonlayer}
%%	\begin{pgfonlayer}{edgelayer}
%%		\draw (8) to (1);
%%		\draw (0) to (9);
%%		\draw (9) to (10);
%%		\draw (10) to (3);
%%		\draw (6) to (7);
%%		\draw (8) to (9);
%%		\draw (10) to (7);
%%		\draw (12) to (11);
%%		\draw (11) to (7);
%%		\draw (8) to (4);
%%		\draw (9) to (11);
%%		\draw (10) to (13);
%%		\draw (13) to (5);
%%		\draw (13) to (2);
%%	\end{pgfonlayer}
%%\end{tikzpicture}
%%$}
%%
%%\item
%%\label{TOF.7}
%%{\hfil
%%$
%%\begin{tikzpicture}
%%	\begin{pgfonlayer}{nodelayer}
%%		\node [style=nothing] (0) at (1, 0) {};
%%		\node [style=nothing] (1) at (0.5, 0) {};
%%		\node [style=nothing] (2) at (0.5, 3.5) {};
%%		\node [style=nothing] (3) at (1, 3.5) {};
%%		\node [style=X] (4) at (1.5, 3) {};
%%		\node [style=oplus] (5) at (1.5, 2.5) {};
%%		\node [style=dot] (6) at (1, 2.5) {};
%%		\node [style=dot] (7) at (0.5, 1) {};
%%		\node [style=oplus] (8) at (1.5, 1) {};
%%		\node [style=X] (9) at (1.5, 1.5) {};
%%		\node [style=X] (10) at (1.5, 0.5) {$1$};
%%		\node [style=X] (11) at (1.5, 2) {$1$};
%%	\end{pgfonlayer}
%%	\begin{pgfonlayer}{edgelayer}
%%		\draw (1) to (7);
%%		\draw (7) to (2);
%%		\draw (3) to (6);
%%		\draw (6) to (0);
%%		\draw (8) to (9);
%%		\draw (8) to (7);
%%		\draw (5) to (4);
%%		\draw (5) to (6);
%%		\draw (10) to (8);
%%		\draw (11) to (5);
%%	\end{pgfonlayer}
%%\end{tikzpicture}
%%=
%%\begin{tikzpicture}
%%	\begin{pgfonlayer}{nodelayer}
%%		\node [style=nothing] (0) at (3, 0) {};
%%		\node [style=nothing] (1) at (2.5, 0) {};
%%		\node [style=nothing] (2) at (2.5, 3.5) {};
%%		\node [style=nothing] (3) at (3, 3.5) {};
%%		\node [style=dot] (4) at (2.5, 1.75) {};
%%		\node [style=dot] (5) at (3, 1.75) {};
%%		\node [style=X] (6) at (3.5, 1.25) {$1$};
%%		\node [style=X] (7) at (3.5, 2.25) {};
%%		\node [style=oplus] (8) at (3.5, 1.75) {};
%%	\end{pgfonlayer}
%%	\begin{pgfonlayer}{edgelayer}
%%		\draw (1) to (4);
%%		\draw (4) to (2);
%%		\draw (3) to (5);
%%		\draw (5) to (0);
%%		\draw (6) to (8);
%%		\draw (8) to (7);
%%		\draw (8) to (5);
%%		\draw (5) to (4);
%%	\end{pgfonlayer}
%%\end{tikzpicture}
%%$}
%%%
%%%\item
%%%\label{TOF.7}
%%%{\hfil
%%%$
%%%\begin{tikzpicture}
%%%	\begin{pgfonlayer}{nodelayer}
%%%		\node [style=nothing] (0) at (0, -0) {};
%%%		\node [style=nothing] (1) at (1.5, -0) {};
%%%		\node [style=X] (2) at (0.4, 0.5) {};
%%%		\node [style=X] (3) at (1.1, 0.5) {};
%%%	\end{pgfonlayer}
%%%	\begin{pgfonlayer}{edgelayer}
%%%		\draw (2) to (3);
%%%		\draw (0) to (1);
%%%	\end{pgfonlayer}
%%%\end{tikzpicture}
%%%=
%%%\begin{tikzpicture}
%%%	\begin{pgfonlayer}{nodelayer}
%%%		\node [style=nothing] (0) at (0, -0) {};
%%%		\node [style=nothing] (1) at (1.5, -0) {};
%%%		\node [style=X] (2) at (0.4, 0.5) {};
%%%		\node [style=X] (3) at (1.1, 0.5) {};
%%%		\node [style=X] (4) at (1, -0) {};
%%%		\node [style=X] (5) at (0.5000002, -0) {};
%%%	\end{pgfonlayer}
%%%	\begin{pgfonlayer}{edgelayer}
%%%		\draw (2) to (3);
%%%		\draw (5) to (0);
%%%		\draw (4) to (1);
%%%	\end{pgfonlayer}
%%%\end{tikzpicture}
%%%$}
%%
%%\item
%%\label{TOF.8}
%%{\hfil
%%$
%%\begin{tikzpicture}
%%	\begin{pgfonlayer}{nodelayer}
%%		\node [style=X] (0) at (0, 0.5) {$1$};
%%		\node [style=X] (1) at (0, 1.5) {$1$};
%%	\end{pgfonlayer}
%%	\begin{pgfonlayer}{edgelayer}
%%		\draw (0) to (1);
%%	\end{pgfonlayer}
%%\end{tikzpicture}
%%=
%%\begin{tikzpicture}
%%	\begin{pgfonlayer}{nodelayer}
%%		\node [style=rn] (0) at (0, 0.5) {};
%%		\node [style=rn] (1) at (0, 1.5) {};
%%	\end{pgfonlayer}
%%\end{tikzpicture}
%%%\hspace*{-.8cm}
%%%\begin{tikzpicture}[scale=.5]
%%%\begin{pgfonlayer}{nodelayer}
%%%\begin{tikzpicture}
%%%\node[cloud, cloud puffs=15.7,minimum width=3cm, draw,] (cloud) at (0,0) {$1_0$};
%%%\end{tikzpicture}
%%%\end{pgfonlayer}
%%%\begin{pgfonlayer}{edgelayer}
%%%\end{pgfonlayer}
%%%\end{tikzpicture}
%%$}
%%
%%\item
%%\label{TOF.9}
%%{\hfil
%%$
%%\begin{tikzpicture}
%%	\begin{pgfonlayer}{nodelayer}
%%		\node [style=nothing] (0) at (-1.75, 0.5) {};
%%		\node [style=nothing] (1) at (-1.25, 0.5) {};
%%		\node [style=nothing] (2) at (-0.75, 0.5) {};
%%		\node [style=dot] (3) at (-1.75, 1) {};
%%		\node [style=dot] (4) at (-1.25, 1) {};
%%		\node [style=oplus] (5) at (-0.75, 1) {};
%%		\node [style=dot] (6) at (-1.75, 1.5) {};
%%		\node [style=oplus] (7) at (-0.75, 1.5) {};
%%		\node [style=dot] (8) at (-1.25, 1.5) {};
%%		\node [style=nothing] (9) at (-1.25, 2) {};
%%		\node [style=nothing] (10) at (-0.75, 2) {};
%%		\node [style=nothing] (11) at (-1.75, 2) {};
%%	\end{pgfonlayer}
%%	\begin{pgfonlayer}{edgelayer}
%%		\draw (0) to (3);
%%		\draw (1) to (4);
%%		\draw (2) to (5);
%%		\draw (3) to (4);
%%		\draw (4) to (5);
%%		\draw (6) to (8);
%%		\draw (8) to (7);
%%		\draw (3) to (6);
%%		\draw (6) to (11);
%%		\draw (4) to (8);
%%		\draw (8) to (9);
%%		\draw (5) to (7);
%%		\draw (7) to (10);
%%	\end{pgfonlayer}
%%\end{tikzpicture}
%%=
%%\begin{tikzpicture}
%%	\begin{pgfonlayer}{nodelayer}
%%		\node [style=nothing] (0) at (-1.75, 0.5) {};
%%		\node [style=nothing] (1) at (-1.25, 0.5) {};
%%		\node [style=nothing] (2) at (-0.75, 0.5) {};
%%		\node [style=nothing] (3) at (-1.25, 2) {};
%%		\node [style=nothing] (4) at (-0.75, 2) {};
%%		\node [style=nothing] (5) at (-1.75, 2) {};
%%	\end{pgfonlayer}
%%	\begin{pgfonlayer}{edgelayer}
%%		\draw (0) to (5);
%%		\draw (1) to (3);
%%		\draw (2) to (4);
%%	\end{pgfonlayer}
%%\end{tikzpicture}
%%$}
%%
%%\item
%%\label{TOF.10}
%%{\hfil
%%$
%%\begin{tikzpicture}
%%	\begin{pgfonlayer}{nodelayer}
%%		\node [style=nothing] (0) at (0, 0.5) {};
%%		\node [style=nothing] (1) at (-0.5, 0.5) {};
%%		\node [style=nothing] (2) at (-1, 0.5) {};
%%		\node [style=nothing] (3) at (-1.5, 0.5) {};
%%		\node [style=dot] (4) at (-1, 1) {};
%%		\node [style=dot] (5) at (-0.5, 1) {};
%%		\node [style=oplus] (6) at (0, 1) {};
%%		\node [style=dot] (7) at (-1.5, 1.5) {};
%%		\node [style=oplus] (8) at (-0.5, 1.5) {};
%%		\node [style=dot] (9) at (-1, 1.5) {};
%%		\node [style=dot] (10) at (-1, 2) {};
%%		\node [style=oplus] (11) at (0, 2) {};
%%		\node [style=dot] (12) at (-0.5, 2) {};
%%		\node [style=nothing] (13) at (-1.5, 2.5) {};
%%		\node [style=nothing] (14) at (-0.5, 2.5) {};
%%		\node [style=nothing] (15) at (-1, 2.5) {};
%%		\node [style=nothing] (16) at (0, 2.5) {};
%%	\end{pgfonlayer}
%%	\begin{pgfonlayer}{edgelayer}
%%		\draw (4) to (5);
%%		\draw (5) to (6);
%%		\draw (7) to (9);
%%		\draw (9) to (8);
%%		\draw (10) to (12);
%%		\draw (12) to (11);
%%		\draw (3) to (7);
%%		\draw (7) to (13);
%%		\draw (15) to (10);
%%		\draw (10) to (9);
%%		\draw (9) to (4);
%%		\draw (4) to (2);
%%		\draw (1) to (5);
%%		\draw (5) to (8);
%%		\draw (8) to (12);
%%		\draw (12) to (14);
%%		\draw (16) to (11);
%%		\draw (11) to (6);
%%		\draw (6) to (0);
%%	\end{pgfonlayer}
%%\end{tikzpicture}
%%=
%%\begin{tikzpicture}
%%	\begin{pgfonlayer}{nodelayer}
%%		\node [style=nothing] (17) at (4.5, 0.5) {};
%%		\node [style=nothing] (18) at (4, 0.5) {};
%%		\node [style=nothing] (19) at (3.5, 0.5) {};
%%		\node [style=nothing] (20) at (3, 0.5) {};
%%		\node [style=nothing] (21) at (3, 2.5) {};
%%		\node [style=nothing] (22) at (4, 2.5) {};
%%		\node [style=nothing] (23) at (3.5, 2.5) {};
%%		\node [style=nothing] (24) at (4.5, 2.5) {};
%%		\node [style=dot] (25) at (3, 1.25) {};
%%		\node [style=dot] (26) at (3.5, 1.25) {};
%%		\node [style=dot] (27) at (3, 1.75) {};
%%		\node [style=dot] (28) at (3.5, 1.75) {};
%%		\node [style=oplus] (29) at (4, 1.75) {};
%%		\node [style=oplus] (30) at (4.5, 1.25) {};
%%	\end{pgfonlayer}
%%	\begin{pgfonlayer}{edgelayer}
%%		\draw (20) to (25);
%%		\draw (25) to (27);
%%		\draw (27) to (21);
%%		\draw (23) to (28);
%%		\draw (28) to (26);
%%		\draw (26) to (19);
%%		\draw (18) to (29);
%%		\draw (29) to (22);
%%		\draw (24) to (30);
%%		\draw (30) to (17);
%%		\draw (30) to (26);
%%		\draw (26) to (25);
%%		\draw (27) to (28);
%%		\draw (28) to (29);
%%	\end{pgfonlayer}
%%\end{tikzpicture}
%%$}
%%
%%\item
%%\label{TOF.11}
%%{\hfil
%%$
%%\begin{tikzpicture}
%%	\begin{pgfonlayer}{nodelayer}
%%		\node [style=nothing] (0) at (0, 0.5) {};
%%		\node [style=nothing] (1) at (-0.5, 0.5) {};
%%		\node [style=nothing] (2) at (-1, 0.5) {};
%%		\node [style=nothing] (3) at (-1.5, 0.5) {};
%%		\node [style=nothing] (4) at (-0.5, 2.5) {};
%%		\node [style=nothing] (5) at (0, 2.5) {};
%%		\node [style=dot] (6) at (-1.5, 1) {};
%%		\node [style=dot] (7) at (-1, 1.5) {};
%%		\node [style=dot] (8) at (-0.5, 1.5) {};
%%		\node [style=oplus] (9) at (-1, 1) {};
%%		\node [style=oplus] (10) at (0, 1.5) {};
%%		\node [style=nothing] (11) at (-1.5, 2.5) {};
%%		\node [style=nothing] (12) at (-1, 2.5) {};
%%		\node [style=oplus] (13) at (-1, 2) {};
%%		\node [style=dot] (14) at (-1.5, 2) {};
%%	\end{pgfonlayer}
%%	\begin{pgfonlayer}{edgelayer}
%%		\draw (6) to (9);
%%		\draw (7) to (8);
%%		\draw (8) to (10);
%%		\draw (0) to (10);
%%		\draw (10) to (5);
%%		\draw (4) to (8);
%%		\draw (8) to (1);
%%		\draw (2) to (9);
%%		\draw (9) to (7);
%%		\draw (6) to (3);
%%		\draw (6) to (14);
%%		\draw (14) to (11);
%%		\draw (12) to (13);
%%		\draw (13) to (7);
%%		\draw (13) to (14);
%%	\end{pgfonlayer}
%%\end{tikzpicture}
%%=
%%\begin{tikzpicture}
%%	\begin{pgfonlayer}{nodelayer}
%%		\node [style=nothing] (0) at (2, 0.5) {};
%%		\node [style=nothing] (1) at (1, 0.5) {};
%%		\node [style=nothing] (2) at (1.5, 0.5) {};
%%		\node [style=nothing] (3) at (0.5, 0.5) {};
%%		\node [style=dot] (4) at (0.5, 1.25) {};
%%		\node [style=dot] (5) at (1.5, 1.25) {};
%%		\node [style=oplus] (6) at (2, 1.25) {};
%%		\node [style=nothing] (7) at (1.5, 2.5) {};
%%		\node [style=nothing] (8) at (1, 2.5) {};
%%		\node [style=nothing] (9) at (0.5, 2.5) {};
%%		\node [style=nothing] (10) at (2, 2.5) {};
%%		\node [style=dot] (11) at (1, 1.75) {};
%%		\node [style=dot] (12) at (1.5, 1.75) {};
%%		\node [style=oplus] (13) at (2, 1.75) {};
%%	\end{pgfonlayer}
%%	\begin{pgfonlayer}{edgelayer}
%%		\draw (3) to (4);
%%		\draw (2) to (5);
%%		\draw (6) to (0);
%%		\draw (6) to (5);
%%		\draw (5) to (4);
%%		\draw (11) to (1);
%%		\draw (5) to (12);
%%		\draw (12) to (7);
%%		\draw (10) to (13);
%%		\draw (13) to (6);
%%		\draw (13) to (12);
%%		\draw (12) to (11);
%%		\draw (4) to (9);
%%		\draw (11) to (8);
%%	\end{pgfonlayer}
%%\end{tikzpicture}
%%$}
%%
%%\item
%%\label{TOF.12}
%%{\hfil
%%$
%%\begin{tikzpicture}
%%	\begin{pgfonlayer}{nodelayer}
%%		\node [style=nothing] (0) at (-0.5, 0.5) {};
%%		\node [style=nothing] (1) at (0, 0.5) {};
%%		\node [style=nothing] (2) at (-1, 0.5) {};
%%		\node [style=nothing] (3) at (-1.5, 0.5) {};
%%		\node [style=nothing] (4) at (-0.5, 2.5) {};
%%		\node [style=nothing] (5) at (-1.5, 2.5) {};
%%		\node [style=nothing] (6) at (0, 2.5) {};
%%		\node [style=nothing] (7) at (-1, 2.5) {};
%%		\node [style=dot] (8) at (-1.5, 1) {};
%%		\node [style=dot] (9) at (-1, 1) {};
%%		\node [style=oplus] (10) at (-0.5, 1) {};
%%		\node [style=oplus] (11) at (0, 1.5) {};
%%		\node [style=dot] (12) at (-1, 1.5) {};
%%		\node [style=dot] (13) at (-0.5, 1.5) {};
%%		\node [style=oplus] (14) at (-0.5, 2) {};
%%		\node [style=dot] (15) at (-1.5, 2) {};
%%		\node [style=dot] (16) at (-1, 2) {};
%%	\end{pgfonlayer}
%%	\begin{pgfonlayer}{edgelayer}
%%		\draw (8) to (9);
%%		\draw (9) to (10);
%%		\draw (12) to (13);
%%		\draw (13) to (11);
%%		\draw (15) to (16);
%%		\draw (16) to (14);
%%		\draw (3) to (8);
%%		\draw (8) to (15);
%%		\draw (15) to (5);
%%		\draw (7) to (16);
%%		\draw (16) to (12);
%%		\draw (12) to (9);
%%		\draw (9) to (2);
%%		\draw (0) to (10);
%%		\draw (10) to (13);
%%		\draw (13) to (14);
%%		\draw (14) to (4);
%%		\draw (6) to (11);
%%		\draw (11) to (1);
%%	\end{pgfonlayer}
%%\end{tikzpicture}
%%=
%%\begin{tikzpicture}
%%	\begin{pgfonlayer}{nodelayer}
%%		\node [style=nothing] (0) at (1.5, 0.25) {};
%%		\node [style=nothing] (1) at (2, 0.25) {};
%%		\node [style=nothing] (2) at (1, 0.25) {};
%%		\node [style=nothing] (3) at (0.5, 0.25) {};
%%		\node [style=nothing] (4) at (1.5, 2.25) {};
%%		\node [style=nothing] (5) at (0.5, 2.25) {};
%%		\node [style=nothing] (6) at (2, 2.25) {};
%%		\node [style=nothing] (7) at (1, 2.25) {};
%%		\node [style=dot] (8) at (1, 1.5) {};
%%		\node [style=dot] (9) at (1.5, 1.5) {};
%%		\node [style=dot] (10) at (0.5, 1) {};
%%		\node [style=dot] (11) at (1, 1) {};
%%		\node [style=oplus] (12) at (2, 1) {};
%%		\node [style=oplus] (13) at (2, 1.5) {};
%%	\end{pgfonlayer}
%%	\begin{pgfonlayer}{edgelayer}
%%		\draw (8) to (9);
%%		\draw (3) to (10);
%%		\draw (10) to (5);
%%		\draw (2) to (11);
%%		\draw (11) to (8);
%%		\draw (8) to (7);
%%		\draw (0) to (9);
%%		\draw (9) to (4);
%%		\draw (1) to (12);
%%		\draw (12) to (13);
%%		\draw (13) to (6);
%%		\draw (13) to (9);
%%		\draw (12) to (11);
%%		\draw (11) to (10);
%%	\end{pgfonlayer}
%%\end{tikzpicture}
%%$}
%%
%%\item
%%\label{TOF.13}
%%{\hfil
%%$
%%\begin{tikzpicture}
%%	\begin{pgfonlayer}{nodelayer}
%%		\node [style=nothing] (0) at (0, 0.5) {};
%%		\node [style=nothing] (1) at (-1, 0.5) {};
%%		\node [style=nothing] (2) at (-0.5, 0.5) {};
%%		\node [style=nothing] (3) at (-1.5, 0.5) {};
%%		\node [style=nothing] (4) at (0, 2.5) {};
%%		\node [style=dot] (5) at (-1.5, 1) {};
%%		\node [style=dot] (6) at (-1, 1) {};
%%		\node [style=dot] (7) at (-0.5, 1.5) {};
%%		\node [style=oplus] (8) at (-0.5, 1) {};
%%		\node [style=oplus] (9) at (0, 1.5) {};
%%		\node [style=nothing] (10) at (-0.5, 2.5) {};
%%		\node [style=nothing] (11) at (-1.5, 2.5) {};
%%		\node [style=nothing] (12) at (-1, 2.5) {};
%%		\node [style=oplus] (13) at (-0.5, 2) {};
%%		\node [style=dot] (14) at (-1, 2) {};
%%		\node [style=dot] (15) at (-1.5, 2) {};
%%	\end{pgfonlayer}
%%	\begin{pgfonlayer}{edgelayer}
%%		\draw (5) to (3);
%%		\draw (6) to (1);
%%		\draw (2) to (8);
%%		\draw (8) to (7);
%%		\draw (4) to (9);
%%		\draw (9) to (0);
%%		\draw (8) to (6);
%%		\draw (6) to (5);
%%		\draw (9) to (7);
%%		\draw (5) to (15);
%%		\draw (15) to (11);
%%		\draw (12) to (14);
%%		\draw (14) to (6);
%%		\draw (7) to (13);
%%		\draw (13) to (10);
%%		\draw (13) to (14);
%%		\draw (14) to (15);
%%	\end{pgfonlayer}
%%\end{tikzpicture}
%%=
%%\begin{tikzpicture}
%%	\begin{pgfonlayer}{nodelayer}
%%		\node [style=nothing] (0) at (2, 0.25) {};
%%		\node [style=nothing] (1) at (1, 0.25) {};
%%		\node [style=nothing] (2) at (1.5, 0.25) {};
%%		\node [style=nothing] (3) at (0.5, 0.25) {};
%%		\node [style=dot] (4) at (0.5, 1) {};
%%		\node [style=dot] (5) at (1, 1) {};
%%		\node [style=oplus] (6) at (2, 1) {};
%%		\node [style=nothing] (7) at (1.5, 2.25) {};
%%		\node [style=nothing] (8) at (1, 2.25) {};
%%		\node [style=nothing] (9) at (2, 2.25) {};
%%		\node [style=nothing] (10) at (0.5, 2.25) {};
%%		\node [style=dot] (11) at (1.5, 1.5) {};
%%		\node [style=oplus] (12) at (2, 1.5) {};
%%	\end{pgfonlayer}
%%	\begin{pgfonlayer}{edgelayer}
%%		\draw (0) to (6);
%%		\draw (1) to (5);
%%		\draw (4) to (3);
%%		\draw (4) to (5);
%%		\draw (5) to (6);
%%		\draw (11) to (12);
%%		\draw (12) to (9);
%%		\draw (12) to (6);
%%		\draw (2) to (11);
%%		\draw (4) to (10);
%%		\draw (8) to (5);
%%		\draw (11) to (7);
%%	\end{pgfonlayer}
%%\end{tikzpicture}
%%$}
%%
%%\item
%%\label{TOF.14}
%%{\hfil
%%$
%%\begin{tikzpicture}
%%	\begin{pgfonlayer}{nodelayer}
%%		\node [style=nothing] (0) at (0, 0.5) {};
%%		\node [style=nothing] (1) at (-0.5, 0.5) {};
%%		\node [style=nothing] (2) at (-0.5, 2.5) {};
%%		\node [style=nothing] (3) at (0, 2.5) {};
%%		\node [style=oplus] (4) at (0, 1) {};
%%		\node [style=oplus] (5) at (0, 2) {};
%%		\node [style=oplus] (6) at (-0.5, 1.5) {};
%%		\node [style=dot] (7) at (-0.5, 2) {};
%%		\node [style=dot] (8) at (0, 1.5) {};
%%		\node [style=dot] (9) at (-0.5, 1) {};
%%	\end{pgfonlayer}
%%	\begin{pgfonlayer}{edgelayer}
%%		\draw (1) to (9);
%%		\draw (9) to (6);
%%		\draw (6) to (7);
%%		\draw (7) to (2);
%%		\draw (3) to (5);
%%		\draw (5) to (8);
%%		\draw (8) to (4);
%%		\draw (4) to (0);
%%		\draw (4) to (9);
%%		\draw (8) to (6);
%%		\draw (5) to (7);
%%	\end{pgfonlayer}
%%\end{tikzpicture}
%%=
%%\begin{tikzpicture}
%%	\begin{pgfonlayer}{nodelayer}
%%		\node [style=nothing] (0) at (1, 0.5) {};
%%		\node [style=nothing] (1) at (0.5, 0.5) {};
%%		\node [style=nothing] (2) at (0.5, 2.5) {};
%%		\node [style=nothing] (3) at (1, 2.5) {};
%%	\end{pgfonlayer}
%%	\begin{pgfonlayer}{edgelayer}
%%		\draw [in=-90, out=90, looseness=1.25] (1) to (3);
%%		\draw [in=-90, out=90, looseness=1.25] (0) to (2);
%%	\end{pgfonlayer}
%%\end{tikzpicture}
%%$}
%%
%%\item
%%\label{TOF.15}
%%{\hfil
%%$
%%\begin{tikzpicture}
%%	\begin{pgfonlayer}{nodelayer}
%%		\node [style=nothing] (0) at (-1.75, 0.5) {};
%%		\node [style=nothing] (1) at (-1.25, 0.5) {};
%%		\node [style=nothing] (2) at (-0.75, 0.5) {};
%%		\node [style=nothing] (3) at (-1.75, 2.5) {};
%%		\node [style=nothing] (4) at (-1.25, 2.5) {};
%%		\node [style=nothing] (5) at (-0.75, 2.5) {};
%%		\node [style=dot] (6) at (-1.75, 1.5) {};
%%		\node [style=dot] (7) at (-1.25, 1.5) {};
%%		\node [style=oplus] (8) at (-0.75, 1.5) {};
%%	\end{pgfonlayer}
%%	\begin{pgfonlayer}{edgelayer}
%%		\draw (0) to (6);
%%		\draw (6) to (3);
%%		\draw (4) to (7);
%%		\draw (7) to (1);
%%		\draw (2) to (8);
%%		\draw (8) to (5);
%%		\draw (8) to (7);
%%		\draw (7) to (6);
%%	\end{pgfonlayer}
%%\end{tikzpicture}
%%=
%%\begin{tikzpicture}
%%	\begin{pgfonlayer}{nodelayer}
%%		\node [style=nothing] (0) at (-1.75, 0.5) {};
%%		\node [style=nothing] (1) at (-1.25, 0.5) {};
%%		\node [style=nothing] (2) at (-0.75, 0.5) {};
%%		\node [style=dot] (3) at (-1.75, 1.5) {};
%%		\node [style=dot] (4) at (-1.25, 1.5) {};
%%		\node [style=oplus] (5) at (-0.75, 1.5) {};
%%		\node [style=nothing] (6) at (-1.75, 2.5) {};
%%		\node [style=nothing] (7) at (-1.25, 2.5) {};
%%		\node [style=nothing] (8) at (-0.75, 2.5) {};
%%	\end{pgfonlayer}
%%	\begin{pgfonlayer}{edgelayer}
%%		\draw [in=-90, out=90, looseness=1.25] (0) to (4);
%%		\draw [in=-90, out=90, looseness=1.25] (4) to (6);
%%		\draw [in=-90, out=90, looseness=1.25] (3) to (7);
%%		\draw [in=90, out=-90, looseness=1.25] (3) to (1);
%%		\draw (2) to (5);
%%		\draw (5) to (8);
%%		\draw (3) to (4);
%%		\draw (4) to (5);
%%	\end{pgfonlayer}
%%\end{tikzpicture}
%%$}
%%
%%\item
%%\label{TOF.16}
%%{\hfil
%%$
%%\begin{tikzpicture}
%%	\begin{pgfonlayer}{nodelayer}
%%		\node [style=nothing] (0) at (2.5, 0.5) {};
%%		\node [style=nothing] (1) at (1, 0.5) {};
%%		\node [style=nothing] (2) at (2, 0.5) {};
%%		\node [style=nothing] (3) at (0.5, 0.5) {};
%%		\node [style=X] (4) at (1.5, 0.75) {};
%%		\node [style=oplus] (5) at (1.5, 1.75) {};
%%		\node [style=oplus] (6) at (1.5, 2.75) {};
%%		\node [style=dot] (7) at (1.5, 2.25) {};
%%		\node [style=dot] (8) at (2, 2.25) {};
%%		\node [style=dot] (9) at (1, 1.75) {};
%%		\node [style=dot] (10) at (0.5, 1.75) {};
%%		\node [style=dot] (11) at (1, 2.75) {};
%%		\node [style=dot] (12) at (0.5, 2.75) {};
%%		\node [style=oplus] (13) at (2.5, 2.25) {};
%%		\node [style=X] (14) at (1.5, 3.75) {};
%%		\node [style=nothing] (15) at (2.5, 4) {};
%%		\node [style=nothing] (16) at (0.5, 4) {};
%%		\node [style=nothing] (17) at (1, 4) {};
%%		\node [style=nothing] (18) at (2, 4) {};
%%	\end{pgfonlayer}
%%	\begin{pgfonlayer}{edgelayer}
%%		\draw (3) to (10);
%%		\draw (10) to (12);
%%		\draw (12) to (16);
%%		\draw (11) to (9);
%%		\draw (15) to (13);
%%		\draw (13) to (0);
%%		\draw (13) to (8);
%%		\draw (8) to (7);
%%		\draw (9) to (5);
%%		\draw (9) to (10);
%%		\draw (12) to (11);
%%		\draw (6) to (11);
%%		\draw (4) to (5);
%%		\draw (5) to (7);
%%		\draw (7) to (6);
%%		\draw (14) to (6);
%%		\draw [style=simple, in=90, out=-90, looseness=1.25] (17) to (8);
%%		\draw [style=simple, in=90, out=-90, looseness=1.25] (8) to (1);
%%		\draw [style=simple, in=270, out=90] (2) to (9);
%%		\draw [style=simple, in=270, out=90] (11) to (18);
%%	\end{pgfonlayer}
%%\end{tikzpicture}
%%=
%%\begin{tikzpicture}
%%	\begin{pgfonlayer}{nodelayer}
%%		\node [style=nothing] (0) at (2.5, 0.5) {};
%%		\node [style=nothing] (1) at (2, 0.5) {};
%%		\node [style=nothing] (2) at (1, 0.5) {};
%%		\node [style=nothing] (3) at (0.5, 0.5) {};
%%		\node [style=X] (4) at (1.5, 1.25) {};
%%		\node [style=oplus] (5) at (1.5, 1.75) {};
%%		\node [style=oplus] (6) at (1.5, 2.75) {};
%%		\node [style=dot] (7) at (1.5, 2.25) {};
%%		\node [style=dot] (8) at (2, 2.25) {};
%%		\node [style=dot] (9) at (1, 1.75) {};
%%		\node [style=dot] (10) at (0.5, 1.75) {};
%%		\node [style=dot] (11) at (1, 2.75) {};
%%		\node [style=dot] (12) at (0.5, 2.75) {};
%%		\node [style=oplus] (13) at (2.5, 2.25) {};
%%		\node [style=X] (14) at (1.5, 3.25) {};
%%		\node [style=nothing] (15) at (2.5, 4) {};
%%		\node [style=nothing] (16) at (0.5, 4) {};
%%		\node [style=nothing] (17) at (2, 4) {};
%%		\node [style=nothing] (18) at (1, 4) {};
%%	\end{pgfonlayer}
%%	\begin{pgfonlayer}{edgelayer}
%%		\draw (3) to (10);
%%		\draw (10) to (12);
%%		\draw (12) to (16);
%%		\draw (11) to (9);
%%		\draw (15) to (13);
%%		\draw (13) to (0);
%%		\draw (13) to (8);
%%		\draw (8) to (7);
%%		\draw (9) to (5);
%%		\draw (9) to (10);
%%		\draw (12) to (11);
%%		\draw (6) to (11);
%%		\draw (4) to (5);
%%		\draw (5) to (7);
%%		\draw (7) to (6);
%%		\draw (14) to (6);
%%		\draw [style=simple] (17) to (8);
%%		\draw [style=simple] (8) to (1);
%%		\draw [style=simple] (2) to (9);
%%		\draw [style=simple] (11) to (18);
%%	\end{pgfonlayer}
%%\end{tikzpicture}
%%$}
%%\end{enumerate}
%%\end{multicols}
%%
%%Where the controlled-not gate is derived:
%%$$
%%\begin{tikzpicture}
%%	\begin{pgfonlayer}{nodelayer}
%%		\node [style=dot] (1) at (1.5, 1) {};
%%		\node [style=oplus] (2) at (2, 1) {};
%%		\node [style=none] (5) at (1.5, 0.25) {};
%%		\node [style=none] (6) at (2, 0.25) {};
%%		\node [style=none] (7) at (2, 1.75) {};
%%		\node [style=none] (8) at (1.5, 1.75) {};
%%	\end{pgfonlayer}
%%	\begin{pgfonlayer}{edgelayer}
%%		\draw (5.center) to (1);
%%		\draw (1) to (8.center);
%%		\draw (7.center) to (2);
%%		\draw (2) to (6.center);
%%		\draw (2) to (1);
%%	\end{pgfonlayer}
%%\end{tikzpicture}
%%:=
%%\begin{tikzpicture}
%%	\begin{pgfonlayer}{nodelayer}
%%		\node [style=dot] (0) at (1, 1) {};
%%		\node [style=dot] (1) at (1.5, 1) {};
%%		\node [style=oplus] (2) at (2, 1) {};
%%		\node [style=X] (3) at (1, 1.5) {$1$};
%%		\node [style=X] (4) at (1, 0.5) {$1$};
%%		\node [style=none] (5) at (1.5, 0.25) {};
%%		\node [style=none] (6) at (2, 0.25) {};
%%		\node [style=none] (7) at (2, 1.75) {};
%%		\node [style=none] (8) at (1.5, 1.75) {};
%%	\end{pgfonlayer}
%%	\begin{pgfonlayer}{edgelayer}
%%		\draw (5.center) to (1);
%%		\draw (1) to (8.center);
%%		\draw (7.center) to (2);
%%		\draw (2) to (6.center);
%%		\draw (2) to (1);
%%		\draw (1) to (0);
%%		\draw (0) to (3);
%%		\draw (4) to (0);
%%	\end{pgfonlayer}
%%\end{tikzpicture}
%%$$
%%
%%\subsubsection{$(\FPar_2,\times)$}
%%\label{subsubsec:presentations:three:par}
%%$(\FPar_2,\times)$ is presented by the generators and equations in \S \ref{subsubsec:presentations:two:par} as well as the additional generator $
%%\begin{tikzpicture}
%%	\begin{pgfonlayer}{nodelayer}
%%		\node [style=none] (0) at (-3.75, 0.5) {};
%%		\node [style=none] (1) at (-3.75, -0.25) {};
%%		\node [style=andin] (2) at (-3.75, -0.25) {};
%%		\node [style=none] (3) at (-4, -1) {};
%%		\node [style=none] (4) at (-3.5, -1) {};
%%	\end{pgfonlayer}
%%	\begin{pgfonlayer}{edgelayer}
%%		\draw (0.center) to (1.center);
%%		\draw [in=-60, out=90, looseness=1.00] (4.center) to (1.center);
%%		\draw [in=90, out=-120, looseness=1.00] (1.center) to (3.center);
%%	\end{pgfonlayer}
%%\end{tikzpicture}
%%$ (the and gate),  so that 
%%$
%%\left(
%%\begin{tikzpicture}
%%	\begin{pgfonlayer}{nodelayer}
%%		\node [style=none] (0) at (-3.75, 0.5) {};
%%		\node [style=none] (1) at (-3.75, -0.25) {};
%%		\node [style=andin] (2) at (-3.75, -0.25) {};
%%		\node [style=none] (3) at (-4, -1) {};
%%		\node [style=none] (4) at (-3.5, -1) {};
%%	\end{pgfonlayer}
%%	\begin{pgfonlayer}{edgelayer}
%%		\draw (0.center) to (1.center);
%%		\draw [in=-60, out=90, looseness=1.00] (4.center) to (1.center);
%%		\draw [in=90, out=-120, looseness=1.00] (1.center) to (3.center);
%%	\end{pgfonlayer}
%%\end{tikzpicture},
%%\begin{tikzpicture}
%%	\begin{pgfonlayer}{nodelayer}
%%		\node [style=none] (0) at (0.75, 0.5) {};
%%		\node [style=none] (1) at (0.75, -0.25) {};
%%		\node [style=X] (2) at (0.75, -0.25) {$1$};
%%	\end{pgfonlayer}
%%	\begin{pgfonlayer}{edgelayer}
%%		\draw (0.center) to (1.center);
%%	\end{pgfonlayer}
%%\end{tikzpicture},
%%\begin{tikzpicture}
%%	\begin{pgfonlayer}{nodelayer}
%%		\node [style=none] (0) at (0.75, -1) {};
%%		\node [style=none] (1) at (0.75, -0.25) {};
%%		\node [style=Z] (2) at (0.75, -0.25) {};
%%		\node [style=none] (3) at (0.5, 0.5) {};
%%		\node [style=none] (4) at (1, 0.5) {};
%%	\end{pgfonlayer}
%%	\begin{pgfonlayer}{edgelayer}
%%		\draw (0.center) to (1.center);
%%		\draw [in=60, out=-90] (4.center) to (1.center);
%%		\draw [in=-90, out=120] (1.center) to (3.center);
%%	\end{pgfonlayer}
%%\end{tikzpicture},
%%\begin{tikzpicture}
%%	\begin{pgfonlayer}{nodelayer}
%%		\node [style=none] (0) at (0.75, -0.25) {};
%%		\node [style=none] (1) at (0.75, 0.5) {};
%%		\node [style=Z] (2) at (0.75, 0.5) {};
%%	\end{pgfonlayer}
%%	\begin{pgfonlayer}{edgelayer}
%%		\draw (0.center) to (1.center);
%%	\end{pgfonlayer}
%%\end{tikzpicture}
%%\right)
%%$ forms a bicommutative bialgebra; and additionally:
%%$$
%%\begin{tikzpicture}
%%	\begin{pgfonlayer}{nodelayer}
%%		\node [style=none] (0) at (-7, 1) {};
%%		\node [style=none] (1) at (-7, 0.5) {};
%%		\node [style=Z] (2) at (-7, -0.25) {};
%%		\node [style=none] (3) at (-7, -0.75) {};
%%		\node [style=andin] (4) at (-7, 0.5) {};
%%	\end{pgfonlayer}
%%	\begin{pgfonlayer}{edgelayer}
%%		\draw (3.center) to (2.center);
%%		\draw [in=-60, out=60, looseness=1.25] (2.center) to (1);
%%		\draw [in=120, out=-120, looseness=1.25] (1) to (2.center);
%%		\draw (1) to (0.center);
%%	\end{pgfonlayer}
%%\end{tikzpicture}
%%\eref{antispecial}
%%\begin{tikzpicture}
%%	\begin{pgfonlayer}{nodelayer}
%%		\node [style=none] (0) at (-7, 1) {};
%%		\node [style=none] (1) at (-7, -0.75) {};
%%	\end{pgfonlayer}
%%	\begin{pgfonlayer}{edgelayer}
%%		\draw (1.center) to (0.center);
%%	\end{pgfonlayer}
%%\end{tikzpicture},
%%\hspace*{.5cm}
%%\begin{tikzpicture}
%%	\begin{pgfonlayer}{nodelayer}
%%		\node [style=andin] (4) at (1.25, 0.5) {};
%%		\node [style=X] (5) at (0.75, -0.5) {};
%%		\node [style=none] (6) at (0.5, -1) {};
%%		\node [style=none] (7) at (1, -1) {};
%%		\node [style=none] (8) at (1.75, -1) {};
%%		\node [style=none] (9) at (1.25, 0.5) {};
%%		\node [style=none] (10) at (1.25, 1.5) {};
%%	\end{pgfonlayer}
%%	\begin{pgfonlayer}{edgelayer}
%%		\draw [in=-30, out=90] (8.center) to (9.center);
%%		\draw [in=90, out=-150] (9.center) to (5);
%%		\draw [in=90, out=-45] (5) to (7.center);
%%		\draw [in=-135, out=90] (6.center) to (5);
%%		\draw (9.center) to (10.center);
%%	\end{pgfonlayer}
%%\end{tikzpicture}
%%  \eref{ring.mul}
%%\begin{tikzpicture}
%%	\begin{pgfonlayer}{nodelayer}
%%		\node [style=none] (0) at (1, 0) {};
%%		\node [style=none] (1) at (0.5, -1.25) {};
%%		\node [style=none] (2) at (1.75, -0.75) {};
%%		\node [style=none] (3) at (1.33, 0.75) {};
%%		\node [style=andin] (4) at (1, 0) {};
%%		\node [style=none] (5) at (1.75, 0) {};
%%		\node [style=none] (6) at (1, -1.25) {};
%%		\node [style=none] (7) at (1.75, -0.75) {};
%%		\node [style=none] (8) at (1.33, 0.75) {};
%%		\node [style=andin] (9) at (1.75, 0) {};
%%		\node [style=X] (10) at (1.33, 0.75) {};
%%		\node [style=none] (11) at (1.33, 1.25) {};
%%		\node [style=none] (12) at (1.75, -1.25) {};
%%		\node [style=Z] (13) at (1.75, -0.75) {};
%%	\end{pgfonlayer}
%%	\begin{pgfonlayer}{edgelayer}
%%		\draw [in=-135, out=90] (0.center) to (3.center);
%%		\draw [in=165, out=-30, looseness=1.25] (0.center) to (2.center);
%%		\draw [in=-45, out=90] (5.center) to (8.center);
%%		\draw [in=45, out=-45, looseness=1.25] (5.center) to (7.center);
%%		\draw (10) to (11.center);
%%		\draw [in=90, out=-150] (4) to (1.center);
%%		\draw [in=-150, out=90] (6.center) to (9);
%%		\draw (12.center) to (13);
%%	\end{pgfonlayer}
%%\end{tikzpicture},
%%\hspace*{.5cm}
%%\begin{tikzpicture}
%%	\begin{pgfonlayer}{nodelayer}
%%		\node [style=none] (0) at (2, 0) {};
%%		\node [style=none] (1) at (1.75, -0.75) {};
%%		\node [style=none] (2) at (2.25, -0.75) {};
%%		\node [style=none] (3) at (2, 0.5) {};
%%		\node [style=none] (4) at (2.25, -1) {};
%%		\node [style=X] (5) at (1.75, -0.75) {};
%%		\node [style=andin] (6) at (2, 0) {};
%%	\end{pgfonlayer}
%%	\begin{pgfonlayer}{edgelayer}
%%		\draw (0.center) to (3.center);
%%		\draw [in=90, out=-45] (0.center) to (2.center);
%%		\draw (4.center) to (2.center);
%%		\draw [in=-135, out=90] (1.center) to (0.center);
%%	\end{pgfonlayer}
%%\end{tikzpicture}
%%\eref{ring.unit}
%%\begin{tikzpicture}
%%	\begin{pgfonlayer}{nodelayer}
%%		\node [style=none] (12) at (2, 0.5) {};
%%		\node [style=none] (14) at (2, -1) {};
%%		\node [style=X] (15) at (2, 0) {};
%%		\node [style=Z] (16) at (2, -0.5) {};
%%	\end{pgfonlayer}
%%	\begin{pgfonlayer}{edgelayer}
%%		\draw (15) to (12.center);
%%		\draw (16) to (14.center);
%%	\end{pgfonlayer}
%%\end{tikzpicture},
%%\hspace*{.5cm}
%%\begin{tikzpicture}
%%	\begin{pgfonlayer}{nodelayer}
%%		\node [style=none] (0) at (0.75, 0.5) {};
%%		\node [style=none] (1) at (0.75, -0.25) {};
%%		\node [style=andin] (2) at (0.75, -0.25) {};
%%		\node [style=none] (3) at (0.5, -1) {};
%%		\node [style=none] (4) at (1, -1) {};
%%		\node [style=X] (5) at (0.75, 0.5) {$1$};
%%	\end{pgfonlayer}
%%	\begin{pgfonlayer}{edgelayer}
%%		\draw (0.center) to (1.center);
%%		\draw [in=-60, out=90] (4.center) to (1.center);
%%		\draw [in=90, out=-120] (1.center) to (3.center);
%%	\end{pgfonlayer}
%%\end{tikzpicture}
%%  \eref{bi.two}
%%\begin{tikzpicture}
%%	\begin{pgfonlayer}{nodelayer}
%%		\node [style=none] (3) at (0.5, -1) {};
%%		\node [style=none] (4) at (1, -1) {};
%%		\node [style=X] (5) at (0.5, 0.5) {$1$};
%%		\node [style=X] (6) at (1, 0.5) {$1$};
%%	\end{pgfonlayer}
%%	\begin{pgfonlayer}{edgelayer}
%%		\draw (3.center) to (5);
%%		\draw (6) to (4.center);
%%	\end{pgfonlayer}
%%\end{tikzpicture}
%%$$
%\subsubsection{$(\FSpan_2,\times)$}
%\label{subsubsec:presentations:three:span}
%
%$(\FSpan_2,\times)$ is presented by the generators and equations of \S \ref{subsubsec:presentations:three:span} as well as the generator 
%$\begin{tikzpicture}
%	\begin{pgfonlayer}{nodelayer}
%		\node [style=none] (0) at (0.75, 0.5) {};
%		\node [style=none] (1) at (0.75, -0.25) {};
%		\node [style=Z] (2) at (0.75, -0.25) {};
%	\end{pgfonlayer}
%	\begin{pgfonlayer}{edgelayer}
%		\draw (0.center) to (1.center);
%	\end{pgfonlayer}
%\end{tikzpicture}$ 
%and the equation making the codiagonal map counital:
%$$
%  \begin{tikzpicture}[rotate=90,yscale=-1]
%	\begin{pgfonlayer}{nodelayer}
%		\node [style=Z] (0) at (-9, -0) {};
%		\node [style=none] (1) at (-8.25, -0) {};
%		\node [style=Z] (2) at (-9.75, 0.25) {};
%		\node [style=none] (3) at (-10, -0.25) {};
%	\end{pgfonlayer}
%	\begin{pgfonlayer}{edgelayer}
%		\draw [in=-150, out=0, looseness=1.00] (3.center) to (0);
%		\draw [in=150, out=0, looseness=1.00] (2.center) to (0);
%		\draw (0) to (1.center);
%	\end{pgfonlayer}
%  \end{tikzpicture}
%  \eref{unit}
%  \begin{tikzpicture}[rotate=90]
%	\begin{pgfonlayer}{nodelayer}
%		\node [style=none] (0) at (-9, 0.25) {};
%		\node [style=none] (1) at (-9.75, 0.25) {};
%	\end{pgfonlayer}
%	\begin{pgfonlayer}{edgelayer}
%		\draw (1) to (0.center);
%	\end{pgfonlayer}
%  \end{tikzpicture}
%$$
%%
%%
%%\end{comment}




%
%This prop is equivalently presented in terms of the 
%
%\begin{figure}[H]
%	\noindent
%	\scalebox{1.0}{%
%		\vbox{%
%			\begin{mdframed}
%				\begin{multicols}{2}
%					\begin{enumerate}[label={\bf [CNOT.\arabic*]}, ref={\bf [CNOT.\arabic*]}, wide = 0pt, leftmargin = 2em]
%						\item
%						\label{CNOT.1}
%						{\hfil
%							$
%			\begin{tikzpicture}
%	\begin{pgfonlayer}{nodelayer}
%		\node [style=nothing] (26) at (0, 6) {};
%		\node [style=nothing] (27) at (-0.5, 6) {};
%		\node [style=oplus] (28) at (0, 6.5) {};
%		\node [style=dot] (29) at (-0.5, 6.5) {};
%		\node [style=dot] (30) at (0, 7) {};
%		\node [style=oplus] (31) at (-0.5, 7) {};
%		\node [style=oplus] (32) at (0, 7.5) {};
%		\node [style=dot] (33) at (-0.5, 7.5) {};
%		\node [style=nothing] (34) at (0, 8) {};
%		\node [style=nothing] (35) at (-0.5, 8) {};
%	\end{pgfonlayer}
%	\begin{pgfonlayer}{edgelayer}
%		\draw [style=simple] (26) to (34);
%		\draw [style=simple] (27) to (35);
%		\draw [style=simple] (28) to (29);
%		\draw [style=simple] (30) to (31);
%		\draw [style=simple] (32) to (33);
%	\end{pgfonlayer}
%\end{tikzpicture}
%							=
%							\begin{tikzpicture}
%	\begin{pgfonlayer}{nodelayer}
%		\node [style=nothing] (0) at (0, 0.5) {};
%		\node [style=nothing] (1) at (-0.5, 0.5) {};
%		\node [style=nothing] (2) at (-0.5, 1.5) {};
%		\node [style=nothing] (3) at (0, 1.5) {};
%	\end{pgfonlayer}
%	\begin{pgfonlayer}{edgelayer}
%		\draw [in=-90, out=90, looseness=1.25] (1) to (3);
%		\draw [in=-90, out=90, looseness=1.25] (0) to (2);
%	\end{pgfonlayer}
%\end{tikzpicture}
%$}
%						
%						
%						\item
%						\label{CNOT.2}
%						\hfil{
%							$
%							\begin{tikzpicture}
%	\begin{pgfonlayer}{nodelayer}
%		\node [style=nothing] (1) at (0, 0) {};
%		\node [style=nothing] (2) at (-0.5, 0) {};
%		\node [style=oplus] (3) at (0, 0.5) {};
%		\node [style=dot] (4) at (-0.5, 0.5) {};
%		\node [style=oplus] (5) at (0, 1) {};
%		\node [style=dot] (6) at (-0.5, 1) {};
%		\node [style=nothing] (7) at (0, 1.5) {};
%		\node [style=nothing] (8) at (-0.5, 1.5) {};
%	\end{pgfonlayer}
%	\begin{pgfonlayer}{edgelayer}
%		\draw [style=simple] (1) to (7);
%		\draw [style=simple] (2) to (8);
%		\draw [style=simple] (3) to (4);
%		\draw [style=simple] (5) to (6);
%	\end{pgfonlayer}
%\end{tikzpicture}
%							=
%							\begin{tikzpicture}
%	\begin{pgfonlayer}{nodelayer}
%		\node [style=nothing] (2) at (0, 0) {};
%		\node [style=nothing] (3) at (-0.5, 0) {};
%		\node [style=nothing] (4) at (0, 1.5) {};
%		\node [style=nothing] (5) at (-0.5, 1.5) {};
%	\end{pgfonlayer}
%	\begin{pgfonlayer}{edgelayer}
%		\draw [style=simple] (2) to (4);
%		\draw [style=simple] (3) to (5);
%	\end{pgfonlayer}
%\end{tikzpicture}
%							$}
%						
%						\item
%						\label{CNOT.3}
%						\hfil{
%							$
%							\begin{tikzpicture}
%	\begin{pgfonlayer}{nodelayer}
%		\node [style=nothing] (3) at (-1, 0) {};
%		\node [style=nothing] (4) at (-0.5, 0) {};
%		\node [style=nothing] (5) at (0, 0) {};
%		\node [style=oplus] (6) at (-1, 0.75) {};
%		\node [style=dot] (7) at (-0.5, 0.75) {};
%		\node [style=dot] (8) at (-0.5, 1.25) {};
%		\node [style=oplus] (9) at (0, 1.25) {};
%		\node [style=nothing] (10) at (-1, 2) {};
%		\node [style=nothing] (11) at (-0.5, 2) {};
%		\node [style=nothing] (12) at (0, 2) {};
%	\end{pgfonlayer}
%	\begin{pgfonlayer}{edgelayer}
%		\draw [style=simple] (3) to (10);
%		\draw [style=simple] (4) to (11);
%		\draw [style=simple] (5) to (12);
%		\draw [style=simple] (6) to (7);
%		\draw [style=simple] (8) to (9);
%	\end{pgfonlayer}
%\end{tikzpicture}
%							=
%							\begin{tikzpicture}
%	\begin{pgfonlayer}{nodelayer}
%		\node [style=nothing] (4) at (-1, 2.75) {};
%		\node [style=nothing] (5) at (-0.5, 2.75) {};
%		\node [style=nothing] (6) at (0, 2.75) {};
%		\node [style=oplus] (7) at (-1, 4) {};
%		\node [style=dot] (8) at (-0.5, 4) {};
%		\node [style=dot] (9) at (-0.5, 3.5) {};
%		\node [style=oplus] (10) at (0, 3.5) {};
%		\node [style=nothing] (11) at (-1, 4.75) {};
%		\node [style=nothing] (12) at (-0.5, 4.75) {};
%		\node [style=nothing] (13) at (0, 4.75) {};
%	\end{pgfonlayer}
%	\begin{pgfonlayer}{edgelayer}
%		\draw [style=simple] (4) to (11);
%		\draw [style=simple] (5) to (12);
%		\draw [style=simple] (6) to (13);
%		\draw [style=simple] (7) to (8);
%		\draw [style=simple] (9) to (10);
%	\end{pgfonlayer}
%\end{tikzpicture}
%							$}
%						
%						\item 
%						\label{CNOT.4}
%						\hfil{
%							\begin{tabular}{c}
%							$
%							\begin{tikzpicture}
%	\begin{pgfonlayer}{nodelayer}
%		\node [style=onein] (5) at (-0.5, 2.75) {};
%		\node [style=nothing] (6) at (0, 2.75) {};
%		\node [style=dot] (7) at (-0.5, 3.25) {};
%		\node [style=oplus] (8) at (0, 3.25) {};
%		\node [style=nothing] (9) at (-0.5, 3.75) {};
%		\node [style=nothing] (10) at (0, 3.75) {};
%	\end{pgfonlayer}
%	\begin{pgfonlayer}{edgelayer}
%		\draw [style=simple] (5) to (9);
%		\draw [style=simple] (6) to (10);
%		\draw [style=simple] (7) to (8);
%	\end{pgfonlayer}
%\end{tikzpicture}
%							=
%							\begin{tikzpicture}
%	\begin{pgfonlayer}{nodelayer}
%		\node [style=onein] (6) at (-0.5, 2.75) {};
%		\node [style=nothing] (7) at (0, 2.75) {};
%		\node [style=dot] (8) at (-0.5, 3.25) {};
%		\node [style=oplus] (9) at (0, 3.25) {};
%		\node [style=oneout] (10) at (-0.5, 3.75) {};
%		\node [style=nothing] (11) at (0, 4.75) {};
%		\node [style=onein] (12) at (-0.5, 4.25) {};
%		\node [style=nothing] (13) at (-0.5, 4.75) {};
%	\end{pgfonlayer}
%	\begin{pgfonlayer}{edgelayer}
%		\draw [style=simple] (6) to (10);
%		\draw [style=simple] (7) to (11);
%		\draw [style=simple] (8) to (9);
%		\draw [style=simple] (12) to (13);
%	\end{pgfonlayer}
%\end{tikzpicture}$\\
%							$ $\\
%							$\begin{tikzpicture}[tikzfig]
%	\begin{pgfonlayer}{nodelayer}
%		\node [style=nothing] (0) at (-0.5, 0.5) {};
%		\node [style=nothing] (1) at (0, 0.5) {};
%		\node [style=dot] (2) at (-0.5, 1) {};
%		\node [style=oplus] (3) at (0, 1) {};
%		\node [style=oneout] (4) at (-0.5, 1.5) {};
%		\node [style=nothing] (5) at (0, 1.5) {};
%	\end{pgfonlayer}
%	\begin{pgfonlayer}{edgelayer}
%		\draw [style=simple] (0) to (4);
%		\draw [style=simple] (1) to (5);
%		\draw [style=simple] (2) to (3);
%	\end{pgfonlayer}
%\end{tikzpicture}
%							=
%							\begin{tikzpicture}
%	\begin{pgfonlayer}{nodelayer}
%		\node [style=oneout] (8) at (-0.5, 7.25) {};
%		\node [style=nothing] (9) at (0, 7.25) {};
%		\node [style=dot] (10) at (-0.5, 6.75) {};
%		\node [style=oplus] (11) at (0, 6.75) {};
%		\node [style=onein] (12) at (-0.5, 6.25) {};
%		\node [style=nothing] (13) at (0, 5.25) {};
%		\node [style=oneout] (14) at (-0.5, 5.75) {};
%		\node [style=nothing] (15) at (-0.5, 5.25) {};
%	\end{pgfonlayer}
%	\begin{pgfonlayer}{edgelayer}
%		\draw [style=simple] (8) to (12);
%		\draw [style=simple] (9) to (13);
%		\draw [style=simple] (10) to (11);
%		\draw [style=simple] (14) to (15);
%	\end{pgfonlayer}
%\end{tikzpicture}$
%							\end{tabular}
%							}
%						
%						\item 
%						\label{CNOT.5}
%						\hfil{
%							$
%							\begin{tikzpicture}
%	\begin{pgfonlayer}{nodelayer}
%		\node [style=nothing] (9) at (-1, 5.25) {};
%		\node [style=nothing] (10) at (-0.5, 5.25) {};
%		\node [style=nothing] (11) at (0, 5.25) {};
%		\node [style=dot] (12) at (-1, 6) {};
%		\node [style=oplus] (13) at (-0.5, 6) {};
%		\node [style=oplus] (14) at (-0.5, 6.5) {};
%		\node [style=dot] (15) at (0, 6.5) {};
%		\node [style=nothing] (16) at (-1, 7.25) {};
%		\node [style=nothing] (17) at (-0.5, 7.25) {};
%		\node [style=nothing] (18) at (0, 7.25) {};
%	\end{pgfonlayer}
%	\begin{pgfonlayer}{edgelayer}
%		\draw [style=simple] (9) to (16);
%		\draw [style=simple] (10) to (17);
%		\draw [style=simple] (11) to (18);
%		\draw [style=simple] (12) to (13);
%		\draw [style=simple] (14) to (15);
%	\end{pgfonlayer}
%\end{tikzpicture}
%							=
%							\begin{tikzpicture}
%	\begin{pgfonlayer}{nodelayer}
%		\node [style=nothing] (10) at (-1, 5.25) {};
%		\node [style=nothing] (11) at (-0.5, 5.25) {};
%		\node [style=nothing] (12) at (0, 5.25) {};
%		\node [style=dot] (13) at (-1, 6.5) {};
%		\node [style=oplus] (14) at (-0.5, 6.5) {};
%		\node [style=oplus] (15) at (-0.5, 6) {};
%		\node [style=dot] (16) at (0, 6) {};
%		\node [style=nothing] (17) at (-1, 7.25) {};
%		\node [style=nothing] (18) at (-0.5, 7.25) {};
%		\node [style=nothing] (19) at (0, 7.25) {};
%	\end{pgfonlayer}
%	\begin{pgfonlayer}{edgelayer}
%		\draw [style=simple] (10) to (17);
%		\draw [style=simple] (11) to (18);
%		\draw [style=simple] (12) to (19);
%		\draw [style=simple] (13) to (14);
%		\draw [style=simple] (15) to (16);
%	\end{pgfonlayer}
%\end{tikzpicture}
%							$}
%						
%						\item 
%						\label{CNOT.6}
%						\hfil{
%							$
%							\begin{tikzpicture}
%	\begin{pgfonlayer}{nodelayer}
%		\node [style=onein] (11) at (0, 5.25) {};
%		\node [style=oneout] (12) at (0, 6.25) {};
%	\end{pgfonlayer}
%	\begin{pgfonlayer}{edgelayer}
%		\draw [style=simple] (11) to (12);
%	\end{pgfonlayer}
%\end{tikzpicture}
%							=
%							\begin{tikzpicture}
%	\begin{pgfonlayer}{nodelayer}
%		\node [style=rn] (12) at (0, 5.25) {};
%		\node [style=rn] (13) at (0, 6.25) {};
%	\end{pgfonlayer}
%\end{tikzpicture}
%							$}
%						
%						\item 
%						\label{CNOT.7}
%						\hfil{
%							\begin{tabular}{c}
%							$\begin{tikzpicture}
%	\begin{pgfonlayer}{nodelayer}
%		\node [style=onein] (13) at (-1, 5.25) {};
%		\node [style=onein] (14) at (-0.5, 5.25) {};
%		\node [style=nothing] (15) at (0, 5.25) {};
%		\node [style=dot] (16) at (-1, 5.75) {};
%		\node [style=oplus] (17) at (-0.5, 5.75) {};
%		\node [style=dot] (18) at (-0.5, 6.25) {};
%		\node [style=oplus] (19) at (0, 6.25) {};
%		\node [style=oneout] (20) at (-1, 6.25) {};
%		\node [style=nothing] (21) at (-0.5, 6.75) {};
%		\node [style=nothing] (22) at (0, 6.75) {};
%	\end{pgfonlayer}
%	\begin{pgfonlayer}{edgelayer}
%		\draw [style=simple] (13) to (20);
%		\draw [style=simple] (14) to (21);
%		\draw [style=simple] (15) to (22);
%		\draw [style=simple] (16) to (17);
%		\draw [style=simple] (18) to (19);
%	\end{pgfonlayer}
%\end{tikzpicture}
%							=
%							\begin{tikzpicture}
%	\begin{pgfonlayer}{nodelayer}
%		\node [style=onein] (14) at (-1, 5.25) {};
%		\node [style=onein] (15) at (-0.5, 5.25) {};
%		\node [style=nothing] (16) at (0, 5.25) {};
%		\node [style=dot] (17) at (-1, 5.75) {};
%		\node [style=oplus] (18) at (-0.5, 5.75) {};
%		\node [style=oneout] (19) at (-1, 6.25) {};
%		\node [style=nothing] (20) at (-0.5, 6.75) {};
%		\node [style=nothing] (21) at (0, 6.75) {};
%	\end{pgfonlayer}
%	\begin{pgfonlayer}{edgelayer}
%		\draw [style=simple] (14) to (19);
%		\draw [style=simple] (15) to (20);
%		\draw [style=simple] (16) to (21);
%		\draw [style=simple] (17) to (18);
%	\end{pgfonlayer}
%\end{tikzpicture}$\\
%							$ $\\
%							$\begin{tikzpicture}
%	\begin{pgfonlayer}{nodelayer}
%		\node [style=oneout] (15) at (-1, 6.75) {};
%		\node [style=oneout] (16) at (-0.5, 6.75) {};
%		\node [style=nothing] (17) at (0, 6.75) {};
%		\node [style=dot] (18) at (-1, 6.25) {};
%		\node [style=oplus] (19) at (-0.5, 6.25) {};
%		\node [style=dot] (20) at (-0.5, 5.75) {};
%		\node [style=oplus] (21) at (0, 5.75) {};
%		\node [style=onein] (22) at (-1, 5.75) {};
%		\node [style=nothing] (23) at (-0.5, 5.25) {};
%		\node [style=nothing] (24) at (0, 5.25) {};
%	\end{pgfonlayer}
%	\begin{pgfonlayer}{edgelayer}
%		\draw [style=simple] (15) to (22);
%		\draw [style=simple] (16) to (23);
%		\draw [style=simple] (17) to (24);
%		\draw [style=simple] (18) to (19);
%		\draw [style=simple] (20) to (21);
%	\end{pgfonlayer}
%\end{tikzpicture}
%							=
%							\begin{tikzpicture}
%	\begin{pgfonlayer}{nodelayer}
%		\node [style=oneout] (16) at (-1, 6.75) {};
%		\node [style=oneout] (17) at (-0.5, 6.75) {};
%		\node [style=nothing] (18) at (0, 6.75) {};
%		\node [style=dot] (19) at (-1, 6.25) {};
%		\node [style=oplus] (20) at (-0.5, 6.25) {};
%		\node [style=onein] (21) at (-1, 5.75) {};
%		\node [style=nothing] (22) at (-0.5, 5.25) {};
%		\node [style=nothing] (23) at (0, 5.25) {};
%	\end{pgfonlayer}
%	\begin{pgfonlayer}{edgelayer}
%		\draw [style=simple] (16) to (21);
%		\draw [style=simple] (17) to (22);
%		\draw [style=simple] (18) to (23);
%		\draw [style=simple] (19) to (20);
%	\end{pgfonlayer}
%\end{tikzpicture}$
%							\end{tabular}
%							}
%						
%						\item 
%						\label{CNOT.8}
%						\hfil{
%							$
%							\begin{tikzpicture}
%	\begin{pgfonlayer}{nodelayer}
%		\node [style=nothing] (17) at (-1, 5.25) {};
%		\node [style=nothing] (18) at (-0.5, 5.25) {};
%		\node [style=nothing] (19) at (0, 5.25) {};
%		\node [style=dot] (20) at (-1, 5.75) {};
%		\node [style=oplus] (21) at (-0.5, 5.75) {};
%		\node [style=dot] (22) at (-0.5, 6.25) {};
%		\node [style=oplus] (23) at (0, 6.25) {};
%		\node [style=dot] (24) at (-1, 6.75) {};
%		\node [style=oplus] (25) at (-0.5, 6.75) {};
%		\node [style=nothing] (26) at (-1, 7.25) {};
%		\node [style=nothing] (27) at (-0.5, 7.25) {};
%		\node [style=nothing] (28) at (0, 7.25) {};
%	\end{pgfonlayer}
%	\begin{pgfonlayer}{edgelayer}
%		\draw [style=simple] (17) to (26);
%		\draw [style=simple] (18) to (27);
%		\draw [style=simple] (19) to (28);
%		\draw [style=simple] (20) to (21);
%		\draw [style=simple] (22) to (23);
%		\draw [style=simple] (24) to (25);
%	\end{pgfonlayer}
%\end{tikzpicture}
%							=
%							\begin{tikzpicture}
%	\begin{pgfonlayer}{nodelayer}
%		\node [style=nothing] (18) at (-1, 5.25) {};
%		\node [style=nothing] (19) at (-0.5, 5.25) {};
%		\node [style=nothing] (20) at (0, 5.25) {};
%		\node [style=dot] (21) at (-0.5, 5.75) {};
%		\node [style=oplus] (22) at (0, 5.75) {};
%		\node [style=dot] (23) at (-1, 6.25) {};
%		\node [style=oplus] (24) at (0, 6.25) {};
%		\node [style=nothing] (25) at (-1, 6.75) {};
%		\node [style=nothing] (26) at (-0.5, 6.75) {};
%		\node [style=nothing] (27) at (0, 6.75) {};
%	\end{pgfonlayer}
%	\begin{pgfonlayer}{edgelayer}
%		\draw [style=simple] (18) to (25);
%		\draw [style=simple] (19) to (26);
%		\draw [style=simple] (20) to (27);
%		\draw [style=simple] (21) to (22);
%		\draw [style=simple] (23) to (24);
%	\end{pgfonlayer}
%\end{tikzpicture}
%							$}
%						
%						\item 
%						\label{CNOT.9}
%						\hfil{
%							$
%							\begin{tikzpicture}
%	\begin{pgfonlayer}{nodelayer}
%		\node [style=onein] (19) at (-1, 5.25) {};
%		\node [style=onein] (20) at (-0.5, 5.25) {};
%		\node [style=nothing] (21) at (0, 5.25) {};
%		\node [style=dot] (22) at (-1, 5.75) {};
%		\node [style=oplus] (23) at (-0.5, 5.75) {};
%		\node [style=oneout] (24) at (-1, 6.25) {};
%		\node [style=oneout] (25) at (-0.5, 6.25) {};
%		\node [style=nothing] (26) at (0, 6.25) {};
%	\end{pgfonlayer}
%	\begin{pgfonlayer}{edgelayer}
%		\draw [style=simple] (19) to (24);
%		\draw [style=simple] (20) to (25);
%		\draw [style=simple] (21) to (26);
%		\draw [style=simple] (22) to (23);
%	\end{pgfonlayer}
%\end{tikzpicture}
%							=
%							\begin{tikzpicture}
%	\begin{pgfonlayer}{nodelayer}
%		\node [style=onein] (20) at (-1, 5.25) {};
%		\node [style=onein] (21) at (-0.5, 5.25) {};
%		\node [style=nothing] (22) at (0, 5.25) {};
%		\node [style=dot] (23) at (-1, 6.25) {};
%		\node [style=oplus] (24) at (-0.5, 6.25) {};
%		\node [style=oneout] (25) at (-1, 7.25) {};
%		\node [style=oneout] (26) at (-0.5, 7.25) {};
%		\node [style=nothing] (27) at (0, 7.25) {};
%		\node [style=oneout] (28) at (0, 6) {};
%		\node [style=onein] (29) at (0, 6.5) {};
%	\end{pgfonlayer}
%	\begin{pgfonlayer}{edgelayer}
%		\draw [style=simple] (20) to (25);
%		\draw [style=simple] (21) to (26);
%		\draw [style=simple] (22) to (28);
%		\draw [style=simple] (29) to (27);
%		\draw [style=simple] (23) to (24);
%	\end{pgfonlayer}
%\end{tikzpicture}
%							$}
%					\end{enumerate}
%				\end{multicols}
%				\
%			\end{mdframed}
%	}}
%	\caption{The identities of \texorpdfstring{$\CNOT$}{CNOT}}
%	\label{fig:CNOT}
%\end{figure}



\chapter{Stabilizer codes as affine cosiotropic relations}
\label{chap:stab}
Linear Lagrangian relations, or more generally, affine Lagrangian relations provide a rich, compositional setting for modelling the evolutions of various physical systems. For example, certain classes of electrical circuits can be interpreted in terms of Lagrangian relations over the field of real rational functions~\cite{passive,affine}. On a quite different note, the state preparation and quantum evolution of $p$-dimensional generalizations of Spekkens' toy theory~\cite{spekkens2016quasi} and (consequently) odd-prime-dimensional stabilizer quantum theory~\cite{gross} have semantics in terms of affine Lagrangian relations over $\F_p$.  Specifically, the state preparation corresponds to the affine Lagrangian relations from the tensor unit, and the evolution corresponds to affine symplectomorphisms.  In this paper we extend this correspondance to the full category of Lagrangian relations, giving these circuits a proper categorical treatment.
% hence the odd-prime-dimensional qudit stabilizer theory, are equivalent to affine Lagrangian relations over (finite) prime fields.  In this document, a {\em prime field} refers to one of the form $\F_p$ for $p$ prime.

Formally, the category of Lagrangian relations is the symmetric monoidal subcategory of linear relations where the objects are symplectic vector spaces and the morphisms are linear relations satisfying an extra condition which can be captured graphically as the following, where $V^\perp$ denotes the orthogonal complement and the grey box denotes the \textit{antipode} from the graphical theory of linear relations:
% subspaces of maximal dimension on which the symplectic form vanishes.
% In terms of string diagrams, as we will show in Section \ref{sec:sym}, we can think of the complementary observables as  occupying two different wires.  The graphical condition of a linear subspace being Lagrangian is expressed as follows, where $V^\perp$ denotes the orthogonal complement of $V$:
$$
\begin{tikzpicture}
	\begin{pgfonlayer}{nodelayer}
		\node [style=map] (0) at (2, -2) {$V$};
		\node [style=none] (1) at (1.75, -1.25) {};
		\node [style=none] (2) at (2.25, -1.25) {};
		\node [style=none] (3) at (1.75, -2.75) {};
		\node [style=none] (4) at (2.25, -2.75) {};
	\end{pgfonlayer}
	\begin{pgfonlayer}{edgelayer}
		\draw [in=120, out=-90] (1.center) to (0);
		\draw [in=-90, out=60] (0) to (2.center);
		\draw [in=-60, out=90] (4.center) to (0);
		\draw [in=90, out=-120] (0) to (3.center);
	\end{pgfonlayer}
\end{tikzpicture}
=
\begin{tikzpicture}
	\begin{pgfonlayer}{nodelayer}
		\node [style=map] (0) at (2, -2) {$V^\perp$};
		\node [style=none] (1) at (1.75, -1.25) {};
		\node [style=none] (2) at (2.25, -1.25) {};
		\node [style=none] (3) at (1.75, -2.75) {};
		\node [style=none] (4) at (2.25, -2.75) {};
		\node [style=none] (5) at (2.25, -0.5) {};
		\node [style=none] (6) at (1.75, -0.5) {};
		\node [style=none] (7) at (2.25, -3.5) {};
		\node [style=none] (8) at (1.75, -3.5) {};
		\node [style=s] (9) at (2.25, -1.25) {};
		\node [style=s] (10) at (2.25, -2.75) {};
	\end{pgfonlayer}
	\begin{pgfonlayer}{edgelayer}
		\draw [in=120, out=-90] (1.center) to (0);
		\draw [in=-90, out=60] (0) to (2.center);
		\draw [in=-60, out=90] (4.center) to (0);
		\draw [in=90, out=-120] (0) to (3.center);
		\draw [in=90, out=-90] (6.center) to (2.center);
		\draw [in=270, out=90] (1.center) to (5.center);
		\draw [in=270, out=90] (7.center) to (3.center);
		\draw [in=270, out=90] (8.center) to (4.center);
	\end{pgfonlayer}
\end{tikzpicture}
$$
We show that any linear relation $V$ determines a Lagrangian relation in terms of `doubling', i.e. taking the tensor product of a linear relation with its complement:
$$
\begin{tikzpicture}
	\begin{pgfonlayer}{nodelayer}
		\node [style=map] (24) at (3.25, -1) {$V$};
		\node [style=none] (25) at (3.25, -0.25) {};
		\node [style=none] (27) at (3.25, -1.75) {};
	\end{pgfonlayer}
	\begin{pgfonlayer}{edgelayer}
		\draw (25.center) to (24);
		\draw (27.center) to (24);
	\end{pgfonlayer}
\end{tikzpicture}
\mapsto
\begin{tikzpicture}
	\begin{pgfonlayer}{nodelayer}
		\node [style=map] (24) at (3.25, -1) {$V^\perp$};
		\node [style=none] (25) at (3.25, -0.25) {};
		\node [style=none] (27) at (3.25, -1.75) {};
		\node [style=map] (28) at (4, -1) {$V$};
		\node [style=none] (29) at (4, -0.25) {};
		\node [style=none] (30) at (4, -1.75) {};
	\end{pgfonlayer}
	\begin{pgfonlayer}{edgelayer}
		\draw (25.center) to (24);
		\draw (27.center) to (24);
		\draw (29.center) to (28);
		\draw (30.center) to (28);
	\end{pgfonlayer}
\end{tikzpicture}
$$
By analogy to the CPM construction for the category of completely positive maps, we call these \textit{pure} Lagrangian relations. In Theorem \ref{theorem:unbiased} we show that only one more class of `discard' generators $d_a$ for each $a$ in the underlying field $k$ is required to generate all Lagrangian relations.
$$
d_a := 
\begin{tikzpicture}
	\begin{pgfonlayer}{nodelayer}
		\node [style=X] (0) at (0, 0.75) {};
		\node [style=scalar] (1) at (0.5, 0) {$a$};
		\node [style=none] (2) at (0.5, -0.75) {};
		\node [style=none] (3) at (-0.5, 0) {};
		\node [style=none] (4) at (-0.5, -0.75) {};
	\end{pgfonlayer}
	\begin{pgfonlayer}{edgelayer}
		\draw [in=-30, out=90] (1) to (0);
		\draw [in=90, out=-150] (0) to (3.center);
		\draw (4.center) to (3.center);
		\draw (2.center) to (1);
	\end{pgfonlayer}
\end{tikzpicture}
$$

From this, we immediately obtain a complete graphical calculus for Lagrangian relations over any field $k$, namely we can apply the complete calculus $\ih_k$ for linear relations~\cite{ihpub} to diagrams built from pure morphisms and discard maps. 
This extends the doubled presentation of bond graphs, given in \cite[5.3]{coya}, which are not universal for Lagrangian relations, and is instead only universal for a fragment of the pure morphisms.
In Corollary \ref{cor:pure}, we also immediately get a \textit{purification theorem} for Lagrangian relations, much like the purification (a.k.a. Stinespring dilation) of quantum channels which can be proven straightforwardly in the CPM construction over Hilbert spaces.

% two more classes of generators are needed to obtain all Lagrangian relations.  Explicitly, for every element $a$ of the base field we need an $a$-labelled phase gate and the Fourier transform:
% $$
% \tikzfig{Sa-gen}
% \hspace*{2cm}
% \tikzfig{fourier}
% $$
% Notice that these maps can not be factored into a linear relation tensored by its orthogonal complement.
% In  Theorem \ref{theorem:unbiased} we show that Lagrangian relations can be presented in a more symmetric way in terms of the CPM construction applied to linear relations where the conjugation is given by the orthogonal complement, and a `discard' map is added for every element of the field.

Furthermore, in the case of prime fields, i.e. finite fields $\mathbb F_p$ for $p$, we show in Corollary \ref{cor} that this is actually an instance of the original CPM construction, for a suitably defined dagger on the category of linear relations.

% Because doubling is faithful and symmetric monoidal, in the undoubled picture, Lagrangian relations can be obtained by adding one class of extra generators to linear relations.  This is similar to how, in some cases, the CPM construction applied to a prop can be presented in terms of adding a discard map to the base category modulo some extra equations \cite{disc}.

In Section \ref{sec:aff} we show that only one more generator is needed to obtain {\em affine} Lagrangian relations.  In the case of odd prime fields, we show in Theorem \ref{theorem:spekkens} that affine Lagrangian relations are prime-dimensional qudit stabilizer circuits, modulo invertible scalars.  This give a graphical calculus extends to previous work on the qubit \cite{backensspek}, and qutrit \cite{qutrit} cases.  We also discuss the relation to electrical circuits.


\paragraph{Related work.} It was previously shown that certain classes of electrical circuits have a semantics in terms of affine Lagrangian relations over the field of the real numbers and the real rational functions ${\mathbb{R}[x,y]/\langle xy-1\rangle}$ \cite{network,passive}. Similarly in \cite[\S VI]{affine}, the authors give an interpretation of non-passive electrical circuits in terms of these `doubled' string diagrams for affine relations over the real rational functions, however the authors did not give a full characterisation for the category of Lagrangian relations in terms of diagrammatic generators.
%These two approaches are actually one and the same.
We restate the interpretations of the electrical components given in  \cite[\S VI]{affine} in terms of the graphical calculus for affine Lagrangian relations in Example \ref{ex:circuits}.

A presentation of odd-prime stabilizer theory in terms of affine symplectomorphisms applied to Lagrangian subspaces appears in~\cite{gross} and several follow-on works relating stabilizer theory to classical phase space via the discrete Wigner function. Our Theorem \ref{theorem:spekkens} is a categorical reformulation of the result of Spekkens' in which he shows that so called odd-prime-dimensional `quadrature epistricted theories' are operationally equivalent to prime-dimensional qudit stabilizer circuits \cite{spekkens2016quasi}; following earlier work in \cite{spekkens}.  This operational equivalence has also been further explored in the non-prime case \cite{catani}. Note that operational equivalence is not the same as categorical equivalence. The notion of operational equivalence used in \cite{spekkens2016quasi,catani}  refers to the equivalence of protocols in which circuits are prepared, evolve and then are measured; whereas ours is more `process-theoretic', i.e. we consider the category that contains states, effects, evolutions, and all possible compositions thereof. A complete presentation for Spekkens' qubit toy model in terms of a category of relations has also been given \cite{backensspek} following the categorical description by \cite{coecke2012spekkens}.  However, the authors do not explicitly establish that this is the category of affine Lagrangian relations over $\F_2$, but merely a subcategory of finite sets and relations. There is also a complete presentation for qutrit stabilizer theory \cite{qutrit} which, by Theorem \ref{theorem:spekkens}, is equivalent to Spekkens' qutrit toy model, up to scalars; the connection to relations, in this case, being unexplored.

%In this document we show that only two new generators need to be added to linear relations to obtain Lagrangian relations, and one more is needed to obtain affine Lagrangian relations.



%In Section \ref{sec:linear} , we review the graphical calculi and theory of linear relations.  After which in Section \ref{sec:sym} we give a brief review of linear symplectic geometry.  In Section \ref{sec:univ} we give a universal set of generators for Lagrangian relations.  We show that this can be stated in terms of the CPM construction applied to linear relations over a field with respect to the conjugation given by the orthogonal complement.  In the case when one is not working with a prime field, then the CPM construction must be slightly modified to allow for multiple traces.
%In section \ref{sec:aff} we give a universal set of generators for affine Lagrangian relations.

\section{Lagrangian relations}
\label{sec:sym}

Now that we have a graphical presentation of linear relations, we can now do same for (linear) Lagrangian relations.  We first recall some of the basic theory of symplectic vector spaces.  This is expounded upon in much greater generality in the not-necessarily-linear case in \cite{weinstein}.  In this entire paper, we only care about the linear and affine cases; and things will assumed to be linear unless otherwise stated.  As previously mentioned, Lagrangian relations (and their affine counterpart) have previously been studied within the context of monoidal categories  to model electrical circuits among other things \cite{passive,network,coya}; although, to the knowledge of the authors, no proof of universality exists in the literature.

\begin{definition}
  Given a field  $k$ and a $k$-vector space $V$, a {\bf symplectic form} on $V$ is a bilinear map $\omega:V\times V\to k$ which is:
\begin{itemize}
 \item {\bf Alternating:} $\forall v \in V$, $\omega(v,v)=0$ \item {\bf Non-degenerate:} if $\exists v \in V \forall w \in V: \omega(v,w)=0$, then $v=0$.
\end{itemize}
  A {\bf symplectic vector space} is a vector space equipped with a symplectic form. A (linear) {\bf symplectomorphism} is a linear isomorphism between symplectic vector spaces that preserves the symplectic form.
\end{definition}


\begin{lemma}
\label{lemma:sform}
Every vector space $k^{2n}$ is equipped with a bilinear form given by the following block matrix:
$$
\omega:=
\begin{bmatrix}
0_n & I_n\\
-I_n & 0_n
\end{bmatrix}
$$
so that $\omega(v,w) := v \omega w^T$.
Moreover, every finite dimensional symplectic vector space over $k$ is symplectomorphic to one of the form $k^{2n}$ with such a symplectic form.
\end{lemma}



%
%There is some sort of skewed duality between the two gradings of a symplectic vector space $k^n$.
%In classical mechanics, this is used to model the duality between momentum and physics for example.  This hints at there being some relation to stabilizer quantum mechanics, where there is a similar such duality.
%
%
%Just as one can define the dual space $V^\perp$ of a subspace $V \subseteq W$ with respect to the regular inner product of vector spaces, there is an analagous notion with the symplectic form.

\begin{definition}

Let $W \subseteq V$ be a linear subspace of a symplectic space $V$.
The {\bf symplectic dual} of the subspace $W$ is defined to be
$
W^\omega:= \{v \in V : \forall w \in W, \omega(v,w)=0 \}
$.
A linear subspace  $W$ of a symplectic vector space $V$ is {\bf isotropic} when $W^\omega \supseteq W$, {\bf coisotropic} when $W^\omega \subseteq W$ and {\bf Lagrangian} when $W^\omega=W$.


\end{definition}

\begin{lemma}
Every symplectomorphism $f:V\to V$ induces a Lagrangian relation $\Gamma_f:=\{ (fv, v) | v \in V \}$.
\end{lemma}

These spaces have a natural grading into two distinct parts $V \oplus W \subseteq k^n \oplus k^n$. By analogy to the case of quantum stabilizer theory, we call the left part the \textit{X-grading} and the right part the \textit{Z-grading}.

As a matter of convention, we consider linear subspaces as being represented as the row space of a matrix. So in particular, a symplectic subspace of $k^{2n}$ is represented by a matrix of the form $[X|Z]$ where $X,Z$ are both $n\times n$-dimensional matrices.
An isotropic subspace can equivalently be described as a matrix $[X|Z]$ so that $[X|Z] \omega [X|Z]^T = 0$.
Moreover, a Lagrangian subspace can be described as a matrix as above which additionally has rank $n$.


\begin{definition}
Given a field $k$, the prop of {\bf Lagrangian relations},  $\Lag\Rel_k$ has morphisms $k^{2n}\to k^{2m}$ as Lagrangian subspaces of the symplectic vector space $k^{n+m} \oplus k^{n+m}$ with symplectic form given above.  Composition is given by relational composition and the tensor product is given by the direct sum.
\end{definition}
%For the purposes of this paper, because it is so much easier to work in a prop, we will draw string diagram in the skeleton of $\Lag\Rel_k$ whose objects are all of the form $k^{2n}$ equipped with the symplectic form of Lemma \ref{lemma:sform}.

The direct sum of Lagrangian subspaces is graphically depicted as follows:
$$
\begin{tikzpicture}
	\begin{pgfonlayer}{nodelayer}
		\node [style=map] (616) at (272, 0) {$V$};
		\node [style=none] (617) at (271.75, 1) {};
		\node [style=none] (618) at (272.25, 1) {};
	\end{pgfonlayer}
	\begin{pgfonlayer}{edgelayer}
		\draw [in=-90, out=60] (616) to (618.center);
		\draw [in=-90, out=120] (616) to (617.center);
	\end{pgfonlayer}
\end{tikzpicture}
\oplus
\begin{tikzpicture}
	\begin{pgfonlayer}{nodelayer}
		\node [style=map] (616) at (272, 0) {$W$};
		\node [style=none] (617) at (271.75, 1) {};
		\node [style=none] (618) at (272.25, 1) {};
	\end{pgfonlayer}
	\begin{pgfonlayer}{edgelayer}
		\draw [in=-90, out=60] (616) to (618.center);
		\draw [in=-90, out=120] (616) to (617.center);
	\end{pgfonlayer}
\end{tikzpicture}
:=
\begin{tikzpicture}
	\begin{pgfonlayer}{nodelayer}
		\node [style=map] (616) at (272, 0) {$V$};
		\node [style=none] (617) at (271.75, 1) {};
		\node [style=none] (618) at (272.75, 1) {};
		\node [style=map] (619) at (272.75, 0) {$W$};
		\node [style=none] (620) at (272, 1) {};
		\node [style=none] (621) at (273, 1) {};
	\end{pgfonlayer}
	\begin{pgfonlayer}{edgelayer}
		\draw [in=-90, out=60] (616) to (618.center);
		\draw [in=-90, out=120] (616) to (617.center);
		\draw [in=-90, out=60] (619) to (621.center);
		\draw [in=-90, out=120] (619) to (620.center);
	\end{pgfonlayer}
\end{tikzpicture}
$$
Where we are grouping the $X$ gradings together on the left and the $Z$ gradings together on the right. Note that this means the embedding of $\Lag\Rel_k$ into $\LinRel_k$ preserves the monoidal product only up to isomorphism. More precisely, we have the following fact.

\begin{lemma}
\label{lemma:strong}
The forgetful functor $E:\Lag\Rel_k \to \LinRel_k$  is a faithful, strong symmetric monoidal.
\end{lemma}

\begin{proof}
  Functoriality and faithfulness is immediate. The strong monoidal structure is given by $E(I) = I$ and
  \[ E(A) \oplus E(B) := A \oplus A \oplus B \oplus B \xrightarrow{1 \oplus \sigma \oplus 1} A \oplus B \oplus A \oplus B =: E(A \oplus B). \]
  The symmetric monoidal structure on $\Lag\Rel_k$ is chosen such that it is consistent with the monoidal structure above.
\end{proof}

Due to the above lemma, we will regard $\Lag\Rel_k$ as a symmetric monoidal subcategory of $\LinRel_k$.
As such, we can ask what the generators of $\Lag\Rel_k$ look like in terms of string diagrams of $\ih_k$ generators. We first describe what it means to be a Lagrangian relation in pictures, where the $X$ block is the wire on the left and $Z$ block is the wire on the right:
\begin{equation}
\label{eq:lag}
\begin{tikzpicture}
	\begin{pgfonlayer}{nodelayer}
		\node [style=map] (0) at (0.75, -1) {$W$};
		\node [style=none] (1) at (0.5, 0) {};
		\node [style=none] (2) at (1, 0) {};
	\end{pgfonlayer}
	\begin{pgfonlayer}{edgelayer}
		\draw [in=120, out=-90] (1.center) to (0);
		\draw [in=-90, out=60] (0) to (2.center);
	\end{pgfonlayer}
\end{tikzpicture}
%=
%\begin{tikzpicture}
%	\begin{pgfonlayer}{nodelayer}
%		\node [style=map] (0) at (0.75, -1) {$V^\perp$};
%		\node [style=none] (1) at (0.5, 0) {};
%		\node [style=none] (2) at (1, 0) {};
%	\end{pgfonlayer}
%	\begin{pgfonlayer}{edgelayer}
%		\draw [in=120, out=-90] (1.center) to (0);
%		\draw [in=-90, out=60] (0) to (2.center);
%	\end{pgfonlayer}
%\end{tikzpicture}
=
\begin{tikzpicture}
	\begin{pgfonlayer}{nodelayer}
		\node [style=map] (0) at (0.75, -1.75) {$W^\perp$};
		\node [style=none] (1) at (0.5, -1) {};
		\node [style=none] (2) at (1, -1) {};
		\node [style=none] (3) at (1, 0) {};
		\node [style=none] (4) at (0.5, 0) {};
		\node [style=s] (5) at (1, -1) {};
	\end{pgfonlayer}
	\begin{pgfonlayer}{edgelayer}
		\draw [in=120, out=-90] (1.center) to (0);
		\draw [in=-90, out=60] (0) to (2.center);
		\draw [in=-90, out=90] (2.center) to (4.center);
		\draw [in=-270, out=-90] (3.center) to (1.center);
	\end{pgfonlayer}
\end{tikzpicture}
\end{equation}
Algebraically, for $W$ a subspace of $V$, the right hand side is interpreted as follows:
\begin{align*}
W^\omega :&= \{(v_1,v_2) \in V : \forall (w_1,w_2) \in W, \omega((v_1,v_2),(w_1,w_2))=0 \}\\
                    &= \{(v_1,v_2) \in V : \forall (w_1,w_2) \in W,  \langle (v_2,-v_1) ,(w_1,w_2)\rangle =0 \}\\
                    &= \{(v_2,-v_1) \in V : \forall (w_1,w_2) \in W,  \langle (v_1,v_2) ,(w_1,w_2)\rangle =0 \}
\end{align*}

The category of Lagrangian relations is compact closed.  Given a relation $V$ between symplectic vector spaces, we can curry it into a state $\hat V$; and similarily, we can uncurry a state $W$ into a process $\widecheck W$,
$$
\begin{tikzpicture}
	\begin{pgfonlayer}{nodelayer}
		\node [style=map] (0) at (0.75, -1.75) {$V$};
		\node [style=none] (1) at (0.5, -1) {};
		\node [style=none] (2) at (1, -1) {};
		\node [style=none] (3) at (0.5, -2.5) {};
		\node [style=none] (4) at (1, -2.5) {};
	\end{pgfonlayer}
	\begin{pgfonlayer}{edgelayer}
		\draw [in=120, out=-90] (1.center) to (0);
		\draw [in=-90, out=60] (0) to (2.center);
		\draw [in=-60, out=90] (4.center) to (0);
		\draw [in=90, out=-120] (0) to (3.center);
	\end{pgfonlayer}
\end{tikzpicture}
\xmapsto{\hat{(\_)} }
\begin{tikzpicture}
	\begin{pgfonlayer}{nodelayer}
		\node [style=map] (0) at (0.75, -1.75) {$V$};
		\node [style=none] (1) at (0, -1) {};
		\node [style=none] (2) at (1.25, -1) {};
		\node [style=none] (4) at (1.25, -2.5) {};
		\node [style=X] (5) at (0, -3) {};
		\node [style=Z] (6) at (0.75, -3) {};
		\node [style=none] (7) at (0.75, -1) {};
		\node [style=none] (8) at (-0.5, -1) {};
		\node [style=none] (9) at (0, -2) {};
	\end{pgfonlayer}
	\begin{pgfonlayer}{edgelayer}
		\draw [in=120, out=-90] (1.center) to (0);
		\draw [in=-90, out=60] (0) to (2.center);
		\draw [in=-45, out=90] (4.center) to (0);
		\draw [in=-90, out=135, looseness=0.75] (5) to (8.center);
		\draw [in=30, out=-90] (4.center) to (6);
		\draw [in=90, out=-90] (7.center) to (9.center);
		\draw [in=150, out=-90] (9.center) to (6);
		\draw [in=45, out=-135] (0) to (5);
	\end{pgfonlayer}
\end{tikzpicture}
\hspace*{1cm}
\begin{tikzpicture}
	\begin{pgfonlayer}{nodelayer}
		\node [style=map] (0) at (1.5, -2) {$W$};
		\node [style=none] (1) at (1.25, -1) {};
		\node [style=none] (2) at (2.25, -1) {};
		\node [style=none] (6) at (1.75, -1) {};
		\node [style=none] (7) at (0.75, -1) {};
	\end{pgfonlayer}
	\begin{pgfonlayer}{edgelayer}
		\draw [in=105, out=-90] (1.center) to (0);
		\draw [in=-90, out=60] (0) to (2.center);
		\draw [in=120, out=-90] (7.center) to (0);
		\draw [in=-90, out=75] (0) to (6.center);
	\end{pgfonlayer}
\end{tikzpicture}
\xmapsto{\widecheck{(\_)} }
\begin{tikzpicture}
	\begin{pgfonlayer}{nodelayer}
		\node [style=map] (0) at (1.75, -2.25) {$W$};
		\node [style=none] (1) at (1, -0.75) {};
		\node [style=none] (2) at (2, -0.75) {};
		\node [style=none] (6) at (1.5, -1.25) {};
		\node [style=none] (7) at (0.75, -1.25) {};
		\node [style=Z] (8) at (1.5, -1.25) {};
		\node [style=X] (9) at (0.75, -1.25) {};
		\node [style=none] (10) at (0.25, -2.5) {};
		\node [style=none] (11) at (1, -2.5) {};
	\end{pgfonlayer}
	\begin{pgfonlayer}{edgelayer}
		\draw [in=105, out=-90, looseness=1.25] (1.center) to (0);
		\draw [in=-90, out=45, looseness=0.75] (0) to (2.center);
		\draw [in=135, out=-30] (7.center) to (0);
		\draw [in=-45, out=75] (0) to (6.center);
		\draw [in=-135, out=90] (10.center) to (9);
		\draw [in=-135, out=90] (11.center) to (8);
	\end{pgfonlayer}
\end{tikzpicture}
$$
It is easy to see that these two constructions are inverse to each other.
This allows us to derive a graphical criteria for abitrary Lagrangian relations, generalizing Equation \ref{eq:lag}:
$$
\begin{tikzpicture}
	\begin{pgfonlayer}{nodelayer}
		\node [style=map] (0) at (0.75, -1.75) {$V$};
		\node [style=none] (1) at (0, -0.75) {};
		\node [style=none] (2) at (1.25, -0.75) {};
		\node [style=none] (4) at (1.25, -2.5) {};
		\node [style=X] (5) at (0, -3) {};
		\node [style=Z] (6) at (0.75, -3) {};
		\node [style=none] (7) at (0.75, -0.75) {};
		\node [style=none] (8) at (-0.5, -0.75) {};
		\node [style=none] (9) at (0, -2) {};
	\end{pgfonlayer}
	\begin{pgfonlayer}{edgelayer}
		\draw [in=120, out=-90] (1.center) to (0);
		\draw [in=-90, out=60] (0) to (2.center);
		\draw [in=-45, out=90] (4.center) to (0);
		\draw [in=-90, out=135, looseness=0.75] (5) to (8.center);
		\draw [in=30, out=-90] (4.center) to (6);
		\draw [in=90, out=-90] (7.center) to (9.center);
		\draw [in=150, out=-90] (9.center) to (6);
		\draw [in=45, out=-135] (0) to (5);
	\end{pgfonlayer}
\end{tikzpicture}
=
\begin{tikzpicture}
	\begin{pgfonlayer}{nodelayer}
		\node [style=map] (43) at (13.5, -1.75) {$V^\perp$};
		\node [style=none] (44) at (12.75, -0.75) {};
		\node [style=none] (45) at (14, -0.75) {};
		\node [style=none] (46) at (14, -2.5) {};
		\node [style=Z] (47) at (12.75, -3) {};
		\node [style=X] (48) at (13.5, -3) {};
		\node [style=none] (49) at (13.5, -0.75) {};
		\node [style=none] (50) at (12.25, -0.75) {};
		\node [style=none] (51) at (12.75, -2) {};
		\node [style=none] (52) at (13.5, 0.75) {};
		\node [style=none] (53) at (14, 0.75) {};
		\node [style=none] (54) at (12.25, 0.75) {};
		\node [style=none] (55) at (12.75, 0.75) {};
		\node [style=s] (56) at (14, -0.75) {};
		\node [style=s] (57) at (13.5, -0.75) {};
		\node [style=none] (58) at (13.25, -3.5) {};
	\end{pgfonlayer}
	\begin{pgfonlayer}{edgelayer}
		\draw [in=120, out=-90] (44.center) to (43);
		\draw [in=-90, out=60] (43) to (45.center);
		\draw [in=-45, out=90] (46.center) to (43);
		\draw [in=-90, out=135, looseness=0.75] (47) to (50.center);
		\draw [in=30, out=-90] (46.center) to (48);
		\draw [in=90, out=-90] (49.center) to (51.center);
		\draw [in=150, out=-90] (51.center) to (48);
		\draw [in=45, out=-135] (43) to (47);
		\draw [in=270, out=90] (45.center) to (55.center);
		\draw [in=270, out=90] (49.center) to (54.center);
		\draw [in=270, out=90] (44.center) to (53.center);
		\draw [in=270, out=90] (50.center) to (52.center);
	\end{pgfonlayer}
\end{tikzpicture}
\iff
\begin{tikzpicture}
	\begin{pgfonlayer}{nodelayer}
		\node [style=map] (0) at (2, -2) {$V$};
		\node [style=none] (1) at (1.75, -1.25) {};
		\node [style=none] (2) at (2.25, -1.25) {};
		\node [style=none] (3) at (1.75, -2.75) {};
		\node [style=none] (4) at (2.25, -2.75) {};
	\end{pgfonlayer}
	\begin{pgfonlayer}{edgelayer}
		\draw [in=120, out=-90] (1.center) to (0);
		\draw [in=-90, out=60] (0) to (2.center);
		\draw [in=-60, out=90] (4.center) to (0);
		\draw [in=90, out=-120] (0) to (3.center);
	\end{pgfonlayer}
\end{tikzpicture}
=
\begin{tikzpicture}
	\begin{pgfonlayer}{nodelayer}
		\node [style=map] (0) at (2.5, -1.75) {$V$};
		\node [style=none] (1) at (2, -0.5) {};
		\node [style=none] (2) at (3, -0.5) {};
		\node [style=none] (3) at (3, -2.5) {};
		\node [style=X] (4) at (1.75, -3) {};
		\node [style=Z] (5) at (2.5, -3) {};
		\node [style=none] (6) at (2.5, -0.75) {};
		\node [style=none] (7) at (1.5, -0.75) {};
		\node [style=none] (8) at (1.75, -2) {};
		\node [style=X] (9) at (1.5, -0.75) {};
		\node [style=Z] (10) at (2.5, -0.75) {};
		\node [style=none] (11) at (0.5, -3.25) {};
		\node [style=none] (12) at (1, -3.25) {};
	\end{pgfonlayer}
	\begin{pgfonlayer}{edgelayer}
		\draw [in=120, out=-90] (1.center) to (0);
		\draw [in=-90, out=60] (0) to (2.center);
		\draw [in=-45, out=90] (3.center) to (0);
		\draw [in=-45, out=150] (4) to (7.center);
		\draw [in=30, out=-90] (3.center) to (5);
		\draw [in=90, out=-45, looseness=1.25] (6.center) to (8.center);
		\draw [in=150, out=-90] (8.center) to (5);
		\draw [in=45, out=-135] (0) to (4);
		\draw [in=90, out=-150] (10) to (12.center);
		\draw [in=90, out=-120] (9) to (11.center);
	\end{pgfonlayer}
\end{tikzpicture}
=
\begin{tikzpicture}
	\begin{pgfonlayer}{nodelayer}
		\node [style=map] (59) at (17, -1.75) {$V^\perp$};
		\node [style=none] (60) at (16.25, -0.75) {};
		\node [style=none] (61) at (17.5, -0.75) {};
		\node [style=none] (62) at (17.5, -2.5) {};
		\node [style=Z] (63) at (16.25, -3) {};
		\node [style=X] (64) at (17, -3) {};
		\node [style=none] (65) at (17, -0.75) {};
		\node [style=none] (66) at (15.75, -0.75) {};
		\node [style=none] (67) at (16.25, -2) {};
		\node [style=none] (68) at (16.5, 0.75) {};
		\node [style=none] (69) at (17.5, 1) {};
		\node [style=none] (70) at (15.5, 0.75) {};
		\node [style=none] (71) at (16, 1) {};
		\node [style=s] (72) at (17.5, -0.75) {};
		\node [style=s] (73) at (17, -0.75) {};
		\node [style=X] (74) at (15.5, 0.75) {};
		\node [style=Z] (75) at (16.5, 0.75) {};
		\node [style=none] (76) at (15, -3.25) {};
		\node [style=none] (77) at (15.5, -3.25) {};
		\node [style=none] (78) at (16.5, 1.5) {};
	\end{pgfonlayer}
	\begin{pgfonlayer}{edgelayer}
		\draw [in=120, out=-90] (60.center) to (59);
		\draw [in=-90, out=60] (59) to (61.center);
		\draw [in=-45, out=90] (62.center) to (59);
		\draw [in=-90, out=135, looseness=0.75] (63) to (66.center);
		\draw [in=30, out=-90] (62.center) to (64);
		\draw [in=90, out=-90] (65.center) to (67.center);
		\draw [in=150, out=-90] (67.center) to (64);
		\draw [in=45, out=-135] (59) to (63);
		\draw [in=-90, out=90, looseness=1.25] (61.center) to (71.center);
		\draw [in=-60, out=90] (65.center) to (70.center);
		\draw [in=270, out=90] (60.center) to (69.center);
		\draw [in=-45, out=90, looseness=1.25] (66.center) to (68.center);
		\draw [in=-135, out=90, looseness=0.50] (76.center) to (74);
		\draw [in=-150, out=90] (77.center) to (75);
	\end{pgfonlayer}
\end{tikzpicture}
=
\begin{tikzpicture}
	\begin{pgfonlayer}{nodelayer}
		\node [style=map] (0) at (4.5, -1.75) {$V^\perp$};
		\node [style=none] (1) at (4, -1) {};
		\node [style=none] (2) at (5, -1) {};
		\node [style=none] (3) at (5, -2.5) {};
		\node [style=X] (4) at (4.75, -3) {};
		\node [style=none] (5) at (3.75, -2.25) {};
		\node [style=none] (6) at (5, 0) {};
		\node [style=none] (7) at (4, 0) {};
		\node [style=s] (8) at (5, -1) {};
		\node [style=none] (9) at (3.25, -3.25) {};
		\node [style=none] (10) at (4, -3.25) {};
		\node [style=Z] (11) at (3.75, -2.25) {};
	\end{pgfonlayer}
	\begin{pgfonlayer}{edgelayer}
		\draw [in=135, out=-90] (1.center) to (0);
		\draw [in=-90, out=60] (0) to (2.center);
		\draw [in=-45, out=90] (3.center) to (0);
		\draw [in=30, out=-90] (3.center) to (4);
		\draw [in=165, out=-15, looseness=1.25] (5.center) to (4);
		\draw [in=-90, out=90, looseness=1.25] (2.center) to (7.center);
		\draw [in=270, out=90] (1.center) to (6.center);
		\draw [in=240, out=90] (10.center) to (0);
		\draw [in=90, out=-150] (11) to (9.center);
	\end{pgfonlayer}
\end{tikzpicture}
=
\begin{tikzpicture}
	\begin{pgfonlayer}{nodelayer}
		\node [style=map] (0) at (2, -2) {$V^\perp$};
		\node [style=none] (1) at (1.75, -1.25) {};
		\node [style=none] (2) at (2.25, -1.25) {};
		\node [style=none] (3) at (1.75, -2.75) {};
		\node [style=none] (4) at (2.25, -2.75) {};
		\node [style=none] (5) at (2.25, -0.5) {};
		\node [style=none] (6) at (1.75, -0.5) {};
		\node [style=none] (7) at (2.25, -3.5) {};
		\node [style=none] (8) at (1.75, -3.5) {};
		\node [style=s] (9) at (2.25, -1.25) {};
		\node [style=s] (10) at (2.25, -2.75) {};
	\end{pgfonlayer}
	\begin{pgfonlayer}{edgelayer}
		\draw [in=120, out=-90] (1.center) to (0);
		\draw [in=-90, out=60] (0) to (2.center);
		\draw [in=-60, out=90] (4.center) to (0);
		\draw [in=90, out=-120] (0) to (3.center);
		\draw [in=90, out=-90] (6.center) to (2.center);
		\draw [in=270, out=90] (1.center) to (5.center);
		\draw [in=270, out=90] (7.center) to (3.center);
		\draw [in=270, out=90] (8.center) to (4.center);
	\end{pgfonlayer}
\end{tikzpicture}
$$
For this reason, we will depict Lagrangian relations as processes, where the input wires are on the bottom and output wires on on the top.

\begin{lemma}
There is a faithful, strong symmetric monoidal functor $L:\LinRel_k\to\Lag\Rel_k$ given by the following action on the generators of $\ih_k$; doubling, and then changing the colours of one of the copies:
$$
\begin{tikzpicture}
	\begin{pgfonlayer}{nodelayer}
		\node [style=map] (0) at (-3, -1) {$V$};
		\node [style=none] (1) at (-3, -0.25) {};
		\node [style=none] (2) at (-3, -1.75) {};
	\end{pgfonlayer}
	\begin{pgfonlayer}{edgelayer}
		\draw (1.center) to (0);
		\draw (0) to (2.center);
	\end{pgfonlayer}
\end{tikzpicture}
\mapsto
\begin{tikzpicture}
	\begin{pgfonlayer}{nodelayer}
		\node [style=map] (0) at (-3, -1) {$V^\perp$};
		\node [style=none] (1) at (-3, -0.25) {};
		\node [style=none] (2) at (-3, -1.75) {};
		\node [style=map] (3) at (-2.25, -1) {$V$};
		\node [style=none] (4) at (-2.25, -0.25) {};
		\node [style=none] (5) at (-2.25, -1.75) {};
	\end{pgfonlayer}
	\begin{pgfonlayer}{edgelayer}
		\draw (1.center) to (0);
		\draw (0) to (2.center);
		\draw (4.center) to (3);
		\draw (3) to (5.center);
	\end{pgfonlayer}
\end{tikzpicture}
$$
%
%$$
%\begin{tikzpicture}
%	\begin{pgfonlayer}{nodelayer}
%		\node [style=map] (0) at (-3, -1) {$V^\perp$};
%		\node [style=none] (1) at (-3, -0.25) {};
%		\node [style=none] (2) at (-3, -1.75) {};
%		\node [style=map] (3) at (-2.25, -1) {$V$};
%		\node [style=none] (4) at (-2.25, -0.25) {};
%		\node [style=none] (5) at (-2.25, -1.75) {};
%	\end{pgfonlayer}
%	\begin{pgfonlayer}{edgelayer}
%		\draw (1.center) to (0);
%		\draw (0) to (2.center);
%		\draw (4.center) to (3);
%		\draw (3) to (5.center);
%	\end{pgfonlayer}
%\end{tikzpicture}
%=
%\begin{tikzpicture}
%	\begin{pgfonlayer}{nodelayer}
%		\node [style=map] (0) at (-3, -1) {$V$};
%		\node [style=none] (1) at (-3, -0.25) {};
%		\node [style=none] (2) at (-3, -1.75) {};
%		\node [style=map] (3) at (-2.25, -1) {$V^\perp$};
%		\node [style=none] (4) at (-2.25, -0.25) {};
%		\node [style=none] (5) at (-2.25, -1.75) {};
%		\node [style=none] (6) at (-2.25, 0.75) {};
%		\node [style=none] (7) at (-3, 0.75) {};
%		\node [style=none] (8) at (-2.25, -2.75) {};
%		\node [style=none] (9) at (-3, -2.75) {};
%	\end{pgfonlayer}
%	\begin{pgfonlayer}{edgelayer}
%		\draw (1.center) to (0);
%		\draw (0) to (2.center);
%		\draw (4.center) to (3);
%		\draw (3) to (5.center);
%		\draw [in=270, out=90] (1.center) to (6.center);
%		\draw [in=270, out=90] (4.center) to (7.center);
%		\draw [in=270, out=90] (8.center) to (2.center);
%		\draw [in=270, out=90] (9.center) to (5.center);
%	\end{pgfonlayer}
%\end{tikzpicture}
%=
%\begin{tikzpicture}
%	\begin{pgfonlayer}{nodelayer}
%		\node [style=map] (0) at (0.5, -1) {$V$};
%		\node [style=none] (1) at (0.5, 0.25) {};
%		\node [style=none] (2) at (0.5, -1.75) {};
%		\node [style=map] (3) at (1.25, -1) {$V^\perp$};
%		\node [style=none] (4) at (1.25, 0.25) {};
%		\node [style=none] (5) at (1.25, -1.75) {};
%		\node [style=none] (6) at (1.25, 1.25) {};
%		\node [style=none] (7) at (0.5, 1.25) {};
%		\node [style=none] (8) at (1.25, -2.75) {};
%		\node [style=none] (9) at (0.5, -2.75) {};
%		\node [style=s] (10) at (1.25, 0.25) {};
%		\node [style=s] (11) at (1.25, -0.25) {};
%	\end{pgfonlayer}
%	\begin{pgfonlayer}{edgelayer}
%		\draw (1.center) to (0);
%		\draw (0) to (2.center);
%		\draw (4.center) to (3);
%		\draw (3) to (5.center);
%		\draw [in=270, out=90] (1.center) to (6.center);
%		\draw [in=270, out=90] (4.center) to (7.center);
%		\draw [in=270, out=90] (8.center) to (2.center);
%		\draw [in=270, out=90] (9.center) to (5.center);
%	\end{pgfonlayer}
%\end{tikzpicture}
%=
%\begin{tikzpicture}
%	\begin{pgfonlayer}{nodelayer}
%		\node [style=map] (0) at (-3, -1) {$V$};
%		\node [style=none] (1) at (-3, -0.25) {};
%		\node [style=none] (2) at (-3, -1.75) {};
%		\node [style=map] (3) at (-2.25, -1) {$V^\perp$};
%		\node [style=none] (4) at (-2.25, -0.25) {};
%		\node [style=none] (5) at (-2.25, -1.75) {};
%		\node [style=none] (6) at (-2.25, 0.75) {};
%		\node [style=none] (7) at (-3, 0.75) {};
%		\node [style=none] (8) at (-2.25, -2.75) {};
%		\node [style=none] (9) at (-3, -2.75) {};
%		\node [style=s] (10) at (-2.25, -0.25) {};
%		\node [style=s] (11) at (-2.25, -1.75) {};
%	\end{pgfonlayer}
%	\begin{pgfonlayer}{edgelayer}
%		\draw (1.center) to (0);
%		\draw (0) to (2.center);
%		\draw (4.center) to (3);
%		\draw (3) to (5.center);
%		\draw [in=270, out=90] (1.center) to (6.center);
%		\draw [in=270, out=90] (4.center) to (7.center);
%		\draw [in=270, out=90] (8.center) to (2.center);
%		\draw [in=270, out=90] (9.center) to (5.center);
%	\end{pgfonlayer}
%\end{tikzpicture}
%$$
\end{lemma}


To check this is a functor, all we have to show is that it produces Lagrangian relations. This follows immediately from the naturality of $-1$.
This functor is symmetric monoidal and faithful but not full, as for example, the following Lagrangian relation is not in the image of $L$:
$$
\begin{tikzpicture}
	\begin{pgfonlayer}{nodelayer}
		\node [style=Z] (0) at (0.5, 0) {};
		\node [style=none] (1) at (0.5, 1) {};
		\node [style=none] (2) at (0.5, -1) {};
		\node [style=X] (3) at (1.5, 0) {};
		\node [style=X] (4) at (1, 0.5) {};
		\node [style=none] (5) at (1.5, 1) {};
		\node [style=none] (6) at (1.5, -1) {};
	\end{pgfonlayer}
	\begin{pgfonlayer}{edgelayer}
		\draw (2.center) to (0);
		\draw (0) to (1.center);
		\draw (6.center) to (3);
		\draw (3) to (5.center);
		\draw (3) to (4);
		\draw (4) to (0);
	\end{pgfonlayer}
\end{tikzpicture}
=
\begin{tikzpicture}
	\begin{pgfonlayer}{nodelayer}
		\node [style=Z] (14) at (4.5, -0.25) {};
		\node [style=none] (15) at (4.5, 0.5) {};
		\node [style=none] (16) at (4.5, -0.75) {};
		\node [style=X] (17) at (3.5, -0.25) {};
		\node [style=none] (18) at (3.5, 0.5) {};
		\node [style=none] (19) at (3.5, -0.75) {};
		\node [style=none] (21) at (3.5, 1.5) {};
		\node [style=none] (22) at (4.5, 1.5) {};
		\node [style=none] (23) at (3.5, -1.75) {};
		\node [style=none] (24) at (4.5, -1.75) {};
		\node [style=X] (25) at (4, 0.25) {};
	\end{pgfonlayer}
	\begin{pgfonlayer}{edgelayer}
		\draw (16.center) to (14);
		\draw (14) to (15.center);
		\draw (19.center) to (17);
		\draw (17) to (18.center);
		\draw [in=270, out=90] (18.center) to (22.center);
		\draw [in=270, out=90] (15.center) to (21.center);
		\draw [in=270, out=90] (23.center) to (16.center);
		\draw [in=270, out=90] (24.center) to (19.center);
		\draw (17) to (25);
		\draw (14) to (25);
	\end{pgfonlayer}
\end{tikzpicture}
=
\begin{tikzpicture}
	\begin{pgfonlayer}{nodelayer}
		\node [style=Z] (96) at (21.5, -0.25) {};
		\node [style=none] (97) at (21.5, 0.75) {};
		\node [style=none] (98) at (21.5, -0.75) {};
		\node [style=X] (99) at (20.5, -0.25) {};
		\node [style=none] (100) at (20.5, 0.75) {};
		\node [style=none] (101) at (20.5, -0.75) {};
		\node [style=Z] (102) at (21, 0.75) {};
		\node [style=none] (103) at (20.5, 1.75) {};
		\node [style=none] (104) at (21.5, 1.75) {};
		\node [style=none] (105) at (20.5, -1.75) {};
		\node [style=none] (106) at (21.5, -1.75) {};
		\node [style=s] (107) at (21.25, 0.25) {};
	\end{pgfonlayer}
	\begin{pgfonlayer}{edgelayer}
		\draw (98.center) to (96);
		\draw (96) to (97.center);
		\draw (101.center) to (99);
		\draw (99) to (100.center);
		\draw [in=-135, out=60] (99) to (102);
		\draw [in=270, out=90] (100.center) to (104.center);
		\draw [in=270, out=90] (97.center) to (103.center);
		\draw [in=270, out=90] (105.center) to (98.center);
		\draw [in=270, out=90] (106.center) to (101.center);
		\draw [in=-90, out=120] (96) to (107);
		\draw [in=-45, out=90] (107) to (102);
	\end{pgfonlayer}
\end{tikzpicture}
=
\begin{tikzpicture}
	\begin{pgfonlayer}{nodelayer}
		\node [style=Z] (0) at (2.5, 0) {};
		\node [style=none] (1) at (2.5, 0.75) {};
		\node [style=none] (2) at (2.5, -0.75) {};
		\node [style=X] (3) at (1.5, 0) {};
		\node [style=none] (5) at (1.5, 0.75) {};
		\node [style=none] (6) at (1.5, -0.75) {};
		\node [style=Z] (7) at (2, 0.5) {};
		\node [style=none] (8) at (1.5, 1.75) {};
		\node [style=none] (9) at (2.5, 1.75) {};
		\node [style=none] (10) at (1.5, -1.75) {};
		\node [style=none] (11) at (2.5, -1.75) {};
		\node [style=s] (12) at (2.5, -0.75) {};
		\node [style=s] (13) at (2.5, 0.75) {};
	\end{pgfonlayer}
	\begin{pgfonlayer}{edgelayer}
		\draw (2.center) to (0);
		\draw (0) to (1.center);
		\draw (6.center) to (3);
		\draw (3) to (5.center);
		\draw (3) to (7);
		\draw (0) to (7);
		\draw [in=270, out=90] (5.center) to (9.center);
		\draw [in=270, out=90] (1.center) to (8.center);
		\draw [in=270, out=90] (10.center) to (2.center);
		\draw [in=270, out=90] (11.center) to (6.center);
	\end{pgfonlayer}
\end{tikzpicture}
$$
%
%And similarly, we also have:
%
%
%$$
%\begin{tikzpicture}
%	\begin{pgfonlayer}{nodelayer}
%		\node [style=Z] (0) at (0.5, 0) {};
%		\node [style=none] (1) at (0.5, 1) {};
%		\node [style=none] (2) at (0.5, -1) {};
%		\node [style=X] (3) at (1.5, 0) {};
%		\node [style=Z] (4) at (1, 0.5) {};
%		\node [style=none] (5) at (1.5, 1) {};
%		\node [style=none] (6) at (1.5, -1) {};
%	\end{pgfonlayer}
%	\begin{pgfonlayer}{edgelayer}
%		\draw (2.center) to (0);
%		\draw (0) to (1.center);
%		\draw (6.center) to (3);
%		\draw (3) to (5.center);
%		\draw (3) to (4);
%		\draw (4) to (0);
%	\end{pgfonlayer}
%\end{tikzpicture}
%=
%\begin{tikzpicture}
%	\begin{pgfonlayer}{nodelayer}
%		\node [style=Z] (0) at (2.5, 0) {};
%		\node [style=none] (1) at (2.5, 0.75) {};
%		\node [style=none] (2) at (2.5, -0.75) {};
%		\node [style=X] (3) at (1.5, 0) {};
%		\node [style=none] (5) at (1.5, 0.75) {};
%		\node [style=none] (6) at (1.5, -0.75) {};
%		\node [style=X] (7) at (2, 0.5) {};
%		\node [style=none] (8) at (1.5, 1.75) {};
%		\node [style=none] (9) at (2.5, 1.75) {};
%		\node [style=none] (10) at (1.5, -1.75) {};
%		\node [style=none] (11) at (2.5, -1.75) {};
%		\node [style=s] (12) at (2.5, -0.75) {};
%		\node [style=s] (13) at (2.5, 0.75) {};
%	\end{pgfonlayer}
%	\begin{pgfonlayer}{edgelayer}
%		\draw (2.center) to (0);
%		\draw (0) to (1.center);
%		\draw (6.center) to (3);
%		\draw (3) to (5.center);
%		\draw (3) to (7);
%		\draw (0) to (7);
%		\draw [in=270, out=90] (5.center) to (9.center);
%		\draw [in=270, out=90] (1.center) to (8.center);
%		\draw [in=270, out=90] (10.center) to (2.center);
%		\draw [in=270, out=90] (11.center) to (6.center);
%	\end{pgfonlayer}
%\end{tikzpicture}
%$$
%
%
%\begin{theorem}
%These two extra generators, along with the image of the $L$ generate $\Lag\Rel(\F_p)$, for any prime $p$.
%\end{theorem}
%
%In other words, $\Lag\Rel(\F_p)$ arises as ${\sf CPM}(\Rel(\F_p), \perp,\{\Zdot, \Xdot\})$. The category of completely positive maps with respect to the conjugation functor $\perp$ and the two compact structures of linear relations induced by (co)addition and (co)copying.
%
%It is useful to realise, that by Euler decomposition, the Hadamard gate is derived from these generators:
%
%$$
%\begin{tikzpicture}[xscale=-1]
%	\begin{pgfonlayer}{nodelayer}
%		\node [style=none] (0) at (0, 0) {};
%		\node [style=none] (1) at (0.75, 0) {};
%		\node [style=s] (2) at (0.75, -0.5) {};
%		\node [style=none] (3) at (0.75, -1.5) {};
%		\node [style=none] (4) at (0, -1.5) {};
%		\node [style=none] (5) at (0, -0.5) {};
%	\end{pgfonlayer}
%	\begin{pgfonlayer}{edgelayer}
%		\draw [in=270, out=90] (3.center) to (5.center);
%		\draw (5.center) to (0.center);
%		\draw (1.center) to (2);
%		\draw [in=90, out=-90] (2) to (4.center);
%	\end{pgfonlayer}
%\end{tikzpicture}
%$$
%
%And interpreted as a matrix, this is the symplectic form.
%
%
%
%\begin{definition}
%Let $\Aff\Lag\Rel(k)$, denote the affinization of Lagrangian relations.  That is to say, the following pushout of props:
%
%$$
%\Lag\Rel(k) \xleftarrow{S} \Rel(\Mat(k)) \xrightarrow{S} \Lag\Rel(k)  \xrightarrow{G} \Rel(\Mat(k)) \xrightarrow{F} \Rel(\Aff\Mat(k))
%$$
%
%DRAW DISGUSTING PUSHOUT CUBE
%
%
%Where $ G:\Lag\Rel(k)  \rightarrow \Rel(\Mat(k))$ is the forgetful functor and $F: \Rel(\Mat(k))\to  \Rel(\Aff\Mat(k))$ is the free functor.
%\end{definition}
%
%
%This is just a formal way of saying that we are adding an affine shift for each of the gradations of Lagrangian relations.
%
%Notice that this defines another functor $\Aff\Mat(k) \to \Aff\Lag\Rel(k)$ extending the $L$.



%\newpage
%
%
%SYMPLECTOMORPHISMS
%  GENERATED BY FOURIER, PHASE SHIFT, CNOT
%
%
%DEFINE TABLEAU FOR LAGRANGIAN RELATION
%
%
%Just as relations can be represented by matrices over the Boolean semiring, linear relations over $k$ can be represented by matrices over $\F_p$.
%For the sake of simplicity, let us only consider linear relations which are states, as processes can be obtained from currying.
%
%
%Recall that $k$-linear relations $0\to n$ are in one-to-one correspondance with linear subspaces of $k^n$.
%In particular, a linear subspace generated by $m$ column vectors, $v_1,\cdots, v_m$, in $k^n$ is represented by an $n\times m$ matrix.
%
%Recall that a Lagrangian relation $0 \to k^{2n}$ can be characterized as an isotropic subspace of dimension $n$,
%This can be recast in terms of this matrix representation.  In particular $n$-dimensional Lagrangian relations are in one-to-one correspondance with $n\times 2n$ dimensional matrix $M$ with rank $n$  when all rows are orthogonal with respect to the symplectic form.  Take $v_i = z_i x_i$, then for all $1 \leq i,j \leq n$,
%then this symplectic orthogonality is restated as $0=\omega(v_i,v_j) = \langle x_i, z_j \rangle - \langle z_i, x_j \rangle $ which holds if and only if $\langle x_i, z_j \rangle = \langle z_i, x_j \rangle $.
%Similarly, the dimension $n$ condition is equivalent to asking that $\langle v_i,v_j \rangle = 0$ for all $i,j$.


\section{Generators for Lagrangian relations}
\label{sec:univ}

%
%Linear subspaces can be represented in terms of the row space of a matrix.
%In particular, an $n$-dimensional Lagrangian subspace is represented by a block diagonal matrix of the form $[X|Z]$ for $X,Z$ both $n\times n$ matrices.
%This correspondance is made one to one, when the symplectic form is required to vanish, so that $[X|Z] w [X|Z]^n =0$ as well as  $[X|Z]$  having dimension $n$.

In this section, we shall give a universal set of generators for $\Lag\Rel_k$; although, we do not directly give a complete set of identities.  Instead we defer to the completeness of the underlying category $\ih_k\cong\LinRel_k$.


Consider the following symplectomorphisms; the discrete Fourier transform $F$,  the $a$-shift gate $S_a$ and the controlled-$a$ gate $C_a$:
$$
\left\llbracket
\begin{tikzpicture}
	\begin{pgfonlayer}{nodelayer}
		\node [style=none] (0) at (0.5, 1) {};
		\node [style=none] (1) at (0.5, -0.25) {};
		\node [style=none] (2) at (1, -0.25) {};
		\node [style=none] (3) at (1, 1) {};
		\node [style=s] (4) at (1, 0.5) {};
		\node [style=none] (5) at (0.5, 0.5) {};
	\end{pgfonlayer}
	\begin{pgfonlayer}{edgelayer}
		\draw (4) to (3.center);
		\draw [in=90, out=-90] (4) to (1.center);
		\draw [in=-90, out=90] (2.center) to (5.center);
		\draw (5.center) to (0.center);
	\end{pgfonlayer}
\end{tikzpicture}
\right\rrbracket
=
\begin{bmatrix}
0   & 1 \\
-1  & 0
\end{bmatrix}
\hspace*{.2cm}
\left\llbracket
\begin{tikzpicture}
	\begin{pgfonlayer}{nodelayer}
		\node [style=X] (0) at (0.5, 1.25) {};
		\node [style=Z] (1) at (1.5, -0.25) {};
		\node [style=scalar] (2) at (1, 0.5) {$a$};
		\node [style=none] (3) at (0.5, 1.75) {};
		\node [style=none] (4) at (1.5, 1.75) {};
		\node [style=none] (5) at (1.5, -0.75) {};
		\node [style=none] (6) at (0.5, -0.75) {};
	\end{pgfonlayer}
	\begin{pgfonlayer}{edgelayer}
		\draw (5.center) to (1);
		\draw (1) to (4.center);
		\draw [in=-90, out=135] (1) to (2);
		\draw [in=-45, out=90] (2) to (0);
		\draw (3.center) to (0);
		\draw (0) to (6.center);
	\end{pgfonlayer}
\end{tikzpicture}
\right\rrbracket
=
\begin{bmatrix}
1 &a\\
0 & 1
\end{bmatrix}
\hspace*{.2cm}
\left\llbracket
\begin{tikzpicture}
	\begin{pgfonlayer}{nodelayer}
		\node [style=Z] (430) at (219.75, 0) {};
		\node [style=X] (431) at (220.75, 1.5) {};
		\node [style=none] (432) at (219.75, 0) {};
		\node [style=none] (433) at (221.25, -0.75) {};
		\node [style=none] (434) at (219.25, 2.25) {};
		\node [style=none] (435) at (220.75, 2.25) {};
		\node [style=scalar] (436) at (220.25, 0.75) {$a$};
		\node [style=X] (437) at (217.5, 0) {};
		\node [style=Z] (438) at (218.5, 1.5) {};
		\node [style=none] (439) at (217.5, -0.75) {};
		\node [style=none] (440) at (219, -0.75) {};
		\node [style=none] (441) at (217.25, 2.25) {};
		\node [style=none] (442) at (218.5, 1.5) {};
		\node [style=scalarop] (443) at (218, 0.75) {$a$};
		\node [style=none] (445) at (219.75, -0.75) {};
		\node [style=none] (446) at (218.5, 2.25) {};
	\end{pgfonlayer}
	\begin{pgfonlayer}{edgelayer}
		\draw [in=-105, out=30] (430) to (436);
		\draw [in=-150, out=90] (436) to (431);
		\draw [in=90, out=-60] (431) to (433.center);
		\draw (431) to (435.center);
		\draw [in=135, out=-90, looseness=0.75] (434.center) to (430);
		\draw (439.center) to (437);
		\draw [in=-105, out=30] (437) to (443);
		\draw [in=-150, out=90] (443) to (438);
		\draw [in=90, out=-45, looseness=0.75] (438) to (440.center);
		\draw [in=120, out=-90] (441.center) to (437);
		\draw [in=270, out=90] (442.center) to (446.center);
		\draw [in=270, out=90] (445.center) to (432.center);
	\end{pgfonlayer}
\end{tikzpicture}
\right\rrbracket
=
\begin{bmatrix}
1 & -a & 0 & 0 \\
0 & 1 & 0 & 0 \\
0 & 0 & 1 & 0 \\
0 & 0 & a & 1
\end{bmatrix}
%\begin{bmatrix}
%1 & -a & 0 & 0 \\
%0 & 1 & 0 & 0 \\
%0 & 0 & 1 & 0 \\
%0 & 0 & a & 0
%\end{bmatrix}
$$


%When composed with identities, these have having the following action on Lagrangian subspaces when applied to certain wires:

Use the notation $G^{(j)}$ to denote a gate $G$ being applied to wire $j$; and the notation $C_a^{(i,j)}$ to denote the controlled-$a$ gate controlling on wire $i$ targetting wire $j$.

Note the right action of these gates in terms of matrix multiplication of Lagrangian subspaces for any nonzero $a \in k$ (as observed in \cite[p. 4]{aaronson}):

\begin{itemize}
\item
$F^{(i)}$ sets columns $x_i$ to $-z_i$ and $z_i$ to $x_i$.

\item
$S_a^{(i)}$ sets $z_i$ to $z_i+a\cdot x_i$.

%\item
%$M_a^{(i)}$ sets $x_i$ to $a\cdot x_i$ and $z_i$ to $a^{-1}\cdot x_i$.

\item
$C_a^{(i,j)}$ sets $x_j$ to $x_j- a \cdot x_i$ and $z_i$ to $z_i+a\cdot z_j$.

\end{itemize}

Using these symplectomorphisms regarded as Lagrangian relations, we have:


%
%
%
%\begin{align*}
%&\left[
%\begin{array}{*{7}c|*{7}c}
%x_{1,1} & \cdots & x_{1,a} & \cdots & x_{1,b} & \cdots & x_{1,n} &  z_{1,1} & \cdots & z_{1,a} & \cdots & z_{1,b} & \cdots & z_{1,n} \\-
%\vdots   & \ddots  & \vdots  & \ddots  & \vdots   &  \ddots & \vdots   & \vdots    & \ddots  & \vdots  & \ddots  & \vdots  & \ddots  & \vdots   \\
%x_{n,1} & \cdots & x_{n,a} & \cdots & x_{n,b} & \cdots & x_{n,n} &  z_{n,1} & \cdots & z_{n,a} & \cdots & z_{n,b} & \cdots & z_{n,n}
%\end{array}
%\right]\\
%&\mapsto\\
%&\left[
%\begin{array}{*{7}c|*{7}c}
%x_{1,1} & \cdots & x_{1,a} & \cdots & x_{1,b} - kx_{1,a} & \cdots & x_{1,n} &  z_{1,1} & \cdots & z_{1,a} +k z_{1,b} & \cdots & z_{1,b} & \cdots & z_{1,n} \\
%\vdots   & \ddots  & \vdots  & \ddots  & \vdots                    &  \ddots & \vdots   & \vdots    & \ddots  & \vdots                  & \ddots  & \vdots  & \ddots  & \vdots   \\
%x_{n,1} & \cdots & x_{n,a} & \cdots & x_{n,b} -kx_{n,a} & \cdots & x_{n,n} &  z_{n,1} & \cdots & z_{n,a} +k  z_{n,b}& \cdots & z_{n,b} & \cdots & z_{n,n}
%\end{array}
%\right]
%\end{align*}
%
%
%
%\begin{align*}
%&\left[
%\begin{array}{*{5}c|*{5}c}
%x_{1,1} & \cdots & x_{1,a}  & \cdots & x_{1,n} &  z_{1,1} & \cdots & z_{1,a} & \cdots  & z_{1,n} \\
%\vdots   & \ddots  & \vdots   &  \ddots & \vdots   & \vdots    & \ddots  & \vdots  & \ddots   & \vdots   \\  
%x_{n,1} & \cdots & x_{n,a} & \cdots & x_{n,n} &  z_{n,1} & \cdots & z_{n,a} & \cdots   & z_{n,n}
%\end{array}
%\right]\\
%&\mapsto\\
%&\left[
%\begin{array}{*{7}c|*{7}c}
%x_{1,1} & \cdots & x_{1,a-1} & -z_{1,a} &  x_{1,a+1}   & \cdots & x_{1,n} &  z_{1,1} & \cdots & z_{1,a-1} & x_{1,a} &  z_{1,a+1} & \cdots  & z_{1,n} \\
%\vdots   & \ddots  & \vdots      & \vdots   & \vdots           &  \ddots & \vdots   & \vdots    & \ddots  & \vdots      & \vdots    & \vdots        & \ddots   & \vdots   \\  
%x_{n,1} & \cdots & x_{n,a-1} & -z_{n,a} &  x_{n,a+1} & \cdots & x_{n,n} &  z_{n,1} & \cdots &  z_{n,a-1} & x_{n,a} &  z_{n,a+1} & \cdots   & z_{n,n}
%\end{array}
%\right]\\
%\end{align*}
%
%
%



\begin{theorem}
\label{theorem:generators}
For any field $k$ the maps in $L(\LinRel_k)$ as well as $F$ and $S_a$ for all $a \in k$ generate $\Lag\Rel_k$.
\end{theorem}

\begin{proof}
Consider the matrix $[X|Z]$ of an arbitrary Lagrangian relation over field $k$ seen as a state.
%
%$$
%[X | Z]
%=
%\left[
%\begin{array}{ccc|ccc}
%x_{1,1} & \cdots & x_{n,n} & z_{1,1} & \cdots & z_{1,n}\\
%\vdots    & \ddots & \vdots    & \vdots   & \ddots & \vdots \\
%x_{n,1} & \cdots & x_{n,n} & z_{n,1} & \cdots & z_{n,n}\\
%\end{array}
%\right]
%$$
We show how one can reduce $[X|Z]$ to the block matrix $[I|0]$ by right multiplication with the aforementioned symplectomorphisms.
To do so, we first reduce it to a matrix $[I|Z']$
by only applying row operations (keeping the subspace the same) as well as the Fourier transform.
This involve repeatedly do Gaussian elimination and then applying the Fourier transform to wires when the pivot is in the $Z$ block.
We are guaranteed to obtain a matrix $[I|Z']$ because the dimension of Lagrangian subspace is $n$.
A very similar observation was made in \cite[Lem. 6]{aaronson}.

$$
\begin{tikzpicture}
	\begin{pgfonlayer}{nodelayer}
		\node [style=Z] (309) at (26.5, 8.75) {};
		\node [style=none] (310) at (26.5, 8) {};
		\node [style=none] (311) at (25.75, 10.25) {};
		\node [style=none] (312) at (27, 9.5) {};
		\node [style=none] (313) at (30.75, 8.25) {};
		\node [style=Z] (314) at (30.75, 9) {};
		\node [style=none] (315) at (30.5, 10) {};
		\node [style=none] (316) at (31.5, 9.5) {};
		\node [style=none] (317) at (32, 8.25) {};
		\node [style=Z] (318) at (28.5, 8.75) {};
		\node [style=none] (319) at (28.5, 8) {};
		\node [style=none] (320) at (28, 9.5) {};
		\node [style=none] (321) at (29.25, 10.25) {};
		\node [style=none] (322) at (27.5, 10) {};
		\node [style=none] (323) at (33.25, 8.25) {};
		\node [style=Z] (324) at (33.25, 9) {};
		\node [style=none] (325) at (32.75, 10) {};
		\node [style=none] (326) at (33.75, 10) {};
		\node [style=none] (327) at (32.5, 9) {,};
		\node [style=none] (328) at (29.75, 9) {,};
		\node [style=none] (329) at (27.5, 11) {(1)};
		\node [style=none] (330) at (31.25, 11) {(2)};
		\node [style=none] (331) at (33.25, 11) {(3)};
	\end{pgfonlayer}
	\begin{pgfonlayer}{edgelayer}
		\draw (310.center) to (309);
		\draw [in=-90, out=150, looseness=0.75] (309) to (311.center);
		\draw [in=30, out=-90] (312.center) to (309);
		\draw [in=135, out=-90] (315.center) to (314);
		\draw (313.center) to (314);
		\draw [in=45, out=180, looseness=0.75] (316.center) to (314);
		\draw [in=0, out=90, looseness=0.75] (317.center) to (316.center);
		\draw (319.center) to (318);
		\draw [in=-90, out=150] (318) to (320.center);
		\draw [in=30, out=-90] (321.center) to (318);
		\draw [in=0, out=90, looseness=0.75] (320.center) to (322.center);
		\draw [in=90, out=180, looseness=0.75] (322.center) to (312.center);
		\draw [in=150, out=-90] (325.center) to (324);
		\draw (323.center) to (324);
		\draw [in=30, out=-90] (326.center) to (324);
	\end{pgfonlayer}
\end{tikzpicture}
$$

As the Fourier transform is a symplectomorphism $[I|Z']$ is isotropic, so that:

$$
0
=
\begin{bmatrix}
I | Z'
\end{bmatrix}
\omega
\begin{bmatrix}
I | Z'
\end{bmatrix}^T
$$
which holds if and only if 
$$
0=
\begin{bmatrix}
I | Z'
\end{bmatrix}
\begin{bmatrix}
Z' | -I 
\end{bmatrix}^T
=
{Z'}^T-Z'
$$

That is to say $Z'$ is symmetric, meaning that $Z'$ describes the adjacency matrix of a graph coloured by the elements of $k$.
In the language of stabilizer circuits, this is called a {\em graph state}.  In the case of prime fields, this observation was made in \cite[Eq. 18]{gross}.  Graph states were originally discussed in \cite{hein2006entanglement}.


%
%The next part of the algorithm is reducing the second block of the matrix to 0. In the case of $\F_2$, this is reduced to applying controlled-Z gates to the wires which contain an edge in the adjacency matrix of the graph of $Z'$.
%In the case of a general field, we must take a slightly more nuanced approach.
%
%
%
%
%% Input: Matrix representation (X|Z) of Lagrangian subspace
%% Row reduce (X|Z)
%% For all rows i, if the pivot is in the Z block, apply the Fourier transform to row i
%% Because Lagrangian subspaces of k^{2n} have dimension n, this matrix has full rank, 
%%   Therefore, row reduce further until  the X block is the identity
%%Now the Z block is a symmetric matrix
%% For all i from 1 to n
%%   For all j from 1 to i
%%     Apply Fourier transform to row j
%%     Apply C_{z_{i,j}} gate from row i to j
%%    Apply inverse fourier transform to row j


We prove that graph states can be reduced to the subspace $[I|0]$ by induction on the dimension of the subspace.
This base case is trivial.

Suppose we have a $(n+1)$-dimensional Lagrangian subspaces described by a graph state, then:
\newpage

\resizebox{\linewidth}{!}{
  \begin{minipage}{\linewidth}
\begin{align*}
&\left[
\begin{array}{*{5}c|*{6}c}
1         & 0        & 0         & \cdots & 0         & z_{1,1} & z_{1,2} & z_{1,3} & \cdots & z_{1,n}\\
0         & 1        & 0         & \cdots & 0         & z_{1,2} & z_{2,2} & z_{2,3} & \cdots & z_{2,n}\\
0         & 0        & 1         & \ddots & \vdots & z_{1,3} & z_{2,3} & z_{3,3} & \cdots & z_{3,n}\\
\vdots & \vdots & \ddots & \ddots & 0         & \vdots   & \vdots    & \vdots    & \ddots &  \vdots \\
0         & 0         & \cdots & 0        & 1          & z_{1,n} & z_{2,n} & z_{3,n} & \cdots & z_{n,n}\\
\end{array}
\right]
\xmapsto{(F^{(1)})^{-1}}
\left[
\begin{array}{*{5}c|*{6}c}
z_{1,1}         & 0        & 0         & \cdots & 0         & -1 & z_{1,2} & z_{1,3} & \cdots & z_{1,n}\\
z_{1,2}         & 1        & 0         & \cdots & 0         & 0 & z_{2,2} & z_{2,3} & \cdots & z_{2,n}\\
z_{1,3}         & 0        & 1         & \ddots & \vdots & 0 & z_{2,3} & z_{3,3} & \cdots & z_{3,n}\\
\vdots & \vdots & \ddots & \ddots & 0         & \vdots   & \vdots    & \vdots    & \ddots &  \vdots \\
z_{1,n}         & 0         & \cdots & 0        & 1          & 0 & z_{2,n} & z_{3,n} & \cdots & z_{n,n}\\
\end{array}
\right]\\
&\xmapsto{C_{z_{1,2}}^{(2,1)}}\\
&\left[
\begin{array}{*{5}c|*{6}c}
z_{1,1}-0           & 0         & 0         & \cdots & 0         & -1 & z_{1,2}-z_{1,2} & z_{1,3} & \cdots & z_{1,n}\\
z_{1,2}-z_{1,2} & 1         & 0         & \cdots & 0         & 0 & z_{2,2}-0            & z_{2,3} & \cdots & z_{2,n}\\
z_{1,3}-0           & 0         & 1         & \ddots & \vdots & 0 & z_{2,3}-0            & z_{3,3} & \cdots & z_{3,n}\\
\vdots                 & \vdots & \ddots & \ddots & 0         & \vdots   & \vdots        & \vdots    & \ddots &  \vdots \\
z_{1,n}-0           & 0         & \cdots & 0        & 1          & 0 & z_{2,n}-0            & z_{3,n} & \cdots & z_{n,n}\\
\end{array}
\right]
=
\left[
\begin{array}{*{5}c|*{6}c}
z_{1,1}             & 0         & 0         & \cdots & 0         & -1 & 0                         & z_{1,3} & \cdots & z_{1,n}\\
0                       & 1         & 0         & \cdots & 0         & 0 & z_{2,2}                & z_{2,3} & \cdots & z_{2,n}\\
z_{1,3}             & 0         & 1         & \ddots & \vdots & 0 & z_{2,3}                & z_{3,3} & \cdots & z_{3,n}\\
\vdots                & \vdots & \ddots & \ddots & 0         & \vdots   & \vdots        & \vdots    & \ddots &  \vdots \\
z_{1,n}             & 0         & \cdots & 0        & 1          & 0 & z_{2,n}                & z_{3,n} & \cdots & z_{n,n}\\
\end{array}
\right]\\
&\xmapsto{\prod_{i>1}^n C_{z_{1,i}}^{(i,1)}}\\
&\left[
\begin{array}{*{5}c|*{6}c}
z_{1,1}             & 0         & 0         & \cdots & 0         & -1 & 0                         & 0           & \cdots & 0\\
0                       & 1         & 0         & \cdots & 0         & 0 & z_{2,2}                & z_{2,3} & \cdots & z_{2,n}\\
0                       & 0         & 1         & \ddots & \vdots & 0 & z_{2,3}                & z_{3,3} & \cdots & z_{3,n}\\
\vdots               & \vdots & \ddots & \ddots & 0         & \vdots   & \vdots        & \vdots    & \ddots &  \vdots \\
0                       & 0         & \cdots & 0        & 1          & 0 & z_{2,n}                & z_{3,n} & \cdots & z_{n,n}\\
\end{array}
\right]
\xmapsto{F^{(1)}}
\left[
\begin{array}{*{5}c|*{6}c}
1                       & 0         & 0         & \cdots & 0         & z_{1,1} & 0                          & 0           & \cdots & 0\\
0                       & 1         & 0         & \cdots & 0         & 0           & z_{2,2}                & z_{2,3} & \cdots & z_{2,n}\\
0                       & 0         & 1         & \ddots & \vdots & 0           & z_{2,3}                & z_{3,3} & \cdots & z_{3,n}\\
\vdots               & \vdots & \ddots & \ddots & 0         & \vdots   & \vdots                   & \vdots    & \ddots &  \vdots \\
0                       & 0         & \cdots & 0        & 1          & 0           & z_{2,n}                & z_{3,n} & \cdots & z_{n,n}\\
\end{array}
\right]\\
&\xmapsto{S_{-z_{1,1}}^{(1)}  }\\
&\left[
\begin{array}{*{5}c|*{6}c}
1                       & 0         & 0         & \cdots & 0         & 0 & 0                          & 0           & \cdots & 0\\
0                       & 1         & 0         & \cdots & 0         & 0           & z_{2,2}                & z_{2,3} & \cdots & z_{2,n}\\
0                       & 0         & 1         & \ddots & \vdots & 0           & z_{2,3}                & z_{3,3} & \cdots & z_{3,n}\\
\vdots               & \vdots & \ddots & \ddots & 0         & \vdots   & \vdots                   & \vdots    & \ddots &  \vdots \\
0                       & 0         & \cdots & 0        & 1          & 0           & z_{2,n}                & z_{3,n} & \cdots & z_{n,n}\\
\end{array}
\right]\\
\end{align*}
  \end{minipage}
}


Therefore all Lagrangian relations can be reduced to the subspace $[I|0]$ by right multiplication by symplectomorphisms.
In the $n$-dimensional case, this subspace is given by the circuit
$L(
\begin{tikzpicture}[scale=.5]
	\begin{pgfonlayer}{nodelayer}
		\node [style=X] (0) at (6, 3) {};
		\node [style=none] (1) at (6, 3.5) {};
	\end{pgfonlayer}
	\begin{pgfonlayer}{edgelayer}
		\draw (0) to (1.center);
	\end{pgfonlayer}
\end{tikzpicture}
^{\otimes n}$).




Thus, we have already described all the generators of Lagrangian relations. The gates $F$, $C_a$ for all $a \in k$, along with the cup and cap and the zero state generate all Lagrangian relations.  Note that $C_a$ and the zero state are both in the image of the $L$.

\end{proof} 


%
%This is proved by using these symplectomorphisms to reduce a Lagrangian relation by Guassian elimination to the state
%$L(
%\begin{tikzpicture}[scale=.5]
%	\begin{pgfonlayer}{nodelayer}
%		\node [style=X] (0) at (6, 3) {};
%		\node [style=none] (1) at (6, 3.5) {};
%	\end{pgfonlayer}
%	\begin{pgfonlayer}{edgelayer}
%		\draw (0) to (1.center);
%	\end{pgfonlayer}
%\end{tikzpicture}
%^{\otimes n}$).
%

We can also give a presentation of this category which is very close to Selinger's CPM construction~\cite{cpm}. There are several equivalent ways to define the CPM construction. For our purposes, the most convenient one is the presentation used in both in~\cite{pqp,cqm1}, which defines $\CPM[\X]$ as the subcategory of a dagger compact closed category $\X$ whose objects are of the form $A^* \otimes A$ for $A \in \X$ and whose morphisms are generated by (i) `pure' morphisms, i.e. morphisms of the form $f_* \otimes f$ for $f \in \X$ and a covariant functor $(\_)_*$, and (ii) a `discard' morphism $d_A$ for every $A \in \X$ given by the counit $d_A := \epsilon_A : A^* \otimes A \to I$ of the compact closed structure on $A$.

We nearly obtain such a presentation for $\Lag\Rel_k$ using the covariant functor $(-)^\perp$ to define pure morphisms, with the only caveat being we need to consider a family of discard morphisms: each discard morphism being parametrised by a field element.

\begin{theorem}
\label{theorem:unbiased}
$\Lag\Rel_k$ is the monoidal subcategory of $\LinRel_k$ whose objects are of the form $k^n \oplus k^n$, for all natural numbers $n$, and whose morphisms are generated by \textit{pure} morphisms of the form $V^\perp \oplus V$ for $V \in \LinRel_k$ and for each $a \in k$, a `discard' morphism:
$$
d_a :=
\begin{tikzpicture}
	\begin{pgfonlayer}{nodelayer}
		\node [style=X] (0) at (0, 0.75) {};
		\node [style=scalar] (1) at (0.5, 0) {$a$};
		\node [style=none] (2) at (0.5, -0.75) {};
		\node [style=none] (3) at (-0.5, 0) {};
		\node [style=none] (4) at (-0.5, -0.75) {};
	\end{pgfonlayer}
	\begin{pgfonlayer}{edgelayer}
		\draw [in=-30, out=90] (1) to (0);
		\draw [in=90, out=-150] (0) to (3.center);
		\draw (4.center) to (3.center);
		\draw (2.center) to (1);
	\end{pgfonlayer}
\end{tikzpicture}
$$
\end{theorem}


\begin{proof}
  We just have to show that $F$ and $S_a$ can be constructed using these generators. The $S_a$ gate and it's colour-reversed version $V_a$ can be obtained by composing a pure morphism with $d_a$ and $d_{-a}$, respectively:
$$
\begin{tikzpicture}
	\begin{pgfonlayer}{nodelayer}
		\node [style=none] (0) at (1, -1.25) {};
		\node [style=Z] (1) at (1, -0.75) {};
		\node [style=X] (2) at (-0.75, -0.75) {};
		\node [style=X] (3) at (-0.375, 0.75) {};
		\node [style=none] (4) at (-0.75, -1.25) {};
		\node [style=none] (5) at (-0.75, 1.25) {};
		\node [style=none] (6) at (1, 1.25) {};
		\node [style=scalar] (7) at (0.5, 0) {$a$};
		\node [style=none] (8) at (-1.25, 0) {};
		\node [style=none] (9) at (1.75, 0) {$=$};
		\node [style=none] (10) at (3.5, -1) {};
		\node [style=Z] (11) at (3.5, -0.5) {};
		\node [style=X] (12) at (2.5, 0.5) {};
		\node [style=none] (13) at (2.5, -1) {};
		\node [style=scalar] (14) at (3, 0) {$a$};
		\node [style=none] (15) at (2.5, 1) {};
		\node [style=none] (16) at (3.5, 1) {};
	\end{pgfonlayer}
	\begin{pgfonlayer}{edgelayer}
		\draw (1) to (0.center);
		\draw (1) to (6.center);
		\draw (2) to (5.center);
		\draw (2) to (4.center);
		\draw [in=-90, out=165] (1) to (7);
		\draw [in=0, out=90] (7) to (3);
		\draw [in=-90, out=150] (2) to (8.center);
		\draw [in=-180, out=90] (8.center) to (3);
		\draw (11) to (10.center);
		\draw [in=-90, out=165] (11) to (14);
		\draw [in=-15, out=90] (14) to (12);
		\draw (13.center) to (12);
		\draw (11) to (16.center);
		\draw (12) to (15.center);
	\end{pgfonlayer}
\end{tikzpicture}
\ = S_a
\qquad\qquad
\begin{tikzpicture}
	\begin{pgfonlayer}{nodelayer}
		\node [style=none] (0) at (1, -1.25) {};
		\node [style=X] (1) at (1, -0.75) {};
		\node [style=Z] (2) at (-0.75, -0.75) {};
		\node [style=Z] (3) at (-0.375, 0.75) {};
		\node [style=none] (4) at (-0.75, -1.25) {};
		\node [style=none] (5) at (-0.75, 1.25) {};
		\node [style=none] (6) at (1, 1.25) {};
		\node [style=scalar] (7) at (0.5, 0) {$a$};
		\node [style=none] (8) at (-1.25, 0) {};
		\node [style=none] (9) at (1.75, 0) {$=$};
		\node [style=none] (10) at (3.5, -1) {};
		\node [style=X] (11) at (3.5, -0.5) {};
		\node [style=Z] (12) at (2.5, 0.5) {};
		\node [style=none] (13) at (2.5, -1) {};
		\node [style=scalar] (14) at (3, 0) {$a$};
		\node [style=none] (15) at (2.5, 1) {};
		\node [style=none] (16) at (3.5, 1) {};
		\node [style=none] (17) at (-2.75, -1.25) {};
		\node [style=X] (18) at (-2.75, -0.75) {};
		\node [style=Z] (19) at (-4.5, -0.75) {};
		\node [style=X] (20) at (-4.125, 0.75) {};
		\node [style=none] (21) at (-4.5, -1.25) {};
		\node [style=none] (22) at (-4.5, 1.25) {};
		\node [style=none] (23) at (-2.75, 1.25) {};
		\node [style=scalar] (24) at (-3.25, 0) {-$a$};
		\node [style=none] (25) at (-5, 0) {};
		\node [style=none] (26) at (-2, 0) {$=$};
	\end{pgfonlayer}
	\begin{pgfonlayer}{edgelayer}
		\draw (1) to (0.center);
		\draw (1) to (6.center);
		\draw (2) to (5.center);
		\draw (2) to (4.center);
		\draw [in=-90, out=165] (1) to (7);
		\draw [in=0, out=90] (7) to (3);
		\draw [in=-90, out=150] (2) to (8.center);
		\draw [in=-180, out=90] (8.center) to (3);
		\draw (11) to (10.center);
		\draw [in=-90, out=165] (11) to (14);
		\draw [in=-15, out=90] (14) to (12);
		\draw (13.center) to (12);
		\draw (11) to (16.center);
		\draw (12) to (15.center);
		\draw (18) to (17.center);
		\draw (18) to (23.center);
		\draw (19) to (22.center);
		\draw (19) to (21.center);
		\draw [in=-90, out=165] (18) to (24);
		\draw [in=0, out=90] (24) to (20);
		\draw [in=-90, out=150] (19) to (25.center);
		\draw [in=-180, out=90] (25.center) to (20);
	\end{pgfonlayer}
\end{tikzpicture}
\  =: V_a
$$

We can then obtain $F$ as $S_1 \circ V_1 \circ S_1$, which can be proven as a variation of the familiar `3 CNOT' rule for quantum circuits (see e.g.~\cite[\S 3.2.1]{coecke2008interacting}):
$$
S_1 \circ V_1 \circ S_1
=
\begin{tikzpicture}[xscale=-1]
	\begin{pgfonlayer}{nodelayer}
		\node [style=Z] (0) at (3, 1.25) {};
		\node [style=X] (1) at (4, 1.75) {};
		\node [style=Z] (2) at (4, 2.5) {};
		\node [style=X] (3) at (3, 2) {};
		\node [style=Z] (4) at (3, 2.75) {};
		\node [style=X] (5) at (4, 3.25) {};
		\node [style=none] (6) at (3, 4.5) {};
		\node [style=none] (7) at (4, 4.5) {};
		\node [style=none] (8) at (3, 0) {};
		\node [style=none] (9) at (4, 0) {};
	\end{pgfonlayer}
	\begin{pgfonlayer}{edgelayer}
		\draw (6.center) to (4);
		\draw (4) to (3);
		\draw (0) to (3);
		\draw (8.center) to (0);
		\draw (9.center) to (1);
		\draw (1) to (2);
		\draw (2) to (5);
		\draw (5) to (7.center);
		\draw (3) to (2);
		\draw (4) to (5);
		\draw (0) to (1);
	\end{pgfonlayer}
\end{tikzpicture}
=
\begin{tikzpicture}[xscale=-1]
	\begin{pgfonlayer}{nodelayer}
		\node [style=Z] (0) at (3, 1.25) {};
		\node [style=X] (1) at (4, 0.75) {};
		\node [style=Z] (2) at (4, 2.5) {};
		\node [style=X] (3) at (3, 2) {};
		\node [style=Z] (4) at (3, 3.75) {};
		\node [style=X] (5) at (4, 3.25) {};
		\node [style=none] (6) at (3, 4.5) {};
		\node [style=none] (7) at (4, 4.5) {};
		\node [style=none] (8) at (3, 0) {};
		\node [style=none] (9) at (4, 0) {};
		\node [style=s] (10) at (3.5, 3.5) {};
		\node [style=s] (11) at (3.5, 1) {};
	\end{pgfonlayer}
	\begin{pgfonlayer}{edgelayer}
		\draw (6.center) to (4);
		\draw (4) to (3);
		\draw (0) to (3);
		\draw (8.center) to (0);
		\draw (9.center) to (1);
		\draw (1) to (2);
		\draw (2) to (5);
		\draw (5) to (7.center);
		\draw (3) to (2);
		\draw (1) to (11);
		\draw (11) to (0);
		\draw (5) to (10);
		\draw (10) to (4);
	\end{pgfonlayer}
\end{tikzpicture}
=
\begin{tikzpicture}[xscale=-1]
	\begin{pgfonlayer}{nodelayer}
		\node [style=X] (28) at (6.5, 0) {};
		\node [style=Z] (31) at (5, 4) {};
		\node [style=none] (33) at (5, 4.5) {};
		\node [style=none] (34) at (6.5, 4.5) {};
		\node [style=none] (35) at (5, -0.5) {};
		\node [style=none] (36) at (6.5, -0.5) {};
		\node [style=s] (37) at (6, 0.5) {};
		\node [style=s] (38) at (5.5, 3.5) {};
		\node [style=X] (39) at (6, 2.25) {};
		\node [style=Z] (40) at (6, 3) {};
		\node [style=X] (41) at (6.5, 2.25) {};
		\node [style=Z] (42) at (6.5, 3) {};
		\node [style=X] (43) at (5, 1) {};
		\node [style=Z] (44) at (5, 1.75) {};
		\node [style=X] (45) at (5.5, 1) {};
		\node [style=Z] (46) at (5.5, 1.75) {};
	\end{pgfonlayer}
	\begin{pgfonlayer}{edgelayer}
		\draw (33.center) to (31);
		\draw (36.center) to (28);
		\draw (28) to (37);
		\draw (31) to (38);
		\draw (40) to (41);
		\draw (41) to (42);
		\draw (40) to (39);
		\draw (39) to (42);
		\draw (44) to (45);
		\draw (45) to (46);
		\draw (44) to (43);
		\draw (43) to (46);
		\draw (34.center) to (42);
		\draw (40) to (38);
		\draw (46) to (39);
		\draw (41) to (28);
		\draw (37) to (45);
		\draw (35.center) to (43);
		\draw (31) to (44);
	\end{pgfonlayer}
\end{tikzpicture}
=
\begin{tikzpicture}[xscale=-1]
	\begin{pgfonlayer}{nodelayer}
		\node [style=X] (67) at (11.5, 0) {};
		\node [style=Z] (68) at (10, 4.25) {};
		\node [style=none] (69) at (10, 4.75) {};
		\node [style=none] (70) at (11.5, 4.75) {};
		\node [style=none] (71) at (10, -0.5) {};
		\node [style=none] (72) at (11.5, -0.5) {};
		\node [style=s] (73) at (11, 0.5) {};
		\node [style=s] (74) at (10.5, 3.75) {};
		\node [style=Z] (75) at (11, 3.25) {};
		\node [style=X] (76) at (11.5, 2.5) {};
		\node [style=Z] (77) at (11.5, 3.25) {};
		\node [style=X] (78) at (10, 1) {};
		\node [style=Z] (79) at (10, 1.75) {};
		\node [style=X] (80) at (10.5, 1) {};
		\node [style=X] (81) at (10.5, 1.75) {};
		\node [style=Z] (82) at (10.5, 2.5) {};
		\node [style=X] (83) at (11, 1.75) {};
		\node [style=Z] (84) at (11, 2.5) {};
	\end{pgfonlayer}
	\begin{pgfonlayer}{edgelayer}
		\draw (69.center) to (68);
		\draw (72.center) to (67);
		\draw (67) to (73);
		\draw (68) to (74);
		\draw (75) to (76);
		\draw (76) to (77);
		\draw (79) to (80);
		\draw (79) to (78);
		\draw (70.center) to (77);
		\draw (75) to (74);
		\draw (76) to (67);
		\draw (73) to (80);
		\draw (71.center) to (78);
		\draw (68) to (79);
		\draw (82) to (83);
		\draw (83) to (84);
		\draw (82) to (81);
		\draw (81) to (84);
		\draw (78) to (81);
		\draw (80) to (83);
		\draw (84) to (77);
		\draw (75) to (82);
	\end{pgfonlayer}
\end{tikzpicture}
=
\begin{tikzpicture}[xscale=-1]
	\begin{pgfonlayer}{nodelayer}
		\node [style=X] (0) at (11, 0.75) {};
		\node [style=Z] (1) at (10.25, 4.75) {};
		\node [style=none] (2) at (10.25, 5.25) {};
		\node [style=none] (3) at (11, 5.25) {};
		\node [style=none] (4) at (10.25, 0.25) {};
		\node [style=none] (5) at (11, 0.25) {};
		\node [style=s] (6) at (11, 1.5) {};
		\node [style=s] (7) at (10.25, 4) {};
		\node [style=Z] (8) at (10.25, 3.25) {};
		\node [style=X] (9) at (11, 0.75) {};
		\node [style=Z] (10) at (11, 3.25) {};
		\node [style=X] (11) at (10.25, 2.25) {};
		\node [style=Z] (12) at (10.25, 4.75) {};
		\node [style=X] (13) at (11, 2.25) {};
		\node [style=X] (14) at (10.25, 2.25) {};
		\node [style=Z] (15) at (10.25, 3.25) {};
		\node [style=X] (16) at (11, 2.25) {};
		\node [style=Z] (17) at (11, 3.25) {};
	\end{pgfonlayer}
	\begin{pgfonlayer}{edgelayer}
		\draw (2.center) to (1);
		\draw (5.center) to (0);
		\draw (0) to (6);
		\draw (1) to (7);
		\draw [in=135, out=-60] (8) to (9);
		\draw [bend right, looseness=1.25] (9) to (10);
		\draw [in=120, out=-45] (12) to (13);
		\draw [bend right, looseness=1.25] (12) to (11);
		\draw (3.center) to (10);
		\draw (8) to (7);
		\draw (6) to (13);
		\draw (4.center) to (11);
		\draw [in=150, out=-30, looseness=0.75] (15) to (16);
		\draw (16) to (17);
		\draw (15) to (14);
		\draw (14) to (17);
	\end{pgfonlayer}
\end{tikzpicture}
=
\begin{tikzpicture}[xscale=-1]
	\begin{pgfonlayer}{nodelayer}
		\node [style=X] (0) at (19.5, 1.25) {};
		\node [style=Z] (1) at (18, 4.25) {};
		\node [style=none] (2) at (18, 4.75) {};
		\node [style=none] (3) at (19.5, 4.75) {};
		\node [style=none] (4) at (18, 0.75) {};
		\node [style=none] (5) at (19.5, 0.75) {};
		\node [style=X] (6) at (19.5, 1.25) {};
		\node [style=Z] (7) at (19.5, 4.25) {};
		\node [style=X] (8) at (18, 1.25) {};
		\node [style=Z] (9) at (18, 4.25) {};
		\node [style=X] (10) at (19.5, 1.25) {};
		\node [style=X] (11) at (18, 1.25) {};
		\node [style=X] (12) at (19.5, 1.25) {};
		\node [style=Z] (13) at (19.5, 4.25) {};
		\node [style=s] (14) at (17.75, 3) {};
		\node [style=s] (15) at (18.75, 3) {};
		\node [style=s] (16) at (19.75, 3) {};
		\node [style=s] (17) at (18.25, 3) {};
	\end{pgfonlayer}
	\begin{pgfonlayer}{edgelayer}
		\draw (2.center) to (1);
		\draw (5.center) to (0);
		\draw [bend right=45] (6) to (7);
		\draw [in=120, out=-135, looseness=1.25] (9) to (8);
		\draw (3.center) to (7);
		\draw (4.center) to (8);
		\draw [in=-120, out=15, looseness=0.75] (11) to (13);
		\draw [in=90, out=-105] (9) to (14);
		\draw [in=90, out=-45, looseness=0.75] (9) to (15);
		\draw [in=90, out=-90] (14) to (11);
		\draw [in=-90, out=120, looseness=0.75] (6) to (15);
		\draw [in=-15, out=90] (12) to (9);
		\draw [in=-90, out=150, looseness=0.75] (12) to (17);
		\draw [in=285, out=90] (17) to (9);
		\draw [in=-90, out=75, looseness=0.75] (12) to (16);
		\draw [in=-75, out=90] (16) to (13);
	\end{pgfonlayer}
\end{tikzpicture}
=
\begin{tikzpicture}
	\begin{pgfonlayer}{nodelayer}
		\node [style=none] (0) at (24, 4.5) {};
		\node [style=none] (1) at (23.5, 4) {};
		\node [style=none] (2) at (24, 2.75) {};
		\node [style=none] (3) at (23.5, 2.75) {};
		\node [style=s] (4) at (24, 4) {};
		\node [style=none] (5) at (23.5, 4.5) {};
	\end{pgfonlayer}
	\begin{pgfonlayer}{edgelayer}
		\draw [in=-90, out=90, looseness=1.25] (2.center) to (1.center);
		\draw (1.center) to (5.center);
		\draw (0.center) to (4);
		\draw [in=90, out=-90, looseness=1.25] (4) to (3.center);
	\end{pgfonlayer}
\end{tikzpicture}
=
F
$$
\end{proof}

In the ZX-calculus literature, this decomposition of the Fourier transform is known as {\it Euler decomposition} \cite{duncan2009graph}.
A variant of this decomposition is given in \cite[p.6]{control}; although in the context of plain old linear relations instead of Lagrangian relations, so an antipode is missing in their case.  A similar observation was made in \cite[(34)]{ranchin2014depicting} in terms of qudit controlled boost gates; however, the connection to phase-shift gates and Euler decomposition was not made.

From Theorem~\ref{theorem:generators}, we know that we can build any Lagrangian relation using pure Lagrangian relations and discard maps. Since the former is closed under composition and monoidal product, the following can be shown immediately from string diagram deformation.

\begin{corollary}[Phase purification]\label{cor:pure}
  Any linear Lagrangian relation can be written in the following form, for $V$ a linear relation:
$$
\begin{tikzpicture}
	\begin{pgfonlayer}{nodelayer}
		\node [style=X] (0) at (-0.25, 1.75) {};
		\node [style=scalar] (1) at (1, 0.75) {$a_1$};
		\node [style=none] (2) at (1, 0) {};
		\node [style=none] (3) at (-2.25, 0.5) {};
		\node [style=none] (4) at (-2.25, 0) {};
		\node [style=map, minimum width=2cm, minimum height=1cm] (5) at (-1.5, -0.5) {$V^\perp$};
		\node [style=map, minimum width=2cm, minimum height=1cm] (6) at (1.75, -0.5) {$V$};
		\node [style=X] (7) at (1, 1.75) {};
		\node [style=scalar] (8) at (2, 0.75) {$a_k$};
		\node [style=none] (9) at (2, 0) {};
		\node [style=none] (10) at (-1.25, 0.5) {};
		\node [style=none] (11) at (-1.25, 0) {};
		\node [style=none] (12) at (1.5, 0.25) {...};
		\node [style=none] (13) at (-0.75, 0) {};
		\node [style=none] (14) at (-0.75, 2.25) {};
		\node [style=none] (17) at (2.5, 0) {};
		\node [style=none] (18) at (2.5, 2.25) {};
		\node [style=none] (19) at (-1.75, 0.25) {...};
		\node [style=none] (20) at (-1.5, -1.5) {};
		\node [style=none] (21) at (-1.5, -0.75) {};
		\node [style=none] (22) at (1.75, -1.5) {};
		\node [style=none] (23) at (1.75, -0.75) {};
	\end{pgfonlayer}
	\begin{pgfonlayer}{edgelayer}
		\draw [in=-30, out=90, looseness=0.75] (1) to (0);
		\draw [in=90, out=-165, looseness=0.50] (0) to (3.center);
		\draw (4.center) to (3.center);
		\draw (2.center) to (1);
		\draw [in=-30, out=90, looseness=0.75] (8) to (7);
		\draw [in=90, out=-165, looseness=0.50] (7) to (10.center);
		\draw (11.center) to (10.center);
		\draw (9.center) to (8);
		\draw [in=270, out=90] (13.center) to (14.center);
		\draw [in=270, out=90] (17.center) to (18.center);
		\draw [in=270, out=90] (20.center) to (21.center);
		\draw [in=270, out=90] (22.center) to (23.center);
	\end{pgfonlayer}
\end{tikzpicture}
$$
\end{corollary}

In the case when we are working with prime fields, then Lagrangian relations are exactly an instance of the CPM construction. Namely, in the category of linear relations, $(-)^*$ is given by relational converse, so we can define a dagger functor $(-)^\dagger := ((-)^\perp)^*$ such that $(-)_* = (-)^\perp$. It only remains to show that all of the discarding maps arise from a single fixed cap. This can be done as follows, for $k = \F_p$:
$$
\begin{tikzpicture}
	\begin{pgfonlayer}{nodelayer}
		\node [style=X] (3) at (3, 2) {};
		\node [style=scalar] (7) at (3.5, 1.5) {$n$};
		\node [style=none] (8) at (2.5, 1.5) {};
		\node [style=none] (9) at (2.5, 1) {};
		\node [style=none] (10) at (3.5, 1) {};
	\end{pgfonlayer}
	\begin{pgfonlayer}{edgelayer}
		\draw [in=-15, out=90] (7) to (3);
		\draw [in=-165, out=90] (8.center) to (3);
		\draw (9.center) to (8.center);
		\draw (10.center) to (7);
	\end{pgfonlayer}
\end{tikzpicture}
=
\begin{tikzpicture}
	\begin{pgfonlayer}{nodelayer}
		\node [style=X] (0) at (5.25, 1.75) {};
		\node [style=Z] (1) at (6.25, 0.5) {};
		\node [style=none] (2) at (4.25, 0) {};
		\node [style=none] (3) at (6.25, 0) {};
		\node [style=none] (4) at (5.75, 1.25) {$\iddots$};
		\node [style=none] (5) at (5.9, 1) {$n$};
		\node [style=X] (6) at (4.75, 2.25) {};
	\end{pgfonlayer}
	\begin{pgfonlayer}{edgelayer}
		\draw [in=0, out=90, looseness=1.25] (1) to (0);
		\draw [in=-90, out=165, looseness=1.25] (1) to (0);
		\draw [in=-120, out=90] (2.center) to (6);
		\draw [in=105, out=-15] (6) to (0);
		\draw (3.center) to (1);
	\end{pgfonlayer}
\end{tikzpicture}
=
\begin{tikzpicture}
	\begin{pgfonlayer}{nodelayer}
		\node [style=none] (21) at (6.25, 2) {};
		\node [style=none] (22) at (5.25, 2.25) {};
		\node [style=X] (23) at (5.25, 2.25) {};
		\node [style=Z] (24) at (6.25, 2) {};
		\node [style=none] (25) at (5, 3) {};
		\node [style=none] (26) at (6.5, 2.75) {};
		\node [style=Z] (28) at (6.5, 2.75) {};
		\node [style=none] (30) at (6.25, 0.75) {};
		\node [style=none] (31) at (5.25, 1) {};
		\node [style=X] (32) at (5.25, 1) {};
		\node [style=Z] (33) at (6.25, 0.75) {};
		\node [style=none] (34) at (5.25, 0.25) {};
		\node [style=none] (35) at (6.25, 0.25) {};
		\node [style=none] (36) at (5.7, 1.6) {$\vdots$};
		\node [style=none] (37) at (5.95, 1.55) {$n$};
		\node [style=X] (38) at (5, 3) {};
	\end{pgfonlayer}
	\begin{pgfonlayer}{edgelayer}
		\draw [in=-90, out=135, looseness=0.75] (23) to (25.center);
		\draw [in=-90, out=60] (24) to (26.center);
		\draw (34.center) to (32);
		\draw (32) to (23);
		\draw (35.center) to (33);
		\draw (33) to (24);
		\draw (24) to (23);
		\draw (33) to (32);
	\end{pgfonlayer}
\end{tikzpicture}
=
\begin{tikzpicture}
	\begin{pgfonlayer}{nodelayer}
		\node [style=X] (622) at (274.75, 2.25) {};
		\node [style=none] (623) at (275.5, 1.25) {};
		\node [style=none] (624) at (274.25, 1.25) {};
		\node [style=X] (625) at (274.25, 1.25) {};
		\node [style=Z] (626) at (275.5, 1.25) {};
		\node [style=none] (627) at (274.25, 2.5) {};
		\node [style=none] (628) at (275.5, 2.5) {};
		\node [style=X] (629) at (274.25, 2.5) {};
		\node [style=Z] (630) at (275.5, 2.5) {};
		\node [style=X] (631) at (274.75, 0.75) {};
		\node [style=none] (632) at (275.5, -0.25) {};
		\node [style=none] (633) at (274.25, -0.25) {};
		\node [style=X] (634) at (274.25, -0.25) {};
		\node [style=Z] (635) at (275.5, -0.25) {};
		\node [style=none] (636) at (274.25, -1) {};
		\node [style=none] (637) at (275.5, -1) {};
		\node [style=none] (638) at (274.8, 1.59) {$\vdots$};
		\node [style=none] (639) at (275.05, 1.5) {$n$};
	\end{pgfonlayer}
	\begin{pgfonlayer}{edgelayer}
		\draw [in=-15, out=120] (623.center) to (622);
		\draw [in=-165, out=165, looseness=1.50] (624.center) to (622);
		\draw (625) to (627.center);
		\draw (626) to (628.center);
		\draw [in=-30, out=120] (632.center) to (631);
		\draw [in=-165, out=150, looseness=1.50] (633.center) to (631);
		\draw (636.center) to (634);
		\draw (634) to (625);
		\draw (637.center) to (635);
		\draw (635) to (626);
	\end{pgfonlayer}
\end{tikzpicture}
$$

%
%\begin{definition} \cite{cpm}:
%Given a strongly compact closed category $\X$ equipped with a dagger functor $(\_)^\dag:\X^\op\to\X$, there is a strongly compact closed category, $\CPM(\X,(\_)^\dag)$ with:
%\begin{description}
%\item[Objects:] Same as in $\X$
%\item[Maps:]
%\item[Identity:]
%\item[Composition:]
%\item[Tensor:]
%
%
%\end{description}
%\end{definition}


\begin{corollary}
\label{cor}
For $p$ prime, $\Lag\Rel_{\F_p} \cong \CPM[\LinRel_{\F_p}]$.
\end{corollary}

\section{Affine Lagrangian relations}
\label{sec:aff}

Affine Lagrangian relations are perhaps of more practical interest than plain old Lagrangian relations.  As we will discuss in this section, these give a semantics for qudit stabilizer circuits as well as certain electrical circuits.  We use our universal set of generators for Lagrangian relations as well as the presentation for affine relations to get a universal set of generators for affine Lagrangian relations.


\begin{definition}
Let $\Aff\Lag\Rel_k$ denote the monoidal category whose objects are symplectic vector spaces, and whose morphisms are generated by the image of $\Lag\Rel_k \xrightarrow{E} \LinRel_k \to \Aff\Rel_k$ as well as all affine shifts and whose tensor product is the direct sum.
\end{definition}
Because the tensor product is defined in the same way as in $\Lag\Rel_k$, as in Lemma \ref{lemma:strong}, the forgetful functor  $\Aff\Lag\Rel_k\to \Aff\Rel_k$ is faithful, but only {\em strong} monoidal.



\begin{definition}
Let $\alr_k$ denote the monoidal subcategory of $\aih_k$ with objects $2n$, generated by the morphisms in the image of $\Lag\Rel_k\xrightarrow{E} \LinRel_k \cong \ih_k \to \aih_k$ as well as the following generator:
$$
\begin{tikzpicture}
	\begin{pgfonlayer}{nodelayer}
		\node [style=X] (0) at (0, 0) {$1$};
		\node [style=Z] (1) at (0.5, 0) {};
		\node [style=none] (2) at (0, 0.75) {};
		\node [style=none] (3) at (0.5, 0.75) {};
	\end{pgfonlayer}
	\begin{pgfonlayer}{edgelayer}
		\draw (1) to (3.center);
		\draw (2.center) to (0);
	\end{pgfonlayer}
\end{tikzpicture}
$$
\end{definition}

\begin{lemma}
$\alr_k$ is a presentation of $\Aff\Lag\Rel_k$.
\end{lemma}
\begin{proof}
All the affine shifts can be produced from tensoring and composing these two maps on the right:
$$
\begin{tikzpicture}
	\begin{pgfonlayer}{nodelayer}
		\node [style=X] (0) at (0, 0) {$1$};
		\node [style=Z] (1) at (0.5, 0) {};
		\node [style=none] (2) at (0, 0.75) {};
		\node [style=none] (3) at (0.5, 0.75) {};
		\node [style=none] (4) at (0.5, 1.5) {};
		\node [style=none] (5) at (0, 1.5) {};
		\node [style=s] (6) at (0.5, 0.75) {};
	\end{pgfonlayer}
	\begin{pgfonlayer}{edgelayer}
		\draw (1) to (3.center);
		\draw (2.center) to (0);
		\draw [in=270, out=90] (3.center) to (5.center);
		\draw [in=270, out=90] (2.center) to (4.center);
	\end{pgfonlayer}
\end{tikzpicture}
=
\begin{tikzpicture}
	\begin{pgfonlayer}{nodelayer}
		\node [style=X] (0) at (0.5, 0) {$1$};
		\node [style=Z] (1) at (0, 0) {};
		\node [style=none] (2) at (0.5, 0.75) {};
		\node [style=none] (3) at (0, 0.75) {};
	\end{pgfonlayer}
	\begin{pgfonlayer}{edgelayer}
		\draw (1) to (3.center);
		\draw (2.center) to (0);
	\end{pgfonlayer}
\end{tikzpicture} \hspace*{.1cm} \in  \alr_k
\hspace*{.5cm}
\implies
\hspace*{.5cm}
\begin{tikzpicture}
	\begin{pgfonlayer}{nodelayer}
		\node [style=Z] (447) at (224.25, 0.75) {};
		\node [style=X] (448) at (222.75, 0.75) {};
		\node [style=none] (449) at (223.75, -0.25) {};
		\node [style=none] (450) at (222.25, -0.25) {};
		\node [style=none] (451) at (224.25, 1.5) {};
		\node [style=none] (452) at (222.75, 1.5) {};
		\node [style=X] (453) at (223.25, -1) {$1$};
		\node [style=Z] (454) at (224.75, -1) {};
		\node [style=scalar] (455) at (223.25, -0.25) {$a$};
		\node [style=scalarop] (456) at (224.75, -0.25) {$a$};
		\node [style=none] (457) at (223.75, -1.5) {};
		\node [style=none] (458) at (222.25, -1.5) {};
	\end{pgfonlayer}
	\begin{pgfonlayer}{edgelayer}
		\draw [in=90, out=-150] (447) to (449.center);
		\draw [in=-150, out=90] (450.center) to (448);
		\draw (447) to (451.center);
		\draw (448) to (452.center);
		\draw (457.center) to (449.center);
		\draw (458.center) to (450.center);
		\draw (453) to (455);
		\draw (454) to (456);
		\draw [in=-30, out=90] (456) to (447);
		\draw [in=-30, out=90] (455) to (448);
	\end{pgfonlayer}
\end{tikzpicture}
=
\begin{tikzpicture}
	\begin{pgfonlayer}{nodelayer}
		\node [style=X] (17) at (5, 0) {$a$};
		\node [style=none] (18) at (5.5, -1.5) {};
		\node [style=none] (19) at (5, -1.5) {};
		\node [style=none] (20) at (5.5, 1.5) {};
		\node [style=none] (21) at (5, 1.5) {};
	\end{pgfonlayer}
	\begin{pgfonlayer}{edgelayer}
		\draw (19.center) to (17);
		\draw (17) to (21.center);
		\draw (18.center) to (20.center);
	\end{pgfonlayer}
\end{tikzpicture},
\hspace*{.5cm}
\begin{tikzpicture}
	\begin{pgfonlayer}{nodelayer}
		\node [style=Z] (459) at (226.25, 0.75) {};
		\node [style=X] (460) at (227.75, 0.75) {};
		\node [style=none] (461) at (226.75, -0.25) {};
		\node [style=none] (462) at (228.25, -0.25) {};
		\node [style=none] (463) at (226.25, 1.5) {};
		\node [style=none] (464) at (227.75, 1.5) {};
		\node [style=X] (465) at (227.25, -1) {$1$};
		\node [style=Z] (466) at (225.75, -1) {};
		\node [style=scalar] (467) at (227.25, -0.25) {$a$};
		\node [style=scalarop] (468) at (225.75, -0.25) {$a$};
		\node [style=none] (469) at (226.75, -1.5) {};
		\node [style=none] (470) at (228.25, -1.5) {};
	\end{pgfonlayer}
	\begin{pgfonlayer}{edgelayer}
		\draw [in=90, out=-30] (459) to (461.center);
		\draw [in=-30, out=90] (462.center) to (460);
		\draw (459) to (463.center);
		\draw (460) to (464.center);
		\draw (469.center) to (461.center);
		\draw (470.center) to (462.center);
		\draw (465) to (467);
		\draw (466) to (468);
		\draw [in=-150, out=90] (468) to (459);
		\draw [in=-150, out=90] (467) to (460);
	\end{pgfonlayer}
\end{tikzpicture}
=
\begin{tikzpicture}
	\begin{pgfonlayer}{nodelayer}
		\node [style=X] (0) at (1.5, 0) {$a$};
		\node [style=none] (1) at (1, -1.5) {};
		\node [style=none] (2) at (1.5, -1.5) {};
		\node [style=none] (3) at (1, 1.5) {};
		\node [style=none] (4) at (1.5, 1.5) {};
	\end{pgfonlayer}
	\begin{pgfonlayer}{edgelayer}
		\draw (2.center) to (0);
		\draw (0) to (4.center);
		\draw (1.center) to (3.center);
	\end{pgfonlayer}
\end{tikzpicture}
\hspace*{.1cm}\in\alr_k
$$
\end{proof}
Therefore, we are justified in using string diagrams in $\alr_k$ to reason about morphisms in $\Aff\Lag\Rel_k$.

We will restate the interpretations given in \cite{affine} of some components for electrical circuits in terms  affine relations  in terms of the generators for graphical calculus for Lagrangian relations.  This interpretation is also explored in \cite{passive,network}; albeit, not enjoying the graphical calculus for affine relations.
\begin{example}
\label{ex:circuits}
For any non-negative real $a$, wires, $a$-weighted resistors, inductors and capacitors have the following interpretations in $\Aff\Lag\Rel_{\mathbb{R}(x)}$:
$$
\left\llbracket
\begin{tikzpicture}
	\begin{pgfonlayer}{nodelayer}
		\node [style=none] (0) at (21, 4.25) {};
		\node [style=none] (1) at (22, 4.25) {};
		\node [style=none] (2) at (21, 2.75) {};
		\node [style=none] (3) at (22, 2.75) {};
		\node [style=dot] (4) at (21.5, 3.5) {};
		\node [style=none] (5) at (21.5, 4) {$\cdots$};
		\node [style=none] (6) at (21.5, 3) {$\cdots$};
	\end{pgfonlayer}
	\begin{pgfonlayer}{edgelayer}
		\draw [in=150, out=-90] (0.center) to (4);
		\draw [in=90, out=-150] (4) to (2.center);
		\draw [in=-30, out=90] (3.center) to (4);
		\draw [in=-90, out=30] (4) to (1.center);
	\end{pgfonlayer}
\end{tikzpicture}
\right\rrbracket
=
\begin{tikzpicture}
	\begin{pgfonlayer}{nodelayer}
		\node [style=none] (501) at (237.5, 4.25) {};
		\node [style=none] (502) at (238.5, 4.25) {};
		\node [style=none] (503) at (237.5, 2.75) {};
		\node [style=none] (504) at (238.5, 2.75) {};
		\node [style=X] (505) at (238, 3.5) {};
		\node [style=none] (506) at (238, 4) {$\cdots$};
		\node [style=none] (507) at (238, 3) {$\cdots$};
		\node [style=none] (508) at (236.25, 4.25) {};
		\node [style=none] (509) at (237.25, 4.25) {};
		\node [style=none] (510) at (236.25, 2.75) {};
		\node [style=none] (511) at (237.25, 2.75) {};
		\node [style=Z] (512) at (236.75, 3.5) {};
		\node [style=none] (513) at (236.75, 4) {$\cdots$};
		\node [style=none] (514) at (236.75, 3) {$\cdots$};
	\end{pgfonlayer}
	\begin{pgfonlayer}{edgelayer}
		\draw [in=150, out=-90] (501.center) to (505);
		\draw [in=90, out=-150] (505) to (503.center);
		\draw [in=-30, out=90] (504.center) to (505);
		\draw [in=-90, out=30] (505) to (502.center);
		\draw [in=150, out=-90] (508.center) to (512);
		\draw [in=90, out=-150] (512) to (510.center);
		\draw [in=-30, out=90] (511.center) to (512);
		\draw [in=-90, out=30] (512) to (509.center);
	\end{pgfonlayer}
\end{tikzpicture}
\hspace*{.5cm}
\left\llbracket
\tikz \draw (0,0) to[R=$a$] (0,2);
\hspace*{,3cm}
\right\rrbracket
=
\begin{tikzpicture}
	\begin{pgfonlayer}{nodelayer}
		\node [style=Z] (0) at (23, 3.75) {};
		\node [style=X] (1) at (22, 5.25) {};
		\node [style=scalar] (2) at (22.5, 4.5) {$a$};
		\node [style=none] (3) at (22, 5.75) {};
		\node [style=none] (4) at (23, 5.75) {};
		\node [style=none] (5) at (23, 3.25) {};
		\node [style=none] (6) at (22, 3.25) {};
	\end{pgfonlayer}
	\begin{pgfonlayer}{edgelayer}
		\draw [in=-90, out=135] (0) to (2);
		\draw [in=315, out=90] (2) to (1);
		\draw (1) to (3.center);
		\draw (1) to (6.center);
		\draw (5.center) to (0);
		\draw (4.center) to (0);
	\end{pgfonlayer}
\end{tikzpicture}
\hspace*{.5cm}
\left\llbracket
\tikz \draw (0,0) to[L=$a$] (0,2);
\hspace*{,3cm}
\right\rrbracket
=
\begin{tikzpicture}
	\begin{pgfonlayer}{nodelayer}
		\node [style=Z] (0) at (23, 3.75) {};
		\node [style=X] (1) at (22, 5.25) {};
		\node [style=scalar] (2) at (22.5, 4.5) {$ax$};
		\node [style=none] (3) at (22, 5.75) {};
		\node [style=none] (4) at (23, 5.75) {};
		\node [style=none] (5) at (23, 3.25) {};
		\node [style=none] (6) at (22, 3.25) {};
	\end{pgfonlayer}
	\begin{pgfonlayer}{edgelayer}
		\draw [in=-90, out=135] (0) to (2);
		\draw [in=315, out=90] (2) to (1);
		\draw (1) to (3.center);
		\draw (1) to (6.center);
		\draw (5.center) to (0);
		\draw (4.center) to (0);
	\end{pgfonlayer}
\end{tikzpicture}
\hspace*{.5cm}
\left\llbracket
\tikz \draw (0,0) to[C=$a$] (0,2);
\hspace*{,3cm}
\right\rrbracket
=
\begin{tikzpicture}
	\begin{pgfonlayer}{nodelayer}
		\node [style=Z] (0) at (23.25, 3.75) {};
		\node [style=X] (1) at (21.75, 5.25) {};
		\node [style=scalar] (2) at (22.5, 4.5) {$-ax$};
		\node [style=none] (3) at (21.75, 5.75) {};
		\node [style=none] (4) at (23.25, 5.75) {};
		\node [style=none] (5) at (23.25, 3.25) {};
		\node [style=none] (6) at (21.75, 3.25) {};
	\end{pgfonlayer}
	\begin{pgfonlayer}{edgelayer}
		\draw [in=-90, out=135] (0) to (2);
		\draw [in=315, out=90] (2) to (1);
		\draw (1) to (3.center);
		\draw (1) to (6.center);
		\draw (5.center) to (0);
		\draw (4.center) to (0);
	\end{pgfonlayer}
\end{tikzpicture}
$$
Similarly for $a$-valued voltage and current sources (again, for $a$ a non-negative real number):
$$
\left\llbracket
\begin{tikzpicture}
	\begin{pgfonlayer}{nodelayer}
		\node [style=none] (0) at (0, 2) {};
		\node [style=isourceAMshape,rotate=90] (1) at (0, 1) {};
		\node [style=none] (2) at (0, 0) {};
	\end{pgfonlayer}
	\begin{pgfonlayer}{edgelayer}
		\draw (2.center) to (1);
		\draw (1) to (0.center);
		\node [style=none] (3) at (-.7, 1) {$a$};
	\end{pgfonlayer}
\end{tikzpicture}
\hspace*{,3cm}
\right\rrbracket
=
\begin{tikzpicture}
	\begin{pgfonlayer}{nodelayer}
		\node [style=X] (1) at (22, 5.25) {};
		\node [style=scalar] (2) at (22.5, 4.5) {$ax$};
		\node [style=none] (3) at (22, 5.75) {};
		\node [style=none] (4) at (23, 5.75) {};
		\node [style=none] (5) at (23, 3.25) {};
		\node [style=none] (6) at (22, 3.25) {};
		\node [style=X] (7) at (22.5, 3.75) {$1$};
	\end{pgfonlayer}
	\begin{pgfonlayer}{edgelayer}
		\draw [in=315, out=90] (2) to (1);
		\draw (1) to (3.center);
		\draw (1) to (6.center);
		\draw (7) to (2);
		\draw (5.center) to (4.center);
	\end{pgfonlayer}
\end{tikzpicture}
=
\begin{tikzpicture}
	\begin{pgfonlayer}{nodelayer}
		\node [style=X] (32) at (129.25, -0.25) {$ax$};
		\node [style=none] (33) at (129.25, 1) {};
		\node [style=none] (34) at (129.75, 1) {};
		\node [style=none] (35) at (129.75, -1.5) {};
		\node [style=none] (36) at (129.25, -1.5) {};
	\end{pgfonlayer}
	\begin{pgfonlayer}{edgelayer}
		\draw (32) to (33.center);
		\draw (32) to (36.center);
		\draw (35.center) to (34.center);
	\end{pgfonlayer}
\end{tikzpicture}
\hspace{,3cm}
\left\llbracket
\begin{tikzpicture}
	\begin{pgfonlayer}{nodelayer}
		\node [style=none] (0) at (0, 2) {};
		\node [style=vsourceAMshape,rotate=-90] (1) at (0, 1) {};
		\node [style=none] (2) at (0, 0) {};
		\node [style=none] (3) at (-.7, 1) {$a$};
	\end{pgfonlayer}
	\begin{pgfonlayer}{edgelayer}
		\draw (2.center) to (1);
		\draw (1) to (0.center);
	\end{pgfonlayer}
\end{tikzpicture}
\hspace*{,3cm}
\right\rrbracket
=
\begin{tikzpicture}
	\begin{pgfonlayer}{nodelayer}
		\node [style=none] (28) at (9, 1) {};
		\node [style=none] (29) at (10, 1) {};
		\node [style=none] (30) at (10, -1.5) {};
		\node [style=none] (31) at (9, -1.5) {};
		\node [style=Z] (32) at (9, 0.25) {};
		\node [style=Z] (33) at (9, -0.75) {};
		\node [style=X] (34) at (9.5, -1.25) {$1$};
		\node [style=scalar] (35) at (9.5, -0.5) {$a$};
		\node [style=Z] (36) at (10, 0.5) {};
	\end{pgfonlayer}
	\begin{pgfonlayer}{edgelayer}
		\draw (31.center) to (33);
		\draw (32) to (28.center);
		\draw (36) to (29.center);
		\draw [in=90, out=-150] (36) to (35);
		\draw (35) to (34);
		\draw (30.center) to (36);
	\end{pgfonlayer}
\end{tikzpicture}
=
\begin{tikzpicture}
	\begin{pgfonlayer}{nodelayer}
		\node [style=none] (37) at (11, 1) {};
		\node [style=none] (38) at (12, 1) {};
		\node [style=none] (39) at (12, -1.5) {};
		\node [style=none] (40) at (11, -1.5) {};
		\node [style=Z] (41) at (11, 0.25) {};
		\node [style=Z] (42) at (11, -0.75) {};
		\node [style=X] (43) at (11.5, -1.25) {$1$};
		\node [style=Z] (45) at (12, -0.25) {};
		\node [style=scalar] (46) at (12, 0.5) {$a$};
		\node [style=scalarop] (47) at (12, -1) {$a$};
	\end{pgfonlayer}
	\begin{pgfonlayer}{edgelayer}
		\draw (40.center) to (42);
		\draw (41) to (37.center);
		\draw (45) to (46);
		\draw (46) to (38.center);
		\draw [in=-150, out=90] (43) to (45);
		\draw (39.center) to (47);
		\draw (47) to (45);
	\end{pgfonlayer}
\end{tikzpicture}
=
\begin{tikzpicture}
	\begin{pgfonlayer}{nodelayer}
		\node [style=none] (73) at (17, 0.75) {};
		\node [style=none] (74) at (17.75, 0.75) {};
		\node [style=none] (75) at (17.75, -2) {};
		\node [style=none] (76) at (17, -2) {};
		\node [style=Z] (77) at (17, 0.25) {};
		\node [style=Z] (78) at (17, -0.5) {};
		\node [style=X] (79) at (17.75, 0.25) {$a$};
		\node [style=scalarop] (80) at (17.75, -1.5) {$a$};
		\node [style=X] (83) at (17.75, -0.5) {$1$};
		\node [style=s] (84) at (17.75, -1) {};
	\end{pgfonlayer}
	\begin{pgfonlayer}{edgelayer}
		\draw (76.center) to (78);
		\draw (77) to (73.center);
		\draw [in=-90, out=90] (75.center) to (80);
		\draw (79) to (74.center);
		\draw (84) to (83);
		\draw [in=-90, out=90] (80) to (84);
	\end{pgfonlayer}
\end{tikzpicture}
=
\begin{tikzpicture}
	\begin{pgfonlayer}{nodelayer}
		\node [style=none] (84) at (18.75, 1.25) {};
		\node [style=none] (85) at (19.5, 1.25) {};
		\node [style=none] (86) at (19.5, -1.5) {};
		\node [style=none] (87) at (18.75, -1.5) {};
		\node [style=Z] (88) at (18.75, 0.25) {};
		\node [style=Z] (89) at (18.75, -0.5) {};
		\node [style=X] (90) at (19.5, 0.25) {$1$};
		\node [style=X] (91) at (19.5, -0.5) {$1$};
		\node [style=scalarop] (92) at (19.5, -1) {$-a$};
		\node [style=scalar] (93) at (18.75, -1) {$-a$};
		\node [style=scalar] (94) at (19.5, 0.75) {$a$};
		\node [style=scalarop] (95) at (18.75, 0.75) {$a$};
	\end{pgfonlayer}
	\begin{pgfonlayer}{edgelayer}
		\draw (86.center) to (92);
		\draw (92) to (91);
		\draw (89) to (93);
		\draw (87.center) to (93);
		\draw (88) to (95);
		\draw (95) to (84.center);
		\draw (90) to (94);
		\draw (94) to (85.center);
	\end{pgfonlayer}
\end{tikzpicture}
=
\begin{tikzpicture}
	\begin{pgfonlayer}{nodelayer}
		\node [style=none] (602) at (268.25, 1.75) {};
		\node [style=none] (603) at (269, 1.75) {};
		\node [style=none] (604) at (269, -1) {};
		\node [style=none] (605) at (267.25, -1) {};
		\node [style=Z] (606) at (268.25, 0.75) {};
		\node [style=Z] (607) at (267.75, 0.25) {};
		\node [style=X] (608) at (269, 0.75) {$1$};
		\node [style=X] (609) at (269.5, 0.25) {};
		\node [style=scalarop] (610) at (269, -0.5) {$-a$};
		\node [style=scalar] (611) at (267.25, -0.5) {$-a$};
		\node [style=scalar] (612) at (269, 1.25) {$a$};
		\node [style=scalarop] (613) at (268.25, 1.25) {$a$};
		\node [style=Z] (614) at (268.25, -0.5) {};
		\node [style=X] (615) at (270, -0.5) {$1$};
	\end{pgfonlayer}
	\begin{pgfonlayer}{edgelayer}
		\draw (604.center) to (610);
		\draw [in=-150, out=90] (610) to (609);
		\draw [in=90, out=-150] (607) to (611);
		\draw (605.center) to (611);
		\draw (606) to (613);
		\draw (613) to (602.center);
		\draw (608) to (612);
		\draw (612) to (603.center);
		\draw [in=-30, out=90] (615) to (609);
		\draw [in=-30, out=90] (614) to (607);
	\end{pgfonlayer}
\end{tikzpicture}
$$
\end{example}
Note that these generators do not generate the whole category of Lagrangian relations; for instance, the coefficients are required to be non-negative.

\subsection{Stabilizer circuits and Spekkens' toy model}


In this subsection, we show that, when $p$ is an odd prime, the prop of affine Lagrangian relations over $\F_p$  is isomoprhic to $p$-dimensional qudit stabilizer circuits, modulo invertible scalars.  We first consider an intermediary fragment between the Fourier-free, phase free fragments and stabilizer circuits.  


\begin{definition}
The qudit {\bf boost operator} is the following unitary on $d$ in $\Mat(\C)$, 
$
{\cal X} := \sum_{a =0}^{d-1} |  a+1  \rangle\langle a |
$.
\end{definition}

In the qubit case, the boost operator is just the not gate.  Adding the affine shift to $\ih_{\F_p}$, corresponds to adding the boost gate to the  Fourier-free, phase-free ZX-calculus, extending Lemma \ref{lemma:phasefree}.  This is a qudit generalization of the observation made in \cite{distzx}:

\begin{lemma}
For $p$ prime, $\aih_{\F_p}$ is isomorphic as a prop  to the Fourier-free, $p$-dimensional qudit ZX-calculus with the boost operator modulo invertible scalars.
\end{lemma}




We can go further with affine Lagrangian relations.  Inspired by the work of Spekkens \cite{spekkens,spekkens2016quasi}:


\begin{definition}
When $p$ is prime, let {\bf Spekkens' qudit toy model} of dimension $p$  denote the prop $\Aff\Lag\Rel_{\F_p}$.
\end{definition}

We first give a short review of the qudit stabilizer formalism, before establishing the equivalence between Spekkens' toy model and stabilizer circuits in the odd prime qudit case.  All of the material from Definition \ref{definition:begin} to \ref{lemma:end} are contained in \cite{generators}.


\begin{definition}
\label{definition:begin}
The qudit {\bf shift operator} is the following unitary on $d$ in $\Mat(\C)$, 
${\cal Z} := \sum_{a =0}^{d-1} e^{2\pi i a/d} |  a  \rangle\langle a |$.



An $n$-qudit {\bf Weyl operator} an $d^n$-dimensional unitary generated by the shift, boost and identity operators as well as the scalar $e^{\pi i /d}$ under tensor product and matrix multiplication. /

The $n$-qudit {\bf Weyl group},${\frak P}_d^{ n}$, is generated by the $n$-qudit Weyl operators under matrix multiplication.

An $n$-qudit {\bf Clifford operator} $U$ is an $d^n$-dimensional unitary so that $U {\frak P}_d^{ n} U^\dag = {\frak P}_d^{ n}$.

The $n$-qudit {\bf Clifford group} is formed by the $n$-qudit Clifford operators under matrix multiplication.

An $n$-qudit {\bf stabilizer state} is a state $ U |0\rangle^{\otimes n}$ for an $n$-qudit Clifford $U$.

Given any $n$-qudit stabilizer state $|\psi \rangle$,  the {\bf stabilizer group} of $|\psi \rangle$  is the (Abelian) subgroup of ${\frak S}_{|\psi\rangle} \subset {\frak P}_d^{ n}$ whose elements are the $U \in {\frak P}_d^{ n}$ for which $U|\psi \rangle=|\psi \rangle$.
\end{definition}


\begin{lemma}
Two stabilizer states with the same stabilizer groups are the same, up to global phases.
\end{lemma}

\begin{lemma}
\label{lemma:end}
For natural numbers $n,d \geq 2$ the $n$-dimensional qudit stabilizer group modulo invertible scalars is generated under tensor and composition of $I_d$ as well as the boost operator ${\cal X}$, the controlled-boost operator  ${\cal C}$, the Fourier transform ${\cal F}$ and the phase-shift operator ${\cal S}$:
$$
{\cal C}  := \sum_{a,b = 0 }^{d-1} |a,a+b \rangle\langle a, b|
\hspace*{.5cm}
{\cal F}  := \frac{1}{\sqrt d}\sum_{a,b= 0 }^{d-1} e^{2\pi i ab/d} |b \rangle\langle a|
\hspace*{.5cm}
{\cal S} := \sum_{a =0}^{d-1} e^{\pi i a (a+d)/d} |  a  \rangle\langle a |
$$
Notice that the boost operator can be obtained by ${\cal Z }={\cal F}{\cal X} {\cal F}^{2}$.
\end{lemma}

%\bar{e^{\pi i a (a+d)/d}}|  a  \rangle\langle a |
%e^{-\pi i a (a+d)/d}|  a  \rangle\langle a |
%e^{-\pi i a (a+d)/d}|  a  \rangle\langle a |


\begin{definition}
Let $\Stab_p$ denote the subcategory of $\Mat(\C)$ generated by the $p$-dimensional qudit Clifford group as well as the vectors $| 0\rangle$, $\langle 0|$, quotiented by invertible scalars.
\end{definition}


The following isomorphism is described in \cite{gross}, when restricted to the nonempty case.  This comes from the projective representation of the $n$ qudit odd-prime-dimensional Clifford group in terms of the affine symplectomorphisms over $\F_p^n$.  However, since there is only one empty relation and one zero matrix of every type, we get the following result immediately:

\begin{lemma}
For every odd prime $p$, there is an isomorphism $G:\Aff\Lag\Rel_{\F_p}(0, n) \to \Stab_p(0,n)$ determined by:
$$
\begin{tikzpicture}
	\begin{pgfonlayer}{nodelayer}
		\node [style=X] (393) at (206, -2) {$\pi$};
	\end{pgfonlayer}
\end{tikzpicture}
\mapsto 
0
\hspace*{.5cm}
\begin{tikzpicture}
	\begin{pgfonlayer}{nodelayer}
		\node [style=X] (389) at (205, -2) {};
		\node [style=none] (390) at (205, -1.5) {};
		\node [style=Z] (391) at (204.5, -2) {};
		\node [style=none] (392) at (204.5, -1.5) {};
	\end{pgfonlayer}
	\begin{pgfonlayer}{edgelayer}
		\draw (389) to (390.center);
		\draw (391) to (392.center);
	\end{pgfonlayer}
\end{tikzpicture}
 \mapsto |0\rangle
\hspace*{.5cm}
\begin{tikzpicture}
	\begin{pgfonlayer}{nodelayer}
		\node [style=none] (391) at (206, -1) {};
		\node [style=none] (392) at (206, -2) {};
		\node [style=none] (393) at (206.5, -1) {};
		\node [style=none] (394) at (206.5, -2) {};
		\node [style=X] (395) at (206.5, -1.5) {$1$};
	\end{pgfonlayer}
	\begin{pgfonlayer}{edgelayer}
		\draw (392.center) to (391.center);
		\draw (393.center) to (395);
		\draw (395) to (394.center);
	\end{pgfonlayer}
\end{tikzpicture}
 \mapsto {\cal X}
\hspace*{.5cm}
C_1 \mapsto {\cal C}
\hspace*{.5cm}
F \mapsto {\cal F}
\hspace*{.5cm}
S_1 \mapsto {\cal S}
$$
\end{lemma}

We extend this isomorphim of states to an isomorphism of props:

\begin{theorem}
\label{theorem:spekkens}
When $p$ is an odd prime, the mapping $H:\Aff\Lag\Rel_{\F_p} \to \Stab_p$ defined by:
$$
\begin{tikzpicture}
	\begin{pgfonlayer}{nodelayer}
		\node [style=map] (21) at (2, -2) {$f$};
		\node [style=none] (22) at (1.75, -1.25) {};
		\node [style=none] (23) at (2.25, -1.25) {};
		\node [style=none] (24) at (1.75, -2.75) {};
		\node [style=none] (25) at (2.25, -2.75) {};
	\end{pgfonlayer}
	\begin{pgfonlayer}{edgelayer}
		\draw [in=-90, out=120] (21) to (22.center);
		\draw [in=90, out=-120] (21) to (24.center);
		\draw [in=-60, out=90] (25.center) to (21);
		\draw [in=-90, out=60] (21) to (23.center);
	\end{pgfonlayer}
\end{tikzpicture}
\mapsto
\begin{tikzpicture}
	\begin{pgfonlayer}{nodelayer}
		\node [style=map] (7) at (12, 0) {$G\left(\hat f\right)$};
		\node [style=none] (8) at (11.25, 1.5) {};
		\node [style=map] (10) at (13, 1) {$\eta$};
		\node [style=none] (12) at (13.5, -0.5) {};
	\end{pgfonlayer}
	\begin{pgfonlayer}{edgelayer}
		\draw [in=135, out=-90] (8.center) to (7);
		\draw [in=-150, out=60, looseness=0.75] (7) to (10);
		\draw [in=-45, out=90] (12.center) to (10);
	\end{pgfonlayer}
\end{tikzpicture}
$$
is a symmetric monoidal equivalence, where $\eta$ is the cap of the compact closed structure induced by the $Z$ observable.
\end{theorem}

%
%%See Appendix \ref{proof:theorem:spekkens} for the proof.  
%The main difficulty in proving this is to show that $H$ is a functor.  This is shown by observing that stabilizer states with the same stabilizer group only differ by a global scalar.


\begin{proof}
It preserves identities by the snake equations.
Now we must show it preserves composition.
Consider some composable pair in $\Aff\Lag\Rel_{\F_p}$:
$$
\F_p^n \xrightarrow{f} \F_p^m \xrightarrow{g} \F_p^\ell
$$
If the composite is empty, then the result follows immediately.  Suppose otherwise.
We know that:
$$
\begin{tikzpicture}
	\begin{pgfonlayer}{nodelayer}
		\node [style=map] (189) at (103, 0) {$\hat {f;g}$};
		\node [style=none] (190) at (102.25, 1.5) {};
		\node [style=none] (191) at (103.25, 1.5) {};
		\node [style=none] (192) at (102.75, 1.5) {};
		\node [style=none] (193) at (103.75, 1.5) {};
	\end{pgfonlayer}
	\begin{pgfonlayer}{edgelayer}
		\draw [in=135, out=-90] (190.center) to (189);
		\draw [in=-90, out=75] (189) to (191.center);
		\draw [in=-90, out=45] (189) to (193.center);
		\draw [in=105, out=-90] (192.center) to (189);
	\end{pgfonlayer}
\end{tikzpicture}
=
\begin{tikzpicture}
	\begin{pgfonlayer}{nodelayer}
		\node [style=map] (163) at (96.5, -2.25) {$f;g$};
		\node [style=none] (164) at (96.75, -1.5) {};
		\node [style=none] (165) at (95.75, -1.5) {};
		\node [style=Z] (166) at (96.25, -3) {};
		\node [style=X] (167) at (95.75, -3) {};
		\node [style=none] (168) at (95.75, -2.25) {};
		\node [style=none] (169) at (95.25, -2.25) {};
		\node [style=none] (170) at (96.25, -1.5) {};
		\node [style=none] (171) at (95.25, -1.5) {};
	\end{pgfonlayer}
	\begin{pgfonlayer}{edgelayer}
		\draw [in=-90, out=60] (163) to (164.center);
		\draw [in=-90, out=120] (163) to (165.center);
		\draw [in=-120, out=30] (167) to (163);
		\draw [in=30, out=-60, looseness=1.25] (163) to (166);
		\draw [in=-90, out=150] (166) to (168.center);
		\draw [in=-90, out=135] (167) to (169.center);
		\draw (169.center) to (171.center);
		\draw [in=-90, out=90] (168.center) to (170.center);
	\end{pgfonlayer}
\end{tikzpicture}
=
\begin{tikzpicture}
	\begin{pgfonlayer}{nodelayer}
		\node [style=map] (211) at (110.5, -2.25) {$g$};
		\node [style=none] (212) at (110.75, -1.5) {};
		\node [style=none] (213) at (109.75, -1.5) {};
		\node [style=Z] (214) at (110.25, -3) {};
		\node [style=X] (215) at (109.75, -3) {};
		\node [style=none] (216) at (109.75, -2.25) {};
		\node [style=none] (217) at (109.25, -2.25) {};
		\node [style=none] (218) at (110.25, -1.5) {};
		\node [style=none] (219) at (109.25, -1.5) {};
		\node [style=map] (220) at (112.5, -2.25) {$f$};
		\node [style=none] (221) at (112.75, -1.5) {};
		\node [style=none] (222) at (111.75, -1.5) {};
		\node [style=Z] (223) at (112.25, -3) {};
		\node [style=X] (224) at (111.75, -3) {};
		\node [style=none] (225) at (111.75, -2.25) {};
		\node [style=none] (226) at (111.25, -2.25) {};
		\node [style=none] (227) at (112.25, -1.5) {};
		\node [style=none] (228) at (111.25, -1.5) {};
		\node [style=Z] (229) at (111.5, -0.5) {};
		\node [style=X] (230) at (110.5, -0.5) {};
		\node [style=none] (231) at (110.5, 0.25) {};
		\node [style=none] (232) at (111.5, 0.25) {};
		\node [style=none] (233) at (109.25, 0.25) {};
		\node [style=none] (234) at (112.75, 0.25) {};
	\end{pgfonlayer}
	\begin{pgfonlayer}{edgelayer}
		\draw [in=-90, out=60] (211) to (212.center);
		\draw [in=-90, out=120] (211) to (213.center);
		\draw [in=-120, out=30] (215) to (211);
		\draw [in=30, out=-60, looseness=1.25] (211) to (214);
		\draw [in=-90, out=150] (214) to (216.center);
		\draw [in=-90, out=135] (215) to (217.center);
		\draw (217.center) to (219.center);
		\draw [in=-90, out=90] (216.center) to (218.center);
		\draw [in=-90, out=60] (220) to (221.center);
		\draw [in=-90, out=120] (220) to (222.center);
		\draw [in=-120, out=30] (224) to (220);
		\draw [in=30, out=-60, looseness=1.25] (220) to (223);
		\draw [in=-90, out=150] (223) to (225.center);
		\draw [in=-90, out=135] (224) to (226.center);
		\draw (226.center) to (228.center);
		\draw [in=-90, out=90] (225.center) to (227.center);
		\draw [in=-30, out=90] (227.center) to (229);
		\draw [in=90, out=-150] (229) to (212.center);
		\draw [in=-30, out=90] (228.center) to (230);
		\draw [in=90, out=-150] (230) to (213.center);
		\draw (219.center) to (233.center);
		\draw [in=-90, out=90] (218.center) to (232.center);
		\draw [in=-90, out=90, looseness=0.75] (222.center) to (231.center);
		\draw (221.center) to (234.center);
	\end{pgfonlayer}
\end{tikzpicture}
=
\begin{tikzpicture}
	\begin{pgfonlayer}{nodelayer}
		\node [style=map] (64) at (28.75, 0) {$\hat g$};
		\node [style=none] (65) at (28, 1.25) {};
		\node [style=X] (66) at (29.25, 1) {};
		\node [style=Z] (67) at (29.75, 1) {};
		\node [style=none] (68) at (28.75, 1.25) {};
		\node [style=map] (69) at (30.25, 0) {$\hat f$};
		\node [style=none] (70) at (30.25, 1.25) {};
		\node [style=none] (71) at (31, 1.25) {};
		\node [style=none] (72) at (30.25, 2.5) {};
		\node [style=none] (73) at (28.75, 2.5) {};
		\node [style=none] (74) at (31, 2.5) {};
		\node [style=none] (75) at (28, 2.5) {};
	\end{pgfonlayer}
	\begin{pgfonlayer}{edgelayer}
		\draw [in=135, out=-90] (65.center) to (64);
		\draw [in=-150, out=45] (64) to (67);
		\draw [in=-165, out=105, looseness=1.25] (64) to (66);
		\draw [in=-90, out=60] (64) to (68.center);
		\draw [in=-30, out=60, looseness=1.25] (69) to (67);
		\draw [in=135, out=-30] (66) to (69);
		\draw [in=-90, out=120] (69) to (70.center);
		\draw [in=45, out=-90] (71.center) to (69);
		\draw (65.center) to (75.center);
		\draw [in=270, out=90] (68.center) to (72.center);
		\draw (71.center) to (74.center);
		\draw [in=270, out=90] (70.center) to (73.center);
	\end{pgfonlayer}
\end{tikzpicture}
$$

We have the following equality of diagrams in $\Stab_p$, 
We draw the wires exiting $G$ to be connected to the corresponding wires in the $X$ block of the subspace.
\begin{align*}
\begin{tikzpicture}
	\begin{pgfonlayer}{nodelayer}
		\node [style=map] (29) at (23, 0) {$G\left(\hat {f;g}\right)$};
		\node [style=none] (30) at (22.25, 1.5) {};
		\node [style=none] (33) at (23.75, 1.5) {};
	\end{pgfonlayer}
	\begin{pgfonlayer}{edgelayer}
		\draw [in=135, out=-90] (30.center) to (29);
		\draw [in=-90, out=45] (29) to (33.center);
	\end{pgfonlayer}
\end{tikzpicture}
&=
\begin{tikzpicture}
	\begin{pgfonlayer}{nodelayer}
		\node [style=map] (244) at (170.75, -4.5) {$\hat g$};
		\node [style=none] (245) at (170, -3.25) {};
		\node [style=X] (246) at (171.25, -3.5) {};
		\node [style=Z] (247) at (171.75, -3.5) {};
		\node [style=none] (248) at (170.75, -3.25) {};
		\node [style=map] (249) at (172.25, -4.5) {$\hat f$};
		\node [style=none] (250) at (172.25, -3.25) {};
		\node [style=none] (251) at (173, -3.25) {};
		\node [style=none] (252) at (172.25, -2) {};
		\node [style=none] (253) at (170.75, -2) {};
		\node [style=none] (254) at (173, -2) {};
		\node [style=none] (255) at (170, -2) {};
		\node [style=none] (256) at (169.75, -2) {};
		\node [style=none] (257) at (173.25, -2) {};
		\node [style=none] (258) at (173.25, -5) {};
		\node [style=none] (259) at (169.75, -5) {};
		\node [style=none] (260) at (170, -4.75) {$G$};
		\node [style=none] (261) at (170.75, -2) {};
		\node [style=none] (262) at (170, -2) {};
		\node [style=none] (263) at (170.75, -1.25) {};
		\node [style=none] (264) at (170, -1.25) {};
	\end{pgfonlayer}
	\begin{pgfonlayer}{edgelayer}
		\draw [in=135, out=-90] (245.center) to (244);
		\draw [in=-150, out=45] (244) to (247);
		\draw [in=-165, out=105, looseness=1.25] (244) to (246);
		\draw [in=-90, out=60] (244) to (248.center);
		\draw [in=-30, out=60, looseness=1.25] (249) to (247);
		\draw [in=135, out=-30] (246) to (249);
		\draw [in=-90, out=120] (249) to (250.center);
		\draw [in=45, out=-90] (251.center) to (249);
		\draw (245.center) to (255.center);
		\draw [in=270, out=90] (248.center) to (252.center);
		\draw (251.center) to (254.center);
		\draw [in=270, out=90] (250.center) to (253.center);
		\draw (257.center) to (256.center);
		\draw (256.center) to (259.center);
		\draw (259.center) to (258.center);
		\draw (258.center) to (257.center);
		\draw (262.center) to (264.center);
		\draw (261.center) to (263.center);
	\end{pgfonlayer}
\end{tikzpicture}
=
\begin{tikzpicture}
	\begin{pgfonlayer}{nodelayer}
		\node [style=map] (12) at (200, -1.75) {$\hat g$};
		\node [style=none] (13) at (199.25, -0.25) {};
		\node [style=none] (14) at (199.75, -0.25) {};
		\node [style=map] (15) at (202, -1.75) {$\hat f$};
		\node [style=none] (16) at (202.25, -0.25) {};
		\node [style=none] (17) at (202.75, -0.25) {};
		\node [style=none] (18) at (199, -0.25) {};
		\node [style=none] (19) at (203, -0.25) {};
		\node [style=none] (20) at (203, -2.25) {};
		\node [style=none] (21) at (199, -2.25) {};
		\node [style=none] (22) at (199.25, -2) {$G$};
		\node [style=none] (23) at (200.75, -0.25) {};
		\node [style=none] (24) at (199.25, -0.25) {};
		\node [style=none] (25) at (200.75, 1) {};
		\node [style=none] (26) at (199.25, 1) {};
		\node [style=none] (27) at (201.25, -0.25) {};
		\node [style=none] (28) at (201.75, -0.25) {};
		\node [style=none] (29) at (200.75, -0.25) {};
		\node [style=none] (30) at (200.25, -0.25) {};
		\node [style=none] (31) at (199.75, -0.25) {};
		\node [style=none] (32) at (200.25, -0.25) {};
		\node [style=map] (33) at (200, 0.5) {$\eta$};
	\end{pgfonlayer}
	\begin{pgfonlayer}{edgelayer}
		\draw [in=135, out=-90] (13.center) to (12);
		\draw [in=-90, out=105] (12) to (14.center);
		\draw [in=-90, out=75] (15) to (16.center);
		\draw [in=45, out=-90] (17.center) to (15);
		\draw (19.center) to (18.center);
		\draw (18.center) to (21.center);
		\draw (21.center) to (20.center);
		\draw (20.center) to (19.center);
		\draw (24.center) to (26.center);
		\draw (23.center) to (25.center);
		\draw [in=-90, out=45] (12) to (28.center);
		\draw [in=-90, out=75] (12) to (27.center);
		\draw [in=-90, out=105] (15) to (29.center);
		\draw [in=-90, out=135] (15) to (30.center);
		\draw [in=90, out=-120] (33) to (31.center);
		\draw [in=90, out=-60] (33) to (32.center);
	\end{pgfonlayer}
\end{tikzpicture}
=
\begin{tikzpicture}
	\begin{pgfonlayer}{nodelayer}
		\node [style=map] (34) at (204.75, 0) {$G\left(\hat g\right)$};
		\node [style=none] (35) at (204, 1.25) {};
		\node [style=map] (36) at (205.5, 1) {$\eta$};
		\node [style=map] (37) at (206.25, 0) {$G\left(\hat f\right)$};
		\node [style=none] (38) at (207, 1.25) {};
		\node [style=none] (39) at (207, 2.5) {};
		\node [style=none] (40) at (204, 2.5) {};
	\end{pgfonlayer}
	\begin{pgfonlayer}{edgelayer}
		\draw [in=135, out=-90] (35.center) to (34);
		\draw [in=-120, out=45, looseness=0.75] (34) to (36);
		\draw [in=-60, out=135, looseness=0.75] (37) to (36);
		\draw [in=45, out=-90] (38.center) to (37);
		\draw (35.center) to (40.center);
		\draw (38.center) to (39.center);
	\end{pgfonlayer}
\end{tikzpicture}
\end{align*}
This second equality is the only nontrivial part.  It follows by observing that both stabilizer states are stabilized by the same generalized Pauli operators, and thus they are the same.  This is because  the generalized Pauli operators can be pulled through $G$, by \cite[Lemma 4]{gross}, where they act the same on the caps of Lagrangian relations and in matrices.
%More explicitly, each white cap acting on the $i$th and $j$ wire is just removing vectors $[X_k|Z_k]$ from the stabilizer subspace which $X_{i.k}\neq -X_{i.k}$ or $Z_{i.k}\neq Z_{i.k}$ and then projecting out the $i$th and $j$th columns.

Explicity, the boost and shift operators commute with the cap as follows:
$$
\begin{tabular}{c}
\begin{tikzpicture}
	\begin{pgfonlayer}{nodelayer}
		\node [style=none] (515) at (241, -0.25) {};
		\node [style=none] (516) at (241, -1) {};
		\node [style=none] (517) at (239.5, -1) {};
		\node [style=X] (518) at (240, 0.75) {};
		\node [style=Z] (519) at (241.5, 0.75) {};
		\node [style=none] (520) at (240.5, -0.25) {};
		\node [style=none] (521) at (242, -0.25) {};
		\node [style=none] (522) at (240.5, -1) {};
		\node [style=none] (523) at (242, -1) {};
		\node [style=none] (524) at (239.5, -0.25) {};
		\node [style=X] (525) at (239.5, -0.25) {$a$};
	\end{pgfonlayer}
	\begin{pgfonlayer}{edgelayer}
		\draw (516.center) to (515.center);
		\draw [in=-165, out=90] (515.center) to (519);
		\draw [in=-15, out=90] (520.center) to (518);
		\draw [in=-15, out=90] (521.center) to (519);
		\draw (523.center) to (521.center);
		\draw (520.center) to (522.center);
		\draw (517.center) to (524.center);
		\draw [in=-165, out=90] (524.center) to (518);
	\end{pgfonlayer}
\end{tikzpicture}
=
\begin{tikzpicture}
	\begin{pgfonlayer}{nodelayer}
		\node [style=none] (526) at (244.25, -0.25) {};
		\node [style=none] (527) at (244.25, -1) {};
		\node [style=none] (528) at (242.75, -1) {};
		\node [style=X] (529) at (243.25, 0.75) {$a$};
		\node [style=Z] (530) at (244.75, 0.75) {};
		\node [style=none] (531) at (243.75, -0.25) {};
		\node [style=none] (532) at (245.25, -0.25) {};
		\node [style=none] (533) at (243.75, -1) {};
		\node [style=none] (534) at (245.25, -1) {};
		\node [style=none] (535) at (242.75, -0.25) {};
	\end{pgfonlayer}
	\begin{pgfonlayer}{edgelayer}
		\draw (527.center) to (526.center);
		\draw [in=-165, out=90] (526.center) to (530);
		\draw [in=-15, out=90] (531.center) to (529);
		\draw [in=-15, out=90] (532.center) to (530);
		\draw (534.center) to (532.center);
		\draw (531.center) to (533.center);
		\draw (528.center) to (535.center);
		\draw [in=-165, out=90] (535.center) to (529);
	\end{pgfonlayer}
\end{tikzpicture}
=
\begin{tikzpicture}
	\begin{pgfonlayer}{nodelayer}
		\node [style=none] (536) at (247.5, -0.25) {};
		\node [style=none] (537) at (247.5, -1) {};
		\node [style=none] (538) at (246, -1) {};
		\node [style=X] (539) at (246.5, 0.75) {};
		\node [style=Z] (540) at (248, 0.75) {};
		\node [style=none] (541) at (247, -0.25) {};
		\node [style=none] (542) at (248.5, -0.25) {};
		\node [style=none] (543) at (247, -1) {};
		\node [style=none] (544) at (248.5, -1) {};
		\node [style=none] (545) at (246, -0.25) {};
		\node [style=X] (546) at (247, -0.25) {$a$};
	\end{pgfonlayer}
	\begin{pgfonlayer}{edgelayer}
		\draw (537.center) to (536.center);
		\draw [in=-165, out=90] (536.center) to (540);
		\draw [in=-15, out=90] (541.center) to (539);
		\draw [in=-15, out=90] (542.center) to (540);
		\draw (544.center) to (542.center);
		\draw (541.center) to (543.center);
		\draw (538.center) to (545.center);
		\draw [in=-165, out=90] (545.center) to (539);
	\end{pgfonlayer}
\end{tikzpicture}\\
\begin{tikzpicture}
	\begin{pgfonlayer}{nodelayer}
		\node [style=none] (547) at (249.25, -3.75) {};
		\node [style=none] (548) at (249.25, -4.5) {};
		\node [style=none] (549) at (247.75, -4.5) {};
		\node [style=X] (550) at (248.25, -2.75) {};
		\node [style=Z] (551) at (249.75, -2.75) {};
		\node [style=none] (552) at (248.75, -3.75) {};
		\node [style=none] (553) at (250.25, -3.75) {};
		\node [style=none] (554) at (248.75, -4.5) {};
		\node [style=none] (555) at (250.25, -4.5) {};
		\node [style=none] (556) at (247.75, -3.75) {};
		\node [style=X] (557) at (249.25, -3.75) {$a$};
	\end{pgfonlayer}
	\begin{pgfonlayer}{edgelayer}
		\draw (548.center) to (547.center);
		\draw [in=-165, out=90] (547.center) to (551);
		\draw [in=-15, out=90] (552.center) to (550);
		\draw [in=-15, out=90] (553.center) to (551);
		\draw (555.center) to (553.center);
		\draw (552.center) to (554.center);
		\draw (549.center) to (556.center);
		\draw [in=-165, out=90] (556.center) to (550);
	\end{pgfonlayer}
\end{tikzpicture}
=
\begin{tikzpicture}
	\begin{pgfonlayer}{nodelayer}
		\node [style=none] (558) at (253, -4.5) {};
		\node [style=none] (559) at (251.5, -4.5) {};
		\node [style=X] (560) at (252, -2) {};
		\node [style=none] (561) at (252.5, -3) {};
		\node [style=none] (562) at (252.5, -4.5) {};
		\node [style=none] (563) at (254, -3) {};
		\node [style=none] (564) at (251.5, -3) {};
		\node [style=X] (565) at (253, -3.75) {$a$};
		\node [style=X] (566) at (253.5, -2) {};
		\node [style=s] (567) at (253, -3) {};
		\node [style=none] (568) at (254, -4.5) {};
	\end{pgfonlayer}
	\begin{pgfonlayer}{edgelayer}
		\draw [in=-15, out=90] (561.center) to (560);
		\draw [in=90, out=-90] (561.center) to (562.center);
		\draw (559.center) to (564.center);
		\draw [in=-165, out=90] (564.center) to (560);
		\draw [in=-15, out=90] (563.center) to (566);
		\draw (558.center) to (565);
		\draw (565) to (567);
		\draw [in=-165, out=90] (567) to (566);
		\draw (568.center) to (563.center);
	\end{pgfonlayer}
\end{tikzpicture}
=
\begin{tikzpicture}
	\begin{pgfonlayer}{nodelayer}
		\node [style=none] (569) at (256.75, -4.5) {};
		\node [style=none] (570) at (255.25, -4.5) {};
		\node [style=X] (571) at (255.75, -2) {};
		\node [style=none] (572) at (256.25, -3) {};
		\node [style=none] (573) at (256.25, -4.5) {};
		\node [style=none] (574) at (257.75, -3) {};
		\node [style=none] (575) at (255.25, -3) {};
		\node [style=X] (576) at (257.25, -2) {};
		\node [style=none] (577) at (257.75, -4.5) {};
		\node [style=s] (578) at (256.75, -3.75) {};
		\node [style=X] (579) at (256.75, -3) {$-a$};
	\end{pgfonlayer}
	\begin{pgfonlayer}{edgelayer}
		\draw [in=-15, out=90] (572.center) to (571);
		\draw [in=90, out=-90] (572.center) to (573.center);
		\draw (570.center) to (575.center);
		\draw [in=-165, out=90] (575.center) to (571);
		\draw [in=-15, out=90] (574.center) to (576);
		\draw (577.center) to (574.center);
		\draw (569.center) to (578);
		\draw (578) to (579);
		\draw [in=-165, out=90] (579) to (576);
	\end{pgfonlayer}
\end{tikzpicture}
=
\begin{tikzpicture}
	\begin{pgfonlayer}{nodelayer}
		\node [style=none] (580) at (261.25, -3.75) {};
		\node [style=none] (581) at (259.75, -3.75) {};
		\node [style=X] (582) at (260.25, -2) {};
		\node [style=none] (583) at (260.75, -3) {};
		\node [style=none] (584) at (260.75, -3.75) {};
		\node [style=none] (585) at (262.25, -3) {};
		\node [style=none] (586) at (259.75, -3) {};
		\node [style=X] (587) at (261.75, -2) {};
		\node [style=none] (588) at (262.25, -3.75) {};
		\node [style=s] (589) at (261.25, -3) {};
		\node [style=X] (590) at (262.25, -3) {$-a$};
	\end{pgfonlayer}
	\begin{pgfonlayer}{edgelayer}
		\draw [in=-15, out=90] (583.center) to (582);
		\draw [in=90, out=-90] (583.center) to (584.center);
		\draw (581.center) to (586.center);
		\draw [in=-165, out=90] (586.center) to (582);
		\draw [in=-15, out=90] (585.center) to (587);
		\draw (588.center) to (585.center);
		\draw (580.center) to (589);
		\draw [in=-165, out=90] (589) to (587);
	\end{pgfonlayer}
\end{tikzpicture}
=
\begin{tikzpicture}
	\begin{pgfonlayer}{nodelayer}
		\node [style=none] (591) at (265.25, -3) {};
		\node [style=none] (592) at (265.25, -3.75) {};
		\node [style=none] (593) at (263.75, -3.75) {};
		\node [style=X] (594) at (264.25, -2) {};
		\node [style=Z] (595) at (265.75, -2) {};
		\node [style=none] (596) at (264.75, -3) {};
		\node [style=none] (597) at (266.25, -3) {};
		\node [style=none] (598) at (264.75, -3.75) {};
		\node [style=none] (599) at (266.25, -3.75) {};
		\node [style=none] (600) at (263.75, -3) {};
		\node [style=X] (601) at (266.25, -3) {$-a$};
	\end{pgfonlayer}
	\begin{pgfonlayer}{edgelayer}
		\draw (592.center) to (591.center);
		\draw [in=-165, out=90] (591.center) to (595);
		\draw [in=-15, out=90] (596.center) to (594);
		\draw [in=-15, out=90] (597.center) to (595);
		\draw (599.center) to (597.center);
		\draw (596.center) to (598.center);
		\draw (593.center) to (600.center);
		\draw [in=-165, out=90] (600.center) to (594);
	\end{pgfonlayer}
\end{tikzpicture}
\end{tabular}
$$
Which are analagous to the following commutations in  stabilizer circuits:
$$
\begin{tikzpicture}
	\begin{pgfonlayer}{nodelayer}
		\node [style=map] (161) at (151.25, 6) {$\eta$};
		\node [style=none] (162) at (150.75, 5.25) {};
		\node [style=none] (163) at (151.75, 5.25) {};
		\node [style=none] (164) at (150.75, 4.5) {};
		\node [style=none] (165) at (151.75, 4.5) {};
		\node [style=map] (166) at (150.75, 5.25) {${\cal Z}^a$};
	\end{pgfonlayer}
	\begin{pgfonlayer}{edgelayer}
		\draw (165.center) to (163.center);
		\draw [in=-30, out=90] (163.center) to (161);
		\draw [in=90, out=-150] (161) to (162.center);
		\draw (162.center) to (164.center);
	\end{pgfonlayer}
\end{tikzpicture}
=
\begin{tikzpicture}
	\begin{pgfonlayer}{nodelayer}
		\node [style=map] (155) at (153.5, 6) {$\eta$};
		\node [style=none] (156) at (153, 5.25) {};
		\node [style=none] (157) at (154, 5.25) {};
		\node [style=none] (158) at (153, 4.5) {};
		\node [style=none] (159) at (154, 4.5) {};
		\node [style=map] (160) at (154, 5.25) {${\cal Z}^a$};
	\end{pgfonlayer}
	\begin{pgfonlayer}{edgelayer}
		\draw (159.center) to (157.center);
		\draw [in=-30, out=90] (157.center) to (155);
		\draw [in=90, out=-150] (155) to (156.center);
		\draw (156.center) to (158.center);
	\end{pgfonlayer}
\end{tikzpicture}
\hspace*{.5cm}
\begin{tikzpicture}
	\begin{pgfonlayer}{nodelayer}
		\node [style=map] (167) at (151.25, 3.75) {$\eta$};
		\node [style=none] (168) at (150.75, 3) {};
		\node [style=none] (169) at (151.75, 3) {};
		\node [style=none] (170) at (150.75, 2.25) {};
		\node [style=none] (171) at (151.75, 2.25) {};
		\node [style=map] (172) at (150.75, 3) {${\cal X}^a$};
	\end{pgfonlayer}
	\begin{pgfonlayer}{edgelayer}
		\draw (171.center) to (169.center);
		\draw [in=-30, out=90] (169.center) to (167);
		\draw [in=90, out=-150] (167) to (168.center);
		\draw (168.center) to (170.center);
	\end{pgfonlayer}
\end{tikzpicture}
=
\begin{tikzpicture}
	\begin{pgfonlayer}{nodelayer}
		\node [style=map] (173) at (153.5, 4) {$\eta$};
		\node [style=none] (174) at (153, 3.25) {};
		\node [style=none] (175) at (154, 3.25) {};
		\node [style=none] (176) at (153, 2.5) {};
		\node [style=none] (177) at (154, 2.5) {};
		\node [style=map] (178) at (154, 3.25) {${\cal X}^{-a}$};
	\end{pgfonlayer}
	\begin{pgfonlayer}{edgelayer}
		\draw (177.center) to (175.center);
		\draw [in=-30, out=90] (175.center) to (173);
		\draw [in=90, out=-150] (173) to (174.center);
		\draw (174.center) to (176.center);
	\end{pgfonlayer}
\end{tikzpicture}
$$
Where moreover, for any Lagrangian relation $V$ and  $a,b \in \F_p$, we already know:
$$
\begin{tikzpicture}
	\begin{pgfonlayer}{nodelayer}
		\node [style=none] (294) at (183.25, -0.75) {};
		\node [style=none] (295) at (185.75, -0.75) {};
		\node [style=none] (296) at (185.75, -3) {};
		\node [style=none] (297) at (183.25, -3) {};
		\node [style=none] (298) at (183.5, -2.75) {$G$};
		\node [style=map] (299) at (184.5, -2.5) {$V$};
		\node [style=none] (300) at (183.5, -0.75) {};
		\node [style=none] (301) at (184, -0.75) {};
		\node [style=none] (302) at (185, -0.75) {};
		\node [style=none] (303) at (185.5, -0.75) {};
		\node [style=X] (304) at (183.5, -1.5) {$a$};
		\node [style=X] (305) at (185, -1.5) {$b$};
		\node [style=none] (306) at (184, -1.5) {};
		\node [style=none] (307) at (185.5, -1.5) {};
		\node [style=none] (308) at (184, -0.25) {};
		\node [style=none] (309) at (183.5, -0.25) {};
		\node [style=none] (310) at (184, -0.75) {};
		\node [style=none] (311) at (183.5, -0.75) {};
	\end{pgfonlayer}
	\begin{pgfonlayer}{edgelayer}
		\draw (295.center) to (294.center);
		\draw (294.center) to (297.center);
		\draw (297.center) to (296.center);
		\draw (296.center) to (295.center);
		\draw [in=-90, out=45] (299) to (307.center);
		\draw (307.center) to (303.center);
		\draw (302.center) to (305);
		\draw [in=75, out=-90] (305) to (299);
		\draw [in=-90, out=105] (299) to (306.center);
		\draw (306.center) to (301.center);
		\draw (300.center) to (304);
		\draw [in=135, out=-90] (304) to (299);
		\draw (311.center) to (309.center);
		\draw (308.center) to (310.center);
	\end{pgfonlayer}
\end{tikzpicture}
=
\begin{tikzpicture}
	\begin{pgfonlayer}{nodelayer}
		\node [style=none] (351) at (195, -0.75) {};
		\node [style=none] (352) at (198, -0.75) {};
		\node [style=none] (353) at (198, -2.25) {};
		\node [style=none] (354) at (195, -2.25) {};
		\node [style=none] (355) at (195.25, -2) {$G$};
		\node [style=map] (356) at (196.5, -1.75) {$V$};
		\node [style=none] (357) at (195.25, -0.75) {};
		\node [style=none] (358) at (196.5, -0.75) {};
		\node [style=none] (359) at (197.25, -0.75) {};
		\node [style=none] (360) at (197.75, -0.75) {};
		\node [style=none] (361) at (196.5, 0.25) {};
		\node [style=none] (362) at (196.5, -0.75) {};
		\node [style=none] (363) at (195.25, -0.75) {};
		\node [style=map] (364) at (195.25, 0.25) {${\cal X}^a {\cal Z}^b$};
		\node [style=none] (365) at (195.25, 1) {};
		\node [style=none] (366) at (196.5, 1) {};
	\end{pgfonlayer}
	\begin{pgfonlayer}{edgelayer}
		\draw (352.center) to (351.center);
		\draw (351.center) to (354.center);
		\draw (354.center) to (353.center);
		\draw (353.center) to (352.center);
		\draw (361.center) to (362.center);
		\draw [in=270, out=90] (364) to (365.center);
		\draw [in=90, out=-90] (364) to (363.center);
		\draw [in=-90, out=135, looseness=0.75] (356) to (357.center);
		\draw [in=105, out=-90, looseness=0.75] (358.center) to (356);
		\draw [in=-90, out=75] (356) to (359.center);
		\draw [in=45, out=-90, looseness=0.75] (360.center) to (356);
		\draw (361.center) to (366.center);
	\end{pgfonlayer}
\end{tikzpicture}
$$

Therefore, functoriality follows by uncurrying the left and right hand sides of the previous equation.  Fullness and faithfulness follow immediately from $G$ being an isomorphism.
\end{proof}




\begin{definition}
Define a conjugation functor $\bar{(\_)}:\Aff\Lag\Rel_k\to \Aff\Lag\Rel_k$ the identity on $L(\LinRel_k)$ and $X$, but taking $F\mapsto F^{-1}$, $S_a \mapsto S_a^{-1}$ and $X\mapsto X$.
\end{definition}

The following fact follows from a mechanical calculation:


\begin{lemma}
For odd prime $p$, the conjugation functor $\bar{(\_)}:\Aff\Lag\Rel_{\F_p}\to \Aff\Lag\Rel_{\F_p}$ corresponds to complex conjugation in $\Stab_p$.
\end{lemma}





As we mentioned in the introduction, this is a categorical reformulation of the result of Spekkens' in which he shows that odd-prime-dimensional `quadrature epistricted theories' are operationally equivalent to prime-dimensional qudit stabilizer circuits \cite{spekkens2016quasi}.



A complete presentation for Spekkens' qubit toy model in terms of a category of relations was given \cite{backensspek} in a style which mirrors that of the qubit ZX-calculus~\cite{coecke2008interacting}. We now show how the generators of that presentation appear in our `doubled' formulation.




\begin{theorem}
For odd prime $p$, $\Aff\Lag\Rel_{\F_p}$ is isomorphic to $\Stab_p$ and 
$\Aff\Lag\Rel_{\F_2}$ is isomorphic to Spekkens toy model.
\end{theorem}

\begin{corollary}
For odd prime $p$, $\Lag\Rel_{\F_p}$ is a presentation for stabilizer circuits without affine phases. 
\end{corollary}

\begin{theorem}
$\Aff\Lag\Rel_{k}$ is generated by two spiders both decorated by the additive group of $k^2$:
$$
\left\llbracket
\begin{tikzpicture}
	\begin{pgfonlayer}{nodelayer}
		\node [style=none] (0) at (21, 5) {};
		\node [style=none] (1) at (22, 5) {};
		\node [style=none] (2) at (21, 2.5) {};
		\node [style=none] (3) at (22, 2.5) {};
		\node [style=Z] (4) at (21.5, 3.75) {$\hspace*{.05cm}n,m\hspace*{.05cm}$};
		\node [style=none] (5) at (21.5, 4.5) {$\cdots$};
		\node [style=none] (6) at (21.5, 3) {$\cdots$};
		\node [style=none] (7) at (21.5, 4.75) {};
		\node [style=none] (8) at (21.5, 2.75) {};
	\end{pgfonlayer}
	\begin{pgfonlayer}{edgelayer}
		\draw [in=150, out=-90, looseness=0.75] (0.center) to (4);
		\draw [in=90, out=-150, looseness=0.75] (4) to (2.center);
		\draw [in=-30, out=90, looseness=0.75] (3.center) to (4);
		\draw [in=-90, out=30, looseness=0.75] (4) to (1.center);
	\end{pgfonlayer}
\end{tikzpicture}
\right\rrbracket
=
\begin{tikzpicture}
	\begin{pgfonlayer}{nodelayer}
		\node [style=none] (9) at (231.75, 0.5) {};
		\node [style=none] (10) at (231.75, -3) {};
		\node [style=Z] (11) at (231, -2) {};
		\node [style=none] (12) at (231.32, -1.25) {$\cdots$};
		\node [style=none] (13) at (231, -2.5) {$\cdots$};
		\node [style=none] (14) at (230.75, 0.5) {};
		\node [style=none] (15) at (229.25, -3) {};
		\node [style=X] (16) at (230, -0.5) {$n$};
		\node [style=none] (17) at (230, 0) {$\cdots$};
		\node [style=none] (18) at (229.72, -1.25) {$\cdots$};
		\node [style=none] (19) at (230, -3) {};
		\node [style=none] (20) at (230.25, -3) {};
		\node [style=none] (21) at (229.25, 0.5) {};
		\node [style=none] (22) at (231, 0.5) {};
		\node [style=scalar] (23) at (230.5, -1.25) {$m$};
		\node [style=none] (24) at (230, 0.25) {};
		\node [style=none] (25) at (231, -2.75) {};
		\node [style=none] (26) at (229.7, -1.5) {};
		\node [style=none] (27) at (231.3, -1) {};
	\end{pgfonlayer}
	\begin{pgfonlayer}{edgelayer}
		\draw [in=-30, out=90] (10.center) to (11);
		\draw [in=-90, out=30, looseness=0.75] (11) to (9.center);
		\draw [in=90, out=-150] (16) to (15.center);
		\draw [in=-90, out=30] (16) to (14.center);
		\draw (19.center) to (16);
		\draw [in=-150, out=90] (20.center) to (11);
		\draw [in=-90, out=150] (16) to (21.center);
		\draw (11) to (22.center);
		\draw [in=-75, out=150] (11) to (23);
		\draw [in=330, out=90] (23) to (16);
	\end{pgfonlayer}
\end{tikzpicture}
\hspace*{.5cm}
\left\llbracket
\begin{tikzpicture}
	\begin{pgfonlayer}{nodelayer}
		\node [style=none] (0) at (21, 5) {};
		\node [style=none] (1) at (22, 5) {};
		\node [style=none] (2) at (21, 2.5) {};
		\node [style=none] (3) at (22, 2.5) {};
		\node [style=X] (4) at (21.5, 3.75) {$\hspace*{.05cm}n,m\hspace*{.05cm}$};
		\node [style=none] (5) at (21.5, 4.5) {$\cdots$};
		\node [style=none] (6) at (21.5, 3) {$\cdots$};
		\node [style=none] (7) at (21.5, 4.75) {};
		\node [style=none] (8) at (21.5, 2.75) {};
	\end{pgfonlayer}
	\begin{pgfonlayer}{edgelayer}
		\draw [in=150, out=-90, looseness=0.75] (0.center) to (4);
		\draw [in=90, out=-150, looseness=0.75] (4) to (2.center);
		\draw [in=-30, out=90, looseness=0.75] (3.center) to (4);
		\draw [in=-90, out=30, looseness=0.75] (4) to (1.center);
	\end{pgfonlayer}
\end{tikzpicture}
\right\rrbracket
:=
\begin{tikzpicture}
	\begin{pgfonlayer}{nodelayer}
		\node [style=none] (0) at (232.75, 0.5) {};
		\node [style=none] (1) at (232.75, -3) {};
		\node [style=Z] (2) at (233.5, -2) {};
		\node [style=none] (3) at (233.25, -1.25) {$\cdots$};
		\node [style=none] (4) at (233.5, -2.5) {$\cdots$};
		\node [style=none] (5) at (233.75, 0.5) {};
		\node [style=none] (6) at (235.25, -3) {};
		\node [style=X] (7) at (234.5, -0.5) {$n$};
		\node [style=none] (8) at (234.5, 0) {$\cdots$};
		\node [style=none] (9) at (234.82, -1.25) {$\cdots$};
		\node [style=none] (10) at (234.5, -3) {};
		\node [style=none] (11) at (234.25, -3) {};
		\node [style=none] (12) at (235.25, 0.5) {};
		\node [style=none] (13) at (233.5, 0.5) {};
		\node [style=scalar] (14) at (234, -1.25) {$m$};
		\node [style=none] (15) at (233.25, -1) {};
		\node [style=none] (16) at (234.5, 0.25) {};
		\node [style=none] (17) at (233.5, -2.75) {};
		\node [style=none] (18) at (234.78, -1.5) {};
	\end{pgfonlayer}
	\begin{pgfonlayer}{edgelayer}
		\draw [in=-150, out=90] (1.center) to (2);
		\draw [in=-90, out=150, looseness=0.75] (2) to (0.center);
		\draw [in=90, out=-30] (7) to (6.center);
		\draw [in=-90, out=150] (7) to (5.center);
		\draw (10.center) to (7);
		\draw [in=-30, out=90] (11.center) to (2);
		\draw [in=-90, out=30] (7) to (12.center);
		\draw (2) to (13.center);
		\draw [in=-105, out=30] (2) to (14);
		\draw [in=-150, out=90] (14) to (7);
	\end{pgfonlayer}
\end{tikzpicture}
$$

The Fourier transform is derived by Euler composition:
$$
\left\llbracket
\begin{tikzpicture}
	\begin{pgfonlayer}{nodelayer}
		\node [style=none] (0) at (1.25, -1) {};
		\node [style=map] (1) at (1.25, -1.5) {$F$};
		\node [style=none] (2) at (1.25, -2) {};
	\end{pgfonlayer}
	\begin{pgfonlayer}{edgelayer}
		\draw (2.center) to (1);
		\draw (1) to (0.center);
	\end{pgfonlayer}
\end{tikzpicture}
\right\rrbracket
=
\begin{tikzpicture}
	\begin{pgfonlayer}{nodelayer}
		\node [style=none] (0) at (0.5, 1) {};
		\node [style=none] (1) at (0.5, -0.25) {};
		\node [style=none] (2) at (1, -0.25) {};
		\node [style=none] (3) at (1, 1) {};
		\node [style=s] (4) at (1, 0.5) {};
		\node [style=none] (5) at (0.5, 0.5) {};
	\end{pgfonlayer}
	\begin{pgfonlayer}{edgelayer}
		\draw (4) to (3.center);
		\draw [in=90, out=-90] (4) to (1.center);
		\draw [in=-90, out=90] (2.center) to (5.center);
		\draw (5.center) to (0.center);
	\end{pgfonlayer}
\end{tikzpicture}
=
\begin{tikzpicture}[xscale=-1]
	\begin{pgfonlayer}{nodelayer}
		\node [style=X] (0) at (19.5, 1.25) {};
		\node [style=Z] (1) at (18, 4.25) {};
		\node [style=none] (2) at (18, 4.75) {};
		\node [style=none] (3) at (19.5, 4.75) {};
		\node [style=none] (4) at (18, 0.75) {};
		\node [style=none] (5) at (19.5, 0.75) {};
		\node [style=X] (6) at (19.5, 1.25) {};
		\node [style=Z] (7) at (19.5, 4.25) {};
		\node [style=X] (8) at (18, 1.25) {};
		\node [style=Z] (9) at (18, 4.25) {};
		\node [style=X] (10) at (19.5, 1.25) {};
		\node [style=X] (11) at (18, 1.25) {};
		\node [style=X] (12) at (19.5, 1.25) {};
		\node [style=Z] (13) at (19.5, 4.25) {};
		\node [style=s] (14) at (17.75, 3) {};
		\node [style=s] (15) at (18.75, 3) {};
		\node [style=s] (16) at (19.75, 3) {};
		\node [style=s] (17) at (18.25, 3) {};
	\end{pgfonlayer}
	\begin{pgfonlayer}{edgelayer}
		\draw (2.center) to (1);
		\draw (5.center) to (0);
		\draw [bend right=45] (6) to (7);
		\draw [in=135, out=-135, looseness=1.25] (9) to (8);
		\draw (3.center) to (7);
		\draw (4.center) to (8);
		\draw [in=-120, out=15, looseness=0.75] (11) to (13);
		\draw [in=90, out=-105] (9) to (14);
		\draw [in=90, out=-45, looseness=0.75] (9) to (15);
		\draw [in=90, out=-90] (14) to (11);
		\draw [in=-90, out=120, looseness=0.75] (6) to (15);
		\draw [in=-15, out=90] (12) to (9);
		\draw [in=-90, out=150, looseness=0.75] (12) to (17);
		\draw [in=285, out=90] (17) to (9);
		\draw [in=-90, out=75, looseness=0.75] (12) to (16);
		\draw [in=-75, out=90] (16) to (13);
	\end{pgfonlayer}
\end{tikzpicture}
=
\begin{tikzpicture}
	\begin{pgfonlayer}{nodelayer}
		\node [style=none] (0) at (0, 0.75) {};
		\node [style=none] (1) at (0.75, 0.75) {};
		\node [style=none] (2) at (0, 3.25) {};
		\node [style=none] (3) at (0.75, 3.25) {};
		\node [style=Z] (4) at (0.75, 1.25) {};
		\node [style=X] (5) at (0, 1.75) {};
		\node [style=Z] (6) at (0.75, 2.25) {};
		\node [style=X] (7) at (0, 2.75) {};
		\node [style=Z] (8) at (0, 2.25) {};
		\node [style=X] (9) at (0.75, 1.75) {};
	\end{pgfonlayer}
	\begin{pgfonlayer}{edgelayer}
		\draw (4) to (5);
		\draw (6) to (7);
		\draw (8) to (9);
		\draw (1.center) to (4);
		\draw (4) to (9);
		\draw (9) to (6);
		\draw (6) to (3.center);
		\draw (2.center) to (7);
		\draw (7) to (8);
		\draw (8) to (5);
		\draw (5) to (0.center);
	\end{pgfonlayer}
\end{tikzpicture}
=
\left\llbracket
\begin{tikzpicture}
	\begin{pgfonlayer}{nodelayer}
		\node [style=none] (0) at (1.25, 0) {};
		\node [style=none] (1) at (1.25, -3.5) {};
		\node [style=Z] (2) at (1.25, -2.75) {$\hspace*{.05cm}0,1\hspace*{.05cm}$};
		\node [style=Z] (3) at (1.25, -0.75) {$\hspace*{.05cm}0,1\hspace*{.05cm}$};
		\node [style=X] (4) at (1.25, -1.75) {$\hspace*{.05cm}0,-1\hspace*{.05cm}$};
	\end{pgfonlayer}
	\begin{pgfonlayer}{edgelayer}
		\draw (1.center) to (2);
		\draw (2) to (4);
		\draw (4) to (3);
		\draw (3) to (0.center);
	\end{pgfonlayer}
\end{tikzpicture}
\right\rrbracket
$$



%Transporting the complex conjugation along the isomorphism $\Aff\Lag\Rel_k \cong \Stab_p$ yields a conjugation functor $\Aff\Lag\Rel_k\to \Aff\Lag\Rel_k$ such that:

%In $\Stab_p$ the spiders are linear maps:
%
%$$
%\left\llbracket
%\begin{tikzpicture}
%	\begin{pgfonlayer}{nodelayer}
%		\node [style=none] (0) at (21, 5) {};
%		\node [style=none] (1) at (22, 5) {};
%		\node [style=none] (2) at (21, 2.5) {};
%		\node [style=none] (3) at (22, 2.5) {};
%		\node [style=Z] (4) at (21.5, 3.75) {$\hspace*{.05cm}n,m\hspace*{.05cm}$};
%		\node [style=none] (5) at (21.5, 4.5) {$\cdots$};
%		\node [style=none] (6) at (21.5, 3) {$\cdots$};
%		\node [style=none] (7) at (21.5, 4.75) {};
%		\node [style=none] (8) at (21.5, 2.75) {};
%	\end{pgfonlayer}
%	\begin{pgfonlayer}{edgelayer}
%		\draw [in=150, out=-90, looseness=0.75] (0.center) to (4);
%		\draw [in=90, out=-150, looseness=0.75] (4) to (2.center);
%		\draw [in=-30, out=90, looseness=0.75] (3.center) to (4);
%		\draw [in=-90, out=30, looseness=0.75] (4) to (1.center);
%	\end{pgfonlayer}
%\end{tikzpicture}
%\right\rrbracket
%=
%\sum{}
%$$
%
%
%
%GIVE ACTION OF COMPLEX CONJUGATION
\end{theorem}






The spider fusion is pointwise  where $(a,b)=(n+k,m+\ell)$:

$$
\begin{tikzpicture}
	\begin{pgfonlayer}{nodelayer}
		\node [style=none] (0) at (1.5, -0.5) {};
		\node [style=none] (1) at (0.5, -0.5) {};
		\node [style=none] (2) at (1, -0.5) {$\cdots$};
		\node [style=none] (3) at (0.5, -2.75) {};
		\node [style=Z] (4) at (1, -1.25) {$n,m$};
		\node [style=none] (5) at (2, -0.5) {};
		\node [style=none] (6) at (1.5, -2.75) {$\cdots$};
		\node [style=none] (7) at (1, -2.75) {};
		\node [style=Z] (8) at (1.5, -2) {$k,\ell$};
		\node [style=none] (9) at (2, -2.75) {};
		\node [style=none] (10) at (1.25, -1.5) {\reflectbox{$\ddots$}};
	\end{pgfonlayer}
	\begin{pgfonlayer}{edgelayer}
		\draw [in=-124, out=90] (3.center) to (4);
		\draw [in=-90, out=56] (4) to (0.center);
		\draw [in=124, out=-90] (1.center) to (4);
		\draw [in=-124, out=90] (7.center) to (8);
		\draw [in=90, out=-56] (8) to (9.center);
		\draw [in=-90, out=56] (8) to (5.center);
		\draw [bend left=45, looseness=1.25] (8) to (4);
		\draw [bend left=45, looseness=1.25] (4) to (8);
	\end{pgfonlayer}
\end{tikzpicture}
=
\begin{tikzpicture}
	\begin{pgfonlayer}{nodelayer}
		\node [style=none] (11) at (4, -0.5) {};
		\node [style=none] (12) at (3, -0.5) {};
		\node [style=none] (13) at (3.5, -0.5) {$\cdots$};
		\node [style=none] (14) at (2.5, -2) {};
		\node [style=none] (15) at (3.5, -1.25) {};
		\node [style=none] (16) at (4.5, -0.5) {};
		\node [style=none] (17) at (3.5, -2) {$\cdots$};
		\node [style=none] (18) at (3, -2) {};
		\node [style=Z] (19) at (3.5, -1.25) {$a,b$};
		\node [style=none] (20) at (4, -2) {};
	\end{pgfonlayer}
	\begin{pgfonlayer}{edgelayer}
		\draw [in=-150, out=90] (14.center) to (15);
		\draw [in=-90, out=56] (15) to (11.center);
		\draw [in=124, out=-90] (12.center) to (15);
		\draw [in=-124, out=90] (18.center) to (19);
		\draw [in=90, out=-56] (19) to (20.center);
		\draw [in=-90, out=30] (19) to (16.center);
	\end{pgfonlayer}
\end{tikzpicture}
\hspace*{1cm}
\begin{tikzpicture}
	\begin{pgfonlayer}{nodelayer}
		\node [style=none] (0) at (1.5, -0.5) {};
		\node [style=none] (1) at (0.5, -0.5) {};
		\node [style=none] (2) at (1, -0.5) {$\cdots$};
		\node [style=none] (3) at (0.5, -2.75) {};
		\node [style=X] (4) at (1, -1.25) {$n,m$};
		\node [style=none] (5) at (2, -0.5) {};
		\node [style=none] (6) at (1.5, -2.75) {$\cdots$};
		\node [style=none] (7) at (1, -2.75) {};
		\node [style=X] (8) at (1.5, -2) {$k,\ell$};
		\node [style=none] (9) at (2, -2.75) {};
		\node [style=none] (10) at (1.25, -1.5) {\reflectbox{$\ddots$}};
	\end{pgfonlayer}
	\begin{pgfonlayer}{edgelayer}
		\draw [in=-124, out=90] (3.center) to (4);
		\draw [in=-90, out=56] (4) to (0.center);
		\draw [in=124, out=-90] (1.center) to (4);
		\draw [in=-124, out=90] (7.center) to (8);
		\draw [in=90, out=-56] (8) to (9.center);
		\draw [in=-90, out=56] (8) to (5.center);
		\draw [bend left=45, looseness=1.25] (8) to (4);
		\draw [bend left=45, looseness=1.25] (4) to (8);
	\end{pgfonlayer}
\end{tikzpicture}
=
\begin{tikzpicture}
	\begin{pgfonlayer}{nodelayer}
		\node [style=none] (11) at (4, -0.5) {};
		\node [style=none] (12) at (3, -0.5) {};
		\node [style=none] (13) at (3.5, -0.5) {$\cdots$};
		\node [style=none] (14) at (2.5, -2) {};
		\node [style=none] (15) at (3.5, -1.25) {};
		\node [style=none] (16) at (4.5, -0.5) {};
		\node [style=none] (17) at (3.5, -2) {$\cdots$};
		\node [style=none] (18) at (3, -2) {};
		\node [style=X] (19) at (3.5, -1.25) {$a,b$};
		\node [style=none] (20) at (4, -2) {};
	\end{pgfonlayer}
	\begin{pgfonlayer}{edgelayer}
		\draw [in=-150, out=90] (14.center) to (15);
		\draw [in=-90, out=56] (15) to (11.center);
		\draw [in=124, out=-90] (12.center) to (15);
		\draw [in=-124, out=90] (18.center) to (19);
		\draw [in=90, out=-56] (19) to (20.center);
		\draw [in=-90, out=30] (19) to (16.center);
	\end{pgfonlayer}
\end{tikzpicture}
$$

Call the first component of the phase group the {\bf affine phase} and the second component the {\bf linear phase}.  The white spider corresponds to the $Z$ basis and the grey spider corresponds to the $X$ basis.


As stabilizer circuits, the black spider is interpreted as follows (the white spiders are the same, but in the Fourier basis):

$$
\left\llbracket\
\begin{tikzpicture}
	\begin{pgfonlayer}{nodelayer}
		\node [style=none] (0) at (21, 5) {};
		\node [style=none] (1) at (22, 5) {};
		\node [style=none] (2) at (21, 2.5) {};
		\node [style=none] (3) at (22, 2.5) {};
		\node [style=Z] (4) at (21.5, 3.75) {$\hspace*{.05cm}n,m\hspace*{.05cm}$};
		\node [style=none] (5) at (21.5, 4.5) {$\cdots$};
		\node [style=none] (6) at (21.5, 3) {$\cdots$};
		\node [style=none] (7) at (21.5, 4.75) {};
		\node [style=none] (8) at (21.5, 2.75) {};
	\end{pgfonlayer}
	\begin{pgfonlayer}{edgelayer}
		\draw [in=150, out=-90, looseness=0.75] (0.center) to (4);
		\draw [in=90, out=-150, looseness=0.75] (4) to (2.center);
		\draw [in=-30, out=90, looseness=0.75] (3.center) to (4);
		\draw [in=-90, out=30, looseness=0.75] (4) to (1.center);
	\end{pgfonlayer}
\end{tikzpicture}\
\right\rrbracket
\propto
\sum_{j=0}^{p-1}  e^{ \pi \cdot i \cdot j \cdot m\cdot (m+p)/ p} | j+n,\ldots,j+n \rangle\langle j, \ldots, j|
$$

Notice that this implies that the phase groups of the frobenius algebras in odd prime dimensional stabilizer generating spiders are now the torus $(\Z/p\Z)^2$; this is in contrast to the qubit case where the phase groups are $\Z/4\Z$.


\begin{lemma}
The conjugation on $ \Stab_p \cong \Aff\Lag\Rel_{\F_p}$  generalizes to a conjugation $\bar{(\_)}:\Aff\Lag\Rel_k\to \Aff\Lag\Rel_k$ for any field $k$:
$$
\begin{tikzpicture}
	\begin{pgfonlayer}{nodelayer}
		\node [style=none] (0) at (21, 5) {};
		\node [style=none] (1) at (22, 5) {};
		\node [style=none] (2) at (21, 2.5) {};
		\node [style=none] (3) at (22, 2.5) {};
		\node [style=Z] (4) at (21.5, 3.75) {$\hspace*{.05cm}n,m\hspace*{.05cm}$};
		\node [style=none] (5) at (21.5, 4.5) {$\cdots$};
		\node [style=none] (6) at (21.5, 3) {$\cdots$};
		\node [style=none] (7) at (21.5, 4.75) {};
		\node [style=none] (8) at (21.5, 2.75) {};
	\end{pgfonlayer}
	\begin{pgfonlayer}{edgelayer}
		\draw [in=150, out=-90, looseness=0.75] (0.center) to (4);
		\draw [in=90, out=-150, looseness=0.75] (4) to (2.center);
		\draw [in=-30, out=90, looseness=0.75] (3.center) to (4);
		\draw [in=-90, out=30, looseness=0.75] (4) to (1.center);
	\end{pgfonlayer}
\end{tikzpicture}
\mapsto
\begin{tikzpicture}
	\begin{pgfonlayer}{nodelayer}
		\node [style=none] (0) at (20.75, 5) {};
		\node [style=none] (1) at (22.25, 5) {};
		\node [style=none] (2) at (20.75, 2.5) {};
		\node [style=none] (3) at (22.25, 2.5) {};
		\node [style=Z] (4) at (21.5, 3.75) {};
		\node [style=none] (5) at (21.5, 4.75) {$\cdots$};
		\node [style=none] (6) at (21.5, 2.75) {$\cdots$};
		\node [style=Z] (9) at (21.5, 3.75) {$\hspace*{.05cm}n,m\hspace*{.05cm}$};
	\end{pgfonlayer}
	\begin{pgfonlayer}{edgelayer}
		\draw [in=150, out=-90, looseness=0.75] (0.center) to (4);
		\draw [in=90, out=-150, looseness=0.75] (4) to (2.center);
		\draw [in=-30, out=90, looseness=0.75] (3.center) to (4);
		\draw [in=-90, out=30, looseness=0.75] (4) to (1.center);
	\end{pgfonlayer}
\end{tikzpicture}
\hspace*{1cm}
\begin{tikzpicture}
	\begin{pgfonlayer}{nodelayer}
		\node [style=none] (0) at (21, 5) {};
		\node [style=none] (1) at (22, 5) {};
		\node [style=none] (2) at (21, 2.5) {};
		\node [style=none] (3) at (22, 2.5) {};
		\node [style=X] (4) at (21.5, 3.75) {$\hspace*{.05cm}n,m\hspace*{.05cm}$};
		\node [style=none] (5) at (21.5, 4.5) {$\cdots$};
		\node [style=none] (6) at (21.5, 3) {$\cdots$};
		\node [style=none] (7) at (21.5, 4.75) {};
		\node [style=none] (8) at (21.5, 2.75) {};
	\end{pgfonlayer}
	\begin{pgfonlayer}{edgelayer}
		\draw [in=150, out=-90, looseness=0.75] (0.center) to (4);
		\draw [in=90, out=-150, looseness=0.75] (4) to (2.center);
		\draw [in=-30, out=90, looseness=0.75] (3.center) to (4);
		\draw [in=-90, out=30, looseness=0.75] (4) to (1.center);
	\end{pgfonlayer}
\end{tikzpicture}
\mapsto
\begin{tikzpicture}
	\begin{pgfonlayer}{nodelayer}
		\node [style=none] (0) at (20.75, 5) {};
		\node [style=none] (1) at (22.25, 5) {};
		\node [style=none] (2) at (20.75, 2.5) {};
		\node [style=none] (3) at (22.25, 2.5) {};
		\node [style=X] (4) at (21.5, 3.75) {};
		\node [style=none] (5) at (21.5, 4.75) {$\cdots$};
		\node [style=none] (6) at (21.5, 2.75) {$\cdots$};
		\node [style=X] (9) at (21.5, 3.75) {$\hspace*{.05cm}-n,m\hspace*{.05cm}$};
	\end{pgfonlayer}
	\begin{pgfonlayer}{edgelayer}
		\draw [in=150, out=-90, looseness=0.75] (0.center) to (4);
		\draw [in=90, out=-150, looseness=0.75] (4) to (2.center);
		\draw [in=-30, out=90, looseness=0.75] (3.center) to (4);
		\draw [in=-90, out=30, looseness=0.75] (4) to (1.center);
	\end{pgfonlayer}
\end{tikzpicture}
$$
\end{lemma}


In the quantum setting, a basis for the affine Lagrangian subspace corresponds to the stabilizer tableau for a pure stabilizer state (ie a stabilizer tableau on $n$ qudits with dimension $n$).
The novelty in interpreting stabilizer states in this categorical framework is that it reveals that the relational composition of tableaus is the composition of stabilizer circuits.


\section{String diagrams for measurement and coisotropic relations}
\label{sec:coisotrel}

Up until now has been a review of the literature.
In this section we show that by only requiring that the morphisms are affine {\em coisotropic} subspaces   (subspaces $V$ so that $V^\omega \subseteq V$) instead of affine Lagrangian subspaces, we can capture the maximally mixed state/discarding; with which we can recover state preparation and measurement compositionally.


\begin{theorem}[Relational purification]~\\
The prop $\Isot\Rel_k$ of isotropic relations is generated by adding the doubled zero relation to the image of the forgetful functor $\Lag\Rel_k\to\LinRel_k$, ie. the following generator in $\LinRel_k$:
$
\begin{tikzpicture}
	\begin{pgfonlayer}{nodelayer}
		\node [style=X] (0) at (0.5, 0.5) {};
		\node [style=X] (1) at (1, 0.5) {};
		\node [style=none] (2) at (0.5, 0) {};
		\node [style=none] (3) at (1, 0) {};
	\end{pgfonlayer}
	\begin{pgfonlayer}{edgelayer}
		\draw (1) to (3.center);
		\draw (0) to (2.center);
	\end{pgfonlayer}
\end{tikzpicture}
$
\end{theorem}
\begin{proof}
This generator is an isotropic subspace of $(k^{2n},\omega)$ since:
$$
\left(
\begin{tikzpicture}
	\begin{pgfonlayer}{nodelayer}
		\node [style=X] (0) at (0.5, 0.5) {};
		\node [style=X] (1) at (1, 0.5) {};
		\node [style=none] (2) at (0.5, 0) {};
		\node [style=none] (3) at (1, 0) {};
	\end{pgfonlayer}
	\begin{pgfonlayer}{edgelayer}
		\draw (1) to (3.center);
		\draw (0) to (2.center);
	\end{pgfonlayer}
\end{tikzpicture}
\right)^\omega
=
\begin{tikzpicture}
	\begin{pgfonlayer}{nodelayer}
		\node [style=Z] (0) at (0.5, 0.5) {};
		\node [style=Z] (1) at (1, 0.5) {};
		\node [style=none] (2) at (0.5, 0) {};
		\node [style=none] (3) at (1, 0) {};
		\node [style=s] (4) at (1, 0) {};
		\node [style=none] (5) at (1, -0.75) {};
		\node [style=none] (7) at (0.5, -0.75) {};
	\end{pgfonlayer}
	\begin{pgfonlayer}{edgelayer}
		\draw (1) to (3.center);
		\draw (0) to (2.center);
		\draw [in=270, out=90] (7.center) to (4.center);
		\draw [in=90, out=-90] (2.center) to (5.center);
	\end{pgfonlayer}
\end{tikzpicture}
=
\begin{tikzpicture}
	\begin{pgfonlayer}{nodelayer}
		\node [style=Z] (0) at (0.5, 0.5) {};
		\node [style=Z] (1) at (1, 0.5) {};
		\node [style=none] (2) at (0.5, 0) {};
		\node [style=none] (3) at (1, 0) {};
	\end{pgfonlayer}
	\begin{pgfonlayer}{edgelayer}
		\draw (1) to (3.center);
		\draw (0) to (2.center);
	\end{pgfonlayer}
\end{tikzpicture}
\supset
\begin{tikzpicture}
	\begin{pgfonlayer}{nodelayer}
		\node [style=X] (0) at (0.5, 0.5) {};
		\node [style=X] (1) at (1, 0.5) {};
		\node [style=none] (2) at (0.5, 0) {};
		\node [style=none] (3) at (1, 0) {};
	\end{pgfonlayer}
	\begin{pgfonlayer}{edgelayer}
		\draw (1) to (3.center);
		\draw (0) to (2.center);
	\end{pgfonlayer}
\end{tikzpicture}
$$



Suppose that we have an isotropic subspace $V$ of $(k^{2n},\omega)$ with dimension $n-1$. 

By applying Fourier transforms, we obtain a symplectomorphic subspace generated by a matrix whose pivots are all in the $Z$ block.  Therefore, we can row reduce this matrix to obtain one of the following form:

$$
\left[\begin{array}{cc|cc}
I_{n-1} & Z_B & X_A & X_B 
\end{array}\right]
$$

By applying controlled shift gates from the first $n-1$ wires to the last wire, we obtain an isotropic subspace $V'\cong V$ generated by a matrix of the following form:


$$
\left[\begin{array}{cc|cc}
I_{n-1} & 0 & X_A' & X_B' 
\end{array}\right]
$$

Since all of the rows of this subspace are orthogonal with respect to the symplectic form, we have:

\begin{align*}
0 &=
\left[\begin{array}{cc|cc}
I_{n-1} & 0 & X_A' & X_B' 
\end{array}\right]
\omega
\left[\begin{array}{cc|cc}
I_{n-1} & 0 & X_A' & X_B' 
\end{array}\right]^T\\
&=
\left[\begin{array}{cc|cc}
I_{n-1} & 0 & X_A' & X_B' 
\end{array}\right]
\left[\begin{array}{cc|cc}
 -X_A' & -X_B'  & I_{n-1} & 0
\end{array}\right]^T\\
&=
I_{n-1}(-X_A')^T +  0( -X_B' )^T +X_A'I_{n-1} + X_B' 0 \\
&=
(-X_A')^T +X_A'
\end{align*}
Which holds if and only if $X_A'=(X_A')^T$ is symmetric.

Therefore, the following matrix generates a Lagrangian subspace of $k^{2(n+1)}$ because the $Z$ block is the identity and the $X$ block is symmetric:
$$
\left[\begin{array}{ccc|ccc}
I_{n-1} & 0    & 0 & X_A'       & X_B' & 0\\
0           & 1 & 0 & (X_B')^T & 0     & 1 \\
0           & 0    & 1  & 0            & 1 & 0
\end{array}\right]
$$
Let $W$ be the Lagrangian subspace generated by this matrix.  Then
$$
\begin{tikzpicture}
	\begin{pgfonlayer}{nodelayer}
		\node [style=X] (0) at (0.5, 0.5) {};
		\node [style=X] (1) at (1.5, 0.5) {};
		\node [style=map] (8) at (0.75, -0.5) {$W$};
		\node [style=none] (9) at (1, 0.5) {};
		\node [style=none] (10) at (0, 0.5) {};
		\node [style=none] (11) at (0, 1) {};
		\node [style=none] (12) at (1, 1) {};
	\end{pgfonlayer}
	\begin{pgfonlayer}{edgelayer}
		\draw [bend left, looseness=0.75] (8) to (10.center);
		\draw [bend left=15] (8) to (0);
		\draw [bend right=15] (8) to (9.center);
		\draw [bend right, looseness=0.75] (8) to (1);
		\draw (11.center) to (10.center);
		\draw (9.center) to (12.center);
	\end{pgfonlayer}
\end{tikzpicture}
=
\begin{tikzpicture}
	\begin{pgfonlayer}{nodelayer}
		\node [style=map] (15) at (3.25, -0.5) {$V'$};
		\node [style=none] (16) at (3.5, 0.25) {};
		\node [style=none] (17) at (3, 0.25) {};
	\end{pgfonlayer}
	\begin{pgfonlayer}{edgelayer}
		\draw [bend left=15, looseness=0.75] (15) to (17.center);
		\draw [bend right=15, looseness=0.75] (15) to (16.center);
	\end{pgfonlayer}
\end{tikzpicture}
$$
This follows because composing $W$ with the cozero maps on the last wire of the $X$ and $Z$ blocks picks out the rows where the last entries of the  $Z$ and $X$ blocks are both postselected to be $0$; that is, those of the generator matrix of $V'$. Then by applying the inverse controlled shift and inverse Fourier transform, to $V'$ we get back $V'$ again.  This yields a Lagrangian dilation of $V$.

Suppose that we have an isotropic subspace $V$ of $(k^{2n},\omega)$ with dimension $n-k$; by induction, $V$ can be purified to a  Lagrangian subspace of $k^{2(n+k)}$.
\end{proof}

Since, the symplectic complement reverses the order of inclusions, it extends to an isomorphism $\Co\Isot\Rel_k\cong \Isot\Rel_k$ so that we get a dual purification result:


\begin{corollary}
The prop $\Co\Isot\Rel_k$ of affine coisotropic relations is generated by adding the doubled discard relation to the image of the embedding $\Lag\Rel_k\to\LinRel_k$, ie. the linear relation
$
\begin{tikzpicture}[yscale=-1]
	\begin{pgfonlayer}{nodelayer}
		\node [style=Z] (0) at (0, 0) {};
		\node [style=Z] (1) at (0.5, 0) {};
		\node [style=none] (2) at (0, 0.5) {};
		\node [style=none] (3) at (0.5, 0.5) {};
	\end{pgfonlayer}
	\begin{pgfonlayer}{edgelayer}
		\draw (1.center) to (3.center);
		\draw (0.center) to (2.center);
	\end{pgfonlayer}
\end{tikzpicture}
$

\end{corollary}


From the same argument that yields $\Aff\Lag\Rel_k$ from $\Lag\Rel_k$ in \cite{lagrel}:

\begin{lemma}
The props $\Aff\Isot\Rel_k$ and $\Aff\Co\Isot\Rel_k$ are generated by adding the generator $X$ to $\Isot\Rel_k$ and $\Co\Isot\Rel_k$ respectively seen as categories of affine relations with trivial affine shift.
\end{lemma}

Unlike in the linear case, these two props are not isomorphic, as the doubled discard and  doubled cozero maps interact differently with the $X$ gate.  For example:

$$
\begin{tikzpicture}
	\begin{pgfonlayer}{nodelayer}
		\node [style=X] (0) at (1.75, -0.75) {$1$};
		\node [style=Z] (3) at (1.75, 0) {};
		\node [style=Z] (4) at (0.75, 0) {};
		\node [style=none] (9) at (1.75, -1.5) {};
		\node [style=none] (10) at (0.75, -1.5) {};
	\end{pgfonlayer}
	\begin{pgfonlayer}{edgelayer}
		\draw (0) to (9.center);
		\draw (0) to (3);
		\draw (10.center) to (4);
	\end{pgfonlayer}
\end{tikzpicture}
=
\begin{tikzpicture}
	\begin{pgfonlayer}{nodelayer}
		\node [style=Z] (3) at (1.75, 0) {};
		\node [style=Z] (4) at (0.75, 0) {};
		\node [style=none] (9) at (1.75, -1.5) {};
		\node [style=none] (10) at (0.75, -1.5) {};
	\end{pgfonlayer}
	\begin{pgfonlayer}{edgelayer}
		\draw (10.center) to (4);
		\draw (9.center) to (3);
	\end{pgfonlayer}
\end{tikzpicture}
\hspace*{.5cm}
\text{but}
\hspace*{.5cm}
\begin{tikzpicture}
	\begin{pgfonlayer}{nodelayer}
		\node [style=X] (0) at (1.75, -0.75) {$1$};
		\node [style=X] (3) at (1.75, 0) {};
		\node [style=X] (4) at (0.75, 0) {};
		\node [style=none] (9) at (1.75, -1.5) {};
		\node [style=none] (10) at (0.75, -1.5) {};
	\end{pgfonlayer}
	\begin{pgfonlayer}{edgelayer}
		\draw (0) to (9.center);
		\draw (0) to (3);
		\draw (10.center) to (4);
	\end{pgfonlayer}
\end{tikzpicture}
\neq
\begin{tikzpicture}
	\begin{pgfonlayer}{nodelayer}
		\node [style=X] (3) at (1.75, 0) {};
		\node [style=X] (4) at (0.75, 0) {};
		\node [style=none] (9) at (1.75, -1.5) {};
		\node [style=none] (10) at (0.75, -1.5) {};
	\end{pgfonlayer}
	\begin{pgfonlayer}{edgelayer}
		\draw (10.center) to (4);
		\draw (9.center) to (3);
	\end{pgfonlayer}
\end{tikzpicture}
$$



TODO INTRODUCE EARLIER WHERE WE USE IT FOR SECOND THEOREM 
\begin{definition}
Given a field $k$ and natural number $n$, the {\bf symplectic Pauli group} on $n$ wires, $P_k^n$ is generated by the following affine Lagrangian relations under tensor product, for $z,x \in k$:

$$
W(a,b):=
\begin{tikzpicture}
	\begin{pgfonlayer}{nodelayer}
		\node [style=X] (0) at (0, 1) {$a$};
		\node [style=none] (1) at (0, 2) {};
		\node [style=none] (2) at (0, 0) {};
		\node [style=none] (3) at (1, 2) {};
		\node [style=none] (4) at (1, 0) {};
		\node [style=X] (5) at (1, 1) {$b$};
	\end{pgfonlayer}
	\begin{pgfonlayer}{edgelayer}
		\draw (4.center) to (5.center);
		\draw (5.center) to (3.center);
		\draw (1.center) to (0.center);
		\draw (0.center) to (2.center);
	\end{pgfonlayer}
\end{tikzpicture}
$$

Call elements of the generalized Pauli group {\bf symplectic Weyl operators}.
Given a tuple $((z_1,\cdots, z_n),(x_1,\cdots, x_n)) \in (k^n)^2$ we can similarly define a Weyl operator $W((z_1,\cdots, z_n),(x_1,\cdots, x_n) )= W(z_1,x_1)\oplus \cdots \oplus W(z_n,x_n)$.
\end{definition}

%Note that $\bar{W(z,x)} = W(-z,x)$


\begin{definition}
Given some state $f:0\to n$ in $\CPM(\Aff\Lag\Rel_k)$, the {\bf symplectic stabilizer group} of $f$ is the subgroup of the generalized Pauli group generated by the Weyl operators $a \in P_k^n$ so that:% $U(f);(1_n\otimes a) = U(f)$, where $U:\CPM(\Aff\Lag\Rel_k) \to \Aff\Lag\Rel_k$ is the forgetful functor. 

%Graphically for a dilation $g$ of $f$, then $a$ is a stabilizer of $f$ when:
$$
\begin{tikzpicture}
	\begin{pgfonlayer}{nodelayer}
		\node [style=none] (0) at (3.75, 3.25) {};
		\node [style=none] (1) at (2.25, 3.25) {};
		\node [style=map] (2) at (3, 2.25) {$f$};
		\node [style=map] (3) at (2.25, 3.25) {$a$};
		\node [style=none] (4) at (2.25, 3.75) {};
		\node [style=none] (5) at (3.75, 3.75) {};
	\end{pgfonlayer}
	\begin{pgfonlayer}{edgelayer}
		\draw [in=-90, out=135] (2) to (1.center);
		\draw [in=-90, out=45] (2) to (0.center);
		\draw (0.center) to (5.center);
		\draw (3) to (4.center);
	\end{pgfonlayer}
\end{tikzpicture}
=
\begin{tikzpicture}
	\begin{pgfonlayer}{nodelayer}
		\node [style=map] (0) at (2.25, 3.25) {$a$};
		\node [style=none] (1) at (3.75, 3.25) {};
		\node [style=Z] (4) at (3.25, 2.75) {};
		\node [style=map] (5) at (2.5, 2) {$g$};
		\node [style=map] (6) at (4, 2) {$\bar g$};
		\node [style=none] (7) at (3.75, 3.25) {};
		\node [style=none] (8) at (2.25, 3.75) {};
		\node [style=none] (9) at (3.75, 3.75) {};
	\end{pgfonlayer}
	\begin{pgfonlayer}{edgelayer}
		\draw [bend left] (5) to (4);
		\draw [in=60, out=0, looseness=1.50] (4) to (6);
		\draw [in=-90, out=120] (6) to (1.center);
		\draw [in=270, out=120] (5) to (0.center);
		\draw (7) to (9.center);
		\draw (0.center) to (8.center);
	\end{pgfonlayer}
\end{tikzpicture}
=
\begin{tikzpicture}
	\begin{pgfonlayer}{nodelayer}
		\node [style=none] (0) at (2.25, 3.25) {};
		\node [style=none] (1) at (3.75, 3.25) {};
		\node [style=Z] (4) at (3.25, 2.75) {};
		\node [style=map] (5) at (2.5, 2) {$g$};
		\node [style=map] (6) at (4, 2) {$\bar g$};
	\end{pgfonlayer}
	\begin{pgfonlayer}{edgelayer}
		\draw [bend left] (5) to (4);
		\draw [in=60, out=0, looseness=1.50] (4) to (6);
		\draw [in=-90, out=120] (6) to (1.center);
		\draw [in=270, out=120] (5) to (0.center);
	\end{pgfonlayer}
\end{tikzpicture}
=
\begin{tikzpicture}
	\begin{pgfonlayer}{nodelayer}
		\node [style=none] (0) at (2.25, 3.25) {};
		\node [style=none] (1) at (3.75, 3.25) {};
		\node [style=map] (2) at (3, 2.25) {$f$};
	\end{pgfonlayer}
	\begin{pgfonlayer}{edgelayer}
		\draw [in=-90, out=45] (2) to (1.center);
		\draw [in=-90, out=135] (2) to (0.center);
	\end{pgfonlayer}
\end{tikzpicture}
$$


\end{definition}

%
%\begin{lemma}
%\label{lem:cpmstab}
%Given a an affine Lagrangian dilation $(m,f:0\to n +m)$ of $\hat f:0\to n$ in $\CPM(\Aff\Lag\Rel_k)$,   $W(z,x)$ in the stabilizer group of $\hat f$ if and only if $((z,0),(x,0))\in f$.
%\end{lemma}
%
%\begin{proof}
%The converse is trivial.  For the forward direction, take some state $f$ as above.  If $\hat f$ has no stabilizer group (because it is empty), then the claim follows vacuously. 
%Take some  $W(a,b) \in P_k^n$ in the stabilizer group of $\hat f$.
%Then there exists some $W(c,d) \in P_k^m$ such that:
%
%
%$$
%\begin{tikzpicture}
%	\begin{pgfonlayer}{nodelayer}
%		\node [style=map] (0) at (2, 2.25) {$f$};
%		\node [style=none] (6) at (1, 3) {};
%		\node [style=none] (7) at (1.5, 3) {};
%		\node [style=none] (8) at (2.5, 3) {};
%		\node [style=none] (9) at (3, 3) {};
%		\node [style=none] (10) at (1, 4.5) {};
%		\node [style=none] (11) at (1.5, 4.5) {};
%		\node [style=none] (12) at (2.5, 4.5) {};
%		\node [style=none] (13) at (3, 4.5) {};
%		\node [style=X] (14) at (1, 3.75) {$a$};
%		\node [style=X] (15) at (1.5, 3.75) {$b$};
%	\end{pgfonlayer}
%	\begin{pgfonlayer}{edgelayer}
%		\draw [in=135, out=-90] (6.center) to (0);
%		\draw [in=-90, out=105] (0) to (7.center);
%		\draw [in=-90, out=75] (0) to (8.center);
%		\draw [in=45, out=-90] (9.center) to (0);
%		\draw (6.center) to (10.center);
%		\draw (7.center) to (11.center);
%		\draw (8.center) to (12.center);
%		\draw (9.center) to (13.center);
%	\end{pgfonlayer}
%\end{tikzpicture}
%=
%\begin{tikzpicture}
%	\begin{pgfonlayer}{nodelayer}
%		\node [style=map] (0) at (2, 2.25) {$f$};
%		\node [style=map] (1) at (2, 3.75) {$W(a,b)+ 1_m$};
%		\node [style=none] (6) at (1, 3) {};
%		\node [style=none] (7) at (1.5, 3) {};
%		\node [style=none] (8) at (2.5, 3) {};
%		\node [style=none] (9) at (3, 3) {};
%		\node [style=none] (10) at (1, 4.5) {};
%		\node [style=none] (11) at (1.5, 4.5) {};
%		\node [style=none] (12) at (2.5, 4.5) {};
%		\node [style=none] (13) at (3, 4.5) {};
%	\end{pgfonlayer}
%	\begin{pgfonlayer}{edgelayer}
%		\draw [in=135, out=-90] (6.center) to (0);
%		\draw [in=-90, out=105] (0) to (7.center);
%		\draw [in=-90, out=75] (0) to (8.center);
%		\draw [in=45, out=-90] (9.center) to (0);
%		\draw (6.center) to (10.center);
%		\draw (7.center) to (11.center);
%		\draw (8.center) to (12.center);
%		\draw (9.center) to (13.center);
%	\end{pgfonlayer}
%\end{tikzpicture}
%=
%\begin{tikzpicture}
%	\begin{pgfonlayer}{nodelayer}
%		\node [style=map] (0) at (2, 2.25) {$f$};
%		\node [style=map] (1) at (2, 3.75) {$1_n+W(c,d)$};
%		\node [style=none] (6) at (1, 3) {};
%		\node [style=none] (7) at (1.5, 3) {};
%		\node [style=none] (8) at (2.5, 3) {};
%		\node [style=none] (9) at (3, 3) {};
%		\node [style=none] (10) at (1, 4.5) {};
%		\node [style=none] (11) at (1.5, 4.5) {};
%		\node [style=none] (12) at (2.5, 4.5) {};
%		\node [style=none] (13) at (3, 4.5) {};
%	\end{pgfonlayer}
%	\begin{pgfonlayer}{edgelayer}
%		\draw [in=135, out=-90] (6.center) to (0);
%		\draw [in=-90, out=105] (0) to (7.center);
%		\draw [in=-90, out=75] (0) to (8.center);
%		\draw [in=45, out=-90] (9.center) to (0);
%		\draw (6.center) to (10.center);
%		\draw (7.center) to (11.center);
%		\draw (8.center) to (12.center);
%		\draw (9.center) to (13.center);
%	\end{pgfonlayer}
%\end{tikzpicture}
%=
%\begin{tikzpicture}
%	\begin{pgfonlayer}{nodelayer}
%		\node [style=map] (0) at (2, 2.25) {$f$};
%		\node [style=none] (6) at (1, 3) {};
%		\node [style=none] (7) at (1.5, 3) {};
%		\node [style=none] (8) at (2.5, 3) {};
%		\node [style=none] (9) at (3, 3) {};
%		\node [style=none] (10) at (1, 4.5) {};
%		\node [style=none] (11) at (1.5, 4.5) {};
%		\node [style=none] (12) at (2.5, 4.5) {};
%		\node [style=none] (13) at (3, 4.5) {};
%		\node [style=X] (14) at (2.5, 3.75) {$c$};
%		\node [style=X] (15) at (3, 3.75) {$d$};
%	\end{pgfonlayer}
%	\begin{pgfonlayer}{edgelayer}
%		\draw [in=135, out=-90] (6.center) to (0);
%		\draw [in=-90, out=105] (0) to (7.center);
%		\draw [in=-90, out=75] (0) to (8.center);
%		\draw [in=45, out=-90] (9.center) to (0);
%		\draw (6.center) to (10.center);
%		\draw (7.center) to (11.center);
%		\draw (8.center) to (12.center);
%		\draw (9.center) to (13.center);
%	\end{pgfonlayer}
%\end{tikzpicture}
%$$
%
%
%
%Let $((Z_A,Z_B),(X_A,X_B))$ be the affine shift of  $f$, %  Then $f;W((z_A,x_B),(z_A,x_B)));d_{n+m}=1_0$.
%%Pick some element $((x_A,x_B),(z_A,z_B))$ in $f$ so that $f;W((x_A,x_B),(z_A,z_B)));d_{n+m}=1_0$.
%then:
%$$
%\begin{tikzpicture}
%	\begin{pgfonlayer}{nodelayer}
%		\node [style=map] (0) at (2, 2.25) {$f$};
%		\node [style=map] (1) at (2, 3.75) {$W((Z_A,Z_B),(X_A,X_B))^{-1}$};
%		\node [style=X] (2) at (1, 4.5) {};
%		\node [style=X] (3) at (1.5, 4.5) {};
%		\node [style=X] (4) at (2.5, 4.5) {};
%		\node [style=X] (5) at (3, 4.5) {};
%		\node [style=none] (6) at (1, 3) {};
%		\node [style=none] (7) at (1.5, 3) {};
%		\node [style=none] (8) at (2.5, 3) {};
%		\node [style=none] (9) at (3, 3) {};
%	\end{pgfonlayer}
%	\begin{pgfonlayer}{edgelayer}
%		\draw [in=135, out=-90] (6.center) to (0);
%		\draw [in=-90, out=105] (0) to (7.center);
%		\draw [in=-90, out=75] (0) to (8.center);
%		\draw [in=45, out=-90] (9.center) to (0);
%		\draw (6.center) to (2);
%		\draw (7.center) to (3);
%		\draw (8.center) to (4);
%		\draw (9.center) to (5);
%	\end{pgfonlayer}
%\end{tikzpicture}
%=
%\begin{tikzpicture}
%	\begin{pgfonlayer}{nodelayer}
%		\node [style=map] (0) at (2, 3) {$f$};
%		\node [style=X] (1) at (0.5, 4) {$-Z_A$};
%		\node [style=X] (2) at (1.5, 4) {$-Z_B$};
%		\node [style=X] (3) at (2.5, 4) {$-X_A$};
%		\node [style=X] (4) at (3.5, 4) {$-X_B$};
%		\node [style=none] (5) at (0.5, 4) {};
%		\node [style=none] (6) at (1.5, 4) {};
%		\node [style=none] (7) at (2.5, 4) {};
%		\node [style=none] (8) at (3.5, 4) {};
%	\end{pgfonlayer}
%	\begin{pgfonlayer}{edgelayer}
%		\draw [in=150, out=-90] (5.center) to (0);
%		\draw [in=-90, out=120] (0) to (6.center);
%		\draw [in=-90, out=60] (0) to (7.center);
%		\draw [in=30, out=-90] (8.center) to (0);
%	\end{pgfonlayer}
%\end{tikzpicture}
%=
%1_0
%$$
%
%Therefore $((Z_A,X_A),(-Z_A,X_A))$ is the affine shift of $\hat f$, so that:
%
%\begin{align*}
%1_0
%&=
%\begin{tikzpicture}
%	\begin{pgfonlayer}{nodelayer}
%		\node [style=map] (0) at (2.75, 2.25) {$f$};
%		\node [style=none] (6) at (1.75, 4.25) {};
%		\node [style=none] (7) at (2.5, 4.25) {};
%		\node [style=map] (10) at (4.5, 2.25) {$\bar f$};
%		\node [style=X] (15) at (3, 3.75) {};
%		\node [style=Z] (16) at (4.75, 3.75) {};
%		\node [style=X] (17) at (1.75, 4.25) {$-Z_A$};
%		\node [style=X] (18) at (2.5, 4.25) {$-X_A$};
%		\node [style=X] (21) at (3.5, 4.25) {$Z_A$};
%		\node [style=X] (22) at (4.25, 4.25) {$-X_A$};
%	\end{pgfonlayer}
%	\begin{pgfonlayer}{edgelayer}
%		\draw [in=135, out=-90] (6.center) to (0);
%		\draw [in=-90, out=60] (0) to (7.center);
%		\draw [in=-165, out=105, looseness=1.25] (0) to (15);
%		\draw [in=315, out=30, looseness=1.50] (10) to (16);
%		\draw [in=270, out=75] (10) to (22);
%		\draw [in=-90, out=150, looseness=0.75] (10) to (21);
%		\draw [in=-15, out=105] (10) to (15);
%		\draw [in=-165, out=30, looseness=1.25] (0) to (16);
%	\end{pgfonlayer}
%\end{tikzpicture}
%=
%\begin{tikzpicture}
%	\begin{pgfonlayer}{nodelayer}
%		\node [style=map] (0) at (2.75, 2.25) {$f$};
%		\node [style=map] (10) at (4.5, 2.25) {$\bar f$};
%		\node [style=X] (15) at (3, 3.75) {};
%		\node [style=Z] (16) at (4.75, 3.75) {};
%		\node [style=X] (21) at (3.5, 4.25) {$Z_A$};
%		\node [style=X] (22) at (4.25, 4.25) {$-X_A$};
%		\node [style=none] (23) at (1.75, 4.75) {};
%		\node [style=none] (24) at (2.5, 4.75) {};
%		\node [style=X] (25) at (1.75, 4.75) {$-Z_A$};
%		\node [style=X] (26) at (2.5, 4.75) {$-X_A$};
%		\node [style=none] (27) at (1.75, 4) {};
%		\node [style=none] (28) at (2.5, 4) {};
%		\node [style=X] (29) at (1.75, 4) {$a$};
%		\node [style=X] (30) at (2.5, 4) {$b$};
%	\end{pgfonlayer}
%	\begin{pgfonlayer}{edgelayer}
%		\draw [in=-165, out=105, looseness=1.25] (0) to (15);
%		\draw [in=315, out=30, looseness=1.50] (10) to (16);
%		\draw [in=270, out=75] (10) to (22);
%		\draw [in=-90, out=150, looseness=0.75] (10) to (21);
%		\draw [in=-15, out=105] (10) to (15);
%		\draw [in=-165, out=30, looseness=1.25] (0) to (16);
%		\draw [in=-90, out=60] (0) to (30);
%		\draw (30) to (26);
%		\draw [in=-90, out=135] (0) to (29);
%		\draw (29) to (25);
%	\end{pgfonlayer}
%\end{tikzpicture}
%=
%\begin{tikzpicture}
%	\begin{pgfonlayer}{nodelayer}
%		\node [style=map] (0) at (2.75, 2.25) {$f$};
%		\node [style=none] (6) at (1.75, 4.25) {};
%		\node [style=none] (7) at (2.5, 4.25) {};
%		\node [style=map] (10) at (4.5, 2.25) {$\bar f$};
%		\node [style=X] (15) at (3, 3.75) {};
%		\node [style=Z] (16) at (4.75, 3.75) {};
%		\node [style=X] (17) at (1.75, 4.25) {$-Z_A$};
%		\node [style=X] (18) at (2.5, 4.25) {$-X_A$};
%		\node [style=X] (21) at (3.5, 4.25) {$Z_A$};
%		\node [style=X] (22) at (4.25, 4.25) {$-X_A$};
%		\node [style=X] (23) at (2.5, 3) {$c$};
%		\node [style=X] (24) at (3.5, 3) {$d$};
%	\end{pgfonlayer}
%	\begin{pgfonlayer}{edgelayer}
%		\draw [in=135, out=-90] (6.center) to (0);
%		\draw [in=-90, out=60] (0) to (7.center);
%		\draw [in=315, out=30, looseness=1.50] (10) to (16);
%		\draw [in=270, out=75] (10) to (22);
%		\draw [in=-90, out=150, looseness=0.75] (10) to (21);
%		\draw [in=-15, out=105] (10) to (15);
%		\draw [in=225, out=30] (0) to (24);
%		\draw [in=45, out=-165] (16) to (24);
%		\draw [in=90, out=-150] (15) to (23);
%		\draw [in=-90, out=105] (0) to (23);
%	\end{pgfonlayer}
%\end{tikzpicture}\\
%&=
%\begin{tikzpicture}
%	\begin{pgfonlayer}{nodelayer}
%		\node [style=map] (0) at (2.75, 2.25) {$f$};
%		\node [style=none] (6) at (1.75, 4.25) {};
%		\node [style=none] (7) at (2.5, 4.25) {};
%		\node [style=map] (10) at (4.5, 2.25) {$\bar f$};
%		\node [style=X] (15) at (3, 3.75) {};
%		\node [style=Z] (16) at (4.75, 3.75) {};
%		\node [style=X] (17) at (1.75, 4.25) {$-Z_A$};
%		\node [style=X] (18) at (2.5, 4.25) {$-X_A$};
%		\node [style=X] (21) at (3.5, 4.25) {$Z_A$};
%		\node [style=X] (22) at (4.25, 4.25) {$-X_A$};
%		\node [style=X] (23) at (4.25, 3) {$c$};
%		\node [style=X] (24) at (5.25, 3) {$-d$};
%	\end{pgfonlayer}
%	\begin{pgfonlayer}{edgelayer}
%		\draw [in=135, out=-90] (6.center) to (0);
%		\draw [in=-90, out=60] (0) to (7.center);
%		\draw [in=-165, out=105, looseness=1.25] (0) to (15);
%		\draw [in=270, out=75] (10) to (22);
%		\draw [in=-90, out=150, looseness=0.75] (10) to (21);
%		\draw [in=-165, out=30, looseness=1.25] (0) to (16);
%		\draw (10) to (23);
%		\draw [in=0, out=105, looseness=0.75] (23) to (15);
%		\draw [in=-90, out=30] (10) to (24);
%		\draw [in=90, out=-30] (16) to (24);
%	\end{pgfonlayer}
%\end{tikzpicture}
%=
%\begin{tikzpicture}
%	\begin{pgfonlayer}{nodelayer}
%		\node [style=X] (29) at (7.25, 3.75) {};
%		\node [style=Z] (30) at (8.5, 3.75) {};
%		\node [style=X] (31) at (6.5, 3) {$Z_B$};
%		\node [style=X] (32) at (7.5, 3) {$X_B$};
%		\node [style=X] (33) at (8.5, 2.25) {$-Z_B$};
%		\node [style=X] (34) at (9.5, 2.25) {$X_B$};
%		\node [style=X] (35) at (8.5, 3) {$c$};
%		\node [style=X] (36) at (9.5, 3) {$-d$};
%	\end{pgfonlayer}
%	\begin{pgfonlayer}{edgelayer}
%		\draw [in=0, out=105] (35) to (29);
%		\draw [in=90, out=0] (30) to (36);
%		\draw [in=-165, out=90] (31) to (29);
%		\draw [in=180, out=90] (32) to (30);
%		\draw (34) to (36);
%		\draw (33) to (35);
%	\end{pgfonlayer}
%\end{tikzpicture}
%=
%\begin{tikzpicture}
%	\begin{pgfonlayer}{nodelayer}
%		\node [style=X] (35) at (8.5, 3) {$c$};
%		\node [style=X] (36) at (9.5, 3) {$-d$};
%	\end{pgfonlayer}
%\end{tikzpicture}
%\end{align*}
%
%Which is true if and only if $c=-d=d=0$ if and only if $W(c,d)=1$.
%%
%%
%%Take $U:\CPM(\Aff\Lag\Rel_k) \to \Aff\Lag\Rel_k$ to be the forgetful functor.  Then
%%
%%\begin{align*}
%%1 &=U( \CPM(f);(1_n \oplus !_m); \CPM(W(x_A,z_A);d_{n}) )\\
%%   &=U( \CPM(f);(1_n \oplus !_m); (1_n\oplus W(a,b)) ; \CPM(W(x_A,z_A);d_{n}) )\\
%%   &=U( \CPM(f);(1_n \oplus !_m); (1_n\oplus W(a,b)) ; \CPM(W(x_A,z_A);d_{n}) )\\
%%   &=  (\bar f \oplus f);(1_n \oplus (1\oplus W(c,d )  ;\eta_m) \oplus 1_n ); \bar {W(x_A,z_A)};d_{n} \oplus W(x_A,z_A);d_{n}\\
%%   &=  d_m^T;W(x_B,-z_B)\oplus d_m^T;W(x_B,z_B)   ; (1\oplus W(c,d) )  ;\eta_m\\
%%   &= d_m^T;W(x_B,-z_B);W(c,d);W(-x_B,z_B);d_m\\
%%   &= d_m^T;W(c,d);d_m\\
%%\end{align*}
%%
%%Which is true if and only if $W(c,d)=1$.
%
%\end{proof}

This is needed to prove the essential uniqueness of purification for $\CPM(\Aff\Lag\Rel_k)$, generalizing the case for stabilizers:

\begin{proposition}[Essential uniqueness of quantum purification]~\\
\label{prop:uniqueness}
States in $\CPM(\Aff\Lag\Rel_k)$ are uniquely determined by their stabilizer groups.
\end{proposition}
\begin{proof}
%By construction, the stabilizer group of a state is unique.

Take affine Lagrangian dilations $(m,f:0\to n+m)$ of $\hat f:0\to m$ in $\CPM(\Aff\Lag\Rel_k)$ and $(g:0\to n+\ell)$ of $\hat g:0\to m$ in $\CPM(\Aff\Lag\Rel_k)$, so that $\hat f$ and $\hat g$ have the same stabilizer groups. Without loss of generality take $m\geq \ell$. %From the previous result:
%
%$$
% ((Z_A, 0 ),(X_A,0))  \in f \iff
%  ((Z_A, 0 ),(X_A,0))  \in g
%$$
%


%
%Then $f$ and $g$ are generated respectively by the following affine subspaces:
%
%$$
%\left[ Z_n Z_m | X_n X_m \right] + [a_n a_m]
%\hspace*{1cm}
%\left[ Z_n Z_\ell  | X_n X_\ell \right] + [a_n a_\ell]
%$$

%Then there are isometries $u:m\to m$, $v:\ell \to m$ such that
Consider the unitary  $u:m\to m$  and isometry $v:\ell\to m$, which perform Gaussian elimination on the ancillary space of $f$ and $g$.  Compose $u$ and $v$ on the ancillary spaces of $f$ and $g$, respectively; and then regard them as pure states.  These pure states both have the same stabilizer groups.  Therefore they span the same affine subspace:

$$
\begin{tikzpicture}
	\begin{pgfonlayer}{nodelayer}
		\node [style=map] (0) at (2.5, 2.25) {$f$};
		\node [style=none] (1) at (1.75, 4.25) {};
		\node [style=none] (2) at (2.75, 4.25) {};
		\node [style=map] (3) at (2.5, 3.25) {$u$};
		\node [style=none] (4) at (2.25, 4.25) {};
		\node [style=none] (5) at (3.25, 4.25) {};
	\end{pgfonlayer}
	\begin{pgfonlayer}{edgelayer}
		\draw [in=135, out=-90] (1.center) to (0);
		\draw [in=-90, out=60] (0) to (2.center);
		\draw [in=255, out=105] (0) to (3);
		\draw [in=-90, out=120] (3) to (4.center);
		\draw [in=-45, out=30, looseness=1.25] (0) to (3);
		\draw [in=75, out=-90] (5.center) to (3);
	\end{pgfonlayer}
\end{tikzpicture}
=
\begin{tikzpicture}
	\begin{pgfonlayer}{nodelayer}
		\node [style=map] (0) at (2.5, 2.25) {$g$};
		\node [style=none] (1) at (1.75, 4.25) {};
		\node [style=none] (2) at (2.75, 4.25) {};
		\node [style=map] (3) at (2.5, 3.25) {$v$};
		\node [style=none] (4) at (2.25, 4.25) {};
		\node [style=none] (5) at (3.25, 4.25) {};
	\end{pgfonlayer}
	\begin{pgfonlayer}{edgelayer}
		\draw [in=135, out=-90] (1.center) to (0);
		\draw [in=-90, out=60] (0) to (2.center);
		\draw [in=255, out=105] (0) to (3);
		\draw [in=-90, out=120] (3) to (4.center);
		\draw [in=-45, out=30, looseness=1.25] (0) to (3);
		\draw [in=75, out=-90] (5.center) to (3);
	\end{pgfonlayer}
\end{tikzpicture}
$$


Therefore:
%
%$$
% \CPM (f);(1_n \oplus !_m) 
%=\CPM (f;(1_n\oplus u) );(1_n \oplus !_m) 
%=\CPM (g;(1\oplus v) );(1_n \oplus !_m)
%= \CPM (g);(1_n \oplus !_\ell) 
%$$


$$
\begin{tikzpicture}
	\begin{pgfonlayer}{nodelayer}
		\node [style=map] (0) at (2.75, 2.25) {$f$};
		\node [style=none] (6) at (1.75, 4.25) {};
		\node [style=none] (7) at (2.5, 4.25) {};
		\node [style=map] (10) at (4.5, 2.25) {$\bar f$};
		\node [style=X] (15) at (3, 3.75) {};
		\node [style=Z] (16) at (4.75, 3.75) {};
		\node [style=none] (17) at (1.75, 4.25) {};
		\node [style=none] (18) at (2.5, 4.25) {};
		\node [style=none] (21) at (3.5, 4.25) {};
		\node [style=none] (22) at (4.25, 4.25) {};
	\end{pgfonlayer}
	\begin{pgfonlayer}{edgelayer}
		\draw [in=135, out=-90] (6.center) to (0);
		\draw [in=-90, out=60] (0) to (7.center);
		\draw [in=-165, out=105, looseness=1.25] (0) to (15);
		\draw [in=315, out=30, looseness=1.50] (10) to (16);
		\draw [in=270, out=75] (10) to (22);
		\draw [in=-90, out=150, looseness=0.75] (10) to (21);
		\draw [in=-15, out=105] (10) to (15);
		\draw [in=-165, out=30, looseness=1.25] (0) to (16);
	\end{pgfonlayer}
\end{tikzpicture}
=
\begin{tikzpicture}
	\begin{pgfonlayer}{nodelayer}
		\node [style=map] (10) at (6.75, 2.25) {$f$};
		\node [style=none] (11) at (5.5, 3.5) {};
		\node [style=none] (12) at (6, 3.5) {};
		\node [style=map] (13) at (8.5, 2.25) {$\bar f$};
		\node [style=X] (14) at (6.5, 4.25) {};
		\node [style=Z] (15) at (8.75, 4.25) {};
		\node [style=none] (18) at (7.75, 4.75) {};
		\node [style=none] (19) at (8.25, 4.75) {};
		\node [style=map] (20) at (6.5, 3.5) {$u$};
		\node [style=map] (21) at (9, 3.25) {$\bar u$};
		\node [style=none] (22) at (5.5, 4.75) {};
		\node [style=none] (23) at (6, 4.75) {};
	\end{pgfonlayer}
	\begin{pgfonlayer}{edgelayer}
		\draw [in=135, out=-90] (11.center) to (10);
		\draw [in=-90, out=60] (10) to (12.center);
		\draw [in=270, out=75] (13) to (19.center);
		\draw [in=-90, out=150, looseness=0.75] (13) to (18.center);
		\draw [in=195, out=105, looseness=1.25] (13) to (21);
		\draw [bend left, looseness=0.75] (21) to (13);
		\draw [bend right=45] (21) to (15);
		\draw [in=0, out=120, looseness=0.75] (21) to (14);
		\draw [in=60, out=180, looseness=0.50] (15) to (20);
		\draw [bend right=15] (20) to (10);
		\draw [in=330, out=45, looseness=1.25] (10) to (20);
		\draw [bend left] (20) to (14);
		\draw (12.center) to (23.center);
		\draw (11.center) to (22.center);
	\end{pgfonlayer}
\end{tikzpicture}
=
\begin{tikzpicture}
	\begin{pgfonlayer}{nodelayer}
		\node [style=map] (10) at (6.75, 2.25) {$g$};
		\node [style=none] (11) at (5.5, 3.5) {};
		\node [style=none] (12) at (6, 3.5) {};
		\node [style=map] (13) at (8.5, 2.25) {$\bar g$};
		\node [style=X] (14) at (6.5, 4.25) {};
		\node [style=Z] (15) at (8.75, 4.25) {};
		\node [style=none] (18) at (7.75, 4.75) {};
		\node [style=none] (19) at (8.25, 4.75) {};
		\node [style=map] (20) at (6.5, 3.5) {$v$};
		\node [style=map] (21) at (9, 3.25) {$\bar v$};
		\node [style=none] (22) at (5.5, 4.75) {};
		\node [style=none] (23) at (6, 4.75) {};
	\end{pgfonlayer}
	\begin{pgfonlayer}{edgelayer}
		\draw [in=135, out=-90] (11.center) to (10);
		\draw [in=-90, out=60] (10) to (12.center);
		\draw [in=270, out=75] (13) to (19.center);
		\draw [in=-90, out=150, looseness=0.75] (13) to (18.center);
		\draw [in=195, out=105, looseness=1.25] (13) to (21);
		\draw [bend left, looseness=0.75] (21) to (13);
		\draw [bend right=45] (21) to (15);
		\draw [in=0, out=120, looseness=0.75] (21) to (14);
		\draw [in=60, out=180, looseness=0.50] (15) to (20);
		\draw [bend right=15] (20) to (10);
		\draw [in=330, out=45, looseness=1.25] (10) to (20);
		\draw [bend left] (20) to (14);
		\draw (12.center) to (23.center);
		\draw (11.center) to (22.center);
	\end{pgfonlayer}
\end{tikzpicture}
=
\begin{tikzpicture}
	\begin{pgfonlayer}{nodelayer}
		\node [style=map] (0) at (2.75, 2.25) {$g$};
		\node [style=none] (6) at (1.75, 4.25) {};
		\node [style=none] (7) at (2.5, 4.25) {};
		\node [style=map] (10) at (4.5, 2.25) {$\bar g$};
		\node [style=X] (15) at (3, 3.75) {};
		\node [style=Z] (16) at (4.75, 3.75) {};
		\node [style=none] (17) at (1.75, 4.25) {};
		\node [style=none] (18) at (2.5, 4.25) {};
		\node [style=none] (21) at (3.5, 4.25) {};
		\node [style=none] (22) at (4.25, 4.25) {};
	\end{pgfonlayer}
	\begin{pgfonlayer}{edgelayer}
		\draw [in=135, out=-90] (6.center) to (0);
		\draw [in=-90, out=60] (0) to (7.center);
		\draw [in=-165, out=105, looseness=1.25] (0) to (15);
		\draw [in=315, out=30, looseness=1.50] (10) to (16);
		\draw [in=270, out=75] (10) to (22);
		\draw [in=-90, out=150, looseness=0.75] (10) to (21);
		\draw [in=-15, out=105] (10) to (15);
		\draw [in=-165, out=30, looseness=1.25] (0) to (16);
	\end{pgfonlayer}
\end{tikzpicture}
$$

\end{proof}




\begin{theorem}[Essential uniqueness of relational purification]~\\
\label{them:dilation}
 $\CPM(\Aff\Lag\Rel_k) \cong \Aff\Co\Isot\Rel_k$
\end{theorem}


\begin{proof}

%%
%%
%%First note that $\Aff\Lag\Rel_k\to\CPM(\Aff\Lag\Rel_k)$ is faithful, so that when there are two $f,g:0\to n$ in $\Aff\Lag\Rel_k$,  such that for for all Weyl operators $x$,
%%
%%$$\CPM(f);(1_n \oplus x) = \CPM(f) \iff \CPM(g);(1_n \oplus x) = \CPM(g)$$
%%
%%Then we can conclude  that $\CPM(f) = \CPM(g)$.
%%That is to say, that the pure states in $\CPM(\Aff\Lag\Rel_k)$ are determined by their stabilizer groups.
%%We  show the same for mixed states.
%%
%%Suppose that we have two affine Lagrangian relations $f:0\to n +m$ and $g:0\to n+\ell$ such that for any Weyl operator $x$ on $n$:
%%
%%$$U(\CPM(f);(1_n \oplus !_m)); (1_n \oplus x) = U(\CPM(f);(1_n \oplus !_m))
%%\iff
%%U(\CPM(g);(1_n \oplus !_\ell)); (1_n \oplus x) = U(\CPM(g);(1_n \oplus !_\ell))$$
%%Where  $U:\CPM(\Aff\Lag\Rel_k)\to \Aff\Lag\Rel_k$ is the forgetful functor and $!_n$ is the discard map on $n$ wires.
%%
%%We seek to show that $ \CPM(f);(1_n \oplus !_m)= \CPM(g);(1_n \oplus !_\ell) $.
%%
%%First, we show for a Weyl operator $x$:
%%$$
%%f; (x \oplus 1_m) = f
%%\iff
%%U(\CPM(f);(1_n \oplus !_m)); (1_n \oplus x) = U(\CPM(f);(1_n \oplus !_m))
%%$$
%%
%%The foreward implication is immediate
%%
%%
%%%We need this lemma
%%%%For any stabilizer state $f:0\to n+m$, and Weyl operator $x$ on $n$ qudits, 
%%%%%Either there exists some Weyl operator $y$ on $m$ qudits so that $f;(x\oplus 1)=f;(1\oplus y)$
%%%%%Or f;(x\oplus 1)=0
%%%This follows from the following lemma
%%%%For any affine relation $f:0\to n+m$, either $f;(|x>|y>) = 1$ or it is zero.
%%
%%
%%%For the converse direction TODO, follows from quantum literature on mixed stabilizer states, projectors
%%%Suppose (x,z) is a stabilizer of f
%%%If (x,z) is a stabilizer of \bar f, show that z=z^{-1}, meaning that z=1
%%
%%
%%For the converse direction, suppose that there is a Weyl operator $x$ on $n$ qudits such that:
%%$$U(\CPM(f);(1_n \oplus !_m)); (1_n \oplus x) = U(\CPM(f);(1_n \oplus !_m))$$
%%
%%If $f$ is empty, then the claim follows immediately.  Suppose otherwise, that it is nonempty.
%%Then there exists another Weyl operator $y$ on $m$ qudits such that $f;(1_n \oplus x) = f;(y\oplus 1_m)$.
%%Therefore, %$\bar f ; (y \oplus 1_m)$, so that:
%%
%%$$
%%U(\CPM(f);(1_n \oplus !_m))=
%%U(\CPM(f);(1_n \oplus !_m)); (1_n \oplus x)=
%%(\bar f \oplus f);(1_n \oplus ((1\oplus y )  ;\eta_m) \oplus 1_n )
%%$$
%%
%%%Let $z: n \to 0$ denote the cozero relation and $\eta_m$ the counit of the compact closed structure.
%%%Because $\bar x \oplus x$  is a stabilizer, we know that $\bar z \oplus z$ is inside this linear subspace so that 
%%For $a,b \in k^n$, let $W(a,b)=(X^{\otimes n})^a;(Z^{\otimes n})^b$ and let $h$ denote the cozero relation.
%%
%%Then there are some $a,b,c,d \in k$ such that $W(a,b)=x$ and $W(c,d)=y$.  Thus,
%%
%%
%%%Then we have 
%%% Take some $t \in f;(1_ \oplus (x;d))$.
%%% Then there are some $e,f \in k$ such that $W(e,f);d = t^T$
%%
%%%%%%PROOF IS WRONG, USE ONE IN TITOUANES PAPER
%%%
%%%\begin{align*}
%%% 1=&  (\bar f \oplus f);(1_n \oplus ((1\oplus y )  ;\eta_m) \oplus 1_n ); \bar {W(a,b);h} \oplus {W(a,b)};h\\
%%%   =&  (\bar f \oplus f);(1_n \oplus ((1\oplus y )  ;\eta_m) \oplus 1_n );  W(a,-b);h \oplus {W(a,b)};h\\
%%%   =&h^T;W(-a,-b);W(c,d);W(a,b);h\\
%%%   =&h^T;W(c,d);h
%%%\end{align*}
%%
%%We have that
%%
%%$$
%%1=  (\bar f \oplus f);(1_n \oplus ((1\oplus y )  ;\eta_m) \oplus 1_n ); \bar {W(a,b);h} \oplus {W(a,b)};h
%%$$
%%
%%Thus since $(W(e,f);h)^T$ is in the subspace $f (1\oplus W(a,b) ;h)$, it follows that:
%%
%%
%%\begin{align*}
%%1  =& h^T;W(-e,-b);W(c,d);W(e,f);h\\
%%   =&h^T;W(c,d);h
%%\end{align*}
%%
%%
%%Which is true if and only if $y=W(c,d)=1$, otherwise, this would evaluate to the empty relation.
%%%% \bar{W(a,b)} = W(a,-b)
%%%
%%%\begin{align*}
%%% &1=  (\bar f \oplus f);(1_n \oplus ((1\oplus y )  ;\eta_m) \oplus 1_n ); \bar z \oplus z\\
%%%&\iff    1 = z^\dag;y;z\\
%%%&\iff    1=y
%%%\end{align*}
%%Therefore, the reverse implication holds.
%%
%%
%%%%%%%%%%%%%%%%%%%%%%%%%%%%%%%%%%%%%%%%%%%%%%%%%
%%% conjugation takes phases  weyl operators (n,m) -> (-n,-m)                                          %
%%% Therefore, weyl operators should commute with the cap, because they will cancel     %
%%%%%%%%%%%%%%%%%%%%%%%%%%%%%%%%%%%%%%%%%%%%%%%%%
%%
%%By substituting $f$ with $g$ it follows that:
%%Given affine Lagrangian relations $f:0\to n$, $g:0\to n$, suppose that $\CPM(f);(1_n \oplus !_m))$ and $\CPM(f);(1_n \oplus !_\ell))$ have then same stabilizer groups.  Then
%%
%%\begin{align*}
%%  f; (x \oplus 1_m) = f
%% \iff &
%%U(\CPM(f);(1_n \oplus !_m)); (1_n \oplus x) = U(\CPM(f);(1_n \oplus !_m))\\
%%\iff & U(\CPM(g);(1_n \oplus !_\ell)); (1_n \oplus x) = U(\CPM(g);(1_n \oplus !_\ell))\\
%%\iff & f; (x \oplus 1_\ell) = g
%%\end{align*}
%%
%%
%%Without loss of generality take $m\geq \ell$.
%%Then there are isometries $u:m\to m$, $v:\ell \to m$ such that
%%$$f(1_n\oplus u) = g (1_n\oplus v)$$
%%
%%Therefore,
%%
%%$$
%% \CPM (f);(1_n \oplus !_m) 
%%=\CPM (f;(1_n\oplus u) );(1_n \oplus !_m) 
%%=\CPM (g;(1\oplus v) );(1_n \oplus !_m)
%%= \CPM (g);(1_n \oplus !_\ell) 
%%$$
%
%%Which proves the initial biequivalence.
%
%
%%Define stabilizers for al
%We already know that states in $\Aff\Co\Isot\Rel_k$ and $\CPM(\Aff\Lag\Rel_k)$ are determined by their stabilizers.
%Therefore it suffices to show that the stabilizers in $\CPM(\Aff\Lag\Rel_k)$ and  $\Aff\Co\Isot\Rel_k$ agree.%; that is:
%
%
%%$$f;(x \oplus d_m)=f;(1_n \oplus d_m) \iff U(\CPM(f);(1_n \oplus !_m)); (1_n \oplus x) = U(\CPM(f);(1_n \oplus !_m)) $$
%
%
%If $f$ is empty, then the claim follows immediately.  Suppose otherwise.
%The forward direction follows immediately.
%Conversely, we know there is some Weyl operator $y$ such that  
%$$f;(x\oplus 1) =  f;(1\oplus y)\hspace*{.5cm} \text{but}\hspace*{.5cm}
%f;(x \oplus d_m) = f;(1_n \oplus (y;d_m)) = f;(1_n\oplus d_m)$$
%

Because both categories are compact closed, it suffices to exhibit a functorial bijection between the states of both categories.
Consider the map $\CPM(\Aff\Lag\Rel_k) \cong \Aff\Co\Isot\Rel_k$, sending:
$$
\begin{tikzpicture}
	\begin{pgfonlayer}{nodelayer}
		\node [style=map] (0) at (2.75, 2.25) {$f$};
		\node [style=none] (1) at (1.75, 4.25) {};
		\node [style=none] (2) at (2.5, 4.25) {};
		\node [style=map] (3) at (4.5, 2.25) {$\bar f$};
		\node [style=X] (4) at (3, 3.75) {};
		\node [style=Z] (5) at (4.75, 3.75) {};
		\node [style=none] (6) at (1.75, 4.25) {};
		\node [style=none] (7) at (2.5, 4.25) {};
		\node [style=none] (8) at (3.5, 4.25) {};
		\node [style=none] (9) at (4.25, 4.25) {};
	\end{pgfonlayer}
	\begin{pgfonlayer}{edgelayer}
		\draw [in=135, out=-90] (1.center) to (0);
		\draw [in=-90, out=60] (0) to (2.center);
		\draw [in=-165, out=105, looseness=1.25] (0) to (4);
		\draw [in=315, out=30, looseness=1.50] (3) to (5);
		\draw [in=270, out=75] (3) to (9.center);
		\draw [in=-90, out=150, looseness=0.75] (3) to (8.center);
		\draw [in=-15, out=105] (3) to (4);
		\draw [in=-165, out=30, looseness=1.25] (0) to (5);
	\end{pgfonlayer}
\end{tikzpicture}
\mapsto
\begin{tikzpicture}
	\begin{pgfonlayer}{nodelayer}
		\node [style=map] (0) at (2.5, 2.25) {$f$};
		\node [style=none] (1) at (1.75, 4.25) {};
		\node [style=none] (2) at (2.75, 4.25) {};
		\node [style=Z] (10) at (2.25, 3.5) {};
		\node [style=Z] (11) at (3.25, 3.5) {};
	\end{pgfonlayer}
	\begin{pgfonlayer}{edgelayer}
		\draw [in=135, out=-90] (1.center) to (0);
		\draw [in=-90, out=60] (0) to (2.center);
		\draw [in=-90, out=105] (0) to (10);
		\draw [in=-90, out=45] (0) to (11);
	\end{pgfonlayer}
\end{tikzpicture}
$$

%If $a$ is a stabilizer of the right hand side then it is a 
By Proposition \ref{prop:uniqueness}, this mapping is a full functor. For faithfulness, take maps $\hat f$ and $\hat g$ sent to the same affine coisotropic relation with dilations $(m,f:0\to n\oplus m)$ and  $(\ell,g:0\to n\oplus \ell )$ where $m\geq \ell$.
Then there is a unitary  $u:m\to m$  and isometry $v:\ell\to m$, which perform Gaussian elimination on the ancillary systems of $f$ and $g$ so that:

$$
\begin{tikzpicture}
	\begin{pgfonlayer}{nodelayer}
		\node [style=map] (0) at (2.5, 2.25) {$f$};
		\node [style=none] (1) at (1.75, 4.25) {};
		\node [style=none] (2) at (2.75, 4.25) {};
		\node [style=map] (3) at (2.5, 3.25) {$u$};
		\node [style=none] (4) at (2.25, 4.25) {};
		\node [style=none] (5) at (3.25, 4.25) {};
	\end{pgfonlayer}
	\begin{pgfonlayer}{edgelayer}
		\draw [in=135, out=-90] (1.center) to (0);
		\draw [in=-90, out=60] (0) to (2.center);
		\draw [in=255, out=105] (0) to (3);
		\draw [in=-90, out=120] (3) to (4.center);
		\draw [in=-45, out=30, looseness=1.25] (0) to (3);
		\draw [in=75, out=-90] (5.center) to (3);
	\end{pgfonlayer}
\end{tikzpicture}
=
\begin{tikzpicture}
	\begin{pgfonlayer}{nodelayer}
		\node [style=map] (0) at (2.5, 2.25) {$g$};
		\node [style=none] (1) at (1.75, 4.25) {};
		\node [style=none] (2) at (2.75, 4.25) {};
		\node [style=map] (3) at (2.5, 3.25) {$v$};
		\node [style=none] (4) at (2.25, 4.25) {};
		\node [style=none] (5) at (3.25, 4.25) {};
	\end{pgfonlayer}
	\begin{pgfonlayer}{edgelayer}
		\draw [in=135, out=-90] (1.center) to (0);
		\draw [in=-90, out=60] (0) to (2.center);
		\draw [in=255, out=105] (0) to (3);
		\draw [in=-90, out=120] (3) to (4.center);
		\draw [in=-45, out=30, looseness=1.25] (0) to (3);
		\draw [in=75, out=-90] (5.center) to (3);
	\end{pgfonlayer}
\end{tikzpicture}
$$

So that, as before, they are both dilations of $\hat f = \hat g$.
\end{proof}







% We can interpret the nonempty states in $\Aff\Co\Isot\Rel_k$ as stabilizer codes, but the formalism allows us to uncurry states into processes and compose them.  We will discuss this relationship with stabilizer codes in and error correction in further detail in the next section.

\begin{corollary}
\label{cor:stabcode}
For odd prime $p$, $\Aff\Co\Isot\Rel_{\F_p}\cong \CPM(\Aff\Lag\Rel_{\F_p})\cong \CPM(\Stab_p)$, that is, mixed stabilizer circuits modulo invertible scalars, ie {\bf stabilizer codes}.
\end{corollary}

This formalizes the relationship between mixed stabilizer circuits and stabilizer tableaus with not-necessarily-full rank in  a compositional way. This presentation is similar in spirit to the way in which adding quantum discarding can often be presented by adding a generator which freely discarding the isometries, formalized by the discard construction \cite{discard}. Although in our case the quantum discarding is interpreted as the literal discard relation, therefore our semantics is still in affine relations.



Every stabilizer code has an associated affine isotropic subspace and affine coisotropic subspace taking the symplectic complement of the linear component of the affine subspace. However, as remarked earlier, their compositions as affine relations are different. The interpretation of the doubled zero postselection as the quantum discard map is not sound with respect to the composition of stabilizer circuits. 
 Even though pure stabilizer states have larger stabilizer groups than mixed stabilizer states (stabilizer codes); the corresponding $\F_p$-linear subspaces have smaller dimensions.


\begin{corollary}
Moreover, $\Aff\Co\Isot\Rel_{\F_2}\cong \CPM(\Aff\Lag\Rel_{\F_2})$ is Spekkens' toy model with mixed states.
\end{corollary}

Recall for $p$ prime, $\Aff\Isot\Rel_{\F_p} \not \cong \Aff\Co\Isot\Rel_{\F_p}$. Therefore affine isotropic relations provides an epistemically corestricted {\em toy theory} which is neither Spekkens qubit toy model, nor qudit stabilizer quantum mechanics.

{\em Absolutely remarkably}, and seemingly out of nowhere, the odd prime $p$-dimensional qudit  stabilizer codes with trivial affine phase (no Pauli gates) can be expressed, modulo invertible phase, in terms of an iterated $\CPM$ construction with respect to the orthogonal complement at the inner level, and the complex conjugation at the outer level:
\begin{corollary}
For a prime number $p$, $\Isot\Rel_{\F_p}\cong\Co\Isot\Rel_{\F_p}\cong \CPM(\CPM(\LinRel_{\F_p},\perp),\bar{(\_)})$.
\end{corollary}
The astounding symmetry involved here begs the question if iterating the $\CPM$ construction more times yields anything physically interesting. Perhaps the work of \cite{CPMho} can shed some light on this question. 

In the quantum setting, by connecting the discard map to a spider, one obtains a circuit which decoheres the state into a basis.  In stabilizer circuits, discarding a white spider projects onto the $X$ basis, and discarding in the black spider projects onto  the $Z$ basis. 

\begin{definition}
The $X$ and $Z$ projectors are defined as follows in $\Aff\Co\Isot\Rel_{\F_p}$:
$$
p_X:=
\begin{tikzpicture}
	\begin{pgfonlayer}{nodelayer}
		\node [style=X] (0) at (0.5, -0.75) {};
		\node [style=none] (2) at (0.25, 0) {};
		\node [style=none] (4) at (1.25, 0.5) {};
		\node [style=Z] (5) at (0.75, 0) {};
		\node [style=Z] (6) at (1.75, 0) {};
		\node [style=Z] (7) at (1.5, -0.75) {};
		\node [style=none] (8) at (0.75, 0) {};
		\node [style=none] (9) at (1.25, 0) {};
		\node [style=none] (10) at (1.75, 0) {};
		\node [style=none] (11) at (0.5, -1.5) {};
		\node [style=none] (13) at (1.5, -1.5) {};
		\node [style=none] (14) at (0.25, 0.5) {};
	\end{pgfonlayer}
	\begin{pgfonlayer}{edgelayer}
		\draw [in=-90, out=120] (7) to (9.center);
		\draw (7) to (13.center);
		\draw [in=60, out=-90] (10.center) to (7);
		\draw [in=60, out=-90] (8.center) to (0);
		\draw [in=-90, out=120] (0) to (2.center);
		\draw (0) to (11.center);
		\draw (9.center) to (4.center);
		\draw (2.center) to (14.center);
	\end{pgfonlayer}
\end{tikzpicture}
=
\begin{tikzpicture}
	\begin{pgfonlayer}{nodelayer}
		\node [style=none] (4) at (1, 0.5) {};
		\node [style=Z] (5) at (0.25, 0) {};
		\node [style=none] (9) at (1, -1.25) {};
		\node [style=none] (11) at (0.25, -1.25) {};
		\node [style=none] (14) at (0.25, 0.5) {};
		\node [style=Z] (15) at (0.25, -0.75) {};
	\end{pgfonlayer}
	\begin{pgfonlayer}{edgelayer}
		\draw (9.center) to (4.center);
		\draw (14.center) to (5);
		\draw (11.center) to (15);
	\end{pgfonlayer}
\end{tikzpicture}
\hspace*{.5cm}
p_Z:=
\begin{tikzpicture}
	\begin{pgfonlayer}{nodelayer}
		\node [style=Z] (0) at (0.5, -0.75) {};
		\node [style=none] (2) at (0.25, 0) {};
		\node [style=none] (4) at (1.25, 0.5) {};
		\node [style=Z] (5) at (0.75, 0) {};
		\node [style=Z] (6) at (1.75, 0) {};
		\node [style=X] (7) at (1.5, -0.75) {};
		\node [style=none] (8) at (0.75, 0) {};
		\node [style=none] (9) at (1.25, 0) {};
		\node [style=none] (10) at (1.75, 0) {};
		\node [style=none] (11) at (0.5, -1.5) {};
		\node [style=none] (13) at (1.5, -1.5) {};
		\node [style=none] (14) at (0.25, 0.5) {};
	\end{pgfonlayer}
	\begin{pgfonlayer}{edgelayer}
		\draw [in=-90, out=120] (7) to (9.center);
		\draw (7) to (13.center);
		\draw [in=60, out=-90] (10.center) to (7);
		\draw [in=60, out=-90] (8.center) to (0);
		\draw [in=-90, out=120] (0) to (2.center);
		\draw (0) to (11.center);
		\draw (9.center) to (4.center);
		\draw (2.center) to (14.center);
	\end{pgfonlayer}
\end{tikzpicture}
=
\begin{tikzpicture}[scale=-1]
	\begin{pgfonlayer}{nodelayer}
		\node [style=none] (4) at (1, 0.5) {};
		\node [style=Z] (5) at (0.25, 0) {};
		\node [style=none] (9) at (1, -1.25) {};
		\node [style=none] (11) at (0.25, -1.25) {};
		\node [style=none] (14) at (0.25, 0.5) {};
		\node [style=Z] (15) at (0.25, -0.75) {};
	\end{pgfonlayer}
	\begin{pgfonlayer}{edgelayer}
		\draw (9.center) to (4.center);
		\draw (14.center) to (5);
		\draw (11.center) to (15);
	\end{pgfonlayer}
\end{tikzpicture}
$$
\end{definition}

%By splitting these idempotents, we can obtain a two-sorted semantics for which classical and quantum datum live together.
 
\begin{definition}
%Given a monoidal category $\C$ and a set of idempotents $P$ in $\C$ define the category $k_X(\C)$ to be the monoidal category generated by splitting the idempotents in $X$.
%
%Consider the prop $\Aff\Co\Isot\Rel_k^M:=k_{\{p_Z\}}(\Aff\Co\Isot\Rel_k)$.  Let $Q:=(1,1)$ $C:=(p_z,1)$ denote the generators for the multicoloured prop.
Let $\Aff\Co\Isot\Rel_k^M$ denote the two coloured prop generated by splitting $p_Z$ in $\Aff\Co\Isot\Rel_k$.
Let $Q=(1,1_1)$ denote the original generating object and $C=(1_,p_Z)$ the object obtained by splitting $p_Z$.
\end{definition}

We could have instead split $p_X$, or split both $p_X$ and $p_Z$; however, all three of these multicoloured props are equivalent.  This equivalence is witnessed via the Fourier transform. Indeed this suffices to split all nonzero projectors up to isomorphism because all projectors of the same dimension are isomorphic as affine coisotropic subspaces.  Therefore we can construct a nonzero projector of each possible dimension by composition with $p_X$ and affine symplectomorphisms.  We chose not to split the zero projector for the same reason why we did not chose to have it as an object in $\Aff\Rel_k$: for ease of notation.

\begin{remark}
The object $Q$ can be interpreted as a quantum channel and the object $C$ as a classical channel. Or equivalently, $C^{\otimes n}$ is interpreted as the space of logical qubits and  $Q^{\otimes m}$  as the space of physical qubits.
\end{remark}

This category has a nice presentation:

\begin{theorem}
The full subcategory of $\Aff\Co\Isot\Rel_k^M$ generated by tensor powers of $C$ is isomorphic to $\Aff\Rel_k$.
Thereforefore $\Aff\Co\Isot\Rel_k^M$ is isomorphic to adding the following linear relations to the image of $\Aff\Co\Isot\Rel_k^M\to \Aff\Co\Isot\Rel_k$ in the way which makes this into a two-coloured prop:
$$
\begin{tikzpicture}[xscale=-1]
	\begin{pgfonlayer}{nodelayer}
		\node [style=none] (4) at (0.25, 0.5) {};
		\node [style=Z] (5) at (1, 0) {};
		\node [style=none] (9) at (0.25, -0.5) {};
		\node [style=none] (14) at (1, 0.5) {};
	\end{pgfonlayer}
	\begin{pgfonlayer}{edgelayer}
		\draw (9.center) to (4.center);
		\draw (14.center) to (5);
	\end{pgfonlayer}
\end{tikzpicture}
\hspace*{.5cm}\text{and}\hspace*{.5cm}
\begin{tikzpicture}[scale=-1]
	\begin{pgfonlayer}{nodelayer}
		\node [style=none] (4) at (0.25, 0.5) {};
		\node [style=Z] (5) at (1, 0) {};
		\node [style=none] (9) at (0.25, -0.5) {};
		\node [style=none] (14) at (1, 0.5) {};
	\end{pgfonlayer}
	\begin{pgfonlayer}{edgelayer}
		\draw (9.center) to (4.center);
		\draw (14.center) to (5);
	\end{pgfonlayer}
\end{tikzpicture}
$$
\end{theorem}

 The classical state ``lives'' on a single wire and the stabilizer state ``lives'' on the doubled wires.
Because of this, the  aforementioned circuits are interpreted in terms of state preparation and measurement in the $Z$ basis. For example, given any dit $x \in \F_p$, to prepare the state $|x\rangle$ is to take the composite:

$$
\begin{tikzpicture}
	\begin{pgfonlayer}{nodelayer}
		\node [style=none] (0) at (1.25, 0.5) {};
		\node [style=Z] (1) at (0.5, 0) {};
		\node [style=none] (2) at (1.25, -0.5) {};
		\node [style=none] (3) at (0.5, 0.5) {};
		\node [style=X] (4) at (1.25, -0.5) {$x$};
	\end{pgfonlayer}
	\begin{pgfonlayer}{edgelayer}
		\draw (2.center) to (0.center);
		\draw (3.center) to (1);
	\end{pgfonlayer}
\end{tikzpicture}
=
\begin{tikzpicture}
	\begin{pgfonlayer}{nodelayer}
		\node [style=none] (0) at (1.25, 0.5) {};
		\node [style=Z] (1) at (0.5, 0) {};
		\node [style=none] (2) at (1.25, 0) {};
		\node [style=none] (3) at (0.5, 0.5) {};
		\node [style=X] (4) at (1.25, 0) {$x$};
	\end{pgfonlayer}
	\begin{pgfonlayer}{edgelayer}
		\draw (2.center) to (0.center);
		\draw (3.center) to (1);
	\end{pgfonlayer}
\end{tikzpicture}
$$

The state preparation and discarding in the $Z$ basis are obtained by composition of these morphisms with the Fourier transform; yielding morphisms which dicard the $X$ wire instead of the $Z$ wire:

$$
\begin{tikzpicture}
	\begin{pgfonlayer}{nodelayer}
		\node [style=none] (4) at (0.25, 0.5) {};
		\node [style=Z] (5) at (1, 0) {};
		\node [style=none] (9) at (0.25, -0.5) {};
		\node [style=none] (14) at (1, 0.5) {};
	\end{pgfonlayer}
	\begin{pgfonlayer}{edgelayer}
		\draw (9.center) to (4.center);
		\draw (14.center) to (5);
	\end{pgfonlayer}
\end{tikzpicture}
\hspace*{.5cm}\text{and}\hspace*{.5cm}
\begin{tikzpicture}[yscale=-1]
	\begin{pgfonlayer}{nodelayer}
		\node [style=none] (4) at (0.25, 0.5) {};
		\node [style=Z] (5) at (1, 0) {};
		\node [style=none] (9) at (0.25, -0.5) {};
		\node [style=none] (14) at (1, 0.5) {};
	\end{pgfonlayer}
	\begin{pgfonlayer}{edgelayer}
		\draw (9.center) to (4.center);
		\draw (14.center) to (5);
	\end{pgfonlayer}
\end{tikzpicture}
$$


\begin{remark}
In $\FHilb$, the state preparation and measurement in the $X$ and $Z$ bases are given by the following ZX-diagrams:

$$
\begin{tikzpicture}
	\begin{pgfonlayer}{nodelayer}
		\node [style=Z] (182) at (49, 6) {};
		\node [style=none] (183) at (48.75, 7) {};
		\node [style=Z] (184) at (49.5, 6.75) {};
		\node [style=none] (185) at (49, 5.25) {};
		\node [style=none] (186) at (49.75, 5.25) {};
	\end{pgfonlayer}
	\begin{pgfonlayer}{edgelayer}
		\draw [in=120, out=-90] (183.center) to (182);
		\draw (182) to (184);
		\draw (185.center) to (182);
		\draw [in=-60, out=90, looseness=0.75] (186.center) to (184);
	\end{pgfonlayer}
\end{tikzpicture},
\hspace*{.5cm}
\begin{tikzpicture}[yscale=-1]
	\begin{pgfonlayer}{nodelayer}
		\node [style=Z] (182) at (49, 6) {};
		\node [style=none] (183) at (48.75, 7) {};
		\node [style=Z] (184) at (49.5, 6.75) {};
		\node [style=none] (185) at (49, 5.25) {};
		\node [style=none] (186) at (49.75, 5.25) {};
	\end{pgfonlayer}
	\begin{pgfonlayer}{edgelayer}
		\draw [in=120, out=-90] (183.center) to (182);
		\draw (182) to (184);
		\draw (185.center) to (182);
		\draw [in=-60, out=90, looseness=0.75] (186.center) to (184);
	\end{pgfonlayer}
\end{tikzpicture},
\hspace*{.5cm}
\begin{tikzpicture}
	\begin{pgfonlayer}{nodelayer}
		\node [style=none] (188) at (50.75, 7) {};
		\node [style=Z] (189) at (51.5, 6.75) {};
		\node [style=none] (190) at (51, 5.25) {};
		\node [style=none] (191) at (51.75, 5.25) {};
		\node [style=X] (192) at (51, 6) {};
	\end{pgfonlayer}
	\begin{pgfonlayer}{edgelayer}
		\draw [in=-60, out=90, looseness=0.75] (191.center) to (189);
		\draw (192) to (190.center);
		\draw [in=-90, out=120] (192) to (188.center);
		\draw (192) to (189);
	\end{pgfonlayer}
\end{tikzpicture},
\hspace*{.5cm}
\begin{tikzpicture}[yscale=-1]
	\begin{pgfonlayer}{nodelayer}
		\node [style=none] (188) at (50.75, 7) {};
		\node [style=Z] (189) at (51.5, 6.75) {};
		\node [style=none] (190) at (51, 5.25) {};
		\node [style=none] (191) at (51.75, 5.25) {};
		\node [style=X] (192) at (51, 6) {};
	\end{pgfonlayer}
	\begin{pgfonlayer}{edgelayer}
		\draw [in=-60, out=90, looseness=0.75] (191.center) to (189);
		\draw (192) to (190.center);
		\draw [in=-90, out=120] (192) to (188.center);
		\draw (192) to (189);
	\end{pgfonlayer}
\end{tikzpicture}
$$

The strong complementarity of the $X$ and $Z$ variables in this setting follows from the fact that their corresponding Frobenius algebras interact to form a hopf algebra:
$$
\begin{tikzpicture}
	\begin{pgfonlayer}{nodelayer}
		\node [style=Z] (285) at (77.25, 6) {};
		\node [style=none] (286) at (77, 7) {};
		\node [style=Z] (287) at (77.75, 6.75) {};
		\node [style=none] (288) at (77, 4.5) {};
		\node [style=Z] (289) at (77.75, 4.75) {};
		\node [style=X] (290) at (77.25, 5.5) {};
	\end{pgfonlayer}
	\begin{pgfonlayer}{edgelayer}
		\draw [in=120, out=-90] (286.center) to (285);
		\draw (285) to (287);
		\draw [in=90, out=-120] (290) to (288.center);
		\draw (290) to (289);
		\draw (290) to (285);
		\draw [bend left] (287) to (289);
	\end{pgfonlayer}
\end{tikzpicture}
=
\begin{tikzpicture}
	\begin{pgfonlayer}{nodelayer}
		\node [style=Z] (291) at (78.75, 6.5) {};
		\node [style=none] (292) at (78.75, 7) {};
		\node [style=none] (293) at (78.75, 4.5) {};
		\node [style=X] (294) at (78.75, 5) {};
		\node [style=s] (295) at (79, 5.75) {};
	\end{pgfonlayer}
	\begin{pgfonlayer}{edgelayer}
		\draw (292.center) to (291);
		\draw (294) to (293.center);
		\draw [bend left] (294) to (291);
		\draw [in=90, out=-60] (291) to (295);
		\draw [in=-90, out=60] (294) to (295);
	\end{pgfonlayer}
\end{tikzpicture}
=
\begin{tikzpicture}
	\begin{pgfonlayer}{nodelayer}
		\node [style=Z] (297) at (79.75, 6) {};
		\node [style=none] (298) at (79.75, 7) {};
		\node [style=none] (299) at (79.75, 4.5) {};
		\node [style=X] (300) at (79.75, 5.5) {};
	\end{pgfonlayer}
	\begin{pgfonlayer}{edgelayer}
		\draw (298.center) to (297);
		\draw (300) to (299.center);
	\end{pgfonlayer}
\end{tikzpicture}
$$
The wire separates showing that preparing a state in the $X$ basis and measuring in the $Z$ basis preserves no information. However, in the symplectic picture, this result is purely topological:

$$
\begin{tikzpicture}
	\begin{pgfonlayer}{nodelayer}
		\node [style=Z] (219) at (59, 6.25) {};
		\node [style=none] (220) at (59, 7.5) {};
		\node [style=Z] (221) at (59.5, 7) {};
		\node [style=none] (222) at (59.5, 5.75) {};
	\end{pgfonlayer}
	\begin{pgfonlayer}{edgelayer}
		\draw (220.center) to (219);
		\draw (222.center) to (221);
	\end{pgfonlayer}
\end{tikzpicture}
=
\begin{tikzpicture}
	\begin{pgfonlayer}{nodelayer}
		\node [style=Z] (219) at (59, 6.25) {};
		\node [style=none] (220) at (59, 6.75) {};
		\node [style=Z] (221) at (59, 5.75) {};
		\node [style=none] (222) at (59, 5.25) {};
	\end{pgfonlayer}
	\begin{pgfonlayer}{edgelayer}
		\draw (220.center) to (219);
		\draw (222.center) to (221);
	\end{pgfonlayer}
\end{tikzpicture}
$$
\end{remark}


For this reason, we can prove the correctness of the quantum teleportation algorithm using only spider fusion:


\begin{example}
Given any prime $p$, the following string diagram in $\Aff\Co\Isot\Rel_{\F_p}^M$ depicts the $p$-dimensional qudit quantum teleportation protocol where Alice on the left teleports a qudit to Bob, on the right. They share an EPR pair (on the bottom of the diagram)  and two classical dits (drawn in red).  


$$
\begin{tikzpicture}
	\begin{pgfonlayer}{nodelayer}
		\node [style=none] (223) at (62.5, 1.75) {};
		\node [style=none] (224) at (63, 3.25) {};
		\node [style=none] (225) at (62, 3.25) {};
		\node [style=none] (226) at (61.5, 1.75) {};
		\node [style=Z] (227) at (63, 4.75) {};
		\node [style=Z] (228) at (61.5, 4.75) {};
		\node [style=none] (229) at (66.75, 3.25) {};
		\node [style=none] (230) at (67.25, 3.25) {};
		\node [style=none] (231) at (66.75, 9) {};
		\node [style=none] (232) at (67.25, 9) {};
		\node [style=Z] (233) at (61.5, 3.25) {};
		\node [style=X] (234) at (62, 3.75) {};
		\node [style=Z] (235) at (63, 3.75) {};
		\node [style=X] (236) at (62.5, 3.25) {};
		\node [style=none] (237) at (62, 4.75) {};
		\node [style=none] (238) at (62.5, 4.75) {};
		\node [style=X] (239) at (64, 2.25) {};
		\node [style=Z] (240) at (65.5, 2.25) {};
		\node [style=Z] (241) at (66.75, 7.75) {};
		\node [style=X] (242) at (67.25, 7.75) {};
		\node [style=X] (243) at (66.75, 8.5) {};
		\node [style=Z] (244) at (67.25, 8.5) {};
		\node [style=Z] (245) at (65.5, 6.75) {};
		\node [style=Z] (246) at (66, 6.75) {};
		\node [style=none] (247) at (66.5, 6.75) {};
		\node [style=none] (248) at (65, 6.75) {};
		\node [style=none] (249) at (64.5, 9) {};
		\node [style=none] (250) at (64.5, 1.75) {};
		\node [style=none] (251) at (63.5, 9) {Alice};
		\node [style=none] (252) at (65.5, 9) {Bob};
		\node [style=none] (254) at (61, 4.75) {};
		\node [style=none] (255) at (67.75, 4.75) {};
		\node [style=none] (256) at (61, 6.75) {};
		\node [style=none] (257) at (67.75, 6.75) {};
		\node [style=none] (258) at (59, 6.75) {Phase correction};
		\node [style=none] (259) at (59, 4.75) {Measurement};
	\end{pgfonlayer}
	\begin{pgfonlayer}{edgelayer}
		\draw (234) to (233);
		\draw (236) to (235);
		\draw (224.center) to (235);
		\draw (235) to (227);
		\draw (226.center) to (233);
		\draw (233) to (228);
		\draw (225.center) to (234);
		\draw (223.center) to (236);
		\draw (236) to (238.center);
		\draw (234) to (237.center);
		\draw [in=15, out=-90, looseness=0.75] (229.center) to (239);
		\draw [in=-90, out=165, looseness=0.50] (239) to (225.center);
		\draw [in=165, out=-90, looseness=0.50] (224.center) to (240);
		\draw [in=-90, out=15, looseness=0.75] (240) to (230.center);
		\draw (244) to (242);
		\draw (243) to (241);
		\draw (229.center) to (241);
		\draw (243) to (231.center);
		\draw (232.center) to (244);
		\draw (242) to (230.center);
		\draw [in=-150, out=90, looseness=0.75] (245) to (244);
		\draw [in=-150, out=90] (246) to (241);
		\draw [color=red, in=-90, out=90, looseness=0.50] (238.center) to (247.center);
		\draw [color=red, in=-90, out=90] (237.center) to (248.center);
		\draw [in=210, out=90, looseness=0.75] (248.center) to (243);
		\draw [in=-150, out=90] (247.center) to (242);
		\draw [style=dotted] (250.center) to (249.center);
		\draw [style=dotted] (255.center) to (254.center);
		\draw [style=dotted] (257.center) to (256.center);
	\end{pgfonlayer}
\end{tikzpicture}
=
\begin{tikzpicture}
	\begin{pgfonlayer}{nodelayer}
		\node [style=none] (301) at (81.75, 1.5) {};
		\node [style=none] (302) at (81.25, 3) {};
		\node [style=none] (303) at (80.75, 1.5) {};
		\node [style=none] (304) at (84.5, 3) {};
		\node [style=none] (305) at (85, 3) {};
		\node [style=none] (306) at (84.5, 8.5) {};
		\node [style=none] (307) at (85, 8.5) {};
		\node [style=X] (308) at (81.25, 4.5) {};
		\node [style=X] (309) at (82.75, 1.75) {};
		\node [style=Z] (310) at (83.75, 1.75) {};
		\node [style=X] (311) at (85, 7.25) {};
		\node [style=X] (312) at (84.5, 7.25) {};
		\node [style=Z] (313) at (83, 5) {};
		\node [style=X] (314) at (82, 4.5) {};
	\end{pgfonlayer}
	\begin{pgfonlayer}{edgelayer}
		\draw (302.center) to (308);
		\draw [in=15, out=-90, looseness=0.75] (304.center) to (309);
		\draw [in=-90, out=165, looseness=0.50] (309) to (302.center);
		\draw [in=-90, out=15, looseness=0.75] (310) to (305.center);
		\draw (312) to (306.center);
		\draw (311) to (305.center);
		\draw (311) to (307.center);
		\draw (304.center) to (312);
		\draw [in=90, out=-120] (308) to (303.center);
		\draw [color=red, in=225, out=90, looseness=0.75] (308) to (312);
		\draw [in=225, out=120, looseness=0.50, color=red] (314) to (311);
		\draw [in=15, out=-165] (313) to (314);
		\draw [in=270, out=90] (301.center) to (314);
		\draw [in=-60, out=135, looseness=0.75] (310) to (313);
	\end{pgfonlayer}
\end{tikzpicture}
=
\begin{tikzpicture}
	\begin{pgfonlayer}{nodelayer}
		\node [style=none] (166) at (42.5, 1.5) {};
		\node [style=none] (167) at (42, 1.5) {};
		\node [style=none] (168) at (43.5, 8.5) {};
		\node [style=none] (169) at (44, 8.5) {};
		\node [style=X] (170) at (42, 4.5) {};
		\node [style=X] (171) at (42.5, 3.5) {};
		\node [style=X] (172) at (44, 6.25) {};
		\node [style=X] (173) at (43.5, 7.25) {};
	\end{pgfonlayer}
	\begin{pgfonlayer}{edgelayer}
		\draw [in=-120, out=90, looseness=0.75] (166.center) to (171);
		\draw [in=270, out=60] (173) to (168.center);
		\draw [in=270, out=60, looseness=0.75] (172) to (169.center);
		\draw [color=red, bend left=15, looseness=0.50] (171) to (172);
		\draw [in=90, out=-120, looseness=0.50] (170) to (167.center);
		\draw [bend right=15, looseness=0.50] (171) to (172);
		\draw [bend left=15, looseness=0.50] (173) to (170);
		\draw [color=red, bend left=15, looseness=0.50] (170) to (173);
	\end{pgfonlayer}
\end{tikzpicture}
=
\begin{tikzpicture}
	\begin{pgfonlayer}{nodelayer}
		\node [style=none] (174) at (45.5, 1.5) {};
		\node [style=none] (175) at (45, 1.5) {};
		\node [style=none] (176) at (45, 8.5) {};
		\node [style=none] (177) at (45.5, 8.5) {};
	\end{pgfonlayer}
	\begin{pgfonlayer}{edgelayer}
		\draw (177.center) to (174.center);
		\draw (175.center) to (176.center);
	\end{pgfonlayer}
\end{tikzpicture}
$$

\end{example}

Because $\Aff\Co\Isot\Rel_{\F_p}^M$ is a subcategory of relations, composable maps are ordered by subspace inclusion (ie, it is poset-enriched). Moreover, since all possible outcomes are equally likely we can identify when the measurement statistics of one process arise from the marginalization of of the measurement statistics of another process:

\begin{remark}
Take two odd-prime dimensional qudit stabilizer circuits with state preparations and measurement $f,g$ interpreted as parallel maps in  $\Aff\Co\Isot\Rel_{\F_p}^M$.
Then $f$ is a coarse-graining of $g$ when $f \subset g$ is a (strict)  affine subspace.
\end{remark}
%\begin{proof}
%Consider the general setting of relations between sets.  Take two relations $R,S \subset X$.  Then $R \subseteq S$ if and only if for all $x \in R$ then $x \in S$.  
%\end{proof}


\begin{example}
For an extreme example, the identity circuit on a classical wire is contained within  the circuit obtained by preparing in the $Z$ basis and measuring in the $X$:

$$
\begin{tikzpicture}
	\begin{pgfonlayer}{nodelayer}
		\node [style=none] (220) at (59, 6.75) {};
		\node [style=none] (222) at (59, 5.25) {};
	\end{pgfonlayer}
	\begin{pgfonlayer}{edgelayer}
		\draw (220.center) to (222);
	\end{pgfonlayer}
\end{tikzpicture}
\subset
\begin{tikzpicture}
	\begin{pgfonlayer}{nodelayer}
		\node [style=Z] (219) at (59, 6.25) {};
		\node [style=none] (220) at (59, 6.75) {};
		\node [style=Z] (221) at (59, 5.75) {};
		\node [style=none] (222) at (59, 5.25) {};
	\end{pgfonlayer}
	\begin{pgfonlayer}{edgelayer}
		\draw (220.center) to (219);
		\draw (222.center) to (221);
	\end{pgfonlayer}
\end{tikzpicture}
=
\begin{tikzpicture}
	\begin{pgfonlayer}{nodelayer}
		\node [style=Z] (219) at (59, 6.25) {};
		\node [style=none] (220) at (59, 7.5) {};
		\node [style=Z] (221) at (59.5, 7) {};
		\node [style=none] (222) at (59.5, 5.75) {};
	\end{pgfonlayer}
	\begin{pgfonlayer}{edgelayer}
		\draw (220.center) to (219);
		\draw (222.center) to (221);
	\end{pgfonlayer}
\end{tikzpicture}
$$

This is because, given any input state, the circuit on the right hand side can produce any output state; however, the identity circuit forces the inputs to be the outputs.
\end{example}


\begin{example}
Similarly, the identity on a quantum wire is a coarse graining of the decoherence map:
$$
\begin{tikzpicture}
	\begin{pgfonlayer}{nodelayer}
		\node [style=none] (261) at (69.25, 6.75) {};
		\node [style=none] (263) at (69.25, 5.25) {};
		\node [style=none] (264) at (69.75, 6.75) {};
		\node [style=none] (265) at (69.75, 5.25) {};
	\end{pgfonlayer}
	\begin{pgfonlayer}{edgelayer}
		\draw (265.center) to (264.center);
		\draw (263.center) to (261.center);
	\end{pgfonlayer}
\end{tikzpicture}
\subset
\begin{tikzpicture}
	\begin{pgfonlayer}{nodelayer}
		\node [style=Z] (260) at (69.25, 6.25) {};
		\node [style=none] (261) at (69.25, 6.75) {};
		\node [style=Z] (262) at (69.25, 5.75) {};
		\node [style=none] (263) at (69.25, 5.25) {};
		\node [style=none] (264) at (69.75, 6.75) {};
		\node [style=none] (265) at (69.75, 5.25) {};
	\end{pgfonlayer}
	\begin{pgfonlayer}{edgelayer}
		\draw (261.center) to (260);
		\draw (263.center) to (262);
		\draw (265.center) to (264.center);
	\end{pgfonlayer}
\end{tikzpicture}
$$
\end{example}

This two coloured prop also has a semantics in terms of electrical circuits, when changing the field from $\F_p$ to the field of fractions of real polynomials $\R(x)$.  It gives a well-structured semantics for a large fragment of the impedence calculus \cite{impedence}:

\begin{remark}
$\Aff\Co\Isot\Rel_{\R(x)}^M$ is a semantics for the fragment of the impedence calculus generated by resistors, inductors, capacitors, controlled and uncontrolled voltage/current sources as well as ammeters and voltmeters.
\end{remark}

This can be verified by factoring the generators in \cite{impedence} into the appropriate form.  Perhaps this can be extended to general impedence boxes, but this not known to the author.
However, we shall not discuss this any further because it is outside of the scope of this paper.


\section{Error correction}
\label{sec:qec}

%Recall the observation from Corollary \ref{cor:stabcode} that nonempty states in $\Aff\Co\Isot\Rel_{\F_p}$ correspond to stabilizer codes. Moreover an affine coisotropic subspace of $\F_p^{2n}$ with dimension $m$  interpreted as a state in $\CPM(\Stab_p)$ encodes $m-n$ logical qubits in $n$ physical qubits.  In the literature, these are the $[n,m-n]$ odd prime dimensional qudit stabilizer codes \cite{????}.  In this section we show how to model error correction protocols within this framework.  Starting with stabilizer codes and then extending this interpretation to topological stabilizer codes by looking at the quantized Weinstein category.

Quantum channels are inherently noisy, and it is an active area of study to protect quantum channels against errors.  Quantum error detection/correction protocols are often performed using stabilizer circuits, which is amenable to the symplectic formalism.  We will use our newfound categorical semantics for stabilier codes to describe quantum error correction protocols. 

\subsection{Stabilizer codes}



In this subsection we will relate the terminology for stabilizer codes to the symplectic formalism, following Gross \cite{gross}.





%Fix odd prime $p$ dimensional qudit dimension.  Take a pure stabilizer state $|\phi \rangle$ on $n$ qudits with correponding affine Lagrangian subspace $S$ and homogeneous decomposition  $L+a=S$ into linear and affine components.  And fix a basis  $\{g_1,\ldots, g_n\}$ for $L$.
%Then $1 + W(g_i+a):\C^{d^n}\to \C^{d^n}$ is a rank 1 projector onto the $+1$-eigenspace of $W(g_i)$ so that:
%$$
%| \phi \rangle\langle\phi |= \prod_{i=1}^{n} \dfrac{1 + W(g_i+a)}{p}
%$$
%
%
%Suppose that we have a stabilizer code $f$ on $n$ qudits of rank $n-k$ with corresponding affine coisotropic subspace $S$ with homogeneous decomposition  $L+a=S$ into linear and affine components.  Let  $\{g_1,\ldots, g_{n-k}\}$ be a basis of the isotropic subspace $L^{\omega}$ of dimension $n-k$.  Then
%$$
%f = \prod_{i=1}^{n-k} \dfrac{1 + W(g_i+a)}{p}
%$$
%Therefore, even though we view stabilizer codes as maps in $\Aff\Co\Isot\Rel_{\F_p}$, the projectors comes from viewing stabilizer codes as an affine isotropic subspaces.  In the case of full rank projectors, then there is no distinction between the cositropic subspace and isotropic subspaces  because Lagrangian subspaces are self dual with respect to the symplectic complement.
%
%Given a stabilizer code, we show how to implement quantum error correction prototcol in the symplectic formalism.
%Fix an affine Lagrangian isometry $e_f:k\to n$ which is a dilation  $(k,e_f)$ of an affine coisotropic relation $f: 0 \to n$ of dimension $n+k$ so that:
%$$
%\begin{tikzpicture}
%	\begin{pgfonlayer}{nodelayer}
%		\node [style=map] (0) at (1.5, 0) {$e_f$};
%		\node [style=none] (1) at (1.25, 0.75) {};
%		\node [style=none] (3) at (1.75, 0.75) {};
%		\node [style=none] (4) at (1.25, -0.75) {};
%		\node [style=none] (5) at (1.75, -0.75) {};
%		\node [style=Z] (6) at (1.25, -0.75) {};
%		\node [style=Z] (7) at (1.75, -0.75) {};
%	\end{pgfonlayer}
%	\begin{pgfonlayer}{edgelayer}
%		\draw [in=-90, out=120] (0) to (1.center);
%		\draw [in=270, out=60] (0) to (3.center);
%		\draw [in=-60, out=90] (5.center) to (0);
%		\draw [in=90, out=-120] (0) to (4.center);
%	\end{pgfonlayer}
%\end{tikzpicture}
%=
%\begin{tikzpicture}
%	\begin{pgfonlayer}{nodelayer}
%		\node [style=map] (0) at (0, 0) {$f$};
%		\node [style=none] (1) at (-0.5, 1) {};
%		\node [style=none] (3) at (0.5, 1) {};
%	\end{pgfonlayer}
%	\begin{pgfonlayer}{edgelayer}
%		\draw [in=-90, out=135, looseness=0.75] (0) to (1.center);
%		\draw [in=-90, out=45, looseness=0.75] (0) to (3.center);
%	\end{pgfonlayer}
%\end{tikzpicture}
%$$
%
%We will think of $f$ as our $[n,n-k]$ qudit stabilizer code.
%This dilation encodes $k$ logical qubits into $n$ physical qubits.
%Chosing the $X$ basis to perform state preparations, we have that  that given any $x=(x_1,\ldots, x_k) \in \F_p^k$, the logical state $|x_1,\ldots, x_n\rangle$ is encoded as:
%
%$$
%\begin{tikzpicture}
%	\begin{pgfonlayer}{nodelayer}
%		\node [style=map] (0) at (1.5, 0) {$e_f$};
%		\node [style=none] (1) at (1.25, 0.75) {};
%		\node [style=none] (2) at (1.75, 0.75) {};
%		\node [style=none] (3) at (1.25, -0.75) {};
%		\node [style=none] (4) at (1.75, -0.75) {};
%		\node [style=Z] (5) at (1.25, -0.75) {};
%		\node [style=X] (6) at (1.75, -0.75) {$x$};
%	\end{pgfonlayer}
%	\begin{pgfonlayer}{edgelayer}
%		\draw [in=-90, out=120] (0) to (1.center);
%		\draw [in=270, out=60] (0) to (2.center);
%		\draw [in=-60, out=90] (4.center) to (0);
%		\draw [in=90, out=-120] (0) to (3.center);
%	\end{pgfonlayer}
%\end{tikzpicture}
%$$
%
%
%To construct the decoding map, we we factorize $e_f$ into $| 0 \rangle$ state preparations followed by a symplectomorphism $g$:
%
%$$
%\begin{tikzpicture}
%	\begin{pgfonlayer}{nodelayer}
%		\node [style=map] (0) at (1.5, 0) {$e_f$};
%		\node [style=none] (1) at (1.25, 0.75) {};
%		\node [style=none] (3) at (1.75, 0.75) {};
%		\node [style=none] (4) at (1.25, -0.75) {};
%		\node [style=none] (5) at (1.75, -0.75) {};
%		\node [style=none] (6) at (1.25, -0.75) {};
%		\node [style=none] (7) at (1.75, -0.75) {};
%	\end{pgfonlayer}
%	\begin{pgfonlayer}{edgelayer}
%		\draw [in=-90, out=120] (0) to (1.center);
%		\draw [in=270, out=60] (0) to (3.center);
%		\draw [in=-60, out=90] (5.center) to (0);
%		\draw [in=90, out=-120] (0) to (4.center);
%	\end{pgfonlayer}
%\end{tikzpicture}
%=
%\begin{tikzpicture}
%	\begin{pgfonlayer}{nodelayer}
%		\node [style=none] (0) at (3.5, -1) {};
%		\node [style=none] (1) at (1.5, 1) {};
%		\node [style=none] (2) at (2.5, 1) {};
%		\node [style=none] (3) at (3.5, -1.25) {$k$};
%		\node [style=none] (4) at (1.5, 1.25) {$n$};
%		\node [style=none] (5) at (2.5, 1.25) {$n$};
%		\node [style=none] (6) at (1.5, -1) {};
%		\node [style=Z] (7) at (1.25, -0.5) {};
%		\node [style=X] (8) at (2.75, -0.5) {};
%		\node [style=map] (9) at (2, 0.25) {$g$};
%		\node [style=none] (10) at (2, -0.5) {$n-k$};
%		\node [style=none] (11) at (1.5, -1.25) {$k$};
%		\node [style=none] (12) at (0.5, -0.5) {$n-k$};
%	\end{pgfonlayer}
%	\begin{pgfonlayer}{edgelayer}
%		\draw [in=300, out=105, looseness=0.75] (8) to (9);
%		\draw [in=-120, out=90] (6.center) to (9);
%		\draw [in=90, out=-30] (9) to (0.center);
%		\draw [in=-90, out=45] (9) to (2.center);
%		\draw [in=210, out=90, looseness=0.75] (7) to (9);
%		\draw [in=-90, out=135] (9) to (1.center);
%	\end{pgfonlayer}
%\end{tikzpicture}
%$$
%
%Therefore, we get a non-postselected circuit $h=Q^{\otimes n} \to Q^{\otimes k}\otimes C^{\otimes n-k}$  $\Aff\Co\Isot\Rel_{\F_p}^M$ when measuring the first $n-k$ wires of $g^\dag$ in the $X$ basis:
%
%$$
%h=
%\begin{tikzpicture}
%	\begin{pgfonlayer}{nodelayer}
%		\node [style=none] (0) at (3.5, 1) {};
%		\node [style=none] (1) at (1.5, -1) {};
%		\node [style=none] (2) at (2.5, -1) {};
%		\node [style=none] (3) at (3.5, 1.5) {$k$};
%		\node [style=none] (4) at (1.5, -1.25) {$n$};
%		\node [style=none] (5) at (2.5, -1.25) {$n$};
%		\node [style=none] (6) at (1.5, 1) {};
%		\node [style=Z] (7) at (1.25, 0.5) {};
%		\node [style=map] (9) at (2, -0.25) {$g^\dag$};
%		\node [style=none] (10) at (2.75, 1.5) {$n-k$};
%		\node [style=none] (11) at (1.5, 1.5) {$k$};
%		\node [style=none] (12) at (0.5, 0.5) {$n-k$};
%		\node [style=none] (13) at (2.75, 1) {};
%	\end{pgfonlayer}
%	\begin{pgfonlayer}{edgelayer}
%		\draw [in=120, out=-90] (6.center) to (9);
%		\draw [in=-90, out=30] (9) to (0.center);
%		\draw [in=90, out=-45] (9) to (2.center);
%		\draw [in=-210, out=-90, looseness=0.75] (7) to (9);
%		\draw [in=90, out=-135] (9) to (1.center);
%		\draw [in=-90, out=60, looseness=0.75] (9) to (13.center);
%	\end{pgfonlayer}
%\end{tikzpicture}
%$$
%The result of the $n-k$ measurements can be used to detect.
%Suppose we encode some logical qubits and then apply the measurements above.
%If there are no errors, then one will measure $|0,\ldots, 0\rangle$ on the syndrome because:
%
%$$
%\begin{tikzpicture}
%	\begin{pgfonlayer}{nodelayer}
%		\node [style=none] (0) at (3.5, 0) {};
%		\node [style=none] (2) at (1.5, 0) {};
%		\node [style=Z] (3) at (1.25, -0.5) {};
%		\node [style=map] (4) at (2, -1.25) {$g^\dag$};
%		\node [style=none] (8) at (2.75, 0) {};
%		\node [style=none] (9) at (3.5, -3.5) {};
%		\node [style=none] (11) at (1.5, -3.5) {};
%		\node [style=Z] (12) at (1.25, -3) {};
%		\node [style=X] (13) at (2.75, -3) {};
%		\node [style=map] (14) at (2, -2.25) {$g$};
%	\end{pgfonlayer}
%	\begin{pgfonlayer}{edgelayer}
%		\draw [in=120, out=-90] (2.center) to (4);
%		\draw [in=-90, out=30] (4) to (0.center);
%		\draw [in=-210, out=-90, looseness=0.75] (3) to (4);
%		\draw [in=-90, out=60, looseness=0.75] (4) to (8.center);
%		\draw [in=300, out=105, looseness=0.75] (13) to (14);
%		\draw [in=-120, out=90] (11.center) to (14);
%		\draw [in=90, out=-30] (14) to (9.center);
%		\draw [in=210, out=90, looseness=0.75] (12) to (14);
%		\draw [in=240, out=120] (14) to (4);
%		\draw [bend left] (4) to (14);
%	\end{pgfonlayer}
%\end{tikzpicture}
%=
%\begin{tikzpicture}
%	\begin{pgfonlayer}{nodelayer}
%		\node [style=none] (0) at (3, 0) {};
%		\node [style=none] (6) at (1.5, 0) {};
%		\node [style=Z] (7) at (1, -0.5) {};
%		\node [style=none] (13) at (2.25, 0) {};
%		\node [style=none] (14) at (3, -3.5) {};
%		\node [style=none] (20) at (1.5, -3.5) {};
%		\node [style=Z] (21) at (1, -3) {};
%		\node [style=X] (22) at (2.25, -3) {};
%	\end{pgfonlayer}
%	\begin{pgfonlayer}{edgelayer}
%		\draw (21) to (7);
%		\draw (20.center) to (6.center);
%		\draw (22) to (13.center);
%		\draw (14.center) to (0.center);
%	\end{pgfonlayer}
%\end{tikzpicture}
%=
%\begin{tikzpicture}
%	\begin{pgfonlayer}{nodelayer}
%		\node [style=none] (0) at (3, 0) {};
%		\node [style=none] (6) at (1.5, 0) {};
%		\node [style=none] (13) at (2.25, 0) {};
%		\node [style=none] (14) at (3, -1.5) {};
%		\node [style=none] (20) at (1.5, -1.5) {};
%		\node [style=X] (22) at (2.25, -1) {};
%	\end{pgfonlayer}
%	\begin{pgfonlayer}{edgelayer}
%		\draw (20.center) to (6.center);
%		\draw (22) to (13.center);
%		\draw (14.center) to (0.center);
%	\end{pgfonlayer}
%\end{tikzpicture}
%$$
%So there is no need to do any error correction
%Suppose that there is a Pauli error $W(E) \in P_{\F_p}^n$ which occurs before the measurements.
%Take $q$ to be the marginalization of $h^\dag$ on the $k$ wires:
%$$
%\begin{tikzpicture}
%	\begin{pgfonlayer}{nodelayer}
%		\node [style=none] (0) at (3.5, -1) {};
%		\node [style=none] (1) at (1.5, 1) {};
%		\node [style=none] (2) at (2.5, 1) {};
%		\node [style=none] (3) at (3.5, -1.5) {$k$};
%		\node [style=none] (4) at (1.5, 1.25) {$n$};
%		\node [style=none] (5) at (2.5, 1.25) {$n$};
%		\node [style=none] (6) at (1.5, -1) {};
%		\node [style=Z] (7) at (1.25, -0.5) {};
%		\node [style=map] (9) at (2, 0.25) {$g$};
%		\node [style=none] (10) at (2, -0.5) {$n-k$};
%		\node [style=none] (11) at (1.5, -1.5) {$k$};
%		\node [style=none] (12) at (0.5, -0.5) {$n-k$};
%		\node [style=Z] (13) at (1.5, -1) {};
%		\node [style=Z] (14) at (3.5, -1) {};
%		\node [style=none] (15) at (2.5, -0.5) {};
%		\node [style=none] (16) at (2.5, -1.75) {};
%	\end{pgfonlayer}
%	\begin{pgfonlayer}{edgelayer}
%		\draw [in=-120, out=90] (6.center) to (9);
%		\draw [in=90, out=-30] (9) to (0.center);
%		\draw [in=-90, out=45] (9) to (2.center);
%		\draw [in=210, out=90, looseness=0.75] (7) to (9);
%		\draw [in=-90, out=135] (9) to (1.center);
%		\draw [in=300, out=90, looseness=0.75] (15.center) to (9);
%		\draw (16.center) to (15.center);
%	\end{pgfonlayer}
%\end{tikzpicture}
%$$
%Let $q=L+a$ to be the homogenisation of $f$. Then the measurements performed by $h$ in the $X$ basis pick out a basis $\{g_1,\ldots, g_{n-k}\}$ of $L^\omega$ which yields the following measurement outcome  $(\omega(g_1+a,E), \ldots, \omega(g_{n-k}+a,E))$.
%
%If $S_E\neq 0$, then we know that an error has occured, and that the we can't trust the decoding is correct. 
%But  how does one correct for these errors?  We must change the protocol. 
%
%First we prepare an $n-k$ qudit ancilla in the state $| 0, \ldots, 0\rangle$, apply a controlled-shift onto this ancilla and then measure it in the $X$ basis:
%
%$$
%\begin{tikzpicture}
%	\begin{pgfonlayer}{nodelayer}
%		\node [style=none] (0) at (3.5, 0.75) {};
%		\node [style=none] (1) at (1.5, 0.75) {};
%		\node [style=none] (4) at (3, 0.75) {};
%		\node [style=none] (5) at (3.5, -3.75) {};
%		\node [style=none] (6) at (1.5, -3.75) {};
%		\node [style=Z] (7) at (1.25, -3.25) {};
%		\node [style=X] (8) at (2.75, -3.25) {};
%		\node [style=map] (9) at (2, -2.5) {$g$};
%		\node [style=map] (10) at (2, -1.75) {$W(X)$};
%		\node [style=Z] (11) at (1, -0.5) {};
%		\node [style=X] (12) at (3, -0.5) {};
%		\node [style=none] (13) at (1, 0.75) {};
%		\node [style=none] (14) at (2.5, 0.75) {};
%		\node [style=none] (15) at (0.5, 0.75) {};
%		\node [style=Z] (16) at (0.5, 0.75) {};
%		\node [style=Z] (17) at (0.5, 0) {};
%		\node [style=Z] (18) at (0.5, -0.75) {};
%		\node [style=X] (19) at (2.5, -0.75) {};
%		\node [style=X] (20) at (2.5, 0) {};
%	\end{pgfonlayer}
%	\begin{pgfonlayer}{edgelayer}
%		\draw [in=300, out=105, looseness=0.75] (8) to (9);
%		\draw [in=-120, out=90] (6.center) to (9);
%		\draw [in=90, out=-30] (9) to (5.center);
%		\draw [in=210, out=90, looseness=0.75] (7) to (9);
%		\draw (9) to (10);
%		\draw [in=-90, out=120] (10) to (1.center);
%		\draw [in=-90, out=15, looseness=1.25] (10) to (0.center);
%		\draw [in=-90, out=60, looseness=0.75] (10) to (12);
%		\draw [in=-60, out=135, looseness=0.75] (10) to (11);
%		\draw [in=270, out=90] (11) to (13.center);
%		\draw (12) to (4.center);
%		\draw (11) to (17);
%		\draw (17) to (16);
%		\draw (18) to (17);
%		\draw (12) to (20);
%		\draw (20) to (19);
%		\draw [style=red] (20) to (14.center);
%	\end{pgfonlayer}
%\end{tikzpicture}
%=
%\begin{tikzpicture}
%	\begin{pgfonlayer}{nodelayer}
%		\node [style=none] (0) at (3.5, 0.5) {};
%		\node [style=none] (1) at (1.5, 0.5) {};
%		\node [style=none] (4) at (3, 0.5) {};
%		\node [style=none] (5) at (3.5, -3.75) {};
%		\node [style=none] (6) at (1.5, -3.75) {};
%		\node [style=Z] (7) at (1.25, -3.25) {};
%		\node [style=X] (8) at (2.75, -3.25) {};
%		\node [style=map] (9) at (2, -2.5) {$g$};
%		\node [style=map] (10) at (2, -1.75) {$W(X)$};
%		\node [style=Z] (11) at (1, -0.75) {};
%		\node [style=X] (12) at (2.5, -0.75) {};
%		\node [style=none] (13) at (1, 0.5) {};
%		\node [style=none] (14) at (2.5, 0.5) {};
%		\node [style=none] (15) at (0.5, 0.5) {};
%		\node [style=Z] (16) at (0.5, 0.5) {};
%	\end{pgfonlayer}
%	\begin{pgfonlayer}{edgelayer}
%		\draw [in=300, out=105, looseness=0.75] (8) to (9);
%		\draw [in=-120, out=90] (6.center) to (9);
%		\draw [in=90, out=-30] (9) to (5.center);
%		\draw [in=210, out=90, looseness=0.75] (7) to (9);
%		\draw (9) to (10);
%		\draw [in=-90, out=120] (10) to (1.center);
%		\draw [in=-90, out=30, looseness=0.75] (10) to (0.center);
%		\draw [in=-90, out=60, looseness=0.75] (10) to (12);
%		\draw [in=-60, out=135, looseness=0.75] (10) to (11);
%		\draw [in=-90, out=135] (11) to (15.center);
%		\draw [in=270, out=90] (11) to (13.center);
%		\draw [in=-90, out=60] (12) to (4.center);
%		\draw [style=red] (12) to (14.center);
%	\end{pgfonlayer}
%\end{tikzpicture}
%$$
%
%Call the measurement $(x_1,\ldots, x_{n-k}) \in \F_p^{n-k}$ the {\bf syndrome}.  Now we know if an error has occured 



Fix odd prime $p$ dimensional qudit dimension.  
Let $\mathbf{Z_j}$ and $\mathbf{X_j}$ denote the qudit generalized Pauli matrices applied to the $j$th qudit.
For $a_z,a_x,z,x\in \F_p^n $, define the Pauli operator as the complex matrix:

$$\mathbf{W}_{a_z,a_x}(z,x):= e^{2\pi\cdot i ( \omega ((a_z,a_x),(z,x)) -\langle z,x\rangle/2 )/p}   \bigotimes_{j=0}^{n-1}  \mathbf{Z}_j^{z_j} \mathbf{X}_j^{x_j}$$

The generalized Pauli group on $n$ qudits is the group $\{ \mathbf{W}_a(g): a,g \in \F_p^{2n}\}$ under multiplication. Therefore, given an $n$ qudit stabilizer state $|\phi \rangle$, the density matrix $| \phi \rangle\langle\phi |$ is the joint projector onto a maximal Abelian subgroup of the generalized Pauli group on $n$ qudits, called the {\bf stabilizer group} of  $|\phi \rangle$.  Note that when we defined the symplectic Pauli group, the phase was neglected because all global scalars are quotiented.


Consider a nonempty  affine Lagrangian subspace $S \subseteq \F_p^{2n}$ for $|\phi\rangle$ with homogenized coordinates $S=L+a$.  The stabilizer group for $| \phi\rangle $ is given by $\{ \mathbf{W}_a(g): g \in L \}$.  The projector is therefore:
 
$$
| \phi \rangle\langle\phi |= \dfrac{1}{p^n}\sum_{g \in L} \mathbf{W}_a(g)
$$
%
%Fixing a basis for $L=\Span_{\F_p}\{g_1,\ldots, g_n\}$ (chosing a stabilizer tableau), owing to the stabilizer formalism, we can express this projector much more effeciently:
%$$
%| \phi \rangle\langle\phi | = \prod_{j=1}^{n} \dfrac{\mathbbm{1} + \mathbf{W}_a(g_j)}{2}
%$$
%

Now consider  a stabilizer code on $n$ qudits  with corresponding nonempty affine coisotropic subspace $S$ with dimension $n+k$.  The stabilizer code is a projector onto a non-maximal Abelian subgroup of the $n$-qudit generalized Pauli group. Let $S=L+a$ be the homogenized coordinates,
% with a chosen basis  for the {\em isotropic} subspace $L^{\omega}=\Span_{\F_p}\{g_1,\ldots, g_n\} \subseteq \F_p^{2n}$.  Then the stabilizer code is the joint projector onto all $\{ \mathbf{W}_a(g)|\forall g \in L\}$:
%$$
%\dfrac{1}{p^{n-k}} \sum_{g \in L^\omega} \mathbf{W}_a(g)= \prod_{j=1}^{n-k} \dfrac{\mathbbm{1} + \mathbf{W}_a(g_j)}{2}
%$$
Then the stabilizer code is the joint projector onto all elements of the group $\{ \mathbf{W}_a(g)|\forall g \in L^\omega\}$:
$$
\dfrac{1}{p^{n-k}} \sum_{g \in L^\omega} \mathbf{W}_a(g)
$$

Recall that we view stabilizer codes as maps $L+a$ in $\Aff\Co\Isot\Rel_{\F_p}$, due to this category having the correct notion of composition of circuits.  However, the corresponding stabilizer code is a projection onto the stabilizer group $\{ \mathbf{W}_a(g)|\forall g \in L^\omega\}$.  The stabilizer group corresponds now to the affine isotropic space $L^\omega +a $.



 To our knowledge, in the literature, stabilizer codes are always regarded as affine isotropic subspaces, as opposed to affine coisotropic subspaces (with the exception of \cite{booth}, following a private communication between the coauthors and the author of this paper). In the case of full rank projectors, then there is no distinction between the cosisotropic subspace and the symplectically-dual isotropic subspace  because Lagrangian subspaces are self dual with respect to the symplectic complement.

\subsection{Error correction protocols}
We show how to implement quantum error correction prototcols for stabilizer codes using the string diagrams we developed in the previous section. 
Fix an odd prime $p$.
Consider an affine coisotropic subspace $S=L+a \subseteq \F_p^{2n}$ where $L$ has dimension $n+k$.  Then the associated projector onto $\{ \mathbf{W}_a(g)|\forall g \in L^\omega\}$ is called a $[n,n-k]$-stabilizer code. Fix a unitary purification $U$ of $S$:

$$
\begin{tikzpicture}
	\begin{pgfonlayer}{nodelayer}
		\node [style=map] (249) at (105.1, 3.25) {$S$};
		\node [style=none] (250) at (104.85, 4) {};
		\node [style=none] (251) at (105.35, 4) {};
		\node [style=none] (256) at (104.85, 4.25) {$n$};
		\node [style=none] (257) at (105.35, 4.25) {$n$};
	\end{pgfonlayer}
	\begin{pgfonlayer}{edgelayer}
		\draw [in=-90, out=120] (249) to (250.center);
		\draw [in=-90, out=60] (249) to (251.center);
	\end{pgfonlayer}
\end{tikzpicture}
=
\begin{tikzpicture}
	\begin{pgfonlayer}{nodelayer}
		\node [style=map] (249) at (105.1, 3.25) {$U$};
		\node [style=none] (250) at (104.85, 4) {};
		\node [style=none] (251) at (105.35, 4) {};
		\node [style=none] (252) at (103.85, 2) {};
		\node [style=none] (253) at (105.6, 2) {};
		\node [style=Z] (254) at (103.85, 2) {};
		\node [style=none] (255) at (106.35, 1.55) {$n-k$};
		\node [style=none] (256) at (104.85, 4.25) {$n$};
		\node [style=none] (257) at (105.35, 4.25) {$n$};
		\node [style=X] (258) at (106.35, 2) {};
		\node [style=Z] (259) at (104.6, 2) {};
		\node [style=none] (260) at (105.6, 1.5) {$k$};
		\node [style=none] (261) at (103.75, 1.5) {$k$};
		\node [style=none] (262) at (104.6, 1.5) {$n-k$};
		\node [style=Z] (263) at (105.6, 2) {};
	\end{pgfonlayer}
	\begin{pgfonlayer}{edgelayer}
		\draw [in=-90, out=120] (249) to (250.center);
		\draw [in=-90, out=60] (249) to (251.center);
		\draw [in=-60, out=90] (253.center) to (249);
		\draw [in=90, out=-150] (249) to (252.center);
		\draw [in=-30, out=90] (258) to (249);
		\draw [in=90, out=-120] (249) to (259);
	\end{pgfonlayer}
\end{tikzpicture}
$$



%%Fix a basis $\{b_1,\ldots, b_{n-k}\} $ for the affine isotropic subspace  $L^\perp+a\subseteq\F_p^{2n}$.  This choice of basis yields injection into an isometry $V:n-k\to n$; which can be factored into $k$ $|0\rangle$ states followed by a unitary $U:n\to n$, so that:
%$$
%\begin{tikzpicture}
%	\begin{pgfonlayer}{nodelayer}
%		\node [style=map] (42) at (29.7, 3.25) {$V$};
%		\node [style=none] (43) at (29.45, 4) {};
%		\node [style=none] (44) at (29.95, 4) {};
%		\node [style=none] (45) at (29.45, 2.25) {};
%		\node [style=none] (46) at (29.95, 2) {};
%		\node [style=Z] (47) at (29.45, 2.25) {};
%		\node [style=none] (48) at (29.95, 1.7) {$n-k$};
%		\node [style=none] (49) at (29.45, 4.25) {$n$};
%		\node [style=none] (50) at (29.95, 4.25) {$n$};
%		\node [style=none] (51) at (29, 2.73) {$n-k$};
%	\end{pgfonlayer}
%	\begin{pgfonlayer}{edgelayer}
%		\draw [in=-90, out=120] (42) to (43.center);
%		\draw [in=-90, out=60] (42) to (44.center);
%		\draw [in=-60, out=90] (46.center) to (42);
%		\draw [in=90, out=-120] (42) to (45.center);
%	\end{pgfonlayer}
%\end{tikzpicture}
%=
%\begin{tikzpicture}
%	\begin{pgfonlayer}{nodelayer}
%		\node [style=map] (96) at (59.8, 2.5) {$U$};
%		\node [style=none] (97) at (59.55, 3.25) {};
%		\node [style=none] (98) at (60.05, 3.25) {};
%		\node [style=none] (99) at (58.8, 1.25) {};
%		\node [style=none] (100) at (61.05, 1) {};
%		\node [style=Z] (101) at (58.8, 1.25) {};
%		\node [style=none] (102) at (61.05, 0.55) {$n-k$};
%		\node [style=none] (103) at (59.55, 3.5) {$n$};
%		\node [style=none] (104) at (60.05, 3.5) {$n$};
%		\node [style=X] (105) at (60.55, 1.25) {};
%		\node [style=Z] (106) at (59.3, 1.25) {};
%		\node [style=none] (107) at (60.3, 0.75) {$k$};
%		\node [style=none] (108) at (58.45, 0.75) {$k$};
%		\node [style=none] (109) at (59.55, 0.75) {$n-k$};
%	\end{pgfonlayer}
%	\begin{pgfonlayer}{edgelayer}
%		\draw [in=-90, out=120] (96) to (97.center);
%		\draw [in=-90, out=60] (96) to (98.center);
%		\draw [in=-30, out=90] (100.center) to (96);
%		\draw [in=90, out=-150] (96) to (99.center);
%		\draw [in=-60, out=90, looseness=0.75] (105) to (96);
%		\draw [in=90, out=-120] (96) to (106);
%	\end{pgfonlayer}
%\end{tikzpicture}
%\hspace*{.5cm}\text{where}\hspace*{.5cm}
%\begin{tikzpicture}
%	\begin{pgfonlayer}{nodelayer}
%		\node [style=map] (24) at (25.5, 3.25) {$U$};
%		\node [style=none] (25) at (25.25, 4) {};
%		\node [style=none] (26) at (25.75, 4) {};
%		\node [style=none] (27) at (24.5, 2.25) {};
%		\node [style=none] (28) at (26, 2.25) {};
%		\node [style=Z] (29) at (24.5, 2.25) {};
%		\node [style=Z] (33) at (26.5, 2.25) {};
%		\node [style=Z] (34) at (25, 2.25) {};
%		\node [style=X] (38) at (26, 2.25) {};
%	\end{pgfonlayer}
%	\begin{pgfonlayer}{edgelayer}
%		\draw [in=-90, out=120] (24) to (25.center);
%		\draw [in=-90, out=60] (24) to (26.center);
%		\draw [in=-60, out=90] (28.center) to (24);
%		\draw [in=90, out=-150] (24) to (27.center);
%		\draw [in=-30, out=90] (33) to (24);
%		\draw [in=90, out=-120] (24) to (34);
%	\end{pgfonlayer}
%\end{tikzpicture}
%=
%\begin{tikzpicture}
%	\begin{pgfonlayer}{nodelayer}
%		\node [style=map] (39) at (27.75, 3) {$S$};
%		\node [style=none] (40) at (27.5, 3.75) {};
%		\node [style=none] (41) at (28, 3.75) {};
%	\end{pgfonlayer}
%	\begin{pgfonlayer}{edgelayer}
%		\draw [in=-90, out=120] (39) to (40.center);
%		\draw [in=-90, out=60] (39) to (41.center);
%	\end{pgfonlayer}
%\end{tikzpicture}
%$$

The Lagrangian dilation is called an encoder of $S$:
$$
\begin{tikzpicture}
	\begin{pgfonlayer}{nodelayer}
		\node [style=map] (249) at (105.1, 3.25) {$U$};
		\node [style=none] (250) at (104.85, 4) {};
		\node [style=none] (251) at (105.35, 4) {};
		\node [style=none] (252) at (103.85, 1.75) {};
		\node [style=none] (253) at (105.6, 1.75) {};
		\node [style=none] (255) at (106.35, 1.55) {$n-k$};
		\node [style=none] (256) at (104.85, 4.25) {$n$};
		\node [style=none] (257) at (105.35, 4.25) {$n$};
		\node [style=X] (258) at (106.35, 2) {};
		\node [style=Z] (259) at (104.6, 2) {};
		\node [style=none] (260) at (105.6, 1.25) {$k$};
		\node [style=none] (262) at (104.6, 1.5) {$n-k$};
		\node [style=none] (265) at (103.85, 1.25) {$k$};
	\end{pgfonlayer}
	\begin{pgfonlayer}{edgelayer}
		\draw [in=-90, out=120] (249) to (250.center);
		\draw [in=-90, out=60] (249) to (251.center);
		\draw [in=-60, out=90] (253.center) to (249);
		\draw [in=90, out=-150] (249) to (252.center);
		\draw [in=-30, out=90] (258) to (249);
		\draw [in=90, out=-120] (249) to (259);
	\end{pgfonlayer}
\end{tikzpicture}
$$



Splitting this projector fixes a basis $\{b_1,\ldots, b_{n-k}\}$ for $L^\omega$:
$$
\begin{tikzpicture}
	\begin{pgfonlayer}{nodelayer}
		\node [style=map] (266) at (108.7, 3.25) {$U$};
		\node [style=none] (267) at (108.45, 4) {};
		\node [style=none] (268) at (108.95, 4) {};
		\node [style=none] (269) at (107.45, 2) {};
		\node [style=none] (270) at (109.2, 2) {};
		\node [style=Z] (271) at (107.45, 2) {};
		\node [style=none] (272) at (109.95, 1.05) {$n-k$};
		\node [style=none] (273) at (108.45, 4.25) {$n$};
		\node [style=none] (274) at (108.95, 4.25) {$n$};
		\node [style=Z] (276) at (108.2, 2) {};
		\node [style=none] (277) at (109.2, 1.5) {$k$};
		\node [style=none] (278) at (107.35, 1.5) {$k$};
		\node [style=none] (279) at (108.2, 1.5) {$n-k$};
		\node [style=none] (280) at (109.95, 1.5) {};
		\node [style=none] (281) at (109.95, 2) {};
		\node [style=Z] (282) at (109.2, 2) {};
	\end{pgfonlayer}
	\begin{pgfonlayer}{edgelayer}
		\draw [in=-90, out=120] (266) to (267.center);
		\draw [in=-90, out=60] (266) to (268.center);
		\draw [in=-60, out=90] (270.center) to (266);
		\draw [in=90, out=-150] (266) to (269.center);
		\draw [in=90, out=-120] (266) to (276);
		\draw [in=-30, out=90] (281.center) to (266);
		\draw [style=red] (280.center) to (281.center);
	\end{pgfonlayer}
\end{tikzpicture}
$$



Suppose that Alice encodes a state and sends it to Bob on a noisy quantum channel with pauli error $W(e)$.  To detect the error apply the non-destructive measurement on the last $n-k$ wires in the $Z$ basis conjugated by U:

$$
\begin{tikzpicture}
	\begin{pgfonlayer}{nodelayer}
		\node [style=map] (206) at (95, 2.75) {$U$};
		\node [style=none] (207) at (94, 1.25) {};
		\node [style=none] (208) at (95.25, 1.25) {};
		\node [style=X] (209) at (95.75, 1.5) {};
		\node [style=Z] (210) at (94.5, 1.5) {};
		\node [style=none] (211) at (96.75, 6) {};
		\node [style=none] (212) at (97.25, 6) {};
		\node [style=none] (213) at (98.75, 6) {};
		\node [style=none] (214) at (99.25, 6) {};
		\node [style=Z] (215) at (97.75, 6.25) {};
		\node [style=X] (216) at (97.25, 6) {};
		\node [style=Z] (217) at (97.75, 5.5) {};
		\node [style=none] (218) at (98, 8) {};
		\node [style=none] (219) at (98, 8) {};
		\node [style=Z] (220) at (99.25, 6) {};
		\node [style=X] (221) at (99.75, 6.25) {};
		\node [style=X] (222) at (99.75, 5.5) {};
		\node [style=none] (223) at (98, 8) {};
		\node [style=none] (224) at (98, 8) {};
		\node [style=Z] (225) at (97.75, 7) {};
		\node [style=none] (226) at (99.75, 9) {};
		\node [style=none] (227) at (99.75, 7) {};
		\node [style=map] (228) at (96.25, 3.75) {$W(e)$};
		\node [style=map] (229) at (98, 4.75) {$U^\dag$};
		\node [style=map] (230) at (98, 8) {$U$};
		\node [style=none] (231) at (97, 9) {};
		\node [style=none] (232) at (97.5, 9) {};
		\node [style=none] (233) at (98.5, 9) {};
		\node [style=none] (234) at (99, 9) {};
		\node [style=none] (235) at (96.25, 9) {};
		\node [style=none] (236) at (96.25, 1.25) {};
		\node [style=none] (237) at (100.25, 7) {};
		\node [style=none] (238) at (93.25, 7) {};
		\node [style=none] (239) at (100.25, 1.5) {};
		\node [style=none] (240) at (93.5, 1.5) {};
		\node [style=none] (241) at (92, 7.25) {Syndrome};
		\node [style=none] (242) at (92.25, 1.5) {Encoding};
		\node [style=none] (243) at (92.75, 2.75) {Alice};
		\node [style=none] (244) at (98.5, 2.75) {Bob};
		\node [style=none] (245) at (100.25, 3.75) {};
		\node [style=none] (246) at (93.25, 3.75) {};
		\node [style=none] (247) at (92.5, 3.75) {Error};
		\node [style=none] (248) at (92, 6.75) {Measurement};
	\end{pgfonlayer}
	\begin{pgfonlayer}{edgelayer}
		\draw [in=-60, out=90, looseness=0.75] (208.center) to (206);
		\draw [in=90, out=-165] (206) to (207.center);
		\draw [in=-30, out=90] (209) to (206);
		\draw [in=90, out=-135] (206) to (210);
		\draw (215) to (216);
		\draw (217) to (215);
		\draw [in=-150, out=90] (211.center) to (219.center);
		\draw [in=-120, out=90, looseness=1.25] (216) to (218.center);
		\draw [in=-60, out=90, looseness=0.75] (213.center) to (224.center);
		\draw [in=90, out=-30, looseness=0.75] (223.center) to (220);
		\draw (220) to (221);
		\draw (221) to (222);
		\draw [style=red] (227.center) to (226.center);
		\draw (215) to (225);
		\draw (221) to (227.center);
		\draw [in=270, out=75, looseness=0.75] (206) to (228);
		\draw [in=-105, out=90, looseness=0.50] (228) to (229);
		\draw [in=-90, out=150] (229) to (211.center);
		\draw [in=135, out=-90] (212.center) to (229);
		\draw [in=-90, out=60] (229) to (213.center);
		\draw [in=30, out=-90, looseness=0.75] (214.center) to (229);
		\draw [in=-90, out=135] (230) to (231.center);
		\draw [in=-90, out=105] (230) to (232.center);
		\draw [in=-90, out=75] (230) to (233.center);
		\draw [in=-90, out=45] (230) to (234.center);
		\draw [style=dotted] (236.center) to (235.center);
		\draw [style=dotted] (238.center) to (237.center);
		\draw [style=dotted] (240.center) to (239.center);
		\draw [style=dotted] (246.center) to (245.center);
	\end{pgfonlayer}
\end{tikzpicture}
=
\begin{tikzpicture}
	\begin{pgfonlayer}{nodelayer}
		\node [style=none] (68) at (53.25, 4) {};
		\node [style=none] (69) at (54.75, 4) {};
		\node [style=X] (70) at (55.25, 4.25) {};
		\node [style=Z] (71) at (53.75, 4.25) {};
		\node [style=none] (72) at (54.25, 7) {};
		\node [style=none] (73) at (54.25, 7) {};
		\node [style=none] (74) at (54.25, 7) {};
		\node [style=Z] (75) at (55, 6.25) {};
		\node [style=none] (76) at (54.25, 7) {};
		\node [style=none] (77) at (54.25, 7) {};
		\node [style=none] (78) at (56, 8) {};
		\node [style=none] (79) at (55, 6.25) {};
		\node [style=map] (80) at (54.25, 5.5) {$U;W(e);U^\dag$};
		\node [style=map] (81) at (54.25, 7) {$U$};
		\node [style=none] (82) at (53.25, 8) {};
		\node [style=none] (83) at (53.75, 8) {};
		\node [style=none] (84) at (54.75, 8) {};
		\node [style=none] (85) at (55.25, 8) {};
	\end{pgfonlayer}
	\begin{pgfonlayer}{edgelayer}
		\draw [in=90, out=-30, looseness=0.75] (76.center) to (75);
		\draw [style=red, in=-90, out=45] (79.center) to (78.center);
		\draw [in=-90, out=135] (81) to (82.center);
		\draw [in=-90, out=105] (81) to (83.center);
		\draw [in=-90, out=75] (81) to (84.center);
		\draw [in=-90, out=45] (81) to (85.center);
		\draw [in=-90, out=45, looseness=0.75] (80) to (79.center);
		\draw [bend left=60, looseness=1.25] (80) to (81);
		\draw [in=90, out=-30] (80) to (70);
		\draw [bend left] (80) to (81);
		\draw [bend left] (81) to (80);
		\draw [in=-60, out=90] (69.center) to (80);
		\draw [in=-120, out=90] (71) to (80);
		\draw [in=-150, out=90] (68.center) to (80);
	\end{pgfonlayer}
\end{tikzpicture}
=
\begin{tikzpicture}
	\begin{pgfonlayer}{nodelayer}
		\node [style=none] (86) at (57, 4.25) {};
		\node [style=none] (87) at (58.5, 4.25) {};
		\node [style=X] (88) at (59, 4.5) {};
		\node [style=Z] (89) at (57.5, 4.5) {};
		\node [style=none] (90) at (58, 7) {};
		\node [style=none] (91) at (58, 7) {};
		\node [style=none] (92) at (58, 7) {};
		\node [style=none] (94) at (58, 7) {};
		\node [style=none] (95) at (58, 7) {};
		\node [style=none] (96) at (59.75, 8) {};
		\node [style=none] (97) at (59.75, 6.25) {};
		\node [style=map] (98) at (58, 5.75) {$U;W(e);U^\dag$};
		\node [style=map] (99) at (58, 7) {$U$};
		\node [style=none] (100) at (57, 8) {};
		\node [style=none] (101) at (57.5, 8) {};
		\node [style=none] (102) at (58.5, 8) {};
		\node [style=none] (103) at (59, 8) {};
		\node [style=X] (104) at (59.75, 6.25) {$t$};
	\end{pgfonlayer}
	\begin{pgfonlayer}{edgelayer}
		\draw [style=red] (97.center) to (96.center);
		\draw [in=-90, out=135] (99) to (100.center);
		\draw [in=-90, out=105] (99) to (101.center);
		\draw [in=-90, out=75] (99) to (102.center);
		\draw [in=-90, out=45] (99) to (103.center);
		\draw [bend left=60] (98) to (99);
		\draw [in=90, out=-30] (98) to (88);
		\draw [bend left] (98) to (99);
		\draw [bend left] (99) to (98);
		\draw [in=-60, out=90] (87.center) to (98);
		\draw [in=-120, out=90] (89) to (98);
		\draw [in=-150, out=90] (86.center) to (98);
		\draw [bend right=60] (98) to (99);
	\end{pgfonlayer}
\end{tikzpicture}
=
\begin{tikzpicture}
	\begin{pgfonlayer}{nodelayer}
		\node [style=none] (105) at (61.25, 4.25) {};
		\node [style=none] (106) at (62.75, 4.25) {};
		\node [style=X] (107) at (63.25, 4.5) {};
		\node [style=Z] (108) at (61.75, 4.5) {};
		\node [style=none] (114) at (64, 8) {};
		\node [style=none] (115) at (64, 6.25) {};
		\node [style=map] (116) at (62.25, 5.75) {$U$};
		\node [style=X] (122) at (64, 6.25) {$t$};
		\node [style=map] (123) at (62.25, 6.5) {$W(e)$};
		\node [style=none] (124) at (61.25, 8) {};
		\node [style=none] (125) at (61.75, 8) {};
		\node [style=none] (126) at (62.75, 8) {};
		\node [style=none] (127) at (63.25, 8) {};
	\end{pgfonlayer}
	\begin{pgfonlayer}{edgelayer}
		\draw [style=red] (115.center) to (114.center);
		\draw [in=90, out=-30] (116) to (107);
		\draw [in=-60, out=90] (106.center) to (116);
		\draw [in=-120, out=90] (108) to (116);
		\draw [in=-150, out=90] (105.center) to (116);
		\draw (116) to (123);
		\draw [in=-90, out=135] (123) to (124.center);
		\draw [in=-90, out=45] (123) to (127.center);
		\draw [in=60, out=-90] (126.center) to (123);
		\draw [in=-90, out=120] (123) to (125.center);
	\end{pgfonlayer}
\end{tikzpicture}
$$

%
%Moveorver, given that $U$ can be decomposed into a symplectormorphim $U'$ followed by affine shift $s$ $U;W(e);U^\dag = U'()$
%The bitstring $t=(\omega(b_1+a,e_1),\omega(b_{k}+a,e)) \in \F_p^{k}$ which is measured is called the {\bf syndrome}.
%If $e \in L^\omega+a$, then it commutes with everything in $L+a$, therefore the syndrome will be the zero vector:
%We call such errors {\bf undetectable}, as the act the same as the trivial error (which is itself vacuously undetectable).

The tuple $t \in \F_p^{n-k}$ is called the {\bf syndrome}. The syndrome measures the displacement of the basis elements $b_i$ by errors.
An error $W(e)$ is undetectable if and only if the syndrome is the zero vector; this is because $e$ commutes with everything in $L+a$ meaning that $e \in L^\omega+a$.  In particular, the trivial error is undetectable; so undetectable errors are indistinguishible from having no errors at all.


To correct errors, we want to construct a classically controlled operation from the syndrome.
Given any syndrome measurement $t\in \F_p^{n-k}$, pick an error $W(e)$ which one wishes to correct, where additionally, the trivial syndrome corrects nothing.  This determines a function $f:\F_p^{n-k}\to\F_p^{2n}$ sending $t\mapsto e$.
Take $t=(t_1,\ldots, t_{n-k}) \in \F_p^{n-k}$ such that $f(t)=(c_{1,z},\ldots,c_{n,z},c_{1,x},\ldots,c_{n,x} )$.


If $f$ is an affine transformation, then we can construct the classically controlled error correction operation $c_f$ so that given a syndrome $t$, it applies the operation $W(-f(t))$.  This restriction on the function affine comes from the fact that within the model we have constructed, only affine classical processing is allowed.  The final error correction protocol is as follows:

$$
\begin{tikzpicture}
	\begin{pgfonlayer}{nodelayer}
		\node [style=map] (395) at (145.75, 2.75) {$U$};
		\node [style=none] (396) at (144.75, 1.25) {};
		\node [style=none] (397) at (146, 1.25) {};
		\node [style=X] (398) at (146.5, 1.5) {};
		\node [style=Z] (399) at (145.25, 1.5) {};
		\node [style=none] (400) at (147.5, 6) {};
		\node [style=none] (401) at (148, 6) {};
		\node [style=none] (402) at (149.5, 6) {};
		\node [style=none] (403) at (150, 6) {};
		\node [style=Z] (404) at (148.5, 6.25) {};
		\node [style=X] (405) at (148, 6) {};
		\node [style=Z] (406) at (148.5, 5.5) {};
		\node [style=none] (407) at (148.75, 8) {};
		\node [style=none] (408) at (148.75, 8) {};
		\node [style=Z] (409) at (150, 6) {};
		\node [style=X] (410) at (150.5, 6.25) {};
		\node [style=X] (411) at (150.5, 5.5) {};
		\node [style=none] (412) at (148.75, 8) {};
		\node [style=none] (413) at (148.75, 8) {};
		\node [style=Z] (414) at (148.5, 7) {};
		\node [style=none] (415) at (150.25, 8.5) {};
		\node [style=none] (416) at (150.5, 7) {};
		\node [style=map] (417) at (147, 3.75) {$W(e)$};
		\node [style=map] (418) at (148.75, 4.75) {$U^\dag$};
		\node [style=map] (419) at (148.75, 8) {$U$};
		\node [style=none] (420) at (147, 11) {};
		\node [style=none] (421) at (147, 1.25) {};
		\node [style=none] (422) at (151, 7) {};
		\node [style=none] (423) at (144, 7) {};
		\node [style=none] (424) at (151, 1.5) {};
		\node [style=none] (425) at (144.25, 1.5) {};
		\node [style=none] (426) at (142.75, 7.25) {Syndrome};
		\node [style=none] (427) at (143, 1.5) {Encoding};
		\node [style=none] (428) at (143.5, 2.75) {Alice};
		\node [style=none] (429) at (149.25, 2.75) {Bob};
		\node [style=none] (430) at (151, 3.75) {};
		\node [style=none] (431) at (144, 3.75) {};
		\node [style=none] (432) at (143.25, 3.75) {Error};
		\node [style=none] (433) at (142.75, 6.75) {Measurement};
		\node [style=Z] (434) at (149.75, 8.5) {};
		\node [style=map] (435) at (148.75, 10) {$c_f$};
		\node [style=none] (436) at (147.75, 11) {};
		\node [style=none] (437) at (148.25, 11) {};
		\node [style=none] (438) at (149.25, 11) {};
		\node [style=none] (439) at (149.75, 11) {};
		\node [style=none] (440) at (151, 10) {};
		\node [style=none] (441) at (143.75, 10) {};
		\node [style=none] (442) at (142.5, 10) {Error correction};
		\node [style=Z] (443) at (150.5, 7.75) {};
		\node [style=none] (444) at (150.75, 11) {};
	\end{pgfonlayer}
	\begin{pgfonlayer}{edgelayer}
		\draw [in=-60, out=90, looseness=0.75] (397.center) to (395);
		\draw [in=90, out=-165] (395) to (396.center);
		\draw [in=-30, out=90] (398) to (395);
		\draw [in=90, out=-135] (395) to (399);
		\draw (404) to (405);
		\draw (406) to (404);
		\draw [in=-150, out=90] (400.center) to (408.center);
		\draw [in=-120, out=90, looseness=1.25] (405) to (407.center);
		\draw [in=-60, out=90, looseness=0.75] (402.center) to (413.center);
		\draw [in=90, out=-30, looseness=0.75] (412.center) to (409);
		\draw (409) to (410);
		\draw (410) to (411);
		\draw (404) to (414);
		\draw (410) to (416.center);
		\draw [in=270, out=75, looseness=0.75] (395) to (417);
		\draw [in=-105, out=90, looseness=0.50] (417) to (418);
		\draw [in=-90, out=150] (418) to (400.center);
		\draw [in=135, out=-90] (401.center) to (418);
		\draw [in=-90, out=60] (418) to (402.center);
		\draw [in=30, out=-90, looseness=0.75] (403.center) to (418);
		\draw [style=dotted] (421.center) to (420.center);
		\draw [style=dotted] (423.center) to (422.center);
		\draw [style=dotted] (425.center) to (424.center);
		\draw [style=dotted] (431.center) to (430.center);
		\draw [in=-90, out=120, looseness=0.75] (435) to (437.center);
		\draw [in=135, out=-90] (436.center) to (435);
		\draw [in=-90, out=60, looseness=0.75] (435) to (438.center);
		\draw [in=-90, out=45, looseness=0.75] (435) to (439.center);
		\draw [in=-165, out=150] (419) to (435);
		\draw [in=30, out=-30, looseness=1.25] (435) to (419);
		\draw [bend right=45] (419) to (435);
		\draw [bend right=45, looseness=0.75] (435) to (419);
		\draw [in=-15, out=90, looseness=0.75] (415.center) to (435);
		\draw [in=-105, out=90, looseness=0.75] (434) to (435);
		\draw [style=dotted] (441.center) to (440.center);
		\draw [style=red, in=-90, out=120] (443) to (415.center);
		\draw [style=red] (416.center) to (443);
		\draw [style=red, in=-90, out=60, looseness=0.50] (443) to (444.center);
	\end{pgfonlayer}
\end{tikzpicture}
=
\begin{tikzpicture}
	\begin{pgfonlayer}{nodelayer}
		\node [style=none] (491) at (163.25, 6) {};
		\node [style=none] (492) at (164.75, 6) {};
		\node [style=X] (493) at (165.25, 6.25) {};
		\node [style=Z] (494) at (163.75, 6.25) {};
		\node [style=map] (495) at (164.25, 7.5) {$U$};
		\node [style=X] (496) at (165.25, 9) {$t$};
		\node [style=map] (497) at (164.25, 8.25) {$W(e)$};
		\node [style=none] (498) at (164.25, 10) {};
		\node [style=none] (499) at (164.25, 10) {};
		\node [style=none] (500) at (164.25, 10) {};
		\node [style=none] (501) at (164.25, 10) {};
		\node [style=none] (502) at (165.25, 9) {};
		\node [style=Z] (503) at (164, 9) {};
		\node [style=map] (504) at (164.25, 10) {$c_f$};
		\node [style=none] (505) at (163.25, 11) {};
		\node [style=none] (506) at (163.75, 11) {};
		\node [style=none] (507) at (164.75, 11) {};
		\node [style=none] (508) at (165.25, 11) {};
		\node [style=X] (509) at (166, 9) {$t$};
		\node [style=none] (510) at (166, 11) {};
	\end{pgfonlayer}
	\begin{pgfonlayer}{edgelayer}
		\draw [in=90, out=-30] (495) to (493);
		\draw [in=-60, out=90] (492.center) to (495);
		\draw [in=-120, out=90] (494) to (495);
		\draw [in=-150, out=90] (491.center) to (495);
		\draw (495) to (497);
		\draw [in=-165, out=165, looseness=1.50] (497) to (498.center);
		\draw [in=-15, out=30] (497) to (501.center);
		\draw [in=60, out=-30] (500.center) to (497);
		\draw [in=-150, out=150, looseness=1.25] (497) to (499.center);
		\draw [in=-90, out=120, looseness=0.75] (504) to (506.center);
		\draw [in=135, out=-90] (505.center) to (504);
		\draw [in=-90, out=60, looseness=0.75] (504) to (507.center);
		\draw [in=-90, out=45, looseness=0.75] (504) to (508.center);
		\draw [in=0, out=90] (502.center) to (504);
		\draw [in=-120, out=90, looseness=1.25] (503) to (504);
		\draw [style=red] (509) to (510.center);
	\end{pgfonlayer}
\end{tikzpicture}
=
\begin{tikzpicture}
	\begin{pgfonlayer}{nodelayer}
		\node [style=none] (466) at (155.75, 6) {};
		\node [style=none] (467) at (157.25, 6) {};
		\node [style=X] (468) at (157.75, 6.25) {};
		\node [style=Z] (469) at (156.25, 6.25) {};
		\node [style=map] (470) at (156.75, 7.5) {$U$};
		\node [style=map] (471) at (156.75, 8.25) {$W(e)$};
		\node [style=none] (472) at (155.75, 10.25) {};
		\node [style=none] (473) at (157.25, 10.25) {};
		\node [style=none] (474) at (156.25, 10.25) {};
		\node [style=none] (475) at (157.75, 10.25) {};
		\node [style=map] (476) at (156.75, 9) {$W(-f(t))$};
		\node [style=X] (477) at (158.5, 9) {$t$};
		\node [style=none] (478) at (158.5, 10.25) {};
	\end{pgfonlayer}
	\begin{pgfonlayer}{edgelayer}
		\draw [in=90, out=-30] (470) to (468);
		\draw [in=-60, out=90] (467.center) to (470);
		\draw [in=-120, out=90] (469) to (470);
		\draw [in=-150, out=90] (466.center) to (470);
		\draw (470) to (471);
		\draw (471) to (476);
		\draw [in=-90, out=60] (476) to (473.center);
		\draw [in=-90, out=30, looseness=0.75] (476) to (475.center);
		\draw [in=-90, out=120] (476) to (474.center);
		\draw [in=-90, out=150, looseness=0.75] (476) to (472.center);
		\draw [style=red] (477) to (478.center);
	\end{pgfonlayer}
\end{tikzpicture}
=
\begin{tikzpicture}
	\begin{pgfonlayer}{nodelayer}
		\node [style=none] (479) at (159.5, 6.25) {};
		\node [style=none] (480) at (161, 6.25) {};
		\node [style=X] (481) at (161.5, 6.5) {};
		\node [style=Z] (482) at (160, 6.5) {};
		\node [style=map] (483) at (160.5, 7.75) {$U$};
		\node [style=none] (484) at (159.5, 10) {};
		\node [style=none] (485) at (161, 10) {};
		\node [style=none] (486) at (160, 10) {};
		\node [style=none] (487) at (161.5, 10) {};
		\node [style=map] (488) at (160.5, 8.75) {$W(e-f(t))$};
		\node [style=X] (489) at (162.25, 8.75) {$t$};
		\node [style=none] (490) at (162.25, 10) {};
	\end{pgfonlayer}
	\begin{pgfonlayer}{edgelayer}
		\draw [in=90, out=-30] (483) to (481);
		\draw [in=-60, out=90] (480.center) to (483);
		\draw [in=-120, out=90] (482) to (483);
		\draw [in=-150, out=90] (479.center) to (483);
		\draw [in=-90, out=60] (488) to (485.center);
		\draw [in=-90, out=30, looseness=0.75] (488) to (487.center);
		\draw [in=-90, out=120] (488) to (486.center);
		\draw [in=-90, out=150, looseness=0.75] (488) to (484.center);
		\draw (483) to (488);
		\draw [style=red] (489) to (490.center);
	\end{pgfonlayer}
\end{tikzpicture}
$$


Note that this all works for $p=2$ when the stabilizer code has trivial phases: ie, it is a CSS code.



\chapter{Categorified Frobenius algebras}
\label{chap:grothendieck}


%
%%
%%%The Grothendieck construction establishes an equivalence of categories between pseudofunctors from categories $\X$ into $\Cat$ and fibrations over $\X$.  There is a two-sided variation of this construction:
%%%
%%%
%%%\begin{theorem}{Benabou-Grothendieck}
%%%Take a lax normal functor $F:\mathcal{I}\to\Prof$ from a 1-category $\mathcal{I}$.
%%%
%%%Then the following pullback exists:
%%%$$
%%%\xymatrix{
%%%\int F \ar[d]_{} \ar[r]^{} & \Prof_*  \ar[d]\\
%%%\mathcal{I}^{\op} \ar[r] ^F          & \Prof
%%%}
%%%$$
%%%
%%%Where $\int F$ is a 1-category and $\int F \to \X$ is a functor.
%%%This extends to an isomorphism of categories:
%%%$$\int:\Cat_{l,n}(\X, \Prof)\cong \Cat/\X:\delta$$
%%%\end{theorem}
%%%
%%%$\Cat_{l,n}(\X, \Prof)$ is the category of lax normal functors from $\X$ to $\Prof$ and lax natural transformations. $\Cat/\X$ is the slice category over $\X$.
%%%
%%%
%%%Explicitly:
%%%\begin{lemma}
%%%$\int F$ has:
%%%\begin{description}
%%%\item[Objects:] Pairs $(Y \in \X_0, Y^\sharp \in F(Y))$
%%%\item[Morphisms:] 
%%%
%%%
%%%
%%%\item[Composition]
%%%\end{description}
%%%The functor $\int F\to \mathcal{I}^\op$ is the first projection from the pullback.
%%%
%%%\end{lemma}
%%%and conversely
%%%\begin{lemma}
%%%Given a functor $\delta: \mathcal{J}\to\mathcal{I}^\op$ $\delta \pi$ is the lax functor such that
%%%\begin{description}
%%%\item
%%%\item
%%%\itemend{description}
%%%\end{lemma}
%%%
%%%
%%%
%%%The Grothendieck construction has been extended to the monoidal case: establishing an equivalence between monoidal fibrations and monoidal pseudofunctors \cite[].  Similarly \cite[] prove the polycategorical Benabou-Grothendieck equivalence.  We adapt the work of both authors, by establishing a monoidal Grothendieck-Benabou equivalence:
%%%%
%%%%
%%%%
%%%%\begin{theorem}[Monoidal Grothendieck-Benabou construction.]
%%%%There is an equivalence of categories between the lax monoidal, lax normal functor category:
%%%%
%%%%\begin{description}
%%%%\item[Objects:]
%%%%\item[Maps:]
%%%%\item[Identity:]
%%%%\item[Composition:]
%%%%\end{description}
%%%%
%%%%and  the monoidal slice cateogry:
%%%%
%%%%\begin{description}
%%%%\item[Objects:]
%%%%\item[Maps:]
%%%%\item[Identity:]
%%%%\item[Composition:]
%%%%\end{description}
%%%%
%%%%\end{theorem}
%%%%\begin{proof}
%%%%Proof strategy: Start with a monoidal functor $\Y\to \X$ between strict monoidal categories.  Build a lax normal lax monoidal functor $\X\to \Prof$
%%%%
%%%%
%%%%Let $\X$ be a monoidal category, regarded as a monoidal bicategory with one object, and take a lax monoidal lax functor $F:\X\to\Prof$.  Then the following pullback exists in the category of monoidal bicategories, lax monoidal lax functors and lax natural transformations:
%%%%$$
%%%%\xymatrix{
%%%%\int F \ar[d]_{} \ar[r]^{} & \Prof_*  \ar[d]\\
%%%%\X \ar[r] ^F          & \Prof
%%%%}
%%%%$$
%%%%
%%%%$\int F$ has the same categorical structure as in the non-monoidal 
%%%%
%%%%
%%%%\end{proof}
%%%
%%%
%%%
%%%If $F$ is a normal frobenius monoidal lax normal functor
%%%with laxator $\ell$ monoidal laxator $\mu$ oplaxator $\nu$
%%%, then $\int F$ has an induced monoidal structure with:
%%%
%%% $F(X\otimes Y)\to F(X)\otimes F(Y) \to F(X')\otimes F(Y') \to F(X'\otimes Y')$
%%%
%%%\begin{description}
%%%\item[Tensor product:] On Objects:
%%%$$
%%%(Y,Y^\sharp) \otimes (Z,Z^\sharp)
%%%:=
%%%(Y\otimes Z, \nu(\mu(Y^\sharp, Z^\sharp)))
%%%$$
%%%
%%%On morphisms:
%%%$$
%%%(f,f^\sharp) \otimes (g,g^\sharp)
%%%:=
%%%(f\otimes g, \nu(\mu(f^\sharp, g^\sharp)) )
%%%$$
%%%
%%%\item[Tensor unit:]
%%%$$
%%%(I,* \in F(I))=\mathbb{1})
%%%$$
%%%\item[Unitors:]
%%%$$
%%%u_{(X,X^\sharp)}^R: (X,X^\sharp)\otimes (I,*)=(X\otimes I, \mu^{\otimes}(X^\sharp, *))  \to (X,X^\sharp)
%%%$$
%%%is given by
%%%$$
%%%(u_X^{R}:X\otimes I\to X,  f\in F(u_X^{R})(X^\sharp, \mu^{\otimes}(X^\sharp, *) )   )
%%%$$
%%%
%%%Where $f$ is the identity on $X$ along the isomorphism $ \mu^{\otimes}(X^\sharp, *) = $
%%%
%%%$
%%%(u_X^{R})(X^\sharp, X^\sharp)
%%%\cong
%%%(u_X^{R})(X^\sharp, \mu^{\otimes}(X^\sharp, *))
%%%$
%%%
%%%
%%%\item[Associator:]
%%%
%%%$$
%%%(\alpha_{X,Y,Z},  g):((X\otimes Y)\otimes Z, \mu^\otimes(\mu^{\otimes}(X^\sharp, Y^\sharp),Z^\sharp) \to 
%%%(X\otimes (Y\otimes Z),\mu^\otimes(X^\sharp, \mu^\otimes(Y^\sharp,Z^\sharp)))
%%%$$
%%%
%%%\end{description}
%%%
%%%
%%%Where the projection map $p:\int F\to\mathcal I$ moreover is strong monoidal:
%%%
%%%$$
%%%p ((X,X^\sharp )\times (Y,Y^\sharp )) = 
%%%$$
%%%
%%%
%%% Since we are regarding $\Prof$ as a quasistrict monoidal bicategory, if $\mathcal  I$ is strict monoidal, then so is $\int F$ so that the projection $\int F\to \mathcal I $ is strict monoidal.
%%%
%%%
%%%Conversely, suppose there is a strong monoidal functor $\pi:\mathcal{J}\to\mathcal{I}^\op$ between monoidal categories.
%%%Then $\delta \pi:\mathcal{I}^\op \to \Prof$ is a Frobenius monoidal, lax normal functor.  The monoidal laxators are given by:
%%%
%%%
%%%
%%%
%%%
%%%
%%%Given a monoidal category $p:\X\to \mathbb{1}$, the Frobenius monoidal lax structure of $\delta p : \mathbb{1}\to \Prof$ regarded as a lax monoidal functor is precisely the data of a representable lax special \dag-Frobenius algebra in $\Prof$:
%%%
%%%GIVE AXIOMS
%%%
%%%
%%%This is essentially the two-sided version of the coherence data of a monoidal category. 
%%%
%%%bicateggorical spider theorem
%%%
%%%Since $\mathbb{1}$ is strict monoidal, $\int \delta p$ is as well.
%%%Therefore we can deduce that this is the strictification of $\X$.
%%%
%%%Slice category definition of grothendieck construction
%%%
%%%Recalling the string diagrams for pointed profunctors, we have that the strictification of $\X$ is generated by.
%%%
%%%
%%%The monoidal Benabou-Grothendieck construction is a very general construction for creating string diagrams for the strictification of monoidal functors.  Given some algebraic structure $F$ in  $\Prof$, the lax normal, lax monoidal structure can be regarded as the data for a normal form.  Then each object in $\int F$ contains the information need
%%%
%%%
%%%
%%%
%%%
%%%
%%%
%%%
%%%
%%%
%%%
%%%
%%%
%%%
%%%
%%%
%%%
%%%
%%%
%%%
%%%
%%%
%%%
%%%
%%%
%%%
%%%
%%%
%%%Monoidal categories $\X$ are in bijection with pseudomonoids in Cat.
%%%These are in bijection with extraspecial representable dagger frobenius algebras in Prof
%%%which are in bijection with lax seperable normal dagger frobenius monoidal lax functors $F_\X:\mathbb{1}\to\Prof$.
%%%
%%%Since $\mathbb 1$ is strict monoidal so is $\int F_\X$.
%%%Moreover, there is a $\dag$-Frobenius monoidal pseudo functor $\iota:\X\to \Prof_*$ making the diagram commute:
%%%
%%%$$
%%%\xymatrix{
%%%\X  \ar[drr]^\iota \ar[ddr]  & &\\
%%%       &  \int F_\X \ar[d]_{} \ar[r]^{} & \Prof_*  \ar[d]\\
%%%       &  \mathbb{1} \ar[r] ^F          & \Prof
%%%}
%%%$$
%%%
%%%Therefore, the universal map $G:\X\to F_\X$ is a Frobenius monoidal pseudofunctor.  It can also be shown to be strong monoidal, and moreover an equivalence of categories.  Therefore $\int F_\X$ is the monoidal strictification of $\X$.
%%%
%%%
%%%This extends to an equivalence of categories:
%%%
%%%Monoidal functors $\X\to \Y$ are in bijection with pseudomonoid homomorphisms in Cat.  These are in bijection with monoidal natural transformations 
%%%$F_\X\Rightarrow F_\Y$.
%%%These are in bijection with strict monoidal functors $\int F_\X\to \int F_\Y$. These are in 
%%%
%%%Surely intertwiners between pseudomonoid homorphisms correspond to strict monoidal natural transformations.
%%%
%%%
%%%
%%%
%%%
%%%
%%%
%%%
%%%
%%%
%%%
%%%First show
%%%
%%%\begin{description}
%%%\item[0-cells:] Frobenius monoidal functors \mathcal{I}\to\Prof$
%%%\item[1-cells:] Frobenius monoidal lax natural transformations.
%%%\item[2-cells:] Intertwiners
%%%\end{description}
%%%
%%%is 2-equivalent
%%%
%%%\begin{description}
%%%\item[0-cells:] \int F
%%%\item[1-cells:] strong monoidal functors $\int F \to \int G$ making the triangle commute.
%%%\item[2-cells:] monoidal natural transformations
%%%\end{description}
%%%
%%%
%%%
%%%
%%%
%%%
%%%
%%%
%%%
%%%
%%%
%%%Consider the bicategory of:
%%%
%%%\begin{description}
%%%\item[0-cells:] Monoidal categories
%%%\item[1-cells:] Monoidal functors.
%%%\item[2-cells:] Monoidal natural transformations
%%%\end{description}
%%%
%%%is 2-isomorphic 
%%%
%%%\begin{description}
%%%\item[0-cells:] Pseudomonoids in Cat
%%%\item[1-cells:] Pseudomonoid homomorphisms.
%%%\item[2-cells:] Intertwiners
%%%\end{description}
%%%
%%%is 2-isomorphic 
%%%
%%%\begin{description}
%%%\item[0-cells:] XXX Frobenius pseudomonoid 
%%%\item[1-cells:] Pseudomonoid homomorphisms.
%%%Composition by conjugation
%%%\item[2-cells:] Intertwiners
%%%\end{description}
%%%
%%%is 2-isomorphic 
%%%
%%%\begin{description}
%%%\item[0-cells:] Frobenius monoidal functors $\mathbb{1}\to\Prof$
%%%\item[1-cells:] Frobenius monoidal lax natural transformations.
%%%\item[2-cells:] Intertwiners
%%%\end{description}
%%%
%%%is 2-equivalent
%%%
%%%\begin{description}
%%%\item[0-cells:] \int F
%%%\item[1-cells:] \alpha:\int F\to \int G
%%%\item[2-cells:] Intertwiners
%%%\end{description}
%%%
%%%is 2-isomorphic 
%%%
%%%\begin{description}
%%%\item[0-cells:] Strict monoidal categories
%%%\item[1-cells:] monoidal 
%%%\item[2-cells:] Intertwiners
%%%\end{description}
%%%
%%%
%%%
%%%
%%%
%%%
%%%
%%%
%%%
%%%Scalable ZX-calculus.
%%%
%%%hierarchical string diagrams
%%%
%%%
%%%
%%%
%%%
%%%Take the lax Frobenius monoidal $F:\N \to\Prof$
%%%sending $n=\prod_i p_i^{a_i }\mapsto \prod \Span(\Mat_{\F_{p_i}})$  for where the tensor in $\N$ is multiplication.
%%%
%%%
%%%There is a faithful functor $\int F\to\FHilb$ picking out phase free ZX diagrams with arbitrary finite dimension. This is because for the full subcategory off prime prower dimension $\int F |_p \hookrightarrow \int F$, $\int F |_p\cong \Span(\Mat_{\F_{p_i}})$. Moreover, the maps given by the laxators are change of basis vectors.
%%%
%%%
%%%
%%%If instead we do the same trick but sending $p$ to odd prime dimensional qudit complete-ZX diagrams, then we regain the qufinite presentation of the ZX-calculus of \cite{wang???}.
%%%
%%%
%%%
%%%For stabilizers, we can do the same modulo scalars, but with $F:\N/\{2\} \to\Prof$ picking out odd prime dimensional stabilizer diagrams.
%%%
%%%
%%
%%
%%
%%
%%
%%
%%
%%
%%
%%
%%
%%\begin{definition}
%%Given bicategories $\X$ and $\Y$, a lax normal functor $\X\to\Y$ is:
%%
%%TODO
%%\end{definition}
%%
%%
%%\begin{definition}
%%Given monoidal bicategories $\X$ and $\Y$, a  Frobenius monoidal lax normal functor is a lax normal functor $\X\to\Y$, equipped with 2-cells $\mu$ and $\nu$ called the monoidal laxator and oplaxators, and coherences called the left and right frobeniusators interacting with the compositors todo 
%%
%%TODO
%%\end{definition}
%%
%%
%%
%%\begin{definition}
%%A morphism between Frobenius monoidal lax normal functors $F,G\X\to\Y$ is.  Given two monoidal bicategories, this notion induces the Frobenius monoidal lax normal functor category, denoted $[\X,\Y]_{fln}$
%%\end{definition}
%%
%%\begin{lemma}
%%$[\X,\Y]_{fln}$ is a monoidal category with
%%\end{lemma}
%%
%%
%%
%%\begin{definition}
%%A monoidal displayed category is a monoidal category $\D$ equipped with a  Frobenius monoidal lax functor $\mathcal{D}\to\Prof$.
%%\end{definition}
%%
%%\begin{theorem}{Monoidal Grothendieck-Benabou construction}
%%Given a monoidal category $\X$, there is a monoidal equivalence between the Frobenius monoidal lax normal functor category $[\X,\Prof]_{fln}$ and the strict monoidal coslice category over $\X$.
%%\end{theorem}
%%
%%
%%
%%
%%
%%
%%
%%\begin{proof}
%%Fix a monoidal category $\X$.
%%
%%Take a strict monoidal functor $p:\Y\to\X$.
%%
%%
%%Given an object $X$ of $\X$, the indexed category of $p$ over $X$, $p^{-1}(X)$ has:
%%
%%Objects in $Y \in \Y$ such that $p^{-1}(Y)=X$.
%%
%%Morphisms $f:Y\to Y'$ such that $p^{-1}(f)=1_X$.
%%
%%Composition and identities in $\Y$.
%%
%%
%%Given a morphism $f:X\to X'$ to in $\X$, the reindexing profunctor $p^{-1}(f):p^{-1}(X)^\op  \times p^{-1}(X')\to \Set$ sends:
%%
%%objects: $p^{-1}(f)(Y,Y')= \{ g:Y\to Y' | p(g)=f \}$
%%maps: $p^{-1}(f)(h:Y\to Y', k:Z \to Z')=\lambda x \in p^{-1}(f)(Y,Y'). h;x;k$.
%%
%%
%%This has the structure of a lax normal functor $P^{-1}:\X\to\Prof$.
%%
%%
%%For objects $X,X',X'' \in \X$ the compositors at $X,X',X''$ at components $f:X\to X'$ and $g:X'\to X''$ are functions $\int^Z p^{-1}(f)(X,Z) \times p^{-1}(g)(Z,X'') \to  p^{-1}(f;g)(X,X'')$,
%%sending elements $(h,k)$ of the equivalence class to their composite $h;k$.  This is a function because $p(h;k) = p(h);p(k)=f;g$.
%%
%%
%%For each $X \in \X$, $p^{-1}(f)=1_X$ is the identity profunctor on $p^{-1}(X)$, so this functor is normal.
%%The desired commutative diagrams hold making this into a lax normal functor.
%%
%%Furthermore, we now show that this lax normal functor preserves the monoidal structure lax-Frobeniusly.
%%
%% 
%%
%%%1_{p^{-1}(X)} \to 
%%
%%%
%%%The components of the monoidal laxtator at $(f,g)$
%%%
%%%$$
%%%p^{-1}(f) \times p^{-1}(g) \Rightarrow p^{-1}(f \otimes g)
%%%$$
%%%
%%%are given by functions
%%%$$
%%%p^{-1}(f)\times p^{-1}(g) = \int^{Y} p^{-1}(f) (X,Y) \times p^{-1}(g) (W,Z)
%%%\Rightarrow 
%%%p^{-1}(f) (X,Y)\times p^{-1}(g) (W,Z)
%%%= \{h:X\to Y | p(h)=f \}\times  \{k: W\to Z | p(k)=g \}
%%%\Rightarrow
%%%\{\ell :X\otimes W \to Z \otimes Y | p(\ell)= f\otimes g  \}
%%%$$
%%%
%%%taking $(h,k) \mapsto h\otimes k$
%%%
%%%and oplaxator:
%%%
%%%$$
%%% p^{-1}(f \otimes g) \Rightarrow p^{-1}(f) \times p^{-1}(g)
%%%$$
%%%
%%%by functions:
%%%
%%%$$
%%% p^{-1}(f \otimes g)
%%%=
%%% \{h:X\to Y | p(h)=f\otimes g \}
%%%$$
%%%
%%
%%
%%
%%
%%The components of the monoidal laxtator at $(f:X\to X',g:X''\to X''')$
%%
%%$$
%%p^{-1}(f) \times p^{-1}(g) \Rightarrow p^{-1}(f \otimes g)
%%$$
%%
%%are given by functions 
%%sending elements $(h,k)\in p^{-1}(f)\times p^{-1}(g)$ to $h\otimes k \in p^{-1}(f \otimes g)$.  This is a function because $p(h\otimes k) = p(h)\otimes p(k)= h\otimes k$.
%%
%%
%%Unitor
%%
%%
%%
%%For the oplaxator at $(f:X\to X',g:X''\to X''')$
%%
%%
%%$$
%%p^{-1}(f \otimes g) \Rightarrow p^{-1}(f) \times p^{-1}(g)  
%%$$
%%are given by functions 
%%sending elements $h \in p^{-1}(f\otimes g)$
%%
%%
%%% $h;k$.  This is a function because $p(h;k) = p(h);p(k)=f;g$.
%%\end{proof}
%
%
%
%\newcommand{\Cat}{{\sf Cat}}
%
%In this Section, we will propose how to categorify categorical quantum mechanics by regarding monoidal categories themselves as a certain kind of special-commutative \dag-Frobenius algebra.  This will give a categorical account of proof nets for monoidal categories which we reviewed in the first Chapter.  This chapter is much more exploratory than the others, and indeed there is much more work that should be worked out in the future.
%
%
%We first need the following definition to motivate where we are going:
%
%\begin{definition}
%The symmetric monoidal bicategory of {\bf pointed Categories} is the coslice category $\Cat_*:=1/\Cat$.  Explicitly, this has:
%
%\begin{description}
%\item[0-cells:] A pointed category is a pair consisting of a category along with a chosen object of that category: 
%
%$$(\X, X\in\X_0)$$
%
%\item[1-cells:] A pointed functor between pointed categories is a pair consisting of a functor between the underlying categories and a morphism that preserves the points, as follows:
%
%$$(F:\X\to\Y, f\in \Y(F(X),Y) ):(\X, X\in\X_0)\to (\Y, Y\in\Y_0)$$
%
%\item[2-cells:] Given two parallel pointed functors, 
%
%$$
% (F:\X\to\Y, f\in \Y(F(X),Y) ),  (G:\X\to\Y, g\in \Y(G(X),Y) ):(\X, X\in\X_0)\to $(\Y, Y\in\Y_0)$
%$$
%
%a pointed natural transformation is natural transformation  $\phi:F\to G$ that preserves the distinguished map, so that $\phi_X:g=f$.
%\end{description}
%
%
%Composition of the 1-cells and 2-cells is given pointwise; and the monoidal structure is given by the Cartesian product.
%
%\end{definition}
%
%There is a graphical calculus for pointed categories.  If $\X$ is a monoidal category, then for every map $f:X\otimes Y\to Z$, there is a pointed functor:
%
%
%$$(\_\otimes \_ :\X^2\to\X, f\in \Y(F(X),Y) ):(\X^2, (X,Y)\in\X^2_0)\to (\X, Z\in\X_0)$$
%
%Drawn as follows:
%
%TODO DRAW
%
%
%Moreover, for every state $f:I\to X$, there is a pointed functor
%
%
%$$(I:1\to\X, f\in \Y(I,X) ):(1,* \in 1)\to (\X, I\in\X_0)$$
%
%
%
%Drawn as follows:
%
%DRAW
%
%
%
%Where $1$ is the strict monoidal category with one object and one morphism and $I:1\to \X$ is the functor which picks out the tensor unit and its identity in $\X$.
%
%
%Now, the pentagon equation means that the following diagram commutes:
%
%DRAW ASSOCIATOR WITH TREES AND ELEMENTS
%
%And the left and right unit equations mean that the following diagrams commute:
%
%DRAW LEFT AND RIGHT UNITORS WITH ELEMENTS 
%
%
%Instead of defining a monoidal category as we did in the first chapter, we could have instead defined a monoidal category as a pseudomonoid in $\Cat$.  Then the tensor product would become the multiplication and the tensor unit would become the unit:
%
%\begin{definition}
%Draw pseudomonoid coherences
%
%
%
%
%
%\end{definition}
%
%However, this does not generalize to other algenraic structures in $\Cat$.  For this we will need the following definition:
%
%
%\begin{definition}
%
%A {\bf pseudofunctor }
%A {\bf lax monoidal pseudofunctor}
%
%
%\end{definition}
%
%With this in mind, we have a concise definition of a monoidal category:
%
%
%\begin{lemma}
%A monoidal category is the data of a  lax monoidal pseudofunctor $:1\to \Cat$, where $1$ is the terminal monoidal category with one object and one morphism.
%\end{lemma}
%
%
%
%This definition may seem terse, as it moves around the coherence data into a different place; however, this is much more amenable to generalization.
%Indeed,  a {\bf monoidal indexed category} is a monoidal category $\X$ equipped with a lax monoidal functor $F:\X\to \Cat$.  
%
%
%\begin{definition}
%To every monoidal indexed category $F:\X\to \Cat$, the {\bf monoidal Grothendieck category} $\int F$ is a monoidal category with:
%
%
%\begin{description}
%
%\item[Objects: ]
%
%\item[Maps: ]
%
%Where the composition is:
%
%And the identity is:
%
%
%\item[Monoidal structure:]
%
%\end{description}
%
%\end{definition}
%
%
%Note the the projection  $\int F\to \X$ is strict monoidal.
%
%
%This projection is actually more than just a strict monoidal functor, it is a fibration!  In \cite{???}, they show that given a fixed monoidal category $\X$, the slice category of strict monoidal fibrations is equivalent to the lax monoidal pseudofunctor category from $\X\to \Cat$.  We wont discuss this further, because it is not immediately useful for our purposes.
%
%
%If we return to our working example $F_\X:1\to \Cat$ picking out a monoidal category, we find that there is a strict monoidal isomorhism between $\int F_\X$ and $\X$:
%
%TODO Draw composition and tensor product in terms of tubes.
%
%
%
%However, we could also take a different monoidal pseudofunctor into $\Cat$.
%
%\begin{definition}
%
%Given a monoidal category $\X$, define a strong monoidal pseudofunctor: $\Delta_\X:\Delta\to \Cat$
%
%sending $n\mapsto \X^n$, where the laxator left associates trees and eliminates units.
%
%\end{definition}
%
%Explicitly:
%
%\begin{lemma}
%$\int \Delta_\X$ is the strict monoidal category with:
%\begin{description}
%\item[Objects:] Finite lists of objects in $\X$.
%
%\item[Maps:]  Maps are binary forests, which are left associated where the units are elminated whenever possible.
%
%\item[Tensor:]  The tensor product is given by the cartesian product in pointed functors.
%
%\item[Composition:] The composition is given by nesting and then reducing the all connected components to the same normal form.
%
%
%\end{description}
%\end{lemma}
%
%
%However, this is a very one-sided construction, as the string diagrams in $\int \Delta_\X$ has the shapes of trees. 
%
%Draw laxator:
%
%
%One way to get around this is look at $\X$ as an object in $\Prof$, rather than $\Cat$.  There are two yoneda embeddings $y^* \Cat^\co \to \Prof$ and $y_*:\Cat^\op \to \Prof$.  Therefore, we can regard a monoidal category as both a pseudomonoid in $\Prof$ or a pseudocomonoid in $\Prof$. 
%
%
%
%\begin{definition}
%The monoidal bicategory of pointed profunctors, $\Prof_*$, has
%
%
%\begin{description}
%\item[0-cells:]
%
%\item[1-cells:]
%
%\item[2-cells:]
%\end{description}
%\end{definition}
%
%
%Pointed profunctors has a very similar graphical calculus to pointed categories.  Given a monoidal category one can regard not only factorizations of the domains of maps $f:X\otimes Y\to Z$, and states $g:I\to X$ as pointed profunctors, but also factorizations of the codomains of maps   $g:X \to Y\times Z$, and effects $h:X\to I$:
%
%
%TODO
%
%
%Now, it is known that autonomous monoidal categories correspond to pseudofrobenius algebras in prof which interact with the units and counits of the adjunctions between the the different components of the yoneda embedding, see \cite{???,???}.  However, the literature is quite terse and lacking; indeed, in the literature, many people miss this extra coherence equation governing the interaction of the pseudofrobenius algebra with the adjoints.  Moreover, a general monoidal category would not induce a pseudo frobenius algebra, but a Lax frobenius algebra. 
%
%The lax frobeniusators are given by the following 2-cells; which are hardly ever going to be invertible:
%
%TODO
%
%Given a monoidal category $\X$, it is therefore and open question as to what notion of functor is needed to regard the monoidal structure of  $\X$ as some sort of monoidal functor $1\to\Prof$.  There are lots of coherence equations to work out, but I conjecture that it is a lax normal functor that is lax monoidal and oplax monoidal such that the lax and oplax structures interact to form a lax frobenius algebra where the monoidal lax structure  is XXXX adjoint to the oplax structure. 
%
%
%As much as it would be nice to complete the picture, we don't really need this.  
%
%\begin{definition}
%Given a category $\X$ and a bicategory $\Y$, a lax normal functor is a pseudofunctor where the compositors are no longer required to be isomorphisms.
%A {\bf displayed category } is a category $\X$ equipped with a lax normal functor $\X\to \Prof$.
%\end{definition}
%
%\begin{lemma}
%Given a displayed category $F:\X\to \Prof$, the Grothendieck-Benabou category $\Pi \X$ has:
%
%\begin{description}
%\item[Objects:]
%\item[Maps:]
%\item[Composition:]
%\item[Identities:]
%\end{description}
%
%
%\end{lemma}
%
%
%Moreover, the projection $\Pi \X\to \X$ is a functor.  Given some fixed category $\X$, the correspondence extends between an equivalence of categories between the lax normal functor category from $\X$ to $\Prof$ and the slice category over $\X$.  This correspondence exists in several places in the literature and is thought to be due to Benabou.  It has recently started being called the Grothendieck-Benabou construction.
%
%
%\begin{lemma}
%Take a displayed category $F:\X\to \Prof$ such that $\X$ is monoidal and $F$ has the structure of a strong monoidal 2-functor.
%
%Then $\Pi \X$ has a monoidal structure given by:
%
%\end{lemma}
%
%
%%%%%This is where the important stuff is
%
%\begin{defintion}
%Given a monoidal category $\X$, there is a strong monoidal lax functor $G_\X:frob\to \Prof$ with
%
%\begin{description}
%\item[Objects:] $n\mapsto \X^n$
%\item[Compositor:] 
%\item[Unitor:] 
%\item[Tensorator:] 
%\item[Tensor-unitor:] 
%\end{description}
%
%
%\end{defintion}
%
%
%\begin{lemma}
%The Grothendieck benabou category $\Pi_\X$ is monoidal with:
%
%
%\begin{description}
%\item[Objects:] Finite lists of objects in $\X$.
%
%\item[Maps:] Two-sided forests in spider normal form, alongside elements:
%
%\item[Composition:] Composition is given by spider fusion of shapes.
%
%\item[Tensor:] The tensor product is given by the cartesian product in pointed profunctors.
%
%
%
%\end{description}
%\end{lemma}
%
%Proof nets live within the subcategory whose shapes are connected.  To actually get things to be connected, we have to force things to be connected:
%
%
%\begin{lemma}
%
%we can define a different tensor product on this category that squeezed to top and bottom wires together:
%
%\begin{definition}
%Given an object $X=[X_0,\cdots, X_{n-1}]$, define the projector on $X$ to be the map which inserts identities into the following diagram:
%
%Take the full subcategory of the Karoubi envelope, $\Lambda \X$, consisting of these projectors.
%
%This is the same as pr
%\end{definition}
%
%
%
%
%
%
%
%
%
%
%
%
%
%
%
%
%
%
%
%
%
%
%
%
%
%
%
%
%
%
%
%
%
%
%
%
%
%
%
%
%
%
%
%
%
%

In this Chapter, we attempt a high-level reconstruction of  proof nets by categorifying Frobenius algebras and appealing to a conjectural categorified spider theorem.  Starting with a monoidal category, we obtain a monoidal category which is not itself the strictification, but which is closely related.

We  hope that the highly conceptual level of the construction opens up avenues for finding string diagrams for more general algebraic structures; our goal being to find canonical ways to glue monoidal categories together and obtain new kinds of string diagrams thereof.  This kind of problem has previously been considered in the setting where one wants to glue together monoidal categories along multiple different monoidal functors \cite{lobski}.  The motivation therein is to give a semantics for ways to explain systems which are stratified into multiple different layers, although they obtain a monoidal bicategory, rather than an ordinary monoidal category where one has to keep track of coherence information.


Indeed, throughout this thesis, we have been drawing string diagrams in monoidal, or symmetric monoidal categories in which our familiar notions of monoids and Frobenius algebras and so on live.  However, if we move to the higher dimensional setting of monoidal bicategories, all of the equalities defining the axioms for these structures become natural transformations subject to  coherence conditions.   


For this Chapter, unlike the rest, I will assume that the reader has some familiarity with monoidal bicategories.


\section{Pointed categories and multicategories}
 First, we categorify a monoid in a monoidal category:

\begin{definition}
A {\bf pseudomonoid} in a monoidal bicategory is an object $\mathcal C$ equipped with two 1-cells ${C} \otimes \mathcal{C} \to \mathcal{C}$ and $\one \to \mathcal C$ drawn as follows:
$$
\begin{tikzpicture}
	\begin{pgfonlayer}{nodelayer}
		\node [style=Z] (4) at (2.5, 0) {};
		\node [style=none] (5) at (2.5, 1) {};
		\node [style=none] (6) at (2, -1) {};
		\node [style=none] (7) at (3, -1) {};
		\node [style=Z] (11) at (4.5, 0) {};
		\node [style=none] (12) at (4.5, 1) {};
		\node [style=none] (14) at (3.5, 0) {$,$};
	\end{pgfonlayer}
	\begin{pgfonlayer}{edgelayer}
		\draw [in=-135, out=90] (6.center) to (4);
		\draw (4) to (5.center);
		\draw [in=-45, out=90] (7.center) to (4);
		\draw (11) to (12.center);
	\end{pgfonlayer}
\end{tikzpicture}
$$





 as well as three 2-cells, the associator, left and right unitors:

$$
\begin{tikzpicture}
	\begin{pgfonlayer}{nodelayer}
		\node [style=Z]  (0) at (4, 0) {};
		\node [style=Z]  (1) at (4.5, 1) {};
		\node [style=none] (2) at (3.5, -1) {};
		\node [style=none] (3) at (4.5, -1) {};
		\node [style=none] (4) at (5.5, -1) {};
		\node [style=none] (5) at (4.5, 2) {};
		\node [style=Z]  (6) at (7.5, 0) {};
		\node [style=Z]  (7) at (7, 1) {};
		\node [style=none] (8) at (8, -1) {};
		\node [style=none] (9) at (7, -1) {};
		\node [style=none] (10) at (6, -1) {};
		\node [style=none] (11) at (7, 2) {};
		\node [style=none] (12) at (5.75, 1) {$\xRightarrow{\alpha}$};
		\node [style=Z]  (13) at (9.5, 1) {};
		\node [style=none] (15) at (10, 0) {};
		\node [style=none] (16) at (9, 0) {};
		\node [style=none] (18) at (9.5, 1.75) {};
		\node [style=Z]  (19) at (9, 0) {};
		\node [style=none] (20) at (10.25, 1) {$\xRightarrow{u^L}$};
		\node [style=Z]  (21) at (12.5, 1) {};
		\node [style=none] (22) at (12, 0) {};
		\node [style=none] (23) at (13, 0) {};
		\node [style=none] (24) at (12.5, 1.75) {};
		\node [style=Z]  (25) at (13, 0) {};
		\node [style=none] (26) at (11.75, 1) {$\xLeftarrow{u^R}$};
		\node [style=none] (27) at (11, 1.75) {};
		\node [style=none] (28) at (11, 0) {};
		\node [style=none] (29) at (8.25, 0.75) {,};
	\end{pgfonlayer}
	\begin{pgfonlayer}{edgelayer}
		\draw [in=90, out=-135] (0) to (2.center);
		\draw [in=-45, out=90] (3.center) to (0);
		\draw [in=-135, out=90] (0) to (1);
		\draw (1) to (5.center);
		\draw [in=-45, out=90] (4.center) to (1);
		\draw [in=90, out=-45] (6) to (8.center);
		\draw [in=-135, out=90] (9.center) to (6);
		\draw [in=-45, out=90] (6) to (7);
		\draw (7) to (11.center);
		\draw [in=-135, out=90] (10.center) to (7);
		\draw [in=90, out=-45] (13) to (15.center);
		\draw [in=-135, out=90] (16.center) to (13);
		\draw (13) to (18.center);
		\draw [in=90, out=-135] (21) to (22.center);
		\draw [in=-45, out=90] (23.center) to (21);
		\draw (21) to (24.center);
		\draw (28.center) to (27.center);
	\end{pgfonlayer}
\end{tikzpicture}
$$

Satisfying the maclane pentagon coherence equation (where the dashed blue box indicates where the nonidentity natural transformation is being applied):

$$
\begin{tikzpicture}
	\begin{pgfonlayer}{nodelayer}
		\node [style=none] (0) at (3.5, -4) {};
		\node [style=none] (1) at (4, -4) {};
		\node [style=none] (2) at (4.5, -4) {};
		\node [style=none] (3) at (5, -4) {};
		\node [style=none] (4) at (4.25, -1.5) {};
		\node [style=Z] (5) at (4.25, -3.25) {};
		\node [style=Z] (6) at (3.75, -2.75) {};
		\node [style=Z] (7) at (4.25, -2.25) {};
		\node [style=none] (8) at (2.75, -2.5) {$\xRightarrow{\alpha}$};
		\node [style=none] (9) at (5.75, -4) {};
		\node [style=none] (10) at (6.25, -4) {};
		\node [style=none] (11) at (6.75, -4) {};
		\node [style=none] (12) at (7.25, -4) {};
		\node [style=none] (13) at (6.5, -1.5) {};
		\node [style=Z] (14) at (6.5, -3.25) {};
		\node [style=Z] (15) at (6.5, -2.25) {};
		\node [style=none] (16) at (5.25, -2.5) {$\xRightarrow{\alpha}$};
		\node [style=Z] (17) at (7, -2.75) {};
		\node [style=none] (18) at (8, -2.5) {$\xRightarrow{\alpha}$};
		\node [style=none] (19) at (2.75, 0.75) {$\xRightarrow{\alpha}$};
		\node [style=none] (20) at (5.5, 0.75) {$\cong$};
		\node [style=none] (21) at (8, 0.75) {$\xRightarrow{\alpha}$};
		\node [style=none] (22) at (1.75, -3.5) {};
		\node [style=none] (23) at (1.75, -2.5) {};
		\node [style=none] (24) at (0.5, -2.5) {};
		\node [style=none] (25) at (0.5, -3.5) {};
		\node [style=none] (26) at (4.5, -3) {};
		\node [style=none] (27) at (4.5, -2) {};
		\node [style=none] (28) at (3.5, -2) {};
		\node [style=none] (29) at (3.5, -3) {};
		\node [style=none] (30) at (7.5, -3.5) {};
		\node [style=none] (31) at (7.5, -2.5) {};
		\node [style=none] (32) at (6.25, -2.5) {};
		\node [style=none] (33) at (6.25, -3.5) {};
		\node [style=none] (34) at (3.25, -0.75) {};
		\node [style=none] (35) at (3.75, -0.75) {};
		\node [style=none] (36) at (4.25, -0.75) {};
		\node [style=none] (37) at (4.75, -0.75) {};
		\node [style=Z] (38) at (3.5, 0) {};
		\node [style=Z] (39) at (4.5, 0.5) {};
		\node [style=Z] (40) at (4, 1) {};
		\node [style=none] (41) at (4, 1.75) {};
		\node [style=none] (42) at (5, -0.25) {};
		\node [style=none] (43) at (5, 0.75) {};
		\node [style=none] (44) at (3.25, 0.75) {};
		\node [style=none] (45) at (3.25, -0.25) {};
		\node [style=none] (46) at (7, 0.25) {};
		\node [style=none] (47) at (7, 1.25) {};
		\node [style=none] (48) at (6, 1.25) {};
		\node [style=none] (49) at (6, 0.25) {};
		\node [style=none] (50) at (2.25, 0.25) {};
		\node [style=none] (51) at (2.25, 1.25) {};
		\node [style=none] (52) at (1, 1.25) {};
		\node [style=none] (53) at (1, 0.25) {};
		\node [style=none] (54) at (6, -0.75) {};
		\node [style=none] (55) at (6.5, -0.75) {};
		\node [style=none] (56) at (7, -0.75) {};
		\node [style=none] (57) at (7.5, -0.75) {};
		\node [style=Z] (58) at (6.25, 0.5) {};
		\node [style=Z] (59) at (7.25, 0) {};
		\node [style=Z] (60) at (6.75, 1) {};
		\node [style=none] (61) at (6.75, 1.75) {};
		\node [style=none] (62) at (10, -0.5) {};
		\node [style=none] (63) at (9.5, -0.5) {};
		\node [style=none] (64) at (9, -0.5) {};
		\node [style=none] (65) at (8.5, -0.5) {};
		\node [style=Z] (66) at (9.75, 0.25) {};
		\node [style=Z] (67) at (9.25, 0.75) {};
		\node [style=Z] (68) at (8.75, 1.25) {};
		\node [style=none] (69) at (8.75, 2) {};
		\node [style=none] (70) at (1.5, -1.25) {$\shortparallel$};
		\node [style=none] (71) at (9.25, -1.25) {$\shortparallel$};
		\node [style=none] (72) at (10, -4) {};
		\node [style=none] (73) at (9.5, -4) {};
		\node [style=none] (74) at (9, -4) {};
		\node [style=none] (75) at (8.5, -4) {};
		\node [style=Z] (76) at (9.75, -3.25) {};
		\node [style=Z] (77) at (9.25, -2.75) {};
		\node [style=Z] (78) at (8.75, -2.25) {};
		\node [style=none] (79) at (8.75, -1.5) {};
		\node [style=none] (80) at (0.5, -4) {};
		\node [style=none] (81) at (1, -4) {};
		\node [style=none] (82) at (1.5, -4) {};
		\node [style=none] (83) at (2, -4) {};
		\node [style=Z] (84) at (0.75, -3.25) {};
		\node [style=Z] (85) at (1.25, -2.75) {};
		\node [style=Z] (86) at (1.75, -2.25) {};
		\node [style=none] (87) at (1.75, -1.5) {};
		\node [style=none] (88) at (0.5, -0.75) {};
		\node [style=none] (89) at (1, -0.75) {};
		\node [style=none] (90) at (1.5, -0.75) {};
		\node [style=none] (91) at (2, -0.75) {};
		\node [style=Z] (92) at (0.75, 0) {};
		\node [style=Z] (93) at (1.25, 0.5) {};
		\node [style=Z] (94) at (1.75, 1) {};
		\node [style=none] (95) at (1.75, 1.75) {};
	\end{pgfonlayer}
	\begin{pgfonlayer}{edgelayer}
		\draw [in=90, out=-60] (5) to (2.center);
		\draw [in=-120, out=90] (1.center) to (5);
		\draw (5) to (6);
		\draw [in=-165, out=90] (6) to (7);
		\draw (7) to (4.center);
		\draw [in=90, out=-45, looseness=0.75] (7) to (3.center);
		\draw [in=90, out=-120] (6) to (0.center);
		\draw [in=90, out=-60] (14) to (11.center);
		\draw [in=-120, out=90] (10.center) to (14);
		\draw (15) to (13.center);
		\draw [in=90, out=-165] (17) to (14);
		\draw [in=-45, out=90, looseness=0.75] (12.center) to (17);
		\draw (17) to (15);
		\draw [in=90, out=-150] (15) to (9.center);
		\draw [color=blue,dashed] (22.center) to (23.center) to (24.center) to (25.center) to cycle;
		\draw [color=blue,dashed] (26.center) to (27.center) to (28.center) to (29.center) to cycle;
		\draw [color=blue,dashed] (30.center) to (31.center) to (32.center) to (33.center) to cycle;
		\draw [in=90, out=-60] (38) to (35.center);
		\draw [in=90, out=-120] (38) to (34.center);
		\draw (41.center) to (40);
		\draw [in=90, out=-15] (40) to (39);
		\draw [in=90, out=-60] (39) to (37.center);
		\draw [in=-120, out=90] (36.center) to (39);
		\draw [in=90, out=-135] (40) to (38);
		\draw [color=blue,dashed] (42.center) to (43.center) to (44.center) to (45.center) to cycle;
		\draw [color=blue,dashed] (46.center) to (47.center) to (48.center) to (49.center) to cycle;
		\draw [color=blue,dashed] (50.center) to (51.center) to (52.center) to (53.center) to cycle;
		\draw [in=90, out=-60] (58) to (55.center);
		\draw [in=90, out=-120] (58) to (54.center);
		\draw (61.center) to (60);
		\draw [in=90, out=-45] (60) to (59);
		\draw [in=90, out=-60] (59) to (57.center);
		\draw [in=-120, out=90] (56.center) to (59);
		\draw [in=90, out=-165] (60) to (58);
		\draw [in=90, out=-120] (66) to (63.center);
		\draw [in=90, out=-60] (66) to (62.center);
		\draw (69.center) to (68);
		\draw [in=90, out=-15] (68) to (67);
		\draw [in=90, out=-15] (67) to (66);
		\draw [in=90, out=-120] (67) to (64.center);
		\draw [in=90, out=-120] (68) to (65.center);
		\draw [in=90, out=-120] (76) to (73.center);
		\draw [in=90, out=-60] (76) to (72.center);
		\draw (79.center) to (78);
		\draw [in=90, out=-15] (78) to (77);
		\draw [in=90, out=-15] (77) to (76);
		\draw [in=90, out=-120] (77) to (74.center);
		\draw [in=90, out=-120] (78) to (75.center);
		\draw [in=90, out=-60] (84) to (81.center);
		\draw [in=90, out=-120] (84) to (80.center);
		\draw (87.center) to (86);
		\draw [in=90, out=-165] (86) to (85);
		\draw [in=90, out=-165] (85) to (84);
		\draw [in=90, out=-60] (85) to (82.center);
		\draw [in=90, out=-60] (86) to (83.center);
		\draw [in=90, out=-60] (92) to (89.center);
		\draw [in=90, out=-120] (92) to (88.center);
		\draw (95.center) to (94);
		\draw [in=90, out=-165] (94) to (93);
		\draw [in=90, out=-165] (93) to (92);
		\draw [in=90, out=-60] (93) to (90.center);
		\draw [in=90, out=-60] (94) to (91.center);
	\end{pgfonlayer}
\end{tikzpicture}
$$


As well as the unit coherences:

$$
\begin{tikzpicture}
	\begin{pgfonlayer}{nodelayer}
		\node [style=none] (47) at (2.75, 0.75) {$\xRightarrow{\alpha}$};
		\node [style=none] (100) at (1.5, -1.25) {$\shortparallel$};
		\node [style=none] (120) at (0.75, -0.75) {};
		\node [style=none] (121) at (2.25, -0.75) {};
		\node [style=Z] (122) at (1.5, -0.25) {};
		\node [style=Z] (123) at (1.25, 0.5) {};
		\node [style=Z] (124) at (1.75, 1.25) {};
		\node [style=none] (125) at (1.75, 1.75) {};
		\node [style=none] (126) at (5, -0.5) {};
		\node [style=none] (127) at (5, 0.75) {};
		\node [style=none] (128) at (3.75, 0.75) {};
		\node [style=none] (129) at (3.75, -0.5) {};
		\node [style=none] (130) at (4.75, -0.75) {};
		\node [style=none] (131) at (3.25, -0.75) {};
		\node [style=Z] (132) at (4, -0.25) {};
		\node [style=Z] (133) at (4.25, 0.5) {};
		\node [style=Z] (134) at (3.75, 1.25) {};
		\node [style=none] (135) at (3.75, 1.75) {};
		\node [style=none] (136) at (2.25, 0.25) {};
		\node [style=none] (137) at (2.25, 1.5) {};
		\node [style=none] (138) at (1, 1.5) {};
		\node [style=none] (139) at (1, 0.25) {};
		\node [style=Z] (144) at (7.25, 0.75) {};
		\node [style=none] (145) at (7.25, 1.75) {};
		\node [style=none] (146) at (6.75, -0.75) {};
		\node [style=none] (147) at (7.75, -0.75) {};
		\node [style=none] (148) at (6, 0.75) {$\xRightarrow{ u^L}$};
		\node [style=none] (149) at (7.25, -1.25) {$\shortparallel$};
		\node [style=none] (150) at (0.5, -4.25) {};
		\node [style=none] (151) at (2, -4.25) {};
		\node [style=Z] (152) at (1.25, -3.75) {};
		\node [style=Z] (153) at (1, -3) {};
		\node [style=Z] (154) at (1.5, -2.25) {};
		\node [style=none] (155) at (1.5, -1.75) {};
		\node [style=none] (156) at (1.5, -4) {};
		\node [style=none] (157) at (1.5, -2.75) {};
		\node [style=none] (158) at (0.25, -2.75) {};
		\node [style=none] (159) at (0.25, -4) {};
		\node [style=none] (160) at (4.25, -3.25) {$\xRightarrow{u^R}$};
		\node [style=Z] (161) at (7.25, -3) {};
		\node [style=none] (162) at (7.25, -1.75) {};
		\node [style=none] (163) at (6.75, -4.25) {};
		\node [style=none] (164) at (7.75, -4.25) {};
	\end{pgfonlayer}
	\begin{pgfonlayer}{edgelayer}
		\draw (125.center) to (124);
		\draw [in=90, out=-165] (124) to (123);
		\draw [in=90, out=-45] (123) to (122);
		\draw [in=90, out=-135] (123) to (120.center);
		\draw [in=90, out=-45, looseness=0.75] (124) to (121.center);
		\draw [color=blue,dashed] (126.center) to (127.center) to (128.center) to (129.center) to cycle;
		\draw (135.center) to (134);
		\draw [in=90, out=-15] (134) to (133);
		\draw [in=90, out=-135] (133) to (132);
		\draw [in=90, out=-45] (133) to (130.center);
		\draw [in=90, out=-135, looseness=0.75] (134) to (131.center);
		\draw [color=blue,dashed] (136.center) to (137.center) to (138.center) to (139.center) to cycle;
		\draw (145.center) to (144);
		\draw [in=90, out=-45] (144) to (147.center);
		\draw [in=-135, out=90] (146.center) to (144);
		\draw (155.center) to (154);
		\draw [in=90, out=-165] (154) to (153);
		\draw [in=90, out=-45] (153) to (152);
		\draw [in=90, out=-135] (153) to (150.center);
		\draw [in=90, out=-45, looseness=0.75] (154) to (151.center);
		\draw [color=blue,dashed] (156.center) to (157.center) to (158.center) to (159.center) to cycle;
		\draw (162.center) to (161);
		\draw [in=90, out=-45] (161) to (164.center);
		\draw [in=-135, out=90] (163.center) to (161);
	\end{pgfonlayer}
\end{tikzpicture}
$$

A pseudomonoid is {\bf strict} when the associator and unitors are idenities.

\end{definition}

If we take $\Cat$ to be a monoidal bicategory with respect to the Cartesian product, then monoidal categories have a slick definition:

\begin{lemma}
Monoidal categories are pseudomonoids in $\Cat$ and strict monoidal categories are strict pseudomonoids in $\Cat$.
\end{lemma}

By looking at the points inside of $\Cat$, this way of viewing a monoidal category hints at some connection to proof nets:

\begin{definition}
The symmetric monoidal bicategory of {\bf pointed categories} is the coslice bicategory $\Cat_*:=\one/\Cat$.  Explicitly, this has:

\begin{description}
\item[0-cells:] Pointed categories, pairs consisting of a category along with a chosen object of that category: 

$$(\X, X\in\X_0)$$

\item[1-cells:] Pointed functor between pointed categories,   pairs consisting of a functor between the underlying categories and a morphism that preserves the point:

$$(F:\X\to\Y, f\in \Y(F(X),Y) ):(\X, X\in\X_0)\to (\Y, Y\in\Y_0)$$

\item[2-cells:] Given two parallel pointed functors $(F,f),  (G,g):(\X,X)\to (\Y,X)$,
a pointed natural transformation is natural transformation  $\phi:F\to G$ that preserves the distinguished map, so that $\phi_X:g=f$.
\end{description}


Composition of the 1-cells and 2-cells is given pointwise; and the monoidal structure is given by the Cartesian product.

\end{definition}

There is a graphical calculus for pointed profunctors, which we will first specialize to pointed categories.  To my knowledge the graphical calculus is not fully worked out in the literature.  Initially called ``internal string diagrams''  by \cite{vicary} when applied to  $\Vect$-enriched Profunctors, these have also been called ``open diagrams'' by \cite{mario} for $\Sets$-enriched profunctors.  We will return to the case of $\Sets$-enriched profunctors shortly.

Given the pointed functor,$(1_\X, f \in \X(1_{\X}(X),Y))$, draw the identity as a cylinder and the map inside the cylinder as follows:


$$
\begin{tikzpicture}[scale=1.5]
    \node[Cyl, bot, top, height scale=1.0] (A) at (0,0) {};
    \begin{scope}[internal string scope]
        \node (i) at  ($(A.bot)+(0,-.3)$)  {\tiny $X$};
        \node (j) at ($(A.top)+(0,.3)$)  {\tiny $Y$};
        \node [tiny label] (g) at (A.center) {$f$};
        \draw (A.bot) to (A.top);
    \end{scope}
\end{tikzpicture}
$$
Think of the maps inside the category as ``living'' within the cylinder; wherein one can apply rewrite rules coming from the equational theory of the category.  

Functors are drawn as membranes between separating the cylinder for the domain and codomain category; functoriality means that things can pass up through the membrane:


$$
\begin{tikzpicture}[scale=1.5]
    \node[Cyl, bot, top,xscale=1.5,yscale=1.1] (A) at (0,0) {};
    \node[Cyl, bot, anchor=top,xscale=1.5,yscale=1.1] (B) at (A.bot) {};
    \node (g) at ($(A.bot)+(-.5,0)$) {\tiny $F$};
    \begin{scope}[internal string scope]
        \node (i) at  ($(B.bot)+(0,-.3)$)  {\tiny $X$};
        \node (j) at ($(A.top)+(0,.3)$)  {\tiny $F(Y)$};
        \node [tiny label] (g) at (B.center) {$f$};
        \draw (B.bot) to (A.top);
    \end{scope}
\end{tikzpicture}
=
\begin{tikzpicture}[scale=1.5]
    \node[Cyl, bot, top,xscale=1.5,yscale=1.1] (A) at (0,0) {};
    \node[Cyl, bot, anchor=top,xscale=1.5,yscale=1.1] (B) at (A.bot) {};
    \node (g) at ($(A.bot)+(-.5,0)$) {\tiny $F$};
    \begin{scope}[internal string scope]
        \node (i) at  ($(B.bot)+(0,-.3)$)  {\tiny $X$};
        \node (j) at ($(A.top)+(0,.3)$)  {\tiny $F(Y)$};
        \node [tiny label] (g) at (A.center) {$F(f)$};
        \draw (B.bot) to (A.top);
    \end{scope}
\end{tikzpicture}
$$

If $\X$ is a monoidal category, then one can tensor maps within the cobordism using the tensor product of $\X$, so that for $f:W\to X$ and $g:Y\to Z$, we have the the pointed functor  $(1_\X, f\times  \in \X(1_{\X}(W\otimes X),(Y\otimes Z)))$  with the following graphical representation:

$$
\begin{tikzpicture}[scale=1.5]
    \node[Cyl, bot, top,xscale=1.5,yscale=1.1] (A) at (0,0) {};
    \begin{scope}[internal string scope]
        \node [tiny label] (f) at ($(A.center)+(-.15,0)$) {$f$};
        \node [tiny label] (g) at ($(A.center)+(.15,0)$) {$g$};
        \draw ($(A.bot)+(-.15,0)$) to (f) to ($(A.top)+(-.15,0)$);
        \draw ($(A.bot)+(.15,0)$) to (g) to ($(A.top)+(.15,0)$) ;
        \node (i) at (A.top) [above] {\tiny $Y\otimes Z$};
        \node (j) at (A.bot) [below] {\tiny $W\otimes X$};
    \end{scope}
\end{tikzpicture}
$$

Moreover, for every map with a binary tensor factorization of the domain  $f:X\otimes Y\to Z$, we can use the external tensor product of $\Cat$ to obtain a pointed functor:

$$
(\_\otimes \_ :\X^2\to\X, f\in \Y(X\otimes Y ,Z) ):(\X^2, (X,Y)\in\X^2_0)\to (\X, Z\in\X_0)$$

Drawn as follows:

$$
\begin{tikzpicture}[scale=1.5]
    \node[Pants, bot, top, height scale=1.0] (A) at (0,0) {};
    \begin{scope}[internal string scope]
        \node (i) at (A.belt) [above] {\tiny $Z$};
        \node (j) at (A.leftleg) [below] {\tiny $X$};
        \node (k) at (A.rightleg) [below] { \tiny $Y$};
        \node [tiny label] (g) at (0,0.02\cobheight) {$f$};
        \draw (i.south)
            to (g.center)
            to [out=-140, in=up] (j.north);
        \draw (g.center)
            to [out=-40, in=up] (k.north);
    \end{scope}
\end{tikzpicture}
$$


And for every state $f:I\to X$, there is a pointed functor


$$(I:\one\to\X, f\in \Y(I,X) ):(1,* \in \one)\to (\X, I\in\X_0)$$



Drawn as follows:

$$
\begin{tikzpicture}[scale=1.5]
\setlength\cupheight{1.5\cupheight}
\node (i) at (0,0) [Cup, top] {};
\node (j) at (0,-\cobheight) [Bot3D, invisible] {};
\node (g) [tiny label] at (0,-0.4\cobheight) {$f$};
\begin{scope}[internal string scope]
\node (di) at (i.center) [above] {\tiny $A$};
\draw (i.center) to (g.center) {};
\end{scope}
\end{tikzpicture}
$$

Consider the action of the associator and unitor on the points:



$$
\begin{tikzpicture}[scale=1.5]
    \node[Pants, bot, top] (B) at (0,0) {};
    \node[Pants, bot, anchor=belt] (A) at (B.leftleg) {};
    \node[SwishL, bot, anchor=top] (C) at (B.rightleg) {};
    \begin{scope}[internal string scope]
        \node (i) at (B.belt) {};
        \node (j) at (A.leftleg){};
        \node (k) at (A.rightleg) {};
        \node (l) at (C.bot) {} ;
        \node [tiny label] (f) at (B.center) {$f$};
        \node [tiny label] (g) at (A.center) {$g$};
        \node [tiny label] (h) at (C.center) {$h$};
        \draw (f.center) to (i.center);
        \draw (j.center) to [out=up, in=-135] (g.center);
        \draw (k.center) to [out=up, in=-35] (g.center);
        \draw (g.center) to [out=90, in=-135] node [left=-4pt] {} (f.center);
        \draw (l.center)
            to [out=90, in=-80] (h.center) 
            to [out=up, in=down, out looseness=0.7]
                (B.rightleg)
            to [out=up, in=-45]
                node [right=-4pt, pos=0.11] {}
 (f.center);
    \end{scope}
    \end{tikzpicture}
    \xRightarrow{\alpha}
\begin{tikzpicture}[scale=1.5]
    \node[Pants, bot, top] (B) at (0,0) {};
    \node[Pants, bot, anchor=belt] (A) at (B.rightleg) {};
    \node[SwishR, bot, anchor=top] (C) at (B.leftleg) {};
    \begin{scope}[internal string scope]
        \node (i) at (B.belt) {};
        \node (j) at (C.bot)  {};
        \node (k) at (A.leftleg) {};
        \node (l) at (A.rightleg)  {};
        \node [tiny label] (f) at (0.05\cobwidth,0.18\cobheight) {$f$};
        \node [tiny label] (g)  at (-0.4\cobwidth,-0.1\cobheight) {$g$};
        \draw (j.center)
            to [out=up, in=-130] (g.center);
        \draw (k.center)
            to [out=up, in=down] (B-rightleg.in-leftthird)
            to [out=up, in=-40] (g.center);
        \draw (l.center) to [out=up, in=down] (B-rightleg.in-rightthird)
            to [out=up, in=-60] (f.center);
        \draw (f.center) to [out=90, in=-90, looseness=2] (i.center);
        \draw (f.center) to [in=90, out=-120] (g.center);
        \node [tiny label] (h) at (0.8\cobwidth,-0.25\cobheight) {$h$};
    \end{scope}
    \end{tikzpicture}\ ,
\hspace*{.5cm}
\begin{tikzpicture}[scale=1.5]
    \node[Pants, top, bot] (A) at (0,0) {};
    \node[Cup] (B) at (A.leftleg) {};
    \node[Cyl, bot, anchor=top] (C) at (A.rightleg) {};
    \begin{scope}[internal string scope]
        \node [tiny label] (f) at (0,0) {$f$};
        \node [tiny label] (g) at ([yshift=-0.3\cobheight] B) {$g$};
        \node [tiny label] (h) at (C) {$h$};
        \node (i) at (A.belt) [above] {};
        \node (i2) at (C.bottom) [below] {};
        \draw (f.center) to [out=-140, in=90] node [left=-3pt] {} (g.center);
        \draw (f.center) to (i);
        \draw (f.center)
            to [out=-40, in=90, looseness=0.9]
                node [right=-2pt, pos=0.4] {} (h.center);
        \draw (h.center) to (i2);
    \end{scope}
\end{tikzpicture}
\xRightarrow{u^L}
\begin{tikzpicture}[scale=1.5]
    \node[Cyl, tall, bot, top] (A) at (0,0) {};
    \begin{scope}[internal string scope]
        \node [tiny label] (f) at (0,0.4\cobheight) {$f$};
        \node [tiny label] (g) at (-0.20\cobwidth, -0.1\cobheight) {$g$};
        \node [tiny label] (h) at (0.20\cobwidth,-0.5\cobheight) {$h$};
        \node (i) at (A.top) [above] {};
        \node (i2) at (A.bot) [below] {};
        \draw (f.center) to [out=-120, in=90] (g.center);
        \draw (f.center) to (i);
        \draw (f.center)
            to [out=-60, in=90, looseness=0.9] (h.center);
        \draw (h.center) to [out=-90, in=up, looseness=1.2] (i2.north);
    \end{scope}
\end{tikzpicture}\ ,
\hspace*{.5cm}
\begin{tikzpicture}[scale=1.5]
    \node[Pants, top, bot] (A) at (0,0) {};
    \node[Cup] (B) at (A.rightleg) {};
    \node[Cyl, bot, anchor=top] (C) at (A.leftleg) {};
    \begin{scope}[internal string scope]
        \node [tiny label] (f) at (0,0) {$f$};
        \node [tiny label] (g) at ([yshift=-0.3\cobheight] B) {$g$};
        \node [tiny label] (h) at (C) {$h$};
        \node (i) at (A.belt) [above] {};
        \node (i2) at (C.bottom) [below] {};
        \draw (f.center) to [out=-40, in=90] node [right=-1pt] {} (g.center);
        \draw (f.center) to (i);
        \draw (f.center)
            to [out=-140, in=90, looseness=0.9]
                %node [left=-2pt, pos=0.4] {$Y$}
                (h.center);
        \draw (h.center) to (i2);
    \end{scope}
\end{tikzpicture}
\xRightarrow{u^R}
\begin{tikzpicture}[scale=1.5]
    \node[Cyl, tall, bot, top] (A) at (0,0) {};
    \begin{scope}[internal string scope]
        \node [tiny label] (f) at (0,0.4\cobheight) {$f$};
        \node [tiny label] (g) at (0.2\cobwidth, -0.1\cobheight) {$g$};
        \node [tiny label] (h) at (-0.2\cobwidth,-0.5\cobheight) {$h$};
        \node (i) at (A.top) [above] {};
        \node (i2) at (A.bot) [below] {};
        \draw (f.center) to [out=-60, in=90] (g.center);
        \draw (f.center) to (i);
        \draw (f.center)
            to [out=-120, in=90, looseness=0.9] (h.center);
        \draw (h.center) to [out=-90, in=90, looseness=1.2] (i2);
    \end{scope}
\end{tikzpicture}
$$


One way in which this diverges from proof nets is that these generators form a monoidal bicategory, not a monoidal category; forcing one to keep track of which 2-cells have been applied.  A-priori, in this kind of setting, there is no guarantee that two different ways to get to the same diagram are equal.



Luckily, for monoidal categories, this issue can be resovled via the multicategories.  Informally, a multicategory is like a category, except for the multimaps now go from lists of objects in the domain to a single object in the codomain.  Composition of multimaps corresponds to plugging a single output into an input as if one is nesting trees.  Strict monoidal categories correspond to the representable multicategories where every list of objects can be tensored together.


Formally multicategories can be constructed in a very similar way to internal categories (see \cite[Defininition 4.2.2]{leinster} for a more general, thorough treatment):

\begin{definition}
{\sf MultiSpan} is the bicategory with:

\begin{description}
\item[0-cells:] Sets
\item[1-cells:] 

$$
\dfrac{[X] \xleftarrow{f} A \xrightarrow{g} Y \hspace*{.5cm}\text{in $\Sets$}}
{X \xrightarrow{(A,f)} Y  \hspace*{.5cm}\text{in {\sf MultiSpan}}}
$$

The identity on $X$ is given by the span:
$$ [X] \xleftarrow{\eta_X } X = X$$


The composition of multispans
$$X \xrightarrow{(f,A,g)} Y\xrightarrow{(h,B,k)} Z $$

is given by taking the following pullback
$$
\xymatrix{
     &         &      &[A] {}_{[g]}\times_h B \ar[dr]^{\pi_1} \ar[dl]_{\pi_0} \\
     &         &[A] \ar[dr]^{[g]} \ar[dl]_{[f]}&      & B \ar[dl]_h \ar[ddrrr]^k &   & \\
     & [[X]] \ar[dl]_{\mu_X} &      &[Y] &    &    &  \\
[X] &         &      &     &    &    &&Z
}
$$

Yielding a 1-cell:

$$
X\xrightarrow{(\pi_0;[f];\mu_X, [A] {}_{[g]}\times_h B, \pi_1;k)} Z
$$
Where, recall that $[-]:\Sets\to\Sets $ is the list monad where the unit $\eta_X(x)\mapsto [x]$ inserts into the singleton list  and the multiplication $\mu$ flattens lists of lists.

\item[2-cells:] The 2-cell structure is the same as for spans and coherence 2-cells are essentially the same as for spans.

\end{description}
\end{definition}


\begin{definition}
A (small) {\bf multicategory} is a monad is {\sf MultiSpan}.
\end{definition}

So if we have a monad on the 1-cell $[ {\sf Ob}] \xleftarrow{\sf dom} {\sf Ar} \xrightarrow{\sf codom} {\sf Ob}$.  As opposed to the setting for internal categories, the domain is now a list of objects.  The way that the composition and unit are defined plugs the single object in the codomain into the list of objects in the domain.


%An equivalent way to define a pseudomonoid in $\Cat$ would be to ask for a monoidal pseudofunctor $\one\to \Cat$ from the terminal monoidal category into $\Cat$.  
%This is the same data as a multifunctor $\one \to \Cat$ from the one object multicategory to $\Cat$ regarded as a multicategory.

%Indeed, 

An equivalent way to define a pseudomonoid in $\Cat$ would be to ask for a monoidal pseudofunctor $\one\to \Cat$ from the terminal monoidal category into $\Cat$.  If one were to ask for a multifunctor from $\one\to \Cat$, this would give you what is sometimes called an unbiased monoidal category. This is very much like a monoidal category, except instead of there being a tensor product bi-functor, there is is a tensor product functor $\X^n\to \X$ for every arity $n \in \N$.


Hermida \cite{hermida}, showed that when one regards an (unbiased) monoidal category as a multifunctor $F:\one\to\Cat$, by computing the following pullback in the bicategory of 2-multicategories, the Grothendieck category $\int F_\X$ is precisely a representable multicategory:

$$
\xymatrix{
\int  F \ar[r]^{\pi_1} \ar[d]_{\pi_0} & \Cat_* \ar[d] \\
\one \ar[r]_F & \Cat
}
$$



And moreover, the projection map $\pi_0:F_\X\to \one$ is a fibration of multicategories.  He shows that there is an equivalence of bicategories between the multifunctor category $[\one,\Cat]$ and the subcategory of the slice category of multifibrations over $\one$.  
In some sense, this equivalence can be regarded as a different version of the coherence theorem for monoidal categories.
However, from a string diagrammatic perspective, this is unsatisfying.  Indeed, despite representable multicategories being in bijection with monoidal categories, the way that composition is defined in multicategories biases the inputs over the outputs; moreover, we are only allowed to compose along one object at a time.



\section{Pointed profunctors and polycategories}

To attempt to rectify this with the 2-sided nature of proof nets, we recall the category of internal profunctors described in Definition \ref{def:internalprof}, which is the 2-sided version of $\Cat$.  In this section it will be easier to work with profunctors {\em enriched} in $\Sets$, as opposed to profunctors internal to $\Sets$.  This is more general as well, because it allows us to work with locally small categories, rather than merely small categories:


\begin{definition}
The category of $\Prof$, of profunctors internal to $\Sets$ has:
\begin{description}
\item[0-cells:] Categories
\item[1-cells:]  The morphisms are {\em profunctors} given by the following correspondence:
$$
\dfrac
{F:\X^\op\times\Y\to\Sets \quad \in\ \Cat}
{F:\X\proarrow \Y\quad \in\ \Prof}
$$

The composition of profunctors $P:\X\proarrow \Y$ and $Q:\Y\proarrow \Z$ is given by the coend:
$$
P;Q := \int^{Y \in \Y } P(-,Y) \times Q(Y,=):\X \proarrow \Z
$$
Where the coend of a functor $F:\X^\op\times\X\to \Sets$ is given by the coequalizer diagram (in $\Cat$):
$$
\coprod_{X_1,X_2 \in \X} {\X}(X_1,X_2) \times F(X_1,X_2) \rightrightarrows \prod_{X \in \X} F(X,X) \to \int^{X \in \X} F(X,X)
$$


The identity profunctor on $\X$ is given by the hom functor
$$\X(-,=): \X^\op\times\X\to\Sets$$


Intutitively, this is a categorification of trace of a matrix where the natural numbers are replaced with categories, and the commutative ring is replaced with $\Sets$.  Thus, the composition of profunctors categorifies matrix multiplication.


\item[2-cells:]  2-cells between parallel profunctors $P,Q:\X\proarrow \Y$ are natural transformations between the underlying profunctors $P,Q:\X^\op\times \Y \to \Sets$.

\item[Compact closed structure:] The symmetric monoidal structure of $\Prof$ is given by extending the Cartesian structure in $\Cat$.   The units and counits of the compact closed structure are given by the hom functor.

\end{description}
\end{definition}


A comprehensive review of the basic theory of profunctors and their calculus is contained in \cite{fosco}.
There are two classes of profunctors which will be of interest to us:

\begin{definition}


A profunctor $\X\proarrow \Y$ is {\bf representable} when it is naturally isomorphic to the profunctor $F_*:=\Y(F-,=)$ for a functor $F: \X\to \Y$.


Dually, a profunctor $\Y\proarrow \X$ is {\bf corepresentable} when it is naturally isomorphic to the profunctor $F^*:=\Y(-,F=)$ for a functor $F: \X\to \mathbb{D}$.


\end{definition}

There are two embeddings of $\Cat$ into $\Prof$ that preserve the monoidal structure (formally, they are strong monoidal pseudofunctors):

\begin{definition}
The representable embedding $(\_)_*:\Cat^\co \to \Prof$ is the identity on objects,  covariant on 1-cells and contravariant on 2-cells.  It takes functors $F:\X\to \Y$ to profunctors  $F_*:\X\proarrow \Y$.


Dually, the corepresentable embedding $(\_)^*:\Cat^\op \to \Prof$  is the identity on objects, contravariant on 1-cells and covariant on 2-cells. It takes functors $F:\X\to \Y$ to profunctors  $F^*:\Y\proarrow \X$.

\end{definition}

These two embeddings interact nicely:

\begin{lemma}
Given any functor $F:\X\to \Y$, there is an adjunction $F_* \dashv F^*$ with unit $\eta^F$ and counit $\epsilon^F$.
\end{lemma}

For example, if we draw the two embeddings of the pseudomonoid as follows:
$$
\begin{tikzpicture}
	\begin{pgfonlayer}{nodelayer}
		\node [style=map]  (0) at (1, 0) {$\otimes_*$};
		\node [style=none] (1) at (1, 1) {};
		\node [style=none] (2) at (0.5, -1) {};
		\node [style=none] (3) at (1.5, -1) {};
		\node [style=Z] (4) at (2.5, 0) {};
		\node [style=none] (5) at (2.5, 1) {};
		\node [style=none] (6) at (2, -1) {};
		\node [style=none] (7) at (3, -1) {};
		\node [style=none] (8) at (1.75, 0) {$=:$};
		\node [style=map]  (9) at (4, 0) {$I_*$};
		\node [style=none] (10) at (4, 1) {};
		\node [style=Z] (11) at (5.5, 0) {};
		\node [style=none] (12) at (5.5, 1) {};
		\node [style=none] (13) at (4.75, 0) {$=:$};
		\node [style=none] (14) at (3.25, 0) {$,$};
		\node [style=map]  (15) at (7.25, 0) {$\otimes^*$};
		\node [style=none] (16) at (7.25, -1) {};
		\node [style=none] (17) at (6.75, 1) {};
		\node [style=none] (18) at (7.75, 1) {};
		\node [style=Z] (19) at (8.75, 0) {};
		\node [style=none] (20) at (8.75, -1) {};
		\node [style=none] (21) at (8.25, 1) {};
		\node [style=none] (22) at (9.25, 1) {};
		\node [style=none] (23) at (8, 0) {$=:$};
		\node [style=map]  (24) at (10.25, 0) {$I^*$};
		\node [style=none] (25) at (10.25, -1) {};
		\node [style=Z] (26) at (11.75, 0) {};
		\node [style=none] (27) at (11.75, -1) {};
		\node [style=none] (28) at (11, 0) {$=:$};
		\node [style=none] (29) at (9.5, 0) {$,$};
		\node [style=none] (30) at (6.25, 0) {$,$};
	\end{pgfonlayer}
	\begin{pgfonlayer}{edgelayer}
		\draw [in=-135, out=90] (2.center) to (0);
		\draw (0) to (1.center);
		\draw [in=-45, out=90] (3.center) to (0);
		\draw [in=-135, out=90] (6.center) to (4);
		\draw (4) to (5.center);
		\draw [in=-45, out=90] (7.center) to (4);
		\draw (9) to (10.center);
		\draw (11) to (12.center);
		\draw [in=135, out=-90] (17.center) to (15);
		\draw [in=90, out=-90] (15) to (16.center);
		\draw [in=45, out=-90] (18.center) to (15);
		\draw [in=135, out=-90] (21.center) to (19);
		\draw (19) to (20.center);
		\draw [in=45, out=-90] (22.center) to (19);
		\draw (24) to (25.center);
		\draw (26) to (27.center);
	\end{pgfonlayer}
\end{tikzpicture}
$$

Then we have the following 2-cells:


$$
\begin{tikzpicture}
	\begin{pgfonlayer}{nodelayer}
		\node [style=Z] (4) at (2.5, 0) {};
		\node [style=none] (5) at (2.5, 0.75) {};
		\node [style=Z] (6) at (2.5, -1) {};
		\node [style=none] (7) at (2.5, -1.75) {};
		\node [style=none] (9) at (3.75, 0.75) {};
		\node [style=none] (11) at (3.75, -1.75) {};
		\node [style=none] (12) at (3.25, -0.5) {$\xRightarrow{\eta^\otimes}$};
		\node [style=none] (13) at (4.25, -0.5) {,};
		\node [style=none] (14) at (5.25, 0.75) {};
		\node [style=none] (15) at (5.25, -1.75) {};
		\node [style=none] (16) at (6, 0.75) {};
		\node [style=none] (17) at (6, -1.75) {};
		\node [style=Z] (18) at (7.25, -1) {};
		\node [style=Z] (19) at (7.25, 0) {};
		\node [style=none] (20) at (6.75, 0.75) {};
		\node [style=none] (21) at (7.75, 0.75) {};
		\node [style=none] (22) at (6.75, -1.75) {};
		\node [style=none] (23) at (7.75, -1.75) {};
		\node [style=none] (24) at (6.5, -0.5) {$\xRightarrow{\epsilon^\otimes}$};
		\node [style=none] (25) at (8.75, -0.5) {,};
		\node [style=none] (26) at (11.25, 0.75) {};
		\node [style=none] (27) at (11.25, -1.75) {};
		\node [style=Z] (30) at (9.75, -1) {};
		\node [style=Z] (31) at (9.75, 0) {};
		\node [style=none] (36) at (10.5, -0.5) {$\xRightarrow{\epsilon^I}$};
		\node [style=none] (37) at (12, -0.5) {,};
		\node [style=Z] (40) at (14.5, -1) {};
		\node [style=Z] (41) at (14.5, 0) {};
		\node [style=none] (42) at (13.75, -0.5) {$\xRightarrow{\epsilon^I}$};
		\node [style=none] (43) at (12.5, 0) {};
		\node [style=none] (44) at (13.25, 0) {};
		\node [style=none] (45) at (13.25, -1) {};
		\node [style=none] (46) at (12.5, -1) {};
	\end{pgfonlayer}
	\begin{pgfonlayer}{edgelayer}
		\draw (4) to (5.center);
		\draw (6) to (7.center);
		\draw [bend right=45, looseness=1.25] (4) to (6);
		\draw [bend left=45, looseness=1.25] (4) to (6);
		\draw (11.center) to (9.center);
		\draw (15.center) to (14.center);
		\draw (17.center) to (16.center);
		\draw [in=-30, out=90] (23.center) to (18);
		\draw [in=90, out=-150] (18) to (22.center);
		\draw (18) to (19);
		\draw [in=-90, out=30] (19) to (21.center);
		\draw [in=-90, out=150] (19) to (20.center);
		\draw (27.center) to (26.center);
		\draw (30) to (31);
		\draw (40) to (41);
		\draw [dashed] (45.center) to (44.center);
		\draw [dashed] (44.center) to (43.center);
		\draw [dashed] (43.center) to (46.center);
		\draw [dashed] (46.center) to (45.center);
	\end{pgfonlayer}
\end{tikzpicture}
$$

Indeed for any monoidal category, the counit for the tensor product adjunction induces lax-Frobeniusators:

$$
\phi^L:=
\left(
\begin{tikzpicture}
	\begin{pgfonlayer}{nodelayer}
		\node [style=Z]  (96) at (13.25, 1) {};
		\node [style=Z]  (97) at (14.25, 1.75) {};
		\node [style=none] (98) at (14.25, 2.5) {};
		\node [style=none] (99) at (13, 2.5) {};
		\node [style=none] (100) at (13.25, -0.5) {};
		\node [style=none] (101) at (14.5, -0.5) {};
		\node [style=none] (102) at (14.75, 0) {};
		\node [style=none] (103) at (14.75, 0.5) {};
		\node [style=none] (104) at (13, 0.5) {};
		\node [style=none] (105) at (13, 0) {};
		\node [style=Z]  (106) at (16.5, 1.5) {};
		\node [style=Z]  (107) at (17.5, 2.25) {};
		\node [style=none] (108) at (17.5, 2.75) {};
		\node [style=none] (109) at (16.25, 2.75) {};
		\node [style=none] (110) at (17, 0.75) {};
		\node [style=none] (111) at (17, 0.75) {};
		\node [style=none] (112) at (16.5, -1) {};
		\node [style=none] (113) at (17.5, -1) {};
		\node [style=Z]  (114) at (17, 0.75) {};
		\node [style=Z]  (115) at (17, -0.25) {};
		\node [style=none] (116) at (17.75, 0.25) {};
		\node [style=none] (117) at (17.75, 1.75) {};
		\node [style=none] (118) at (16, 1.75) {};
		\node [style=none] (119) at (16, 0.25) {};
		\node [style=Z]  (120) at (20.25, 1.5) {};
		\node [style=none] (122) at (20.25, 2.75) {};
		\node [style=none] (123) at (19, 2.75) {};
		\node [style=none] (124) at (19.75, 0.75) {};
		\node [style=none] (125) at (19.75, 0.75) {};
		\node [style=none] (126) at (19.25, -1) {};
		\node [style=none] (127) at (20.25, -1) {};
		\node [style=Z]  (128) at (19.75, 0.75) {};
		\node [style=Z]  (129) at (19.75, -0.25) {};
		\node [style=Z]  (134) at (20.25, 2.25) {};
		\node [style=none] (135) at (20.75, 1.25) {};
		\node [style=none] (136) at (20.75, 2.5) {};
		\node [style=none] (137) at (19.75, 2.5) {};
		\node [style=none] (138) at (19.75, 1.25) {};
		\node [style=none] (144) at (22, -0.5) {};
		\node [style=none] (145) at (23, -0.5) {};
		\node [style=Z]  (147) at (22.5, 0.25) {};
		\node [style=none] (148) at (22, 2) {};
		\node [style=none] (149) at (23, 2) {};
		\node [style=Z]  (150) at (22.5, 1.25) {};
		\node [style=none] (151) at (15.5, 1) {$\xRightarrow{\epsilon^\otimes}$};
		\node [style=none] (152) at (18.5, 1) {$\xRightarrow{\alpha_*^{-1}}$};
		\node [style=none] (153) at (21.5, 1) {$\xRightarrow{\eta^\otimes}$};
	\end{pgfonlayer}
	\begin{pgfonlayer}{edgelayer}
		\draw [in=-75, out=90] (101.center) to (97);
		\draw [in=15, out=-165, looseness=0.75] (97) to (96);
		\draw (96) to (100.center);
		\draw [in=270, out=105] (96) to (99.center);
		\draw (98.center) to (97);
		\draw [color=blue, dashed] (102.center) to (103.center);
		\draw [color=blue, dashed] (103.center) to (104.center);
		\draw [color=blue, dashed] (104.center) to (105.center);
		\draw [color=blue, dashed] (105.center) to (102.center);
		\draw [in=-75, out=45, looseness=0.75] (111.center) to (107);
		\draw [in=15, out=-165, looseness=0.75] (107) to (106);
		\draw [in=135, out=-90] (106) to (110.center);
		\draw [in=270, out=105] (106) to (109.center);
		\draw (108.center) to (107);
		\draw [in=90, out=-150] (115) to (112.center);
		\draw (115) to (114);
		\draw [in=90, out=-30] (115) to (113.center);
		\draw [color=blue, dashed] (116.center) to (117.center);
		\draw [color=blue, dashed] (117.center) to (118.center);
		\draw [color=blue, dashed] (118.center) to (119.center);
		\draw [color=blue, dashed] (119.center) to (116.center);
		\draw [in=45, out=-90] (120) to (124.center);
		\draw [in=90, out=-150] (129) to (126.center);
		\draw (129) to (128);
		\draw [in=90, out=-30] (129) to (127.center);
		\draw [bend left=45, looseness=1.25] (134) to (120);
		\draw [bend left=45, looseness=1.25] (120) to (134);
		\draw (122.center) to (134);
		\draw [in=150, out=-90, looseness=0.75] (123.center) to (128);
		\draw [color=blue, dashed] (135.center) to (136.center);
		\draw [color=blue, dashed] (136.center) to (137.center);
		\draw [color=blue, dashed] (137.center) to (138.center);
		\draw [color=blue, dashed] (138.center) to (135.center);
		\draw [in=90, out=-150] (147) to (144.center);
		\draw [in=90, out=-30] (147) to (145.center);
		\draw [in=-90, out=150] (150) to (148.center);
		\draw [in=-90, out=30] (150) to (149.center);
		\draw (147) to (150);
	\end{pgfonlayer}
\end{tikzpicture}
\right)
$$

$$
\phi^R:=
\left(
\begin{tikzpicture}
	\begin{pgfonlayer}{nodelayer}
		\node [style=Z]  (0) at (14.5, 1) {};
		\node [style=Z]  (1) at (13.5, 1.75) {};
		\node [style=none] (2) at (13.5, 2.5) {};
		\node [style=none] (3) at (14.75, 2.5) {};
		\node [style=none] (4) at (14.5, -0.5) {};
		\node [style=none] (5) at (13.25, -0.5) {};
		\node [style=none] (6) at (13, 0) {};
		\node [style=none] (7) at (13, 0.5) {};
		\node [style=none] (8) at (14.75, 0.5) {};
		\node [style=none] (9) at (14.75, 0) {};
		\node [style=Z]  (10) at (17.25, 1.5) {};
		\node [style=Z]  (11) at (16.25, 2.25) {};
		\node [style=none] (12) at (16.25, 2.75) {};
		\node [style=none] (13) at (17.5, 2.75) {};
		\node [style=none] (14) at (16.75, 0.75) {};
		\node [style=none] (15) at (16.75, 0.75) {};
		\node [style=none] (16) at (17.25, -1) {};
		\node [style=none] (17) at (16.25, -1) {};
		\node [style=Z]  (18) at (16.75, 0.75) {};
		\node [style=Z]  (19) at (16.75, -0.25) {};
		\node [style=none] (20) at (16, 0.25) {};
		\node [style=none] (21) at (16, 1.75) {};
		\node [style=none] (22) at (17.75, 1.75) {};
		\node [style=none] (23) at (17.75, 0.25) {};
		\node [style=Z]  (24) at (19.5, 1.5) {};
		\node [style=none] (25) at (19.5, 2.75) {};
		\node [style=none] (26) at (20.75, 2.75) {};
		\node [style=none] (27) at (20, 0.75) {};
		\node [style=none] (28) at (20, 0.75) {};
		\node [style=none] (29) at (20.5, -1) {};
		\node [style=none] (30) at (19.5, -1) {};
		\node [style=Z]  (31) at (20, 0.75) {};
		\node [style=Z]  (32) at (20, -0.25) {};
		\node [style=Z]  (33) at (19.5, 2.25) {};
		\node [style=none] (34) at (19, 1.25) {};
		\node [style=none] (35) at (19, 2.5) {};
		\node [style=none] (36) at (20, 2.5) {};
		\node [style=none] (37) at (20, 1.25) {};
		\node [style=none] (38) at (23, -0.5) {};
		\node [style=none] (39) at (22, -0.5) {};
		\node [style=Z]  (40) at (22.5, 0.25) {};
		\node [style=none] (41) at (23, 2) {};
		\node [style=none] (42) at (22, 2) {};
		\node [style=Z]  (43) at (22.5, 1.25) {};
		\node [style=none] (44) at (15.5, 1) {$\xRightarrow{\epsilon^\otimes}$};
		\node [style=none] (45) at (18.5, 1) {$\xRightarrow{\alpha_*}$};
		\node [style=none] (46) at (21.5, 1) {$\xRightarrow{\eta^\otimes}$};
	\end{pgfonlayer}
	\begin{pgfonlayer}{edgelayer}
		\draw [in=-105, out=90] (5.center) to (1);
		\draw [in=165, out=-15, looseness=0.75] (1) to (0);
		\draw (0) to (4.center);
		\draw [in=-90, out=75] (0) to (3.center);
		\draw (2.center) to (1);
		\draw [color=blue, dashed] (6.center) to (7.center);
		\draw [color=blue, dashed] (7.center) to (8.center);
		\draw [color=blue, dashed] (8.center) to (9.center);
		\draw [color=blue, dashed] (9.center) to (6.center);
		\draw [in=-105, out=135, looseness=0.75] (15.center) to (11);
		\draw [in=165, out=-15, looseness=0.75] (11) to (10);
		\draw [in=45, out=-90] (10) to (14.center);
		\draw [in=-90, out=75] (10) to (13.center);
		\draw (12.center) to (11);
		\draw [in=90, out=-30] (19) to (16.center);
		\draw (19) to (18);
		\draw [in=90, out=-150] (19) to (17.center);
		\draw [color=blue, dashed] (20.center) to (21.center);
		\draw [color=blue, dashed] (21.center) to (22.center);
		\draw [color=blue, dashed] (22.center) to (23.center);
		\draw [color=blue, dashed] (23.center) to (20.center);
		\draw [in=135, out=-90] (24) to (27.center);
		\draw [in=90, out=-30] (32) to (29.center);
		\draw (32) to (31);
		\draw [in=90, out=-150] (32) to (30.center);
		\draw [bend right=45, looseness=1.25] (33) to (24);
		\draw [bend right=45, looseness=1.25] (24) to (33);
		\draw (25.center) to (33);
		\draw [in=30, out=-90, looseness=0.75] (26.center) to (31);
		\draw [color=blue, dashed] (34.center) to (35.center);
		\draw [color=blue, dashed] (35.center) to (36.center);
		\draw [color=blue, dashed] (36.center) to (37.center);
		\draw [color=blue, dashed] (37.center) to (34.center);
		\draw [in=90, out=-30] (40) to (38.center);
		\draw [in=90, out=-150] (40) to (39.center);
		\draw [in=-90, out=30] (43) to (41.center);
		\draw [in=-90, out=150] (43) to (42.center);
		\draw (40) to (43);
	\end{pgfonlayer}
\end{tikzpicture}
\right)
$$

These Frobeniusators interact with the (co)associators and (co)unitors of the (co)omonoid and adjoints to to satisfy several coherences forming what Franco et al call ``map monoidal object'' \cite[Rem. 6.3]{dualsinvert}. They also remark that when the monoidal category is additionally autonomous, so that it has duals, the Frobeniusators are invertible.  As a technical note,  map monoidal objects in $\Prof$ are not precisely monoidal categories, as this biconditional only holds for Cauchy-complete categories.


For context, a similar unpublised result Shulman discussed on the n-category cafe \cite{shula} as well as in his paper \cite{shulb}, characterizes biclosed linearly distributive categories and  *-autonomous categories as, respectively, lax and pseudo-Frobenius algebras in the compact closed bicategory multivariable adjunctions.  This refines the similar result of  \cite{Street2004}, where it is shown that  Cauchy complete  *-autonomous categories are in bijection with Frobenius pseudomonoids in Prof.



We can repeatedly apply natural transformations to reduce diagrams composed of the pseudo-Frobenius algbra structure coming from a monoidal category:

\begin{definition}
Fix a monoidal category $\X$. Say that a connected diagram in $\Prof$ composed of the generators of the corresponding pseudomonoid is in {\bf spider normal form} when it is any of the four following types of diagrams (the last of which we will call a scalar spider):


$$
\begin{tikzpicture}
	\begin{pgfonlayer}{nodelayer}
		\node [style=Z] (0) at (1.25, 3) {};
		\node [style=Z] (1) at (0.5, 4) {};
		\node [style=Z] (2) at (1.25, 2.25) {};
		\node [style=Z] (3) at (0.5, 1.25) {};
		\node [style=none] (4) at (1.5, 4) {};
		\node [style=none] (5) at (1.5, 1.25) {};
		\node [style=none] (6) at (0.25, 0.5) {};
		\node [style=none] (7) at (1.5, 4.75) {};
		\node [style=none] (8) at (1.5, 0.5) {};
		\node [style=none] (9) at (0.75, 4.75) {};
		\node [style=none] (10) at (0.25, 4.75) {};
		\node [style=none] (11) at (0.75, 0.5) {};
		\node [style=none] (12) at (1, 3.25) {};
		\node [style=none] (13) at (0.5, 3.75) {};
		\node [style=none] (14) at (0.5, 1.5) {};
		\node [style=none] (15) at (1, 2) {};
		\node [style=none] (16) at (0.75, 3.5) {$\ddots$};
		\node [style=none] (17) at (0.75, 1.75) {$\reflectbox{$\ddots$}$};
		\node [style=none] (18) at (1.2, 0.5) {$\cdots$};
		\node [style=none] (19) at (1.2, 4.75) {$\cdots$};
	\end{pgfonlayer}
	\begin{pgfonlayer}{edgelayer}
		\draw (7.center) to (4.center);
		\draw [in=105, out=-90] (10.center) to (1);
		\draw [in=60, out=-90, looseness=0.75] (4.center) to (0);
		\draw [in=-90, out=75] (1) to (9.center);
		\draw [in=300, out=90] (5.center) to (2);
		\draw [in=90, out=-120] (3) to (6.center);
		\draw [in=90, out=-60] (3) to (11.center);
		\draw (8.center) to (5.center);
		\draw (0) to (2);
		\draw (3) to (14.center);
		\draw (15.center) to (2);
		\draw (13.center) to (1);
		\draw (0) to (12.center);
	\end{pgfonlayer}
\end{tikzpicture}
=:
\begin{tikzpicture}
	\begin{pgfonlayer}{nodelayer}
		\node [style=none] (0) at (1.5, 1.75) {};
		\node [style=none] (1) at (2.75, 1.75) {};
		\node [style=none] (2) at (2, 1.75) {};
		\node [style=none] (3) at (2.45, 1.75) {$\cdots$};
		\node [style=none] (4) at (2.75, 3.25) {};
		\node [style=none] (5) at (2, 3.25) {};
		\node [style=none] (6) at (1.5, 3.25) {};
		\node [style=none] (7) at (2.45, 3.25) {$\cdots$};
		\node [style=Z] (8) at (2, 2.5) {};
	\end{pgfonlayer}
	\begin{pgfonlayer}{edgelayer}
		\draw [in=-90, out=45] (8) to (4.center);
		\draw (8) to (5.center);
		\draw [in=135, out=-90] (6.center) to (8);
		\draw [in=90, out=-150] (8) to (0.center);
		\draw (2.center) to (8);
		\draw [in=90, out=-30] (8) to (1.center);
	\end{pgfonlayer}
\end{tikzpicture}
\ ,
\hspace*{.5cm}
\begin{tikzpicture}
	\begin{pgfonlayer}{nodelayer}
		\node [style=Z] (3) at (2.5, 3) {};
		\node [style=none] (4) at (2.5, 2.25) {};
	\end{pgfonlayer}
	\begin{pgfonlayer}{edgelayer}
		\draw (4.center) to (3);
	\end{pgfonlayer}
\end{tikzpicture}
\ ,
\hspace*{.5cm}
\begin{tikzpicture}
	\begin{pgfonlayer}{nodelayer}
		\node [style=Z] (5) at (3.5, 2.25) {};
		\node [style=none] (6) at (3.5, 3) {};
	\end{pgfonlayer}
	\begin{pgfonlayer}{edgelayer}
		\draw (6.center) to (5);
	\end{pgfonlayer}
\end{tikzpicture}
\ ,
\hspace*{.5cm}
\begin{tikzpicture}
	\begin{pgfonlayer}{nodelayer}
		\node [style=Z] (0) at (1.5, 3) {};
		\node [style=Z] (2) at (1.5, 2.25) {};
	\end{pgfonlayer}
	\begin{pgfonlayer}{edgelayer}
		\draw (0) to (2);
	\end{pgfonlayer}
\end{tikzpicture}
$$


Say that a not necessarily connected diagram is in {\bf stratified spider normal form} when it can be composed into a strictly progressive sequence of nonscalar spiders followed by a strictly progressive sequence of scalar spiders:

$$
\begin{tikzpicture}
	\begin{pgfonlayer}{nodelayer}
		\node [style=none] (7) at (4.5, 1.75) {};
		\node [style=none] (8) at (5.75, 1.75) {};
		\node [style=none] (9) at (5, 1.75) {};
		\node [style=none] (10) at (5.45, 1.75) {$\cdots$};
		\node [style=none] (11) at (5.75, 3.25) {};
		\node [style=none] (12) at (5, 3.25) {};
		\node [style=none] (13) at (4.5, 3.25) {};
		\node [style=none] (14) at (5.45, 9.5) {$\cdots$};
		\node [style=Z] (15) at (5, 2.5) {};
		\node [style=none] (16) at (6.5, 3.25) {};
		\node [style=none] (17) at (7.75, 3.25) {};
		\node [style=none] (18) at (7, 3.25) {};
		\node [style=none] (19) at (7.45, 1.75) {$\cdots$};
		\node [style=none] (20) at (7.75, 4.75) {};
		\node [style=none] (21) at (7, 4.75) {};
		\node [style=none] (22) at (6.5, 4.75) {};
		\node [style=none] (23) at (7.45, 9.5) {$\cdots$};
		\node [style=Z] (24) at (7, 4) {};
		\node [style=none] (25) at (9, 4.75) {};
		\node [style=none] (26) at (10.25, 4.75) {};
		\node [style=none] (27) at (9.5, 4.75) {};
		\node [style=none] (28) at (9.95, 1.75) {$\cdots$};
		\node [style=none] (29) at (10.25, 6.25) {};
		\node [style=none] (30) at (9.5, 6.25) {};
		\node [style=none] (31) at (9, 6.25) {};
		\node [style=none] (32) at (9.95, 9.5) {$\cdots$};
		\node [style=Z] (33) at (9.5, 5.5) {};
		\node [style=none] (34) at (5, 9.5) {};
		\node [style=none] (35) at (4.5, 9.5) {};
		\node [style=none] (36) at (5.75, 9.5) {};
		\node [style=none] (37) at (7, 9.5) {};
		\node [style=none] (38) at (6.5, 9.5) {};
		\node [style=none] (39) at (7.75, 9.5) {};
		\node [style=none] (40) at (7, 1.75) {};
		\node [style=none] (41) at (6.5, 1.75) {};
		\node [style=none] (42) at (7.75, 1.75) {};
		\node [style=none] (43) at (9.5, 1.75) {};
		\node [style=none] (44) at (9, 1.75) {};
		\node [style=none] (45) at (10.25, 1.75) {};
		\node [style=none] (46) at (8.4, 4.45) {\reflectbox{$\ddots$}};
		\node [style=Z] (47) at (11.5, 7) {};
		\node [style=Z] (48) at (11.5, 6.25) {};
		\node [style=none] (49) at (9.5, 9.5) {};
		\node [style=none] (50) at (9, 9.5) {};
		\node [style=none] (51) at (10.25, 9.5) {};
		\node [style=Z] (52) at (12.25, 8) {};
		\node [style=Z] (53) at (12.25, 7.25) {};
		\node [style=Z] (54) at (13.75, 9.25) {};
		\node [style=Z] (55) at (13.75, 8.5) {};
		\node [style=none] (56) at (13, 8.2) {\reflectbox{$\ddots$}};
	\end{pgfonlayer}
	\begin{pgfonlayer}{edgelayer}
		\draw [in=-90, out=45] (15) to (11.center);
		\draw (15) to (12.center);
		\draw [in=135, out=-90] (13.center) to (15);
		\draw [in=90, out=-150] (15) to (7.center);
		\draw (9.center) to (15);
		\draw [in=90, out=-30] (15) to (8.center);
		\draw [in=-90, out=45] (24) to (20.center);
		\draw (24) to (21.center);
		\draw [in=135, out=-90] (22.center) to (24);
		\draw [in=90, out=-150] (24) to (16.center);
		\draw (18.center) to (24);
		\draw [in=90, out=-30] (24) to (17.center);
		\draw [in=-90, out=45] (33) to (29.center);
		\draw (33) to (30.center);
		\draw [in=135, out=-90] (31.center) to (33);
		\draw [in=90, out=-150] (33) to (25.center);
		\draw (27.center) to (33);
		\draw [in=90, out=-30] (33) to (26.center);
		\draw (35.center) to (13.center);
		\draw (12.center) to (34.center);
		\draw (11.center) to (36.center);
		\draw (22.center) to (38.center);
		\draw (37.center) to (21.center);
		\draw (20.center) to (39.center);
		\draw (41.center) to (16.center);
		\draw (18.center) to (40.center);
		\draw (42.center) to (17.center);
		\draw (25.center) to (44.center);
		\draw (43.center) to (27.center);
		\draw (26.center) to (45.center);
		\draw (47) to (48);
		\draw (29.center) to (51.center);
		\draw (30.center) to (49.center);
		\draw (50.center) to (31.center);
		\draw (52) to (53);
		\draw (54) to (55);
	\end{pgfonlayer}
\end{tikzpicture}
$$

\end{definition}

\begin{lemma}
Given a monoidal category $\X$, and a connected diagram in $\Prof$ generated by the $1$-cells of the pseudo-Frobenius structure induced by $\X$ one can always reduce the diagram to spider normal form by repeated application of

$$\phi^L,\phi^R,\eta^\otimes,\alpha_*,\alpha^*,(u^L)_*,(u^L)^*,(u^R)_*,(u^R)^*$$

As well as the symmetric monoidal structure of $\Prof$.  

Furthermore, given any {\em not-necessarily connected} diagram can be reduced to stratified spider normal form using the same collection of $2$-cells, where the order of the scalars spiders with respect to the tensor product is preserved by normalization.
\end{lemma}


This lemma is an obvious corollary  of spider theorem for special Frobenius algebras; however, the following is not  so immediate:


\begin{conjecture}
The stratified spider normal form is strictly confluent so that any parallel $2$-cells witnessing the reduction to the spider normal form are equal.
\end{conjecture}


In some sense, it seems as if this should be seen to follow from the coherence theorem for monoidal categories; however, the author is unable to prove it.
The rest of this chapter relies on this categorification of the spider theorem being true.  


A similar result is proven for diagrams generated by the free pseudo-Frobenius algebra \cite{dunn}.  However, because the monoid and the comonoid are not required to be adjoint to each other, they require that the shapes be simply connected.  They also require that the shape has a nontrivial boundary, so that scalars are not allowed.

From now on, let us assume that the scalars in our monoidal category are central, so that they commute with all other maps.  Because of the way we moved the scalar spiders out of the way to perform spider fusion unobstructed, we have to make this concession if we want to have any meaningful semantics.

Let us ``look inside'' $\Prof$ just as we did for $\Cat$, to see how normalization acts on points.

\begin{definition}
The symmetric monoidal bicategory of {\bf pointed profunctors}, $\Prof_*$ has:

\begin{description}
\item[0-cells:] Pointed categories:

$$(\X, X\in\X_0)$$

\item[1-cells:] A pointed functor  between pointed categories is a pair consisting of a profunctor between the underlying categories and a morphism that preserves the point:

$$(F:\X\proarrow\Y, f\in F(X,Y)):(\X, X\in\X_0)\to (\Y, Y\in\Y_0)$$

\item[2-cells:] Given two parallel pointed functors $(F, f),  (G, g):(\X, X)\to (\Y,Y)$,
a pointed 2-cell is 2-cell $\phi:F\Rightarrow G$ of profunctors that preserves the distinguished map, so that $\phi_X:g=f$.
\end{description}

\end{definition}


The graphical calculus for pointed profunctors is essentially the same as for pointed categories.  Except now, due to the two different yoneda embeddings, we can factorize the domain and codomain of maps.  For example, consider the action unit and counit for the tensor product:


$$
  \begin{tikzpicture}[scale=1.5]
    \node[Pants, top] (pants) {};
    \node[Copants, bot, lowercob, anchor=leftleg] (copants) at (pants.leftleg) {};
   % \node[Top3D] at (copants.rightleg) {};
    \node[Bot3D] at (pants.rightleg) {};
    \node[Bot3D] at (pants.leftleg) {};
   \begin{scope}[internal string scope]
     \node[sq tiny label] (f) at (pants.center) {$f$};
     \node[sq tiny label] (g) at (copants.center) {$g$};
     \draw (f.center) to (pants.belt);
     \draw[bend right] (f.center) to (pants.leftleg);
     \draw[bend left] (f.center) to (pants.rightleg);
     \draw (g.center) to (copants.belt);
     \draw[bend left] (g.center) to (copants.leftleg);
     \draw[bend right] (g.center) to (copants.rightleg);
   \end{scope}
  \end{tikzpicture}
\xRightarrow{\epsilon^\otimes}
  \begin{tikzpicture}[scale=1.5]
    \node[Cyl,xscale=1.2,top,anchor=bot] (tube) {};
    \node[Cyl,xscale=1.2,bot,anchor=top] (tube1) at (tube.bot) {};
    \begin{scope}[internal string scope]
     \node[sq tiny label] (f) at (tube.center) {$f$};
     \node[sq tiny label] (g) at (tube1.center) {$g$};
     \draw (tube1.bot) to (g.center);
     \draw (f.center) to (tube.top);
     \draw[bend left] (f.center) to (g.center);
     \draw[bend right] (f.center) to (g.center);
    \end{scope}
  \end{tikzpicture}\ ,
\hspace*{1cm}
  \begin{tikzpicture}[scale=1.5]
    \node[Cyl,top,anchor=bot] (tube) {};
    \node[Cyl,bot,anchor=top] (tube1) at (tube.bot) {};
    \begin{scope}[internal string scope]
     \node[sq tiny label] (f) at (tube.bot) {$f$};
     \draw (f.center) to (tube1.bot);
     \draw (f.center) to (tube.top);
    \end{scope}
  \end{tikzpicture}\ \
  \begin{tikzpicture}[scale=1.5]
    \node[Cyl,top,anchor=bot] (tube) {};
    \node[Cyl,bot,anchor=top] (tube1) at (tube.bot) {};
    \begin{scope}[internal string scope]
     \node[sq tiny label] (g) at (tube.bot) {$g$};
     \draw (g.center) to (tube1.bot);
     \draw (g.center) to (tube.top);
    \end{scope}
  \end{tikzpicture}
\xRightarrow{\eta^\otimes}
  \begin{tikzpicture}[scale=1.5]
    \node[Pants,xscale=1.5,bot] (pants) {};
    \node[Copants,xscale=1.5, top, anchor=belt] (copants) at (pants.belt) {};
   % \node[Top3D] at (copants.rightleg) {};
   % \node[Bot3D] at (pants.rightleg) {};
    \node[Bot3D,xscale=1.5] at (pants.belt) {};
    \begin{scope}[internal string scope]
    \node[sq tiny label] (f) at ($(pants.belt)+(-0.22,.2)$) {$f$};
     \node[sq tiny label] (g) at ($(pants.belt)+(0.22,.2)$) {$g$};
     \draw[in=-90, out=90, looseness=1.3]  (f.center) to ($(copants.leftleg)+(0,0)$);
     \draw[in=-90, out=90, looseness=1.3]  (g.center) to ($(copants.rightleg)+(0,0)$);
     \draw[in=90, out=-90, looseness=1.3]  (f.center) to ($(pants.leftleg)+(0,0)$);
     \draw[in=90, out=-90, looseness=1.3]  (g.center) to ($(pants.rightleg)+(0,0)$);
    \end{scope}
  \end{tikzpicture}
$$
And similarly for the tensor unit:
$$
  \begin{tikzpicture}[scale=1.5]
    \node[Cup, top,scale=1.2] (cup) at (0,1.5) {};
    \node[Cap, bot,scale=1.2] (cap) at (0,0) {};
   % \node[Top3D] at (copants.rightleg) {};
    \begin{scope}[internal string scope]
    \node[sq tiny label] (f) at ($(cap.center)+(0,.2)$) {$g$};
    \node[sq tiny label] (g) at ($(cup.center)+(0,-.27)$)  {$f$};
     \draw (g.center) to (cup.center);
     \draw (f.center) to (cap.center);
    \end{scope}
  \end{tikzpicture}
\xRightarrow{\epsilon^I}
  \begin{tikzpicture}[scale=1.5]
    \node[Cyl,xscale=1.2,top,anchor=bot] (tube) {};
    \node[Cyl,xscale=1.2,bot,anchor=top] (tube1) at (tube.bot) {};
    \begin{scope}[internal string scope]
     \node[sq tiny label] (f) at (tube.center) {$f$};
     \node[sq tiny label] (g) at (tube1.center) {$g$};
     \draw (tube1.bot) to (g.center);
     \draw (f.center) to (tube.top);
    \end{scope}
  \end{tikzpicture}\ ,
\hspace*{1cm}
  \begin{tikzpicture}[scale=1.5]
	\begin{pgfonlayer}{nodelayer}
		\node [style=none] (0) at (-1, 1) {};
		\node [style=none] (1) at (-1, 0) {};
		\node [style=none] (2) at (0, 0) {};
		\node [style=none] (3) at (0, 1) {};
	\end{pgfonlayer}
	\begin{pgfonlayer}{edgelayer}
		\draw[style=dashed] (2.center) to (3.center) to (0.center) to (1.center) to cycle;
	\end{pgfonlayer}
\end{tikzpicture}
\xRightarrow{\eta^I}
  \begin{tikzpicture}[scale=1.5]
    \node[Cup] (cup) {};
    \node[Cap, bot]  at (cup) {};
   % \node[Top3D] at (copants.rightleg) {};
    \begin{scope}[internal string scope]
    \end{scope}
  \end{tikzpicture}
$$



These diagrams are very closed to proof nets.  
One might na\"ively try to obtain proof nets for monoidal categories in terms of quotienting the subcategory of $\Prof_*$ generated by the the monoidal structure by forcing the unit and counit of the adjunction to be equalities; however, the prospect of obtaining a monoidal category as a quotient of a monoidal bicategory is highly nontrivial.


If we are more clever, we can obtain the associator regarded as an element of the $\hom$-profunctor:
$$
(\X(-,=), \alpha_{X,Y,Z}):
(\X(-,=), (X\otimes Y)\otimes Z)\proarrow (\X(-,=),X\otimes(Y\otimes Z))
$$
By normalizing this diagram:

$$
\begin{tikzpicture}[scale=1.5]
    \node[Pants, bot, top] (B) at (0,0) {};
    \node[Pants, bot, anchor=belt] (A) at (B.rightleg) {};
    \node[SwishR, bot, anchor=top] (C) at (B.leftleg) {};
    \node[Copants, bot, anchor=rightleg] (D) at (A.leftleg) {};
    \node[SwishR, bot, anchor=top] (E) at (A.rightleg) {};
    \node[Copants, bot, anchor=rightleg] (F) at (E.bot) {};
    \begin{scope}[internal string scope]
    \draw [in=270, out=90] ($(F.belt)+(-.125,0)$) to ($(D.belt)+(-.1,0)$) to (C.bot) to (B.leftleg) to ($(B.belt)+(-.125,0)$);
    \draw [in=270, out=90] (F.belt) to ($(D.belt)+(.1,0)$) to (A.leftleg) to ($(B.rightleg)+(-.125,0)$) to (B.belt);
    \draw [in=270, out=90] ($(F.belt)+(.125,0)$)  to (E.bot) to (A.rightleg) to ($(B.rightleg)+(.125,0)$) to ($(B.belt)+(.125,0)$);
    \node (i) at (B.belt) [above] {\tiny $X\otimes(Y\otimes Z)$};
    \node (i) at (F.belt) [below] {\tiny $(X\otimes Y)\otimes Z$};
    \end{scope}
    \draw [color=blue,dashed] ($(B.leftleg)+(-1,0)$) to ($(B.rightleg)+(1,0)$) to ($(F.rightleg)+(1,0)$) to ($(F.leftleg)+(-1,0)$) to cycle;
\end{tikzpicture}
\xRightarrow{\phi^L}
\begin{tikzpicture}[scale=1.5]
    \node[Pants, bot, top] (A) at (0,0) {};
    \node[Copants, bot, anchor=leftleg] (B) at (A.leftleg) {};
    \node[Pants, bot, anchor=belt] (C) at (B.belt) {};
    \node[Copants, bot, anchor=leftleg] (D) at (C.leftleg) {};
    \begin{scope}[internal string scope]
    \draw [in=270, out=90] ($(D.belt)+(-.125,0)$) to ($(C.leftleg)+(-.1,0)$) to ($(B.belt)+(-.125,0)$) to ($(B.leftleg)+(0,0)$) to  ($(A.belt)+(-.125,0)$);
    \draw [in=270, out=90] ($(D.belt)+(0,0)$) to ($(C.leftleg)+(.1,0)$) to ($(B.belt)+(0,0)$) to ($(B.rightleg)+(-.1,0)$) to  ($(A.belt)+(0,0)$);
    \draw [in=270, out=90] ($(D.belt)+(.125,0)$) to ($(C.rightleg)+(0,0)$) to ($(B.belt)+(.125,0)$) to ($(B.rightleg)+(.1,0)$) to  ($(A.belt)+(.125,0)$);
    \node (i) at (A.belt) [above] {\tiny $X\otimes(Y\otimes Z)$};
    \node (i) at (D.belt) [below] {\tiny $(X\otimes Y)\otimes Z$};
    \end{scope}
\end{tikzpicture}
\xRightarrow{\eta^\otimes;\eta^\otimes}
\begin{tikzpicture}[scale=1.5]
    \node[Cyl, top] (A) at (0,0) {};
    \node[Cyl, anchor=top] (B) at (A.bot) {};
    \node[Cyl, anchor=top] (C) at (B.bot) {};
    \node[Cyl, bot, anchor=top] (D) at (C.bot) {};
    \begin{scope}[internal string scope]
    \draw ($(A.top)+(.125,0)$) to ($(D.bot)+(.125,0)$) ;
    \draw ($(A.top)+(0,0)$) to ($(D.bot)+(0,0)$);
    \draw ($(A.top)+(-.125,0)$) to ($(D.bot)+(-.125,0)$) ;
    \node (i) at (A.top) [above] {\tiny $X\otimes(Y\otimes Z)$};
    \node (i) at (D.bot) [below] {\tiny $(X\otimes Y)\otimes Z$};
    \end{scope}
\end{tikzpicture}
$$


Similarly, we get the inverse associator by connecting the shapes in the other way:

$$
\begin{tikzpicture}[scale=1.5]
    \node[Pants, bot, top] (B) at (0,0) {};
    \node[Pants, bot, anchor=belt] (A) at (B.leftleg) {};
    \node[SwishL, bot, anchor=top] (C) at (B.rightleg) {};
    \node[Copants, bot, anchor=leftleg] (D) at (A.rightleg) {};
    \node[SwishL, bot, anchor=top] (E) at (A.leftleg) {};
    \node[Copants, bot, anchor=leftleg] (F) at (E.bot) {};
    \begin{scope}[internal string scope]
    \draw [in=270, out=90] ($(F.belt)+(.125,0)$) to ($(D.belt)+(.1,0)$) to (C.bot) to (B.rightleg) to ($(B.belt)+(.125,0)$);
    \draw [in=270, out=90] (F.belt) to ($(D.belt)+(-.1,0)$) to (A.rightleg) to ($(B.leftleg)+(.125,0)$) to (B.belt);
    \draw [in=270, out=90] ($(F.belt)+(-.125,0)$)  to (E.bot) to (A.leftleg) to ($(B.leftleg)+(-.125,0)$) to ($(B.belt)+(-.125,0)$);
    \node (i) at (B.belt) [above] {\tiny $(X\otimes Y)\otimes Z$};
    \node (i) at (F.belt) [below] {\tiny $X\otimes(Y\otimes Z)$};
    \end{scope}
    \draw [color=blue,dashed] ($(B.leftleg)+(-1,0)$) to ($(B.rightleg)+(1,0)$) to ($(F.rightleg)+(1,0)$) to ($(F.leftleg)+(-1,0)$) to cycle;
\end{tikzpicture}
\xRightarrow{\phi^R}
\begin{tikzpicture}[scale=1.5]
    \node[Pants, bot, top] (A) at (0,0) {};
    \node[Copants, bot, anchor=leftleg] (B) at (A.leftleg) {};
    \node[Pants, bot, anchor=belt] (C) at (B.belt) {};
    \node[Copants, bot, anchor=leftleg] (D) at (C.leftleg) {};
    \begin{scope}[internal string scope]
    \draw [in=270, out=90] ($(D.belt)+(-.1,0)$) to ($(C.leftleg)+(0,0)$) to ($(B.belt)+(-.1,0)$) to ($(B.leftleg)+(-.1,0)$) to  ($(A.belt)+(-.1,0)$);
    \draw [in=270, out=90] ($(D.belt)+(0,0)$) to ($(C.rightleg)+(-.1,0)$) to ($(B.belt)+(0,0)$) to ($(B.leftleg)+(.1,0)$) to  ($(A.belt)+(0,0)$);
    \draw [in=270, out=90] ($(D.belt)+(.125,0)$) to ($(C.rightleg)+(.1,0)$) to ($(B.belt)+(.125,0)$) to ($(B.rightleg)+(0,0)$) to  ($(A.belt)+(.125,0)$);
    \node (i) at (A.belt) [above]  {\tiny $(X\otimes Y)\otimes Z$};
    \node (i) at (D.belt) [below] {\tiny $X\otimes(Y\otimes Z)$};
    \end{scope}
\end{tikzpicture}
\xRightarrow{\eta^\otimes;\eta^\otimes}
\begin{tikzpicture}[scale=1.5]
    \node[Cyl, top] (A) at (0,0) {};
    \node[Cyl, anchor=top] (B) at (A.bot) {};
    \node[Cyl, anchor=top] (C) at (B.bot) {};
    \node[Cyl, bot, anchor=top] (D) at (C.bot) {};
    \begin{scope}[internal string scope]
    \draw ($(A.top)+(.125,0)$) to ($(D.bot)+(.125,0)$) ;
    \draw ($(A.top)+(0,0)$) to ($(D.bot)+(0,0)$);
    \draw ($(A.top)+(-.125,0)$) to ($(D.bot)+(-.125,0)$) ;
    \node (i) at (A.top) [above]  {\tiny $(X\otimes Y)\otimes Z$};
    \node (i) at (D.bot) [below] {\tiny $X\otimes(Y\otimes Z)$};
    \end{scope}
\end{tikzpicture}
$$


We can do a similar thing for the left and right unitors:

$$
\begin{tikzpicture}[scale=1.5]
    \node[Copants, bot] (A) at (0,0) {};
    \node[Cyl, top, bot, anchor=bot] (B) at (A.rightleg) {};
    \node[Cap,bot] (C) at (A.leftleg) {};
    \begin{scope}[internal string scope]
    \draw [in=270, out=90] (A.belt) to (A.rightleg) to (B.top);
    \node (i) at (A.belt) [below] {\tiny $ I\otimes A$};
    \node (i) at (B.top) [above] {\tiny $ A$};
    \end{scope}
\end{tikzpicture}
\xRightarrow{(u^L)_*}
\begin{tikzpicture}[scale=1.5]
    \node[Cyl, top] (A) at (0,0) {};
    \node[Cyl, bot, anchor=top] (B) at (A.bot) {};
    \begin{scope}[internal string scope]
    \draw [in=270, out=90] ($(A.top)+(0,-.1)$) to ($(B.bot)+(0,.1)$);
    \node (i) at (B.bot) [below] {\tiny $ I\otimes A$};
    \node (i) at (A.top) [above] {\tiny $ A$};
    \end{scope}
\end{tikzpicture}\ ,\hspace*{.5cm}
\begin{tikzpicture}[scale=1.5]
    \node[Copants, bot] (A) at (0,0) {};
    \node[Cyl, top, bot, anchor=bot] (B) at (A.leftleg) {};
    \node[Cap,bot] (C) at (A.rightleg) {};
    \begin{scope}[internal string scope]
    \draw [in=270, out=90] (A.belt) to (A.leftleg) to (B.top);
    \node (i) at (A.belt) [below] {\tiny $ A\otimes I$};
    \node (i) at (B.top) [above] {\tiny $ A$};
    \end{scope}
\end{tikzpicture}
\xRightarrow{(u^R)_*}
\begin{tikzpicture}[scale=1.5]
    \node[Cyl, top] (A) at (0,0) {};
    \node[Cyl, bot, anchor=top] (B) at (A.bot) {};
    \begin{scope}[internal string scope]
    \draw [in=270, out=90] ($(A.top)+(0,-.1)$) to ($(B.bot)+(0,.1)$);
    \node (i) at (B.bot) [below] {\tiny $ I\otimes A$};
    \node (i) at (A.top) [above] {\tiny $ A$};
    \end{scope}
\end{tikzpicture}
$$

As well as their inverses:



$$
\begin{tikzpicture}[scale=1.5]
    \node[Pants, top, bot] (A) at (0,0) {};
    \node[Cyl, bot, anchor=top] (B) at (A.rightleg) {};
    \node[Cup] (C) at (A.leftleg) {};
    \begin{scope}[internal string scope]
    \draw [in=90, out=270] (A.belt) to (A.rightleg) to (B.bot);
    \node (i) at (A.belt) [above] {\tiny $ I\otimes A$};
    \node (i) at (B.bot) [below] {\tiny $ A$};
    \end{scope}
\end{tikzpicture}
\xRightarrow{(u^L)^*}
\begin{tikzpicture}[scale=1.5]
    \node[Cyl, top] (A) at (0,0) {};
    \node[Cyl, bot, anchor=top] (B) at (A.bot) {};
    \begin{scope}[internal string scope]
    \draw [in=270, out=90] ($(A.top)+(0,-.1)$) to ($(B.bot)+(0,.1)$);
    \node (i) at (B.bot) [below] {\tiny $A$};
    \node (i) at (A.top) [above] {\tiny $ I\otimes  A$};
    \end{scope}
\end{tikzpicture}\ ,\hspace*{.5cm}
\begin{tikzpicture}[scale=1.5]
    \node[Pants, top, bot] (A) at (0,0) {};
    \node[Cyl, bot, anchor=top] (B) at (A.leftleg) {};
    \node[Cup,bot] (C) at (A.rightleg) {};
    \begin{scope}[internal string scope]
    \draw [in=90, out=270] (A.belt) to (A.leftleg) to (B.bot);
    \node (i) at (A.belt) [above] {\tiny $ A\otimes I$};
    \node (i) at (B.bot) [below] {\tiny $ A$};
    \end{scope}
\end{tikzpicture}
\xRightarrow{(u^R)^*}
\begin{tikzpicture}[scale=1.5]
    \node[Cyl, top] (A) at (0,0) {};
    \node[Cyl, bot, anchor=top] (B) at (A.bot) {};
    \begin{scope}[internal string scope]
    \draw [in=270, out=90] ($(A.top)+(0,-.1)$) to ($(B.bot)+(0,.1)$);
    \node (i) at (B.bot) [below] {\tiny $ I\otimes A$};
    \node (i) at (A.top) [above] {\tiny $ A$};
    \end{scope}
\end{tikzpicture}
$$


In the setting of proof nets, the tensor is inverse to the cotensor and the unit introduction is inverse to the unit removal. 



\section{Monoidal categories as displayed categories}

 However, slightly closer than in $\Cat_*$, in $\Prof_*$ these 2-cells are at least adjoint.  We want to turn the adjunctions into equalities somehow.  We attempt this by regarding the normalization of the diagrams in $\Prof_*$ as the maps themselves.  
We need the following definition to this end:

\begin{definition}
A {\bf displayed category} is an ordinary category $\D$ equipped with a lax normal functor $F:\D\to \Prof$.
That is to say, $F$ has the data of:

\begin{itemize}
\item A function $F:\D_0\to \Prof_0$ taking objects of $\D$ to categories.
\item For every pair of objects $X,Y \in \D_0$, a function $F_{X,Y}:\D(X,Y)\to \Prof(F(X),F(Y))$  such that $1_{F(X)}=F_{X,X}(1_X)$.
\item For every triple of objects $X,Y,Z \in \D_0$, a $2$-cell, with components at $(f:X\to Y,g:Y\to Z)$:
$$F_{X,Y,Z}(f,g):F_{X,Y}(f);F_{Y,Z}(g) \Rightarrow F_{X,Z}(f;g)$$
\end{itemize}

Such that for any diagram $W\xrightarrow{f} X \xrightarrow{g} Y \xrightarrow{h} Z$ in $\X$ the following diagram commutes in $\Prof$:


$$
\xymatrix{
(F_{W,X}(f);F_{X,Y}(g));F_{Y,Z} \ar[d]_{F_{W,X,Y}(f;g);1_{F_{Y,Z}(h)}} \ar[rr]^{(\alpha_;)_{F_{W,X}(f),F_{X,Y}(g),F_{Y,Z}(h)}}
  & & F_{W,X}(f);(F_{X,Y}(g);F_{Y,Z}(h)) \ar[d]^{1_{F_{W,X}(f)};F_{X,Y,Z}(g,h)}\\
F_{W,Y}(f;g);F_{Y,Z}(h) \ar[dr]_{F_{W,Y,Z}(f;g,h) \ \ }
  && F_{W,X}(f);F_{X,Z}(g;h) \ar[dl]^{\ \ F_{W,X,Z}(f,g;h)}\\
  & F(W,Z)(f;g;h)
}
$$

Where $\alpha_;$ is the associator for composition in $\Prof$.
\end{definition}

We can recast the spider theorem in this light:

\begin{lemma}
Given any monoidal category $\X$, there is a displayed category

$F_\X:\sfa\to \Prof$ such that:
\begin{itemize}
\item $(F_\X)(n) = \X^n$
\item $(F_\X){n,m}$ takes diagrams in $\sfa$ to (stratified) spiders in $\Prof$.
\item The natural transformation $(F_\X)_{n,m,k}$ performs (stratified) spider fusion.
\end{itemize}


\end{lemma}

This follows immediately from the  (conjectural) spider theorem.
The confluence of the spider normalization corresponds to commutation of the pentagon in the definition of a displayed category.

This way of framing the categorified spider theorem lends itself to the following canonical construction, usually attributed to Benabou \cite{benabou}. This is a variation of the eponymous construction of Grothendieck for pseudofunctors from ordinary categories into $\Cat$, which we alluded to early in the multicategorical setting:



\begin{theorem}[B\'enabou-Grothendieck construction]

Given a displayed category $F:\D\to \Prof$, the B\'enabou-Grothendieck category,  $\Pi F$ is given by the pullback:

$$
\xymatrix{
\Pi  F \ar[r]^{\pi_1} \ar[d]_{\pi_0} & \Prof_* \ar[d] \\
\D \ar[r]_F & \Prof
}
$$

Where $\Prof_* \to \Prof$ is the canonical projection.
Concretely $\Pi F$ has:


\begin{description}
\item[Objects:] Pairs $(X\in \D_0, X^\sharp \in (F(X))_0)$
\item[Maps:] The maps are  pointed profunctors:
$$(f, f^\sharp):(X,X^\sharp )\to (Y,Y^\sharp)$$
 where $f \in \D(X,Y)$ and $f^\sharp \in F_{X,Y}(f)(Y^\sharp,X^\sharp)$
\item[Identity:] $1_{(X,X^\sharp)} := (1_X, 1_{X^\sharp})$
\item[Composition:] Given a composable pair:
$$(X,X^\sharp)\xrightarrow{(f, f^\sharp)} (Y,Y^\sharp)\xrightarrow{(g, g^\sharp)} (Z,Z^\sharp)$$
The composite is defined as follows:
$$(f, f^\sharp);(g, g^\sharp):= (F_{X,Y,Z}(f,g))(f^\sharp, g^\sharp)$$
\end{description}


Moreover, the first projection map $\pi_0:\Pi F\to \D$  is a (strict) functor.
\end{theorem}



We can actually go the other direction.  Given some fixed $\D$, this extends to an equivalence of categories between the slice category $\Cat/\D$ and the lax normal functor category $[\D,\Prof]$.  Benabou gives a detailed argument in proving this equivalence in his canonical reference \cite{benaboudist}.  We won't restate this equivalence of categories, as it takes considerable effort to expose.





The idea of regarding monoidal categories as displayed categories is partially motivated by the work of \cite{blanco}, transporting the work of Hermida in the multicategorical setting to the polycategorical setting:  where the inputs and outputs are now both lists of objects.  There is a similar construction of internal polycategories to internal categories and internal multicategories due to \cite{kowlowski}; however, it is quite a bit more complicated to describe formally.  


In \cite{blanco}, they construct the B\'enabou-Grothendieck {\em *-polycatgory}, coming from a {\em displayed polycategory} $\one\to \Prof$, where $\Prof$ is regarded as a 2-polycategory.  In this *-polycategorical setting, their notion of a Frobenius algebra is precisely  a displayed categories from the terminal *-polycategory.
Moreover, they identify representable *-polycategories as the  *-polycategories which are poly-bifibrations over the terminal *-polycategory, where representable *-polycategories are in bijection with strict *-autonomous categories.  This notion of bifibration is closely related to the B\'enabou-Grothendieck construction. 


However, we can not directly appeal to this result if we want to recapture proof nets.  This is because the composition in polycategories is defined only with respect to a single object at a time. The counit for the tensor product does not make sense in this setting.  Thus, we are not able to get the associators by untensoring twice and then retensoring in the other direction:  such a map can not even be formed!

The displayed {\em category} $F_\X:\sfa\to \Prof$ which we constructed is in some sense trying to recapture this way of looking at things.  Let us see what happens when we compute the B\'enabou-Grothendieck category of $F_\X$.

\begin{lemma}
The indexed category $\Pi F_{\X}$ has a concrete presentation:

\begin{description}
\item[Objects:] Finite lists of objects in $\X$.
%\item[Maps:] Given two finite lists $X=[X_0,\ldots, X_{n-1}]$ and $Y=[Y_0,\ldots, Y_{m-1}]$ of objects in $\X$, a map from $X\to Y$ is a pair $(f,f^\sharp)$ where $f:n\to m$ is a map in $\sfa$ equipped with an element $f^\sharp \in  (F_{\X})_{n,m}( F_{\X})$.
%
%
%$f^\sharp$ can be described concretely by induction on the connected components on $n$. If $f$ is a single connected component, then $f^\sharp$ is a map with domains and codomains left-factorized as follows
%$$\otimes_{i=0}^{n-1} X_i \to \otimes_{i=0}^{m-1} Y_i$$
%
%Moreover, if $f$ is multiple connected components, then $f^\sharp$ is a finite list of maps whose domains and codomains factorized in this way.

\item[Maps:] Given two finite lists $X=[X_0,\ldots, X_{n-1}]$ and $Y=[Y_0,\ldots, Y_{m-1}]$ of objects in $\X$, a map from $X\to Y$ is a pointed profunctor

$$
(P,f): (\X^n,X) \proarrow (\Y^m,Y)
$$

Generated by the (co)tensor and (co)unit of the monoidal structure of $\X$.

The equality of these maps is modulo the equivalence relation generated by the congrence
$(P,f)\sim(Q,g)$
 when $Q$ and $P$ are normalized to the same stratified spider $\nu_0:P\Rightarrow S \Leftarrow Q:\nu_1$, and moreover, the normalization agrees on points so that $\nu_0(f)=\nu_1(g)$.
%
%\end{itemize}

\item[Identity:]  The identity on $(\X^n,X)$ is the identity in pointed profunctors

$$
1_{(\X^n,X)} = (\X(-,=)^n,1_X)
$$


%Given a list of objects $X=[X_0,\ldots, X_{n-1}]$  in $\X$, $$1_X:=([1_{X_0},\ldots, 1_{X_{n-1}}], \X(-,=)^n)$$
%vizualized as $n$ tubes occupied by identities:
%
%$$
%\begin{tikzpicture}[scale=1.5]
%    \node[Cyl, top,bot] (A) at (0,0) {};
%    \begin{scope}[internal string scope]
%    \draw [in=270, out=90] ($(A.top)+(0,-.01)$) to ($(A.bot)+(0,.01)$);
%    \node (i) at (A.bot) [below] {\tiny $X_0$};
%    \node (i) at (A.top) [above] {\tiny $ X_0$};
%    \end{scope}
%\end{tikzpicture}\
%\begin{tikzpicture}[scale=1.5]
%    \node[Cyl, top,bot] (A) at (0,0) {};
%    \begin{scope}[internal string scope]
%    \draw [in=270, out=90] ($(A.top)+(0,-.01)$) to ($(A.bot)+(0,.01)$);
%    \node (i) at (A.bot) [below] {\tiny $X_1$};
%    \node (i) at (A.top) [above] {\tiny $ X_1$};
%    \end{scope}
%\end{tikzpicture}\ \
%\cdots
%\begin{tikzpicture}[scale=1.5]
%    \node[Cyl, top,bot] (A) at (0,0) {};
%    \begin{scope}[internal string scope]
%    \draw [in=270, out=90] ($(A.top)+(0,-.01)$) to ($(A.bot)+(0,.01)$);
%    \node (i) at (A.bot) [below] {\tiny $X_{n-1}$};
%    \node (i) at (A.top) [above] {\tiny $ X_{n-1}$};
%    \end{scope}
%\end{tikzpicture}
%$$


\item[Composition:] The composition is the composition of pointed profunctors.

\end{description}

This is moreover a strict monoidal category.  The tensor product is given by the tensor product in $\Prof_*$, so that the projection functor $\pi_0:\Pi F_\X\to \sfa$ is strict monoidal.


\end{lemma}


In $\Pi F_\X$, the components of the unitors and associators that we drew before as 2-cells are now honest maps which are inverse to each other.% Moreover, the different ways of associating and introducing/removing units are now genuine inverses because their composite reduces to the identity.
 Moreover, $\eta^\otimes$ now induces the equality:
%$$
%\begin{tikzpicture}[scale=1.5]
%    \node[Pants, top,bot] (A) at (0,0) {};
%    \node[Copants,bot, anchor=leftleg] (B) at (A.leftleg) {};
%    \begin{scope}[internal string scope]
%    \draw [in=270, out=90] ($(B.belt)+(-.1,0)$) to (A.leftleg) to ($(A.belt)+(-.1,0)$) ;
%    \draw [in=270, out=90] ($(B.belt)+(.1,0)$) to (A.rightleg) to ($(A.belt)+(.1,0)$) ;
%    \end{scope}
%\end{tikzpicture}
%=
%\begin{tikzpicture}[scale=1.5]
%    \node[Cyl, top] (A) at (0,0) {};
%    \node[Cyl, bot, anchor=top] (B) at (A.bot) {};
%    \begin{scope}[internal string scope]
%    \draw [in=270, out=90] ($(A.top)+(0,-.1)$) to ($(B.bot)+(0,.1)$);
%    \end{scope}
%\end{tikzpicture}
%$$



$$
  \begin{tikzpicture}[scale=1.5]
    \node[Pants, top] (pants) {};
    \node[Copants, bot, lowercob, anchor=leftleg] (copants) at (pants.leftleg) {};
   % \node[Top3D] at (copants.rightleg) {};
    \node[Bot3D] at (pants.rightleg) {};
    \node[Bot3D] at (pants.leftleg) {};
   \begin{scope}[internal string scope]
     \node[sq tiny label] (f) at (pants.center) {$f$};
     \node[sq tiny label] (g) at (copants.center) {$g$};
     \draw (f.center) to (pants.belt);
     \draw[bend right] (f.center) to (pants.leftleg);
     \draw[bend left] (f.center) to (pants.rightleg);
     \draw (g.center) to (copants.belt);
     \draw[bend left] (g.center) to (copants.leftleg);
     \draw[bend right] (g.center) to (copants.rightleg);
   \end{scope}
  \end{tikzpicture}
=
  \begin{tikzpicture}[scale=1.5]
    \node[Cyl,xscale=1.2,top,anchor=bot] (tube) {};
    \node[Cyl,xscale=1.2,bot,anchor=top] (tube1) at (tube.bot) {};
    \begin{scope}[internal string scope]
     \node[sq tiny label] (f) at (tube.center) {$f$};
     \node[sq tiny label] (g) at (tube1.center) {$g$};
     \draw (tube1.bot) to (g.center);
     \draw (f.center) to (tube.top);
     \draw[bend left] (f.center) to (g.center);
     \draw[bend right] (f.center) to (g.center);
    \end{scope}
  \end{tikzpicture}
$$


However, this is still not proof nets.  We can only normalize connected components.  For example, the following equation does not hold because the profunctors in which the string diagrams are drawn are not connected:

$$
\begin{tikzpicture}[scale=1.5]
    \node[Cyl, top] (A) at (0,0) {};
    \node[Cyl, bot, anchor=top] (B) at (A.bot) {};
    \node[Cyl, top] (AA) at (.75,0) {};
    \node[Cyl, bot, anchor=top] (BB) at (AA.bot) {};
    \begin{scope}[internal string scope]
    \draw [in=270, out=90] ($(A.top)+(0,-.1)$) to ($(B.bot)+(0,.1)$);
    \draw [in=270, out=90] ($(AA.top)+(0,-.1)$) to ($(BB.bot)+(0,.1)$);
    \end{scope}
\end{tikzpicture}
\ \ \neq\ 
\begin{tikzpicture}[scale=1.5]
    \node[Copants, top,bot] (A) at (0,0) {};
    \node[Pants,bot, anchor=belt] (B) at (A.belt) {};
    \begin{scope}[internal string scope]
    \draw [in=270, out=90] (B.leftleg) to ($(B.belt)+(-.1,0)$) to (A.leftleg);
    \draw [in=270, out=90] (B.rightleg) to ($(B.belt)+(.1,0)$) to (A.rightleg);
    \end{scope}
\end{tikzpicture}
$$

However, if in context $\Gamma$ we knew that the profunctors were connected then we could apply this rewrite rule:


$$
\begin{tikzpicture}[scale=1.5]
    \node[Cyl] (A0) at (0,0) {};
    \node[Cyl,anchor=center] (A1) at ($(A.center)+(1,0)$) {};
    \node[Cyl,anchor=top] (B0) at (A0.bot) {};
    \node[Cyl,anchor=top] (B1) at (A1.bot) {};
    \node (x1) at ($(A1.top)+(.5,0)$) {};
    \node (x2) at ($(A1.top)+(.5,.4)$) {};
    \node (x3) at ($(A0.top)+(-1.5,.4)$) {};
    \node (x4) at ($(B0.bot)+(-1.5,-.4)$) {};
    \node (x5) at ($(B1.bot)+(.5,-.4)$) {};
    \node (x6) at ($(B1.bot)+(.5,0)$) {};
    \node (x7) at ($(B0.bot)+(-.5,0)$) {};
    \node (x8) at ($(A0.top)+(-.5,0)$) {};
    \node[Cyl,top, anchor=bot] (C) at ($(x3.center)+(.75,0)$) {};
    \node[Cyl,top, anchor=bot] (D) at ($(x2.center)+(-.75,0)$) {};
    \node[Cyl,bot, anchor=top] (E) at ($(x4.center)+(.75,0)$) {};
    \node[Cyl,bot, anchor=top] (F) at ($(x5.center)+(-.75,0)$) {};
    \node(G) at($(x3.center)+(1.5,.5)$)  {$\cdots$};
    \node(H) at($(x4.center)+(1.5,-.5)$)  {$\cdots$};
    \begin{scope}[internal string scope]
    \draw (C.bot) to (C.top);
    \draw (D.bot) to (D.top);
    \draw (E.bot) to (E.top);
    \draw (F.bot) to (F.top);
    \draw (A0.top) to (B0.bot);
    \draw (A1.top) to (B1.bot);
    \end{scope}
    \draw [color=black, fill=blue, fill opacity=.4] (x1.center) to  (x1.center) to [bend left=15]  (x2.center) to (x3.center) to  [bend left=-15] (x4.center) to (x5.center) to [bend left=15]  (x6.center) to (x7.center) to  [bend left=15] (x8.center) to cycle;
    \node(K) at($(A0.bot)+(-1.1,0)$)  {$\Gamma$};
\end{tikzpicture}
=
\begin{tikzpicture}[scale=1.5]
    \node[Copants,bot] (A) at (0,0) {};
    \node[Pants, anchor=belt] (B) at (A.belt) {};
    \node (x1) at ($(A.rightleg)+(.5,0)$) {};
    \node (x2) at ($(A.rightleg)+(.5,.4)$) {};
    \node (x3) at ($(A.leftleg)+(-1.5,.4)$) {};
    \node (x4) at ($(B.leftleg)+(-1.5,-.4)$) {};
    \node (x5) at ($(B.rightleg)+(.5,-.4)$) {};
    \node (x6) at ($(B.rightleg)+(.5,0)$) {};
    \node (x7) at ($(B.leftleg)+(-.5,0)$) {};
    \node (x8) at ($(A.leftleg)+(-.5,0)$) {};
    \node[Cyl,top, anchor=bot] (C) at ($(x3.center)+(.75,0)$) {};
    \node[Cyl,top, anchor=bot] (D) at ($(x2.center)+(-.75,0)$) {};
    \node[Cyl,bot, anchor=top] (E) at ($(x4.center)+(.75,0)$) {};
    \node[Cyl,bot, anchor=top] (F) at ($(x5.center)+(-.75,0)$) {};
    \node(G) at($(x3.center)+(1.5,.5)$)  {$\cdots$};
    \node(H) at($(x4.center)+(1.5,-.5)$)  {$\cdots$};
    \begin{scope}[internal string scope]
    \draw [in=270, out=90] (B.leftleg) to ($(B.belt)+(-.1,0)$) to (A.leftleg);
    \draw [in=270, out=90] (B.rightleg) to ($(B.belt)+(.1,0)$) to (A.rightleg);
    \draw (C.bot) to (C.top);
    \draw (D.bot) to (D.top);
    \draw (E.bot) to (E.top);
    \draw (F.bot) to (F.top);
    \end{scope}
    \draw [color=black, fill=blue, fill opacity=.4] (x1.center) to  (x1.center) to [bend left=15]  (x2.center) to (x3.center) to  [bend left=-15] (x4.center) to (x5.center) to [bend left=15]  (x6.center) to (x7.center) to  [bend left=15] (x8.center) to cycle;
    \node(K) at($(A.belt)+(-1.55,0)$)  {$\Gamma$};
\end{tikzpicture}
$$


Similarly:

$$
  \begin{tikzpicture}[scale=1.5]
    \node[Cup, top,scale=1.2] (cup) at (0,1.5) {};
    \node[Cap, bot,scale=1.2] (cap) at (0,0) {};
   % \node[Top3D] at (copants.rightleg) {};
    \begin{scope}[internal string scope]
    \end{scope}
  \end{tikzpicture}
\neq
  \begin{tikzpicture}[scale=1.5]
    \node[Cyl,xscale=1.2,top,anchor=bot] (tube) {};
    \node[Cyl,xscale=1.2,bot,anchor=top] (tube1) at (tube.bot) {};
    \begin{scope}[internal string scope]
    \end{scope}
  \end{tikzpicture}
$$

However, we know that the profunctors are connected in context $\Gamma$, we have:


$$
\begin{tikzpicture}[scale=1.5]
    \node[Cyl] (A0) at (0,0) {};
    \node[Cyl,anchor=top] (B0) at (A0.bot) {};
    \node (x1) at ($(A0.top)+(1.5,0)$) {};
    \node (x2) at ($(A0.top)+(1.5,.4)$) {};
    \node (x3) at ($(A0.top)+(-1.5,.4)$) {};
    \node (x4) at ($(B0.bot)+(-1.5,-.4)$) {};
    \node (x5) at ($(B0.bot)+(1.5,-.4)$) {};
    \node (x6) at ($(B0.bot)+(1.5,0)$) {};
    \node (x7) at ($(B0.bot)+(-.5,0)$) {};
    \node (x8) at ($(A0.top)+(-.5,0)$) {};
    \node[Cyl,top, anchor=bot] (C) at ($(x3.center)+(.75,0)$) {};
    \node[Cyl,top, anchor=bot] (D) at ($(x2.center)+(-.75,0)$) {};
    \node[Cyl,bot, anchor=top] (E) at ($(x4.center)+(.75,0)$) {};
    \node[Cyl,bot, anchor=top] (F) at ($(x5.center)+(-.75,0)$) {};
    \node(G) at($(x3.center)+(1.5,.5)$)  {$\cdots$};
    \node(H) at($(x4.center)+(1.5,-.5)$)  {$\cdots$};
    \begin{scope}[internal string scope]
    \draw (C.bot) to (C.top);
    \draw (D.bot) to (D.top);
    \draw (E.bot) to (E.top);
    \draw (F.bot) to (F.top);
    \end{scope}
    \draw [color=black, fill=blue, fill opacity=.4] (x1.center) to  (x1.center) to [bend left=15]  (x2.center) to (x3.center) to  [bend left=-15] (x4.center) to (x5.center) to [bend left=15]  (x6.center) to (x7.center) to  [bend left=15] (x8.center) to cycle;
    \node(K) at($(A0.bot)+(-1.1,0)$)  {$\Gamma$};
\end{tikzpicture}
=
\
\begin{tikzpicture}[scale=1.5]
    \node (A00) at (0,0) {};
    \node (B00) at (A0.center) {};
    \node [Cup] (A0) at ($(A00.center)+(0,.75)$) {};
    \node [Cap,anchor=center]  (B0) at ($(B00.center)+(0.,-.75)$) {};
    \node (x1) at ($(A0.center)+(1.5,0)$) {};
    \node (x2) at ($(A0.center)+(1.5,.4)$) {};
    \node (x3) at ($(A0.center)+(-1.5,.4)$) {};
    \node (x4) at ($(B0.center)+(-1.5,-.4)$) {};
    \node (x5) at ($(B0.center)+(1.5,-.4)$) {};
    \node (x6) at ($(B0.center)+(1.5,0)$) {};
    \node (x7) at ($(B0.center)+(-.5,0)$) {};
    \node (x8) at ($(A0.center)+(-.5,0)$) {};
    \node[Cyl,top, anchor=bot] (C) at ($(x3.center)+(.75,0)$) {};
    \node[Cyl,top, anchor=bot] (D) at ($(x2.center)+(-.75,0)$) {};
    \node[Cyl,bot, anchor=top] (E) at ($(x4.center)+(.75,0)$) {};
    \node[Cyl,bot, anchor=top] (F) at ($(x5.center)+(-.75,0)$) {};
    \node(G) at($(x3.center)+(1.5,.5)$)  {$\cdots$};
    \node(H) at($(x4.center)+(1.5,-.5)$)  {$\cdots$};
    \begin{scope}[internal string scope]
    \draw (C.bot) to (C.top);
    \draw (D.bot) to (D.top);
    \draw (E.bot) to (E.top);
    \draw (F.bot) to (F.top);
    \end{scope}
    \draw [color=black, fill=blue, fill opacity=.4] (x1.center) to  (x1.center) to [bend left=15]  (x2.center) to (x3.center) to  [bend left=-15] (x4.center) to (x5.center) to [bend left=15]  (x6.center) to (x7.center) to  [bend left=15] (x8.center) to cycle;
    \node(K) at($(A0.center)+(-1.1,-.75)$)  {$\Gamma$};
\end{tikzpicture}
$$

To obtain proof nets, we would want all components to be connected.


In order to attempt this, remark that for any  finite nonempty list of objects $X=[X_0,\ldots, X_{n-1}]$ in $\X$ there is an idempotent $(s_X, s_X^\sharp)=e_X:X\to X$ in   $\Pi F_{\X}$ where $s_X$ is the fully connected spider from $\X^n\to\X^n$ and $s_X^\sharp$ is the tensor factorized identity $s_X^\sharp = \otimes_{i=0}^{n-1}1 _{X_i}$ regarded as an element of the spider.  Define $e_{[]}=(\one, *)$.


Consider for the sake of illustration:

$$
e_{[X_0,X_1,X_2]}
=
\begin{tikzpicture}[scale=1.5]
    \node[Copants,top,bot] (A) at (0,0) {};
    \node[Copants, bot,anchor=leftleg] (B) at (A.belt) {};
    \node[Pants, bot ,anchor=belt] (C) at (B.belt) {};
    \node[Pants, bot ,anchor=belt] (D) at (C.leftleg) {};
    \node[SwishR, bot,top, anchor=bot] (E) at (B.rightleg) {};
    \node[SwishL, bot, anchor=top] (F) at (C.rightleg) {};
    \begin{scope}[internal string scope]
    \draw [in=270, out=90] ($(D.leftleg)+(0,0)$) to ($(C.leftleg)+(-.1,0)$) to ($(B.belt)+(-.125,0)$) to ($(A.belt)+(-.1,0)$) to ($(A.leftleg)+(0,0)$);
    \draw [in=270, out=90] ($(D.rightleg)+(0,0)$) to ($(C.leftleg)+(.1,0)$) to ($(B.belt)+(0,0)$) to ($(A.belt)+(.1,0)$) to ($(A.rightleg)+(0,0)$);
    \draw [in=270, out=90] ($(F.bot)+(0,0)$) to ($(C.rightleg)+(0,0)$) to ($(B.belt)+(.125,0)$) to (E.bot) to (E.top);
    \node (i) at (A.leftleg) [above] {\tiny $ X_0$};
    \node (i) at (A.rightleg) [above] {\tiny $ X_1$};
    \node (i) at (E.top) [above] {\tiny $ X_2$};
    \node (i) at (D.leftleg) [below] {\tiny $ X_0$};
    \node (i) at (D.rightleg) [below] {\tiny $ X_1$};
    \node (i) at (F.bot) [below] {\tiny $ X_3$};
    \end{scope}
\end{tikzpicture}
$$



We can regard these idempotents as the {\em property} that the boundary be connected, and obtain a new strict monoidal category by splitting them:


\begin{definition}
Take $N\X:={\sf Split}_{\{e_X\ | \ X \in [\X_0]\}}(\Pi F_{\X})$; that is, the full subcategory of the  the Karoubi envelope of $\Pi F_{\X}$ with objects $(X,e_X)$, for every finite list of objects in $X$. Concretely $N\X$ has:

\begin{description}
\item[Objects:] Nonempty finite lists of objects in $\X$.

\item[Maps:] 
The maps $(P,f);X\to Y  \in N\X$ are given by maps  $(P,f):X\to Y  \in \Pi F_\X$ where the top boundaries of $P$ are connected together, as well as the bottom boundaries are connected together.


\item[Composition:] Same as in $\Pi F_\X$.

\item[Identity:] Same as in $\Pi F_\X$.

\item[Monoidal structure:] Given two $(P,f);W\to X$ and $(Q,g);Y\to Z$

$$
(P,f)\otimes(Q,g) := e_{W,Y};(P\times Q, (f,g)); e_{X,Z}
$$

\end{description}

\end{definition}


This monoidal category is very close to the strictification of a monoidal category; however, it is not quite there.  There is no way to eliminate the scalars  in this category.  The following equation does not hold:


$$
 \begin{tikzpicture}[scale=1.5]
	\begin{pgfonlayer}{nodelayer}
		\node [style=none] (0) at (-1, 1) {};
		\node [style=none] (1) at (-1, 0) {};
		\node [style=none] (2) at (0, 0) {};
		\node [style=none] (3) at (0, 1) {};
	\end{pgfonlayer}
	\begin{pgfonlayer}{edgelayer}
		\draw[style=dashed] (2.center) to (3.center) to (0.center) to (1.center) to cycle;
	\end{pgfonlayer}
\end{tikzpicture}
\neq
  \begin{tikzpicture}[scale=1.5]
    \node[Cup] (cup) {};
    \node[Cap, bot]  at (cup) {};
   % \node[Top3D] at (copants.rightleg) {};
    \begin{scope}[internal string scope]
    \end{scope}
  \end{tikzpicture}
$$


Moreover, the top and bottom boundaries need not be connected {\em to each other}, so  we still don't recover the following desired equations:

$$
  \begin{tikzpicture}[scale=1.5]
    \node[Cup, top,scale=1.2] (cup) at (0,1.5) {};
    \node[Cap, bot,scale=1.2] (cap) at (0,0) {};
   % \node[Top3D] at (copants.rightleg) {};
    \begin{scope}[internal string scope]
    \end{scope}
  \end{tikzpicture}
\neq
  \begin{tikzpicture}[scale=1.5]
    \node[Cyl,xscale=1.2,top,anchor=bot] (tube) {};
    \node[Cyl,xscale=1.2,bot,anchor=top] (tube1) at (tube.bot) {};
    \begin{scope}[internal string scope]
    \end{scope}
  \end{tikzpicture}\ ,
\hspace*{.5cm}
\begin{tikzpicture}[scale=1.5]
    \node[Cyl, top] (A) at (0,0) {};
    \node[Cyl, bot, anchor=top] (B) at (A.bot) {};
    \node[Cyl, top] (AA) at (.75,0) {};
    \node[Cyl, bot, anchor=top] (BB) at (AA.bot) {};
    \begin{scope}[internal string scope]
    \draw [in=270, out=90] ($(A.top)+(0,-.1)$) to ($(B.bot)+(0,.1)$);
    \draw [in=270, out=90] ($(AA.top)+(0,-.1)$) to ($(BB.bot)+(0,.1)$);
    \end{scope}
\end{tikzpicture}
\ \ \neq\ 
\begin{tikzpicture}[scale=1.5]
    \node[Copants, top,bot] (A) at (0,0) {};
    \node[Pants,bot, anchor=belt] (B) at (A.belt) {};
    \begin{scope}[internal string scope]
    \draw [in=270, out=90] (B.leftleg) to ($(B.belt)+(-.1,0)$) to (A.leftleg);
    \draw [in=270, out=90] (B.rightleg) to ($(B.belt)+(.1,0)$) to (A.rightleg);
    \end{scope}
\end{tikzpicture}
$$


If we additionally imposed the following equations as a quotient on $N\X$:

$$
 \begin{tikzpicture}[scale=1.5]
	\begin{pgfonlayer}{nodelayer}
		\node [style=none] (0) at (-1, 1) {};
		\node [style=none] (1) at (-1, 0) {};
		\node [style=none] (2) at (0, 0) {};
		\node [style=none] (3) at (0, 1) {};
	\end{pgfonlayer}
	\begin{pgfonlayer}{edgelayer}
		\draw[style=dashed] (2.center) to (3.center) to (0.center) to (1.center) to cycle;
	\end{pgfonlayer}
\end{tikzpicture}
=
  \begin{tikzpicture}[scale=1.5]
    \node[Cup] (cup) {};
    \node[Cap, bot]  at (cup) {};
   % \node[Top3D] at (copants.rightleg) {};
    \begin{scope}[internal string scope]
    \end{scope}
  \end{tikzpicture}\ ,
\hspace*{1cm}
  \begin{tikzpicture}[scale=1.5]
    \node[Cup, top,scale=1.2] (cup) at (0,1.5) {};
    \node[Cap, bot,scale=1.2] (cap) at (0,0) {};
   % \node[Top3D] at (copants.rightleg) {};
    \begin{scope}[internal string scope]
    \end{scope}
  \end{tikzpicture}
=
  \begin{tikzpicture}[scale=1.5]
    \node[Cyl,xscale=1.2,top,anchor=bot] (tube) {};
    \node[Cyl,xscale=1.2,bot,anchor=top] (tube1) at (tube.bot) {};
    \begin{scope}[internal string scope]
    \end{scope}
  \end{tikzpicture}$$

Which would yield a monoidal category in which all components are connected.  This appears to recapture proof nets for monoidal categories in the setting where the scalars are central; however, the only reason this works is by direct appeal to the coherence result of \cite{wilson}. This doesn't give any deep insight into why thinning links are not needed for monoidal categories, but are needed for linearly distributive categories.

The problem with units in *-autonomous categories and linearly distributive categories has long been known.  For example, \cite{houston} devotes his entire thesis to developing unitless multiplicative linear logic for this reason.  However, it is disappointing that our na\"ive approach doesn't seem to work in the degenerate case of monoidal categories where there is only one tensor product.


Nevertheless, I think there is a lot more to be done here; a proof of the uniqueness of the stratified spider normal form, or some variation of such,  being the most important. 

%At least in the case when the original monoidal category has central scalars, then modulo the confluence of the spider theorem, 

Despite the drawbacks to this approach, it is highly amenable to different notions of algebraic structure in $\Prof$ other than monoidal categories. One could have instead asked that the monoid and comonoid be commutative; corresponding to categorification of the spider theorem for special commutative Frobenius algebras to get similar graphical calculi.

Similarly, by asking that the frobeniusators be pseudo, one would be asking for duals; by asking for commutativity and pseudo-ness of the Frobeniusators, this would ask for compact closedness.



Proof nets for both of these other things already exist, however, this approach to coherence begs the question of string diagrams for more exotic structures.
Perhaps categories like $\Pi F_\X$ or $N\X$ are the right level of complexity to work at.
  For example a biactegory is essentially a monoidal category with compatible left and right strengths that commute with the tensor product. However, regarded as extra structure on a lax Frobenius algebra in $\Prof$, this is a categorification of phased spiders. A categorified phased-spider normal form would apparently give some notion of string diagrams for monoidal biactegories (presumably which would be a special case of the proof nets for linear actegories found in \cite[\S6.1]{polyact}).  More generally, one could ask for the categorification of the various other normal forms we have discussed throughout this thesis... does the bialgebra law have some sort of categorified analogue?






















%
%
%
%
%As a historical note, proof nets were originally invented by Girard as a graphical calculus for multiplicative linear logic \cite{girard}, a resource-sensitive refinement of logic where the antecedents and consequents of sequents can not be freely copied and deleted.
%
%For example, in multiplicative linear logic, two sequents $X \vdash Y$ and $Y \vdash Z$ are cut together in a way that consumes $Y$:
%
%$$
%\dfrac{\Gamma, X\vdash \Delta,Y \hspace{.5cm}  Y,\Gamma'\vdash Z,\Delta'}{\Gamma,X,\Gamma' \vdash \Delta, Z, \Delta'}
%$$
%
%The symbols $\Gamma,\Gamma'$ and $\Delta,\Delta'$ represent the contexts in the antecedents and sequents of the sequents.
%
%Later, Cockett Blute and Seely formalized  multiplicative linear logic in the representable setting in terms of linearly distributive categories \cite{ldc}.
%In a linearly distributive category, there are two monoidal structures, one which tensors the antecedents of a sequent  the other which tensors the consequents.  The former is called tensor (drawn as $\otimes$) and the latter is called par (drawn as $\oplus$).  This two monoidal structures distribute over each other ``linearly'' according to left and right linear distributors:
%$$
%\delta_{X,Y,Z}^L: X \otimes (Y\oplus Z) \to (X\otimes Y)\oplus Z
%\hspace*{.5cm}
%\delta_{X,Y,Z}^R: (X\oplus Y) \otimes Z \to X\oplus (Y \otimes Z)
%$$
%
%This distributivity allows one to cut together sequents:
%
%$$
%\dfrac{\Gamma \otimes X\vdash \Delta \oplus Y \hspace{.5cm}  Y\otimes \Gamma'\vdash Z\oplus \Delta'}{(\Gamma\otimes X) \otimes \Gamma' \vdash \Delta \oplus (Z \oplus \Delta')}
%$$
%
%In their proof net notation, there are tensoring and untensoring operations for both the tensor and par:
%
%$$
%\begin{tikzpicture}
%	\begin{pgfonlayer}{nodelayer}
%		\node [style=none] (105) at (18.75, 5) {};
%		\node [style=none] (106) at (17.75, 5) {};
%		\node [style=none] (107) at (18.25, 4.25) {};
%		\node [style=none] (108) at (18.25, 3.5) {};
%		\node [style=none] (109) at (17.75, 5.25) {$X$};
%		\node [style=none] (110) at (18.75, 5.25) {$Y$};
%		\node [style=none] (111) at (18.25, 3.25) {$X\otimes Y$};
%		\node [style=otimes] (112) at (18.25, 4.25) {};
%	\end{pgfonlayer}
%	\begin{pgfonlayer}{edgelayer}
%		\draw (108.center) to (107.center);
%		\draw [in=-90, out=30] (107.center) to (105.center);
%		\draw [in=-90, out=150] (107.center) to (106.center);
%	\end{pgfonlayer}
%\end{tikzpicture}
%\ ,\hspace*{.5cm}
%\begin{tikzpicture}
%	\begin{pgfonlayer}{nodelayer}
%		\node [style=none] (113) at (20.75, 3.5) {};
%		\node [style=none] (114) at (19.75, 3.5) {};
%		\node [style=none] (115) at (20.25, 4.25) {};
%		\node [style=none] (116) at (20.25, 5) {};
%		\node [style=none] (117) at (19.75, 3.25) {$X$};
%		\node [style=none] (118) at (20.75, 3.25) {$Y$};
%		\node [style=none] (119) at (20.25, 5.25) {$X\otimes Y$};
%		\node [style=otimes] (120) at (20.25, 4.25) {};
%	\end{pgfonlayer}
%	\begin{pgfonlayer}{edgelayer}
%		\draw (116.center) to (115.center);
%		\draw [in=90, out=-30] (115.center) to (113.center);
%		\draw [in=90, out=-150] (115.center) to (114.center);
%	\end{pgfonlayer}
%\end{tikzpicture}
%\ ,\hspace*{.5cm}
%\begin{tikzpicture}
%	\begin{pgfonlayer}{nodelayer}
%		\node [style=none] (105) at (18.75, 5) {};
%		\node [style=none] (106) at (17.75, 5) {};
%		\node [style=none] (107) at (18.25, 4.25) {};
%		\node [style=none] (108) at (18.25, 3.5) {};
%		\node [style=none] (109) at (17.75, 5.25) {$X$};
%		\node [style=none] (110) at (18.75, 5.25) {$Y$};
%		\node [style=none] (111) at (18.25, 3.25) {$X\oplus Y$};
%		\node [style=oplus, fill=white] (112) at (18.25, 4.25) {};
%	\end{pgfonlayer}
%	\begin{pgfonlayer}{edgelayer}
%		\draw (108.center) to (107.center);
%		\draw [in=-90, out=30] (107.center) to (105.center);
%		\draw [in=-90, out=150] (107.center) to (106.center);
%	\end{pgfonlayer}
%\end{tikzpicture}
%\ ,\hspace*{.5cm}
%\begin{tikzpicture}
%	\begin{pgfonlayer}{nodelayer}
%		\node [style=none] (113) at (20.75, 3.5) {};
%		\node [style=none] (114) at (19.75, 3.5) {};
%		\node [style=none] (115) at (20.25, 4.25) {};
%		\node [style=none] (116) at (20.25, 5) {};
%		\node [style=none] (117) at (19.75, 3.25) {$X$};
%		\node [style=none] (118) at (20.75, 3.25) {$Y$};
%		\node [style=none] (119) at (20.25, 5.25) {$X\oplus Y$};
%		\node [style=oplus, fill=white] (120) at (20.25, 4.25) {};
%	\end{pgfonlayer}
%	\begin{pgfonlayer}{edgelayer}
%		\draw (116.center) to (115.center);
%		\draw [in=90, out=-30] (115.center) to (113.center);
%		\draw [in=90, out=-150] (115.center) to (114.center);
%	\end{pgfonlayer}
%\end{tikzpicture}
%$$
%
%
%
%
%In the proof net notation, the linear distributors then are drawn as follows:
%
%
%$$
%\delta_{X,Y,Z}^L
%=
%\begin{tikzpicture}
%	\begin{pgfonlayer}{nodelayer}
%		\node [style=none] (141) at (25, 5) {};
%		\node [style=none] (142) at (26, 5) {};
%		\node [style=none] (143) at (25.5, 4.25) {};
%		\node [style=none] (144) at (25.5, 3.5) {};
%		\node [style=none] (145) at (25.5, 3.25) {$X\otimes (Y\oplus Z)$};
%		\node [style=otimes] (146) at (25.5, 4.25) {};
%		\node [style=none] (147) at (25.5, 5.75) {};
%		\node [style=none] (148) at (26.5, 5.75) {};
%		\node [style=none] (149) at (26, 5) {};
%		\node [style=oplus, fill=white] (151) at (26, 5) {};
%		\node [style=none] (152) at (26.5, 5.75) {};
%		\node [style=none] (153) at (25.5, 5.75) {};
%		\node [style=none] (154) at (26, 6.5) {};
%		\node [style=none] (155) at (26, 7.25) {};
%		\node [style=oplus, fill=white] (156) at (26, 6.5) {};
%		\node [style=none] (157) at (26, 5) {};
%		\node [style=none] (158) at (25, 5) {};
%		\node [style=none] (159) at (25.5, 5.75) {};
%		\node [style=otimes] (160) at (25.5, 5.75) {};
%		\node [style=none] (161) at (26, 7.5) {$(X\otimes Y)\oplus Z$};
%	\end{pgfonlayer}
%	\begin{pgfonlayer}{edgelayer}
%		\draw (144.center) to (143.center);
%		\draw [in=-90, out=150] (143.center) to (141.center);
%		\draw [in=-90, out=30] (143.center) to (142.center);
%		\draw [in=-90, out=30] (149.center) to (148.center);
%		\draw (155.center) to (154.center);
%		\draw [in=90, out=-30] (154.center) to (152.center);
%		\draw [in=90, out=-150] (154.center) to (153.center);
%		\draw (159.center) to (157.center);
%		\draw [in=90, out=-150] (159.center) to (158.center);
%	\end{pgfonlayer}
%\end{tikzpicture}
%\ , \hspace*{.5cm}
%\delta_{X,Y,Z}^R 
%=
%\begin{tikzpicture}
%	\begin{pgfonlayer}{nodelayer}
%		\node [style=none] (121) at (23.25, 5) {};
%		\node [style=none] (122) at (22.25, 5) {};
%		\node [style=none] (123) at (22.75, 4.25) {};
%		\node [style=none] (124) at (22.75, 3.5) {};
%		\node [style=none] (125) at (22.75, 3.25) {$(X\oplus Y)\otimes Z$};
%		\node [style=otimes] (126) at (22.75, 4.25) {};
%		\node [style=none] (127) at (22.75, 5.75) {};
%		\node [style=none] (128) at (21.75, 5.75) {};
%		\node [style=none] (129) at (22.25, 5) {};
%		\node [style=none] (130) at (22.25, 7.5) {$X\oplus (Y\otimes Z)$};
%		\node [style=oplus, fill=white] (131) at (22.25, 5) {};
%		\node [style=none] (132) at (21.75, 5.75) {};
%		\node [style=none] (133) at (22.75, 5.75) {};
%		\node [style=none] (134) at (22.25, 6.5) {};
%		\node [style=none] (135) at (22.25, 7.25) {};
%		\node [style=oplus, fill=white] (136) at (22.25, 6.5) {};
%		\node [style=none] (137) at (22.25, 5) {};
%		\node [style=none] (138) at (23.25, 5) {};
%		\node [style=none] (139) at (22.75, 5.75) {};
%		\node [style=otimes] (140) at (22.75, 5.75) {};
%	\end{pgfonlayer}
%	\begin{pgfonlayer}{edgelayer}
%		\draw (124.center) to (123.center);
%		\draw [in=-90, out=30] (123.center) to (121.center);
%		\draw [in=-90, out=150] (123.center) to (122.center);
%		\draw [in=-90, out=150] (129.center) to (128.center);
%		\draw (135.center) to (134.center);
%		\draw [in=90, out=-150] (134.center) to (132.center);
%		\draw [in=90, out=-30] (134.center) to (133.center);
%		\draw (139.center) to (137.center);
%		\draw [in=90, out=-30] (139.center) to (138.center);
%	\end{pgfonlayer}
%\end{tikzpicture}
%$$
%
%
%They also have generators for the unit introduction and removal for both monoidal structures; however, there is a combinatorial connectivity condition involving thinning links.
%
%A 3-dimensional graphical calculus  was later explored in  \cite{dunn}. See for example the following projection from their 3-dimensional graphical calculus into proof-nets:
%
%\begin{figure}[H]
%\centering
%\includegraphics[width=130mm]{pictures/proof_net.png}
%\caption{3D representation of proof nets \cite[Fig. 1]{dunn} }
%\end{figure}
%
%The thinning links (drawn as dotted lines on the right hand side of this figure) can be seen to arise as a way to recover the connectivity information of the units which is lost as degeneracies in the projection from 3 to 2-dimensions.
%
%
%This is an essential difference for proof nets for monoidal categories and proof nets for linearly distributive categories. In particular, monoidal categories can be regarded as special cases of linearly distributive categories where both monoidal structures coincide and the linear distributors are the associators.  In this case, the distributor is redundant, and the 3rd dimension adds no new information. Therefore,  the calculus of thinning links which keeps track of the connectivity information of the units is not needed.  Thus, in the monoidal setting one recovers proof nets we reviewed in the previous subsection.
%
%
%
%On a separate note, linearly distributive categories have also been used to explore quantum causality \cite{sander} as well as to give toy models for infinite dimensional quantum processes  \cite{muc}.  Therefore, there is also motivation for understanding proof nets in the non-monoidal setting in the context of exploring quantum mechanics from a categorical perspective, although it is not the focus of this thesis.





\chapter{Conclusion}
\label{chap:conclusion}

\bibliography{bibliography} 
\bibliographystyle{eptcs}
\end{document}
