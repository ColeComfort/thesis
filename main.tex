% !TEX TS-program = pdflatex
% !TEX encoding = UTF-8 Unicode

% This is a simple template for a LaTeX document using the "article" class.
% See "book", "report", "letter" for other types of document.
\errorcontextlines=500

\documentclass[12pt]{ociamthesis}  % default square logo 

%\documentclass[a4paper,11pt,oneside]{bo2ok}
%\usepackage[DIV=14,BCOR=2mm,headinclude=true,footinclude=false]{typearea}


\usepackage{hhline}
\usepackage{bookmark}
\usepackage[table]{xcolor}% http://ctan.org/pkg/xcolor
\usepackage[all,cmtip]{xy} 
\usepackage{float}

\usepackage{tikzit}
\input{thesis.tikzstyles}
\input{thesis.tikzdefs}

\usepackage{comment}


\usepackage{mdframed}
\usepackage{arydshln}
\usepackage{multicol}
\renewcommand{\tilde}{\widetilde}
\usepackage{everypage}
\usepackage{lipsum}
\usepackage{amsthm}
\usepackage[inline]{enumitem}   
\usepackage{scalerel,stackengine}
\stackMath
\renewcommand\hat[1]{%
\savestack{\tmpbox}{\stretchto{%
  \scaleto{%
    \scalerel*[\widthof{\ensuremath{#1}}]{\kern-.6pt\bigwedge\kern-.6pt}%
    {\rule[-\textheight/2]{1ex}{\textheight}}%WIDTH-LIMITED BIG WEDGE
  }{\textheight}% 
}{0.5ex}}%
\stackon[1pt]{#1}{\tmpbox}%
}
\parskip 1ex



\newcommand{\bcell}{\cellcolor{black!10}}

\makeatletter
% This command ignores the optional argument for itemize and enumerate lists
\newcommand{\inlineitem}[1][]{%
\ifnum\enit@type=\tw@
    {\descriptionlabel{#1}}
  \hspace{\labelsep}%
\else
  \ifnum\enit@type=\z@
       \refstepcounter{\@listctr}\fi
    \quad\@itemlabel\hspace{\labelsep}%
\fi}
\makeatother
\parindent=0pt
 

%\usepackage{extpfeil}
%\newextarrow{\xleftarrowtail}{500(40)}{\leftarrow\relbar<}
%\newextarrow{\xrightarrowtail}{500(40)}{>\relbar\rightarrow}



\makeatletter
\def\proarrowfill@#1#2#3#4#5{%
  $\m@th\thickmuskip0mu\medmuskip\thickmuskip\thinmuskip\thickmuskip
   \relax#5#1\mkern-7mu%
   \cleaders\hbox{$#5\mkern-2mu#2\mkern-2mu$}\hfill
   \mathclap{#3}\mathclap{#2}%
   \cleaders\hbox{$#5\mkern-2mu#2\mkern-2mu$}\hfill
   \mkern-7mu#4$%
}
\def\rightproarrowfill@{%
  \proarrowfill@\relbar\relbar\mapstochar\rightarrow}
\newcommand\xproarrow[2][]{%
  \ext@arrow 0055{\rightproarrowfill@}{#1}{#2}}
\makeatother

\newcommand{\proarrow}{\xproarrow{}}



%\newcommand\xrightarrowtail[2][]{\ensurestackMath{\mathrel{%
%  \stackengine{1pt}{%
 %   \stackengine{0pt}{\rightarrowtail}{\scriptstyle#2}{O}{c}{F}{F}{S}%
%  }{\scriptstyle#1}{U}{c}{F}{F}{S}%
%}}}



%\newcommand\xleftarrowtail[2][]{\ensurestackMath{\mathrel{%
%  \stackengine{1pt}{%
%    \stackengine{0pt}{\leftarrowtail}{\scriptstyle#2}{O}{c}{F}{F}{S}%
%  }{\scriptstyle#1}{U}{c}{F}{F}{S}%
%}}}


%\newcommand{\xrightarrowtail}[1]{\!\!\stackrel{#1}{\xymatrix@C=0.78em{\ar@{>->}[r]&}}\!\!\!}
%\newcommand{\xleftarrowtail}[1]{\!\!\!\stackrel{#1}{\xymatrix@C=0.78em{&\ar@{>->}[l]}}\!\!}



\newcommand{\alr}{{\sf alr}}
\newcommand{\lr}{{\sf lr}}
\newcommand{\rel}{{\sf r}}
\newcommand{\aih}{{\sf aih}}
\newcommand{\ih}{{\sf ih}}


\newcommand{\xrightarrowtail}[1]{\!\!{\xymatrix@C=1em{\ar@{>->}[r]^{#1}&}}\!\!\!}
\newcommand{\xleftarrowtail}[1]{\!\!\!{\xymatrix@C=1em{&\ar@{>->}[l]_{#1}}}\!\!}


\newcommand{\xrightarrowiso}[1]{\!\!{\xymatrix@C=1em{\ar@{->}[r]^{#1}_\cong&}}\!\!\!}
\newcommand{\xleftarrowiso}[1]{\!\!\!{\xymatrix@C=1em{&\ar@{->}[l]_{#1}^\cong}}\!\!}




%\theoremstyle{theorem} 
  \newtheorem{theorem}{Theorem}[section]
  \newtheorem{corollary}[theorem]{Corollary}
  \newtheorem{lemma}[theorem]{Lemma}
  \newtheorem{proposition}[theorem]{Proposition}
  
%\theoremstyle{definition}    conjecture
  \newtheorem{definition}[theorem]{Definition}
  \newtheorem{example}[theorem]{Example}
  \newtheorem{conjecture}[theorem]{Conjecture}
  \newtheorem{remark}[theorem]{Remark}
  
  
\newcommand{\Mat}{\mathsf{Mat}}


\newcommand{\AND}{{\sf and}}
\newcommand{\Set}{\Sets}
\newcommand{\Mnd}{{\sf Mnd}}
\newcommand{\Map}{{\sf Map}}
\newcommand{\Monot}{{\sf Monot}}

\newcommand{\dom}{{\sf dom}}
\newcommand{\cod}{{\sf cod}}

\newcommand{\cnot}{\mathsf{cnot}}
\newcommand{\tof}{\mathsf{tof}}
\newcommand{\Not}{\mathsf{not}}
\newcommand{\zeroin}{|0\rangle}
\newcommand{\zeroout}{\langle 0|}
\newcommand{\CNOT}{\mathsf{CNOT}}
\newcommand{\Sets}{\mathsf{Set}}
\newcommand{\FSets}{\mathsf{FSet}}
\newcommand{\FinOrdMonot}{\mathsf{FinOrdMonot}}
%\newcommand{\FSet}{\mathsf{FinOrd}}
\newcommand{\FinOrd}{\mathsf{FinOrd}}
\newcommand{\FinMonot}{\mathsf{FinMonot}}
\newcommand{\Fin}{\mathsf{Fd}}
\newcommand{\TOF}{\mathsf{TOF}}
\newcommand{\Span}{\mathsf{Span}}
\newcommand{\dec}{\mathsf{dec}}
\newcommand{\Rel}{\mathsf{Rel}}
\newcommand{\FRel}{\mathsf{FRel}}
\newcommand{\op}{\mathsf{op}}
\newcommand{\co}{\mathsf{co}}
\newcommand{\Hilb}{\mathsf{Hilb}}
\newcommand{\FdHilb}{\mathsf{FHilb}}
\newcommand{\FHilb}{\mathsf{FHilb}}
\newcommand{\CPM}{\mathsf{CPM}}
\newcommand{\CP}{\mathsf{CP}}
\newcommand{\FPinj}{\mathsf{FPinj}}
\newcommand{\FPar}{\mathsf{FPar}}
\newcommand{\FSpan}{\mathsf{FSpan}}
\newcommand{\Pinj}{\mathsf{Pinj}}
\newcommand{\Par}{\mathsf{Par}}
\newcommand{\Aff}{\mathsf{Aff}}
\newcommand{\ParIso}{\mathsf{ParIso}}

\newcommand{\Total}{\mathsf{Total}}
%\newcommand{\CFrob}{\mathsf{CFrob}}
\newcommand{\tr}{\mathsf{Tr}}
\newcommand{\ox}{\otimes}
\newcommand{\Csp}{{\sf Cospan}}
\newcommand{\Corel}{{\sf Corel}}
\newcommand{\Bool}{\mathbb{B}}
\newcommand{\Iso}{{\sf Iso}}
\renewcommand{\P}{{\sf p}}
\newcommand{\pmul}{{\sf pmul}}

\newcommand{\Prof}{{\sf Prof}}
\newcommand{\Mod}{{\sf Mod}}

\newcommand{\unit}{{\sf unit}}
\newcommand{\comm}{{\sf comm}}
\newcommand{\assoc}{{\sf assoc}}
\newcommand{\inj}{{\sf Inj}}
\newcommand{\surj}{{\sf Surj}}
\newcommand{\PSurj}{{\sf PSurj}}

\newcommand{\pre}{{\sf pre}}
\newcommand{\poly}{{\sf poly}}
\newcommand{\sub}{{\sf sub}}

\newcommand{\C}{\mathbb{C}}
\newcommand{\R}{\mathbb{R}}
\newcommand{\CoPara}{{\sf CoPara}}

\newcommand{\ch}{{\sf ch}}
\newcommand{\m}{{\sf m}}
\newcommand{\cm}{{\sf cm}}
\newcommand{\cb}{{\sf cb}}
\newcommand{\pcm}{{\sf pcm}}
\renewcommand{\r}{{\sf r}}
%\newcommand{\scfrob}{{\sf scfrob}}

\newcommand{\bi}{{\sf b1}}
\newcommand{\bii}{{\sf b2}}
\newcommand{\biii}{{\sf b3}}
\newcommand{\biv}{{\sf b4}}

\newcommand{\Kl}{{\sf Kl}}
\newcommand{\Mon}{{\sf Mon}}

\newcommand{\ev}{{\sf ev}}

\renewcommand{\P}{{\sf p}}
\newcommand{\f}{\mathsf{f}}

\newcommand{\F}{\mathbb{F}}
\newcommand{\X}{\mathbb{X}}
\newcommand{\Y}{\mathbb{Y}}
\newcommand{\Z}{\mathbb{Z}}
\newcommand{\N}{\mathbb{N}}
\newcommand{\T}{\mathbb{T}}
\newcommand{\s}{\mathbb{S}}
\newcommand{\U}{\mathbb{U}}

\newcommand{\IH}{\mathbb{IH}}


\newcommand{\M}{\mathcal{M}}
\newcommand{\E}{\mathcal{E}}



\renewcommand{\sp}{\mathsf{sp}}
\newcommand{\pr}{\mathsf{p}}
\newcommand{\iso}{\mathsf{i}}







\newcounter{eq}

\makeatletter
\newcommand{\ltxlabel}{\ltx@label}
\makeatother

\newcommand{\eqzxa}[1]{%
\refstepcounter{eq}%
\ltxlabel{#1}%
\eqstack{#1}%
}




\newcommand{\eqstack}[1]{%
\stackrel{\scalebox{.6}{(\ref{#1})}}{=}%
}

\newcommand{\eq}[1]{\stackrel{\scalebox{.6}{#1}}{=}}

\newcommand{\defeq}[1]{\stackrel{\scalebox{.6}{#1}}{:=}}


\newcommand{\eref}{\eqstack}

\newcommand{\erefop}[1]{%
\stackrel{\scalebox{.6}{(\ref{#1})${}^\op$}}{=}%
}

\newcommand{\ZXA}{\mathsf{ZX}\textit{\&}}


\newcommand{\Vect}{\mathsf{Vect}}
\newcommand{\FVect}{\mathsf{FVect}}
\newcommand{\Lag}{\mathsf{Lag}}
\newcommand{\im}{\mathsf{im}}
\renewcommand{\ker}{\mathsf{ker}}
\newcommand{\ZX}{\mathsf{ZX}}
\newcommand{\ZH}{\mathsf{ZH}}
\DeclareMathSymbol{\bot}{\mathord}{symbols}{"3F}


\newcommand{\pullbackcorner}[1][dl]{\save*!/#1-1pc/#1:(-1,1)@^{|-}\restore}

\renewcommand{\epsilon}{\varepsilon}
\renewcommand{\phi}{\varphi}
\renewcommand{\bar}[1]{\overline{#1}\hspace*{.01cm}}

\newcommand{\Stab}{{\sf Stab}}
\newcommand{\LinRel}{\sf LinRel}


\newcommand{\Isot}{{\sf Isot}}
\newcommand{\Co}{{\sf Co}}

\newcommand{\B}{\mathbb{B}}

\newcommand{\STOCH}{\mathsf{STOCH}}

%\renewcommand\floatpagefraction{.9}
%\renewcommand\topfraction{.9}
%\renewcommand\bottomfraction{.9}
%\renewcommand\textfraction{.1}   
%\Setcounter{totalnumber}{50}
%\Setcounter{topnumber}{50}
%\Setcounter{bottomnumber}{50}


\usepackage{amsmath}
\usepackage{pict2e}

\newcommand{\lbparen}{\{
}

\newcommand{\rbparen}{ \}
}



\newcommand{\Cat}{{\sf Cat}}
\newcommand{\D}{\mathcal{D}}
%\newcommand{\sfa}{{\sf SFA}}
\newcommand{\one}{\mathbbm{1}}






\newdir{|>}{-<5pt,0pt>{
\begin{tikzpicture}[scale=.7]
	\begin{pgfonlayer}{nodelayer}
		\node [style=none] (0) at (0, 0) {};
		\node [style=none] (1) at (1, 0) {};
		\node [style=none] (2) at (-1, -0.25) {};
	\end{pgfonlayer}
	\begin{pgfonlayer}{edgelayer}
		\draw (2.center) to (0.center);
		\draw (0.center) to (1.center);
	\end{pgfonlayer}
\end{tikzpicture}
}}
\newdir{|<}{-<5pt,0pt>{
\begin{tikzpicture}[scale=.9]
	\begin{pgfonlayer}{nodelayer}
		\node [style=none] (0) at (0, -0.25) {};
		\node [style=none] (1) at (-1, -0.25) {};
		\node [style=none] (2) at (1, 0) {};
	\end{pgfonlayer}
	\begin{pgfonlayer}{edgelayer}
		\draw (2.center) to (0.center);
		\draw (0.center) to (1.center);
	\end{pgfonlayer}
\end{tikzpicture}
}}




\newcommand{\zcirc}{\begin{tikzpicture}
	\begin{pgfonlayer}{nodelayer}
		\node [style=Z] (0) at (0, 0) {};
	\end{pgfonlayer}
\end{tikzpicture}}

\newcommand{\xcirc}{\begin{tikzpicture}
	\begin{pgfonlayer}{nodelayer}
		\node [style=X] (0) at (0, 0) {};
	\end{pgfonlayer}
\end{tikzpicture}}



\newcommand{\skewpullbackcorner}[1][dl]{\save*!/#1-1.1pc/#1:(-.5,1)@^{|>}\restore}
\newcommand{\skewpushoutcorner}[1][dl]{\save*!/#1-1pc/#1:(-1,1)@^{|<}\restore}


\DeclareFontFamily{U}{mathx}{\hyphenchar\font45}
\DeclareFontShape{U}{mathx}{m}{n}{
      <5> <6> <7> <8> <9> <10>
      <10.95> <12> <14.4> <17.28> <20.74> <24.88>
      mathx10
      }{}
\DeclareSymbolFont{mathx}{U}{mathx}{m}{n}
\DeclareFontSubstitution{U}{mathx}{m}{n}
\DeclareMathAccent{\widecheck}{0}{mathx}{"71}
\DeclareMathAccent{\wideparen}{0}{mathx}{"75}

\def\cs#1{\texttt{\char`\\#1}}


\usepackage{amsmath}


\usepackage{hyperref}

\newcommand\numeq[2]%
  {\label{#2}\stackrel{\scriptscriptstyle(\mkern-1.5mu#1\mkern-1.5mu)}{=}}

\newcommand{\cubetopbl}{A}
\newcommand{\cubetopbr}{B}
\newcommand{\cubetopfl}{C}
\newcommand{\cubetopfr}{D}
\newcommand{\cubebotbl}{E}
\newcommand{\cubebotbr}{F}
\newcommand{\cubebotfl}{G}
\newcommand{\cubebotfr}{H}

\xymatrixrowsep{.5cm}
\xymatrixcolsep{.65cm}


\newcommand{\fa}{{\sf fa}}
\newcommand{\sfa}{{\sf sfa}}
\newcommand{\scfa}{{\sf scfa}}
\newcommand{\cfa}{{\sf cfa}}

\newcommand{\disc}
{{
\begin{tikzpicture}[scale=.5]
	\begin{pgfonlayer}{nodelayer}
		\node [style=none] (7) at (24.2, -0.25) {};
		\node [style=none] (8) at (24.2, 0.275) {};
		\node [style=none] (9) at (24.4, 0.025) {};
		\node [style=none] (10) at (24, 0.025) {};
		\node [style=none] (11) at (24.35, 0.1) {};
		\node [style=none] (12) at (24.05, 0.1) {};
		\node [style=none] (13) at (24.3, 0.175) {};
		\node [style=none] (14) at (24.1, 0.175) {};
	\end{pgfonlayer}
	\begin{pgfonlayer}{edgelayer}
		\draw (8.center) to (7.center);
		\draw (10.center) to (9.center);
		\draw (12.center) to (11.center);
		\draw (14.center) to (13.center);
	\end{pgfonlayer}
\end{tikzpicture}
}}


\tikzset{meter/.append style={draw, inner sep=10, rectangle, font=\vphantom{A}, minimum width=30, line width=.8,
 path picture={\draw[black] ([shift={(.1,.3)}]path picture bounding box.south west) to[bend left=50] ([shift={(-.1,.3)}]path picture bounding box.south east);\draw[black,-latex] ([shift={(0,.1)}]path picture bounding box.south) -- ([shift={(.3,-.1)}]path picture bounding box.north);}}}



\tikzset{
  unit/.style={shape=rectangle, rounded corners,inner sep=0.3em,draw=black,fill=white, font={$I$}}
}


\usepackage{bbm}

\usepackage{Cobordism}
\pgfsetlayers{bottom,background,main,internal,foreground,label,cobordism,top,selectionbox,edgelayer,nodelayer}
%\input{styles.tikzdefs}
%\input{styles.tikzstyles}


\usepackage[utf8]{inputenc} % set input encoding (not needed with XeLaTeX)


\title{Relational semantics for quantum protocols}
\author{Cole Comfort}
\college{New College}  %your college
\degree{Doctor of Philosophy in Computer Science} 
\degreedate{????? 2023}    

%\renewcommand{\submittedtext}{change the default text here if needed}
%\date{} % Activate to display a given date or no date (if empty),
         % otherwise the current date is printed 

\newcommand{\Set}{\Sets}
\newcommand{\Mnd}{{\sf Mnd}}

\begin{document}
\maketitle

\tableofcontents


\chapter{Introduction}


Nonlocality of quantum mechanics. Some fragments of quantum mechanics do have local hidden variable models.  We exhibit such models for two fragment. The first such fragment is that generated by the Toffoli gate as well as state preparation and post selection in the $Z$ and $X$ basis.  The second fragment is odd prime qudit dimensional stabilizer circuits.  The first fragment is a full subcategory of matrices under the natural numbers.  This is equivalent to a full subcategory of spans of finite sets.
Stabilizer circuits are projectively equivalent to a subcategory of relations between matrices over finite fields.

There is a conceptual advantage to modelling these fragments in terms of categories of spans or relations:  this perspective illuminates some of the most elegant symmetries of the ZX and ZH.

For example, \dag-Frobenius algebras in $\FHilb$, which are used to express orthonormal bases are shown to arise from a colimit in such a fragment.  Moreover, the Euler decomposition of the Fourier transform is shown to arise from the hopf law.

The CPM construction, which is traditionally used to model density matrices using monoidal categories, also surprisingly plays various roles in our presentations of these fragments of quantum circuits.  Not only does it allow one to mix, quantum circuits, a very closely related construction adds the plus state to partial, reversible boolean circuits.  Perhaps more surprisingly, the CPM construction also captures the notion of the Fourier transform, and Euler decomposition; when it is applied to phase-free circuits: ie, circuits which correspond to linear subspaces of $\F_p$-vector spaces.

Finally, we conclude the thesis by regarding {\em quantum protocols themselves} as a generalized category of relations.  The machinery involved in this case is the 2-category of profunctors: which can be thought of as bimodules of categories.  This gives a structural account of the scalable ZX-calculus, which is used to parametrize families of circuits using string diagrams.  In particular, this is a limit over profunctors: an instance of the so called Benabou-Grothendieck construction.  Because of the generality at which this is proven, we suggest applications in other domains where string diagrams are used. 


\chapter{Background}

\section{Category theory}





We assume that the reader has a basic understanding of monoidal bicategories.  For reference, refer to ?????JAMIE AND CHRIS HEUNEN

\subsection{Symmetric monoidal categories}


\subsection{Internal category theory}

%\begin{definition}
%\label{def:monad}
%%monad
%\end{definition}
%
%
%
%\begin{definition}
%\label{def:span}
%
%%2-category of spans, cospans
%\end{definition}
%
%
%\begin{definition}
%\label{def:rel}
%
%%2-category of relations, corelations
%\end{definition}


Categories of spans, cospans and relations are central themes to this thesis.

\begin{definition}
Given a finitely complete category $\X$, let $\Span(\X)$ denote the 2-category of spans in $\X$.  Let $\Span^\sim(\X)$ denote the 1-category of spans given by quotienting 1-cells by isomorphism.

Similarly, given a finitely cocomplete category $\X$, let $\Csp(\X)$ denote the 2-category of cospans in $\X$.  Let $\Csp^\sim(\X)$ denote the 1-category of cospans given by quotienting 1-cells by isomorphism.

Given a regular category $\X$ let $\Rel(\X)$ denote the category of relations in $\X$.
\end{definition}

This allows us to regard categories as monads:
\begin{definition}
\label{def:monad}
%monad
\end{definition}


\begin{definition}
\label{def:internalcat}

%Internal category
Given a category $\mathcal V$ with finite pullbacks $\mathcal V$, a $\mathcal V$-{\bf internal category} is a monad in $\Span(\mathcal V)$.
\end{definition}

Internal categories are indeed categories.  The collection of objects is given by the feet of the span, the set of morphisms by the apex, the domain and codomain by the left and right legs respectively.  The components of the unit of the comonad give the identity morphisms and the multiplication of the monad gives the composition.

\begin{lemma}
\label{lem:internalcat}

Monads internal to $\Set$ are in bijection with small categories.
\end{lemma}


It is not the case that monad maps correspond to functors between internal categories.  A canonical way to obtain such a notion requires the machinery of double categories, which is outside of the purview of this thesis.  However, (globular) 2-categories suffice to construct internal profunctors which is what we need the machinery for.


This allows us to compose internal categories with the same set of objects via distributive law via the composite monad induced by a distributive law on monads in $\Span(\mathcal{V})$.


Just as we can regard categories as certain monoids; by changing the setting with which we are working internal to, we can do the same for {\em monoidal} categories.

\begin{definition}
\label{def:monoid}
Let $\Mon$ denote the category with set-monoids as objects and monoid homorphisms as morphisms.
\end{definition}



\begin{lemma}
\label{def:internalmonoidalcat}

Monads internal to $\Mon$ are in bijection with small monoidal categories.
\end{lemma}

We want to be able to take distributive laws of two categories which both share some structure.  For example, distributive laws of symmetric monoidal categories where the braiding of both categories are identified with each other.  For this, we need to work in bimodules of internal categories:

\begin{definition}
Given a category $\mathcal V$ with finite pullbacks and coequalizers preserving them, let $\mathcal V-\Prof:=\Mod(\Mnd(\mathcal V))$ denote the 2-category of $\mathcal V$-{\bf internal profunctors}.  
The 1-cells of $\Prof$ are called (internal) {\bf  profunctors}.
The tensor product of bimodules of internal categories is the (internal) {\bf coend}.
\end{definition}




%
%String diagrams are the canonical graphical calculi for {\em strict monoidal categories}. Objects are drawn as wires, morphisms are drawn as boxes; the tensor product is giving by connecting the wires together and the tensor product is given by monoidal pasting.  The coherence for strict monoidal categories is equivalent to planar isotopy of these diagrams. As a matter of convention, we will draw the order of composition from bottom to top and the tensor from left to right.
%
%
%GIVE EXAMPLE
%
%String diagrams for monoidal categories can be augmented to describe morphisms in non-strict monoidal categories by adding four connectives and equations:
%
%\begin{definition}
%Given a (non-strict) monoidal category the monoidal category of proof nets in $\X$ is generated by the string diagrams for $\X$ addition to the following generators for all objects $X,Y$
%
%modulo the equations
%\end{definition}
%
%
%\begin{lemma}
%There is a fully faithful monoidal functor from $\C$ to proof nets over $\C$ given by:
%
%Draw action
%
%
%give coherence rules
%\end{lemma}
%
%Although this has long been known, the idea of proof nets for monoidal categories has recently been rediscovered \cite{wilson}, where the coauthors exhibit proof nets as the residue of a novel algebraic proof of MacLane's coherence theorem for monoidal categories.In the ZX-calculus literature, proof nets for strict monoidal categories have also been rediscovered as the scalable ZX calculus.  In the scalable ZX-calculus, the nets for the units have been ommited, and they use the proof nets to index wires when specifying quantum protocols diagrammatically.
%
% we give a novel conceptual proof in Section \ref{??} which constructs proof nets from string diagrams in a canonical way.  This way of viewing things can be generalized to other settings that monoidal categories.


\begin{definition}
\label{def:pro}
A {\bf prop} is a  strict  symmetric monoidal category generated by one object under the tensor product.  A {\bf multicoloured prop} is a strict (symmetric) monoidal category generated by some specified class of object under the tensor product.
\end{definition}


\begin{definition}
\label{def:monoidaltheory}

%A {\bf symmetric monoidal theory} is a triple $T=({\sf Ob},\Sigma ,E \)$. $\sf Ob$ is a set of {\em objects}. $\Sigma\in [{\sf Ob}]\times [{\sf Ob}]^{G}$ is a set of {\bf generators} $g$ with associated arities $(X,Y) \in [{\sf Ob}]\times [{\sf Ob}]$,  denoted $g:X\to Y$, where $[\_]$ is the finite list monad. Let $\Sigma'$ denote the set $\Sigma\sqcup\{c_X: [X,X]\to[X,X] | \forall X \in {\sf Ob} \}$, where the $c_X$ are regarded as the braiding maps.  Moreover, let $\Sigma^*$ denote the 
%$E \subseteq \{(f:X\to Y,g:X\to Y) \in \Sigma^2\}$

A {\bf symmetric monoidal theory} is a triple $T=({\sf Ob},\Sigma ,E )$ where: ${\sf Ob}$ is regarded as the set of generating objects; $\Sigma$, the set of generating morphisms with arities in $[{\sf Ob}]\times [{\sf Ob}]$, denoted $f:X\to Y$;  and $E$ the set of generating equations between parallel maps generated by $\Sigma \sqcup C$  where $C=\{c_X:[X,X]\to [X,X]|\forall X \in {\sf Ob}\}$ is the set of distinguished braiding maps.

Every symmetric monoidal theory uniquely defines a multicoloured prop $\bar T$ with object set $\sf Ob$ by generating the free strict monoidal category with morphisms in $\Sigma\sqcup C$   corresponding to the braiding maps, and then quotienting by the equations in $E$ as well as the axioms of a strict symmetric  monoidal category.  We say that $\bar T$ is generated by, or presented by $T$.
\end{definition}

In practice, we won't explicitly regard a symmetric monoidal theory as a triple; rather, we will present multicoloured props by first drawing the string diagrams for all of the generators and then the string diagrams for the equations. For example:

\begin{example}
Let $\cm$ denote the prop generated by the free commutative monoid on one object; that is, the prop with generators:


and equations:



\end{example}

Throughout this thesis, we give presentations for concrete, symmetric monoidal categories, such as the following simple example:

\begin{lemma}
$\cm$ is equivalent as a symmetric monoidal category to the symmetric monoidal category of finite sets and functions with respect to the coproduct as the monoidal struture.
\end{lemma}
%
%There are various ways to combine symmetric monoidal theories theories, by coproduct, pushout and distributive laws:
%
%\begin{lemma}
%Given two symmetric  monoidal theories $({\sf Ob}_1, \Sigma_1,E_1)$  and $({\sf Ob}_1, \Sigma_2,E_2)$  the coproduct  $\bar{(\Sigma_1,E_1)}+\bar{(\Sigma_2,E_2)}$ is generated by the (symmetric) monoidal theory $(\Sigma_1+\Sigma_2,E_1+E_2)$.
%\end{lemma}


\begin{lemma}
pushout of symmetric monoidal theories
\end{lemma}

%Notice that the coproduct is a special case of the pushout.

\begin{definition}
distributive law of symmetric monoidal theories
\end{definition}






\begin{definition}
Consider the following two distributive laws: 
\begin{align*}
\cm^\op  \otimes_\P \cm;&
  \begin{tikzpicture}
	\begin{pgfonlayer}{nodelayer}
		\node [style=X] (0) at (-3.75, -1) {};
		\node [style=none] (1) at (-4, -1.75) {};
		\node [style=none] (2) at (-3.5, -1.75) {};
		\node [style=Z] (3) at (-3.75, -0.25) {};
		\node [style=none] (4) at (-4, 0.5) {};
		\node [style=none] (5) at (-3.5, 0.5) {};
	\end{pgfonlayer}
	\begin{pgfonlayer}{edgelayer}
		\draw [in=90, out=-60, looseness=1.00] (0) to (2.center);
		\draw [in=-120, out=90, looseness=1.00] (1.center) to (0);
		\draw (0) to (3);
		\draw [in=60, out=-90, looseness=1.00] (5.center) to (3);
		\draw [in=-90, out=120, looseness=1.00] (3) to (4.center);
	\end{pgfonlayer}
  \end{tikzpicture}
  \eqzxa{bi.one}
  \begin{tikzpicture}
	\begin{pgfonlayer}{nodelayer}
		\node [style=X] (0) at (-4, 0.5) {};
		\node [style=Z] (1) at (-4, -0.25) {};
		\node [style=X] (2) at (-4.5, 0.5) {};
		\node [style=Z] (3) at (-4.5, -0.25) {};
		\node [style=none] (4) at (-4, -1) {};
		\node [style=none] (5) at (-4.5, -1) {};
		\node [style=none] (6) at (-4.5, 1.25) {};
		\node [style=none] (7) at (-4, 1.25) {};
	\end{pgfonlayer}
	\begin{pgfonlayer}{edgelayer}
		\draw [bend left, looseness=1.25] (0) to (1);
		\draw [bend right, looseness=1.25] (2) to (3);
		\draw (1) to (2);
		\draw (3) to (0);
		\draw (0) to (7.center);
		\draw (6.center) to (2);
		\draw (3) to (5.center);
		\draw (4.center) to (1);
	\end{pgfonlayer}
\end{tikzpicture},
\hspace*{.5cm}
  \begin{tikzpicture}
	\begin{pgfonlayer}{nodelayer}
		\node [style=Z] (0) at (-4, -0) {};
		\node [style=X] (1) at (-4, -0.75) {};
		\node [style=none] (2) at (-4.25, -1.5) {};
		\node [style=none] (3) at (-3.75, -1.5) {};
	\end{pgfonlayer}
	\begin{pgfonlayer}{edgelayer}
		\draw [in=-60, out=90, looseness=1.00] (3.center) to (1);
		\draw (1) to (0);
		\draw [in=90, out=-120, looseness=1.00] (1) to (2.center);
	\end{pgfonlayer}
  \end{tikzpicture}
  \eqzxa{bi.two}
  \begin{tikzpicture}
	\begin{pgfonlayer}{nodelayer}
		\node [style=Z] (0) at (-4.25, -0.75) {};
		\node [style=none] (1) at (-4.25, -1.5) {};
		\node [style=none] (2) at (-3.5, -1.5) {};
		\node [style=Z] (3) at (-3.5, -0.75) {};
	\end{pgfonlayer}
	\begin{pgfonlayer}{edgelayer}
		\draw (2.center) to (3);
		\draw (0) to (1.center);
	\end{pgfonlayer}
  \end{tikzpicture},
  \hspace*{.5cm}
   \begin{tikzpicture}[yscale=-1]
	\begin{pgfonlayer}{nodelayer}
		\node [style=X] (0) at (-4, -0) {};
		\node [style=Z] (1) at (-4, -0.75) {};
		\node [style=none] (2) at (-4.25, -1.5) {};
		\node [style=none] (3) at (-3.75, -1.5) {};
	\end{pgfonlayer}
	\begin{pgfonlayer}{edgelayer}
		\draw [in=-60, out=90, looseness=1.00] (3.center) to (1);
		\draw (1) to (0);
		\draw [in=90, out=-120, looseness=1.00] (1) to (2.center);
	\end{pgfonlayer}
  \end{tikzpicture}
  \erefop{bi.two}
   \begin{tikzpicture}[yscale=-1]
	\begin{pgfonlayer}{nodelayer}
		\node [style=X] (0) at (-4.25, -0.75) {};
		\node [style=none] (1) at (-4.25, -1.5) {};
		\node [style=none] (2) at (-3.5, -1.5) {};
		\node [style=X] (3) at (-3.5, -0.75) {};
	\end{pgfonlayer}
	\begin{pgfonlayer}{edgelayer}
		\draw (2.center) to (3);
		\draw (0) to (1.center);
	\end{pgfonlayer}
  \end{tikzpicture},
\hspace*{.5cm}
  \begin{tikzpicture}[rotate=90]
	\begin{pgfonlayer}{nodelayer}
		\node [style=Z] (0) at (-8.25, -0) {};
		\node [style=X] (1) at (-9.25, -0) {};
	\end{pgfonlayer}
	\begin{pgfonlayer}{edgelayer}
		\draw (0) to (1);
	\end{pgfonlayer}
\end{tikzpicture}
  \eqzxa{extra}
\\
 \cm \otimes_\P \cm^\op;&
    \begin{tikzpicture}[rotate=90]
	\begin{pgfonlayer}{nodelayer}
		\node [style=X] (0) at (-6.25, 0.25) {};
		\node [style=none] (1) at (-7, 0.25) {};
		\node [style=none] (2) at (-4.75, 0.25) {};
		\node [style=X] (3) at (-5.5, 0.25) {};
	\end{pgfonlayer}
	\begin{pgfonlayer}{edgelayer}
		\draw (0) to (1.center);
		\draw (3) to (2.center);
		\draw [bend right, looseness=1.25] (3) to (0);
		\draw [bend right, looseness=1.25] (0) to (3);
	\end{pgfonlayer}
  \end{tikzpicture}
  \eqzxa{special}
  \begin{tikzpicture}[rotate=90]
	\begin{pgfonlayer}{nodelayer}
		\node [style=none] (0) at (-7, 0.25) {};
		\node [style=none] (1) at (-6, 0.25) {};
	\end{pgfonlayer}
	\begin{pgfonlayer}{edgelayer}
		\draw (1.center) to (0.center);
	\end{pgfonlayer}
  \end{tikzpicture},
  \hspace*{.5cm}
  \begin{tikzpicture}[rotate=90]
	\begin{pgfonlayer}{nodelayer}
		\node [style=X] (0) at (-7, -0) {};
		\node [style=X] (1) at (-6.25, 0.5) {};
		\node [style=none] (2) at (-7, 0.75) {};
		\node [style=none] (3) at (-7.75, 0.75) {};
		\node [style=none] (4) at (-7.75, -0) {};
		\node [style=none] (5) at (-6.25, -0.25) {};
		\node [style=none] (6) at (-5.5, -0.25) {};
		\node [style=none] (7) at (-5.5, 0.5) {};
	\end{pgfonlayer}
	\begin{pgfonlayer}{edgelayer}
		\draw (6.center) to (5.center);
		\draw [in=-30, out=180, looseness=1.00] (5.center) to (0);
		\draw (1) to (0);
		\draw [in=0, out=150, looseness=1.00] (1) to (2.center);
		\draw (2.center) to (3.center);
		\draw (0) to (4.center);
		\draw (1) to (7.center);
	\end{pgfonlayer}
  \end{tikzpicture}
 =
  \begin{tikzpicture}[rotate=90,xscale=-1]
	\begin{pgfonlayer}{nodelayer}
		\node [style=X] (0) at (-7, -0) {};
		\node [style=X] (1) at (-6.25, 0.5) {};
		\node [style=none] (2) at (-7, 0.75) {};
		\node [style=none] (3) at (-7.75, 0.75) {};
		\node [style=none] (4) at (-7.75, -0) {};
		\node [style=none] (5) at (-6.25, -0.25) {};
		\node [style=none] (6) at (-5.5, -0.25) {};
		\node [style=none] (7) at (-5.5, 0.5) {};
	\end{pgfonlayer}
	\begin{pgfonlayer}{edgelayer}
		\draw (6.center) to (5.center);
		\draw [in=-30, out=180, looseness=1.00] (5.center) to (0);
		\draw (1) to (0);
		\draw [in=0, out=150, looseness=1.00] (1) to (2.center);
		\draw (2.center) to (3.center);
		\draw (0) to (4.center);
		\draw (1) to (7.center);
	\end{pgfonlayer}
  \end{tikzpicture}
  \eqzxa{frob}
  \begin{tikzpicture}[rotate=90]
	\begin{pgfonlayer}{nodelayer}
		\node [style=none] (0) at (-4.75, -0.25) {};
		\node [style=X] (1) at (-5.5, -0) {};
		\node [style=none] (2) at (-7, -0.25) {};
		\node [style=X] (3) at (-6.25, 0) {};
		\node [style=none] (4) at (-4.75, 0.25) {};
		\node [style=none] (5) at (-7, 0.25) {};
	\end{pgfonlayer}
	\begin{pgfonlayer}{edgelayer}
		\draw [in=-30, out=180, looseness=1.25] (0.center) to (1);
		\draw (3) to (1);
		\draw [in=180, out=30, looseness=1.25] (1) to (4.center);
		\draw [in=0, out=-150, looseness=1.25] (3) to (2.center);
		\draw [in=0, out=150, looseness=1.25] (3) to (5.center);
	\end{pgfonlayer}
\end{tikzpicture}
  \end{align*}

The former yields, {\sf cb}, the prop for the free {\bf bicommutative bialgebra} and the latter yields, {\sf scfa}, the prop for the free {\bf special commutative Frobenius algebra}.

\end{definition}




\begin{lemma} \cite[\S 5.3, 5.4]{lack}
{\sf cb} is a presentation for $(\Span^{\sim}(\FSets),+)$ and {\sf scfa} is a presentation for $(\Csp^\sim(\FSets),+)$.

\end{lemma}

There is an equivalent way of thinking of $(\Span^{\sim}(\FSets),+)$, which we will also use throughout this thesis:

\begin{lemma}
{\sf cb} is a presentation for $\Mat_\N$ under the direct sum.
\end{lemma}


Because $\N$ is the initial commutative semiring ring:
\begin{lemma}
Given any semiring $S$, $\Mat_S$ is presented by the prop $\ch_S$ given by $\ch$ as well as generators $r$ for each $r \in S$ modulo the equations of the ring $S$:

TODO
\end{lemma}
This way of looking at the prop for the free bicommutative bialgebra, naturally leads to a concrete discription of the free bicommutative hopf algebra.

\begin{definition}
A  {\bf bicommutative  Hopf algebra} is a bicommutative bialgebra with an antipode map $s:1\to1$ satisfying the following equation:

$$
\begin{tikzpicture}
	\begin{pgfonlayer}{nodelayer}
		\node [style=none] (0) at (2.5, 5) {};
		\node [style=X] (1) at (2.5, 4.25) {};
		\node [style=Z] (2) at (2.5, 2.75) {};
		\node [style=none] (3) at (2.5, 2) {};
		\node [style=map] (4) at (2, 3.5) {$s$};
	\end{pgfonlayer}
	\begin{pgfonlayer}{edgelayer}
		\draw [in=150, out=-90] (4) to (2);
		\draw [bend right=60, looseness=1.25] (2) to (1);
		\draw [in=90, out=-150] (1) to (4);
		\draw (2) to (3.center);
		\draw (1) to (0.center);
	\end{pgfonlayer}
\end{tikzpicture}
=
\begin{tikzpicture}
	\begin{pgfonlayer}{nodelayer}
		\node [style=none] (0) at (2.5, 2) {};
		\node [style=Z] (1) at (2.5, 2.75) {};
		\node [style=X] (2) at (2.5, 4.25) {};
		\node [style=none] (3) at (2.5, 5) {};
	\end{pgfonlayer}
	\begin{pgfonlayer}{edgelayer}
		\draw (2) to (3.center);
		\draw (1) to (0.center);
	\end{pgfonlayer}
\end{tikzpicture}
%=
%\begin{tikzpicture}
%	\begin{pgfonlayer}{nodelayer}
%		\node [style=none] (0) at (2, 5) {};
%		\node [style=X] (1) at (2, 4.25) {};
%		\node [style=Z] (2) at (2, 2.75) {};
%		\node [style=none] (3) at (2, 2) {};
%		\node [style=map] (4) at (2.5, 3.5) {$s$};
%	\end{pgfonlayer}
%	\begin{pgfonlayer}{edgelayer}
%		\draw [in=30, out=-90] (4) to (2);
%		\draw [bend left=60, looseness=1.25] (2) to (1);
%		\draw [in=90, out=-30] (1) to (4);
%		\draw (2) to (3.center);
%		\draw (1) to (0.center);
%	\end{pgfonlayer}
%\end{tikzpicture}
$$



Let $\ch$ denote the prop for the free commutative hopf algebra.
\end{definition}

\begin{lemma}
$\ch$ is a presentation for $\Mat_\Z$ under the direct sum.
\end{lemma}

Because $\Z$ is the initial commutative ring:
\begin{lemma}
Given any ring $R$, $\Mat_R$ is presented by the prop $\ch_R$ given by $\ch$ as well as generators $r$ for each $r \in R$ modulo the equations of the ring $R$:

TODO
\end{lemma}

Distributive laws are particularly nice because they allow us to factorize maps into normal forms:
\begin{lemma}
\label{lem:distfact}
TODO: Distributive laws and factorization theorems
\end{lemma}


For example:

\begin{lemma}[Spider theorem]
Given parallel string diagrams generated by the components of a Frobenius algebra, then they are equal if and only if they have the same connected components.  A connected component with $n$ inputs and $m$ outputs has the normal form where the $n$ inputs are left associated, and plugged into the left coassociated $m$ outputs.


Graphically, the connected components are normalized to the following shape which we contract using the spider notation:

$$
\begin{tikzpicture}
	\begin{pgfonlayer}{nodelayer}
		\node [style=Z] (0) at (1.25, 3) {};
		\node [style=Z] (1) at (0.5, 4) {};
		\node [style=Z] (2) at (1.25, 2.25) {};
		\node [style=Z] (3) at (0.5, 1.25) {};
		\node [style=none] (4) at (1.5, 4) {};
		\node [style=none] (5) at (1.5, 1.25) {};
		\node [style=none] (6) at (0.25, 0.5) {};
		\node [style=none] (7) at (1.5, 4.75) {};
		\node [style=none] (8) at (1.5, 0.5) {};
		\node [style=none] (9) at (0.75, 4.75) {};
		\node [style=none] (10) at (0.25, 4.75) {};
		\node [style=none] (11) at (0.75, 0.5) {};
		\node [style=none] (12) at (1, 3.25) {};
		\node [style=none] (13) at (0.5, 3.75) {};
		\node [style=none] (14) at (0.5, 1.5) {};
		\node [style=none] (15) at (1, 2) {};
		\node [style=none] (16) at (0.75, 3.5) {$\ddots$};
		\node [style=none] (17) at (0.75, 1.75) {$\reflectbox{$\ddots$}$};
		\node [style=none] (18) at (1.2, 0.5) {$\cdots$};
		\node [style=none] (19) at (1.2, 4.75) {$\cdots$};
	\end{pgfonlayer}
	\begin{pgfonlayer}{edgelayer}
		\draw (7.center) to (4.center);
		\draw [in=105, out=-90] (10.center) to (1);
		\draw [in=60, out=-90, looseness=0.75] (4.center) to (0);
		\draw [in=-90, out=75] (1) to (9.center);
		\draw [in=300, out=90] (5.center) to (2);
		\draw [in=90, out=-120] (3) to (6.center);
		\draw [in=90, out=-60] (3) to (11.center);
		\draw (8.center) to (5.center);
		\draw (0) to (2);
		\draw (3) to (14.center);
		\draw (15.center) to (2);
		\draw (13.center) to (1);
		\draw (0) to (12.center);
	\end{pgfonlayer}
\end{tikzpicture}
=:
\begin{tikzpicture}
	\begin{pgfonlayer}{nodelayer}
		\node [style=none] (0) at (1.5, 1.75) {};
		\node [style=none] (1) at (2.75, 1.75) {};
		\node [style=none] (2) at (2, 1.75) {};
		\node [style=none] (3) at (2.45, 1.75) {$\cdots$};
		\node [style=none] (4) at (2.75, 3.25) {};
		\node [style=none] (5) at (2, 3.25) {};
		\node [style=none] (6) at (1.5, 3.25) {};
		\node [style=none] (7) at (2.45, 3.25) {$\cdots$};
		\node [style=Z] (8) at (2, 2.5) {};
	\end{pgfonlayer}
	\begin{pgfonlayer}{edgelayer}
		\draw [in=-90, out=45] (8) to (4.center);
		\draw (8) to (5.center);
		\draw [in=135, out=-90] (6.center) to (8);
		\draw [in=90, out=-150] (8) to (0.center);
		\draw (2.center) to (8);
		\draw [in=90, out=-30] (8) to (1.center);
	\end{pgfonlayer}
\end{tikzpicture}
$$

\end{lemma}


%
%\subsection{Enriched category theory}
%In this chapter, we develop the theory of enriched profunctors.  Recall in Definition \ref{def:internalprof}, in order to define distributive laws of props, we briefly described internal profunctors.  This imposes size restrictions which are sometimes undesirable. The cousin of internal category theory is enriched category theory:  in the $\mathcal V$-enriched setting, there is only the requirement that between two objects there is a $\mathcal V$-category.  
%
%
%The following data gives enough structure to develop the theory of enriched categories: 
%\begin{definition}
%A {\bf Benabou cosmos } is a complete, cocomplete symmetric monoidal closed category.
%\end{definition}
%
%
%
%\begin{definition}
%Given a Benabou cosmos ${\mathcal V}$,  a $\mathcal V$-{\bf profunctor} $\X \proarrow \Y$ is a functor  $\X^\op \times \Y \to \mathcal V$.
%\end{definition}
%
%
%\begin{definition}
%Given an endo $\mathcal V$-{\bf profunctor} $P:\X \proarrow \X$, the  {\bf coend} $\int^{X} P(X,X) $ is given by the coequalizer:
%
%$
%  \xymatrix{
%      \coprod_{X\to X'} P(X',X)  \ar@<-0.5ex>[r]\ar@<0.5ex>[r] &
%      \coprod_{X \in \X_0} P(X,X) \ar[r] &
%  	\int^{X} P(X,X) \\
%  }
%$
%
%\end{definition}
%
%
%
%\begin{lemma}
%Monoidal bicategory
%\end{lemma}
%
%
%\begin{definition}
%The monoidal bicategory $\Prof$ has:
%
%%embeddings and adjoints
%\end{definition}
%
%
%
%\begin{definition}
%The monoidal bicategory $\Prof^*$ has:
%
%
%\end{definition}
%
%
%\begin{theorem}
%Quasitrictification theorem for monoidal 2-category
%\end{theorem}
%
%\begin{corollary}
%Graphical calculus for quasitrict monoidal 2-category
%\end{corollary}
%
%
%\begin{lemma}
%Graphical calculus for pointed profunctors
%
%%
%%monoidal functors
%\end{lemma}



\subsection{Cartesian bicategories}


\label{sec:rest}

Cartesian bicategories capture the behaviour of $\Rel$ inherited from the cartesian structure of $\Set$.  The induced frobenius algebra structure can be interpreted as a ``possibilistic copying.'' However, there are classes of categories in between cartesian categories and cartesian bicategories which capture partially invertible (like $\Pinj$) and partial notions of copying (like $\Par$).  In this section, we review these notions and give examples which will serve to motivate their usage in quantum computing later in this thesis.


%Restriction and inverse categories provide a categorical semantics for partial computing and reversible computing, respectively.  We review how weakened products can be constructed in both settings; relating one to the other.

\begin{definition}\cite[\S 2.1.1]{cockett}
A {\bf restriction category} is a category along with a restriction operator:

\hfil
$
(A \xrightarrow{f} B )\mapsto (A \xrightarrow{\bar f} A)
$\\
such that:


\begin{center}
%\begin{mdframed}
\begin{multicols}{4}
\begin{enumerate}[label={\bf [R.\arabic*]}, ref={\bf [R.\arabic*]}]
\item $\bar f f  = f$
\label{R.1}
\item $\bar f \bar g = \bar g \bar f$
\label{R.2}
\item $\bar f \bar g = \bar{\bar f g}$
\label{R.3}
\item $f \bar g = \bar{fg} f$
\label{R.4}
\end{enumerate}
\end{multicols}
%\end{mdframed}
\end{center}


Maps of the form $\bar f$ are called restriction idempotents.
The canonical example of a restriction category is $\Par$, sets and partial maps.  The restriction in this case, just restricts partial functions to their domain of definition.


Restriction categories are poset enriched where $f \leq g \iff \bar f g = f$.


A map $f$ in a restriction category is called a {\bf partial isomorphism}, in case there exists a map $g$ called the partial inverse of $f$ so that $fg=\bar f$ and $gf = \bar g$.  Similarly, a map $f$ in a restriction category is {\bf total} if $\bar f =1$.  Denote the subcategories of partial isomorphisms and total maps of a restriction category $\X$, respectively by $\ParIso(\X)$ and $\Total(\X)$.



%A {\bf split restriction category} is a restriction category in which all restriction idempotents split.
\end{definition}



\begin{example} \cite[p. 101]{pcat} \cite[\S 5]{restiii}
A {\bf counital copy category} (or a p-category with a one element object) is a monoidal category with a family of commutative comonoids on every object compatible with the monoidal structure, with a natural comultiplication.  This gives a restriction via copying and then discarding:
$$
\begin{tikzpicture}
	\begin{pgfonlayer}{nodelayer}
		\node [style=none] (0) at (0.75, -2.5) {};
		\node [style=none] (1) at (0.75, -0.5) {};
		\node [style=map] (2) at (0.75, -1.5) {$\bar f$};
	\end{pgfonlayer}
	\begin{pgfonlayer}{edgelayer}
		\draw [style=simple] (0.center) to (2);
		\draw [style=simple] (2) to (1.center);
	\end{pgfonlayer}
\end{tikzpicture}
:=
\begin{tikzpicture}
	\begin{pgfonlayer}{nodelayer}
		\node [style=map] (0) at (0, 2.5) {$f$};
		\node [style=X] (1) at (0, 3.5) {};
		\node [style=X] (2) at (0.5, 1.5) {};
		\node [style=none] (3) at (1, 3.5) {};
		\node [style=none] (4) at (0.5, 0.5) {};
	\end{pgfonlayer}
	\begin{pgfonlayer}{edgelayer}
		\draw [style=simple] (1) to (0);
		\draw [style=simple, in=117, out=-90] (0) to (2);
		\draw [style=simple] (2) to (4.center);
		\draw [style=simple, in=-90, out=60] (2) to (3.center);
	\end{pgfonlayer}
\end{tikzpicture}
$$
\end{example}


\begin{definition}\cite[\S 3.1]{cockett}
A {\bf stable system of monics} $\M$ of $\X$ is a collection of monics in $\X$ containing all isomorphisms; where for any cospan $ X\xrightarrow{f} Z \xleftarrowtail{m} Y$  in $\X$, where $m'$ is in $\M$, the following pullback exists:

%\hfil$
%\xymatrixrowsep{.005in}
%\xymatrixcolsep{.13in}
%  \xymatrix{
%    W \ar[r]^{f'} \ar@{>->}[d]_{m'} & Y  \ar@{>->}[d]^m \\
%    X \ar[r]_{f} & Z
%  }
%$\\

\hfil$
\xymatrixrowsep{.005in}
\xymatrixcolsep{.13in}
  \xymatrix{
  	& W \ar@{>->}[dl]_{m'} \ar[dr]^{f'}\\
  	X \ar[dr]_f &  & Y \ar@{>->}[dl]^m\\
  	& Z
  }
$

Where $m'$ is in $\M$.

\end{definition}

Stable systems of monics allow one to represent the domains of definition of a partial functions as a subobjects:

\begin{definition}\cite[\S 3.1]{cockett}
Given a stable system of monics $\M$ in a category $\X$, the {\bf partial map category} $\Par(\X,\M)$ is given by the same objects as in $\X$ where morphisms $X\to Y$, given by isomorphism classes of spans $X\xleftarrowtail{m} Z \xrightarrow{f} Y$ where $f$ is a map in $\X$ and $m$ is a map in $\M$.  Composition is given by pullback and the identity is given by the trivial span.


Partial map categories have a restriction structure given by:  $(X\xleftarrowtail{m} Z \xrightarrow{f} Y) \mapsto (X\xleftarrowtail{m} Z \xrightarrowtail{m} X)$.  Moreover, a partial isomorphism is a span $X\xleftarrowtail{e} Z \xrightarrowtail{m} Y$ where $e,m \in \M$; the partial inverse given by  $Y\xleftarrowtail{m} Z \xrightarrowtail{e} X$.
\end{definition}


$\Par$ is equivalently the partial map category $\Par(\Sets,\M)$ where $\M$ is all monics in $\Sets$.




%\begin{lemma} \cite[Prop. 3.1]{cockett}
%Partial map categories are split restriction categories.
%\end{lemma}


%If restriction idempotents split then X is a cartesian restriction category if and only if Tot(X) is a cartesian category



%If  $\X$ is finitely complete, then $\Span^\sim(\X)$ exists, and thus, there is a faithful functor $\Par(\X,\M)\to \Span^\sim(\X)$.




\begin{definition}\cite[\S 2.3.2]{cockett}
An {\bf inverse category} is a restriction category in which all maps are partial isomorphisms.
\end{definition}

\begin{example}
The subcategory of partial isomorphisms of $\Par$ is  $\Pinj$.
\end{example}

Inverse categories can be presented with a dagger functor taking maps to their partial inverses.

\begin{definition}
\label{def:dag}
A {\bf dagger category} is a category $\X$ equipped with an identity-on-objects involution $(\_)^\dag:\X^\op\to\X$ called the dagger.

An isomorphism $f$ in a dagger category is {\bf unitary} when $f^{-1}=f^\dag$.
\end{definition}


\begin{theorem}\cite[Thm. 2.20]{cockett}
A restriction category $\X$ is an inverse category if and only if there is a dagger functor $(\_)^\circ:\X^\op\to\X$ such that for all $X\xleftarrow{f} Z \xrightarrow{g} Y$:
\begin{center}
\begin{tabular}{cc}
 $f f^\circ f = f$ & 
 $f f ^\circ gg^\circ = gg^\circ f f ^\circ $
\end{tabular}
\end{center}
\end{theorem}

The unitary maps in an inverse category are the total maps.



In the case of restriction categories, one must weaken the notion of the product to lax products using the partial order enrichment:


\begin{definition}\cite{restiii}
A restriction category has {\bf binary restriction products}, when for all objects  $X,Y$, there exists an object $X\times Y$ and total maps $X \xleftarrow{\pi_0}  X\times Y \xrightarrow{\pi_1} Y$, so that for all objects $Z$ and all maps $X \xleftarrow{f} Z \xrightarrow{g} Y$, there exists a unique $Z\xrightarrow{\langle f,g \rangle} X\times Y$ making the diagram commute:
\hfil
$
\xymatrixrowsep{0.2cm}
\xymatrixcolsep{0.4cm}
\xymatrix{
&& Z\ar@{..>}[dd]|-{\langle f, g\rangle} \ar@/_/[ddll]_f \ar@/^/[ddrr]^g &&\\
& \ar@{}[dr]|-{\geq} && \ar@{}[dl] |-{\leq} &\\
X &&  X\times Y \ar[rr]_{\pi_1} \ar[ll]^{\pi_0}  && Y
}
$

so that $\bar{\langle f, g\rangle \pi_0} f = \langle f, g\rangle \pi_0$ and $\bar{\langle f, g\rangle \pi_1} g = \langle f, g\rangle \pi_1$;
where additionally $\bar{\langle f, g\rangle} =  \bar f \bar g$.

%%DRAW DIAGRAM
%\begin{center}
%\begin{tabular}{ccc}
%  $\langle f, g\rangle \pi_0 \leq f$ &
%  $\langle f, g\rangle \pi_1 \leq g$ &
%  $\bar{\langle f, g\rangle} =  \bar f \bar g$
%\end{tabular}
%\end{center}

A restriction category has a {\bf restriction terminal object} $\top$ when for all objects $X$, there exists a unique total map $!_X:X\to\top$ such that $f !_Y = \bar f !_X$.

A restriction category with a restriction terminal object and binary restriction products is a {\bf Cartesian restriction category}.


An object $A$ in a restriction category with restriction products is {\bf discrete} when the diagonal map $\Delta_X:=\langle 1_X, 1_X\rangle$ is a partial isomorphism. A restriction category is discrete when all objects are discrete. 
\end{definition}




\begin{theorem}\cite[Thm. 5.2]{restiii}
Counital copy categories are in bijection with Cartesian restriction categories.
\end{theorem}


\begin{proposition} 
\label{prop:cartesian}
If $\X$ is a  Cartesian restriction category, then $\Total(\X)$ is Cartesian.
\end{proposition}



$\Par$ is a canonical example of a Cartesian restriction category; the restriction product is given by the Cartesian product on underlying sets and the terminal object is  the singleton set. In fact it is also discrete, because the converse of the diagonal relation is a partial function.




The weakened notion of products in restriction categories is the right generalization of products for inverse categories because it does not impose enough equations governing the interaction between the diagonal map and its partial inverse.  In order to mix the dagger structure of an inverse category with symmetric monoidal structure, we first need the following definition:


\begin{definition}
A {\bf dagger symmetric monoidal  category}a symmetric monoidal dagger category where all the coherence isomorphisms of the symmetric monoidal structure are unitary. And for all morphisms $f,g$: $(f \otimes g)^\dag = f^\dag \otimes g^\dag$.
\end{definition}


\begin{definition}\cite[Def. 4.3.1]{giles}
A symmetric monoidal inverse category $\X$ is a {\bf discrete inverse category} when it is a dagger symmetric monoidal  category, 
equipped with a commutative semigroup and cocommutative cosemigroup on every object  compatible with the tensor product:

$$
\begin{tikzpicture}
	\begin{pgfonlayer}{nodelayer}
		\node [style=none] (0) at (0, 2.5) {};
		\node [style=none] (1) at (1, 2.5) {};
		\node [style=X] (2) at (0.5, 1.5) {};
		\node [style=none] (3) at (0.5, 0.5) {};
	\end{pgfonlayer}
	\begin{pgfonlayer}{edgelayer}
		\draw [style=simple] (3.center) to (2);
		\draw [style=simple, in=-90, out=117] (2) to (0.center);
		\draw [style=simple, in=63, out=-90] (1.center) to (2);
	\end{pgfonlayer}
\end{tikzpicture}
=
\begin{tikzpicture}
	\begin{pgfonlayer}{nodelayer}
		\node [style=X] (0) at (0, 2.5) {};
		\node [style=X] (1) at (1, 2.5) {};
		\node [style=none] (2) at (0.5, 1.5) {};
		\node [style=none] (3) at (0.5, 0.5) {};
		\node [style=none] (4) at (0, 3.5) {};
		\node [style=none] (5) at (1, 3.5) {};
		\node [style=none] (6) at (0, 4.5) {};
		\node [style=none] (7) at (1, 4.5) {};
		\node [style=otimes] (8) at (0.5, 1.5) {};
		\node [style=otimes] (9) at (1, 3.5) {};
		\node [style=otimes] (10) at (0, 3.5) {};
	\end{pgfonlayer}
	\begin{pgfonlayer}{edgelayer}
		\draw [style=simple] (3.center) to (2.center);
		\draw [style=simple, in=-90, out=135] (2.center) to (0);
		\draw [style=simple] (0) to (5.center);
		\draw [style=simple, in=120, out=-120, looseness=1.25] (4.center) to (0);
		\draw [style=simple, in=-60, out=60, looseness=1.25] (1) to (5.center);
		\draw [style=simple] (1) to (4.center);
		\draw [style=simple, in=45, out=-90] (1) to (2.center);
		\draw [style=simple] (4.center) to (6.center);
		\draw [style=simple] (5.center) to (7.center);
	\end{pgfonlayer}
\end{tikzpicture}
\hspace*{1cm}
\begin{tikzpicture}
	\begin{pgfonlayer}{nodelayer}
		\node [style=X] (0) at (0, 1.5) {};
		\node [style=none] (1) at (-0.5, 2.5) {};
		\node [style=none] (2) at (0.5, 2.5) {};
		\node [style=none] (3) at (0, 0.5) {};
	\end{pgfonlayer}
	\begin{pgfonlayer}{edgelayer}
		\draw [style=dashed] (3.center) to (0);
		\draw [style=dashed, in=-90, out=117] (0) to (1.center);
		\draw [style=dashed, in=63, out=-90] (2.center) to (0);
	\end{pgfonlayer}
\end{tikzpicture}
=
\begin{tikzpicture}
	\begin{pgfonlayer}{nodelayer}
		\node [style=none] (0) at (0, 1.5) {};
		\node [style=none] (1) at (-0.5, 2.5) {};
		\node [style=none] (2) at (0.5, 2.5) {};
		\node [style=none] (3) at (0, 0.5) {};
		\node [style=otimes] (4) at (0, 1.5) {};
	\end{pgfonlayer}
	\begin{pgfonlayer}{edgelayer}
		\draw [style=dashed] (3.center) to (0.center);
		\draw [style=dashed, in=-90, out=117] (0.center) to (1.center);
		\draw [style=dashed, in=63, out=-90] (2.center) to (0.center);
	\end{pgfonlayer}
\end{tikzpicture}
$$

Where the semigroup and cosemigroup are daggers of each other and interact interact to form a (co)unitless special commutative Frobenius algebra.

GIVE STRING DIAGRAMS


Where the diagonal is natural so that for arbitrary $f$:
$$
\begin{tikzpicture}
	\begin{pgfonlayer}{nodelayer}
		\node [style=Z] (0) at (0, 12) {};
		\node [style=map] (1) at (0, 11.25) {$f$};
		\node [style=none] (2) at (-0.5, 12.75) {};
		\node [style=none] (3) at (0.5, 12.75) {};
		\node [style=none] (4) at (0, 10.5) {};
	\end{pgfonlayer}
	\begin{pgfonlayer}{edgelayer}
		\draw (4.center) to (1);
		\draw (1) to (0);
		\draw [in=-90, out=150] (0) to (2.center);
		\draw [in=-90, out=30] (0) to (3.center);
	\end{pgfonlayer}
\end{tikzpicture}
=
\begin{tikzpicture}
	\begin{pgfonlayer}{nodelayer}
		\node [style=Z] (5) at (2, 11.25) {};
		\node [style=none] (7) at (1.5, 12) {};
		\node [style=none] (8) at (2.5, 12) {};
		\node [style=none] (9) at (2, 10.5) {};
		\node [style=map] (10) at (2.5, 12) {$f$};
		\node [style=map] (11) at (1.5, 12) {$f$};
		\node [style=none] (12) at (1.5, 12.75) {};
		\node [style=none] (13) at (2.5, 12.75) {};
	\end{pgfonlayer}
	\begin{pgfonlayer}{edgelayer}
		\draw [in=-90, out=150] (5) to (7.center);
		\draw [in=-90, out=30] (5) to (8.center);
		\draw (9.center) to (5);
		\draw (11) to (12.center);
		\draw (10) to (13.center);
	\end{pgfonlayer}
\end{tikzpicture}
$$
\end{definition}

In a discrete inverse category, restriction idempotents are strengths for the multiplication and comultiplication so that:
$$
\begin{tikzpicture}
	\begin{pgfonlayer}{nodelayer}
		\node [style=X] (0) at (3, 1.75) {};
		\node [style=map] (1) at (3, 1) {$\bar f$};
		\node [style=none] (2) at (3, 0.5) {};
		\node [style=none] (3) at (2.5, 2.5) {};
		\node [style=none] (4) at (3.5, 2.5) {};
	\end{pgfonlayer}
	\begin{pgfonlayer}{edgelayer}
		\draw [style=simple, in=63, out=-90] (4.center) to (0);
		\draw [style=simple, in=-90, out=117] (0) to (3.center);
		\draw [style=simple] (1) to (0);
		\draw [style=simple] (1) to (2.center);
	\end{pgfonlayer}
\end{tikzpicture}
=
\begin{tikzpicture}
	\begin{pgfonlayer}{nodelayer}
		\node [style=X] (0) at (3, 2) {};
		\node [style=none] (1) at (3, 1.5) {};
		\node [style=none] (2) at (2.5, 3) {};
		\node [style=none] (3) at (3.5, 3) {};
		\node [style=map] (4) at (2.5, 3) {$\bar f$};
		\node [style=none] (5) at (3.5, 3.5) {};
		\node [style=none] (6) at (2.5, 3.5) {};
	\end{pgfonlayer}
	\begin{pgfonlayer}{edgelayer}
		\draw [style=simple, in=63, out=-90] (3.center) to (0);
		\draw [style=simple, in=-90, out=117] (0) to (2.center);
		\draw [style=simple] (6.center) to (2.center);
		\draw [style=simple] (5.center) to (3.center);
		\draw [style=simple] (0) to (1.center);
	\end{pgfonlayer}
\end{tikzpicture}
=
\begin{tikzpicture}
	\begin{pgfonlayer}{nodelayer}
		\node [style=X] (0) at (3, 2) {};
		\node [style=none] (1) at (3, 1.5) {};
		\node [style=none] (2) at (3.5, 3) {};
		\node [style=none] (3) at (2.5, 3) {};
		\node [style=map] (4) at (3.5, 3) {$\bar f$};
		\node [style=none] (5) at (2.5, 3.5) {};
		\node [style=none] (6) at (3.5, 3.5) {};
	\end{pgfonlayer}
	\begin{pgfonlayer}{edgelayer}
		\draw [style=simple, in=117, out=-90] (3.center) to (0);
		\draw [style=simple, in=-90, out=63] (0) to (2.center);
		\draw [style=simple] (6.center) to (2.center);
		\draw [style=simple] (5.center) to (3.center);
		\draw [style=simple] (0) to (1.center);
	\end{pgfonlayer}
\end{tikzpicture}
\hspace*{.6cm}
\begin{tikzpicture}
	\begin{pgfonlayer}{nodelayer}
		\node [style=X] (0) at (3, 3) {};
		\node [style=none] (1) at (3, 3.5) {};
		\node [style=none] (2) at (3.5, 2) {};
		\node [style=none] (3) at (2.5, 2) {};
		\node [style=map] (4) at (3.5, 2) {$\bar f$};
		\node [style=none] (5) at (2.5, 1.5) {};
		\node [style=none] (6) at (3.5, 1.5) {};
	\end{pgfonlayer}
	\begin{pgfonlayer}{edgelayer}
		\draw [style=simple, in=-117, out=90] (3.center) to (0);
		\draw [style=simple, in=90, out=-63] (0) to (2.center);
		\draw [style=simple] (6.center) to (2.center);
		\draw [style=simple] (5.center) to (3.center);
		\draw [style=simple] (0) to (1.center);
	\end{pgfonlayer}
\end{tikzpicture}
=
\begin{tikzpicture}
	\begin{pgfonlayer}{nodelayer}
		\node [style=X] (0) at (3, 3) {};
		\node [style=none] (1) at (3, 3.5) {};
		\node [style=none] (2) at (2.5, 2) {};
		\node [style=none] (3) at (3.5, 2) {};
		\node [style=map] (4) at (2.5, 2) {$\bar f$};
		\node [style=none] (5) at (3.5, 1.5) {};
		\node [style=none] (6) at (2.5, 1.5) {};
	\end{pgfonlayer}
	\begin{pgfonlayer}{edgelayer}
		\draw [style=simple, in=-63, out=90] (3.center) to (0);
		\draw [style=simple, in=90, out=-117] (0) to (2.center);
		\draw [style=simple] (6.center) to (2.center);
		\draw [style=simple] (5.center) to (3.center);
		\draw [style=simple] (0) to (1.center);
	\end{pgfonlayer}
\end{tikzpicture}
=
\begin{tikzpicture}
	\begin{pgfonlayer}{nodelayer}
		\node [style=X] (0) at (3, 1.25) {};
		\node [style=map] (1) at (3, 2) {$\bar f$};
		\node [style=none] (2) at (3, 2.5) {};
		\node [style=none] (3) at (2.5, 0.5) {};
		\node [style=none] (4) at (3.5, 0.5) {};
	\end{pgfonlayer}
	\begin{pgfonlayer}{edgelayer}
		\draw [style=simple, in=-63, out=90] (4.center) to (0);
		\draw [style=simple, in=90, out=-117] (0) to (3.center);
		\draw [style=simple] (1) to (0);
		\draw [style=simple] (1) to (2.center);
	\end{pgfonlayer}
\end{tikzpicture}
$$


Discrete inverse categories are the ``right'' notion of weakened products for monoidal inverse categories:

\begin{theorem}\cite[Thm. 5.2.6]{giles}
There is an equivalence of categories between the category of discrete inverse categories and the category of discrete Cartesian categories.
\end{theorem}

To go from  discrete Cartesian restriction categories to discrete inverse categories, one takes the subcategory of partial isomorphisms.
The other direction is less trivial; in particular, this involves adding a restriction terminal object via the following construction which ``adds a history'' to a partial isomorphisms.  We will first introduce the more general copara construction, we will need this machinery later.

\begin{definition}
Given a monoidal category $\X$, the copara construction, $\CoPara(\X)$ is the monoidal category with:

\begin{description}
\item[Objects:] Same as in $\X$.

\item[Maps:]  
\hfil $
\dfrac{ X\xrightarrow{f} Y \otimes S \in \X           }
         { X\xrightarrow{(f,S)} Y \in  \CoPara(\X) }
$

\item[Composition]:  
\hfil $
\dfrac{X\xrightarrow{(f,S)} Y , \hspace*{.5cm} Y\xrightarrow{(g,T)} Z }
         {(f,S);(g;T) := (f;(g\otimes 1_S);\alpha^{-1}_{Z,S,T} ,S\otimes T) } 
$

\hfil Or using proof net notation:
\hspace*{.5cm}
$
\begin{tikzpicture}
	\begin{pgfonlayer}{nodelayer}
		\node [style=map] (0) at (0, 1.5) {$f$};
		\node [style=none] (1) at (-0.5, 2.5) {};
		\node [style=none] (2) at (0.5, 2.5) {};
		\node [style=none] (3) at (0, 0.5) {};
	\end{pgfonlayer}
	\begin{pgfonlayer}{edgelayer}
		\draw [in=117, out=-90] (1.center) to (0);
		\draw [in=-90, out=63] (0) to (2.center);
		\draw (0) to (3.center);
	\end{pgfonlayer}
\end{tikzpicture}
;
\begin{tikzpicture}
	\begin{pgfonlayer}{nodelayer}
		\node [style=map] (0) at (0, 1.5) {$g$};
		\node [style=none] (1) at (-0.5, 2.5) {};
		\node [style=none] (2) at (0.5, 2.5) {};
		\node [style=none] (3) at (0, 0.5) {};
	\end{pgfonlayer}
	\begin{pgfonlayer}{edgelayer}
		\draw [in=117, out=-90] (1.center) to (0);
		\draw [in=-90, out=63] (0) to (2.center);
		\draw (0) to (3.center);
	\end{pgfonlayer}
\end{tikzpicture}
:=
\begin{tikzpicture}
	\begin{pgfonlayer}{nodelayer}
		\node [style=map] (0) at (0, 1.5) {$f$};
		\node [style=none] (1) at (0.5, 2.5) {};
		\node [style=none] (2) at (0, 0.5) {};
		\node [style=map] (3) at (-0.5, 2.5) {$g$};
		\node [style=none] (4) at (-1, 3.5) {};
		\node [style=otimes] (5) at (0, 3.5) {};
		\node [style=none] (6) at (-0.5, 2.5) {};
		\node [style=none] (7) at (-1, 4.5) {};
		\node [style=none] (8) at (0, 4.5) {};
	\end{pgfonlayer}
	\begin{pgfonlayer}{edgelayer}
		\draw [in=-90, out=63] (0) to (1.center);
		\draw (0) to (2.center);
		\draw [in=117, out=-90] (4.center) to (3);
		\draw (3) to (5);
		\draw [in=117, out=-90] (6.center) to (0);
		\draw [in=-63, out=90] (1.center) to (5);
		\draw (5) to (8.center);
		\draw (4.center) to (7.center);
	\end{pgfonlayer}
\end{tikzpicture}
$

\item[Identity:]
$
\dfrac{ 1_X \in  \CoPara(\X)}{(u^R_A)^{-1} \in \X}
$

\item[Tensor product:]\

\hspace*{-2cm}
$
\dfrac{X\xrightarrow{(f,S)} Y, \hspace*{.5cm} Z\xrightarrow{(g,T)} W}
{(f,S)\otimes (g;T) := ((f\otimes g);(1_{X\otimes S} \otimes c_{W,T});\alpha_{X,S,T\otimes W};(1_X\otimes \alpha_{S,T,W}^{-1};(c_{S,T}));\alpha_{Y,W,S\otimes T}^{-1} ,S\otimes T)} 
$

\hfil Or in proof net notation:
$
\begin{tikzpicture}
	\begin{pgfonlayer}{nodelayer}
		\node [style=map] (0) at (0, 1.5) {$f$};
		\node [style=none] (1) at (-0.5, 2.5) {};
		\node [style=none] (2) at (0.5, 2.5) {};
		\node [style=none] (3) at (0, 0.5) {};
	\end{pgfonlayer}
	\begin{pgfonlayer}{edgelayer}
		\draw [in=117, out=-90] (1.center) to (0);
		\draw [in=-90, out=63] (0) to (2.center);
		\draw (0) to (3.center);
	\end{pgfonlayer}
\end{tikzpicture}
\otimes
\begin{tikzpicture}
	\begin{pgfonlayer}{nodelayer}
		\node [style=map] (0) at (0, 1.5) {$g$};
		\node [style=none] (1) at (-0.5, 2.5) {};
		\node [style=none] (2) at (0.5, 2.5) {};
		\node [style=none] (3) at (0, 0.5) {};
	\end{pgfonlayer}
	\begin{pgfonlayer}{edgelayer}
		\draw [in=117, out=-90] (1.center) to (0);
		\draw [in=-90, out=63] (0) to (2.center);
		\draw (0) to (3.center);
	\end{pgfonlayer}
\end{tikzpicture}
:=
\begin{tikzpicture}
	\begin{pgfonlayer}{nodelayer}
		\node [style=map] (9) at (3.5, 1.5) {$f$};
		\node [style=map] (13) at (4.5, 1.5) {$g$};
		\node [style=otimes] (17) at (4.5, 2.5) {};
		\node [style=otimes] (18) at (3.5, 2.5) {};
		\node [style=otimes] (190) at (4, 0.75) {};
		\node  (19) at (4, 0.75) {};
		\node [style=none] (20) at (3.5, 3) {};
		\node [style=none] (21) at (4.5, 3) {};
		\node [style=none] (22) at (4, 0.25) {};
	\end{pgfonlayer}
	\begin{pgfonlayer}{edgelayer}
		\draw (13) to (18);
		\draw [bend right] (18) to (9);
		\draw (9) to (17);
		\draw [bend left] (17) to (13);
		\draw [in=45, out=-90] (13) to (19);
		\draw [in=-90, out=135] (19) to (9);
		\draw (21.center) to (17);
		\draw (22.center) to (19);
		\draw (18) to (20.center);
	\end{pgfonlayer}
\end{tikzpicture}
$

\end{description}
\end{definition}

The coherence data for the monoidal structure is inhereted in a straightforward way from $\X$. Moreover, if $\X$ is symmetric monoidal, then it is easy to see how ${\CoPara}(\X)$ is as well.

\begin{definition}\cite[Def. 5.1.1]{giles}
Given a discrete inverse category $\X$,  its {\bf Cartesian completion} $\tilde \X$ is the quotient of ${\CoPara}(\X)$ by either of the following equivalent congruence relations:
$$
\begin{tikzpicture}
	\begin{pgfonlayer}{nodelayer}
		\node [style=map] (0) at (0, 1.5) {$f$};
		\node [style=none] (1) at (0, 0.5) {};
		\node [style=map] (2) at (0, 3) {$f^\circ$};
		\node [style=map] (3) at (0, 4) {$g$};
		\node [style=X] (4) at (-0.5, 2.25) {};
		\node [style=X] (5) at (-0.5, 5) {};
		\node [style=none] (6) at (-0.5, 6) {};
		\node [style=none] (7) at (0.25, 6) {};
	\end{pgfonlayer}
	\begin{pgfonlayer}{edgelayer}
		\draw (0) to (1.center);
		\draw [in=75, out=-90] (7.center) to (3);
		\draw (6.center) to (5);
		\draw [in=120, out=-120] (5) to (4);
		\draw (4) to (2);
		\draw [in=60, out=-60, looseness=1.25] (2) to (0);
		\draw (0) to (4);
		\draw (3) to (2);
		\draw (3) to (5);
	\end{pgfonlayer}
\end{tikzpicture}
=
\begin{tikzpicture}
	\begin{pgfonlayer}{nodelayer}
		\node [style=map] (0) at (0, 1.5) {$g$};
		\node [style=none] (1) at (-0.5, 2.5) {};
		\node [style=none] (2) at (0.5, 2.5) {};
		\node [style=none] (3) at (0, 0.5) {};
	\end{pgfonlayer}
	\begin{pgfonlayer}{edgelayer}
		\draw [in=117, out=-90] (1.center) to (0);
		\draw [in=-90, out=63] (0) to (2.center);
		\draw (0) to (3.center);
	\end{pgfonlayer}
\end{tikzpicture}
\hspace*{.3cm}
or
\hspace*{.3cm}
\begin{tikzpicture}
	\begin{pgfonlayer}{nodelayer}
		\node [style=map] (0) at (0, 1.5) {$g$};
		\node [style=none] (1) at (0, 0.5) {};
		\node [style=map] (2) at (0, 3) {$g^\circ$};
		\node [style=map] (3) at (0, 4) {$f$};
		\node [style=X] (4) at (-0.5, 2.25) {};
		\node [style=X] (5) at (-0.5, 5) {};
		\node [style=none] (6) at (-0.5, 6) {};
		\node [style=none] (7) at (0.25, 6) {};
	\end{pgfonlayer}
	\begin{pgfonlayer}{edgelayer}
		\draw (0) to (1.center);
		\draw [in=75, out=-90] (7.center) to (3);
		\draw (6.center) to (5);
		\draw [in=120, out=-120] (5) to (4);
		\draw (4) to (2);
		\draw [in=60, out=-60, looseness=1.25] (2) to (0);
		\draw (0) to (4);
		\draw (3) to (2);
		\draw (3) to (5);
	\end{pgfonlayer}
\end{tikzpicture}
=
\begin{tikzpicture}
	\begin{pgfonlayer}{nodelayer}
		\node [style=map] (0) at (0, 1.5) {$f$};
		\node [style=none] (1) at (-0.5, 2.5) {};
		\node [style=none] (2) at (0.5, 2.5) {};
		\node [style=none] (3) at (0, 0.5) {};
	\end{pgfonlayer}
	\begin{pgfonlayer}{edgelayer}
		\draw [in=117, out=-90] (1.center) to (0);
		\draw [in=-90, out=63] (0) to (2.center);
		\draw (0) to (3.center);
	\end{pgfonlayer}
\end{tikzpicture}
$$

$\tilde \X$ is regarded as a discrete cartesian restriction category with:
\begin{description}
\item[Restriction: ] %Restriction is given by the underlying restriction of $\X$, so that:
\hfil
$
\bar{\left(
\begin{tikzpicture}
	\begin{pgfonlayer}{nodelayer}
		\node [style=map] (0) at (0, 1.5) {$f$};
		\node [style=none] (1) at (0, 0.5) {};
		\node [style=none] (2) at (-0.5, 2.5) {};
		\node [style=none] (3) at (0.5, 2.5) {};
	\end{pgfonlayer}
	\begin{pgfonlayer}{edgelayer}
		\draw [style=simple] (1.center) to (0);
		\draw [style=simple, in=117, out=-90] (2.center) to (0);
		\draw [style=simple, in=63, out=-90] (3.center) to (0);
	\end{pgfonlayer}
\end{tikzpicture}
\right)}
:=
\begin{tikzpicture}
	\begin{pgfonlayer}{nodelayer}
		\node [style=map] (0) at (0.5, 1.5) {$\bar f$};
		\node [style=none] (1) at (0.5, 0.5) {};
		\node [style=none] (2) at (0.5, 2.5) {};
		\node [style=none] (4) at (1, 2.5) {};
		\node [style=map] (6) at (1, 2) {$I$};
	\end{pgfonlayer}
	\begin{pgfonlayer}{edgelayer}
		\draw [style=simple] (1.center) to (0);
		\draw [style=simple] (2.center) to (0);
		\draw (6) to (4.center);
	\end{pgfonlayer}
\end{tikzpicture}
$

\item[Restriction product:]
\hfil
$
\langle f,g \rangle:=
\begin{tikzpicture}
	\begin{pgfonlayer}{nodelayer}
		\node [style=map] (0) at (-0.25, 2.5) {$f$};
		\node [style=none] (1) at (-0.25, 3.5) {};
		\node [style=none] (2) at (0.75, 3.5) {};
		\node [style=none] (3) at (-0.25, 3.5) {};
		\node [style=map] (4) at (0.75, 2.5) {$g$};
		\node [style=none] (5) at (0.75, 3.5) {};
		\node [style=otimes] (6) at (0.75, 3.5) {};
		\node [style=otimes] (7) at (-0.25, 3.5) {};
		\node [style=X] (8) at (0.25, 1.5) {};
		\node [style=none] (9) at (-0.25, 4.5) {};
		\node [style=none] (10) at (0.75, 4.5) {};
		\node [style=none] (11) at (0.25, 0.5) {};
	\end{pgfonlayer}
	\begin{pgfonlayer}{edgelayer}
		\draw [style=simple, in=117, out=-120] (1.center) to (0);
		\draw [style=simple] (2.center) to (0);
		\draw [style=simple] (3.center) to (4);
		\draw [style=simple, in=63, out=-60] (5.center) to (4);
		\draw [style=simple, in=56, out=-90] (4) to (8);
		\draw [style=simple, in=-90, out=124] (8) to (0);
		\draw [style=simple] (9.center) to (1.center);
		\draw [style=simple] (2.center) to (10.center);
		\draw [style=simple] (8) to (11.center);
	\end{pgfonlayer}
\end{tikzpicture}
$

\item[Restriction terminal map:]
\hfil
$
\begin{tikzpicture}
	\begin{pgfonlayer}{nodelayer}
		\node [style=none] (0) at (1.25, 0.5) {};
		\node [style=none] (1) at (1.25, 2) {};
		\node [style=none] (2) at (0.5, 2) {};
		\node [style=map] (3) at (0.5, 1.25) {$I$};
	\end{pgfonlayer}
	\begin{pgfonlayer}{edgelayer}
		\draw [style=simple] (0.center) to (1.center);
		\draw (3) to (2.center);
	\end{pgfonlayer}
\end{tikzpicture}
$
\end{description}

\end{definition}


\begin{example}\cite[Ex. 5.3.3]{giles}
$\tilde \Pinj$ is $\Par$.
\end{example}
\begin{proof}
For a partial function $f:X\to Y$, $\{(x,(y,x)) | (x,y) \in f \}/\sim$ is a partial isomorphism.
\end{proof}



\begin{lemma}
\label{lemma:xtildefaithful}
The canonical functor $\iota:\X\to \tilde \X$ is faithful.
\end{lemma}

\begin{proof}
Suppose that $\iota(f)\sim\iota(g)$, Then:

\begin{align*}
\begin{tikzpicture}
	\begin{pgfonlayer}{nodelayer}
		\node [style=map] (0) at (-0.5, 1.5) {$g$};
		\node [style=none] (1) at (-0.5, 2.5) {};
		\node [style=none] (2) at (-0.5, 0.5) {};
	\end{pgfonlayer}
	\begin{pgfonlayer}{edgelayer}
		\draw (1.center) to (0);
		\draw (0) to (2.center);
	\end{pgfonlayer}
\end{tikzpicture}
=
\begin{tikzpicture}
	\begin{pgfonlayer}{nodelayer}
		\node [style=map] (0) at (-0.5, 1.25) {$f$};
		\node [style=none] (1) at (-0.5, 0.5) {};
		\node [style=map] (2) at (-0.25, 3) {$f^\circ$};
		\node [style=map] (3) at (-0.25, 3.75) {$g$};
		\node [style=X] (4) at (-0.5, 2) {};
		\node [style=X] (5) at (-0.5, 4.75) {};
		\node [style=none] (6) at (-0.5, 5.75) {};
	\end{pgfonlayer}
	\begin{pgfonlayer}{edgelayer}
		\draw (0) to (1.center);
		\draw (6.center) to (5);
		\draw [in=120, out=-120, looseness=0.75] (5) to (4);
		\draw [in=-90, out=56] (4) to (2);
		\draw (0) to (4);
		\draw (3) to (2);
		\draw [in=-63, out=90] (3) to (5);
	\end{pgfonlayer}
\end{tikzpicture}
=
\begin{tikzpicture}
	\begin{pgfonlayer}{nodelayer}
		\node [style=none] (0) at (-0.5, 0.5) {};
		\node [style=map] (1) at (-0.25, 4.5) {$g$};
		\node [style=X] (2) at (-0.5, 2.75) {};
		\node [style=X] (3) at (-0.5, 5.5) {};
		\node [style=none] (4) at (-0.5, 6.5) {};
		\node [style=map] (5) at (-0.25, 3.75) {$f^\circ$};
		\node [style=map] (6) at (-0.5, 1.25) {$f$};
		\node [style=map] (7) at (-0.5, 2) {$f^\circ f$};
	\end{pgfonlayer}
	\begin{pgfonlayer}{edgelayer}
		\draw (4.center) to (3);
		\draw [in=120, out=-120, looseness=0.75] (3) to (2);
		\draw [in=-63, out=90] (1) to (3);
		\draw [in=-90, out=56] (2) to (5);
		\draw (1) to (5);
		\draw [style=simple] (2) to (7);
		\draw [style=simple] (7) to (6);
		\draw [style=simple] (6) to (0.center);
	\end{pgfonlayer}
\end{tikzpicture}
=
\begin{tikzpicture}
	\begin{pgfonlayer}{nodelayer}
		\node [style=none] (0) at (-0.5, 0.5) {};
		\node [style=map] (1) at (0, 4) {$g$};
		\node [style=X] (2) at (-0.5, 2.25) {};
		\node [style=X] (3) at (-0.5, 5) {};
		\node [style=none] (4) at (-0.5, 6) {};
		\node [style=map] (5) at (0, 3.25) {$f^\circ$};
		\node [style=map] (6) at (-0.5, 1.25) {$f$};
		\node [style=map] (7) at (-1, 3.25) {$f^\circ f$};
		\node [style=none] (8) at (-1, 4) {};
	\end{pgfonlayer}
	\begin{pgfonlayer}{edgelayer}
		\draw (4.center) to (3);
		\draw [in=-60, out=90] (1) to (3);
		\draw [in=-90, out=56] (2) to (5);
		\draw (1) to (5);
		\draw [style=simple] (6) to (0.center);
		\draw [style=simple, in=-90, out=120] (2) to (7);
		\draw [style=simple] (2) to (6);
		\draw [style=simple, in=90, out=-120] (3) to (8.center);
		\draw [style=simple] (8.center) to (7);
	\end{pgfonlayer}
\end{tikzpicture}
=
\begin{tikzpicture}
	\begin{pgfonlayer}{nodelayer}
		\node [style=none] (0) at (-0.5, 0.5) {};
		\node [style=map] (1) at (0, 3.25) {$g$};
		\node [style=X] (2) at (-0.5, 2.25) {};
		\node [style=X] (3) at (-0.5, 4.25) {};
		\node [style=none] (4) at (-0.5, 5.25) {};
		\node [style=map] (5) at (-0.5, 1.25) {$ff^\circ$};
		\node [style=map] (6) at (-1, 3.25) {$f$};
	\end{pgfonlayer}
	\begin{pgfonlayer}{edgelayer}
		\draw (4.center) to (3);
		\draw [in=-60, out=90] (1) to (3);
		\draw [style=simple] (5) to (0.center);
		\draw [style=simple, in=-90, out=120] (2) to (6);
		\draw [style=simple] (2) to (5);
		\draw [style=simple, in=60, out=-90] (1) to (2);
		\draw [style=simple, in=90, out=-120] (3) to (6);
	\end{pgfonlayer}
\end{tikzpicture}
=
\begin{tikzpicture}
	\begin{pgfonlayer}{nodelayer}
		\node [style=none] (0) at (-0.5, 0.5) {};
		\node [style=map] (1) at (0, 3.25) {$g$};
		\node [style=X] (2) at (-0.5, 1.5) {};
		\node [style=X] (3) at (-0.5, 4.25) {};
		\node [style=none] (4) at (-0.5, 5.25) {};
		\node [style=map] (5) at (-1, 3.25) {$f$};
		\node [style=map] (6) at (-1, 2.5) {$ff^\circ$};
		\node [style=none] (7) at (0, 2.5) {};
	\end{pgfonlayer}
	\begin{pgfonlayer}{edgelayer}
		\draw (4.center) to (3);
		\draw [in=-60, out=90] (1) to (3);
		\draw [style=simple, in=90, out=-120] (3) to (5);
		\draw (1) to (7.center);
		\draw [in=60, out=-90] (7.center) to (2);
		\draw (2) to (0.center);
		\draw [in=-90, out=120] (2) to (6);
		\draw (6) to (5);
	\end{pgfonlayer}
\end{tikzpicture}
=
\begin{tikzpicture}
	\begin{pgfonlayer}{nodelayer}
		\node [style=none] (0) at (-0.5, 0.5) {};
		\node [style=map] (1) at (0, 2.5) {$g$};
		\node [style=X] (2) at (-0.5, 1.5) {};
		\node [style=X] (3) at (-0.5, 3.5) {};
		\node [style=none] (4) at (-0.5, 4.5) {};
		\node [style=map] (5) at (-1, 2.5) {$f$};
	\end{pgfonlayer}
	\begin{pgfonlayer}{edgelayer}
		\draw (4.center) to (3);
		\draw [in=-60, out=90] (1) to (3);
		\draw [style=simple, in=90, out=-120] (3) to (5);
		\draw (2) to (0.center);
		\draw [in=60, out=-90] (1) to (2);
		\draw [in=-90, out=120] (2) to (5);
	\end{pgfonlayer}
\end{tikzpicture}
=
\begin{tikzpicture}
	\begin{pgfonlayer}{nodelayer}
		\node [style=none] (0) at (-0.5, 0.5) {};
		\node [style=map] (1) at (-0.2, 2.5) {$g$};
		\node [style=X] (2) at (-0.5, 1.5) {};
		\node [style=X] (3) at (-0.5, 3.5) {};
		\node [style=none] (4) at (-0.5, 4.5) {};
		\node [style=map] (5) at (-0.8, 2.5) {$f$};
	\end{pgfonlayer}
	\begin{pgfonlayer}{edgelayer}
		\draw (4.center) to (3);
		\draw [in=-120, out=90, looseness=1.25] (1) to (3);
		\draw [style=simple, in=90, out=-60, looseness=1.25] (3) to (5);
		\draw (2) to (0.center);
		\draw [in=120, out=-90, looseness=1.25] (1) to (2);
		\draw [in=-90, out=60, looseness=1.25] (2) to (5);
	\end{pgfonlayer}
\end{tikzpicture}
=
\begin{tikzpicture}
	\begin{pgfonlayer}{nodelayer}
		\node [style=none] (0) at (-0.5, 0.5) {};
		\node [style=map] (1) at (0, 2.5) {$f$};
		\node [style=X] (2) at (-0.5, 1.5) {};
		\node [style=X] (3) at (-0.5, 3.5) {};
		\node [style=none] (4) at (-0.5, 4.5) {};
		\node [style=map] (5) at (-1, 2.5) {$g$};
	\end{pgfonlayer}
	\begin{pgfonlayer}{edgelayer}
		\draw (4.center) to (3);
		\draw [in=-60, out=90] (1) to (3);
		\draw [style=simple, in=90, out=-120] (3) to (5);
		\draw (2) to (0.center);
		\draw [in=60, out=-90] (1) to (2);
		\draw [in=-90, out=120] (2) to (5);
	\end{pgfonlayer}
\end{tikzpicture}
=
\begin{tikzpicture}
	\begin{pgfonlayer}{nodelayer}
		\node [style=map] (0) at (-0.5, 1.25) {$g$};
		\node [style=none] (1) at (-0.5, 0.5) {};
		\node [style=map] (2) at (-0.25, 3) {$g^\circ$};
		\node [style=map] (3) at (-0.25, 3.75) {$f$};
		\node [style=X] (4) at (-0.5, 2) {};
		\node [style=X] (5) at (-0.5, 4.75) {};
		\node [style=none] (6) at (-0.5, 5.75) {};
	\end{pgfonlayer}
	\begin{pgfonlayer}{edgelayer}
		\draw (0) to (1.center);
		\draw (6.center) to (5);
		\draw [in=120, out=-120, looseness=0.75] (5) to (4);
		\draw [in=-90, out=56] (4) to (2);
		\draw (0) to (4);
		\draw (3) to (2);
		\draw [in=-63, out=90] (3) to (5);
	\end{pgfonlayer}
\end{tikzpicture}
=
\begin{tikzpicture}
	\begin{pgfonlayer}{nodelayer}
		\node [style=map] (0) at (-0.5, 1.5) {$f$};
		\node [style=none] (1) at (-0.5, 2.5) {};
		\node [style=none] (2) at (-0.5, 0.5) {};
	\end{pgfonlayer}
	\begin{pgfonlayer}{edgelayer}
		\draw (1.center) to (0);
		\draw (0) to (2.center);
	\end{pgfonlayer}
\end{tikzpicture}
\end{align*}


\end{proof}


%
%\begin{lemma}
%The induced Frobenius algebra structure in $\tilde \X$ is counital.
%\end{lemma}
%\begin{proof}
%For all $X$, the map $X \to (X\otimes X) \otimes I$ in $\tilde\X$ induced by the Frobenius algebra in $\X$ has a counit given by the  unitor $X\to I\otimes X$ since, in $\X$:
%$$
%\begin{tikzpicture}
%	\begin{pgfonlayer}{nodelayer}
%		\node [style=X] (0) at (0, 3.75) {};
%		\node [style=none] (1) at (0, 3) {};
%		\node [style=X] (2) at (-0.25, 4.75) {};
%		\node [style=X] (3) at (0, 5.75) {};
%		\node [style=X] (4) at (-0.25, 7.25) {};
%		\node [style=none] (5) at (-0.25, 8) {};
%		\node [style=none] (6) at (0.25, 8) {};
%		\node [style=none] (7) at (0, 6.5) {};
%	\end{pgfonlayer}
%	\begin{pgfonlayer}{edgelayer}
%		\draw (0) to (1.center);
%		\draw [in=-90, out=120] (0) to (2);
%		\draw (2) to (3);
%		\draw (5.center) to (4);
%		\draw [in=120, out=-120, looseness=0.75] (4) to (2);
%		\draw [in=60, out=-60] (3) to (0);
%		\draw (3) to (7.center);
%		\draw [in=-45, out=90, looseness=0.75] (7.center) to (4);
%		\draw [style=dashed, in=-90, out=75] (7.center) to (6.center);
%	\end{pgfonlayer}
%\end{tikzpicture}
%=
%\begin{tikzpicture}
%	\begin{pgfonlayer}{nodelayer}
%		\node [style=none] (0) at (0, 3) {};
%		\node [style=none] (1) at (0, 4.5) {};
%		\node [style=none] (2) at (0.5, 4.5) {};
%		\node [style=none] (3) at (0, 3.75) {};
%	\end{pgfonlayer}
%	\begin{pgfonlayer}{edgelayer}
%		\draw [style=dashed, in=-90, out=15] (3.center) to (2.center);
%		\draw [style=simple] (3.center) to (0.center);
%		\draw [style=simple] (3.center) to (1.center);
%	\end{pgfonlayer}
%\end{tikzpicture}
%$$
%\end{proof}


\begin{definition}
A {\bf bicategory of relations} is  TODO
\end{definition}

TODO

\begin{lemma}
Given a regular category $\X$, $\Rel(\X)$ is a cartesian bicategory with respect to the cartesian product.
\end{lemma}

\begin{example}
$\Rel=\Rel(\Set)$ is the canonical example of a bicategory of relations.  $\FRel=\Rel(\FSet)$ is a bicategory of relations as well.
\end{example}

Therefore, now we can see how cartesian restriction categories,  discrete inverse categories and a bicategories of relations are related.  At each level, we ask for a compatible familty of some components of a special commutative Frobenius algebra, along with various naturality requirements.  The more components of the Frobenius algebra we ask for, the weaker notion of copying we obtain.


CARTESIAN BICATEGORIES WITH SUMS


We will also be concerned with categories of relations with linear or affine structure.  These both have very elegant presentations:


\begin{definition}
Given a field $k$, the $\dag$-compact closed prop of linear relations over $k$, the category of {\bf linear relations} over $k$, $\LinRel_{k}$ is $\Rel(\Mat_k)$ under the direct sum.

Explicity, $\LinRel_{k}$ has:

\begin{description}
\item[Objects:] Natural numbers.

\item[Maps:] A linear relation $n\to m$ is a linear subspace of $k^n \oplus k^m$.

\item[Composition:] Relational composition, so that for $R \subseteq k^n \oplus k^m$  and $S \subseteq k^m \oplus k^\ell$:
$$
R;S := \{  (x,z) \in k^{n} \oplus k^{\ell} : \exists y \in k^{m}, (x,y) \in R \wedge (y,z) \in S \} \subseteq k^n \oplus k^\ell
$$ 

\item[Tensor product:] Direct sum, so that for $R \subseteq k^n \oplus k^m$ and $S \subseteq k^\ell \oplus k^q$:

$$R\oplus S : =
\left\{
\left(
\begin{pmatrix}
a_1\\a_2
\end{pmatrix},
\begin{pmatrix}
b_1\\b_2
\end{pmatrix}
:
\forall (a_1,b_1) \in R, (a_2,b_2) \in S
\right)
\right\} \subseteq k^{n+\ell}\oplus k^{m+q}
$$

\item[Dagger:] Relational converse, so that for $R \subseteq k^{n}\oplus k^m$:

$$
R^T := \{ (b,a) : \forall (a,b) \in R \} \subseteq k^{m} \oplus k^n
$$
\end{description}
\end{definition}



\begin{lemma}[\cite{ihpub}]
Given a field $k$, $\LinRel_{k}$ is generated by the generators and equations of the presentation of $\Mat_k$ as well as those of $\Mat_k^{\op}$ (drawn as the vertically flipped generators of $\Mat_k$) modulo the equations for all $a \in k$, $a\neq 0$:

$$
\begin{tikzpicture}
	\begin{pgfonlayer}{nodelayer}
		\node [style=none] (0) at (1.5, -0.5) {};
		\node [style=none] (1) at (0.5, -0.5) {};
		\node [style=none] (2) at (1, -0.5) {$\cdots$};
		\node [style=none] (3) at (0.5, -2.75) {};
		\node [style=Z] (4) at (1, -1.25) {};
		\node [style=none] (5) at (2, -0.5) {};
		\node [style=none] (6) at (1.5, -2.75) {$\cdots$};
		\node [style=none] (7) at (1, -2.75) {};
		\node [style=Z] (8) at (1.5, -2) {};
		\node [style=none] (9) at (2, -2.75) {};
		\node [style=none] (10) at (1.25, -1.5) {\reflectbox{$\ddots$}};
	\end{pgfonlayer}
	\begin{pgfonlayer}{edgelayer}
		\draw [in=-124, out=90] (3.center) to (4);
		\draw [in=-90, out=56] (4) to (0.center);
		\draw [in=124, out=-90] (1.center) to (4);
		\draw [in=-124, out=90] (7.center) to (8);
		\draw [in=90, out=-56] (8) to (9.center);
		\draw [in=-90, out=56] (8) to (5.center);
		\draw [bend left=45, looseness=1.25] (8) to (4);
		\draw [bend left=45, looseness=1.25] (4) to (8);
	\end{pgfonlayer}
\end{tikzpicture}
=
\begin{tikzpicture}
	\begin{pgfonlayer}{nodelayer}
		\node [style=none] (11) at (4, -0.5) {};
		\node [style=none] (12) at (3, -0.5) {};
		\node [style=none] (13) at (3.5, -0.5) {$\cdots$};
		\node [style=none] (14) at (2.5, -2) {};
		\node [style=Z] (15) at (3.5, -1.25) {};
		\node [style=none] (16) at (4.5, -0.5) {};
		\node [style=none] (17) at (3.5, -2) {$\cdots$};
		\node [style=none] (18) at (3, -2) {};
		\node [style=Z] (19) at (3.5, -1.25) {};
		\node [style=none] (20) at (4, -2) {};
	\end{pgfonlayer}
	\begin{pgfonlayer}{edgelayer}
		\draw [in=-150, out=90] (14.center) to (15);
		\draw [in=-90, out=56] (15) to (11.center);
		\draw [in=124, out=-90] (12.center) to (15);
		\draw [in=-124, out=90] (18.center) to (19);
		\draw [in=90, out=-56] (19) to (20.center);
		\draw [in=-90, out=30] (19) to (16.center);
	\end{pgfonlayer}
\end{tikzpicture},
\hspace*{1cm}
\begin{tikzpicture}
	\begin{pgfonlayer}{nodelayer}
		\node [style=none] (0) at (1.5, -0.5) {};
		\node [style=none] (1) at (0.5, -0.5) {};
		\node [style=none] (2) at (1, -0.5) {$\cdots$};
		\node [style=none] (3) at (0.5, -2.75) {};
		\node [style=X] (4) at (1, -1.25) {};
		\node [style=none] (5) at (2, -0.5) {};
		\node [style=none] (6) at (1.5, -2.75) {$\cdots$};
		\node [style=none] (7) at (1, -2.75) {};
		\node [style=X] (8) at (1.5, -2) {};
		\node [style=none] (9) at (2, -2.75) {};
		\node [style=none] (10) at (1.25, -1.5) {\reflectbox{$\ddots$}};
	\end{pgfonlayer}
	\begin{pgfonlayer}{edgelayer}
		\draw [in=-124, out=90] (3.center) to (4);
		\draw [in=-90, out=56] (4) to (0.center);
		\draw [in=124, out=-90] (1.center) to (4);
		\draw [in=-124, out=90] (7.center) to (8);
		\draw [in=90, out=-56] (8) to (9.center);
		\draw [in=-90, out=56] (8) to (5.center);
		\draw [bend left=45, looseness=1.25] (8) to (4);
		\draw [bend left=45, looseness=1.25] (4) to (8);
	\end{pgfonlayer}
\end{tikzpicture}
=
\begin{tikzpicture}
	\begin{pgfonlayer}{nodelayer}
		\node [style=none] (11) at (4, -0.5) {};
		\node [style=none] (12) at (3, -0.5) {};
		\node [style=none] (13) at (3.5, -0.5) {$\cdots$};
		\node [style=none] (14) at (2.5, -2) {};
		\node [style=X] (15) at (3.5, -1.25) {};
		\node [style=none] (16) at (4.5, -0.5) {};
		\node [style=none] (17) at (3.5, -2) {$\cdots$};
		\node [style=none] (18) at (3, -2) {};
		\node [style=X] (19) at (3.5, -1.25) {};
		\node [style=none] (20) at (4, -2) {};
	\end{pgfonlayer}
	\begin{pgfonlayer}{edgelayer}
		\draw [in=-150, out=90] (14.center) to (15);
		\draw [in=-90, out=56] (15) to (11.center);
		\draw [in=124, out=-90] (12.center) to (15);
		\draw [in=-124, out=90] (18.center) to (19);
		\draw [in=90, out=-56] (19) to (20.center);
		\draw [in=-90, out=30] (19) to (16.center);
	\end{pgfonlayer}
\end{tikzpicture},
\hspace*{1cm}
\begin{tikzpicture}
	\begin{pgfonlayer}{nodelayer}
		\node [style=Z] (0) at (3.75, -1) {};
	\end{pgfonlayer}
\end{tikzpicture}
=
\begin{tikzpicture}
	\begin{pgfonlayer}{nodelayer}
		\node [style=X] (0) at (3.75, -1) {};
	\end{pgfonlayer}
\end{tikzpicture}
=
\begin{tikzpicture}
	\begin{pgfonlayer}{nodelayer}
		\node [style=none] (0) at (2, 0) {};
		\node [style=none] (1) at (2, -1) {};
		\node [style=none] (2) at (3, -1) {};
		\node [style=none] (3) at (3, 0) {};
	\end{pgfonlayer}
	\begin{pgfonlayer}{edgelayer}
		\draw[style=dashed] (3.center) to (0.center);
		\draw[style=dashed] (0.center) to (1.center);
		\draw[style=dashed] (1.center) to (2.center);
		\draw[style=dashed] (2.center) to (3.center);
	\end{pgfonlayer}
\end{tikzpicture},
\hspace*{1cm}
\begin{tikzpicture}
	\begin{pgfonlayer}{nodelayer}
		\node [style=none] (3) at (17, 1.5) {};
		\node [style=none] (4) at (17, -0.75) {};
		\node [style=scalarop] (5) at (17, 0.75) {$a$};
		\node [style=scalar] (6) at (17, 0) {$a$};
	\end{pgfonlayer}
	\begin{pgfonlayer}{edgelayer}
		\draw (4.center) to (6);
		\draw (6) to (5);
		\draw (5) to (3.center);
	\end{pgfonlayer}
\end{tikzpicture}
=
\begin{tikzpicture}
	\begin{pgfonlayer}{nodelayer}
		\node [style=none] (3) at (17, 1.5) {};
		\node [style=none] (4) at (17, -0.75) {};
		\node [style=scalar] (5) at (17, 0.75) {$a$};
		\node [style=scalarop] (6) at (17, 0) {$a$};
	\end{pgfonlayer}
	\begin{pgfonlayer}{edgelayer}
		\draw (4.center) to (6);
		\draw (6) to (5);
		\draw (5) to (3.center);
	\end{pgfonlayer}
\end{tikzpicture}
=
\begin{tikzpicture}
	\begin{pgfonlayer}{nodelayer}
		\node [style=none] (3) at (17, 1.5) {};
		\node [style=none] (4) at (17, -0.75) {};
	\end{pgfonlayer}
	\begin{pgfonlayer}{edgelayer}
		\draw (4.center) to (3.center);
	\end{pgfonlayer}
\end{tikzpicture}
$$
\end{lemma}

We can also capture affine relations as a prop, by regarding the empty relation as a subobject:


\begin{definition}
The prop of affine relations over $k$, $\Aff\Rel_{k}$ is constructed in the same way as $\LinRel_k$ except a map $n\to m$ is instead a (possibly empty) affine subspace of $k^n\oplus k^m$.
\end{definition}


\begin{lemma}[\cite{affine}]
$\Aff\Rel_{\F_p}$ is presented by the prop $\aih_k$ given by $\ih_k$ in addition to the following generator:

$$
\begin{tikzpicture}
	\begin{pgfonlayer}{nodelayer}
		\node [style=none] (0) at (4, -0.5) {};
		\node [style=none] (1) at (3, -0.5) {};
		\node [style=none] (2) at (3.5, -0.5) {$\cdots$};
		\node [style=X] (4) at (3.5, -1.25) {};
		\node [style=none] (6) at (3.5, -2) {$\cdots$};
		\node [style=none] (7) at (3, -2) {};
		\node [style=none] (8) at (3.5, -1.25) {};
		\node [style=none] (9) at (4, -2) {};
	\end{pgfonlayer}
	\begin{pgfonlayer}{edgelayer}
		\draw [in=-90, out=56] (4) to (0.center);
		\draw [in=124, out=-90] (1.center) to (4);
		\draw [in=-124, out=90] (7.center) to (8.center);
		\draw [in=90, out=-56] (8.center) to (9.center);
	\end{pgfonlayer}
\end{tikzpicture}
:=
\begin{tikzpicture}
	\begin{pgfonlayer}{nodelayer}
		\node [style=none] (0) at (4, -0.5) {};
		\node [style=none] (1) at (3, -0.5) {};
		\node [style=none] (2) at (3.5, -0.5) {$\cdots$};
		\node [style=X] (4) at (3.5, -1.25) {$0$};
		\node [style=none] (6) at (3.5, -2) {$\cdots$};
		\node [style=none] (7) at (3, -2) {};
		\node [style=none] (8) at (3.5, -1.25) {};
		\node [style=none] (9) at (4, -2) {};
	\end{pgfonlayer}
	\begin{pgfonlayer}{edgelayer}
		\draw [in=-90, out=56] (4) to (0.center);
		\draw [in=124, out=-90] (1.center) to (4);
		\draw [in=-124, out=90] (7.center) to (8.center);
		\draw [in=90, out=-56] (8.center) to (9.center);
	\end{pgfonlayer}
\end{tikzpicture},
\hspace*{1cm}
\begin{tikzpicture}
	\begin{pgfonlayer}{nodelayer}
		\node [style=none] (0) at (1.5, -0.5) {};
		\node [style=none] (1) at (0.5, -0.5) {};
		\node [style=none] (2) at (1, -0.5) {$\cdots$};
		\node [style=none] (3) at (0.5, -2.75) {};
		\node [style=X] (4) at (1, -1.25) {$a$};
		\node [style=none] (5) at (2, -0.5) {};
		\node [style=none] (6) at (1.5, -2.75) {$\cdots$};
		\node [style=none] (7) at (1, -2.75) {};
		\node [style=X] (8) at (1.5, -2) {$b$};
		\node [style=none] (9) at (2, -2.75) {};
		\node [style=none] (10) at (1.25, -1.5) {\reflectbox{$\ddots$}};
	\end{pgfonlayer}
	\begin{pgfonlayer}{edgelayer}
		\draw [in=-124, out=90] (3.center) to (4);
		\draw [in=-90, out=56] (4) to (0.center);
		\draw [in=124, out=-90] (1.center) to (4);
		\draw [in=-124, out=90] (7.center) to (8);
		\draw [in=90, out=-56] (8) to (9.center);
		\draw [in=-90, out=56] (8) to (5.center);
		\draw [bend left=45, looseness=1.25] (8) to (4);
		\draw [bend left=45, looseness=1.25] (4) to (8);
	\end{pgfonlayer}
\end{tikzpicture}
=
\begin{tikzpicture}
	\begin{pgfonlayer}{nodelayer}
		\node [style=none] (11) at (4, -0.5) {};
		\node [style=none] (12) at (3, -0.5) {};
		\node [style=none] (13) at (3.5, -0.5) {$\cdots$};
		\node [style=none] (14) at (2.5, -2) {};
		\node [style=X] (15) at (3.5, -1.25) {$a+b$};
		\node [style=none] (16) at (4.5, -0.5) {};
		\node [style=none] (17) at (3.5, -2) {$\cdots$};
		\node [style=none] (18) at (3, -2) {};
		\node [style=none] (19) at (3.5, -1.25) {};
		\node [style=none] (20) at (4, -2) {};
	\end{pgfonlayer}
	\begin{pgfonlayer}{edgelayer}
		\draw [in=-150, out=90] (14.center) to (15);
		\draw [in=-90, out=56] (15) to (11.center);
		\draw [in=124, out=-90] (12.center) to (15);
		\draw [in=-124, out=90] (18.center) to (19);
		\draw [in=90, out=-56] (19) to (20.center);
		\draw [in=-90, out=30] (19) to (16.center);
	\end{pgfonlayer}
\end{tikzpicture},
\hspace*{1cm}
\begin{tikzpicture}
	\begin{pgfonlayer}{nodelayer}
		\node [style=X] (0) at (3.5, -1.25) {$1$};
		\node [style=none] (1) at (4, -0.5) {};
		\node [style=none] (2) at (4, -2) {};
	\end{pgfonlayer}
	\begin{pgfonlayer}{edgelayer}
		\draw (2.center) to (1.center);
	\end{pgfonlayer}
\end{tikzpicture}
=
\begin{tikzpicture}
	\begin{pgfonlayer}{nodelayer}
		\node [style=X] (4) at (3.5, -1.25) {$1$};
		\node [style=none] (9) at (4, -0.5) {};
		\node [style=none] (10) at (4, -2) {};
		\node [style=Z] (11) at (4, -1.5) {};
		\node [style=X] (12) at (4, -1) {};
	\end{pgfonlayer}
	\begin{pgfonlayer}{edgelayer}
		\draw (10.center) to (11);
		\draw (12) to (9.center);
	\end{pgfonlayer}
\end{tikzpicture}
$$


\end{lemma}

\section{Categorical quantum mechanics}
Categorical quantum mechanics reformulates finite dimensional quantum mechanics in terms of monoidal categories.  Frobenius algebras and bialgebras play a central role, much like in cartesian bicategories.  We assume basic knowledge of quantum computing in order to understand this section.



\begin{definition}

A {\bf dagger compact closed category } is a compact closed category equipped with a dagger symmetric monoidal  functor whose underlying symmetric monoidal closed structure forms a symmetric monoidal dagger category, where moreover, for all $X$, $\epsilon^\dagger = \eta;c$.
\end{definition}

\begin{example}
$\FHilb$ is a dagger compact closed category, where the monoidal product is the bilinear tensor, the dagger is given by the Hermetian adjoint and the unit is given by the normalized sum of tensor product of basis elements:

$$
\dfrac{1}{\sqrt d}\sum_{i=0}^{d-1} |i\rangle \otimes   |i\rangle
$$
\end{example}


\begin{example}
Given a finitely complete category $\X$, $\Span^\sim (\X)$ is dagger compact closed where the tensor is the cartesian product, the dagger is the transpose and the unit is the span $I \xleftarrow{!_X}  X \xrightarrow{\Delta_X} X \times X$.

Similarly, given a regular category $\X$, $\Rel(\X)$ is  dagger compact closed.
\end{example}

There is a very important algebraic structure on $\dag$-compact closed categories:

\begin{definition}
\label{def:specialdagfa}
%special dag-Frobenius algebras
A {\bf dagger Frobenius algebra}  is a Frobenius algebra in a dagger category whose monoid structure is the dagger of the comonoid structure.

\end{definition}

These allow us to treat bases diagramatically:
\begin{lemma}
\label{lem:specialdagfa}
%special dag-Frobenius algebras in FHilb are orthonomal bases/quantum observables

Special commutative dagger Frobenius algebras in $\FHilb$ are in bijection with orthonormal bases.
\end{lemma}

In quantum computing, quantum observables are Hermetian matrices and spectrum of measurement outcomes the orthonormal eigenbasis of the observable.  Therefore, $\dag$-Frobenius algebras play a prominent role in the diagrammatic analysis of quantum computing.

This naturally leads to the following:

\begin{definition}
\label{def:phases}
Given a $\dag$-Frobenius algebra on an object $X$, a {\bf phase} for the Frobenius algebra is a unitary endomorphism $\theta:X\to X$ which commutes with the multiplication and comultiplication, so that:
$$
\begin{tikzpicture}
	\begin{pgfonlayer}{nodelayer}
		\node [style=Z] (14) at (-1.75, 12) {};
		\node [style=none] (15) at (-1.25, 11.25) {};
		\node [style=none] (16) at (-2.25, 11.25) {};
		\node [style=none] (17) at (-1.75, 12.75) {};
		\node [style=map] (18) at (-1.25, 11.25) {$\theta$};
		\node [style=none] (19) at (-1.25, 10.5) {};
		\node [style=none] (20) at (-2.25, 10.5) {};
	\end{pgfonlayer}
	\begin{pgfonlayer}{edgelayer}
		\draw [in=90, out=-30] (14) to (15.center);
		\draw [in=90, out=-150] (14) to (16.center);
		\draw (17.center) to (14);
		\draw (18) to (19.center);
		\draw (20.center) to (16.center);
	\end{pgfonlayer}
\end{tikzpicture}
=
\begin{tikzpicture}
	\begin{pgfonlayer}{nodelayer}
		\node [style=Z] (0) at (0, 11.25) {};
		\node [style=map] (1) at (0, 12) {$\theta$};
		\node [style=none] (2) at (-0.5, 10.5) {};
		\node [style=none] (3) at (0.5, 10.5) {};
		\node [style=none] (4) at (0, 12.75) {};
	\end{pgfonlayer}
	\begin{pgfonlayer}{edgelayer}
		\draw (4.center) to (1);
		\draw (1) to (0);
		\draw [in=90, out=-150] (0) to (2.center);
		\draw [in=90, out=-30] (0) to (3.center);
	\end{pgfonlayer}
\end{tikzpicture}
=
\begin{tikzpicture}
	\begin{pgfonlayer}{nodelayer}
		\node [style=Z] (5) at (2, 12) {};
		\node [style=none] (7) at (1.5, 11.25) {};
		\node [style=none] (8) at (2.5, 11.25) {};
		\node [style=none] (9) at (2, 12.75) {};
		\node [style=map] (11) at (1.5, 11.25) {$\theta$};
		\node [style=none] (12) at (1.5, 10.5) {};
		\node [style=none] (13) at (2.5, 10.5) {};
	\end{pgfonlayer}
	\begin{pgfonlayer}{edgelayer}
		\draw [in=90, out=-150] (5) to (7.center);
		\draw [in=90, out=-30] (5) to (8.center);
		\draw (9.center) to (5);
		\draw (11) to (12.center);
		\draw (13.center) to (8.center);
	\end{pgfonlayer}
\end{tikzpicture}
\hspace*{1cm}
\begin{tikzpicture}
	\begin{pgfonlayer}{nodelayer}
		\node [style=Z] (33) at (4, 11.25) {};
		\node [style=none] (34) at (4.5, 12) {};
		\node [style=none] (35) at (3.5, 12) {};
		\node [style=none] (36) at (4, 10.5) {};
		\node [style=map] (37) at (4.5, 12) {$\theta$};
		\node [style=none] (38) at (4.5, 12.75) {};
		\node [style=none] (39) at (3.5, 12.75) {};
	\end{pgfonlayer}
	\begin{pgfonlayer}{edgelayer}
		\draw [in=-90, out=30] (33) to (34.center);
		\draw [in=-90, out=150] (33) to (35.center);
		\draw (36.center) to (33);
		\draw (37) to (38.center);
		\draw (39.center) to (35.center);
	\end{pgfonlayer}
\end{tikzpicture}
=
\begin{tikzpicture}
	\begin{pgfonlayer}{nodelayer}
		\node [style=Z] (21) at (5.75, 12) {};
		\node [style=map] (22) at (5.75, 11.25) {$\theta$};
		\node [style=none] (23) at (5.25, 12.75) {};
		\node [style=none] (24) at (6.25, 12.75) {};
		\node [style=none] (25) at (5.75, 10.5) {};
	\end{pgfonlayer}
	\begin{pgfonlayer}{edgelayer}
		\draw (25.center) to (22);
		\draw (22) to (21);
		\draw [in=-90, out=150] (21) to (23.center);
		\draw [in=-90, out=30] (21) to (24.center);
	\end{pgfonlayer}
\end{tikzpicture}
=
\begin{tikzpicture}
	\begin{pgfonlayer}{nodelayer}
		\node [style=Z] (26) at (7.75, 11.25) {};
		\node [style=none] (27) at (7.25, 12) {};
		\node [style=none] (28) at (8.25, 12) {};
		\node [style=none] (29) at (7.75, 10.5) {};
		\node [style=map] (30) at (7.25, 12) {$\theta$};
		\node [style=none] (31) at (7.25, 12.75) {};
		\node [style=none] (32) at (8.25, 12.75) {};
	\end{pgfonlayer}
	\begin{pgfonlayer}{edgelayer}
		\draw [in=-90, out=150] (26) to (27.center);
		\draw [in=-90, out=30] (26) to (28.center);
		\draw (29.center) to (26);
		\draw (30) to (31.center);
		\draw (32.center) to (28.center);
	\end{pgfonlayer}
\end{tikzpicture}
$$

Phases for Frobenius algebras are preserved by composition; and they form a group called the {\bf phase group} for the Frobenius algebra.  The phase group associated with a commutative Frobenius algebra is therefore Abelian.
\end{definition}

The motivating example is in $\FHilb$
\begin{example}
Given an orthonormal basis $\{| j \rangle \}_{j \in I}$ in $\FHilb$, the phases are generated by unitaries $\sum_{j} e^{ \theta \pi i}|  j \rangle\langle j|$ for all $\theta \in [0, 2\pi)$. 

 $e^{ \theta \pi i}$, $\theta \in [0, 2\pi)$ carves out the unit circle in the complex plane; thus the phase group is isomorphic to the circle (hence the name).
\end{example}

\begin{lemma}[Phased spider theorem]
There is a normal form for the string diagrams generated by the components of a Frobenius algebra and its phase group.
\end{lemma}

The normal form is the same as the vanilla spider theorem, except decorated with a single element of the phase group in the middle, between where the monoid and comonoid meet:

$$
\begin{tikzpicture}
	\begin{pgfonlayer}{nodelayer}
		\node [style=Z] (9) at (4.75, 3) {};
		\node [style=Z] (10) at (4, 4) {};
		\node [style=Z] (11) at (4.75, 2) {};
		\node [style=Z] (12) at (4, 1) {};
		\node [style=none] (13) at (5, 4) {};
		\node [style=none] (14) at (5, 1) {};
		\node [style=none] (15) at (3.75, 0.25) {};
		\node [style=none] (16) at (5, 4.75) {};
		\node [style=none] (17) at (5, 0.25) {};
		\node [style=none] (18) at (4.25, 4.75) {};
		\node [style=none] (19) at (3.75, 4.75) {};
		\node [style=none] (20) at (4.25, 0.25) {};
		\node [style=none] (21) at (4.5, 3.25) {};
		\node [style=none] (22) at (4, 3.75) {};
		\node [style=none] (23) at (4, 1.25) {};
		\node [style=none] (24) at (4.5, 1.75) {};
		\node [style=none] (25) at (4.25, 3.5) {$\ddots$};
		\node [style=none] (26) at (4.25, 1.5) {$\reflectbox{$\ddots$}$};
		\node [style=none] (27) at (4.7, 0.25) {$\cdots$};
		\node [style=none] (28) at (4.7, 4.75) {$\cdots$};
		\node [style=map] (29) at (4.75, 2.5) {$\theta$};
	\end{pgfonlayer}
	\begin{pgfonlayer}{edgelayer}
		\draw (16.center) to (13.center);
		\draw [in=105, out=-90] (19.center) to (10);
		\draw [in=60, out=-90, looseness=0.75] (13.center) to (9);
		\draw [in=-90, out=75] (10) to (18.center);
		\draw [in=300, out=90] (14.center) to (11);
		\draw [in=90, out=-120] (12) to (15.center);
		\draw [in=90, out=-60] (12) to (20.center);
		\draw (17.center) to (14.center);
		\draw (9) to (11);
		\draw (12) to (23.center);
		\draw (24.center) to (11);
		\draw (22.center) to (10);
		\draw (9) to (21.center);
	\end{pgfonlayer}
\end{tikzpicture}
=:
\begin{tikzpicture}
	\begin{pgfonlayer}{nodelayer}
		\node [style=none] (0) at (1.5, 1.75) {};
		\node [style=none] (1) at (2.75, 1.75) {};
		\node [style=none] (2) at (2, 1.75) {};
		\node [style=none] (3) at (2.45, 1.75) {$\cdots$};
		\node [style=none] (4) at (2.75, 3.25) {};
		\node [style=none] (5) at (2, 3.25) {};
		\node [style=none] (6) at (1.5, 3.25) {};
		\node [style=none] (7) at (2.45, 3.25) {$\cdots$};
		\node [style=Z] (8) at (2, 2.5) {$\theta$};
	\end{pgfonlayer}
	\begin{pgfonlayer}{edgelayer}
		\draw [in=-90, out=45] (8) to (4.center);
		\draw (8) to (5.center);
		\draw [in=135, out=-90] (6.center) to (8);
		\draw [in=90, out=-150] (8) to (0.center);
		\draw (2.center) to (8);
		\draw [in=90, out=-30] (8) to (1.center);
	\end{pgfonlayer}
\end{tikzpicture}
$$


The normal form induces a phased spider fusion rule:

$$
\begin{tikzpicture}
	\begin{pgfonlayer}{nodelayer}
		\node [style=none] (0) at (1.5, -0.5) {};
		\node [style=none] (1) at (0.5, -0.5) {};
		\node [style=none] (2) at (1, -0.5) {$\cdots$};
		\node [style=none] (3) at (0.5, -2.75) {};
		\node [style=Z] (4) at (1, -1.25) {$\theta$};
		\node [style=none] (5) at (2, -0.5) {};
		\node [style=none] (6) at (1.5, -2.75) {$\cdots$};
		\node [style=none] (7) at (1, -2.75) {};
		\node [style=Z] (8) at (1.5, -2) {$\phi$};
		\node [style=none] (9) at (2, -2.75) {};
		\node [style=none] (10) at (1.25, -1.5) {\reflectbox{$\ddots$}};
	\end{pgfonlayer}
	\begin{pgfonlayer}{edgelayer}
		\draw [in=-124, out=90] (3.center) to (4);
		\draw [in=-90, out=56] (4) to (0.center);
		\draw [in=124, out=-90] (1.center) to (4);
		\draw [in=-124, out=90] (7.center) to (8);
		\draw [in=90, out=-56] (8) to (9.center);
		\draw [in=-90, out=56] (8) to (5.center);
		\draw (8) to (4);
	\end{pgfonlayer}
\end{tikzpicture}
=
\begin{tikzpicture}
	\begin{pgfonlayer}{nodelayer}
		\node [style=none] (11) at (4, -0.5) {};
		\node [style=none] (12) at (3, -0.5) {};
		\node [style=none] (13) at (3.5, -0.5) {$\cdots$};
		\node [style=none] (14) at (2.5, -2) {};
		\node [style=none] (15) at (3.5, -1.25) {};
		\node [style=none] (16) at (4.5, -0.5) {};
		\node [style=none] (17) at (3.5, -2) {$\cdots$};
		\node [style=none] (18) at (3, -2) {};
		\node [style=Z] (19) at (3.5, -1.25) {$\theta;\phi$};
		\node [style=none] (20) at (4, -2) {};
	\end{pgfonlayer}
	\begin{pgfonlayer}{edgelayer}
		\draw [in=-150, out=90] (14.center) to (15);
		\draw [in=-90, out=56] (15) to (11.center);
		\draw [in=124, out=-90] (12.center) to (15);
		\draw [in=-124, out=90] (18.center) to (19);
		\draw [in=90, out=-56] (19) to (20.center);
		\draw [in=-90, out=30] (19) to (16.center);
	\end{pgfonlayer}
\end{tikzpicture}
$$

This notation is compatible with the non-phased spider notation, where a spider drawn with no phase corresponds to a phased spider whose phase is the identity:

$$
\begin{tikzpicture}
	\begin{pgfonlayer}{nodelayer}
		\node [style=none] (0) at (4, -0.5) {};
		\node [style=none] (1) at (3, -0.5) {};
		\node [style=none] (2) at (3.5, -0.5) {$\cdots$};
		\node [style=none] (4) at (3.5, -1.25) {};
		\node [style=none] (6) at (3.5, -2) {$\cdots$};
		\node [style=none] (7) at (3, -2) {};
		\node [style=Z] (8) at (3.5, -1.25) {};
		\node [style=none] (9) at (4, -2) {};
	\end{pgfonlayer}
	\begin{pgfonlayer}{edgelayer}
		\draw [in=-90, out=56] (4.center) to (0.center);
		\draw [in=124, out=-90] (1.center) to (4.center);
		\draw [in=-124, out=90] (7.center) to (8);
		\draw [in=90, out=-56] (8) to (9.center);
	\end{pgfonlayer}
\end{tikzpicture}
=
\begin{tikzpicture}
	\begin{pgfonlayer}{nodelayer}
		\node [style=none] (0) at (4, -0.5) {};
		\node [style=none] (1) at (3, -0.5) {};
		\node [style=none] (2) at (3.5, -0.5) {$\cdots$};
		\node [style=none] (4) at (3.5, -1.25) {};
		\node [style=none] (6) at (3.5, -2) {$\cdots$};
		\node [style=none] (7) at (3, -2) {};
		\node [style=Z] (8) at (3.5, -1.25) {$1$};
		\node [style=none] (9) at (4, -2) {};
	\end{pgfonlayer}
	\begin{pgfonlayer}{edgelayer}
		\draw [in=-90, out=56] (4.center) to (0.center);
		\draw [in=124, out=-90] (1.center) to (4.center);
		\draw [in=-124, out=90] (7.center) to (8);
		\draw [in=90, out=-56] (8) to (9.center);
	\end{pgfonlayer}
\end{tikzpicture}
$$



\begin{definition}
\label{def:complementary}
%Interacting Hopf-Frobenius algebras/ strongly complementary observables

Two bases in $\FHilb$ are {\bf strongly complementary} when their corresponding Frobenius algebras interact to form a Hopf algebra whose antipode is equivalently any of the following maps:
$$
\begin{tikzpicture}
	\begin{pgfonlayer}{nodelayer}
		\node [style=Z] (0) at (0.5, 0) {};
		\node [style=X] (1) at (1, 0.5) {};
		\node [style=none] (2) at (0, 1) {};
		\node [style=none] (3) at (1.5, -0.5) {};
	\end{pgfonlayer}
	\begin{pgfonlayer}{edgelayer}
		\draw [in=-90, out=135] (0) to (2.center);
		\draw (0) to (1);
		\draw [in=90, out=-45] (1) to (3.center);
	\end{pgfonlayer}
\end{tikzpicture}=
\begin{tikzpicture}
	\begin{pgfonlayer}{nodelayer}
		\node [style=X] (0) at (0.5, 0) {};
		\node [style=Z] (1) at (1, 0.5) {};
		\node [style=none] (2) at (0, 1) {};
		\node [style=none] (3) at (1.5, -0.5) {};
	\end{pgfonlayer}
	\begin{pgfonlayer}{edgelayer}
		\draw [in=-90, out=135] (0) to (2.center);
		\draw (0) to (1);
		\draw [in=90, out=-45] (1) to (3.center);
	\end{pgfonlayer}
\end{tikzpicture}=
\begin{tikzpicture}
	\begin{pgfonlayer}{nodelayer}
		\node [style=Z] (0) at (1, 0) {};
		\node [style=X] (1) at (0.5, 0.5) {};
		\node [style=none] (2) at (1.5, 1) {};
		\node [style=none] (3) at (0, -0.5) {};
	\end{pgfonlayer}
	\begin{pgfonlayer}{edgelayer}
		\draw [in=-90, out=45] (0) to (2.center);
		\draw (0) to (1);
		\draw [in=90, out=-135] (1) to (3.center);
	\end{pgfonlayer}
\end{tikzpicture}=
\begin{tikzpicture}
	\begin{pgfonlayer}{nodelayer}
		\node [style=X] (0) at (1, 0) {};
		\node [style=Z] (1) at (0.5, 0.5) {};
		\node [style=none] (2) at (1.5, 1) {};
		\node [style=none] (3) at (0, -0.5) {};
	\end{pgfonlayer}
	\begin{pgfonlayer}{edgelayer}
		\draw [in=-90, out=45] (0) to (2.center);
		\draw (0) to (1);
		\draw [in=90, out=-135] (1) to (3.center);
	\end{pgfonlayer}
\end{tikzpicture}
$$

\end{definition}
%example: this can be used to construct the CNOT gate





\begin{definition}
\label{def:zx}
Given some fixed dimension $d$, the qudit ZX-calculus is a collection of related graphical calculi with faithful interpretations into $\FHilb$ generated by the phased Frobenius algebras for the standard basis and Fourier bases.

A symmetric monoidal theory is a {\bf fragment of the ZX-calculus} when it is a symmetric monoidal subtheory of the $\ZX$-calculus with a faithful interpretation in $\FHilb$.
\end{definition}


\begin{definition}
The {\bf phase-free} qudit $\ZX$-calculus
is the fragment of the ZX-calculus generated by both Frobenius algebras with trivial phases interpreted as:

$$
\left\llbracket\ 
\begin{tikzpicture}[scale=.84]
	\begin{pgfonlayer}{nodelayer}
		\node [style=none] (0) at (4, -0.5) {};
		\node [style=none] (1) at (3, -0.5) {};
		\node [style=none] (2) at (3.5, -0.75) {$\cdots$};
		\node [style=Z] (4) at (3.5, -1.25) {};
		\node [style=none] (6) at (3.5, -1.75) {$\cdots$};
		\node [style=none] (7) at (3, -2) {};
		\node [style=Z] (8) at (3.5, -1.25) {};
		\node [style=none] (9) at (4, -2) {};
		\node [style=none] (10) at (3.5, -2) {$n$};
		\node [style=none] (11) at (3.5, -0.5) {$m$};
	\end{pgfonlayer}
	\begin{pgfonlayer}{edgelayer}
		\draw [in=-90, out=56] (4) to (0.center);
		\draw [in=124, out=-90] (1.center) to (4);
		\draw [in=-124, out=90] (7.center) to (8);
		\draw [in=90, out=-56] (8) to (9.center);
	\end{pgfonlayer}
\end{tikzpicture}
\ \right\rrbracket
\propto
\sum_{i=0}^{p-1} | i, \ldots, i\rangle \langle i,\ldots, i|
\hspace*{1cm} 
\left\llbracket\ 
\begin{tikzpicture}[scale=.84]
	\begin{pgfonlayer}{nodelayer}
		\node [style=none] (0) at (4, -0.5) {};
		\node [style=none] (1) at (3, -0.5) {};
		\node [style=none] (2) at (3.5, -0.75) {$\cdots$};
		\node [style=X] (4) at (3.5, -1.25) {};
		\node [style=none] (6) at (3.5, -1.75) {$\cdots$};
		\node [style=none] (7) at (3, -2) {};
		\node [style=none] (8) at (3.5, -1.25) {};
		\node [style=none] (9) at (4, -2) {};
		\node [style=none] (10) at (3.5, -2) {$n$};
		\node [style=none] (11) at (3.5, -0.5) {$m$};
	\end{pgfonlayer}
	\begin{pgfonlayer}{edgelayer}
		\draw [in=-90, out=56] (4) to (0.center);
		\draw [in=124, out=-90] (1.center) to (4);
		\draw [in=-124, out=90] (7.center) to (8);
		\draw [in=90, out=-56] (8) to (9.center);
	\end{pgfonlayer}
\end{tikzpicture}
\ \right\rrbracket
\propto
\sum_{\sum  x_i = \sum y _j \mod p} | y_1 ,\ldots, y_n \rangle \langle  x_1,\ldots, x_n|
$$
\end{definition}


\begin{lemma}[\cite{cole}]
$\LinRel_{\F_p}$ is isomorphic to the $p$-dimensional qudit phase-free ZX-calculus modulo invertible scalars.
\end{lemma}
\begin{proof}

\end{proof}


We can go add some phases to get a bit more expressitivity:
\begin{definition}
TODO X PHASE ZX CALCULUS
\end{definition}
\begin{lemma}
$\Aff\Rel_{\F_p}$ is isomorphic to the $p$-dimensional qudit ZX-calculus with Pauli $X$ phases modulo invertible scalars.
\end{lemma}
This is given by the interpretation:
$$
\left\llbracket\ 
\begin{tikzpicture}[scale=.84]
	\begin{pgfonlayer}{nodelayer}
		\node [style=none] (0) at (4, -0.5) {};
		\node [style=none] (1) at (3, -0.5) {};
		\node [style=none] (2) at (3.5, -0.75) {$\cdots$};
		\node [style=Z] (4) at (3.5, -1.25) {};
		\node [style=none] (6) at (3.5, -1.75) {$\cdots$};
		\node [style=none] (7) at (3, -2) {};
		\node [style=Z] (8) at (3.5, -1.25) {};
		\node [style=none] (9) at (4, -2) {};
		\node [style=none] (10) at (3.5, -2) {$n$};
		\node [style=none] (11) at (3.5, -0.5) {$m$};
	\end{pgfonlayer}
	\begin{pgfonlayer}{edgelayer}
		\draw [in=-90, out=56] (4) to (0.center);
		\draw [in=124, out=-90] (1.center) to (4);
		\draw [in=-124, out=90] (7.center) to (8);
		\draw [in=90, out=-56] (8) to (9.center);
	\end{pgfonlayer}
\end{tikzpicture}
\ \right\rrbracket
\propto
\sum_{i=0}^{p-1} | i, \ldots, i\rangle \langle i,\ldots, i|
\hspace*{1cm} 
\left\llbracket\ 
\begin{tikzpicture}[scale=.84]
	\begin{pgfonlayer}{nodelayer}
		\node [style=none] (0) at (4, -0.5) {};
		\node [style=none] (1) at (3, -0.5) {};
		\node [style=none] (2) at (3.5, -0.75) {$\cdots$};
		\node [style=X] (4) at (3.5, -1.25) {$a$};
		\node [style=none] (6) at (3.5, -1.75) {$\cdots$};
		\node [style=none] (7) at (3, -2) {};
		\node [style=none] (8) at (3.5, -1.25) {};
		\node [style=none] (9) at (4, -2) {};
		\node [style=none] (10) at (3.5, -2) {$n$};
		\node [style=none] (11) at (3.5, -0.5) {$m$};
	\end{pgfonlayer}
	\begin{pgfonlayer}{edgelayer}
		\draw [in=-90, out=56] (4) to (0.center);
		\draw [in=124, out=-90] (1.center) to (4);
		\draw [in=-124, out=90] (7.center) to (8);
		\draw [in=90, out=-56] (8) to (9.center);
	\end{pgfonlayer}
\end{tikzpicture}
\ \right\rrbracket
\propto
\sum_{\sum  x_i = \sum y _j +a \mod p} | y_1 ,\ldots, y_n \rangle \langle  x_1,\ldots, x_n|
$$

%Talk about the alternative scaling that we use to simplify things so that phase free ZX is presented by spans of Hopf algebras


%density matrices


\begin{definition}
\label{def:cpm}

%Dagger category... equivalent to ioo compact closed conjugation 

Given a $\dag$-compact closed category $\X$, then  $\CPM(\X)$ is the quotient of ${\CoPara}(\X)$ by the congruence relation:
$$
(f,S) \sim (g,T) \iff
\begin{tikzpicture}
	\begin{pgfonlayer}{nodelayer}
		\node [style=none] (0) at (0.75, 11.75) {};
		\node [style=none] (1) at (0.75, 10.75) {};
		\node [style=map] (3) at (0.75, 10.75) {$f$};
		\node [style=map] (4) at (0.75, 11.75) {$f^\dag$};
		\node [style=none] (5) at (0.75, 10) {};
		\node [style=none] (6) at (0.75, 12.5) {};
		\node [style=none] (7) at (0, 12.5) {};
		\node [style=none] (8) at (0, 10) {};
	\end{pgfonlayer}
	\begin{pgfonlayer}{edgelayer}
		\draw (6.center) to (4);
		\draw (4) to (3);
		\draw (3) to (5.center);
		\draw [in=-90, out=135, looseness=0.75] (3) to (7.center);
		\draw [in=-135, out=90] (8.center) to (4);
	\end{pgfonlayer}
\end{tikzpicture}
=
\begin{tikzpicture}
	\begin{pgfonlayer}{nodelayer}
		\node [style=none] (0) at (0.75, 11.75) {};
		\node [style=none] (1) at (0.75, 10.75) {};
		\node [style=map] (3) at (0.75, 10.75) {$g$};
		\node [style=map] (4) at (0.75, 11.75) {$g^\dag$};
		\node [style=none] (5) at (0.75, 10) {};
		\node [style=none] (6) at (0.75, 12.5) {};
		\node [style=none] (7) at (0, 12.5) {};
		\node [style=none] (8) at (0, 10) {};
	\end{pgfonlayer}
	\begin{pgfonlayer}{edgelayer}
		\draw (6.center) to (4);
		\draw (4) to (3);
		\draw (3) to (5.center);
		\draw [in=-90, out=135, looseness=0.75] (3) to (7.center);
		\draw [in=-135, out=90] (8.center) to (4);
	\end{pgfonlayer}
\end{tikzpicture}
$$

This category has a $\dag$-compact closed structure which is inherited from $\X$ in the obvious way.

The image of doubling functor $\X\to \CPM(\X)$ sending $f \mapsto (f,I)$ takes maps in $\X$ to {\bf pure maps}. The maps which are not pure are called {\bf mixed}. The map $d_X=((u^L_X)^{-1}, X)$ is called the {\bf discarding map} on $X$.  

All maps can be obtained by composing pure maps with discard maps.  Any such factorization is called a {\bf dilation}.
\end{definition}



\begin{example}
$\CPM(\FHilb)$ is the dagger compact closed category of density matrices between finite dimensional Hilbert spaces.
\end{example}

Density matrices model mixed quantum circuits.  The circuits in the image of the doubling functor are interpreted as the rays of pure quantum processes, unexposed to a classical system.

Oftentimes, we will refer to the maps in  $\CPM(\X)$, not in terms of the individual representatives of the equivalence class; rather, we think of them in the ``doubled picture'' where the congruence lives so that a map $(f,S)$ is drawn as:
$$
\begin{tikzpicture}
	\begin{pgfonlayer}{nodelayer}
		\node [style=none] (1) at (1.5, 10.5) {};
		\node [style=map] (3) at (1.5, 10.5) {$f$};
		\node [style=map] (4) at (2.5, 10.5) {$(f^\dag)^*$};
		\node [style=none] (5) at (1.5, 9.75) {};
		\node [style=none] (6) at (2, 11.75) {};
		\node [style=none] (7) at (1, 11.75) {};
		\node [style=none] (8) at (2.5, 9.75) {};
	\end{pgfonlayer}
	\begin{pgfonlayer}{edgelayer}
		\draw [in=135, out=-90, looseness=1.25] (6.center) to (4);
		\draw [in=75, out=405, looseness=2.00] (4) to (3);
		\draw (3) to (5.center);
		\draw [in=-90, out=135, looseness=0.75] (3) to (7.center);
		\draw (8.center) to (4);
	\end{pgfonlayer}
\end{tikzpicture}
$$
Then composition of equivalence classes is composition in $\X$.  

There are several variations on the $\CPM$ construction which are defined using different congurence relations on ${\CoPara}(\X)$, where this ``doubled picture'' doesn't make sense. 

For example, thre is an infinite dimensional version which uses a different congruence. as the quotient we have given is not a congruence relation for general $\dag$-symmetric monoidal categories \cite{cpinf}.  There is a seperate equivalence relation which is defined in terms of universally quantifying over all of the maps which fit where we have drawn the braiding.


Similarly the discard construction \cite{discard} quotients by the congruence which freely discards isometries.  All three of these constructions are isomorphic when $\X=\FHilb$.

%
%
%\begin{remark}
%We can present $\CPM(\X)$ in terms of adding generators and relations to $\X$, regarding the morphisms of $\X$ as the pure maps: adding a mixing map for each object $X$ in $\X$, with represetative $d_X=((u^L_X)^{-1}, X)$.
%
%In the case of $\X=\FHilb$, $d_X$ is interpreted as quantum discarding.
%\end{remark}


\begin{definition}
Given an orthonormal basis $B$ in $\FHilb$, the projector $p_B$ onto this basis, is first given by copying in the basis and then discarding.  In the doubled picture:
$$
\begin{tikzpicture}
	\begin{pgfonlayer}{nodelayer}
		\node [style=none] (0) at (0, 0.25) {};
		\node [style=none] (1) at (0.75, 0.25) {};
		\node [style=Z] (2) at (0, 1) {};
		\node [style=Z] (3) at (0.75, 1) {};
		\node [style=none] (4) at (-0.5, 2) {};
		\node [style=none] (5) at (0.25, 2) {};
		\node [style=Z] (12) at (0.75, 1.75) {};
	\end{pgfonlayer}
	\begin{pgfonlayer}{edgelayer}
		\draw (0.center) to (2);
		\draw [in=-90, out=150, looseness=0.75] (2) to (4.center);
		\draw (1.center) to (3);
		\draw [in=270, out=90] (3) to (5.center);
		\draw [bend right=45, looseness=1.25] (3) to (12);
		\draw [in=30, out=-150] (12) to (2);
	\end{pgfonlayer}
\end{tikzpicture}
=
\begin{tikzpicture}
	\begin{pgfonlayer}{nodelayer}
		\node [style=none] (6) at (2.25, 0.5) {};
		\node [style=Z] (8) at (2.5, 1.25) {};
		\node [style=none] (10) at (2.25, 2) {};
		\node [style=none] (13) at (2.75, 0.5) {};
		\node [style=Z] (14) at (2.5, 1.25) {};
		\node [style=none] (15) at (2.75, 2) {};
	\end{pgfonlayer}
	\begin{pgfonlayer}{edgelayer}
		\draw [in=-120, out=90] (6.center) to (8);
		\draw [in=-90, out=135] (8) to (10.center);
		\draw [in=-60, out=90] (13.center) to (14);
		\draw [in=-90, out=45] (14) to (15.center);
	\end{pgfonlayer}
\end{tikzpicture}
$$
\end{definition}

Projectors are idempotent, so they can be slpit:

\begin{remark}
In the full subcategory of the Karoubi envelope of $\FHilb$ generated by splitting projectors onto orthonormal bases.  
The new objects which are created are regarded as classical systems.

 The projections $(X,p_B)\to (X,1_X)$ and injections $(X,1_X)\to (X,p_B)$ are interpreted as state preparation and measurement operations.
\end{remark}

SINGLE PICTURE AND DOUBLED PICTURE INTERACTION

\begin{lemma}
Meauring and preparing strongly complementary observables preserves no information (actually we only need the hopf and not the bialgebra).
\end{lemma}

We can prove the correctness of quantum teleportation.

\begin{example}

\end{example}




\chapter{The classical fragment of the ZH-calculus}
\section{Introduction}

In this chapter a complete set of identities is provided for the fragment, $\ZXA$, of the $\ZX$-calculus, generated by black and white spiders, the not gate and the {\sf and} gate. We show that this is a universal and complete presentation of ``qubit multirelations,'' or equivalently $2^n \times 2^m$ dimensional matrices over $\N$.
 To prove completeness and universality requires much exposition.  Along the way we show that the category of classical channels of a discrete inverse category is the Cartesian completion of that discrete inverse category.  We then show that the corresponding environment structure is precisely the free counit completion of the chosen Frobenius structure.  This allows us to present the Cartesian completion of, $\TOF$, the category generated by the Toffoli gate, $|1\rangle$ and $\langle 1|$ by only adding the $|+\rangle$ state and the unitality equation.  By freely adding both the unit and counit to $\TOF$, corresponding to $\sqrt{2}|+\rangle$ and  $\sqrt{2}\langle +|$, this yields an isomorphism with spans between ordinals $2^n$, $n\in \N$, or equivalently, ``qubit multirelations.''

The identities which are given by this two way translation are {\em almost} the union of the complete identities for Boolean functions \cite[Thm. 10]{lafont} (functions of type $\F_2^n \to \F_2$) and the identities for $\Span^\sim(\Mat(\F_2))$ \cite[Def. 5.1]{ihpub}.  These classes of circuits, and these identities for that matter, are nothing new; however, we provide a completeness result, as well as a structural account of how the full classical qubit fragment of $\FHilb$ can be obtained from adding discarding and codiscarding to the full classically reversible Boolean fragment.  In fact, some of these identities are presented in \cite[Chap. 5]{herrmann},  and they are used in the $\ZH$-calculus \cite{zh,zhpi}, as well as in some presentations of the $\ZX$-calculus with the triangle generator as a primitive \cite{munson2019note,ringZX}.  This is particularity unsurprising for the latter, \cite{ringZX}, where the author proves completeness of the $\ZX$-calculus over arbitrary semirings, which subsumes the completeness result herein.  Albeit, the presentation given here is substantially simpler.  It worth mentioning that $\ZXA$ is not a $\ZX*$-calculus in the sense of \cite{zxstar}, because the {\sf and} gate is not a spider.  $\ZXA$ should be instead though of as the ``classical fragment'' of the phase-free $\ZH$-calculus: retaining the monoid for ``and'' without $H$-boxes.   From this presentation only natural-number H-boxes can be derived.

We assume familiarity with the theory of  monoidal categories and categorical quantum mechanics.
Most of the paper will be devoted to reviewing the required categorical machinery of restriction and inverse categories, and developing it further, in order to prove the main result.  With all of mathematics reviewed and developed in generality, the desired result follows from abstract nonsense after a mechanical calculation. 

In Section \ref{sec:rest}, the theory of restriction categories and inverse categories is reviewed.  In Section \ref{sec:cpm}, we construct classical channels in the setting of discrete inverse categories, showing that the ``environment structures'' of the classical channels corresponds to adding counits to the base discrete inverse category.  Finally, in Section \ref{sec:ZXA}, we actually compute the (co)unit completion of $\TOF$.  We show that this category has a much more canonical presentation, $\ZXA$, in terms of interacting monoids/comonoids which very much resembles the $\ZH$-calculus.  We also show that this category is isomorphic to the category spans between ordinals $2^n$.





\section{Categorical quantum mechanics and completely positive maps}
\label{sec:cpm}
The $\sf CPM$ construction gives a notion of quantum channels for any $\dag$-compact closed category \cite{cpm}.
The \dag-Frobenius algebras in the base category induce idempotents in $\sf CPM$ corresponding to decohering quantum channels.  By considering the full subcategory of the Karoubi envelope whose objects are such idempotents one obtains the $\STOCH$ construction of \cite{coecke2016categories}: yielding classical channels between finite dimensional $C^*$-algebras when applied to $\FHilb$.   However, the $\sf CPM$ construction  can not be applied to $\Hilb$ in general because unlike $\FHilb$, it is not compact closed. 
The $\CP^\infty$ construction  \cite{coecke2016pictures} generalizes the $\sf CPM$ construction to (non compact closed) $\dag$-symmetric monoidal categories, by unbending the cups/caps and, identifying two super-maps  when they act the same on all positive test maps: recovering the usual notion of purely quantum channels.

%$$
%\begin{tikzpicture}
%	\begin{pgfonlayer}{nodelayer}
%		\node [style=none] (0) at (0.5, -0) {};
%		\node [style=none] (1) at (0.5, -0.75) {};
%		\node [style=X] (2) at (1.25, -0) {};
%		\node [style=X] (3) at (1.25, -0.75) {};
%		\node [style=Z] (4) at (2.75, -0) {};
%		\node [style=Z] (5) at (2.75, -0.75) {};
%		\node [style=none] (6) at (3.5, -0) {};
%		\node [style=none] (7) at (3.5, -0.75) {};
%		\node [style=map] (8) at (2, -0) {$f$};
%		\node [style=map] (9) at (2, -0.75) {$f_*$};
%	\end{pgfonlayer}
%	\begin{pgfonlayer}{edgelayer}
%		\draw [bend right=60, looseness=1.25] (3) to (2);
%		\draw (2) to (0.center);
%		\draw (1.center) to (3);
%		\draw (5) to (3);
%		\draw [bend right=60, looseness=1.50] (5) to (4);
%		\draw (4) to (6.center);
%		\draw (5) to (7.center);
%		\draw (4) to (2);
%		\draw [style=simple, in=-30, out=30, looseness=1.25] (9) to (8);
%	\end{pgfonlayer}
%\end{tikzpicture}
%\iff
%\begin{tikzpicture}
%	\begin{pgfonlayer}{nodelayer}
%		\node [style=none] (0) at (0.5, -0.5) {};
%		\node [style=X] (1) at (1.25, -0.5) {};
%		\node [style=Z] (2) at (3, -0.5) {};
%		\node [style=none] (3) at (3.75, -0.5) {};
%		\node [style=map] (4) at (2, -0) {$f$};
%		\node [style=map] (5) at (2, -1) {$f_*$};
%		\node [style=none] (6) at (2, -0) {};
%		\node [style=none] (7) at (2, -1) {};
%	\end{pgfonlayer}
%	\begin{pgfonlayer}{edgelayer}
%		\draw (1) to (0.center);
%		\draw (2) to (3.center);
%		\draw [style=simple, in=-60, out=180, looseness=1.00] (5) to (1);
%		\draw [style=simple, in=-120, out=0, looseness=1.00] (5) to (2);
%		\draw [style=simple, in=120, out=0, looseness=1.00] (4) to (2);
%		\draw [style=simple, in=180, out=60, looseness=1.00] (1) to (4);
%		\draw [style=simple, bend left=75, looseness=1.50] (6.center) to (7.center);
%	\end{pgfonlayer}
%\end{tikzpicture}
%$$

\begin{figure}



$$
\begin{tikzpicture}
	\begin{pgfonlayer}{nodelayer}
		\node [style=map] (0) at (0.25, 1.25) {$f$};
		\node [style=none] (1) at (0.25, 0.5) {};
		\node [style=none] (2) at (0, 2) {};
		\node [style=none] (3) at (0.5, 2) {};
	\end{pgfonlayer}
	\begin{pgfonlayer}{edgelayer}
		\draw [style=simple] (1.center) to (0);
		\draw [style=simple, in=-90, out=124] (0) to (2.center);
		\draw [style=simple, in=56, out=-90] (3.center) to (0);
	\end{pgfonlayer}
\end{tikzpicture}
;
\begin{tikzpicture}
	\begin{pgfonlayer}{nodelayer}
		\node [style=map] (0) at (0.25, 1.25) {$g$};
		\node [style=none] (1) at (0.25, 0.5) {};
		\node [style=none] (2) at (0, 2) {};
		\node [style=none] (3) at (0.5, 2) {};
	\end{pgfonlayer}
	\begin{pgfonlayer}{edgelayer}
		\draw [style=simple] (1.center) to (0);
		\draw [style=simple, in=-90, out=124] (0) to (2.center);
		\draw [style=simple, in=56, out=-90] (3.center) to (0);
	\end{pgfonlayer}
\end{tikzpicture}
:=
\begin{tikzpicture}
	\begin{pgfonlayer}{nodelayer}
		\node [style=map] (0) at (0.25, 1.25) {$f$};
		\node [style=none] (1) at (0.25, 0.5) {};
		\node [style=none] (2) at (0, 2) {};
		\node [style=none] (3) at (0.5, 2.75) {};
		\node [style=none] (4) at (0.5, 2.75) {};
		\node [style=none] (5) at (-0.25, 3.25) {};
		\node [style=map] (6) at (0, 2) {$g$};
		\node [style=otimes] (7) at (0.5, 2.75) {};
		\node [style=none] (8) at (0.5, 3.25) {};
	\end{pgfonlayer}
	\begin{pgfonlayer}{edgelayer}
		\draw [style=simple] (1.center) to (0);
		\draw [style=simple, in=-90, out=124] (0) to (2.center);
		\draw [style=simple, in=56, out=-90] (3.center) to (0);
		\draw [style=simple, in=-90, out=124] (6) to (5.center);
		\draw [style=simple, in=56, out=-120] (4.center) to (6);
		\draw [style=simple] (8.center) to (3.center);
	\end{pgfonlayer}
\end{tikzpicture}
\hspace*{.5cm}
\begin{tikzpicture}
	\begin{pgfonlayer}{nodelayer}
		\node [style=map] (0) at (0.25, 1) {$h$};
		\node [style=none] (1) at (0.25, 0.25) {};
		\node [style=none] (2) at (-0.25, 3) {};
		\node [style=map] (3) at (0.25, 2.25) {$h^\circ$};
		\node [style=none] (4) at (0.25, 3) {};
		\node [style=none] (5) at (-0.25, 0.25) {};
	\end{pgfonlayer}
	\begin{pgfonlayer}{edgelayer}
		\draw [style=simple] (1.center) to (0);
		\draw [style=simple, in=-90, out=124, looseness=0.75] (0) to (2.center);
		\draw [style=simple] (4.center) to (3);
		\draw [style=simple, in=90, out=-124, looseness=0.75] (3) to (5.center);
		\draw [bend right, looseness=0.75] (0) to (3);
	\end{pgfonlayer}
\end{tikzpicture}
=
\begin{tikzpicture}
	\begin{pgfonlayer}{nodelayer}
		\node [style=map] (0) at (0.25, 1.25) {$k$};
		\node [style=none] (1) at (0.25, 0.5) {};
		\node [style=none] (2) at (-0.25, 3.25) {};
		\node [style=map] (3) at (0.25, 2.5) {$k^\circ$};
		\node [style=none] (4) at (0.25, 3.25) {};
		\node [style=none] (5) at (-0.25, 0.5) {};
	\end{pgfonlayer}
	\begin{pgfonlayer}{edgelayer}
		\draw [style=simple] (1.center) to (0);
		\draw [style=simple, in=-90, out=124, looseness=0.75] (0) to (2.center);
		\draw [style=simple] (4.center) to (3);
		\draw [style=simple, in=90, out=-124, looseness=0.75] (3) to (5.center);
		\draw [bend right, looseness=0.75] (0) to (3);
	\end{pgfonlayer}
\end{tikzpicture}
\hspace*{.5cm}
\begin{tikzpicture}
	\begin{pgfonlayer}{nodelayer}
		\node [style=X] (0) at (0, 1) {};
		\node [style=none] (1) at (0, 0.5) {};
		\node [style=none] (2) at (-0.25, 1.75) {};
		\node [style=none] (3) at (0.25, 1.75) {};
	\end{pgfonlayer}
	\begin{pgfonlayer}{edgelayer}
		\draw (1.center) to (0);
		\draw [in=-90, out=108] (0) to (2.center);
		\draw [in=72, out=-90, looseness=0.75] (3.center) to (0);
	\end{pgfonlayer}
\end{tikzpicture}
$$

\caption{
Composition of representatives $f;g$;  equivalence relation $h\sim k$; decoherence map.}
\label{fig:kraus}
\end{figure}


To generalize the $\STOCH$ construction to  \dag-semi-Frobenius algebras, one must combine the  %\linebreak[4]
 $\STOCH$ and $\CP^\infty$ constructions, as the compact closed structure is no longer taken for granted.   We show that the Cartesian completion is the same as first applying a modified version of the  $\CP^\infty$ construction (without quantifying over all test maps, as seen in Figure \ref{fig:kraus}) to a discrete inverse category and then taking the full subcategory of the Karoubi envelope whose objects are  the decoherence maps \footnote{Although, composition in this version of the ${\sf CP}^\infty$ construction, without universally quantifying over test maps, when applied to a discrete inverse category is not obviously well-defined unless the base category embeds in a compact closed category.}.  The following Lemma is needed to prove this fact:




\begin{lemma}
\label{lem:latching}

Given two parallel maps $X\xrightarrow{f,g} Y\otimes Z$ in a discrete inverse category:

$$
\begin{tikzpicture}
	\begin{pgfonlayer}{nodelayer}
		\node [style=map] (0) at (0, 1.25) {$f$};
		\node [style=none] (1) at (0, 0.5) {};
		\node [style=none] (2) at (-0.25, 2) {};
		\node [style=none] (3) at (0.25, 2) {};
	\end{pgfonlayer}
	\begin{pgfonlayer}{edgelayer}
		\draw (1.center) to (0);
		\draw [in=60, out=-90] (3.center) to (0);
		\draw [in=-90, out=120] (0) to (2.center);
	\end{pgfonlayer}
\end{tikzpicture}
=
\begin{tikzpicture}
	\begin{pgfonlayer}{nodelayer}
		\node [style=map] (0) at (0, 1.25) {$g$};
		\node [style=none] (1) at (0, 0.5) {};
		\node [style=none] (2) at (-0.25, 2) {};
		\node [style=none] (3) at (0.25, 2) {};
	\end{pgfonlayer}
	\begin{pgfonlayer}{edgelayer}
		\draw (1.center) to (0);
		\draw [in=60, out=-90] (3.center) to (0);
		\draw [in=-90, out=120] (0) to (2.center);
	\end{pgfonlayer}
\end{tikzpicture}
\iff
\begin{tikzpicture}
	\begin{pgfonlayer}{nodelayer}
		\node [style=map] (0) at (0, 1.25) {$f$};
		\node [style=X] (1) at (-0.5, 2.25) {};
		\node [style=none] (2) at (-0.5, 3) {};
		\node [style=none] (3) at (0.25, 3) {};
		\node [style=none] (4) at (-0.75, 0.5) {};
		\node [style=none] (5) at (0, 0.5) {};
	\end{pgfonlayer}
	\begin{pgfonlayer}{edgelayer}
		\draw (5.center) to (0);
		\draw (0) to (1);
		\draw (1) to (2.center);
		\draw [in=75, out=-90] (3.center) to (0);
		\draw [in=90, out=-105] (1) to (4.center);
	\end{pgfonlayer}
\end{tikzpicture}
=
\begin{tikzpicture}
	\begin{pgfonlayer}{nodelayer}
		\node [style=map] (0) at (0, 1.25) {$g$};
		\node [style=X] (1) at (-0.5, 2.25) {};
		\node [style=none] (2) at (-0.5, 3) {};
		\node [style=none] (3) at (0.25, 3) {};
		\node [style=none] (4) at (-0.75, 0.5) {};
		\node [style=none] (5) at (0, 0.5) {};
	\end{pgfonlayer}
	\begin{pgfonlayer}{edgelayer}
		\draw (5.center) to (0);
		\draw (0) to (1);
		\draw (1) to (2.center);
		\draw [in=75, out=-90] (3.center) to (0);
		\draw [in=90, out=-105] (1) to (4.center);
	\end{pgfonlayer}
\end{tikzpicture}
$$
\end{lemma}

\begin{proof}
The one direction is trivial, for the other direction:

\begin{align*}
\begin{tikzpicture}
	\begin{pgfonlayer}{nodelayer}
		\node [style=map] (0) at (0, 1.25) {$f$};
		\node [style=none] (1) at (0, 0.5) {};
		\node [style=none] (2) at (-0.25, 2) {};
		\node [style=none] (3) at (0.25, 2) {};
	\end{pgfonlayer}
	\begin{pgfonlayer}{edgelayer}
		\draw (1.center) to (0);
		\draw [in=60, out=-90] (3.center) to (0);
		\draw [in=-90, out=120] (0) to (2.center);
	\end{pgfonlayer}
\end{tikzpicture}
=
\begin{tikzpicture}
	\begin{pgfonlayer}{nodelayer}
		\node [style=map] (0) at (0, 1.25) {$f$};
		\node [style=none] (1) at (0, 0.5) {};
		\node [style=X] (2) at (-0.25, 2) {};
		\node [style=X] (3) at (0.25, 2) {};
		\node [style=X] (4) at (0.25, 2.75) {};
		\node [style=X] (5) at (-0.25, 2.75) {};
		\node [style=none] (6) at (-0.25, 3.25) {};
		\node [style=none] (7) at (0.25, 3.25) {};
	\end{pgfonlayer}
	\begin{pgfonlayer}{edgelayer}
		\draw (1.center) to (0);
		\draw [in=60, out=-90] (3) to (0);
		\draw [in=-90, out=120] (0) to (2);
		\draw (6.center) to (5);
		\draw (4) to (7.center);
		\draw [in=-60, out=60] (3) to (4);
		\draw [in=120, out=-120] (5) to (2);
		\draw [in=60, out=-60] (5) to (2);
		\draw [in=-120, out=120] (3) to (4);
	\end{pgfonlayer}
\end{tikzpicture}
=
\begin{tikzpicture}
	\begin{pgfonlayer}{nodelayer}
		\node [style=X] (0) at (0, 1) {};
		\node [style=X] (1) at (0.5, 2.75) {};
		\node [style=none] (2) at (-0.5, 3.25) {};
		\node [style=none] (3) at (0.5, 3.25) {};
		\node [style=none] (4) at (0, 0.5) {};
		\node [style=X] (5) at (-0.5, 2.75) {};
		\node [style=map] (6) at (0.5, 1.75) {$f$};
		\node [style=map] (7) at (-0.5, 1.75) {$f$};
	\end{pgfonlayer}
	\begin{pgfonlayer}{edgelayer}
		\draw (2.center) to (5);
		\draw (1) to (3.center);
		\draw (4.center) to (0);
		\draw [in=-90, out=135] (0) to (7);
		\draw [in=45, out=-90] (6) to (0);
		\draw (7) to (1);
		\draw [in=120, out=-120] (5) to (7);
		\draw (6) to (5);
		\draw [in=60, out=-60] (1) to (6);
	\end{pgfonlayer}
\end{tikzpicture}
=
\begin{tikzpicture}
	\begin{pgfonlayer}{nodelayer}
		\node [style=X] (0) at (0, 1) {};
		\node [style=X] (1) at (0.25, 3.25) {};
		\node [style=none] (2) at (-0.75, 3.75) {};
		\node [style=none] (3) at (0.25, 3.75) {};
		\node [style=none] (4) at (0, 0.5) {};
		\node [style=X] (5) at (-0.75, 3.25) {};
		\node [style=map] (6) at (0.25, 1.75) {$f$};
		\node [style=map] (7) at (-0.25, 2.5) {$f$};
		\node [style=none] (8) at (-0.75, 2.25) {};
	\end{pgfonlayer}
	\begin{pgfonlayer}{edgelayer}
		\draw (2.center) to (5);
		\draw (1) to (3.center);
		\draw (4.center) to (0);
		\draw [in=-90, out=135] (0) to (7);
		\draw [in=45, out=-90] (6) to (0);
		\draw [in=-124, out=60] (7) to (1);
		\draw [in=120, out=-60] (5) to (7);
		\draw [in=60, out=-60] (1) to (6);
		\draw [in=-90, out=120, looseness=0.75] (6) to (8.center);
		\draw [in=-90, out=90] (8.center) to (5);
	\end{pgfonlayer}
\end{tikzpicture}
=
\begin{tikzpicture}
	\begin{pgfonlayer}{nodelayer}
		\node [style=X] (0) at (0, 1) {};
		\node [style=X] (1) at (0.25, 3.25) {};
		\node [style=none] (2) at (-0.75, 3.75) {};
		\node [style=none] (3) at (0.25, 3.75) {};
		\node [style=none] (4) at (0, 0.5) {};
		\node [style=X] (5) at (-0.75, 3.25) {};
		\node [style=map] (6) at (0.25, 1.75) {$f$};
		\node [style=map] (7) at (-0.25, 2.5) {$g$};
		\node [style=none] (8) at (-0.75, 2.25) {};
	\end{pgfonlayer}
	\begin{pgfonlayer}{edgelayer}
		\draw (2.center) to (5);
		\draw (1) to (3.center);
		\draw (4.center) to (0);
		\draw [in=-90, out=135] (0) to (7);
		\draw [in=45, out=-90] (6) to (0);
		\draw [in=-124, out=60] (7) to (1);
		\draw [in=120, out=-60] (5) to (7);
		\draw [in=60, out=-60] (1) to (6);
		\draw [in=-90, out=120, looseness=0.75] (6) to (8.center);
		\draw [in=-90, out=90] (8.center) to (5);
	\end{pgfonlayer}
\end{tikzpicture}
=
\begin{tikzpicture}
	\begin{pgfonlayer}{nodelayer}
		\node [style=X] (0) at (0, 1) {};
		\node [style=X] (1) at (0.5, 2.75) {};
		\node [style=none] (2) at (-0.5, 3.25) {};
		\node [style=none] (3) at (0.5, 3.25) {};
		\node [style=none] (4) at (0, 0.5) {};
		\node [style=X] (5) at (-0.5, 2.75) {};
		\node [style=map] (6) at (0.5, 1.75) {$f$};
		\node [style=map] (7) at (-0.5, 1.75) {$g$};
	\end{pgfonlayer}
	\begin{pgfonlayer}{edgelayer}
		\draw (2.center) to (5);
		\draw (1) to (3.center);
		\draw (4.center) to (0);
		\draw [in=-90, out=135] (0) to (7);
		\draw [in=45, out=-90] (6) to (0);
		\draw (7) to (1);
		\draw [in=120, out=-120] (5) to (7);
		\draw (6) to (5);
		\draw [in=60, out=-60] (1) to (6);
	\end{pgfonlayer}
\end{tikzpicture}
=
\begin{tikzpicture}
	\begin{pgfonlayer}{nodelayer}
		\node [style=X] (0) at (0, 1) {};
		\node [style=X] (1) at (0.5, 2.75) {};
		\node [style=none] (2) at (-0.5, 3.25) {};
		\node [style=none] (3) at (0.5, 3.25) {};
		\node [style=none] (4) at (0, 0.5) {};
		\node [style=X] (5) at (-0.5, 2.75) {};
		\node [style=map] (6) at (0.5, 1.75) {$g$};
		\node [style=map] (7) at (-0.5, 1.75) {$g$};
	\end{pgfonlayer}
	\begin{pgfonlayer}{edgelayer}
		\draw (2.center) to (5);
		\draw (1) to (3.center);
		\draw (4.center) to (0);
		\draw [in=-90, out=135] (0) to (7);
		\draw [in=45, out=-90] (6) to (0);
		\draw (7) to (1);
		\draw [in=120, out=-120] (5) to (7);
		\draw (6) to (5);
		\draw [in=60, out=-60] (1) to (6);
	\end{pgfonlayer}
\end{tikzpicture}
=
\begin{tikzpicture}
	\begin{pgfonlayer}{nodelayer}
		\node [style=map] (0) at (0, 1.25) {$g$};
		\node [style=none] (1) at (0, 0.5) {};
		\node [style=none] (2) at (-0.25, 2) {};
		\node [style=none] (3) at (0.25, 2) {};
	\end{pgfonlayer}
	\begin{pgfonlayer}{edgelayer}
		\draw (1.center) to (0);
		\draw [in=60, out=-90] (3.center) to (0);
		\draw [in=-90, out=120] (0) to (2.center);
	\end{pgfonlayer}
\end{tikzpicture}
\end{align*}
\end{proof}





\begin{lemma}
\label{theorem:cpstartheorem}
Given two maps $X \xrightarrow{f} Y\otimes S$ and $X \xrightarrow{g} Y\otimes T$, in a discrete inverse category:

\begin{align*}
\begin{tikzpicture}
	\begin{pgfonlayer}{nodelayer}
		\node [style=map] (0) at (0.5, 1.75) {$g$};
		\node [style=none] (1) at (0.5, 1) {};
		\node [style=map] (2) at (0.5, 3.25) {$g^\circ$};
		\node [style=map] (3) at (0.5, 4) {$f$};
		\node [style=X] (4) at (0.25, 2.5) {};
		\node [style=X] (5) at (0.25, 4.75) {};
		\node [style=none] (6) at (0.25, 5.25) {};
		\node [style=none] (7) at (0.75, 5.25) {};
	\end{pgfonlayer}
	\begin{pgfonlayer}{edgelayer}
		\draw (1.center) to (0);
		\draw [in=-90, out=124] (0) to (4);
		\draw (4) to (2);
		\draw [in=60, out=-60] (2) to (0);
		\draw [in=-120, out=120] (4) to (5);
		\draw (5) to (6.center);
		\draw [in=60, out=-90] (7.center) to (3);
		\draw (3) to (5);
		\draw (3) to (2);
	\end{pgfonlayer}
\end{tikzpicture}
=
\begin{tikzpicture}
	\begin{pgfonlayer}{nodelayer}
		\node [style=map] (0) at (0.5, 1.75) {$f$};
		\node [style=none] (1) at (0.5, 1) {};
		\node [style=none] (2) at (0.25, 2.5) {};
		\node [style=none] (3) at (0.75, 2.5) {};
	\end{pgfonlayer}
	\begin{pgfonlayer}{edgelayer}
		\draw (1.center) to (0);
		\draw [in=-90, out=60] (0) to (3.center);
		\draw [in=120, out=-90] (2.center) to (0);
	\end{pgfonlayer}
\end{tikzpicture}
\iff
\begin{tikzpicture}
	\begin{pgfonlayer}{nodelayer}
		\node [style=X] (0) at (0, 2.25) {};
		\node [style=X] (1) at (0, 3) {};
		\node [style=map] (2) at (0.5, 1.75) {$f$};
		\node [style=map] (3) at (0.5, 3.5) {$f^\circ$};
		\node [style=none] (4) at (-0.25, 1) {};
		\node [style=none] (5) at (0.5, 1) {};
		\node [style=none] (6) at (-0.25, 4.25) {};
		\node [style=none] (7) at (0.5, 4.25) {};
	\end{pgfonlayer}
	\begin{pgfonlayer}{edgelayer}
		\draw (5.center) to (2);
		\draw (2) to (0);
		\draw [in=90, out=-101] (0) to (4.center);
		\draw (0) to (1);
		\draw [in=-90, out=101] (1) to (6.center);
		\draw (7.center) to (3);
		\draw (3) to (1);
		\draw [in=-75, out=75] (2) to (3);
	\end{pgfonlayer}
\end{tikzpicture}
=
\begin{tikzpicture}
	\begin{pgfonlayer}{nodelayer}
		\node [style=X] (0) at (0, 2.25) {};
		\node [style=X] (1) at (0, 3) {};
		\node [style=map] (2) at (0.5, 1.75) {$g$};
		\node [style=map] (3) at (0.5, 3.5) {$g^\circ$};
		\node [style=none] (4) at (-0.25, 1) {};
		\node [style=none] (5) at (0.5, 1) {};
		\node [style=none] (6) at (-0.25, 4.25) {};
		\node [style=none] (7) at (0.5, 4.25) {};
	\end{pgfonlayer}
	\begin{pgfonlayer}{edgelayer}
		\draw (5.center) to (2);
		\draw (2) to (0);
		\draw [in=90, out=-101] (0) to (4.center);
		\draw (0) to (1);
		\draw [in=-90, out=101] (1) to (6.center);
		\draw (7.center) to (3);
		\draw (3) to (1);
		\draw [in=-75, out=75] (2) to (3);
	\end{pgfonlayer}
\end{tikzpicture}
\iff
\begin{tikzpicture}
	\begin{pgfonlayer}{nodelayer}
		\node [style=X] (0) at (0, 2.25) {};
		\node [style=X] (1) at (0, 3) {};
		\node [style=map] (2) at (0.5, 1.75) {$f$};
		\node [style=map] (3) at (0.5, 3.5) {$f^\circ$};
		\node [style=none] (4) at (-0.25, 0.75) {};
		\node [style=none] (5) at (-0.25, 4.5) {};
		\node [style=X] (6) at (1, 1.25) {};
		\node [style=X] (7) at (1, 4) {};
		\node [style=none] (8) at (1, 4.5) {};
		\node [style=none] (9) at (1, 0.75) {};
	\end{pgfonlayer}
	\begin{pgfonlayer}{edgelayer}
		\draw (2) to (0);
		\draw [in=90, out=-101] (0) to (4.center);
		\draw (0) to (1);
		\draw [in=-90, out=101] (1) to (5.center);
		\draw (3) to (1);
		\draw [in=-75, out=75] (2) to (3);
		\draw (6) to (2);
		\draw [bend right=15] (6) to (7);
		\draw (7) to (3);
		\draw (7) to (8.center);
		\draw (6) to (9.center);
	\end{pgfonlayer}
\end{tikzpicture}
=
\begin{tikzpicture}
	\begin{pgfonlayer}{nodelayer}
		\node [style=X] (0) at (0, 2.25) {};
		\node [style=X] (1) at (0, 3) {};
		\node [style=map] (2) at (0.5, 1.75) {$g$};
		\node [style=map] (3) at (0.5, 3.5) {$g^\circ$};
		\node [style=none] (4) at (-0.25, 0.75) {};
		\node [style=none] (5) at (-0.25, 4.5) {};
		\node [style=X] (6) at (1, 1.25) {};
		\node [style=X] (7) at (1, 4) {};
		\node [style=none] (8) at (1, 4.5) {};
		\node [style=none] (9) at (1, 0.75) {};
	\end{pgfonlayer}
	\begin{pgfonlayer}{edgelayer}
		\draw (2) to (0);
		\draw [in=90, out=-101] (0) to (4.center);
		\draw (0) to (1);
		\draw [in=-90, out=101] (1) to (5.center);
		\draw (3) to (1);
		\draw [in=-75, out=75] (2) to (3);
		\draw (6) to (2);
		\draw [bend right=15] (6) to (7);
		\draw (7) to (3);
		\draw (7) to (8.center);
		\draw (6) to (9.center);
	\end{pgfonlayer}
\end{tikzpicture}
\end{align*}
\end{lemma}


\begin{proof}
First note:

\begin{align*}
\begin{tikzpicture}
	\begin{pgfonlayer}{nodelayer}
		\node [style=X] (0) at (0, 2.25) {};
		\node [style=X] (1) at (0, 3) {};
		\node [style=map] (2) at (0.5, 1.75) {$f$};
		\node [style=map] (3) at (0.5, 3.5) {$f^\circ$};
		\node [style=none] (4) at (-0.25, 1.25) {};
		\node [style=none] (5) at (-0.25, 4) {};
		\node [style=none] (6) at (0.5, 4) {};
		\node [style=none] (7) at (0.5, 1.25) {};
	\end{pgfonlayer}
	\begin{pgfonlayer}{edgelayer}
		\draw (2) to (0);
		\draw [in=90, out=-101] (0) to (4.center);
		\draw (0) to (1);
		\draw [in=-90, out=101] (1) to (5.center);
		\draw (3) to (1);
		\draw [in=-75, out=75] (2) to (3);
		\draw (2) to (7.center);
		\draw (3) to (6.center);
	\end{pgfonlayer}
\end{tikzpicture}
=
\begin{tikzpicture}
	\begin{pgfonlayer}{nodelayer}
		\node [style=X] (0) at (0, 2.25) {};
		\node [style=X] (1) at (0, 3) {};
		\node [style=map] (2) at (0.5, 1) {$f$};
		\node [style=map] (3) at (0.5, 4.25) {$f^\circ$};
		\node [style=none] (4) at (-0.25, 0.5) {};
		\node [style=none] (5) at (-0.25, 4.75) {};
		\node [style=none] (6) at (0.5, 4.75) {};
		\node [style=none] (7) at (0.5, 0.5) {};
		\node [style=X] (8) at (0.25, 1.75) {};
		\node [style=X] (9) at (0.25, 3.5) {};
	\end{pgfonlayer}
	\begin{pgfonlayer}{edgelayer}
		\draw [in=90, out=-101] (0) to (4.center);
		\draw (0) to (1);
		\draw [in=-90, out=101] (1) to (5.center);
		\draw [in=-75, out=75] (2) to (3);
		\draw (2) to (7.center);
		\draw (3) to (6.center);
		\draw (2) to (8);
		\draw (8) to (0);
		\draw [bend right=15] (8) to (9);
		\draw (9) to (3);
		\draw (9) to (1);
	\end{pgfonlayer}
\end{tikzpicture}
=
\begin{tikzpicture}
	\begin{pgfonlayer}{nodelayer}
		\node [style=X] (0) at (0, 2.25) {};
		\node [style=X] (1) at (0, 3) {};
		\node [style=map] (2) at (0.5, 1) {$f$};
		\node [style=map] (3) at (0.5, 4.25) {$f^\circ$};
		\node [style=none] (4) at (-0.25, 0.5) {};
		\node [style=none] (5) at (-0.25, 4.75) {};
		\node [style=none] (6) at (0.5, 4.75) {};
		\node [style=none] (7) at (0.5, 0.5) {};
		\node [style=X] (8) at (0.25, 1.75) {};
		\node [style=X] (9) at (0.25, 3.5) {};
		\node [style=X] (10) at (0.75, 1.75) {};
		\node [style=X] (11) at (0.75, 3.5) {};
	\end{pgfonlayer}
	\begin{pgfonlayer}{edgelayer}
		\draw [in=90, out=-101] (0) to (4.center);
		\draw (0) to (1);
		\draw [in=-90, out=101] (1) to (5.center);
		\draw (2) to (7.center);
		\draw (3) to (6.center);
		\draw (2) to (8);
		\draw (8) to (0);
		\draw [bend right=15] (8) to (9);
		\draw (9) to (3);
		\draw (9) to (1);
		\draw (2) to (10);
		\draw [in=-135, out=135, looseness=1.25] (10) to (11);
		\draw (11) to (3);
		\draw [in=75, out=-75] (11) to (10);
	\end{pgfonlayer}
\end{tikzpicture}
=
\begin{tikzpicture}
	\begin{pgfonlayer}{nodelayer}
		\node [style=X] (0) at (1.5, 2.25) {};
		\node [style=none] (1) at (1.25, 0.5) {};
		\node [style=none] (2) at (2.5, 0.5) {};
		\node [style=none] (3) at (1.25, 4.75) {};
		\node [style=X] (4) at (1.5, 3) {};
		\node [style=none] (5) at (2.5, 4.75) {};
		\node [style=map] (6) at (2, 1.75) {$f$};
		\node [style=map] (7) at (2.75, 1.75) {$f$};
		\node [style=map] (8) at (2, 3.5) {$f^\circ$};
		\node [style=map] (9) at (2.75, 3.5) {$f^\circ$};
		\node [style=X] (10) at (2.5, 1) {};
		\node [style=X] (11) at (2.5, 4.25) {};
	\end{pgfonlayer}
	\begin{pgfonlayer}{edgelayer}
		\draw [in=90, out=-101] (0) to (1.center);
		\draw (0) to (4);
		\draw [in=-90, out=101] (4) to (3.center);
		\draw (2.center) to (10);
		\draw (10) to (6);
		\draw (6) to (0);
		\draw (8) to (4);
		\draw (8) to (11);
		\draw (11) to (9);
		\draw [in=120, out=-120] (9) to (7);
		\draw (7) to (10);
		\draw [in=-60, out=60] (7) to (9);
		\draw [in=75, out=-75] (8) to (6);
		\draw (5.center) to (11);
	\end{pgfonlayer}
\end{tikzpicture}
=
\begin{tikzpicture}
	\begin{pgfonlayer}{nodelayer}
		\node [style=X] (0) at (1.5, 3.75) {};
		\node [style=none] (1) at (1.25, 0.5) {};
		\node [style=none] (2) at (2.5, 0.5) {};
		\node [style=none] (3) at (1.25, 6.25) {};
		\node [style=X] (4) at (1.5, 4.5) {};
		\node [style=none] (5) at (2.5, 6.25) {};
		\node [style=map] (6) at (2, 3.25) {$f$};
		\node [style=map] (7) at (2, 5) {$f^\circ$};
		\node [style=X] (8) at (2.5, 1) {};
		\node [style=X] (9) at (2.5, 5.75) {};
		\node [style=map] (10) at (2, 2.5) {$f^\circ$};
		\node [style=map] (11) at (2, 1.75) {$f$};
	\end{pgfonlayer}
	\begin{pgfonlayer}{edgelayer}
		\draw [in=90, out=-101] (0) to (1.center);
		\draw (0) to (4);
		\draw [in=-90, out=101] (4) to (3.center);
		\draw (2.center) to (8);
		\draw (6) to (0);
		\draw (7) to (4);
		\draw (7) to (9);
		\draw [in=75, out=-75] (7) to (6);
		\draw [in=120, out=-120] (10) to (11);
		\draw [in=-60, out=60] (11) to (10);
		\draw (8) to (11);
		\draw [in=-75, out=75, looseness=0.75] (8) to (9);
		\draw (9) to (5.center);
		\draw (6) to (10);
	\end{pgfonlayer}
\end{tikzpicture}
=
\begin{tikzpicture}
	\begin{pgfonlayer}{nodelayer}
		\node [style=X] (0) at (1.5, 2.25) {};
		\node [style=none] (1) at (1.25, 0.5) {};
		\node [style=none] (2) at (2.5, 0.5) {};
		\node [style=none] (3) at (1.25, 4.75) {};
		\node [style=X] (4) at (1.5, 3) {};
		\node [style=none] (5) at (2.5, 4.75) {};
		\node [style=map] (6) at (2, 1.75) {$f$};
		\node [style=map] (7) at (2, 3.5) {$f^\circ$};
		\node [style=X] (8) at (2.5, 1) {};
		\node [style=X] (9) at (2.5, 4.25) {};
	\end{pgfonlayer}
	\begin{pgfonlayer}{edgelayer}
		\draw [in=90, out=-101] (0) to (1.center);
		\draw (0) to (4);
		\draw [in=-90, out=101] (4) to (3.center);
		\draw (2.center) to (8);
		\draw (6) to (0);
		\draw (7) to (4);
		\draw (7) to (9);
		\draw [in=75, out=-75] (7) to (6);
		\draw [in=-75, out=75, looseness=0.75] (8) to (9);
		\draw (9) to (5.center);
		\draw (8) to (6);
	\end{pgfonlayer}
\end{tikzpicture}
\end{align*}

So that we only have to prove the first biconditional.
Suppose that the left hand side holds, then:

\begin{align*}
\begin{tikzpicture}
	\begin{pgfonlayer}{nodelayer}
		\node [style=X] (0) at (0, 2.25) {};
		\node [style=X] (1) at (0, 3) {};
		\node [style=map] (2) at (0.5, 1.75) {$f$};
		\node [style=map] (3) at (0.5, 3.5) {$f^\circ$};
		\node [style=none] (4) at (-0.25, 1) {};
		\node [style=none] (5) at (0.5, 1) {};
		\node [style=none] (6) at (-0.25, 4.25) {};
		\node [style=none] (7) at (0.5, 4.25) {};
	\end{pgfonlayer}
	\begin{pgfonlayer}{edgelayer}
		\draw (5.center) to (2);
		\draw (2) to (0);
		\draw [in=90, out=-101] (0) to (4.center);
		\draw (0) to (1);
		\draw [in=-90, out=101] (1) to (6.center);
		\draw (7.center) to (3);
		\draw (3) to (1);
		\draw [in=-75, out=75] (2) to (3);
	\end{pgfonlayer}
\end{tikzpicture}
=
\begin{tikzpicture}
	\begin{pgfonlayer}{nodelayer}
		\node [style=map] (0) at (0.5, 1.75) {$g$};
		\node [style=none] (1) at (0.5, 1) {};
		\node [style=map] (2) at (0.5, 3.25) {$g^\circ$};
		\node [style=map] (3) at (0.5, 4) {$f$};
		\node [style=X] (4) at (0.25, 2.5) {};
		\node [style=X] (5) at (0.25, 4.75) {};
		\node [style=map] (6) at (0.5, 9.75) {$g^\circ$};
		\node [style=map] (7) at (0.5, 8.25) {$g$};
		\node [style=map] (8) at (0.5, 7.5) {$f^\circ$};
		\node [style=X] (9) at (0.25, 9) {};
		\node [style=X] (10) at (0.25, 6.75) {};
		\node [style=none] (11) at (0.5, 10.5) {};
		\node [style=X] (12) at (0, 5.5) {};
		\node [style=X] (13) at (0, 6) {};
		\node [style=none] (14) at (-0.25, 1) {};
		\node [style=none] (15) at (-0.25, 10.5) {};
		\node [style=none] (16) at (-0.25, 4.75) {};
		\node [style=none] (17) at (-0.25, 6.75) {};
	\end{pgfonlayer}
	\begin{pgfonlayer}{edgelayer}
		\draw (1.center) to (0);
		\draw [in=-90, out=124] (0) to (4);
		\draw (4) to (2);
		\draw [in=60, out=-60] (2) to (0);
		\draw [in=-120, out=120] (4) to (5);
		\draw (3) to (5);
		\draw (3) to (2);
		\draw (11.center) to (6);
		\draw [in=90, out=-124] (6) to (9);
		\draw (9) to (7);
		\draw [in=-60, out=60] (7) to (6);
		\draw [in=120, out=-120] (9) to (10);
		\draw (8) to (10);
		\draw (8) to (7);
		\draw [in=-75, out=75, looseness=0.75] (3) to (8);
		\draw [in=72, out=-90] (10) to (13);
		\draw (13) to (12);
		\draw [in=90, out=-72] (12) to (5);
		\draw [in=90, out=-108] (12) to (16.center);
		\draw (16.center) to (14.center);
		\draw [in=-90, out=108] (13) to (17.center);
		\draw (17.center) to (15.center);
	\end{pgfonlayer}
\end{tikzpicture}
=
\begin{tikzpicture}
	\begin{pgfonlayer}{nodelayer}
		\node [style=map] (0) at (2.75, 4.25) {$f$};
		\node [style=none] (1) at (2.25, 10.25) {};
		\node [style=none] (2) at (2.25, 0.5) {};
		\node [style=none] (3) at (2.75, 0.5) {};
		\node [style=X] (4) at (2.5, 7.25) {};
		\node [style=X] (5) at (2.5, 5) {};
		\node [style=none] (6) at (2.75, 10.25) {};
		\node [style=X] (7) at (2.5, 2.75) {};
		\node [style=map] (8) at (2.75, 5.75) {$f^\circ$};
		\node [style=map] (9) at (2.75, 3.5) {$g^\circ$};
		\node [style=map] (10) at (2.75, 6.5) {$g$};
		\node [style=map] (11) at (2.75, 1.25) {$g$};
		\node [style=map] (12) at (2.75, 9.5) {$g^\circ$};
		\node [style=X] (13) at (2.5, 8.75) {};
		\node [style=X] (14) at (2.5, 2) {};
		\node [style=X] (15) at (2.5, 8) {};
	\end{pgfonlayer}
	\begin{pgfonlayer}{edgelayer}
		\draw (3.center) to (11);
		\draw (7) to (9);
		\draw [in=60, out=-60] (9) to (11);
		\draw (0) to (9);
		\draw (6.center) to (12);
		\draw (4) to (10);
		\draw [in=-60, out=60] (10) to (12);
		\draw [in=120, out=-120] (4) to (5);
		\draw (8) to (5);
		\draw (8) to (10);
		\draw [in=-60, out=60] (0) to (8);
		\draw [in=-99, out=90] (2.center) to (14);
		\draw (14) to (7);
		\draw (14) to (11);
		\draw (4) to (13);
		\draw (13) to (12);
		\draw [in=-90, out=99] (13) to (1.center);
		\draw [in=-90, out=114] (0) to (5);
		\draw [bend left=45, looseness=0.50] (7) to (15);
	\end{pgfonlayer}
\end{tikzpicture}
=
\begin{tikzpicture}
	\begin{pgfonlayer}{nodelayer}
		\node [style=X] (0) at (-0.5, 5.75) {};
		\node [style=none] (1) at (-0.75, 7.25) {};
		\node [style=map] (2) at (-0.25, 4.25) {$g$};
		\node [style=map] (3) at (-0.25, 1.25) {$g$};
		\node [style=none] (4) at (-0.75, 0.5) {};
		\node [style=none] (5) at (-0.25, 0.5) {};
		\node [style=none] (6) at (-0.25, 7.25) {};
		\node [style=map] (7) at (-0.25, 3.5) {$g^\circ$};
		\node [style=X] (8) at (-0.5, 2) {};
		\node [style=map] (9) at (-0.25, 6.5) {$g^\circ$};
		\node [style=X] (10) at (-0.5, 5) {};
		\node [style=X] (11) at (-0.5, 2.75) {};
	\end{pgfonlayer}
	\begin{pgfonlayer}{edgelayer}
		\draw (5.center) to (3);
		\draw (11) to (7);
		\draw [in=60, out=-60] (7) to (3);
		\draw (6.center) to (9);
		\draw [in=-60, out=60] (2) to (9);
		\draw [in=-99, out=90] (4.center) to (8);
		\draw (8) to (11);
		\draw (8) to (3);
		\draw (0) to (9);
		\draw [in=-90, out=99] (0) to (1.center);
		\draw (10) to (2);
		\draw (10) to (0);
		\draw (2) to (7);
		\draw [in=-120, out=120] (11) to (10);
	\end{pgfonlayer}
\end{tikzpicture}
=
\begin{tikzpicture}
	\begin{pgfonlayer}{nodelayer}
		\node [style=map] (0) at (2.75, 8) {$g^\circ$};
		\node [style=none] (1) at (2.75, 0.5) {};
		\node [style=X] (2) at (2.25, 7.25) {};
		\node [style=X] (3) at (2.25, 2) {};
		\node [style=none] (4) at (2, 0.5) {};
		\node [style=none] (5) at (2, 8.75) {};
		\node [style=map] (6) at (2.75, 1.25) {$g$};
		\node [style=X] (7) at (2.25, 2.75) {};
		\node [style=none] (8) at (2.75, 8.75) {};
		\node [style=X] (9) at (2.25, 6.5) {};
		\node [style=map] (10) at (3, 5) {$g$};
		\node [style=map] (11) at (3, 4.25) {$g^\circ$};
		\node [style=X] (12) at (2.5, 3.25) {};
		\node [style=X] (13) at (3, 3.25) {};
		\node [style=X] (14) at (2.5, 6) {};
		\node [style=X] (15) at (3, 6) {};
	\end{pgfonlayer}
	\begin{pgfonlayer}{edgelayer}
		\draw (1.center) to (6);
		\draw (8.center) to (0);
		\draw [in=-99, out=90] (4.center) to (3);
		\draw (3) to (7);
		\draw (3) to (6);
		\draw (2) to (0);
		\draw [in=-90, out=99] (2) to (5.center);
		\draw (9) to (2);
		\draw [in=-120, out=120] (7) to (9);
		\draw (10) to (11);
		\draw [in=-90, out=76] (6) to (13);
		\draw (7) to (12);
		\draw [bend left] (12) to (14);
		\draw [bend right, looseness=1.25] (15) to (13);
		\draw (14) to (9);
		\draw [in=-76, out=90] (15) to (0);
		\draw (12) to (11);
		\draw (13) to (11);
		\draw (10) to (14);
		\draw [in=90, out=-90] (15) to (10);
	\end{pgfonlayer}
\end{tikzpicture}
=
\begin{tikzpicture}
	\begin{pgfonlayer}{nodelayer}
		\node [style=map] (0) at (3, 7) {$g^\circ$};
		\node [style=none] (1) at (3, 0.5) {};
		\node [style=X] (2) at (2.5, 6.25) {};
		\node [style=X] (3) at (2.5, 2) {};
		\node [style=none] (4) at (2.25, 0.5) {};
		\node [style=none] (5) at (2.25, 7.75) {};
		\node [style=map] (6) at (3, 1.25) {$g$};
		\node [style=none] (7) at (3, 7.75) {};
		\node [style=map] (8) at (3, 4.5) {$g$};
		\node [style=map] (9) at (3, 3.75) {$g^\circ$};
		\node [style=X] (10) at (2.5, 2.75) {};
		\node [style=X] (11) at (3, 2.75) {};
		\node [style=X] (12) at (2.5, 5.5) {};
		\node [style=X] (13) at (3, 5.5) {};
	\end{pgfonlayer}
	\begin{pgfonlayer}{edgelayer}
		\draw (1.center) to (6);
		\draw (7.center) to (0);
		\draw [in=-99, out=90] (4.center) to (3);
		\draw (3) to (6);
		\draw (2) to (0);
		\draw [in=-90, out=99] (2) to (5.center);
		\draw (8) to (9);
		\draw (6) to (11);
		\draw [bend left] (10) to (12);
		\draw [bend right, looseness=1.25] (13) to (11);
		\draw (13) to (0);
		\draw (10) to (9);
		\draw (11) to (9);
		\draw (8) to (12);
		\draw [in=90, out=-90] (13) to (8);
		\draw (3) to (10);
		\draw (12) to (2);
	\end{pgfonlayer}
\end{tikzpicture}
=
\begin{tikzpicture}
	\begin{pgfonlayer}{nodelayer}
		\node [style=map] (0) at (0, 7.25) {$g^\circ$};
		\node [style=map] (1) at (0.25, 4) {$g$};
		\node [style=X] (2) at (-0.5, 6.5) {};
		\node [style=none] (3) at (-0.75, 8) {};
		\node [style=map] (4) at (0, 1.25) {$g$};
		\node [style=none] (5) at (-0.75, 5) {};
		\node [style=X] (6) at (-0.25, 5) {};
		\node [style=X] (7) at (-0.25, 2.25) {};
		\node [style=none] (8) at (0, 8) {};
		\node [style=X] (9) at (-0.5, 5.75) {};
		\node [style=X] (10) at (0.25, 5) {};
		\node [style=none] (11) at (0, 0.5) {};
		\node [style=map] (12) at (0.25, 3.25) {$g^\circ$};
		\node [style=X] (13) at (0.25, 2.25) {};
		\node [style=none] (14) at (-0.75, 0.5) {};
	\end{pgfonlayer}
	\begin{pgfonlayer}{edgelayer}
		\draw (11.center) to (4);
		\draw (8.center) to (0);
		\draw (2) to (0);
		\draw [in=-90, out=99] (2) to (3.center);
		\draw (9) to (2);
		\draw (1) to (12);
		\draw [in=-90, out=76] (4) to (13);
		\draw [bend left] (7) to (6);
		\draw [bend right, looseness=1.25] (10) to (13);
		\draw [in=-60, out=90] (6) to (9);
		\draw [in=-76, out=90] (10) to (0);
		\draw (7) to (12);
		\draw (13) to (12);
		\draw (1) to (6);
		\draw [in=90, out=-90] (10) to (1);
		\draw [in=104, out=-90] (7) to (4);
		\draw [in=90, out=-120] (9) to (5.center);
		\draw (14.center) to (5.center);
	\end{pgfonlayer}
\end{tikzpicture}
=
\begin{tikzpicture}
	\begin{pgfonlayer}{nodelayer}
		\node [style=map] (0) at (0, 5.25) {$g^\circ$};
		\node [style=X] (1) at (-0.5, 4.5) {};
		\node [style=none] (2) at (-0.75, 6) {};
		\node [style=map] (3) at (0, 1.5) {$g$};
		\node [style=none] (4) at (-0.75, 3) {};
		\node [style=none] (5) at (0, 6) {};
		\node [style=X] (6) at (-0.5, 3.75) {};
		\node [style=none] (7) at (0, 1) {};
		\node [style=none] (8) at (-0.75, 1) {};
		\node [style=map] (9) at (0, 2.25) {$g^\circ$};
		\node [style=map] (10) at (0, 3) {$g$};
	\end{pgfonlayer}
	\begin{pgfonlayer}{edgelayer}
		\draw (7.center) to (3);
		\draw (5.center) to (0);
		\draw (1) to (0);
		\draw [in=-90, out=99] (1) to (2.center);
		\draw (6) to (1);
		\draw [in=90, out=-120] (6) to (4.center);
		\draw (8.center) to (4.center);
		\draw [in=-120, out=120] (3) to (9);
		\draw [in=60, out=-60] (9) to (3);
		\draw (9) to (10);
		\draw (10) to (6);
		\draw [in=75, out=-75] (0) to (10);
	\end{pgfonlayer}
\end{tikzpicture}
=
\begin{tikzpicture}
	\begin{pgfonlayer}{nodelayer}
		\node [style=X] (0) at (0, 2.25) {};
		\node [style=X] (1) at (0, 3) {};
		\node [style=map] (2) at (0.5, 1.75) {$g$};
		\node [style=map] (3) at (0.5, 3.5) {$g^\circ$};
		\node [style=none] (4) at (-0.25, 1) {};
		\node [style=none] (5) at (0.5, 1) {};
		\node [style=none] (6) at (-0.25, 4.25) {};
		\node [style=none] (7) at (0.5, 4.25) {};
	\end{pgfonlayer}
	\begin{pgfonlayer}{edgelayer}
		\draw (5.center) to (2);
		\draw (2) to (0);
		\draw [in=90, out=-101] (0) to (4.center);
		\draw (0) to (1);
		\draw [in=-90, out=101] (1) to (6.center);
		\draw (7.center) to (3);
		\draw (3) to (1);
		\draw [in=-75, out=75] (2) to (3);
	\end{pgfonlayer}
\end{tikzpicture}
\end{align*}

Conversely, suppose that the right hand side holds.  Then:

\begin{align*}
\begin{tikzpicture}
	\begin{pgfonlayer}{nodelayer}
		\node [style=map] (0) at (0.5, 1.75) {$g$};
		\node [style=none] (1) at (0.5, 1) {};
		\node [style=map] (2) at (0.5, 3.25) {$g^\circ$};
		\node [style=map] (3) at (0.5, 4) {$f$};
		\node [style=X] (4) at (0.25, 2.5) {};
		\node [style=none] (5) at (0.75, 5.75) {};
		\node [style=none] (6) at (0.25, 5.75) {};
		\node [style=X] (7) at (0.25, 4.75) {};
		\node [style=X] (8) at (0.25, 5.25) {};
		\node [style=none] (9) at (-0.25, 4.25) {};
		\node [style=none] (10) at (-0.25, 1) {};
	\end{pgfonlayer}
	\begin{pgfonlayer}{edgelayer}
		\draw (1.center) to (0);
		\draw [in=-90, out=124] (0) to (4);
		\draw (4) to (2);
		\draw [in=60, out=-60] (2) to (0);
		\draw [in=60, out=-90] (5.center) to (3);
		\draw (3) to (2);
		\draw [in=-120, out=120] (4) to (7);
		\draw (3) to (7);
		\draw (10.center) to (9.center);
		\draw [in=-135, out=90] (9.center) to (8);
		\draw (8) to (6.center);
		\draw (8) to (7);
	\end{pgfonlayer}
\end{tikzpicture}
=
\begin{tikzpicture}
	\begin{pgfonlayer}{nodelayer}
		\node [style=map] (0) at (0.5, 1.25) {$g$};
		\node [style=none] (1) at (0.5, 0.5) {};
		\node [style=map] (2) at (0.5, 3.25) {$g^\circ$};
		\node [style=map] (3) at (0.5, 4) {$f$};
		\node [style=X] (4) at (0.25, 2.5) {};
		\node [style=none] (5) at (0.75, 5.25) {};
		\node [style=none] (6) at (0.25, 5.25) {};
		\node [style=X] (7) at (0.25, 4.75) {};
		\node [style=none] (8) at (0, 0.5) {};
		\node [style=X] (9) at (0.25, 2) {};
	\end{pgfonlayer}
	\begin{pgfonlayer}{edgelayer}
		\draw (1.center) to (0);
		\draw (4) to (2);
		\draw [in=60, out=-60] (2) to (0);
		\draw [in=60, out=-90] (5.center) to (3);
		\draw (3) to (2);
		\draw [in=-120, out=120] (4) to (7);
		\draw (3) to (7);
		\draw (9) to (0);
		\draw (9) to (4);
		\draw [in=90, out=-120] (9) to (8.center);
		\draw (7) to (6.center);
	\end{pgfonlayer}
\end{tikzpicture}
=
\begin{tikzpicture}
	\begin{pgfonlayer}{nodelayer}
		\node [style=map] (0) at (0.5, 1.25) {$f$};
		\node [style=none] (1) at (0.5, 0.5) {};
		\node [style=map] (2) at (0.5, 3.25) {$f^\circ$};
		\node [style=map] (3) at (0.5, 4) {$f$};
		\node [style=X] (4) at (0.25, 2.5) {};
		\node [style=none] (5) at (0.75, 5.25) {};
		\node [style=none] (6) at (0.25, 5.25) {};
		\node [style=X] (7) at (0.25, 4.75) {};
		\node [style=none] (8) at (0, 0.5) {};
		\node [style=X] (9) at (0.25, 2) {};
	\end{pgfonlayer}
	\begin{pgfonlayer}{edgelayer}
		\draw (1.center) to (0);
		\draw (4) to (2);
		\draw [in=60, out=-60] (2) to (0);
		\draw [in=60, out=-90] (5.center) to (3);
		\draw (3) to (2);
		\draw [in=-120, out=120] (4) to (7);
		\draw (3) to (7);
		\draw (9) to (0);
		\draw (9) to (4);
		\draw [in=90, out=-120] (9) to (8.center);
		\draw (7) to (6.center);
	\end{pgfonlayer}
\end{tikzpicture}
=
\begin{tikzpicture}
	\begin{pgfonlayer}{nodelayer}
		\node [style=map] (0) at (2.5, 1.25) {$f$};
		\node [style=X] (1) at (2, 6.25) {};
		\node [style=none] (2) at (1.75, 0.5) {};
		\node [style=none] (3) at (2.75, 6.75) {};
		\node [style=none] (4) at (2.5, 0.5) {};
		\node [style=X] (5) at (2, 2.5) {};
		\node [style=X] (6) at (2, 2) {};
		\node [style=none] (7) at (2, 6.75) {};
		\node [style=map] (8) at (2.75, 4.75) {$f$};
		\node [style=map] (9) at (2.75, 4) {$f^\circ$};
		\node [style=X] (10) at (2.25, 5.75) {};
		\node [style=X] (11) at (2.25, 3) {};
		\node [style=X] (12) at (2.75, 5.75) {};
		\node [style=X] (13) at (2.75, 3) {};
	\end{pgfonlayer}
	\begin{pgfonlayer}{edgelayer}
		\draw (4.center) to (0);
		\draw [in=-120, out=120] (5) to (1);
		\draw (6) to (0);
		\draw (6) to (5);
		\draw [in=90, out=-120] (6) to (2.center);
		\draw (1) to (7.center);
		\draw (8) to (9);
		\draw (11) to (5);
		\draw [in=74, out=-90] (13) to (0);
		\draw [in=-120, out=120] (11) to (10);
		\draw (10) to (1);
		\draw (3.center) to (12);
		\draw [in=120, out=-120, looseness=1.25] (12) to (13);
		\draw (9) to (11);
		\draw (8) to (10);
		\draw (9) to (13);
		\draw (8) to (12);
	\end{pgfonlayer}
\end{tikzpicture}
=
\begin{tikzpicture}
	\begin{pgfonlayer}{nodelayer}
		\node [style=map] (0) at (2.75, 1.25) {$f$};
		\node [style=none] (1) at (2, 0.5) {};
		\node [style=none] (2) at (2.75, 6.25) {};
		\node [style=none] (3) at (2.75, 0.5) {};
		\node [style=X] (4) at (2.25, 2) {};
		\node [style=none] (5) at (2.25, 6.25) {};
		\node [style=map] (6) at (2.75, 4.5) {$f$};
		\node [style=map] (7) at (2.75, 3.75) {$f^\circ$};
		\node [style=X] (8) at (2.25, 5.5) {};
		\node [style=X] (9) at (2.25, 2.75) {};
		\node [style=X] (10) at (2.75, 5.5) {};
		\node [style=X] (11) at (2.75, 2.75) {};
	\end{pgfonlayer}
	\begin{pgfonlayer}{edgelayer}
		\draw (3.center) to (0);
		\draw (4) to (0);
		\draw [in=90, out=-120] (4) to (1.center);
		\draw (6) to (7);
		\draw (11) to (0);
		\draw [in=-120, out=120] (9) to (8);
		\draw (2.center) to (10);
		\draw [in=120, out=-120, looseness=1.25] (10) to (11);
		\draw (7) to (9);
		\draw (6) to (8);
		\draw (7) to (11);
		\draw (6) to (10);
		\draw (9) to (4);
		\draw (8) to (5.center);
	\end{pgfonlayer}
\end{tikzpicture}
=
\begin{tikzpicture}
	\begin{pgfonlayer}{nodelayer}
		\node [style=map] (0) at (2.5, 2) {$f$};
		\node [style=none] (1) at (1.75, 1.25) {};
		\node [style=none] (2) at (2.75, 6.75) {};
		\node [style=none] (3) at (2.5, 1.25) {};
		\node [style=none] (4) at (2.25, 6.75) {};
		\node [style=map] (5) at (2.75, 4.5) {$f$};
		\node [style=map] (6) at (2.75, 3.75) {$f^\circ$};
		\node [style=X] (7) at (2.25, 5.5) {};
		\node [style=X] (8) at (2.25, 2.75) {};
		\node [style=X] (9) at (2.75, 5.5) {};
		\node [style=X] (10) at (2.75, 2.75) {};
		\node [style=none] (11) at (1.75, 5.5) {};
		\node [style=X] (12) at (2.25, 6.25) {};
	\end{pgfonlayer}
	\begin{pgfonlayer}{edgelayer}
		\draw (3.center) to (0);
		\draw (5) to (6);
		\draw [in=60, out=-90] (10) to (0);
		\draw [in=-120, out=120] (8) to (7);
		\draw (2.center) to (9);
		\draw [in=120, out=-120, looseness=1.25] (9) to (10);
		\draw (6) to (8);
		\draw (5) to (7);
		\draw (6) to (10);
		\draw (5) to (9);
		\draw (7) to (4.center);
		\draw [in=90, out=-135] (12) to (11.center);
		\draw (11.center) to (1.center);
		\draw [in=120, out=-90] (8) to (0);
	\end{pgfonlayer}
\end{tikzpicture}
=
\begin{tikzpicture}
	\begin{pgfonlayer}{nodelayer}
		\node [style=map] (0) at (2.5, 2) {$f$};
		\node [style=none] (1) at (2, 1.25) {};
		\node [style=none] (2) at (2.75, 4.75) {};
		\node [style=none] (3) at (2.5, 1.25) {};
		\node [style=none] (4) at (2.25, 4.75) {};
		\node [style=map] (5) at (2.5, 3.5) {$f$};
		\node [style=map] (6) at (2.5, 2.75) {$f^\circ$};
		\node [style=none] (7) at (2, 3.5) {};
		\node [style=X] (8) at (2.25, 4.25) {};
	\end{pgfonlayer}
	\begin{pgfonlayer}{edgelayer}
		\draw (3.center) to (0);
		\draw (5) to (6);
		\draw [in=90, out=-135] (8) to (7.center);
		\draw (7.center) to (1.center);
		\draw [bend left, looseness=1.25] (0) to (6);
		\draw [bend left, looseness=1.25] (6) to (0);
		\draw (5) to (8);
		\draw (8) to (4.center);
		\draw [in=60, out=-90] (2.center) to (5);
	\end{pgfonlayer}
\end{tikzpicture}
=
\begin{tikzpicture}
	\begin{pgfonlayer}{nodelayer}
		\node [style=none] (0) at (2, 2.75) {};
		\node [style=none] (1) at (2.75, 4.75) {};
		\node [style=none] (2) at (2.5, 2.75) {};
		\node [style=none] (3) at (2.25, 4.75) {};
		\node [style=map] (4) at (2.5, 3.5) {$f$};
		\node [style=none] (5) at (2, 3.5) {};
		\node [style=X] (6) at (2.25, 4.25) {};
	\end{pgfonlayer}
	\begin{pgfonlayer}{edgelayer}
		\draw [in=90, out=-135, looseness=0.75] (6) to (5.center);
		\draw (5.center) to (0.center);
		\draw (4) to (6);
		\draw (6) to (3.center);
		\draw [in=60, out=-90] (1.center) to (4);
		\draw (4) to (2.center);
	\end{pgfonlayer}
\end{tikzpicture}
\end{align*}

Thus, by Lemma \ref{lem:latching}:
$$
\begin{tikzpicture}
	\begin{pgfonlayer}{nodelayer}
		\node [style=map] (0) at (0.5, 1.75) {$g$};
		\node [style=none] (1) at (0.5, 1) {};
		\node [style=map] (2) at (0.5, 3.25) {$g^\circ$};
		\node [style=map] (3) at (0.5, 4) {$f$};
		\node [style=X] (4) at (0.25, 2.5) {};
		\node [style=X] (5) at (0.25, 4.75) {};
		\node [style=none] (6) at (0.25, 5.25) {};
		\node [style=none] (7) at (0.75, 5.25) {};
	\end{pgfonlayer}
	\begin{pgfonlayer}{edgelayer}
		\draw (1.center) to (0);
		\draw [in=-90, out=124] (0) to (4);
		\draw (4) to (2);
		\draw [in=60, out=-60] (2) to (0);
		\draw [in=-120, out=120] (4) to (5);
		\draw (5) to (6.center);
		\draw [in=60, out=-90] (7.center) to (3);
		\draw (3) to (5);
		\draw (3) to (2);
	\end{pgfonlayer}
\end{tikzpicture}
=
\begin{tikzpicture}
	\begin{pgfonlayer}{nodelayer}
		\node [style=map] (0) at (0.5, 1.75) {$f$};
		\node [style=none] (1) at (0.5, 1) {};
		\node [style=none] (2) at (0.25, 2.5) {};
		\node [style=none] (3) at (0.75, 2.5) {};
	\end{pgfonlayer}
	\begin{pgfonlayer}{edgelayer}
		\draw (1.center) to (0);
		\draw [in=-90, out=60] (0) to (3.center);
		\draw [in=120, out=-90] (2.center) to (0);
	\end{pgfonlayer}
\end{tikzpicture}
$$


\end{proof}


%\subsection{Environment structures}
\label{sec:env}

The natural question arises: can we characterize classical channels in this setting, algebraically in terms of a discarding morphism, without performing any doubling.  In other words, is there some notion of ``environment structure'' \cite{coecke2010environment} for the {\em classical} channels of discrete inverse categories:


\begin{definition}
Given a discrete inverse category $\X$, define the counital completion of $\X$, $c(\X)$ to have the same objects and maps of $\X$, except with a freely adjoined counit $!_X:X\to I$ to the chosen semi-Frobenius algebra on $X$, for each object in $\X$ compatible with the monoidal structure.
\end{definition}

\begin{lemma}
$c(\X)$ is a discrete Cartesian restriction category.
\end{lemma}
\begin{proof}
This is clearly a counital copy category, with a restriction terminal object given by the tensor unit.  Moreover, because the Frobenius structure is special, it is also discrete.
\end{proof}



\begin{lemma}
\label{lemma:envstruct}
Given a discrete inverse category $\X$, $c(\X)$ and $\tilde \X$ are isomorphic as discrete Cartesian restriction categories.
\end{lemma}

\begin{proof}
Define an identity on objects functor $F:c(\X)\to \tilde \X$ in the obvious way, sending the counits to the ancillary space.
Similarly, define an identity on objects functor from $G:\tilde \X \to c(\X)$ given by plugging counits into the ancillary space.
These maps are clearly inverses to each other and preserve discrete Cartesian restriction structure; however, once again we mush show that they are actually  functors.


To see  that $F$ is a functor, it suffices to observe that every object in  $\tilde \X $ is equipped with a counital Frobenius algebra, compatible with the monoidal structure, where the unit is in the image of the freely adjoined counit under $F$.


%%%%%%%%%%%%%%%%%%%%%%%%%%%%%


To prove that $G$ is a functor, take some $(f,S)\sim (g,T)$ in $\tilde \X$.
Therefore, in $\tilde \X$, since the Frobenius structure is counital:
$$
\begin{tikzpicture}
	\begin{pgfonlayer}{nodelayer}
		\node [style=map] (0) at (0, 3.25) {$f^\circ$};
		\node [style=X] (1) at (-0.5, 2.25) {};
		\node [style=map] (2) at (0, 1.25) {$f$};
		\node [style=none] (3) at (0, 4.25) {};
		\node [style=none] (4) at (-1, 0.5) {};
		\node [style=none] (5) at (0, 0.5) {};
		\node [style=none] (6) at (0, 3.75) {};
		\node [style=none] (7) at (0.5, 4.25) {};
	\end{pgfonlayer}
	\begin{pgfonlayer}{edgelayer}
		\draw [style=simple] (0) to (3.center);
		\draw [style=simple] (1) to (2);
		\draw [style=simple, in=90, out=-104] (1) to (4.center);
		\draw [style=simple] (5.center) to (2);
		\draw [style=simple, in=60, out=-60, looseness=0.75] (0) to (2);
		\draw [in=-117, out=90] (1) to (0);
		\draw [style=dashed, in=30, out=-90] (7.center) to (6.center);
	\end{pgfonlayer}
\end{tikzpicture}
\sim
\begin{tikzpicture}
	\begin{pgfonlayer}{nodelayer}
		\node [style=map] (0) at (0, 4.25) {$f^\circ$};
		\node [style=X] (1) at (-0.5, 3.25) {};
		\node [style=X] (2) at (-0.5, 2.5) {};
		\node [style=map] (3) at (0, 1.5) {$f$};
		\node [style=none] (4) at (0, 5.25) {};
		\node [style=none] (5) at (-1, 0.5) {};
		\node [style=none] (6) at (0, 0.5) {};
		\node [style=none] (7) at (0.5, 5.25) {};
		\node [style=none] (8) at (-0.75, 4.5) {};
	\end{pgfonlayer}
	\begin{pgfonlayer}{edgelayer}
		\draw [style=simple] (1) to (0);
		\draw [style=simple] (0) to (4.center);
		\draw [style=simple] (1) to (2);
		\draw [style=simple] (2) to (3);
		\draw [style=simple, in=90, out=-104] (2) to (5.center);
		\draw [style=simple] (6.center) to (3);
		\draw [style=simple, in=60, out=-60, looseness=0.75] (0) to (3);
		\draw [style=simple, in=90, out=-90, looseness=0.75] (7.center) to (8.center);
		\draw [style=simple, in=-90, out=101] (1) to (8.center);
	\end{pgfonlayer}
\end{tikzpicture}
=
\begin{tikzpicture}
	\begin{pgfonlayer}{nodelayer}
		\node [style=map] (0) at (0, 4.25) {$g^\circ$};
		\node [style=X] (1) at (-0.5, 3.25) {};
		\node [style=X] (2) at (-0.5, 2.5) {};
		\node [style=map] (3) at (0, 1.5) {$g$};
		\node [style=none] (4) at (0, 5.25) {};
		\node [style=none] (5) at (-1, 0.5) {};
		\node [style=none] (6) at (0, 0.5) {};
		\node [style=none] (7) at (0.5, 5.25) {};
		\node [style=none] (8) at (-0.75, 4.5) {};
	\end{pgfonlayer}
	\begin{pgfonlayer}{edgelayer}
		\draw [style=simple] (1) to (0);
		\draw [style=simple] (0) to (4.center);
		\draw [style=simple] (1) to (2);
		\draw [style=simple] (2) to (3);
		\draw [style=simple, in=90, out=-104] (2) to (5.center);
		\draw [style=simple] (6.center) to (3);
		\draw [style=simple, in=60, out=-60, looseness=0.75] (0) to (3);
		\draw [style=simple, in=90, out=-90, looseness=0.75] (7.center) to (8.center);
		\draw [style=simple, in=-90, out=101] (1) to (8.center);
	\end{pgfonlayer}
\end{tikzpicture}
\sim
\begin{tikzpicture}
	\begin{pgfonlayer}{nodelayer}
		\node [style=map] (0) at (0, 3.25) {$g^\circ$};
		\node [style=X] (1) at (-0.5, 2.25) {};
		\node [style=map] (2) at (0, 1.25) {$g$};
		\node [style=none] (3) at (0, 4.25) {};
		\node [style=none] (4) at (-1, 0.5) {};
		\node [style=none] (5) at (0, 0.5) {};
		\node [style=none] (6) at (0, 3.75) {};
		\node [style=none] (7) at (0.5, 4.25) {};
	\end{pgfonlayer}
	\begin{pgfonlayer}{edgelayer}
		\draw [style=simple] (0) to (3.center);
		\draw [style=simple] (1) to (2);
		\draw [style=simple, in=90, out=-104] (1) to (4.center);
		\draw [style=simple] (5.center) to (2);
		\draw [style=simple, in=60, out=-60, looseness=0.75] (0) to (2);
		\draw [in=-117, out=90] (1) to (0);
		\draw [style=dashed, in=30, out=-90] (7.center) to (6.center);
	\end{pgfonlayer}
\end{tikzpicture}
$$


However, since the functor $\X\to \tilde \X $ is faithful by Lemma \ref{lemma:xtildefaithful}, using the alternate equivalence relation of $\tilde \X$ by  Lemma \ref{theorem:cpstartheorem}, we have that in $\X$:

$$
\begin{tikzpicture}
	\begin{pgfonlayer}{nodelayer}
		\node [style=map] (0) at (0, 3.25) {$f^\circ$};
		\node [style=X] (1) at (-0.5, 2.25) {};
		\node [style=map] (2) at (0, 1.25) {$f$};
		\node [style=none] (3) at (0, 4.25) {};
		\node [style=none] (4) at (-1, 0.5) {};
		\node [style=none] (5) at (0, 0.5) {};
	\end{pgfonlayer}
	\begin{pgfonlayer}{edgelayer}
		\draw [style=simple] (0) to (3.center);
		\draw [style=simple] (1) to (2);
		\draw [style=simple, in=90, out=-104] (1) to (4.center);
		\draw [style=simple] (5.center) to (2);
		\draw [style=simple, in=60, out=-60, looseness=0.75] (0) to (2);
		\draw [in=-117, out=90] (1) to (0);
	\end{pgfonlayer}
\end{tikzpicture}
=
\begin{tikzpicture}
	\begin{pgfonlayer}{nodelayer}
		\node [style=map] (0) at (0, 3.5) {$g^\circ$};
		\node [style=X] (1) at (-0.5, 2.5) {};
		\node [style=map] (2) at (0, 1.5) {$g$};
		\node [style=none] (3) at (0, 4.5) {};
		\node [style=none] (4) at (-1, 0.5) {};
		\node [style=none] (5) at (0, 0.5) {};
	\end{pgfonlayer}
	\begin{pgfonlayer}{edgelayer}
		\draw [style=simple] (0) to (3.center);
		\draw [style=simple] (1) to (2);
		\draw [style=simple, in=90, out=-104] (1) to (4.center);
		\draw [style=simple] (5.center) to (2);
		\draw [style=simple, in=60, out=-60, looseness=0.75] (0) to (2);
		\draw [in=-117, out=90] (1) to (0);
	\end{pgfonlayer}
\end{tikzpicture}
\hspace*{.2cm}\text{and thus}\hspace*{.1cm}
\begin{tikzpicture}
	\begin{pgfonlayer}{nodelayer}
		\node [style=map] (0) at (0, 1.5) {$f$};
		\node [style=X] (1) at (-0.5, 2.5) {};
		\node [style=map] (2) at (0, 3.5) {$f^\circ$};
		\node [style=none] (3) at (0, 0.5) {};
		\node [style=none] (4) at (-1, 4.5) {};
		\node [style=none] (5) at (0, 4.5) {};
	\end{pgfonlayer}
	\begin{pgfonlayer}{edgelayer}
		\draw [style=simple] (0) to (3.center);
		\draw [style=simple] (1) to (2);
		\draw [style=simple, in=-90, out=104] (1) to (4.center);
		\draw [style=simple] (5.center) to (2);
		\draw [style=simple, in=-60, out=60, looseness=0.75] (0) to (2);
		\draw [in=117, out=-90] (1) to (0);
	\end{pgfonlayer}
\end{tikzpicture}
=
\begin{tikzpicture}
	\begin{pgfonlayer}{nodelayer}
		\node [style=map] (0) at (0, 1.5) {$g$};
		\node [style=X] (1) at (-0.5, 2.5) {};
		\node [style=map] (2) at (0, 3.5) {$g^\circ$};
		\node [style=none] (3) at (0, 0.5) {};
		\node [style=none] (4) at (-1, 4.25) {};
		\node [style=none] (5) at (0, 4.25) {};
	\end{pgfonlayer}
	\begin{pgfonlayer}{edgelayer}
		\draw [style=simple] (0) to (3.center);
		\draw [style=simple] (1) to (2);
		\draw [style=simple, in=-90, out=104] (1) to (4.center);
		\draw [style=simple] (5.center) to (2);
		\draw [style=simple, in=-60, out=60, looseness=0.75] (0) to (2);
		\draw [in=117, out=-90] (1) to (0);
	\end{pgfonlayer}
\end{tikzpicture}
$$


Therefore in $c(\X)$:

\begin{align*}
\begin{tikzpicture}
	\begin{pgfonlayer}{nodelayer}
		\node [style=map] (0) at (0, 1.5) {$f$};
		\node [style=X] (1) at (-0.5, 2.5) {};
		\node [style=map] (2) at (0, 3.5) {$f^\circ$};
		\node [style=none] (3) at (0, 0.5) {};
		\node [style=none] (4) at (-1, 4.5) {};
		\node [style=none] (5) at (0, 4.25) {};
		\node [style=X] (6) at (0, 4.25) {};
	\end{pgfonlayer}
	\begin{pgfonlayer}{edgelayer}
		\draw [style=simple] (0) to (3.center);
		\draw [style=simple] (1) to (2);
		\draw [style=simple, in=-90, out=104] (1) to (4.center);
		\draw [style=simple] (5.center) to (2);
		\draw [style=simple, in=-60, out=60, looseness=0.75] (0) to (2);
		\draw [in=117, out=-90] (1) to (0);
	\end{pgfonlayer}
\end{tikzpicture}
\eq{Rem. \ref{cor:copy}}
\begin{tikzpicture}
	\begin{pgfonlayer}{nodelayer}
		\node [style=map] (0) at (0, 1.5) {$f$};
		\node [style=X] (1) at (-0.5, 2.5) {};
		\node [style=map] (2) at (0, 3.5) {$\bar {f^\circ}$};
		\node [style=none] (3) at (0, 0.5) {};
		\node [style=none] (4) at (-1, 4.5) {};
		\node [style=X] (5) at (-0.25, 4.5) {};
		\node [style=X] (6) at (0.25, 4.5) {};
	\end{pgfonlayer}
	\begin{pgfonlayer}{edgelayer}
		\draw [style=simple] (0) to (3.center);
		\draw [style=simple] (1) to (2);
		\draw [style=simple, in=-90, out=104] (1) to (4.center);
		\draw [style=simple, in=-60, out=60, looseness=0.75] (0) to (2);
		\draw [in=117, out=-90] (1) to (0);
		\draw [in=104, out=-90] (5) to (2);
		\draw [in=-90, out=76] (2) to (6);
	\end{pgfonlayer}
\end{tikzpicture}
=
\begin{tikzpicture}
	\begin{pgfonlayer}{nodelayer}
		\node [style=map] (0) at (1.5, 1.5) {$f$};
		\node [style=X] (1) at (1, 2.5) {};
		\node [style=none] (2) at (1.5, 0.5) {};
		\node [style=none] (3) at (0.5, 4.5) {};
		\node [style=X] (4) at (1.25, 4.5) {};
		\node [style=X] (5) at (1.75, 4.5) {};
		\node [style=map] (6) at (2, 3.75) {$\bar {f^\circ}$};
		\node [style=X] (7) at (1.25, 5.25) {};
		\node [style=X] (8) at (1.75, 5.25) {};
		\node [style=X] (9) at (1.25, 3) {};
		\node [style=X] (10) at (1.75, 3) {};
	\end{pgfonlayer}
	\begin{pgfonlayer}{edgelayer}
		\draw [style=simple] (0) to (2.center);
		\draw [style=simple, in=-90, out=104] (1) to (3.center);
		\draw [in=117, out=-90] (1) to (0);
		\draw [in=-120, out=120, looseness=1.25] (9) to (4);
		\draw [in=-120, out=120, looseness=1.25] (10) to (5);
		\draw [in=-75, out=72] (10) to (6);
		\draw [in=-120, out=45] (9) to (6);
		\draw [in=-45, out=120] (6) to (4);
		\draw [in=75, out=-72] (5) to (6);
		\draw (5) to (8);
		\draw (7) to (4);
		\draw (9) to (1);
		\draw [in=60, out=-90] (10) to (0);
	\end{pgfonlayer}
\end{tikzpicture}
=
\begin{tikzpicture}
	\begin{pgfonlayer}{nodelayer}
		\node [style=map] (0) at (0, 1.5) {$f$};
		\node [style=none] (1) at (0, 1) {};
		\node [style=none] (2) at (-1, 5.75) {};
		\node [style=X] (3) at (-0.25, 4) {};
		\node [style=X] (4) at (0.25, 4) {};
		\node [style=map] (5) at (0.5, 3.25) {$\bar {f^\circ}$};
		\node [style=X] (6) at (-0.25, 5.5) {};
		\node [style=X] (7) at (0.25, 5.5) {};
		\node [style=X] (8) at (-0.25, 2.5) {};
		\node [style=X] (9) at (0.25, 2.5) {};
		\node [style=X] (10) at (-0.25, 4.75) {};
	\end{pgfonlayer}
	\begin{pgfonlayer}{edgelayer}
		\draw [style=simple] (0) to (1.center);
		\draw [in=-120, out=120, looseness=1.25] (8) to (3);
		\draw [in=-120, out=120, looseness=1.25] (9) to (4);
		\draw [in=-75, out=72] (9) to (5);
		\draw [in=-120, out=45] (8) to (5);
		\draw [in=-45, out=120] (5) to (3);
		\draw [in=75, out=-72] (4) to (5);
		\draw (4) to (7);
		\draw (6) to (3);
		\draw [in=60, out=-90] (9) to (0);
		\draw [in=127, out=-90] (2.center) to (10);
		\draw [in=120, out=-90] (8) to (0);
	\end{pgfonlayer}
\end{tikzpicture}
=
\begin{tikzpicture}
	\begin{pgfonlayer}{nodelayer}
		\node [style=map] (0) at (0, 1.5) {$f$};
		\node [style=none] (1) at (0, 1) {};
		\node [style=none] (2) at (-1, 4) {};
		\node [style=map] (3) at (0, 2.25) {$\bar {f^\circ}$};
		\node [style=X] (4) at (-0.25, 3.75) {};
		\node [style=X] (5) at (0.25, 3.75) {};
		\node [style=X] (6) at (-0.25, 3) {};
	\end{pgfonlayer}
	\begin{pgfonlayer}{edgelayer}
		\draw [style=simple] (0) to (1.center);
		\draw [in=127, out=-90] (2.center) to (6);
		\draw [in=120, out=-120, looseness=1.25] (3) to (0);
		\draw [in=-75, out=75, looseness=1.25] (0) to (3);
		\draw [in=-90, out=120] (3) to (6);
		\draw [in=-90, out=75] (3) to (5);
		\draw (4) to (6);
	\end{pgfonlayer}
\end{tikzpicture}
=
\begin{tikzpicture}
	\begin{pgfonlayer}{nodelayer}
		\node [style=none] (0) at (0, 1.5) {};
		\node [style=none] (1) at (-1, 4) {};
		\node [style=X] (2) at (0.25, 3.75) {};
		\node [style=X] (3) at (-0.25, 3) {};
		\node [style=map] (4) at (0, 2) {$f$};
		\node [style=X] (5) at (-0.25, 3.75) {};
	\end{pgfonlayer}
	\begin{pgfonlayer}{edgelayer}
		\draw [in=127, out=-90] (1.center) to (3);
		\draw [style=simple] (4) to (0.center);
		\draw (5) to (3);
		\draw [in=75, out=-90] (2) to (4);
		\draw [in=-90, out=120] (4) to (3);
	\end{pgfonlayer}
\end{tikzpicture}
=
\begin{tikzpicture}
	\begin{pgfonlayer}{nodelayer}
		\node [style=none] (0) at (0, 1.5) {};
		\node [style=X] (1) at (0.25, 3) {};
		\node [style=map] (2) at (0, 2) {$f$};
		\node [style=none] (3) at (-0.25, 3) {};
		\node [style=none] (4) at (-0.25, 3.5) {};
	\end{pgfonlayer}
	\begin{pgfonlayer}{edgelayer}
		\draw [style=simple] (2) to (0.center);
		\draw [in=75, out=-90] (1) to (2);
		\draw (4.center) to (3.center);
		\draw [in=104, out=-90] (3.center) to (2);
	\end{pgfonlayer}
\end{tikzpicture}
\end{align*}

So that combining the previous two equations:

\begin{align*}
\begin{tikzpicture}
	\begin{pgfonlayer}{nodelayer}
		\node [style=none] (0) at (0, 1.5) {};
		\node [style=X] (1) at (0.25, 3) {};
		\node [style=map] (2) at (0, 2) {$f$};
		\node [style=none] (3) at (-0.25, 3) {};
		\node [style=none] (4) at (-0.25, 3.5) {};
	\end{pgfonlayer}
	\begin{pgfonlayer}{edgelayer}
		\draw [style=simple] (2) to (0.center);
		\draw [in=75, out=-90] (1) to (2);
		\draw (4.center) to (3.center);
		\draw [in=104, out=-90] (3.center) to (2);
	\end{pgfonlayer}
\end{tikzpicture}
=
\begin{tikzpicture}
	\begin{pgfonlayer}{nodelayer}
		\node [style=map] (0) at (0, 1.5) {$f$};
		\node [style=X] (1) at (-0.5, 2.5) {};
		\node [style=map] (2) at (0, 3.5) {$f^\circ$};
		\node [style=none] (3) at (0, 0.5) {};
		\node [style=none] (4) at (-1, 4.5) {};
		\node [style=none] (5) at (0, 4.25) {};
		\node [style=X] (6) at (0, 4.25) {};
	\end{pgfonlayer}
	\begin{pgfonlayer}{edgelayer}
		\draw [style=simple] (0) to (3.center);
		\draw [style=simple] (1) to (2);
		\draw [style=simple, in=-90, out=104] (1) to (4.center);
		\draw [style=simple] (5.center) to (2);
		\draw [style=simple, in=-60, out=60, looseness=0.75] (0) to (2);
		\draw [in=117, out=-90] (1) to (0);
	\end{pgfonlayer}
\end{tikzpicture}
=
\begin{tikzpicture}
	\begin{pgfonlayer}{nodelayer}
		\node [style=map] (0) at (0, 1.5) {$g$};
		\node [style=X] (1) at (-0.5, 2.5) {};
		\node [style=map] (2) at (0, 3.5) {$g^\circ$};
		\node [style=none] (3) at (0, 0.5) {};
		\node [style=none] (4) at (-1, 4.5) {};
		\node [style=none] (5) at (0, 4.25) {};
		\node [style=X] (6) at (0, 4.25) {};
	\end{pgfonlayer}
	\begin{pgfonlayer}{edgelayer}
		\draw [style=simple] (0) to (3.center);
		\draw [style=simple] (1) to (2);
		\draw [style=simple, in=-90, out=104] (1) to (4.center);
		\draw [style=simple] (5.center) to (2);
		\draw [style=simple, in=-60, out=60, looseness=0.75] (0) to (2);
		\draw [in=117, out=-90] (1) to (0);
	\end{pgfonlayer}
\end{tikzpicture}
=
\begin{tikzpicture}
	\begin{pgfonlayer}{nodelayer}
		\node [style=none] (0) at (0, 1.5) {};
		\node [style=X] (1) at (0.25, 3) {};
		\node [style=map] (2) at (0, 2) {$g$};
		\node [style=none] (3) at (-0.25, 3) {};
		\node [style=none] (4) at (-0.25, 3.5) {};
	\end{pgfonlayer}
	\begin{pgfonlayer}{edgelayer}
		\draw [style=simple] (2) to (0.center);
		\draw [in=75, out=-90] (1) to (2);
		\draw (4.center) to (3.center);
		\draw [in=104, out=-90] (3.center) to (2);
	\end{pgfonlayer}
\end{tikzpicture}
\end{align*}


\end{proof}



\section{\texorpdfstring{$\ZXA$}{ZX\&}}
\label{sec:ZXA}

In this section, we give a complete presentation, $\ZXA$, for the full monoidal subcategory of spans of finite sets where the objects are powers of the two element set.  This is performed by adding a counit and unit to the semi-Frobenius algebra structure of the category $\TOF$ (described in \cite{tof}), and then performing a two way translation between this prop and $\ZXA$ which we  prove is an  isomorphism. 



%\subsection{The category \texorpdfstring{$\TOF$}{TOF}}
%\label{sec:tof}

\begin{definition}
\label{def:tof}
$\TOF$ \cite{tof} is the PROP, generated by the 1 ancillary bits $| 1\rangle$ and $\langle 1|$ as well as the Toffoli gate, satisfying the identities given in Figure \ref{fig:TOF}.
		

\begin{figure}[H]
\noindent
\scalebox{1.0}{%
\vbox{%
\begin{mdframed}
\begin{multicols}{2}
\begin{enumerate}[label={\bf [TOF.\arabic*]}, ref={\bf [TOF.\arabic*]}, wide = 0pt, leftmargin = 2em]
\item
\label{TOF.1}
{\hfil
$
\begin{tabular}{c}
\begin{tikzpicture}
	\begin{pgfonlayer}{nodelayer}
		\node [style=nothing] (2) at (0, 1.5) {};
		\node [style=nothing] (3) at (-0.5, 1.5) {};
		\node [style=oplus] (4) at (0, 2) {};
		\node [style=dot] (5) at (-0.5, 2) {};
		\node [style=dot] (6) at (-1, 2) {};
		\node [style=onein] (7) at (-1, 1.5) {};
		\node [style=nothing] (8) at (-1, 2.5) {};
		\node [style=nothing] (9) at (-0.5, 2.5) {};
		\node [style=nothing] (10) at (0, 2.5) {};
	\end{pgfonlayer}
	\begin{pgfonlayer}{edgelayer}
		\draw (7) to (6);
		\draw (6) to (8);
		\draw (9) to (5);
		\draw (3) to (5);
		\draw (2) to (4);
		\draw (4) to (10);
		\draw (4) to (5);
		\draw (5) to (6);
	\end{pgfonlayer}
\end{tikzpicture}
=
\begin{tikzpicture}
	\begin{pgfonlayer}{nodelayer}
		\node [style=nothing] (3) at (0, 1.5) {};
		\node [style=nothing] (4) at (-0.5, 1.5) {};
		\node [style=oplus] (5) at (0, 2) {};
		\node [style=dot] (6) at (-0.5, 2) {};
		\node [style=onein] (7) at (-1, 2) {};
		\node [style=nothing] (8) at (-1, 2.5) {};
		\node [style=nothing] (9) at (-0.5, 2.5) {};
		\node [style=nothing] (10) at (0, 2.5) {};
	\end{pgfonlayer}
	\begin{pgfonlayer}{edgelayer}
		\draw (4) to (6);
		\draw (3) to (5);
		\draw (5) to (6);
		\draw (9) to (6);
		\draw (7) to (8);
		\draw (5) to (10);
	\end{pgfonlayer}
\end{tikzpicture}
\\
{}\\
\begin{tikzpicture}
	\begin{pgfonlayer}{nodelayer}
		\node [style=nothing] (0) at (0, 1.5) {};
		\node [style=nothing] (1) at (-0.5, 1.5) {};
		\node [style=oplus] (2) at (0, 1) {};
		\node [style=dot] (3) at (-0.5, 1) {};
		\node [style=dot] (4) at (-1, 1) {};
		\node [style=oneout] (5) at (-1, 1.5) {};
		\node [style=nothing] (6) at (-1, 0.5) {};
		\node [style=nothing] (7) at (-0.5, 0.5) {};
		\node [style=nothing] (8) at (0, 0.5) {};
	\end{pgfonlayer}
	\begin{pgfonlayer}{edgelayer}
		\draw (5) to (4);
		\draw (4) to (6);
		\draw (7) to (3);
		\draw (1) to (3);
		\draw (0) to (2);
		\draw (2) to (8);
		\draw (2) to (3);
		\draw (3) to (4);
	\end{pgfonlayer}
\end{tikzpicture}
=
\begin{tikzpicture}
	\begin{pgfonlayer}{nodelayer}
		\node [style=nothing] (1) at (0, 1) {};
		\node [style=nothing] (2) at (-0.5, 1) {};
		\node [style=oplus] (3) at (0, 0.5) {};
		\node [style=dot] (4) at (-0.5, 0.5) {};
		\node [style=oneout] (5) at (-1, 0.5) {};
		\node [style=nothing] (6) at (-1, 0) {};
		\node [style=nothing] (7) at (-0.5, 0) {};
		\node [style=nothing] (8) at (0, 0) {};
	\end{pgfonlayer}
	\begin{pgfonlayer}{edgelayer}
		\draw (2) to (4);
		\draw (1) to (3);
		\draw (3) to (4);
		\draw (7) to (4);
		\draw (5) to (6);
		\draw (3) to (8);
	\end{pgfonlayer}
\end{tikzpicture}
\end{tabular}
$}


\item
\label{TOF.2}
{\hfil
$
\begin{tabular}{c}
\begin{tikzpicture}
	\begin{pgfonlayer}{nodelayer}
		\node [style=nothing] (2) at (-1.25, 0) {};
		\node [style=nothing] (3) at (-0.75, 0) {};
		\node [style=nothing] (4) at (-1.75, 2) {};
		\node [style=nothing] (5) at (-1.25, 2) {};
		\node [style=nothing] (6) at (-0.75, 2) {};
		\node [style=dot] (7) at (-1.75, 1) {};
		\node [style=dot] (8) at (-1.25, 1) {};
		\node [style=oplus] (9) at (-0.75, 1) {};
		\node [style=zeroin] (10) at (-1.75, 0) {};
	\end{pgfonlayer}
	\begin{pgfonlayer}{edgelayer}
		\draw (7) to (4);
		\draw (5) to (8);
		\draw (8) to (2);
		\draw (3) to (9);
		\draw (9) to (6);
		\draw (9) to (8);
		\draw (8) to (7);
		\draw (10) to (7);
	\end{pgfonlayer}
\end{tikzpicture}
=
\begin{tikzpicture}
	\begin{pgfonlayer}{nodelayer}
		\node [style=nothing] (3) at (-1.25, 0) {};
		\node [style=nothing] (4) at (-0.75, 0) {};
		\node [style=nothing] (5) at (-1.75, 1.5) {};
		\node [style=nothing] (6) at (-1.25, 1.5) {};
		\node [style=nothing] (7) at (-0.75, 1.5) {};
		\node [style=zeroin] (8) at (-1.75, 0) {};
	\end{pgfonlayer}
	\begin{pgfonlayer}{edgelayer}
		\draw (8) to (5);
		\draw (3) to (6);
		\draw (4) to (7);
	\end{pgfonlayer}
\end{tikzpicture}\\
{}\\
\begin{tikzpicture}
	\begin{pgfonlayer}{nodelayer}
		\node [style=nothing] (0) at (1, 0) {};
		\node [style=nothing] (1) at (1.5, 0) {};
		\node [style=nothing] (2) at (0.5, 2) {};
		\node [style=nothing] (3) at (1, 2) {};
		\node [style=nothing] (4) at (1.5, 2) {};
		\node [style=dot] (5) at (0.5, 1) {};
		\node [style=dot] (6) at (1, 1) {};
		\node [style=oplus] (7) at (1.5, 1) {};
		\node [style=zeroin] (8) at (0.5, 0) {};
	\end{pgfonlayer}
	\begin{pgfonlayer}{edgelayer}
		\draw (5) to (2);
		\draw (3) to (6);
		\draw (6) to (0);
		\draw (1) to (7);
		\draw (7) to (4);
		\draw (7) to (6);
		\draw (6) to (5);
		\draw (8) to (5);
	\end{pgfonlayer}
\end{tikzpicture}
=
\begin{tikzpicture}
	\begin{pgfonlayer}{nodelayer}
		\node [style=nothing] (0) at (-1.25, 2) {};
		\node [style=nothing] (1) at (-0.75, 2) {};
		\node [style=zeroin] (2) at (-1.75, 0.5) {};
		\node [style=nothing] (3) at (-1.25, 0.5) {};
		\node [style=nothing] (4) at (-0.75, 0.5) {};
		\node [style=nothing] (5) at (-1.75, 2) {};
	\end{pgfonlayer}
	\begin{pgfonlayer}{edgelayer}
		\draw (5) to (2);
		\draw (0) to (3);
		\draw (1) to (4);
	\end{pgfonlayer}
\end{tikzpicture}
\end{tabular}
$}

\item
\label{TOF.3}
{\hfil
$
\begin{tikzpicture}
	\begin{pgfonlayer}{nodelayer}
		\node [style=nothing] (0) at (-0.5, 0.5) {};
		\node [style=nothing] (1) at (0, 0.5) {};
		\node [style=nothing] (2) at (-1, 0.5) {};
		\node [style=nothing] (3) at (-1.5, 0.5) {};
		\node [style=nothing] (4) at (-2, 0.5) {};
		\node [style=dot] (5) at (-1.5, 1) {};
		\node [style=oplus] (6) at (-1, 1) {};
		\node [style=oplus] (7) at (-1, 1.5) {};
		\node [style=dot] (8) at (-0.5, 1.5) {};
		\node [style=dot] (9) at (-2, 1) {};
		\node [style=dot] (10) at (0, 1.5) {};
		\node [style=nothing] (11) at (-0.5, 2) {};
		\node [style=nothing] (12) at (-1.5, 2) {};
		\node [style=nothing] (13) at (-2, 2) {};
		\node [style=nothing] (14) at (0, 2) {};
		\node [style=nothing] (15) at (-1, 2) {};
	\end{pgfonlayer}
	\begin{pgfonlayer}{edgelayer}
		\draw (4) to (9);
		\draw (9) to (13);
		\draw (3) to (5);
		\draw (5) to (12);
		\draw (2) to (6);
		\draw (6) to (7);
		\draw (7) to (15);
		\draw (0) to (8);
		\draw (8) to (11);
		\draw (1) to (10);
		\draw (10) to (14);
		\draw (10) to (8);
		\draw (8) to (7);
		\draw (6) to (5);
		\draw (5) to (9);
	\end{pgfonlayer}
\end{tikzpicture}
=
\begin{tikzpicture}
	\begin{pgfonlayer}{nodelayer}
		\node [style=nothing] (1) at (-0.5, 0) {};
		\node [style=nothing] (2) at (0, 0) {};
		\node [style=nothing] (3) at (-1, 0) {};
		\node [style=nothing] (4) at (-1.5, 0) {};
		\node [style=nothing] (5) at (-2, 0) {};
		\node [style=dot] (6) at (-1.5, 1) {};
		\node [style=dot] (7) at (-0.5, 0.5) {};
		\node [style=dot] (8) at (-2, 1) {};
		\node [style=dot] (9) at (0, 0.5) {};
		\node [style=nothing] (10) at (-0.5, 1.5) {};
		\node [style=nothing] (11) at (-1.5, 1.5) {};
		\node [style=nothing] (12) at (-2, 1.5) {};
		\node [style=nothing] (13) at (0, 1.5) {};
		\node [style=nothing] (14) at (-1, 1.5) {};
		\node [style=oplus] (15) at (-1, 1) {};
		\node [style=oplus] (16) at (-1, 0.5) {};
	\end{pgfonlayer}
	\begin{pgfonlayer}{edgelayer}
		\draw (5) to (8);
		\draw (8) to (12);
		\draw (4) to (6);
		\draw (6) to (11);
		\draw (1) to (7);
		\draw (7) to (10);
		\draw (2) to (9);
		\draw (9) to (13);
		\draw (9) to (7);
		\draw (6) to (8);
		\draw (3) to (16);
		\draw (16) to (15);
		\draw (15) to (14);
		\draw (15) to (6);
		\draw (7) to (16);
	\end{pgfonlayer}
\end{tikzpicture}
$}


\item
\label{TOF.4}
{\hfil
$
\begin{tikzpicture}
	\begin{pgfonlayer}{nodelayer}
		\node [style=nothing] (2) at (-0.5, 0) {};
		\node [style=nothing] (3) at (0, 0) {};
		\node [style=nothing] (4) at (-1, 0) {};
		\node [style=nothing] (5) at (-1.5, 0) {};
		\node [style=nothing] (6) at (-2, 0) {};
		\node [style=dot] (7) at (-1.5, 0.5) {};
		\node [style=dot] (8) at (-1, 0.5) {};
		\node [style=dot] (9) at (-1, 1) {};
		\node [style=dot] (10) at (-0.5, 1) {};
		\node [style=oplus] (11) at (-2, 0.5) {};
		\node [style=oplus] (12) at (0, 1) {};
		\node [style=nothing] (13) at (-0.5, 1.5) {};
		\node [style=nothing] (14) at (-1.5, 1.5) {};
		\node [style=nothing] (15) at (-2, 1.5) {};
		\node [style=nothing] (16) at (0, 1.5) {};
		\node [style=nothing] (17) at (-1, 1.5) {};
	\end{pgfonlayer}
	\begin{pgfonlayer}{edgelayer}
		\draw (6) to (11);
		\draw (11) to (15);
		\draw (5) to (7);
		\draw (7) to (14);
		\draw (4) to (8);
		\draw (8) to (9);
		\draw (9) to (17);
		\draw (2) to (10);
		\draw (10) to (13);
		\draw (3) to (12);
		\draw (12) to (16);
		\draw (12) to (10);
		\draw (10) to (9);
		\draw (8) to (7);
		\draw (7) to (11);
	\end{pgfonlayer}
\end{tikzpicture}
=
\begin{tikzpicture}
	\begin{pgfonlayer}{nodelayer}
		\node [style=nothing] (3) at (-0.5, 0) {};
		\node [style=nothing] (4) at (0, 0) {};
		\node [style=nothing] (5) at (-1, 0) {};
		\node [style=nothing] (6) at (-1.5, 0) {};
		\node [style=nothing] (7) at (-2, 0) {};
		\node [style=dot] (8) at (-1.5, 1) {};
		\node [style=dot] (9) at (-0.5, 0.5) {};
		\node [style=oplus] (10) at (-2, 1) {};
		\node [style=oplus] (11) at (0, 0.5) {};
		\node [style=nothing] (12) at (-0.5, 1.5) {};
		\node [style=nothing] (13) at (-1.5, 1.5) {};
		\node [style=nothing] (14) at (-2, 1.5) {};
		\node [style=nothing] (15) at (0, 1.5) {};
		\node [style=nothing] (16) at (-1, 1.5) {};
		\node [style=dot] (17) at (-1, 1) {};
		\node [style=dot] (18) at (-1, 0.5) {};
	\end{pgfonlayer}
	\begin{pgfonlayer}{edgelayer}
		\draw (7) to (10);
		\draw (10) to (14);
		\draw (6) to (8);
		\draw (8) to (13);
		\draw (3) to (9);
		\draw (9) to (12);
		\draw (4) to (11);
		\draw (11) to (15);
		\draw (11) to (9);
		\draw (8) to (10);
		\draw (5) to (18);
		\draw (18) to (17);
		\draw (17) to (16);
		\draw (17) to (8);
		\draw (9) to (18);
	\end{pgfonlayer}
\end{tikzpicture}
$}

\item
\label{TOF.5}
{\hfil
$
\begin{tikzpicture}
	\begin{pgfonlayer}{nodelayer}
		\node [style=nothing] (4) at (-1, 2) {};
		\node [style=nothing] (5) at (-0.5, 2) {};
		\node [style=nothing] (6) at (-1.5, 2) {};
		\node [style=nothing] (7) at (-2, 2) {};
		\node [style=nothing] (8) at (-1, 3.5) {};
		\node [style=nothing] (9) at (-1.5, 3.5) {};
		\node [style=nothing] (10) at (-2, 3.5) {};
		\node [style=nothing] (11) at (-0.5, 3.5) {};
		\node [style=oplus] (12) at (-2, 2.5) {};
		\node [style=oplus] (13) at (-0.5, 3) {};
		\node [style=dot] (14) at (-1.5, 2.5) {};
		\node [style=dot] (15) at (-1, 2.5) {};
		\node [style=dot] (16) at (-1.5, 3) {};
		\node [style=dot] (17) at (-1, 3) {};
	\end{pgfonlayer}
	\begin{pgfonlayer}{edgelayer}
		\draw (7) to (12);
		\draw (12) to (10);
		\draw (9) to (16);
		\draw (16) to (14);
		\draw (14) to (6);
		\draw (4) to (15);
		\draw (15) to (17);
		\draw (17) to (8);
		\draw (11) to (13);
		\draw (13) to (5);
		\draw (14) to (15);
		\draw (14) to (12);
		\draw (16) to (17);
		\draw (17) to (13);
	\end{pgfonlayer}
\end{tikzpicture}
=
\begin{tikzpicture}
	\begin{pgfonlayer}{nodelayer}
		\node [style=nothing] (5) at (-1, 2) {};
		\node [style=nothing] (6) at (-0.5, 2) {};
		\node [style=nothing] (7) at (-1.5, 2) {};
		\node [style=nothing] (8) at (-2, 2) {};
		\node [style=nothing] (9) at (-1, 3.5) {};
		\node [style=nothing] (10) at (-1.5, 3.5) {};
		\node [style=nothing] (11) at (-2, 3.5) {};
		\node [style=nothing] (12) at (-0.5, 3.5) {};
		\node [style=oplus] (13) at (-2, 3) {};
		\node [style=dot] (14) at (-1.5, 3) {};
		\node [style=dot] (15) at (-1, 3) {};
		\node [style=oplus] (16) at (-0.5, 2.5) {};
		\node [style=dot] (17) at (-1, 2.5) {};
		\node [style=dot] (18) at (-1.5, 2.5) {};
	\end{pgfonlayer}
	\begin{pgfonlayer}{edgelayer}
		\draw (14) to (15);
		\draw (14) to (13);
		\draw (18) to (17);
		\draw (17) to (16);
		\draw (8) to (13);
		\draw (13) to (11);
		\draw (10) to (14);
		\draw (14) to (18);
		\draw (18) to (7);
		\draw (5) to (17);
		\draw (17) to (15);
		\draw (15) to (9);
		\draw (12) to (16);
		\draw (16) to (6);
	\end{pgfonlayer}
\end{tikzpicture}
$}


\item
\label{TOF.6}
{\hfil
$
\begin{tikzpicture}
	\begin{pgfonlayer}{nodelayer}
		\node [style=nothing] (6) at (-1, 2) {};
		\node [style=nothing] (7) at (-1.5, 2) {};
		\node [style=nothing] (8) at (-2, 2) {};
		\node [style=nothing] (9) at (-1, 3.5) {};
		\node [style=nothing] (10) at (-1.5, 3.5) {};
		\node [style=nothing] (11) at (-2, 3.5) {};
		\node [style=nothing] (12) at (-0.5, 3.5) {};
		\node [style=oplus] (13) at (-0.5, 2.5) {};
		\node [style=dot] (14) at (-1.5, 3) {};
		\node [style=dot] (15) at (-1, 3) {};
		\node [style=dot] (16) at (-1, 2.5) {};
		\node [style=oplus] (17) at (-0.5, 3) {};
		\node [style=nothing] (18) at (-0.5, 2) {};
		\node [style=dot] (19) at (-2, 2.5) {};
	\end{pgfonlayer}
	\begin{pgfonlayer}{edgelayer}
		\draw (14) to (7);
		\draw (6) to (15);
		\draw (15) to (16);
		\draw (16) to (9);
		\draw (12) to (13);
		\draw (14) to (15);
		\draw (16) to (13);
		\draw (18) to (17);
		\draw (17) to (13);
		\draw (14) to (10);
		\draw (15) to (17);
		\draw (16) to (19);
		\draw (19) to (11);
		\draw (19) to (8);
	\end{pgfonlayer}
\end{tikzpicture}
=
\begin{tikzpicture}
	\begin{pgfonlayer}{nodelayer}
		\node [style=nothing] (7) at (-1, 2) {};
		\node [style=nothing] (8) at (-1.5, 2) {};
		\node [style=nothing] (9) at (-2, 2) {};
		\node [style=nothing] (10) at (-1, 3.5) {};
		\node [style=nothing] (11) at (-1.5, 3.5) {};
		\node [style=nothing] (12) at (-2, 3.5) {};
		\node [style=nothing] (13) at (-0.5, 3.5) {};
		\node [style=oplus] (14) at (-0.5, 3) {};
		\node [style=dot] (15) at (-1.5, 2.5) {};
		\node [style=dot] (16) at (-1, 2.5) {};
		\node [style=dot] (17) at (-1, 3) {};
		\node [style=oplus] (18) at (-0.5, 2.5) {};
		\node [style=nothing] (19) at (-0.5, 2) {};
		\node [style=dot] (20) at (-2, 3) {};
	\end{pgfonlayer}
	\begin{pgfonlayer}{edgelayer}
		\draw (15) to (8);
		\draw (7) to (16);
		\draw (16) to (17);
		\draw (17) to (10);
		\draw (13) to (14);
		\draw (15) to (16);
		\draw (17) to (14);
		\draw (19) to (18);
		\draw (18) to (14);
		\draw (15) to (11);
		\draw (16) to (18);
		\draw (17) to (20);
		\draw (20) to (12);
		\draw (20) to (9);
	\end{pgfonlayer}
\end{tikzpicture}
$}

\item
\label{TOF.7}
{\hfil
$
\begin{tikzpicture}
	\begin{pgfonlayer}{nodelayer}
		\node [style=nothing] (8) at (0, 2) {};
		\node [style=nothing] (9) at (-0.5, 2) {};
		\node [style=nothing] (10) at (-0.5, 4.5) {};
		\node [style=nothing] (11) at (0, 4.5) {};
		\node [style=zeroout] (12) at (0.5, 4.5) {};
		\node [style=oplus] (13) at (0.5, 4) {};
		\node [style=dot] (14) at (0, 4) {};
		\node [style=dot] (15) at (-0.5, 2.5) {};
		\node [style=oplus] (16) at (0.5, 2.5) {};
		\node [style=zeroout] (17) at (0.5, 3) {};
		\node [style=onein] (18) at (0.5, 2) {};
		\node [style=onein] (19) at (0.5, 3.5) {};
	\end{pgfonlayer}
	\begin{pgfonlayer}{edgelayer}
		\draw (9) to (15);
		\draw (15) to (10);
		\draw (11) to (14);
		\draw (14) to (8);
		\draw (16) to (17);
		\draw (16) to (15);
		\draw (13) to (12);
		\draw (13) to (14);
		\draw (18) to (16);
		\draw (19) to (13);
	\end{pgfonlayer}
\end{tikzpicture}
=
\begin{tikzpicture}
	\begin{pgfonlayer}{nodelayer}
		\node [style=nothing] (9) at (0, 2) {};
		\node [style=nothing] (10) at (-0.5, 2) {};
		\node [style=nothing] (11) at (-0.5, 3) {};
		\node [style=nothing] (12) at (0, 3) {};
		\node [style=dot] (13) at (-0.5, 2.5) {};
		\node [style=dot] (14) at (0, 2.5) {};
		\node [style=onein] (15) at (0.5, 2) {};
		\node [style=zeroout] (16) at (0.5, 3) {};
		\node [style=oplus] (17) at (0.5, 2.5) {};
	\end{pgfonlayer}
	\begin{pgfonlayer}{edgelayer}
		\draw (10) to (13);
		\draw (13) to (11);
		\draw (12) to (14);
		\draw (14) to (9);
		\draw (15) to (17);
		\draw (17) to (16);
		\draw (17) to (14);
		\draw (14) to (13);
	\end{pgfonlayer}
\end{tikzpicture}
$}
%
%\item
%\label{TOF.7}
%{\hfil
%$
%\begin{tikzpicture}
%	\begin{pgfonlayer}{nodelayer}
%		\node [style=nothing] (0) at (0, -0) {};
%		\node [style=nothing] (1) at (1.5, -0) {};
%		\node [style=onein] (2) at (0.4, 0.5) {};
%		\node [style=zeroout] (3) at (1.1, 0.5) {};
%	\end{pgfonlayer}
%	\begin{pgfonlayer}{edgelayer}
%		\draw (2) to (3);
%		\draw (0) to (1);
%	\end{pgfonlayer}
%\end{tikzpicture}
%=
%\begin{tikzpicture}
%	\begin{pgfonlayer}{nodelayer}
%		\node [style=nothing] (0) at (0, -0) {};
%		\node [style=nothing] (1) at (1.5, -0) {};
%		\node [style=onein] (2) at (0.4, 0.5) {};
%		\node [style=zeroout] (3) at (1.1, 0.5) {};
%		\node [style=onein] (4) at (1, -0) {};
%		\node [style=oneout] (5) at (0.5000002, -0) {};
%	\end{pgfonlayer}
%	\begin{pgfonlayer}{edgelayer}
%		\draw (2) to (3);
%		\draw (5) to (0);
%		\draw (4) to (1);
%	\end{pgfonlayer}
%\end{tikzpicture}
%$}

\item
\label{TOF.8}
{\hfil
$
\begin{tikzpicture}
	\begin{pgfonlayer}{nodelayer}
		\node [style=onein] (10) at (0, 2) {};
		\node [style=oneout] (11) at (0, 3) {};
	\end{pgfonlayer}
	\begin{pgfonlayer}{edgelayer}
		\draw (10) to (11);
	\end{pgfonlayer}
\end{tikzpicture}
=
\begin{tikzpicture}
	\begin{pgfonlayer}{nodelayer}
		\node [style=rn] (11) at (0, 2) {};
		\node [style=rn] (12) at (0, 3) {};
	\end{pgfonlayer}
\end{tikzpicture}
%\hspace*{-.8cm}
%\begin{tikzpicture}[scale=.5]
%\begin{pgfonlayer}{nodelayer}
%\begin{tikzpicture}
%\node[cloud, cloud puffs=15.7,minimum width=3cm, draw,] (cloud) at (0,0) {$1_0$};
%\end{tikzpicture}
%\end{pgfonlayer}
%\begin{pgfonlayer}{edgelayer}
%\end{pgfonlayer}
%\end{tikzpicture}
$}

\item
\label{TOF.9}
{\hfil
$
\begin{tikzpicture}
	\begin{pgfonlayer}{nodelayer}
		\node [style=nothing] (12) at (-1.75, 2) {};
		\node [style=nothing] (13) at (-1.25, 2) {};
		\node [style=nothing] (14) at (-0.75, 2) {};
		\node [style=dot] (15) at (-1.75, 2.5) {};
		\node [style=dot] (16) at (-1.25, 2.5) {};
		\node [style=oplus] (17) at (-0.75, 2.5) {};
		\node [style=dot] (18) at (-1.75, 3) {};
		\node [style=oplus] (19) at (-0.75, 3) {};
		\node [style=dot] (20) at (-1.25, 3) {};
		\node [style=nothing] (21) at (-1.25, 3.5) {};
		\node [style=nothing] (22) at (-0.75, 3.5) {};
		\node [style=nothing] (23) at (-1.75, 3.5) {};
	\end{pgfonlayer}
	\begin{pgfonlayer}{edgelayer}
		\draw (12) to (15);
		\draw (13) to (16);
		\draw (14) to (17);
		\draw (15) to (16);
		\draw (16) to (17);
		\draw (18) to (20);
		\draw (20) to (19);
		\draw (15) to (18);
		\draw (18) to (23);
		\draw (16) to (20);
		\draw (20) to (21);
		\draw (17) to (19);
		\draw (19) to (22);
	\end{pgfonlayer}
\end{tikzpicture}
=
\begin{tikzpicture}
	\begin{pgfonlayer}{nodelayer}
		\node [style=nothing] (13) at (-1.75, 2) {};
		\node [style=nothing] (14) at (-1.25, 2) {};
		\node [style=nothing] (15) at (-0.75, 2) {};
		\node [style=nothing] (16) at (-1.25, 3.5) {};
		\node [style=nothing] (17) at (-0.75, 3.5) {};
		\node [style=nothing] (18) at (-1.75, 3.5) {};
	\end{pgfonlayer}
	\begin{pgfonlayer}{edgelayer}
		\draw (13) to (18);
		\draw (14) to (16);
		\draw (15) to (17);
	\end{pgfonlayer}
\end{tikzpicture}
$}

\item
\label{TOF.10}
{\hfil
$
\begin{tikzpicture}
	\begin{pgfonlayer}{nodelayer}
		\node [style=nothing] (14) at (0, 2) {};
		\node [style=nothing] (15) at (-0.5, 2) {};
		\node [style=nothing] (16) at (-1, 2) {};
		\node [style=nothing] (17) at (-1.5, 2) {};
		\node [style=dot] (18) at (-1, 2.5) {};
		\node [style=dot] (19) at (-0.5, 2.5) {};
		\node [style=oplus] (20) at (0, 2.5) {};
		\node [style=dot] (21) at (-1.5, 3) {};
		\node [style=oplus] (22) at (-0.5, 3) {};
		\node [style=dot] (23) at (-1, 3) {};
		\node [style=dot] (24) at (-1, 3.5) {};
		\node [style=oplus] (25) at (0, 3.5) {};
		\node [style=dot] (26) at (-0.5, 3.5) {};
		\node [style=nothing] (27) at (-1.5, 4) {};
		\node [style=nothing] (28) at (-0.5, 4) {};
		\node [style=nothing] (29) at (-1, 4) {};
		\node [style=nothing] (30) at (0, 4) {};
	\end{pgfonlayer}
	\begin{pgfonlayer}{edgelayer}
		\draw (18) to (19);
		\draw (19) to (20);
		\draw (21) to (23);
		\draw (23) to (22);
		\draw (24) to (26);
		\draw (26) to (25);
		\draw (17) to (21);
		\draw (21) to (27);
		\draw (29) to (24);
		\draw (24) to (23);
		\draw (23) to (18);
		\draw (18) to (16);
		\draw (15) to (19);
		\draw (19) to (22);
		\draw (22) to (26);
		\draw (26) to (28);
		\draw (30) to (25);
		\draw (25) to (20);
		\draw (20) to (14);
	\end{pgfonlayer}
\end{tikzpicture}
=
\begin{tikzpicture}
	\begin{pgfonlayer}{nodelayer}
		\node [style=nothing] (15) at (0, 2) {};
		\node [style=nothing] (16) at (-0.5, 2) {};
		\node [style=nothing] (17) at (-1, 2) {};
		\node [style=nothing] (18) at (-1.5, 2) {};
		\node [style=nothing] (19) at (-1.5, 3.5) {};
		\node [style=nothing] (20) at (-0.5, 3.5) {};
		\node [style=nothing] (21) at (-1, 3.5) {};
		\node [style=nothing] (22) at (0, 3.5) {};
		\node [style=dot] (23) at (-1.5, 2.5) {};
		\node [style=dot] (24) at (-1, 2.5) {};
		\node [style=dot] (25) at (-1.5, 3) {};
		\node [style=dot] (26) at (-1, 3) {};
		\node [style=oplus] (27) at (-0.5, 3) {};
		\node [style=oplus] (28) at (0, 2.5) {};
	\end{pgfonlayer}
	\begin{pgfonlayer}{edgelayer}
		\draw (18) to (23);
		\draw (23) to (25);
		\draw (25) to (19);
		\draw (21) to (26);
		\draw (26) to (24);
		\draw (24) to (17);
		\draw (16) to (27);
		\draw (27) to (20);
		\draw (22) to (28);
		\draw (28) to (15);
		\draw (28) to (24);
		\draw (24) to (23);
		\draw (25) to (26);
		\draw (26) to (27);
	\end{pgfonlayer}
\end{tikzpicture}
$}

\item
\label{TOF.11}
{\hfil
$
\begin{tikzpicture}
	\begin{pgfonlayer}{nodelayer}
		\node [style=nothing] (16) at (0, 2) {};
		\node [style=nothing] (17) at (-0.5, 2) {};
		\node [style=nothing] (18) at (-1, 2) {};
		\node [style=nothing] (19) at (-1.5, 2) {};
		\node [style=nothing] (20) at (-0.5, 4) {};
		\node [style=nothing] (21) at (0, 4) {};
		\node [style=dot] (22) at (-1.5, 2.5) {};
		\node [style=dot] (23) at (-1, 3) {};
		\node [style=dot] (24) at (-0.5, 3) {};
		\node [style=oplus] (25) at (-1, 2.5) {};
		\node [style=oplus] (26) at (0, 3) {};
		\node [style=nothing] (27) at (-1.5, 4) {};
		\node [style=nothing] (28) at (-1, 4) {};
		\node [style=oplus] (29) at (-1, 3.5) {};
		\node [style=dot] (30) at (-1.5, 3.5) {};
	\end{pgfonlayer}
	\begin{pgfonlayer}{edgelayer}
		\draw (22) to (25);
		\draw (23) to (24);
		\draw (24) to (26);
		\draw (16) to (26);
		\draw (26) to (21);
		\draw (20) to (24);
		\draw (24) to (17);
		\draw (18) to (25);
		\draw (25) to (23);
		\draw (22) to (19);
		\draw (22) to (30);
		\draw (30) to (27);
		\draw (28) to (29);
		\draw (29) to (23);
		\draw (29) to (30);
	\end{pgfonlayer}
\end{tikzpicture}
=
\begin{tikzpicture}
	\begin{pgfonlayer}{nodelayer}
		\node [style=nothing] (17) at (0, 2) {};
		\node [style=nothing] (18) at (-1, 2) {};
		\node [style=nothing] (19) at (-0.5, 2) {};
		\node [style=nothing] (20) at (-1.5, 2) {};
		\node [style=dot] (21) at (-1.5, 2.5) {};
		\node [style=dot] (22) at (-0.5, 2.5) {};
		\node [style=oplus] (23) at (0, 2.5) {};
		\node [style=nothing] (24) at (-0.5, 3.5) {};
		\node [style=nothing] (25) at (-1, 3.5) {};
		\node [style=nothing] (26) at (-1.5, 3.5) {};
		\node [style=nothing] (27) at (0, 3.5) {};
		\node [style=dot] (28) at (-1, 3) {};
		\node [style=dot] (29) at (-0.5, 3) {};
		\node [style=oplus] (30) at (0, 3) {};
	\end{pgfonlayer}
	\begin{pgfonlayer}{edgelayer}
		\draw (20) to (21);
		\draw (19) to (22);
		\draw (23) to (17);
		\draw (23) to (22);
		\draw (22) to (21);
		\draw (28) to (18);
		\draw (22) to (29);
		\draw (29) to (24);
		\draw (27) to (30);
		\draw (30) to (23);
		\draw (30) to (29);
		\draw (29) to (28);
		\draw (21) to (26);
		\draw (28) to (25);
	\end{pgfonlayer}
\end{tikzpicture}
$}

\item
\label{TOF.12}
{\hfil
$
\begin{tikzpicture}
	\begin{pgfonlayer}{nodelayer}
		\node [style=nothing] (18) at (-0.5, 2) {};
		\node [style=nothing] (19) at (0, 2) {};
		\node [style=nothing] (20) at (-1, 2) {};
		\node [style=nothing] (21) at (-1.5, 2) {};
		\node [style=nothing] (22) at (-0.5, 4) {};
		\node [style=nothing] (23) at (-1.5, 4) {};
		\node [style=nothing] (24) at (0, 4) {};
		\node [style=nothing] (25) at (-1, 4) {};
		\node [style=dot] (26) at (-1.5, 2.5) {};
		\node [style=dot] (27) at (-1, 2.5) {};
		\node [style=oplus] (28) at (-0.5, 2.5) {};
		\node [style=oplus] (29) at (0, 3) {};
		\node [style=dot] (30) at (-1, 3) {};
		\node [style=dot] (31) at (-0.5, 3) {};
		\node [style=oplus] (32) at (-0.5, 3.5) {};
		\node [style=dot] (33) at (-1.5, 3.5) {};
		\node [style=dot] (34) at (-1, 3.5) {};
	\end{pgfonlayer}
	\begin{pgfonlayer}{edgelayer}
		\draw (26) to (27);
		\draw (27) to (28);
		\draw (30) to (31);
		\draw (31) to (29);
		\draw (33) to (34);
		\draw (34) to (32);
		\draw (21) to (26);
		\draw (26) to (33);
		\draw (33) to (23);
		\draw (25) to (34);
		\draw (34) to (30);
		\draw (30) to (27);
		\draw (27) to (20);
		\draw (18) to (28);
		\draw (28) to (31);
		\draw (31) to (32);
		\draw (32) to (22);
		\draw (24) to (29);
		\draw (29) to (19);
	\end{pgfonlayer}
\end{tikzpicture}
=
\begin{tikzpicture}
	\begin{pgfonlayer}{nodelayer}
		\node [style=nothing] (19) at (-0.5, 3.5) {};
		\node [style=nothing] (20) at (0, 3.5) {};
		\node [style=nothing] (21) at (-1, 3.5) {};
		\node [style=nothing] (22) at (-1.5, 3.5) {};
		\node [style=nothing] (23) at (-0.5, 5) {};
		\node [style=nothing] (24) at (-1.5, 5) {};
		\node [style=nothing] (25) at (0, 5) {};
		\node [style=nothing] (26) at (-1, 5) {};
		\node [style=dot] (27) at (-1, 4.5) {};
		\node [style=dot] (28) at (-0.5, 4.5) {};
		\node [style=dot] (29) at (-1.5, 4) {};
		\node [style=dot] (30) at (-1, 4) {};
		\node [style=oplus] (31) at (0, 4) {};
		\node [style=oplus] (32) at (0, 4.5) {};
	\end{pgfonlayer}
	\begin{pgfonlayer}{edgelayer}
		\draw (27) to (28);
		\draw (22) to (29);
		\draw (29) to (24);
		\draw (21) to (30);
		\draw (30) to (27);
		\draw (27) to (26);
		\draw (19) to (28);
		\draw (28) to (23);
		\draw (20) to (31);
		\draw (31) to (32);
		\draw (32) to (25);
		\draw (32) to (28);
		\draw (31) to (30);
		\draw (30) to (29);
	\end{pgfonlayer}
\end{tikzpicture}
$}

\item
\label{TOF.13}
{\hfil
$
\begin{tikzpicture}
	\begin{pgfonlayer}{nodelayer}
		\node [style=nothing] (20) at (0, 3.5) {};
		\node [style=nothing] (21) at (-1, 3.5) {};
		\node [style=nothing] (22) at (-0.5, 3.5) {};
		\node [style=nothing] (23) at (-1.5, 3.5) {};
		\node [style=nothing] (24) at (0, 5.5) {};
		\node [style=dot] (25) at (-1.5, 4) {};
		\node [style=dot] (26) at (-1, 4) {};
		\node [style=dot] (27) at (-0.5, 4.5) {};
		\node [style=oplus] (28) at (-0.5, 4) {};
		\node [style=oplus] (29) at (0, 4.5) {};
		\node [style=nothing] (30) at (-0.5, 5.5) {};
		\node [style=nothing] (31) at (-1.5, 5.5) {};
		\node [style=nothing] (32) at (-1, 5.5) {};
		\node [style=oplus] (33) at (-0.5, 5) {};
		\node [style=dot] (34) at (-1, 5) {};
		\node [style=dot] (35) at (-1.5, 5) {};
	\end{pgfonlayer}
	\begin{pgfonlayer}{edgelayer}
		\draw (25) to (23);
		\draw (26) to (21);
		\draw (22) to (28);
		\draw (28) to (27);
		\draw (24) to (29);
		\draw (29) to (20);
		\draw (28) to (26);
		\draw (26) to (25);
		\draw (29) to (27);
		\draw (25) to (35);
		\draw (35) to (31);
		\draw (32) to (34);
		\draw (34) to (26);
		\draw (27) to (33);
		\draw (33) to (30);
		\draw (33) to (34);
		\draw (34) to (35);
	\end{pgfonlayer}
\end{tikzpicture}
=
\begin{tikzpicture}
	\begin{pgfonlayer}{nodelayer}
		\node [style=nothing] (21) at (0, 3.5) {};
		\node [style=nothing] (22) at (-1, 3.5) {};
		\node [style=nothing] (23) at (-0.5, 3.5) {};
		\node [style=nothing] (24) at (-1.5, 3.5) {};
		\node [style=dot] (25) at (-1.5, 4) {};
		\node [style=dot] (26) at (-1, 4) {};
		\node [style=oplus] (27) at (0, 4) {};
		\node [style=nothing] (28) at (-0.5, 5) {};
		\node [style=nothing] (29) at (-1, 5) {};
		\node [style=nothing] (30) at (0, 5) {};
		\node [style=nothing] (31) at (-1.5, 5) {};
		\node [style=dot] (32) at (-0.5, 4.5) {};
		\node [style=oplus] (33) at (0, 4.5) {};
	\end{pgfonlayer}
	\begin{pgfonlayer}{edgelayer}
		\draw (21) to (27);
		\draw (22) to (26);
		\draw (25) to (24);
		\draw (25) to (26);
		\draw (26) to (27);
		\draw (32) to (33);
		\draw (33) to (30);
		\draw (33) to (27);
		\draw (23) to (32);
		\draw (25) to (31);
		\draw (29) to (26);
		\draw (32) to (28);
	\end{pgfonlayer}
\end{tikzpicture}
$}

\item
\label{TOF.14}
{\hfil
$
\begin{tikzpicture}
	\begin{pgfonlayer}{nodelayer}
		\node [style=nothing] (22) at (0, 3.5) {};
		\node [style=nothing] (23) at (-0.5, 3.5) {};
		\node [style=nothing] (24) at (-0.5, 5.5) {};
		\node [style=nothing] (25) at (0, 5.5) {};
		\node [style=oplus] (26) at (0, 4) {};
		\node [style=oplus] (27) at (0, 5) {};
		\node [style=oplus] (28) at (-0.5, 4.5) {};
		\node [style=dot] (29) at (-0.5, 5) {};
		\node [style=dot] (30) at (0, 4.5) {};
		\node [style=dot] (31) at (-0.5, 4) {};
	\end{pgfonlayer}
	\begin{pgfonlayer}{edgelayer}
		\draw (23) to (31);
		\draw (31) to (28);
		\draw (28) to (29);
		\draw (29) to (24);
		\draw (25) to (27);
		\draw (27) to (30);
		\draw (30) to (26);
		\draw (26) to (22);
		\draw (26) to (31);
		\draw (30) to (28);
		\draw (27) to (29);
	\end{pgfonlayer}
\end{tikzpicture}
=
\begin{tikzpicture}
	\begin{pgfonlayer}{nodelayer}
		\node [style=nothing] (23) at (0, 3.5) {};
		\node [style=nothing] (24) at (-0.5, 3.5) {};
		\node [style=nothing] (25) at (-0.5, 4.5) {};
		\node [style=nothing] (26) at (0, 4.5) {};
	\end{pgfonlayer}
	\begin{pgfonlayer}{edgelayer}
		\draw [in=-90, out=90, looseness=1.25] (24) to (26);
		\draw [in=-90, out=90, looseness=1.25] (23) to (25);
	\end{pgfonlayer}
\end{tikzpicture}
$}

\item
\label{TOF.15}
{\hfil
$
\begin{tikzpicture}
	\begin{pgfonlayer}{nodelayer}
		\node [style=nothing] (24) at (-1.75, 3.5) {};
		\node [style=nothing] (25) at (-1.25, 3.5) {};
		\node [style=nothing] (26) at (-0.75, 3.5) {};
		\node [style=nothing] (27) at (-1.75, 5.5) {};
		\node [style=nothing] (28) at (-1.25, 5.5) {};
		\node [style=nothing] (29) at (-0.75, 5.5) {};
		\node [style=dot] (30) at (-1.75, 4.5) {};
		\node [style=dot] (31) at (-1.25, 4.5) {};
		\node [style=oplus] (32) at (-0.75, 4.5) {};
	\end{pgfonlayer}
	\begin{pgfonlayer}{edgelayer}
		\draw (24) to (30);
		\draw (30) to (27);
		\draw (28) to (31);
		\draw (31) to (25);
		\draw (26) to (32);
		\draw (32) to (29);
		\draw (32) to (31);
		\draw (31) to (30);
	\end{pgfonlayer}
\end{tikzpicture}
=
\begin{tikzpicture}
	\begin{pgfonlayer}{nodelayer}
		\node [style=nothing] (25) at (-1.75, 3.5) {};
		\node [style=nothing] (26) at (-1.25, 3.5) {};
		\node [style=nothing] (27) at (-0.75, 3.5) {};
		\node [style=dot] (28) at (-1.75, 4.5) {};
		\node [style=dot] (29) at (-1.25, 4.5) {};
		\node [style=oplus] (30) at (-0.75, 4.5) {};
		\node [style=nothing] (31) at (-1.75, 5.5) {};
		\node [style=nothing] (32) at (-1.25, 5.5) {};
		\node [style=nothing] (33) at (-0.75, 5.5) {};
	\end{pgfonlayer}
	\begin{pgfonlayer}{edgelayer}
		\draw [in=-90, out=90, looseness=1.25] (25) to (29);
		\draw [in=-90, out=90, looseness=1.25] (29) to (31);
		\draw [in=-90, out=90, looseness=1.25] (28) to (32);
		\draw [in=90, out=-90, looseness=1.25] (28) to (26);
		\draw (27) to (30);
		\draw (30) to (33);
		\draw (28) to (29);
		\draw (29) to (30);
	\end{pgfonlayer}
\end{tikzpicture}
$}

\item
\label{TOF.16}
{\hfil
$
\begin{tikzpicture}
	\begin{pgfonlayer}{nodelayer}
		\node [style=nothing] (26) at (0, 3.5) {};
		\node [style=nothing] (27) at (-0.5, 3.5) {};
		\node [style=nothing] (28) at (-1.5, 3.5) {};
		\node [style=nothing] (29) at (-2, 3.5) {};
		\node [style=zeroin] (30) at (-1, 3.5) {};
		\node [style=oplus] (31) at (-1, 4) {};
		\node [style=oplus] (32) at (-1, 5) {};
		\node [style=dot] (33) at (-1, 4.5) {};
		\node [style=dot] (34) at (-0.5, 4.5) {};
		\node [style=dot] (35) at (-1.5, 4) {};
		\node [style=dot] (36) at (-2, 4) {};
		\node [style=dot] (37) at (-1.5, 5) {};
		\node [style=dot] (38) at (-2, 5) {};
		\node [style=oplus] (39) at (0, 4.5) {};
		\node [style=zeroout] (40) at (-1, 5.5) {};
		\node [style=nothing] (41) at (0, 5.5) {};
		\node [style=nothing] (42) at (-2, 5.5) {};
		\node [style=nothing] (43) at (-0.5, 5.5) {};
		\node [style=nothing] (44) at (-1.5, 5.5) {};
	\end{pgfonlayer}
	\begin{pgfonlayer}{edgelayer}
		\draw (29) to (36);
		\draw (36) to (38);
		\draw (38) to (42);
		\draw (37) to (35);
		\draw (41) to (39);
		\draw (39) to (26);
		\draw (39) to (34);
		\draw (34) to (33);
		\draw (35) to (31);
		\draw (35) to (36);
		\draw (38) to (37);
		\draw (32) to (37);
		\draw (30) to (31);
		\draw (31) to (33);
		\draw (33) to (32);
		\draw (40) to (32);
		\draw [style=simple] (43) to (34);
		\draw [style=simple] (34) to (27);
		\draw [style=simple] (28) to (35);
		\draw [style=simple] (37) to (44);
	\end{pgfonlayer}
\end{tikzpicture}
=
\begin{tikzpicture}
	\begin{pgfonlayer}{nodelayer}
		\node [style=nothing] (27) at (0, 3.5) {};
		\node [style=nothing] (28) at (-0.5, 3.5) {};
		\node [style=nothing] (29) at (-1.5, 3.5) {};
		\node [style=nothing] (30) at (-2, 3.5) {};
		\node [style=zeroin] (31) at (-1, 4.25) {};
		\node [style=oplus] (32) at (-1, 4.75) {};
		\node [style=oplus] (33) at (-1, 5.75) {};
		\node [style=dot] (34) at (-1, 5.25) {};
		\node [style=dot] (35) at (-0.5, 5.25) {};
		\node [style=dot] (36) at (-1.5, 4.75) {};
		\node [style=dot] (37) at (-2, 4.75) {};
		\node [style=dot] (38) at (-1.5, 5.75) {};
		\node [style=dot] (39) at (-2, 5.75) {};
		\node [style=oplus] (40) at (0, 5.25) {};
		\node [style=zeroout] (41) at (-1, 6.25) {};
		\node [style=nothing] (42) at (0, 7) {};
		\node [style=nothing] (43) at (-2, 7) {};
		\node [style=nothing] (44) at (-0.5, 7) {};
		\node [style=nothing] (45) at (-1.5, 7) {};
		\node [style=none] (46) at (-1.5, 6.5) {};
		\node [style=none] (47) at (-0.5, 6.5) {};
		\node [style=none] (48) at (-0.5, 4) {};
		\node [style=none] (49) at (-1.5, 4) {};
	\end{pgfonlayer}
	\begin{pgfonlayer}{edgelayer}
		\draw (30) to (37);
		\draw (37) to (39);
		\draw (39) to (43);
		\draw (38) to (36);
		\draw (42) to (40);
		\draw (40) to (27);
		\draw (40) to (35);
		\draw (35) to (34);
		\draw (36) to (32);
		\draw (36) to (37);
		\draw (39) to (38);
		\draw (33) to (38);
		\draw (31) to (32);
		\draw (32) to (34);
		\draw (34) to (33);
		\draw (33) to (41);
		\draw [in=90, out=-90, looseness=0.50] (44) to (46.center);
		\draw [in=90, out=-90, looseness=0.75] (45) to (47.center);
		\draw (47.center) to (35);
		\draw (35) to (48.center);
		\draw [in=90, out=-105, looseness=0.50] (48.center) to (29);
		\draw [in=-90, out=90, looseness=0.50] (28) to (49.center);
		\draw (49.center) to (36);
		\draw (38) to (46.center);
	\end{pgfonlayer}
\end{tikzpicture}
$}
\end{enumerate}
\end{multicols}
\
\end{mdframed}
}}
\caption{The identities of \texorpdfstring{$\TOF$}{TOF}}
\label{fig:TOF}
\end{figure}


The Toffoli gate and the 1-ancillary bits allow $\cnot$, $\Not$, $\zeroin$, $\zeroout$, and flipped $\tof$ gate and flipped $\cnot$ gate can defined in this setting:

\[ \begin{array}{ccc}
\begin{tikzpicture}
	\begin{pgfonlayer}{nodelayer}
		\node [style=nothing] (28) at (0, 3.5) {};
		\node [style=nothing] (29) at (-0.5, 3.5) {};
		\node [style=nothing] (30) at (-0.5, 4.5) {};
		\node [style=nothing] (31) at (0, 4.5) {};
		\node [style=oplus] (32) at (0, 4) {};
		\node [style=dot] (33) at (-0.5, 4) {};
	\end{pgfonlayer}
	\begin{pgfonlayer}{edgelayer}
		\draw (29) to (33);
		\draw (33) to (30);
		\draw (31) to (32);
		\draw (32) to (28);
		\draw (32) to (33);
	\end{pgfonlayer}
\end{tikzpicture}
:=
\begin{tikzpicture}
	\begin{pgfonlayer}{nodelayer}
		\node [style=nothing] (29) at (0, 3.5) {};
		\node [style=nothing] (30) at (-0.5, 3.5) {};
		\node [style=nothing] (31) at (-0.5, 4.5) {};
		\node [style=nothing] (32) at (0, 4.5) {};
		\node [style=oplus] (33) at (0, 4) {};
		\node [style=dot] (34) at (-0.5, 4) {};
		\node [style=onein] (35) at (-1, 3.5) {};
		\node [style=oneout] (36) at (-1, 4.5) {};
		\node [style=dot] (37) at (-1, 4) {};
	\end{pgfonlayer}
	\begin{pgfonlayer}{edgelayer}
		\draw (30) to (34);
		\draw (34) to (31);
		\draw (32) to (33);
		\draw (33) to (29);
		\draw (33) to (34);
		\draw (35) to (37);
		\draw (37) to (36);
		\draw (37) to (34);
	\end{pgfonlayer}
\end{tikzpicture}
,
&
\begin{tikzpicture}
	\begin{pgfonlayer}{nodelayer}
		\node [style=nothing] (0) at (0, 0.5) {};
		\node [style=nothing] (1) at (0, 1.5) {};
		\node [style=oplus] (2) at (0, 1) {};
	\end{pgfonlayer}
	\begin{pgfonlayer}{edgelayer}
		\draw (1) to (2);
		\draw (2) to (0);
	\end{pgfonlayer}
\end{tikzpicture}
:=
\begin{tikzpicture}
	\begin{pgfonlayer}{nodelayer}
		\node [style=nothing] (1) at (0, 0) {};
		\node [style=nothing] (2) at (0, 1) {};
		\node [style=oplus] (3) at (0, 0.5) {};
		\node [style=dot] (4) at (-0.5, 0.5) {};
		\node [style=onein] (5) at (-0.5, 0) {};
		\node [style=oneout] (6) at (-0.5, 1) {};
	\end{pgfonlayer}
	\begin{pgfonlayer}{edgelayer}
		\draw (2) to (3);
		\draw (3) to (1);
		\draw (3) to (4);
		\draw (5) to (4);
		\draw (4) to (6);
	\end{pgfonlayer}
\end{tikzpicture}
,
&
\begin{tikzpicture}
	\begin{pgfonlayer}{nodelayer}
		\node [style=zeroin] (2) at (0, 0) {};
		\node [style=nothing] (3) at (0, 1) {};
	\end{pgfonlayer}
	\begin{pgfonlayer}{edgelayer}
		\draw (2) to (3);
	\end{pgfonlayer}
\end{tikzpicture}
:=
\begin{tikzpicture}
	\begin{pgfonlayer}{nodelayer}
		\node [style=nothing] (3) at (0, 1) {};
		\node [style=onein] (4) at (0, 0) {};
		\node [style=oplus] (5) at (0, 0.5) {};
	\end{pgfonlayer}
	\begin{pgfonlayer}{edgelayer}
		\draw (3) to (5);
		\draw (5) to (4);
	\end{pgfonlayer}
\end{tikzpicture}\\
\begin{tikzpicture}
	\begin{pgfonlayer}{nodelayer}
		\node [style=nothing] (4) at (0, 0) {};
		\node [style=zeroout] (5) at (0, 1) {};
	\end{pgfonlayer}
	\begin{pgfonlayer}{edgelayer}
		\draw (4) to (5);
	\end{pgfonlayer}
\end{tikzpicture}
:=
\begin{tikzpicture}
	\begin{pgfonlayer}{nodelayer}
		\node [style=nothing] (5) at (0, 0) {};
		\node [style=oneout] (6) at (0, 1) {};
		\node [style=oplus] (7) at (0, 0.5) {};
	\end{pgfonlayer}
	\begin{pgfonlayer}{edgelayer}
		\draw (5) to (7);
		\draw (7) to (6);
	\end{pgfonlayer}
\end{tikzpicture}
,
&
\begin{tikzpicture}
	\begin{pgfonlayer}{nodelayer}
		\node [style=nothing] (6) at (0, 0) {};
		\node [style=nothing] (7) at (-0.5, 0) {};
		\node [style=nothing] (8) at (-1, 0) {};
		\node [style=nothing] (9) at (-1, 1.5) {};
		\node [style=nothing] (10) at (-0.5, 1.5) {};
		\node [style=nothing] (11) at (0, 1.5) {};
		\node [style=dot] (12) at (0, 0.75) {};
		\node [style=dot] (13) at (-0.5, 0.75) {};
		\node [style=oplus] (14) at (-1, 0.75) {};
	\end{pgfonlayer}
	\begin{pgfonlayer}{edgelayer}
		\draw (8) to (14);
		\draw (14) to (9);
		\draw (10) to (13);
		\draw (13) to (7);
		\draw (6) to (12);
		\draw (12) to (11);
		\draw (14) to (13);
		\draw (12) to (13);
	\end{pgfonlayer}
\end{tikzpicture}
:=
\begin{tikzpicture}
	\begin{pgfonlayer}{nodelayer}
		\node [style=nothing] (7) at (0, 0) {};
		\node [style=nothing] (8) at (-0.5, 0) {};
		\node [style=nothing] (9) at (-1, 0) {};
		\node [style=nothing] (10) at (-1, 2) {};
		\node [style=nothing] (11) at (-0.5, 2) {};
		\node [style=nothing] (12) at (0, 2) {};
		\node [style=dot] (13) at (-1, 1) {};
		\node [style=dot] (14) at (-0.5, 1) {};
		\node [style=oplus] (15) at (0, 1) {};
	\end{pgfonlayer}
	\begin{pgfonlayer}{edgelayer}
		\draw [in=-90, out=90, looseness=1.25] (9) to (15);
		\draw [in=-90, out=90, looseness=1.25] (15) to (10);
		\draw (8) to (14);
		\draw (14) to (11);
		\draw [in=90, out=-90, looseness=1.25] (12) to (13);
		\draw [in=90, out=-90, looseness=1.25] (13) to (7);
		\draw (13) to (14);
		\draw (14) to (15);
	\end{pgfonlayer}
\end{tikzpicture}
,
&
\begin{tikzpicture}
	\begin{pgfonlayer}{nodelayer}
		\node [style=nothing] (8) at (0, 0) {};
		\node [style=nothing] (9) at (-0.5, 0) {};
		\node [style=nothing] (10) at (-0.5, 1) {};
		\node [style=nothing] (11) at (0, 1) {};
		\node [style=dot] (12) at (0, 0.5) {};
		\node [style=oplus] (13) at (-0.5, 0.5) {};
	\end{pgfonlayer}
	\begin{pgfonlayer}{edgelayer}
		\draw (9) to (13);
		\draw (13) to (10);
		\draw (11) to (12);
		\draw (12) to (8);
		\draw (12) to (13);
	\end{pgfonlayer}
\end{tikzpicture}
:=
\begin{tikzpicture}
	\begin{pgfonlayer}{nodelayer}
		\node [style=nothing] (9) at (0, 0) {};
		\node [style=nothing] (10) at (-0.5, 0) {};
		\node [style=nothing] (11) at (-0.5, 1.5) {};
		\node [style=nothing] (12) at (0, 1.5) {};
		\node [style=dot] (13) at (-0.5, 0.75) {};
		\node [style=oplus] (14) at (0, 0.75) {};
	\end{pgfonlayer}
	\begin{pgfonlayer}{edgelayer}
		\draw [in=-90, out=90, looseness=1.25] (10) to (14);
		\draw [in=-90, out=90, looseness=1.25] (9) to (13);
		\draw [in=-90, out=90, looseness=1.25] (13) to (12);
		\draw [in=-90, out=90, looseness=1.25] (14) to (11);
		\draw (13) to (14);
	\end{pgfonlayer}
\end{tikzpicture}
\end{array}  \]

\end{definition}




\begin{theorem}\cite{tof}
$\TOF$ is isomorphic to the category of partial isomorphisms between ordinals $2^n$, $n\in \N$.
\end{theorem}



One can moreover construct generalized controlled not gates with arbitrarily many control wires in the obvious way.  Let $[x,X]$ denote a generalized Toffoli gate acting on the $x$th wire, controlled on the wires indexed by a set $X$. Then we can partially commute generalized controlled-not gates:

\begin{lemma} \cite[Lem. 7.2.6]{cole}
\label{lemma:Iwama}
Let $[x,X]$ and $[y,Y]$ be generalized controlled not gates in $\TOF$ where $x\notin Y$.  We can perform the identities of Iwama et al. \cite{iwama}, to commute them past each other with a trailing generalized controlled not gate as a side effect:
$$
 [y,{X\cup Y}] [y,{Y\sqcup\{x\}}] [x,X]
$$
\end{lemma}

In $\TOF$, one can define the diagonal map as follows:
$$
\begin{tikzpicture}
	\begin{pgfonlayer}{nodelayer}
		\node [style=fanout] (10) at (0, 1) {};
		\node [style=none] (11) at (-0.5, 1.75) {};
		\node [style=none] (12) at (0.5, 1.75) {};
		\node [style=none] (13) at (0, 0.25) {};
	\end{pgfonlayer}
	\begin{pgfonlayer}{edgelayer}
		\draw (13.center) to (10);
		\draw [in=-90, out=124] (10) to (11.center);
		\draw [in=56, out=-90] (12.center) to (10);
	\end{pgfonlayer}
\end{tikzpicture}
:=
\begin{tikzpicture}
	\begin{pgfonlayer}{nodelayer}
		\node [style=zeroin] (11) at (0, 1) {};
		\node [style=oplus] (12) at (0, 1.5) {};
		\node [style=dot] (13) at (-0.5, 1.5) {};
		\node [style=none] (14) at (-0.5, 0.75) {};
		\node [style=none] (15) at (-0.5, 2) {};
		\node [style=none] (16) at (0, 2) {};
	\end{pgfonlayer}
	\begin{pgfonlayer}{edgelayer}
		\draw (16.center) to (11);
		\draw (12) to (13);
		\draw (15.center) to (14.center);
	\end{pgfonlayer}
\end{tikzpicture}
$$


\begin{lemma}\cite[\S 5.3.2]{cole}
The diagonal map is a natural special commutative  \dag-symmetric % \linebreak[4] 
monoidal  nonunital Frobenius algebra.
\end{lemma}

It is also natural on target qubits:
\begin{lemma}\cite[Lem. B.0.2 (iii)]{cole}
\label{lemma:natoplus}
$$
\begin{tikzpicture}
	\begin{pgfonlayer}{nodelayer}
		\node [style=fanin] (12) at (0, 2.25) {};
		\node [style=oplus] (13) at (0, 1.75) {};
		\node [style=dot] (14) at (-0.75, 1.75) {};
		\node [style=none] (15) at (-0.75, 1.25) {};
		\node [style=none] (16) at (0, 1.25) {};
		\node [style=none] (17) at (0.25, 3) {};
		\node [style=none] (18) at (-0.25, 3) {};
		\node [style=none] (19) at (-0.75, 3) {};
	\end{pgfonlayer}
	\begin{pgfonlayer}{edgelayer}
		\draw (15.center) to (14);
		\draw (14) to (19.center);
		\draw (13) to (14);
		\draw (16.center) to (13);
		\draw (13) to (12);
		\draw [in=-90, out=108] (12) to (18.center);
		\draw [in=72, out=-90] (17.center) to (12);
	\end{pgfonlayer}
\end{tikzpicture}
=
\begin{tikzpicture}
	\begin{pgfonlayer}{nodelayer}
		\node [style=fanin] (13) at (0, 0.75) {};
		\node [style=none] (14) at (-0.75, 0.25) {};
		\node [style=none] (15) at (0, 0.25) {};
		\node [style=none] (16) at (0.25, 1.5) {};
		\node [style=none] (17) at (-0.25, 1.5) {};
		\node [style=none] (18) at (-0.75, 2.5) {};
		\node [style=dot] (19) at (-0.75, 1.5) {};
		\node [style=oplus] (20) at (-0.25, 1.5) {};
		\node [style=none] (21) at (0.25, 2) {};
		\node [style=dot] (22) at (-0.75, 2) {};
		\node [style=oplus] (23) at (0.25, 2) {};
		\node [style=none] (24) at (0.25, 2.5) {};
		\node [style=none] (25) at (-0.25, 2.5) {};
	\end{pgfonlayer}
	\begin{pgfonlayer}{edgelayer}
		\draw [in=-90, out=108] (13) to (17.center);
		\draw [in=72, out=-90] (16.center) to (13);
		\draw (20) to (19);
		\draw (18.center) to (14.center);
		\draw (23) to (22);
		\draw (24.center) to (21.center);
		\draw (21.center) to (16.center);
		\draw (13) to (15.center);
		\draw (17.center) to (25.center);
	\end{pgfonlayer}
\end{tikzpicture}
$$
\end{lemma}


\subsubsection{Adding a unit and counit to  \texorpdfstring{$\TOF$}{TOF}}


By adding a unit and counit, we obtain a full subcategory of spans of sets and finite ordinals:

\begin{lemma}
\label{lemma:unitcounit}

 The full subcategory of $\Span^\sim(\FinOrd)$ generated by powers of 2 is presented by the pushout,  $\hat \TOF$, of the following diagram of props:

$$c(\TOF)^\op \leftarrow \TOF \rightarrow c(\TOF)$$
\end{lemma}



\begin{proof}
Recall that $\TOF$ is presented by the subcategory $\FPinj_2$ of $(\Span^\sim (\FinOrd),\times)$ with morphisms of the form $ 2^n \xleftarrowtail{e} k \xrightarrowtail{e'} 2^m$ for arbitrary natural numbers $n,m,k$ and monics $e$ and $e'$.



Similarly, $\tilde \TOF$ is presented by the subcategory $\FPar_2$ of  $(\Span^\sim (\FinOrd),\times)$ with morphisms of the form $2^\ell \xleftarrow{f} 2^n \xleftarrowtail{e} k \xrightarrowtail{e'} 2^m$ for arbitrary natural numbers $\ell, n,m,k$ and monics $e$ and $e'$ and function $f$.
Let $\FSpan_2$ denote the full subcategory of $(\Span^\sim(\FinOrd),\times)$ generated by powers of two.
Consider the pushout $\X$ of the following diagram of props:

$$\FPar_2^\op \xleftarrowtail{}  \FPinj_2 \xrightarrowtail{} \FPar_2$$ 


Consider the functor $F:\X\to\FSpan_2$ induced by the universal property of the pushout.  We show that this functor is an isomorphism.
This functor is clearly the identity on objects.

For fullness consider some span $2^n \xleftarrow{f} k \xrightarrow{g} 2^m$. We can construct a function $f':2^{\lceil \log_2 k \rceil} \rightarrow 2^n$ and monic $e_f: k \xrightarrowtail{} 2^{\lceil \log_2 k \rceil}$ so that $f=ef'$.  Similarly, we can construct some  $g':2^{\lceil \log_2 k \rceil} \rightarrow 2^n$ and monic $e_g: k \xrightarrowtail{} 2^{\lceil \log_2 k \rceil}$ so that $g=e_gg'$.  Therefore:


\begin{align*}
F&\left(
\xymatrix{
         & 2^{\lceil \log_2 k \rceil} \ar[dl]_{f'} \ar@{=}[dr]\\
2^n &                                                                                 &2^{\lceil \log_2 k \rceil}
};
\xymatrix{
         & k \ar@{>->}[dl]_{e_f} \ar@{>->}[dr]^{e_m}\\
2^{\lceil \log_2 k \rceil} &                                                                                 & 2^{\lceil \log_2 k \rceil}
};
\xymatrix{
                                       & 2^{\lceil \log_2 k \rceil} \ar[dr]^{g'} \ar@{=}[dl]\\
2^{\lceil \log_2 k \rceil} &                                                                                 & 2^m
}
\right)\\
&=
\xymatrix{
         &                                                                               &                                             &  k \ar@{=}[dl] \ar@{=}[dr] \ar@/_2.0pc/[dddlll]_{f} \ar@/^2.0pc/[dddrrr]^{g}  \\
         &                                                                               &k \ar@{=}[dr] \ar@{>->}[dl]_{e_f} &                                                & k \ar@{=}[dl] \ar@{>->}[dr]^{e_g}\\
         & 2^{\lceil \log_2 k \rceil}\ar[dl]^{f'}\ar@{=}[dr]   &                                             & k \ar@{>->}[dl]_{e_f} \ar@{>->}[dr]^{e_g} &                                       & 2^{\lceil \log_2 k \rceil}\ar[dr]_{g'}\ar@{=}[dl] \\
2^n  &                                                                                & 2^{\lceil \log_2 k \rceil}     &                                                & 2^{\lceil \log_2 k \rceil} &                   & 2^m
}
\end{align*}

So $F$ is full.


For faithfulness suppose we have any two isomorphic spans in $F(\X)$:

$$
\xymatrix{
                 &                                       & k \ar@{>->}[dl]_{e_1}\ar@{->}[dddd]_\cong^{\alpha} \ar@{>->}[dr]^{e_2} \\ 
                 & 2^{n_2} \ar[dl]_{f_1}   &                                                                              & 2^{n_3} \ar[dr]^{f_2}\\ 
2^{n_1}   &                                       &                                                                              &                 & 2^{n_4}\\
                 & 2^{n_2'} \ar[ul]^{f_1'} &                                                                              & 2^{n_3'} \ar[ur]_{f_2'}\\ 
                 &                                       & k \ar@{>->}[ul]^{e_1'} \ar@{>->}[ur]_{e_2'} \\ 
}
$$



In $\X$, we have:

\begin{align*}
\xymatrix{
                & 2^{n_2} \ar[dl]_{f_1} \ar@{=}[dr] \\
2^{n_1} &                                                             & 2^{n_2}
};&
\xymatrix{
               & k \ar@{>->}[dl]_{e_1} \ar@{>->}[dr]^{e_2}\\
2^{n_2} &                                               & 2^{n_3}
};
\xymatrix{
                & 2^{n_3} \ar@{=}[dl] \ar[dr]^{f_2} \\
2^{n_3} &                                                             & 2^{n_4}
}\\
&=
\xymatrix{
                 &                                                           & k \ar@{>->}[dl]_{e_1} \ar@{=}[dr]  \ar@/_2.0pc/[ddll]_{\alpha e_1' f_1'}\\
                & 2^{n_2} \ar[dl]_{f_1} \ar@{=}[dr]   &                         & k \ar@{>->}[dl]_{e_1 } \ar@{>->}[dr]^{e_2}  \\
2^{n_1} &                                                             & 2^{n_2}          &                                                 & 2^{n_3}
};
\xymatrix{
                & 2^{n_3} \ar@{=}[dl] \ar[dr]^{f_2} \\
2^{n_3} &                                                             & 2^{n_4}
}\\
&=
\xymatrix{
                 &                                                           & k \ar@{>->}[dl]_{\alpha e_1'} \ar@{>->}[ddrr]^{e_2} \\
                & 2^{n_2'} \ar[dl]_{f_1'}                        &                         &  \\
2^{n_1} &                                                             &           &                                                 & 2^{n_3}
};
\xymatrix{
                & 2^{n_3} \ar@{=}[dl] \ar[dr]^{f_2} \\
2^{n_3} &                                                             & 2^{n_4}
}\\
&=
\xymatrix{
                & 2^{n_2'} \ar[dl]_{f_1'} \ar@{=}[dr] \\
2^{n_1} &                                                             & 2^{n_2'}
};
\xymatrix{
               & k \ar@{>->}[dl]_{\alpha e_1'} \ar@{>->}[dr]^{e_2}\\
2^{n_2'} &                                               & 2^{n_3}
};
\xymatrix{
                & 2^{n_3} \ar@{=}[dl] \ar[dr]^{f_2} \\
2^{n_3} &                                                             & 2^{n_4}
}\\
&=
\xymatrix{
                & 2^{n_2'} \ar[dl]_{f_1'} \ar@{=}[dr] \\
2^{n_1} &                                                             & 2^{n_2'}
};
\xymatrix{
                                       & k \ar@{>->}[dl]_{\alpha e_1'} \ar@{>->}[dr]^{\alpha e_2'} \ar[dd]_\cong^\alpha\\
2^{n_2'} &                                                                         & 2^{n_3'} \\
                                       & k  \ar@{>->}[ul]^{e_1'} \ar@{>->}[ur]_{e_2'}
};
\xymatrix{
                & 2^{n_3'} \ar@{=}[dl] \ar[dr]^{f_2} \\
2^{n_3'} &                                                             & 2^{n_4}
}\\
&=
\xymatrix{
                & 2^{n_2'} \ar[dl]_{f_1'} \ar@{=}[dr] \\
2^{n_1} &                                                             & 2^{n_2'}
};
\xymatrix{
               & k \ar@{>->}[dl]_{ e_1'} \ar@{>->}[dr]^{ e_2'}\\
2^{n_2'} &                                               & 2^{n_3'}
};
\xymatrix{
                & 2^{n_3'} \ar@{=}[dl] \ar[dr]^{f_2} \\
2^{n_3'} &                                                             & 2^{n_4}
}
\end{align*}

Therefore $\FSpan_2 \cong \X$.


Two show that $\hat \TOF \cong \FSpan_2$, consider the following diagram where each horizontal face is a pushout:


$$
\xymatrixrowsep{6mm}\xymatrixcolsep{4mm}
\xymatrix{
                                       & {(\FPinj_2,\times)} \ar[dl] \ar@/^.5pc/[rr] \ar@{=}[d]  &                                                  & (\FPar_2,\times) \ar[d]^{\cong} \ar[dl] \\
 (\FPar_2,\times)^\op \ar@/_1pc/[rr]  \ar[d]_{\cong}           &                   {(\FPinj_2,\times)}\ar[dl] \ar@/^.5pc/[rr]    \ar[d]^\cong                                                                       & (\FSpan_2,\times)    \ar@{-->}[d]^(.35){\cong}    & \tilde{(\FPinj_2,\times)} \ar[dl]       \ar[d]^\cong       \\
\tilde{(\FPinj_2,\times)}^\op \ar@/_1pc/[rr]            \ar[d]_{\cong}                               &      \TOF \ar[dl] \ar@/^.5pc/[rr]  \ar@{=}[d]       &                                  \ar@{-->}[d]^(.35){\cong}             & \tilde \TOF  \ar[d]_{\cong} \ar[dl]\\
\tilde{\TOF}^\op \ar@/_1pc/[rr]   \ar[d]_{\cong}   &                  \TOF \ar[dl] \ar@/^.5pc/[rr]                                                                      &  \ar@{-->}[d]^(.35){\cong}  & c(\TOF)  \ar[dl]\\
c(\TOF)^\op        \ar@/_1pc/[rr]                          &                                                                                             &          \hat\TOF &                        &            \\
}
$$


All of the rear and left faces commute. Moreover, the vertical maps are isomorphisms, therefore the maps induced by universal property of the pushout are isomorphisms.







\end{proof}



If $f$ is a partial isomorphism between finite sets, then the white spiders correspond to the classical structure for the chosen computational basis.  For the interpretation into $\FHilb$ via the $\ell_2$ functor, this means that in the  qubit case, the unit and counit correspond to $\sqrt{2}|+\rangle$ and $\sqrt{2}\langle +|$. 


We give a more elegant presentation of this category in terms of interacting monoids and %\linebreak[4]
 comonoids:

\begin{definition}
Consider the self dual prop $\ZXA$ generated by the addition spider with phases in $\{0,\pi\}$, the copy spider and the monoid for conjunction satisfying the  identities given in Figure \ref{fig:ZXA}.


\begin{figure}%[t]
	\noindent
	\scalebox{1.0}{%
		\vbox{%
			\begin{mdframed}
				\begin{multicols}{2}
					\begin{enumerate}[label={\bf [ZX{\it \&}.\arabic*]}, ref={\bf [ZX{\it \&}.\arabic*]}, wide = 0pt, leftmargin = 2em]
						\item
						\label{ZXA.1}
						{\hfil
							$
\begin{tikzpicture}
	\begin{pgfonlayer}{nodelayer}
		\node [style=none] (0) at (0, 0.5) {};
		\node [style=none] (1) at (1, 0.5) {};
		\node [style=none] (2) at (0, 2.75) {};
		\node [style=none] (3) at (1, 2.75) {};
		\node [style=Z] (4) at (0.5, 1.25) {$\alpha$};
		\node [style=Z] (5) at (0.5, 2) {$\beta$};
		\node [style=none] (6) at (0.5, 2.5) {$\vdots$};
		\node [style=none] (7) at (0.5, 0.75) {$\vdots$};
	\end{pgfonlayer}
	\begin{pgfonlayer}{edgelayer}
		\draw [style=simple, in=-56, out=90] (1.center) to (4);
		\draw [style=simple, in=90, out=-124] (4) to (0.center);
		\draw [style=simple, in=-90, out=124] (5) to (2.center);
		\draw [style=simple, in=-90, out=56] (5) to (3.center);
		\draw [style=simple] (5) to (4);
	\end{pgfonlayer}
\end{tikzpicture}
=
\begin{tikzpicture}
	\begin{pgfonlayer}{nodelayer}
		\node [style=none] (0) at (0, 0.5) {};
		\node [style=none] (1) at (1, 0.5) {};
		\node [style=none] (2) at (0.5, 0.5) {$\vdots$};
		\node [style=none] (3) at (0.5, 2) {$\vdots$};
		\node [style=none] (4) at (1, 2) {};
		\node [style=Z] (5) at (0.5, 1.25) {$\alpha+\beta$};
		\node [style=none] (6) at (0, 2) {};
	\end{pgfonlayer}
	\begin{pgfonlayer}{edgelayer}
		\draw [style=simple, in=56, out=-90] (4.center) to (5);
		\draw [style=simple, in=-90, out=124] (5) to (6.center);
		\draw [style=simple, in=90, out=-124] (5) to (0.center);
		\draw [style=simple, in=-56, out=90] (1.center) to (5);
	\end{pgfonlayer}
\end{tikzpicture}
							$
						}

						\item
						\label{ZXA.2}
						{\hfil
							$
\begin{tikzpicture}
	\begin{pgfonlayer}{nodelayer}
		\node [style=none] (0) at (0, 2.25) {};
		\node [style=none] (1) at (1, 2.25) {};
		\node [style=Z] (2) at (0.5, 1.5) {$\alpha$};
		\node [style=none] (3) at (0.5, 2) {$\vdots$};
		\node [style=none] (4) at (0.25, 0.5) {};
		\node [style=none] (5) at (0.75, 0.5) {};
		\node [style=none] (6) at (0.75, 1) {};
		\node [style=none] (7) at (0.25, 1) {};
	\end{pgfonlayer}
	\begin{pgfonlayer}{edgelayer}
		\draw [style=simple, in=56, out=-90] (1.center) to (2);
		\draw [style=simple, in=-90, out=124] (2) to (0.center);
		\draw [style=simple, in=90, out=-63] (2) to (6.center);
		\draw [style=simple, in=90, out=-90] (6.center) to (4.center);
		\draw [style=simple, in=-90, out=90] (5.center) to (7.center);
		\draw [style=simple, in=-117, out=90] (7.center) to (2);
	\end{pgfonlayer}
\end{tikzpicture}
=
\begin{tikzpicture}
	\begin{pgfonlayer}{nodelayer}
		\node [style=none] (0) at (0, 2) {};
		\node [style=none] (1) at (1, 2) {};
		\node [style=Z] (2) at (0.5, 1.25) {$\alpha$};
		\node [style=none] (3) at (0.5, 1.75) {$\vdots$};
		\node [style=none] (4) at (0.25, 0.5) {};
		\node [style=none] (5) at (0.75, 0.5) {};
	\end{pgfonlayer}
	\begin{pgfonlayer}{edgelayer}
		\draw [style=simple, in=56, out=-90] (1.center) to (2);
		\draw [style=simple, in=-90, out=124] (2) to (0.center);
		\draw [style=simple, in=-56, out=90] (5.center) to (2);
		\draw [style=simple, in=-124, out=90] (4.center) to (2);
	\end{pgfonlayer}
\end{tikzpicture}
							$
						}

						\item
						\label{ZXA.3}
						{\hfil
							$
\begin{tikzpicture}
	\begin{pgfonlayer}{nodelayer}
		\node [style=none] (0) at (0, 0.5) {};
		\node [style=none] (1) at (1, 0.5) {};
		\node [style=none] (2) at (0, 2.75) {};
		\node [style=none] (3) at (1, 2.75) {};
		\node [style=X] (4) at (0.5, 1.25) {};
		\node [style=X] (5) at (0.5, 2) {};
		\node [style=none] (6) at (0.5, 2.5) {$\vdots$};
		\node [style=none] (7) at (0.5, 0.75) {$\vdots$};
		\node [style=none] (8) at (0.45, 1.625) {\scalebox{.8}{$\vdots$}};
	\end{pgfonlayer}
	\begin{pgfonlayer}{edgelayer}
		\draw [style=simple, in=-56, out=90] (1.center) to (4);
		\draw [style=simple, in=90, out=-124] (4) to (0.center);
		\draw [style=simple, in=-135, out=135, looseness=1.25] (4) to (5);
		\draw [style=simple, in=45, out=-45, looseness=1.25] (5) to (4);
		\draw [style=simple, in=-90, out=124] (5) to (2.center);
		\draw [style=simple, in=-90, out=56] (5) to (3.center);
	\end{pgfonlayer}
\end{tikzpicture}
=
\begin{tikzpicture}
	\begin{pgfonlayer}{nodelayer}
		\node [style=none] (0) at (0, 0.5) {};
		\node [style=none] (1) at (1, 0.5) {};
		\node [style=X] (2) at (0.5, 1.25) {};
		\node [style=none] (3) at (0.5, 0.75) {$\vdots$};
		\node [style=none] (4) at (0.5, 1.75) {$\vdots$};
		\node [style=none] (5) at (1, 2) {};
		\node [style=X] (6) at (0.5, 1.25) {};
		\node [style=none] (7) at (0, 2) {};
	\end{pgfonlayer}
	\begin{pgfonlayer}{edgelayer}
		\draw [style=simple, in=-56, out=90] (1.center) to (2);
		\draw [style=simple, in=90, out=-124] (2) to (0.center);
		\draw [style=simple, in=56, out=-90] (5.center) to (6);
		\draw [style=simple, in=-90, out=124] (6) to (7.center);
	\end{pgfonlayer}
\end{tikzpicture}
							$
						}



						\item
						\label{ZXA.4}
						{\hfil
							$
\begin{tikzpicture}
	\begin{pgfonlayer}{nodelayer}
		\node [style=none] (0) at (0, 0.5) {};
		\node [style=none] (1) at (1, 0.5) {};
		\node [style=X] (2) at (0.5, 1.25) {};
		\node [style=none] (3) at (0.5, 0.75) {$\vdots$};
		\node [style=none] (4) at (0.25, 2.25) {};
		\node [style=none] (5) at (0.75, 2.25) {};
		\node [style=none] (6) at (0.75, 1.75) {};
		\node [style=none] (7) at (0.25, 1.75) {};
	\end{pgfonlayer}
	\begin{pgfonlayer}{edgelayer}
		\draw [style=simple, in=-56, out=90] (1.center) to (2);
		\draw [style=simple, in=90, out=-124] (2) to (0.center);
		\draw [style=simple, in=-90, out=63] (2) to (6.center);
		\draw [style=simple, in=-90, out=90] (6.center) to (4.center);
		\draw [style=simple, in=90, out=-90] (5.center) to (7.center);
		\draw [style=simple, in=117, out=-90] (7.center) to (2);
	\end{pgfonlayer}
\end{tikzpicture}
=
\begin{tikzpicture}
	\begin{pgfonlayer}{nodelayer}
		\node [style=none] (0) at (0, 0.5) {};
		\node [style=none] (1) at (1, 0.5) {};
		\node [style=X] (2) at (0.5, 1.25) {};
		\node [style=none] (3) at (0.5, 0.75) {$\vdots$};
		\node [style=none] (4) at (0.25, 2) {};
		\node [style=none] (5) at (0.75, 2) {};
	\end{pgfonlayer}
	\begin{pgfonlayer}{edgelayer}
		\draw [style=simple, in=-56, out=90] (1.center) to (2);
		\draw [style=simple, in=90, out=-124] (2) to (0.center);
		\draw [style=simple, in=56, out=-90] (5.center) to (2);
		\draw [style=simple, in=124, out=-90] (4.center) to (2);
	\end{pgfonlayer}
\end{tikzpicture}
							$
						}
						
						
						\item
						\label{ZXA.5}
						{\hfil
							$
\begin{tikzpicture}
	\begin{pgfonlayer}{nodelayer}
		\node [style=Z] (0) at (-1, 1) {};
		\node [style=none] (1) at (-1.25, 0.5) {};
		\node [style=none] (2) at (-0.75, 0.5) {};
		\node [style=X] (3) at (-1, 1.75) {};
		\node [style=none] (4) at (-1.25, 2.25) {};
		\node [style=none] (5) at (-0.75, 2.25) {};
	\end{pgfonlayer}
	\begin{pgfonlayer}{edgelayer}
		\draw [in=63, out=-90] (5.center) to (3);
		\draw (3) to (0);
		\draw [in=90, out=-117] (0) to (1.center);
		\draw [in=-63, out=90] (2.center) to (0);
		\draw [in=-90, out=117] (3) to (4.center);
	\end{pgfonlayer}
\end{tikzpicture}
=
\begin{tikzpicture}
	\begin{pgfonlayer}{nodelayer}
		\node [style=X] (0) at (-1, 1) {};
		\node [style=X] (1) at (-0.25, 1) {};
		\node [style=Z] (2) at (-0.25, 1.75) {};
		\node [style=Z] (3) at (-1, 1.75) {};
		\node [style=none] (4) at (-1, 2.25) {};
		\node [style=none] (5) at (-0.25, 2.25) {};
		\node [style=none] (6) at (-1, 0.5) {};
		\node [style=none] (7) at (-0.25, 0.5) {};
	\end{pgfonlayer}
	\begin{pgfonlayer}{edgelayer}
		\draw (7.center) to (1);
		\draw (1) to (3);
		\draw [in=120, out=-120, looseness=1.25] (3) to (0);
		\draw (0) to (2);
		\draw (2) to (5.center);
		\draw [in=60, out=-60, looseness=1.25] (2) to (1);
		\draw (0) to (6.center);
		\draw (3) to (4.center);
	\end{pgfonlayer}
\end{tikzpicture}
							$
						}
						
											
\item
	\label{ZXA.6}

	{\hfil\hspace*{.5cm}
							$
\begin{tikzpicture}
	\begin{pgfonlayer}{nodelayer}
		\node [style=none] (0) at (-0.25, 2) {};
		\node [style=X] (1) at (0, 1.25) {};
		\node [style=Z] (2) at (0, 0.5) {};
		\node [style=none] (3) at (0.25, 2) {};
	\end{pgfonlayer}
	\begin{pgfonlayer}{edgelayer}
		\draw [style=simple, in=-90, out=124] (1) to (0.center);
		\draw [style=simple, in=60, out=-90] (3.center) to (1);
		\draw [style=simple] (1) to (2);
	\end{pgfonlayer}
\end{tikzpicture}
=
\begin{tikzpicture}
	\begin{pgfonlayer}{nodelayer}
		\node [style=none] (0) at (-0.25, 1) {};
		\node [style=Z] (1) at (-0.25, 0.5) {};
		\node [style=none] (2) at (0.25, 1) {};
		\node [style=Z] (3) at (0.25, 0.5) {};
	\end{pgfonlayer}
	\begin{pgfonlayer}{edgelayer}
		\draw [style=simple] (3) to (2.center);
		\draw [style=simple] (1) to (0.center);
	\end{pgfonlayer}
\end{tikzpicture}
							$
						}
						
						\item
						\label{ZXA.7}
						{\hfil
							$
\begin{tikzpicture}
	\begin{pgfonlayer}{nodelayer}
		\node [style=Z] (0) at (0, 4.5) {};
		\node [style=X] (1) at (0, 5.25) {};
	\end{pgfonlayer}
	\begin{pgfonlayer}{edgelayer}
		\draw (1) to (0);
	\end{pgfonlayer}
\end{tikzpicture}
=
\begin{tikzpicture}
	\begin{pgfonlayer}{nodelayer}
	\end{pgfonlayer}
\end{tikzpicture}
							$
						}
						
						
							\item
						\label{ZXA.8}
						{\hfil
							$
\begin{tikzpicture}
	\begin{pgfonlayer}{nodelayer}
		\node [style=Z] (0) at (-1, 3) {};
		\node [style=X] (1) at (-1, 2.25) {};
		\node [style=none] (2) at (-1, 3.5) {};
		\node [style=none] (3) at (-1, 1.75) {};
	\end{pgfonlayer}
	\begin{pgfonlayer}{edgelayer}
		\draw (2.center) to (0);
		\draw (1) to (3.center);
	\end{pgfonlayer}
\end{tikzpicture}
=
\begin{tikzpicture}
	\begin{pgfonlayer}{nodelayer}
		\node [style=Z] (0) at (-1, 3) {};
		\node [style=X] (1) at (-1, 2.25) {};
		\node [style=none] (2) at (-1, 3.5) {};
		\node [style=none] (3) at (-1, 1.75) {};
	\end{pgfonlayer}
	\begin{pgfonlayer}{edgelayer}
		\draw (2.center) to (0);
		\draw [in=120, out=-120, looseness=1.25] (0) to (1);
		\draw [in=-60, out=60, looseness=1.25] (1) to (0);
		\draw (1) to (3.center);
	\end{pgfonlayer}
\end{tikzpicture}
							$
						}

\item
						\label{ZXA.9}
						{\hfil
							$
\begin{tikzpicture}
	\begin{pgfonlayer}{nodelayer}
		\node [style=none] (0) at (0, 3) {};
		\node [style=none] (1) at (0, 3.5) {};
		\node [style=none] (2) at (0, 2.25) {};
		\node [style=none] (3) at (-0.25, 1.5) {};
		\node [style=none] (4) at (0.25, 1.5) {};
		\node [style=none] (5) at (0.5, 1.5) {};
		\node [style=none] (6) at (1, 1.5) {};
		\node [style=none] (7) at (-0.5, 1.5) {};
		\node [style=none] (8) at (-1, 1.5) {};
		\node [style=none] (9) at (-0.8, 1.7) {$\vdots$};
		\node [style=none] (10) at (-0.055, 1.7) {$\vdots$};
		\node [style=none] (11) at (0.68, 1.7) {$\vdots$};
		\node [style=andin] (12) at (0, 2.25) {};
		\node [style=andin] (13) at (0, 3) {};
	\end{pgfonlayer}
	\begin{pgfonlayer}{edgelayer}
		\draw [style=simple, in=-90, out=90] (0.center) to (1.center);
		\draw [style=simple, in=-90, out=90] (2.center) to (0.center);
		\draw [style=simple, in=-63, out=90] (4.center) to (2.center);
		\draw [style=simple, in=90, out=-117] (2.center) to (3.center);
		\draw [style=simple, in=-120, out=90] (7.center) to (0.center);
		\draw [style=simple, in=90, out=-135] (0.center) to (8.center);
		\draw [style=simple, in=-60, out=90] (5.center) to (0.center);
		\draw [style=simple, in=-45, out=90] (6.center) to (0.center);
	\end{pgfonlayer}
\end{tikzpicture}
=
\begin{tikzpicture}
	\begin{pgfonlayer}{nodelayer}
		\node [style=andin] (0) at (0, 3) {};
		\node [style=none] (1) at (1, 1.5) {};
		\node [style=none] (2) at (0.25, 1.5) {};
		\node [style=none] (3) at (-0.5, 1.5) {};
		\node [style=none] (4) at (-1, 1.5) {};
		\node [style=none] (5) at (0.5, 1.5) {};
		\node [style=none] (6) at (-0.8, 1.7) {$\vdots$};
		\node [style=none] (7) at (0, 3) {};
		\node [style=none] (8) at (0.68, 1.7) {$\vdots$};
		\node [style=none] (9) at (-0.055, 1.7) {$\vdots$};
		\node [style=none] (10) at (-0.25, 1.5) {};
		\node [style=none] (11) at (0, 3.5) {};
		\node [style=none] (12) at (0, 3) {};
	\end{pgfonlayer}
	\begin{pgfonlayer}{edgelayer}
		\draw [style=simple, in=-45, out=90] (1.center) to (7.center);
		\draw [style=simple, in=90, out=-135] (7.center) to (4.center);
		\draw [style=simple, in=90, out=-105] (12.center) to (10.center);
		\draw [style=simple, in=-120, out=90] (3.center) to (7.center);
		\draw [style=simple, in=-60, out=90] (5.center) to (7.center);
		\draw [style=simple, in=-75, out=90] (2.center) to (12.center);
		\draw [style=simple, in=-90, out=90] (7.center) to (11.center);
		\draw [style=simple, in=-90, out=90] (12.center) to (7.center);
	\end{pgfonlayer}
\end{tikzpicture}
							$
						}



						\item
						\label{ZXA.10}
						{\hfil
							$
\begin{tikzpicture}
	\begin{pgfonlayer}{nodelayer}
		\node [style=none] (0) at (-1, 2) {};
		\node [style=andin] (10) at (-1, 2) {};
		\node [style=none] (1) at (-1, 2.5) {};
		\node [style=none] (2) at (-0.75, 1.25) {};
		\node [style=Z] (3) at (-1.25, 1.25) {$\pi$};
	\end{pgfonlayer}
	\begin{pgfonlayer}{edgelayer}
		\draw (0) to (1.center);
		\draw [in=90, out=-108] (0) to (3);
		\draw [in=-72, out=90] (2.center) to (0);
	\end{pgfonlayer}
\end{tikzpicture}
=
\begin{tikzpicture}
	\begin{pgfonlayer}{nodelayer}
		\node [style=none] (0) at (-1, 2.5) {};
		\node [style=none] (1) at (-1, 1.75) {};
	\end{pgfonlayer}
	\begin{pgfonlayer}{edgelayer}
		\draw (0.center) to (1.center);
	\end{pgfonlayer}
\end{tikzpicture}
							$
						}

						\item
						\label{ZXA.11}
						{\hfil
							$
\begin{tikzpicture}
	\begin{pgfonlayer}{nodelayer}
		\node [style=andin] (10) at (0, 2.5) {};
		\node [style=none] (0) at (0, 2.5) {};
		\node [style=none] (1) at (-0.25, 2) {};
		\node [style=none] (2) at (0.25, 2) {};
		\node [style=none] (3) at (0, 3) {};
		\node [style=none] (4) at (-0.25, 1.5) {};
		\node [style=none] (5) at (0.25, 1.5) {};
	\end{pgfonlayer}
	\begin{pgfonlayer}{edgelayer}
		\draw [in=-63, out=90] (2.center) to (0);
		\draw [in=90, out=-117, looseness=1.25] (0) to (1.center);
		\draw (3.center) to (0);
		\draw [in=-90, out=90, looseness=1.25] (5.center) to (1.center);
		\draw [in=90, out=-90, looseness=1.25] (2.center) to (4.center);
	\end{pgfonlayer}
\end{tikzpicture}
=
\begin{tikzpicture}
	\begin{pgfonlayer}{nodelayer}
		\node [style=andin] (10) at (0, 2.5) {};
		\node [style=none] (0) at (0, 2.5) {};
		\node [style=none] (1) at (-0.25, 2) {};
		\node [style=none] (2) at (0.25, 2) {};
		\node [style=none] (3) at (0, 3) {};
	\end{pgfonlayer}
	\begin{pgfonlayer}{edgelayer}
		\draw [in=-63, out=90] (2.center) to (0);
		\draw [in=90, out=-117, looseness=1.25] (0) to (1.center);
		\draw (3.center) to (0);
	\end{pgfonlayer}
\end{tikzpicture}
							$
						}

						\item
						\label{ZXA.12}
						{\hfil
							$
\begin{tikzpicture}
	\begin{pgfonlayer}{nodelayer}
		\node [style=andin] (10) at (-1, 1) {};
		\node [style=none] (0) at (-1, 1) {};
		\node [style=none] (1) at (-1.25, 0.5) {};
		\node [style=none] (2) at (-0.75, 0.5) {};
		\node [style=X] (3) at (-1, 1.75) {};
		\node [style=none] (4) at (-1.25, 2.25) {};
		\node [style=none] (5) at (-0.75, 2.25) {};
	\end{pgfonlayer}
	\begin{pgfonlayer}{edgelayer}
		\draw [in=63, out=-90] (5.center) to (3);
		\draw (3) to (0);
		\draw [in=90, out=-117] (0) to (1.center);
		\draw [in=-63, out=90] (2.center) to (0);
		\draw [in=-90, out=117] (3) to (4.center);
	\end{pgfonlayer}
\end{tikzpicture}
=
\begin{tikzpicture}
	\begin{pgfonlayer}{nodelayer}
		\node [style=X] (0) at (-1, 1) {};
		\node [style=X] (1) at (-0.25, 1) {};
		\node [style=andin] (2) at (-0.25, 2) {};
		\node [style=andin] (3) at (-1, 2) {};
		\node [style=none] (4) at (-1, 2.5) {};
		\node [style=none] (5) at (-0.25, 2.5) {};
		\node [style=none] (6) at (-1, 0.5) {};
		\node [style=none] (7) at (-0.25, 0.5) {};
	\end{pgfonlayer}
	\begin{pgfonlayer}{edgelayer}
		\draw (7.center) to (1);
		\draw [in=-60, out=127] (1) to (3.center);
		\draw [in=120, out=-120, looseness=1.25] (3.center) to (0);
		\draw [in=-120, out=53] (0) to (2.center);
		\draw (2.center) to (5.center);
		\draw [in=60, out=-60, looseness=1.25] (2.center) to (1);
		\draw (0) to (6.center);
		\draw (3.center) to (4.center);
	\end{pgfonlayer}
\end{tikzpicture}
							$
						}


					\item
					\label{ZXA.13}
						{\hfil
							$
\begin{tikzpicture}
	\begin{pgfonlayer}{nodelayer}
		\node [style=none] (0) at (-0.5, 4.25) {};
		\node [style=none] (1) at (0, 3.5) {};
		\node [style=none] (2) at (-1, 3.5) {};
		\node [style=X] (3) at (-0.5, 5) {};
		\node [style=andin] (4) at (-0.5, 4.25) {};
	\end{pgfonlayer}
	\begin{pgfonlayer}{edgelayer}
		\draw [in=90, out=-135] (0.center) to (2.center);
		\draw [in=-41, out=90] (1.center) to (0.center);
		\draw (3) to (0.center);
	\end{pgfonlayer}
\end{tikzpicture}
		=
\begin{tikzpicture}
	\begin{pgfonlayer}{nodelayer}
		\node [style=none] (0) at (-0.5, 3.5) {};
		\node [style=none] (1) at (-1, 3.5) {};
		\node [style=X] (2) at (-1, 4.25) {};
		\node [style=X] (3) at (-0.5, 4.25) {};
	\end{pgfonlayer}
	\begin{pgfonlayer}{edgelayer}
		\draw (3) to (0.center);
		\draw (2) to (1.center);
	\end{pgfonlayer}
\end{tikzpicture}
$
						}

						\item
						\label{ZXA.14}
						{\hfil
							$
\begin{tikzpicture}
	\begin{pgfonlayer}{nodelayer}
		\node [style=none] (0) at (-0.25, 2) {};
		\node [style=X] (1) at (0, 1.25) {};
		\node [style=Z] (2) at (0, 0.5) {$\pi$};
		\node [style=none] (3) at (0.25, 2) {};
	\end{pgfonlayer}
	\begin{pgfonlayer}{edgelayer}
		\draw [style=simple, in=-90, out=124] (1) to (0.center);
		\draw [style=simple, in=60, out=-90] (3.center) to (1);
		\draw [style=simple] (1) to (2);
	\end{pgfonlayer}
\end{tikzpicture}
=
\begin{tikzpicture}
	\begin{pgfonlayer}{nodelayer}
		\node [style=none] (0) at (-0.25, 1) {};
		\node [style=Z] (1) at (-0.25, 0.5) {$\pi$};
		\node [style=none] (2) at (0.25, 1) {};
		\node [style=Z] (3) at (0.25, 0.5) {$\pi$};
	\end{pgfonlayer}
	\begin{pgfonlayer}{edgelayer}
		\draw [style=simple] (3) to (2.center);
		\draw [style=simple] (1) to (0.center);
	\end{pgfonlayer}
\end{tikzpicture}
							$
						}
						
						
						

					

						\item
						\label{ZXA.15}
						{\hfil
							$
\begin{tikzpicture}
	\begin{pgfonlayer}{nodelayer}
		\node [style=none] (0) at (-1, 2) {};
		\node [style=none] (1) at (-1, 1) {};
	\end{pgfonlayer}
	\begin{pgfonlayer}{edgelayer}
		\draw (0.center) to (1.center);
	\end{pgfonlayer}
\end{tikzpicture}
=
\begin{tikzpicture}
	\begin{pgfonlayer}{nodelayer}
		\node [style=X] (0) at (-1, 1.5) {};
		\node [style=andin] (1) at (-1, 2.5) {};
		\node [style=none] (2) at (-1, 3) {};
		\node [style=none] (3) at (-1, 1) {};
	\end{pgfonlayer}
	\begin{pgfonlayer}{edgelayer}
		\draw (2.center) to (1.center);
		\draw [in=120, out=-120, looseness=1.25] (1.center) to (0);
		\draw [in=-60, out=60, looseness=1.25] (0) to (1.center);
		\draw (0) to (3.center);
	\end{pgfonlayer}
\end{tikzpicture}
							$
						}


						\item
						\label{ZXA.16}
						{\hfil
							$
\begin{tikzpicture}
	\begin{pgfonlayer}{nodelayer}
		\node [style=andin] (1) at (-1, 2) {};
		\node [style=Z] (2) at (-1, 2.75) {$\pi$};
		\node [style=none] (3) at (-1.25, 1.25) {};
		\node [style=none] (4) at (-0.75, 1.25) {};
	\end{pgfonlayer}
	\begin{pgfonlayer}{edgelayer}
		\draw [in=90, out=-108] (1.center) to (3.center);
		\draw [in=-72, out=90] (4.center) to (1.center);
		\draw (1.center) to (2);
	\end{pgfonlayer}
\end{tikzpicture}
=
\begin{tikzpicture}
	\begin{pgfonlayer}{nodelayer}
		\node [style=Z] (2) at (-1.25, 2) {$\pi$};
		\node [style=none] (3) at (-1.25, 1.25) {};
		\node [style=none] (4) at (-0.75, 1.25) {};
		\node [style=Z] (5) at (-0.75, 2) {$\pi$};
	\end{pgfonlayer}
	\begin{pgfonlayer}{edgelayer}
		\draw (5) to (4.center);
		\draw (3.center) to (2);
	\end{pgfonlayer}
\end{tikzpicture}
							$
						}

						\item
						\label{ZXA.17}
						{\hfil
							$
\begin{tikzpicture}
	\begin{pgfonlayer}{nodelayer}
		\node [style=Z] (3) at (0, 3) {};
		\node [style=andin] (4) at (-0.25, 3.75) {};
		\node [style=none] (5) at (-0.5, 3) {};
		\node [style=none] (6) at (-0.25, 2.5) {};
		\node [style=none] (7) at (0.25, 2.5) {};
		\node [style=none] (8) at (-0.5, 2.5) {};
		\node [style=none] (9) at (-0.25, 4.25) {};
	\end{pgfonlayer}
	\begin{pgfonlayer}{edgelayer}
		\draw [in=-72, out=90] (3) to (4.center);
		\draw (4.center) to (9.center);
		\draw [in=90, out=-108] (4.center) to (5.center);
		\draw (5.center) to (8.center);
		\draw [in=90, out=-117] (3) to (6.center);
		\draw [in=90, out=-63] (3) to (7.center);
	\end{pgfonlayer}
\end{tikzpicture}
=
\begin{tikzpicture}
	\begin{pgfonlayer}{nodelayer}
		\node [style=none] (4) at (0.25, 3) {};
		\node [style=andin] (5) at (-0.35, 3.75) {};
		\node [style=none] (6) at (-0.25, 3) {};
		\node [style=andin] (7) at (0.35, 3.75) {};
		\node [style=none] (8) at (-0.25, 3) {};
		\node [style=X] (9) at (-0.25, 3) {};
		\node [style=Z] (10) at (0, 4.5) {};
		\node [style=none] (11) at (0, 5) {};
		\node [style=none] (12) at (-0.25, 2.5) {};
		\node [style=none] (13) at (0.5, 2.5) {};
		\node [style=none] (14) at (0.25, 2.5) {};
	\end{pgfonlayer}
	\begin{pgfonlayer}{edgelayer}
		\draw [in=-72, out=90] (4.center) to (5.center);
		\draw [in=120, out=-108] (5.center) to (6.center);
		\draw [in=45, out=-108] (7.center) to (8.center);
		\draw (4.center) to (14.center);
		\draw (12.center) to (6.center);
		\draw [in=-117, out=90] (5.center) to (10);
		\draw (10) to (11.center);
		\draw [in=90, out=-63] (10) to (7.center);
		\draw [in=-75, out=90, looseness=1.25] (13.center) to (7.center);
	\end{pgfonlayer}
\end{tikzpicture}
							$
						}

						

						
						
	


						
					\end{enumerate}
				\end{multicols}
				\
			\end{mdframed}
	}}
	\caption{The identities of \texorpdfstring{$\ZXA$}{ZX\&}, where \texorpdfstring{$\alpha,\beta \in \{0,\pi\}$}{alpha and beta are either 0 or pi} and a blank grey spider has angle 0.}
	\label{fig:ZXA}
\end{figure}

\end{definition}
One can interpret the generators as logical connectives and open wires as variables, similar to the regular logic \cite{butz}, or the logic of a Cartesian bicategory \cite{carboni}, except we forget the 2-cells in $\ZXA$.  The decorated black spiders correspond to fixed variables and xor.  White (co)multiplications (co)copy variables; the white unit is existential quantification and the counit is discarding. The relations are open $\Sigma_1$ Boolean formulas augmented with copying and discarding as well as duals; the open variables correspond to distinguished inputs and outputs.




 % However, some axioms such as \ref{ZXA.16}, \ref{ZXA.13}, are tautologies without postselection.
%Interestingly, \ref{ZXA.17} is witnessing that the and gate is a morphism in the two silded Kleisli category of the distributive law between the black and white spider.
%\ref{ZXA.1}-\ref{ZXA.4}, \ref{ZXA.9} are structural.
%\ref{ZXA.14} asserts that negation commutes with copying and discarding.
%\ref{ZXA.10} asserts that $(\top\wedge x) \iff x$. 
%\ref{ZXA.5} asserts that copying commutes with addition.
%\ref{ZXA.12} asserts that copying commutes with conjunction
%\ref{ZXA.8} is asserting $(x+x)=y \iff y= \bot$, or equivalently, that addition has characteristic 2.
%\ref{ZXA.15} is asserting $(x\wedge x) =y \iff y=x$. 
%\ref{ZXA.16} is asserting $(x\wedge y) = \top \iff (x=\top) \wedge (y=\top)$.
%\ref{ZXA.17} is asserting $x\wedge (y+z) \iff x\wedge y+x\wedge z$, ie. that multiplication distributes over addition.
%\ref{ZXA.7} is asserting the tautology $\top=\top \iff \top$
%%\ref{ZXA.13old} is asserting $(\bot \wedge x) = y\iff \bot= y$\\
%\ref{ZXA.13} is asserting that discarding $x\wedge y$ is the same as discarding $x$ and discarding $y$; or that the and gate is causal.
%Note that existentially quantifying and then discarding is not a tautology, rather it is the dimension, 2.

The identities of $\ZXA$ can also be interpreted by freely taking the coproduct of the free prop of commutative (co)monoids \dag-PROP $3\times 2$ times, modulo various (undirected) distributive laws, and monoid maps.  The distributive laws are summarized in Figure \ref{fig:table} (the duals under diagonal are omitted). Te spider rules implicitly identify the (co)units of the \dag-compact closed structure induced by $Z$ and $X$; which is needed for completeness.
%
%\begin{figure}[h]
%
%\begin{minipage}[b]{\textwidth}
%\setlength\dashlinedash{0.2pt}
%\setlength\dashlinegap{1.5pt}
%\setlength\arrayrulewidth{0.3pt}
%\resizebox{\textwidth}{!}{%
%\begin{tabular}{l|l:l:l:p{45mm}:p{38mm}:l}
%    $\lambda$    & $Z$    & $X$    & $\&$      & $Z^\dag$                                        & $X^\dag$                           & $\&^\dag$\\ \hline
%$Z$       & Comm. monoid &        &           & \mbox{Extra special comm.}\linebreak[4] \mbox{\dag-Frobenius algebra} &                                   Hopf algebra with $s=1$ &  Special bialgebra  \\ \hdashline
%$X$       & \bcell & Comm. monoid &           & Hopf algebra with $s=1$              &  \mbox{Comm. \dag-Frobenius}\linebreak[4] \mbox{algebra} &         \\ \hdashline
%$\&$      & \bcell & \bcell & Comm. monoid    & Special bialgebra                                       &                                    &         \\ \hdashline
%$Z^\dag$  & \bcell & \bcell &   \bcell  & Cocomm. comonoid                                          &                                    &         \\ \hdashline
%$X^\dag$  & \bcell & \bcell &   \bcell  &             \bcell                              & Cocomm. comonoid                             &         \\ \hdashline
%$\&^\dag$ & \bcell & \bcell &   \bcell  &              \bcell                             &           \bcell                   &  Cocomm. comonoid       \\
%\end{tabular}
%}
%\end{minipage}
%\caption{Generating distributive laws of \texorpdfstring{$\ZXA$}{ZX\&}.}
%\label{fig:table}
%\end{figure}






\begin{figure}[H]

\begin{minipage}[b]{\textwidth}
\setlength\dashlinedash{0.2pt}
\setlength\dashlinegap{1.5pt}
\setlength\arrayrulewidth{0.3pt}
\resizebox{\textwidth}{!}{%
\begin{tabular}{l|l:l:l:p{45mm}:p{38mm}:l}
    $\lambda$    & $Z$    & $X$    & $\&$      & $Z^\dag$                                        & $X^\dag$                           & $\&^\dag$\\ \hline
$Z$       & Comm. monoid &        &           & \noindent\begin{tabular}{@{}l} Extra special comm.\\ \dag-Frobenius algebra\end{tabular} &                                   Hopf algebra with $s=1$ &  Special bialgebra  \\ \hdashline
$X$       & \bcell & Comm. monoid &           & Hopf algebra with $s=1$              &  \noindent\begin{tabular}{@{}l} Comm. \dag-Frobenius\\ algebra \end{tabular}&         \\ \hdashline
$\&$      & \bcell & \bcell & Comm. monoid    & Special bialgebra                                       &                                    &         \\ \hdashline
$Z^\dag$  & \bcell & \bcell &   \bcell  & Cocomm. comonoid                                          &                                    &         \\ \hdashline
$X^\dag$  & \bcell & \bcell &   \bcell  &             \bcell                              & Cocomm. comonoid                             &         \\ \hdashline
$\&^\dag$ & \bcell & \bcell &   \bcell  &              \bcell                             &           \bcell                   &  Cocomm. comonoid       \\
\end{tabular}
}
\end{minipage}
\caption{Generating distributive laws of \texorpdfstring{$\ZXA$}{ZX\&}.}
\label{fig:table}
\end{figure}


Additionally, \ref{ZXA.16} states that the counit of $\&^\dag$ is copied by $\&$; ie. the counit is a monad map from $\&$ to the trivial monad.  
\ref{ZXA.17} expreses the multiplication part of the distributive law of Lawvere theories between the props for multiplication and addition mod 2 (see \cite{lawvere} for distributive laws of Lawvere theories).

%
%$$
%\hat \&:=
%\begin{tikzpicture}
%	\begin{pgfonlayer}{nodelayer}
%		\node [style=none] (0) at (1.5, -0.75) {};
%		\node [style=none] (1) at (1.5, -1.25) {};
%		\node [style=none] (2) at (2.25, -1.25) {};
%		\node [style=none] (3) at (2.25, -0.75) {};
%		\node [style=none] (4) at (0.5, -0.75) {};
%		\node [style=andin] (5) at (1.5, -0.75) {};
%	\end{pgfonlayer}
%	\begin{pgfonlayer}{edgelayer}
%		\draw [style=simple] (3.center) to (0.center);
%		\draw [style=simple] (2.center) to (1.center);
%		\draw [style=simple, in=-165, out=180, looseness=2.75] (1.center) to (0.center);
%		\draw [style=simple] (4.center) to (0.center);
%	\end{pgfonlayer}
%\end{tikzpicture}
%\hspace*{.2cm}
%\text{where}
%\hspace*{.2cm}
%\begin{tikzpicture}
%	\begin{pgfonlayer}{nodelayer}
%		\node [style=none] (0) at (1.5, -0.75) {};
%		\node [style=none] (1) at (1.5, -1.25) {};
%		\node [style=none] (2) at (2.25, -1.25) {};
%		\node [style=none] (3) at (2.25, -0.75) {};
%		\node [style=none] (4) at (0.5, -0.75) {};
%		\node [style=andin] (5) at (1.5, -0.75) {};
%		\node [style=Z] (6) at (0.5, -0.75) {};
%		\node [style=none] (7) at (-0.25, -1.25) {};
%		\node [style=none] (8) at (-0.25, -0.25) {};
%	\end{pgfonlayer}
%	\begin{pgfonlayer}{edgelayer}
%		\draw [style=simple, in=0, out=180, looseness=1.75] (3.center) to (0.center);
%		\draw [style=simple] (2.center) to (1.center);
%		\draw [style=simple, in=-165, out=180, looseness=2.50] (1.center) to (0.center);
%		\draw [in=-146, out=0, looseness=1.00] (7.center) to (4.center);
%		\draw [in=0, out=146, looseness=1.00] (4.center) to (8.center);
%		\draw [style=simple] (4.center) to (0.center);
%	\end{pgfonlayer}
%\end{tikzpicture}
%\eq{\ref{ZXA.17}}
%\begin{tikzpicture}
%	\begin{pgfonlayer}{nodelayer}
%		\node [style=none] (0) at (1.5, -0.75) {};
%		\node [style=none] (1) at (2.5, -1.5) {};
%		\node [style=none] (2) at (2.5, -0.75) {};
%		\node [style=none] (3) at (0.5, -0.75) {};
%		\node [style=andin] (4) at (1.5, -0.75) {};
%		\node [style=none] (5) at (1.5, -2) {};
%		\node [style=andin] (6) at (1.5, -1.5) {};
%		\node [style=none] (7) at (1.5, -1.5) {};
%		\node [style=none] (8) at (2.5, -1.5) {};
%		\node [style=none] (9) at (2.5, -0.75) {};
%		\node [style=none] (10) at (3, -0.75) {};
%		\node [style=none] (11) at (3, -1.5) {};
%		\node [style=Z] (12) at (2.5, -0.75) {};
%		\node [style=X] (13) at (2.5, -1.5) {};
%		\node [style=none] (14) at (0.5, -1.5) {};
%	\end{pgfonlayer}
%	\begin{pgfonlayer}{edgelayer}
%		\draw [style=simple, in=0, out=135, looseness=1.50] (2.center) to (0.center);
%		\draw [style=simple] (3.center) to (0.center);
%		\draw [style=simple, in=0, out=-135, looseness=1.50] (9.center) to (7.center);
%		\draw [style=simple, in=0, out=-150, looseness=1.00] (8.center) to (5.center);
%		\draw [style=simple, in=-165, out=180, looseness=3.25] (5.center) to (7.center);
%		\draw (11.center) to (1.center);
%		\draw (10.center) to (2.center);
%		\draw [style=simple, in=-165, out=150, looseness=2.25] (1.center) to (0.center);
%		\draw [style=simple] (14.center) to (7.center);
%	\end{pgfonlayer}
%\end{tikzpicture}
%,
%\begin{tikzpicture}
%	\begin{pgfonlayer}{nodelayer}
%		\node [style=none] (0) at (1.5, -0.75) {};
%		\node [style=none] (1) at (1.5, -1.25) {};
%		\node [style=none] (2) at (2.25, -1.25) {};
%		\node [style=none] (3) at (2.25, -0.75) {};
%		\node [style=Z] (4) at (0.5, -0.75) {};
%		\node [style=andin] (5) at (1.5, -0.75) {};
%		\node [style=none] (6) at (1, -0.75) {};
%	\end{pgfonlayer}
%	\begin{pgfonlayer}{edgelayer}
%		\draw [style=simple, in=0, out=180, looseness=1.75] (3.center) to (0.center);
%		\draw [style=simple] (2.center) to (1.center);
%		\draw [style=simple, in=-165, out=180, looseness=2.75] (1.center) to (0.center);
%		\draw [style=simple] (4.center) to (6.center);
%		\draw [style=simple] (6.center) to (0.center);
%	\end{pgfonlayer}
%\end{tikzpicture}
%\eq{Lem.\ref{lem:oldaxiom}}
%\begin{tikzpicture}
%	\begin{pgfonlayer}{nodelayer}
%		\node [style=none] (0) at (2.5, -0.75) {};
%		\node [style=none] (1) at (2.5, -0.25) {};
%		\node [style=Z] (2) at (1.75, -0.25) {};
%		\node [style=X] (3) at (1.75, -0.75) {};
%	\end{pgfonlayer}
%	\begin{pgfonlayer}{edgelayer}
%		\draw [style=simple] (0.center) to (3);
%		\draw [style=simple] (2) to (1.center);
%	\end{pgfonlayer}
%\end{tikzpicture}
%$$


%\subsubsection{Basic properties of the (co)unitual completion of \texorpdfstring{$\TOF$}{TOF}}
Before, we prove there is a functor from $\ZXA$ to $\hat \TOF$, we establish some basic properties of $\hat \TOF$.

First, the $\cnot$ gate is its own mate on the second wire:
\begin{lemma}
\label{prop:twist}
$$
\begin{tikzpicture}
	\begin{pgfonlayer}{nodelayer}
		\node [style=dot] (17) at (0, 3) {};
		\node [style=oplus] (18) at (1, 3) {};
		\node [style=none] (19) at (0, 2) {};
		\node [style=none] (20) at (0, 4) {};
		\node [style=none] (21) at (0.5, 2.75) {};
		\node [style=none] (22) at (1, 2.75) {};
		\node [style=none] (23) at (1.5, 3.25) {};
		\node [style=none] (24) at (1, 3.25) {};
		\node [style=none] (25) at (0.5, 3.25) {};
		\node [style=none] (26) at (1.5, 2.75) {};
		\node [style=none] (27) at (1, 2) {};
		\node [style=none] (28) at (1, 4) {};
	\end{pgfonlayer}
	\begin{pgfonlayer}{edgelayer}
		\draw (19.center) to (17);
		\draw (17) to (20.center);
		\draw (18) to (17);
		\draw [in=-90, out=-90, looseness=1.50] (22.center) to (21.center);
		\draw [in=90, out=90, looseness=1.50] (23.center) to (24.center);
		\draw (24.center) to (18);
		\draw (22.center) to (18);
		\draw (23.center) to (26.center);
		\draw (21.center) to (25.center);
		\draw [in=-90, out=90] (27.center) to (26.center);
		\draw [in=90, out=-90, looseness=1.25] (28.center) to (25.center);
	\end{pgfonlayer}
\end{tikzpicture}
=
\begin{tikzpicture}
	\begin{pgfonlayer}{nodelayer}
		\node [style=dot] (18) at (0, 3) {};
		\node [style=oplus] (19) at (0.75, 3) {};
		\node [style=none] (20) at (0, 2) {};
		\node [style=none] (21) at (0, 4) {};
		\node [style=none] (22) at (0.75, 4) {};
		\node [style=none] (23) at (0.75, 2) {};
	\end{pgfonlayer}
	\begin{pgfonlayer}{edgelayer}
		\draw (20.center) to (18);
		\draw (18) to (21.center);
		\draw (19) to (18);
		\draw (23.center) to (19);
		\draw (19) to (22.center);
	\end{pgfonlayer}
\end{tikzpicture}
$$
\end{lemma}

\begin{proof}
\begin{align*}
\begin{tikzpicture}
	\begin{pgfonlayer}{nodelayer}
		\node [style=oplus] (0) at (1, 3) {};
		\node [style=none] (1) at (0, 2) {};
		\node [style=none] (2) at (0, 4) {};
		\node [style=none] (3) at (0.5, 2.75) {};
		\node [style=none] (4) at (1, 2.75) {};
		\node [style=none] (5) at (1.5, 3.25) {};
		\node [style=none] (6) at (1, 3.25) {};
		\node [style=none] (7) at (0.5, 3.25) {};
		\node [style=none] (8) at (1.5, 2.75) {};
		\node [style=none] (9) at (1, 2) {};
		\node [style=none] (10) at (1, 4) {};
		\node [style=dot] (11) at (0, 3) {};
	\end{pgfonlayer}
	\begin{pgfonlayer}{edgelayer}
		\draw [in=-90, out=-90, looseness=1.50] (4.center) to (3.center);
		\draw [in=90, out=90, looseness=1.50] (5.center) to (6.center);
		\draw (6.center) to (0);
		\draw (4.center) to (0);
		\draw (5.center) to (8.center);
		\draw (3.center) to (7.center);
		\draw [in=-90, out=90] (9.center) to (8.center);
		\draw [in=90, out=-90, looseness=1.25] (10.center) to (7.center);
		\draw (11) to (0);
		\draw (11) to (2.center);
		\draw (11) to (1.center);
	\end{pgfonlayer}
\end{tikzpicture}
&=
\begin{tikzpicture}
	\begin{pgfonlayer}{nodelayer}
		\node [style=dot] (1) at (0, 3.25) {};
		\node [style=oplus] (2) at (1, 3.25) {};
		\node [style=none] (3) at (0, 2) {};
		\node [style=none] (4) at (0, 4.5) {};
		\node [style=none] (5) at (0.5, 3.5) {};
		\node [style=none] (6) at (1.5, 3) {};
		\node [style=none] (7) at (1.25, 2) {};
		\node [style=none] (8) at (0.75, 4.5) {};
		\node [style=fanout] (9) at (0.75, 2.75) {};
		\node [style=fanin] (10) at (1.25, 3.75) {};
		\node [style=X] (11) at (0.75, 2) {};
		\node [style=X] (12) at (1.25, 4.5) {};
	\end{pgfonlayer}
	\begin{pgfonlayer}{edgelayer}
		\draw (3.center) to (1);
		\draw (1) to (4.center);
		\draw (2) to (1);
		\draw [in=-90, out=90] (7.center) to (6.center);
		\draw [in=90, out=-90, looseness=1.25] (8.center) to (5.center);
		\draw [in=-72, out=90] (6.center) to (10);
		\draw [in=90, out=-117, looseness=1.25] (10) to (2);
		\draw [in=63, out=-90, looseness=1.25] (2) to (9);
		\draw [in=-90, out=108] (9) to (5.center);
		\draw (11) to (9);
		\draw (10) to (12);
	\end{pgfonlayer}
\end{tikzpicture}
\eq{\ref{CNOT.2}}
\begin{tikzpicture}
	\begin{pgfonlayer}{nodelayer}
		\node [style=dot] (2) at (0, 3.25) {};
		\node [style=oplus] (3) at (1, 3.25) {};
		\node [style=none] (4) at (0, 2) {};
		\node [style=none] (5) at (0, 5) {};
		\node [style=none] (6) at (1.5, 3) {};
		\node [style=none] (7) at (1.25, 2) {};
		\node [style=none] (8) at (0.5, 5) {};
		\node [style=fanout] (9) at (0.75, 2.75) {};
		\node [style=fanin] (10) at (1.25, 3.75) {};
		\node [style=X] (11) at (0.75, 2) {};
		\node [style=X] (12) at (1.25, 5) {};
		\node [style=dot] (13) at (0, 3.75) {};
		\node [style=oplus] (14) at (0.5, 3.75) {};
		\node [style=oplus] (15) at (0.5, 4.5) {};
		\node [style=dot] (16) at (0, 4.5) {};
	\end{pgfonlayer}
	\begin{pgfonlayer}{edgelayer}
		\draw (4.center) to (2);
		\draw (2) to (5.center);
		\draw (3) to (2);
		\draw [in=-90, out=90] (7.center) to (6.center);
		\draw [in=-72, out=90] (6.center) to (10);
		\draw [in=90, out=-117, looseness=1.25] (10) to (3);
		\draw [in=63, out=-90, looseness=1.25] (3) to (9);
		\draw (11) to (9);
		\draw (10) to (12);
		\draw (14) to (13);
		\draw (15) to (16);
		\draw (8.center) to (15);
		\draw (15) to (14);
		\draw [in=-90, out=104] (9) to (14);
	\end{pgfonlayer}
\end{tikzpicture}
\eq{Lem. \ref{lemma:natoplus}}
\begin{tikzpicture}
	\begin{pgfonlayer}{nodelayer}
		\node [style=none] (3) at (0, 1.5) {};
		\node [style=none] (4) at (0, 4.5) {};
		\node [style=none] (5) at (1.5, 3) {};
		\node [style=none] (6) at (1.25, 1.5) {};
		\node [style=none] (7) at (0.5, 4.5) {};
		\node [style=fanout] (8) at (0.75, 2.75) {};
		\node [style=fanin] (9) at (1.25, 3.75) {};
		\node [style=X] (10) at (0.75, 1.5) {};
		\node [style=X] (11) at (1.25, 4.5) {};
		\node [style=oplus] (12) at (0.75, 2.25) {};
		\node [style=dot] (13) at (0, 2.25) {};
		\node [style=dot] (14) at (0, 3.5) {};
		\node [style=oplus] (15) at (0.5, 3.5) {};
	\end{pgfonlayer}
	\begin{pgfonlayer}{edgelayer}
		\draw [in=-90, out=90] (6.center) to (5.center);
		\draw [in=-72, out=90] (5.center) to (9);
		\draw (9) to (11);
		\draw (12) to (13);
		\draw (15) to (14);
		\draw (4.center) to (14);
		\draw (15) to (7.center);
		\draw [in=108, out=-90] (15) to (8);
		\draw (8) to (9);
		\draw (8) to (12);
		\draw (12) to (10);
		\draw (3.center) to (13);
		\draw (13) to (14);
	\end{pgfonlayer}
\end{tikzpicture}
\eq{Frob.}
\begin{tikzpicture}
	\begin{pgfonlayer}{nodelayer}
		\node [style=none] (4) at (0, 1.25) {};
		\node [style=none] (5) at (0, 4.75) {};
		\node [style=none] (6) at (1, 2) {};
		\node [style=none] (7) at (1, 1.25) {};
		\node [style=none] (8) at (0.5, 4.75) {};
		\node [style=X] (9) at (0.5, 1.5) {};
		\node [style=X] (10) at (1, 4.25) {};
		\node [style=oplus] (11) at (0.5, 2) {};
		\node [style=dot] (12) at (0, 2) {};
		\node [style=dot] (13) at (0, 4.25) {};
		\node [style=oplus] (14) at (0.5, 4.25) {};
		\node [style=fanin] (15) at (0.75, 2.75) {};
		\node [style=fanout] (16) at (0.75, 3.5) {};
	\end{pgfonlayer}
	\begin{pgfonlayer}{edgelayer}
		\draw [in=-90, out=90] (7.center) to (6.center);
		\draw (11) to (12);
		\draw (14) to (13);
		\draw (5.center) to (13);
		\draw (14) to (8.center);
		\draw (11) to (9);
		\draw (4.center) to (12);
		\draw (12) to (13);
		\draw [in=63, out=-90, looseness=1.25] (10) to (16);
		\draw [in=-90, out=117, looseness=1.25] (16) to (14);
		\draw (16) to (15);
		\draw [in=90, out=-108] (15) to (11);
		\draw [in=-72, out=90] (6.center) to (15);
	\end{pgfonlayer}
\end{tikzpicture}\\
&\eq{unit}
\begin{tikzpicture}
	\begin{pgfonlayer}{nodelayer}
		\node [style=none] (5) at (0, 1.25) {};
		\node [style=none] (6) at (0, 4.75) {};
		\node [style=none] (7) at (1, 2) {};
		\node [style=none] (8) at (1, 1.25) {};
		\node [style=none] (9) at (0.75, 4.75) {};
		\node [style=X] (10) at (0.5, 1.25) {};
		\node [style=oplus] (11) at (0.5, 2) {};
		\node [style=dot] (12) at (0, 2) {};
		\node [style=dot] (13) at (0, 4) {};
		\node [style=oplus] (14) at (0.75, 4) {};
		\node [style=fanin] (15) at (0.75, 2.75) {};
	\end{pgfonlayer}
	\begin{pgfonlayer}{edgelayer}
		\draw [in=-90, out=90] (8.center) to (7.center);
		\draw (11) to (12);
		\draw (14) to (13);
		\draw (6.center) to (13);
		\draw (14) to (9.center);
		\draw (11) to (10);
		\draw (5.center) to (12);
		\draw (12) to (13);
		\draw [in=90, out=-108] (15) to (11);
		\draw [in=-72, out=90] (7.center) to (15);
		\draw (14) to (15);
	\end{pgfonlayer}
\end{tikzpicture}
\eq{Lem. \ref{lemma:natoplus}}
\begin{tikzpicture}
	\begin{pgfonlayer}{nodelayer}
		\node [style=none] (6) at (0, 4) {};
		\node [style=none] (7) at (1, 2.5) {};
		\node [style=none] (8) at (1, 1) {};
		\node [style=none] (9) at (0.75, 4) {};
		\node [style=oplus] (10) at (0.5, 2.5) {};
		\node [style=dot] (11) at (0, 2.5) {};
		\node [style=fanin] (12) at (0.75, 3.25) {};
		\node [style=dot] (13) at (0, 2) {};
		\node [style=oplus] (14) at (0.5, 2) {};
		\node [style=none] (15) at (0, 1) {};
		\node [style=X] (16) at (0.5, 1) {};
		\node [style=dot] (17) at (0, 1.5) {};
		\node [style=oplus] (18) at (1, 1.5) {};
	\end{pgfonlayer}
	\begin{pgfonlayer}{edgelayer}
		\draw [in=-90, out=90] (8.center) to (7.center);
		\draw (10) to (11);
		\draw [in=90, out=-108] (12) to (10);
		\draw [in=-72, out=90] (7.center) to (12);
		\draw (14) to (13);
		\draw (18) to (17);
		\draw (16) to (14);
		\draw (14) to (10);
		\draw (6.center) to (11);
		\draw (11) to (17);
		\draw (17) to (13);
		\draw (13) to (15.center);
		\draw (9.center) to (12);
	\end{pgfonlayer}
\end{tikzpicture}
\eq{\ref{CNOT.2}}
\begin{tikzpicture}
	\begin{pgfonlayer}{nodelayer}
		\node [style=none] (7) at (0, 3.75) {};
		\node [style=none] (8) at (1, 2) {};
		\node [style=none] (9) at (0.75, 3.75) {};
		\node [style=fanin] (10) at (0.75, 3.25) {};
		\node [style=none] (11) at (0, 2) {};
		\node [style=X] (12) at (0.5, 2) {};
		\node [style=dot] (13) at (0, 2.5) {};
		\node [style=oplus] (14) at (1, 2.5) {};
	\end{pgfonlayer}
	\begin{pgfonlayer}{edgelayer}
		\draw (14) to (13);
		\draw (9.center) to (10);
		\draw [in=90, out=-105] (10) to (12);
		\draw (11.center) to (7.center);
		\draw [in=90, out=-72] (10) to (14);
		\draw (14) to (8.center);
	\end{pgfonlayer}
\end{tikzpicture}
\eq{unit}
\begin{tikzpicture}
	\begin{pgfonlayer}{nodelayer}
		\node [style=none] (8) at (0, 6.25) {};
		\node [style=none] (9) at (0.5, 5.25) {};
		\node [style=none] (10) at (0.5, 6.25) {};
		\node [style=none] (11) at (0, 5.25) {};
		\node [style=dot] (12) at (0, 5.75) {};
		\node [style=oplus] (13) at (0.5, 5.75) {};
	\end{pgfonlayer}
	\begin{pgfonlayer}{edgelayer}
		\draw (13) to (12);
		\draw (11.center) to (8.center);
		\draw (13) to (9.center);
		\draw (10.center) to (13);
	\end{pgfonlayer}
\end{tikzpicture}
\end{align*}
\end{proof}


Therefore, 

\begin{lemma}
\label{lemma:cnotslide}
$$
\begin{tikzpicture}
	\begin{pgfonlayer}{nodelayer}
		\node [style=oplus] (9) at (1, 5.75) {};
		\node [style=none] (10) at (0.5, 6.25) {};
		\node [style=none] (11) at (0.5, 5.25) {};
		\node [style=none] (12) at (1, 6.25) {};
		\node [style=none] (13) at (1.5, 5.5) {};
		\node [style=none] (14) at (1, 5.5) {};
		\node [style=none] (15) at (1.5, 6.25) {};
		\node [style=dot] (16) at (0.5, 5.75) {};
	\end{pgfonlayer}
	\begin{pgfonlayer}{edgelayer}
		\draw [in=-90, out=-90, looseness=1.50] (13.center) to (14.center);
		\draw (14.center) to (9);
		\draw (16) to (9);
		\draw (16) to (11.center);
		\draw (16) to (10.center);
		\draw (15.center) to (13.center);
		\draw (9) to (12.center);
	\end{pgfonlayer}
\end{tikzpicture}
\eq{Prop. \ref{prop:twist}}
\begin{tikzpicture}
	\begin{pgfonlayer}{nodelayer}
		\node [style=dot] (10) at (-0.25, 6.25) {};
		\node [style=oplus] (11) at (0.75, 6.25) {};
		\node [style=none] (12) at (-0.25, 5.25) {};
		\node [style=none] (13) at (-0.25, 7.25) {};
		\node [style=none] (14) at (0.25, 6) {};
		\node [style=none] (15) at (0.75, 6) {};
		\node [style=none] (16) at (1.25, 6.5) {};
		\node [style=none] (17) at (0.75, 6.5) {};
		\node [style=none] (18) at (0.25, 6.5) {};
		\node [style=none] (19) at (0.25, 7.25) {};
		\node [style=none] (20) at (1.25, 5.75) {};
		\node [style=none] (21) at (1.75, 5.75) {};
		\node [style=none] (22) at (1.75, 7.25) {};
	\end{pgfonlayer}
	\begin{pgfonlayer}{edgelayer}
		\draw (12.center) to (10);
		\draw (10) to (13.center);
		\draw (11) to (10);
		\draw [in=-90, out=-90, looseness=1.50] (15.center) to (14.center);
		\draw [in=90, out=90, looseness=1.50] (16.center) to (17.center);
		\draw (17.center) to (11);
		\draw (15.center) to (11);
		\draw (14.center) to (18.center);
		\draw [in=90, out=-90, looseness=1.25] (19.center) to (18.center);
		\draw [in=-90, out=-90, looseness=1.50] (21.center) to (20.center);
		\draw (22.center) to (21.center);
		\draw (20.center) to (16.center);
	\end{pgfonlayer}
\end{tikzpicture}
\eq{yanking}
\begin{tikzpicture}
	\begin{pgfonlayer}{nodelayer}
		\node [style=dot] (11) at (-0.25, 5.75) {};
		\node [style=oplus] (12) at (0.75, 5.75) {};
		\node [style=none] (13) at (-0.25, 5.25) {};
		\node [style=none] (14) at (-0.25, 6.25) {};
		\node [style=none] (15) at (0.25, 5.5) {};
		\node [style=none] (16) at (0.75, 5.5) {};
		\node [style=none] (17) at (0.75, 6.25) {};
		\node [style=none] (18) at (0.25, 6.25) {};
	\end{pgfonlayer}
	\begin{pgfonlayer}{edgelayer}
		\draw (13.center) to (11);
		\draw (11) to (14.center);
		\draw (12) to (11);
		\draw [in=-90, out=-90, looseness=1.50] (16.center) to (15.center);
		\draw (17.center) to (12);
		\draw (16.center) to (12);
		\draw (15.center) to (18.center);
	\end{pgfonlayer}
\end{tikzpicture}
$$
\end{lemma}


Thus

\begin{lemma}
\label{lemma:whiteunit}

$$
\begin{tikzpicture}
	\begin{pgfonlayer}{nodelayer}
		\node [style=dot] (12) at (-0.25, 5.75) {};
		\node [style=oplus] (13) at (0.25, 5.75) {};
		\node [style=none] (14) at (-0.25, 5.25) {};
		\node [style=none] (15) at (-0.25, 6.25) {};
		\node [style=none] (16) at (0.25, 6.25) {};
		\node [style=X] (17) at (0.25, 5.25) {};
	\end{pgfonlayer}
	\begin{pgfonlayer}{edgelayer}
		\draw (14.center) to (12);
		\draw (12) to (15.center);
		\draw (13) to (12);
		\draw (16.center) to (13);
		\draw (13) to (17);
	\end{pgfonlayer}
\end{tikzpicture}
=
\begin{tikzpicture}
	\begin{pgfonlayer}{nodelayer}
		\node [style=none] (13) at (-0.25, 5.25) {};
		\node [style=none] (14) at (-0.25, 6) {};
		\node [style=none] (15) at (0.25, 6) {};
		\node [style=X] (16) at (0.25, 5.25) {};
	\end{pgfonlayer}
	\begin{pgfonlayer}{edgelayer}
		\draw (15.center) to (16);
		\draw (13.center) to (14.center);
	\end{pgfonlayer}
\end{tikzpicture}
$$
\end{lemma}

\begin{proof}
\begin{align*}
\begin{tikzpicture}
	\begin{pgfonlayer}{nodelayer}
		\node [style=dot] (14) at (-0.25, 5.75) {};
		\node [style=oplus] (15) at (0.25, 5.75) {};
		\node [style=none] (16) at (-0.25, 5.25) {};
		\node [style=none] (17) at (-0.25, 6.25) {};
		\node [style=none] (18) at (0.25, 6.25) {};
		\node [style=X] (19) at (0.25, 5.25) {};
	\end{pgfonlayer}
	\begin{pgfonlayer}{edgelayer}
		\draw (16.center) to (14);
		\draw (14) to (17.center);
		\draw (15) to (14);
		\draw (18.center) to (15);
		\draw (15) to (19);
	\end{pgfonlayer}
\end{tikzpicture}
&\eq{unit}
\begin{tikzpicture}
	\begin{pgfonlayer}{nodelayer}
		\node [style=dot] (15) at (-0.25, 6.5) {};
		\node [style=oplus] (16) at (0.25, 6.5) {};
		\node [style=none] (17) at (-0.25, 5.25) {};
		\node [style=none] (18) at (-0.25, 7) {};
		\node [style=none] (19) at (0.25, 7) {};
		\node [style=X] (20) at (0.5, 5.25) {};
		\node [style=fanout] (21) at (0.5, 5.75) {};
		\node [style=X] (22) at (0.75, 6.5) {};
	\end{pgfonlayer}
	\begin{pgfonlayer}{edgelayer}
		\draw (17.center) to (15);
		\draw (15) to (18.center);
		\draw (16) to (15);
		\draw (19.center) to (16);
		\draw [in=117, out=-90] (16) to (21);
		\draw (21) to (20);
		\draw [in=63, out=-90] (22) to (21);
	\end{pgfonlayer}
\end{tikzpicture}
=
\begin{tikzpicture}
	\begin{pgfonlayer}{nodelayer}
		\node [style=dot] (16) at (-0.25, 5.75) {};
		\node [style=oplus] (17) at (0.25, 5.75) {};
		\node [style=none] (18) at (-0.25, 5.25) {};
		\node [style=none] (19) at (-0.25, 6.25) {};
		\node [style=none] (20) at (0.25, 6.25) {};
		\node [style=X] (21) at (0.75, 5.75) {};
	\end{pgfonlayer}
	\begin{pgfonlayer}{edgelayer}
		\draw (18.center) to (16);
		\draw (16) to (19.center);
		\draw (17) to (16);
		\draw (20.center) to (17);
		\draw [in=-90, out=-90, looseness=2.25] (17) to (21);
	\end{pgfonlayer}
\end{tikzpicture}\
\eq{\ref{lemma:cnotslide}}
\begin{tikzpicture}
	\begin{pgfonlayer}{nodelayer}
		\node [style=dot] (17) at (0, 5.75) {};
		\node [style=oplus] (18) at (0.75, 5.75) {};
		\node [style=none] (19) at (0, 5.25) {};
		\node [style=none] (20) at (0, 6.25) {};
		\node [style=none] (21) at (0.25, 6.25) {};
		\node [style=X] (22) at (0.75, 6.25) {};
		\node [style=none] (23) at (0.25, 5.5) {};
		\node [style=none] (24) at (0.75, 5.5) {};
	\end{pgfonlayer}
	\begin{pgfonlayer}{edgelayer}
		\draw (19.center) to (17);
		\draw (17) to (20.center);
		\draw (18) to (17);
		\draw (22) to (18);
		\draw (18) to (24.center);
		\draw (23.center) to (21.center);
		\draw [in=-90, out=-90, looseness=1.25] (23.center) to (24.center);
	\end{pgfonlayer}
\end{tikzpicture}\
=
\begin{tikzpicture}
	\begin{pgfonlayer}{nodelayer}
		\node [style=dot] (18) at (0, 6.25) {};
		\node [style=oplus] (19) at (0.75, 6.25) {};
		\node [style=none] (20) at (0, 5.25) {};
		\node [style=none] (21) at (0, 6.75) {};
		\node [style=none] (22) at (0.25, 6.25) {};
		\node [style=X] (23) at (0.75, 6.75) {};
		\node [style=fanout] (24) at (0.5, 5.75) {};
		\node [style=X] (25) at (0.5, 5.25) {};
		\node [style=none] (26) at (0.25, 6.75) {};
	\end{pgfonlayer}
	\begin{pgfonlayer}{edgelayer}
		\draw (20.center) to (18);
		\draw (18) to (21.center);
		\draw (19) to (18);
		\draw (23) to (19);
		\draw [in=63, out=-90] (19) to (24);
		\draw [in=-90, out=117] (24) to (22.center);
		\draw (24) to (25);
		\draw (26.center) to (22.center);
	\end{pgfonlayer}
\end{tikzpicture}
\eq{\ref{CNOT.2}}
\begin{tikzpicture}
	\begin{pgfonlayer}{nodelayer}
		\node [style=dot] (19) at (-0.25, 6.25) {};
		\node [style=oplus] (20) at (0.75, 6.25) {};
		\node [style=none] (21) at (-0.25, 5.25) {};
		\node [style=none] (22) at (0.25, 6.25) {};
		\node [style=X] (23) at (0.75, 6.75) {};
		\node [style=fanout] (24) at (0.5, 5.75) {};
		\node [style=X] (25) at (0.5, 5.25) {};
		\node [style=oplus] (26) at (0.25, 7.25) {};
		\node [style=dot] (27) at (-0.25, 7.25) {};
		\node [style=none] (28) at (0.25, 7.75) {};
		\node [style=none] (29) at (-0.25, 7.75) {};
		\node [style=dot] (30) at (-0.25, 6.75) {};
		\node [style=oplus] (31) at (0.25, 6.75) {};
	\end{pgfonlayer}
	\begin{pgfonlayer}{edgelayer}
		\draw (21.center) to (19);
		\draw (20) to (19);
		\draw (23) to (20);
		\draw [in=63, out=-90] (20) to (24);
		\draw [in=-90, out=117] (24) to (22.center);
		\draw (24) to (25);
		\draw (26) to (27);
		\draw (31) to (30);
		\draw (28.center) to (26);
		\draw (26) to (31);
		\draw (31) to (22.center);
		\draw (19) to (30);
		\draw (30) to (27);
		\draw (27) to (29.center);
	\end{pgfonlayer}
\end{tikzpicture}\
\eq{Lem. \ref{lemma:natoplus}}
\begin{tikzpicture}
	\begin{pgfonlayer}{nodelayer}
		\node [style=X] (20) at (0.75, 7.25) {};
		\node [style=fanout] (21) at (0.5, 6.25) {};
		\node [style=none] (22) at (0.25, 8.25) {};
		\node [style=none] (23) at (-0.25, 8.25) {};
		\node [style=dot] (24) at (-0.25, 6.75) {};
		\node [style=oplus] (25) at (0.25, 6.75) {};
		\node [style=X] (26) at (0.5, 5.25) {};
		\node [style=dot] (27) at (-0.25, 5.75) {};
		\node [style=oplus] (28) at (0.5, 5.75) {};
		\node [style=none] (29) at (-0.25, 5.25) {};
		\node [style=none] (30) at (0.75, 6.75) {};
	\end{pgfonlayer}
	\begin{pgfonlayer}{edgelayer}
		\draw (25) to (24);
		\draw (28) to (27);
		\draw (27) to (29.center);
		\draw (26) to (28);
		\draw (28) to (21);
		\draw [in=-90, out=63] (21) to (30.center);
		\draw (30.center) to (20);
		\draw (25) to (22.center);
		\draw (23.center) to (24);
		\draw [in=117, out=-90] (25) to (21);
		\draw (27) to (24);
	\end{pgfonlayer}
\end{tikzpicture}\\
&\eq{unit}
\begin{tikzpicture}
	\begin{pgfonlayer}{nodelayer}
		\node [style=none] (21) at (0.25, 6.75) {};
		\node [style=none] (22) at (-0.25, 6.75) {};
		\node [style=dot] (23) at (-0.25, 6.25) {};
		\node [style=oplus] (24) at (0.25, 6.25) {};
		\node [style=X] (25) at (0.25, 5.25) {};
		\node [style=dot] (26) at (-0.25, 5.75) {};
		\node [style=oplus] (27) at (0.25, 5.75) {};
		\node [style=none] (28) at (-0.25, 5.25) {};
	\end{pgfonlayer}
	\begin{pgfonlayer}{edgelayer}
		\draw (24) to (23);
		\draw (27) to (26);
		\draw (26) to (28.center);
		\draw (25) to (27);
		\draw (24) to (21.center);
		\draw (22.center) to (23);
		\draw (26) to (23);
		\draw (24) to (27);
	\end{pgfonlayer}
\end{tikzpicture}
\eq{\ref{CNOT.2}}
\begin{tikzpicture}
	\begin{pgfonlayer}{nodelayer}
		\node [style=none] (22) at (0.25, 6) {};
		\node [style=none] (23) at (-0.25, 6) {};
		\node [style=X] (24) at (0.25, 5.25) {};
		\node [style=none] (25) at (-0.25, 5.25) {};
	\end{pgfonlayer}
	\begin{pgfonlayer}{edgelayer}
		\draw (22.center) to (24);
		\draw (25.center) to (23.center);
	\end{pgfonlayer}
\end{tikzpicture}
\end{align*}
\end{proof}



\begin{proposition}
\label{prop:TOFZXA}
Consider the interpretation $\llbracket\_\rrbracket_{\ZXA}:\ZXA\to\hat \TOF$ taking:

$$
\begin{tikzpicture}
	\begin{pgfonlayer}{nodelayer}
		\node [style=none] (7) at (0.25, 2) {};
		\node [style=none] (8) at (0.75, 2) {};
		\node [style=Z] (9) at (0.5, 1.25) {};
		\node [style=none] (10) at (0.5, 0.5) {};
	\end{pgfonlayer}
	\begin{pgfonlayer}{edgelayer}
		\draw [style=simple, in=-90, out=124] (9) to (7.center);
		\draw [style=simple, in=-90, out=56] (9) to (8.center);
		\draw [style=simple] (9) to (10.center);
	\end{pgfonlayer}
\end{tikzpicture}
\mapsto
\begin{tikzpicture}
	\begin{pgfonlayer}{nodelayer}
		\node [style=none] (8) at (0, 2.75) {};
		\node [style=none] (9) at (0, 1.25) {};
		\node [style=dot] (10) at (0.5, 2) {};
		\node [style=oplus] (11) at (0, 2) {};
		\node [style=X] (12) at (0.5, 1.25) {};
		\node [style=none] (13) at (0.5, 2.75) {};
	\end{pgfonlayer}
	\begin{pgfonlayer}{edgelayer}
		\draw [style=simple] (13.center) to (10);
		\draw [style=simple] (10) to (12);
		\draw [style=simple] (10) to (11);
		\draw [style=simple] (11) to (9.center);
		\draw [style=simple] (11) to (8.center);
	\end{pgfonlayer}
\end{tikzpicture}
\hspace*{.5cm}
\begin{tikzpicture}
	\begin{pgfonlayer}{nodelayer}
		\node [style=none] (9) at (0.25, 1.25) {};
		\node [style=none] (10) at (0.75, 1.25) {};
		\node [style=Z] (11) at (0.5, 2) {};
		\node [style=none] (12) at (0.5, 2.75) {};
	\end{pgfonlayer}
	\begin{pgfonlayer}{edgelayer}
		\draw [style=simple, in=90, out=-124] (11) to (9.center);
		\draw [style=simple, in=90, out=-56] (11) to (10.center);
		\draw [style=simple] (11) to (12.center);
	\end{pgfonlayer}
\end{tikzpicture}
\mapsto
\begin{tikzpicture}
	\begin{pgfonlayer}{nodelayer}
		\node [style=none] (15) at (0, 1.25) {};
		\node [style=none] (16) at (0, 2.75) {};
		\node [style=dot] (17) at (0.5, 2) {};
		\node [style=oplus] (18) at (0, 2) {};
		\node [style=X] (19) at (0.5, 2.75) {};
		\node [style=none] (20) at (0.5, 1.25) {};
	\end{pgfonlayer}
	\begin{pgfonlayer}{edgelayer}
		\draw [style=simple] (20.center) to (17);
		\draw [style=simple] (17) to (19);
		\draw [style=simple] (17) to (18);
		\draw [style=simple] (18) to (16.center);
		\draw [style=simple] (18) to (15.center);
	\end{pgfonlayer}
\end{tikzpicture}
\hspace*{.5cm}
\begin{tikzpicture}
	\begin{pgfonlayer}{nodelayer}
		\node [style=Z] (0) at (0, 1) {};
		\node [style=none] (1) at (0, 2) {};
	\end{pgfonlayer}
	\begin{pgfonlayer}{edgelayer}
		\draw [style=simple] (1.center) to (0);
	\end{pgfonlayer}
\end{tikzpicture}
\mapsto
\begin{tikzpicture}
	\begin{pgfonlayer}{nodelayer}
		\node [style=zeroin] (0) at (0, 1) {};
		\node [style=none] (1) at (0, 2) {};
	\end{pgfonlayer}
	\begin{pgfonlayer}{edgelayer}
		\draw [style=simple] (1.center) to (0);
	\end{pgfonlayer}
\end{tikzpicture}
\hspace*{.5cm}
\begin{tikzpicture}
	\begin{pgfonlayer}{nodelayer}
		\node [style=Z] (1) at (0.5, 2) {};
		\node [style=none] (2) at (0.5, 1) {};
	\end{pgfonlayer}
	\begin{pgfonlayer}{edgelayer}
		\draw [style=simple] (2.center) to (1);
	\end{pgfonlayer}
\end{tikzpicture}
\mapsto
\begin{tikzpicture}
	\begin{pgfonlayer}{nodelayer}
		\node [style=zeroout] (1) at (0, 2) {};
		\node [style=none] (2) at (0, 1) {};
	\end{pgfonlayer}
	\begin{pgfonlayer}{edgelayer}
		\draw [style=simple] (2.center) to (1);
	\end{pgfonlayer}
\end{tikzpicture}
\hspace*{.5cm}
\begin{tikzpicture}
	\begin{pgfonlayer}{nodelayer}
		\node [style=none] (2) at (0.25, 2) {};
		\node [style=none] (3) at (0.75, 2) {};
		\node [style=X] (4) at (0.5, 1.25) {};
		\node [style=none] (5) at (0.5, 0.5) {};
	\end{pgfonlayer}
	\begin{pgfonlayer}{edgelayer}
		\draw [style=simple, in=-90, out=124] (4) to (2.center);
		\draw [style=simple, in=-90, out=56] (4) to (3.center);
		\draw [style=simple] (4) to (5.center);
	\end{pgfonlayer}
\end{tikzpicture}
\mapsto
\begin{tikzpicture}
	\begin{pgfonlayer}{nodelayer}
		\node [style=none] (3) at (0, 2) {};
		\node [style=none] (4) at (0, 0.5) {};
		\node [style=oplus] (5) at (0.5, 1.25) {};
		\node [style=dot] (6) at (0, 1.25) {};
		\node [style=zeroin] (7) at (0.5, 0.5) {};
		\node [style=none] (8) at (0.5, 2) {};
	\end{pgfonlayer}
	\begin{pgfonlayer}{edgelayer}
		\draw [style=simple] (8.center) to (5);
		\draw [style=simple] (5) to (7);
		\draw [style=simple] (5) to (6);
		\draw [style=simple] (6) to (4.center);
		\draw [style=simple] (6) to (3.center);
	\end{pgfonlayer}
\end{tikzpicture}
\hspace*{.5cm}
\begin{tikzpicture}
	\begin{pgfonlayer}{nodelayer}
		\node [style=none] (4) at (0.25, 2.25) {};
		\node [style=none] (5) at (0.75, 2.25) {};
		\node [style=X] (6) at (0.5, 3) {};
		\node [style=none] (7) at (0.5, 3.75) {};
	\end{pgfonlayer}
	\begin{pgfonlayer}{edgelayer}
		\draw [style=simple, in=90, out=-124] (6) to (4.center);
		\draw [style=simple, in=90, out=-56] (6) to (5.center);
		\draw [style=simple] (6) to (7.center);
	\end{pgfonlayer}
\end{tikzpicture}
\mapsto
\begin{tikzpicture}
	\begin{pgfonlayer}{nodelayer}
		\node [style=none] (0) at (6.5, -7.75) {};
		\node [style=none] (1) at (6.5, -6.25) {};
		\node [style=oplus] (2) at (7, -7) {};
		\node [style=dot] (3) at (6.5, -7) {};
		\node [style=zeroout] (4) at (7, -6.25) {};
		\node [style=none] (5) at (7, -7.75) {};
	\end{pgfonlayer}
	\begin{pgfonlayer}{edgelayer}
		\draw [style=simple] (5.center) to (2);
		\draw [style=simple] (2) to (4);
		\draw [style=simple] (2) to (3);
		\draw [style=simple] (3) to (1.center);
		\draw [style=simple] (3) to (0.center);
	\end{pgfonlayer}
\end{tikzpicture}
\hspace*{.5cm}
\begin{tikzpicture}
	\begin{pgfonlayer}{nodelayer}
		\node [style=X] (1) at (0, 1) {};
		\node [style=none] (2) at (0, 2) {};
	\end{pgfonlayer}
	\begin{pgfonlayer}{edgelayer}
		\draw [style=simple] (2.center) to (1);
	\end{pgfonlayer}
\end{tikzpicture}
\mapsto
\begin{tikzpicture}
	\begin{pgfonlayer}{nodelayer}
		\node [style=X] (2) at (0, 13.25) {};
		\node [style=none] (3) at (0, 14.25) {};
	\end{pgfonlayer}
	\begin{pgfonlayer}{edgelayer}
		\draw [style=simple] (3.center) to (2);
	\end{pgfonlayer}
\end{tikzpicture}
\hspace*{.5cm}
\begin{tikzpicture}
	\begin{pgfonlayer}{nodelayer}
		\node [style=X] (0) at (0, 2) {};
		\node [style=none] (1) at (0, 1) {};
	\end{pgfonlayer}
	\begin{pgfonlayer}{edgelayer}
		\draw [style=simple] (1.center) to (0);
	\end{pgfonlayer}
\end{tikzpicture}
\mapsto
\begin{tikzpicture}
	\begin{pgfonlayer}{nodelayer}
		\node [style=X] (0) at (0, 2) {};
		\node [style=none] (1) at (0, 1) {};
	\end{pgfonlayer}
	\begin{pgfonlayer}{edgelayer}
		\draw [style=simple] (1.center) to (0);
	\end{pgfonlayer}
\end{tikzpicture}
$$
$$
\begin{tikzpicture}
	\begin{pgfonlayer}{nodelayer}
		\node [style=Z] (0) at (0, 1.5) {$\pi$};
		\node [style=none] (1) at (0, 2.5) {};
		\node [style=none] (2) at (0, 0.5) {};
	\end{pgfonlayer}
	\begin{pgfonlayer}{edgelayer}
		\draw [style=simple] (1.center) to (0);
		\draw [style=simple] (0) to (2.center);
	\end{pgfonlayer}
\end{tikzpicture}
\mapsto
\begin{tikzpicture}
	\begin{pgfonlayer}{nodelayer}
		\node [style=oplus] (1) at (0, 1) {};
		\node [style=none] (2) at (0, 2) {};
		\node [style=none] (3) at (0, 0) {};
	\end{pgfonlayer}
	\begin{pgfonlayer}{edgelayer}
		\draw [style=simple] (2.center) to (1);
		\draw [style=simple] (1) to (3.center);
	\end{pgfonlayer}
\end{tikzpicture}
\hspace*{.5cm}
\begin{tikzpicture}
	\begin{pgfonlayer}{nodelayer}
		\node [style=none] (2) at (0, 0) {};
		\node [style=none] (3) at (0.5, 0.75) {};
		\node [style=andin] (30) at (0.5, 0.75) {};
		\node [style=none] (4) at (0.5, 1.5) {};
		\node [style=none] (5) at (1, 0) {};
	\end{pgfonlayer}
	\begin{pgfonlayer}{edgelayer}
		\draw [style=simple] (4.center) to (3);
		\draw [style=simple, in=90, out=-124] (3) to (2.center);
		\draw [style=simple, in=90, out=-56] (3) to (5.center);
	\end{pgfonlayer}
\end{tikzpicture}
\mapsto
\begin{tikzpicture}
	\begin{pgfonlayer}{nodelayer}
		\node [style=dot] (0) at (0, 1.5) {};
		\node [style=dot] (1) at (0.5, 1.5) {};
		\node [style=oplus] (2) at (1, 1.5) {};
		\node [style=X] (3) at (0, 2.25) {};
		\node [style=X] (4) at (0.5, 2.25) {};
		\node [style=none] (5) at (1, 2.5) {};
		\node [style=zeroin] (6) at (1, 0.75) {};
		\node [style=none] (7) at (0, 0.5) {};
		\node [style=none] (8) at (0.5, 0.5) {};
	\end{pgfonlayer}
	\begin{pgfonlayer}{edgelayer}
		\draw [style=simple] (5.center) to (2);
		\draw [style=simple] (2) to (6);
		\draw [style=simple] (2) to (1);
		\draw [style=simple] (1) to (0);
		\draw [style=simple] (3) to (0);
		\draw [style=simple] (0) to (7.center);
		\draw [style=simple] (8.center) to (1);
		\draw [style=simple] (1) to (4);
	\end{pgfonlayer}
\end{tikzpicture}
\hspace*{.5cm}
\begin{tikzpicture}
	\begin{pgfonlayer}{nodelayer}
		\node [style=none] (0) at (0, 2) {};
		\node [style=andout] (1) at (0.5, 1.25) {};
		\node [style=none] (2) at (0.5, 0.5) {};
		\node [style=none] (3) at (1, 2) {};
	\end{pgfonlayer}
	\begin{pgfonlayer}{edgelayer}
		\draw [style=simple] (2.center) to (1.center);
		\draw [style=simple, in=-90, out=124] (1.center) to (0.center);
		\draw [style=simple, in=-90, out=56] (1.center) to (3.center);
	\end{pgfonlayer}
\end{tikzpicture}
\mapsto
\begin{tikzpicture}
	\begin{pgfonlayer}{nodelayer}
		\node [style=dot] (1) at (0, 1) {};
		\node [style=dot] (2) at (0.5, 1) {};
		\node [style=oplus] (3) at (1, 1) {};
		\node [style=X] (4) at (0, 0.25) {};
		\node [style=X] (5) at (0.5, 0.25) {};
		\node [style=none] (6) at (1, 0) {};
		\node [style=zeroout] (7) at (1, 1.75) {};
		\node [style=none] (8) at (0, 2) {};
		\node [style=none] (9) at (0.5, 2) {};
	\end{pgfonlayer}
	\begin{pgfonlayer}{edgelayer}
		\draw [style=simple] (6.center) to (3);
		\draw [style=simple] (3) to (7);
		\draw [style=simple] (3) to (2);
		\draw [style=simple] (2) to (1);
		\draw [style=simple] (4) to (1);
		\draw [style=simple] (1) to (8.center);
		\draw [style=simple] (9.center) to (2);
		\draw [style=simple] (2) to (5);
	\end{pgfonlayer}
\end{tikzpicture}
$$

This interpretation is a strict symmetric \dag-monoidal functor.
\end{proposition}


\begin{proof}
We prove that all of the axioms of $\ZXA$ hold in $\hat \TOF$:
\begin{enumerate}
\item[\ref{ZXA.1}:]
\begin{description}
\item[Unitality:] By Lemma \ref{lemma:whiteunit}:

\begin{align*}
\left\llbracket
\begin{tikzpicture}
	\begin{pgfonlayer}{nodelayer}
		\node [style=none] (0) at (0, 2) {};
		\node [style=none] (1) at (1, 2) {};
		\node [style=Z] (2) at (0.5, 1.25) {};
		\node [style=none] (3) at (0.5, 0.5) {};
		\node [style=Z] (4) at (0, 2) {};
	\end{pgfonlayer}
	\begin{pgfonlayer}{edgelayer}
		\draw [style=simple, in=-90, out=124] (2) to (0.center);
		\draw [style=simple, in=-90, out=56] (2) to (1.center);
		\draw [style=simple] (2) to (3.center);
	\end{pgfonlayer}
\end{tikzpicture}
\right\rrbracket_{\ZXA}
&=
\begin{tikzpicture}
	\begin{pgfonlayer}{nodelayer}
		\node [style=none] (0) at (0, 0.5) {};
		\node [style=dot] (1) at (0.75, 1.25) {};
		\node [style=oplus] (2) at (0, 1.25) {};
		\node [style=X] (3) at (0.75, 0.5) {};
		\node [style=none] (4) at (0.75, 2) {};
		\node [style=zeroout] (5) at (0, 2) {};
	\end{pgfonlayer}
	\begin{pgfonlayer}{edgelayer}
		\draw [style=simple] (4.center) to (1);
		\draw [style=simple] (1) to (3);
		\draw [style=simple] (1) to (2);
		\draw [style=simple] (2) to (0.center);
		\draw [style=simple] (5) to (2);
	\end{pgfonlayer}
\end{tikzpicture}
\eq{comm.}
\begin{tikzpicture}
	\begin{pgfonlayer}{nodelayer}
		\node [style=none] (1) at (0.75, 0.5) {};
		\node [style=dot] (2) at (0.75, 1.25) {};
		\node [style=oplus] (3) at (0, 1.25) {};
		\node [style=X] (4) at (0, 0.5) {};
		\node [style=none] (5) at (0.75, 2) {};
		\node [style=zeroout] (6) at (0, 2) {};
	\end{pgfonlayer}
	\begin{pgfonlayer}{edgelayer}
		\draw [style=simple] (5.center) to (2);
		\draw [style=simple] (2) to (3);
		\draw [style=simple] (6) to (3);
		\draw [style=simple] (1.center) to (2);
		\draw [style=simple] (3) to (4);
	\end{pgfonlayer}
\end{tikzpicture}
\eq{unit}
\begin{tikzpicture}
	\begin{pgfonlayer}{nodelayer}
		\node [style=none] (2) at (0.75, 0.5) {};
		\node [style=X] (3) at (0, 0.5) {};
		\node [style=none] (4) at (0.75, 2) {};
		\node [style=zeroout] (5) at (0, 2) {};
	\end{pgfonlayer}
	\begin{pgfonlayer}{edgelayer}
		\draw [style=simple] (4.center) to (2.center);
		\draw [style=simple] (3) to (5);
	\end{pgfonlayer}
\end{tikzpicture}\
\eq{Rem. \ref{cor:copy}}
\begin{tikzpicture}
	\begin{pgfonlayer}{nodelayer}
		\node [style=none] (3) at (0.5, 3) {};
		\node [style=none] (4) at (0.5, 2) {};
	\end{pgfonlayer}
	\begin{pgfonlayer}{edgelayer}
		\draw [style=simple] (3.center) to (4.center);
	\end{pgfonlayer}
\end{tikzpicture}
=
\left\llbracket
\begin{tikzpicture}
	\begin{pgfonlayer}{nodelayer}
		\node [style=none] (4) at (0.5, 3) {};
		\node [style=none] (5) at (0.5, 2) {};
	\end{pgfonlayer}
	\begin{pgfonlayer}{edgelayer}
		\draw [style=simple] (4.center) to (5.center);
	\end{pgfonlayer}
\end{tikzpicture}
\right\rrbracket_{\ZXA}
\end{align*}

\item[Associativity:]
\begin{align*}
\left\llbracket
\begin{tikzpicture}
	\begin{pgfonlayer}{nodelayer}
		\node [style=none] (5) at (0, 3.5) {};
		\node [style=none] (6) at (1, 3.5) {};
		\node [style=Z] (7) at (0.5, 2.75) {};
		\node [style=none] (8) at (0.5, 2) {};
		\node [style=Z] (9) at (0, 3.5) {};
		\node [style=none] (10) at (0.5, 4.25) {};
		\node [style=none] (11) at (1, 4.25) {};
		\node [style=none] (12) at (-0.5, 4.25) {};
	\end{pgfonlayer}
	\begin{pgfonlayer}{edgelayer}
		\draw [style=simple, in=-90, out=124] (7) to (5.center);
		\draw [style=simple, in=-90, out=56] (7) to (6.center);
		\draw [style=simple] (7) to (8.center);
		\draw [style=simple, in=56, out=-90] (10.center) to (5.center);
		\draw [style=simple, in=-90, out=124] (5.center) to (12.center);
		\draw [style=simple] (11.center) to (6.center);
	\end{pgfonlayer}
\end{tikzpicture}
\right\rrbracket_{\ZXA}
=
\begin{tikzpicture}
	\begin{pgfonlayer}{nodelayer}
		\node [style=none] (6) at (0, 2) {};
		\node [style=dot] (7) at (1.25, 2.75) {};
		\node [style=oplus] (8) at (0, 2.75) {};
		\node [style=X] (9) at (1.25, 2) {};
		\node [style=none] (10) at (1.25, 5) {};
		\node [style=dot] (11) at (0.75, 4.25) {};
		\node [style=X] (12) at (0.75, 3.5) {};
		\node [style=oplus] (13) at (0, 4.25) {};
		\node [style=none] (14) at (0.75, 5) {};
		\node [style=none] (15) at (0, 5) {};
	\end{pgfonlayer}
	\begin{pgfonlayer}{edgelayer}
		\draw [style=simple] (10.center) to (7);
		\draw [style=simple] (7) to (9);
		\draw [style=simple] (7) to (8);
		\draw [style=simple] (8) to (6.center);
		\draw [style=simple] (14.center) to (11);
		\draw [style=simple] (11) to (12);
		\draw [style=simple] (11) to (13);
		\draw [style=simple] (13) to (15.center);
		\draw [style=simple] (13) to (8);
	\end{pgfonlayer}
\end{tikzpicture}
\eq{\ref{CNOT.8}}
\begin{tikzpicture}
	\begin{pgfonlayer}{nodelayer}
		\node [style=none] (7) at (0, 2) {};
		\node [style=X] (8) at (1.5, 2) {};
		\node [style=none] (9) at (1.5, 5) {};
		\node [style=dot] (10) at (0.75, 3.5) {};
		\node [style=X] (11) at (0.75, 2) {};
		\node [style=oplus] (12) at (0, 3.5) {};
		\node [style=none] (13) at (0.75, 5) {};
		\node [style=none] (14) at (0, 5) {};
		\node [style=oplus] (15) at (0.75, 4.25) {};
		\node [style=dot] (16) at (1.5, 4.25) {};
		\node [style=oplus] (17) at (0.75, 2.75) {};
		\node [style=dot] (18) at (1.5, 2.75) {};
	\end{pgfonlayer}
	\begin{pgfonlayer}{edgelayer}
		\draw [style=simple] (13.center) to (10);
		\draw [style=simple] (10) to (11);
		\draw [style=simple] (10) to (12);
		\draw [style=simple] (12) to (14.center);
		\draw [style=simple] (16) to (15);
		\draw [style=simple] (18) to (17);
		\draw [style=simple] (9.center) to (16);
		\draw [style=simple] (16) to (18);
		\draw [style=simple] (18) to (8);
		\draw [style=simple] (7.center) to (12);
	\end{pgfonlayer}
\end{tikzpicture}
\eq{Rem. \ref{cor:copy}}
\begin{tikzpicture}
	\begin{pgfonlayer}{nodelayer}
		\node [style=none] (8) at (0, 3.25) {};
		\node [style=dot] (9) at (0.75, 4) {};
		\node [style=oplus] (10) at (0, 4) {};
		\node [style=X] (11) at (0.75, 3.25) {};
		\node [style=X] (12) at (1.5, 4) {};
		\node [style=none] (13) at (0.75, 5.5) {};
		\node [style=none] (14) at (1.5, 5.5) {};
		\node [style=oplus] (15) at (0.75, 4.75) {};
		\node [style=dot] (16) at (1.5, 4.75) {};
		\node [style=none] (17) at (0, 5.5) {};
	\end{pgfonlayer}
	\begin{pgfonlayer}{edgelayer}
		\draw [style=simple] (9) to (11);
		\draw [style=simple] (9) to (10);
		\draw [style=simple] (10) to (8.center);
		\draw [style=simple] (14.center) to (16);
		\draw [style=simple] (16) to (12);
		\draw [style=simple] (16) to (15);
		\draw [style=simple] (15) to (13.center);
		\draw [style=simple] (9) to (15);
		\draw [style=simple] (17.center) to (10);
	\end{pgfonlayer}
\end{tikzpicture}
=
\left\llbracket
\begin{tikzpicture}
	\begin{pgfonlayer}{nodelayer}
		\node [style=none] (9) at (0.5, 4.75) {};
		\node [style=none] (10) at (-0.5, 4.75) {};
		\node [style=Z] (11) at (0, 4) {};
		\node [style=none] (12) at (0, 3.25) {};
		\node [style=Z] (13) at (0.5, 4.75) {};
		\node [style=none] (14) at (0, 5.5) {};
		\node [style=none] (15) at (-0.5, 5.5) {};
		\node [style=none] (16) at (1, 5.5) {};
	\end{pgfonlayer}
	\begin{pgfonlayer}{edgelayer}
		\draw [style=simple, in=-90, out=56] (11) to (9.center);
		\draw [style=simple, in=-90, out=124] (11) to (10.center);
		\draw [style=simple] (11) to (12.center);
		\draw [style=simple, in=124, out=-90] (14.center) to (9.center);
		\draw [style=simple, in=-90, out=56] (9.center) to (16.center);
		\draw [style=simple] (15.center) to (10.center);
	\end{pgfonlayer}
\end{tikzpicture}
\right\rrbracket_{\ZXA}
\end{align*}

\item[Frobenius:]
\begin{align*}
\left\llbracket
\begin{tikzpicture}
	\begin{pgfonlayer}{nodelayer}
		\node [style=none] (10) at (1, 4.75) {};
		\node [style=Z] (11) at (0.5, 4) {};
		\node [style=none] (12) at (0.5, 3.25) {};
		\node [style=none] (13) at (0, 4.75) {};
		\node [style=Z] (14) at (0, 4.75) {};
		\node [style=none] (15) at (0.5, 4) {};
		\node [style=none] (16) at (-0.5, 4) {};
		\node [style=none] (17) at (1, 5.5) {};
		\node [style=none] (18) at (0, 5.5) {};
		\node [style=none] (19) at (-0.5, 3.25) {};
	\end{pgfonlayer}
	\begin{pgfonlayer}{edgelayer}
		\draw [style=simple, in=-90, out=56] (11) to (10.center);
		\draw [style=simple] (11) to (12.center);
		\draw [style=simple, in=-56, out=90] (15.center) to (13.center);
		\draw [style=simple, in=90, out=-124] (13.center) to (16.center);
		\draw [style=simple] (17.center) to (10.center);
		\draw [style=simple] (13.center) to (18.center);
		\draw [style=simple] (16.center) to (19.center);
	\end{pgfonlayer}
\end{tikzpicture}
\right\rrbracket_{\ZXA}
=
\begin{tikzpicture}
	\begin{pgfonlayer}{nodelayer}
		\node [style=none] (11) at (0, 3.25) {};
		\node [style=dot] (12) at (0.75, 4) {};
		\node [style=oplus] (13) at (0, 4) {};
		\node [style=X] (14) at (0.75, 3.25) {};
		\node [style=none] (15) at (0, 5.5) {};
		\node [style=oplus] (16) at (0.75, 4.75) {};
		\node [style=X] (17) at (1.5, 5.5) {};
		\node [style=dot] (18) at (1.5, 4.75) {};
		\node [style=none] (19) at (0.75, 5.5) {};
		\node [style=none] (20) at (1.5, 3.25) {};
	\end{pgfonlayer}
	\begin{pgfonlayer}{edgelayer}
		\draw [style=simple] (12) to (14);
		\draw [style=simple] (12) to (13);
		\draw [style=simple] (13) to (11.center);
		\draw [style=simple] (15.center) to (13);
		\draw [style=simple] (18) to (17);
		\draw [style=simple] (18) to (16);
		\draw [style=simple] (18) to (20.center);
		\draw [style=simple] (12) to (16);
		\draw [style=simple] (16) to (19.center);
	\end{pgfonlayer}
\end{tikzpicture}
\eq{Lem \ref{lemma:Iwama}}
\begin{tikzpicture}
	\begin{pgfonlayer}{nodelayer}
		\node [style=dot] (12) at (0.75, 4.75) {};
		\node [style=oplus] (13) at (0, 4.75) {};
		\node [style=dot] (14) at (1.5, 5.5) {};
		\node [style=X] (15) at (1.5, 6.25) {};
		\node [style=none] (16) at (0, 6.25) {};
		\node [style=none] (17) at (0.75, 6.25) {};
		\node [style=none] (18) at (0, 3.25) {};
		\node [style=X] (19) at (0.75, 3.25) {};
		\node [style=none] (20) at (1.5, 3.25) {};
		\node [style=oplus] (21) at (0, 5.5) {};
		\node [style=oplus] (22) at (0.75, 4) {};
		\node [style=dot] (23) at (1.5, 4) {};
	\end{pgfonlayer}
	\begin{pgfonlayer}{edgelayer}
		\draw [style=simple] (12) to (13);
		\draw [style=simple] (15) to (14);
		\draw [style=simple] (20.center) to (23);
		\draw [style=simple] (23) to (14);
		\draw [style=simple] (14) to (21);
		\draw [style=simple] (17.center) to (12);
		\draw [style=simple] (12) to (22);
		\draw [style=simple] (22) to (19);
		\draw [style=simple] (18.center) to (13);
		\draw [style=simple] (13) to (21);
		\draw [style=simple] (21) to (16.center);
		\draw [style=simple] (23) to (22);
	\end{pgfonlayer}
\end{tikzpicture}
\eq{Lem. \ref{lemma:whiteunit}}
\begin{tikzpicture}
	\begin{pgfonlayer}{nodelayer}
		\node [style=dot] (13) at (0.75, 4.75) {};
		\node [style=oplus] (14) at (0, 4.75) {};
		\node [style=dot] (15) at (1.5, 5.5) {};
		\node [style=X] (16) at (1.5, 6.25) {};
		\node [style=none] (17) at (0, 6.25) {};
		\node [style=none] (18) at (0.75, 6.25) {};
		\node [style=none] (19) at (0, 3.25) {};
		\node [style=X] (20) at (0.75, 3.25) {};
		\node [style=none] (21) at (1.5, 3.25) {};
		\node [style=oplus] (22) at (0, 5.5) {};
	\end{pgfonlayer}
	\begin{pgfonlayer}{edgelayer}
		\draw [style=simple] (13) to (14);
		\draw [style=simple] (16) to (15);
		\draw [style=simple] (15) to (22);
		\draw [style=simple] (18.center) to (13);
		\draw [style=simple] (19.center) to (14);
		\draw [style=simple] (14) to (22);
		\draw [style=simple] (22) to (17.center);
		\draw [style=simple] (15) to (21.center);
		\draw [style=simple] (20) to (13);
	\end{pgfonlayer}
\end{tikzpicture}
\eq{\ref{CNOT.5}}
\begin{tikzpicture}
	\begin{pgfonlayer}{nodelayer}
		\node [style=dot] (14) at (0.75, 6.25) {};
		\node [style=oplus] (15) at (0, 6.25) {};
		\node [style=dot] (16) at (0.75, 4) {};
		\node [style=X] (17) at (0.75, 4.75) {};
		\node [style=none] (18) at (0, 7) {};
		\node [style=none] (19) at (0.75, 7) {};
		\node [style=none] (20) at (0, 3.25) {};
		\node [style=X] (21) at (0.75, 5.5) {};
		\node [style=none] (22) at (0.75, 3.25) {};
		\node [style=oplus] (23) at (0, 4) {};
	\end{pgfonlayer}
	\begin{pgfonlayer}{edgelayer}
		\draw [style=simple] (14) to (15);
		\draw [style=simple] (17) to (16);
		\draw [style=simple] (16) to (23);
		\draw [style=simple] (19.center) to (14);
		\draw [style=simple] (20.center) to (15);
		\draw [style=simple] (23) to (18.center);
		\draw [style=simple] (16) to (22.center);
		\draw [style=simple] (21) to (14);
	\end{pgfonlayer}
\end{tikzpicture}
=
\left\llbracket
\begin{tikzpicture}
	\begin{pgfonlayer}{nodelayer}
		\node [style=none] (15) at (1, 5.75) {};
		\node [style=Z] (16) at (0.5, 5) {};
		\node [style=none] (17) at (0, 5.75) {};
		\node [style=none] (18) at (0.5, 5) {};
		\node [style=none] (19) at (1, 3.25) {};
		\node [style=none] (20) at (0.5, 4) {};
		\node [style=Z] (21) at (0.5, 4) {};
		\node [style=none] (22) at (0, 3.25) {};
	\end{pgfonlayer}
	\begin{pgfonlayer}{edgelayer}
		\draw [style=simple, in=-90, out=60] (16) to (15.center);
		\draw [style=simple, in=-90, out=120] (18.center) to (17.center);
		\draw [style=simple, in=90, out=-60] (21) to (19.center);
		\draw [style=simple, in=90, out=-120] (20.center) to (22.center);
		\draw [style=simple] (16) to (20.center);
	\end{pgfonlayer}
\end{tikzpicture}
\right\rrbracket_{\ZXA}
\end{align*}


\item[Phase amalgamation:]

\begin{align*}
\left\llbracket
\begin{tikzpicture}
	\begin{pgfonlayer}{nodelayer}
		\node [style=none] (16) at (0.75, 3.25) {};
		\node [style=Z] (17) at (0.75, 4.25) {$\pi$};
		\node [style=Z] (18) at (0.75, 5.25) {$\pi$};
		\node [style=none] (19) at (0.75, 6.25) {};
	\end{pgfonlayer}
	\begin{pgfonlayer}{edgelayer}
		\draw [style=simple] (19.center) to (18);
		\draw [style=simple] (18) to (17);
		\draw [style=simple] (17) to (16.center);
	\end{pgfonlayer}
\end{tikzpicture}
\right\rrbracket_{\ZXA}
&=
\begin{tikzpicture}
	\begin{pgfonlayer}{nodelayer}
		\node [style=none] (17) at (0.75, 3.25) {};
		\node [style=none] (18) at (0.75, 6.25) {};
		\node [style=oplus] (19) at (0.75, 4.25) {};
		\node [style=oplus] (20) at (0.75, 5.25) {};
	\end{pgfonlayer}
	\begin{pgfonlayer}{edgelayer}
		\draw [style=simple] (18.center) to (20);
		\draw [style=simple] (20) to (19);
		\draw [style=simple] (19) to (17.center);
	\end{pgfonlayer}
\end{tikzpicture}
=
\begin{tikzpicture}
	\begin{pgfonlayer}{nodelayer}
		\node [style=none] (18) at (0.75, 3.25) {};
		\node [style=none] (19) at (0.75, 4.25) {};
	\end{pgfonlayer}
	\begin{pgfonlayer}{edgelayer}
		\draw [style=simple] (19.center) to (18.center);
	\end{pgfonlayer}
\end{tikzpicture}
=
\left\llbracket
\begin{tikzpicture}
	\begin{pgfonlayer}{nodelayer}
		\node [style=none] (19) at (0.75, 3.25) {};
		\node [style=none] (20) at (0.75, 4.25) {};
	\end{pgfonlayer}
	\begin{pgfonlayer}{edgelayer}
		\draw [style=simple] (20.center) to (19.center);
	\end{pgfonlayer}
\end{tikzpicture}
\right\rrbracket_{\ZXA}
\end{align*}



\end{description}


\item[\ref{ZXA.2}:]
\begin{align*}
\left\llbracket
\begin{tikzpicture}
	\begin{pgfonlayer}{nodelayer}
		\node [style=none] (20) at (0, 4.25) {};
		\node [style=none] (21) at (0, 3.25) {};
		\node [style=Z] (22) at (0, 4.25) {};
		\node [style=none] (23) at (0.5, 5) {};
		\node [style=none] (24) at (-0.5, 5) {};
		\node [style=none] (25) at (0.5, 5.75) {};
		\node [style=none] (26) at (-0.5, 5.75) {};
	\end{pgfonlayer}
	\begin{pgfonlayer}{edgelayer}
		\draw [style=simple, in=56, out=-90] (23.center) to (20.center);
		\draw [style=simple, in=-90, out=124] (20.center) to (24.center);
		\draw [style=simple] (21.center) to (20.center);
		\draw [style=simple, in=90, out=-90] (25.center) to (24.center);
		\draw [style=simple, in=90, out=-90] (26.center) to (23.center);
	\end{pgfonlayer}
\end{tikzpicture}
\right\rrbracket_{\ZXA}
=
\begin{tikzpicture}
	\begin{pgfonlayer}{nodelayer}
		\node [style=oplus] (21) at (0, 4.25) {};
		\node [style=dot] (22) at (0.75, 4.25) {};
		\node [style=X] (23) at (0.75, 3.5) {};
		\node [style=none] (24) at (0, 3.25) {};
		\node [style=none] (25) at (0, 5) {};
		\node [style=none] (26) at (0.75, 5) {};
		\node [style=none] (27) at (0, 6) {};
		\node [style=none] (28) at (0.75, 6) {};
	\end{pgfonlayer}
	\begin{pgfonlayer}{edgelayer}
		\draw [style=simple] (26.center) to (22);
		\draw [style=simple] (23) to (22);
		\draw [style=simple] (22) to (21);
		\draw [style=simple] (21) to (24.center);
		\draw [style=simple] (21) to (25.center);
		\draw [style=simple, in=90, out=-90] (28.center) to (25.center);
		\draw [style=simple, in=90, out=-90] (27.center) to (26.center);
	\end{pgfonlayer}
\end{tikzpicture}
\eq{\ref{TOF.14}}
\begin{tikzpicture}
	\begin{pgfonlayer}{nodelayer}
		\node [style=oplus] (22) at (0, 4.25) {};
		\node [style=dot] (23) at (0.75, 4.25) {};
		\node [style=X] (24) at (0.75, 3.5) {};
		\node [style=none] (25) at (0, 3.25) {};
		\node [style=none] (26) at (0, 6.5) {};
		\node [style=none] (27) at (0.75, 6.5) {};
		\node [style=oplus] (28) at (0, 4.75) {};
		\node [style=dot] (29) at (0.75, 5.75) {};
		\node [style=dot] (30) at (0.75, 4.75) {};
		\node [style=oplus] (31) at (0.75, 5.25) {};
		\node [style=oplus] (32) at (0, 5.75) {};
		\node [style=dot] (33) at (0, 5.25) {};
	\end{pgfonlayer}
	\begin{pgfonlayer}{edgelayer}
		\draw [style=simple] (24) to (23);
		\draw [style=simple] (23) to (22);
		\draw [style=simple] (22) to (25.center);
		\draw [style=simple] (30) to (28);
		\draw [style=simple] (33) to (31);
		\draw [style=simple] (29) to (32);
		\draw [style=simple] (32) to (33);
		\draw [style=simple] (33) to (28);
		\draw [style=simple] (30) to (31);
		\draw [style=simple] (31) to (29);
		\draw [style=simple] (27.center) to (29);
		\draw [style=simple] (30) to (23);
		\draw [style=simple] (22) to (28);
		\draw [style=simple] (32) to (26.center);
	\end{pgfonlayer}
\end{tikzpicture}
\eq{\ref{CNOT.2}}
\begin{tikzpicture}
	\begin{pgfonlayer}{nodelayer}
		\node [style=X] (2) at (1.25, 4.5) {};
		\node [style=none] (3) at (0.5, 4.25) {};
		\node [style=none] (4) at (0.5, 6.5) {};
		\node [style=none] (5) at (1.25, 6.5) {};
		\node [style=dot] (7) at (1.25, 5.75) {};
		\node [style=oplus] (9) at (1.25, 5.25) {};
		\node [style=oplus] (10) at (0.5, 5.75) {};
		\node [style=dot] (11) at (0.5, 5.25) {};
	\end{pgfonlayer}
	\begin{pgfonlayer}{edgelayer}
		\draw [style=simple] (11) to (9);
		\draw [style=simple] (7) to (10);
		\draw [style=simple] (10) to (11);
		\draw [style=simple] (9) to (7);
		\draw [style=simple] (5.center) to (7);
		\draw [style=simple] (10) to (4.center);
		\draw (2) to (9);
		\draw (3.center) to (11);
	\end{pgfonlayer}
\end{tikzpicture}
\eq{Lem. \ref{lemma:whiteunit}}
\begin{tikzpicture}
	\begin{pgfonlayer}{nodelayer}
		\node [style=oplus] (0) at (0, 4) {};
		\node [style=dot] (1) at (0.75, 4) {};
		\node [style=X] (2) at (0.75, 3.25) {};
		\node [style=none] (3) at (0, 3) {};
		\node [style=none] (4) at (0, 4.75) {};
		\node [style=none] (5) at (0.75, 4.75) {};
	\end{pgfonlayer}
	\begin{pgfonlayer}{edgelayer}
		\draw [style=simple] (5.center) to (1);
		\draw [style=simple] (2) to (1);
		\draw [style=simple] (1) to (0);
		\draw [style=simple] (0) to (3.center);
		\draw [style=simple] (0) to (4.center);
	\end{pgfonlayer}
\end{tikzpicture}
=
\left\llbracket
\begin{tikzpicture}
	\begin{pgfonlayer}{nodelayer}
		\node [style=none] (0) at (0, 2) {};
		\node [style=none] (1) at (0, 1) {};
		\node [style=Z] (2) at (0, 2) {};
		\node [style=none] (3) at (0.5, 2.75) {};
		\node [style=none] (4) at (-0.5, 2.75) {};
	\end{pgfonlayer}
	\begin{pgfonlayer}{edgelayer}
		\draw [style=simple, in=56, out=-90] (3.center) to (0.center);
		\draw [style=simple, in=-90, out=124] (0.center) to (4.center);
		\draw [style=simple] (1.center) to (0.center);
	\end{pgfonlayer}
\end{tikzpicture}
\right\rrbracket_{\ZXA}
\end{align*}


\item[\ref{ZXA.3}:]
This is immediate.

\item[\ref{ZXA.4}:]
This is immediate.


\item[\ref{ZXA.5}:]
\begin{align*}
\left\llbracket
\begin{tikzpicture}
	\begin{pgfonlayer}{nodelayer}
		\node [style=X] (0) at (-1, 1) {};
		\node [style=X] (1) at (-0.25, 1) {};
		\node [style=Z] (2) at (-0.25, 1.75) {};
		\node [style=Z] (3) at (-1, 1.75) {};
		\node [style=none] (4) at (-1, 2.25) {};
		\node [style=none] (5) at (-0.25, 2.25) {};
		\node [style=none] (6) at (-1, 0.5) {};
		\node [style=none] (7) at (-0.25, 0.5) {};
	\end{pgfonlayer}
	\begin{pgfonlayer}{edgelayer}
		\draw (7.center) to (1);
		\draw (1) to (3);
		\draw [in=120, out=-120, looseness=1.25] (3) to (0);
		\draw (0) to (2);
		\draw (2) to (5.center);
		\draw [in=60, out=-60, looseness=1.25] (2) to (1);
		\draw (0) to (6.center);
		\draw (3) to (4.center);
	\end{pgfonlayer}
\end{tikzpicture}
\right\rrbracket_{\ZXA}
&=
\begin{tikzpicture}
	\begin{pgfonlayer}{nodelayer}
		\node [style=none] (0) at (-1, 0.5) {};
		\node [style=none] (1) at (0.5, 0.5) {};
		\node [style=none] (2) at (-1, 3) {};
		\node [style=none] (3) at (0.5, 3) {};
		\node [style=zeroin] (4) at (-0.5, 0.75) {};
		\node [style=zeroin] (5) at (0, 0.75) {};
		\node [style=oplus] (6) at (-0.5, 1.25) {};
		\node [style=oplus] (7) at (0, 1.25) {};
		\node [style=dot] (8) at (-1, 1.25) {};
		\node [style=dot] (9) at (0.5, 1.25) {};
		\node [style=dot] (10) at (0, 1.75) {};
		\node [style=dot] (11) at (-0.5, 2.25) {};
		\node [style=oplus] (12) at (-1, 1.75) {};
		\node [style=oplus] (13) at (0.5, 2.25) {};
		\node [style=X] (14) at (-0.5, 2.75) {};
		\node [style=X] (15) at (0, 2.75) {};
	\end{pgfonlayer}
	\begin{pgfonlayer}{edgelayer}
		\draw (1.center) to (9);
		\draw (9) to (13);
		\draw (13) to (3.center);
		\draw (15) to (10);
		\draw (10) to (7);
		\draw (7) to (5);
		\draw (4) to (6);
		\draw (6) to (11);
		\draw (11) to (14);
		\draw (2.center) to (12);
		\draw (12) to (8);
		\draw (8) to (0.center);
		\draw (8) to (6);
		\draw (7) to (9);
		\draw (10) to (12);
		\draw (11) to (13);
	\end{pgfonlayer}
\end{tikzpicture}
\eq{Lem \ref{lemma:Iwama}}
\begin{tikzpicture}
	\begin{pgfonlayer}{nodelayer}
		\node [style=none] (0) at (-1, 0.5) {};
		\node [style=none] (1) at (0.5, 0.5) {};
		\node [style=none] (2) at (-1, 4) {};
		\node [style=none] (3) at (0.5, 4) {};
		\node [style=zeroin] (4) at (-0.5, 0.75) {};
		\node [style=zeroin] (5) at (0, 0.75) {};
		\node [style=oplus] (6) at (-0.5, 1.25) {};
		\node [style=oplus] (7) at (0, 2.75) {};
		\node [style=dot] (8) at (-1, 1.25) {};
		\node [style=dot] (9) at (0.5, 2.75) {};
		\node [style=dot] (10) at (0, 2.25) {};
		\node [style=dot] (11) at (-0.5, 3.25) {};
		\node [style=oplus] (12) at (-1, 2.25) {};
		\node [style=oplus] (13) at (0.5, 3.25) {};
		\node [style=X] (14) at (-0.5, 3.75) {};
		\node [style=X] (15) at (0, 3.75) {};
		\node [style=oplus] (16) at (-1, 1.75) {};
		\node [style=dot] (17) at (0.5, 1.75) {};
	\end{pgfonlayer}
	\begin{pgfonlayer}{edgelayer}
		\draw (1.center) to (9);
		\draw (9) to (13);
		\draw (13) to (3.center);
		\draw (15) to (10);
		\draw (10) to (7);
		\draw (7) to (5);
		\draw (4) to (6);
		\draw (6) to (11);
		\draw (11) to (14);
		\draw (2.center) to (12);
		\draw (12) to (8);
		\draw (8) to (0.center);
		\draw (8) to (6);
		\draw (7) to (9);
		\draw (10) to (12);
		\draw (11) to (13);
		\draw (17) to (16);
	\end{pgfonlayer}
\end{tikzpicture}
\eq{\ref{TOF.2}}
\begin{tikzpicture}
	\begin{pgfonlayer}{nodelayer}
		\node [style=none] (0) at (-1, 0.5) {};
		\node [style=none] (1) at (0.5, 0.5) {};
		\node [style=none] (2) at (-1, 3.5) {};
		\node [style=none] (3) at (0.5, 3.5) {};
		\node [style=zeroin] (4) at (-0.5, 0.75) {};
		\node [style=zeroin] (5) at (0, 0.75) {};
		\node [style=oplus] (6) at (-0.5, 1.25) {};
		\node [style=oplus] (7) at (0, 2.25) {};
		\node [style=dot] (8) at (-1, 1.25) {};
		\node [style=dot] (9) at (0.5, 2.25) {};
		\node [style=dot] (10) at (-0.5, 2.75) {};
		\node [style=oplus] (11) at (0.5, 2.75) {};
		\node [style=X] (12) at (-0.5, 3.25) {};
		\node [style=X] (13) at (0, 3.25) {};
		\node [style=oplus] (14) at (-1, 1.75) {};
		\node [style=dot] (15) at (0.5, 1.75) {};
	\end{pgfonlayer}
	\begin{pgfonlayer}{edgelayer}
		\draw (1.center) to (9);
		\draw (9) to (11);
		\draw (11) to (3.center);
		\draw (7) to (5);
		\draw (4) to (6);
		\draw (6) to (10);
		\draw (10) to (12);
		\draw (8) to (0.center);
		\draw (8) to (6);
		\draw (7) to (9);
		\draw (10) to (11);
		\draw (15) to (14);
		\draw (2.center) to (14);
		\draw (14) to (8);
		\draw (7) to (13);
	\end{pgfonlayer}
\end{tikzpicture}
\eq{unit}
\begin{tikzpicture}
	\begin{pgfonlayer}{nodelayer}
		\node [style=none] (0) at (-1, 0.5) {};
		\node [style=none] (1) at (0, 0.5) {};
		\node [style=none] (2) at (-1, 3) {};
		\node [style=none] (3) at (0, 3) {};
		\node [style=zeroin] (4) at (-0.5, 0.75) {};
		\node [style=oplus] (5) at (-0.5, 1.25) {};
		\node [style=dot] (6) at (-1, 1.25) {};
		\node [style=dot] (7) at (-0.5, 2.25) {};
		\node [style=oplus] (8) at (0, 2.25) {};
		\node [style=X] (9) at (-0.5, 2.75) {};
		\node [style=oplus] (10) at (-1, 1.75) {};
		\node [style=dot] (11) at (0, 1.75) {};
	\end{pgfonlayer}
	\begin{pgfonlayer}{edgelayer}
		\draw (8) to (3.center);
		\draw (4) to (5);
		\draw (5) to (7);
		\draw (7) to (9);
		\draw (6) to (0.center);
		\draw (6) to (5);
		\draw (7) to (8);
		\draw (11) to (10);
		\draw (2.center) to (10);
		\draw (10) to (6);
		\draw (8) to (11);
		\draw (11) to (1.center);
	\end{pgfonlayer}
\end{tikzpicture}
=
\begin{tikzpicture}
	\begin{pgfonlayer}{nodelayer}
		\node [style=none] (0) at (-0.75, 0.5) {};
		\node [style=none] (1) at (0, 0.5) {};
		\node [style=none] (2) at (-1, 3) {};
		\node [style=none] (3) at (0, 3) {};
		\node [style=dot] (4) at (-0.5, 2.25) {};
		\node [style=oplus] (5) at (0, 2.25) {};
		\node [style=X] (6) at (-0.5, 2.75) {};
		\node [style=oplus] (7) at (-1, 1.75) {};
		\node [style=dot] (8) at (0, 1.75) {};
		\node [style=fanout] (9) at (-0.75, 1) {};
		\node [style=none] (10) at (-0.5, 1.75) {};
	\end{pgfonlayer}
	\begin{pgfonlayer}{edgelayer}
		\draw (5) to (3.center);
		\draw (4) to (6);
		\draw (4) to (5);
		\draw (8) to (7);
		\draw (2.center) to (7);
		\draw (5) to (8);
		\draw (8) to (1.center);
		\draw (0.center) to (9);
		\draw [in=-90, out=108] (9) to (7);
		\draw (4) to (10.center);
		\draw [in=72, out=-90] (10.center) to (9);
	\end{pgfonlayer}
\end{tikzpicture}
\eq{\ref{CNOT.2}}
\begin{tikzpicture}
	\begin{pgfonlayer}{nodelayer}
		\node [style=none] (0) at (-0.75, 0.5) {};
		\node [style=none] (1) at (0, 0.5) {};
		\node [style=none] (2) at (-1, 4) {};
		\node [style=none] (3) at (0, 4) {};
		\node [style=dot] (4) at (-0.5, 3.25) {};
		\node [style=oplus] (5) at (0, 3.25) {};
		\node [style=X] (6) at (-0.5, 3.75) {};
		\node [style=oplus] (7) at (-1, 2.75) {};
		\node [style=dot] (8) at (0, 2.75) {};
		\node [style=fanout] (9) at (-0.75, 1) {};
		\node [style=oplus] (10) at (-0.5, 2.25) {};
		\node [style=dot] (11) at (0, 2.25) {};
		\node [style=oplus] (12) at (-0.5, 1.75) {};
		\node [style=dot] (13) at (0, 1.75) {};
		\node [style=none] (14) at (-1, 1.75) {};
	\end{pgfonlayer}
	\begin{pgfonlayer}{edgelayer}
		\draw (5) to (3.center);
		\draw (4) to (6);
		\draw (4) to (5);
		\draw (8) to (7);
		\draw (2.center) to (7);
		\draw (5) to (8);
		\draw (8) to (1.center);
		\draw (0.center) to (9);
		\draw (11) to (10);
		\draw (13) to (12);
		\draw (4) to (10);
		\draw (10) to (12);
		\draw [in=60, out=-90] (12) to (9);
		\draw (7) to (14.center);
		\draw [in=120, out=-90] (14.center) to (9);
	\end{pgfonlayer}
\end{tikzpicture}\\
&\eq{Lem. \ref{lemma:natoplus}}
\begin{tikzpicture}
	\begin{pgfonlayer}{nodelayer}
		\node [style=none] (0) at (-0.75, 0.5) {};
		\node [style=none] (1) at (0, 0.5) {};
		\node [style=none] (2) at (-1, 3.75) {};
		\node [style=none] (3) at (0, 3.75) {};
		\node [style=dot] (4) at (-0.5, 3) {};
		\node [style=oplus] (5) at (0, 3) {};
		\node [style=X] (6) at (-0.5, 3.5) {};
		\node [style=fanout] (7) at (-0.75, 1.75) {};
		\node [style=oplus] (8) at (-0.5, 2.5) {};
		\node [style=dot] (9) at (0, 2.5) {};
		\node [style=none] (10) at (-1, 2.5) {};
		\node [style=oplus] (11) at (-0.75, 1) {};
		\node [style=dot] (12) at (0, 1) {};
	\end{pgfonlayer}
	\begin{pgfonlayer}{edgelayer}
		\draw (5) to (3.center);
		\draw (4) to (6);
		\draw (4) to (5);
		\draw (0.center) to (7);
		\draw (9) to (8);
		\draw (4) to (8);
		\draw [in=120, out=-90] (10.center) to (7);
		\draw (12) to (11);
		\draw [in=-90, out=60] (7) to (8);
		\draw (9) to (12);
		\draw (12) to (1.center);
		\draw (9) to (5);
		\draw (2.center) to (10.center);
	\end{pgfonlayer}
\end{tikzpicture}
\eq{Lem. \ref{lemma:whiteunit}}
\begin{tikzpicture}
	\begin{pgfonlayer}{nodelayer}
		\node [style=none] (0) at (-0.75, 0.5) {};
		\node [style=none] (1) at (0, 0.5) {};
		\node [style=none] (2) at (-1, 4.25) {};
		\node [style=none] (3) at (0, 4.25) {};
		\node [style=dot] (4) at (-0.5, 3) {};
		\node [style=oplus] (5) at (0, 3) {};
		\node [style=X] (6) at (-0.5, 4) {};
		\node [style=fanout] (7) at (-0.75, 1.75) {};
		\node [style=oplus] (8) at (-0.5, 2.5) {};
		\node [style=dot] (9) at (0, 2.5) {};
		\node [style=none] (10) at (-1, 2.5) {};
		\node [style=oplus] (11) at (-0.75, 1) {};
		\node [style=dot] (12) at (0, 1) {};
		\node [style=oplus] (13) at (-0.5, 3.5) {};
		\node [style=dot] (14) at (0, 3.5) {};
	\end{pgfonlayer}
	\begin{pgfonlayer}{edgelayer}
		\draw (5) to (3.center);
		\draw (4) to (6);
		\draw (4) to (5);
		\draw (0.center) to (7);
		\draw (9) to (8);
		\draw (4) to (8);
		\draw [in=120, out=-90] (10.center) to (7);
		\draw (12) to (11);
		\draw [in=-90, out=60] (7) to (8);
		\draw (9) to (12);
		\draw (12) to (1.center);
		\draw (9) to (5);
		\draw (2.center) to (10.center);
		\draw (14) to (13);
	\end{pgfonlayer}
\end{tikzpicture}
\eq{\ref{TOF.14}}
\begin{tikzpicture}
	\begin{pgfonlayer}{nodelayer}
		\node [style=none] (0) at (-0.75, 0.5) {};
		\node [style=none] (1) at (0, 0.5) {};
		\node [style=none] (2) at (-1, 3.75) {};
		\node [style=none] (3) at (0, 3.75) {};
		\node [style=X] (4) at (-0.5, 3.5) {};
		\node [style=fanout] (5) at (-0.75, 1.75) {};
		\node [style=none] (6) at (-1, 2.5) {};
		\node [style=oplus] (7) at (-0.75, 1) {};
		\node [style=dot] (8) at (0, 1) {};
		\node [style=none] (9) at (0, 3.5) {};
		\node [style=none] (10) at (0, 2.5) {};
		\node [style=none] (11) at (-0.5, 2.5) {};
	\end{pgfonlayer}
	\begin{pgfonlayer}{edgelayer}
		\draw (0.center) to (5);
		\draw [in=120, out=-90] (6.center) to (5);
		\draw (8) to (7);
		\draw (8) to (1.center);
		\draw (2.center) to (6.center);
		\draw [in=90, out=-90] (9.center) to (11.center);
		\draw [in=-90, out=90] (10.center) to (4);
		\draw (3.center) to (9.center);
		\draw (10.center) to (8);
		\draw [in=60, out=-90] (11.center) to (5);
	\end{pgfonlayer}
\end{tikzpicture}
=
\begin{tikzpicture}
	\begin{pgfonlayer}{nodelayer}
		\node [style=none] (0) at (-0.75, 0.5) {};
		\node [style=none] (1) at (-0.25, 0.5) {};
		\node [style=X] (2) at (-0.25, 1.5) {};
		\node [style=fanout] (3) at (-0.75, 1.75) {};
		\node [style=none] (4) at (-1, 2.5) {};
		\node [style=oplus] (5) at (-0.75, 1) {};
		\node [style=dot] (6) at (-0.25, 1) {};
		\node [style=none] (7) at (-0.5, 2.5) {};
	\end{pgfonlayer}
	\begin{pgfonlayer}{edgelayer}
		\draw (0.center) to (3);
		\draw [in=120, out=-90] (4.center) to (3);
		\draw (6) to (5);
		\draw (6) to (1.center);
		\draw [in=60, out=-90] (7.center) to (3);
		\draw (2) to (6);
	\end{pgfonlayer}
\end{tikzpicture}
=
\left\llbracket
\begin{tikzpicture}
	\begin{pgfonlayer}{nodelayer}
		\node [style=Z] (0) at (-1, 1) {};
		\node [style=none] (1) at (-1.25, 0.5) {};
		\node [style=none] (2) at (-0.75, 0.5) {};
		\node [style=X] (3) at (-1, 1.75) {};
		\node [style=none] (4) at (-1.25, 2.25) {};
		\node [style=none] (5) at (-0.75, 2.25) {};
	\end{pgfonlayer}
	\begin{pgfonlayer}{edgelayer}
		\draw [in=63, out=-90] (5.center) to (3);
		\draw (3) to (0);
		\draw [in=90, out=-117] (0) to (1.center);
		\draw [in=-63, out=90] (2.center) to (0);
		\draw [in=-90, out=117] (3) to (4.center);
	\end{pgfonlayer}
\end{tikzpicture}
\right\rrbracket_{\ZXA}
\end{align*}



\item[\ref{ZXA.6}:]

$$
\left\llbracket
\begin{tikzpicture}
	\begin{pgfonlayer}{nodelayer}
		\node [style=none] (0) at (-0.25, 2) {};
		\node [style=X] (1) at (0, 1.25) {};
		\node [style=Z] (2) at (0, 0.5) {};
		\node [style=none] (3) at (0.25, 2) {};
	\end{pgfonlayer}
	\begin{pgfonlayer}{edgelayer}
		\draw [style=simple, in=-90, out=124] (1) to (0.center);
		\draw [style=simple, in=60, out=-90] (3.center) to (1);
		\draw [style=simple] (1) to (2);
	\end{pgfonlayer}
\end{tikzpicture}
\right\rrbracket_{\ZXA}
=
\begin{tikzpicture}
	\begin{pgfonlayer}{nodelayer}
		\node [style=dot] (0) at (0, 1) {};
		\node [style=zeroin] (1) at (0, 0.5) {};
		\node [style=zeroin] (2) at (0.75, 0.5) {};
		\node [style=none] (3) at (0.75, 1.5) {};
		\node [style=none] (4) at (0, 1.5) {};
		\node [style=oplus] (5) at (0.75, 1) {};
	\end{pgfonlayer}
	\begin{pgfonlayer}{edgelayer}
		\draw [style=simple] (0) to (4.center);
		\draw [style=simple] (0) to (1);
		\draw [style=simple] (2) to (5);
		\draw [style=simple] (5) to (3.center);
		\draw [style=simple] (5) to (0);
	\end{pgfonlayer}
\end{tikzpicture}
\eq{\ref{TOF.2}}
\begin{tikzpicture}
	\begin{pgfonlayer}{nodelayer}
		\node [style=zeroin] (0) at (0, 0.5) {};
		\node [style=zeroin] (1) at (0.75, 0.5) {};
		\node [style=none] (2) at (0.75, 1.5) {};
		\node [style=none] (3) at (0, 1.5) {};
	\end{pgfonlayer}
	\begin{pgfonlayer}{edgelayer}
		\draw [style=simple] (2.center) to (1);
		\draw [style=simple] (0) to (3.center);
	\end{pgfonlayer}
\end{tikzpicture}
=
\left\llbracket
\begin{tikzpicture}
	\begin{pgfonlayer}{nodelayer}
		\node [style=none] (0) at (-0.25, 1) {};
		\node [style=Z] (1) at (-0.25, 0.5) {};
		\node [style=none] (2) at (0.25, 1) {};
		\node [style=Z] (3) at (0.25, 0.5) {};
	\end{pgfonlayer}
	\begin{pgfonlayer}{edgelayer}
		\draw [style=simple] (3) to (2.center);
		\draw [style=simple] (1) to (0.center);
	\end{pgfonlayer}
\end{tikzpicture}
\right\rrbracket_{\ZXA}
$$



\item[\ref{ZXA.7}:]
This is immediate.

%
%
%\item[\ref{ZXA.13old}:]
%\begin{align*}
%\left\llbracket
%\begin{tikzpicture}
%	\begin{pgfonlayer}{nodelayer}
%		\node [style=none] (0) at (4.25, 0.5) {};
%		\node [style=none] (1) at (3.5, -0) {};
%		\node [style=Z] (2) at (3.5, 1) {};
%		\node [style=none] (3) at (5, 0.5) {};
%		\node [style=none] (4) at (3, -0) {};
%		\node [style=andin] (5) at (4.25, 0.5) {};
%	\end{pgfonlayer}
%	\begin{pgfonlayer}{edgelayer}
%		\draw [in=0, out=135, looseness=1.00] (0.center) to (2);
%		\draw [in=-131, out=0, looseness=1.00] (1.center) to (0.center);
%		\draw (3.center) to (0.center);
%		\draw (1.center) to (4.center);
%	\end{pgfonlayer}
%\end{tikzpicture}
%\right\rrbracket_{\ZXA}
%&=
%\begin{tikzpicture}
%	\begin{pgfonlayer}{nodelayer}
%		\node [style=dot] (0) at (4, -1) {};
%		\node [style=dot] (1) at (4, -1.5) {};
%		\node [style=oplus] (2) at (4, -2) {};
%		\node [style=X] (3) at (4.5, -1.5) {};
%		\node [style=X] (4) at (4.5, -1) {};
%		\node [style=zeroin] (5) at (3.5, -2) {};
%		\node [style=none] (6) at (4.75, -2) {};
%		\node [style=none] (7) at (3.25, -1.5) {};
%		\node [style=zeroin] (8) at (3.5, -1) {};
%	\end{pgfonlayer}
%	\begin{pgfonlayer}{edgelayer}
%		\draw (4) to (0);
%		\draw (0) to (8);
%		\draw (7.center) to (3);
%		\draw (0) to (2);
%		\draw (5) to (6.center);
%	\end{pgfonlayer}
%\end{tikzpicture}
%\eq{\ref{TOF.2}}
%\begin{tikzpicture}
%	\begin{pgfonlayer}{nodelayer}
%		\node [style=X] (0) at (4.5, -1.5) {};
%		\node [style=X] (1) at (4.5, -1) {};
%		\node [style=zeroin] (2) at (3.5, -2) {};
%		\node [style=none] (3) at (4.75, -2) {};
%		\node [style=none] (4) at (3.25, -1.5) {};
%		\node [style=zeroin] (5) at (3.5, -1) {};
%	\end{pgfonlayer}
%	\begin{pgfonlayer}{edgelayer}
%		\draw (4.center) to (0);
%		\draw (2) to (3.center);
%		\draw (1) to (5);
%	\end{pgfonlayer}
%\end{tikzpicture}
%\eq{Rem. \ref{cor:copy}}
%\begin{tikzpicture}
%	\begin{pgfonlayer}{nodelayer}
%		\node [style=X] (0) at (4.5, -1.5) {};
%		\node [style=zeroin] (1) at (5.25, -1.5) {};
%		\node [style=none] (2) at (6, -1.5) {};
%		\node [style=none] (3) at (3.75, -1.5) {};
%	\end{pgfonlayer}
%	\begin{pgfonlayer}{edgelayer}
%		\draw (3.center) to (0);
%		\draw (1) to (2.center);
%	\end{pgfonlayer}
%\end{tikzpicture}
%=
%\left\llbracket
%\begin{tikzpicture}
%	\begin{pgfonlayer}{nodelayer}
%		\node [style=Z] (0) at (4.75, -0) {};
%		\node [style=none] (1) at (5.5, -0) {};
%		\node [style=none] (2) at (3.25, -0) {};
%		\node [style=X] (3) at (4, -0) {};
%	\end{pgfonlayer}
%	\begin{pgfonlayer}{edgelayer}
%		\draw (1.center) to (0);
%		\draw (3) to (2.center);
%	\end{pgfonlayer}
%\end{tikzpicture}
%\right\rrbracket_{\ZXA}
%\end{align*}



\item[\ref{ZXA.8}:]
\begin{align*}
\left\llbracket
\begin{tikzpicture}
	\begin{pgfonlayer}{nodelayer}
		\node [style=Z] (0) at (-1, 3) {};
		\node [style=X] (1) at (-1, 2.25) {};
		\node [style=none] (2) at (-1, 3.5) {};
		\node [style=none] (3) at (-1, 1.75) {};
	\end{pgfonlayer}
	\begin{pgfonlayer}{edgelayer}
		\draw (2.center) to (0);
		\draw [in=120, out=-120, looseness=1.25] (0) to (1);
		\draw [in=-60, out=60, looseness=1.25] (1) to (0);
		\draw (1) to (3.center);
	\end{pgfonlayer}
\end{tikzpicture}
\right\rrbracket_{\ZXA}
=
\begin{tikzpicture}
	\begin{pgfonlayer}{nodelayer}
		\node [style=none] (0) at (-1, 3) {};
		\node [style=dot] (1) at (-1, 1.25) {};
		\node [style=dot] (2) at (-0.5, 2.25) {};
		\node [style=oplus] (3) at (-1, 2.25) {};
		\node [style=oplus] (4) at (-0.5, 1.25) {};
		\node [style=X] (5) at (-0.5, 2.75) {};
		\node [style=zeroin] (6) at (-0.5, 0.75) {};
		\node [style=none] (7) at (-1, 0.5) {};
	\end{pgfonlayer}
	\begin{pgfonlayer}{edgelayer}
		\draw (5) to (2);
		\draw (2) to (3);
		\draw (3) to (0.center);
		\draw (3) to (1);
		\draw (1) to (4);
		\draw (4) to (6);
		\draw (4) to (2);
		\draw (1) to (7.center);
	\end{pgfonlayer}
\end{tikzpicture}
\eq{Lem. \ref{lemma:whiteunit}}
\begin{tikzpicture}
	\begin{pgfonlayer}{nodelayer}
		\node [style=dot] (0) at (-1, 1.25) {};
		\node [style=dot] (1) at (-0.5, 1.75) {};
		\node [style=oplus] (2) at (-1, 1.75) {};
		\node [style=oplus] (3) at (-0.5, 1.25) {};
		\node [style=zeroin] (4) at (-0.5, 0.75) {};
		\node [style=none] (5) at (-1, 0.5) {};
		\node [style=none] (6) at (-1, 3) {};
		\node [style=X] (7) at (-0.5, 2.75) {};
		\node [style=dot] (8) at (-1, 2.25) {};
		\node [style=oplus] (9) at (-0.5, 2.25) {};
	\end{pgfonlayer}
	\begin{pgfonlayer}{edgelayer}
		\draw (1) to (2);
		\draw (2) to (0);
		\draw (0) to (3);
		\draw (3) to (4);
		\draw (3) to (1);
		\draw (0) to (5.center);
		\draw (8) to (9);
		\draw (7) to (9);
		\draw (9) to (1);
		\draw (2) to (8);
		\draw (8) to (6.center);
	\end{pgfonlayer}
\end{tikzpicture}
\eq{\ref{TOF.14}}
\begin{tikzpicture}
	\begin{pgfonlayer}{nodelayer}
		\node [style=zeroin] (0) at (-0.5, 0.75) {};
		\node [style=none] (1) at (-1, 0.5) {};
		\node [style=none] (2) at (-1, 2.5) {};
		\node [style=X] (3) at (-0.5, 2.25) {};
		\node [style=none] (4) at (-1, 0.75) {};
		\node [style=none] (5) at (-1, 2.25) {};
	\end{pgfonlayer}
	\begin{pgfonlayer}{edgelayer}
		\draw [in=90, out=-90] (3) to (4.center);
		\draw [in=-90, out=90] (0) to (5.center);
		\draw (5.center) to (2.center);
		\draw (4.center) to (1.center);
	\end{pgfonlayer}
\end{tikzpicture}
=
\begin{tikzpicture}
	\begin{pgfonlayer}{nodelayer}
		\node [style=zeroin] (0) at (-1, 2) {};
		\node [style=X] (1) at (-1, 1.25) {};
		\node [style=none] (2) at (-1, 0.5) {};
		\node [style=none] (3) at (-1, 2.75) {};
	\end{pgfonlayer}
	\begin{pgfonlayer}{edgelayer}
		\draw (1) to (2.center);
		\draw (0) to (3.center);
	\end{pgfonlayer}
\end{tikzpicture}
=
\left\llbracket
\begin{tikzpicture}
	\begin{pgfonlayer}{nodelayer}
		\node [style=Z] (0) at (-1, 3) {};
		\node [style=X] (1) at (-1, 2.25) {};
		\node [style=none] (2) at (-1, 3.5) {};
		\node [style=none] (3) at (-1, 1.75) {};
	\end{pgfonlayer}
	\begin{pgfonlayer}{edgelayer}
		\draw (2.center) to (0);
		\draw (1) to (3.center);
	\end{pgfonlayer}
\end{tikzpicture}
\right\rrbracket_{\ZXA}
\end{align*}


\item[\ref{ZXA.9}:]
\begin{align*}
\left\llbracket
\begin{tikzpicture}
	\begin{pgfonlayer}{nodelayer}
		\node [style=andin] (1) at (0, 3) {};
		\node [style=andin] (2) at (0.5, 4) {};
		\node [style=none] (3) at (0.5, 4.75) {};
		\node [style=none] (4) at (0.75, 3) {};
		\node [style=none] (5) at (-0.5, 2) {};
		\node [style=none] (6) at (0.25, 2) {};
		\node [style=none] (7) at (0.75, 2) {};
	\end{pgfonlayer}
	\begin{pgfonlayer}{edgelayer}
		\draw [style=simple] (3.center) to (2.center);
		\draw [style=simple, in=90, out=-117] (2.center) to (1.center);
		\draw [style=simple, in=90, out=-117] (1.center) to (5.center);
		\draw [style=simple, in=90, out=-76] (1.center) to (6.center);
		\draw [style=simple] (7.center) to (4.center);
		\draw [style=simple, in=-63, out=90] (4.center) to (2.center);
	\end{pgfonlayer}
\end{tikzpicture}
\right\rrbracket_{\ZXA}
&=
\begin{tikzpicture}
	\begin{pgfonlayer}{nodelayer}
		\node [style=dot] (2) at (0, 1) {};
		\node [style=dot] (3) at (0.5, 1) {};
		\node [style=oplus] (4) at (1, 1) {};
		\node [style=X] (5) at (0, 1.75) {};
		\node [style=X] (6) at (0.5, 1.75) {};
		\node [style=none] (7) at (1, 2) {};
		\node [style=zeroin] (8) at (1, 0.25) {};
		\node [style=none] (9) at (0.5, -0.5) {};
		\node [style=X] (10) at (-0.5, 1.25) {};
		\node [style=none] (11) at (-1, -0.5) {};
		\node [style=dot] (12) at (-0.5, 0.5) {};
		\node [style=dot] (13) at (-1, 0.5) {};
		\node [style=oplus] (14) at (0, 0.5) {};
		\node [style=zeroin] (15) at (0, -0.25) {};
		\node [style=X] (16) at (-1, 1.25) {};
		\node [style=none] (17) at (-0.5, -0.5) {};
	\end{pgfonlayer}
	\begin{pgfonlayer}{edgelayer}
		\draw [style=simple] (7.center) to (4);
		\draw [style=simple] (4) to (8);
		\draw [style=simple] (4) to (3);
		\draw [style=simple] (3) to (2);
		\draw [style=simple] (5) to (2);
		\draw [style=simple] (9.center) to (3);
		\draw [style=simple] (3) to (6);
		\draw [style=simple] (14) to (15);
		\draw [style=simple] (14) to (12);
		\draw [style=simple] (12) to (13);
		\draw [style=simple] (16) to (13);
		\draw [style=simple] (13) to (11.center);
		\draw [style=simple] (17.center) to (12);
		\draw [style=simple] (12) to (10);
		\draw [style=simple] (14) to (2);
	\end{pgfonlayer}
\end{tikzpicture}
\eq{Lem \ref{lemma:Iwama}}
\begin{tikzpicture}
	\begin{pgfonlayer}{nodelayer}
		\node [style=dot] (3) at (0, 1.5) {};
		\node [style=dot] (4) at (0.5, 1.5) {};
		\node [style=oplus] (5) at (1, 1.5) {};
		\node [style=none] (6) at (1, 3) {};
		\node [style=zeroin] (7) at (1, 0.25) {};
		\node [style=none] (8) at (0.5, 0) {};
		\node [style=none] (9) at (-1, 0) {};
		\node [style=zeroin] (10) at (0, 0.25) {};
		\node [style=none] (11) at (-0.5, 0) {};
		\node [style=X] (12) at (-0.5, 2.75) {};
		\node [style=X] (13) at (-1, 2.75) {};
		\node [style=X] (14) at (0, 2.75) {};
		\node [style=X] (15) at (0.5, 2.75) {};
		\node [style=dot] (16) at (-1, 2) {};
		\node [style=dot] (17) at (-0.5, 2) {};
		\node [style=oplus] (18) at (0, 2) {};
		\node [style=dot] (19) at (-1, 0.75) {};
		\node [style=dot] (20) at (-0.5, 0.75) {};
		\node [style=oplus] (21) at (1, 0.75) {};
		\node [style=dot] (22) at (0.5, 0.75) {};
	\end{pgfonlayer}
	\begin{pgfonlayer}{edgelayer}
		\draw [style=simple] (6.center) to (5);
		\draw [style=simple] (5) to (4);
		\draw [style=simple] (4) to (3);
		\draw [style=simple] (18) to (17);
		\draw [style=simple] (17) to (16);
		\draw [style=simple] (21) to (20);
		\draw [style=simple] (20) to (19);
		\draw [style=simple] (5) to (21);
		\draw [style=simple] (8.center) to (22);
		\draw [style=simple] (21) to (7);
		\draw [style=simple] (4) to (22);
		\draw [style=simple] (15) to (4);
		\draw [style=simple] (14) to (18);
		\draw [style=simple] (18) to (3);
		\draw [style=simple] (10) to (3);
		\draw [style=simple] (12) to (17);
		\draw [style=simple] (17) to (20);
		\draw [style=simple] (20) to (11.center);
		\draw [style=simple] (9.center) to (19);
		\draw [style=simple] (19) to (16);
		\draw [style=simple] (13) to (16);
	\end{pgfonlayer}
\end{tikzpicture}
\eq{Rem. \ref{cor:copy}}
\begin{tikzpicture}
	\begin{pgfonlayer}{nodelayer}
		\node [style=dot] (4) at (0, 1.5) {};
		\node [style=dot] (5) at (0.5, 1.5) {};
		\node [style=oplus] (6) at (1, 1.5) {};
		\node [style=none] (7) at (1, 2.5) {};
		\node [style=zeroin] (8) at (1, 0.25) {};
		\node [style=none] (9) at (0.5, 0) {};
		\node [style=none] (10) at (-1, 0) {};
		\node [style=zeroin] (11) at (0, 0.25) {};
		\node [style=none] (12) at (-0.5, 0) {};
		\node [style=X] (13) at (-0.5, 2.25) {};
		\node [style=X] (14) at (-1, 2.25) {};
		\node [style=X] (15) at (0, 2.25) {};
		\node [style=X] (16) at (0.5, 2.25) {};
		\node [style=dot] (17) at (-1, 0.75) {};
		\node [style=dot] (18) at (-0.5, 0.75) {};
		\node [style=oplus] (19) at (1, 0.75) {};
		\node [style=dot] (20) at (0.5, 0.75) {};
	\end{pgfonlayer}
	\begin{pgfonlayer}{edgelayer}
		\draw [style=simple] (7.center) to (6);
		\draw [style=simple] (6) to (5);
		\draw [style=simple] (5) to (4);
		\draw [style=simple] (19) to (18);
		\draw [style=simple] (18) to (17);
		\draw [style=simple] (6) to (19);
		\draw [style=simple] (9.center) to (20);
		\draw [style=simple] (19) to (8);
		\draw [style=simple] (5) to (20);
		\draw [style=simple] (16) to (5);
		\draw [style=simple] (11) to (4);
		\draw [style=simple] (18) to (12.center);
		\draw [style=simple] (10.center) to (17);
		\draw [style=simple] (15) to (4);
		\draw [style=simple] (18) to (13);
		\draw [style=simple] (14) to (17);
	\end{pgfonlayer}
\end{tikzpicture}
\eq{\ref{TOF.2}}
\begin{tikzpicture}
	\begin{pgfonlayer}{nodelayer}
		\node [style=none] (5) at (1, 2.25) {};
		\node [style=zeroin] (6) at (1, 0.25) {};
		\node [style=none] (7) at (0.5, 0) {};
		\node [style=none] (8) at (-1, 0) {};
		\node [style=zeroin] (9) at (0, 1.25) {};
		\node [style=none] (10) at (-0.5, 0) {};
		\node [style=X] (11) at (-0.5, 2.25) {};
		\node [style=X] (12) at (-1, 2.25) {};
		\node [style=X] (13) at (0, 2.25) {};
		\node [style=X] (14) at (0.5, 2.25) {};
		\node [style=dot] (15) at (-1, 0.75) {};
		\node [style=dot] (16) at (-0.5, 0.75) {};
		\node [style=oplus] (17) at (1, 0.75) {};
		\node [style=dot] (18) at (0.5, 0.75) {};
	\end{pgfonlayer}
	\begin{pgfonlayer}{edgelayer}
		\draw [style=simple] (17) to (16);
		\draw [style=simple] (16) to (15);
		\draw [style=simple] (7.center) to (18);
		\draw [style=simple] (17) to (6);
		\draw [style=simple] (16) to (10.center);
		\draw [style=simple] (8.center) to (15);
		\draw [style=simple] (16) to (11);
		\draw [style=simple] (12) to (15);
		\draw [style=simple] (5.center) to (17);
		\draw [style=simple] (9) to (13);
		\draw [style=simple] (14) to (18);
	\end{pgfonlayer}
\end{tikzpicture}\\
&\eq{\ref{TOF.2}}
\begin{tikzpicture}
	\begin{pgfonlayer}{nodelayer}
		\node [style=X] (6) at (0.5, 4) {};
		\node [style=zeroin] (7) at (1, 1.25) {};
		\node [style=X] (8) at (0, 4) {};
		\node [style=none] (9) at (-0.5, 1) {};
		\node [style=dot] (10) at (0.5, 3.25) {};
		\node [style=none] (11) at (1, 4.25) {};
		\node [style=X] (12) at (-1, 4) {};
		\node [style=X] (13) at (-0.5, 4) {};
		\node [style=dot] (14) at (-1, 3.25) {};
		\node [style=none] (15) at (0, 1) {};
		\node [style=oplus] (16) at (1, 3.25) {};
		\node [style=zeroin] (17) at (0.5, 1.25) {};
		\node [style=none] (18) at (-1, 1) {};
		\node [style=dot] (19) at (-1, 2.5) {};
		\node [style=dot] (20) at (-0.5, 2.5) {};
		\node [style=dot] (21) at (0, 2.5) {};
		\node [style=oplus] (22) at (1, 2.5) {};
	\end{pgfonlayer}
	\begin{pgfonlayer}{edgelayer}
		\draw [style=simple] (11.center) to (16);
		\draw [style=simple] (16) to (7);
		\draw [style=simple] (16) to (10);
		\draw [style=simple] (10) to (14);
		\draw [style=simple] (12) to (14);
		\draw [style=simple] (10) to (6);
		\draw (14) to (18.center);
		\draw (22) to (21);
		\draw (21) to (20);
		\draw (19) to (20);
		\draw (21) to (15.center);
		\draw (9.center) to (20);
		\draw (10) to (17);
		\draw (6) to (10);
		\draw (8) to (21);
		\draw (20) to (13);
	\end{pgfonlayer}
\end{tikzpicture}
\eq{Lem \ref{lemma:Iwama}}
\begin{tikzpicture}
	\begin{pgfonlayer}{nodelayer}
		\node [style=X] (7) at (0.5, 5.5) {};
		\node [style=zeroin] (8) at (1, 1.25) {};
		\node [style=X] (9) at (0, 5.5) {};
		\node [style=none] (10) at (-0.5, 1) {};
		\node [style=dot] (11) at (0.5, 3.25) {};
		\node [style=none] (12) at (1, 4.25) {};
		\node [style=X] (13) at (-1, 5.5) {};
		\node [style=X] (14) at (-0.5, 5.5) {};
		\node [style=dot] (15) at (-1, 3.25) {};
		\node [style=none] (16) at (0, 1) {};
		\node [style=oplus] (17) at (1, 3.25) {};
		\node [style=zeroin] (18) at (0.5, 1.25) {};
		\node [style=none] (19) at (-1, 1) {};
		\node [style=dot] (20) at (0, 4.75) {};
		\node [style=dot] (21) at (-0.5, 4.75) {};
		\node [style=dot] (22) at (-1, 2.5) {};
		\node [style=dot] (23) at (-0.5, 2.5) {};
		\node [style=dot] (24) at (0, 2.5) {};
		\node [style=oplus] (25) at (0.5, 4.75) {};
		\node [style=oplus] (26) at (1, 2.5) {};
	\end{pgfonlayer}
	\begin{pgfonlayer}{edgelayer}
		\draw [style=simple] (12.center) to (17);
		\draw [style=simple] (17) to (8);
		\draw [style=simple] (17) to (11);
		\draw [style=simple] (11) to (15);
		\draw [style=simple] (13) to (15);
		\draw [style=simple] (11) to (7);
		\draw (15) to (19.center);
		\draw [style=simple] (20) to (21);
		\draw (26) to (24);
		\draw (24) to (23);
		\draw (22) to (23);
		\draw (23) to (21);
		\draw (21) to (14);
		\draw (9) to (20);
		\draw (20) to (24);
		\draw (24) to (16.center);
		\draw (10.center) to (23);
		\draw (25) to (20);
		\draw (11) to (18);
	\end{pgfonlayer}
\end{tikzpicture}
\eq{Rem. \ref{cor:copy}}
\begin{tikzpicture}
	\begin{pgfonlayer}{nodelayer}
		\node [style=X] (8) at (0.5, 5.5) {};
		\node [style=zeroin] (9) at (1, 4) {};
		\node [style=dot] (10) at (-0.5, 3.5) {};
		\node [style=X] (11) at (0, 4.25) {};
		\node [style=dot] (12) at (0, 3.5) {};
		\node [style=none] (13) at (-0.5, 2.5) {};
		\node [style=dot] (14) at (0.5, 4.75) {};
		\node [style=none] (15) at (1, 5.75) {};
		\node [style=X] (16) at (-1, 5.5) {};
		\node [style=X] (17) at (-0.5, 4.25) {};
		\node [style=dot] (18) at (-1, 4.75) {};
		\node [style=none] (19) at (0, 2.5) {};
		\node [style=oplus] (20) at (1, 4.75) {};
		\node [style=oplus] (21) at (0.5, 3.5) {};
		\node [style=zeroin] (22) at (0.5, 2.75) {};
		\node [style=none] (23) at (-1, 2.5) {};
	\end{pgfonlayer}
	\begin{pgfonlayer}{edgelayer}
		\draw [style=simple] (15.center) to (20);
		\draw [style=simple] (20) to (9);
		\draw [style=simple] (20) to (14);
		\draw [style=simple] (14) to (18);
		\draw [style=simple] (16) to (18);
		\draw [style=simple] (14) to (8);
		\draw [style=simple] (12) to (10);
		\draw [style=simple] (17) to (10);
		\draw [style=simple] (10) to (13.center);
		\draw [style=simple] (19.center) to (12);
		\draw [style=simple] (12) to (11);
		\draw [style=simple] (21) to (22);
		\draw (18) to (23.center);
		\draw (21) to (14);
		\draw (12) to (21);
	\end{pgfonlayer}
\end{tikzpicture}
=
\left\llbracket
\begin{tikzpicture}
	\begin{pgfonlayer}{nodelayer}
		\node [style=andin] (9) at (0.25, 3) {};
		\node [style=andin] (10) at (-0.25, 4) {};
		\node [style=none] (11) at (-0.25, 4.75) {};
		\node [style=none] (12) at (-0.5, 3) {};
		\node [style=none] (13) at (0.75, 2) {};
		\node [style=none] (14) at (0, 2) {};
		\node [style=none] (15) at (-0.5, 2) {};
	\end{pgfonlayer}
	\begin{pgfonlayer}{edgelayer}
		\draw [style=simple] (11.center) to (10.center);
		\draw [style=simple, in=90, out=-63] (10.center) to (9.center);
		\draw [style=simple, in=90, out=-63] (9.center) to (13.center);
		\draw [style=simple, in=90, out=-104] (9.center) to (14.center);
		\draw [style=simple] (15.center) to (12.center);
		\draw [style=simple, in=-117, out=90] (12.center) to (10.center);
	\end{pgfonlayer}
\end{tikzpicture}
\right\rrbracket_{\ZXA}
\end{align*}

\item[\ref{ZXA.10}:]
\begin{align*}
\left\llbracket
\begin{tikzpicture}
	\begin{pgfonlayer}{nodelayer}
		\node [style=andin] (100) at (-1, 5.5) {};
		\node [style=none] (10) at (-1, 5.5) {};
		\node [style=none] (11) at (-1, 6) {};
		\node [style=none] (12) at (-0.75, 4.75) {};
		\node [style=Z] (13) at (-1.25, 4.75) {$\pi$};
	\end{pgfonlayer}
	\begin{pgfonlayer}{edgelayer}
		\draw (10) to (11.center);
		\draw [in=90, out=-108] (10) to (13);
		\draw [in=-72, out=90] (12.center) to (10);
	\end{pgfonlayer}
\end{tikzpicture}
\right\rrbracket_{\ZXA}
=
\begin{tikzpicture}
	\begin{pgfonlayer}{nodelayer}
		\node [style=X] (11) at (0, 7) {};
		\node [style=X] (12) at (-0.5, 7) {};
		\node [style=none] (13) at (0, 4.75) {};
		\node [style=zeroin] (14) at (0.5, 5.5) {};
		\node [style=oplus] (15) at (0.5, 6.25) {};
		\node [style=dot] (16) at (0, 6.25) {};
		\node [style=dot] (17) at (-0.5, 6.25) {};
		\node [style=none] (18) at (0.5, 7.25) {};
		\node [style=zeroin] (19) at (-0.5, 5) {};
		\node [style=oplus] (20) at (-0.5, 5.75) {};
	\end{pgfonlayer}
	\begin{pgfonlayer}{edgelayer}
		\draw [style=simple] (16) to (17);
		\draw [style=simple] (16) to (11);
		\draw [style=simple] (15) to (14);
		\draw [style=simple] (13.center) to (16);
		\draw [style=simple] (12) to (17);
		\draw (16) to (15);
		\draw (20) to (19);
		\draw (20) to (17);
		\draw (18.center) to (15);
	\end{pgfonlayer}
\end{tikzpicture}
=
\begin{tikzpicture}
	\begin{pgfonlayer}{nodelayer}
		\node [style=X] (12) at (0, 7) {};
		\node [style=X] (13) at (-0.5, 7) {};
		\node [style=none] (14) at (0, 4.75) {};
		\node [style=zeroin] (15) at (0.5, 5.5) {};
		\node [style=oplus] (16) at (0.5, 6.25) {};
		\node [style=dot] (17) at (0, 6.25) {};
		\node [style=dot] (18) at (-0.5, 6.25) {};
		\node [style=none] (19) at (0.5, 7.25) {};
		\node [style=onein] (20) at (-0.5, 5.5) {};
	\end{pgfonlayer}
	\begin{pgfonlayer}{edgelayer}
		\draw [style=simple] (17) to (18);
		\draw [style=simple] (17) to (12);
		\draw [style=simple] (16) to (15);
		\draw [style=simple] (14.center) to (17);
		\draw [style=simple] (13) to (18);
		\draw (17) to (16);
		\draw (19.center) to (16);
		\draw (18) to (20);
	\end{pgfonlayer}
\end{tikzpicture}
\eq{\ref{TOF.1}}
\begin{tikzpicture}
	\begin{pgfonlayer}{nodelayer}
		\node [style=X] (13) at (0, 7) {};
		\node [style=X] (14) at (-0.5, 7) {};
		\node [style=none] (15) at (0, 4.75) {};
		\node [style=zeroin] (16) at (0.5, 5.5) {};
		\node [style=none] (17) at (0.5, 7.25) {};
		\node [style=onein] (18) at (-0.5, 5.5) {};
		\node [style=oplus] (19) at (0.5, 6.25) {};
		\node [style=dot] (20) at (0, 6.25) {};
	\end{pgfonlayer}
	\begin{pgfonlayer}{edgelayer}
		\draw [style=simple] (19) to (16);
		\draw (17.center) to (19);
		\draw [style=simple] (15.center) to (20);
		\draw (20) to (19);
		\draw [style=simple] (20) to (13);
		\draw (14) to (18);
	\end{pgfonlayer}
\end{tikzpicture}
\eq{Rem. \ref{cor:copy}}
\begin{tikzpicture}
	\begin{pgfonlayer}{nodelayer}
		\node [style=X] (14) at (0, 8) {};
		\node [style=none] (15) at (0, 5.75) {};
		\node [style=zeroin] (16) at (0.5, 6.5) {};
		\node [style=none] (17) at (0.5, 8.25) {};
		\node [style=oplus] (18) at (0.5, 7.25) {};
		\node [style=dot] (19) at (0, 7.25) {};
	\end{pgfonlayer}
	\begin{pgfonlayer}{edgelayer}
		\draw [style=simple] (18) to (16);
		\draw (17.center) to (18);
		\draw [style=simple] (15.center) to (19);
		\draw (19) to (18);
		\draw [style=simple] (19) to (14);
	\end{pgfonlayer}
\end{tikzpicture}
\eq{Lem. \ref{lemma:whiteunit}}
\begin{tikzpicture}
	\begin{pgfonlayer}{nodelayer}
		\node [style=none] (15) at (0, 5.75) {};
		\node [style=none] (16) at (0, 6.75) {};
	\end{pgfonlayer}
	\begin{pgfonlayer}{edgelayer}
		\draw (16.center) to (15.center);
	\end{pgfonlayer}
\end{tikzpicture}=
\left\llbracket
\begin{tikzpicture}
	\begin{pgfonlayer}{nodelayer}
		\node [style=none] (16) at (0, 5.75) {};
		\node [style=none] (17) at (0, 6.75) {};
	\end{pgfonlayer}
	\begin{pgfonlayer}{edgelayer}
		\draw (17.center) to (16.center);
	\end{pgfonlayer}
\end{tikzpicture}
\right\rrbracket_{\ZXA}
\end{align*}

\item[\ref{ZXA.11}:]

\begin{align*}
\left\llbracket
\begin{tikzpicture}
	\begin{pgfonlayer}{nodelayer}
		\node [style=andin] (170) at (0, 6.75) {};
		\node [style=none] (17) at (0, 6.75) {};
		\node [style=none] (18) at (-0.25, 6.25) {};
		\node [style=none] (19) at (0.25, 6.25) {};
		\node [style=none] (20) at (0, 7.25) {};
		\node [style=none] (21) at (-0.25, 5.75) {};
		\node [style=none] (22) at (0.25, 5.75) {};
	\end{pgfonlayer}
	\begin{pgfonlayer}{edgelayer}
		\draw [in=-63, out=90] (19.center) to (17);
		\draw [in=90, out=-117, looseness=1.25] (17) to (18.center);
		\draw (20.center) to (17);
		\draw [in=-90, out=90, looseness=1.25] (22.center) to (18.center);
		\draw [in=90, out=-90, looseness=1.25] (19.center) to (21.center);
	\end{pgfonlayer}
\end{tikzpicture}
\right\rrbracket_{\ZXA}
=
\begin{tikzpicture}
	\begin{pgfonlayer}{nodelayer}
		\node [style=none] (18) at (0.5, 5.75) {};
		\node [style=dot] (19) at (0, 6.75) {};
		\node [style=dot] (20) at (0.5, 6.75) {};
		\node [style=oplus] (21) at (1, 6.75) {};
		\node [style=X] (22) at (0, 7.5) {};
		\node [style=X] (23) at (0.5, 7.5) {};
		\node [style=none] (24) at (1, 7.75) {};
		\node [style=zeroin] (25) at (1, 6) {};
		\node [style=none] (26) at (0, 5.75) {};
	\end{pgfonlayer}
	\begin{pgfonlayer}{edgelayer}
		\draw (25) to (21);
		\draw (20) to (23);
		\draw (24.center) to (21);
		\draw (21) to (20);
		\draw (20) to (19);
		\draw (19) to (22);
		\draw [in=90, out=-90] (19) to (18.center);
		\draw [in=90, out=-90] (20) to (26.center);
	\end{pgfonlayer}
\end{tikzpicture}
\eq{\ref{TOF.15}}
\begin{tikzpicture}
	\begin{pgfonlayer}{nodelayer}
		\node [style=none] (19) at (0.5, 6.75) {};
		\node [style=dot] (20) at (0, 7.75) {};
		\node [style=dot] (21) at (0.5, 7.75) {};
		\node [style=oplus] (22) at (1, 7.75) {};
		\node [style=X] (23) at (0.5, 8.75) {};
		\node [style=X] (24) at (0, 8.75) {};
		\node [style=none] (25) at (1, 8.75) {};
		\node [style=zeroin] (26) at (1, 7) {};
		\node [style=none] (27) at (0, 6.75) {};
		\node [style=none] (28) at (0.5, 5.75) {};
		\node [style=none] (29) at (0, 5.75) {};
	\end{pgfonlayer}
	\begin{pgfonlayer}{edgelayer}
		\draw (26) to (22);
		\draw [in=-90, out=90] (21) to (24);
		\draw (25.center) to (22);
		\draw (22) to (21);
		\draw (21) to (20);
		\draw [in=-90, out=90] (20) to (23);
		\draw [in=90, out=-90] (20) to (19.center);
		\draw [in=90, out=-90] (21) to (27.center);
		\draw [in=90, out=-90] (19.center) to (29.center);
		\draw [in=-90, out=90] (28.center) to (27.center);
	\end{pgfonlayer}
\end{tikzpicture}
=
\begin{tikzpicture}
	\begin{pgfonlayer}{nodelayer}
		\node [style=none] (20) at (0, 5.75) {};
		\node [style=dot] (21) at (0, 6.75) {};
		\node [style=dot] (22) at (0.5, 6.75) {};
		\node [style=oplus] (23) at (1, 6.75) {};
		\node [style=X] (24) at (0, 7.5) {};
		\node [style=X] (25) at (0.5, 7.5) {};
		\node [style=none] (26) at (1, 7.75) {};
		\node [style=zeroin] (27) at (1, 6) {};
		\node [style=none] (28) at (0.5, 5.75) {};
	\end{pgfonlayer}
	\begin{pgfonlayer}{edgelayer}
		\draw (27) to (23);
		\draw (22) to (25);
		\draw (26.center) to (23);
		\draw (23) to (22);
		\draw (22) to (21);
		\draw (21) to (24);
		\draw (21) to (20.center);
		\draw (22) to (28.center);
	\end{pgfonlayer}
\end{tikzpicture}
=
\left\llbracket
\begin{tikzpicture}
	\begin{pgfonlayer}{nodelayer}
		\node [style=andin] (210) at (0, 6.25) {};
		\node [style=none] (21) at (0, 6.25) {};
		\node [style=none] (22) at (-0.25, 5.75) {};
		\node [style=none] (23) at (0.25, 5.75) {};
		\node [style=none] (24) at (0, 6.75) {};
	\end{pgfonlayer}
	\begin{pgfonlayer}{edgelayer}
		\draw [in=-63, out=90] (23.center) to (21);
		\draw [in=90, out=-117, looseness=1.25] (21) to (22.center);
		\draw (24.center) to (21);
	\end{pgfonlayer}
\end{tikzpicture}
\right\rrbracket_{\ZXA}
\end{align*}

\item[\ref{ZXA.12}:]

\begin{align*}
\left\llbracket
\begin{tikzpicture}
	\begin{pgfonlayer}{nodelayer}
		\node [style=X] (22) at (-1, 6.25) {};
		\node [style=X] (23) at (-0.25, 6.25) {};
		\node [style=andin] (240) at (-0.25, 7.25) {};
		\node [style=andin] (250) at (-1, 7.25) {};
		\node [style=none] (24) at (-0.25, 7.25) {};
		\node [style=none] (25) at (-1, 7.25) {};
		\node [style=none] (26) at (-1, 7.75) {};
		\node [style=none] (27) at (-0.25, 7.75) {};
		\node [style=none] (28) at (-1, 5.75) {};
		\node [style=none] (29) at (-0.25, 5.75) {};
	\end{pgfonlayer}
	\begin{pgfonlayer}{edgelayer}
		\draw (29.center) to (23);
		\draw [in=-60, out=127] (23) to (25);
		\draw [in=120, out=-120, looseness=1.25] (25) to (22);
		\draw [in=-120, out=53] (22) to (24);
		\draw (24) to (27.center);
		\draw [in=60, out=-60, looseness=1.25] (24) to (23);
		\draw (22) to (28.center);
		\draw (25) to (26.center);
	\end{pgfonlayer}
\end{tikzpicture}
\right\rrbracket_{\ZXA}
&=
\begin{tikzpicture}
	\begin{pgfonlayer}{nodelayer}
		\node [style=dot] (23) at (-0.5, 7.5) {};
		\node [style=dot] (24) at (-1, 7.5) {};
		\node [style=oplus] (25) at (-1.5, 7.5) {};
		\node [style=zeroin] (26) at (-1.5, 7) {};
		\node [style=X] (27) at (-0.5, 8) {};
		\node [style=X] (28) at (-1, 8) {};
		\node [style=none] (29) at (-1.5, 8.25) {};
		\node [style=dot] (30) at (0.5, 7.5) {};
		\node [style=oplus] (31) at (1, 7.5) {};
		\node [style=none] (32) at (1, 8.25) {};
		\node [style=dot] (33) at (0, 7.5) {};
		\node [style=X] (34) at (0.5, 8) {};
		\node [style=X] (35) at (0, 8) {};
		\node [style=zeroin] (36) at (1, 7) {};
		\node [style=fanout] (37) at (-0.75, 6.5) {};
		\node [style=fanout] (38) at (0.25, 6.5) {};
		\node [style=none] (39) at (0.25, 5.75) {};
		\node [style=none] (40) at (-0.75, 5.75) {};
	\end{pgfonlayer}
	\begin{pgfonlayer}{edgelayer}
		\draw (27) to (23);
		\draw (23) to (24);
		\draw (24) to (28);
		\draw (29.center) to (25);
		\draw (25) to (26);
		\draw (25) to (24);
		\draw (35) to (33);
		\draw (33) to (30);
		\draw (30) to (34);
		\draw (32.center) to (31);
		\draw (31) to (36);
		\draw (31) to (30);
		\draw [in=99, out=-90] (24) to (37);
		\draw [in=-90, out=63] (37) to (33);
		\draw [in=117, out=-90] (23) to (38);
		\draw [in=-90, out=81] (38) to (30);
		\draw (37) to (40.center);
		\draw (39.center) to (38);
	\end{pgfonlayer}
\end{tikzpicture}
\eq{\ref{TOF.4}}
\begin{tikzpicture}
	\begin{pgfonlayer}{nodelayer}
		\node [style=X] (26) at (-0.5, 8.5) {};
		\node [style=X] (27) at (-1, 8.5) {};
		\node [style=none] (28) at (-1.5, 8.75) {};
		\node [style=dot] (29) at (0.5, 8) {};
		\node [style=oplus] (30) at (1, 8) {};
		\node [style=none] (31) at (1, 8.75) {};
		\node [style=dot] (32) at (0, 8) {};
		\node [style=X] (33) at (0.5, 8.5) {};
		\node [style=X] (34) at (0, 8.5) {};
		\node [style=zeroin] (35) at (1, 7.5) {};
		\node [style=fanout] (36) at (-0.5, 7) {};
		\node [style=fanout] (37) at (0, 7) {};
		\node [style=none] (38) at (0, 5.75) {};
		\node [style=none] (39) at (-0.5, 5.75) {};
		\node [style=dot] (40) at (0, 6.5) {};
		\node [style=zeroin] (41) at (-1.5, 5.75) {};
		\node [style=oplus] (42) at (-1.5, 6.5) {};
		\node [style=dot] (43) at (-0.5, 6.5) {};
	\end{pgfonlayer}
	\begin{pgfonlayer}{edgelayer}
		\draw (34) to (32);
		\draw (32) to (29);
		\draw (29) to (33);
		\draw (31.center) to (30);
		\draw (30) to (35);
		\draw (30) to (29);
		\draw [in=-90, out=63] (36) to (32);
		\draw [in=-90, out=60] (37) to (29);
		\draw (36) to (39.center);
		\draw (38.center) to (37);
		\draw (40) to (43);
		\draw (42) to (41);
		\draw (42) to (43);
		\draw [in=-90, out=117] (37) to (26);
		\draw [in=120, out=-90] (27) to (36);
		\draw (42) to (28.center);
	\end{pgfonlayer}
\end{tikzpicture}
\eq{unit}
\begin{tikzpicture}
	\begin{pgfonlayer}{nodelayer}
		\node [style=none] (27) at (-1, 8.25) {};
		\node [style=dot] (28) at (0, 7.25) {};
		\node [style=oplus] (29) at (0.5, 7.25) {};
		\node [style=none] (30) at (0.5, 8.25) {};
		\node [style=dot] (31) at (-0.5, 7.25) {};
		\node [style=X] (32) at (0, 7.75) {};
		\node [style=X] (33) at (-0.5, 7.75) {};
		\node [style=zeroin] (34) at (0.5, 6.75) {};
		\node [style=none] (35) at (0, 5.75) {};
		\node [style=none] (36) at (-0.5, 5.75) {};
		\node [style=dot] (37) at (0, 6.75) {};
		\node [style=zeroin] (38) at (-1, 6) {};
		\node [style=oplus] (39) at (-1, 6.75) {};
		\node [style=dot] (40) at (-0.5, 6.75) {};
	\end{pgfonlayer}
	\begin{pgfonlayer}{edgelayer}
		\draw (33) to (31);
		\draw (31) to (28);
		\draw (28) to (32);
		\draw (30.center) to (29);
		\draw (29) to (34);
		\draw (29) to (28);
		\draw (37) to (40);
		\draw (39) to (38);
		\draw (39) to (40);
		\draw (39) to (27.center);
		\draw (31) to (40);
		\draw (37) to (28);
		\draw (37) to (35.center);
		\draw (36.center) to (40);
	\end{pgfonlayer}
\end{tikzpicture}
=
\begin{tikzpicture}
	\begin{pgfonlayer}{nodelayer}
		\node [style=none] (28) at (-1, 8.25) {};
		\node [style=dot] (29) at (-1.5, 7.25) {};
		\node [style=oplus] (30) at (-0.5, 7.25) {};
		\node [style=none] (31) at (-0.5, 8.25) {};
		\node [style=dot] (32) at (-2, 7.25) {};
		\node [style=X] (33) at (-1.5, 7.75) {};
		\node [style=X] (34) at (-2, 7.75) {};
		\node [style=zeroin] (35) at (-0.5, 6.75) {};
		\node [style=none] (36) at (-1.5, 5.75) {};
		\node [style=none] (37) at (-2, 5.75) {};
		\node [style=dot] (38) at (-1.5, 6.75) {};
		\node [style=zeroin] (39) at (-1, 6) {};
		\node [style=oplus] (40) at (-1, 6.75) {};
		\node [style=dot] (41) at (-2, 6.75) {};
	\end{pgfonlayer}
	\begin{pgfonlayer}{edgelayer}
		\draw (34) to (32);
		\draw (32) to (29);
		\draw (29) to (33);
		\draw (31.center) to (30);
		\draw (30) to (35);
		\draw (30) to (29);
		\draw (38) to (41);
		\draw (40) to (39);
		\draw (40) to (41);
		\draw (40) to (28.center);
		\draw (32) to (41);
		\draw (38) to (29);
		\draw (38) to (36.center);
		\draw (37.center) to (41);
	\end{pgfonlayer}
\end{tikzpicture}\\
&\eq{\ref{TOF.2}}
\begin{tikzpicture}
	\begin{pgfonlayer}{nodelayer}
		\node [style=none] (29) at (-1, 8.75) {};
		\node [style=dot] (30) at (-1.5, 7.75) {};
		\node [style=oplus] (31) at (-0.5, 7.75) {};
		\node [style=none] (32) at (-0.5, 8.75) {};
		\node [style=dot] (33) at (-2, 7.75) {};
		\node [style=X] (34) at (-1.5, 8.25) {};
		\node [style=X] (35) at (-2, 8.25) {};
		\node [style=zeroin] (36) at (-0.5, 6) {};
		\node [style=none] (37) at (-1.5, 5.75) {};
		\node [style=none] (38) at (-2, 5.75) {};
		\node [style=dot] (39) at (-1.5, 7.25) {};
		\node [style=zeroin] (40) at (-1, 6) {};
		\node [style=oplus] (41) at (-1, 7.25) {};
		\node [style=dot] (42) at (-2, 7.25) {};
		\node [style=dot] (43) at (-1, 6.75) {};
		\node [style=oplus] (44) at (-0.5, 6.75) {};
	\end{pgfonlayer}
	\begin{pgfonlayer}{edgelayer}
		\draw (35) to (33);
		\draw (33) to (30);
		\draw (30) to (34);
		\draw (32.center) to (31);
		\draw (31) to (36);
		\draw (31) to (30);
		\draw (39) to (42);
		\draw (41) to (40);
		\draw (41) to (42);
		\draw (41) to (29.center);
		\draw (33) to (42);
		\draw (39) to (30);
		\draw (39) to (37.center);
		\draw (38.center) to (42);
		\draw (44) to (43);
	\end{pgfonlayer}
\end{tikzpicture}
\eq{Lem \ref{lemma:Iwama}}
\begin{tikzpicture}
	\begin{pgfonlayer}{nodelayer}
		\node [style=dot] (30) at (0.25, 6.75) {};
		\node [style=zeroin] (31) at (1.25, 6) {};
		\node [style=none] (32) at (1.75, 7.75) {};
		\node [style=oplus] (33) at (1.25, 6.75) {};
		\node [style=none] (34) at (1.25, 7.75) {};
		\node [style=X] (35) at (0.25, 7.25) {};
		\node [style=X] (36) at (0.75, 7.25) {};
		\node [style=zeroin] (37) at (1.75, 6) {};
		\node [style=none] (38) at (0.25, 5.75) {};
		\node [style=none] (39) at (0.75, 5.75) {};
		\node [style=dot] (40) at (0.75, 6.75) {};
		\node [style=dot] (41) at (1.25, 7.25) {};
		\node [style=oplus] (42) at (1.75, 7.25) {};
	\end{pgfonlayer}
	\begin{pgfonlayer}{edgelayer}
		\draw (40) to (30);
		\draw (33) to (31);
		\draw (33) to (30);
		\draw (33) to (34.center);
		\draw (40) to (39.center);
		\draw (38.center) to (30);
		\draw (42) to (41);
		\draw (32.center) to (42);
		\draw (42) to (37);
		\draw (36) to (40);
		\draw (30) to (35);
	\end{pgfonlayer}
\end{tikzpicture}=
\left\llbracket
\begin{tikzpicture}
	\begin{pgfonlayer}{nodelayer}
		\node [style=andin] (310) at (-1, 6.25) {};
		\node [style=none] (31) at (-1, 6.25) {};
		\node [style=none] (32) at (-1.25, 5.75) {};
		\node [style=none] (33) at (-0.75, 5.75) {};
		\node [style=X] (34) at (-1, 7) {};
		\node [style=none] (35) at (-1.25, 7.5) {};
		\node [style=none] (36) at (-0.75, 7.5) {};
	\end{pgfonlayer}
	\begin{pgfonlayer}{edgelayer}
		\draw [in=63, out=-90] (36.center) to (34);
		\draw (34) to (31);
		\draw [in=90, out=-117] (31) to (32.center);
		\draw [in=-63, out=90] (33.center) to (31);
		\draw [in=-90, out=117] (34) to (35.center);
	\end{pgfonlayer}
\end{tikzpicture}
\right\rrbracket_{\ZXA}
\end{align*}



\item[\ref{ZXA.13}:]
\begin{align*}
\left\llbracket
\begin{tikzpicture}
	\begin{pgfonlayer}{nodelayer}
		\node [style=none] (32) at (-0.5, 6.5) {};
		\node [style=none] (33) at (0, 5.75) {};
		\node [style=none] (34) at (-1, 5.75) {};
		\node [style=X] (35) at (-0.5, 7.25) {};
		\node [style=andin] (36) at (-0.5, 6.5) {};
	\end{pgfonlayer}
	\begin{pgfonlayer}{edgelayer}
		\draw [in=90, out=-135] (32.center) to (34.center);
		\draw [in=-41, out=90] (33.center) to (32.center);
		\draw (35) to (32.center);
	\end{pgfonlayer}
\end{tikzpicture}
\right\rrbracket_{\ZXA}
		&=
\begin{tikzpicture}
	\begin{pgfonlayer}{nodelayer}
		\node [style=none] (33) at (-0.5, 5.75) {};
		\node [style=none] (34) at (-1, 5.75) {};
		\node [style=dot] (35) at (-1, 6.5) {};
		\node [style=dot] (36) at (-0.5, 6.5) {};
		\node [style=oplus] (37) at (0, 6.5) {};
		\node [style=zeroin] (38) at (0, 6) {};
		\node [style=X] (39) at (0, 7) {};
		\node [style=X] (40) at (-0.5, 7) {};
		\node [style=X] (41) at (-1, 7) {};
	\end{pgfonlayer}
	\begin{pgfonlayer}{edgelayer}
		\draw (37) to (35);
		\draw (41) to (34.center);
		\draw (33.center) to (40);
		\draw (39) to (38);
	\end{pgfonlayer}
\end{tikzpicture}
\eq{\ref{TOF.2}}
\begin{tikzpicture}
	\begin{pgfonlayer}{nodelayer}
		\node [style=none] (34) at (-0.5, 5.75) {};
		\node [style=none] (35) at (-1, 5.75) {};
		\node [style=zeroin] (36) at (0, 6) {};
		\node [style=X] (37) at (0, 6.75) {};
		\node [style=X] (38) at (-0.5, 6.75) {};
		\node [style=X] (39) at (-1, 6.75) {};
	\end{pgfonlayer}
	\begin{pgfonlayer}{edgelayer}
		\draw (39) to (35.center);
		\draw (34.center) to (38);
		\draw (37) to (36);
	\end{pgfonlayer}
\end{tikzpicture}
\eq{Rem. \ref{cor:copy}}
\begin{tikzpicture}
	\begin{pgfonlayer}{nodelayer}
		\node [style=none] (35) at (-0.5, 5.75) {};
		\node [style=none] (36) at (-1, 5.75) {};
		\node [style=X] (37) at (-0.5, 6.75) {};
		\node [style=X] (38) at (-1, 6.75) {};
	\end{pgfonlayer}
	\begin{pgfonlayer}{edgelayer}
		\draw (38) to (36.center);
		\draw (35.center) to (37);
	\end{pgfonlayer}
\end{tikzpicture}
=
\left\llbracket
\begin{tikzpicture}
	\begin{pgfonlayer}{nodelayer}
		\node [style=none] (36) at (-0.5, 5.75) {};
		\node [style=none] (37) at (-1, 5.75) {};
		\node [style=X] (38) at (-1, 6.5) {};
		\node [style=X] (39) at (-0.5, 6.5) {};
	\end{pgfonlayer}
	\begin{pgfonlayer}{edgelayer}
		\draw (39) to (36.center);
		\draw (38) to (37.center);
	\end{pgfonlayer}
\end{tikzpicture}
\right\rrbracket_{\ZXA}
\end{align*}


\item[\ref{ZXA.14}:]

$$
\left\llbracket
\begin{tikzpicture}
	\begin{pgfonlayer}{nodelayer}
		\node [style=none] (37) at (-0.25, 7.25) {};
		\node [style=X] (38) at (0, 6.5) {};
		\node [style=Z] (39) at (0, 5.75) {$\pi$};
		\node [style=none] (40) at (0.25, 7.25) {};
	\end{pgfonlayer}
	\begin{pgfonlayer}{edgelayer}
		\draw [style=simple, in=-90, out=124] (38) to (37.center);
		\draw [style=simple, in=60, out=-90] (40.center) to (38);
		\draw [style=simple] (38) to (39);
	\end{pgfonlayer}
\end{tikzpicture}
\right\rrbracket_{\ZXA}
=
\begin{tikzpicture}
	\begin{pgfonlayer}{nodelayer}
		\node [style=dot] (38) at (1, 6.25) {};
		\node [style=oplus] (39) at (1.75, 6.25) {};
		\node [style=onein] (40) at (1, 5.75) {};
		\node [style=zeroin] (41) at (1.75, 5.75) {};
		\node [style=none] (42) at (1, 6.75) {};
		\node [style=none] (43) at (1.75, 6.75) {};
	\end{pgfonlayer}
	\begin{pgfonlayer}{edgelayer}
		\draw (43.center) to (39);
		\draw (39) to (41);
		\draw (39) to (38);
		\draw (38) to (42.center);
		\draw (38) to (40);
	\end{pgfonlayer}
\end{tikzpicture}
\eq{\ref{TOF.1}}
\begin{tikzpicture}
	\begin{pgfonlayer}{nodelayer}
		\node [style=onein] (39) at (1, 5.75) {};
		\node [style=none] (40) at (1, 6.75) {};
		\node [style=none] (41) at (1.75, 6.75) {};
		\node [style=onein] (42) at (1.75, 5.75) {};
	\end{pgfonlayer}
	\begin{pgfonlayer}{edgelayer}
		\draw (39) to (40.center);
		\draw (41.center) to (42);
	\end{pgfonlayer}
\end{tikzpicture}
=
\left\llbracket
\begin{tikzpicture}
	\begin{pgfonlayer}{nodelayer}
		\node [style=none] (40) at (-0.25, 6.25) {};
		\node [style=Z] (41) at (-0.25, 5.75) {$\pi$};
		\node [style=none] (42) at (0.25, 6.25) {};
		\node [style=Z] (43) at (0.25, 5.75) {$\pi$};
	\end{pgfonlayer}
	\begin{pgfonlayer}{edgelayer}
		\draw [style=simple] (43) to (42.center);
		\draw [style=simple] (41) to (40.center);
	\end{pgfonlayer}
\end{tikzpicture}
\right\rrbracket_{\ZXA}
$$
%
%$$
%\left\llbracket
%\begin{tikzpicture}
%	\begin{pgfonlayer}{nodelayer}
%		\node [style=X] (0) at (3, -0.25) {};
%		\node [style=none] (1) at (2, -0) {};
%		\node [style=none] (2) at (2, -0.5) {};
%		\node [style=Z] (3) at (3.75, -0.25) {$\pi$};
%		\node [style=none] (4) at (4.5, -0.25) {};
%	\end{pgfonlayer}
%	\begin{pgfonlayer}{edgelayer}
%		\draw (3) to (0);
%		\draw [in=0, out=166, looseness=1.00] (0) to (1.center);
%		\draw [in=-166, out=0, looseness=1.00] (2.center) to (0);
%		\draw (4.center) to (3);
%	\end{pgfonlayer}
%\end{tikzpicture}
%\right\rrbracket_{\ZXA}
%=
%\begin{tikzpicture}
%	\begin{pgfonlayer}{nodelayer}
%		\node [style=fanin] (0) at (2, -0) {};
%		\node [style=oplus] (1) at (2.5, -0) {};
%		\node [style=none] (2) at (3, -0) {};
%		\node [style=none] (3) at (1.25, 0.25) {};
%		\node [style=none] (4) at (1.25, -0.25) {};
%	\end{pgfonlayer}
%	\begin{pgfonlayer}{edgelayer}
%		\draw (2.center) to (1);
%		\draw (1) to (0);
%		\draw [in=0, out=162, looseness=1.00] (0) to (3.center);
%		\draw [in=-162, out=0, looseness=1.00] (4.center) to (0);
%	\end{pgfonlayer}
%\end{tikzpicture}
%\eq{nat.}
%\begin{tikzpicture}
%	\begin{pgfonlayer}{nodelayer}
%		\node [style=fanin] (0) at (2, -0) {};
%		\node [style=none] (1) at (2.5, -0) {};
%		\node [style=none] (2) at (1.25, 0.25) {};
%		\node [style=none] (3) at (1.25, -0.25) {};
%		\node [style=oplus] (4) at (1.25, 0.25) {};
%		\node [style=oplus] (5) at (1.25, -0.25) {};
%		\node [style=none] (6) at (0.5, 0.25) {};
%		\node [style=none] (7) at (0.5, -0.25) {};
%	\end{pgfonlayer}
%	\begin{pgfonlayer}{edgelayer}
%		\draw [in=0, out=162, looseness=1.00] (0) to (2.center);
%		\draw [in=-162, out=0, looseness=1.00] (3.center) to (0);
%		\draw (1.center) to (0);
%		\draw (2.center) to (6.center);
%		\draw (7.center) to (3.center);
%	\end{pgfonlayer}
%\end{tikzpicture}
%=
%\left\llbracket
%\begin{tikzpicture}
%	\begin{pgfonlayer}{nodelayer}
%		\node [style=none] (0) at (2, -0) {};
%		\node [style=none] (1) at (2, -0.5) {};
%		\node [style=Z] (2) at (2.75, -0) {$\pi$};
%		\node [style=Z] (3) at (2.75, -0.5) {$\pi$};
%		\node [style=X] (4) at (3.5, -0.25) {};
%		\node [style=none] (5) at (4.25, -0.25) {};
%	\end{pgfonlayer}
%	\begin{pgfonlayer}{edgelayer}
%		\draw (3) to (1.center);
%		\draw (2) to (0.center);
%		\draw [in=162, out=0, looseness=1.00] (2) to (4);
%		\draw (4) to (5.center);
%		\draw [in=0, out=-162, looseness=1.00] (4) to (3);
%	\end{pgfonlayer}
%\end{tikzpicture}
%\right\rrbracket_{\ZXA}
%$$
%
%
%$$
%\left\llbracket
%\begin{tikzpicture}
%	\begin{pgfonlayer}{nodelayer}
%		\node [style=Z] (0) at (3, -0.25) {};
%		\node [style=none] (1) at (2, -0) {};
%		\node [style=none] (2) at (2, -0.5) {};
%		\node [style=X] (3) at (3.75, -0.25) {};
%	\end{pgfonlayer}
%	\begin{pgfonlayer}{edgelayer}
%		\draw (3) to (0);
%		\draw [in=0, out=166, looseness=1.00] (0) to (1.center);
%		\draw [in=-166, out=0, looseness=1.00] (2.center) to (0);
%	\end{pgfonlayer}
%\end{tikzpicture}
%\right\rrbracket_{\ZXA}
%=
%\begin{tikzpicture}
%	\begin{pgfonlayer}{nodelayer}
%		\node [style=dot] (0) at (1, -0) {};
%		\node [style=oplus] (1) at (1, -0.5) {};
%		\node [style=X] (2) at (1.5, -0) {};
%		\node [style=X] (3) at (1.5, -0.5) {};
%		\node [style=none] (4) at (0.5, -0.5) {};
%		\node [style=none] (5) at (0.5, -0) {};
%	\end{pgfonlayer}
%	\begin{pgfonlayer}{edgelayer}
%		\draw (0) to (1);
%		\draw (3) to (1);
%		\draw (1) to (4.center);
%		\draw (5.center) to (0);
%		\draw (0) to (2);
%	\end{pgfonlayer}
%\end{tikzpicture}
%\eq{Lem. \ref{lemma:whiteunit}}
%\begin{tikzpicture}
%	\begin{pgfonlayer}{nodelayer}
%		\node [style=X] (0) at (1.5, -0) {};
%		\node [style=X] (1) at (1.5, -0.5) {};
%		\node [style=none] (2) at (0.5, -0.5) {};
%		\node [style=none] (3) at (0.5, -0) {};
%	\end{pgfonlayer}
%	\begin{pgfonlayer}{edgelayer}
%		\draw (1) to (2.center);
%		\draw (3.center) to (0);
%	\end{pgfonlayer}
%\end{tikzpicture}
%=
%\left\llbracket
%\begin{tikzpicture}
%	\begin{pgfonlayer}{nodelayer}
%		\node [style=none] (0) at (2, -0) {};
%		\node [style=none] (1) at (2, -0.5) {};
%		\node [style=X] (2) at (2.75, -0) {};
%		\node [style=X] (3) at (2.75, -0.5) {};
%	\end{pgfonlayer}
%	\begin{pgfonlayer}{edgelayer}
%		\draw (3) to (1.center);
%		\draw (2) to (0.center);
%	\end{pgfonlayer}
%\end{tikzpicture}
%\right\rrbracket_{\ZXA}
%$$
%
%
%$$
%\left\llbracket
%\begin{tikzpicture}
%	\begin{pgfonlayer}{nodelayer}
%		\node [style=X] (0) at (3, -0.25) {};
%		\node [style=none] (1) at (2, -0) {};
%		\node [style=none] (2) at (2, -0.5) {};
%		\node [style=Z] (3) at (3.75, -0.25) {};
%	\end{pgfonlayer}
%	\begin{pgfonlayer}{edgelayer}
%		\draw (3) to (0);
%		\draw [in=0, out=166, looseness=1.00] (0) to (1.center);
%		\draw [in=-166, out=0, looseness=1.00] (2.center) to (0);
%	\end{pgfonlayer}
%\end{tikzpicture}
%\right\rrbracket_{\ZXA}
%=
%\begin{tikzpicture}
%	\begin{pgfonlayer}{nodelayer}
%		\node [style=oplus] (0) at (3, -1.75) {};
%		\node [style=dot] (1) at (3, -1) {};
%		\node [style=zeroout] (2) at (3.5, -1) {};
%		\node [style=zeroout] (3) at (3.5, -1.75) {};
%		\node [style=none] (4) at (2.5, -1) {};
%		\node [style=none] (5) at (2.5, -1.75) {};
%	\end{pgfonlayer}
%	\begin{pgfonlayer}{edgelayer}
%		\draw (3) to (0);
%		\draw (0) to (5.center);
%		\draw (4.center) to (1);
%		\draw (1) to (2);
%		\draw (1) to (0);
%	\end{pgfonlayer}
%\end{tikzpicture}
%\eq{\ref{TOF.2}}
%\left\llbracket
%\begin{tikzpicture}
%	\begin{pgfonlayer}{nodelayer}
%		\node [style=none] (0) at (2, -0) {};
%		\node [style=none] (1) at (2, -0.5) {};
%		\node [style=Z] (2) at (2.75, -0) {};
%		\node [style=Z] (3) at (2.75, -0.5) {};
%	\end{pgfonlayer}
%	\begin{pgfonlayer}{edgelayer}
%		\draw (3) to (1.center);
%		\draw (2) to (0.center);
%	\end{pgfonlayer}
%\end{tikzpicture}
%\right\rrbracket_{\ZXA}
%$$



\item[\ref{ZXA.15}:]
\begin{align*}
\left\llbracket
\begin{tikzpicture}
	\begin{pgfonlayer}{nodelayer}
		\node [style=X] (41) at (-1, 6.25) {};
		\node [style=none] (42) at (-1, 7.25) {};
		\node [style=andin] (420) at (-1, 7.25) {};
		\node [style=none] (43) at (-1, 7.75) {};
		\node [style=none] (44) at (-1, 5.75) {};
	\end{pgfonlayer}
	\begin{pgfonlayer}{edgelayer}
		\draw (43.center) to (42);
		\draw [in=120, out=-120, looseness=1.25] (42) to (41);
		\draw [in=-60, out=60, looseness=1.25] (41) to (42);
		\draw (41) to (44.center);
	\end{pgfonlayer}
\end{tikzpicture}
\right\rrbracket_{\ZXA}
=
\begin{tikzpicture}
	\begin{pgfonlayer}{nodelayer}
		\node [style=dot] (42) at (0, 7) {};
		\node [style=dot] (43) at (0.5, 7) {};
		\node [style=oplus] (44) at (1, 7) {};
		\node [style=X] (45) at (0, 7.5) {};
		\node [style=X] (46) at (0.5, 7.5) {};
		\node [style=zeroin] (47) at (1, 6.5) {};
		\node [style=none] (48) at (1, 7.75) {};
		\node [style=dot] (49) at (0, 6.5) {};
		\node [style=oplus] (50) at (0.5, 6.5) {};
		\node [style=zeroin] (51) at (0.5, 6) {};
		\node [style=none] (52) at (0, 5.75) {};
	\end{pgfonlayer}
	\begin{pgfonlayer}{edgelayer}
		\draw (48.center) to (44);
		\draw (44) to (47);
		\draw (44) to (43);
		\draw (43) to (46);
		\draw (45) to (42);
		\draw (42) to (43);
		\draw (43) to (50);
		\draw (50) to (49);
		\draw (49) to (42);
		\draw (50) to (51);
		\draw (49) to (52.center);
	\end{pgfonlayer}
\end{tikzpicture}
\eq{Lem. \ref{lemma:Iwama}}
\begin{tikzpicture}
	\begin{pgfonlayer}{nodelayer}
		\node [style=dot] (43) at (0, 7) {};
		\node [style=dot] (44) at (0.5, 7) {};
		\node [style=oplus] (45) at (1, 7) {};
		\node [style=X] (46) at (0, 8) {};
		\node [style=X] (47) at (0.5, 8) {};
		\node [style=zeroin] (48) at (1, 6) {};
		\node [style=none] (49) at (1, 8.25) {};
		\node [style=zeroin] (50) at (0.5, 6) {};
		\node [style=none] (51) at (0, 5.75) {};
		\node [style=dot] (52) at (0, 7.5) {};
		\node [style=oplus] (53) at (0.5, 7.5) {};
		\node [style=dot] (54) at (0, 6.5) {};
		\node [style=oplus] (55) at (1, 6.5) {};
	\end{pgfonlayer}
	\begin{pgfonlayer}{edgelayer}
		\draw (49.center) to (45);
		\draw (45) to (48);
		\draw (45) to (44);
		\draw (44) to (47);
		\draw (46) to (43);
		\draw (43) to (44);
		\draw (53) to (52);
		\draw (55) to (54);
		\draw (44) to (50);
		\draw (51.center) to (54);
		\draw (54) to (43);
	\end{pgfonlayer}
\end{tikzpicture}
\eq{Rem. \ref{cor:copy}}
\begin{tikzpicture}
	\begin{pgfonlayer}{nodelayer}
		\node [style=dot] (44) at (0, 7) {};
		\node [style=dot] (45) at (0.5, 7) {};
		\node [style=oplus] (46) at (1, 7) {};
		\node [style=X] (47) at (0, 7.5) {};
		\node [style=X] (48) at (0.5, 7.5) {};
		\node [style=zeroin] (49) at (1, 6) {};
		\node [style=none] (50) at (1, 7.75) {};
		\node [style=zeroin] (51) at (0.5, 6) {};
		\node [style=none] (52) at (0, 5.75) {};
		\node [style=dot] (53) at (0, 6.5) {};
		\node [style=oplus] (54) at (1, 6.5) {};
	\end{pgfonlayer}
	\begin{pgfonlayer}{edgelayer}
		\draw (50.center) to (46);
		\draw (46) to (49);
		\draw (46) to (45);
		\draw (45) to (48);
		\draw (47) to (44);
		\draw (44) to (45);
		\draw (54) to (53);
		\draw (45) to (51);
		\draw (52.center) to (53);
		\draw (53) to (44);
	\end{pgfonlayer}
\end{tikzpicture}
\eq{\ref{TOF.2}}
\begin{tikzpicture}
	\begin{pgfonlayer}{nodelayer}
		\node [style=X] (45) at (0, 7) {};
		\node [style=X] (46) at (0.5, 7) {};
		\node [style=zeroin] (47) at (1, 6) {};
		\node [style=none] (48) at (1, 7.25) {};
		\node [style=zeroin] (49) at (0.5, 6) {};
		\node [style=none] (50) at (0, 5.75) {};
		\node [style=dot] (51) at (0, 6.5) {};
		\node [style=oplus] (52) at (1, 6.5) {};
	\end{pgfonlayer}
	\begin{pgfonlayer}{edgelayer}
		\draw (52) to (51);
		\draw (50.center) to (51);
		\draw (48.center) to (52);
		\draw (52) to (47);
		\draw (49) to (46);
		\draw (45) to (51);
	\end{pgfonlayer}
\end{tikzpicture}
\eq{Rem. \ref{cor:copy}}
\begin{tikzpicture}
	\begin{pgfonlayer}{nodelayer}
		\node [style=X] (46) at (0, 7) {};
		\node [style=zeroin] (47) at (0.5, 6) {};
		\node [style=none] (48) at (0.5, 7.25) {};
		\node [style=none] (49) at (0, 5.75) {};
		\node [style=dot] (50) at (0, 6.5) {};
		\node [style=oplus] (51) at (0.5, 6.5) {};
	\end{pgfonlayer}
	\begin{pgfonlayer}{edgelayer}
		\draw (51) to (50);
		\draw (49.center) to (50);
		\draw (48.center) to (51);
		\draw (51) to (47);
		\draw (46) to (50);
	\end{pgfonlayer}
\end{tikzpicture}
\eq{Lem. \ref{lemma:whiteunit}}
\begin{tikzpicture}
	\begin{pgfonlayer}{nodelayer}
		\node [style=none] (47) at (-1, 6.75) {};
		\node [style=none] (48) at (-1, 5.75) {};
	\end{pgfonlayer}
	\begin{pgfonlayer}{edgelayer}
		\draw (47.center) to (48.center);
	\end{pgfonlayer}
\end{tikzpicture}
=
\left\llbracket
\begin{tikzpicture}
	\begin{pgfonlayer}{nodelayer}
		\node [style=none] (47) at (-1, 6.75) {};
		\node [style=none] (48) at (-1, 5.75) {};
	\end{pgfonlayer}
	\begin{pgfonlayer}{edgelayer}
		\draw (47.center) to (48.center);
	\end{pgfonlayer}
\end{tikzpicture}
\right\rrbracket_{\ZXA}
\end{align*}


\item[\ref{ZXA.16}:]
This is precisely \ref{TOF.7}.

\item[\ref{ZXA.17}:]
\begin{align*}
\left\llbracket
\begin{tikzpicture}
	\begin{pgfonlayer}{nodelayer}
		\node [style=Z] (48) at (0, 6.25) {};
		\node [style=andin] (49) at (-0.25, 7) {};
		\node [style=none] (50) at (-0.5, 6.25) {};
		\node [style=none] (51) at (-0.25, 5.75) {};
		\node [style=none] (52) at (0.25, 5.75) {};
		\node [style=none] (53) at (-0.5, 5.75) {};
		\node [style=none] (54) at (-0.25, 7.5) {};
	\end{pgfonlayer}
	\begin{pgfonlayer}{edgelayer}
		\draw [in=-72, out=90] (48) to (49.center);
		\draw (49.center) to (54.center);
		\draw [in=90, out=-108] (49.center) to (50.center);
		\draw (50.center) to (53.center);
		\draw [in=90, out=-117] (48) to (51.center);
		\draw [in=90, out=-63] (48) to (52.center);
	\end{pgfonlayer}
\end{tikzpicture}
\right\rrbracket_{\ZXA}
&=
\begin{tikzpicture}
	\begin{pgfonlayer}{nodelayer}
		\node [style=X] (49) at (1.5, 6.75) {};
		\node [style=oplus] (50) at (1.5, 8) {};
		\node [style=zeroin] (51) at (1.5, 7.5) {};
		\node [style=dot] (52) at (1, 8) {};
		\node [style=dot] (53) at (0.5, 8) {};
		\node [style=X] (54) at (1, 8.5) {};
		\node [style=X] (55) at (0.5, 8.5) {};
		\node [style=dot] (56) at (1.5, 6.25) {};
		\node [style=oplus] (57) at (1, 6.25) {};
		\node [style=none] (58) at (1.5, 5.75) {};
		\node [style=none] (59) at (1, 5.75) {};
		\node [style=none] (60) at (0.5, 5.75) {};
		\node [style=none] (61) at (1.5, 8.75) {};
	\end{pgfonlayer}
	\begin{pgfonlayer}{edgelayer}
		\draw (56) to (57);
		\draw (57) to (59.center);
		\draw (58.center) to (56);
		\draw (61.center) to (50);
		\draw (50) to (51);
		\draw (50) to (52);
		\draw (52) to (54);
		\draw (52) to (53);
		\draw (53) to (55);
		\draw (53) to (60.center);
		\draw (57) to (52);
		\draw (49) to (56);
	\end{pgfonlayer}
\end{tikzpicture}
=
\begin{tikzpicture}
	\begin{pgfonlayer}{nodelayer}
		\node [style=X] (50) at (2, 7.5) {};
		\node [style=oplus] (51) at (1.5, 7) {};
		\node [style=zeroin] (52) at (1.5, 6) {};
		\node [style=dot] (53) at (1, 7) {};
		\node [style=dot] (54) at (0.5, 7) {};
		\node [style=X] (55) at (1, 7.5) {};
		\node [style=X] (56) at (0.5, 7.5) {};
		\node [style=dot] (57) at (2, 6.5) {};
		\node [style=oplus] (58) at (1, 6.5) {};
		\node [style=none] (59) at (2, 5.75) {};
		\node [style=none] (60) at (1, 5.75) {};
		\node [style=none] (61) at (0.5, 5.75) {};
		\node [style=none] (62) at (1.5, 7.75) {};
	\end{pgfonlayer}
	\begin{pgfonlayer}{edgelayer}
		\draw (57) to (58);
		\draw (58) to (60.center);
		\draw (59.center) to (57);
		\draw (62.center) to (51);
		\draw (51) to (52);
		\draw (51) to (53);
		\draw (53) to (55);
		\draw (53) to (54);
		\draw (54) to (56);
		\draw (54) to (61.center);
		\draw (58) to (53);
		\draw (50) to (57);
	\end{pgfonlayer}
\end{tikzpicture}\
\eq{Lem. \ref{lemma:Iwama}}
\begin{tikzpicture}
	\begin{pgfonlayer}{nodelayer}
		\node [style=X] (51) at (2, 8) {};
		\node [style=oplus] (52) at (1.5, 7) {};
		\node [style=zeroin] (53) at (1.5, 6) {};
		\node [style=dot] (54) at (1, 7) {};
		\node [style=dot] (55) at (0.5, 7) {};
		\node [style=X] (56) at (1, 8) {};
		\node [style=X] (57) at (0.5, 8) {};
		\node [style=dot] (58) at (2, 7.5) {};
		\node [style=oplus] (59) at (1, 7.5) {};
		\node [style=none] (60) at (2, 5.75) {};
		\node [style=none] (61) at (1, 5.75) {};
		\node [style=none] (62) at (0.5, 5.75) {};
		\node [style=none] (63) at (1.5, 8.25) {};
		\node [style=oplus] (64) at (1.5, 6.5) {};
		\node [style=dot] (65) at (2, 6.5) {};
		\node [style=dot] (66) at (0.5, 6.5) {};
	\end{pgfonlayer}
	\begin{pgfonlayer}{edgelayer}
		\draw (58) to (59);
		\draw (59) to (61.center);
		\draw (60.center) to (58);
		\draw (63.center) to (52);
		\draw (52) to (53);
		\draw (52) to (54);
		\draw (54) to (56);
		\draw (54) to (55);
		\draw (55) to (57);
		\draw (55) to (62.center);
		\draw (59) to (54);
		\draw (51) to (58);
		\draw (65) to (64);
		\draw (64) to (66);
	\end{pgfonlayer}
\end{tikzpicture}
\eq{Rem. \ref{cor:copy}}
\begin{tikzpicture}
	\begin{pgfonlayer}{nodelayer}
		\node [style=X] (52) at (2, 7.5) {};
		\node [style=oplus] (53) at (1.5, 7) {};
		\node [style=zeroin] (54) at (1.5, 6) {};
		\node [style=dot] (55) at (1, 7) {};
		\node [style=dot] (56) at (0.5, 7) {};
		\node [style=X] (57) at (1, 7.5) {};
		\node [style=X] (58) at (0.5, 7.5) {};
		\node [style=none] (59) at (2, 5.75) {};
		\node [style=none] (60) at (1, 5.75) {};
		\node [style=none] (61) at (0.5, 5.75) {};
		\node [style=none] (62) at (1.5, 7.75) {};
		\node [style=oplus] (63) at (1.5, 6.5) {};
		\node [style=dot] (64) at (2, 6.5) {};
		\node [style=dot] (65) at (0.5, 6.5) {};
	\end{pgfonlayer}
	\begin{pgfonlayer}{edgelayer}
		\draw (62.center) to (53);
		\draw (53) to (54);
		\draw (53) to (55);
		\draw (55) to (57);
		\draw (55) to (56);
		\draw (56) to (58);
		\draw (56) to (61.center);
		\draw (64) to (63);
		\draw (63) to (65);
		\draw (52) to (64);
		\draw (64) to (59.center);
		\draw (60.center) to (55);
	\end{pgfonlayer}
\end{tikzpicture}\\
&\eq{Rem. \ref{cor:copy}}
\begin{tikzpicture}
	\begin{pgfonlayer}{nodelayer}
		\node [style=X] (53) at (2, 8) {};
		\node [style=oplus] (54) at (1, 7) {};
		\node [style=zeroin] (55) at (1, 6) {};
		\node [style=dot] (56) at (0.5, 7) {};
		\node [style=dot] (57) at (0, 7) {};
		\node [style=X] (58) at (0.5, 8) {};
		\node [style=X] (59) at (0, 8) {};
		\node [style=none] (60) at (2, 5.75) {};
		\node [style=none] (61) at (0.5, 5.75) {};
		\node [style=none] (62) at (0, 5.75) {};
		\node [style=none] (63) at (1, 8.25) {};
		\node [style=oplus] (64) at (1, 6.5) {};
		\node [style=dot] (65) at (2, 6.5) {};
		\node [style=dot] (66) at (0, 6.5) {};
		\node [style=zeroin] (67) at (1.5, 6) {};
		\node [style=X] (68) at (1.5, 8) {};
		\node [style=oplus] (69) at (1.5, 7.5) {};
		\node [style=dot] (70) at (2, 7.5) {};
		\node [style=dot] (71) at (0, 7.5) {};
	\end{pgfonlayer}
	\begin{pgfonlayer}{edgelayer}
		\draw (63.center) to (54);
		\draw (54) to (55);
		\draw (54) to (56);
		\draw (56) to (58);
		\draw (56) to (57);
		\draw (57) to (59);
		\draw (57) to (62.center);
		\draw (65) to (64);
		\draw (64) to (66);
		\draw (53) to (65);
		\draw (65) to (60.center);
		\draw (61.center) to (56);
		\draw (67) to (68);
		\draw (70) to (69);
		\draw (69) to (71);
	\end{pgfonlayer}
\end{tikzpicture}
\eq{\ref{TOF.2}}
\begin{tikzpicture}
	\begin{pgfonlayer}{nodelayer}
		\node [style=X] (54) at (2, 8.5) {};
		\node [style=oplus] (55) at (1, 7) {};
		\node [style=zeroin] (56) at (1, 6) {};
		\node [style=dot] (57) at (0.5, 7) {};
		\node [style=dot] (58) at (0, 7) {};
		\node [style=X] (59) at (0.5, 8.5) {};
		\node [style=X] (60) at (0, 8.5) {};
		\node [style=none] (61) at (2, 5.75) {};
		\node [style=none] (62) at (0.5, 5.75) {};
		\node [style=none] (63) at (0, 5.75) {};
		\node [style=none] (64) at (1, 8.75) {};
		\node [style=oplus] (65) at (1, 6.5) {};
		\node [style=dot] (66) at (2, 6.5) {};
		\node [style=dot] (67) at (0, 6.5) {};
		\node [style=zeroin] (68) at (1.5, 6) {};
		\node [style=X] (69) at (1.5, 8.5) {};
		\node [style=oplus] (70) at (1.5, 8) {};
		\node [style=dot] (71) at (2, 8) {};
		\node [style=dot] (72) at (0, 8) {};
		\node [style=oplus] (73) at (1, 7.5) {};
		\node [style=dot] (74) at (1.5, 7.5) {};
	\end{pgfonlayer}
	\begin{pgfonlayer}{edgelayer}
		\draw (64.center) to (55);
		\draw (55) to (56);
		\draw (55) to (57);
		\draw (57) to (59);
		\draw (57) to (58);
		\draw (58) to (60);
		\draw (58) to (63.center);
		\draw (66) to (65);
		\draw (65) to (67);
		\draw (54) to (66);
		\draw (66) to (61.center);
		\draw (62.center) to (57);
		\draw (68) to (69);
		\draw (71) to (70);
		\draw (70) to (72);
		\draw (74) to (73);
	\end{pgfonlayer}
\end{tikzpicture}
\eq{Lem. \ref{lemma:Iwama}}
\begin{tikzpicture}
	\begin{pgfonlayer}{nodelayer}
		\node [style=X] (55) at (2, 9) {};
		\node [style=oplus] (56) at (1, 7) {};
		\node [style=zeroin] (57) at (1, 6) {};
		\node [style=dot] (58) at (0.5, 7) {};
		\node [style=dot] (59) at (0, 7) {};
		\node [style=X] (60) at (0.5, 9) {};
		\node [style=X] (61) at (0, 9) {};
		\node [style=none] (62) at (2, 5.75) {};
		\node [style=none] (63) at (0.5, 5.75) {};
		\node [style=none] (64) at (0, 5.75) {};
		\node [style=none] (65) at (1, 9.25) {};
		\node [style=zeroin] (66) at (1.5, 6) {};
		\node [style=X] (67) at (1.5, 9) {};
		\node [style=oplus] (68) at (1.5, 7.5) {};
		\node [style=dot] (69) at (2, 7.5) {};
		\node [style=dot] (70) at (0, 7.5) {};
		\node [style=oplus] (71) at (1, 8) {};
		\node [style=dot] (72) at (1.5, 8) {};
		\node [style=oplus] (73) at (1, 8.5) {};
		\node [style=dot] (74) at (0, 8.5) {};
		\node [style=dot] (75) at (2, 8.5) {};
		\node [style=oplus] (76) at (1, 6.5) {};
		\node [style=dot] (77) at (2, 6.5) {};
		\node [style=dot] (78) at (0, 6.5) {};
	\end{pgfonlayer}
	\begin{pgfonlayer}{edgelayer}
		\draw (65.center) to (56);
		\draw (56) to (57);
		\draw (56) to (58);
		\draw (58) to (60);
		\draw (58) to (59);
		\draw (59) to (61);
		\draw (59) to (64.center);
		\draw (63.center) to (58);
		\draw (66) to (67);
		\draw (69) to (68);
		\draw (68) to (70);
		\draw (72) to (71);
		\draw (75) to (73);
		\draw (73) to (74);
		\draw (76) to (78);
		\draw (77) to (76);
		\draw (55) to (77);
		\draw (77) to (62.center);
	\end{pgfonlayer}
\end{tikzpicture}\\
&\eq{\ref{TOF.9}}
\begin{tikzpicture}
	\begin{pgfonlayer}{nodelayer}
		\node [style=X] (56) at (2, 8) {};
		\node [style=oplus] (57) at (1, 6.5) {};
		\node [style=zeroin] (58) at (1, 6) {};
		\node [style=dot] (59) at (0.5, 6.5) {};
		\node [style=dot] (60) at (0, 6.5) {};
		\node [style=X] (61) at (0.5, 8) {};
		\node [style=X] (62) at (0, 8) {};
		\node [style=none] (63) at (2, 5.75) {};
		\node [style=none] (64) at (0.5, 5.75) {};
		\node [style=none] (65) at (0, 5.75) {};
		\node [style=none] (66) at (1, 8.25) {};
		\node [style=zeroin] (67) at (1.5, 6) {};
		\node [style=X] (68) at (1.5, 8) {};
		\node [style=oplus] (69) at (1.5, 7) {};
		\node [style=dot] (70) at (2, 7) {};
		\node [style=dot] (71) at (0, 7) {};
		\node [style=oplus] (72) at (1, 7.5) {};
		\node [style=dot] (73) at (1.5, 7.5) {};
	\end{pgfonlayer}
	\begin{pgfonlayer}{edgelayer}
		\draw (66.center) to (57);
		\draw (57) to (58);
		\draw (57) to (59);
		\draw (59) to (61);
		\draw (59) to (60);
		\draw (60) to (62);
		\draw (60) to (65.center);
		\draw (64.center) to (59);
		\draw (67) to (68);
		\draw (70) to (69);
		\draw (69) to (71);
		\draw (73) to (72);
		\draw (63.center) to (56);
	\end{pgfonlayer}
\end{tikzpicture}
\eq{Rem. \ref{cor:copy}}
\begin{tikzpicture}
	\begin{pgfonlayer}{nodelayer}
		\node [style=dot] (57) at (0.5, 6.75) {};
		\node [style=X] (58) at (1, 8.25) {};
		\node [style=X] (59) at (3, 8.25) {};
		\node [style=none] (60) at (1, 5.75) {};
		\node [style=dot] (61) at (3, 6.75) {};
		\node [style=dot] (62) at (1, 7.25) {};
		\node [style=none] (63) at (3, 5.75) {};
		\node [style=oplus] (64) at (1.5, 7.25) {};
		\node [style=none] (65) at (1.5, 8.5) {};
		\node [style=none] (66) at (0.5, 5.75) {};
		\node [style=X] (67) at (2, 8.25) {};
		\node [style=oplus] (68) at (2, 6.75) {};
		\node [style=X] (69) at (0.5, 8.25) {};
		\node [style=dot] (70) at (0.5, 7.25) {};
		\node [style=zeroin] (71) at (2, 6) {};
		\node [style=zeroin] (72) at (1.5, 6) {};
		\node [style=oplus] (73) at (1.5, 7.75) {};
		\node [style=dot] (74) at (2, 7.75) {};
		\node [style=zeroin] (75) at (2.5, 6) {};
		\node [style=X] (76) at (2.5, 8.25) {};
	\end{pgfonlayer}
	\begin{pgfonlayer}{edgelayer}
		\draw (65.center) to (64);
		\draw (64) to (72);
		\draw (64) to (62);
		\draw (62) to (58);
		\draw (60.center) to (62);
		\draw (62) to (70);
		\draw (70) to (69);
		\draw (70) to (66.center);
		\draw (67) to (71);
		\draw (68) to (57);
		\draw (61) to (68);
		\draw (74) to (73);
		\draw (59) to (61);
		\draw (61) to (63.center);
		\draw (76) to (75);
	\end{pgfonlayer}
\end{tikzpicture}
\eq{Rem. \ref{cor:copy}}
\begin{tikzpicture}
	\begin{pgfonlayer}{nodelayer}
		\node [style=dot] (58) at (0.5, 6.75) {};
		\node [style=X] (59) at (1, 8.75) {};
		\node [style=X] (60) at (3, 8.75) {};
		\node [style=none] (61) at (1, 5.75) {};
		\node [style=dot] (62) at (3, 6.75) {};
		\node [style=dot] (63) at (1, 7.75) {};
		\node [style=none] (64) at (3, 5.75) {};
		\node [style=oplus] (65) at (1.5, 7.75) {};
		\node [style=none] (66) at (1.5, 9) {};
		\node [style=none] (67) at (0.5, 5.75) {};
		\node [style=X] (68) at (2, 8.75) {};
		\node [style=oplus] (69) at (2, 6.75) {};
		\node [style=X] (70) at (0.5, 8.75) {};
		\node [style=dot] (71) at (0.5, 7.75) {};
		\node [style=zeroin] (72) at (2, 6) {};
		\node [style=zeroin] (73) at (1.5, 6) {};
		\node [style=oplus] (74) at (1.5, 8.25) {};
		\node [style=dot] (75) at (2, 8.25) {};
		\node [style=zeroin] (76) at (2.5, 6) {};
		\node [style=X] (77) at (2.5, 8.75) {};
		\node [style=dot] (78) at (0.5, 7.25) {};
		\node [style=oplus] (79) at (2.5, 7.25) {};
	\end{pgfonlayer}
	\begin{pgfonlayer}{edgelayer}
		\draw (66.center) to (65);
		\draw (65) to (73);
		\draw (65) to (63);
		\draw (63) to (59);
		\draw (61.center) to (63);
		\draw (63) to (71);
		\draw (71) to (70);
		\draw (71) to (67.center);
		\draw (68) to (72);
		\draw (69) to (58);
		\draw (62) to (69);
		\draw (75) to (74);
		\draw (60) to (62);
		\draw (62) to (64.center);
		\draw (77) to (76);
		\draw (79) to (78);
	\end{pgfonlayer}
\end{tikzpicture}\\
&\eq{\ref{TOF.2}}
\begin{tikzpicture}
	\begin{pgfonlayer}{nodelayer}
		\node [style=dot] (59) at (0.5, 6.75) {};
		\node [style=X] (60) at (1, 9.25) {};
		\node [style=X] (61) at (3, 9.25) {};
		\node [style=none] (62) at (1, 5.75) {};
		\node [style=dot] (63) at (3, 6.75) {};
		\node [style=dot] (64) at (1, 8.25) {};
		\node [style=none] (65) at (3, 5.75) {};
		\node [style=oplus] (66) at (1.5, 8.25) {};
		\node [style=none] (67) at (1.5, 9.5) {};
		\node [style=none] (68) at (0.5, 5.75) {};
		\node [style=X] (69) at (2, 9.25) {};
		\node [style=oplus] (70) at (2, 6.75) {};
		\node [style=X] (71) at (0.5, 9.25) {};
		\node [style=dot] (72) at (0.5, 8.25) {};
		\node [style=zeroin] (73) at (2, 6) {};
		\node [style=zeroin] (74) at (1.5, 6) {};
		\node [style=oplus] (75) at (1.5, 8.75) {};
		\node [style=dot] (76) at (2, 8.75) {};
		\node [style=zeroin] (77) at (2.5, 6) {};
		\node [style=X] (78) at (2.5, 9.25) {};
		\node [style=oplus] (79) at (2, 7.25) {};
		\node [style=dot] (80) at (2.5, 7.25) {};
		\node [style=dot] (81) at (3, 7.25) {};
		\node [style=dot] (82) at (0.5, 7.75) {};
		\node [style=oplus] (83) at (2.5, 7.75) {};
	\end{pgfonlayer}
	\begin{pgfonlayer}{edgelayer}
		\draw (67.center) to (66);
		\draw (66) to (74);
		\draw (66) to (64);
		\draw (64) to (60);
		\draw (62.center) to (64);
		\draw (64) to (72);
		\draw (72) to (71);
		\draw (72) to (68.center);
		\draw (69) to (73);
		\draw (70) to (59);
		\draw (63) to (70);
		\draw (76) to (75);
		\draw (61) to (63);
		\draw (63) to (65.center);
		\draw (78) to (77);
		\draw (81) to (80);
		\draw (80) to (79);
		\draw (83) to (82);
	\end{pgfonlayer}
\end{tikzpicture}
\eq{\ref{TOF.2}}
\begin{tikzpicture}
	\begin{pgfonlayer}{nodelayer}
		\node [style=dot] (59) at (0.5, 6.75) {};
		\node [style=X] (60) at (1, 9.25) {};
		\node [style=X] (61) at (3, 9.25) {};
		\node [style=none] (62) at (1, 5.75) {};
		\node [style=dot] (63) at (3, 6.75) {};
		\node [style=dot] (64) at (1, 8.25) {};
		\node [style=none] (65) at (3, 5.75) {};
		\node [style=oplus] (66) at (1.5, 8.25) {};
		\node [style=none] (67) at (1.5, 9.5) {};
		\node [style=none] (68) at (0.5, 5.75) {};
		\node [style=X] (69) at (2, 9.25) {};
		\node [style=oplus] (70) at (2, 6.75) {};
		\node [style=X] (71) at (0.5, 9.25) {};
		\node [style=dot] (72) at (0.5, 8.25) {};
		\node [style=zeroin] (73) at (2, 6) {};
		\node [style=zeroin] (74) at (1.5, 6) {};
		\node [style=oplus] (75) at (1.5, 8.75) {};
		\node [style=dot] (76) at (2, 8.75) {};
		\node [style=zeroin] (77) at (2.5, 6) {};
		\node [style=X] (78) at (2.5, 9.25) {};
		\node [style=oplus] (79) at (2, 7.25) {};
		\node [style=dot] (80) at (2.5, 7.25) {};
		\node [style=dot] (81) at (3, 7.25) {};
		\node [style=dot] (82) at (0.5, 7.75) {};
		\node [style=oplus] (83) at (2.5, 7.75) {};
	\end{pgfonlayer}
	\begin{pgfonlayer}{edgelayer}
		\draw (67.center) to (66);
		\draw (66) to (74);
		\draw (66) to (64);
		\draw (64) to (60);
		\draw (62.center) to (64);
		\draw (64) to (72);
		\draw (72) to (71);
		\draw (72) to (68.center);
		\draw (69) to (73);
		\draw (70) to (59);
		\draw (63) to (70);
		\draw (76) to (75);
		\draw (61) to (63);
		\draw (63) to (65.center);
		\draw (78) to (77);
		\draw (81) to (80);
		\draw (80) to (79);
		\draw (83) to (82);
	\end{pgfonlayer}
\end{tikzpicture}
\eq{\ref{ZXA.11}}
\begin{tikzpicture}
	\begin{pgfonlayer}{nodelayer}
		\node [style=dot] (60) at (-0.5, 7) {};
		\node [style=dot] (61) at (0, 7) {};
		\node [style=dot] (62) at (1, 8) {};
		\node [style=oplus] (63) at (0.5, 8) {};
		\node [style=oplus] (64) at (0.5, 7) {};
		\node [style=dot] (65) at (2, 7) {};
		\node [style=dot] (66) at (1.5, 7) {};
		\node [style=oplus] (67) at (1, 7) {};
		\node [style=oplus] (68) at (1.5, 7.5) {};
		\node [style=dot] (69) at (0, 7.5) {};
		\node [style=X] (70) at (0, 8) {};
		\node [style=X] (71) at (-0.5, 8) {};
		\node [style=X] (72) at (2, 8) {};
		\node [style=X] (73) at (1.5, 8) {};
		\node [style=zeroin] (74) at (0.5, 6) {};
		\node [style=zeroin] (75) at (1, 6) {};
		\node [style=zeroin] (76) at (1.5, 6) {};
		\node [style=none] (77) at (0.5, 8.75) {};
		\node [style=none] (78) at (2, 5.75) {};
		\node [style=none] (79) at (0, 5.75) {};
		\node [style=none] (80) at (-0.5, 5.75) {};
		\node [style=X] (81) at (1, 8.5) {};
		\node [style=dot] (82) at (2, 6.5) {};
		\node [style=dot] (83) at (0, 6.5) {};
		\node [style=oplus] (84) at (1, 6.5) {};
	\end{pgfonlayer}
	\begin{pgfonlayer}{edgelayer}
		\draw (70) to (79.center);
		\draw (80.center) to (71);
		\draw (77.center) to (74);
		\draw (72) to (78.center);
		\draw (76) to (73);
		\draw (65) to (67);
		\draw (62) to (63);
		\draw (64) to (60);
		\draw (69) to (68);
		\draw (81) to (62);
		\draw (62) to (67);
		\draw (67) to (75);
		\draw (82) to (84);
		\draw (84) to (83);
	\end{pgfonlayer}
\end{tikzpicture}\\
&\eq{Lem. \ref{lemma:Iwama}}
\begin{tikzpicture}
	\begin{pgfonlayer}{nodelayer}
		\node [style=dot] (61) at (-0.5, 7.5) {};
		\node [style=dot] (62) at (0, 7.5) {};
		\node [style=dot] (63) at (1, 8) {};
		\node [style=oplus] (64) at (0.5, 8) {};
		\node [style=oplus] (65) at (0.5, 7.5) {};
		\node [style=dot] (66) at (2, 7.5) {};
		\node [style=dot] (67) at (1.5, 7.5) {};
		\node [style=oplus] (68) at (1, 7.5) {};
		\node [style=oplus] (69) at (1.5, 6.5) {};
		\node [style=dot] (70) at (0, 6.5) {};
		\node [style=X] (71) at (0, 8) {};
		\node [style=X] (72) at (-0.5, 8) {};
		\node [style=X] (73) at (2, 8) {};
		\node [style=X] (74) at (1.5, 8) {};
		\node [style=zeroin] (75) at (0.5, 7) {};
		\node [style=zeroin] (76) at (1, 7) {};
		\node [style=zeroin] (77) at (1.5, 6) {};
		\node [style=none] (78) at (0.5, 8.75) {};
		\node [style=none] (79) at (2, 5.75) {};
		\node [style=none] (80) at (0, 5.75) {};
		\node [style=none] (81) at (-0.5, 5.75) {};
		\node [style=X] (82) at (1, 8.5) {};
	\end{pgfonlayer}
	\begin{pgfonlayer}{edgelayer}
		\draw (71) to (80.center);
		\draw (81.center) to (72);
		\draw (78.center) to (75);
		\draw (73) to (79.center);
		\draw (77) to (74);
		\draw (66) to (68);
		\draw (63) to (64);
		\draw (65) to (61);
		\draw (70) to (69);
		\draw (82) to (63);
		\draw (63) to (68);
		\draw (68) to (76);
	\end{pgfonlayer}
\end{tikzpicture}
=
\left\llbracket
\begin{tikzpicture}
	\begin{pgfonlayer}{nodelayer}
		\node [style=none] (62) at (0.25, 6.25) {};
		\node [style=andin] (63) at (-0.35, 7) {};
		\node [style=none] (64) at (-0.25, 6.25) {};
		\node [style=andin] (65) at (0.35, 7) {};
		\node [style=none] (66) at (-0.25, 6.25) {};
		\node [style=X] (67) at (-0.25, 6.25) {};
		\node [style=Z] (68) at (0, 7.75) {};
		\node [style=none] (69) at (0, 8.25) {};
		\node [style=none] (70) at (-0.25, 5.75) {};
		\node [style=none] (71) at (0.5, 5.75) {};
		\node [style=none] (72) at (0.25, 5.75) {};
	\end{pgfonlayer}
	\begin{pgfonlayer}{edgelayer}
		\draw [in=-72, out=90] (62.center) to (63.center);
		\draw [in=120, out=-108] (63.center) to (64.center);
		\draw [in=45, out=-108] (65.center) to (66.center);
		\draw (62.center) to (72.center);
		\draw (70.center) to (64.center);
		\draw [in=-117, out=90] (63.center) to (68);
		\draw (68) to (69.center);
		\draw [in=90, out=-63] (68) to (65.center);
		\draw [in=-75, out=90, looseness=1.25] (71.center) to (65.center);
	\end{pgfonlayer}
\end{tikzpicture}
\right\rrbracket_{\ZXA}
\end{align*}


\end{enumerate}

\end{proof}



To prove functoriality in the other direction, we prove some basic properties of $\ZXA$.

\begin{lemma}
\label{lem:blackdot}
$$
\begin{tikzpicture}
	\begin{pgfonlayer}{nodelayer}
		\node [style=Z] (23) at (0, 5.25) {};
	\end{pgfonlayer}
\end{tikzpicture}
=
\begin{tikzpicture}
	\begin{pgfonlayer}{nodelayer}
		\node [style=none] (0) at (0, -0) {};
	\end{pgfonlayer}
\end{tikzpicture}
$$
\end{lemma}
\begin{proof}
\begin{align*}
\begin{tikzpicture}
	\begin{pgfonlayer}{nodelayer}
		\node [style=Z] (0) at (0, -0) {};
	\end{pgfonlayer}
\end{tikzpicture}
\eq{\ref{ZXA.1}}
\begin{tikzpicture}
	\begin{pgfonlayer}{nodelayer}
		\node [style=Z] (1) at (-0.5, -0.75) {};
		\node [style=Z] (2) at (0, -0.75) {};
	\end{pgfonlayer}
	\begin{pgfonlayer}{edgelayer}
		\draw [in=90, out=90, looseness=2.25] (2) to (1);
	\end{pgfonlayer}
\end{tikzpicture}
\eq{\ref{ZXA.3}}
\begin{tikzpicture}
	\begin{pgfonlayer}{nodelayer}
		\node [style=X] (2) at (0, 0) {};
		\node [style=Z] (3) at (-0.25, -0.75) {};
		\node [style=Z] (4) at (0.25, -0.75) {};
		\node [style=X] (5) at (0, 0.75) {};
	\end{pgfonlayer}
	\begin{pgfonlayer}{edgelayer}
		\draw (5) to (2);
		\draw [in=90, out=-124] (2) to (3);
		\draw [in=-56, out=90] (4) to (2);
	\end{pgfonlayer}
\end{tikzpicture}
\eq{\ref{ZXA.6}}
\begin{tikzpicture}
	\begin{pgfonlayer}{nodelayer}
		\node [style=X] (3) at (0, 0.75) {};
		\node [style=X] (4) at (0, 0) {};
		\node [style=Z] (5) at (0, -1.75) {};
		\node [style=X] (6) at (0, -1) {};
	\end{pgfonlayer}
	\begin{pgfonlayer}{edgelayer}
		\draw (3) to (4);
		\draw (5) to (6);
		\draw [bend left, looseness=1.25] (6) to (4);
		\draw [bend left, looseness=1.25] (4) to (6);
	\end{pgfonlayer}
\end{tikzpicture}
\eq{\ref{ZXA.3}}
\begin{tikzpicture}
	\begin{pgfonlayer}{nodelayer}
		\node [style=X] (4) at (0, 6.75) {};
		\node [style=Z] (5) at (0, 5.75) {};
	\end{pgfonlayer}
	\begin{pgfonlayer}{edgelayer}
		\draw (5) to (4);
	\end{pgfonlayer}
\end{tikzpicture}
\eq{\ref{ZXA.7}}
\end{align*}
\end{proof}




\begin{lemma}
The phase fusion of the black spider in $\ZXA$, 
$$
\begin{tikzpicture}
	\begin{pgfonlayer}{nodelayer}
		\node [style=Z] (5) at (-0.5, 5.75) {$\pi$};
		\node [style=Z] (6) at (0, 5.75) {$\pi$};
		\node [style=Z] (7) at (-0.25, 6.5) {};
		\node [style=none] (8) at (-0.25, 7) {};
	\end{pgfonlayer}
	\begin{pgfonlayer}{edgelayer}
		\draw [in=-108, out=90] (5) to (7);
		\draw [in=90, out=-72] (7) to (6);
		\draw (7) to (8.center);
	\end{pgfonlayer}
\end{tikzpicture}
=
\begin{tikzpicture}
	\begin{pgfonlayer}{nodelayer}
		\node [style=Z] (6) at (-0.25, 5.75) {};
		\node [style=none] (7) at (-0.25, 6.25) {};
	\end{pgfonlayer}
	\begin{pgfonlayer}{edgelayer}
		\draw (6) to (7.center);
	\end{pgfonlayer}
\end{tikzpicture}
$$
in the presence of the other axioms is equivalent to asserting:
$$
\begin{tikzpicture}
	\begin{pgfonlayer}{nodelayer}
		\node [style=Z] (7) at (-0.5, 5.75) {$\pi$};
		\node [style=X] (8) at (-0.5, 6.5) {};
	\end{pgfonlayer}
	\begin{pgfonlayer}{edgelayer}
		\draw (8) to (7);
	\end{pgfonlayer}
\end{tikzpicture}
=
\begin{tikzpicture}
	\begin{pgfonlayer}{nodelayer}
		\node [style=none] (8) at (-0.5, 5.75) {};
	\end{pgfonlayer}
\end{tikzpicture}
$$
Or in other terms, the phase fusion of the black spider is equivalent to the interaction of the unit for and and the counit for copying as a bialgebra.
\end{lemma}

\begin{proof}
For the one direction, suppose that phase fusion holds:


\begin{align*}
\begin{tikzpicture}
	\begin{pgfonlayer}{nodelayer}
		\node [style=Z] (9) at (-0.5, 6) {$\pi$};
		\node [style=X] (10) at (-0.5, 6.75) {};
	\end{pgfonlayer}
	\begin{pgfonlayer}{edgelayer}
		\draw (10) to (9);
	\end{pgfonlayer}
\end{tikzpicture}
\eq{\ref{ZXA.3}}
\begin{tikzpicture}
	\begin{pgfonlayer}{nodelayer}
		\node [style=Z] (10) at (-0.5, 6) {$\pi$};
		\node [style=X] (11) at (-0.5, 6.75) {};
	\end{pgfonlayer}
	\begin{pgfonlayer}{edgelayer}
		\draw (11) to (10);
		\draw [in=135, out=45, loop] (11) to ();
	\end{pgfonlayer}
\end{tikzpicture}
\eq{\ref{ZXA.1}}
\begin{tikzpicture}
	\begin{pgfonlayer}{nodelayer}
		\node [style=Z] (11) at (-0.5, 6) {$\pi$};
		\node [style=X] (12) at (-0.5, 6.75) {};
		\node [style=Z] (13) at (-0.5, 7.75) {};
		\node [style=Z] (14) at (-0.5, 8.5) {};
	\end{pgfonlayer}
	\begin{pgfonlayer}{edgelayer}
		\draw (12) to (11);
		\draw [bend left=45, looseness=1.25] (12) to (13);
		\draw [bend left, looseness=1.25] (13) to (12);
		\draw (13) to (14);
	\end{pgfonlayer}
\end{tikzpicture}
\eq{\ref{ZXA.8}}
\begin{tikzpicture}
	\begin{pgfonlayer}{nodelayer}
		\node [style=Z] (12) at (-1, 6) {$\pi$};
		\node [style=Z] (13) at (-0.5, 7) {};
		\node [style=Z] (14) at (-0.5, 7.75) {};
		\node [style=Z] (15) at (0, 6) {$\pi$};
	\end{pgfonlayer}
	\begin{pgfonlayer}{edgelayer}
		\draw (13) to (14);
		\draw [in=90, out=-117] (13) to (12);
		\draw [in=-63, out=90] (15) to (13);
	\end{pgfonlayer}
\end{tikzpicture}
=
\begin{tikzpicture}
	\begin{pgfonlayer}{nodelayer}
		\node [style=Z] (13) at (-0.5, 6) {};
		\node [style=Z] (14) at (-0.5, 6.75) {};
	\end{pgfonlayer}
	\begin{pgfonlayer}{edgelayer}
		\draw (13) to (14);
	\end{pgfonlayer}
\end{tikzpicture}
\eq{\ref{ZXA.7},\ref{lem:blackdot}}
\begin{tikzpicture}
	\begin{pgfonlayer}{nodelayer}
		\node [style=none] (14) at (-0.5, 6) {};
	\end{pgfonlayer}
\end{tikzpicture}
\end{align*}


Conversely if the unit part of the bialgebra rule holds:

\begin{align*}
\begin{tikzpicture}
	\begin{pgfonlayer}{nodelayer}
		\node [style=Z] (15) at (-0.5, 6) {$\pi$};
		\node [style=Z] (16) at (0, 6) {$\pi$};
		\node [style=Z] (17) at (-0.25, 6.75) {};
		\node [style=none] (18) at (-0.25, 7.25) {};
	\end{pgfonlayer}
	\begin{pgfonlayer}{edgelayer}
		\draw [in=-108, out=90] (15) to (17);
		\draw [in=90, out=-72] (17) to (16);
		\draw (17) to (18.center);
	\end{pgfonlayer}
\end{tikzpicture}
\eq{\ref{ZXA.14}}
\begin{tikzpicture}
	\begin{pgfonlayer}{nodelayer}
		\node [style=Z] (16) at (-0.25, 7.5) {};
		\node [style=none] (17) at (-0.25, 8) {};
		\node [style=Z] (18) at (-0.25, 6) {$\pi$};
		\node [style=X] (19) at (-0.25, 6.75) {};
	\end{pgfonlayer}
	\begin{pgfonlayer}{edgelayer}
		\draw (16) to (17.center);
		\draw [in=120, out=-120, looseness=1.25] (16) to (19);
		\draw (19) to (18);
		\draw [in=-60, out=60, looseness=1.25] (19) to (16);
	\end{pgfonlayer}
\end{tikzpicture}
\eq{\ref{ZXA.8}}
\begin{tikzpicture}
	\begin{pgfonlayer}{nodelayer}
		\node [style=Z] (17) at (-0.25, 7.5) {};
		\node [style=none] (18) at (-0.25, 8) {};
		\node [style=Z] (19) at (-0.25, 6) {$\pi$};
		\node [style=X] (20) at (-0.25, 6.75) {};
	\end{pgfonlayer}
	\begin{pgfonlayer}{edgelayer}
		\draw (17) to (18.center);
		\draw (20) to (19);
	\end{pgfonlayer}
\end{tikzpicture}
=
\begin{tikzpicture}
	\begin{pgfonlayer}{nodelayer}
		\node [style=Z] (18) at (-0.25, 6) {};
		\node [style=none] (19) at (-0.25, 6.5) {};
	\end{pgfonlayer}
	\begin{pgfonlayer}{edgelayer}
		\draw (18) to (19.center);
	\end{pgfonlayer}
\end{tikzpicture}
\end{align*}


\end{proof}


\begin{lemma}
\label{lem:oldaxiom}
$$
\begin{tikzpicture}
	\begin{pgfonlayer}{nodelayer}
		\node [style=Z] (19) at (0, 6.25) {};
		\node [style=none] (20) at (0.5, 7.25) {};
		\node [style=andin] (21) at (0.5, 7.25) {};
		\node [style=none] (22) at (0.5, 8) {};
		\node [style=none] (23) at (1, 6.25) {};
		\node [style=none] (24) at (1, 6) {};
	\end{pgfonlayer}
	\begin{pgfonlayer}{edgelayer}
		\draw [in=90, out=-63] (20.center) to (23.center);
		\draw (22.center) to (20.center);
		\draw [in=90, out=-117] (20.center) to (19);
		\draw (24.center) to (23.center);
	\end{pgfonlayer}
\end{tikzpicture}
=
\begin{tikzpicture}
	\begin{pgfonlayer}{nodelayer}
		\node [style=Z] (20) at (0.5, 7.5) {};
		\node [style=none] (21) at (0.5, 8.25) {};
		\node [style=none] (22) at (0.5, 6) {};
		\node [style=X] (23) at (0.5, 6.75) {};
	\end{pgfonlayer}
	\begin{pgfonlayer}{edgelayer}
		\draw (21.center) to (20);
		\draw (22.center) to (23);
	\end{pgfonlayer}
\end{tikzpicture}
$$
\end{lemma}

\begin{proof}
$$
\begin{tikzpicture}
	\begin{pgfonlayer}{nodelayer}
		\node [style=Z] (21) at (0, 6.25) {};
		\node [style=none] (22) at (0.5, 7.25) {};
		\node [style=andin] (23) at (0.5, 7.25) {};
		\node [style=none] (24) at (0.5, 8) {};
		\node [style=none] (25) at (1, 6.25) {};
		\node [style=none] (26) at (1, 6) {};
	\end{pgfonlayer}
	\begin{pgfonlayer}{edgelayer}
		\draw [in=90, out=-63] (22.center) to (25.center);
		\draw (24.center) to (22.center);
		\draw [in=90, out=-117] (22.center) to (21);
		\draw (26.center) to (25.center);
	\end{pgfonlayer}
\end{tikzpicture}
\eq{\ref{ZXA.1}}
\begin{tikzpicture}
	\begin{pgfonlayer}{nodelayer}
		\node [style=Z] (22) at (0, 7) {};
		\node [style=none] (23) at (0.5, 8.75) {};
		\node [style=none] (24) at (1, 6) {};
		\node [style=none] (25) at (1, 7) {};
		\node [style=andin] (26) at (0.5, 8) {};
		\node [style=none] (27) at (0.5, 8) {};
		\node [style=Z] (28) at (-0.25, 6.25) {$\pi$};
		\node [style=Z] (29) at (0.25, 6.25) {$\pi$};
	\end{pgfonlayer}
	\begin{pgfonlayer}{edgelayer}
		\draw (24.center) to (25.center);
		\draw (23.center) to (27.center);
		\draw [in=90, out=-63] (27.center) to (25.center);
		\draw [in=90, out=-117] (27.center) to (22);
		\draw [in=90, out=-108] (22) to (28);
		\draw [in=-72, out=90] (29) to (22);
	\end{pgfonlayer}
\end{tikzpicture}
\eq{\ref{ZXA.17}}
\begin{tikzpicture}
	\begin{pgfonlayer}{nodelayer}
		\node [style=none] (23) at (-2.25, 7.5) {};
		\node [style=none] (24) at (-0.75, 7.5) {};
		\node [style=X] (25) at (-0.75, 6.5) {};
		\node [style=none] (26) at (-1.5, 9) {};
		\node [style=none] (27) at (-0.75, 6) {};
		\node [style=Z] (28) at (-1.5, 6.25) {$\pi$};
		\node [style=Z] (29) at (-2.25, 6.25) {$\pi$};
		\node [style=Z] (30) at (-1.5, 8.25) {};
		\node [style=andin] (31) at (-2.25, 7.5) {};
		\node [style=andin] (32) at (-0.75, 7.5) {};
	\end{pgfonlayer}
	\begin{pgfonlayer}{edgelayer}
		\draw (26.center) to (30);
		\draw [in=90, out=-150] (30) to (23.center);
		\draw (23.center) to (29);
		\draw [in=90, out=-121] (24.center) to (28);
		\draw [in=-30, out=90] (24.center) to (30);
		\draw [in=146, out=-45] (23.center) to (25);
		\draw (25) to (24.center);
		\draw (25) to (27.center);
	\end{pgfonlayer}
\end{tikzpicture}
\eq{\ref{ZXA.10}}
\begin{tikzpicture}
	\begin{pgfonlayer}{nodelayer}
		\node [style=none] (24) at (1, 6) {};
		\node [style=X] (25) at (1, 6.75) {};
		\node [style=none] (26) at (1, 8.5) {};
		\node [style=Z] (27) at (1, 7.75) {};
	\end{pgfonlayer}
	\begin{pgfonlayer}{edgelayer}
		\draw (24.center) to (25);
		\draw (27) to (26.center);
		\draw [in=120, out=-120, looseness=1.25] (27) to (25);
		\draw [in=-60, out=60, looseness=1.25] (25) to (27);
	\end{pgfonlayer}
\end{tikzpicture}
\eq{\ref{ZXA.8}}
\begin{tikzpicture}
	\begin{pgfonlayer}{nodelayer}
		\node [style=Z] (25) at (0.5, 7.5) {};
		\node [style=none] (26) at (0.5, 8.25) {};
		\node [style=none] (27) at (0.5, 6) {};
		\node [style=X] (28) at (0.5, 6.75) {};
	\end{pgfonlayer}
	\begin{pgfonlayer}{edgelayer}
		\draw (26.center) to (25);
		\draw (27.center) to (28);
	\end{pgfonlayer}
\end{tikzpicture}
$$
\end{proof}


\begin{proposition}
\label{prop:ZXATOF}
Consider the interpretation $\llbracket\_\rrbracket_{\hat \TOF}:\hat \TOF\to \ZXA$ taking:

\begin{center}
\begin{tabular}{c}
$
\begin{tikzpicture}
	\begin{pgfonlayer}{nodelayer}
		\node [style=dot] (63) at (0, 6.5) {};
		\node [style=oplus] (64) at (0.5, 6.5) {};
		\node [style=dot] (65) at (-0.5, 6.5) {};
		\node [style=none] (66) at (0.5, 7.25) {};
		\node [style=none] (67) at (0, 7.25) {};
		\node [style=none] (68) at (-0.5, 7.25) {};
		\node [style=none] (69) at (-0.5, 5.75) {};
		\node [style=none] (70) at (0, 5.75) {};
		\node [style=none] (71) at (0.5, 5.75) {};
	\end{pgfonlayer}
	\begin{pgfonlayer}{edgelayer}
		\draw [style=simple] (66.center) to (64);
		\draw [style=simple] (64) to (63);
		\draw [style=simple] (63) to (65);
		\draw [style=simple] (65) to (68.center);
		\draw [style=simple] (67.center) to (63);
		\draw [style=simple] (63) to (70.center);
		\draw [style=simple] (69.center) to (65);
		\draw [style=simple] (64) to (71.center);
	\end{pgfonlayer}
\end{tikzpicture}
\mapsto
\begin{tikzpicture}
	\begin{pgfonlayer}{nodelayer}
		\node [style=none] (64) at (0, 5.75) {};
		\node [style=none] (65) at (1, 5.75) {};
		\node [style=none] (66) at (1.5, 5.75) {};
		\node [style=Z] (67) at (1.5, 7.75) {};
		\node [style=X] (68) at (1, 6.25) {};
		\node [style=X] (69) at (0, 6.25) {};
		\node [style=andin] (70) at (0.5, 7.25) {};
		\node [style=none] (71) at (0, 8.5) {};
		\node [style=none] (72) at (1.5, 8.5) {};
		\node [style=none] (73) at (1, 8.5) {};
	\end{pgfonlayer}
	\begin{pgfonlayer}{edgelayer}
		\draw [style=simple, in=90, out=180, looseness=0.75] (67) to (70.center);
		\draw [style=simple, in=45, out=-120] (70.center) to (69);
		\draw [style=simple] (69) to (64.center);
		\draw [style=simple] (65.center) to (68);
		\draw [style=simple] (67) to (66.center);
		\draw [style=simple] (73.center) to (68);
		\draw [style=simple] (72.center) to (67);
		\draw [style=simple, in=-60, out=135] (68) to (70.center);
		\draw [style=simple] (71.center) to (69);
	\end{pgfonlayer}
\end{tikzpicture}
\hspace*{.5cm}
\begin{tikzpicture}
	\begin{pgfonlayer}{nodelayer}
		\node [style=onein] (67) at (0, 5.75) {};
		\node [style=none] (68) at (0, 6.75) {};
	\end{pgfonlayer}
	\begin{pgfonlayer}{edgelayer}
		\draw [style=simple] (68.center) to (67);
	\end{pgfonlayer}
\end{tikzpicture}
\mapsto
\begin{tikzpicture}
	\begin{pgfonlayer}{nodelayer}
		\node [style=Z] (68) at (0, 5.75) {$\pi$};
		\node [style=none] (69) at (0, 6.75) {};
	\end{pgfonlayer}
	\begin{pgfonlayer}{edgelayer}
		\draw [style=simple] (69.center) to (68);
	\end{pgfonlayer}
\end{tikzpicture}
\hspace*{.5cm}
\begin{tikzpicture}
	\begin{pgfonlayer}{nodelayer}
		\node [style=oneout] (70) at (0, 6.75) {};
		\node [style=none] (71) at (0, 5.75) {};
	\end{pgfonlayer}
	\begin{pgfonlayer}{edgelayer}
		\draw [style=simple] (71.center) to (70);
	\end{pgfonlayer}
\end{tikzpicture}
\mapsto
\begin{tikzpicture}
	\begin{pgfonlayer}{nodelayer}
		\node [style=Z] (0) at (0, 1.5) {$\pi$};
		\node [style=none] (1) at (0, 0.5) {};
	\end{pgfonlayer}
	\begin{pgfonlayer}{edgelayer}
		\draw [style=simple] (1.center) to (0);
	\end{pgfonlayer}
\end{tikzpicture}
\hspace*{.5cm}
\begin{tikzpicture}
	\begin{pgfonlayer}{nodelayer}
		\node [style=X] (1) at (0, 0) {};
		\node [style=none] (2) at (0, 1) {};
	\end{pgfonlayer}
	\begin{pgfonlayer}{edgelayer}
		\draw [style=simple] (2.center) to (1);
	\end{pgfonlayer}
\end{tikzpicture}
\mapsto
\begin{tikzpicture}
	\begin{pgfonlayer}{nodelayer}
		\node [style=X] (2) at (0, 0) {};
		\node [style=none] (3) at (0, 1) {};
	\end{pgfonlayer}
	\begin{pgfonlayer}{edgelayer}
		\draw [style=simple] (3.center) to (2);
	\end{pgfonlayer}
\end{tikzpicture}
\hspace*{.5cm}
\begin{tikzpicture}
	\begin{pgfonlayer}{nodelayer}
		\node [style=X] (4) at (0, 1) {};
		\node [style=none] (5) at (0, 0) {};
	\end{pgfonlayer}
	\begin{pgfonlayer}{edgelayer}
		\draw [style=simple] (5.center) to (4);
	\end{pgfonlayer}
\end{tikzpicture}
\mapsto
\begin{tikzpicture}
	\begin{pgfonlayer}{nodelayer}
		\node [style=X] (5) at (0, 1) {};
		\node [style=none] (6) at (0, 0) {};
	\end{pgfonlayer}
	\begin{pgfonlayer}{edgelayer}
		\draw [style=simple] (6.center) to (5);
	\end{pgfonlayer}
\end{tikzpicture}
$
\end{tabular}
\end{center}

This interepretation is a strict symmetric \dag-monoidal functor.
\end{proposition}

\begin{proof}
First, observe:
\begin{align*}
\left\llbracket
\begin{tikzpicture}
	\begin{pgfonlayer}{nodelayer}
		\node [style=dot] (6) at (0, 7) {};
		\node [style=oplus] (7) at (0.5, 7) {};
		\node [style=none] (8) at (0.5, 7.5) {};
		\node [style=none] (9) at (0, 7.5) {};
		\node [style=none] (10) at (0, 6.5) {};
		\node [style=none] (11) at (0.5, 6.5) {};
	\end{pgfonlayer}
	\begin{pgfonlayer}{edgelayer}
		\draw (7) to (8.center);
		\draw (7) to (11.center);
		\draw (7) to (6);
		\draw (6) to (9.center);
		\draw (6) to (10.center);
	\end{pgfonlayer}
\end{tikzpicture}
\right\rrbracket_{\hat{\TOF}}
&=
\begin{tikzpicture}
	\begin{pgfonlayer}{nodelayer}
		\node [style=none] (7) at (1, -0.5) {};
		\node [style=none] (8) at (1.5, -0.5) {};
		\node [style=Z] (9) at (1.5, 2) {};
		\node [style=X] (10) at (1, 0.5) {};
		\node [style=X] (11) at (0, 0.5) {};
		\node [style=none] (12) at (0.5, 1.5) {};
		\node [style=none] (13) at (1.5, 2.75) {};
		\node [style=none] (14) at (1, 2.75) {};
		\node [style=Z] (15) at (0, 1.25) {$\pi$};
		\node [style=Z] (16) at (0, -0.25) {$\pi$};
		\node [style=andin] (17) at (0.5, 1.5) {};
	\end{pgfonlayer}
	\begin{pgfonlayer}{edgelayer}
		\draw [style=simple, in=90, out=180, looseness=0.75] (9) to (12.center);
		\draw [style=simple, in=45, out=-120] (12.center) to (11);
		\draw [style=simple] (7.center) to (10);
		\draw [style=simple] (9) to (8.center);
		\draw [style=simple] (14.center) to (10);
		\draw [style=simple] (13.center) to (9);
		\draw [style=simple, in=-60, out=135] (10) to (12.center);
		\draw (15) to (11);
		\draw (11) to (16);
	\end{pgfonlayer}
\end{tikzpicture}
\eq{\ref{ZXA.14}}
\begin{tikzpicture}
	\begin{pgfonlayer}{nodelayer}
		\node [style=none] (8) at (1.5, 8.25) {};
		\node [style=Z] (9) at (0.25, 4.5) {$\pi$};
		\node [style=none] (10) at (1, 4.25) {};
		\node [style=Z] (11) at (1.5, 7.5) {};
		\node [style=X] (12) at (1, 6) {};
		\node [style=none] (13) at (0.5, 7) {};
		\node [style=none] (14) at (1, 8.25) {};
		\node [style=Z] (15) at (0.25, 5.25) {$\pi$};
		\node [style=none] (16) at (1.5, 4.25) {};
		\node [style=Z] (17) at (0, 6) {$\pi$};
		\node [style=andin] (18) at (0.5, 7) {};
	\end{pgfonlayer}
	\begin{pgfonlayer}{edgelayer}
		\draw [style=simple, in=90, out=180, looseness=0.75] (11) to (13.center);
		\draw [style=simple] (10.center) to (12);
		\draw [style=simple] (11) to (16.center);
		\draw [style=simple] (14.center) to (12);
		\draw [style=simple] (8.center) to (11);
		\draw [style=simple, in=-60, out=135] (12) to (13.center);
		\draw [in=90, out=-135] (13.center) to (17);
		\draw (15) to (9);
	\end{pgfonlayer}
\end{tikzpicture}
\eq{\ref{ZXA.1}}
\begin{tikzpicture}
	\begin{pgfonlayer}{nodelayer}
		\node [style=none] (9) at (1.5, 8.25) {};
		\node [style=Z] (10) at (0.5, 5.25) {};
		\node [style=none] (11) at (1, 4.75) {};
		\node [style=Z] (12) at (1.5, 7.5) {};
		\node [style=X] (13) at (1, 6) {};
		\node [style=none] (14) at (0.5, 7) {};
		\node [style=none] (15) at (1, 8.25) {};
		\node [style=none] (16) at (1.5, 4.75) {};
		\node [style=Z] (17) at (0, 6) {$\pi$};
		\node [style=andin] (18) at (0.5, 7) {};
	\end{pgfonlayer}
	\begin{pgfonlayer}{edgelayer}
		\draw [style=simple, in=90, out=180, looseness=0.75] (12) to (14.center);
		\draw [style=simple] (11.center) to (13);
		\draw [style=simple] (12) to (16.center);
		\draw [style=simple] (15.center) to (13);
		\draw [style=simple] (9.center) to (12);
		\draw [style=simple, in=-60, out=135] (13) to (14.center);
		\draw [in=90, out=-135] (14.center) to (17);
	\end{pgfonlayer}
\end{tikzpicture}\\
&
\eq{Lem. \ref{lem:blackdot}, \ref{ZXA.7}}
\begin{tikzpicture}
	\begin{pgfonlayer}{nodelayer}
		\node [style=none] (10) at (1.5, 8.25) {};
		\node [style=none] (11) at (1, 5.25) {};
		\node [style=Z] (12) at (1.5, 7.5) {};
		\node [style=X] (13) at (1, 6) {};
		\node [style=none] (14) at (0.5, 7) {};
		\node [style=none] (15) at (1, 8.25) {};
		\node [style=none] (16) at (1.5, 5.25) {};
		\node [style=Z] (17) at (0, 6) {$\pi$};
		\node [style=andin] (18) at (0.5, 7) {};
	\end{pgfonlayer}
	\begin{pgfonlayer}{edgelayer}
		\draw [style=simple, in=90, out=180, looseness=0.75] (12) to (14.center);
		\draw [style=simple] (11.center) to (13);
		\draw [style=simple] (12) to (16.center);
		\draw [style=simple] (15.center) to (13);
		\draw [style=simple] (10.center) to (12);
		\draw [style=simple, in=-60, out=135] (13) to (14.center);
		\draw [in=90, out=-135] (14.center) to (17);
	\end{pgfonlayer}
\end{tikzpicture}
\eq{\ref{ZXA.10}}
\begin{tikzpicture}
	\begin{pgfonlayer}{nodelayer}
		\node [style=none] (11) at (1.5, 8.25) {};
		\node [style=none] (12) at (1, 5.25) {};
		\node [style=Z] (13) at (1.5, 7.5) {};
		\node [style=X] (14) at (1, 6) {};
		\node [style=none] (15) at (1, 8.25) {};
		\node [style=none] (16) at (1.5, 5.25) {};
	\end{pgfonlayer}
	\begin{pgfonlayer}{edgelayer}
		\draw [style=simple] (12.center) to (14);
		\draw [style=simple] (13) to (16.center);
		\draw [style=simple] (15.center) to (14);
		\draw [style=simple] (11.center) to (13);
		\draw [in=120, out=-135, looseness=1.25] (13) to (14);
	\end{pgfonlayer}
\end{tikzpicture}
\eq{\ref{ZXA.4}}
\begin{tikzpicture}
	\begin{pgfonlayer}{nodelayer}
		\node [style=none] (12) at (1.5, 8.25) {};
		\node [style=none] (13) at (1, 6.75) {};
		\node [style=Z] (14) at (1.5, 7.5) {};
		\node [style=X] (15) at (1, 7.5) {};
		\node [style=none] (16) at (1, 8.25) {};
		\node [style=none] (17) at (1.5, 6.75) {};
	\end{pgfonlayer}
	\begin{pgfonlayer}{edgelayer}
		\draw [style=simple] (13.center) to (15);
		\draw [style=simple] (14) to (17.center);
		\draw [style=simple, in=90, out=-90] (16.center) to (15);
		\draw [style=simple] (12.center) to (14);
		\draw (14) to (15);
	\end{pgfonlayer}
\end{tikzpicture}
\end{align*}

Thus:
\begin{align*}
\left\llbracket
\begin{tikzpicture}
	\begin{pgfonlayer}{nodelayer}
		\node [style=oplus] (13) at (0.5, 7) {};
		\node [style=none] (14) at (0.5, 7.5) {};
		\node [style=none] (15) at (0.5, 6.5) {};
	\end{pgfonlayer}
	\begin{pgfonlayer}{edgelayer}
		\draw (13) to (14.center);
		\draw (13) to (15.center);
	\end{pgfonlayer}
\end{tikzpicture}
\right\rrbracket_{\hat{\TOF}}
&=
\begin{tikzpicture}
	\begin{pgfonlayer}{nodelayer}
		\node [style=none] (14) at (1.5, 1.5) {};
		\node [style=Z] (15) at (1.5, 4) {};
		\node [style=X] (16) at (1, 2.5) {};
		\node [style=X] (17) at (0, 2.5) {};
		\node [style=none] (18) at (0.5, 3.5) {};
		\node [style=none] (19) at (1.5, 4.75) {};
		\node [style=Z] (20) at (0, 3.25) {$\pi$};
		\node [style=Z] (21) at (0, 1.75) {$\pi$};
		\node [style=Z] (22) at (1, 1.75) {$\pi$};
		\node [style=Z] (23) at (1, 3.25) {$\pi$};
		\node [style=andin] (24) at (0.5, 3.5) {};
	\end{pgfonlayer}
	\begin{pgfonlayer}{edgelayer}
		\draw [style=simple, in=90, out=180, looseness=0.75] (15) to (18.center);
		\draw [style=simple, in=45, out=-120] (18.center) to (17);
		\draw [style=simple] (15) to (14.center);
		\draw [style=simple] (19.center) to (15);
		\draw [style=simple, in=-60, out=135] (16) to (18.center);
		\draw (20) to (17);
		\draw (17) to (21);
		\draw (23) to (16);
		\draw (16) to (22);
	\end{pgfonlayer}
\end{tikzpicture}
=
\begin{tikzpicture}
	\begin{pgfonlayer}{nodelayer}
		\node [style=none] (15) at (1.5, 1.5) {};
		\node [style=Z] (16) at (1.5, 2.25) {};
		\node [style=X] (17) at (1, 2.25) {};
		\node [style=none] (18) at (1.5, 3) {};
		\node [style=Z] (19) at (1, 1.75) {$\pi$};
		\node [style=Z] (20) at (1, 2.75) {$\pi$};
	\end{pgfonlayer}
	\begin{pgfonlayer}{edgelayer}
		\draw [style=simple] (16) to (15.center);
		\draw [style=simple] (18.center) to (16);
		\draw (20) to (17);
		\draw (17) to (19);
		\draw (16) to (17);
	\end{pgfonlayer}
\end{tikzpicture}
\eq{\ref{ZXA.14}}
\begin{tikzpicture}
	\begin{pgfonlayer}{nodelayer}
		\node [style=none] (16) at (1.5, 1.5) {};
		\node [style=Z] (17) at (1.5, 2.5) {};
		\node [style=none] (18) at (1.5, 4) {};
		\node [style=Z] (19) at (1, 3) {$\pi$};
		\node [style=Z] (20) at (1, 3.75) {$\pi$};
		\node [style=Z] (21) at (1, 2) {$\pi$};
	\end{pgfonlayer}
	\begin{pgfonlayer}{edgelayer}
		\draw [style=simple] (17) to (16.center);
		\draw [style=simple] (18.center) to (17);
		\draw (20) to (19);
		\draw [in=90, out=180, looseness=1.25] (17) to (21);
	\end{pgfonlayer}
\end{tikzpicture}
\eq{\ref{ZXA.1}}
\begin{tikzpicture}
	\begin{pgfonlayer}{nodelayer}
		\node [style=none] (17) at (1.5, 1.5) {};
		\node [style=Z] (18) at (1.5, 2) {$\pi$};
		\node [style=none] (19) at (1.5, 2.5) {};
		\node [style=Z] (20) at (1, 2) {};
	\end{pgfonlayer}
	\begin{pgfonlayer}{edgelayer}
		\draw [style=simple] (18) to (17.center);
		\draw [style=simple] (19.center) to (18);
	\end{pgfonlayer}
\end{tikzpicture}
\eq{Lem. \ref{lem:blackdot}, \ref{ZXA.7}}
\begin{tikzpicture}
	\begin{pgfonlayer}{nodelayer}
		\node [style=none] (18) at (1.5, 2) {};
		\node [style=Z] (19) at (1.5, 2.5) {$\pi$};
		\node [style=none] (20) at (1.5, 3) {};
	\end{pgfonlayer}
	\begin{pgfonlayer}{edgelayer}
		\draw [style=simple] (19) to (18.center);
		\draw [style=simple] (20.center) to (19);
	\end{pgfonlayer}
\end{tikzpicture}
\end{align*}

Thus:
\begin{align*}
\left\llbracket
\begin{tikzpicture}
	\begin{pgfonlayer}{nodelayer}
		\node [style=none] (19) at (1.5, 5.25) {};
		\node [style=none] (20) at (1.5, 8.25) {};
		\node [style=Z] (21) at (1.5, 7.5) {};
		\node [style=X] (22) at (1, 6) {};
		\node [style=Z] (23) at (0, 6) {$\pi$};
		\node [style=andin] (24) at (0.5, 7) {};
		\node [style=none] (25) at (0.5, 7) {};
		\node [style=none] (26) at (1, 5.25) {};
		\node [style=none] (27) at (1, 8.25) {};
	\end{pgfonlayer}
	\begin{pgfonlayer}{edgelayer}
		\draw [style=simple, in=90, out=180, looseness=0.75] (21) to (25.center);
		\draw [style=simple] (26.center) to (22);
		\draw [style=simple] (21) to (19.center);
		\draw [style=simple] (27.center) to (22);
		\draw [style=simple] (20.center) to (21);
		\draw [style=simple, in=-60, out=135] (22) to (25.center);
		\draw [in=90, out=-135] (25.center) to (23);
	\end{pgfonlayer}
\end{tikzpicture}
\right\rrbracket_{\hat{\TOF}}
=
\begin{tikzpicture}
	\begin{pgfonlayer}{nodelayer}
		\node [style=Z] (20) at (1.5, 4.25) {};
		\node [style=X] (21) at (1, 2.75) {};
		\node [style=X] (22) at (0, 2.75) {};
		\node [style=none] (23) at (0.5, 3.75) {};
		\node [style=none] (24) at (1.5, 5) {};
		\node [style=Z] (25) at (0, 3.5) {$\pi$};
		\node [style=Z] (26) at (0, 2) {$\pi$};
		\node [style=Z] (27) at (1, 2) {$\pi$};
		\node [style=Z] (28) at (1, 3.5) {$\pi$};
		\node [style=andin] (29) at (0.5, 3.75) {};
		\node [style=Z] (30) at (1.5, 2) {$\pi$};
	\end{pgfonlayer}
	\begin{pgfonlayer}{edgelayer}
		\draw [style=simple, in=90, out=180, looseness=0.75] (20) to (23.center);
		\draw [style=simple, in=45, out=-120] (23.center) to (22);
		\draw [style=simple] (24.center) to (20);
		\draw [style=simple, in=-60, out=135] (21) to (23.center);
		\draw (25) to (22);
		\draw (22) to (26);
		\draw (28) to (21);
		\draw (21) to (27);
		\draw (20) to (30);
	\end{pgfonlayer}
\end{tikzpicture}
=
\begin{tikzpicture}
	\begin{pgfonlayer}{nodelayer}
		\node [style=Z] (21) at (1.5, 2.75) {$\pi$};
		\node [style=none] (22) at (1.5, 3.5) {};
		\node [style=Z] (23) at (1.5, 2) {$\pi$};
	\end{pgfonlayer}
	\begin{pgfonlayer}{edgelayer}
		\draw [style=simple] (22.center) to (21);
		\draw (21) to (23);
	\end{pgfonlayer}
\end{tikzpicture}
\eq{\ref{ZXA.1}}
\begin{tikzpicture}
	\begin{pgfonlayer}{nodelayer}
		\node [style=Z] (22) at (1.5, 2) {};
		\node [style=none] (23) at (1.5, 2.75) {};
	\end{pgfonlayer}
	\begin{pgfonlayer}{edgelayer}
		\draw [style=simple] (23.center) to (22);
	\end{pgfonlayer}
\end{tikzpicture}
\end{align*}


We prove that all of the axioms of $\hat \TOF$ hold in $\ZXA$ :
\begin{enumerate}
\item[\ref{TOF.1}:]
\begin{align*}
\left\llbracket
\begin{tikzpicture}
	\begin{pgfonlayer}{nodelayer}
		\node [style=nothing] (23) at (0, 2) {};
		\node [style=nothing] (24) at (-0.5, 2) {};
		\node [style=oplus] (25) at (0, 2.5) {};
		\node [style=dot] (26) at (-0.5, 2.5) {};
		\node [style=dot] (27) at (-1, 2.5) {};
		\node [style=onein] (28) at (-1, 2) {};
		\node [style=nothing] (29) at (-1, 3) {};
		\node [style=nothing] (30) at (-0.5, 3) {};
		\node [style=nothing] (31) at (0, 3) {};
	\end{pgfonlayer}
	\begin{pgfonlayer}{edgelayer}
		\draw (28) to (27);
		\draw (27) to (29);
		\draw (30) to (26);
		\draw (24) to (26);
		\draw (23) to (25);
		\draw (25) to (31);
		\draw (25) to (26);
		\draw (26) to (27);
	\end{pgfonlayer}
\end{tikzpicture}
\right\rrbracket_{\hat{\TOF}}
=
\begin{tikzpicture}
	\begin{pgfonlayer}{nodelayer}
		\node [style=andin] (24) at (-0.5, 4.25) {};
		\node [style=X] (25) at (-1, 3.5) {};
		\node [style=X] (26) at (0, 3.5) {};
		\node [style=Z] (27) at (0.5, 5) {};
		\node [style=none] (28) at (0.5, 2.5) {};
		\node [style=none] (29) at (0.5, 5.75) {};
		\node [style=none] (30) at (0, 5.75) {};
		\node [style=none] (31) at (-1, 5.75) {};
		\node [style=Z] (32) at (-1, 2.75) {$\pi$};
		\node [style=none] (33) at (0, 2.5) {};
	\end{pgfonlayer}
	\begin{pgfonlayer}{edgelayer}
		\draw (29.center) to (27);
		\draw (27) to (28.center);
		\draw (33.center) to (26);
		\draw [in=90, out=180, looseness=0.75] (27) to (24.center);
		\draw (24.center) to (25);
		\draw (24.center) to (26);
		\draw (25) to (32);
		\draw (25) to (31.center);
		\draw (30.center) to (26);
	\end{pgfonlayer}
\end{tikzpicture}
\eq{\ref{ZXA.14}}
\begin{tikzpicture}
	\begin{pgfonlayer}{nodelayer}
		\node [style=andin] (25) at (-0.5, 4.25) {};
		\node [style=X] (26) at (0, 3.5) {};
		\node [style=Z] (27) at (0.5, 5) {};
		\node [style=none] (28) at (0.5, 2.5) {};
		\node [style=none] (29) at (0.5, 5.75) {};
		\node [style=none] (30) at (0, 5.75) {};
		\node [style=none] (31) at (-1, 5.75) {};
		\node [style=Z] (32) at (-1, 3.5) {$\pi$};
		\node [style=none] (33) at (0, 2.5) {};
		\node [style=Z] (34) at (-1, 5) {$\pi$};
	\end{pgfonlayer}
	\begin{pgfonlayer}{edgelayer}
		\draw (29.center) to (27);
		\draw (27) to (28.center);
		\draw (33.center) to (26);
		\draw [in=90, out=180, looseness=0.75] (27) to (25.center);
		\draw (25.center) to (26);
		\draw (30.center) to (26);
		\draw (31.center) to (34);
		\draw [in=-124, out=90] (32) to (25.center);
	\end{pgfonlayer}
\end{tikzpicture}
\eq{\ref{ZXA.10}}
\begin{tikzpicture}
	\begin{pgfonlayer}{nodelayer}
		\node [style=X] (26) at (0, 4) {};
		\node [style=Z] (27) at (0.5, 5) {};
		\node [style=none] (28) at (0.5, 3.5) {};
		\node [style=none] (29) at (0.5, 5.5) {};
		\node [style=none] (30) at (0, 5.5) {};
		\node [style=none] (31) at (-0.5, 5.5) {};
		\node [style=none] (32) at (0, 3.5) {};
		\node [style=Z] (33) at (-0.5, 5) {$\pi$};
	\end{pgfonlayer}
	\begin{pgfonlayer}{edgelayer}
		\draw (29.center) to (27);
		\draw (27) to (28.center);
		\draw (32.center) to (26);
		\draw (30.center) to (26);
		\draw (31.center) to (33);
		\draw [in=-108, out=120, looseness=1.25] (26) to (27);
	\end{pgfonlayer}
\end{tikzpicture}
\eq{\ref{ZXA.3}}
\begin{tikzpicture}
	\begin{pgfonlayer}{nodelayer}
		\node [style=X] (27) at (0, 5) {};
		\node [style=Z] (28) at (0.5, 5) {};
		\node [style=none] (29) at (0.5, 4.5) {};
		\node [style=none] (30) at (0.5, 5.5) {};
		\node [style=none] (31) at (0, 5.5) {};
		\node [style=none] (32) at (-0.5, 5.5) {};
		\node [style=none] (33) at (0, 4.5) {};
		\node [style=Z] (34) at (-0.5, 5) {$\pi$};
	\end{pgfonlayer}
	\begin{pgfonlayer}{edgelayer}
		\draw (30.center) to (28);
		\draw (28) to (29.center);
		\draw (33.center) to (27);
		\draw (31.center) to (27);
		\draw (32.center) to (34);
		\draw (27) to (28);
	\end{pgfonlayer}
\end{tikzpicture}
=
\left\llbracket
\begin{tikzpicture}
	\begin{pgfonlayer}{nodelayer}
		\node [style=nothing] (28) at (0, 2) {};
		\node [style=nothing] (29) at (-0.5, 2) {};
		\node [style=oplus] (30) at (0, 2.5) {};
		\node [style=dot] (31) at (-0.5, 2.5) {};
		\node [style=onein] (32) at (-1, 2.5) {};
		\node [style=nothing] (33) at (-1, 3) {};
		\node [style=nothing] (34) at (-0.5, 3) {};
		\node [style=nothing] (35) at (0, 3) {};
	\end{pgfonlayer}
	\begin{pgfonlayer}{edgelayer}
		\draw (29) to (31);
		\draw (28) to (30);
		\draw (30) to (31);
		\draw (34) to (31);
		\draw (32) to (33);
		\draw (30) to (35);
	\end{pgfonlayer}
\end{tikzpicture}
\right\rrbracket_{\hat{\TOF}}
\end{align*}

\item[\ref{TOF.2}:]
\begin{align*}
\left\llbracket
\begin{tikzpicture}
	\begin{pgfonlayer}{nodelayer}
		\node [style=nothing] (29) at (-1.25, 2) {};
		\node [style=nothing] (30) at (-0.75, 2) {};
		\node [style=nothing] (31) at (-1.75, 4) {};
		\node [style=nothing] (32) at (-1.25, 4) {};
		\node [style=nothing] (33) at (-0.75, 4) {};
		\node [style=dot] (34) at (-1.75, 3) {};
		\node [style=dot] (35) at (-1.25, 3) {};
		\node [style=oplus] (36) at (-0.75, 3) {};
		\node [style=zeroin] (37) at (-1.75, 2) {};
	\end{pgfonlayer}
	\begin{pgfonlayer}{edgelayer}
		\draw (34) to (31);
		\draw (32) to (35);
		\draw (35) to (29);
		\draw (30) to (36);
		\draw (36) to (33);
		\draw (36) to (35);
		\draw (35) to (34);
		\draw (37) to (34);
	\end{pgfonlayer}
\end{tikzpicture}
\right\rrbracket_{\hat{\TOF}}
=
\begin{tikzpicture}
	\begin{pgfonlayer}{nodelayer}
		\node [style=andin] (30) at (-0.5, 4.25) {};
		\node [style=X] (31) at (-1, 3.5) {};
		\node [style=X] (32) at (0, 3.5) {};
		\node [style=Z] (33) at (0.5, 5) {};
		\node [style=none] (34) at (0.5, 2.5) {};
		\node [style=none] (35) at (0.5, 5.75) {};
		\node [style=none] (36) at (0, 5.75) {};
		\node [style=none] (37) at (-1, 5.75) {};
		\node [style=none] (38) at (0, 2.5) {};
		\node [style=Z] (39) at (-1, 2.75) {};
	\end{pgfonlayer}
	\begin{pgfonlayer}{edgelayer}
		\draw (35.center) to (33);
		\draw (33) to (34.center);
		\draw (38.center) to (32);
		\draw [in=90, out=180, looseness=0.75] (33) to (30.center);
		\draw (30.center) to (31);
		\draw (30.center) to (32);
		\draw (31) to (37.center);
		\draw (36.center) to (32);
		\draw (31) to (39);
	\end{pgfonlayer}
\end{tikzpicture}
\eq{\ref{ZXA.6}}
\begin{tikzpicture}
	\begin{pgfonlayer}{nodelayer}
		\node [style=andin] (31) at (-0.5, 4.25) {};
		\node [style=X] (32) at (0, 3.5) {};
		\node [style=Z] (33) at (0.5, 5) {};
		\node [style=none] (34) at (0.5, 3) {};
		\node [style=none] (35) at (0.5, 5.5) {};
		\node [style=none] (36) at (0, 5.5) {};
		\node [style=none] (37) at (-1, 5.5) {};
		\node [style=none] (38) at (0, 3) {};
		\node [style=Z] (39) at (-1, 3.5) {};
		\node [style=Z] (40) at (-1, 5) {};
	\end{pgfonlayer}
	\begin{pgfonlayer}{edgelayer}
		\draw (35.center) to (33);
		\draw (33) to (34.center);
		\draw (38.center) to (32);
		\draw [in=90, out=180, looseness=0.75] (33) to (31.center);
		\draw (31.center) to (32);
		\draw (36.center) to (32);
		\draw [in=90, out=-124] (31.center) to (39);
		\draw (37.center) to (40);
	\end{pgfonlayer}
\end{tikzpicture}
\eq{Lem. \ref{lem:oldaxiom}}
\begin{tikzpicture}
	\begin{pgfonlayer}{nodelayer}
		\node [style=X] (32) at (0, 4.25) {};
		\node [style=Z] (33) at (0.5, 6.25) {};
		\node [style=none] (34) at (0.5, 3.75) {};
		\node [style=none] (35) at (0.5, 6.75) {};
		\node [style=none] (36) at (0, 6.75) {};
		\node [style=none] (37) at (-0.5, 6.75) {};
		\node [style=none] (38) at (0, 3.75) {};
		\node [style=Z] (39) at (-0.5, 6.25) {};
		\node [style=Z] (40) at (-0.5, 5.5) {};
		\node [style=X] (41) at (-0.5, 5) {};
	\end{pgfonlayer}
	\begin{pgfonlayer}{edgelayer}
		\draw (35.center) to (33);
		\draw (33) to (34.center);
		\draw (38.center) to (32);
		\draw (36.center) to (32);
		\draw (37.center) to (39);
		\draw [in=90, out=-143, looseness=0.75] (33) to (40);
		\draw [in=-90, out=124] (32) to (41);
	\end{pgfonlayer}
\end{tikzpicture}
\eq{\ref{ZXA.1}}
\begin{tikzpicture}
	\begin{pgfonlayer}{nodelayer}
		\node [style=X] (33) at (0, 4.25) {};
		\node [style=none] (34) at (0.5, 3.75) {};
		\node [style=none] (35) at (0.5, 6) {};
		\node [style=none] (36) at (0, 6) {};
		\node [style=none] (37) at (-0.5, 6) {};
		\node [style=none] (38) at (0, 3.75) {};
		\node [style=Z] (39) at (-0.5, 5.5) {};
		\node [style=X] (40) at (-0.5, 5) {};
	\end{pgfonlayer}
	\begin{pgfonlayer}{edgelayer}
		\draw (38.center) to (33);
		\draw (36.center) to (33);
		\draw (37.center) to (39);
		\draw [in=-90, out=124] (33) to (40);
		\draw (35.center) to (34.center);
	\end{pgfonlayer}
\end{tikzpicture}
\eq{\ref{ZXA.3}}
\begin{tikzpicture}
	\begin{pgfonlayer}{nodelayer}
		\node [style=none] (34) at (0.5, 3.75) {};
		\node [style=none] (35) at (0.5, 4.75) {};
		\node [style=none] (36) at (0, 4.75) {};
		\node [style=none] (37) at (-0.5, 4.75) {};
		\node [style=none] (38) at (0, 3.75) {};
		\node [style=Z] (39) at (-0.5, 4.25) {};
	\end{pgfonlayer}
	\begin{pgfonlayer}{edgelayer}
		\draw (37.center) to (39);
		\draw (35.center) to (34.center);
		\draw (36.center) to (38.center);
	\end{pgfonlayer}
\end{tikzpicture}
=
\left\llbracket
\begin{tikzpicture}
	\begin{pgfonlayer}{nodelayer}
		\node [style=nothing] (35) at (-1.25, 3.75) {};
		\node [style=nothing] (36) at (-0.75, 3.75) {};
		\node [style=nothing] (37) at (-1.75, 5.25) {};
		\node [style=nothing] (38) at (-1.25, 5.25) {};
		\node [style=nothing] (39) at (-0.75, 5.25) {};
		\node [style=zeroin] (40) at (-1.75, 3.75) {};
	\end{pgfonlayer}
	\begin{pgfonlayer}{edgelayer}
		\draw (40) to (37);
		\draw (35) to (38);
		\draw (36) to (39);
	\end{pgfonlayer}
\end{tikzpicture}
\right\rrbracket_{\hat{\TOF}}
\end{align*}

\item[\ref{TOF.3}:]
This follows from the spider law.

\item[\ref{TOF.4}:]
This follows from the spider law.

\item[\ref{TOF.5}:]
This follows from the spider law.

\item[\ref{TOF.6}:]
This follows from the spider law.

\item[\ref{TOF.7}:]
\begin{align*}
\left\llbracket
\begin{tikzpicture}
	\begin{pgfonlayer}{nodelayer}
		\node [style=nothing] (36) at (0, 3.75) {};
		\node [style=nothing] (37) at (-0.5, 3.75) {};
		\node [style=nothing] (38) at (-0.5, 6.25) {};
		\node [style=nothing] (39) at (0, 6.25) {};
		\node [style=zeroout] (40) at (0.5, 6.25) {};
		\node [style=oplus] (41) at (0.5, 5.75) {};
		\node [style=dot] (42) at (0, 5.75) {};
		\node [style=dot] (43) at (-0.5, 4.25) {};
		\node [style=oplus] (44) at (0.5, 4.25) {};
		\node [style=zeroout] (45) at (0.5, 4.75) {};
		\node [style=onein] (46) at (0.5, 3.75) {};
		\node [style=onein] (47) at (0.5, 5.25) {};
	\end{pgfonlayer}
	\begin{pgfonlayer}{edgelayer}
		\draw (37) to (43);
		\draw (43) to (38);
		\draw (39) to (42);
		\draw (42) to (36);
		\draw (44) to (45);
		\draw (44) to (43);
		\draw (41) to (40);
		\draw (41) to (42);
		\draw (46) to (44);
		\draw (47) to (41);
	\end{pgfonlayer}
\end{tikzpicture}
\right\rrbracket_{\hat{\TOF}}
=
\begin{tikzpicture}
	\begin{pgfonlayer}{nodelayer}
		\node [style=X] (37) at (-0.5, 4.5) {};
		\node [style=X] (38) at (0, 6.25) {};
		\node [style=Z] (39) at (0.5, 6.25) {};
		\node [style=Z] (40) at (0.5, 4.5) {};
		\node [style=Z] (41) at (0.5, 5) {};
		\node [style=Z] (42) at (0.5, 6.75) {};
		\node [style=none] (43) at (0, 3.75) {};
		\node [style=none] (44) at (-0.5, 3.75) {};
		\node [style=none] (45) at (0, 7) {};
		\node [style=none] (46) at (-0.5, 7) {};
		\node [style=Z] (47) at (0.5, 4) {$\pi$};
		\node [style=Z] (48) at (0.5, 5.75) {$\pi$};
	\end{pgfonlayer}
	\begin{pgfonlayer}{edgelayer}
		\draw (46.center) to (44.center);
		\draw (43.center) to (45.center);
		\draw (42) to (48);
		\draw (41) to (47);
		\draw (40) to (37);
		\draw (39) to (38);
	\end{pgfonlayer}
\end{tikzpicture}
\eq{\ref{ZXA.1}}
\begin{tikzpicture}
	\begin{pgfonlayer}{nodelayer}
		\node [style=X] (38) at (-0.5, 4.5) {};
		\node [style=X] (39) at (0, 5.25) {};
		\node [style=none] (40) at (0, 3.75) {};
		\node [style=none] (41) at (-0.5, 3.75) {};
		\node [style=none] (42) at (0, 6) {};
		\node [style=none] (43) at (-0.5, 6) {};
		\node [style=Z] (44) at (0.5, 4.5) {$\pi$};
		\node [style=Z] (45) at (0.5, 5.25) {$\pi$};
	\end{pgfonlayer}
	\begin{pgfonlayer}{edgelayer}
		\draw (43.center) to (41.center);
		\draw (40.center) to (42.center);
		\draw (44) to (38);
		\draw (45) to (39);
	\end{pgfonlayer}
\end{tikzpicture}
\eq{\ref{ZXA.16}}
\begin{tikzpicture}
	\begin{pgfonlayer}{nodelayer}
		\node [style=X] (39) at (-0.5, 4.25) {};
		\node [style=X] (40) at (0, 5) {};
		\node [style=none] (41) at (0, 3.75) {};
		\node [style=none] (42) at (-0.5, 3.75) {};
		\node [style=none] (43) at (0, 6.75) {};
		\node [style=none] (44) at (-0.5, 6.75) {};
		\node [style=Z] (45) at (0.5, 6.5) {$\pi$};
		\node [style=andin] (46) at (0.5, 5.75) {};
		\node [style=none] (47) at (1, 5.25) {};
		\node [style=none] (48) at (1, 4.75) {};
	\end{pgfonlayer}
	\begin{pgfonlayer}{edgelayer}
		\draw (44.center) to (42.center);
		\draw (41.center) to (43.center);
		\draw (45) to (46);
		\draw [in=30, out=-124] (46) to (40);
		\draw [in=90, out=-45] (46) to (47.center);
		\draw (47.center) to (48.center);
		\draw [in=0, out=-90, looseness=0.50] (48.center) to (39);
	\end{pgfonlayer}
\end{tikzpicture}
\eq{\ref{ZXA.1}}
\begin{tikzpicture}
	\begin{pgfonlayer}{nodelayer}
		\node [style=X] (40) at (-0.75, 4.25) {};
		\node [style=X] (41) at (0.25, 4.25) {};
		\node [style=none] (42) at (0.25, 3.75) {};
		\node [style=none] (43) at (-0.75, 3.75) {};
		\node [style=none] (44) at (0.25, 6.75) {};
		\node [style=none] (45) at (-0.75, 6.75) {};
		\node [style=andin] (46) at (-0.25, 5.25) {};
		\node [style=Z] (47) at (0.75, 5.5) {$\pi$};
		\node [style=Z] (48) at (0.75, 6.5) {};
		\node [style=Z] (49) at (0.75, 6) {};
	\end{pgfonlayer}
	\begin{pgfonlayer}{edgelayer}
		\draw (45.center) to (43.center);
		\draw (42.center) to (44.center);
		\draw (46.center) to (41);
		\draw (48) to (49);
		\draw (49) to (47);
		\draw [in=90, out=180, looseness=0.75] (49) to (46.center);
		\draw [in=63, out=-117] (46.center) to (40);
	\end{pgfonlayer}
\end{tikzpicture}
=
\left\llbracket
\begin{tikzpicture}
	\begin{pgfonlayer}{nodelayer}
		\node [style=nothing] (41) at (0, 3.75) {};
		\node [style=nothing] (42) at (-0.5, 3.75) {};
		\node [style=nothing] (43) at (-0.5, 4.75) {};
		\node [style=nothing] (44) at (0, 4.75) {};
		\node [style=dot] (45) at (-0.5, 4.25) {};
		\node [style=dot] (46) at (0, 4.25) {};
		\node [style=onein] (47) at (0.5, 3.75) {};
		\node [style=zeroout] (48) at (0.5, 4.75) {};
		\node [style=oplus] (49) at (0.5, 4.25) {};
	\end{pgfonlayer}
	\begin{pgfonlayer}{edgelayer}
		\draw (42) to (45);
		\draw (45) to (43);
		\draw (44) to (46);
		\draw (46) to (41);
		\draw (47) to (49);
		\draw (49) to (48);
		\draw (49) to (46);
		\draw (46) to (45);
	\end{pgfonlayer}
\end{tikzpicture}
\right\rrbracket_{\hat{\TOF}}
\end{align*}


\item[\ref{TOF.8}:]
This follows immediately from Lemma \ref{lem:blackdot} and \ref{ZXA.7}.

\item[\ref{TOF.9}:]

\begin{align*}
\left\llbracket
\begin{tikzpicture}
	\begin{pgfonlayer}{nodelayer}
		\node [style=nothing] (42) at (-1.75, 3.75) {};
		\node [style=nothing] (43) at (-1.25, 3.75) {};
		\node [style=nothing] (44) at (-0.75, 3.75) {};
		\node [style=dot] (45) at (-1.75, 4.25) {};
		\node [style=dot] (46) at (-1.25, 4.25) {};
		\node [style=oplus] (47) at (-0.75, 4.25) {};
		\node [style=dot] (48) at (-1.75, 4.75) {};
		\node [style=oplus] (49) at (-0.75, 4.75) {};
		\node [style=dot] (50) at (-1.25, 4.75) {};
		\node [style=nothing] (51) at (-1.25, 5.25) {};
		\node [style=nothing] (52) at (-0.75, 5.25) {};
		\node [style=nothing] (53) at (-1.75, 5.25) {};
	\end{pgfonlayer}
	\begin{pgfonlayer}{edgelayer}
		\draw (42) to (45);
		\draw (43) to (46);
		\draw (44) to (47);
		\draw (45) to (46);
		\draw (46) to (47);
		\draw (48) to (50);
		\draw (50) to (49);
		\draw (45) to (48);
		\draw (48) to (53);
		\draw (46) to (50);
		\draw (50) to (51);
		\draw (47) to (49);
		\draw (49) to (52);
	\end{pgfonlayer}
\end{tikzpicture}
\right\rrbracket_{\hat{\TOF}}
&=
\begin{tikzpicture}
	\begin{pgfonlayer}{nodelayer}
		\node [style=X] (43) at (-0.75, 6.25) {};
		\node [style=X] (44) at (0.25, 6.25) {};
		\node [style=none] (45) at (0.25, 3.75) {};
		\node [style=none] (46) at (-0.75, 3.75) {};
		\node [style=none] (47) at (0.25, 8.5) {};
		\node [style=none] (48) at (-0.75, 8.5) {};
		\node [style=andin] (49) at (-0.25, 7.25) {};
		\node [style=Z] (50) at (0.75, 8) {};
		\node [style=none] (51) at (0.75, 8.5) {};
		\node [style=none] (52) at (0.75, 3.75) {};
		\node [style=X] (53) at (0.25, 4.25) {};
		\node [style=andin] (54) at (-0.25, 5.25) {};
		\node [style=Z] (55) at (0.75, 6) {};
		\node [style=X] (56) at (-0.75, 4.25) {};
	\end{pgfonlayer}
	\begin{pgfonlayer}{edgelayer}
		\draw (48.center) to (46.center);
		\draw (45.center) to (47.center);
		\draw (49.center) to (44);
		\draw [in=90, out=180, looseness=0.75] (50) to (49.center);
		\draw [in=63, out=-117] (49.center) to (43);
		\draw (54.center) to (53);
		\draw [in=90, out=180, looseness=0.75] (55) to (54.center);
		\draw [in=63, out=-117] (54.center) to (56);
		\draw (51.center) to (52.center);
	\end{pgfonlayer}
\end{tikzpicture}
\eq{\ref{ZXA.3}}
\begin{tikzpicture}
	\begin{pgfonlayer}{nodelayer}
		\node [style=X] (1) at (-0.75, 8.25) {};
		\node [style=X] (2) at (0.25, 8.25) {};
		\node [style=none] (3) at (0.75, 6.5) {};
		\node [style=none] (4) at (-1.25, 6.5) {};
		\node [style=none] (5) at (0.75, 10) {};
		\node [style=none] (6) at (-1.25, 10) {};
		\node [style=andin] (7) at (-0.25, 9.25) {};
		\node [style=Z] (8) at (1.25, 8.25) {};
		\node [style=none] (9) at (1.25, 10) {};
		\node [style=none] (10) at (1.25, 6.5) {};
		\node [style=andout] (11) at (-0.25, 7.25) {};
		\node [style=X] (12) at (-1.25, 8.25) {};
		\node [style=X] (13) at (0.75, 8.25) {};
	\end{pgfonlayer}
	\begin{pgfonlayer}{edgelayer}
		\draw (7.center) to (2);
		\draw [in=90, out=105, looseness=1.50] (8) to (7.center);
		\draw [in=63, out=-117] (7.center) to (1);
		\draw (9.center) to (10.center);
		\draw (2) to (11.center);
		\draw (11.center) to (1);
		\draw [in=-105, out=-90, looseness=1.75] (11.center) to (8);
		\draw (5.center) to (13);
		\draw (13) to (3.center);
		\draw (13) to (2);
		\draw (1) to (12);
		\draw (12) to (6.center);
		\draw (12) to (4.center);
	\end{pgfonlayer}
\end{tikzpicture}
=
\begin{tikzpicture}
	\begin{pgfonlayer}{nodelayer}
		\node [style=X] (2) at (-1, 9.25) {};
		\node [style=X] (3) at (-0.25, 9.25) {};
		\node [style=none] (4) at (0.25, 6.5) {};
		\node [style=none] (5) at (-1.5, 6.5) {};
		\node [style=none] (6) at (0.25, 10.5) {};
		\node [style=none] (7) at (-1.5, 10.5) {};
		\node [style=none] (8) at (-1, 8.25) {};
		\node [style=Z] (9) at (0.75, 7) {};
		\node [style=none] (10) at (0.75, 10.5) {};
		\node [style=none] (11) at (0.75, 6.5) {};
		\node [style=none] (12) at (-0.25, 8.25) {};
		\node [style=X] (13) at (-1.5, 9.75) {};
		\node [style=X] (14) at (0.25, 9.75) {};
		\node [style=none] (15) at (-1, 7.5) {};
		\node [style=none] (16) at (-0.25, 7.75) {};
		\node [style=andout] (17) at (-1, 8.25) {};
		\node [style=andout] (18) at (-0.25, 8.25) {};
	\end{pgfonlayer}
	\begin{pgfonlayer}{edgelayer}
		\draw (8.center) to (3);
		\draw [in=-120, out=120, looseness=1.25] (8.center) to (2);
		\draw (10.center) to (11.center);
		\draw [in=60, out=-60, looseness=1.25] (3) to (12.center);
		\draw (12.center) to (2);
		\draw (6.center) to (14);
		\draw (14) to (4.center);
		\draw [in=90, out=180, looseness=1.75] (14) to (3);
		\draw [in=0, out=90, looseness=1.75] (2) to (13);
		\draw (13) to (7.center);
		\draw (13) to (5.center);
		\draw [in=-90, out=153, looseness=0.75] (9) to (16.center);
		\draw [in=-90, out=180] (9) to (15.center);
		\draw (15.center) to (8.center);
		\draw (16.center) to (12.center);
	\end{pgfonlayer}
\end{tikzpicture}
\eq{\ref{ZXA.12}}
\begin{tikzpicture}
	\begin{pgfonlayer}{nodelayer}
		\node [style=none] (3) at (0, 6.5) {};
		\node [style=none] (4) at (-1, 6.5) {};
		\node [style=none] (5) at (0, 9) {};
		\node [style=none] (6) at (-1, 9) {};
		\node [style=Z] (7) at (0.5, 7.25) {};
		\node [style=none] (8) at (0.5, 9) {};
		\node [style=none] (9) at (0.5, 6.5) {};
		\node [style=X] (10) at (-1, 8.5) {};
		\node [style=X] (11) at (0, 8.5) {};
		\node [style=X] (12) at (-0.5, 7.25) {};
		\node [style=none] (13) at (-0.5, 8) {};
		\node [style=andout] (14) at (-0.5, 8) {};
	\end{pgfonlayer}
	\begin{pgfonlayer}{edgelayer}
		\draw (8.center) to (9.center);
		\draw (5.center) to (11);
		\draw (11) to (3.center);
		\draw (10) to (6.center);
		\draw (10) to (4.center);
		\draw [bend right] (7) to (12);
		\draw [bend right] (12) to (7);
		\draw (11) to (13.center);
		\draw (13.center) to (10);
		\draw (13.center) to (12);
	\end{pgfonlayer}
\end{tikzpicture}\\
&\eq{\ref{ZXA.8}}
\begin{tikzpicture}
	\begin{pgfonlayer}{nodelayer}
		\node [style=none] (4) at (0, 6.5) {};
		\node [style=none] (5) at (-1, 6.5) {};
		\node [style=none] (6) at (0, 9) {};
		\node [style=none] (7) at (-1, 9) {};
		\node [style=Z] (8) at (0.5, 7.25) {};
		\node [style=none] (9) at (0.5, 9) {};
		\node [style=none] (10) at (0.5, 6.5) {};
		\node [style=X] (11) at (-1, 8.5) {};
		\node [style=X] (12) at (0, 8.5) {};
		\node [style=X] (13) at (-0.5, 7.25) {};
		\node [style=none] (14) at (-0.5, 8) {};
		\node [style=andout] (15) at (-0.5, 8) {};
	\end{pgfonlayer}
	\begin{pgfonlayer}{edgelayer}
		\draw (9.center) to (10.center);
		\draw (6.center) to (12);
		\draw (12) to (4.center);
		\draw (11) to (7.center);
		\draw (11) to (5.center);
		\draw (12) to (14.center);
		\draw (14.center) to (11);
		\draw (14.center) to (13);
	\end{pgfonlayer}
\end{tikzpicture}
\eq{\ref{ZXA.1}}
\begin{tikzpicture}
	\begin{pgfonlayer}{nodelayer}
		\node [style=none] (0) at (0, 3) {};
		\node [style=none] (1) at (-1, 3) {};
		\node [style=none] (2) at (0, 5.5) {};
		\node [style=none] (3) at (-1, 5.5) {};
		\node [style=none] (4) at (0.5, 5.5) {};
		\node [style=none] (5) at (0.5, 3) {};
		\node [style=X] (6) at (-1, 5) {};
		\node [style=X] (7) at (0, 5) {};
		\node [style=X] (8) at (-0.5, 3.75) {};
		\node [style=none] (9) at (-0.5, 4.5) {};
		\node [style=andout] (10) at (-0.5, 4.5) {};
	\end{pgfonlayer}
	\begin{pgfonlayer}{edgelayer}
		\draw (4.center) to (5.center);
		\draw (2.center) to (7);
		\draw (7) to (0.center);
		\draw (6) to (3.center);
		\draw (6) to (1.center);
		\draw (7) to (9.center);
		\draw (9.center) to (6);
		\draw (9.center) to (8);
	\end{pgfonlayer}
\end{tikzpicture}
\eq{\ref{ZXA.13}}
\begin{tikzpicture}
	\begin{pgfonlayer}{nodelayer}
		\node [style=none] (1) at (0, 3.75) {};
		\node [style=none] (2) at (-1.5, 3.75) {};
		\node [style=none] (3) at (0, 5.5) {};
		\node [style=none] (4) at (-1.5, 5.5) {};
		\node [style=none] (5) at (0.5, 5.5) {};
		\node [style=none] (6) at (0.5, 3.75) {};
		\node [style=X] (7) at (-1.5, 5) {};
		\node [style=X] (8) at (0, 5) {};
		\node [style=X] (9) at (-1, 4.25) {};
		\node [style=X] (10) at (-0.5, 4.25) {};
	\end{pgfonlayer}
	\begin{pgfonlayer}{edgelayer}
		\draw (5.center) to (6.center);
		\draw (3.center) to (8);
		\draw (8) to (1.center);
		\draw (7) to (4.center);
		\draw (7) to (2.center);
		\draw [in=90, out=-124] (8) to (10);
		\draw [in=-56, out=90] (9) to (7);
	\end{pgfonlayer}
\end{tikzpicture}
\eq{\ref{ZXA.3}}
\begin{tikzpicture}
	\begin{pgfonlayer}{nodelayer}
		\node [style=nothing] (2) at (-1.75, 0) {};
		\node [style=nothing] (3) at (-1.25, 0) {};
		\node [style=nothing] (4) at (-0.75, 0) {};
		\node [style=nothing] (5) at (-1.25, 1.5) {};
		\node [style=nothing] (6) at (-0.75, 1.5) {};
		\node [style=nothing] (7) at (-1.75, 1.5) {};
	\end{pgfonlayer}
	\begin{pgfonlayer}{edgelayer}
		\draw (2) to (7);
		\draw (3) to (5);
		\draw (4) to (6);
	\end{pgfonlayer}
\end{tikzpicture}
=
\left\llbracket
\begin{tikzpicture}
	\begin{pgfonlayer}{nodelayer}
		\node [style=nothing] (3) at (-1.75, 0) {};
		\node [style=nothing] (4) at (-1.25, 0) {};
		\node [style=nothing] (5) at (-0.75, 0) {};
		\node [style=nothing] (6) at (-1.25, 1.5) {};
		\node [style=nothing] (7) at (-0.75, 1.5) {};
		\node [style=nothing] (8) at (-1.75, 1.5) {};
	\end{pgfonlayer}
	\begin{pgfonlayer}{edgelayer}
		\draw (3) to (8);
		\draw (4) to (6);
		\draw (5) to (7);
	\end{pgfonlayer}
\end{tikzpicture}
\right\rrbracket_{\hat{\TOF}}
\end{align*}


\item[\ref{TOF.10}:]  It is easier to prove that $\ref{TOF.10}$ is redundant.  Given \ref{TOF.9},  \ref{TOF.6} and \ref{TOF.12}, \ref{TOF.10} is equivalent to the following:
$$
\begin{tikzpicture}
	\begin{pgfonlayer}{nodelayer}
		\node [style=dot] (4) at (0, 3) {};
		\node [style=dot] (5) at (0.5, 3) {};
		\node [style=dot] (6) at (-0.5, 3.5) {};
		\node [style=dot] (7) at (0, 3.5) {};
		\node [style=dot] (8) at (0, 4) {};
		\node [style=dot] (9) at (0.5, 4) {};
		\node [style=dot] (10) at (-0.5, 4.5) {};
		\node [style=dot] (11) at (0, 4.5) {};
		\node [style=oplus] (12) at (1, 3) {};
		\node [style=oplus] (13) at (0.5, 3.5) {};
		\node [style=oplus] (14) at (1, 4) {};
		\node [style=oplus] (15) at (0.5, 4.5) {};
		\node [style=none] (16) at (1, 2.5) {};
		\node [style=none] (17) at (0.5, 2.5) {};
		\node [style=none] (18) at (0, 2.5) {};
		\node [style=none] (19) at (-0.5, 2.5) {};
		\node [style=none] (20) at (-0.5, 5) {};
		\node [style=none] (21) at (0, 5) {};
		\node [style=none] (22) at (0.5, 5) {};
		\node [style=none] (23) at (1, 5) {};
	\end{pgfonlayer}
	\begin{pgfonlayer}{edgelayer}
		\draw (16.center) to (23.center);
		\draw (22.center) to (17.center);
		\draw (18.center) to (21.center);
		\draw (20.center) to (19.center);
		\draw (4) to (12);
		\draw (13) to (6);
		\draw (8) to (14);
		\draw (15) to (10);
	\end{pgfonlayer}
\end{tikzpicture}
\eq{\ref{TOF.10}}
\begin{tikzpicture}
	\begin{pgfonlayer}{nodelayer}
		\node [style=dot] (5) at (-0.5, 3.5) {};
		\node [style=dot] (6) at (0, 3.5) {};
		\node [style=dot] (7) at (-0.5, 4) {};
		\node [style=dot] (8) at (0, 4) {};
		\node [style=dot] (9) at (-0.5, 4.5) {};
		\node [style=dot] (10) at (0, 4.5) {};
		\node [style=oplus] (11) at (1, 3.5) {};
		\node [style=oplus] (12) at (0.5, 4) {};
		\node [style=oplus] (13) at (0.5, 4.5) {};
		\node [style=none] (14) at (1, 3) {};
		\node [style=none] (15) at (0.5, 3) {};
		\node [style=none] (16) at (0, 3) {};
		\node [style=none] (17) at (-0.5, 3) {};
		\node [style=none] (18) at (-0.5, 5) {};
		\node [style=none] (19) at (0, 5) {};
		\node [style=none] (20) at (0.5, 5) {};
		\node [style=none] (21) at (1, 5) {};
	\end{pgfonlayer}
	\begin{pgfonlayer}{edgelayer}
		\draw (14.center) to (21.center);
		\draw (20.center) to (15.center);
		\draw (16.center) to (19.center);
		\draw (18.center) to (17.center);
		\draw (11) to (5);
		\draw (7) to (12);
		\draw (13) to (9);
	\end{pgfonlayer}
\end{tikzpicture}
\eq{\ref{TOF.9}}
\begin{tikzpicture}
	\begin{pgfonlayer}{nodelayer}
		\node [style=dot] (6) at (-0.5, 3) {};
		\node [style=dot] (7) at (0, 3) {};
		\node [style=oplus] (8) at (1, 3) {};
		\node [style=none] (9) at (1, 2.5) {};
		\node [style=none] (10) at (0.5, 2.5) {};
		\node [style=none] (11) at (0, 2.5) {};
		\node [style=none] (12) at (-0.5, 2.5) {};
		\node [style=none] (13) at (-0.5, 3.5) {};
		\node [style=none] (14) at (0, 3.5) {};
		\node [style=none] (15) at (0.5, 3.5) {};
		\node [style=none] (16) at (1, 3.5) {};
	\end{pgfonlayer}
	\begin{pgfonlayer}{edgelayer}
		\draw (9.center) to (16.center);
		\draw (15.center) to (10.center);
		\draw (11.center) to (14.center);
		\draw (13.center) to (12.center);
		\draw (6) to (8);
	\end{pgfonlayer}
\end{tikzpicture}
$$

However
$$
\begin{tikzpicture}
	\begin{pgfonlayer}{nodelayer}
		\node [style=dot] (7) at (0, 3) {};
		\node [style=dot] (8) at (0.5, 3) {};
		\node [style=dot] (9) at (-0.5, 3.5) {};
		\node [style=dot] (10) at (0, 3.5) {};
		\node [style=dot] (11) at (0, 4) {};
		\node [style=dot] (12) at (0.5, 4) {};
		\node [style=dot] (13) at (-0.5, 4.5) {};
		\node [style=dot] (14) at (0, 4.5) {};
		\node [style=oplus] (15) at (1, 3) {};
		\node [style=oplus] (16) at (0.5, 3.5) {};
		\node [style=oplus] (17) at (1, 4) {};
		\node [style=oplus] (18) at (0.5, 4.5) {};
		\node [style=none] (19) at (1, 2.5) {};
		\node [style=none] (20) at (0.5, 2.5) {};
		\node [style=none] (21) at (0, 2.5) {};
		\node [style=none] (22) at (-0.5, 2.5) {};
		\node [style=none] (23) at (-0.5, 5) {};
		\node [style=none] (24) at (0, 5) {};
		\node [style=none] (25) at (0.5, 5) {};
		\node [style=none] (26) at (1, 5) {};
	\end{pgfonlayer}
	\begin{pgfonlayer}{edgelayer}
		\draw (19.center) to (26.center);
		\draw (25.center) to (20.center);
		\draw (21.center) to (24.center);
		\draw (23.center) to (22.center);
		\draw (7) to (15);
		\draw (16) to (9);
		\draw (11) to (17);
		\draw (18) to (13);
	\end{pgfonlayer}
\end{tikzpicture}
\eq{\ref{TOF.12}}
\begin{tikzpicture}
	\begin{pgfonlayer}{nodelayer}
		\node [style=dot] (8) at (0, 3) {};
		\node [style=dot] (9) at (0.5, 3) {};
		\node [style=dot] (10) at (-0.5, 3.5) {};
		\node [style=dot] (11) at (0, 3.5) {};
		\node [style=dot] (12) at (0, 4) {};
		\node [style=dot] (13) at (0.5, 4) {};
		\node [style=oplus] (14) at (1, 3) {};
		\node [style=oplus] (15) at (1, 3.5) {};
		\node [style=oplus] (16) at (1, 4) {};
		\node [style=none] (17) at (1, 2.5) {};
		\node [style=none] (18) at (0.5, 2.5) {};
		\node [style=none] (19) at (0, 2.5) {};
		\node [style=none] (20) at (-0.5, 2.5) {};
		\node [style=none] (21) at (-0.5, 4.5) {};
		\node [style=none] (22) at (0, 4.5) {};
		\node [style=none] (23) at (0.5, 4.5) {};
		\node [style=none] (24) at (1, 4.5) {};
	\end{pgfonlayer}
	\begin{pgfonlayer}{edgelayer}
		\draw (17.center) to (24.center);
		\draw (23.center) to (18.center);
		\draw (19.center) to (22.center);
		\draw (21.center) to (20.center);
		\draw (8) to (14);
		\draw (15) to (10);
		\draw (12) to (16);
	\end{pgfonlayer}
\end{tikzpicture}
\eq{\ref{TOF.6}}
\begin{tikzpicture}
	\begin{pgfonlayer}{nodelayer}
		\node [style=dot] (9) at (0, 3) {};
		\node [style=dot] (10) at (0.5, 3) {};
		\node [style=dot] (11) at (-0.5, 3.5) {};
		\node [style=dot] (12) at (0, 3.5) {};
		\node [style=dot] (13) at (0, 4) {};
		\node [style=dot] (14) at (0.5, 4) {};
		\node [style=oplus] (15) at (1, 3) {};
		\node [style=oplus] (16) at (1, 3.5) {};
		\node [style=oplus] (17) at (1, 4) {};
		\node [style=none] (18) at (1, 2.5) {};
		\node [style=none] (19) at (0.5, 2.5) {};
		\node [style=none] (20) at (0, 2.5) {};
		\node [style=none] (21) at (-0.5, 2.5) {};
		\node [style=none] (22) at (-0.5, 4.5) {};
		\node [style=none] (23) at (0, 4.5) {};
		\node [style=none] (24) at (0.5, 4.5) {};
		\node [style=none] (25) at (1, 4.5) {};
	\end{pgfonlayer}
	\begin{pgfonlayer}{edgelayer}
		\draw (18.center) to (25.center);
		\draw (24.center) to (19.center);
		\draw (20.center) to (23.center);
		\draw (22.center) to (21.center);
		\draw (9) to (15);
		\draw (16) to (11);
		\draw (13) to (17);
	\end{pgfonlayer}
\end{tikzpicture}
\eq{\ref{TOF.9}}
\begin{tikzpicture}
	\begin{pgfonlayer}{nodelayer}
		\node [style=dot] (10) at (-0.5, 4) {};
		\node [style=dot] (11) at (0, 4) {};
		\node [style=oplus] (12) at (1, 4) {};
		\node [style=none] (13) at (1, 3.5) {};
		\node [style=none] (14) at (0.5, 3.5) {};
		\node [style=none] (15) at (0, 3.5) {};
		\node [style=none] (16) at (-0.5, 3.5) {};
		\node [style=none] (17) at (-0.5, 4.5) {};
		\node [style=none] (18) at (0, 4.5) {};
		\node [style=none] (19) at (0.5, 4.5) {};
		\node [style=none] (20) at (1, 4.5) {};
	\end{pgfonlayer}
	\begin{pgfonlayer}{edgelayer}
		\draw (13.center) to (20.center);
		\draw (19.center) to (14.center);
		\draw (15.center) to (18.center);
		\draw (17.center) to (16.center);
		\draw (12) to (10);
	\end{pgfonlayer}
\end{tikzpicture}
$$

\item[\ref{TOF.11}:]
\begin{align*}
\left\llbracket
\begin{tikzpicture}
	\begin{pgfonlayer}{nodelayer}
		\node [style=nothing] (11) at (0, 2.5) {};
		\node [style=nothing] (12) at (-0.5, 2.5) {};
		\node [style=nothing] (13) at (-1, 2.5) {};
		\node [style=nothing] (14) at (-1.5, 2.5) {};
		\node [style=nothing] (15) at (-0.5, 4.5) {};
		\node [style=nothing] (16) at (0, 4.5) {};
		\node [style=dot] (17) at (-1.5, 3) {};
		\node [style=dot] (18) at (-1, 3.5) {};
		\node [style=dot] (19) at (-0.5, 3.5) {};
		\node [style=oplus] (20) at (-1, 3) {};
		\node [style=oplus] (21) at (0, 3.5) {};
		\node [style=nothing] (22) at (-1.5, 4.5) {};
		\node [style=nothing] (23) at (-1, 4.5) {};
		\node [style=oplus] (24) at (-1, 4) {};
		\node [style=dot] (25) at (-1.5, 4) {};
	\end{pgfonlayer}
	\begin{pgfonlayer}{edgelayer}
		\draw (17) to (20);
		\draw (18) to (19);
		\draw (19) to (21);
		\draw (11) to (21);
		\draw (21) to (16);
		\draw (15) to (19);
		\draw (19) to (12);
		\draw (13) to (20);
		\draw (20) to (18);
		\draw (17) to (14);
		\draw (17) to (25);
		\draw (25) to (22);
		\draw (23) to (24);
		\draw (24) to (18);
		\draw (24) to (25);
	\end{pgfonlayer}
\end{tikzpicture}
\right\rrbracket_{\hat{\TOF}}
&=
\begin{tikzpicture}
	\begin{pgfonlayer}{nodelayer}
		\node [style=none] (12) at (-0.75, 4.25) {};
		\node [style=X] (13) at (-1.25, 3.5) {};
		\node [style=X] (14) at (-0.25, 3.5) {};
		\node [style=Z] (15) at (0.25, 4.75) {};
		\node [style=Z] (16) at (-1.25, 3) {};
		\node [style=Z] (17) at (-1.25, 4.75) {};
		\node [style=X] (18) at (-1.75, 3) {};
		\node [style=X] (19) at (-1.75, 4.75) {};
		\node [style=none] (20) at (0.25, 5.25) {};
		\node [style=none] (21) at (-0.25, 5.25) {};
		\node [style=none] (22) at (-1.75, 2.5) {};
		\node [style=none] (23) at (-1.25, 2.5) {};
		\node [style=none] (24) at (-0.25, 2.5) {};
		\node [style=none] (25) at (0.25, 2.5) {};
		\node [style=none] (26) at (-1.75, 5.25) {};
		\node [style=none] (27) at (-1.25, 5.25) {};
		\node [style=andin] (28) at (-0.75, 4.25) {};
	\end{pgfonlayer}
	\begin{pgfonlayer}{edgelayer}
		\draw (20.center) to (15);
		\draw [in=90, out=180] (15) to (12.center);
		\draw (12.center) to (14);
		\draw (14) to (24.center);
		\draw (25.center) to (15);
		\draw (12.center) to (13);
		\draw (13) to (17);
		\draw (13) to (16);
		\draw (16) to (18);
		\draw (19) to (17);
		\draw (18) to (22.center);
		\draw (23.center) to (16);
		\draw (27.center) to (17);
		\draw (19) to (26.center);
		\draw (19) to (18);
		\draw (21.center) to (14);
	\end{pgfonlayer}
\end{tikzpicture}
\eq{\ref{ZXA.3}}
\begin{tikzpicture}
	\begin{pgfonlayer}{nodelayer}
		\node [style=none] (13) at (-0.25, 4.5) {};
		\node [style=X] (14) at (-0.75, 3.75) {};
		\node [style=X] (15) at (0.25, 3.75) {};
		\node [style=Z] (16) at (0.75, 5) {};
		\node [style=Z] (17) at (-1.25, 3) {};
		\node [style=Z] (18) at (-1.25, 4.5) {};
		\node [style=X] (19) at (-1.75, 3.75) {};
		\node [style=none] (20) at (0.75, 5.5) {};
		\node [style=none] (21) at (0.25, 5.5) {};
		\node [style=none] (22) at (-2.25, 2.5) {};
		\node [style=none] (23) at (-1.25, 2.5) {};
		\node [style=none] (24) at (0.25, 2.5) {};
		\node [style=none] (25) at (0.75, 2.5) {};
		\node [style=none] (26) at (-2.25, 5.5) {};
		\node [style=none] (27) at (-1.25, 5.5) {};
		\node [style=andin] (28) at (-0.25, 4.5) {};
		\node [style=X] (29) at (-2.25, 3.75) {};
	\end{pgfonlayer}
	\begin{pgfonlayer}{edgelayer}
		\draw (20.center) to (16);
		\draw [in=90, out=180] (16) to (13.center);
		\draw (13.center) to (15);
		\draw (15) to (24.center);
		\draw (25.center) to (16);
		\draw (13.center) to (14);
		\draw (17) to (19);
		\draw (23.center) to (17);
		\draw (27.center) to (18);
		\draw (21.center) to (15);
		\draw (18) to (19);
		\draw (19) to (29);
		\draw (29) to (26.center);
		\draw (29) to (22.center);
		\draw (14) to (18);
		\draw (14) to (17);
	\end{pgfonlayer}
\end{tikzpicture}
=
\begin{tikzpicture}
	\begin{pgfonlayer}{nodelayer}
		\node [style=none] (14) at (-0.25, 4.5) {};
		\node [style=X] (15) at (-0.75, 3.75) {};
		\node [style=X] (16) at (0.25, 3.75) {};
		\node [style=Z] (17) at (0.75, 5) {};
		\node [style=Z] (18) at (-1.25, 3) {};
		\node [style=Z] (19) at (-1.75, 3.75) {};
		\node [style=X] (20) at (-1.25, 4.5) {};
		\node [style=none] (21) at (0.75, 5.5) {};
		\node [style=none] (22) at (0.25, 5.5) {};
		\node [style=none] (23) at (-2.25, 2.5) {};
		\node [style=none] (24) at (-1.25, 2.5) {};
		\node [style=none] (25) at (0.25, 2.5) {};
		\node [style=none] (26) at (0.75, 2.5) {};
		\node [style=none] (27) at (-2.25, 5.5) {};
		\node [style=none] (28) at (-1.75, 5.5) {};
		\node [style=andin] (29) at (-0.25, 4.5) {};
		\node [style=X] (30) at (-2.25, 5) {};
	\end{pgfonlayer}
	\begin{pgfonlayer}{edgelayer}
		\draw (21.center) to (17);
		\draw [in=90, out=180] (17) to (14.center);
		\draw (14.center) to (16);
		\draw (16) to (25.center);
		\draw (26.center) to (17);
		\draw (14.center) to (15);
		\draw (18) to (20);
		\draw (24.center) to (18);
		\draw [in=90, out=-90] (28.center) to (19);
		\draw (22.center) to (16);
		\draw (19) to (20);
		\draw (20) to (30);
		\draw (30) to (27.center);
		\draw (30) to (23.center);
		\draw (15) to (19);
		\draw (15) to (18);
	\end{pgfonlayer}
\end{tikzpicture}
\eq{\ref{ZXA.5}}
\begin{tikzpicture}
	\begin{pgfonlayer}{nodelayer}
		\node [style=X] (15) at (-0.75, 8.75) {};
		\node [style=none] (16) at (-2.75, 10.5) {};
		\node [style=none] (17) at (-0.25, 10.5) {};
		\node [style=none] (18) at (-2.75, 7.5) {};
		\node [style=none] (19) at (-3.25, 7.5) {};
		\node [style=none] (20) at (-0.75, 7.5) {};
		\node [style=none] (21) at (-0.75, 10.5) {};
		\node [style=Z] (22) at (-0.25, 10) {};
		\node [style=none] (23) at (-1.25, 9.5) {};
		\node [style=X] (24) at (-3.25, 10) {};
		\node [style=none] (25) at (-0.25, 7.5) {};
		\node [style=none] (26) at (-3.25, 10.5) {};
		\node [style=Z] (27) at (-2, 9) {};
		\node [style=X] (28) at (-2.75, 8.25) {};
		\node [style=andin] (29) at (-1.25, 9.5) {};
	\end{pgfonlayer}
	\begin{pgfonlayer}{edgelayer}
		\draw (17.center) to (22);
		\draw [in=90, out=180] (22) to (23.center);
		\draw (23.center) to (15);
		\draw (15) to (20.center);
		\draw (25.center) to (22);
		\draw (21.center) to (15);
		\draw (24) to (26.center);
		\draw (24) to (19.center);
		\draw (27) to (28);
		\draw (27) to (24);
		\draw (28) to (18.center);
		\draw (28) to (16.center);
		\draw (23.center) to (27);
	\end{pgfonlayer}
\end{tikzpicture}\\
&\eq{\ref{ZXA.17}}
\begin{tikzpicture}
	\begin{pgfonlayer}{nodelayer}
		\node [style=none] (0) at (-1.25, 10.75) {};
		\node [style=none] (1) at (-3.5, 10.75) {};
		\node [style=X] (2) at (-1.25, 11.5) {};
		\node [style=none] (3) at (-4, 10.75) {};
		\node [style=X] (4) at (-4, 11.5) {};
		\node [style=none] (5) at (-1.25, 15) {};
		\node [style=none] (6) at (-4, 15) {};
		\node [style=none] (7) at (-3.5, 15) {};
		\node [style=none] (8) at (-0.75, 10.75) {};
		\node [style=none] (9) at (-0.75, 15) {};
		\node [style=Z] (10) at (-0.75, 14.5) {};
		\node [style=X] (11) at (-3.5, 12.5) {};
		\node [style=none] (12) at (-1.75, 13.25) {};
		\node [style=Z] (13) at (-2.25, 14) {};
		\node [style=X] (14) at (-1.75, 12.25) {};
		\node [style=andin] (15) at (-2.75, 13.25) {};
		\node [style=none] (16) at (-2.75, 13.25) {};
		\node [style=andin] (17) at (-1.75, 13.25) {};
	\end{pgfonlayer}
	\begin{pgfonlayer}{edgelayer}
		\draw (9.center) to (10);
		\draw (2) to (0.center);
		\draw (8.center) to (10);
		\draw (5.center) to (2);
		\draw (4) to (6.center);
		\draw (4) to (3.center);
		\draw (11) to (1.center);
		\draw (11) to (7.center);
		\draw [in=-90, out=124] (2) to (14);
		\draw (14) to (12.center);
		\draw (12.center) to (13);
		\draw (12.center) to (4);
		\draw (13) to (16.center);
		\draw (16.center) to (11);
		\draw (14) to (16.center);
		\draw [in=90, out=180] (10) to (13);
	\end{pgfonlayer}
\end{tikzpicture}
\eq{\ref{ZXA.1},\ref{ZXA.3}}
\begin{tikzpicture}
	\begin{pgfonlayer}{nodelayer}
		\node [style=none] (1) at (-1.25, 10.75) {};
		\node [style=none] (2) at (-3.5, 10.75) {};
		\node [style=X] (3) at (-1.25, 11.25) {};
		\node [style=none] (4) at (-4, 10.75) {};
		\node [style=X] (5) at (-4, 11.25) {};
		\node [style=none] (6) at (-1.25, 15) {};
		\node [style=none] (7) at (-4, 15) {};
		\node [style=none] (8) at (-3.5, 15) {};
		\node [style=none] (9) at (-0.75, 10.75) {};
		\node [style=none] (10) at (-0.75, 15) {};
		\node [style=Z] (11) at (-0.75, 14.5) {};
		\node [style=X] (12) at (-3.5, 12.5) {};
		\node [style=none] (13) at (-1.75, 13.25) {};
		\node [style=andin] (14) at (-1.75, 13.25) {};
		\node [style=none] (15) at (-2.75, 13.25) {};
		\node [style=Z] (16) at (-0.75, 14) {};
		\node [style=X] (17) at (-1.25, 12.5) {};
		\node [style=andin] (18) at (-2.75, 13.25) {};
	\end{pgfonlayer}
	\begin{pgfonlayer}{edgelayer}
		\draw (10.center) to (11);
		\draw (3) to (1.center);
		\draw (9.center) to (11);
		\draw (6.center) to (3);
		\draw (5) to (7.center);
		\draw (5) to (4.center);
		\draw (12) to (2.center);
		\draw (12) to (8.center);
		\draw (13.center) to (5);
		\draw (15.center) to (12);
		\draw [in=90, out=180] (16) to (13.center);
		\draw [in=180, out=90, looseness=0.75] (15.center) to (11);
		\draw (3) to (15.center);
		\draw (13.center) to (17);
	\end{pgfonlayer}
\end{tikzpicture}
=
\left\llbracket
\begin{tikzpicture}
	\begin{pgfonlayer}{nodelayer}
		\node [style=nothing] (2) at (0, 0) {};
		\node [style=nothing] (3) at (-1, 0) {};
		\node [style=nothing] (4) at (-0.5, 0) {};
		\node [style=nothing] (5) at (-1.5, 0) {};
		\node [style=dot] (6) at (-1.5, 0.5) {};
		\node [style=dot] (7) at (-0.5, 0.5) {};
		\node [style=oplus] (8) at (0, 0.5) {};
		\node [style=nothing] (9) at (-0.5, 1.5) {};
		\node [style=nothing] (10) at (-1, 1.5) {};
		\node [style=nothing] (11) at (-1.5, 1.5) {};
		\node [style=nothing] (12) at (0, 1.5) {};
		\node [style=dot] (13) at (-1, 1) {};
		\node [style=dot] (14) at (-0.5, 1) {};
		\node [style=oplus] (15) at (0, 1) {};
	\end{pgfonlayer}
	\begin{pgfonlayer}{edgelayer}
		\draw (5) to (6);
		\draw (4) to (7);
		\draw (8) to (2);
		\draw (8) to (7);
		\draw (7) to (6);
		\draw (13) to (3);
		\draw (7) to (14);
		\draw (14) to (9);
		\draw (12) to (15);
		\draw (15) to (8);
		\draw (15) to (14);
		\draw (14) to (13);
		\draw (6) to (11);
		\draw (13) to (10);
	\end{pgfonlayer}
\end{tikzpicture}
\right\rrbracket_{\hat{\TOF}}
\end{align*}


\item[\ref{TOF.12}:]
\begingroup
\allowdisplaybreaks
\begin{align*}
\left\llbracket
\begin{tikzpicture}
	\begin{pgfonlayer}{nodelayer}
		\node [style=nothing] (3) at (-0.5, 0) {};
		\node [style=nothing] (4) at (0, 0) {};
		\node [style=nothing] (5) at (-1, 0) {};
		\node [style=nothing] (6) at (-1.5, 0) {};
		\node [style=nothing] (7) at (-0.5, 2) {};
		\node [style=nothing] (8) at (-1.5, 2) {};
		\node [style=nothing] (9) at (0, 2) {};
		\node [style=nothing] (10) at (-1, 2) {};
		\node [style=dot] (11) at (-1.5, 0.5) {};
		\node [style=dot] (12) at (-1, 0.5) {};
		\node [style=oplus] (13) at (-0.5, 0.5) {};
		\node [style=oplus] (14) at (0, 1) {};
		\node [style=dot] (15) at (-1, 1) {};
		\node [style=dot] (16) at (-0.5, 1) {};
		\node [style=oplus] (17) at (-0.5, 1.5) {};
		\node [style=dot] (18) at (-1.5, 1.5) {};
		\node [style=dot] (19) at (-1, 1.5) {};
	\end{pgfonlayer}
	\begin{pgfonlayer}{edgelayer}
		\draw (11) to (12);
		\draw (12) to (13);
		\draw (15) to (16);
		\draw (16) to (14);
		\draw (18) to (19);
		\draw (19) to (17);
		\draw (6) to (11);
		\draw (11) to (18);
		\draw (18) to (8);
		\draw (10) to (19);
		\draw (19) to (15);
		\draw (15) to (12);
		\draw (12) to (5);
		\draw (3) to (13);
		\draw (13) to (16);
		\draw (16) to (17);
		\draw (17) to (7);
		\draw (9) to (14);
		\draw (14) to (4);
	\end{pgfonlayer}
\end{tikzpicture}
\right\rrbracket_{\hat{\TOF}}
&=
\begin{tikzpicture}
	\begin{pgfonlayer}{nodelayer}
		\node [style=none] (4) at (-1.5, 11.25) {};
		\node [style=X] (5) at (-2, 10.5) {};
		\node [style=X] (6) at (-1, 10.5) {};
		\node [style=Z] (7) at (0, 12) {};
		\node [style=andin] (8) at (-1.5, 11.25) {};
		\node [style=X] (9) at (-2, 13) {};
		\node [style=none] (10) at (-1.5, 13.75) {};
		\node [style=X] (11) at (-1, 13) {};
		\node [style=Z] (12) at (0, 14.5) {};
		\node [style=andin] (13) at (-1.5, 13.75) {};
		\node [style=X] (14) at (-1, 12.5) {};
		\node [style=none] (15) at (-0.5, 13.25) {};
		\node [style=X] (16) at (0, 12.5) {};
		\node [style=Z] (17) at (0.5, 14) {};
		\node [style=andin] (18) at (-0.5, 13.25) {};
		\node [style=none] (19) at (-2, 10) {};
		\node [style=none] (20) at (-1, 10) {};
		\node [style=none] (21) at (0, 10) {};
		\node [style=none] (22) at (0.5, 10) {};
		\node [style=none] (23) at (0.5, 15) {};
		\node [style=none] (24) at (0, 15) {};
		\node [style=none] (25) at (-1, 15) {};
		\node [style=none] (26) at (-2, 15) {};
	\end{pgfonlayer}
	\begin{pgfonlayer}{edgelayer}
		\draw [in=90, out=180] (7) to (4.center);
		\draw (4.center) to (5);
		\draw (4.center) to (6);
		\draw [in=90, out=180] (12) to (10.center);
		\draw (10.center) to (9);
		\draw (10.center) to (11);
		\draw [in=90, out=180] (17) to (15.center);
		\draw (15.center) to (14);
		\draw (15.center) to (16);
		\draw (19.center) to (26.center);
		\draw (25.center) to (20.center);
		\draw (24.center) to (21.center);
		\draw (22.center) to (23.center);
	\end{pgfonlayer}
\end{tikzpicture}
\eq{\ref{ZXA.3}}
\begin{tikzpicture}
	\begin{pgfonlayer}{nodelayer}
		\node [style=none] (5) at (-2, 11.25) {};
		\node [style=X] (6) at (-3, 12) {};
		\node [style=X] (7) at (-1.5, 12) {};
		\node [style=Z] (8) at (0, 10.5) {};
		\node [style=andout] (9) at (-2, 11.25) {};
		\node [style=X] (10) at (-2.5, 12) {};
		\node [style=none] (11) at (-2, 12.75) {};
		\node [style=X] (12) at (-1, 12) {};
		\node [style=Z] (13) at (0, 14.5) {};
		\node [style=andin] (14) at (-2, 12.75) {};
		\node [style=X] (15) at (-1, 12.75) {};
		\node [style=none] (16) at (-0.5, 13.5) {};
		\node [style=X] (17) at (0, 12.75) {};
		\node [style=Z] (18) at (0.5, 14.25) {};
		\node [style=andin] (19) at (-0.5, 13.5) {};
		\node [style=none] (20) at (-3, 10) {};
		\node [style=none] (21) at (-1, 10) {};
		\node [style=none] (22) at (0, 10) {};
		\node [style=none] (23) at (0.5, 10) {};
		\node [style=none] (24) at (0.5, 15) {};
		\node [style=none] (25) at (0, 15) {};
		\node [style=none] (26) at (-1, 15) {};
		\node [style=none] (27) at (-3, 15) {};
	\end{pgfonlayer}
	\begin{pgfonlayer}{edgelayer}
		\draw [in=-90, out=180] (8) to (5.center);
		\draw (5.center) to (7);
		\draw [in=90, out=180] (13) to (11.center);
		\draw (11.center) to (10);
		\draw [in=90, out=180] (18) to (16.center);
		\draw (16.center) to (15);
		\draw (16.center) to (17);
		\draw (20.center) to (27.center);
		\draw (26.center) to (21.center);
		\draw (25.center) to (22.center);
		\draw (23.center) to (24.center);
		\draw (7) to (11.center);
		\draw (7) to (12);
		\draw (5.center) to (10);
		\draw (10) to (6);
	\end{pgfonlayer}
\end{tikzpicture}
=
\begin{tikzpicture}
	\begin{pgfonlayer}{nodelayer}
		\node [style=none] (6) at (-1.5, 10.75) {};
		\node [style=X] (7) at (-3.5, 12.75) {};
		\node [style=X] (8) at (-1.5, 12) {};
		\node [style=Z] (9) at (0, 10) {};
		\node [style=andout] (10) at (-1.5, 10.75) {};
		\node [style=X] (11) at (-2.5, 12) {};
		\node [style=none] (12) at (-2.5, 10.75) {};
		\node [style=X] (13) at (-1, 12.75) {};
		\node [style=Z] (14) at (0, 15.25) {};
		\node [style=X] (15) at (-1, 13.25) {};
		\node [style=none] (16) at (-0.5, 14) {};
		\node [style=X] (17) at (0, 13.25) {};
		\node [style=Z] (18) at (0.5, 14.75) {};
		\node [style=andin] (19) at (-0.5, 14) {};
		\node [style=none] (20) at (-3.5, 9.5) {};
		\node [style=none] (21) at (-1, 9.5) {};
		\node [style=none] (22) at (0, 9.5) {};
		\node [style=none] (23) at (0.5, 9.5) {};
		\node [style=none] (24) at (0.5, 15.75) {};
		\node [style=none] (25) at (0, 15.75) {};
		\node [style=none] (26) at (-1, 15.75) {};
		\node [style=none] (27) at (-3.5, 15.75) {};
		\node [style=none] (28) at (-3, 10.75) {};
		\node [style=none] (29) at (-3, 14.75) {};
		\node [style=andout] (30) at (-2.5, 10.75) {};
	\end{pgfonlayer}
	\begin{pgfonlayer}{edgelayer}
		\draw [in=-90, out=180] (9) to (6.center);
		\draw [in=-60, out=60] (6.center) to (8);
		\draw [in=-120, out=120] (12.center) to (11);
		\draw [in=90, out=180] (18) to (16.center);
		\draw (16.center) to (15);
		\draw (16.center) to (17);
		\draw (20.center) to (27.center);
		\draw (26.center) to (21.center);
		\draw (25.center) to (22.center);
		\draw (23.center) to (24.center);
		\draw (8) to (12.center);
		\draw [in=180, out=90] (8) to (13);
		\draw (6.center) to (11);
		\draw [in=0, out=90, looseness=1.25] (11) to (7);
		\draw [in=-90, out=-90, looseness=3.50] (12.center) to (28.center);
		\draw (28.center) to (29.center);
		\draw [in=90, out=180, looseness=0.50] (14) to (29.center);
	\end{pgfonlayer}
\end{tikzpicture}\\
&\eq{\ref{ZXA.12}}
\begin{tikzpicture}
	\begin{pgfonlayer}{nodelayer}
		\node [style=none] (7) at (-0.25, 4.25) {};
		\node [style=none] (8) at (-0.75, 0.75) {};
		\node [style=X] (9) at (-0.75, 3) {};
		\node [style=none] (10) at (-2.75, 6) {};
		\node [style=none] (11) at (-2.25, 1.75) {};
		\node [style=none] (12) at (-2.25, 5) {};
		\node [style=none] (13) at (-2.75, 0.75) {};
		\node [style=none] (14) at (0.75, 0.75) {};
		\node [style=none] (15) at (0.25, 0.75) {};
		\node [style=X] (16) at (-0.75, 3.5) {};
		\node [style=andin] (17) at (-0.25, 4.25) {};
		\node [style=none] (18) at (-0.75, 6) {};
		\node [style=Z] (19) at (0.75, 5) {};
		\node [style=none] (20) at (0.25, 6) {};
		\node [style=Z] (21) at (0.25, 5.5) {};
		\node [style=Z] (22) at (0.25, 1.25) {};
		\node [style=X] (23) at (-2.75, 3.5) {};
		\node [style=X] (24) at (0.25, 3.5) {};
		\node [style=none] (25) at (0.75, 6) {};
		\node [style=none] (26) at (-1.5, 2.5) {};
		\node [style=X] (27) at (-1.5, 1.75) {};
		\node [style=andout] (28) at (-1.5, 2.5) {};
	\end{pgfonlayer}
	\begin{pgfonlayer}{edgelayer}
		\draw [in=90, out=180] (19) to (7.center);
		\draw (7.center) to (16);
		\draw (7.center) to (24);
		\draw (13.center) to (10.center);
		\draw (18.center) to (8.center);
		\draw (20.center) to (15.center);
		\draw (14.center) to (25.center);
		\draw (11.center) to (12.center);
		\draw [in=90, out=180, looseness=0.50] (21) to (12.center);
		\draw (26.center) to (9);
		\draw (26.center) to (23);
		\draw [in=180, out=-60] (27) to (22);
		\draw [in=-90, out=-135, looseness=1.25] (27) to (11.center);
		\draw (26.center) to (27);
	\end{pgfonlayer}
\end{tikzpicture}
\eq{\ref{ZXA.5}}
\begin{tikzpicture}
	\begin{pgfonlayer}{nodelayer}
		\node [style=Z] (8) at (1, 13.25) {};
		\node [style=none] (9) at (-2.75, 8.5) {};
		\node [style=none] (10) at (0.5, 8.5) {};
		\node [style=none] (11) at (-1.5, 10.75) {};
		\node [style=X] (12) at (-2.75, 11.75) {};
		\node [style=X] (13) at (-0.75, 11.75) {};
		\node [style=none] (14) at (-0.75, 8.5) {};
		\node [style=X] (15) at (-0.75, 11.25) {};
		\node [style=none] (16) at (1, 14.25) {};
		\node [style=none] (17) at (-2.25, 13.25) {};
		\node [style=Z] (18) at (0.5, 13.75) {};
		\node [style=andin] (19) at (-0.25, 12.5) {};
		\node [style=none] (20) at (1, 8.5) {};
		\node [style=andout] (21) at (-1.5, 10.75) {};
		\node [style=X] (22) at (-1.5, 9.5) {};
		\node [style=none] (23) at (-0.25, 12.5) {};
		\node [style=none] (24) at (-2.25, 9.5) {};
		\node [style=none] (25) at (0.5, 14.25) {};
		\node [style=none] (26) at (-2.75, 14.25) {};
		\node [style=none] (27) at (-0.75, 14.25) {};
		\node [style=X] (28) at (0, 9.75) {};
		\node [style=X] (29) at (0.5, 9.75) {};
		\node [style=Z] (30) at (0.5, 10.5) {};
		\node [style=Z] (31) at (0, 10.5) {};
	\end{pgfonlayer}
	\begin{pgfonlayer}{edgelayer}
		\draw [in=90, out=180] (8) to (23.center);
		\draw (23.center) to (13);
		\draw (9.center) to (26.center);
		\draw (27.center) to (14.center);
		\draw (20.center) to (16.center);
		\draw (24.center) to (17.center);
		\draw [in=90, out=180, looseness=0.50] (18) to (17.center);
		\draw (11.center) to (15);
		\draw (11.center) to (12);
		\draw [in=-90, out=-135, looseness=1.25] (22) to (24.center);
		\draw (11.center) to (22);
		\draw [in=-60, out=-90, looseness=1.25] (28) to (22);
		\draw (29) to (10.center);
		\draw (29) to (31);
		\draw (30) to (28);
		\draw [bend right, looseness=1.25] (31) to (28);
		\draw [bend right, looseness=1.25] (29) to (30);
		\draw (31) to (23.center);
		\draw (30) to (18);
		\draw (25.center) to (18);
	\end{pgfonlayer}
\end{tikzpicture}
\eq{\ref{ZXA.1},\ref{ZXA.2}}
\begin{tikzpicture}
	\begin{pgfonlayer}{nodelayer}
		\node [style=Z] (9) at (1, 12.25) {};
		\node [style=none] (10) at (-2.75, 8.75) {};
		\node [style=none] (11) at (0.5, 8.75) {};
		\node [style=none] (12) at (-1.5, 10) {};
		\node [style=X] (13) at (-2.75, 11) {};
		\node [style=X] (14) at (-0.75, 11) {};
		\node [style=none] (15) at (-0.75, 8.75) {};
		\node [style=X] (16) at (-0.75, 10.5) {};
		\node [style=none] (17) at (1, 12.75) {};
		\node [style=andin] (18) at (-0.25, 11.5) {};
		\node [style=none] (19) at (1, 8.75) {};
		\node [style=andout] (20) at (-1.5, 10) {};
		\node [style=none] (21) at (-0.25, 11.5) {};
		\node [style=none] (22) at (0.5, 12.75) {};
		\node [style=none] (23) at (-2.75, 12.75) {};
		\node [style=none] (24) at (-0.75, 12.75) {};
		\node [style=X] (25) at (0, 9.75) {};
		\node [style=X] (26) at (0.5, 9.75) {};
		\node [style=Z] (27) at (0.5, 10.5) {};
		\node [style=Z] (28) at (0, 10.5) {};
		\node [style=none] (29) at (-1.5, 9.75) {};
	\end{pgfonlayer}
	\begin{pgfonlayer}{edgelayer}
		\draw [in=90, out=180] (9) to (21.center);
		\draw (21.center) to (14);
		\draw (10.center) to (23.center);
		\draw (24.center) to (15.center);
		\draw (19.center) to (17.center);
		\draw (12.center) to (16);
		\draw (12.center) to (13);
		\draw (26) to (11.center);
		\draw (26) to (28);
		\draw [bend right=15, looseness=1.25] (27) to (25);
		\draw [bend right, looseness=1.25] (28) to (25);
		\draw [bend right, looseness=1.25] (26) to (27);
		\draw (28) to (21.center);
		\draw (22.center) to (27);
		\draw [bend right=15] (25) to (27);
		\draw [in=-90, out=-90, looseness=1.25] (25) to (29.center);
		\draw (29.center) to (12.center);
	\end{pgfonlayer}
\end{tikzpicture}\\
&\eq{\ref{ZXA.8}}
\begin{tikzpicture}
	\begin{pgfonlayer}{nodelayer}
		\node [style=Z] (10) at (1, 12.75) {};
		\node [style=none] (11) at (-2.25, 8.75) {};
		\node [style=none] (12) at (0.5, 8.75) {};
		\node [style=none] (13) at (-1.5, 10.5) {};
		\node [style=X] (14) at (-2.25, 11) {};
		\node [style=X] (15) at (-0.75, 11.5) {};
		\node [style=none] (16) at (-0.75, 8.75) {};
		\node [style=X] (17) at (-0.75, 11) {};
		\node [style=none] (18) at (1, 13.5) {};
		\node [style=andin] (19) at (-0.25, 12.25) {};
		\node [style=none] (20) at (1, 8.75) {};
		\node [style=andout] (21) at (-1.5, 10.5) {};
		\node [style=none] (22) at (-0.25, 12.25) {};
		\node [style=none] (23) at (0.5, 13.5) {};
		\node [style=none] (24) at (-2.25, 13.5) {};
		\node [style=none] (25) at (-0.75, 13.5) {};
		\node [style=X] (26) at (0.5, 9.5) {};
		\node [style=Z] (27) at (0, 10.25) {};
		\node [style=none] (28) at (-1.5, 10.25) {};
	\end{pgfonlayer}
	\begin{pgfonlayer}{edgelayer}
		\draw [in=90, out=180] (10) to (22.center);
		\draw (22.center) to (15);
		\draw (11.center) to (24.center);
		\draw (25.center) to (16.center);
		\draw (20.center) to (18.center);
		\draw (13.center) to (17);
		\draw (13.center) to (14);
		\draw (26) to (12.center);
		\draw (26) to (27);
		\draw [in=-60, out=90, looseness=0.75] (27) to (22.center);
		\draw (28.center) to (13.center);
		\draw (23.center) to (26);
		\draw [in=-90, out=-120] (27) to (28.center);
	\end{pgfonlayer}
\end{tikzpicture}
\eq{\ref{ZXA.17}}
\begin{tikzpicture}
	\begin{pgfonlayer}{nodelayer}
		\node [style=andin] (11) at (0, 5.5) {};
		\node [style=none] (12) at (-1.5, 1.5) {};
		\node [style=none] (13) at (1, 7.25) {};
		\node [style=andout] (14) at (-2.25, 2.75) {};
		\node [style=none] (15) at (-2.25, 2.5) {};
		\node [style=X] (16) at (-1.5, 3.5) {};
		\node [style=X] (17) at (-1.5, 4) {};
		\node [style=X] (18) at (-3, 3.5) {};
		\node [style=none] (19) at (-1.5, 7.25) {};
		\node [style=none] (20) at (-3, 1.5) {};
		\node [style=none] (21) at (-2.25, 2.75) {};
		\node [style=none] (22) at (-3, 7.25) {};
		\node [style=none] (23) at (1, 1.5) {};
		\node [style=X] (24) at (0.5, 4.5) {};
		\node [style=none] (25) at (0.5, 1.5) {};
		\node [style=Z] (26) at (1, 6.75) {};
		\node [style=none] (27) at (0.5, 7.25) {};
		\node [style=none] (28) at (-1, 5.5) {};
		\node [style=none] (29) at (0, 5.5) {};
		\node [style=Z] (30) at (-0.5, 6.25) {};
		\node [style=X] (31) at (-1, 4.5) {};
		\node [style=none] (32) at (0, 4) {};
		\node [style=none] (33) at (0, 2.5) {};
		\node [style=andin] (34) at (-1, 5.5) {};
	\end{pgfonlayer}
	\begin{pgfonlayer}{edgelayer}
		\draw (20.center) to (22.center);
		\draw (19.center) to (12.center);
		\draw (23.center) to (13.center);
		\draw (21.center) to (16);
		\draw (21.center) to (18);
		\draw (24) to (25.center);
		\draw (15.center) to (21.center);
		\draw (27.center) to (24);
		\draw [in=90, out=-124] (30) to (28.center);
		\draw [in=90, out=-56] (30) to (29.center);
		\draw [in=37, out=-120] (29.center) to (31);
		\draw (31) to (28.center);
		\draw (31) to (17);
		\draw (29.center) to (24);
		\draw [in=90, out=-45, looseness=1.25] (28.center) to (32.center);
		\draw (32.center) to (33.center);
		\draw [in=-90, out=-90] (33.center) to (15.center);
		\draw [in=90, out=180, looseness=0.75] (26) to (30);
	\end{pgfonlayer}
\end{tikzpicture}
=
\begin{tikzpicture}
	\begin{pgfonlayer}{nodelayer}
		\node [style=andin] (12) at (0, 5.25) {};
		\node [style=none] (13) at (-1.5, 2) {};
		\node [style=none] (14) at (1, 7.25) {};
		\node [style=none] (15) at (-0.75, 4.5) {};
		\node [style=X] (16) at (-1.5, 2.5) {};
		\node [style=X] (17) at (-2.5, 2.5) {};
		\node [style=none] (18) at (-1.5, 7.25) {};
		\node [style=none] (19) at (-2.5, 2) {};
		\node [style=none] (20) at (-2, 3.25) {};
		\node [style=none] (21) at (-2.5, 7.25) {};
		\node [style=none] (22) at (1, 2) {};
		\node [style=X] (23) at (0.5, 4.25) {};
		\node [style=none] (24) at (0.5, 2) {};
		\node [style=Z] (25) at (1, 6.75) {};
		\node [style=none] (26) at (0.5, 7.25) {};
		\node [style=none] (27) at (-1, 5.25) {};
		\node [style=none] (28) at (0, 5.25) {};
		\node [style=Z] (29) at (-0.5, 6) {};
		\node [style=X] (30) at (-1.5, 4.25) {};
		\node [style=none] (31) at (-0.75, 4.5) {};
		\node [style=andin] (32) at (-1, 5.25) {};
		\node [style=andin] (33) at (-2, 3.25) {};
	\end{pgfonlayer}
	\begin{pgfonlayer}{edgelayer}
		\draw (19.center) to (21.center);
		\draw (18.center) to (13.center);
		\draw (22.center) to (14.center);
		\draw (20.center) to (16);
		\draw (20.center) to (17);
		\draw (23) to (24.center);
		\draw [in=90, out=-90] (15.center) to (20.center);
		\draw (26.center) to (23);
		\draw [in=90, out=-124] (29) to (27.center);
		\draw [in=90, out=-56] (29) to (28.center);
		\draw [in=37, out=-120] (28.center) to (30);
		\draw (30) to (27.center);
		\draw (28.center) to (23);
		\draw [in=90, out=-45, looseness=1.25] (27.center) to (31.center);
		\draw [in=90, out=180] (25) to (29);
	\end{pgfonlayer}
\end{tikzpicture}\\
&\eq{\ref{ZXA.11}}
\begin{tikzpicture}
	\begin{pgfonlayer}{nodelayer}
		\node [style=andin] (13) at (0, 5.5) {};
		\node [style=none] (14) at (-1.5, 1.25) {};
		\node [style=none] (15) at (1, 7.5) {};
		\node [style=none] (16) at (-0.5, 4.5) {};
		\node [style=none] (17) at (-1.5, 7.5) {};
		\node [style=none] (18) at (-2.5, 1.25) {};
		\node [style=none] (19) at (-2, 3.25) {};
		\node [style=none] (20) at (-2.5, 7.5) {};
		\node [style=none] (21) at (1, 1.25) {};
		\node [style=X] (22) at (0.5, 4.5) {};
		\node [style=none] (23) at (0.5, 1.25) {};
		\node [style=Z] (24) at (1, 7) {};
		\node [style=none] (25) at (0.5, 7.5) {};
		\node [style=none] (26) at (-1, 5.5) {};
		\node [style=none] (27) at (0, 5.5) {};
		\node [style=Z] (28) at (-0.5, 6.25) {};
		\node [style=X] (29) at (-1.5, 4.5) {};
		\node [style=none] (30) at (-0.5, 4.5) {};
		\node [style=andin] (31) at (-1, 5.5) {};
		\node [style=andin] (32) at (-2, 3.25) {};
		\node [style=X] (33) at (-2.5, 1.75) {};
		\node [style=X] (34) at (-1.5, 1.75) {};
		\node [style=none] (35) at (-2.25, 2.5) {};
		\node [style=none] (36) at (-1.75, 2.5) {};
	\end{pgfonlayer}
	\begin{pgfonlayer}{edgelayer}
		\draw (18.center) to (20.center);
		\draw (17.center) to (14.center);
		\draw (21.center) to (15.center);
		\draw (22) to (23.center);
		\draw [in=90, out=-90, looseness=1.25] (16.center) to (19.center);
		\draw (25.center) to (22);
		\draw [in=90, out=-124] (28) to (26.center);
		\draw [in=90, out=-56] (28) to (27.center);
		\draw [in=37, out=-120] (27.center) to (29);
		\draw (29) to (26.center);
		\draw (27.center) to (22);
		\draw [in=90, out=-45, looseness=1.25] (26.center) to (30.center);
		\draw [in=90, out=180] (24) to (28);
		\draw [in=90, out=-108] (19.center) to (35.center);
		\draw [in=135, out=-90] (35.center) to (34);
		\draw [in=-72, out=90] (36.center) to (19.center);
		\draw [in=45, out=-90] (36.center) to (33);
	\end{pgfonlayer}
\end{tikzpicture}
\eq{\ref{ZXA.9}}
\begin{tikzpicture}
	\begin{pgfonlayer}{nodelayer}
		\node [style=andin] (14) at (0, 6.25) {};
		\node [style=none] (15) at (-2, 2.5) {};
		\node [style=none] (16) at (1, 8.25) {};
		\node [style=none] (17) at (-0.5, 4.25) {};
		\node [style=X] (18) at (-2.5, 3.25) {};
		\node [style=X] (19) at (-2, 3.25) {};
		\node [style=none] (20) at (-2, 8.25) {};
		\node [style=none] (21) at (-2.5, 2.5) {};
		\node [style=none] (22) at (-2.5, 8.25) {};
		\node [style=none] (23) at (1, 2.5) {};
		\node [style=X] (24) at (0.5, 4.75) {};
		\node [style=none] (25) at (0.5, 2.5) {};
		\node [style=Z] (26) at (1, 7.75) {};
		\node [style=none] (27) at (0.5, 8.25) {};
		\node [style=none] (28) at (-1, 6.25) {};
		\node [style=none] (29) at (0, 6.25) {};
		\node [style=Z] (30) at (-0.5, 7) {};
		\node [style=X] (31) at (-2, 4.25) {};
		\node [style=none] (32) at (-0.5, 4.25) {};
		\node [style=none] (33) at (-1.5, 5.5) {};
		\node [style=andin] (34) at (-1, 6.25) {};
		\node [style=andin] (35) at (-1.5, 5.5) {};
		\node [style=none] (36) at (-1.25, 4.25) {};
	\end{pgfonlayer}
	\begin{pgfonlayer}{edgelayer}
		\draw (21.center) to (22.center);
		\draw (20.center) to (15.center);
		\draw (23.center) to (16.center);
		\draw (24) to (25.center);
		\draw (27.center) to (24);
		\draw [in=90, out=-124] (30) to (28.center);
		\draw [in=90, out=-56] (30) to (29.center);
		\draw [in=37, out=-120] (29.center) to (31);
		\draw (29.center) to (24);
		\draw [in=90, out=-45] (28.center) to (32.center);
		\draw [in=90, out=180] (26) to (30);
		\draw [in=90, out=-124] (28.center) to (33.center);
		\draw (33.center) to (31);
		\draw [in=53, out=-90, looseness=0.75] (17.center) to (18);
		\draw [in=90, out=-60] (33.center) to (36.center);
		\draw [in=49, out=-90, looseness=0.75] (36.center) to (19);
	\end{pgfonlayer}
\end{tikzpicture}
\eq{\ref{ZXA.3}}
\begin{tikzpicture}
	\begin{pgfonlayer}{nodelayer}
		\node [style=andin] (15) at (0, 5.75) {};
		\node [style=none] (16) at (-2, 2.5) {};
		\node [style=none] (17) at (1, 7.5) {};
		\node [style=none] (18) at (-0.5, 4.25) {};
		\node [style=X] (19) at (-2.5, 3.25) {};
		\node [style=X] (20) at (-2, 4.25) {};
		\node [style=none] (21) at (-2, 7.5) {};
		\node [style=none] (22) at (-2.5, 2.5) {};
		\node [style=none] (23) at (-2.5, 7.5) {};
		\node [style=none] (24) at (1, 2.5) {};
		\node [style=X] (25) at (0.5, 4.75) {};
		\node [style=none] (26) at (0.5, 2.5) {};
		\node [style=Z] (27) at (1, 7) {};
		\node [style=none] (28) at (0.5, 7.5) {};
		\node [style=none] (29) at (-1, 5.75) {};
		\node [style=none] (30) at (0, 5.75) {};
		\node [style=Z] (31) at (-0.5, 6.5) {};
		\node [style=X] (32) at (-2, 4.25) {};
		\node [style=none] (33) at (-0.5, 4.25) {};
		\node [style=none] (34) at (-1.5, 5) {};
		\node [style=andin] (35) at (-1, 5.75) {};
		\node [style=andin] (36) at (-1.5, 5) {};
	\end{pgfonlayer}
	\begin{pgfonlayer}{edgelayer}
		\draw (22.center) to (23.center);
		\draw (21.center) to (16.center);
		\draw (24.center) to (17.center);
		\draw (25) to (26.center);
		\draw (28.center) to (25);
		\draw [in=90, out=-124] (31) to (29.center);
		\draw [in=90, out=-56] (31) to (30.center);
		\draw [in=0, out=-120, looseness=0.75] (30.center) to (32);
		\draw (30.center) to (25);
		\draw [in=90, out=-45] (29.center) to (33.center);
		\draw [in=90, out=180, looseness=0.75] (27) to (31);
		\draw [in=90, out=-124] (29.center) to (34.center);
		\draw [in=60, out=-142, looseness=0.75] (34.center) to (32);
		\draw [in=53, out=-90, looseness=0.75] (18.center) to (19);
		\draw [bend left=45] (34.center) to (20);
	\end{pgfonlayer}
\end{tikzpicture}\\
&\eq{\ref{ZXA.15}}
\begin{tikzpicture}
	\begin{pgfonlayer}{nodelayer}
		\node [style=andin] (16) at (0, 6.25) {};
		\node [style=none] (17) at (-1.5, 2.5) {};
		\node [style=none] (18) at (1, 8.25) {};
		\node [style=none] (19) at (-0.5, 4.25) {};
		\node [style=X] (20) at (-2, 3.25) {};
		\node [style=X] (21) at (-1.5, 4.25) {};
		\node [style=none] (22) at (-1.5, 8.25) {};
		\node [style=none] (23) at (-2, 2.5) {};
		\node [style=none] (24) at (-2, 8.25) {};
		\node [style=none] (25) at (1, 2.5) {};
		\node [style=X] (26) at (0.5, 4.75) {};
		\node [style=none] (27) at (0.5, 2.5) {};
		\node [style=Z] (28) at (1, 7.75) {};
		\node [style=none] (29) at (0.5, 8.25) {};
		\node [style=none] (30) at (-1, 6.25) {};
		\node [style=none] (31) at (0, 6.25) {};
		\node [style=Z] (32) at (-0.5, 7) {};
		\node [style=X] (33) at (-1.5, 4.25) {};
		\node [style=none] (34) at (-0.5, 4.25) {};
		\node [style=andin] (35) at (-1, 6.25) {};
	\end{pgfonlayer}
	\begin{pgfonlayer}{edgelayer}
		\draw (23.center) to (24.center);
		\draw (22.center) to (17.center);
		\draw (25.center) to (18.center);
		\draw (26) to (27.center);
		\draw (29.center) to (26);
		\draw [in=90, out=-124] (32) to (30.center);
		\draw [in=90, out=-56] (32) to (31.center);
		\draw [in=0, out=-120, looseness=0.75] (31.center) to (33);
		\draw (31.center) to (26);
		\draw [in=90, out=-45] (30.center) to (34.center);
		\draw [in=90, out=180] (28) to (32);
		\draw [in=53, out=-90, looseness=0.75] (19.center) to (20);
		\draw [in=75, out=-105] (30.center) to (21);
	\end{pgfonlayer}
\end{tikzpicture}
\eq{\ref{ZXA.3}}
\begin{tikzpicture}
	\begin{pgfonlayer}{nodelayer}
		\node [style=none] (17) at (-1.5, 3.5) {};
		\node [style=none] (18) at (1.25, 8.25) {};
		\node [style=none] (19) at (-0.5, 6.25) {};
		\node [style=X] (20) at (-2, 5.75) {};
		\node [style=X] (21) at (-1.5, 5) {};
		\node [style=none] (22) at (-1.5, 8.25) {};
		\node [style=none] (23) at (-2, 3.5) {};
		\node [style=none] (24) at (-2, 8.25) {};
		\node [style=none] (25) at (1.25, 3.5) {};
		\node [style=X] (26) at (0.75, 5.25) {};
		\node [style=none] (27) at (0.75, 3.5) {};
		\node [style=Z] (28) at (1.25, 7.75) {};
		\node [style=none] (29) at (0.75, 8.25) {};
		\node [style=none] (30) at (-1, 7) {};
		\node [style=none] (31) at (0.25, 6.25) {};
		\node [style=X] (32) at (-1.5, 4.25) {};
		\node [style=none] (33) at (-0.5, 6.25) {};
		\node [style=andin] (34) at (-1, 7) {};
		\node [style=Z] (35) at (1.25, 7) {};
		\node [style=andin] (36) at (0.25, 6.25) {};
	\end{pgfonlayer}
	\begin{pgfonlayer}{edgelayer}
		\draw (23.center) to (24.center);
		\draw (22.center) to (17.center);
		\draw (25.center) to (18.center);
		\draw (26) to (27.center);
		\draw (29.center) to (26);
		\draw [in=0, out=-120, looseness=0.75] (31.center) to (32);
		\draw (31.center) to (26);
		\draw [in=90, out=-45] (30.center) to (33.center);
		\draw [in=53, out=-90, looseness=0.75] (19.center) to (20);
		\draw [in=60, out=-120, looseness=1.25] (30.center) to (21);
		\draw [in=90, out=180, looseness=0.75] (28) to (30.center);
		\draw [in=90, out=180] (35) to (31.center);
	\end{pgfonlayer}
\end{tikzpicture}
\eq{\ref{ZXA.11}}
\begin{tikzpicture}
	\begin{pgfonlayer}{nodelayer}
		\node [style=none] (18) at (-0.25, 4.75) {};
		\node [style=none] (19) at (1.25, 8.25) {};
		\node [style=X] (20) at (-0.25, 6) {};
		\node [style=X] (21) at (-1.25, 6) {};
		\node [style=none] (22) at (-0.25, 8.25) {};
		\node [style=none] (23) at (-1.25, 4.75) {};
		\node [style=none] (24) at (-1.25, 8.25) {};
		\node [style=none] (25) at (1.25, 4.75) {};
		\node [style=X] (26) at (0.75, 5.25) {};
		\node [style=none] (27) at (0.75, 4.75) {};
		\node [style=Z] (28) at (1.25, 7.75) {};
		\node [style=none] (29) at (0.75, 8.25) {};
		\node [style=none] (30) at (-0.75, 7) {};
		\node [style=none] (31) at (0.25, 6.25) {};
		\node [style=X] (32) at (-0.25, 5.25) {};
		\node [style=andin] (33) at (-0.75, 7) {};
		\node [style=Z] (34) at (1.25, 7) {};
		\node [style=andin] (35) at (0.25, 6.25) {};
	\end{pgfonlayer}
	\begin{pgfonlayer}{edgelayer}
		\draw (23.center) to (24.center);
		\draw (22.center) to (18.center);
		\draw (25.center) to (19.center);
		\draw (26) to (27.center);
		\draw (29.center) to (26);
		\draw (31.center) to (32);
		\draw (31.center) to (26);
		\draw [in=90, out=180, looseness=0.75] (28) to (30.center);
		\draw [in=90, out=180] (34) to (31.center);
		\draw (30.center) to (21);
		\draw (20) to (30.center);
	\end{pgfonlayer}
\end{tikzpicture}
=
\left\llbracket
\begin{tikzpicture}
	\begin{pgfonlayer}{nodelayer}
		\node [style=nothing] (19) at (-0.5, 0.5) {};
		\node [style=nothing] (20) at (0, 0.5) {};
		\node [style=nothing] (21) at (-1, 0.5) {};
		\node [style=nothing] (22) at (-1.5, 0.5) {};
		\node [style=nothing] (23) at (-0.5, 2) {};
		\node [style=nothing] (24) at (-1.5, 2) {};
		\node [style=nothing] (25) at (0, 2) {};
		\node [style=nothing] (26) at (-1, 2) {};
		\node [style=dot] (27) at (-1, 1.5) {};
		\node [style=dot] (28) at (-0.5, 1.5) {};
		\node [style=dot] (29) at (-1.5, 1) {};
		\node [style=dot] (30) at (-1, 1) {};
		\node [style=oplus] (31) at (0, 1) {};
		\node [style=oplus] (32) at (0, 1.5) {};
	\end{pgfonlayer}
	\begin{pgfonlayer}{edgelayer}
		\draw (27) to (28);
		\draw (22) to (29);
		\draw (29) to (24);
		\draw (21) to (30);
		\draw (30) to (27);
		\draw (27) to (26);
		\draw (19) to (28);
		\draw (28) to (23);
		\draw (20) to (31);
		\draw (31) to (32);
		\draw (32) to (25);
		\draw (32) to (28);
		\draw (31) to (30);
		\draw (30) to (29);
	\end{pgfonlayer}
\end{tikzpicture}
\right\rrbracket_{\hat{\TOF}}
\end{align*}
\endgroup


\item[\ref{TOF.13}:]
\begingroup
\allowdisplaybreaks
\begin{align*}
\left\llbracket
\begin{tikzpicture}
	\begin{pgfonlayer}{nodelayer}
		\node [style=nothing] (21) at (0, 0.5) {};
		\node [style=nothing] (22) at (-1, 0.5) {};
		\node [style=nothing] (23) at (-0.5, 0.5) {};
		\node [style=nothing] (24) at (-1.5, 0.5) {};
		\node [style=nothing] (25) at (0, 2.5) {};
		\node [style=dot] (26) at (-1.5, 1) {};
		\node [style=dot] (27) at (-1, 1) {};
		\node [style=dot] (28) at (-0.5, 1.5) {};
		\node [style=oplus] (29) at (-0.5, 1) {};
		\node [style=oplus] (30) at (0, 1.5) {};
		\node [style=nothing] (31) at (-0.5, 2.5) {};
		\node [style=nothing] (32) at (-1.5, 2.5) {};
		\node [style=nothing] (33) at (-1, 2.5) {};
		\node [style=oplus] (34) at (-0.5, 2) {};
		\node [style=dot] (35) at (-1, 2) {};
		\node [style=dot] (36) at (-1.5, 2) {};
	\end{pgfonlayer}
	\begin{pgfonlayer}{edgelayer}
		\draw (26) to (24);
		\draw (27) to (22);
		\draw (23) to (29);
		\draw (29) to (28);
		\draw (25) to (30);
		\draw (30) to (21);
		\draw (29) to (27);
		\draw (27) to (26);
		\draw (30) to (28);
		\draw (26) to (36);
		\draw (36) to (32);
		\draw (33) to (35);
		\draw (35) to (27);
		\draw (28) to (34);
		\draw (34) to (31);
		\draw (34) to (35);
		\draw (35) to (36);
	\end{pgfonlayer}
\end{tikzpicture}
\right\rrbracket_{\hat{\TOF}}
&=
\begin{tikzpicture}
	\begin{pgfonlayer}{nodelayer}
		\node [style=none] (22) at (-2, 7) {};
		\node [style=none] (23) at (-1, 7) {};
		\node [style=none] (24) at (-2, 12.25) {};
		\node [style=none] (25) at (-1, 12.25) {};
		\node [style=X] (26) at (-2, 8) {};
		\node [style=X] (27) at (-1, 8) {};
		\node [style=none] (28) at (-1.5, 9) {};
		\node [style=Z] (29) at (-0.5, 9.5) {};
		\node [style=andin] (30) at (-1.5, 9) {};
		\node [style=andin] (31) at (-1.5, 11) {};
		\node [style=X] (32) at (-1, 10) {};
		\node [style=Z] (33) at (-0.5, 11.5) {};
		\node [style=X] (34) at (-2, 10) {};
		\node [style=none] (35) at (-1.5, 11) {};
		\node [style=X] (36) at (-0.5, 10.5) {};
		\node [style=Z] (37) at (0, 10.5) {};
		\node [style=none] (38) at (-0.5, 12.25) {};
		\node [style=none] (39) at (-0.5, 7) {};
		\node [style=none] (40) at (0, 12.25) {};
		\node [style=none] (41) at (0, 7) {};
	\end{pgfonlayer}
	\begin{pgfonlayer}{edgelayer}
		\draw [in=90, out=180] (29) to (28.center);
		\draw (28.center) to (26);
		\draw (27) to (28.center);
		\draw [in=90, out=180] (33) to (35.center);
		\draw (35.center) to (34);
		\draw (32) to (35.center);
		\draw (37) to (36);
		\draw (40.center) to (41.center);
		\draw (39.center) to (38.center);
		\draw (25.center) to (23.center);
		\draw (22.center) to (24.center);
	\end{pgfonlayer}
\end{tikzpicture}
\eq{\ref{ZXA.3}}
\begin{tikzpicture}
	\begin{pgfonlayer}{nodelayer}
		\node [style=none] (23) at (-2, 8) {};
		\node [style=none] (24) at (-1, 8) {};
		\node [style=none] (25) at (-2, 12) {};
		\node [style=none] (26) at (-1, 12) {};
		\node [style=andin] (27) at (-1.5, 11) {};
		\node [style=X] (28) at (-1, 10) {};
		\node [style=Z] (29) at (-0.5, 11.5) {};
		\node [style=X] (30) at (-2, 10) {};
		\node [style=none] (31) at (-1.5, 11) {};
		\node [style=X] (32) at (-0.5, 10) {};
		\node [style=Z] (33) at (0, 10) {};
		\node [style=none] (34) at (-0.5, 12) {};
		\node [style=none] (35) at (-0.5, 8) {};
		\node [style=none] (36) at (0, 12) {};
		\node [style=none] (37) at (0, 8) {};
		\node [style=andout] (38) at (-1.5, 9) {};
		\node [style=X] (39) at (-1, 10) {};
		\node [style=Z] (40) at (-0.5, 8.5) {};
		\node [style=X] (41) at (-2, 10) {};
		\node [style=none] (42) at (-1.5, 9) {};
	\end{pgfonlayer}
	\begin{pgfonlayer}{edgelayer}
		\draw [in=90, out=180] (29) to (31.center);
		\draw (31.center) to (30);
		\draw (28) to (31.center);
		\draw (33) to (32);
		\draw (36.center) to (37.center);
		\draw (35.center) to (34.center);
		\draw (26.center) to (24.center);
		\draw (23.center) to (25.center);
		\draw [in=-90, out=180] (40) to (42.center);
		\draw (42.center) to (41);
		\draw (39) to (42.center);
	\end{pgfonlayer}
\end{tikzpicture}\\
&\eq{\ref{ZXA.3}}
\begin{tikzpicture}
	\begin{pgfonlayer}{nodelayer}
		\node [style=none] (24) at (-3, 9.5) {};
		\node [style=none] (25) at (-1, 9.5) {};
		\node [style=none] (26) at (-3, 13.5) {};
		\node [style=none] (27) at (-1, 13.5) {};
		\node [style=andin] (28) at (-2, 12.5) {};
		\node [style=X] (29) at (-1.5, 11.5) {};
		\node [style=Z] (30) at (-0.5, 13) {};
		\node [style=X] (31) at (-2.5, 11.5) {};
		\node [style=none] (32) at (-2, 12.5) {};
		\node [style=X] (33) at (-0.5, 11.5) {};
		\node [style=Z] (34) at (0, 11.5) {};
		\node [style=none] (35) at (-0.5, 13.5) {};
		\node [style=none] (36) at (-0.5, 9.5) {};
		\node [style=none] (37) at (0, 13.5) {};
		\node [style=none] (38) at (0, 9.5) {};
		\node [style=X] (39) at (-1, 11.5) {};
		\node [style=Z] (40) at (-0.5, 10) {};
		\node [style=X] (41) at (-3, 11.5) {};
		\node [style=none] (42) at (-2, 10.5) {};
		\node [style=andout] (43) at (-2, 10.5) {};
	\end{pgfonlayer}
	\begin{pgfonlayer}{edgelayer}
		\draw [in=90, out=180] (30) to (32.center);
		\draw (32.center) to (31);
		\draw (29) to (32.center);
		\draw (34) to (33);
		\draw (37.center) to (38.center);
		\draw (36.center) to (35.center);
		\draw (27.center) to (25.center);
		\draw (24.center) to (26.center);
		\draw [in=-90, out=180] (40) to (42.center);
		\draw (31) to (41);
		\draw (31) to (42.center);
		\draw (29) to (39);
		\draw (29) to (42.center);
	\end{pgfonlayer}
\end{tikzpicture}
=
\begin{tikzpicture}
	\begin{pgfonlayer}{nodelayer}
		\node [style=none] (25) at (-3.5, 9.5) {};
		\node [style=none] (26) at (-1, 9.5) {};
		\node [style=none] (27) at (-3.5, 13.5) {};
		\node [style=none] (28) at (-1, 13.5) {};
		\node [style=X] (29) at (-1.75, 11.5) {};
		\node [style=none] (30) at (-2.5, 9.75) {};
		\node [style=X] (31) at (-2.5, 11.5) {};
		\node [style=none] (32) at (-2.5, 10.5) {};
		\node [style=X] (33) at (-0.5, 11.5) {};
		\node [style=Z] (34) at (0, 11.5) {};
		\node [style=none] (35) at (-0.5, 13.5) {};
		\node [style=none] (36) at (-0.5, 9.5) {};
		\node [style=none] (37) at (0, 13.5) {};
		\node [style=none] (38) at (0, 9.5) {};
		\node [style=X] (39) at (-1, 11.5) {};
		\node [style=Z] (40) at (-0.5, 10) {};
		\node [style=X] (41) at (-3.5, 12.5) {};
		\node [style=none] (42) at (-1.75, 10.5) {};
		\node [style=andout] (43) at (-1.75, 10.5) {};
		\node [style=andout] (44) at (-2.5, 10.5) {};
		\node [style=none] (45) at (-3, 9.75) {};
		\node [style=none] (46) at (-3, 12.25) {};
		\node [style=Z] (47) at (-0.5, 12.75) {};
	\end{pgfonlayer}
	\begin{pgfonlayer}{edgelayer}
		\draw (30.center) to (32.center);
		\draw [in=-120, out=120, looseness=1.25] (32.center) to (31);
		\draw (29) to (32.center);
		\draw (34) to (33);
		\draw (37.center) to (38.center);
		\draw (36.center) to (35.center);
		\draw (28.center) to (26.center);
		\draw (25.center) to (27.center);
		\draw [in=-90, out=180] (40) to (42.center);
		\draw [in=-63, out=90] (31) to (41);
		\draw (31) to (42.center);
		\draw (29) to (39);
		\draw [in=60, out=-60, looseness=1.25] (29) to (42.center);
		\draw [in=90, out=-174, looseness=0.50] (47) to (46.center);
		\draw (46.center) to (45.center);
		\draw [in=-90, out=-90, looseness=1.50] (45.center) to (30.center);
	\end{pgfonlayer}
\end{tikzpicture}\\
&\eq{\ref{ZXA.12}}
\begin{tikzpicture}
	\begin{pgfonlayer}{nodelayer}
		\node [style=none] (26) at (-3.5, 9.5) {};
		\node [style=none] (27) at (-1, 9.5) {};
		\node [style=none] (28) at (-3.5, 14.25) {};
		\node [style=none] (29) at (-1, 14.25) {};
		\node [style=X] (30) at (-0.5, 12.25) {};
		\node [style=Z] (31) at (0, 12.25) {};
		\node [style=none] (32) at (-0.5, 14.25) {};
		\node [style=none] (33) at (-0.5, 9.5) {};
		\node [style=none] (34) at (0, 14.25) {};
		\node [style=none] (35) at (0, 9.5) {};
		\node [style=X] (36) at (-1, 12.25) {};
		\node [style=Z] (37) at (-0.5, 10) {};
		\node [style=X] (38) at (-3.5, 13.25) {};
		\node [style=none] (39) at (-2.75, 10.5) {};
		\node [style=none] (40) at (-2.75, 13) {};
		\node [style=Z] (41) at (-0.5, 13.5) {};
		\node [style=none] (42) at (-2, 11.5) {};
		\node [style=andout] (43) at (-2, 11.5) {};
		\node [style=X] (44) at (-2, 10.5) {};
	\end{pgfonlayer}
	\begin{pgfonlayer}{edgelayer}
		\draw (31) to (30);
		\draw (34.center) to (35.center);
		\draw (33.center) to (32.center);
		\draw (29.center) to (27.center);
		\draw (26.center) to (28.center);
		\draw [in=90, out=-174, looseness=0.50] (41) to (40.center);
		\draw (40.center) to (39.center);
		\draw [in=60, out=-143] (36) to (42.center);
		\draw [in=-49, out=120] (42.center) to (38);
		\draw (42.center) to (44);
		\draw [in=180, out=-60, looseness=1.25] (44) to (37);
		\draw [in=-90, out=-120, looseness=2.00] (44) to (39.center);
	\end{pgfonlayer}
\end{tikzpicture}
\eq{\ref{ZXA.5}}
\begin{tikzpicture}
	\begin{pgfonlayer}{nodelayer}
		\node [style=Z] (27) at (0, 13) {};
		\node [style=none] (28) at (-2, 11.5) {};
		\node [style=Z] (29) at (0.5, 12.25) {};
		\node [style=none] (30) at (0.5, 13.75) {};
		\node [style=X] (31) at (-1, 12.25) {};
		\node [style=none] (32) at (-3.25, 13.75) {};
		\node [style=none] (33) at (-1, 13.75) {};
		\node [style=none] (34) at (0, 13.75) {};
		\node [style=none] (35) at (-2.75, 12.5) {};
		\node [style=X] (36) at (-3.25, 13) {};
		\node [style=X] (37) at (-2, 10.5) {};
		\node [style=none] (38) at (-1, 9.5) {};
		\node [style=andout] (39) at (-2, 11.5) {};
		\node [style=none] (40) at (0.5, 9.5) {};
		\node [style=none] (41) at (-3.25, 9.5) {};
		\node [style=none] (42) at (-2.75, 10.5) {};
		\node [style=none] (43) at (0, 9.5) {};
		\node [style=X] (44) at (-0.5, 10.5) {};
		\node [style=X] (45) at (0, 10.5) {};
		\node [style=Z] (46) at (0, 11.5) {};
		\node [style=Z] (47) at (-0.5, 11.5) {};
	\end{pgfonlayer}
	\begin{pgfonlayer}{edgelayer}
		\draw (30.center) to (40.center);
		\draw (33.center) to (38.center);
		\draw (41.center) to (32.center);
		\draw [in=90, out=-174, looseness=0.50] (27) to (35.center);
		\draw (35.center) to (42.center);
		\draw [in=60, out=-143] (31) to (28.center);
		\draw [in=-49, out=120] (28.center) to (36);
		\draw (28.center) to (37);
		\draw [in=-90, out=-120, looseness=2.00] (37) to (42.center);
		\draw (46) to (44);
		\draw [in=-120, out=120, looseness=1.25] (44) to (47);
		\draw (47) to (45);
		\draw [in=-60, out=60, looseness=1.25] (45) to (46);
		\draw [in=-75, out=-90, looseness=1.25] (44) to (37);
		\draw (45) to (43.center);
		\draw [in=-124, out=90] (46) to (29);
		\draw [in=90, out=-90] (27) to (47);
		\draw (27) to (34.center);
	\end{pgfonlayer}
\end{tikzpicture}\\
&\eq{\ref{ZXA.1},\ref{ZXA.3}}
\begin{tikzpicture}
	\begin{pgfonlayer}{nodelayer}
		\node [style=none] (28) at (-1.5, 10.5) {};
		\node [style=Z] (29) at (0.5, 12.25) {};
		\node [style=none] (30) at (0.5, 14.25) {};
		\node [style=X] (31) at (-1, 12.25) {};
		\node [style=none] (32) at (-2, 14.25) {};
		\node [style=none] (33) at (-1, 14.25) {};
		\node [style=none] (34) at (0, 14.25) {};
		\node [style=X] (35) at (-2, 12.25) {};
		\node [style=none] (36) at (-1, 9.5) {};
		\node [style=andout] (37) at (-1.5, 10.5) {};
		\node [style=none] (38) at (0.5, 9.5) {};
		\node [style=none] (39) at (-2, 9.5) {};
		\node [style=none] (40) at (0, 9.5) {};
		\node [style=X] (41) at (-0.5, 10.5) {};
		\node [style=X] (42) at (0, 10.5) {};
		\node [style=Z] (43) at (0, 11.5) {};
		\node [style=Z] (44) at (-0.5, 11.5) {};
	\end{pgfonlayer}
	\begin{pgfonlayer}{edgelayer}
		\draw (30.center) to (38.center);
		\draw (33.center) to (36.center);
		\draw (39.center) to (32.center);
		\draw (31) to (28.center);
		\draw (28.center) to (35);
		\draw (43) to (41);
		\draw [in=-120, out=120, looseness=1.25] (41) to (44);
		\draw (44) to (42);
		\draw [in=-60, out=60, looseness=1.25] (42) to (43);
		\draw (42) to (40.center);
		\draw [in=-124, out=90] (43) to (29);
		\draw (44) to (41);
		\draw [in=-90, out=-90] (41) to (28.center);
		\draw [in=90, out=-90] (34.center) to (44);
	\end{pgfonlayer}
\end{tikzpicture}
\eq{\ref{ZXA.8}}
\begin{tikzpicture}
	\begin{pgfonlayer}{nodelayer}
		\node [style=none] (29) at (-1.5, 10) {};
		\node [style=Z] (30) at (0.5, 11.75) {};
		\node [style=none] (31) at (0.5, 12.25) {};
		\node [style=X] (32) at (-1, 11) {};
		\node [style=none] (33) at (-2, 12.25) {};
		\node [style=none] (34) at (-1, 12.25) {};
		\node [style=none] (35) at (0, 12.25) {};
		\node [style=X] (36) at (-2, 11) {};
		\node [style=none] (37) at (-1, 9.5) {};
		\node [style=andout] (38) at (-1.5, 10) {};
		\node [style=none] (39) at (0.5, 9.5) {};
		\node [style=none] (40) at (-2, 9.5) {};
		\node [style=none] (41) at (0, 9.5) {};
		\node [style=none] (42) at (-0.5, 10) {};
		\node [style=X] (43) at (0, 10) {};
		\node [style=Z] (44) at (0, 11) {};
		\node [style=none] (45) at (-0.5, 11) {};
	\end{pgfonlayer}
	\begin{pgfonlayer}{edgelayer}
		\draw (31.center) to (39.center);
		\draw (34.center) to (37.center);
		\draw (40.center) to (33.center);
		\draw (32) to (29.center);
		\draw (29.center) to (36);
		\draw [in=90, out=-117] (44) to (42.center);
		\draw [in=117, out=-90] (45.center) to (43);
		\draw [in=-60, out=60, looseness=1.25] (43) to (44);
		\draw (43) to (41.center);
		\draw [in=-124, out=90] (44) to (30);
		\draw [in=-90, out=-90] (42.center) to (29.center);
		\draw [in=90, out=-90] (35.center) to (45.center);
	\end{pgfonlayer}
\end{tikzpicture}\\
&\eq{\ref{ZXA.1}}
\begin{tikzpicture}
	\begin{pgfonlayer}{nodelayer}
		\node [style=none] (30) at (-1.5, 10.75) {};
		\node [style=Z] (31) at (0, 10.25) {};
		\node [style=none] (32) at (0, 12.5) {};
		\node [style=X] (33) at (-1, 11.75) {};
		\node [style=none] (34) at (-2, 12.5) {};
		\node [style=none] (35) at (-1, 12.5) {};
		\node [style=none] (36) at (-0.5, 12.5) {};
		\node [style=X] (37) at (-2, 11.75) {};
		\node [style=none] (38) at (-1, 9.5) {};
		\node [style=andout] (39) at (-1.5, 10.75) {};
		\node [style=none] (40) at (0, 9.5) {};
		\node [style=none] (41) at (-2, 9.5) {};
		\node [style=none] (42) at (-0.5, 9.5) {};
		\node [style=X] (43) at (-0.5, 11.75) {};
		\node [style=Z] (44) at (0, 11.75) {};
	\end{pgfonlayer}
	\begin{pgfonlayer}{edgelayer}
		\draw (32.center) to (40.center);
		\draw (35.center) to (38.center);
		\draw (41.center) to (34.center);
		\draw (33) to (30.center);
		\draw (30.center) to (37);
		\draw (43) to (44);
		\draw (43) to (42.center);
		\draw [in=-90, out=180, looseness=0.75] (31) to (30.center);
		\draw (36.center) to (43);
	\end{pgfonlayer}
\end{tikzpicture}
=
\left\llbracket
\begin{tikzpicture}
	\begin{pgfonlayer}{nodelayer}
		\node [style=nothing] (31) at (0, 9.5) {};
		\node [style=nothing] (32) at (-1, 9.5) {};
		\node [style=nothing] (33) at (-0.5, 9.5) {};
		\node [style=nothing] (34) at (-1.5, 9.5) {};
		\node [style=dot] (35) at (-1.5, 10) {};
		\node [style=dot] (36) at (-1, 10) {};
		\node [style=oplus] (37) at (0, 10) {};
		\node [style=nothing] (38) at (-0.5, 11) {};
		\node [style=nothing] (39) at (-1, 11) {};
		\node [style=nothing] (40) at (0, 11) {};
		\node [style=nothing] (41) at (-1.5, 11) {};
		\node [style=dot] (42) at (-0.5, 10.5) {};
		\node [style=oplus] (43) at (0, 10.5) {};
	\end{pgfonlayer}
	\begin{pgfonlayer}{edgelayer}
		\draw (31) to (37);
		\draw (32) to (36);
		\draw (35) to (34);
		\draw (35) to (36);
		\draw (36) to (37);
		\draw (42) to (43);
		\draw (43) to (40);
		\draw (43) to (37);
		\draw (33) to (42);
		\draw (35) to (41);
		\draw (39) to (36);
		\draw (42) to (38);
	\end{pgfonlayer}
\end{tikzpicture}
\right\rrbracket_{\hat{\TOF}}
\end{align*}
\endgroup

\item[\ref{TOF.14}:]
\begin{align*}
\left\llbracket
\begin{tikzpicture}
	\begin{pgfonlayer}{nodelayer}
		\node [style=nothing] (32) at (0, 9.5) {};
		\node [style=nothing] (33) at (-0.5, 9.5) {};
		\node [style=nothing] (34) at (-0.5, 11.5) {};
		\node [style=nothing] (35) at (0, 11.5) {};
		\node [style=oplus] (36) at (0, 10) {};
		\node [style=oplus] (37) at (0, 11) {};
		\node [style=oplus] (38) at (-0.5, 10.5) {};
		\node [style=dot] (39) at (-0.5, 11) {};
		\node [style=dot] (40) at (0, 10.5) {};
		\node [style=dot] (41) at (-0.5, 10) {};
	\end{pgfonlayer}
	\begin{pgfonlayer}{edgelayer}
		\draw (33) to (41);
		\draw (41) to (38);
		\draw (38) to (39);
		\draw (39) to (34);
		\draw (35) to (37);
		\draw (37) to (40);
		\draw (40) to (36);
		\draw (36) to (32);
		\draw (36) to (41);
		\draw (40) to (38);
		\draw (37) to (39);
	\end{pgfonlayer}
\end{tikzpicture}
\right\rrbracket_{\hat{\TOF}}
&=
\begin{tikzpicture}
	\begin{pgfonlayer}{nodelayer}
		\node [style=X] (33) at (-3.25, 10) {};
		\node [style=Z] (34) at (-2.75, 10) {};
		\node [style=Z] (35) at (-3.25, 10.5) {};
		\node [style=Z] (36) at (-2.75, 11) {};
		\node [style=X] (37) at (-3.25, 11) {};
		\node [style=none] (38) at (-3.25, 11.5) {};
		\node [style=none] (39) at (-2.75, 11.5) {};
		\node [style=none] (40) at (-3.25, 9.5) {};
		\node [style=none] (41) at (-2.75, 9.5) {};
		\node [style=X] (42) at (-2.75, 10.5) {};
	\end{pgfonlayer}
	\begin{pgfonlayer}{edgelayer}
		\draw (39.center) to (36);
		\draw (36) to (42);
		\draw (42) to (34);
		\draw (34) to (41.center);
		\draw (40.center) to (33);
		\draw (33) to (34);
		\draw (42) to (35);
		\draw (35) to (33);
		\draw (35) to (37);
		\draw (37) to (36);
		\draw (37) to (38.center);
	\end{pgfonlayer}
\end{tikzpicture}
=
\begin{tikzpicture}
	\begin{pgfonlayer}{nodelayer}
		\node [style=X] (34) at (-3.25, 10) {};
		\node [style=Z] (35) at (-2.75, 10) {};
		\node [style=Z] (36) at (-2.75, 11) {};
		\node [style=Z] (37) at (-2.75, 12) {};
		\node [style=X] (38) at (-3.25, 12) {};
		\node [style=none] (39) at (-3.25, 12.5) {};
		\node [style=none] (40) at (-2.75, 12.5) {};
		\node [style=none] (41) at (-3.25, 9.5) {};
		\node [style=none] (42) at (-2.75, 9.5) {};
		\node [style=X] (43) at (-3.25, 11) {};
	\end{pgfonlayer}
	\begin{pgfonlayer}{edgelayer}
		\draw (40.center) to (37);
		\draw [in=90, out=-90] (37) to (43);
		\draw [in=90, out=-90] (43) to (35);
		\draw (35) to (42.center);
		\draw (41.center) to (34);
		\draw (34) to (35);
		\draw (43) to (36);
		\draw [in=90, out=-90] (36) to (34);
		\draw [in=-90, out=90] (36) to (38);
		\draw (38) to (37);
		\draw (38) to (39.center);
	\end{pgfonlayer}
\end{tikzpicture}
\eq{\ref{ZXA.5}}
\begin{tikzpicture}
	\begin{pgfonlayer}{nodelayer}
		\node [style=Z] (35) at (-2.75, 11.25) {};
		\node [style=X] (36) at (-3.25, 11.25) {};
		\node [style=none] (37) at (-3.25, 11.75) {};
		\node [style=none] (38) at (-2.75, 11.75) {};
		\node [style=none] (39) at (-3.25, 9.5) {};
		\node [style=none] (40) at (-2.75, 9.5) {};
		\node [style=Z] (41) at (-3.25, 10) {};
		\node [style=X] (42) at (-2.75, 10) {};
	\end{pgfonlayer}
	\begin{pgfonlayer}{edgelayer}
		\draw (38.center) to (35);
		\draw (36) to (35);
		\draw (36) to (37.center);
		\draw (42) to (41);
		\draw [in=-90, out=90] (41) to (35);
		\draw [in=-90, out=90] (42) to (36);
		\draw (41) to (39.center);
		\draw (40.center) to (42);
	\end{pgfonlayer}
\end{tikzpicture}
=
\begin{tikzpicture}
	\begin{pgfonlayer}{nodelayer}
		\node [style=Z] (36) at (-3.25, 10.75) {};
		\node [style=X] (37) at (-2.75, 10.75) {};
		\node [style=none] (38) at (-3.25, 11.75) {};
		\node [style=none] (39) at (-2.75, 11.75) {};
		\node [style=none] (40) at (-3.25, 9.5) {};
		\node [style=none] (41) at (-2.75, 9.5) {};
		\node [style=Z] (42) at (-3.25, 10) {};
		\node [style=X] (43) at (-2.75, 10) {};
	\end{pgfonlayer}
	\begin{pgfonlayer}{edgelayer}
		\draw [in=90, out=-90] (39.center) to (36);
		\draw (37) to (36);
		\draw [in=-90, out=90] (37) to (38.center);
		\draw (43) to (42);
		\draw (42) to (36);
		\draw (43) to (37);
		\draw (42) to (40.center);
		\draw (41.center) to (43);
	\end{pgfonlayer}
\end{tikzpicture}
\eq{\ref{ZXA.1},\ref{ZXA.3},\ref{ZXA.15}}
\begin{tikzpicture}
	\begin{pgfonlayer}{nodelayer}
		\node [style=nothing] (37) at (0, 9.5) {};
		\node [style=nothing] (38) at (-0.5, 9.5) {};
		\node [style=nothing] (39) at (-0.5, 10.5) {};
		\node [style=nothing] (40) at (0, 10.5) {};
	\end{pgfonlayer}
	\begin{pgfonlayer}{edgelayer}
		\draw [in=-90, out=90, looseness=1.25] (38) to (40);
		\draw [in=-90, out=90, looseness=1.25] (37) to (39);
	\end{pgfonlayer}
\end{tikzpicture}
=
\left\llbracket
\begin{tikzpicture}
	\begin{pgfonlayer}{nodelayer}
		\node [style=nothing] (38) at (0, 9.5) {};
		\node [style=nothing] (39) at (-0.5, 9.5) {};
		\node [style=nothing] (40) at (-0.5, 10.5) {};
		\node [style=nothing] (41) at (0, 10.5) {};
	\end{pgfonlayer}
	\begin{pgfonlayer}{edgelayer}
		\draw [in=-90, out=90, looseness=1.25] (39) to (41);
		\draw [in=-90, out=90, looseness=1.25] (38) to (40);
	\end{pgfonlayer}
\end{tikzpicture}
\right\rrbracket_{\hat{\TOF}}
\end{align*}

\item[\ref{TOF.15}:]
\begin{align*}
\left\llbracket
\begin{tikzpicture}
	\begin{pgfonlayer}{nodelayer}
		\node [style=nothing] (39) at (-1.75, 9.5) {};
		\node [style=nothing] (40) at (-1.25, 9.5) {};
		\node [style=nothing] (41) at (-0.75, 9.5) {};
		\node [style=nothing] (42) at (-1.75, 11.5) {};
		\node [style=nothing] (43) at (-1.25, 11.5) {};
		\node [style=nothing] (44) at (-0.75, 11.5) {};
		\node [style=dot] (45) at (-1.75, 10.5) {};
		\node [style=dot] (46) at (-1.25, 10.5) {};
		\node [style=oplus] (47) at (-0.75, 10.5) {};
	\end{pgfonlayer}
	\begin{pgfonlayer}{edgelayer}
		\draw (39) to (45);
		\draw (45) to (42);
		\draw (43) to (46);
		\draw (46) to (40);
		\draw (41) to (47);
		\draw (47) to (44);
		\draw (47) to (46);
		\draw (46) to (45);
	\end{pgfonlayer}
\end{tikzpicture}
\right\rrbracket_{\hat{\TOF}}
&=
\begin{tikzpicture}
	\begin{pgfonlayer}{nodelayer}
		\node [style=none] (40) at (-2, 9.5) {};
		\node [style=none] (41) at (-1, 9.5) {};
		\node [style=none] (42) at (-0.5, 9.5) {};
		\node [style=X] (43) at (-2, 10) {};
		\node [style=X] (44) at (-1, 10) {};
		\node [style=andin] (45) at (-1.5, 10.75) {};
		\node [style=Z] (46) at (-0.5, 11.5) {};
		\node [style=none] (47) at (-0.5, 12) {};
		\node [style=none] (48) at (-2, 12) {};
		\node [style=none] (49) at (-1, 12) {};
		\node [style=none] (50) at (-1.5, 10.75) {};
	\end{pgfonlayer}
	\begin{pgfonlayer}{edgelayer}
		\draw [in=90, out=180] (46) to (50.center);
		\draw (50.center) to (43);
		\draw (43) to (40.center);
		\draw (43) to (48.center);
		\draw (49.center) to (44);
		\draw (44) to (41.center);
		\draw (42.center) to (46);
		\draw (46) to (47.center);
		\draw (50.center) to (44);
	\end{pgfonlayer}
\end{tikzpicture}
\eq{\ref{ZXA.11}}
\begin{tikzpicture}
	\begin{pgfonlayer}{nodelayer}
		\node [style=none] (41) at (-2, 9.5) {};
		\node [style=none] (42) at (-1, 9.5) {};
		\node [style=none] (43) at (-0.5, 9.5) {};
		\node [style=X] (44) at (-2, 10) {};
		\node [style=X] (45) at (-1, 10) {};
		\node [style=andin] (46) at (-1.5, 11.25) {};
		\node [style=Z] (47) at (-0.5, 12) {};
		\node [style=none] (48) at (-0.5, 12.5) {};
		\node [style=none] (49) at (-2, 12.5) {};
		\node [style=none] (50) at (-1, 12.5) {};
		\node [style=none] (51) at (-1.5, 11.25) {};
		\node [style=none] (52) at (-1.75, 10.5) {};
		\node [style=none] (53) at (-1.25, 10.5) {};
	\end{pgfonlayer}
	\begin{pgfonlayer}{edgelayer}
		\draw [in=90, out=180] (47) to (51.center);
		\draw (44) to (41.center);
		\draw (44) to (49.center);
		\draw (50.center) to (45);
		\draw (45) to (42.center);
		\draw (43.center) to (47);
		\draw (47) to (48.center);
		\draw [in=90, out=-108] (51.center) to (52.center);
		\draw [in=146, out=-90] (52.center) to (45);
		\draw [in=34, out=-90, looseness=0.75] (53.center) to (44);
		\draw [in=-72, out=90] (53.center) to (51.center);
	\end{pgfonlayer}
\end{tikzpicture}
=
\begin{tikzpicture}
	\begin{pgfonlayer}{nodelayer}
		\node [style=none] (42) at (-2, 9.5) {};
		\node [style=none] (43) at (-1, 9.5) {};
		\node [style=none] (44) at (-0.5, 9.5) {};
		\node [style=X] (45) at (-1, 10.75) {};
		\node [style=X] (46) at (-2, 10.75) {};
		\node [style=andin] (47) at (-1.5, 11.5) {};
		\node [style=Z] (48) at (-0.5, 12.5) {};
		\node [style=none] (49) at (-0.5, 13) {};
		\node [style=none] (50) at (-2, 13) {};
		\node [style=none] (51) at (-1, 13) {};
		\node [style=none] (52) at (-1.5, 11.5) {};
		\node [style=none] (53) at (-2, 12) {};
		\node [style=none] (54) at (-1, 12) {};
	\end{pgfonlayer}
	\begin{pgfonlayer}{edgelayer}
		\draw [in=90, out=180] (48) to (52.center);
		\draw [in=90, out=-90] (45) to (42.center);
		\draw [in=90, out=-90] (46) to (43.center);
		\draw (44.center) to (48);
		\draw (48) to (49.center);
		\draw [in=90, out=-90] (51.center) to (53.center);
		\draw [in=-90, out=90] (54.center) to (50.center);
		\draw (54.center) to (45);
		\draw (46) to (52.center);
		\draw (52.center) to (45);
		\draw (46) to (53.center);
	\end{pgfonlayer}
\end{tikzpicture}
=
\left\llbracket
\begin{tikzpicture}
	\begin{pgfonlayer}{nodelayer}
		\node [style=nothing] (43) at (-1.75, 9.5) {};
		\node [style=nothing] (44) at (-1.25, 9.5) {};
		\node [style=nothing] (45) at (-0.75, 9.5) {};
		\node [style=dot] (46) at (-1.75, 10.5) {};
		\node [style=dot] (47) at (-1.25, 10.5) {};
		\node [style=oplus] (48) at (-0.75, 10.5) {};
		\node [style=nothing] (49) at (-1.75, 11.5) {};
		\node [style=nothing] (50) at (-1.25, 11.5) {};
		\node [style=nothing] (51) at (-0.75, 11.5) {};
	\end{pgfonlayer}
	\begin{pgfonlayer}{edgelayer}
		\draw [in=-90, out=90, looseness=1.25] (43) to (47);
		\draw [in=-90, out=90, looseness=1.25] (47) to (49);
		\draw [in=-90, out=90, looseness=1.25] (46) to (50);
		\draw [in=90, out=-90, looseness=1.25] (46) to (44);
		\draw (45) to (48);
		\draw (48) to (51);
		\draw (46) to (47);
		\draw (47) to (48);
	\end{pgfonlayer}
\end{tikzpicture}
\right\rrbracket_{\hat{\TOF}}
\end{align*}


\item[\ref{TOF.16}:]

\begingroup
\allowdisplaybreaks
\begin{align*}
\left\llbracket
\begin{tikzpicture}
	\begin{pgfonlayer}{nodelayer}
		\node [style=nothing] (44) at (0, 9.5) {};
		\node [style=nothing] (45) at (-0.5, 9.5) {};
		\node [style=nothing] (46) at (-1.5, 9.5) {};
		\node [style=nothing] (47) at (-2, 9.5) {};
		\node [style=zeroin] (48) at (-1, 9.5) {};
		\node [style=oplus] (49) at (-1, 10) {};
		\node [style=oplus] (50) at (-1, 11) {};
		\node [style=dot] (51) at (-1, 10.5) {};
		\node [style=dot] (52) at (-0.5, 10.5) {};
		\node [style=dot] (53) at (-1.5, 10) {};
		\node [style=dot] (54) at (-2, 10) {};
		\node [style=dot] (55) at (-1.5, 11) {};
		\node [style=dot] (56) at (-2, 11) {};
		\node [style=oplus] (57) at (0, 10.5) {};
		\node [style=zeroout] (58) at (-1, 11.5) {};
		\node [style=nothing] (59) at (0, 11.5) {};
		\node [style=nothing] (60) at (-2, 11.5) {};
		\node [style=nothing] (61) at (-0.5, 11.5) {};
		\node [style=nothing] (62) at (-1.5, 11.5) {};
	\end{pgfonlayer}
	\begin{pgfonlayer}{edgelayer}
		\draw (47) to (54);
		\draw (54) to (56);
		\draw (56) to (60);
		\draw (55) to (53);
		\draw (59) to (57);
		\draw (57) to (44);
		\draw (57) to (52);
		\draw (52) to (51);
		\draw (53) to (49);
		\draw (53) to (54);
		\draw (56) to (55);
		\draw (50) to (55);
		\draw (48) to (49);
		\draw (49) to (51);
		\draw (51) to (50);
		\draw (58) to (50);
		\draw [style=simple] (61) to (52);
		\draw [style=simple] (52) to (45);
		\draw [style=simple] (46) to (53);
		\draw [style=simple] (55) to (62);
	\end{pgfonlayer}
\end{tikzpicture}
\right\rrbracket_{\hat{\TOF}}
&=
\begin{tikzpicture}
	\begin{pgfonlayer}{nodelayer}
		\node [style=X] (44) at (-2, 10) {};
		\node [style=X] (45) at (-1, 10) {};
		\node [style=none] (46) at (-1.5, 10.75) {};
		\node [style=Z] (47) at (-0.5, 11.5) {};
		\node [style=Z] (48) at (-0.5, 10.75) {};
		\node [style=Z] (49) at (-0.5, 14) {};
		\node [style=X] (50) at (-1, 12) {};
		\node [style=Z] (51) at (-0.5, 13.5) {};
		\node [style=X] (52) at (-2, 12) {};
		\node [style=none] (53) at (-1.5, 12.75) {};
		\node [style=X] (54) at (0.5, 12) {};
		\node [style=Z] (55) at (1, 13.5) {};
		\node [style=X] (56) at (-0.5, 12) {};
		\node [style=none] (57) at (0, 12.75) {};
		\node [style=none] (58) at (-2, 9.5) {};
		\node [style=none] (59) at (-1, 9.5) {};
		\node [style=none] (60) at (0.5, 9.5) {};
		\node [style=none] (61) at (1, 9.5) {};
		\node [style=none] (62) at (-1, 14.5) {};
		\node [style=none] (63) at (-2, 14.5) {};
		\node [style=none] (64) at (1, 14.5) {};
		\node [style=none] (65) at (0.5, 14.5) {};
		\node [style=andin] (66) at (-1.5, 10.75) {};
		\node [style=andin] (67) at (-1.5, 12.75) {};
		\node [style=andin] (68) at (0, 12.75) {};
	\end{pgfonlayer}
	\begin{pgfonlayer}{edgelayer}
		\draw (45) to (46.center);
		\draw (46.center) to (44);
		\draw (47) to (48);
		\draw [in=90, out=180] (47) to (46.center);
		\draw (50) to (53.center);
		\draw (53.center) to (52);
		\draw (51) to (49);
		\draw [in=90, out=180] (51) to (53.center);
		\draw (54) to (57.center);
		\draw (57.center) to (56);
		\draw [in=90, out=180] (55) to (57.center);
		\draw (64.center) to (61.center);
		\draw (60.center) to (65.center);
		\draw (51) to (56);
		\draw (56) to (47);
		\draw (62.center) to (59.center);
		\draw (58.center) to (63.center);
	\end{pgfonlayer}
\end{tikzpicture}
\eq{\ref{ZXA.1}}
\begin{tikzpicture}
	\begin{pgfonlayer}{nodelayer}
		\node [style=X] (45) at (-2, 11.5) {};
		\node [style=X] (46) at (-1, 11.5) {};
		\node [style=none] (47) at (-1.5, 10.75) {};
		\node [style=X] (48) at (-1, 11.5) {};
		\node [style=X] (49) at (-2, 11.5) {};
		\node [style=none] (50) at (-1.5, 12.25) {};
		\node [style=X] (51) at (0.5, 11.5) {};
		\node [style=Z] (52) at (1, 13) {};
		\node [style=X] (53) at (-0.5, 11.5) {};
		\node [style=none] (54) at (0, 12.25) {};
		\node [style=none] (55) at (-2, 9.5) {};
		\node [style=none] (56) at (-1, 9.5) {};
		\node [style=none] (57) at (0.5, 9.5) {};
		\node [style=none] (58) at (1, 9.5) {};
		\node [style=none] (59) at (-1, 13.5) {};
		\node [style=none] (60) at (-2, 13.5) {};
		\node [style=none] (61) at (1, 13.5) {};
		\node [style=none] (62) at (0.5, 13.5) {};
		\node [style=andout] (63) at (-1.5, 10.75) {};
		\node [style=andin] (64) at (-1.5, 12.25) {};
		\node [style=andin] (65) at (0, 12.25) {};
		\node [style=none] (66) at (-1.5, 12.75) {};
		\node [style=none] (67) at (-0.5, 12.75) {};
		\node [style=none] (68) at (-1.5, 10.25) {};
		\node [style=none] (69) at (-0.5, 10.25) {};
	\end{pgfonlayer}
	\begin{pgfonlayer}{edgelayer}
		\draw (46) to (47.center);
		\draw (47.center) to (45);
		\draw (48) to (50.center);
		\draw (50.center) to (49);
		\draw (51) to (54.center);
		\draw (54.center) to (53);
		\draw [in=90, out=180] (52) to (54.center);
		\draw (61.center) to (58.center);
		\draw (57.center) to (62.center);
		\draw (59.center) to (56.center);
		\draw (55.center) to (60.center);
		\draw [in=90, out=90, looseness=1.25] (67.center) to (66.center);
		\draw (66.center) to (50.center);
		\draw (47.center) to (68.center);
		\draw [in=-90, out=-90, looseness=1.25] (68.center) to (69.center);
		\draw (69.center) to (53);
		\draw (53) to (67.center);
	\end{pgfonlayer}
\end{tikzpicture}\\
&\eq{\ref{ZXA.3}}
\begin{tikzpicture}
	\begin{pgfonlayer}{nodelayer}
		\node [style=X] (46) at (-2.5, 11.5) {};
		\node [style=X] (47) at (-1.5, 11.5) {};
		\node [style=none] (48) at (-2, 10.75) {};
		\node [style=X] (49) at (-1.5, 11.5) {};
		\node [style=X] (50) at (-2.5, 11.5) {};
		\node [style=none] (51) at (-2, 12.25) {};
		\node [style=X] (52) at (0.5, 11.5) {};
		\node [style=Z] (53) at (1, 13) {};
		\node [style=X] (54) at (-0.5, 11.5) {};
		\node [style=none] (55) at (0, 12.25) {};
		\node [style=none] (56) at (-3, 9.5) {};
		\node [style=none] (57) at (-1, 9.5) {};
		\node [style=none] (58) at (0.5, 9.5) {};
		\node [style=none] (59) at (1, 9.5) {};
		\node [style=none] (60) at (-1, 13.5) {};
		\node [style=none] (61) at (-3, 13.5) {};
		\node [style=none] (62) at (1, 13.5) {};
		\node [style=none] (63) at (0.5, 13.5) {};
		\node [style=andout] (64) at (-2, 10.75) {};
		\node [style=andin] (65) at (-2, 12.25) {};
		\node [style=andin] (66) at (0, 12.25) {};
		\node [style=none] (67) at (-2, 12.75) {};
		\node [style=none] (68) at (-0.5, 12.75) {};
		\node [style=none] (69) at (-2, 10.25) {};
		\node [style=none] (70) at (-0.5, 10.25) {};
		\node [style=X] (71) at (-1, 11.5) {};
		\node [style=X] (72) at (-3, 11.5) {};
	\end{pgfonlayer}
	\begin{pgfonlayer}{edgelayer}
		\draw (47) to (48.center);
		\draw (48.center) to (46);
		\draw (49) to (51.center);
		\draw (51.center) to (50);
		\draw (52) to (55.center);
		\draw (55.center) to (54);
		\draw [in=90, out=180] (53) to (55.center);
		\draw (62.center) to (59.center);
		\draw (58.center) to (63.center);
		\draw (60.center) to (57.center);
		\draw (56.center) to (61.center);
		\draw [in=90, out=90, looseness=1.25] (68.center) to (67.center);
		\draw (67.center) to (51.center);
		\draw (48.center) to (69.center);
		\draw [in=-90, out=-90, looseness=1.25] (69.center) to (70.center);
		\draw (70.center) to (54);
		\draw (54) to (68.center);
		\draw (71) to (47);
		\draw (46) to (72);
	\end{pgfonlayer}
\end{tikzpicture}
=
\begin{tikzpicture}
	\begin{pgfonlayer}{nodelayer}
		\node [style=X] (47) at (-0.75, 11.5) {};
		\node [style=none] (48) at (0.25, 9.5) {};
		\node [style=andin] (49) at (-0.25, 12.25) {};
		\node [style=none] (50) at (-0.75, 12.75) {};
		\node [style=none] (51) at (0.75, 9.5) {};
		\node [style=none] (52) at (-1.75, 10.75) {};
		\node [style=X] (53) at (0.25, 11.5) {};
		\node [style=none] (54) at (0.75, 13.5) {};
		\node [style=none] (55) at (-1.25, 13.5) {};
		\node [style=none] (56) at (-3.5, 9.75) {};
		\node [style=none] (57) at (-3, 12.75) {};
		\node [style=X] (58) at (-1.25, 12.5) {};
		\node [style=Z] (59) at (0.75, 13) {};
		\node [style=none] (60) at (-0.25, 12.25) {};
		\node [style=none] (61) at (-0.75, 10.25) {};
		\node [style=none] (62) at (-1.25, 9.5) {};
		\node [style=none] (63) at (-1.75, 10.25) {};
		\node [style=none] (64) at (-3.5, 13.5) {};
		\node [style=X] (65) at (-3.5, 12.5) {};
		\node [style=none] (66) at (0.25, 13.5) {};
		\node [style=andout] (67) at (-2.5, 10.75) {};
		\node [style=none] (68) at (-2.5, 10.25) {};
		\node [style=none] (69) at (-2.5, 10.75) {};
		\node [style=X] (70) at (-2.5, 11.75) {};
		\node [style=X] (71) at (-1.75, 11.75) {};
		\node [style=andout] (72) at (-1.75, 10.75) {};
		\node [style=none] (73) at (-3, 10.25) {};
	\end{pgfonlayer}
	\begin{pgfonlayer}{edgelayer}
		\draw (53) to (60.center);
		\draw (60.center) to (47);
		\draw [in=90, out=180] (59) to (60.center);
		\draw (54.center) to (51.center);
		\draw (48.center) to (66.center);
		\draw (55.center) to (62.center);
		\draw (56.center) to (64.center);
		\draw [in=90, out=90, looseness=0.50] (50.center) to (57.center);
		\draw (52.center) to (63.center);
		\draw [in=-90, out=-90, looseness=1.25] (63.center) to (61.center);
		\draw (61.center) to (47);
		\draw (47) to (50.center);
		\draw (69.center) to (68.center);
		\draw (71) to (69.center);
		\draw [in=-120, out=120, looseness=1.25] (69.center) to (70);
		\draw [in=60, out=-60, looseness=1.25] (71) to (52.center);
		\draw (52.center) to (70);
		\draw [in=0, out=90] (70) to (65);
		\draw (57.center) to (73.center);
		\draw [bend right=90, looseness=1.50] (73.center) to (68.center);
		\draw [in=90, out=180] (58) to (71);
	\end{pgfonlayer}
\end{tikzpicture}\\
&\eq{\ref{ZXA.12}}
\begin{tikzpicture}
	\begin{pgfonlayer}{nodelayer}
		\node [style=none] (48) at (-2.75, 9.75) {};
		\node [style=X] (49) at (-1.25, 12.5) {};
		\node [style=Z] (50) at (0.75, 13) {};
		\node [style=none] (51) at (0.75, 9.5) {};
		\node [style=none] (52) at (0.25, 13.5) {};
		\node [style=none] (53) at (-2.25, 12.75) {};
		\node [style=andout] (54) at (-1.75, 11.75) {};
		\node [style=X] (55) at (-1.75, 10.75) {};
		\node [style=none] (56) at (-0.25, 12.25) {};
		\node [style=none] (57) at (-1.25, 13.5) {};
		\node [style=none] (58) at (0.75, 13.5) {};
		\node [style=andin] (59) at (-0.25, 12.25) {};
		\node [style=none] (60) at (-2.25, 10.75) {};
		\node [style=none] (61) at (-1.75, 11.75) {};
		\node [style=none] (62) at (-0.75, 10.75) {};
		\node [style=none] (63) at (-2.75, 13.5) {};
		\node [style=none] (64) at (0.25, 9.5) {};
		\node [style=X] (65) at (-0.75, 11.5) {};
		\node [style=none] (66) at (-1.25, 9.5) {};
		\node [style=X] (67) at (0.25, 11.5) {};
		\node [style=none] (68) at (-0.75, 12.75) {};
		\node [style=X] (69) at (-2.75, 12.5) {};
	\end{pgfonlayer}
	\begin{pgfonlayer}{edgelayer}
		\draw (67) to (56.center);
		\draw (56.center) to (65);
		\draw [in=90, out=180] (50) to (56.center);
		\draw (58.center) to (51.center);
		\draw (64.center) to (52.center);
		\draw (57.center) to (66.center);
		\draw (48.center) to (63.center);
		\draw [in=90, out=90, looseness=0.50] (68.center) to (53.center);
		\draw (62.center) to (65);
		\draw (65) to (68.center);
		\draw (53.center) to (60.center);
		\draw [in=-60, out=-90] (62.center) to (55);
		\draw [in=-90, out=-105, looseness=1.75] (55) to (60.center);
		\draw (61.center) to (55);
		\draw (61.center) to (69);
		\draw (61.center) to (49);
	\end{pgfonlayer}
\end{tikzpicture}
\eq{\ref{ZXA.3}}
\begin{tikzpicture}
	\begin{pgfonlayer}{nodelayer}
		\node [style=none] (49) at (0.25, 9.5) {};
		\node [style=andin] (50) at (-0.25, 11.25) {};
		\node [style=none] (51) at (0.75, 9.5) {};
		\node [style=X] (52) at (0.25, 10.25) {};
		\node [style=none] (53) at (0.75, 12.75) {};
		\node [style=none] (54) at (-0.75, 12.75) {};
		\node [style=none] (55) at (-1.75, 9.5) {};
		\node [style=X] (56) at (-0.75, 12) {};
		\node [style=Z] (57) at (0.75, 12) {};
		\node [style=none] (58) at (-0.25, 11.25) {};
		\node [style=none] (59) at (-0.75, 9.5) {};
		\node [style=none] (60) at (-1.75, 12.75) {};
		\node [style=X] (61) at (-1.75, 12) {};
		\node [style=none] (62) at (0.25, 12.75) {};
		\node [style=none] (63) at (-1.25, 11.25) {};
		\node [style=andout] (64) at (-1.25, 11.25) {};
	\end{pgfonlayer}
	\begin{pgfonlayer}{edgelayer}
		\draw (52) to (58.center);
		\draw [in=90, out=180] (57) to (58.center);
		\draw (53.center) to (51.center);
		\draw (49.center) to (62.center);
		\draw (54.center) to (59.center);
		\draw (55.center) to (60.center);
		\draw (63.center) to (61);
		\draw (63.center) to (56);
		\draw [in=-90, out=-120, looseness=2.00] (58.center) to (63.center);
	\end{pgfonlayer}
\end{tikzpicture}\\
&=
\begin{tikzpicture}
	\begin{pgfonlayer}{nodelayer}
		\node [style=none] (50) at (0.25, 9.5) {};
		\node [style=none] (51) at (0.75, 9.5) {};
		\node [style=X] (52) at (0.25, 11.25) {};
		\node [style=none] (53) at (0.75, 13.5) {};
		\node [style=none] (54) at (-0.75, 13.5) {};
		\node [style=none] (55) at (-1.75, 9.5) {};
		\node [style=X] (56) at (-0.75, 10.25) {};
		\node [style=Z] (57) at (0.75, 13) {};
		\node [style=none] (58) at (-0.75, 9.5) {};
		\node [style=none] (59) at (-1.75, 13.5) {};
		\node [style=X] (60) at (-1.75, 10.25) {};
		\node [style=none] (61) at (0.25, 13.5) {};
		\node [style=none] (62) at (-1.25, 11.25) {};
		\node [style=andin] (63) at (-1.25, 11.25) {};
		\node [style=none] (64) at (-0.25, 12.25) {};
		\node [style=andin] (65) at (-0.25, 12.25) {};
	\end{pgfonlayer}
	\begin{pgfonlayer}{edgelayer}
		\draw (53.center) to (51.center);
		\draw (50.center) to (61.center);
		\draw (54.center) to (58.center);
		\draw (55.center) to (59.center);
		\draw [in=90, out=180, looseness=1.50] (57) to (64.center);
		\draw (64.center) to (52);
		\draw [in=-124, out=90] (62.center) to (64.center);
		\draw (62.center) to (56);
		\draw (60) to (62.center);
	\end{pgfonlayer}
\end{tikzpicture}
\eq{\ref{ZXA.11}}
\begin{tikzpicture}
	\begin{pgfonlayer}{nodelayer}
		\node [style=none] (51) at (0.25, 9.5) {};
		\node [style=none] (52) at (0.75, 9.5) {};
		\node [style=X] (53) at (0.25, 11.75) {};
		\node [style=none] (54) at (0.75, 14.25) {};
		\node [style=none] (55) at (-0.75, 14.25) {};
		\node [style=none] (56) at (-1.75, 9.5) {};
		\node [style=X] (57) at (-0.75, 10.25) {};
		\node [style=Z] (58) at (0.75, 13.75) {};
		\node [style=none] (59) at (-0.75, 9.5) {};
		\node [style=none] (60) at (-1.75, 14.25) {};
		\node [style=X] (61) at (-1.75, 10.25) {};
		\node [style=none] (62) at (0.25, 14.25) {};
		\node [style=none] (63) at (-1.25, 11.75) {};
		\node [style=andin] (64) at (-1.25, 11.75) {};
		\node [style=none] (65) at (-0.25, 13) {};
		\node [style=andin] (66) at (-0.25, 13) {};
		\node [style=none] (67) at (-1.5, 11) {};
		\node [style=none] (68) at (-1, 11) {};
	\end{pgfonlayer}
	\begin{pgfonlayer}{edgelayer}
		\draw (54.center) to (52.center);
		\draw (51.center) to (62.center);
		\draw (55.center) to (59.center);
		\draw (56.center) to (60.center);
		\draw [in=90, out=180, looseness=1.50] (58) to (65.center);
		\draw [in=90, out=-120, looseness=1.25] (63.center) to (67.center);
		\draw [in=-60, out=90, looseness=1.25] (68.center) to (63.center);
		\draw [in=45, out=-90] (68.center) to (61);
		\draw [in=135, out=-90] (67.center) to (57);
		\draw (65.center) to (53);
		\draw [in=90, out=-129] (65.center) to (63.center);
	\end{pgfonlayer}
\end{tikzpicture}\\
&\eq{\ref{ZXA.9}}
\begin{tikzpicture}
	\begin{pgfonlayer}{nodelayer}
		\node [style=none] (52) at (-0.25, 9.5) {};
		\node [style=none] (53) at (0.25, 9.5) {};
		\node [style=X] (54) at (-0.25, 11.5) {};
		\node [style=none] (55) at (0.25, 14.25) {};
		\node [style=none] (56) at (-0.75, 14.25) {};
		\node [style=none] (57) at (-1.75, 9.5) {};
		\node [style=X] (58) at (-0.75, 10.25) {};
		\node [style=Z] (59) at (0.25, 13.75) {};
		\node [style=none] (60) at (-0.75, 9.5) {};
		\node [style=none] (61) at (-1.75, 14.25) {};
		\node [style=X] (62) at (-1.75, 10.25) {};
		\node [style=none] (63) at (-0.25, 14.25) {};
		\node [style=none] (64) at (-1.5, 13) {};
		\node [style=andin] (65) at (-1.5, 13) {};
		\node [style=none] (66) at (-1.5, 11) {};
		\node [style=none] (67) at (-1, 11) {};
		\node [style=andin] (68) at (-1, 12.25) {};
		\node [style=none] (69) at (-1, 12.25) {};
	\end{pgfonlayer}
	\begin{pgfonlayer}{edgelayer}
		\draw (55.center) to (53.center);
		\draw (52.center) to (63.center);
		\draw (56.center) to (60.center);
		\draw (57.center) to (61.center);
		\draw [in=90, out=180, looseness=1.25] (59) to (64.center);
		\draw [in=45, out=-90] (67.center) to (62);
		\draw [in=135, out=-90] (66.center) to (58);
		\draw (69.center) to (54);
		\draw (69.center) to (64.center);
		\draw (64.center) to (66.center);
		\draw [in=90, out=-120, looseness=1.25] (69.center) to (67.center);
	\end{pgfonlayer}
\end{tikzpicture}
\eq{\ref{ZXA.11}}
\begin{tikzpicture}
	\begin{pgfonlayer}{nodelayer}
		\node [style=none] (53) at (0, 9.5) {};
		\node [style=none] (54) at (0.5, 9.5) {};
		\node [style=X] (55) at (0, 10.75) {};
		\node [style=none] (56) at (0.5, 14.5) {};
		\node [style=none] (57) at (-0.5, 14.5) {};
		\node [style=none] (58) at (-2, 9.5) {};
		\node [style=X] (59) at (-0.5, 10.25) {};
		\node [style=Z] (60) at (0.5, 14) {};
		\node [style=none] (61) at (-0.5, 9.5) {};
		\node [style=none] (62) at (-2, 14.5) {};
		\node [style=X] (63) at (-2, 10.25) {};
		\node [style=none] (64) at (0, 14.5) {};
		\node [style=none] (65) at (-1.5, 13.25) {};
		\node [style=andin] (66) at (-1.5, 13.25) {};
		\node [style=none] (67) at (-1.75, 11.75) {};
		\node [style=andin] (68) at (-1, 12.5) {};
		\node [style=none] (69) at (-1, 12.5) {};
		\node [style=none] (70) at (-1.25, 11.75) {};
		\node [style=none] (71) at (-0.75, 11.75) {};
	\end{pgfonlayer}
	\begin{pgfonlayer}{edgelayer}
		\draw (56.center) to (54.center);
		\draw (53.center) to (64.center);
		\draw (57.center) to (61.center);
		\draw (58.center) to (62.center);
		\draw [in=90, out=180, looseness=1.25] (60) to (65.center);
		\draw [in=135, out=-90] (67.center) to (59);
		\draw (69.center) to (65.center);
		\draw [in=90, out=-99] (65.center) to (67.center);
		\draw [in=90, out=-72] (69.center) to (71.center);
		\draw [in=-108, out=90] (70.center) to (69.center);
		\draw [in=149, out=-90] (70.center) to (55);
		\draw [in=56, out=-90] (71.center) to (63);
	\end{pgfonlayer}
\end{tikzpicture}\\
&\eq{\ref{ZXA.9}}
\begin{tikzpicture}
	\begin{pgfonlayer}{nodelayer}
		\node [style=none] (54) at (0, 9.5) {};
		\node [style=none] (55) at (0.5, 9.5) {};
		\node [style=X] (56) at (0, 10.75) {};
		\node [style=none] (57) at (0.5, 14.5) {};
		\node [style=none] (58) at (-0.5, 14.5) {};
		\node [style=none] (59) at (-2, 9.5) {};
		\node [style=X] (60) at (-0.5, 10.25) {};
		\node [style=Z] (61) at (0.5, 14) {};
		\node [style=none] (62) at (-0.5, 9.5) {};
		\node [style=none] (63) at (-2, 14.5) {};
		\node [style=X] (64) at (-2, 10.25) {};
		\node [style=none] (65) at (0, 14.5) {};
		\node [style=none] (66) at (-1, 13.25) {};
		\node [style=andin] (67) at (-1, 13.25) {};
		\node [style=none] (68) at (-1.75, 11.75) {};
		\node [style=none] (69) at (-1.25, 11.75) {};
		\node [style=none] (70) at (-0.75, 11.75) {};
		\node [style=none] (71) at (-1.5, 12.5) {};
		\node [style=andin] (72) at (-1.5, 12.5) {};
	\end{pgfonlayer}
	\begin{pgfonlayer}{edgelayer}
		\draw (57.center) to (55.center);
		\draw (54.center) to (65.center);
		\draw (58.center) to (62.center);
		\draw (59.center) to (63.center);
		\draw [in=90, out=180, looseness=1.25] (61) to (66.center);
		\draw [in=135, out=-90] (68.center) to (60);
		\draw [in=149, out=-90] (69.center) to (56);
		\draw [in=56, out=-90] (70.center) to (64);
		\draw [in=90, out=-81] (66.center) to (70.center);
		\draw [in=90, out=-124] (66.center) to (71.center);
		\draw [in=90, out=-72] (71.center) to (69.center);
		\draw [in=90, out=-108] (71.center) to (68.center);
	\end{pgfonlayer}
\end{tikzpicture}
\eq{\ref{ZXA.11}}
\begin{tikzpicture}
	\begin{pgfonlayer}{nodelayer}
		\node [style=none] (55) at (0, 10.5) {};
		\node [style=none] (56) at (0.5, 10.5) {};
		\node [style=X] (57) at (0, 11.75) {};
		\node [style=none] (58) at (0.5, 16.25) {};
		\node [style=none] (59) at (-0.5, 16.25) {};
		\node [style=none] (60) at (-2, 10.5) {};
		\node [style=X] (61) at (-0.5, 11.25) {};
		\node [style=Z] (62) at (0.5, 15.75) {};
		\node [style=none] (63) at (-0.5, 10.5) {};
		\node [style=none] (64) at (-2, 16.25) {};
		\node [style=X] (65) at (-2, 11.25) {};
		\node [style=none] (66) at (0, 16.25) {};
		\node [style=none] (67) at (-1, 15) {};
		\node [style=andin] (68) at (-1, 15) {};
		\node [style=none] (69) at (-1.75, 12.75) {};
		\node [style=none] (70) at (-1.25, 12.75) {};
		\node [style=none] (71) at (-0.75, 12.75) {};
		\node [style=none] (72) at (-1.5, 14.25) {};
		\node [style=andin] (73) at (-1.5, 14.25) {};
		\node [style=none] (74) at (-1.75, 13.5) {};
		\node [style=none] (75) at (-1.25, 13.5) {};
	\end{pgfonlayer}
	\begin{pgfonlayer}{edgelayer}
		\draw (58.center) to (56.center);
		\draw (55.center) to (66.center);
		\draw (59.center) to (63.center);
		\draw (60.center) to (64.center);
		\draw [in=90, out=180, looseness=1.25] (62) to (67.center);
		\draw [in=135, out=-90] (69.center) to (61);
		\draw [in=149, out=-90] (70.center) to (57);
		\draw [in=56, out=-90] (71.center) to (65);
		\draw [in=90, out=-81] (67.center) to (71.center);
		\draw [in=90, out=-124] (67.center) to (72.center);
		\draw [in=90, out=-72] (72.center) to (75.center);
		\draw [in=90, out=-108] (72.center) to (74.center);
		\draw [in=90, out=-90] (74.center) to (70.center);
		\draw [in=90, out=-90] (75.center) to (69.center);
	\end{pgfonlayer}
\end{tikzpicture}\\
&=
\begin{tikzpicture}
	\begin{pgfonlayer}{nodelayer}
		\node [style=none] (56) at (-0.75, 13.75) {};
		\node [style=Z] (57) at (-0.25, 15.25) {};
		\node [style=andin] (58) at (-1.75, 13.5) {};
		\node [style=X] (59) at (-0.75, 12) {};
		\node [style=none] (60) at (-1.25, 15.75) {};
		\node [style=none] (61) at (-2.75, 10.5) {};
		\node [style=X] (62) at (-2.75, 12) {};
		\node [style=none] (63) at (-1.75, 13.5) {};
		\node [style=none] (64) at (-0.25, 10.5) {};
		\node [style=andin] (65) at (-2.25, 12.75) {};
		\node [style=none] (66) at (-1.25, 10.5) {};
		\node [style=none] (67) at (-0.75, 10.5) {};
		\node [style=X] (68) at (-1.25, 12) {};
		\node [style=none] (69) at (-2.25, 12.75) {};
		\node [style=none] (70) at (-2.75, 14) {};
		\node [style=none] (71) at (-0.25, 15.75) {};
		\node [style=none] (72) at (-2.75, 15) {};
		\node [style=none] (73) at (-0.75, 15.25) {};
		\node [style=none] (74) at (-2.75, 15.75) {};
		\node [style=none] (75) at (-0.75, 15.75) {};
	\end{pgfonlayer}
	\begin{pgfonlayer}{edgelayer}
		\draw (71.center) to (64.center);
		\draw (60.center) to (66.center);
		\draw [in=90, out=180, looseness=1.25] (57) to (63.center);
		\draw [in=90, out=-124, looseness=1.25] (63.center) to (69.center);
		\draw [in=90, out=-90] (62) to (67.center);
		\draw [in=-90, out=90] (61.center) to (59);
		\draw (69.center) to (62);
		\draw (68) to (69.center);
		\draw [in=105, out=-60] (63.center) to (59);
		\draw [in=90, out=-90] (72.center) to (56.center);
		\draw [in=90, out=-90] (73.center) to (70.center);
		\draw (75.center) to (73.center);
		\draw (56.center) to (59);
		\draw (62) to (70.center);
		\draw (72.center) to (74.center);
	\end{pgfonlayer}
\end{tikzpicture}
=
\begin{tikzpicture}
	\begin{pgfonlayer}{nodelayer}
		\node [style=X] (57) at (-2, 11.75) {};
		\node [style=X] (58) at (-1, 11.75) {};
		\node [style=none] (59) at (-1.5, 12.5) {};
		\node [style=Z] (60) at (-0.5, 13.25) {};
		\node [style=Z] (61) at (-0.5, 12.5) {};
		\node [style=Z] (62) at (-0.5, 15.75) {};
		\node [style=X] (63) at (-1, 13.75) {};
		\node [style=Z] (64) at (-0.5, 15.25) {};
		\node [style=X] (65) at (-2, 13.75) {};
		\node [style=none] (66) at (-1.5, 14.5) {};
		\node [style=X] (67) at (0.5, 13.75) {};
		\node [style=Z] (68) at (1, 15.25) {};
		\node [style=X] (69) at (-0.5, 13.75) {};
		\node [style=none] (70) at (0, 14.5) {};
		\node [style=none] (71) at (-2, 10.5) {};
		\node [style=none] (72) at (-1, 11.75) {};
		\node [style=none] (73) at (0.5, 11.75) {};
		\node [style=none] (74) at (1, 10.5) {};
		\node [style=none] (75) at (-1, 15.75) {};
		\node [style=none] (76) at (-2, 17) {};
		\node [style=none] (77) at (1, 17) {};
		\node [style=none] (78) at (0.5, 15.75) {};
		\node [style=andin] (79) at (-1.5, 12.5) {};
		\node [style=andin] (80) at (-1.5, 14.5) {};
		\node [style=andin] (81) at (0, 14.5) {};
		\node [style=none] (82) at (-1, 17) {};
		\node [style=none] (83) at (0.5, 17) {};
		\node [style=none] (84) at (0.5, 10.5) {};
		\node [style=none] (85) at (-1, 10.5) {};
	\end{pgfonlayer}
	\begin{pgfonlayer}{edgelayer}
		\draw (58) to (59.center);
		\draw (59.center) to (57);
		\draw (60) to (61);
		\draw [in=90, out=180] (60) to (59.center);
		\draw (63) to (66.center);
		\draw (66.center) to (65);
		\draw (64) to (62);
		\draw [in=90, out=180] (64) to (66.center);
		\draw (67) to (70.center);
		\draw (70.center) to (69);
		\draw [in=90, out=180] (68) to (70.center);
		\draw (77.center) to (74.center);
		\draw (73.center) to (78.center);
		\draw (64) to (69);
		\draw (69) to (60);
		\draw (71.center) to (76.center);
		\draw [in=90, out=-90] (82.center) to (78.center);
		\draw [in=-90, out=90] (75.center) to (83.center);
		\draw [in=90, out=-90] (72.center) to (84.center);
		\draw [in=-90, out=90] (85.center) to (73.center);
		\draw (75.center) to (58);
	\end{pgfonlayer}
\end{tikzpicture}\\
&=
\left\llbracket
\begin{tikzpicture}
	\begin{pgfonlayer}{nodelayer}
		\node [style=nothing] (58) at (0, 10.5) {};
		\node [style=nothing] (59) at (-0.5, 10.5) {};
		\node [style=nothing] (60) at (-1.5, 10.5) {};
		\node [style=nothing] (61) at (-2, 10.5) {};
		\node [style=zeroin] (62) at (-1, 11.25) {};
		\node [style=oplus] (63) at (-1, 11.75) {};
		\node [style=oplus] (64) at (-1, 12.75) {};
		\node [style=dot] (65) at (-1, 12.25) {};
		\node [style=dot] (66) at (-0.5, 12.25) {};
		\node [style=dot] (67) at (-1.5, 11.75) {};
		\node [style=dot] (68) at (-2, 11.75) {};
		\node [style=dot] (69) at (-1.5, 12.75) {};
		\node [style=dot] (70) at (-2, 12.75) {};
		\node [style=oplus] (71) at (0, 12.25) {};
		\node [style=zeroout] (72) at (-1, 13.25) {};
		\node [style=nothing] (73) at (0, 14) {};
		\node [style=nothing] (74) at (-2, 14) {};
		\node [style=nothing] (75) at (-0.5, 14) {};
		\node [style=nothing] (76) at (-1.5, 14) {};
		\node [style=none] (77) at (-1.5, 13.5) {};
		\node [style=none] (78) at (-0.5, 13.5) {};
		\node [style=none] (79) at (-0.5, 11) {};
		\node [style=none] (80) at (-1.5, 11) {};
	\end{pgfonlayer}
	\begin{pgfonlayer}{edgelayer}
		\draw (61) to (68);
		\draw (68) to (70);
		\draw (70) to (74);
		\draw (69) to (67);
		\draw (73) to (71);
		\draw (71) to (58);
		\draw (71) to (66);
		\draw (66) to (65);
		\draw (67) to (63);
		\draw (67) to (68);
		\draw (70) to (69);
		\draw (64) to (69);
		\draw (62) to (63);
		\draw (63) to (65);
		\draw (65) to (64);
		\draw (64) to (72);
		\draw [in=90, out=-90, looseness=0.50] (75) to (77.center);
		\draw [in=90, out=-90, looseness=0.75] (76) to (78.center);
		\draw (78.center) to (66);
		\draw (66) to (79.center);
		\draw [in=90, out=-105, looseness=0.50] (79.center) to (60);
		\draw [in=-90, out=90, looseness=0.50] (59) to (80.center);
		\draw (80.center) to (67);
		\draw (69) to (77.center);
	\end{pgfonlayer}
\end{tikzpicture}
\right\rrbracket_{\hat{\TOF}}
\end{align*}
\endgroup



\end{enumerate}
Where unitality and counitality follow from the fact that the white spiders are Frobenius algebras.  Also, we must also note that both Frobenius algebras induce the same compact closed structure, as is implied by the spider law;  this is immediate.

\end{proof}

\begin{theorem}
\label{theorem:TOFZXAiso}
The interpretation functors $\llbracket\_\rrbracket_{\ZXA}$ and $\llbracket\_\rrbracket_{\hat \TOF}$ are inverses, so that $\hat \TOF$ and $\ZXA$ are isomorphic as strongly compact closed props.
\end{theorem}

\begin{proof}
First we show that $\llbracket\llbracket\_\rrbracket_{\ZXA}\rrbracket_{\hat \TOF}=1$:
\begin{description}
\item[For the white spider:]
The case for the unit and counit is trivial.  For the (co)multiplication we have:
\begin{align*}
\left\llbracket\left\llbracket
\begin{tikzpicture}
	\begin{pgfonlayer}{nodelayer}
		\node [style=X] (59) at (0, 11.5) {};
		\node [style=none] (60) at (-0.5, 12.5) {};
		\node [style=none] (61) at (0.5, 12.5) {};
		\node [style=none] (62) at (0, 10.5) {};
	\end{pgfonlayer}
	\begin{pgfonlayer}{edgelayer}
		\draw [in=63, out=-90] (61.center) to (59);
		\draw [in=-90, out=117] (59) to (60.center);
		\draw (59) to (62.center);
	\end{pgfonlayer}
\end{tikzpicture}
\right\rrbracket_{\ZXA}\right\rrbracket_{\hat \TOF}
&=
\left\llbracket
\begin{tikzpicture}
	\begin{pgfonlayer}{nodelayer}
		\node [style=none] (60) at (-0.5, 11.75) {};
		\node [style=none] (61) at (0, 11.75) {};
		\node [style=none] (62) at (-0.5, 10.5) {};
		\node [style=dot] (63) at (-0.5, 11.25) {};
		\node [style=oplus] (64) at (0, 11.25) {};
		\node [style=zeroin] (65) at (0, 10.75) {};
	\end{pgfonlayer}
	\begin{pgfonlayer}{edgelayer}
		\draw (61.center) to (64);
		\draw (64) to (65);
		\draw (64) to (63);
		\draw (63) to (60.center);
		\draw (63) to (62.center);
	\end{pgfonlayer}
\end{tikzpicture}
\right\rrbracket_{\hat \TOF}
=
\begin{tikzpicture}
	\begin{pgfonlayer}{nodelayer}
		\node [style=andin] (61) at (-1.5, 12.5) {};
		\node [style=X] (62) at (-2, 11.5) {};
		\node [style=X] (63) at (-1, 11.5) {};
		\node [style=Z] (64) at (-0.5, 13) {};
		\node [style=none] (65) at (-1.5, 12.5) {};
		\node [style=Z] (66) at (-2, 10.5) {$\pi$};
		\node [style=Z] (67) at (-2, 12.5) {$\pi$};
		\node [style=none] (68) at (-0.5, 13.75) {};
		\node [style=none] (69) at (-1, 13.75) {};
		\node [style=none] (70) at (-1, 10.5) {};
		\node [style=Z] (71) at (-0.5, 11.5) {};
	\end{pgfonlayer}
	\begin{pgfonlayer}{edgelayer}
		\draw [in=90, out=180] (64) to (65.center);
		\draw (65.center) to (62);
		\draw (63) to (65.center);
		\draw (67) to (62);
		\draw (62) to (66);
		\draw (69.center) to (63);
		\draw (63) to (70.center);
		\draw (68.center) to (64);
		\draw (64) to (71);
	\end{pgfonlayer}
\end{tikzpicture}
=
\begin{tikzpicture}
	\begin{pgfonlayer}{nodelayer}
		\node [style=X] (62) at (-1, 11.5) {};
		\node [style=Z] (63) at (-0.5, 11.5) {};
		\node [style=none] (64) at (-0.5, 12.25) {};
		\node [style=none] (65) at (-1, 12.25) {};
		\node [style=none] (66) at (-1, 10.5) {};
		\node [style=Z] (67) at (-0.5, 10.75) {};
	\end{pgfonlayer}
	\begin{pgfonlayer}{edgelayer}
		\draw (65.center) to (62);
		\draw (62) to (66.center);
		\draw (64.center) to (63);
		\draw (63) to (67);
		\draw (63) to (62);
	\end{pgfonlayer}
\end{tikzpicture}
=
\begin{tikzpicture}
	\begin{pgfonlayer}{nodelayer}
		\node [style=X] (63) at (0, 11.25) {};
		\node [style=none] (64) at (-0.5, 12.25) {};
		\node [style=none] (65) at (0.5, 12.25) {};
		\node [style=none] (66) at (0, 10.5) {};
	\end{pgfonlayer}
	\begin{pgfonlayer}{edgelayer}
		\draw [in=63, out=-90] (65.center) to (63);
		\draw [in=-90, out=117] (63) to (64.center);
		\draw (63) to (66.center);
	\end{pgfonlayer}
\end{tikzpicture}
\end{align*}

\item[For the grey spider:]
The cases for the unit, counit and $\pi$ phase are trivial.  For the (co) multiplication we have:

\begin{align*}
\left\llbracket\left\llbracket
\begin{tikzpicture}
	\begin{pgfonlayer}{nodelayer}
		\node [style=Z] (64) at (0, 11) {};
		\node [style=none] (65) at (-0.5, 11.75) {};
		\node [style=none] (66) at (0.5, 11.75) {};
		\node [style=none] (67) at (0, 10.5) {};
	\end{pgfonlayer}
	\begin{pgfonlayer}{edgelayer}
		\draw [in=63, out=-90] (66.center) to (64);
		\draw [in=-90, out=117] (64) to (65.center);
		\draw (64) to (67.center);
	\end{pgfonlayer}
\end{tikzpicture}
\right\rrbracket_{\ZXA}\right\rrbracket_{\hat \TOF}
&=
\left\llbracket
\begin{tikzpicture}
	\begin{pgfonlayer}{nodelayer}
		\node [style=none] (65) at (-0.5, 11.75) {};
		\node [style=none] (66) at (0, 11.75) {};
		\node [style=none] (67) at (-0.5, 10.5) {};
		\node [style=dot] (68) at (0, 11.25) {};
		\node [style=oplus] (69) at (-0.5, 11.25) {};
		\node [style=X] (70) at (0, 10.75) {};
	\end{pgfonlayer}
	\begin{pgfonlayer}{edgelayer}
		\draw (69) to (68);
		\draw (68) to (66.center);
		\draw (70) to (68);
		\draw (69) to (65.center);
		\draw (69) to (67.center);
	\end{pgfonlayer}
\end{tikzpicture}
\right\rrbracket_{\hat \TOF}
=
\begin{tikzpicture}
	\begin{pgfonlayer}{nodelayer}
		\node [style=X] (66) at (-2, 11.75) {};
		\node [style=X] (67) at (-1, 11.75) {};
		\node [style=none] (68) at (-1.5, 12.5) {};
		\node [style=Z] (69) at (-2.5, 13.25) {};
		\node [style=Z] (70) at (-1, 10.75) {$\pi$};
		\node [style=Z] (71) at (-1, 13.25) {$\pi$};
		\node [style=X] (72) at (-2, 10.75) {};
		\node [style=none] (73) at (-2.5, 10.5) {};
		\node [style=none] (74) at (-2.5, 13.75) {};
		\node [style=none] (75) at (-2, 13.75) {};
		\node [style=andin] (76) at (-1.5, 12.5) {};
	\end{pgfonlayer}
	\begin{pgfonlayer}{edgelayer}
		\draw (70) to (67);
		\draw (67) to (68.center);
		\draw (71) to (67);
		\draw (68.center) to (66);
		\draw (68.center) to (69);
		\draw (75.center) to (66);
		\draw (66) to (72);
		\draw (73.center) to (69);
		\draw (69) to (74.center);
	\end{pgfonlayer}
\end{tikzpicture}
=
\begin{tikzpicture}
	\begin{pgfonlayer}{nodelayer}
		\node [style=X] (67) at (-2, 11.25) {};
		\node [style=Z] (68) at (-2.5, 11.25) {};
		\node [style=X] (69) at (-2, 10.75) {};
		\node [style=none] (70) at (-2.5, 10.5) {};
		\node [style=none] (71) at (-2.5, 11.75) {};
		\node [style=none] (72) at (-2, 11.75) {};
	\end{pgfonlayer}
	\begin{pgfonlayer}{edgelayer}
		\draw (72.center) to (67);
		\draw (67) to (69);
		\draw (70.center) to (68);
		\draw (68) to (71.center);
		\draw (67) to (68);
	\end{pgfonlayer}
\end{tikzpicture}
=
\begin{tikzpicture}
	\begin{pgfonlayer}{nodelayer}
		\node [style=Z] (68) at (0, 11) {};
		\node [style=none] (69) at (-0.5, 11.75) {};
		\node [style=none] (70) at (0.5, 11.75) {};
		\node [style=none] (71) at (0, 10.5) {};
	\end{pgfonlayer}
	\begin{pgfonlayer}{edgelayer}
		\draw [in=63, out=-90] (70.center) to (68);
		\draw [in=-90, out=117] (68) to (69.center);
		\draw (68) to (71.center);
	\end{pgfonlayer}
\end{tikzpicture}
\end{align*}


\item[For the {\sf and} gate:]
\begin{align*}
\left\llbracket\left\llbracket
\begin{tikzpicture}
	\begin{pgfonlayer}{nodelayer}
		\node [style=none] (69) at (0, 11.5) {};
		\node [style=none] (70) at (-0.5, 10.5) {};
		\node [style=none] (71) at (0.5, 10.5) {};
		\node [style=none] (72) at (0, 12.5) {};
		\node [style=andin] (73) at (0, 11.5) {};
	\end{pgfonlayer}
	\begin{pgfonlayer}{edgelayer}
		\draw [in=-63, out=90] (71.center) to (69.center);
		\draw [in=90, out=-117] (69.center) to (70.center);
		\draw (69.center) to (72.center);
	\end{pgfonlayer}
\end{tikzpicture}
\right\rrbracket_{\ZXA}\right\rrbracket_{\hat \TOF}
&=
\left\llbracket
\begin{tikzpicture}
	\begin{pgfonlayer}{nodelayer}
		\node [style=dot] (70) at (-2, 11.25) {};
		\node [style=dot] (71) at (-1.5, 11.25) {};
		\node [style=oplus] (72) at (-1, 11.25) {};
		\node [style=zeroin] (73) at (-1, 10.75) {};
		\node [style=X] (74) at (-2, 11.75) {};
		\node [style=X] (75) at (-1.5, 11.75) {};
		\node [style=none] (76) at (-1, 12) {};
		\node [style=none] (77) at (-2, 10.5) {};
		\node [style=none] (78) at (-1.5, 10.5) {};
	\end{pgfonlayer}
	\begin{pgfonlayer}{edgelayer}
		\draw (75) to (71);
		\draw (70) to (71);
		\draw (71) to (72);
		\draw (72) to (73);
		\draw (72) to (76.center);
		\draw (74) to (70);
		\draw (70) to (77.center);
		\draw (78.center) to (71);
	\end{pgfonlayer}
\end{tikzpicture}
\right\rrbracket_{\hat \TOF}
=
\begin{tikzpicture}
	\begin{pgfonlayer}{nodelayer}
		\node [style=X] (71) at (-2, 11) {};
		\node [style=X] (72) at (-1, 11) {};
		\node [style=none] (73) at (-1.5, 11.75) {};
		\node [style=Z] (74) at (-0.5, 12.5) {};
		\node [style=Z] (75) at (-0.5, 11.5) {};
		\node [style=X] (76) at (-1, 13) {};
		\node [style=X] (77) at (-2, 13) {};
		\node [style=none] (78) at (-0.5, 13.25) {};
		\node [style=none] (79) at (-2, 10.5) {};
		\node [style=none] (80) at (-1, 10.5) {};
		\node [style=andin] (81) at (-1.5, 11.75) {};
	\end{pgfonlayer}
	\begin{pgfonlayer}{edgelayer}
		\draw (73.center) to (71);
		\draw (72) to (73.center);
		\draw (74) to (75);
		\draw [in=90, out=180] (74) to (73.center);
		\draw (76) to (72);
		\draw (72) to (80.center);
		\draw (79.center) to (71);
		\draw (71) to (77);
		\draw (78.center) to (74);
	\end{pgfonlayer}
\end{tikzpicture}
=
\begin{tikzpicture}
	\begin{pgfonlayer}{nodelayer}
		\node [style=none] (72) at (0, 11.5) {};
		\node [style=none] (73) at (-0.5, 10.5) {};
		\node [style=none] (74) at (0.5, 10.5) {};
		\node [style=none] (75) at (0, 12.5) {};
		\node [style=andin] (76) at (0, 11.5) {};
	\end{pgfonlayer}
	\begin{pgfonlayer}{edgelayer}
		\draw [in=-63, out=90] (74.center) to (72.center);
		\draw [in=90, out=-117] (72.center) to (73.center);
		\draw (72.center) to (75.center);
	\end{pgfonlayer}
\end{tikzpicture}
\end{align*}
\end{description}

Next, we show that $\llbracket\llbracket\_\rrbracket_{\hat \TOF}\rrbracket_{\ZXA}=1$:
The ancillae are trivial.  For the Toffoli gate:
\begin{align*}
\left\llbracket\left\llbracket
\begin{tikzpicture}
	\begin{pgfonlayer}{nodelayer}
		\node [style=dot] (73) at (-2, 11.25) {};
		\node [style=dot] (74) at (-1.5, 11.25) {};
		\node [style=oplus] (75) at (-1, 11.25) {};
		\node [style=none] (76) at (-1, 12) {};
		\node [style=none] (77) at (-1, 10.5) {};
		\node [style=none] (78) at (-2, 10.5) {};
		\node [style=none] (79) at (-1.5, 10.5) {};
		\node [style=none] (80) at (-1.5, 12) {};
		\node [style=none] (81) at (-2, 12) {};
	\end{pgfonlayer}
	\begin{pgfonlayer}{edgelayer}
		\draw (75) to (74);
		\draw (74) to (73);
		\draw (81.center) to (73);
		\draw (73) to (78.center);
		\draw (79.center) to (74);
		\draw (74) to (80.center);
		\draw (76.center) to (75);
		\draw (75) to (77.center);
	\end{pgfonlayer}
\end{tikzpicture}
\right\rrbracket_{\hat \TOF}\right\rrbracket_{\ZXA}
&=
\left\llbracket
\begin{tikzpicture}
	\begin{pgfonlayer}{nodelayer}
		\node [style=X] (74) at (-2, 11) {};
		\node [style=X] (75) at (-1, 11) {};
		\node [style=none] (76) at (-1.5, 12) {};
		\node [style=Z] (77) at (-0.5, 13) {};
		\node [style=none] (78) at (-0.5, 10.5) {};
		\node [style=none] (79) at (-0.5, 13.75) {};
		\node [style=none] (80) at (-1, 13.75) {};
		\node [style=none] (81) at (-2, 13.75) {};
		\node [style=none] (82) at (-2, 10.5) {};
		\node [style=none] (83) at (-1, 10.5) {};
		\node [style=andin] (84) at (-1.5, 12) {};
	\end{pgfonlayer}
	\begin{pgfonlayer}{edgelayer}
		\draw (77) to (79.center);
		\draw (77) to (78.center);
		\draw (83.center) to (75);
		\draw (75) to (80.center);
		\draw (81.center) to (74);
		\draw (74) to (82.center);
		\draw (74) to (76.center);
		\draw [in=180, out=90] (76.center) to (77);
		\draw (76.center) to (75);
	\end{pgfonlayer}
\end{tikzpicture}
\right\rrbracket_{\ZXA}
=
\begin{tikzpicture}
	\begin{pgfonlayer}{nodelayer}
		\node [style=dot] (75) at (-2, 12) {};
		\node [style=dot] (76) at (-1, 12) {};
		\node [style=oplus] (77) at (-0.25, 12) {};
		\node [style=X] (78) at (-2, 12.5) {};
		\node [style=X] (79) at (-1, 12.5) {};
		\node [style=oplus] (80) at (-0.25, 12.5) {};
		\node [style=dot] (81) at (0.25, 12.5) {};
		\node [style=X] (82) at (0.25, 12) {};
		\node [style=none] (83) at (0.25, 13) {};
		\node [style=dot] (84) at (-2.5, 11.5) {};
		\node [style=oplus] (85) at (-2, 11.5) {};
		\node [style=zeroin] (86) at (-2, 11) {};
		\node [style=oplus] (87) at (-1, 11.5) {};
		\node [style=dot] (88) at (-1.5, 11.5) {};
		\node [style=zeroin] (89) at (-1, 11) {};
		\node [style=none] (90) at (-1.5, 13) {};
		\node [style=none] (91) at (-2.5, 13) {};
		\node [style=none] (92) at (-1.5, 10.5) {};
		\node [style=none] (93) at (-2.5, 10.5) {};
		\node [style=none] (94) at (-0.25, 10.5) {};
		\node [style=zeroout] (95) at (-0.25, 13) {};
	\end{pgfonlayer}
	\begin{pgfonlayer}{edgelayer}
		\draw (83.center) to (81);
		\draw (81) to (80);
		\draw (80) to (77);
		\draw (82) to (81);
		\draw (77) to (76);
		\draw (76) to (75);
		\draw (75) to (78);
		\draw (79) to (76);
		\draw (85) to (86);
		\draw (85) to (84);
		\draw (87) to (89);
		\draw (87) to (88);
		\draw (76) to (87);
		\draw (85) to (75);
		\draw (91.center) to (84);
		\draw (84) to (93.center);
		\draw (92.center) to (88);
		\draw (88) to (90.center);
		\draw (95) to (80);
		\draw (77) to (94.center);
	\end{pgfonlayer}
\end{tikzpicture}
\eq{unit}
\begin{tikzpicture}
	\begin{pgfonlayer}{nodelayer}
		\node [style=dot] (76) at (-2, 12) {};
		\node [style=dot] (77) at (-1, 12) {};
		\node [style=oplus] (78) at (-0.25, 12) {};
		\node [style=X] (79) at (-2, 12.5) {};
		\node [style=X] (80) at (-1, 12.5) {};
		\node [style=none] (81) at (-0.25, 13) {};
		\node [style=dot] (82) at (-2.5, 11.5) {};
		\node [style=oplus] (83) at (-2, 11.5) {};
		\node [style=zeroin] (84) at (-2, 11) {};
		\node [style=oplus] (85) at (-1, 11.5) {};
		\node [style=dot] (86) at (-1.5, 11.5) {};
		\node [style=zeroin] (87) at (-1, 11) {};
		\node [style=none] (88) at (-1.5, 13) {};
		\node [style=none] (89) at (-2.5, 13) {};
		\node [style=none] (90) at (-1.5, 10.5) {};
		\node [style=none] (91) at (-2.5, 10.5) {};
		\node [style=none] (92) at (-0.25, 10.5) {};
	\end{pgfonlayer}
	\begin{pgfonlayer}{edgelayer}
		\draw (78) to (77);
		\draw (77) to (76);
		\draw (76) to (79);
		\draw (80) to (77);
		\draw (83) to (84);
		\draw (83) to (82);
		\draw (85) to (87);
		\draw (85) to (86);
		\draw (77) to (85);
		\draw (83) to (76);
		\draw (89.center) to (82);
		\draw (82) to (91.center);
		\draw (90.center) to (86);
		\draw (86) to (88.center);
		\draw (78) to (92.center);
		\draw (81.center) to (78);
	\end{pgfonlayer}
\end{tikzpicture}\\
&\eq{Lem. \ref{lemma:Iwama}}
\begin{tikzpicture}
	\begin{pgfonlayer}{nodelayer}
		\node [style=dot] (77) at (-2, 12.5) {};
		\node [style=dot] (78) at (-1, 12.5) {};
		\node [style=oplus] (79) at (-0.25, 12.5) {};
		\node [style=X] (80) at (-2, 13.5) {};
		\node [style=X] (81) at (-1, 13.5) {};
		\node [style=none] (82) at (-0.25, 14) {};
		\node [style=dot] (83) at (-2.5, 13) {};
		\node [style=oplus] (84) at (-2, 13) {};
		\node [style=zeroin] (85) at (-2, 11) {};
		\node [style=oplus] (86) at (-1, 11.5) {};
		\node [style=dot] (87) at (-1.5, 11.5) {};
		\node [style=zeroin] (88) at (-1, 11) {};
		\node [style=none] (89) at (-1.5, 14) {};
		\node [style=none] (90) at (-2.5, 14) {};
		\node [style=none] (91) at (-1.5, 10.5) {};
		\node [style=none] (92) at (-2.5, 10.5) {};
		\node [style=none] (93) at (-0.25, 10.5) {};
		\node [style=dot] (94) at (-2.5, 12) {};
		\node [style=dot] (95) at (-1, 12) {};
		\node [style=oplus] (96) at (-0.25, 12) {};
	\end{pgfonlayer}
	\begin{pgfonlayer}{edgelayer}
		\draw (79) to (78);
		\draw (78) to (77);
		\draw (77) to (80);
		\draw (81) to (78);
		\draw (84) to (85);
		\draw (84) to (83);
		\draw (86) to (88);
		\draw (86) to (87);
		\draw (78) to (86);
		\draw (84) to (77);
		\draw (90.center) to (83);
		\draw (83) to (92.center);
		\draw (91.center) to (87);
		\draw (87) to (89.center);
		\draw (79) to (93.center);
		\draw (82.center) to (79);
		\draw (96) to (95);
		\draw (95) to (94);
	\end{pgfonlayer}
\end{tikzpicture}
\eq{\ref{TOF.2}}
\begin{tikzpicture}
	\begin{pgfonlayer}{nodelayer}
		\node [style=X] (78) at (-2, 13) {};
		\node [style=X] (79) at (-1, 13) {};
		\node [style=none] (80) at (-0.25, 13.5) {};
		\node [style=dot] (81) at (-2.5, 12.5) {};
		\node [style=oplus] (82) at (-2, 12.5) {};
		\node [style=zeroin] (83) at (-2, 11) {};
		\node [style=oplus] (84) at (-1, 11.5) {};
		\node [style=dot] (85) at (-1.5, 11.5) {};
		\node [style=zeroin] (86) at (-1, 11) {};
		\node [style=none] (87) at (-1.5, 13.5) {};
		\node [style=none] (88) at (-2.5, 13.5) {};
		\node [style=none] (89) at (-1.5, 10.5) {};
		\node [style=none] (90) at (-2.5, 10.5) {};
		\node [style=none] (91) at (-0.25, 10.5) {};
		\node [style=dot] (92) at (-2.5, 12) {};
		\node [style=dot] (93) at (-1, 12) {};
		\node [style=oplus] (94) at (-0.25, 12) {};
	\end{pgfonlayer}
	\begin{pgfonlayer}{edgelayer}
		\draw (82) to (83);
		\draw (82) to (81);
		\draw (84) to (86);
		\draw (84) to (85);
		\draw (88.center) to (81);
		\draw (81) to (90.center);
		\draw (89.center) to (85);
		\draw (85) to (87.center);
		\draw (94) to (93);
		\draw (93) to (92);
		\draw (80.center) to (91.center);
		\draw (84) to (79);
		\draw (78) to (82);
	\end{pgfonlayer}
\end{tikzpicture}
\eq{unit}
\begin{tikzpicture}
	\begin{pgfonlayer}{nodelayer}
		\node [style=X] (79) at (-1, 13) {};
		\node [style=none] (80) at (-0.25, 13.5) {};
		\node [style=oplus] (81) at (-1, 11.5) {};
		\node [style=dot] (82) at (-1.5, 11.5) {};
		\node [style=zeroin] (83) at (-1, 11) {};
		\node [style=none] (84) at (-1.5, 13.5) {};
		\node [style=none] (85) at (-2, 13.5) {};
		\node [style=none] (86) at (-1.5, 10.5) {};
		\node [style=none] (87) at (-2, 10.5) {};
		\node [style=none] (88) at (-0.25, 10.5) {};
		\node [style=dot] (89) at (-2, 12) {};
		\node [style=dot] (90) at (-1, 12) {};
		\node [style=oplus] (91) at (-0.25, 12) {};
	\end{pgfonlayer}
	\begin{pgfonlayer}{edgelayer}
		\draw (81) to (83);
		\draw (81) to (82);
		\draw (86.center) to (82);
		\draw (82) to (84.center);
		\draw (91) to (90);
		\draw (90) to (89);
		\draw (80.center) to (88.center);
		\draw (81) to (79);
		\draw (85.center) to (87.center);
	\end{pgfonlayer}
\end{tikzpicture}\\
&\eq{Lem.  \ref{lemma:Iwama}}
\begin{tikzpicture}
	\begin{pgfonlayer}{nodelayer}
		\node [style=X] (80) at (-1, 13) {};
		\node [style=none] (81) at (-0.25, 13.5) {};
		\node [style=oplus] (82) at (-1, 12.5) {};
		\node [style=dot] (83) at (-1.5, 12.5) {};
		\node [style=zeroin] (84) at (-1, 11) {};
		\node [style=none] (85) at (-1.5, 13.5) {};
		\node [style=none] (86) at (-2, 13.5) {};
		\node [style=none] (87) at (-1.5, 10.5) {};
		\node [style=none] (88) at (-2, 10.5) {};
		\node [style=none] (89) at (-0.25, 10.5) {};
		\node [style=dot] (90) at (-2, 12) {};
		\node [style=dot] (91) at (-1, 12) {};
		\node [style=oplus] (92) at (-0.25, 12) {};
		\node [style=dot] (93) at (-2, 11.5) {};
		\node [style=dot] (94) at (-1.5, 11.5) {};
		\node [style=oplus] (95) at (-0.25, 11.5) {};
	\end{pgfonlayer}
	\begin{pgfonlayer}{edgelayer}
		\draw (82) to (84);
		\draw (82) to (83);
		\draw (87.center) to (83);
		\draw (83) to (85.center);
		\draw (92) to (91);
		\draw (91) to (90);
		\draw (81.center) to (89.center);
		\draw (82) to (80);
		\draw (86.center) to (88.center);
		\draw (95) to (93);
	\end{pgfonlayer}
\end{tikzpicture}
\eq{\ref{TOF.2}}
\begin{tikzpicture}
	\begin{pgfonlayer}{nodelayer}
		\node [style=X] (81) at (-1, 12.5) {};
		\node [style=none] (82) at (-0.25, 13) {};
		\node [style=oplus] (83) at (-1, 12) {};
		\node [style=dot] (84) at (-1.5, 12) {};
		\node [style=zeroin] (85) at (-1, 11) {};
		\node [style=none] (86) at (-1.5, 13) {};
		\node [style=none] (87) at (-2, 13) {};
		\node [style=none] (88) at (-1.5, 10.5) {};
		\node [style=none] (89) at (-2, 10.5) {};
		\node [style=none] (90) at (-0.25, 10.5) {};
		\node [style=dot] (91) at (-2, 11.5) {};
		\node [style=dot] (92) at (-1.5, 11.5) {};
		\node [style=oplus] (93) at (-0.25, 11.5) {};
	\end{pgfonlayer}
	\begin{pgfonlayer}{edgelayer}
		\draw (83) to (85);
		\draw (83) to (84);
		\draw (88.center) to (84);
		\draw (84) to (86.center);
		\draw (82.center) to (90.center);
		\draw (83) to (81);
		\draw (87.center) to (89.center);
		\draw (93) to (91);
	\end{pgfonlayer}
\end{tikzpicture}
\eq{unit}
\begin{tikzpicture}
	\begin{pgfonlayer}{nodelayer}
		\node [style=dot] (82) at (-2, 11.25) {};
		\node [style=dot] (83) at (-1.5, 11.25) {};
		\node [style=oplus] (84) at (-1, 11.25) {};
		\node [style=none] (85) at (-1, 12) {};
		\node [style=none] (86) at (-1, 10.5) {};
		\node [style=none] (87) at (-2, 10.5) {};
		\node [style=none] (88) at (-1.5, 10.5) {};
		\node [style=none] (89) at (-1.5, 12) {};
		\node [style=none] (90) at (-2, 12) {};
	\end{pgfonlayer}
	\begin{pgfonlayer}{edgelayer}
		\draw (84) to (83);
		\draw (83) to (82);
		\draw (90.center) to (82);
		\draw (82) to (87.center);
		\draw (88.center) to (83);
		\draw (83) to (89.center);
		\draw (85.center) to (84);
		\draw (84) to (86.center);
	\end{pgfonlayer}
\end{tikzpicture}
\end{align*}
\end{proof}


Recall the following proposition:

\begin{proposition}\cite[Prop. 2.6]{bruni}\footnote{In \cite{bruni}, they do not prove this equivalence is monoidal, but it is an obvious corollary. They also do not consider the finite case.}
The category $\Span^\sim(\FinOrd)$ equipped with the Cartesian product is monoidally equivalent to the category of (finite)  matrices over the natural numbers and the Kronecker product.
\end{proposition}

Thus,

\begin{corollary}
$\ZXA$ is complete for the prop of $2^n\times 2^m$ matrices over the natural numbers.
\end{corollary} 


\section{Conclusion}
There are various other directions which could be pursued.  One could also ask if there is a normal form for $\ZXA$ induced by the presentation in terms of distributive laws and monoid maps, using the correspondence between strict factorization systems and distributive laws in spans \cite{rosebrugh}. %Second, one could compute various other fragments of the ZX-calculus by performing a two way translation between the (co)unital completion of some other discrete inverse category.  %The category $\CNOT$ comes to mind, where presumably, this would yield affine multirelations, generalizing the affine relations of \cite[\S 4.3]{piedeleu}.
\nocite{piedeleu} 
% One could also apply this construction to ``infinite dimensional'' discrete inverse categories, to obtain hypergraph categories which can not be interpreted into Hilbert spaces.
It would also be interesting to investigate the 2-categorical structure of $\ZXA$; presenting the corresponding category of relations as a Frobenius theory \cite{functorial} using the partial order enrichment of $\TOF$.

Another immediate direction would be to add the white $\pi$ phase to $\ZXA$ to obtain an approximately universal graphical calculus for quantum computing using only distributive laws and monoid maps.  In such a fragment, one could construct the {\sf and} gate for the $X$ basis; perhaps expanding the table of distributive laws in Figure \ref{fig:table} to be complete for an approximately universal fragment of quantum computing, furthering the general programme of \cite{ihpub,duncan} decomposing circuits using distributive laws.  This  approach is contrasted to considering H-boxes as primitives, as in the phase-free fragment of the $\ZH$-calculus \cite{zhpi}---in $\ZXA$+the white $\pi$ phase, the unnormalized Hadamard gate is derived. Perhaps proving the minimality of the axioms using this presentation might be easier, although we do not prove minimality in this paper.

It would also be interesting to investigate the connection to the $\ZH$-calculus and triangle fragments of the $\ZX$-calculus; in particular, in regard to natural number labelled H-boxes, as in \cite{natspiders}.  %The triangle, and the natural number labeled H boxes are given below:
These gates can be represented in string diagrams. The diagram of the triangle can be interpreted as the assertion  $x\wedge \neg y =  \bot$ which is equivalent to the material implication  $ x \Rightarrow y$.
\begin{figure}[H]

$$
\begin{tikzpicture}
	\begin{pgfonlayer}{nodelayer}
		\node [style=none] (83) at (0.75, 11.25) {};
		\node [style=none] (84) at (0.75, 10.5) {};
		\node [style=none] (85) at (0.75, 12) {};
		\node [style=triflip] (86) at (0.75, 11.25) {};
	\end{pgfonlayer}
	\begin{pgfonlayer}{edgelayer}
		\draw (85.center) to (83.center);
		\draw (83.center) to (84.center);
	\end{pgfonlayer}
\end{tikzpicture}
:=
\begin{tikzpicture}
	\begin{pgfonlayer}{nodelayer}
		\node [style=none] (84) at (0.75, 11.5) {};
		\node [style=none] (85) at (1.25, 11.5) {};
		\node [style=Z] (86) at (0.75, 12.25) {};
		\node [style=none] (87) at (0.75, 10.5) {};
		\node [style=andin] (88) at (0.75, 11.5) {};
		\node [style=none] (89) at (1.25, 13) {};
		\node [style=Z] (90) at (1.25, 12.25) {$\pi$};
	\end{pgfonlayer}
	\begin{pgfonlayer}{edgelayer}
		\draw [style=simple] (86) to (84.center);
		\draw [style=simple, in=-75, out=-90, looseness=2.75] (85.center) to (84.center);
		\draw [style=simple] (87.center) to (84.center);
		\draw (89.center) to (90);
		\draw (90) to (85.center);
	\end{pgfonlayer}
\end{tikzpicture}
\hspace*{1cm}
\begin{tikzpicture}
	\begin{pgfonlayer}{nodelayer}
		\node [style=H] (85) at (1, 11.25) {$n$};
		\node [style=none] (86) at (1, 10.5) {};
		\node [style=none] (87) at (1, 12) {};
	\end{pgfonlayer}
	\begin{pgfonlayer}{edgelayer}
		\draw (87.center) to (85);
		\draw (85) to (86.center);
	\end{pgfonlayer}
\end{tikzpicture}
:=
\begin{tikzpicture}
	\begin{pgfonlayer}{nodelayer}
		\node [style=none] (1) at (0.75, -0.5) {};
		\node [style=none] (2) at (1.25, -0.5) {};
		\node [style=Z] (3) at (0.75, 2.75) {$\pi$};
		\node [style=none] (4) at (0.75, -1.5) {};
		\node [style=andin] (5) at (0.75, -0.5) {};
		\node [style=none] (6) at (1.25, 3.25) {};
		\node [style=triflip] (7) at (0.75, 2) {};
		\node [style=triflip] (8) at (0.75, 1) {};
		\node [style=none] (9) at (0.5, 1.5) {$n$};
		\node [style=Z] (10) at (0.75, 0.25) {$\pi$};
	\end{pgfonlayer}
	\begin{pgfonlayer}{edgelayer}
		\draw [style=simple, in=-75, out=-90, looseness=2.75] (2.center) to (1.center);
		\draw [style=simple] (4.center) to (1.center);
		\draw [style=dotted] (7) to (8);
		\draw (7) to (3);
		\draw (8) to (1.center);
		\draw [style=simple] (2.center) to (6.center);
	\end{pgfonlayer}
\end{tikzpicture}
$$

\caption{Triangles and H-boxes in \texorpdfstring{$\ZXA$}{ZX\&}, for \texorpdfstring{$n\in \N$}{n a natural number}.}
\label{fig:gens}
\end{figure}

Therefore, to get a universal and complete graphical calculus for boolean relations, one must simply impose the following equation corresponding to the sequent $ (x\Rightarrow y) \wedge ( y \Rightarrow z ) \vdash (x \Rightarrow z)$:

$$
\begin{tikzpicture}
	\begin{pgfonlayer}{nodelayer}
		\node [style=none] (0) at (0.75, 11.25) {};
		\node [style=none] (2) at (0.75, 12) {};
		\node [style=triflip] (3) at (0.75, 11.25) {};
		\node [style=none] (4) at (0.75, 10.25) {};
		\node [style=none] (5) at (0.75, 9.5) {};
		\node [style=triflip] (7) at (0.75, 10.25) {};
	\end{pgfonlayer}
	\begin{pgfonlayer}{edgelayer}
		\draw (2.center) to (0.center);
		\draw (4.center) to (5.center);
		\draw (7) to (3);
	\end{pgfonlayer}
\end{tikzpicture}
=
\begin{tikzpicture}
	\begin{pgfonlayer}{nodelayer}
		\node [style=none] (2) at (0.75, 12) {};
		\node [style=none] (4) at (0.75, 10.75) {};
		\node [style=none] (5) at (0.75, 9.5) {};
		\node [style=triflip] (7) at (0.75, 10.75) {};
	\end{pgfonlayer}
	\begin{pgfonlayer}{edgelayer}
		\draw (4.center) to (5.center);
		\draw (7) to (2.center);
	\end{pgfonlayer}
\end{tikzpicture}
$$


%
%\nocite{coecke2008classical}
%\nocite{cnot}
%\nocite{tof}
%\nocite{Cole}
%\nocite{elltwo}
%%\nocite{sam}
%\nocite{coecke2017two}
%\nocite{carboni}
%\nocite{butz}
%\nocite{pqp}
%\nocite{lack2004composing}


\chapter{Decomposing fragments of the ZX/ZH-calculus}
In this section, we modularly build up to the presentation of $\ZXA$ by taking distributive laws and pushouts of smaller symmetric monoidal theories.  Along the way, we obtain various fragments of quantum circuits with interesting partial and reversible semantics.


%%We first review some basic theory involving the presentation of props.  These results are mostly folklore, however, I will refer the reader to \cite[\S 2]{ih} for a more comprehensive introduction.
%
%\begin{definition}
%A {\bf pro} is a strict monoidal category generated by one object under the tensor product, and a {\bf prop} is a  strict {\em symmetric} monoidal category generated by one object under the tensor product
%
%\end{definition}

%
%\begin{definition}
%A {\bf monoidal theory} is a pair $(\Sigma,E)$ of {\bf generators} $\Sigma$ and {\bf equations} $E$.
%Each generator $f \in \Sigma$ has a chosen domain $\dom (f) \in \N$  and codomain $\cod (f) \in \N$, so that $f$ can be seen as a map from $\dom(f)$ to $\cod (f)$.
%
%The free pro with signature $\Sigma$ has maps in $\Sigma^*$ obtained by inductively  tensoring all the generators and composing all appropriately typed generators in $\Sigma$,
%The equations in $E$ are pairs of parallel maps in $\Sigma^*$.
%Any monoidal theory $(\Sigma,E)$  generates a pro $\bar{(\Sigma,E)}$ given by the free pro with signature $\Sigma$ modulo the equations in $E$.
%
%A {\bf symmetric monoidal theory} is the symmetric version of a monoidal theory, which generates a prop.  Here the set $\Sigma^*$ is obtained by composing and tensoring maps with symmetries, and then quotienting by the axioms of a prop.
%\end{definition}
%
%\begin{lemma}
%Given two (symmetric) monoidal theories $(\Sigma_1,E_1)$  and $(\Sigma_2,E_2)$  the coproduct of pro(p)s  $\bar{(\Sigma_1,E_1)}+\bar{(\Sigma_2,E_2)}$ is generated by the (symmetric) monoidal theory $(\Sigma_1+\Sigma_2,E_1+E_2)$.
%\end{lemma}
%
%
%\begin{lemma}
%Given three  (symmetric) monoidal theories $(\Sigma_1,E_1)$, $(\Sigma_2,E_2)$ and $(\Sigma_3,E_3)$ where $\bar{(\Sigma_3,E_3)}$ is a sub-pro(p) of both $\bar{(\Sigma_1,E_1)}$ and $\bar{(\Sigma_2,E_2)}$.  The pushout of the diagram of pro(p)s
%$$
%\bar{(\Sigma_1,E_1)} \leftarrow \bar{(\Sigma_3,E_3)}\rightarrow \bar{(\Sigma_2,E_2)}
%$$
%is generated by the (symmetric) monoidal theory $(\Sigma_1^* +_{\Sigma_3} \Sigma_2^*, E_1 + E_2)$.
%\end{lemma}

%
%\begin{definition}
%Informally, given two small categories $\X,\Y$ with the same objects; a distributive law on $\Y \otimes \X$ is a way to turn formal composites of maps in $\Y$ followed by $\X$ into a category.  That is to say, a quotient which gives us a way to turn composites of the form $\xrightarrow{f \in \X} \xrightarrow{g \in \Y}$  into ones of the form  $\xrightarrow{g' \in \Y} \xrightarrow{f' \in \X}$.
%
%If $\X$ and $\Y$ have some shared structure witnessed by a subcategory $\Z$ with the same objects, a relaxed distributive law $\Y \otimes_{\Z} \X$ is like a distributive law $\Y \otimes \X$, except where the maps in $\Y$ can be identified as either being in $\X$ or in $\Y$.
%\footnote{These are called distributive laws because small categories are monads in the bicategory $\Span(\Sets)$; where distributive laws in the first sense are distributive laws of the corresponding monads in $\Span(\Sets)$.  In the latter case, these are given by distributive laws of bimodules of spans of sets (ie. distributive laws in small profunctors).}
%\end{definition}
%
%This second notion of distributive law will be needed when $\X$ and $\Y$ are props; because they (minimally) share a subcategory $\P$ of permutations.

%We recall the novel way to compose pro(p), first described in \cite{lack}:
%\begin{definition}
%Suppose there three  (symmetric) monoidal theories $(\Sigma_1,E_1)$, $(\Sigma_2,E_2)$ and $(\Sigma_3,E_3)$ where $\bar{(\Sigma_3,E_3)}$ is a sub-pro(p) of both $\bar{(\Sigma_1,E_1)}$ and $\bar{(\Sigma_2,E_2)}$. A {\bf distributive law of pro(p)s} is a distributive law $\lambda:\bar{(\Sigma_2,E_2)} \otimes_{\bar{(\Sigma_3,E_3)}} \bar{ (\Sigma_1,E_1)}$   in $\Mon$-$\Prof$.  Informally, this is a way to push all the maps in $\Sigma_1^*$ past those of  $\Sigma_2^*$ modulo $\Sigma_3$ and the equations $E_1+E_2$ and the axioms of a pro(p).
%\end{definition}
%
%In \cite{lack} it is required that $\bar{(\Sigma_3,E_3)}$ is a groupoid; however, we must loosen this requirement (note that when this is not a groupoid, there is no correspondence to factorization systems as in \cite{rosebrugh}).  

%\begin{lemma}
%Suppose that we have three (symmetric) monoidal theories and a distribtuive law $\lambda:\bar{(\Sigma_2,E_2)} \otimes_{\bar{(\Sigma_3,E_3)}} \bar{ (\Sigma_1,E_1)}$ as above.
%
%Then the induced pro(p) $\bar{(\Sigma_2,E_2)} \otimes_{\bar{(\Sigma_3,E_3)}} \bar{ (\Sigma_1,E_1)}$ is presented by the monoidal theory\\ ${(\Sigma_1^* +_{\Sigma_3} \Sigma_2^*, E_1 + E_2+E_\lambda)}$, where $E_\lambda$ are all the equations needed to push elements of $\Sigma_1^*$ past those of $ \Sigma_2^*$ up to  $\Sigma_3^*$, dictated by $\lambda$.
%\end{lemma}






\subsection{The phase-free fragment}
\label{sec:one}

In this section we build up to giving a presentation for $(\Span^\sim(\Mat(\F_2)),+)$ in a modular way. This category is shown to be the same as the phase-free Hadamard free fragment of the ZX-calculus. Although this presentation of linear spans has already been discussed in great detail  for arbitrary PIDs \cite{ih}, our particular method of exposition is necessary to motivate the affine and full cases. 


\begin{definition}
Consider the prop $\Iso(\cb_2)$ generated by the controlled not gate modulo the following relations:

$$
\begin{tikzpicture}
	\begin{pgfonlayer}{nodelayer}
		\node [style=dot] (0) at (7, 0) {};
		\node [style=oplus] (1) at (7.5, 0) {};
		\node [style=oplus] (2) at (7.5, 0.5) {};
		\node [style=dot] (3) at (7, 0.5) {};
		\node [style=none] (4) at (7.5, 1) {};
		\node [style=none] (5) at (7, 1) {};
		\node [style=none] (6) at (7, -0.5) {};
		\node [style=none] (7) at (7.5, -0.5) {};
	\end{pgfonlayer}
	\begin{pgfonlayer}{edgelayer}
		\draw (5.center) to (3);
		\draw (3) to (0);
		\draw (0) to (6.center);
		\draw (7.center) to (1);
		\draw (1) to (2);
		\draw (2) to (4.center);
		\draw (2) to (3);
		\draw (0) to (1);
	\end{pgfonlayer}
\end{tikzpicture}
\eqzxa{cnot.one}
\begin{tikzpicture}
	\begin{pgfonlayer}{nodelayer}
		\node [style=none] (4) at (7.5, 1) {};
		\node [style=none] (5) at (7, 1) {};
		\node [style=none] (6) at (7, -0.5) {};
		\node [style=none] (7) at (7.5, -0.5) {};
	\end{pgfonlayer}
	\begin{pgfonlayer}{edgelayer}
		\draw (7.center) to (4.center);
		\draw (5.center) to (6.center);
	\end{pgfonlayer}
\end{tikzpicture}
\hspace*{.5cm}
\begin{tikzpicture}
	\begin{pgfonlayer}{nodelayer}
		\node [style=none] (4) at (7.5, 1) {};
		\node [style=none] (5) at (7, 1) {};
		\node [style=none] (6) at (7, -1) {};
		\node [style=none] (7) at (7.5, -1) {};
		\node [style=dot] (8) at (7.5, 0.5) {};
		\node [style=dot] (9) at (7.5, -0.5) {};
		\node [style=dot] (10) at (7, 0) {};
		\node [style=oplus] (11) at (7.5, 0) {};
		\node [style=oplus] (12) at (7, 0.5) {};
		\node [style=oplus] (13) at (7, -0.5) {};
	\end{pgfonlayer}
	\begin{pgfonlayer}{edgelayer}
		\draw (7.center) to (4.center);
		\draw (5.center) to (6.center);
		\draw (8) to (12);
		\draw (10) to (11);
		\draw (9) to (13);
	\end{pgfonlayer}
\end{tikzpicture}
\eqzxa{cnot.two}
\begin{tikzpicture}
	\begin{pgfonlayer}{nodelayer}
		\node [style=none] (4) at (7, 1) {};
		\node [style=none] (5) at (7.5, 1) {};
		\node [style=none] (6) at (7, -1) {};
		\node [style=none] (7) at (7.5, -1) {};
	\end{pgfonlayer}
	\begin{pgfonlayer}{edgelayer}
		\draw [in=270, out=90] (7.center) to (4.center);
		\draw [in=90, out=-90] (5.center) to (6.center);
	\end{pgfonlayer}
\end{tikzpicture}
\hspace*{.5cm}
\begin{tikzpicture}
	\begin{pgfonlayer}{nodelayer}
		\node [style=dot] (0) at (8, 0) {};
		\node [style=dot] (1) at (8.5, 0.5) {};
		\node [style=dot] (2) at (8.5, -0.5) {};
		\node [style=oplus] (3) at (8.5, 0) {};
		\node [style=oplus] (4) at (9, 0.5) {};
		\node [style=oplus] (5) at (9, -0.5) {};
		\node [style=none] (6) at (9, -1) {};
		\node [style=none] (7) at (8.5, -1) {};
		\node [style=none] (8) at (8, -1) {};
		\node [style=none] (9) at (8, 1) {};
		\node [style=none] (10) at (8.5, 1) {};
		\node [style=none] (11) at (9, 1) {};
	\end{pgfonlayer}
	\begin{pgfonlayer}{edgelayer}
		\draw (9.center) to (8.center);
		\draw (7.center) to (10.center);
		\draw (11.center) to (6.center);
		\draw (5) to (2);
		\draw (3) to (0);
		\draw (1) to (4);
	\end{pgfonlayer}
\end{tikzpicture}
\eqzxa{cnot.three}
\begin{tikzpicture}
	\begin{pgfonlayer}{nodelayer}
		\node [style=dot] (0) at (8, -0.25) {};
		\node [style=dot] (1) at (8, 0.25) {};
		\node [style=oplus] (3) at (8.5, -0.25) {};
		\node [style=oplus] (4) at (9, 0.25) {};
		\node [style=none] (6) at (9, -1) {};
		\node [style=none] (7) at (8.5, -1) {};
		\node [style=none] (8) at (8, -1) {};
		\node [style=none] (9) at (8, 1) {};
		\node [style=none] (10) at (8.5, 1) {};
		\node [style=none] (11) at (9, 1) {};
	\end{pgfonlayer}
	\begin{pgfonlayer}{edgelayer}
		\draw (9.center) to (8.center);
		\draw (7.center) to (10.center);
		\draw (11.center) to (6.center);
		\draw (3) to (0);
		\draw (1) to (4);
	\end{pgfonlayer}
\end{tikzpicture}
\hspace*{.5cm}
\begin{tikzpicture}
	\begin{pgfonlayer}{nodelayer}
		\node [style=dot] (0) at (2, 1.5) {};
		\node [style=oplus] (1) at (2.5, 1.5) {};
		\node [style=none] (3) at (2, 1) {};
		\node [style=none] (4) at (2.5, 1) {};
		\node [style=none] (5) at (2, 2.5) {};
		\node [style=none] (6) at (2.5, 2.5) {};
		\node [style=none] (7) at (3, 2.5) {};
		\node [style=none] (8) at (3, 1) {};
		\node [style=dot] (9) at (3, 2) {};
		\node [style=oplus] (10) at (2.5, 2) {};
	\end{pgfonlayer}
	\begin{pgfonlayer}{edgelayer}
		\draw (0) to (5.center);
		\draw (6.center) to (1);
		\draw (1) to (0);
		\draw (4.center) to (1);
		\draw (3.center) to (0);
		\draw (10) to (9);
		\draw (8.center) to (7.center);
	\end{pgfonlayer}
\end{tikzpicture}
\eqzxa{cnot.four}
\begin{tikzpicture}
	\begin{pgfonlayer}{nodelayer}
		\node [style=dot] (0) at (2, 2) {};
		\node [style=oplus] (1) at (2.5, 2) {};
		\node [style=none] (3) at (2, 2.5) {};
		\node [style=none] (4) at (2.5, 2.5) {};
		\node [style=none] (5) at (2, 1) {};
		\node [style=none] (6) at (2.5, 1) {};
		\node [style=none] (7) at (3, 1) {};
		\node [style=none] (8) at (3, 2.5) {};
		\node [style=dot] (9) at (3, 1.5) {};
		\node [style=oplus] (10) at (2.5, 1.5) {};
	\end{pgfonlayer}
	\begin{pgfonlayer}{edgelayer}
		\draw (0) to (5.center);
		\draw (6.center) to (1);
		\draw (1) to (0);
		\draw (4.center) to (1);
		\draw (3.center) to (0);
		\draw (10) to (9);
		\draw (8.center) to (7.center);
	\end{pgfonlayer}
\end{tikzpicture}
\hspace*{.5cm}
\begin{tikzpicture}
	\begin{pgfonlayer}{nodelayer}
		\node [style=oplus] (0) at (2, 1.5) {};
		\node [style=dot] (1) at (2.5, 1.5) {};
		\node [style=none] (3) at (2, 1) {};
		\node [style=none] (4) at (2.5, 1) {};
		\node [style=none] (5) at (2, 2.5) {};
		\node [style=none] (6) at (2.5, 2.5) {};
		\node [style=none] (7) at (3, 2.5) {};
		\node [style=none] (8) at (3, 1) {};
		\node [style=oplus] (9) at (3, 2) {};
		\node [style=dot] (10) at (2.5, 2) {};
	\end{pgfonlayer}
	\begin{pgfonlayer}{edgelayer}
		\draw (0) to (5.center);
		\draw (6.center) to (1);
		\draw (1) to (0);
		\draw (4.center) to (1);
		\draw (3.center) to (0);
		\draw (10) to (9);
		\draw (8.center) to (7.center);
	\end{pgfonlayer}
\end{tikzpicture}
\eqzxa{cnot.five}
\begin{tikzpicture}
	\begin{pgfonlayer}{nodelayer}
		\node [style=oplus] (0) at (2, 2) {};
		\node [style=dot] (1) at (2.5, 2) {};
		\node [style=none] (3) at (2, 2.5) {};
		\node [style=none] (4) at (2.5, 2.5) {};
		\node [style=none] (5) at (2, 1) {};
		\node [style=none] (6) at (2.5, 1) {};
		\node [style=none] (7) at (3, 1) {};
		\node [style=none] (8) at (3, 2.5) {};
		\node [style=oplus] (9) at (3, 1.5) {};
		\node [style=dot] (10) at (2.5, 1.5) {};
	\end{pgfonlayer}
	\begin{pgfonlayer}{edgelayer}
		\draw (0) to (5.center);
		\draw (6.center) to (1);
		\draw (1) to (0);
		\draw (4.center) to (1);
		\draw (3.center) to (0);
		\draw (10) to (9);
		\draw (8.center) to (7.center);
	\end{pgfonlayer}
\end{tikzpicture}
$$
\end{definition}




\begin{lemma} \cite[Thm. 6]{lafont}
$\Iso(\cb_2)$ is a presentation for the prop $(\Iso(\Mat(\F_2),+))$ with respect to the interpretation:

$$
\left\llbracket
\begin{tikzpicture}
	\begin{pgfonlayer}{nodelayer}
		\node [style=oplus] (5) at (0.5, 2.75) {};
		\node [style=dot] (6) at (0, 2.75) {};
		\node [style=none] (7) at (0.5, 3.5) {};
		\node [style=none] (8) at (0.5, 2) {};
		\node [style=none] (9) at (0, 2) {};
		\node [style=none] (10) at (0, 3.5) {};
	\end{pgfonlayer}
	\begin{pgfonlayer}{edgelayer}
		\draw (8.center) to (5);
		\draw (5) to (7.center);
		\draw (10.center) to (6);
		\draw (6) to (5);
		\draw (6) to (9.center);
	\end{pgfonlayer}
\end{tikzpicture}
\right\rrbracket
=
\begin{tikzpicture}
	\begin{pgfonlayer}{nodelayer}
		\node [style=X] (0) at (-0.25, -1) {};
		\node [style=Z] (1) at (-0.75, -1.75) {};
		\node [style=none] (2) at (0, -2.25) {};
		\node [style=none] (3) at (-0.25, -0.5) {};
		\node [style=none] (4) at (-1, -0.5) {};
		\node [style=none] (5) at (-0.75, -2.25) {};
	\end{pgfonlayer}
	\begin{pgfonlayer}{edgelayer}
		\draw (3.center) to (0);
		\draw [in=90, out=-75] (0) to (2.center);
		\draw (5.center) to (1);
		\draw (1) to (0);
		\draw [in=-90, out=105] (1) to (4.center);
	\end{pgfonlayer}
\end{tikzpicture} 
$$
\end{lemma}



\begin{definition}
Consider the prop $\inj(\cb_2)$ generated by the coproduct of props $\Iso(\cb_2)+\inj$ modulo the equation:
\hspace*{1cm}
$
\begin{tikzpicture}
	\begin{pgfonlayer}{nodelayer}
		\node [style=X] (0) at (0, 0) {};
		\node [style=dot] (1) at (0, 0.5) {};
		\node [style=oplus] (2) at (0.5, 0.5) {};
		\node [style=none] (3) at (0.5, -0.25) {};
		\node [style=none] (4) at (0.5, 1) {};
		\node [style=none] (5) at (0, 1) {};
	\end{pgfonlayer}
	\begin{pgfonlayer}{edgelayer}
		\draw (0) to (1);
		\draw (1) to (5.center);
		\draw (1) to (2);
		\draw (2) to (4.center);
		\draw (3.center) to (2);
	\end{pgfonlayer}
\end{tikzpicture}
\eqzxa{cnot.six}
\begin{tikzpicture}
	\begin{pgfonlayer}{nodelayer}
		\node [style=X] (0) at (0, 0) {};
		\node [style=none] (3) at (0.5, -0.25) {};
		\node [style=none] (4) at (0.5, 0.5) {};
		\node [style=none] (5) at (0, 0.5) {};
	\end{pgfonlayer}
	\begin{pgfonlayer}{edgelayer}
		\draw (0) to (5.center);
		\draw (3.center) to (4.center);
	\end{pgfonlayer}
\end{tikzpicture}
$

\end{definition}


\begin{lemma} \cite[Thm. 7]{lafont}
$\inj(\cb_2)$ is a presentation for the prop $(\inj(\Mat(\F_2)),+)$
\end{lemma}


The white comultiplication can be derived in this fragment:

$$
\left\llbracket
\begin{tikzpicture}
	\begin{pgfonlayer}{nodelayer}
		\node [style=oplus] (0) at (1, 2.75) {};
		\node [style=dot] (1) at (0.5, 2.75) {};
		\node [style=none] (2) at (1, 3.5) {};
		\node [style=none] (3) at (1, 2.25) {};
		\node [style=none] (4) at (0.5, 2) {};
		\node [style=none] (5) at (0.5, 3.5) {};
		\node [style=X] (6) at (1, 2.25) {};
	\end{pgfonlayer}
	\begin{pgfonlayer}{edgelayer}
		\draw (3.center) to (0);
		\draw (0) to (2.center);
		\draw (5.center) to (1);
		\draw (1) to (0);
		\draw (1) to (4.center);
	\end{pgfonlayer}
\end{tikzpicture}
\right\rrbracket
=
\begin{tikzpicture}
	\begin{pgfonlayer}{nodelayer}
		\node [style=X] (0) at (1.25, -1) {};
		\node [style=Z] (1) at (0.75, -1.75) {};
		\node [style=none] (2) at (1.5, -2) {};
		\node [style=none] (3) at (1.25, -0.5) {};
		\node [style=none] (4) at (0.5, -0.5) {};
		\node [style=none] (5) at (0.75, -2.25) {};
		\node [style=X] (6) at (1.5, -2) {};
	\end{pgfonlayer}
	\begin{pgfonlayer}{edgelayer}
		\draw (3.center) to (0);
		\draw [in=90, out=-75] (0) to (2.center);
		\draw (5.center) to (1);
		\draw (1) to (0);
		\draw [in=-90, out=105] (1) to (4.center);
	\end{pgfonlayer}
\end{tikzpicture}
=
\begin{tikzpicture}
	\begin{pgfonlayer}{nodelayer}
		\node [style=Z] (1) at (0.75, -1.75) {};
		\node [style=none] (3) at (1, -1) {};
		\node [style=none] (4) at (0.5, -1) {};
		\node [style=none] (5) at (0.75, -2.25) {};
	\end{pgfonlayer}
	\begin{pgfonlayer}{edgelayer}
		\draw (5.center) to (1);
		\draw [in=-90, out=120] (1) to (4.center);
		\draw [in=-90, out=60] (1) to (3.center);
	\end{pgfonlayer}
\end{tikzpicture}
$$

%This is to be expected because there is a faithful ``'graph functor' monoidal functor $(\inj(\FinSet),+) \to (\Inj(\Span(\FinSet)),+)$.




As a matter of notation, given a category $\X$ with finite limits, we refer to the subcategory of $\Span(\X)$ where the left leg is monic as $\Par(\X)$, and the subcategory of spans where all legs are monic by $\Par\Iso(\X)$.  These two categories, respectively, give semantics for partial maps and partially invertible maps in $\X$ (see \cite{cockett} for more details).


\begin{definition}
\label{def:pariso:cb}
Consider the prop $\ParIso(\cb_2)$ generated by the distributive law of props:
$$
\inj(\cb_2)^\op \otimes_{\Iso(\cb_2)} \inj(\cb_2);
\begin{tikzpicture}
	\begin{pgfonlayer}{nodelayer}
		\node [style=X] (0) at (0, 0) {};
		\node [style=X] (1) at (0, 0.75) {};
	\end{pgfonlayer}
	\begin{pgfonlayer}{edgelayer}
		\draw (0) to (1);
	\end{pgfonlayer}
\end{tikzpicture}
\eref{extra}
$$
\end{definition}


\begin{remark}
\label{rem:pariso:cb}
This is actually a distributive law because the only seemingly nontrivial situation arises when controlled not gates are sandwiched by black units/counits on their target wires.  However the following identity holds by induction on the number of controlled not gates.   For the base case of $n=0$, this follows from the bone law which we added to the distributive law.  For $n>1$, we have the following situation:
$$
\begin{tikzpicture}
	\begin{pgfonlayer}{nodelayer}
		\node [style=X] (0) at (1.5, -0.25) {};
		\node [style=oplus] (1) at (1.5, -0.75) {};
		\node [style=oplus] (2) at (1.5, -1.5) {};
		\node [style=dot] (3) at (1, -0.75) {};
		\node [style=dot] (4) at (0.5, -1.5) {};
		\node [style=none] (5) at (1, 0) {};
		\node [style=none] (6) at (0.5, 0) {};
		\node [style=none] (7) at (0.75, -1) {$\iddots$};
		\node [style=none] (8) at (1.5, -1) {$\vdots$};
		\node [style=X] (9) at (1.5, -2.5) {};
		\node [style=none] (10) at (0.5, -2.75) {};
		\node [style=none] (11) at (1, -2.75) {};
		\node [style=none] (12) at (0.75, -0.25) {$\cdots$};
		\node [style=oplus] (13) at (1.5, -2) {};
		\node [style=dot] (14) at (0, -2) {};
		\node [style=none] (15) at (0, -2.75) {};
		\node [style=none] (16) at (0, 0) {};
		\node [style=none] (18) at (1.25, -0.75) {};
	\end{pgfonlayer}
	\begin{pgfonlayer}{edgelayer}
		\draw (0) to (1);
		\draw (1) to (3);
		\draw (5.center) to (3);
		\draw (6.center) to (4);
		\draw (4) to (2);
		\draw (4) to (10.center);
		\draw (11.center) to (3);
		\draw (9) to (2);
		\draw (14) to (13);
		\draw (15.center) to (14);
		\draw (14) to (16.center);
	\end{pgfonlayer}
\end{tikzpicture}
=
\begin{tikzpicture}
	\begin{pgfonlayer}{nodelayer}
		\node [style=X] (19) at (3.5, -1.25) {};
		\node [style=none] (24) at (3.5, 0.5) {};
		\node [style=none] (25) at (3, 0.5) {};
		\node [style=X] (28) at (3.5, -1.75) {};
		\node [style=none] (29) at (3, -3.5) {};
		\node [style=none] (30) at (3.5, -3.5) {};
		\node [style=none] (34) at (2.5, -3.5) {};
		\node [style=none] (35) at (2.5, 0.5) {};
		\node [style=oplus] (36) at (3.5, -0.75) {};
		\node [style=oplus] (37) at (3.5, 0) {};
		\node [style=oplus] (38) at (3.5, -3) {};
		\node [style=oplus] (39) at (3.5, -2.25) {};
		\node [style=dot] (40) at (2.5, -3) {};
		\node [style=dot] (41) at (3, -2.25) {};
		\node [style=dot] (42) at (3, -0.75) {};
		\node [style=dot] (43) at (2.5, 0) {};
		\node [style=none] (44) at (3.5, -0.25) {$\vdots$};
		\node [style=none] (45) at (3.5, -2.5) {$\vdots$};
		\node [style=none] (46) at (2.75, -1.5) {$\cdots$};
		\node [style=none] (47) at (2.75, -2.5) {$\iddots$};
		\node [style=none] (48) at (2.75, -0.25) {$\ddots$};
	\end{pgfonlayer}
	\begin{pgfonlayer}{edgelayer}
		\draw (24.center) to (37);
		\draw (36) to (19);
		\draw (28) to (39);
		\draw (38) to (30.center);
		\draw (29.center) to (41);
		\draw (41) to (42);
		\draw (42) to (25.center);
		\draw (35.center) to (43);
		\draw (43) to (40);
		\draw (40) to (38);
		\draw (37) to (43);
		\draw (42) to (36);
		\draw (39) to (41);
	\end{pgfonlayer}
\end{tikzpicture}
$$
For $n=1$, that is:
$$
\begin{tikzpicture}
	\begin{pgfonlayer}{nodelayer}
		\node [style=X] (0) at (4, 2) {};
		\node [style=X] (1) at (4, 1) {};
		\node [style=oplus] (2) at (4, 1.5) {};
		\node [style=dot] (3) at (3.5, 1.5) {};
		\node [style=none] (4) at (3.5, 2.25) {};
		\node [style=none] (5) at (3.5, 0.75) {};
	\end{pgfonlayer}
	\begin{pgfonlayer}{edgelayer}
		\draw (1) to (0);
		\draw (4.center) to (5.center);
		\draw (3) to (2);
	\end{pgfonlayer}
\end{tikzpicture}
=
\begin{tikzpicture}
	\begin{pgfonlayer}{nodelayer}
		\node [style=X] (0) at (4, 2.5) {};
		\node [style=X] (1) at (4, 0.5) {};
		\node [style=oplus] (2) at (4, 1.5) {};
		\node [style=dot] (3) at (3.5, 1.5) {};
		\node [style=none] (4) at (3.5, 2.75) {};
		\node [style=none] (5) at (3.5, 0.25) {};
		\node [style=oplus] (6) at (3.5, 2) {};
		\node [style=dot] (7) at (4, 2) {};
		\node [style=oplus] (8) at (3.5, 1) {};
		\node [style=dot] (9) at (4, 1) {};
	\end{pgfonlayer}
	\begin{pgfonlayer}{edgelayer}
		\draw (1) to (0);
		\draw (4.center) to (5.center);
		\draw (3) to (2);
		\draw (7) to (6);
		\draw (9) to (8);
	\end{pgfonlayer}
\end{tikzpicture}
=
\begin{tikzpicture}
	\begin{pgfonlayer}{nodelayer}
		\node [style=X] (0) at (4, 2.5) {};
		\node [style=X] (1) at (4, 1.5) {};
		\node [style=none] (4) at (3.5, 2.75) {};
		\node [style=none] (5) at (3.5, 1.25) {};
	\end{pgfonlayer}
	\begin{pgfonlayer}{edgelayer}
		\draw [in=-90, out=90] (1) to (4.center);
		\draw [in=90, out=-90] (0) to (5.center);
	\end{pgfonlayer}
\end{tikzpicture}
=
\begin{tikzpicture}
	\begin{pgfonlayer}{nodelayer}
		\node [style=X] (0) at (3.5, 1.75) {};
		\node [style=X] (1) at (3.5, 2.25) {};
		\node [style=none] (4) at (3.5, 2.75) {};
		\node [style=none] (5) at (3.5, 1.25) {};
	\end{pgfonlayer}
	\begin{pgfonlayer}{edgelayer}
		\draw [in=-90, out=90] (1) to (4.center);
		\draw [in=90, out=-90] (0) to (5.center);
	\end{pgfonlayer}
\end{tikzpicture}
$$
And for the base case for $n=2$:
$$
\begin{tikzpicture}
	\begin{pgfonlayer}{nodelayer}
		\node [style=X] (0) at (4.5, 2) {};
		\node [style=X] (1) at (4.5, 0.5) {};
		\node [style=oplus] (2) at (4.5, 1) {};
		\node [style=dot] (3) at (3.5, 1) {};
		\node [style=none] (4) at (3.5, 2.25) {};
		\node [style=none] (5) at (3.5, 0.25) {};
		\node [style=oplus] (6) at (4.5, 1.5) {};
		\node [style=dot] (7) at (4, 1.5) {};
		\node [style=none] (8) at (4, 2.25) {};
		\node [style=none] (9) at (4, 0.25) {};
	\end{pgfonlayer}
	\begin{pgfonlayer}{edgelayer}
		\draw (1) to (0);
		\draw (4.center) to (5.center);
		\draw (3) to (2);
		\draw (8.center) to (9.center);
		\draw (7) to (6);
	\end{pgfonlayer}
\end{tikzpicture}
=
\begin{tikzpicture}
	\begin{pgfonlayer}{nodelayer}
		\node [style=X] (0) at (4.5, 2.5) {};
		\node [style=X] (1) at (4.5, 0.5) {};
		\node [style=oplus] (2) at (4.5, 1) {};
		\node [style=dot] (3) at (3.5, 1) {};
		\node [style=none] (4) at (3.5, 2.75) {};
		\node [style=none] (5) at (3.5, 0.25) {};
		\node [style=oplus] (6) at (4.5, 1.5) {};
		\node [style=dot] (7) at (4, 1.5) {};
		\node [style=none] (8) at (4, 2.75) {};
		\node [style=none] (9) at (4, 0.25) {};
		\node [style=oplus] (10) at (4, 2) {};
		\node [style=dot] (11) at (4.5, 2) {};
	\end{pgfonlayer}
	\begin{pgfonlayer}{edgelayer}
		\draw (1) to (0);
		\draw (4.center) to (5.center);
		\draw (3) to (2);
		\draw (8.center) to (9.center);
		\draw (7) to (6);
		\draw (11) to (10);
	\end{pgfonlayer}
\end{tikzpicture}
=
\begin{tikzpicture}
	\begin{pgfonlayer}{nodelayer}
		\node [style=X] (0) at (4.5, 3.5) {};
		\node [style=X] (1) at (4.5, 0.5) {};
		\node [style=oplus] (2) at (4.5, 1) {};
		\node [style=dot] (3) at (3.5, 1) {};
		\node [style=none] (4) at (3.5, 3.75) {};
		\node [style=none] (5) at (3.5, 0.25) {};
		\node [style=oplus] (6) at (4.5, 2.5) {};
		\node [style=dot] (7) at (4, 2.5) {};
		\node [style=none] (8) at (4, 3.75) {};
		\node [style=none] (9) at (4, 0.25) {};
		\node [style=oplus] (10) at (4, 3) {};
		\node [style=dot] (11) at (4.5, 3) {};
		\node [style=oplus] (12) at (4, 2) {};
		\node [style=dot] (13) at (4.5, 2) {};
		\node [style=oplus] (14) at (4, 1.5) {};
		\node [style=dot] (15) at (4.5, 1.5) {};
	\end{pgfonlayer}
	\begin{pgfonlayer}{edgelayer}
		\draw (1) to (0);
		\draw (4.center) to (5.center);
		\draw (3) to (2);
		\draw (8.center) to (9.center);
		\draw (7) to (6);
		\draw (11) to (10);
		\draw (13) to (12);
		\draw (15) to (14);
	\end{pgfonlayer}
\end{tikzpicture}
=
\begin{tikzpicture}
	\begin{pgfonlayer}{nodelayer}
		\node [style=X] (0) at (4.5, 2.5) {};
		\node [style=X] (1) at (4.5, 0.5) {};
		\node [style=oplus] (2) at (4.5, 1) {};
		\node [style=dot] (3) at (3.5, 1) {};
		\node [style=none] (4) at (3.5, 2.75) {};
		\node [style=none] (5) at (3.5, 0.25) {};
		\node [style=none] (8) at (4, 2.75) {};
		\node [style=none] (9) at (4, 0.25) {};
		\node [style=oplus] (14) at (4, 1.5) {};
		\node [style=dot] (15) at (4.5, 1.5) {};
		\node [style=none] (16) at (4, 2.5) {};
	\end{pgfonlayer}
	\begin{pgfonlayer}{edgelayer}
		\draw (4.center) to (5.center);
		\draw (3) to (2);
		\draw (15) to (14);
		\draw (9.center) to (14);
		\draw (1) to (15);
		\draw [in=-90, out=90] (15) to (16.center);
		\draw [in=-90, out=90] (14) to (0);
		\draw (8.center) to (16.center);
	\end{pgfonlayer}
\end{tikzpicture}
=
\begin{tikzpicture}
	\begin{pgfonlayer}{nodelayer}
		\node [style=X] (0) at (4, 2) {};
		\node [style=X] (1) at (4.5, 0.5) {};
		\node [style=oplus] (2) at (4.5, 1) {};
		\node [style=dot] (3) at (3.5, 1) {};
		\node [style=none] (4) at (3.5, 2.75) {};
		\node [style=none] (5) at (3.5, 0.25) {};
		\node [style=none] (8) at (4, 2.75) {};
		\node [style=none] (9) at (4, 0.25) {};
		\node [style=oplus] (14) at (4, 1.5) {};
		\node [style=dot] (15) at (4.5, 1.5) {};
		\node [style=none] (16) at (4.5, 2) {};
	\end{pgfonlayer}
	\begin{pgfonlayer}{edgelayer}
		\draw (4.center) to (5.center);
		\draw (3) to (2);
		\draw (15) to (14);
		\draw (9.center) to (14);
		\draw (1) to (15);
		\draw [in=-90, out=90] (14) to (0);
		\draw (15) to (16.center);
		\draw [in=-90, out=90, looseness=0.75] (16.center) to (8.center);
	\end{pgfonlayer}
\end{tikzpicture}
=
\begin{tikzpicture}
	\begin{pgfonlayer}{nodelayer}
		\node [style=X] (17) at (6, 2.5) {};
		\node [style=X] (18) at (6.5, 1) {};
		\node [style=oplus] (19) at (6.5, 2) {};
		\node [style=dot] (20) at (5.5, 2) {};
		\node [style=none] (21) at (5.5, 3.25) {};
		\node [style=none] (22) at (5.5, 0.25) {};
		\node [style=none] (23) at (6, 3.25) {};
		\node [style=none] (24) at (6, 0.25) {};
		\node [style=oplus] (25) at (6, 1.5) {};
		\node [style=dot] (26) at (6.5, 1.5) {};
		\node [style=none] (27) at (6.5, 2.5) {};
		\node [style=oplus] (28) at (6, 1) {};
		\node [style=dot] (29) at (5.5, 1) {};
	\end{pgfonlayer}
	\begin{pgfonlayer}{edgelayer}
		\draw (21.center) to (22.center);
		\draw (20) to (19);
		\draw (26) to (25);
		\draw (24.center) to (25);
		\draw (18) to (26);
		\draw [in=-90, out=90] (25) to (17);
		\draw (26) to (27.center);
		\draw [in=-90, out=90, looseness=0.75] (27.center) to (23.center);
		\draw (29) to (28);
	\end{pgfonlayer}
\end{tikzpicture}
=
\begin{tikzpicture}
	\begin{pgfonlayer}{nodelayer}
		\node [style=X] (17) at (6, 2) {};
		\node [style=X] (18) at (6.5, 1) {};
		\node [style=oplus] (19) at (6.5, 1.5) {};
		\node [style=dot] (20) at (5.5, 1.5) {};
		\node [style=none] (21) at (5.5, 2.75) {};
		\node [style=none] (22) at (5.5, 0.25) {};
		\node [style=none] (23) at (6, 2.75) {};
		\node [style=none] (24) at (6, 0.25) {};
		\node [style=none] (27) at (6.5, 2) {};
		\node [style=oplus] (28) at (6, 1) {};
		\node [style=dot] (29) at (5.5, 1) {};
	\end{pgfonlayer}
	\begin{pgfonlayer}{edgelayer}
		\draw (21.center) to (22.center);
		\draw (20) to (19);
		\draw [in=-90, out=90, looseness=0.75] (27.center) to (23.center);
		\draw (29) to (28);
		\draw (24.center) to (28);
		\draw (28) to (17);
		\draw (27.center) to (18);
	\end{pgfonlayer}
\end{tikzpicture}
=
\begin{tikzpicture}
	\begin{pgfonlayer}{nodelayer}
		\node [style=X] (17) at (6, 1.5) {};
		\node [style=X] (18) at (6, 2) {};
		\node [style=oplus] (19) at (6, 2.5) {};
		\node [style=dot] (20) at (5.5, 2.5) {};
		\node [style=none] (21) at (5.5, 3) {};
		\node [style=none] (22) at (5.5, 0.5) {};
		\node [style=none] (24) at (6, 0.5) {};
		\node [style=none] (27) at (6, 3) {};
		\node [style=oplus] (28) at (6, 1) {};
		\node [style=dot] (29) at (5.5, 1) {};
	\end{pgfonlayer}
	\begin{pgfonlayer}{edgelayer}
		\draw (21.center) to (22.center);
		\draw (20) to (19);
		\draw (29) to (28);
		\draw (24.center) to (28);
		\draw (28) to (17);
		\draw (27.center) to (18);
	\end{pgfonlayer}
\end{tikzpicture}
$$


The inductive case is essentially the same as the base case.

%
%Note that the distributive law is actually a partially reversible formulation of the bialgebra law in disguise since:
%
%
%$$
%\left\llbracket
%\begin{tikzpicture}
%	\begin{pgfonlayer}{nodelayer}
%		\node [style=X] (0) at (1, -0.25) {};
%		\node [style=oplus] (1) at (1, -0.75) {};
%		\node [style=oplus] (2) at (1, -1.5) {};
%		\node [style=dot] (3) at (0.5, -0.75) {};
%		\node [style=dot] (4) at (0, -1.5) {};
%		\node [style=none] (5) at (0.5, 0) {};
%		\node [style=none] (6) at (0, 0) {};
%		\node [style=none] (7) at (0.25, -1) {$\iddots$};
%		\node [style=none] (8) at (1, -1) {$\vdots$};
%		\node [style=X] (9) at (1, -2) {};
%		\node [style=none] (18) at (0, -2.25) {};
%		\node [style=none] (19) at (0.5, -2.25) {};
%		\node [style=none] (22) at (0.25, -0.25) {$\cdots$};
%	\end{pgfonlayer}
%	\begin{pgfonlayer}{edgelayer}
%		\draw (0) to (1);
%		\draw (1) to (3);
%		\draw (5.center) to (3);
%		\draw (6.center) to (4);
%		\draw (4) to (2);
%		\draw (4) to (18.center);
%		\draw (19.center) to (3);
%		\draw (9) to (2);
%	\end{pgfonlayer}
%\end{tikzpicture}
%\right\rrbracket
%=
%\begin{tikzpicture}
%	\begin{pgfonlayer}{nodelayer}
%		\node [style=none] (0) at (9.25, 0.5) {};
%		\node [style=none] (1) at (8.75, 0.5) {};
%		\node [style=none] (2) at (8.75, -1.75) {};
%		\node [style=none] (3) at (9.25, -1.75) {};
%		\node [style=Z] (4) at (9.25, 0) {};
%		\node [style=Z] (5) at (8.75, -0.75) {};
%		\node [style=none] (6) at (9.5, -1.25) {};
%		\node [style=none] (7) at (10.5, -1.25) {};
%		\node [style=none] (8) at (10.5, 0.25) {};
%		\node [style=none] (9) at (9.5, 0.25) {};
%		\node [style=none] (10) at (10.25, -1) {$!$};
%		\node [style=X] (11) at (10, 0) {};
%		\node [style=X] (12) at (10, -0.75) {};
%	\end{pgfonlayer}
%	\begin{pgfonlayer}{edgelayer}
%		\draw (1.center) to (5);
%		\draw (5) to (2.center);
%		\draw (3.center) to (4);
%		\draw (4) to (0.center);
%		\draw [style=dotted] (6.center) to (7.center);
%		\draw [style=dotted] (7.center) to (8.center);
%		\draw [style=dotted] (8.center) to (9.center);
%		\draw [style=dotted] (9.center) to (6.center);
%		\draw (5) to (12);
%		\draw (12) to (11);
%		\draw (11) to (4);
%	\end{pgfonlayer}
%\end{tikzpicture}
%=
%\begin{tikzpicture}
%	\begin{pgfonlayer}{nodelayer}
%		\node [style=none] (12) at (9.25, 1) {};
%		\node [style=none] (13) at (8.75, 1) {};
%		\node [style=none] (15) at (8.75, -1.75) {};
%		\node [style=none] (16) at (9.25, -1.75) {};
%		\node [style=Z] (18) at (8.75, -1.25) {};
%		\node [style=none] (19) at (9.5, -1.25) {};
%		\node [style=none] (20) at (10.5, -1.25) {};
%		\node [style=none] (21) at (10.5, 0.75) {};
%		\node [style=none] (22) at (9.5, 0.75) {};
%		\node [style=none] (23) at (10.25, -1) {$!$};
%		\node [style=X] (25) at (10, -0.75) {};
%		\node [style=Z] (28) at (10, 0.25) {};
%		\node [style=X] (29) at (9.25, 0.5) {};
%		\node [style=X] (30) at (9.25, -0.25) {};
%	\end{pgfonlayer}
%	\begin{pgfonlayer}{edgelayer}
%		\draw (13.center) to (18);
%		\draw (18) to (15.center);
%		\draw [style=dotted] (19.center) to (20.center);
%		\draw [style=dotted] (20.center) to (21.center);
%		\draw [style=dotted] (21.center) to (22.center);
%		\draw [style=dotted] (22.center) to (19.center);
%		\draw (18) to (25);
%		\draw (12.center) to (29);
%		\draw (29) to (28);
%		\draw (28) to (30);
%		\draw (25) to (28);
%		\draw (16.center) to (30);
%	\end{pgfonlayer}
%\end{tikzpicture}
%=
%\begin{tikzpicture}
%	\begin{pgfonlayer}{nodelayer}
%		\node [style=none] (0) at (9.25, 1.25) {};
%		\node [style=none] (1) at (8.75, 1.25) {};
%		\node [style=none] (2) at (8.75, -1.75) {};
%		\node [style=none] (3) at (9.25, -1.75) {};
%		\node [style=none] (5) at (9.5, -1.25) {};
%		\node [style=none] (6) at (10.5, -1.25) {};
%		\node [style=none] (7) at (10.5, 1) {};
%		\node [style=none] (8) at (9.5, 1) {};
%		\node [style=none] (9) at (10.25, -1) {$!$};
%		\node [style=Z] (11) at (10, 0.5) {};
%		\node [style=X] (12) at (9.25, 0.75) {};
%		\node [style=X] (13) at (9.25, 0) {};
%		\node [style=Z] (14) at (10, -0.75) {};
%		\node [style=X] (15) at (8.75, -0.25) {};
%		\node [style=X] (16) at (8.75, -1.25) {};
%	\end{pgfonlayer}
%	\begin{pgfonlayer}{edgelayer}
%		\draw [style=dotted] (5.center) to (6.center);
%		\draw [style=dotted] (6.center) to (7.center);
%		\draw [style=dotted] (7.center) to (8.center);
%		\draw [style=dotted] (8.center) to (5.center);
%		\draw (0.center) to (12);
%		\draw (12) to (11);
%		\draw (11) to (13);
%		\draw (3.center) to (13);
%		\draw (16) to (14);
%		\draw (14) to (11);
%		\draw (14) to (15);
%		\draw (15) to (1.center);
%		\draw (16) to (2.center);
%	\end{pgfonlayer}
%\end{tikzpicture}
%=
%\begin{tikzpicture}
%	\begin{pgfonlayer}{nodelayer}
%		\node [style=none] (0) at (9.25, 0.75) {};
%		\node [style=none] (1) at (8.75, 0.75) {};
%		\node [style=none] (2) at (8.75, -2) {};
%		\node [style=none] (3) at (9.25, -2) {};
%		\node [style=none] (4) at (9.5, -1.75) {};
%		\node [style=none] (5) at (10.5, -1.75) {};
%		\node [style=none] (6) at (10.5, 0.5) {};
%		\node [style=none] (7) at (9.5, 0.5) {};
%		\node [style=none] (8) at (10.25, -1.5) {$!$};
%		\node [style=Z] (9) at (10, 0.25) {};
%		\node [style=X] (10) at (9.25, 0.25) {};
%		\node [style=X] (11) at (9.25, -0.25) {};
%		\node [style=Z] (12) at (10, -0.75) {};
%		\node [style=X] (13) at (8.75, -0.75) {};
%		\node [style=X] (14) at (8.75, -1.25) {};
%		\node [style=Z] (15) at (10, -0.25) {};
%		\node [style=Z] (16) at (10, -1.25) {};
%	\end{pgfonlayer}
%	\begin{pgfonlayer}{edgelayer}
%		\draw [style=dotted] (4.center) to (5.center);
%		\draw [style=dotted] (5.center) to (6.center);
%		\draw [style=dotted] (6.center) to (7.center);
%		\draw [style=dotted] (7.center) to (4.center);
%		\draw (0.center) to (10);
%		\draw (10) to (9);
%		\draw (3.center) to (11);
%		\draw (13) to (12);
%		\draw (13) to (1.center);
%		\draw (14) to (2.center);
%		\draw (9) to (12);
%		\draw (16) to (14);
%		\draw (11) to (15);
%		\draw (16) to (12);
%	\end{pgfonlayer}
%\end{tikzpicture}
%=
%\left\llbracket
%\begin{tikzpicture}
%	\begin{pgfonlayer}{nodelayer}
%		\node [style=none] (0) at (7, 2.25) {};
%		\node [style=none] (1) at (6.5, 2.25) {};
%		\node [style=none] (2) at (6.75, -.2) {$\iddots$};
%		\node [style=none] (3) at (6.5, -3) {};
%		\node [style=none] (4) at (7, -3) {};
%		\node [style=oplus] (6) at (6.5, -2.25) {};
%		\node [style=X] (7) at (6.5, -1.75) {};
%		\node [style=X] (8) at (6.5, -1.25) {};
%		\node [style=dot] (9) at (7.5, -2.25) {};
%		\node [style=dot] (10) at (7.5, -0.75) {};
%		\node [style=Z] (12) at (7.5, -2.75) {};
%		\node [style=oplus] (13) at (7, 0) {};
%		\node [style=X] (14) at (7, 0.5) {};
%		\node [style=X] (15) at (7, 1) {};
%		\node [style=dot] (16) at (7.5, 0) {};
%		\node [style=dot] (17) at (7.5, 1.5) {};
%		\node [style=none] (18) at (7.5, -0.25) {$\vdots$};
%		\node [style=Z] (19) at (7.5, 2) {};
%		\node [style=oplus] (20) at (7, 1.5) {};
%		\node [style=oplus] (21) at (6.5, -0.75) {};
%	\end{pgfonlayer}
%	\begin{pgfonlayer}{edgelayer}
%		\draw (3.center) to (6);
%		\draw (6) to (7);
%		\draw (6) to (9);
%		\draw (12) to (9);
%		\draw (9) to (10);
%		\draw (13) to (14);
%		\draw (13) to (16);
%		\draw (16) to (17);
%		\draw (4.center) to (13);
%		\draw (19) to (17);
%		\draw (17) to (20);
%		\draw (20) to (15);
%		\draw (0.center) to (20);
%		\draw (10) to (21);
%		\draw (21) to (8);
%		\draw (21) to (1.center);
%	\end{pgfonlayer}
%\end{tikzpicture}
%\right\rrbracket
%=
%\left\llbracket
%\begin{tikzpicture}
%	\begin{pgfonlayer}{nodelayer}
%		\node [style=none] (18) at (7.5, 1.25) {};
%		\node [style=none] (19) at (6.5, 0.5) {};
%		\node [style=none] (20) at (6.7, -0.25) {$\iddots$};
%		\node [style=none] (23) at (7, -2) {};
%		\node [style=none] (24) at (7, -1.25) {};
%		\node [style=X] (30) at (6.5, -1.25) {};
%		\node [style=dot] (32) at (7.5, -0.75) {};
%		\node [style=oplus] (35) at (7, 0) {};
%		\node [style=X] (36) at (7, 0.5) {};
%		\node [style=dot] (38) at (7.5, 0) {};
%		\node [style=none] (41) at (7.5, -0.25) {$\vdots$};
%		\node [style=none] (42) at (7.5, -1.25) {};
%		\node [style=oplus] (43) at (6.5, -0.75) {};
%		\node [style=none] (44) at (7.5, -2) {};
%		\node [style=none] (45) at (7, 1.25) {};
%	\end{pgfonlayer}
%	\begin{pgfonlayer}{edgelayer}
%		\draw (35) to (36);
%		\draw (35) to (38);
%		\draw [in=270, out=90] (24.center) to (35);
%		\draw (38) to (18.center);
%		\draw [in=-90, out=90, looseness=0.75] (23.center) to (42.center);
%		\draw (42.center) to (32);
%		\draw (32) to (43);
%		\draw (43) to (19.center);
%		\draw (43) to (30);
%		\draw [in=-90, out=90] (44.center) to (24.center);
%		\draw [in=-90, out=90] (19.center) to (45.center);
%	\end{pgfonlayer}
%\end{tikzpicture}
%\right\rrbracket
%=
%\left\llbracket
%\begin{tikzpicture}
%	\begin{pgfonlayer}{nodelayer}
%		\node [style=none] (15) at (9.5, 1.5) {};
%		\node [style=none] (19) at (10, -1) {};
%		\node [style=X] (20) at (9, 0.5) {};
%		\node [style=dot] (21) at (9.5, 1) {};
%		\node [style=oplus] (22) at (10, -0.5) {};
%		\node [style=X] (23) at (10, 0) {};
%		\node [style=dot] (24) at (9.5, -0.5) {};
%		\node [style=none] (26) at (9.5, -1) {};
%		\node [style=oplus] (27) at (9, 1) {};
%		\node [style=none] (29) at (9, 1.5) {};
%		\node [style=none] (30) at (9.5, 0.25) {$\ddots$};
%	\end{pgfonlayer}
%	\begin{pgfonlayer}{edgelayer}
%		\draw (22) to (23);
%		\draw (22) to (24);
%		\draw [in=270, out=90] (19.center) to (22);
%		\draw (21) to (27);
%		\draw (27) to (20);
%		\draw (21) to (15.center);
%		\draw (24) to (26.center);
%		\draw (29.center) to (27);
%		\draw (24) to (21);
%	\end{pgfonlayer}
%\end{tikzpicture}
%\right\rrbracket
%$$
%\end{definition}

%Notice that the choice of which wires to straighten out the zig-zag is arbitrary.

\end{remark}


\begin{lemma}
\label{lem:parisocb}
$\Par\Iso(\cb_2)$ is a presentation for the prop $(\Par\Iso(\Mat(\F_2),+))$.
\end{lemma}
%This follows from \cite[??]{ih}.


We can get partial maps by freely adding a counit to the nonunital, noncounital special commutative Frobenius algebra:

\begin{definition}

Let $\Par(\cb_2)$ denote the pushout of the diagram of props:
$$
\Par\Iso(\cb_2)  \leftarrow  \surj^\op \rightarrow   \cm^\op
$$

\end{definition}



%The following Lemma follows after meticulous calculation and the application of \cite[Lem. 3.5]{zxa}:

\begin{lemma}
\label{lem:parcb}

$\Par(\cb_2)$ is a presentation for the prop $(\Par(\Mat(\F_2),+))$.
\end{lemma}



\begin{proof}
One must show that the following diagram commutes:

%
%\renewcommand{\cubetopbl}{$\inj(\cb_2)$}
%\renewcommand{\cubetopbr}{$\inj(\cb_2)^\op \otimes_{\Iso(\cb_2)} \inj(\cb_2)$}
%\renewcommand{\cubetopfl}{$\inj(\cb_2)\otimes_{\Iso(\cb_2)} \surj(\cb_2)^\op$}
%\renewcommand{\cubetopfr}{$\Par(\cb_2)$}\ParIso(\cb_2)
%\renewcommand{\cubebotbl}{$(\inj(\Mat(\F_2)), +)$}
%\renewcommand{\cubebotbr}{$(\Par\Iso(\Mat(\F_2)), +)$}
%\renewcommand{\cubebotfl}{$(\Par\surj(\Mat(\F_2)), +)$}
%\renewcommand{\cubebotfr}{}
%
%$$
%\xymatrixrowsep{3mm}\xymatrixcolsep{1mm}
%\xymatrix{
%                                       & \mbox{\cubetopbl} \ar[rr] \ar[dl] \ar[dd]^(.7){\cong}      &                                                  & \mbox{\cubetopbr}  \ar[dd]^{\cong} \ar[dl] \\
%\mbox{\cubetopfl} \ar[rr]  \ar[dd]_{\cong}           &                                                                                              &\mbox{\cubetopfr} \ar@{-->}[dd]^(.35){\cong}   \skewpullbackcorner[ul]              \\
%                                       &  \mbox{\cubebotbl} \ar[dl] \ar[rr]                    &                                                  & \mbox{\cubebotbr} \ar@/^1pc/[ddl] \ar[dl] \\
%\mbox{\cubebotfl} \ar@/_1pc/[drr] \ar[rr]  &                                                                                             & \mbox{\cubebotfr} \skewpullbackcorner[ul]    \ar@{-->}[d]^{\cong}  \\
%                                                   &                                                                                             & (\Par(\Mat(\F_2)),+)
%}
%$$
%

\renewcommand{\cubetopbl}{$\surj^\op$}
\renewcommand{\cubetopbr}{$\cm^\op$}
\renewcommand{\cubetopfl}{$\ParIso(\cb_2)$}
\renewcommand{\cubetopfr}{$\Par(\cb_2)$}
\renewcommand{\cubebotbl}{$\surj^\op$ }
\renewcommand{\cubebotbr}{$\cm^\op$ }
\renewcommand{\cubebotfl}{$\ParIso(\Mat(\F_2)),+)$ }
\renewcommand{\cubebotfr}{}

$$
\xymatrixrowsep{2mm}\xymatrixcolsep{2mm}
\xymatrix{
                                       & \mbox{\cubetopbl} \ar[rr] \ar[dl] \ar@{=}[dd]     &                                                  & \mbox{\cubetopbr} \ar@{=}[dd] \ar[dl] \\
\mbox{\cubetopfl} \ar[rr]  \ar[dd]_{\cong}           &                                                                                              &\mbox{\cubetopfr} \ar@{-->}[dd]^(.35){\cong}   \skewpullbackcorner[ul]              \\
                                       &  \mbox{\cubebotbl} \ar[dl] \ar[rr]                    &                                                  & \mbox{\cubebotbr} \ar@/^1pc/[ddl] \ar[dl] \\
\mbox{\cubebotfl} \ar@/_1pc/[drr] \ar[rr]  &                                                                                             & \mbox{\cubebotfr} \skewpullbackcorner[ul]    \ar@{-->}[d]^{\cong}  \\
                                                   &                                                                                             & (\Par(\Mat(\F_2)),+)
}
$$

It doesn't take too much work to show that $\ParIso(\cb_2)\cong\ParIso(\Mat(\F_2))$ is a discrete inverse category (defined in \cite[\S 4.3]{giles}).
We know that the counital completion of a discrete inverse category is the same as its Cartesian completion from \cite[Lem. 3.5]{zxa}; moreover, the Cartesian completion of  $\ParIso(\Mat(\F_2))$ is $\Par(\Mat(\F_2))$.  So this diagram commutes as a consequence.

\end{proof}



%\begin{comment}
This props has a particularly elegant presentation which is given in \S \ref{subsubsec:presentations:one:par}.
%\end{comment}



%
%\begin{corollary}
%Equivalently, $(\Par(\Mat(\F_2),+))$ is presented by the coproduct
%$ T_{\eta_X} + \cb_2 $
%modulo:
%
%?????
%
%\end{corollary}


%
%For $R$ a PID, $(\Span(\Mat(R)),+)$ TODO
%
%
%
%
%For $R$ a PID, $(\Rel(\Mat(R)),+)$ TODO
%
%
%  Kernel, Image, orthogonal complement
%  Pushout of TODO
%
%
%Because are self-dual with respect to the transpose functor, we omit the discussion of cospans and corelations.
%


%I will omit the discussion of spans and relations; however, the details are contained in \cite{ih}.




\begin{definition}
Let $\Span(\cb_2)$ denote the pushout of the diagram of props:
$$
\Par(\cb_2)^\op \leftarrow  \ParIso(\cb_2) \rightarrow \Par(\cb_2)
$$
\end{definition}

The following lemma holds because of \cite[Lem. 4.3]{zxa}:


\begin{lemma}
\label{lem:spancb}

$\Span(\cb_2)$ is a presentation for the prop $(\Span^\sim(\Mat(\F_2)), +)$.
\end{lemma}


\begin{proof}\


\renewcommand{\cubetopbl}{$\inj(\cb_2)^\op \otimes_{\Iso(\cb_2)} \inj(\cb_2)$}
\renewcommand{\cubetopbr}{$\Par(\cb_2)$}
\renewcommand{\cubetopfl}{$\Par(\cb_2)^\op$}
\renewcommand{\cubetopfr}{$\Span(\cb_2)$}
\renewcommand{\cubebotbl}{$(\Par\Iso(\Mat(\F_2)),+)$ }
\renewcommand{\cubebotbr}{$(\Par(\Mat(\F_2)),+)$ }
\renewcommand{\cubebotfl}{$(\Par(\Mat(\F_2)),+)^\op$ }
\renewcommand{\cubebotfr}{}

\scalebox{.9}{$
\xymatrixrowsep{2mm}\xymatrixcolsep{1mm}
\xymatrix{
                                       & \mbox{\cubetopbl} \ar[rr] \ar[dl] \ar[dd]^(.7){\cong}      &                                                  & \mbox{\cubetopbr}  \ar[dd]^{\cong} \ar[dl] \\
\mbox{\cubetopfl} \ar[rr]  \ar[dd]_{\cong}           &                                                                                              &\mbox{\cubetopfr} \ar@{-->}[dd]^(.35){\cong}   \skewpullbackcorner[ul]              \\
                                       &  \mbox{\cubebotbl} \ar[dl] \ar[rr]                    &                                                  & \mbox{\cubebotbr} \ar@/^1pc/[ddl] \ar[dl] \\
\mbox{\cubebotfl} \ar@/_1pc/[drr] \ar[rr]  &                                                                                             & \mbox{\cubebotfr} \skewpullbackcorner[ul]    \ar@{-->}[d]^{\cong}_F \\
                                                   &                                                                                             & (\Span^\sim(\Mat(\F_2)),+)
}
$}


The cube easily commutes.  What remains to be shown is that the universal map $F$ is an isomorphism of props.  It is clearly the identity on objects, so we just need to show it is full and faithful.

It is clearly full as any span $ n \xleftarrow{ f}  k \xrightarrow{g } m$, we have:
$$
F\left( (n \xleftarrow{f} k = k);(k = k \xrightarrow{g} m) \right)=n \xleftarrow{ f}  k \xrightarrow{g } m
$$ 
For faithfulness, we must observe given for any two isomorphic maps in $\Span(\Mat(\F_2))$:
$$
\xymatrixrowsep{2mm}\xymatrixcolsep{6mm}
\xymatrix{
          & k \ar[dl]_{f'} \ar[dd]_{\cong}^{h} \ar[dr]^{g'}\\
n  &                                                                                                    & m\\
         & k \ar[ul]^{f} \ar[ur]_{g}\\
}
$$
Then in the domain of $F$, we have:
{
\xymatrixrowsep{0mm}\xymatrixcolsep{1.7mm}
\begin{align*}
&
\xymatrix{
   & k \ar[dl]_f \ar@{=}[dr]\\
n &                                      &k
};
\xymatrix{
   & k \ar[dr]^g \ar@{=}[dl]\\
k &                                      &m
}
%&
 =
\xymatrix{
   & k \ar[dl]_f \ar@{=}[dr]\\
n &                                      &k
};
\xymatrix{
   & k \ar@{=}[dl] \ar@{=}[dr]\\
k &                                             & k\\
   & k \ar[ul]^h \ar[ur]_h \ar[uu]^\cong_h
};
\xymatrix{
   & k \ar[dr]^g \ar@{=}[dl]\\
k &                                      &m
}\\
 &=
\xymatrix{
   & k \ar[dl]_f \ar@{=}[dr]\\
n &                                      &k
};
\xymatrix{
   & k \ar[dl]_h \ar@{=}[dr]\\
k &                                         & k
};
\xymatrix{
   & k \ar[dr]^h \ar@{=}[dl]\\
k &                                         & k
};
\xymatrix{
   & k \ar[dr]^g \ar@{=}[dl]\\
k &                                      &m
}
%\\&
=
\xymatrix{
            &                                                        &k \ar[dl]_{h} \ar@{=}[dr] \ar@/_1.2pc/[ddll]_{f'}\\
            & k \ar@{=}[dr] \ar[dl]^{f}&                                                          & k \ar@{=}[dr] \ar[dl]_{h}\\
n &                                                         & k                                             &                                                         &k
};
\xymatrix{
            &                                                        & k \ar[dr]^{h} \ar@{=}[dl] \ar@/^1.2pc/[ddrr]^{g'}  \\
            & k \ar[dr]^{h}   \ar@{=}[dl] &                                                          & k \ar@{=}[dl] \ar[dr]_{g}\\
k &                                                         & k                                             &                                                         &m
}
\end{align*}
}
 
\end{proof}


Given a PID $k$, the prop $(\Span^\sim(\Mat(k)), +)$ is already known to have a presentation given in terms of ``interacting Hopf algebras" \cite[Definition 3.13]{ih}.  This is also the way in which the phase-free fragment of the ZX-calculus would be presented, in terms of two Frobenius algebras  corresponding to the $Z$ and $X$ observables, interacting to form Hopf algebras in addition to a few more equations.
%\begin{comment}
 We have included this presentation in \S \ref{subsubsec:presentations:one:span}.
%\end{comment}



%
%\begin{corollary}
%Because  prop $(\Mat(\F_2),+)$ is self dual, the prop $\Csp(\cb_2)$ obtained by swapping colours is a presentation for the prop $(\Csp(\Mat(\F_2)),+)$
%\end{corollary}
%
%
%
%
%\begin{definition} 
%
%Consider the prop $\Rel(\cb_2)$ given by the following pushout of the following diagram of props:
%
%$$
%\Csp(\cb_2) \leftarrow \inj(\cb_2)^\op + \inj(\cb_2)  \rightarrow \ParIso(\cb_2)
%$$
%
%\end{definition}
%
%This particular construction of the following presentation is due to \cite[Thm. 3.6]{universal}; however, it was constructed in a slightly different manner before in \cite[Thm. 3.49]{ih}:
%
%\begin{lemma}
%$\Rel(\cb_2)$ is a presentation for the prop $(\Rel(\Mat(\F_2)), +)$
%\end{lemma}
%
%
%
%\begin{proof}
%
%
%
%\renewcommand{\cubetopbl}{$\inj(\cb_2)^\op + \inj(\cb_2)$}
%\renewcommand{\cubetopbr}{$\inj(\cb_2)\otimes_{\Iso(\cb_2)} \inj(\cb_2)^\op$}
%\renewcommand{\cubetopfl}{$\cb_2^\op \otimes_{\Iso(\cb_2)}  \cb_2 $}
%\renewcommand{\cubetopfr}{$\Rel(\cb_2)$}
%\renewcommand{\cubebotbl}{$(\inj(\Mat(\F_2)),+)^\op+(\inj(\Mat(\F_2)),+)$ }
%\renewcommand{\cubebotbr}{$(\Par\Iso(\Mat(\F_2)),+)$ }
%\renewcommand{\cubebotfl}{$(\Csp(\Mat(\F_2)),+)$ }
%\renewcommand{\cubebotfr}{$$}
%
%$$
%\xymatrixrowsep{3mm}\xymatrixcolsep{1mm}
%\xymatrix{
%                                       & \mbox{\cubetopbl} \ar[rr] \ar[dl] \ar[dd]^(.7){\cong}      &                                                  & \mbox{\cubetopbr}  \ar[dd]^{\cong} \ar[dl] \\
%\mbox{\cubetopfl} \ar[rr]  \ar[dd]_{\cong}           &                                                                                              &\mbox{\cubetopfr} \ar@{-->}[dd]^(.35){\cong}   \skewpullbackcorner[ul]              \\
%                                       &  \mbox{\cubebotbl} \ar[dl] \ar[rr]                    &                                                  & \mbox{\cubebotbr} \ar@/^1pc/[ddl] \ar[dl] \\
%\mbox{\cubebotfl} \ar@/_1pc/[drr] \ar[rr]  &                                                                                             & \mbox{\cubebotfr} \skewpullbackcorner[ul]    \ar@{-->}[d]^{\cong}  \\
%                                                   &                                                                                             & (\Rel(\Mat(\F_2)),+)
%}
%$$
%
%
%\end{proof}
%
%


\subsection{Adding the $\cal X$ gates}
\label{sec:two}


\begin{definition}
Consider the prop  $\Aff\cb_2$ given by adjoining the following generator to $\cb_2$
\hfil
$
\begin{tikzpicture}
	\begin{pgfonlayer}{nodelayer}
		\node [style=none] (0) at (-3.75, -0.25) {};
		\node [style=X] (1) at (-3.75, -1) {$1$};
	\end{pgfonlayer}
	\begin{pgfonlayer}{edgelayer}
		\draw (0.center) to (1);
	\end{pgfonlayer}
\end{tikzpicture}
$

modulo the equations:

$$
\begin{tikzpicture}
	\begin{pgfonlayer}{nodelayer}
		\node [style=X] (0) at (0.75, 0.25) {$1$};
		\node [style=Z] (1) at (0.75, 0.75) {};
		\node [style=none] (2) at (0.5, 1.5) {};
		\node [style=none] (3) at (1, 1.5) {};
	\end{pgfonlayer}
	\begin{pgfonlayer}{edgelayer}
		\draw (1) to (0);
		\draw [in=-90, out=60] (1) to (3.center);
		\draw [in=120, out=-90] (2.center) to (1);
	\end{pgfonlayer}
\end{tikzpicture}
\erefop{bi.two}
\begin{tikzpicture}
	\begin{pgfonlayer}{nodelayer}
		\node [style=X] (0) at (0.5, 0.5) {$1$};
		\node [style=none] (1) at (0.5, 1.75) {};
		\node [style=none] (2) at (1, 1.75) {};
		\node [style=X] (3) at (1, 0.5) {$1$};
	\end{pgfonlayer}
	\begin{pgfonlayer}{edgelayer}
		\draw (0) to (1.center);
		\draw (2.center) to (3);
	\end{pgfonlayer}
\end{tikzpicture}
\hspace*{1cm}
\begin{tikzpicture}
	\begin{pgfonlayer}{nodelayer}
		\node [style=X] (0) at (0, 0) {$1$};
		\node [style=Z] (1) at (0, 0.75) {};
	\end{pgfonlayer}
	\begin{pgfonlayer}{edgelayer}
		\draw (1) to (0);
	\end{pgfonlayer}
\end{tikzpicture}
\eref{extra}
$$



\end{definition}



\begin{lemma} \cite[\S 4]{lafont}
 $\Aff\cb_2$ is a presentation for the prop $(\Aff\Mat(\F_2),+)$.
\end{lemma}


Note that this assumes that affine matrices are non-empty, as this is a prop.  This will become a problem later, when we wish to pull back affine spaces.






\begin{definition}
Consider the prop $\Iso(\Aff\cb_2)$ generated by the controlled not gate, and the not gate, modulo the relations of $\Iso(\cb_2)$ as well as the additional relations:
$$
\begin{tikzpicture}
	\begin{pgfonlayer}{nodelayer}
		\node [style=oplus] (0) at (0, 0) {};
		\node [style=oplus] (1) at (0, 0.5) {};
		\node [style=none] (2) at (0, 1) {};
		\node [style=none] (3) at (0, -0.5) {};
	\end{pgfonlayer}
	\begin{pgfonlayer}{edgelayer}
		\draw (2.center) to (3.center);
	\end{pgfonlayer}
\end{tikzpicture}
\eqzxa{cnot.seven}
\begin{tikzpicture}
	\begin{pgfonlayer}{nodelayer}
		\node [style=none] (2) at (0, 1) {};
		\node [style=none] (3) at (0, -0.5) {};
	\end{pgfonlayer}
	\begin{pgfonlayer}{edgelayer}
		\draw (2.center) to (3.center);
	\end{pgfonlayer}
\end{tikzpicture}
\hspace*{.5cm}
\begin{tikzpicture}
	\begin{pgfonlayer}{nodelayer}
		\node [style=dot] (0) at (2, 2) {};
		\node [style=oplus] (1) at (2.5, 2) {};
		\node [style=oplus] (2) at (2, 1.5) {};
		\node [style=none] (3) at (2, 1) {};
		\node [style=none] (4) at (2.5, 1) {};
		\node [style=none] (5) at (2, 2.5) {};
		\node [style=none] (6) at (2.5, 2.5) {};
	\end{pgfonlayer}
	\begin{pgfonlayer}{edgelayer}
		\draw (3.center) to (2);
		\draw (2) to (0);
		\draw (0) to (5.center);
		\draw (6.center) to (1);
		\draw (1) to (0);
		\draw (4.center) to (1);
	\end{pgfonlayer}
\end{tikzpicture}
\eqzxa{cnot.eight}
\begin{tikzpicture}
	\begin{pgfonlayer}{nodelayer}
		\node [style=dot] (0) at (2, 1.5) {};
		\node [style=oplus] (1) at (2.5, 1.5) {};
		\node [style=none] (3) at (2, 1) {};
		\node [style=none] (4) at (2.5, 1) {};
		\node [style=none] (5) at (2, 2.5) {};
		\node [style=none] (6) at (2.5, 2.5) {};
		\node [style=oplus] (7) at (2, 2) {};
		\node [style=oplus] (8) at (2.5, 2) {};
	\end{pgfonlayer}
	\begin{pgfonlayer}{edgelayer}
		\draw (0) to (5.center);
		\draw (6.center) to (1);
		\draw (1) to (0);
		\draw (4.center) to (1);
		\draw (3.center) to (0);
	\end{pgfonlayer}
\end{tikzpicture}
\hspace*{.5cm}
\begin{tikzpicture}
	\begin{pgfonlayer}{nodelayer}
		\node [style=dot] (0) at (2, 2) {};
		\node [style=oplus] (1) at (2.5, 2) {};
		\node [style=none] (3) at (2, 1) {};
		\node [style=none] (4) at (2.5, 1) {};
		\node [style=none] (5) at (2, 2.5) {};
		\node [style=none] (6) at (2.5, 2.5) {};
		\node [style=oplus] (8) at (2.5, 1.5) {};
	\end{pgfonlayer}
	\begin{pgfonlayer}{edgelayer}
		\draw (0) to (5.center);
		\draw (6.center) to (1);
		\draw (1) to (0);
		\draw (4.center) to (1);
		\draw (3.center) to (0);
	\end{pgfonlayer}
\end{tikzpicture}
\eqzxa{cnot.nine}
\begin{tikzpicture}
	\begin{pgfonlayer}{nodelayer}
		\node [style=dot] (0) at (2, 1.5) {};
		\node [style=oplus] (1) at (2.5, 1.5) {};
		\node [style=none] (3) at (2, 1) {};
		\node [style=none] (4) at (2.5, 1) {};
		\node [style=none] (5) at (2, 2.5) {};
		\node [style=none] (6) at (2.5, 2.5) {};
		\node [style=oplus] (8) at (2.5, 2) {};
	\end{pgfonlayer}
	\begin{pgfonlayer}{edgelayer}
		\draw (0) to (5.center);
		\draw (6.center) to (1);
		\draw (1) to (0);
		\draw (4.center) to (1);
		\draw (3.center) to (0);
	\end{pgfonlayer}
\end{tikzpicture}
$$

\end{definition}


\begin{lemma}\cite[Thm. 11]{lafont}
$\Iso(\Aff\cb_2)$ is a presentation for the prop $(\Iso(\Aff\Mat(\F_2),+))$ with respect to the interpretation:

$$
\left\llbracket
\begin{tikzpicture}
	\begin{pgfonlayer}{nodelayer}
		\node [style=oplus] (5) at (0.5, 2.75) {};
		\node [style=dot] (6) at (0, 2.75) {};
		\node [style=none] (7) at (0.5, 3.5) {};
		\node [style=none] (8) at (0.5, 2) {};
		\node [style=none] (9) at (0, 2) {};
		\node [style=none] (10) at (0, 3.5) {};
	\end{pgfonlayer}
	\begin{pgfonlayer}{edgelayer}
		\draw (8.center) to (5);
		\draw (5) to (7.center);
		\draw (10.center) to (6);
		\draw (6) to (5);
		\draw (6) to (9.center);
	\end{pgfonlayer}
\end{tikzpicture}
\right\rrbracket
=
\begin{tikzpicture}
	\begin{pgfonlayer}{nodelayer}
		\node [style=X] (0) at (-0.25, -1) {};
		\node [style=Z] (1) at (-0.75, -1.75) {};
		\node [style=none] (2) at (0, -2.25) {};
		\node [style=none] (3) at (-0.25, -0.5) {};
		\node [style=none] (4) at (-1, -0.5) {};
		\node [style=none] (5) at (-0.75, -2.25) {};
	\end{pgfonlayer}
	\begin{pgfonlayer}{edgelayer}
		\draw (3.center) to (0);
		\draw [in=90, out=-75] (0) to (2.center);
		\draw (5.center) to (1);
		\draw (1) to (0);
		\draw [in=-90, out=105] (1) to (4.center);
	\end{pgfonlayer}
\end{tikzpicture}
\hspace*{1cm}
\left\llbracket
\begin{tikzpicture}
	\begin{pgfonlayer}{nodelayer}
		\node [style=none] (0) at (0, -0.5) {};
		\node [style=none] (1) at (0, -1.5) {};
		\node [style=oplus] (2) at (0, -1) {};
	\end{pgfonlayer}
	\begin{pgfonlayer}{edgelayer}
		\draw (1.center) to (2);
		\draw (2) to (0.center);
	\end{pgfonlayer}
\end{tikzpicture}
\right\rrbracket
=
\begin{tikzpicture}
	\begin{pgfonlayer}{nodelayer}
		\node [style=none] (0) at (1, 1) {};
		\node [style=none] (1) at (0.75, 2) {};
		\node [style=X] (2) at (0.75, 1.5) {};
		\node [style=X] (3) at (0.5, 1) {$1$};
		\node [style=none] (4) at (1, 0.5) {};
	\end{pgfonlayer}
	\begin{pgfonlayer}{edgelayer}
		\draw (1.center) to (2);
		\draw [in=90, out=-45, looseness=0.75] (2) to (0.center);
		\draw [in=90, out=-135, looseness=0.75] (2) to (3);
		\draw (0.center) to (4.center);
	\end{pgfonlayer}
\end{tikzpicture}
$$
\end{lemma}






\begin{definition}
Let $\inj(\Aff\cb_2)$ denote the pushout of the diagram of props:
$$
 \inj(\cb_2) \leftarrow  \Iso(\cb_2)\rightarrow  \Iso(\Aff\cb_2)
$$
\end{definition}




\begin{lemma}
\label{lem:injaffcb}
$\inj(\Aff\cb_2)$ is a presentation for the prop $(\inj(\Aff\Mat(\F_2)),+)$.
\end{lemma}


\begin{proof}
Consider the following diagram:


\renewcommand{\cubetopbl}{$\Iso(\cb_2)$}
\renewcommand{\cubetopbr}{$\Iso(\Aff\cb_2)$}
\renewcommand{\cubetopfl}{$\inj(\cb_2)$}
\renewcommand{\cubetopfr}{$\inj(\Aff\cb_2)$}
\renewcommand{\cubebotbl}{$(\Iso(\Mat(\F_2)),+)$ }
\renewcommand{\cubebotbr}{$(\Iso(\Aff\Mat(\F_2)),+)$ }
\renewcommand{\cubebotfl}{$(\inj(\Mat(\F_2)),+)$ }
\renewcommand{\cubebotfr}{}

$$
\xymatrixrowsep{2mm}\xymatrixcolsep{1.5mm}
\xymatrix{
                                       & \mbox{\cubetopbl} \ar[rr] \ar[dl] \ar[dd]^(.7){\cong}      &                                                  & \mbox{\cubetopbr}  \ar[dd]^{\cong} \ar[dl] \\
\mbox{\cubetopfl} \ar[rr]  \ar[dd]_{\cong}           &                                                                                              &\mbox{\cubetopfr} \ar@{-->}[dd]^(.35){\cong}   \skewpullbackcorner[ul]              \\
                                       &  \mbox{\cubebotbl} \ar[dl] \ar[rr]                    &                                                  & \mbox{\cubebotbr} \ar@/^1pc/[ddl] \ar[dl] \\
\mbox{\cubebotfl} \ar@/_1pc/[drr] \ar[rr]  &                                                                                             & \mbox{\cubebotfr} \skewpullbackcorner[ul]    \ar@{-->}[d]^{\cong}_F  \\
                                                   &                                                                                             & (\inj(\Aff\Mat(\F_2)),+)
}
$$

 The rear and left faces of the cube commute and the vertical maps are all isomorphisms. Therefore, the whole cube commutes via universal property of the pushout, with the upper universal map being an isomorphism.

We seek to show that the lower universal map  $F$ is also an isomorphism.  It is clearly the identity on objects, so we just have to show fullness and faithfulness.

For fullness, consider any map $n\ \xrightarrowtail{(A,x)} m$ in $(\inj(\Aff\Mat(\F_2)),+)$.  Note that this can be factored into:
$$
n\ \xrightarrowtail{(A,0)} m \xrightarrowiso{(1,x)}  m
$$
Which lies in the image of $F$ as $m \xrightarrowiso{(1,x)} m$ is an isomorphism.

For faithfulness, we show that every map in $$(\Iso(\Aff\Mat(\F_2)),+)+_{(\Iso(\Mat(\F_2)),+)} (\inj(\Mat(\F_2)),+)$$ can be factored uniquely in this way. 
There are two cases:
$$
\left( n \ \xrightarrowtail{ A} m ; m \xrightarrowiso{(B, x)} m \right)
= \left( n \ \xrightarrowtail{ A} m ; m \xrightarrowiso{(B, 0)} m; m \xrightarrowiso{(1, x)}  m \right)
= \left( n \ \xrightarrowtail{ A;B}  m\xrightarrowiso{(1, x)}  m \right)
$$
and
\begin{align*}
\left(n \ \xrightarrowtail{ (A,x)} m ; m \xrightarrowiso{B} m \right)
&= \left( n \ \xrightarrowtail{ (A,0)}m; m \xrightarrowiso{(1,x)} m ; m \xrightarrowiso{B} m \right)
= \left( n\  \xrightarrowtail{ A }m; m \xrightarrowiso{(B,B(x))} m  \right)\\
&= \left( n \ \xrightarrowtail{ A;B }m; m \xrightarrowiso{(1,B(x))} m  \right)
\end{align*}
So every map in this pushout has the correct form, which is unique by construction.
\end{proof}


To define partial isomorphisms, we must add a point to the constituent props of the desired distributive law, because the empty set can arise as a subobject by pullback (where the empty set is not properly an object in the prop).

%\begin{definition}
%Given a prop $\X$, let $\X!$ denote the prop generated by adding a scalar $0$,  quotiented by the equation, for all parallel $f,g$: 
%$
%f \otimes 0  =  g \otimes 0
%$ 
%\end{definition}
%
%
%
%That is to say, $\X!$ is the prop with zero maps formally added.  In affine matrices, there is no proper zero object: the one element space is the terminal object and the empty set is the initial object.  By taking spans of affine matrices, the initial object becomes a zero object; however, seeing as we are working with props, the empty set can not be represented using this formalism.  Thus we just add the zero object as a subjobject.





\begin{definition}
\label{def:isoaffcbzero}
Let $\Iso(\Aff\cb_2)^{+1}$ denote the prop obtained by adjoining the following generator to $\Iso(\Aff\cb_2)$ 
$
\begin{tikzpicture}
	\begin{pgfonlayer}{nodelayer}
		\node [style=X] (0) at (0, 0) {$1$};
	\end{pgfonlayer}
\end{tikzpicture}
$
modulo the equations:
$$
\begin{tikzpicture}
	\begin{pgfonlayer}{nodelayer}
		\node [style=X] (0) at (0, 0) {$1$};
		\node [style=X] (3) at (0.5, 0) {$1$};
	\end{pgfonlayer}
\end{tikzpicture}
\eqzxa{zero.one}
\begin{tikzpicture}
	\begin{pgfonlayer}{nodelayer}
		\node [style=X] (0) at (0, 0) {$1$};
	\end{pgfonlayer}
\end{tikzpicture},
\hspace*{.5cm}
\begin{tikzpicture}
	\begin{pgfonlayer}{nodelayer}
		\node [style=X] (0) at (0, 1) {$1$};
		\node [style=none] (1) at (0.5, 0.5) {};
		\node [style=none] (2) at (0.5, 1.5) {};
		\node [style=none] (3) at (1, 1.5) {};
		\node [style=none] (4) at (1, 0.5) {};
		\node [style=dot] (5) at (0.5, 1) {};
		\node [style=oplus] (6) at (1, 1) {};
	\end{pgfonlayer}
	\begin{pgfonlayer}{edgelayer}
		\draw [in=90, out=-90] (2.center) to (1.center);
		\draw [in=-90, out=90] (4.center) to (3.center);
		\draw (6) to (5);
	\end{pgfonlayer}
\end{tikzpicture}
\eqzxa{zero.two}
\begin{tikzpicture}
	\begin{pgfonlayer}{nodelayer}
		\node [style=X] (0) at (0, 1) {$1$};
		\node [style=none] (1) at (0.5, 0.5) {};
		\node [style=none] (2) at (0.5, 1.5) {};
		\node [style=none] (3) at (1, 1.5) {};
		\node [style=none] (4) at (1, 0.5) {};
	\end{pgfonlayer}
	\begin{pgfonlayer}{edgelayer}
		\draw [in=90, out=-90] (2.center) to (1.center);
		\draw [in=-90, out=90] (4.center) to (3.center);
	\end{pgfonlayer}
\end{tikzpicture},
\hspace*{.5cm}
\begin{tikzpicture}
	\begin{pgfonlayer}{nodelayer}
		\node [style=X] (0) at (0, 1) {$1$};
		\node [style=none] (1) at (0.5, 0.5) {};
		\node [style=none] (2) at (1, 1.5) {};
		\node [style=none] (3) at (0.5, 1.5) {};
		\node [style=none] (4) at (1, 0.5) {};
	\end{pgfonlayer}
	\begin{pgfonlayer}{edgelayer}
		\draw [in=90, out=-90] (2.center) to (1.center);
		\draw [in=-90, out=90] (4.center) to (3.center);
	\end{pgfonlayer}
\end{tikzpicture}
\eqzxa{zero.three}
\begin{tikzpicture}
	\begin{pgfonlayer}{nodelayer}
		\node [style=X] (0) at (0, 1) {$1$};
		\node [style=none] (1) at (0.5, 0.5) {};
		\node [style=none] (2) at (0.5, 1.5) {};
		\node [style=none] (3) at (1, 1.5) {};
		\node [style=none] (4) at (1, 0.5) {};
	\end{pgfonlayer}
	\begin{pgfonlayer}{edgelayer}
		\draw [in=90, out=-90] (2.center) to (1.center);
		\draw [in=-90, out=90] (4.center) to (3.center);
	\end{pgfonlayer}
\end{tikzpicture},
\hspace*{.5cm}
\begin{tikzpicture}
	\begin{pgfonlayer}{nodelayer}
		\node [style=X] (0) at (0, 1) {$1$};
		\node [style=none] (1) at (0.5, 0.5) {};
		\node [style=none] (2) at (0.5, 1.5) {};
		\node [style=oplus] (3) at (0.5, 1) {};
	\end{pgfonlayer}
	\begin{pgfonlayer}{edgelayer}
		\draw (2.center) to (1.center);
	\end{pgfonlayer}
\end{tikzpicture}
\eqzxa{zero.four}
\begin{tikzpicture}
	\begin{pgfonlayer}{nodelayer}
		\node [style=X] (0) at (0, 1) {$1$};
		\node [style=none] (1) at (0.5, 0.5) {};
		\node [style=none] (2) at (0.5, 1.5) {};
	\end{pgfonlayer}
	\begin{pgfonlayer}{edgelayer}
		\draw (2.center) to (1.center);
	\end{pgfonlayer}
\end{tikzpicture}
$$
\end{definition}


\begin{lemma}
$\Iso(\Aff\cb_2)^{+1}$ is a presentation for the subcategory of $$(\Span^\sim(\Aff\Fin\Vect(\F_2)), +)$$ generated by spans 


$$\F_2^n = \F_2^n \xrightarrow[\cong]{f} \F_2^n\hspace*{.5cm}\text{and}\hspace*{.5cm}\F_2^n \xleftarrowtail {?}\  \emptyset \ \xrightarrowtail{?}  \F_2^n$$

for all $n \in \N$ and isomorphisms $f$. 
\end{lemma}


%
%\begin{lemma}
%
%The props $\Iso(\Aff\cb_2)^{+1}$ and $\Iso(\Aff\cb_2)!$ are isomorphic.
%
%\end{lemma}

\begin{proof}
Identify this new generator with the span $\F_2^0 \leftarrow \emptyset \rightarrow \F_2^0$.  If there is a factor of 
$
\begin{tikzpicture}
	\begin{pgfonlayer}{nodelayer}
		\node [style=X] (0) at (0, 0) {$1$};
	\end{pgfonlayer}
\end{tikzpicture}
$,   repeatedly apply these identities from left to right until the diagram corresponding to the identity tensored by $
\begin{tikzpicture}
	\begin{pgfonlayer}{nodelayer}
		\node [style=X] (0) at (0, 0) {$1$};
	\end{pgfonlayer}
\end{tikzpicture}
$ is obtained, which is as a normal form.
\end{proof}

\begin{definition}
Let $\inj(\Aff\cb_2)^{+1}$ denote the pushout of the diagram of props:


$$
\inj(\Aff\cb_2) \leftarrow \Iso(\Aff\cb_2) \rightarrow \Iso(\Aff\cb_2)^{+1}
$$


\end{definition}

\begin{lemma}
$\inj(\Aff\cb_2)^{+1}$ is a presentation for the subcategory of $(\Span^\sim(\Aff\Fin\Vect(\F_2)), +)$ generated by spans $\F_2^n = \F_2^n \ \xrightarrowtail{e} \F_2^m$ and $\F_2^n \xleftarrowtail{?} \ \emptyset \ \xrightarrowtail{?}  \F_2^n$, for all $n,m \in \N$ and monics $e$. 
\end{lemma}
%
%\begin{lemma}
%The props $\inj(\Aff\cb_2)^{+1}$ and $\inj(\Aff\cb_2)!$ are isomorphic.
%\end{lemma}

The proof of this lemma is essentially the same for $\Iso(\Aff\cb_2)^{+1}$, although diagrams with a factor of
$
\begin{tikzpicture}
	\begin{pgfonlayer}{nodelayer}
		\node [style=X] (0) at (0, 0) {$1$};
	\end{pgfonlayer}
\end{tikzpicture}
$ are reduced to the following normal form:

$$
\begin{tikzpicture}
	\begin{pgfonlayer}{nodelayer}
		\node [style=X] (0) at (0, 1.25) {$1$};
		\node [style=none] (1) at (0.5, 0.5) {};
		\node [style=none] (2) at (0.5, 1.75) {};
		\node [style=none] (3) at (1, 0.5) {};
		\node [style=none] (4) at (1, 1.75) {};
		\node [style=X] (5) at (1.5, 0.75) {};
		\node [style=X] (6) at (2, 0.75) {};
		\node [style=none] (7) at (1.5, 1.75) {};
		\node [style=none] (8) at (2, 1.75) {};
		\node [style=none] (9) at (0.75, 1.5) {$n$};
		\node [style=none] (10) at (1.75, 1.5) {$m$};
		\node [style=none] (11) at (1.77, 1.25) {$\cdots$};
		\node [style=none] (12) at (0.77, 1.25) {$\cdots$};
	\end{pgfonlayer}
	\begin{pgfonlayer}{edgelayer}
		\draw (2.center) to (1.center);
		\draw (4.center) to (3.center);
		\draw (5) to (7.center);
		\draw (8.center) to (6);
	\end{pgfonlayer}
\end{tikzpicture}
$$

Unlike in the linear case, now we must consider a distributive law over a prop which is not a groupoid: we add a single idempotent corresponding to the empty set to the isomorphisms.  To satisfy the requirement that this prop is a sub-prop of the left and right components of the  distributive law, we also add this idempotent to the injections and the co-injections:


\begin{definition}
\label{def:parisoaffcb}
Consider the prop $\pr\iso\Aff\cb_2$ generated by the distributive law of props:

$$
 (\inj(\Aff\cb_2)^{+1})^\op \otimes_{\Iso(\Aff\cb_2)^{+1}}  \inj(\Aff\cb_2)^{+1}
$$
Given by the equations of $\pr\iso\Aff\cb_2$ as well as:

$$
\begin{tikzpicture}
	\begin{pgfonlayer}{nodelayer}
		\node [style=X] (0) at (0.5, 0.75) {$1$};
		\node [style=X] (1) at (0.5, 0) {};
	\end{pgfonlayer}
	\begin{pgfonlayer}{edgelayer}
		\draw (0) to (1);
	\end{pgfonlayer}
\end{tikzpicture}
=
\begin{tikzpicture}
	\begin{pgfonlayer}{nodelayer}
		\node [style=X] (0) at (0, 0) {$1$};
		\node [style=X] (1) at (0, 0.75) {};
	\end{pgfonlayer}
	\begin{pgfonlayer}{edgelayer}
		\draw (0) to (1);
	\end{pgfonlayer}
\end{tikzpicture}
\eqzxa{zero.five}
\begin{tikzpicture}
	\begin{pgfonlayer}{nodelayer}
		\node [style=X] (0) at (0, 0) {$1$};
	\end{pgfonlayer}
	\begin{pgfonlayer}{edgelayer}
	\end{pgfonlayer}
\end{tikzpicture}
$$

\end{definition}


\begin{remark}
\label{rem:parisoaffcb}
$ (\inj(\Aff\cb_2)^{+1})^\op \otimes_{\Iso(\Aff\cb_2)^{+1}}  \inj(\Aff\cb_2)^{+1}$ is actually a distributive law because the only only nontrivial situation arises when controlled-not gates are sandwiched between black, or black $1$ units/counits on their target wires.  The case where there are no controlled not gates in between is resolved by the new axiom we have added.  When there are more controlled-not gates, they can be pushed past each other as follows:
$$
\begin{tikzpicture}
	\begin{pgfonlayer}{nodelayer}
		\node [style=X] (0) at (2.5, 0.25) {};
		\node [style=oplus] (1) at (2.5, -0.75) {};
		\node [style=oplus] (2) at (2.5, -1.5) {};
		\node [style=dot] (3) at (1.5, -0.75) {};
		\node [style=dot] (4) at (1, -1.5) {};
		\node [style=none] (5) at (1.5, 0.5) {};
		\node [style=none] (6) at (1, 0.5) {};
		\node [style=none] (7) at (1.25, -1) {$\iddots$};
		\node [style=none] (8) at (2.5, -1) {$\vdots$};
		\node [style=X] (9) at (2.5, -2) {$1$};
		\node [style=none] (10) at (1, -2.25) {};
		\node [style=none] (11) at (1.5, -2.25) {};
		\node [style=none] (12) at (1.25, -0.25) {$\cdots$};
		\node [style=none] (17) at (2, -2.25) {};
		\node [style=none] (18) at (2, 0.5) {};
		\node [style=oplus] (19) at (2.5, -0.25) {};
		\node [style=dot] (20) at (2, -0.25) {};
	\end{pgfonlayer}
	\begin{pgfonlayer}{edgelayer}
		\draw (0) to (1);
		\draw (1) to (3);
		\draw (5.center) to (3);
		\draw (6.center) to (4);
		\draw (4) to (2);
		\draw (4) to (10.center);
		\draw (11.center) to (3);
		\draw (9) to (2);
		\draw (17.center) to (20);
		\draw (20) to (18.center);
		\draw (19) to (20);
	\end{pgfonlayer}
\end{tikzpicture}
=
\begin{tikzpicture}
	\begin{pgfonlayer}{nodelayer}
		\node [style=X] (0) at (2.5, 0.25) {};
		\node [style=oplus] (1) at (2.5, -0.75) {};
		\node [style=oplus] (2) at (2.5, -1.5) {};
		\node [style=dot] (3) at (1.5, -0.75) {};
		\node [style=dot] (4) at (1, -1.5) {};
		\node [style=none] (5) at (1.5, 0.5) {};
		\node [style=none] (6) at (1, 0.5) {};
		\node [style=none] (7) at (1.25, -1) {$\iddots$};
		\node [style=none] (8) at (2.5, -1) {$\vdots$};
		\node [style=X] (9) at (2.5, -2.5) {};
		\node [style=none] (10) at (1, -2.75) {};
		\node [style=none] (11) at (1.5, -2.75) {};
		\node [style=none] (12) at (1.25, -0.25) {$\cdots$};
		\node [style=none] (17) at (2, -2.75) {};
		\node [style=none] (18) at (2, 0.5) {};
		\node [style=oplus] (19) at (2.5, -0.25) {};
		\node [style=dot] (20) at (2, -0.25) {};
		\node [style=oplus] (21) at (2.5, -2) {};
	\end{pgfonlayer}
	\begin{pgfonlayer}{edgelayer}
		\draw (0) to (1);
		\draw (1) to (3);
		\draw (5.center) to (3);
		\draw (6.center) to (4);
		\draw (4) to (2);
		\draw (4) to (10.center);
		\draw (11.center) to (3);
		\draw (9) to (2);
		\draw (17.center) to (20);
		\draw (20) to (18.center);
		\draw (19) to (20);
	\end{pgfonlayer}
\end{tikzpicture}
=
\begin{tikzpicture}
	\begin{pgfonlayer}{nodelayer}
		\node [style=X] (0) at (2.5, 0.75) {};
		\node [style=oplus] (1) at (2.5, -0.75) {};
		\node [style=oplus] (2) at (2.5, -1.5) {};
		\node [style=dot] (3) at (1.5, -0.75) {};
		\node [style=dot] (4) at (1, -1.5) {};
		\node [style=none] (5) at (1.5, 1.25) {};
		\node [style=none] (6) at (1, 1.25) {};
		\node [style=none] (7) at (1.25, -1) {$\iddots$};
		\node [style=none] (8) at (2.5, -1) {$\vdots$};
		\node [style=X] (9) at (2.5, -2) {};
		\node [style=none] (10) at (1, -2.25) {};
		\node [style=none] (11) at (1.5, -2.25) {};
		\node [style=none] (12) at (1.25, 0.25) {$\cdots$};
		\node [style=none] (17) at (2, -2.25) {};
		\node [style=none] (18) at (2, 1.25) {};
		\node [style=oplus] (19) at (2.5, 0.25) {};
		\node [style=dot] (20) at (2, 0.25) {};
		\node [style=oplus] (21) at (2, -0.25) {};
		\node [style=oplus] (22) at (2, 0.75) {};
	\end{pgfonlayer}
	\begin{pgfonlayer}{edgelayer}
		\draw (0) to (1);
		\draw (1) to (3);
		\draw (5.center) to (3);
		\draw (6.center) to (4);
		\draw (4) to (2);
		\draw (4) to (10.center);
		\draw (11.center) to (3);
		\draw (9) to (2);
		\draw (17.center) to (20);
		\draw (20) to (18.center);
		\draw (19) to (20);
	\end{pgfonlayer}
\end{tikzpicture}
=
\begin{tikzpicture}
	\begin{pgfonlayer}{nodelayer}
		\node [style=X] (0) at (2.5, 0.25) {};
		\node [style=oplus] (1) at (2.5, -0.75) {};
		\node [style=oplus] (2) at (2.5, -1.5) {};
		\node [style=dot] (3) at (1.5, -0.75) {};
		\node [style=dot] (4) at (1, -1.5) {};
		\node [style=none] (5) at (1.5, 1) {};
		\node [style=none] (6) at (1, 1) {};
		\node [style=none] (7) at (1.25, -1) {$\iddots$};
		\node [style=none] (8) at (2.5, -1) {$\vdots$};
		\node [style=X] (9) at (2.5, -2) {};
		\node [style=none] (10) at (1, -2.75) {};
		\node [style=none] (11) at (1.5, -2.75) {};
		\node [style=none] (12) at (1.25, -0.25) {$\cdots$};
		\node [style=none] (17) at (2, -2.75) {};
		\node [style=none] (18) at (2, 1) {};
		\node [style=oplus] (19) at (2.5, -0.25) {};
		\node [style=dot] (20) at (2, -0.25) {};
		\node [style=oplus] (21) at (2, -2.25) {};
		\node [style=oplus] (22) at (2, 0.5) {};
	\end{pgfonlayer}
	\begin{pgfonlayer}{edgelayer}
		\draw (0) to (1);
		\draw (1) to (3);
		\draw (5.center) to (3);
		\draw (6.center) to (4);
		\draw (4) to (2);
		\draw (4) to (10.center);
		\draw (11.center) to (3);
		\draw (9) to (2);
		\draw (17.center) to (20);
		\draw (20) to (18.center);
		\draw (19) to (20);
	\end{pgfonlayer}
\end{tikzpicture}
=
\begin{tikzpicture}
	\begin{pgfonlayer}{nodelayer}
		\node [style=none] (24) at (4.5, 1.75) {};
		\node [style=none] (25) at (4, 1.75) {};
		\node [style=X] (28) at (5, -1) {};
		\node [style=none] (29) at (4, -3.25) {};
		\node [style=none] (30) at (4.5, -3.25) {};
		\node [style=none] (32) at (5, -3.25) {};
		\node [style=none] (33) at (5, 1.75) {};
		\node [style=oplus] (37) at (5, 1.25) {};
		\node [style=oplus] (38) at (5, -1.5) {};
		\node [style=dot] (39) at (4.5, -1.5) {};
		\node [style=oplus] (40) at (5, -2.25) {};
		\node [style=dot] (41) at (4, -2.25) {};
		\node [style=oplus] (42) at (5, 0) {};
		\node [style=dot] (43) at (4.5, 0) {};
		\node [style=oplus] (44) at (5, 0.75) {};
		\node [style=dot] (45) at (4, 0.75) {};
		\node [style=oplus] (46) at (5, -2.75) {};
		\node [style=X] (47) at (5, -0.5) {};
		\node [style=none] (48) at (4.25, -0.75) {$\cdots$};
		\node [style=none] (49) at (4.25, -1.75) {$\iddots$};
		\node [style=none] (50) at (4.25, 0.5) {$\ddots$};
		\node [style=none] (51) at (5, 0.5) {$\vdots$};
		\node [style=none] (52) at (5, -1.75) {$\vdots$};
	\end{pgfonlayer}
	\begin{pgfonlayer}{edgelayer}
		\draw (38) to (39);
		\draw (40) to (41);
		\draw (42) to (43);
		\draw (44) to (45);
		\draw (29.center) to (25.center);
		\draw (24.center) to (30.center);
		\draw (38) to (28);
		\draw (40) to (32.center);
		\draw (47) to (42);
		\draw (44) to (33.center);
	\end{pgfonlayer}
\end{tikzpicture}
=
\begin{tikzpicture}
	\begin{pgfonlayer}{nodelayer}
		\node [style=none] (24) at (4.5, 1.25) {};
		\node [style=none] (25) at (4, 1.25) {};
		\node [style=X] (28) at (5, -1) {$1$};
		\node [style=none] (29) at (4, -2.75) {};
		\node [style=none] (30) at (4.5, -2.75) {};
		\node [style=none] (32) at (5, -2.75) {};
		\node [style=none] (33) at (5, 1.25) {};
		\node [style=oplus] (38) at (5, -1.5) {};
		\node [style=dot] (39) at (4.5, -1.5) {};
		\node [style=oplus] (40) at (5, -2.25) {};
		\node [style=dot] (41) at (4, -2.25) {};
		\node [style=oplus] (42) at (5, 0) {};
		\node [style=dot] (43) at (4.5, 0) {};
		\node [style=oplus] (44) at (5, 0.75) {};
		\node [style=dot] (45) at (4, 0.75) {};
		\node [style=X] (47) at (5, -0.5) {$1$};
		\node [style=none] (48) at (4.25, -0.75) {$\cdots$};
		\node [style=none] (49) at (4.25, -1.75) {$\iddots$};
		\node [style=none] (50) at (4.25, 0.5) {$\ddots$};
		\node [style=none] (51) at (5, 0.5) {$\vdots$};
		\node [style=none] (52) at (5, -1.75) {$\vdots$};
	\end{pgfonlayer}
	\begin{pgfonlayer}{edgelayer}
		\draw (38) to (39);
		\draw (40) to (41);
		\draw (42) to (43);
		\draw (44) to (45);
		\draw (29.center) to (25.center);
		\draw (24.center) to (30.center);
		\draw (38) to (28);
		\draw (40) to (32.center);
		\draw (47) to (42);
		\draw (44) to (33.center);
	\end{pgfonlayer}
\end{tikzpicture}
$$
%Notice that the choice of which wires to straighten out the zig-zag is arbitrary.
\end{remark}







\begin{lemma}
\label{lem:parisoaffcb}
$\ParIso(\Aff\cb_2)$ is a presentation for the full subcategory $\Par\Iso(\Aff\Fin\Vect(\F_2))^*$ of $\Par\Iso(\Aff\Fin\Vect(\F_2))$ where the objects are nonempty affine vector spaces.
\end{lemma}


\begin{proof}
The obvious functor $\ParIso(\Aff\cb_2)\to \Par\Iso(\Aff\Fin\Vect(\F_2))^*$ is clearly full,  as well as an isomorphism on objects.
It remains to show it is faihful.  It is faithful on maps which are taken to spans with nonempty apex by the same argument as Lemma \ref{lem:parisocb}. For empty case, there is exactly one diagram of each type with a factor of $0$; and similarly, there is exactly one span with an empty apex.
\end{proof}

By \cite{cnot}  in this the identities of Definition \ref{def:isoaffcbzero}
 can be replaced by the following identity, while maintaining completeness:
\hspace*{.4cm}
$
\begin{tikzpicture}
	\begin{pgfonlayer}{nodelayer}
		\node [style=X] (0) at (0, 5) {$1$};
		\node [style=none] (1) at (0.5, 5.75) {};
		\node [style=none] (2) at (0.5, 4.25) {};
	\end{pgfonlayer}
	\begin{pgfonlayer}{edgelayer}
		\draw (2.center) to (1.center);
	\end{pgfonlayer}
\end{tikzpicture}
\eqzxa{zero.six}
\begin{tikzpicture}
	\begin{pgfonlayer}{nodelayer}
		\node [style=X] (0) at (0, 5) {$1$};
		\node [style=none] (1) at (0.5, 5.75) {};
		\node [style=none] (2) at (0.5, 4.25) {};
		\node [style=X] (3) at (0.5, 5.25) {$1$};
		\node [style=X] (4) at (0.5, 4.75) {$1$};
	\end{pgfonlayer}
	\begin{pgfonlayer}{edgelayer}
		\draw (3) to (1.center);
		\draw (4) to (2.center);
	\end{pgfonlayer}
\end{tikzpicture}
$

%Using the identities presented in \cite{cnot}, this has a more compact presentation given in Appendix \ref{subsubsec:presentations:three:pinj}.
%This alternative form is much more in the aesthetic vein of the ZX-calculus,

%\begin{corollary}
%The prop $\ParIso(\Aff\cb_2)$ has a finite presentation with axioms where the axioms are the union of the axioms for $\inj(\Aff\cb_2),\inj(\Aff\cb_2)^\op$, the law $\eta_X\eta_X^\op=1_I$ as well as the $1$-law, replacing the universally quantified law for the 0 scalar:
%$$
%\begin{tikzpicture}
%	\begin{pgfonlayer}{nodelayer}
%		\node [style=none] (0) at (-1, -7) {};
%		\node [style=X] (1) at (-0.5, -7.75) {$1$};
%		\node [style=none] (2) at (-1, -8.5) {};
%	\end{pgfonlayer}
%	\begin{pgfonlayer}{edgelayer}
%		\draw (0.center) to (2.center);
%	\end{pgfonlayer}
%\end{tikzpicture}
%=
%\begin{tikzpicture}
%	\begin{pgfonlayer}{nodelayer}
%		\node [style=none] (0) at (-1, -7) {};
%		\node [style=X] (1) at (-0.5, -7.75) {$1$};
%		\node [style=X] (2) at (-1, -7.5) {};
%		\node [style=X] (3) at (-1, -8) {};
%		\node [style=none] (4) at (-1, -8.5) {};
%	\end{pgfonlayer}
%	\begin{pgfonlayer}{edgelayer}
%		\draw (0.center) to (2);
%		\draw (3) to (4.center);
%	\end{pgfonlayer}
%\end{tikzpicture}
%$$
% 
%
%\end{corollary}


\begin{definition}

Let $\pr\Aff\cb_2$ denote the pushout of the diagram of props:
$$
\pr\iso\Aff\cb_2 \leftarrow \surj^\op \rightarrow \cm^\op
$$

\end{definition}





\begin{lemma}
\label{lem:paraffcb}

$\pr\Aff\cb_2$ is a presentation for the prop $(\Par(\Aff\Fin\Vect(\F_2))^*,+)$.
\end{lemma}

\begin{proof}
%\renewcommand{\cubetopbl}{$\inj(\Aff\cb_2)$}
%\renewcommand{\cubetopbr}{$\inj(\Aff\cb_2)+1\otimes_{\Iso(\Aff\cb_2)+1} \inj(\Aff\cb_2)+1^\op$}
%\renewcommand{\cubetopfl}{$\inj(\Aff\cb_2)\otimes_{\Iso(\Aff\cb_2)} \surj(\Aff\cb_2)^\op$}
%\renewcommand{\cubetopfr}{$\Par(\Aff\cb_2)$}
%\renewcommand{\cubebotbl}{$(\inj(\Aff\Mat(\F_2)),+)$ }
%\renewcommand{\cubebotbr}{$(\ParIso(\Aff\Vect(\F_2))^*,+)$ }
%\renewcommand{\cubebotfl}{$(\Par\surj(\Aff\Mat(\F_2)),+)^\op$ }
%\renewcommand{\cubebotfr}{}
%
%$$
%\xymatrixrowsep{3mm}\xymatrixcolsep{-10mm}
%\xymatrix{
%                                       & \mbox{\cubetopbl} \ar[rr] \ar[dl] \ar[dd]^(.7){\cong}      &                                                  & \mbox{\cubetopbr}  \ar[dd]^{\cong} \ar[dl] \\
%\mbox{\cubetopfl} \ar[rr]  \ar[dd]_{\cong}           &                                                                                              &\mbox{\cubetopfr} \ar@{-->}[dd]^(.35){\cong}   \skewpullbackcorner[ul]              \\
%                                       &  \mbox{\cubebotbl} \ar[dl] \ar[rr]                    &                                                  & \mbox{\cubebotbr} \ar@/^1pc/[ddl] \ar[dl] \\
%\mbox{\cubebotfl} \ar@/_1pc/[drr] \ar[rr]  &                                                                                             & \mbox{\cubebotfr} \skewpullbackcorner[ul]    \ar@{-->}[d]^{\cong}  \\
%                                                   &                                                                                             & (\Par(\Aff\Mat(\F_2)),+)
%}
%$$



\renewcommand{\cubetopbl}{$\surj^\op$}
\renewcommand{\cubetopbr}{$\cm^\op$}
\renewcommand{\cubetopfl}{$\pr\iso\Aff\cb_2$}
\renewcommand{\cubetopfr}{$\pr\Aff\cb_2$}
\renewcommand{\cubebotbl}{$\surj^\op$ }
\renewcommand{\cubebotbr}{$\cm^\op$ }
\renewcommand{\cubebotfl}{$(\ParIso(\Aff\Vect(\F_2))^*,+)$ }
\renewcommand{\cubebotfr}{}

$$
\xymatrixrowsep{2mm}\xymatrixcolsep{1mm}
\xymatrix{
                                       & \mbox{\cubetopbl} \ar[rr] \ar[dl] \ar@{=}[dd]     &                                                  & \mbox{\cubetopbr} \ar@{=}[dd] \ar[dl] \\
\mbox{\cubetopfl} \ar[rr]  \ar[dd]_{\cong}           &                                                                                              &\mbox{\cubetopfr} \ar@{-->}[dd]^(.35){\cong}   \skewpullbackcorner[ul]              \\
                                       &  \mbox{\cubebotbl} \ar[dl] \ar[rr]                    &                                                  & \mbox{\cubebotbr} \ar@/^1pc/[ddl] \ar[dl] \\
\mbox{\cubebotfl} \ar@/_1pc/[drr] \ar[rr]  &                                                                                             & \mbox{\cubebotfr} \skewpullbackcorner[ul]    \ar@{-->}[d]_F^{\cong}  \\
                                                   &                                                                                             & (\Par(\Aff\Vect(\F_2))^*,+) 
}
$$


We know that $\pr\iso\Aff\cb_2\cong \ParIso(\Aff\Vect(\F_2))^*,+)$ is a discrete inverse category by \cite[Prop. 3.4]{cnot}.

The cube commutes by the universal property of the pushout, as before.

We just have to show that the universal map $F$ is an isomorphism.  It is clearly the identity on objects, so we just have to show it is full and faithful.
This follows from essentially the same argument as in the linear case.


\end{proof}


%\begin{comment}
$\pr\Aff\cb_2$ has a particularly elegant presentation given in \S \ref{subsubsec:presentations:two:par}, which is much more in the spirit of the ZX-calculus.
%\end{comment}


\begin{definition}
Let $\sp\Aff\cb_2$ denote the pushout of the diagram of props:
$$
 \pr\Aff\cb_2^\op \leftarrow \pr\iso\Aff\cb_2 \rightarrow \pr\Aff\cb_2
$$

\end{definition}



\begin{lemma}
\label{lem:spanaffcb}
$\sp\Aff\cb_2$ is a presentation for the prop $(\Span^\sim(\Aff\Fin\Vect(\F_2))^*,+)$.
\end{lemma}

\begin{proof} \


\renewcommand{\cubetopbl}{$\pr\iso\Aff\cb_2$}
\renewcommand{\cubetopbr}{$\pr\Aff\cb_2$}
\renewcommand{\cubetopfl}{$\pr\Aff\cb_2^\op$}
\renewcommand{\cubetopfr}{$\sp\Aff\cb_2$}
\renewcommand{\cubebotbl}{$(\ParIso(\Aff\Vect(\F_2))^*,+)$ }
\renewcommand{\cubebotbr}{$(\Par(\Aff\Vect(\F_2))^*,+)$ }
\renewcommand{\cubebotfl}{$(\Par(\Aff\Vect(\F_2))^*,+)^\op$ }
\renewcommand{\cubebotfr}{}

\scalebox{.8}{$
\xymatrixrowsep{2mm}\xymatrixcolsep{.5mm}
\xymatrix{
                                       & \mbox{\cubetopbl} \ar[rr] \ar[dl] \ar[dd]^(.7){\cong}      &                                                  & \mbox{\cubetopbr}  \ar[dd]^{\cong} \ar[dl] \\
\mbox{\cubetopfl} \ar[rr]  \ar[dd]_{\cong}           &                                                                                              &\mbox{\cubetopfr} \ar@{-->}[dd]^(.35){\cong}   \skewpullbackcorner[ul]              \\
                                       &  \mbox{\cubebotbl} \ar[dl] \ar[rr]                    &                                                  & \mbox{\cubebotbr} \ar@/^1pc/[ddl] \ar[dl] \\
\mbox{\cubebotfl} \ar@/_1pc/[drr] \ar[rr]  &                                                                                             & \mbox{\cubebotfr} \skewpullbackcorner[ul]    \ar@{-->}[d]_F^{\cong}  \\
                                                   &                                                                                             & (\Span^\sim(\Aff\Vect(\F_2))^*,+)
}
$}

 The rear and left faces of the cube commute and the vertical maps are all isomorphisms. Therefore, the whole cube commutes by the universal property of the pushout, with the upper universal map being an isomorphism.

We seek to show that the lower universal map  $F$ is also an isomorphism.  It is clearly the identity on objects, so we just have to show fullness and faithfulness.

For fullness, let us first consider the nonempty case; that is a map $\F_2^n \xleftarrow{(A,x)} \F_2^k \xrightarrow{(B,y)}\F^m$ in $(\Span^\sim (\Aff\Vect(\F_2))^*,+)$.  This is in the image of the following diagram under $F$:
$$
(\F_2^n \xleftarrow{(A,x)} \F_2^k  = \F_2^k); (\F_2^k = \F_2^k  \xrightarrow{(B,y)}\F^m)
$$ 
Otherwise, consider a map of the form  $\F_2^n \xleftarrow{?} \emptyset  \xrightarrow{?}\F^m$.  This is in the image of the following diagram:
$$
(\F_2^n \xleftarrow{?} \emptyset \xrightarrow {?} \F_2^0  );(\F_2^0 \xleftarrow{?} \emptyset  \xrightarrow{?}\F^m)
$$
For faithfulness, again, we separate the proof into two cases.  The functor is faithful on diagrams in $(\Span^\sim(\Aff\Vect(\F_2))^*,+)$ with nonempty apex by the same argument as in Lemma \ref{lem:spancb}.
%$$
%\xymatrix{
%          & \F_2^k \ar[dl]_{(A',x')} \ar[dd]_{\cong}^{(C,z)} \ar[dr]^{(B',y')}\\
%\F_2^n  &                                                                                                    & \F_2^m\\
%         & \F_2^k \ar[ul]^{(A,x)} \ar[ur]_{(B,y)}\\
%}
%$$
%We have the following equation in $\Span(\Aff\cb_2)$:
%{
%\xymatrixrowsep{1mm}\xymatrixcolsep{3.5mm}
%\begin{align*}
%\xymatrix{
%          & \F_2^k \ar[dl]_{(A,x)}  \ar@{=}[dr]\\
%\F_2^n  &                                                                                                    & \F_2^k\\
%};
%\xymatrix{
%          & \F_2^k \ar[dr]^{(B,y)}  \ar@{=}[dl]\\
%\F_2^k  &                                                                                                    & \F_2^m\\
%} &=
%\xymatrix{
%          & \F_2^k \ar[dl]_{(A,x)}  \ar@{=}[dr]\\
%\F_2^n  &                                                                                                    & \F_2^k\\
%};
%\xymatrix{
%          & \F_2^k \ar@{=}[dr] \ar@{=}[dl] \\
%\F_2^k  &                                                                                                    & \F_2^k\\
%         & \F_2^k \ar[ul]^{(C,z)} \ar[ur] _{(C,z)} \ar[uu]^{\cong}_{(C,z)}
%};
%\xymatrix{
%          & \F_2^k \ar[dr]^{(B,y)}  \ar@{=}[dl]\\
%\F_2^k  &                                                                                                    & \F_2^m\\
%}\\
%&=
%\xymatrix{
%          & \F_2^k \ar[dl]_{(A,x)}  \ar@{=}[dr]\\
%\F_2^n  &                                                                                                    & \F_2^k\\
%};
%\xymatrix{
%        & \F_2^k \ar[dl]_{(C,z)} \ar@{=}[dr]\\
%\F_2^k  &                                                     & \F_2^k
%};
%\xymatrix{
%        & \F_2^k \ar[dr]^{(C,z)} \ar@{=}[dl]\\
%\F_2^k  &                                                     & \F_2^k
%};
%\xymatrix{
%          & \F_2^k \ar[dr]^{(B,y)}  \ar@{=}[dl]\\
%\F_2^k  &                                                                                                    & \F_2^m\\
%}\\
%&=
%\xymatrix{
%            &                                                        & \F_2^k \ar[dl]_{(C,z)} \ar@{=}[dr] \ar@/_2.0pc/[ddll]_{(A',x')}\\
%            & \F_2^k \ar@{=}[dr] \ar[dl]^{(A,x)}&                                                          & \F_2^k \ar@{=}[dr] \ar[dl]_{(C,z)}\\
%\F_2^k &                                                         & \F_2^k                                             &                                                         &\F_2^k
%};
%\xymatrix{
%            &                                                        & \F_2^k \ar[dr]^{(C,z)} \ar@{=}[dl] \ar@/^2.0pc/[ddrr]^{(B',y')}  \\
%            & \F_2^k \ar[dr]^{(C,z)}   \ar@{=}[dl] &                                                          & \F_2^k \ar@{=}[dl] \ar[dr]_{(B,y)}\\
%\F_2^k &                                                         & \F_2^k                                             &                                                         &\F_2^k
%}
%\end{align*}
%}
The case for spans with empty apex follows immediately as the only endomorphism on the empty set is the identity; thus,  isomorphic spans must be equal on the nose.

\end{proof}

%\begin{comment}
There is a particularly elegant equivalent presentation given in \S \ref{subsubsec:presentations:two:span}.
%\end{comment}
This is almost equivalent to the presentation given in \cite{affine} which gives a presentation for the full subcategory of relations of finite dimensional affine vector spaces where the objects are given by the nonempty vector spaces, and is much more in the spirit of the ZX-calculus.


\subsection{The and gate}
\label{sec:three}

Recall that unlike when the tensor product is the coproduct; when the tensor product is induced by the multiplication, to obtain a prop, one must consider the subcategory generated by tensoring a fixed object with itsef.
%
%Because $\sum_n 1=n$ grows linearly in $n$ and $\prod_n k = k^n$ grows exponentially, giving presentations for multiplicative models will be much more involved because there are more points to deal with, and in particular, more subobjects arise by pullback.


\begin{definition}
Let $L_{\F_2^\times}$ be the prop generated by quotienting $\cb$ by the equation:

$$
\begin{tikzpicture}
	\begin{pgfonlayer}{nodelayer}
		\node [style=none] (0) at (-7, 1) {};
		\node [style=none] (1) at (-7, 0.5) {};
		\node [style=Z] (2) at (-7, -0.25) {};
		\node [style=none] (3) at (-7, -0.75) {};
		\node [style=andin] (4) at (-7, 0.5) {};
	\end{pgfonlayer}
	\begin{pgfonlayer}{edgelayer}
		\draw (3.center) to (2.center);
		\draw [in=-60, out=60, looseness=1.25] (2.center) to (1);
		\draw [in=120, out=-120, looseness=1.25] (1) to (2.center);
		\draw (1) to (0.center);
	\end{pgfonlayer}
\end{tikzpicture}
\eqzxa{antispecial}
\begin{tikzpicture}
	\begin{pgfonlayer}{nodelayer}
		\node [style=none] (0) at (-7, 1) {};
		\node [style=none] (1) at (-7, -0.75) {};
	\end{pgfonlayer}
	\begin{pgfonlayer}{edgelayer}
		\draw (1.center) to (0.center);
	\end{pgfonlayer}
\end{tikzpicture}
$$

Where the components of the  monoid are relabled as follows:
\hspace*{.5cm}
$
\left(
\begin{tikzpicture}
	\begin{pgfonlayer}{nodelayer}
		\node [style=none] (0) at (-3.75, 0.5) {};
		\node [style=none] (1) at (-3.75, -0.25) {};
		\node [style=andin] (2) at (-3.75, -0.25) {};
		\node [style=none] (3) at (-4, -1) {};
		\node [style=none] (4) at (-3.5, -1) {};
	\end{pgfonlayer}
	\begin{pgfonlayer}{edgelayer}
		\draw (0.center) to (1.center);
		\draw [in=-60, out=90, looseness=1.00] (4.center) to (1.center);
		\draw [in=90, out=-120, looseness=1.00] (1.center) to (3.center);
	\end{pgfonlayer}
\end{tikzpicture},
\begin{tikzpicture}
	\begin{pgfonlayer}{nodelayer}
		\node [style=none] (0) at (-3.75, -0.25) {};
		\node [style=X] (1) at (-3.75, -1) {$1$};
	\end{pgfonlayer}
	\begin{pgfonlayer}{edgelayer}
		\draw (0.center) to (1);
	\end{pgfonlayer}
\end{tikzpicture}
\right)
$


\end{definition}


\begin{lemma}
$L_{\F_2}^\times$ is a presentation for the Lawvere theory for the group of units of the field $\F_2$.
\end{lemma}

\begin{definition}
Consider the prop $\f_2$, generated by the distributive law:
$$
L_{\F_2^\times} \otimes_{\cm^\op} \cb_2;
\begin{tikzpicture}
	\begin{pgfonlayer}{nodelayer}
		\node [style=andin] (4) at (1.25, 0.5) {};
		\node [style=X] (5) at (0.75, -0.5) {};
		\node [style=none] (6) at (0.5, -1) {};
		\node [style=none] (7) at (1, -1) {};
		\node [style=none] (8) at (1.75, -1) {};
		\node [style=none] (9) at (1.25, 0.5) {};
		\node [style=none] (10) at (1.25, 1.5) {};
	\end{pgfonlayer}
	\begin{pgfonlayer}{edgelayer}
		\draw [in=-30, out=90] (8.center) to (9.center);
		\draw [in=90, out=-150] (9.center) to (5);
		\draw [in=90, out=-45] (5) to (7.center);
		\draw [in=-135, out=90] (6.center) to (5);
		\draw (9.center) to (10.center);
	\end{pgfonlayer}
\end{tikzpicture}
\eqzxa{ring.mul}
\begin{tikzpicture}
	\begin{pgfonlayer}{nodelayer}
		\node [style=none] (0) at (1, 0) {};
		\node [style=none] (1) at (0.5, -1.25) {};
		\node [style=none] (2) at (1.75, -0.75) {};
		\node [style=none] (3) at (1.33, 0.75) {};
		\node [style=andin] (4) at (1, 0) {};
		\node [style=none] (5) at (1.75, 0) {};
		\node [style=none] (6) at (1, -1.25) {};
		\node [style=none] (7) at (1.75, -0.75) {};
		\node [style=none] (8) at (1.33, 0.75) {};
		\node [style=andin] (9) at (1.75, 0) {};
		\node [style=X] (10) at (1.33, 0.75) {};
		\node [style=none] (11) at (1.33, 1.25) {};
		\node [style=none] (12) at (1.75, -1.25) {};
		\node [style=Z] (13) at (1.75, -0.75) {};
	\end{pgfonlayer}
	\begin{pgfonlayer}{edgelayer}
		\draw [in=-135, out=90] (0.center) to (3.center);
		\draw [in=165, out=-30, looseness=1.25] (0.center) to (2.center);
		\draw [in=-45, out=90] (5.center) to (8.center);
		\draw [in=45, out=-45, looseness=1.25] (5.center) to (7.center);
		\draw (10) to (11.center);
		\draw [in=90, out=-150] (4) to (1.center);
		\draw [in=-150, out=90] (6.center) to (9);
		\draw (12.center) to (13);
	\end{pgfonlayer}
\end{tikzpicture},
\hspace*{.5cm}
\begin{tikzpicture}
	\begin{pgfonlayer}{nodelayer}
		\node [style=none] (0) at (2, 0) {};
		\node [style=none] (1) at (1.75, -0.75) {};
		\node [style=none] (2) at (2.25, -0.75) {};
		\node [style=none] (3) at (2, 0.5) {};
		\node [style=none] (4) at (2.25, -1) {};
		\node [style=X] (5) at (1.75, -0.75) {};
		\node [style=andin] (6) at (2, 0) {};
	\end{pgfonlayer}
	\begin{pgfonlayer}{edgelayer}
		\draw (0.center) to (3.center);
		\draw [in=90, out=-45] (0.center) to (2.center);
		\draw (4.center) to (2.center);
		\draw [in=-135, out=90] (1.center) to (0.center);
	\end{pgfonlayer}
\end{tikzpicture}
\eqzxa{ring.unit}
\begin{tikzpicture}
	\begin{pgfonlayer}{nodelayer}
		\node [style=none] (12) at (2, 0.5) {};
		\node [style=none] (14) at (2, -1) {};
		\node [style=X] (15) at (2, 0) {};
		\node [style=Z] (16) at (2, -0.5) {};
	\end{pgfonlayer}
	\begin{pgfonlayer}{edgelayer}
		\draw (15) to (12.center);
		\draw (16) to (14.center);
	\end{pgfonlayer}
\end{tikzpicture}
$$

\end{definition}


\begin{lemma} \cite[Thm. 10]{lafont}
$\f_2$ is a presentation for the prop $(\FSets_2,\times)$.
\end{lemma}


Therefore in some sense, we are justified in thinking of this prop $(\FSets_2,\times)$ as a sort of categorification of boolean polynomials.

%\begin{proof}
%The generators have the following interpretations in $\Sets_2$;  $X$ corresponds to addition:
%
%$$
%|a,b\rangle \xmapsto{\llbracket \mu_X \rrbracket} | a+b\rangle
%\hspace*{.5cm}
%* \xmapsto{\llbracket \mu_X \rrbracket} | 0 \rangle
%$$
%
%$\&$ corresponds to multiplication:
%
%$$
%|a,b\rangle \xmapsto{\llbracket \mu_\& \rrbracket} | a\cdot b\rangle
%\hspace*{.5cm}
%* \xmapsto{\llbracket\mu_X \rrbracket} | 1 \rangle
%$$
%
%
%The two bicommutative bialgebra laws correspond to the commutation of multiplication and addition with copying; and the monad map corresponds to the fact that multiplication distributes over addition.  Therefore, this is just the presentation of the ring $\Z_2$ as a Lawvere theory. 
%\end{proof}


To find larger fragments, it will be useful to first identify the isomorphisms and the monics of $\f_2$.


\begin{definition}
Given a map $f$ in  $\f_2$, the {\bf oracle} for $f$, ${\mathcal O}_f$ is defined as follows:
$$
\begin{tikzpicture}
	\begin{pgfonlayer}{nodelayer}
		\node [style=Z] (0) at (0.75, 0.25) {};
		\node [style=X] (1) at (1.5, 2.25) {};
		\node [style=map] (2) at (1, 1.5) {$f$};
		\node [style=none] (3) at (0.5, 2.75) {};
		\node [style=none] (4) at (1.5, 2.75) {};
		\node [style=none] (5) at (1.5, -0.25) {};
		\node [style=none] (6) at (0.75, -0.25) {};
		\node [style=Z] (7) at (-0.25, 0.25) {};
		\node [style=none] (8) at (-0.5, 2.75) {};
		\node [style=none] (9) at (-0.25, -0.25) {};
		\node [style=none] (10) at (0.25, 0) {$\cdots$};
		\node [style=none] (11) at (0, 2.5) {$\cdots$};
	\end{pgfonlayer}
	\begin{pgfonlayer}{edgelayer}
		\draw (6.center) to (0);
		\draw [in=-60, out=60] (0) to (2);
		\draw [in=-120, out=90] (2) to (1);
		\draw (1) to (4.center);
		\draw [in=90, out=-60] (1) to (5.center);
		\draw [in=-90, out=120] (0) to (3.center);
		\draw (9.center) to (7);
		\draw [in=-90, out=120] (7) to (8.center);
		\draw [in=45, out=-120] (2) to (7);
	\end{pgfonlayer}
\end{tikzpicture}
$$

\end{definition}

\begin{lemma}
The oracles in $f_2$ are generated by the generalized controlled-not gates:
$$
\begin{tikzpicture}
	\begin{pgfonlayer}{nodelayer}
		\node [style=none] (0) at (1, -0.75) {};
		\node [style=X] (1) at (0.5, -0.75) {$1$};
		\node [style=none] (2) at (0.75, 0.75) {};
		\node [style=X] (3) at (0.75, 0) {};
		\node [style=none] (4) at (1, -1.25) {};
	\end{pgfonlayer}
	\begin{pgfonlayer}{edgelayer}
		\draw [in=-45, out=90, looseness=0.75] (0.center) to (3);
		\draw [in=90, out=-135, looseness=0.75] (3) to (1);
		\draw (3) to (2.center);
		\draw (4.center) to (0.center);
	\end{pgfonlayer}
\end{tikzpicture},
\hspace*{.5cm}
\begin{tikzpicture}[xscale=-1]
	\begin{pgfonlayer}{nodelayer}
		\node [style=X] (0) at (0.75, 0) {};
		\node [style=Z] (1) at (1.25, -0.5) {};
		\node [style=none] (2) at (0.5, -1) {};
		\node [style=none] (3) at (1.25, -1) {};
		\node [style=none] (4) at (1.5, 0.5) {};
		\node [style=none] (5) at (0.75, 0.5) {};
	\end{pgfonlayer}
	\begin{pgfonlayer}{edgelayer}
		\draw (5.center) to (0);
		\draw [in=150, out=-30] (0) to (1);
		\draw [in=-90, out=60, looseness=0.75] (1) to (4.center);
		\draw (1) to (3.center);
		\draw [in=90, out=-120, looseness=0.75] (0) to (2.center);
	\end{pgfonlayer}
\end{tikzpicture},
\hspace*{.5cm}
\begin{tikzpicture}
	\begin{pgfonlayer}{nodelayer}
		\node [style=Z] (0) at (-10.25, 0.25) {};
		\node [style=Z] (1) at (-11.25, 0.25) {};
		\node [style=none] (2) at (-10.75, 1) {};
		\node [style=X] (3) at (-9.75, 1.75) {};
		\node [style=none] (4) at (-11.25, -0.5) {};
		\node [style=none] (5) at (-10.25, -0.5) {};
		\node [style=none] (6) at (-9.75, -0.5) {};
		\node [style=none] (7) at (-9.75, 2.25) {};
		\node [style=none] (8) at (-10.25, 2.25) {};
		\node [style=none] (9) at (-11.25, 2.25) {};
		\node [style=andin] (10) at (-10.75, 1) {};
		\node [style=none] (11) at (-10.75, 2.25) {$n$};
		\node [style=none] (12) at (-10.75, 0.25) {$n$};
		\node [style=none] (13) at (-10.75, 2) {$\cdots$};
		\node [style=none] (14) at (-10.75, 0.5) {$\cdots$};
	\end{pgfonlayer}
	\begin{pgfonlayer}{edgelayer}
		\draw (4.center) to (1);
		\draw (1) to (2.center);
		\draw (2.center) to (0);
		\draw (0) to (5.center);
		\draw (6.center) to (3);
		\draw [in=90, out=-146, looseness=1.50] (3) to (2.center);
		\draw [in=-90, out=120, looseness=1.00] (1) to (9.center);
		\draw [in=-90, out=60, looseness=0.75] (0) to (8.center);
		\draw (3) to (7.center);
	\end{pgfonlayer}
\end{tikzpicture}
$$




%and the equations of Iwama et al, where $[n,X]$ denotes an $|X|$-controlled not gate controlled by the wires indexed by the set X, and targetting the wire $n \notin X$  \cite{iwama} generalized b

%https://web.eecs.umich.edu/~imarkov/pubs/jour/tcad03-iwls.pdf

%\begin{description}
%\item $[x,X][x,X]=1$
%\item  When the target wire are the same $[x,X][x,Y] = [x,Y] [x,X]$
%\item  When $x \not\in Y$ and $y \not\in X$ then $[x,X][y,Y] = [y,Y] [x,X]$
%\item  $[x,X] [y,{\{ x\} \sqcup Y}] = [y,{X \cup Y}][y,{\{ x\} \sqcup Y}] [x,X]$
%\item For $x \neq y$, $[x,] $
%\end{description}
%
%
%TODO


\end{lemma}


\begin{proof}
Any  Boolean function of $n$ arguments can be represented by a polynomial in\\
 $\F_2[x_1,\ldots, x_n]/\langle x_1^2-x_1,\ldots x_n^2-x_n\rangle$.  Every polynomial in this quotient ring has a unique normal form given by sums of products (which is not true for arbitrary finite fields).  Each product corresponds to a generalized controlled-not gate, and the sum corresponds to composing these generalized controlled-not gates in sequence.
\end{proof}

In the quantum circuit notation, the generalized controlled-not gates are drawn as follows (the first being the not gate, and the second being the controlled-not gate):
$$
\begin{tikzpicture}
	\begin{pgfonlayer}{nodelayer}
		\node [style=oplus] (0) at (0, 1.5) {};
		\node [style=none] (1) at (0, 2) {};
		\node [style=none] (2) at (0, 1) {};
	\end{pgfonlayer}
	\begin{pgfonlayer}{edgelayer}
		\draw (0) to (1.center);
		\draw (0) to (2.center);
	\end{pgfonlayer}
\end{tikzpicture}
,
\hspace*{.5cm}
\begin{tikzpicture}
	\begin{pgfonlayer}{nodelayer}
		\node [style=oplus] (0) at (0, 1.5) {};
		\node [style=none] (1) at (0, 2) {};
		\node [style=none] (2) at (0, 1) {};
		\node [style=none] (4) at (-0.5, 2) {};
		\node [style=none] (5) at (-0.5, 1) {};
		\node [style=dot] (6) at (-0.5, 1.5) {};
	\end{pgfonlayer}
	\begin{pgfonlayer}{edgelayer}
		\draw (0) to (1.center);
		\draw (0) to (2.center);
		\draw (0) to (6);
		\draw (6) to (4.center);
		\draw (6) to (5.center);
	\end{pgfonlayer}
\end{tikzpicture}
,
\hspace*{.5cm}
\begin{tikzpicture}
	\begin{pgfonlayer}{nodelayer}
		\node [style=oplus] (0) at (-0.25, 1.5) {};
		\node [style=none] (1) at (-0.25, 2) {};
		\node [style=none] (2) at (-0.25, 1) {};
		\node [style=none] (4) at (-0.75, 2) {};
		\node [style=none] (5) at (-0.75, 1) {};
		\node [style=dot] (6) at (-0.75, 1.5) {};
		\node [style=none] (7) at (-1.75, 2) {};
		\node [style=none] (8) at (-1.75, 1) {};
		\node [style=dot] (9) at (-1.75, 1.5) {};
		\node [style=none] (10) at (-1, 1.5) {};
		\node [style=none] (11) at (-1.5, 1.5) {};
		\node [style=none] (12) at (-1.25, 1.5) {$\cdots$};
		\node [style=none] (13) at (-1.25, 1.75) {$n$};
	\end{pgfonlayer}
	\begin{pgfonlayer}{edgelayer}
		\draw (0) to (1.center);
		\draw (0) to (2.center);
		\draw (0) to (6);
		\draw (6) to (4.center);
		\draw (6) to (5.center);
		\draw (9) to (7.center);
		\draw (9) to (8.center);
		\draw (11.center) to (9);
		\draw (10.center) to (6);
	\end{pgfonlayer}
\end{tikzpicture}
$$


%We will also allow generalized controlled not gates controlled and targetting arbitrary wires, possibly with gaps in the middle.




\begin{lemma}\cite[Thm. 5.1]{toffolireversible}
The prop generated by the oracles in $\f_2$ generate $\Iso(\f_2)$.
\end{lemma}

%This actually follows from https://arxiv.org/pdf/quant-ph/0207001.pdf
%Scratch space is used to construct n-bit cnot gate
%http://theory.caltech.edu/~preskill/ph229/notes/chap6.pdf

Denote a generalized controlled not gate controlled by wires indexed by $X$, operating on $x$ by $\lbparen X,x\rbparen$.


%
%
%Iwama et al,  give a set of identities which are complete for oracles, not general isomorphisms \cite{iwama}.
%Shende restated these identities using the commutator \cite{shende}.






Iwama et al \cite{iwama} originally gave a complete set of identities for circuits generated by generalized controlled not gates where the value of all-but-one output wires are fixed.  It is worth mentioning that Shende et al. later used the commutator to generalize some of these identities \cite[Cor. 26]{shende}.  We conjecture that a very similar set of identities is complete for Boolean isomorphisms: 
%Recall that the {\bf commutator} of two group elements $f,g$ is the element $[f,g]:=fgf^{-1}g^{-1}$; therefore $fg=[f,g]gf$.


%Shende gives a way to compute commutators of controlled isomorphsisms in $\Iso(\FSets_2)$.  In particular, take isomorphisms $f,g$  which only change bits in sets indexed by $X,Y$, respectively.  Then if we control these gates by $Z$ and $W$ respectively, denoted by $f^Z,g^W$, we have :
%$$
%[f,g] = [f^{Y * Z}, g^{X*W}]^{\bar{(X\#Y)} * (W\#Z) }
%$$
%Where $X\# Y $ is the defined as the pointwise xor, and $X * Y$ is defined as the pointwise and.

\begin{conjecture}
%Denote a generalized controlled-not gate controlled from wires indexed by $X$ and operating on the wire $x$ by $\lbparen X,x \rbparen$.

Let  $\Iso(\FSets_2)$ denote the prop generated by all generalized controlled-not gates modulo the following identities:

\begin{itemize}
\item $\lbparen X,x\rbparen ; \lbparen X,x \rbparen= 1$ 
%So the first identity is that  in this case $\lbparen X,x\rbparen\lbparen Y,y\rbparen=\lbparen Y,y\rbparen\lbparen X,x\rbparen$


\item
If  $x \notin Y $ and $ y \notin X$ then $\lbparen X,x\rbparen ;\lbparen Y,y\rbparen =\lbparen Y,y\rbparen; \lbparen X,x\rbparen $.


%
%So the second identity is that in this case:
%$$
%\lbparen  X,x\rbparen  \lbparen  Y ,y\rbparen = \lbparen \lbparen  X \cup Y -y,x\rbparen \rbparen  \lbparen  Y ,y\rbparen  \lbparen  X,x\rbparen  
%$$

\item
If $x \notin Y$, then $\lbparen X,x\rbparen; \lbparen \{x\} \sqcup Y, y\rbparen = \lbparen X\cup Y,y\rbparen ; \lbparen \{x\} \sqcup Y, y\rbparen;  \lbparen X,x\rbparen $.

\item
If $x \notin Y$, then $ \lbparen \{x\} \sqcup Y, y\rbparen ; \lbparen X,x\rbparen = \lbparen X,x\rbparen;   \lbparen \{x\} \sqcup Y, y\rbparen ; \lbparen X\cup Y,y\rbparen $.


%So the third identity is that $\lbparen  X,x\rbparen ^2= 1 $.

\item
If $x \in Y$ and $y \in X$, then
$
\lbparen  X,x \rbparen ; \lbparen  Y,y \rbparen ;  \lbparen  X,x \rbparen 
=
\lbparen  Y,y \rbparen ;  \lbparen  X,x \rbparen ;  \lbparen  Y,y \rbparen 
$.



\end{itemize}

\end{conjecture}



Note that the braid is derived in this fragment by composing 3 controlled not gates, as in Definition \cite{def:isoaff}.  The axioms of a prop are derived, so we are justified in calling $\Iso(\f_2)$ a prop.


Although we aren't sure if these identities are complete, it doesn't matter in the end.  With each generator we add, we add new enough identities to give a complete presentation, given that there is a complete presentation for $\Iso(\f_2)$.  However, eventually once we add enough generators and identities, we get a finite, complete presentation.

\begin{definition}
Let $\inj(\f_2)$ be the prop given by adjoining the black unit to $\Iso(\f_2)$ modulo:

$$
\begin{tikzpicture}
	\begin{pgfonlayer}{nodelayer}
		\node [style=oplus] (0) at (2, 1.5) {};
		\node [style=none] (1) at (2, 2.25) {};
		\node [style=none] (2) at (2, 0.75) {};
		\node [style=none] (3) at (1.5, 2.25) {};
		\node [style=none] (4) at (1.5, 0.75) {};
		\node [style=dot] (5) at (1.5, 1.5) {};
		\node [style=none] (6) at (0.5, 2.25) {};
		\node [style=dot] (7) at (0.5, 1.5) {};
		\node [style=none] (8) at (1.25, 1.5) {};
		\node [style=none] (9) at (0.75, 1.5) {};
		\node [style=none] (10) at (1, 1.5) {$\cdots$};
		\node [style=none] (11) at (1, 1.75) {$n$};
		\node [style=none] (13) at (0, 2.25) {};
		\node [style=dot] (14) at (0, 1.5) {};
		\node [style=X] (15) at (0, 1) {};
		\node [style=none] (16) at (0.5, 0.75) {};
	\end{pgfonlayer}
	\begin{pgfonlayer}{edgelayer}
		\draw (0) to (1.center);
		\draw (0) to (2.center);
		\draw (0) to (5);
		\draw (5) to (3.center);
		\draw (5) to (4.center);
		\draw (7) to (6.center);
		\draw (9.center) to (7);
		\draw (8.center) to (5);
		\draw (14) to (13.center);
		\draw (15) to (14);
		\draw (16.center) to (7);
		\draw (7) to (14);
	\end{pgfonlayer}
\end{tikzpicture}
\eqzxa{mono.ftwo}
\begin{tikzpicture}
	\begin{pgfonlayer}{nodelayer}
		\node [style=none] (1) at (2, 2.25) {};
		\node [style=none] (2) at (2, 0.75) {};
		\node [style=none] (3) at (1.5, 2.25) {};
		\node [style=none] (4) at (1.5, 0.75) {};
		\node [style=none] (6) at (0.5, 2.25) {};
		\node [style=none] (10) at (1, 1.5) {$\cdots$};
		\node [style=none] (11) at (1, 1.75) {$n$};
		\node [style=none] (13) at (0, 2.25) {};
		\node [style=X] (15) at (0, 1) {};
		\node [style=none] (16) at (0.5, 0.75) {};
	\end{pgfonlayer}
	\begin{pgfonlayer}{edgelayer}
		\draw (16.center) to (6.center);
		\draw (13.center) to (15);
		\draw (4.center) to (3.center);
		\draw (1.center) to (2.center);
	\end{pgfonlayer}
\end{tikzpicture}
$$
\end{definition}

\begin{lemma}
\label{lem:injand}
$\inj(\f_2)$ is a presentation for the prop $(\inj(\FSets_2),\times)$.
\end{lemma}

%Note, however, that the monoidal theory for $\inj(\f_2)$ is dependant on that of $\iso(\f_2)$ the identities of which are conjectured, even though we only need one extra identity to get injections.

The pushout of a diagram of sets and functions $2^n \xleftarrowtail{}\  2^k \ \xrightarrowtail{} 2^m$ is not always a power of 2.  Therefore, one should not expect to construct categories of partial isomorphisms via a distributive law  $\inj(\f_2)\otimes_{\Iso(\f_2)} \inj(\f_2)^\op$. Instead one must add all of the nontrivial subobjects to the constituent props forming the distributive law; as opposed to the affine case, there are more than one such subobjects which arise in this way.

\begin{definition}
Consider the pro $\sub_2$ generated by endomorphisms such that for any $n$, $\sub_2(n,n)$ is the set described by all $n$-variable polynomials over $\F_2$.  Denote such a generator by a box with $n$ inputs and $n$ outputs labelled by the corresponding polynomial.

We require that the following equations hold so that
$$\forall n,m \in \N, p,r \in \F_2[x_1,\ldots, x_n],  q \in \F_2[x_{n+1},\ldots, x_{n+m}]:\hspace*{1cm}
$$
$$
\begin{tikzpicture}
	\begin{pgfonlayer}{nodelayer}
		\node [style=none] (0) at (3, 2.25) {};
		\node [style=none] (1) at (3, 3.25) {};
		\node [style=map] (2) at (3, 2.75) {$1$};
		\node [style=none] (6) at (3, 3.5) {$n$};
		\node [style=none] (8) at (3, 2) {$n$};
	\end{pgfonlayer}
	\begin{pgfonlayer}{edgelayer}
		\draw (0.center) to (1.center);
	\end{pgfonlayer}
\end{tikzpicture}
\eqzxa{sub.one}
\begin{tikzpicture}
	\begin{pgfonlayer}{nodelayer}
		\node [style=none] (0) at (3, 2.25) {};
		\node [style=none] (1) at (3, 3.25) {};
		\node [style=none] (6) at (3, 3.5) {$n$};
		\node [style=none] (8) at (3, 2) {$n$};
	\end{pgfonlayer}
	\begin{pgfonlayer}{edgelayer}
		\draw (0.center) to (1.center);
	\end{pgfonlayer}
\end{tikzpicture}
\hspace*{,5cm}
\begin{tikzpicture}
	\begin{pgfonlayer}{nodelayer}
		\node [style=none] (0) at (5, 2) {};
		\node [style=none] (1) at (5, 4) {};
		\node [style=map] (2) at (5, 2.5) {$r$};
		\node [style=map] (3) at (5, 3.5) {$p$};
		\node [style=none] (4) at (5, 4.25) {$n$};
		\node [style=none] (5) at (5, 1.75) {$n$};
	\end{pgfonlayer}
	\begin{pgfonlayer}{edgelayer}
		\draw (0.center) to (1.center);
	\end{pgfonlayer}
\end{tikzpicture}
\eqzxa{sub.two}
\begin{tikzpicture}
	\begin{pgfonlayer}{nodelayer}
		\node [style=none] (0) at (5, 2.25) {};
		\node [style=none] (1) at (5, 4.25) {};
		\node [style=map] (2) at (5, 3.25) {$p+r+pr$};
		\node [style=none] (4) at (5, 4.5) {$n$};
		\node [style=none] (5) at (5, 2) {$n$};
	\end{pgfonlayer}
	\begin{pgfonlayer}{edgelayer}
		\draw (0.center) to (1.center);
	\end{pgfonlayer}
\end{tikzpicture}
\hspace*{.5cm}
\begin{tikzpicture}
	\begin{pgfonlayer}{nodelayer}
		\node [style=none] (0) at (2.4, 2.25) {};
		\node [style=none] (1) at (2.4, 3.25) {};
		\node [style=map] (2) at (2.4, 2.75) {$p$};
		\node [style=none] (3) at (3, 3.25) {};
		\node [style=none] (4) at (3, 2.25) {};
		\node [style=map] (5) at (3, 2.75) {$q$};
		\node [style=none] (6) at (2.4, 3.5) {$n$};
		\node [style=none] (7) at (3, 3.5) {$m$};
		\node [style=none] (8) at (2.4, 2) {$n$};
		\node [style=none] (9) at (3, 2) {$m$};
	\end{pgfonlayer}
	\begin{pgfonlayer}{edgelayer}
		\draw (0.center) to (1.center);
		\draw (3.center) to (4.center);
	\end{pgfonlayer}
\end{tikzpicture}
\eqzxa{sub.three}
\begin{tikzpicture}
	\begin{pgfonlayer}{nodelayer}
		\node [style=none] (0) at (2.5, 2.25) {};
		\node [style=none] (1) at (2.5, 3.25) {};
		\node [style=map] (2) at (2.75, 2.75) {$p\cdot q$};
		\node [style=none] (3) at (3, 3.25) {};
		\node [style=none] (4) at (3, 2.25) {};
		\node [style=none] (6) at (2.5, 3.5) {$n$};
		\node [style=none] (7) at (3, 3.5) {$m$};
		\node [style=none] (8) at (2.5, 2) {$n$};
		\node [style=none] (9) at (3, 2) {$m$};
	\end{pgfonlayer}
	\begin{pgfonlayer}{edgelayer}
		\draw (0.center) to (1.center);
		\draw (3.center) to (4.center);
	\end{pgfonlayer}
\end{tikzpicture}
$$
As well as, for all $n$, the equations of the quotient rings  
$$\F_2[x_1,\ldots, x_n]/\langle x_1^2-x_1,\ldots, x_n^2-x_n \rangle$$

\end{definition}


\begin{lemma}
\label{lem:sub}
$\sub_2$ is a presentation for the pro of symmetric spans of monic functions, ie spans of the following form $2^n \xleftarrow{e} k \xrightarrow{e}2^n$, for all $n,k \in \N$ and monics $e$.
\end{lemma}



\begin{proof}
Each polynomial  $p \in \F_2[x_1,\ldots, x_n]/\langle x_1^2-x_1,\ldots, x_n^2-x_n \rangle$ corresponds to a function $\ev_p:\Z_2^n \to \Z_2$ given by evaluation.  Let $k = |\ev^{-1}(1)|$, then there chose a function $f_p:k \rightarrowtail 2^n$ picking out all the solutions which evaluate to $1$. The functor from $\sub_2$ to this subcategory spans takes polynomials $p \mapsto (2^n \xleftarrowtail{f_p} \ k \ \xrightarrowtail {f_p} 2^n)$.  Any two spans induced by the same polynomial are isomorphic, so this is actually well defined.  It is clearly an isomorphism on objects, and it can easily be shown to be a monoidal functor.

The fullness is easy and the faithfulness comes from the fact that we can reduce every map to a polynomial and then reduce the polynomial to algebraic normal form.

\end{proof}


%
%The first axiom enforces that the trivial polynomial is the identity.
%The second axiom allows one to perform elementary row operations on polynomials.
%The third axiom allows one to multiply all polynomials.
%Because every map $f$ in $\sub_2(n,n)$ has precisely one representative polynomial, and polynomials have a normal form via their reduction, and row reduction is confluent, completeness is immediate.




\begin{definition}
Let $\sub\Iso\f_2$ be the prop generated by a distributive law of pros:
$$
\sub_2 \otimes \Iso(\f_2);
$$
$$
 \forall n,m,k \in \N, \forall p \in \F_2[x_1,\ldots, x_{n+2+m}],
q \in \F_2[x_1,\ldots,x_{n+m+1+k}],
r \in \F_2[x_1,\ldots, x_n]:
$$
$$
\begin{tikzpicture}
	\begin{pgfonlayer}{nodelayer}
		\node [style=map] (0) at (5, 3.5) {$p(x_1,\ldots, x_n, x_{n+1}, x_{n+2}, x_{n+3},\ldots, x_{n+2+m})$};
		\node [style=none] (1) at (4.5, 4.25) {};
		\node [style=none] (2) at (5.5, 4.25) {};
		\node [style=none] (3) at (4.75, 4.25) {};
		\node [style=none] (4) at (5.25, 4.25) {};
		\node [style=none] (5) at (4.5, 2.25) {};
		\node [style=none] (6) at (5.5, 2.25) {};
		\node [style=none] (7) at (4.75, 2.75) {};
		\node [style=none] (8) at (5.25, 2.75) {};
		\node [style=none] (9) at (4.5, 4.5) {$n$};
		\node [style=none] (10) at (4.5, 2) {$n$};
		\node [style=none] (11) at (5.5, 4.5) {$m$};
		\node [style=none] (12) at (5.5, 2) {$m$};
		\node [style=none] (13) at (5.25, 2.25) {};
		\node [style=none] (14) at (4.75, 2.25) {};
	\end{pgfonlayer}
	\begin{pgfonlayer}{edgelayer}
		\draw [in=120, out=-90] (1.center) to (0);
		\draw [in=-90, out=60] (0) to (2.center);
		\draw [in=75, out=-90, looseness=0.75] (4.center) to (0);
		\draw [in=-90, out=105, looseness=0.75] (0) to (3.center);
		\draw [in=300, out=90] (6.center) to (0);
		\draw [in=90, out=-75, looseness=0.75] (0) to (8.center);
		\draw [in=255, out=90, looseness=0.75] (7.center) to (0);
		\draw [in=90, out=-120] (0) to (5.center);
		\draw [in=270, out=90] (13.center) to (7.center);
		\draw [in=270, out=90] (14.center) to (8.center);
	\end{pgfonlayer}
\end{tikzpicture}
\eqzxa{subiso.one}
\begin{tikzpicture}
	\begin{pgfonlayer}{nodelayer}
		\node [style=map] (0) at (5, 3.25) {$p(x_1,\ldots, x_n, x_{n+2}, x_{n+1}, x_{n+3},\ldots, x_{n+2+m})$};
		\node [style=none] (1) at (4.5, 2.5) {};
		\node [style=none] (2) at (5.5, 2.5) {};
		\node [style=none] (3) at (4.75, 2.5) {};
		\node [style=none] (4) at (5.25, 2.5) {};
		\node [style=none] (5) at (4.5, 4.5) {};
		\node [style=none] (6) at (5.5, 4.5) {};
		\node [style=none] (7) at (4.75, 4) {};
		\node [style=none] (8) at (5.25, 4) {};
		\node [style=none] (9) at (4.5, 2.25) {$n$};
		\node [style=none] (10) at (4.5, 4.75) {$n$};
		\node [style=none] (11) at (5.5, 2.25) {$m$};
		\node [style=none] (12) at (5.5, 4.75) {$m$};
		\node [style=none] (13) at (5.25, 4.5) {};
		\node [style=none] (14) at (4.75, 4.5) {};
	\end{pgfonlayer}
	\begin{pgfonlayer}{edgelayer}
		\draw [in=-120, out=90] (1.center) to (0);
		\draw [in=90, out=-60] (0) to (2.center);
		\draw [in=-75, out=90, looseness=0.75] (4.center) to (0);
		\draw [in=90, out=-105, looseness=0.75] (0) to (3.center);
		\draw [in=-300, out=-90] (6.center) to (0);
		\draw [in=-90, out=75, looseness=0.75] (0) to (8.center);
		\draw [in=-255, out=-90, looseness=0.75] (7.center) to (0);
		\draw [in=-90, out=120] (0) to (5.center);
		\draw [in=-270, out=-90] (13.center) to (7.center);
		\draw [in=-270, out=-90] (14.center) to (8.center);
	\end{pgfonlayer}
\end{tikzpicture}
$$
$$
\begin{tikzpicture}
	\begin{pgfonlayer}{nodelayer}
		\node [style=map] (0) at (5, 3.5) {$q(x_1,\ldots, x_{n+m+1+k})$};
		\node [style=none] (1) at (3.75, 4) {};
		\node [style=none] (2) at (6.25, 4) {};
		\node [style=none] (3) at (4.25, 4) {};
		\node [style=none] (4) at (5.75, 4) {};
		\node [style=none] (5) at (3.75, 3) {};
		\node [style=none] (6) at (6.25, 3) {};
		\node [style=none] (7) at (4.25, 3) {};
		\node [style=none] (8) at (5.75, 3) {};
		\node [style=none] (9) at (3.75, 2.25) {$n$};
		\node [style=none] (10) at (4.25, 2.5) {};
		\node [style=none] (11) at (5.75, 2.5) {};
		\node [style=dot] (12) at (4.25, 2.75) {};
		\node [style=oplus] (13) at (5.75, 2.75) {};
		\node [style=none] (14) at (3.75, 2.5) {};
		\node [style=none] (15) at (6.25, 2.5) {};
		\node [style=none] (16) at (5.25, 4) {};
		\node [style=dot] (17) at (5.25, 2.75) {};
		\node [style=none] (18) at (5.25, 2.5) {};
		\node [style=none] (19) at (6.25, 2.25) {$k$};
		\node [style=none] (20) at (4.75, 2.5) {$m$};
		\node [style=none] (21) at (3.75, 5) {$n$};
		\node [style=none] (22) at (6.25, 5) {$k$};
		\node [style=none] (23) at (4.75, 4) {$m$};
		\node [style=none] (24) at (4.25, 4.5) {};
		\node [style=none] (25) at (5.75, 4.5) {};
		\node [style=none] (26) at (3.75, 4.5) {};
		\node [style=none] (27) at (6.25, 4.5) {};
		\node [style=none] (28) at (5.25, 4.5) {};
	\end{pgfonlayer}
	\begin{pgfonlayer}{edgelayer}
		\draw [in=270, out=90] (10.center) to (7.center);
		\draw [in=270, out=90] (11.center) to (8.center);
		\draw (15.center) to (6.center);
		\draw (14.center) to (5.center);
		\draw (13) to (17);
		\draw [style=dotted] (17) to (12);
		\draw (18.center) to (17);
		\draw (1.center) to (26.center);
		\draw (3.center) to (24.center);
		\draw (16.center) to (28.center);
		\draw (4.center) to (25.center);
		\draw (2.center) to (27.center);
		\draw (1.center) to (5.center);
		\draw (12) to (3.center);
		\draw (16.center) to (17);
		\draw (13) to (4.center);
		\draw (2.center) to (6.center);
	\end{pgfonlayer}
\end{tikzpicture}
\eqzxa{subiso.two}
\begin{tikzpicture}
	\begin{pgfonlayer}{nodelayer}
		\node [style=map] (0) at (5, 3) {$q(x_1,\ldots, x_{n+m}, (x_{n+1}\ldots x_{n+m-1})+x_{n+m+1}, x_{n+m+2}, \ldots, x_{n+m+1+k})$};
		\node [style=none] (1) at (3.75, 2.5) {};
		\node [style=none] (2) at (6.25, 2.5) {};
		\node [style=none] (3) at (4.25, 2.5) {};
		\node [style=none] (4) at (5.75, 2.5) {};
		\node [style=none] (5) at (3.75, 3.5) {};
		\node [style=none] (6) at (6.25, 3.5) {};
		\node [style=none] (7) at (4.25, 3.5) {};
		\node [style=none] (8) at (5.75, 3.5) {};
		\node [style=none] (9) at (3.75, 4.25) {$n$};
		\node [style=none] (10) at (4.25, 4) {};
		\node [style=none] (11) at (5.75, 4) {};
		\node [style=dot] (12) at (4.25, 3.75) {};
		\node [style=oplus] (13) at (5.75, 3.75) {};
		\node [style=none] (14) at (3.75, 4) {};
		\node [style=none] (15) at (6.25, 4) {};
		\node [style=none] (16) at (5.25, 2.5) {};
		\node [style=dot] (17) at (5.25, 3.75) {};
		\node [style=none] (18) at (5.25, 4) {};
		\node [style=none] (19) at (6.25, 4.25) {$k$};
		\node [style=none] (20) at (4.75, 4) {$m$};
		\node [style=none] (21) at (3.75, 1.75) {$n$};
		\node [style=none] (22) at (6.25, 1.75) {$k$};
		\node [style=none] (23) at (4.75, 2.5) {$m$};
		\node [style=none] (24) at (4.25, 2) {};
		\node [style=none] (25) at (5.75, 2) {};
		\node [style=none] (26) at (3.75, 2) {};
		\node [style=none] (27) at (6.25, 2) {};
		\node [style=none] (28) at (5.25, 2) {};
	\end{pgfonlayer}
	\begin{pgfonlayer}{edgelayer}
		\draw [in=-270, out=-90] (10.center) to (7.center);
		\draw [in=-270, out=-90] (11.center) to (8.center);
		\draw (15.center) to (6.center);
		\draw (14.center) to (5.center);
		\draw (13) to (17);
		\draw [style=dotted] (17) to (12);
		\draw (18.center) to (17);
		\draw (26.center) to (1.center);
		\draw (24.center) to (3.center);
		\draw (28.center) to (16.center);
		\draw (25.center) to (4.center);
		\draw (27.center) to (2.center);
		\draw (1.center) to (5.center);
		\draw (3.center) to (12);
		\draw (16.center) to (17);
		\draw (4.center) to (13);
		\draw (2.center) to (6.center);
	\end{pgfonlayer}
\end{tikzpicture}
$$
$$
\begin{tikzpicture}
	\begin{pgfonlayer}{nodelayer}
		\node [style=none] (5) at (3.25, 4.25) {};
		\node [style=none] (6) at (3.25, 3.25) {};
		\node [style=map] (9) at (3.25, 3.75) {$r$};
		\node [style=none] (10) at (3.25, 2.25) {};
		\node [style=none] (11) at (3.75, 4.25) {};
		\node [style=none] (12) at (3.75, 2.25) {};
		\node [style=none] (13) at (3.75, 3.25) {};
	\end{pgfonlayer}
	\begin{pgfonlayer}{edgelayer}
		\draw (5.center) to (6.center);
		\draw [in=90, out=-90] (13.center) to (10.center);
		\draw [in=-90, out=90] (12.center) to (6.center);
		\draw (13.center) to (11.center);
	\end{pgfonlayer}
\end{tikzpicture}
\eqzxa{subiso.three}
\begin{tikzpicture}
	\begin{pgfonlayer}{nodelayer}
		\node [style=none] (5) at (3.75, 2.25) {};
		\node [style=none] (6) at (3.75, 3.25) {};
		\node [style=map] (9) at (3.75, 2.75) {$r$};
		\node [style=none] (10) at (3.75, 4.25) {};
		\node [style=none] (11) at (3.25, 2.25) {};
		\node [style=none] (12) at (3.25, 4.25) {};
		\node [style=none] (13) at (3.25, 3.25) {};
	\end{pgfonlayer}
	\begin{pgfonlayer}{edgelayer}
		\draw (5.center) to (6.center);
		\draw [in=270, out=90] (13.center) to (10.center);
		\draw [in=90, out=-90] (12.center) to (6.center);
		\draw (13.center) to (11.center);
	\end{pgfonlayer}
\end{tikzpicture}
$$


\end{definition}

\begin{lemma}
\label{lem:subiso}
$\sub\Iso\f_2$ is a presentation for the subcategory of $(\Span^\sim(\FSets),\times)$ generated by spans of the form $2^n \xleftarrowtail{e} \ k \ \xrightarrowtail {e} 2^m \xrightarrow[\cong]{f} 2^m$, for all $n,m k \in \N$ and all isomorphisms $f$ and monics $e$.
\end{lemma}


\begin{proof}
The obvious functor is clearly monoidal. Moreover, it is full by construction.
For the faithfulness, take two maps $f$ and $g$ in $\sub\Iso\f_2$.  Then one can just push everything to the end and then use the decidability of equality on both factors of the distributive law to show that they are equal.
\end{proof}

% $2^n \xleftarrow{e} k \xrightarrow{e} 2^n$, for all monics $e$ and isomorphisms of the form $2^n = 2^n \xrightarrow{\sim} 2^n$. 




\begin{definition}
Consider the prop $\sub\inj\f_2$ generated by a distributive law of props:
$$
 \sub\Iso\f_2 \otimes_{\Iso(\f_2)} \inj(\f_2);
 \forall n,m \in \N, p \in \F_2[x_1,\ldots, x_{n+1+m}]:
$$
$$
\begin{tikzpicture}
	\begin{pgfonlayer}{nodelayer}
		\node [style=none] (0) at (1.5, 3.5) {};
		\node [style=map] (1) at (2.5, 2.75) {$p(x_1,\ldots,x_{n+1+m})$};
		\node [style=none] (2) at (1.5, 3.75) {$n$};
		\node [style=none] (3) at (3.5, 3.5) {};
		\node [style=none] (4) at (3.5, 3.75) {$m$};
		\node [style=none] (5) at (1.5, 1.75) {};
		\node [style=none] (6) at (3.5, 1.75) {};
		\node [style=none] (7) at (1.5, 1.5) {$n$};
		\node [style=none] (8) at (3.5, 1.5) {$m$};
		\node [style=X] (9) at (2.5, 2) {};
		\node [style=none] (10) at (2.5, 3.5) {};
	\end{pgfonlayer}
	\begin{pgfonlayer}{edgelayer}
		\draw (0.center) to (5.center);
		\draw (3.center) to (6.center);
		\draw (10.center) to (9);
	\end{pgfonlayer}
\end{tikzpicture}
\eqzxa{subinj}
\begin{tikzpicture}
	\begin{pgfonlayer}{nodelayer}
		\node [style=none] (0) at (1.5, 3.75) {};
		\node [style=map] (1) at (2.5, 2.75) {$p(x_1,\ldots,x_n,0,x_{n+2},\ldots,x_{n+1+m})$};
		\node [style=none] (2) at (1.5, 4) {$n$};
		\node [style=none] (3) at (3.5, 3.75) {};
		\node [style=none] (4) at (3.5, 4) {$m$};
		\node [style=none] (5) at (1.5, 2) {};
		\node [style=none] (6) at (3.5, 2) {};
		\node [style=none] (7) at (1.5, 1.75) {$n$};
		\node [style=none] (8) at (3.5, 1.75) {$m$};
		\node [style=X] (9) at (2.5, 3.4) {};
		\node [style=none] (10) at (2.5, 3.75) {};
	\end{pgfonlayer}
	\begin{pgfonlayer}{edgelayer}
		\draw (0.center) to (5.center);
		\draw (3.center) to (6.center);
		\draw (10.center) to (9);
	\end{pgfonlayer}
\end{tikzpicture}
$$
\end{definition}



\begin{lemma}
\label{lem:subinj}
$\sub\inj\f_2$ is a presentation for the subcategory of $(\Span^\sim(\FSets),\times)$ generated by spans of the form $2^n \xleftarrowtail{e}  \ k \ \xrightarrowtail{e} 2^n \ \xrightarrowtail{e'} 2^{m}$ for all $n,m,k \in \N$ and all monics $e,e'$.
\end{lemma}

The proof is completely analogous to the case of $\sub\iso\f_2$.

Any $n$ variable polynomial $p$ can be interpreted as a span of monics via the oracle $\mathcal{O}_p$, where the value of the target wire is restricted to have the value $0$.  Each such polynomial corresponds to a subobject, which complicates the matter further than in the affine case.


\begin{definition}
Consider the prop $\pr\iso\f_2$ given by the distributive law of props:

$$
\sub\inj\f_2^\op \otimes_{\sub\Iso\f_2} \sub\inj\f_2;
\begin{tikzpicture}
	\begin{pgfonlayer}{nodelayer}
		\node [style=map] (0) at (0, 0) {$\mathcal{O}_p$};
		\node [style=none] (1) at (-0.25, -0.75) {};
		\node [style=none] (2) at (-0.25, 0.75) {};
		\node [style=none] (3) at (-0.25, 1) {};
		\node [style=none] (4) at (-0.25, -1) {};
		\node [style=X] (5) at (0.25, 0.75) {};
		\node [style=X] (6) at (0.25, -0.75) {};
	\end{pgfonlayer}
	\begin{pgfonlayer}{edgelayer}
		\draw (3.center) to (2.center);
		\draw (4.center) to (1.center);
		\draw [in=-60, out=90] (6) to (0);
		\draw [in=-90, out=60] (0) to (5);
		\draw [in=120, out=-90] (2.center) to (0);
		\draw [in=90, out=-120, looseness=1.25] (0) to (1.center);
	\end{pgfonlayer}
\end{tikzpicture}
\eqzxa{oracle}
\begin{tikzpicture}
	\begin{pgfonlayer}{nodelayer}
		\node [style=map] (0) at (-0.25, 0) {$p$};
		\node [style=none] (1) at (-0.25, -0.75) {};
		\node [style=none] (2) at (-0.25, 0.75) {};
		\node [style=none] (3) at (-0.25, 1) {};
		\node [style=none] (4) at (-0.25, -1) {};
	\end{pgfonlayer}
	\begin{pgfonlayer}{edgelayer}
		\draw (3.center) to (2.center);
		\draw (4.center) to (1.center);
		\draw (2.center) to (0);
		\draw (0) to (1.center);
	\end{pgfonlayer}
\end{tikzpicture}
$$

\end{definition}

%This is actually a distributive law, because it need only be witnessed by pushing $\eta_X$ past $\eta_X^\op$.  The only time this can't be done is when there is the target  of a generalized controlled-not not gate--or those of several generalized controlled-not gates is in between both of these generators.  In which case, the obstructing generalized controlled-not gates form an oracle for which we can apply this equation.  This is computing the apex of the span when performing a pullback.

%Note that $\pr\iso\f_2$ is not actually partial isomorphisms over $\f_2$, per se, but rather a full subcategory of partial isomorphisms of sets and functions, which is not iself a prop with respect to the Cartesian product.  It is because of such a complication that the aforementioned distributive law isn't quite a pullback.

\begin{lemma}
\label{lem:parisof}
$\pr\iso \f_2$ is a presentation for the full subcategory $(\FPinj_2,\times)$ of $(\ParIso(\FSets),\times)$ with objects powers of two.
\end{lemma}

Unlike the previous lemmas, this is not dependant on a complete presentation for the isomorphisms.
The proof is a consequence of \cite[Thm 7.6.14]{cole} where they give a finite, complete presentation for this category.  The identities up to this point are equivalent to this finite presentation, whether or not the conjectured presentation for the isomorphisms is complete.

%\begin{comment}
$\pr\iso \f_2$ can be presented in terms of finitely many generators and relations.  The identities are contained in \S \ref{subsubsec:presentations:three:pinj}.
%\end{comment}

\begin{definition}

Consider the prop $\pr\f_2$ given by the  pushout of the following diagram of props, given by adding a counit to the diagonal map:
$$\pr\iso\f_2 \leftarrow \surj^\op \rightarrow \cm^\op$$

\end{definition}


\begin{lemma}
\label{lem:parand}
$\pr\f_2$ is a presentation for the the full subcategory $(\FPar_2,\times)$ of $(\Par(\FSets),\times)$ with objects powers of two.
\end{lemma}

\begin{proof}
One has to show that the following diagram commutes:

\renewcommand{\cubetopbl}{$\surj^\op$}
\renewcommand{\cubetopbr}{$\cm^\op$}
\renewcommand{\cubetopfl}{$\pr\iso\f_2$}
\renewcommand{\cubetopfr}{$\pr\f_2$}
\renewcommand{\cubebotbl}{$\surj^\op$ }
\renewcommand{\cubebotbr}{$\cm^\op$ }
\renewcommand{\cubebotfl}{$(\FPinj_2,\times)$ }
\renewcommand{\cubebotfr}{}

$$
\xymatrixrowsep{2mm}\xymatrixcolsep{2mm}
\xymatrix{
                                       & \mbox{\cubetopbl} \ar[rr] \ar[dl] \ar@{=}[dd]     &                                                  & \mbox{\cubetopbr} \ar@{=}[dd] \ar[dl] \\
\mbox{\cubetopfl} \ar[rr]  \ar[dd]_{\cong}           &                                                                                              &\mbox{\cubetopfr} \ar@{-->}[dd]^(.35){\cong}   \skewpullbackcorner[ul]              \\
                                       &  \mbox{\cubebotbl} \ar[dl] \ar[rr]                    &                                                  & \mbox{\cubebotbr} \ar@/^1pc/[ddl] \ar[dl] \\
\mbox{\cubebotfl} \ar@/_1pc/[drr] \ar[rr]  &                                                                                             & \mbox{\cubebotfr} \skewpullbackcorner[ul]    \ar@{-->}[d]^{\cong}  \\
                                                   &                                                                                             & (\FPar_2,\times)
}
$$

Again, the proof is essentially the same as for the linear and affine cases; the only difference being that the Cartesian completion of $\FPinj_2$ is $\FPar_2$.


\end{proof}


%\begin{comment}
There is a particularly elegant finite presentation  contained in \S \ref{subsubsec:presentations:three:par}, which  is much more ZX-flavoured.
%\end{comment}


%
%\begin{corollary}
%Give easier chacterization TODO
%\end{corollary}


\begin{definition}
Let $\sp\f_2$ denote the pushout of the diagram of props:
$$
\pr\f_2^\op\leftarrow \pr\iso\f_2 \rightarrow \pr\f_2
$$
\end{definition}

\begin{lemma}\cite{zxa}
\label{lem:spanand}
$\sp\f_2$ is a presentation for  the full subcategory $(\FSpan_2,\times)$ of $(\Span^\sim(\FSets),\times)$ with objects powers of two.
\end{lemma}

\begin{proof}
One has to show that the following diagram commutes:

\renewcommand{\cubetopbl}{$\pr\iso\f_2$}
\renewcommand{\cubetopbr}{$ \pr\f_2$}
\renewcommand{\cubetopfl}{$\pr\f_2^\op$}
\renewcommand{\cubetopfr}{$\sp\f_2$}
\renewcommand{\cubebotbl}{$(\FPinj_2,\times)$ }
\renewcommand{\cubebotbr}{$(\FPar_2,\times)$ }
\renewcommand{\cubebotfl}{$(\FPar_2,\times)^\op$ }
\renewcommand{\cubebotfr}{}

$$
\xymatrixrowsep{2mm}\xymatrixcolsep{0mm}
\xymatrix{
                                       & \mbox{\cubetopbl} \ar[rr] \ar[dl] \ar[dd]^(.7){\cong}      &                                                  & \mbox{\cubetopbr}  \ar[dd]^{\cong} \ar[dl] \\
\mbox{\cubetopfl} \ar[rr]  \ar[dd]_{\cong}           &                                                                                              &\mbox{\cubetopfr} \ar@{-->}[dd]^(.35){\cong}   \skewpullbackcorner[ul]              \\
                                       &  \mbox{\cubebotbl} \ar[dl] \ar[rr]                    &                                                  & \mbox{\cubebotbr} \ar@/^1pc/[ddl] \ar[dl] \\
\mbox{\cubebotfl} \ar@/_1pc/[drr] \ar[rr]  &                                                                                             & \mbox{\cubebotfr} \skewpullbackcorner[ul]    \ar@{-->}[d]^{\cong}  \\
                                                   &                                                                                             &( \FSpan_2,\times)
}
$$


This follows from \cite[Lem. 4.3]{zxa}.

\end{proof}


\begin{remark}
$\sp\f_2$ is isomorphic to $\ZXA$.
\end{remark}






%
%
%That is to say $\Span(\f_2)$ is a presentation for the prop of ``qubit matrices'' over $\N$.  There is an alternative presentation of this category due to \cite{zxa}:
%\begin{corollary}
%$\Span(\f_2)$ can equivalently can be presented in terms of the coproduct of the monoids $X,Z,\&$ and comonoids $Z^\dag,X^\dag$
%\begin{itemize}
%\item An extra-Frobenius algebra between $X^\op$ and $Y$.
%
%\item A special-Frobenius algebra between $Z^\op$ and $Z$.
%
%\item The Lawvere theory for $\F_2^+$ represented by $X$ and $Z^\op$.
%
%\item The Lawvere theory for $\F_2^\times$  represented by  $\&$ and $Z^\op$.
%
%\item A distributive law $L_{X} \otimes_L L_{\&}  \Rightarrow  L_{\&} \otimes_L L_{X}$  in $\Kl(\T_{\Mon-\Prof}^\times)$, where $Z^\op$ is identified with prop for the diagonal monoid.
%
%\item The naturality of $\eta_{\&}$ with respect to $\mu_{\&}^\op$.
%
%\end{itemize}
%
%
%\end{corollary}


%
%The proof follows from realizing that this equation makes the triangle gate idempotent, which allows one to reduce the value of non-scalar positive natural number H-boxes to $1$, alike to the quotient described in \cite{niel} (H-boxes are first described in the paper \cite{zh}).    The other law forces all nonzero scalar H-boxes, and thus all nonzero scalars to be 0. So this is complete for qubit boolean matrices, and thus, qubit relations.
%


\nocite{ih}
\nocite{coecke2008interacting}
\nocite{zh}
\nocite{tof}

%\appendix 



%
%\begin{lemma}
%
%The phase-free fragment of the ZH calculus is presented by the pushout of the following diagram of props:
%
%
%TODO
%
%modulo the quotient:
%
%\end{lemma}



%\begin{comment}
%






%
%\subsection{Alternative presentations}
%\label{sec:presentations}
%
%In this section, we give alternative presentations of the props presented in the main body of this paper.  With the exception of the alternative presentation of $(\FPinj_2,\times)$, these are presented in terms of a bunch of (co)monoids modulo equations.  This is more in the aesthetic tradition of the ZX-calculus, for example.  
%
%\subsection{Section \ref{sec:one}}
%\label{subsec:presentations:one}
%
%\subsubsection{$(\Par(\Mat(\F_2)),+)$}
%\label{subsubsec:presentations:one:par}
%
%
%$(\Par(\Mat(\F_2)),+)$ is presented by the symmetric monoidal theory with the following generators:
%$$
%\begin{tikzpicture}
%	\begin{pgfonlayer}{nodelayer}
%		\node [style=Z] (0) at (0.75, 5) {};
%		\node [style=none] (1) at (0.5, 4.5) {};
%		\node [style=none] (2) at (1, 4.5) {};
%		\node [style=none] (3) at (0.75, 5.5) {};
%	\end{pgfonlayer}
%	\begin{pgfonlayer}{edgelayer}
%		\draw [in=-135, out=90] (1.center) to (0);
%		\draw [in=90, out=-45] (0) to (2.center);
%		\draw (0) to (3.center);
%	\end{pgfonlayer}
%\end{tikzpicture}
%\hspace*{.5cm}
%\begin{tikzpicture}
%	\begin{pgfonlayer}{nodelayer}
%		\node [style=Z] (0) at (0, 5) {};
%		\node [style=none] (1) at (-0.25, 5.5) {};
%		\node [style=none] (2) at (0.25, 5.5) {};
%		\node [style=none] (3) at (0, 4.5) {};
%	\end{pgfonlayer}
%	\begin{pgfonlayer}{edgelayer}
%		\draw [in=135, out=-90] (1.center) to (0);
%		\draw [in=-90, out=45] (0) to (2.center);
%		\draw (0) to (3.center);
%	\end{pgfonlayer}
%\end{tikzpicture}
%\hspace*{.5cm}
%\begin{tikzpicture}
%	\begin{pgfonlayer}{nodelayer}
%		\node [style=Z] (0) at (0, 5) {};
%		\node [style=none] (3) at (0, 4.5) {};
%	\end{pgfonlayer}
%	\begin{pgfonlayer}{edgelayer}
%		\draw (0) to (3.center);
%	\end{pgfonlayer}
%\end{tikzpicture}
%\hspace*{.5cm}
%\begin{tikzpicture}
%	\begin{pgfonlayer}{nodelayer}
%		\node [style=X] (0) at (0.75, 5) {};
%		\node [style=none] (1) at (0.5, 4.5) {};
%		\node [style=none] (2) at (1, 4.5) {};
%		\node [style=none] (3) at (0.75, 5.5) {};
%	\end{pgfonlayer}
%	\begin{pgfonlayer}{edgelayer}
%		\draw [in=-135, out=90] (1.center) to (0);
%		\draw [in=90, out=-45] (0) to (2.center);
%		\draw (0) to (3.center);
%	\end{pgfonlayer}
%\end{tikzpicture}
%\hspace*{.5cm}
%\begin{tikzpicture}
%	\begin{pgfonlayer}{nodelayer}
%		\node [style=X] (0) at (0, 4.5) {};
%		\node [style=none] (3) at (0, 5) {};
%	\end{pgfonlayer}
%	\begin{pgfonlayer}{edgelayer}
%		\draw (0) to (3.center);
%	\end{pgfonlayer}
%\end{tikzpicture}
%\hspace*{,5cm}
%\begin{tikzpicture}
%	\begin{pgfonlayer}{nodelayer}
%		\node [style=X] (0) at (0.5, 5) {};
%		\node [style=none] (1) at (0.5, 4.5) {};
%	\end{pgfonlayer}
%	\begin{pgfonlayer}{edgelayer}
%		\draw (0) to (1.center);
%	\end{pgfonlayer}
%\end{tikzpicture}
%$$
%Modulo the all the equations of $\cb_2$ in addition to the identity and its transpose:
%$$
%\begin{tikzpicture}
%	\begin{pgfonlayer}{nodelayer}
%		\node [style=X] (0) at (0.5, 5) {};
%		\node [style=none] (1) at (0.5, 4.5) {};
%		\node [style=Z] (2) at (0.5, 4.5) {};
%		\node [style=none] (3) at (0.25, 4) {};
%		\node [style=none] (4) at (0.75, 4) {};
%	\end{pgfonlayer}
%	\begin{pgfonlayer}{edgelayer}
%		\draw (0) to (1.center);
%		\draw [in=-135, out=90] (3.center) to (2);
%		\draw [in=-45, out=90] (4.center) to (2);
%	\end{pgfonlayer}
%\end{tikzpicture}
%  \eref{bi.two}
%\begin{tikzpicture}
%	\begin{pgfonlayer}{nodelayer}
%		\node [style=X] (0) at (0.25, 5) {};
%		\node [style=none] (3) at (0.25, 4) {};
%		\node [style=none] (4) at (0.75, 4) {};
%		\node [style=X] (5) at (0.75, 5) {};
%	\end{pgfonlayer}
%	\begin{pgfonlayer}{edgelayer}
%		\draw (4.center) to (5);
%		\draw (0) to (3.center);
%	\end{pgfonlayer}
%\end{tikzpicture}
%$$
%
%
%\subsubsection{$(\Span(\Mat(\F_2)),+)$}
%\label{subsubsec:presentations:one:span}
%
%$(\Span(\Mat(\F_2)),+)$ is presented by the symmetric monoidal theory with the same generators and equations as \S \ref{subsubsec:presentations:one:par} and their transposes as well as the following equations making the white monoid/comnoid pair into a special commutative Frobenius algebra and the black monoid/comonoid pair into a commutative Frobenius algebra.
%
%
%
%\subsection{Section \ref{sec:two}}
%\label{subsec:presentations:two}
%
%
%\subsubsection{$(\Par(\Aff\Fin\Vect(\F_2))^*,+)$}
%\label{subsubsec:presentations:two:par}
%
%$(\Par(\Aff\Fin\Vect(\F_2))^*,+)$ is presented by the symmetric monoidal theory with the same generators and equations as in \S \ref{subsubsec:presentations:one:par} in addition to the following generator:
%$$
%\begin{tikzpicture}
%	\begin{pgfonlayer}{nodelayer}
%		\node [style=X] (0) at (0, 4) {$1$};
%		\node [style=none] (1) at (0, 4.5) {};
%	\end{pgfonlayer}
%	\begin{pgfonlayer}{edgelayer}
%		\draw (0) to (1.center);
%	\end{pgfonlayer}
%\end{tikzpicture}
%$$
%and the following equations:
%$$
%\begin{tikzpicture}
%	\begin{pgfonlayer}{nodelayer}
%		\node [style=X] (0) at (-4, 0.75) {$1$};
%		\node [style=none] (1) at (-3.5, 0) {};
%		\node [style=none] (2) at (-3.5, 1.5) {};
%	\end{pgfonlayer}
%	\begin{pgfonlayer}{edgelayer}
%		\draw (1.center) to (2.center);
%	\end{pgfonlayer}
%\end{tikzpicture}
%\eqzxa{zero.new}
%\begin{tikzpicture}
%	\begin{pgfonlayer}{nodelayer}
%		\node [style=X] (0) at (-4, 0.75) {$1$};
%		\node [style=none] (1) at (-3.5, 0) {};
%		\node [style=none] (2) at (-3.5, 1.5) {};
%		\node [style=Z] (3) at (-3.5, 0.5) {};
%		\node [style=X] (4) at (-3.5, 1) {};
%	\end{pgfonlayer}
%	\begin{pgfonlayer}{edgelayer}
%		\draw (2.center) to (4);
%		\draw (3) to (1.center);
%	\end{pgfonlayer}
%\end{tikzpicture},
%\hspace*{,5cm}
%\begin{tikzpicture}
%	\begin{pgfonlayer}{nodelayer}
%		\node [style=X] (0) at (0.75, 4) {$1$};
%		\node [style=none] (1) at (0.75, 4.5) {};
%		\node [style=Z] (2) at (0.75, 4.5) {};
%		\node [style=none] (3) at (0.5, 5) {};
%		\node [style=none] (4) at (1, 5) {};
%	\end{pgfonlayer}
%	\begin{pgfonlayer}{edgelayer}
%		\draw (0) to (1.center);
%		\draw [in=135, out=-90] (3.center) to (2);
%		\draw [in=45, out=-90] (4.center) to (2);
%	\end{pgfonlayer}
%\end{tikzpicture}
%  \erefop{bi.two}
%\begin{tikzpicture}
%	\begin{pgfonlayer}{nodelayer}
%		\node [style=X] (0) at (0.5, 4) {$1$};
%		\node [style=none] (1) at (0.5, 5) {};
%		\node [style=none] (2) at (1, 5) {};
%		\node [style=X] (3) at (1, 4) {$1$};
%	\end{pgfonlayer}
%	\begin{pgfonlayer}{edgelayer}
%		\draw (2.center) to (3);
%		\draw (0) to (1.center);
%	\end{pgfonlayer}
%\end{tikzpicture}
%\hspace*{,5cm}
%\begin{tikzpicture}
%	\begin{pgfonlayer}{nodelayer}
%		\node [style=X] (0) at (0.75, 4) {$1$};
%		\node [style=none] (1) at (0.75, 4.5) {};
%		\node [style=Z] (2) at (0.75, 4.5) {};
%	\end{pgfonlayer}
%	\begin{pgfonlayer}{edgelayer}
%		\draw (0) to (1.center);
%	\end{pgfonlayer}
%\end{tikzpicture}
%  \eref{extra}
%$$
%
%
%\subsubsection{$(\Span(\Aff\Fin\Vect(\F_2))^*,+)$}
%\label{subsubsec:presentations:two:span}
%
%$(\Span(\Aff\Fin\Vect(\F_2))^*,+)$ is presented by the generators and identities of  \ref{subsubsec:presentations:two:par} as well as as well as the generator 
%$\begin{tikzpicture}
%	\begin{pgfonlayer}{nodelayer}
%		\node [style=none] (0) at (0.75, 0.5) {};
%		\node [style=none] (1) at (0.75, -0.25) {};
%		\node [style=Z] (2) at (0.75, -0.25) {};
%	\end{pgfonlayer}
%	\begin{pgfonlayer}{edgelayer}
%		\draw (0.center) to (1.center);
%	\end{pgfonlayer}
%\end{tikzpicture}$ 
%and the equation making the codiagonal map counital:
%$$
%  \begin{tikzpicture}[rotate=90,yscale=-1]
%	\begin{pgfonlayer}{nodelayer}
%		\node [style=Z] (0) at (-9, -0) {};
%		\node [style=none] (1) at (-8.25, -0) {};
%		\node [style=Z] (2) at (-9.75, 0.25) {};
%		\node [style=none] (3) at (-10, -0.25) {};
%	\end{pgfonlayer}
%	\begin{pgfonlayer}{edgelayer}
%		\draw [in=-150, out=0, looseness=1.00] (3.center) to (0);
%		\draw [in=150, out=0, looseness=1.00] (2.center) to (0);
%		\draw (0) to (1.center);
%	\end{pgfonlayer}
%  \end{tikzpicture}
%  \eref{unit}
%  \begin{tikzpicture}[rotate=90]
%	\begin{pgfonlayer}{nodelayer}
%		\node [style=none] (0) at (-9, 0.25) {};
%		\node [style=none] (1) at (-9.75, 0.25) {};
%	\end{pgfonlayer}
%	\begin{pgfonlayer}{edgelayer}
%		\draw (1) to (0.center);
%	\end{pgfonlayer}
%  \end{tikzpicture}
%$$
%
%\subsection{Section \ref{sec:three}}
%\label{subsec:presentations:three}
%
%
%
%%\subsubsection{$(\FPinj_2,\times)$}
%%\label{subsubsec:presentations:three:pinj}
%
%%By \cite[\S 7]{cole} $(\FPinj_2,\times)$ is presented by the prop generated by the Toffoli gate (the triple-controlled-not gate) as well as unit for the and gate and its transpose modulo the following equations:
%%
%%\begin{multicols}{2}
%%\begin{enumerate}[label={\bf [TOF.\arabic*]}, ref={\bf [TOF.\arabic*]}, wide = 0pt, leftmargin = 2em]
%%\item
%%\label{TOF.1}
%%{\hfil
%%$
%%\begin{tabular}{cc}
%%\begin{tikzpicture}
%%	\begin{pgfonlayer}{nodelayer}
%%		\node [style=nothing] (0) at (1.5, 0) {};
%%		\node [style=nothing] (1) at (1, 0) {};
%%		\node [style=oplus] (2) at (1.5, 1) {};
%%		\node [style=dot] (3) at (1, 1) {};
%%		\node [style=dot] (4) at (0.5, 1) {};
%%		\node [style=X] (5) at (0.5, 0.5) {$1$};
%%		\node [style=nothing] (6) at (0.5, 1.5) {};
%%		\node [style=nothing] (7) at (1, 1.5) {};
%%		\node [style=nothing] (8) at (1.5, 1.5) {};
%%	\end{pgfonlayer}
%%	\begin{pgfonlayer}{edgelayer}
%%		\draw (5) to (4);
%%		\draw (4) to (6);
%%		\draw (7) to (3);
%%		\draw (1) to (3);
%%		\draw (0) to (2);
%%		\draw (2) to (8);
%%		\draw (2) to (3);
%%		\draw (3) to (4);
%%	\end{pgfonlayer}
%%\end{tikzpicture}
%%=
%%\begin{tikzpicture}
%%	\begin{pgfonlayer}{nodelayer}
%%		\node [style=nothing] (0) at (1.5, 0) {};
%%		\node [style=nothing] (1) at (1, 0) {};
%%		\node [style=oplus] (2) at (1.5, 1) {};
%%		\node [style=dot] (3) at (1, 1) {};
%%		\node [style=X] (4) at (0.5, 1) {$1$};
%%		\node [style=nothing] (5) at (0.5, 1.5) {};
%%		\node [style=nothing] (6) at (1, 1.5) {};
%%		\node [style=nothing] (7) at (1.5, 1.5) {};
%%	\end{pgfonlayer}
%%	\begin{pgfonlayer}{edgelayer}
%%		\draw (1) to (3);
%%		\draw (0) to (2);
%%		\draw (2) to (3);
%%		\draw (6) to (3);
%%		\draw (4) to (5);
%%		\draw (2) to (7);
%%	\end{pgfonlayer}
%%\end{tikzpicture} &
%%\begin{tikzpicture}
%%	\begin{pgfonlayer}{nodelayer}
%%		\node [style=nothing] (0) at (1.5, 2) {};
%%		\node [style=nothing] (1) at (1, 2) {};
%%		\node [style=oplus] (2) at (1.5, 1) {};
%%		\node [style=dot] (3) at (1, 1) {};
%%		\node [style=dot] (4) at (0.5, 1) {};
%%		\node [style=X] (5) at (0.5, 1.5) {$1$};
%%		\node [style=nothing] (6) at (0.5, 0.5) {};
%%		\node [style=nothing] (7) at (1, 0.5) {};
%%		\node [style=nothing] (8) at (1.5, 0.5) {};
%%	\end{pgfonlayer}
%%	\begin{pgfonlayer}{edgelayer}
%%		\draw (5) to (4);
%%		\draw (4) to (6);
%%		\draw (7) to (3);
%%		\draw (1) to (3);
%%		\draw (0) to (2);
%%		\draw (2) to (8);
%%		\draw (2) to (3);
%%		\draw (3) to (4);
%%	\end{pgfonlayer}
%%\end{tikzpicture}
%%=
%%\begin{tikzpicture}
%%	\begin{pgfonlayer}{nodelayer}
%%		\node [style=nothing] (0) at (1.5, 2) {};
%%		\node [style=nothing] (1) at (1, 2) {};
%%		\node [style=oplus] (2) at (1.5, 1) {};
%%		\node [style=dot] (3) at (1, 1) {};
%%		\node [style=X] (4) at (0.5, 1) {$1$};
%%		\node [style=nothing] (5) at (0.5, 0.5) {};
%%		\node [style=nothing] (6) at (1, 0.5) {};
%%		\node [style=nothing] (7) at (1.5, 0.5) {};
%%	\end{pgfonlayer}
%%	\begin{pgfonlayer}{edgelayer}
%%		\draw (1) to (3);
%%		\draw (0) to (2);
%%		\draw (2) to (3);
%%		\draw (6) to (3);
%%		\draw (4) to (5);
%%		\draw (2) to (7);
%%	\end{pgfonlayer}
%%\end{tikzpicture}
%%\end{tabular}
%%$}
%%
%%
%%\item
%%\label{TOF.2}
%%{\hfil
%%$
%%\begin{tabular}{cc}
%%\begin{tikzpicture}
%%	\begin{pgfonlayer}{nodelayer}
%%		\node [style=nothing] (0) at (1, 0.5) {};
%%		\node [style=nothing] (1) at (1.5, 0.5) {};
%%		\node [style=nothing] (2) at (0.5, 2) {};
%%		\node [style=nothing] (3) at (1, 2) {};
%%		\node [style=nothing] (4) at (1.5, 2) {};
%%		\node [style=dot] (5) at (0.5, 1.5) {};
%%		\node [style=dot] (6) at (1, 1.5) {};
%%		\node [style=oplus] (7) at (1.5, 1.5) {};
%%		\node [style=X] (8) at (0.5, 1) {};
%%	\end{pgfonlayer}
%%	\begin{pgfonlayer}{edgelayer}
%%		\draw (5) to (2);
%%		\draw (3) to (6);
%%		\draw (6) to (0);
%%		\draw (1) to (7);
%%		\draw (7) to (4);
%%		\draw (7) to (6);
%%		\draw (6) to (5);
%%		\draw (8) to (5);
%%	\end{pgfonlayer}
%%\end{tikzpicture}
%%=
%%\begin{tikzpicture}
%%	\begin{pgfonlayer}{nodelayer}
%%		\node [style=nothing] (0) at (1, 0.75) {};
%%		\node [style=nothing] (1) at (1.5, 0.75) {};
%%		\node [style=nothing] (2) at (0.5, 2) {};
%%		\node [style=nothing] (3) at (1, 2) {};
%%		\node [style=nothing] (4) at (1.5, 2) {};
%%		\node [style=X] (5) at (0.5, 1.25) {};
%%	\end{pgfonlayer}
%%	\begin{pgfonlayer}{edgelayer}
%%		\draw (5) to (2);
%%		\draw (0) to (3);
%%		\draw (1) to (4);
%%	\end{pgfonlayer}
%%\end{tikzpicture} & 
%%\begin{tikzpicture}
%%	\begin{pgfonlayer}{nodelayer}
%%		\node [style=nothing] (0) at (1, 2.5) {};
%%		\node [style=nothing] (1) at (1.5, 2.5) {};
%%		\node [style=nothing] (2) at (0.5, 1) {};
%%		\node [style=nothing] (3) at (1, 1) {};
%%		\node [style=nothing] (4) at (1.5, 1) {};
%%		\node [style=dot] (5) at (0.5, 1.5) {};
%%		\node [style=dot] (6) at (1, 1.5) {};
%%		\node [style=oplus] (7) at (1.5, 1.5) {};
%%		\node [style=X] (8) at (0.5, 2) {};
%%	\end{pgfonlayer}
%%	\begin{pgfonlayer}{edgelayer}
%%		\draw (5) to (2);
%%		\draw (3) to (6);
%%		\draw (6) to (0);
%%		\draw (1) to (7);
%%		\draw (7) to (4);
%%		\draw (7) to (6);
%%		\draw (6) to (5);
%%		\draw (8) to (5);
%%	\end{pgfonlayer}
%%\end{tikzpicture}
%%=
%%\begin{tikzpicture}
%%	\begin{pgfonlayer}{nodelayer}
%%		\node [style=nothing] (0) at (1, 2.5) {};
%%		\node [style=nothing] (1) at (1.5, 2.5) {};
%%		\node [style=nothing] (2) at (0.5, 1.25) {};
%%		\node [style=nothing] (3) at (1, 1.25) {};
%%		\node [style=nothing] (4) at (1.5, 1.25) {};
%%		\node [style=X] (5) at (0.5, 2) {};
%%	\end{pgfonlayer}
%%	\begin{pgfonlayer}{edgelayer}
%%		\draw (5) to (2);
%%		\draw (0) to (3);
%%		\draw (1) to (4);
%%	\end{pgfonlayer}
%%\end{tikzpicture}
%%\end{tabular}
%%$}
%%
%%\item
%%\label{TOF.3}
%%{\hfil
%%$
%%\begin{tikzpicture}
%%	\begin{pgfonlayer}{nodelayer}
%%		\node [style=nothing] (0) at (-0.5, 0.5) {};
%%		\node [style=nothing] (1) at (0, 0.5) {};
%%		\node [style=nothing] (2) at (-1, 0.5) {};
%%		\node [style=nothing] (3) at (-1.5, 0.5) {};
%%		\node [style=nothing] (4) at (-2, 0.5) {};
%%		\node [style=dot] (5) at (-1.5, 1) {};
%%		\node [style=oplus] (6) at (-1, 1) {};
%%		\node [style=oplus] (7) at (-1, 1.5) {};
%%		\node [style=dot] (8) at (-0.5, 1.5) {};
%%		\node [style=dot] (9) at (-2, 1) {};
%%		\node [style=dot] (10) at (0, 1.5) {};
%%		\node [style=nothing] (11) at (-0.5, 2) {};
%%		\node [style=nothing] (12) at (-1.5, 2) {};
%%		\node [style=nothing] (13) at (-2, 2) {};
%%		\node [style=nothing] (14) at (0, 2) {};
%%		\node [style=nothing] (15) at (-1, 2) {};
%%	\end{pgfonlayer}
%%	\begin{pgfonlayer}{edgelayer}
%%		\draw (4) to (9);
%%		\draw (9) to (13);
%%		\draw (3) to (5);
%%		\draw (5) to (12);
%%		\draw (2) to (6);
%%		\draw (6) to (7);
%%		\draw (7) to (15);
%%		\draw (0) to (8);
%%		\draw (8) to (11);
%%		\draw (1) to (10);
%%		\draw (10) to (14);
%%		\draw (10) to (8);
%%		\draw (8) to (7);
%%		\draw (6) to (5);
%%		\draw (5) to (9);
%%	\end{pgfonlayer}
%%\end{tikzpicture}
%%=
%%\begin{tikzpicture}
%%	\begin{pgfonlayer}{nodelayer}
%%		\node [style=nothing] (0) at (-0.5, 0.5) {};
%%		\node [style=nothing] (1) at (0, 0.5) {};
%%		\node [style=nothing] (2) at (-1, 0.5) {};
%%		\node [style=nothing] (3) at (-1.5, 0.5) {};
%%		\node [style=nothing] (4) at (-2, 0.5) {};
%%		\node [style=dot] (5) at (-1.5, 1.5) {};
%%		\node [style=dot] (6) at (-0.5, 1) {};
%%		\node [style=dot] (7) at (-2, 1.5) {};
%%		\node [style=dot] (8) at (0, 1) {};
%%		\node [style=nothing] (9) at (-0.5, 2) {};
%%		\node [style=nothing] (10) at (-1.5, 2) {};
%%		\node [style=nothing] (11) at (-2, 2) {};
%%		\node [style=nothing] (12) at (0, 2) {};
%%		\node [style=nothing] (13) at (-1, 2) {};
%%		\node [style=oplus] (14) at (-1, 1.5) {};
%%		\node [style=oplus] (15) at (-1, 1) {};
%%	\end{pgfonlayer}
%%	\begin{pgfonlayer}{edgelayer}
%%		\draw (4) to (7);
%%		\draw (7) to (11);
%%		\draw (3) to (5);
%%		\draw (5) to (10);
%%		\draw (0) to (6);
%%		\draw (6) to (9);
%%		\draw (1) to (8);
%%		\draw (8) to (12);
%%		\draw (8) to (6);
%%		\draw (5) to (7);
%%		\draw (2) to (15);
%%		\draw (15) to (14);
%%		\draw (14) to (13);
%%		\draw (14) to (5);
%%		\draw (6) to (15);
%%	\end{pgfonlayer}
%%\end{tikzpicture}
%%$}
%%
%%
%%\item
%%\label{TOF.4}
%%{\hfil
%%$
%%\begin{tikzpicture}
%%	\begin{pgfonlayer}{nodelayer}
%%		\node [style=nothing] (0) at (-0.5, 0.5) {};
%%		\node [style=nothing] (1) at (0, 0.5) {};
%%		\node [style=nothing] (2) at (-1, 0.5) {};
%%		\node [style=nothing] (3) at (-1.5, 0.5) {};
%%		\node [style=nothing] (4) at (-2, 0.5) {};
%%		\node [style=dot] (5) at (-1.5, 1) {};
%%		\node [style=dot] (6) at (-1, 1) {};
%%		\node [style=dot] (7) at (-1, 1.5) {};
%%		\node [style=dot] (8) at (-0.5, 1.5) {};
%%		\node [style=oplus] (9) at (-2, 1) {};
%%		\node [style=oplus] (10) at (0, 1.5) {};
%%		\node [style=nothing] (11) at (-0.5, 2) {};
%%		\node [style=nothing] (12) at (-1.5, 2) {};
%%		\node [style=nothing] (13) at (-2, 2) {};
%%		\node [style=nothing] (14) at (0, 2) {};
%%		\node [style=nothing] (15) at (-1, 2) {};
%%	\end{pgfonlayer}
%%	\begin{pgfonlayer}{edgelayer}
%%		\draw (4) to (9);
%%		\draw (9) to (13);
%%		\draw (3) to (5);
%%		\draw (5) to (12);
%%		\draw (2) to (6);
%%		\draw (6) to (7);
%%		\draw (7) to (15);
%%		\draw (0) to (8);
%%		\draw (8) to (11);
%%		\draw (1) to (10);
%%		\draw (10) to (14);
%%		\draw (10) to (8);
%%		\draw (8) to (7);
%%		\draw (6) to (5);
%%		\draw (5) to (9);
%%	\end{pgfonlayer}
%%\end{tikzpicture}
%%=
%%\begin{tikzpicture}
%%	\begin{pgfonlayer}{nodelayer}
%%		\node [style=nothing] (0) at (-0.5, 0.5) {};
%%		\node [style=nothing] (1) at (0, 0.5) {};
%%		\node [style=nothing] (2) at (-1, 0.5) {};
%%		\node [style=nothing] (3) at (-1.5, 0.5) {};
%%		\node [style=nothing] (4) at (-2, 0.5) {};
%%		\node [style=dot] (5) at (-1.5, 1.5) {};
%%		\node [style=dot] (6) at (-0.5, 1) {};
%%		\node [style=oplus] (7) at (-2, 1.5) {};
%%		\node [style=oplus] (8) at (0, 1) {};
%%		\node [style=nothing] (9) at (-0.5, 2) {};
%%		\node [style=nothing] (10) at (-1.5, 2) {};
%%		\node [style=nothing] (11) at (-2, 2) {};
%%		\node [style=nothing] (12) at (0, 2) {};
%%		\node [style=nothing] (13) at (-1, 2) {};
%%		\node [style=dot] (14) at (-1, 1.5) {};
%%		\node [style=dot] (15) at (-1, 1) {};
%%	\end{pgfonlayer}
%%	\begin{pgfonlayer}{edgelayer}
%%		\draw (4) to (7);
%%		\draw (7) to (11);
%%		\draw (3) to (5);
%%		\draw (5) to (10);
%%		\draw (0) to (6);
%%		\draw (6) to (9);
%%		\draw (1) to (8);
%%		\draw (8) to (12);
%%		\draw (8) to (6);
%%		\draw (5) to (7);
%%		\draw (2) to (15);
%%		\draw (15) to (14);
%%		\draw (14) to (13);
%%		\draw (14) to (5);
%%		\draw (6) to (15);
%%	\end{pgfonlayer}
%%\end{tikzpicture}
%%$}
%%
%%\item
%%\label{TOF.5}
%%{\hfil
%%$
%%\begin{tikzpicture}
%%	\begin{pgfonlayer}{nodelayer}
%%		\node [style=nothing] (0) at (-1, 0.5) {};
%%		\node [style=nothing] (1) at (-0.5, 0.5) {};
%%		\node [style=nothing] (2) at (-1.5, 0.5) {};
%%		\node [style=nothing] (3) at (-2, 0.5) {};
%%		\node [style=nothing] (4) at (-1, 2) {};
%%		\node [style=nothing] (5) at (-1.5, 2) {};
%%		\node [style=nothing] (6) at (-2, 2) {};
%%		\node [style=nothing] (7) at (-0.5, 2) {};
%%		\node [style=oplus] (8) at (-2, 1) {};
%%		\node [style=oplus] (9) at (-0.5, 1.5) {};
%%		\node [style=dot] (10) at (-1.5, 1) {};
%%		\node [style=dot] (11) at (-1, 1) {};
%%		\node [style=dot] (12) at (-1.5, 1.5) {};
%%		\node [style=dot] (13) at (-1, 1.5) {};
%%	\end{pgfonlayer}
%%	\begin{pgfonlayer}{edgelayer}
%%		\draw (3) to (8);
%%		\draw (8) to (6);
%%		\draw (5) to (12);
%%		\draw (12) to (10);
%%		\draw (10) to (2);
%%		\draw (0) to (11);
%%		\draw (11) to (13);
%%		\draw (13) to (4);
%%		\draw (7) to (9);
%%		\draw (9) to (1);
%%		\draw (10) to (11);
%%		\draw (10) to (8);
%%		\draw (12) to (13);
%%		\draw (13) to (9);
%%	\end{pgfonlayer}
%%\end{tikzpicture}
%%=
%%\begin{tikzpicture}
%%	\begin{pgfonlayer}{nodelayer}
%%		\node [style=nothing] (0) at (-1, 0.5) {};
%%		\node [style=nothing] (1) at (-0.5, 0.5) {};
%%		\node [style=nothing] (2) at (-1.5, 0.5) {};
%%		\node [style=nothing] (3) at (-2, 0.5) {};
%%		\node [style=nothing] (4) at (-1, 2) {};
%%		\node [style=nothing] (5) at (-1.5, 2) {};
%%		\node [style=nothing] (6) at (-2, 2) {};
%%		\node [style=nothing] (7) at (-0.5, 2) {};
%%		\node [style=oplus] (8) at (-2, 1.5) {};
%%		\node [style=dot] (9) at (-1.5, 1.5) {};
%%		\node [style=dot] (10) at (-1, 1.5) {};
%%		\node [style=oplus] (11) at (-0.5, 1) {};
%%		\node [style=dot] (12) at (-1, 1) {};
%%		\node [style=dot] (13) at (-1.5, 1) {};
%%	\end{pgfonlayer}
%%	\begin{pgfonlayer}{edgelayer}
%%		\draw (9) to (10);
%%		\draw (9) to (8);
%%		\draw (13) to (12);
%%		\draw (12) to (11);
%%		\draw (3) to (8);
%%		\draw (8) to (6);
%%		\draw (5) to (9);
%%		\draw (9) to (13);
%%		\draw (13) to (2);
%%		\draw (0) to (12);
%%		\draw (12) to (10);
%%		\draw (10) to (4);
%%		\draw (7) to (11);
%%		\draw (11) to (1);
%%	\end{pgfonlayer}
%%\end{tikzpicture}
%%$}
%%
%%
%%\item
%%\label{TOF.6}
%%{\hfil
%%$
%%\begin{tikzpicture}
%%	\begin{pgfonlayer}{nodelayer}
%%		\node [style=nothing] (0) at (-1, 0.5) {};
%%		\node [style=nothing] (1) at (-1.5, 0.5) {};
%%		\node [style=nothing] (2) at (-2, 0.5) {};
%%		\node [style=nothing] (3) at (-1, 2) {};
%%		\node [style=nothing] (4) at (-1.5, 2) {};
%%		\node [style=nothing] (5) at (-2, 2) {};
%%		\node [style=nothing] (6) at (-0.5, 2) {};
%%		\node [style=oplus] (7) at (-0.5, 1) {};
%%		\node [style=dot] (8) at (-1.5, 1.5) {};
%%		\node [style=dot] (9) at (-1, 1.5) {};
%%		\node [style=dot] (10) at (-1, 1) {};
%%		\node [style=oplus] (11) at (-0.5, 1.5) {};
%%		\node [style=nothing] (12) at (-0.5, 0.5) {};
%%		\node [style=dot] (13) at (-2, 1) {};
%%	\end{pgfonlayer}
%%	\begin{pgfonlayer}{edgelayer}
%%		\draw (8) to (1);
%%		\draw (0) to (9);
%%		\draw (9) to (10);
%%		\draw (10) to (3);
%%		\draw (6) to (7);
%%		\draw (8) to (9);
%%		\draw (10) to (7);
%%		\draw (12) to (11);
%%		\draw (11) to (7);
%%		\draw (8) to (4);
%%		\draw (9) to (11);
%%		\draw (10) to (13);
%%		\draw (13) to (5);
%%		\draw (13) to (2);
%%	\end{pgfonlayer}
%%\end{tikzpicture}
%%=
%%\begin{tikzpicture}
%%	\begin{pgfonlayer}{nodelayer}
%%		\node [style=nothing] (0) at (-1, 0.5) {};
%%		\node [style=nothing] (1) at (-1.5, 0.5) {};
%%		\node [style=nothing] (2) at (-2, 0.5) {};
%%		\node [style=nothing] (3) at (-1, 2) {};
%%		\node [style=nothing] (4) at (-1.5, 2) {};
%%		\node [style=nothing] (5) at (-2, 2) {};
%%		\node [style=nothing] (6) at (-0.5, 2) {};
%%		\node [style=oplus] (7) at (-0.5, 1.5) {};
%%		\node [style=dot] (8) at (-1.5, 1) {};
%%		\node [style=dot] (9) at (-1, 1) {};
%%		\node [style=dot] (10) at (-1, 1.5) {};
%%		\node [style=oplus] (11) at (-0.5, 1) {};
%%		\node [style=nothing] (12) at (-0.5, 0.5) {};
%%		\node [style=dot] (13) at (-2, 1.5) {};
%%	\end{pgfonlayer}
%%	\begin{pgfonlayer}{edgelayer}
%%		\draw (8) to (1);
%%		\draw (0) to (9);
%%		\draw (9) to (10);
%%		\draw (10) to (3);
%%		\draw (6) to (7);
%%		\draw (8) to (9);
%%		\draw (10) to (7);
%%		\draw (12) to (11);
%%		\draw (11) to (7);
%%		\draw (8) to (4);
%%		\draw (9) to (11);
%%		\draw (10) to (13);
%%		\draw (13) to (5);
%%		\draw (13) to (2);
%%	\end{pgfonlayer}
%%\end{tikzpicture}
%%$}
%%
%%\item
%%\label{TOF.7}
%%{\hfil
%%$
%%\begin{tikzpicture}
%%	\begin{pgfonlayer}{nodelayer}
%%		\node [style=nothing] (0) at (1, 0) {};
%%		\node [style=nothing] (1) at (0.5, 0) {};
%%		\node [style=nothing] (2) at (0.5, 3.5) {};
%%		\node [style=nothing] (3) at (1, 3.5) {};
%%		\node [style=X] (4) at (1.5, 3) {};
%%		\node [style=oplus] (5) at (1.5, 2.5) {};
%%		\node [style=dot] (6) at (1, 2.5) {};
%%		\node [style=dot] (7) at (0.5, 1) {};
%%		\node [style=oplus] (8) at (1.5, 1) {};
%%		\node [style=X] (9) at (1.5, 1.5) {};
%%		\node [style=X] (10) at (1.5, 0.5) {$1$};
%%		\node [style=X] (11) at (1.5, 2) {$1$};
%%	\end{pgfonlayer}
%%	\begin{pgfonlayer}{edgelayer}
%%		\draw (1) to (7);
%%		\draw (7) to (2);
%%		\draw (3) to (6);
%%		\draw (6) to (0);
%%		\draw (8) to (9);
%%		\draw (8) to (7);
%%		\draw (5) to (4);
%%		\draw (5) to (6);
%%		\draw (10) to (8);
%%		\draw (11) to (5);
%%	\end{pgfonlayer}
%%\end{tikzpicture}
%%=
%%\begin{tikzpicture}
%%	\begin{pgfonlayer}{nodelayer}
%%		\node [style=nothing] (0) at (3, 0) {};
%%		\node [style=nothing] (1) at (2.5, 0) {};
%%		\node [style=nothing] (2) at (2.5, 3.5) {};
%%		\node [style=nothing] (3) at (3, 3.5) {};
%%		\node [style=dot] (4) at (2.5, 1.75) {};
%%		\node [style=dot] (5) at (3, 1.75) {};
%%		\node [style=X] (6) at (3.5, 1.25) {$1$};
%%		\node [style=X] (7) at (3.5, 2.25) {};
%%		\node [style=oplus] (8) at (3.5, 1.75) {};
%%	\end{pgfonlayer}
%%	\begin{pgfonlayer}{edgelayer}
%%		\draw (1) to (4);
%%		\draw (4) to (2);
%%		\draw (3) to (5);
%%		\draw (5) to (0);
%%		\draw (6) to (8);
%%		\draw (8) to (7);
%%		\draw (8) to (5);
%%		\draw (5) to (4);
%%	\end{pgfonlayer}
%%\end{tikzpicture}
%%$}
%%%
%%%\item
%%%\label{TOF.7}
%%%{\hfil
%%%$
%%%\begin{tikzpicture}
%%%	\begin{pgfonlayer}{nodelayer}
%%%		\node [style=nothing] (0) at (0, -0) {};
%%%		\node [style=nothing] (1) at (1.5, -0) {};
%%%		\node [style=X] (2) at (0.4, 0.5) {};
%%%		\node [style=X] (3) at (1.1, 0.5) {};
%%%	\end{pgfonlayer}
%%%	\begin{pgfonlayer}{edgelayer}
%%%		\draw (2) to (3);
%%%		\draw (0) to (1);
%%%	\end{pgfonlayer}
%%%\end{tikzpicture}
%%%=
%%%\begin{tikzpicture}
%%%	\begin{pgfonlayer}{nodelayer}
%%%		\node [style=nothing] (0) at (0, -0) {};
%%%		\node [style=nothing] (1) at (1.5, -0) {};
%%%		\node [style=X] (2) at (0.4, 0.5) {};
%%%		\node [style=X] (3) at (1.1, 0.5) {};
%%%		\node [style=X] (4) at (1, -0) {};
%%%		\node [style=X] (5) at (0.5000002, -0) {};
%%%	\end{pgfonlayer}
%%%	\begin{pgfonlayer}{edgelayer}
%%%		\draw (2) to (3);
%%%		\draw (5) to (0);
%%%		\draw (4) to (1);
%%%	\end{pgfonlayer}
%%%\end{tikzpicture}
%%%$}
%%
%%\item
%%\label{TOF.8}
%%{\hfil
%%$
%%\begin{tikzpicture}
%%	\begin{pgfonlayer}{nodelayer}
%%		\node [style=X] (0) at (0, 0.5) {$1$};
%%		\node [style=X] (1) at (0, 1.5) {$1$};
%%	\end{pgfonlayer}
%%	\begin{pgfonlayer}{edgelayer}
%%		\draw (0) to (1);
%%	\end{pgfonlayer}
%%\end{tikzpicture}
%%=
%%\begin{tikzpicture}
%%	\begin{pgfonlayer}{nodelayer}
%%		\node [style=rn] (0) at (0, 0.5) {};
%%		\node [style=rn] (1) at (0, 1.5) {};
%%	\end{pgfonlayer}
%%\end{tikzpicture}
%%%\hspace*{-.8cm}
%%%\begin{tikzpicture}[scale=.5]
%%%\begin{pgfonlayer}{nodelayer}
%%%\begin{tikzpicture}
%%%\node[cloud, cloud puffs=15.7,minimum width=3cm, draw,] (cloud) at (0,0) {$1_0$};
%%%\end{tikzpicture}
%%%\end{pgfonlayer}
%%%\begin{pgfonlayer}{edgelayer}
%%%\end{pgfonlayer}
%%%\end{tikzpicture}
%%$}
%%
%%\item
%%\label{TOF.9}
%%{\hfil
%%$
%%\begin{tikzpicture}
%%	\begin{pgfonlayer}{nodelayer}
%%		\node [style=nothing] (0) at (-1.75, 0.5) {};
%%		\node [style=nothing] (1) at (-1.25, 0.5) {};
%%		\node [style=nothing] (2) at (-0.75, 0.5) {};
%%		\node [style=dot] (3) at (-1.75, 1) {};
%%		\node [style=dot] (4) at (-1.25, 1) {};
%%		\node [style=oplus] (5) at (-0.75, 1) {};
%%		\node [style=dot] (6) at (-1.75, 1.5) {};
%%		\node [style=oplus] (7) at (-0.75, 1.5) {};
%%		\node [style=dot] (8) at (-1.25, 1.5) {};
%%		\node [style=nothing] (9) at (-1.25, 2) {};
%%		\node [style=nothing] (10) at (-0.75, 2) {};
%%		\node [style=nothing] (11) at (-1.75, 2) {};
%%	\end{pgfonlayer}
%%	\begin{pgfonlayer}{edgelayer}
%%		\draw (0) to (3);
%%		\draw (1) to (4);
%%		\draw (2) to (5);
%%		\draw (3) to (4);
%%		\draw (4) to (5);
%%		\draw (6) to (8);
%%		\draw (8) to (7);
%%		\draw (3) to (6);
%%		\draw (6) to (11);
%%		\draw (4) to (8);
%%		\draw (8) to (9);
%%		\draw (5) to (7);
%%		\draw (7) to (10);
%%	\end{pgfonlayer}
%%\end{tikzpicture}
%%=
%%\begin{tikzpicture}
%%	\begin{pgfonlayer}{nodelayer}
%%		\node [style=nothing] (0) at (-1.75, 0.5) {};
%%		\node [style=nothing] (1) at (-1.25, 0.5) {};
%%		\node [style=nothing] (2) at (-0.75, 0.5) {};
%%		\node [style=nothing] (3) at (-1.25, 2) {};
%%		\node [style=nothing] (4) at (-0.75, 2) {};
%%		\node [style=nothing] (5) at (-1.75, 2) {};
%%	\end{pgfonlayer}
%%	\begin{pgfonlayer}{edgelayer}
%%		\draw (0) to (5);
%%		\draw (1) to (3);
%%		\draw (2) to (4);
%%	\end{pgfonlayer}
%%\end{tikzpicture}
%%$}
%%
%%\item
%%\label{TOF.10}
%%{\hfil
%%$
%%\begin{tikzpicture}
%%	\begin{pgfonlayer}{nodelayer}
%%		\node [style=nothing] (0) at (0, 0.5) {};
%%		\node [style=nothing] (1) at (-0.5, 0.5) {};
%%		\node [style=nothing] (2) at (-1, 0.5) {};
%%		\node [style=nothing] (3) at (-1.5, 0.5) {};
%%		\node [style=dot] (4) at (-1, 1) {};
%%		\node [style=dot] (5) at (-0.5, 1) {};
%%		\node [style=oplus] (6) at (0, 1) {};
%%		\node [style=dot] (7) at (-1.5, 1.5) {};
%%		\node [style=oplus] (8) at (-0.5, 1.5) {};
%%		\node [style=dot] (9) at (-1, 1.5) {};
%%		\node [style=dot] (10) at (-1, 2) {};
%%		\node [style=oplus] (11) at (0, 2) {};
%%		\node [style=dot] (12) at (-0.5, 2) {};
%%		\node [style=nothing] (13) at (-1.5, 2.5) {};
%%		\node [style=nothing] (14) at (-0.5, 2.5) {};
%%		\node [style=nothing] (15) at (-1, 2.5) {};
%%		\node [style=nothing] (16) at (0, 2.5) {};
%%	\end{pgfonlayer}
%%	\begin{pgfonlayer}{edgelayer}
%%		\draw (4) to (5);
%%		\draw (5) to (6);
%%		\draw (7) to (9);
%%		\draw (9) to (8);
%%		\draw (10) to (12);
%%		\draw (12) to (11);
%%		\draw (3) to (7);
%%		\draw (7) to (13);
%%		\draw (15) to (10);
%%		\draw (10) to (9);
%%		\draw (9) to (4);
%%		\draw (4) to (2);
%%		\draw (1) to (5);
%%		\draw (5) to (8);
%%		\draw (8) to (12);
%%		\draw (12) to (14);
%%		\draw (16) to (11);
%%		\draw (11) to (6);
%%		\draw (6) to (0);
%%	\end{pgfonlayer}
%%\end{tikzpicture}
%%=
%%\begin{tikzpicture}
%%	\begin{pgfonlayer}{nodelayer}
%%		\node [style=nothing] (17) at (4.5, 0.5) {};
%%		\node [style=nothing] (18) at (4, 0.5) {};
%%		\node [style=nothing] (19) at (3.5, 0.5) {};
%%		\node [style=nothing] (20) at (3, 0.5) {};
%%		\node [style=nothing] (21) at (3, 2.5) {};
%%		\node [style=nothing] (22) at (4, 2.5) {};
%%		\node [style=nothing] (23) at (3.5, 2.5) {};
%%		\node [style=nothing] (24) at (4.5, 2.5) {};
%%		\node [style=dot] (25) at (3, 1.25) {};
%%		\node [style=dot] (26) at (3.5, 1.25) {};
%%		\node [style=dot] (27) at (3, 1.75) {};
%%		\node [style=dot] (28) at (3.5, 1.75) {};
%%		\node [style=oplus] (29) at (4, 1.75) {};
%%		\node [style=oplus] (30) at (4.5, 1.25) {};
%%	\end{pgfonlayer}
%%	\begin{pgfonlayer}{edgelayer}
%%		\draw (20) to (25);
%%		\draw (25) to (27);
%%		\draw (27) to (21);
%%		\draw (23) to (28);
%%		\draw (28) to (26);
%%		\draw (26) to (19);
%%		\draw (18) to (29);
%%		\draw (29) to (22);
%%		\draw (24) to (30);
%%		\draw (30) to (17);
%%		\draw (30) to (26);
%%		\draw (26) to (25);
%%		\draw (27) to (28);
%%		\draw (28) to (29);
%%	\end{pgfonlayer}
%%\end{tikzpicture}
%%$}
%%
%%\item
%%\label{TOF.11}
%%{\hfil
%%$
%%\begin{tikzpicture}
%%	\begin{pgfonlayer}{nodelayer}
%%		\node [style=nothing] (0) at (0, 0.5) {};
%%		\node [style=nothing] (1) at (-0.5, 0.5) {};
%%		\node [style=nothing] (2) at (-1, 0.5) {};
%%		\node [style=nothing] (3) at (-1.5, 0.5) {};
%%		\node [style=nothing] (4) at (-0.5, 2.5) {};
%%		\node [style=nothing] (5) at (0, 2.5) {};
%%		\node [style=dot] (6) at (-1.5, 1) {};
%%		\node [style=dot] (7) at (-1, 1.5) {};
%%		\node [style=dot] (8) at (-0.5, 1.5) {};
%%		\node [style=oplus] (9) at (-1, 1) {};
%%		\node [style=oplus] (10) at (0, 1.5) {};
%%		\node [style=nothing] (11) at (-1.5, 2.5) {};
%%		\node [style=nothing] (12) at (-1, 2.5) {};
%%		\node [style=oplus] (13) at (-1, 2) {};
%%		\node [style=dot] (14) at (-1.5, 2) {};
%%	\end{pgfonlayer}
%%	\begin{pgfonlayer}{edgelayer}
%%		\draw (6) to (9);
%%		\draw (7) to (8);
%%		\draw (8) to (10);
%%		\draw (0) to (10);
%%		\draw (10) to (5);
%%		\draw (4) to (8);
%%		\draw (8) to (1);
%%		\draw (2) to (9);
%%		\draw (9) to (7);
%%		\draw (6) to (3);
%%		\draw (6) to (14);
%%		\draw (14) to (11);
%%		\draw (12) to (13);
%%		\draw (13) to (7);
%%		\draw (13) to (14);
%%	\end{pgfonlayer}
%%\end{tikzpicture}
%%=
%%\begin{tikzpicture}
%%	\begin{pgfonlayer}{nodelayer}
%%		\node [style=nothing] (0) at (2, 0.5) {};
%%		\node [style=nothing] (1) at (1, 0.5) {};
%%		\node [style=nothing] (2) at (1.5, 0.5) {};
%%		\node [style=nothing] (3) at (0.5, 0.5) {};
%%		\node [style=dot] (4) at (0.5, 1.25) {};
%%		\node [style=dot] (5) at (1.5, 1.25) {};
%%		\node [style=oplus] (6) at (2, 1.25) {};
%%		\node [style=nothing] (7) at (1.5, 2.5) {};
%%		\node [style=nothing] (8) at (1, 2.5) {};
%%		\node [style=nothing] (9) at (0.5, 2.5) {};
%%		\node [style=nothing] (10) at (2, 2.5) {};
%%		\node [style=dot] (11) at (1, 1.75) {};
%%		\node [style=dot] (12) at (1.5, 1.75) {};
%%		\node [style=oplus] (13) at (2, 1.75) {};
%%	\end{pgfonlayer}
%%	\begin{pgfonlayer}{edgelayer}
%%		\draw (3) to (4);
%%		\draw (2) to (5);
%%		\draw (6) to (0);
%%		\draw (6) to (5);
%%		\draw (5) to (4);
%%		\draw (11) to (1);
%%		\draw (5) to (12);
%%		\draw (12) to (7);
%%		\draw (10) to (13);
%%		\draw (13) to (6);
%%		\draw (13) to (12);
%%		\draw (12) to (11);
%%		\draw (4) to (9);
%%		\draw (11) to (8);
%%	\end{pgfonlayer}
%%\end{tikzpicture}
%%$}
%%
%%\item
%%\label{TOF.12}
%%{\hfil
%%$
%%\begin{tikzpicture}
%%	\begin{pgfonlayer}{nodelayer}
%%		\node [style=nothing] (0) at (-0.5, 0.5) {};
%%		\node [style=nothing] (1) at (0, 0.5) {};
%%		\node [style=nothing] (2) at (-1, 0.5) {};
%%		\node [style=nothing] (3) at (-1.5, 0.5) {};
%%		\node [style=nothing] (4) at (-0.5, 2.5) {};
%%		\node [style=nothing] (5) at (-1.5, 2.5) {};
%%		\node [style=nothing] (6) at (0, 2.5) {};
%%		\node [style=nothing] (7) at (-1, 2.5) {};
%%		\node [style=dot] (8) at (-1.5, 1) {};
%%		\node [style=dot] (9) at (-1, 1) {};
%%		\node [style=oplus] (10) at (-0.5, 1) {};
%%		\node [style=oplus] (11) at (0, 1.5) {};
%%		\node [style=dot] (12) at (-1, 1.5) {};
%%		\node [style=dot] (13) at (-0.5, 1.5) {};
%%		\node [style=oplus] (14) at (-0.5, 2) {};
%%		\node [style=dot] (15) at (-1.5, 2) {};
%%		\node [style=dot] (16) at (-1, 2) {};
%%	\end{pgfonlayer}
%%	\begin{pgfonlayer}{edgelayer}
%%		\draw (8) to (9);
%%		\draw (9) to (10);
%%		\draw (12) to (13);
%%		\draw (13) to (11);
%%		\draw (15) to (16);
%%		\draw (16) to (14);
%%		\draw (3) to (8);
%%		\draw (8) to (15);
%%		\draw (15) to (5);
%%		\draw (7) to (16);
%%		\draw (16) to (12);
%%		\draw (12) to (9);
%%		\draw (9) to (2);
%%		\draw (0) to (10);
%%		\draw (10) to (13);
%%		\draw (13) to (14);
%%		\draw (14) to (4);
%%		\draw (6) to (11);
%%		\draw (11) to (1);
%%	\end{pgfonlayer}
%%\end{tikzpicture}
%%=
%%\begin{tikzpicture}
%%	\begin{pgfonlayer}{nodelayer}
%%		\node [style=nothing] (0) at (1.5, 0.25) {};
%%		\node [style=nothing] (1) at (2, 0.25) {};
%%		\node [style=nothing] (2) at (1, 0.25) {};
%%		\node [style=nothing] (3) at (0.5, 0.25) {};
%%		\node [style=nothing] (4) at (1.5, 2.25) {};
%%		\node [style=nothing] (5) at (0.5, 2.25) {};
%%		\node [style=nothing] (6) at (2, 2.25) {};
%%		\node [style=nothing] (7) at (1, 2.25) {};
%%		\node [style=dot] (8) at (1, 1.5) {};
%%		\node [style=dot] (9) at (1.5, 1.5) {};
%%		\node [style=dot] (10) at (0.5, 1) {};
%%		\node [style=dot] (11) at (1, 1) {};
%%		\node [style=oplus] (12) at (2, 1) {};
%%		\node [style=oplus] (13) at (2, 1.5) {};
%%	\end{pgfonlayer}
%%	\begin{pgfonlayer}{edgelayer}
%%		\draw (8) to (9);
%%		\draw (3) to (10);
%%		\draw (10) to (5);
%%		\draw (2) to (11);
%%		\draw (11) to (8);
%%		\draw (8) to (7);
%%		\draw (0) to (9);
%%		\draw (9) to (4);
%%		\draw (1) to (12);
%%		\draw (12) to (13);
%%		\draw (13) to (6);
%%		\draw (13) to (9);
%%		\draw (12) to (11);
%%		\draw (11) to (10);
%%	\end{pgfonlayer}
%%\end{tikzpicture}
%%$}
%%
%%\item
%%\label{TOF.13}
%%{\hfil
%%$
%%\begin{tikzpicture}
%%	\begin{pgfonlayer}{nodelayer}
%%		\node [style=nothing] (0) at (0, 0.5) {};
%%		\node [style=nothing] (1) at (-1, 0.5) {};
%%		\node [style=nothing] (2) at (-0.5, 0.5) {};
%%		\node [style=nothing] (3) at (-1.5, 0.5) {};
%%		\node [style=nothing] (4) at (0, 2.5) {};
%%		\node [style=dot] (5) at (-1.5, 1) {};
%%		\node [style=dot] (6) at (-1, 1) {};
%%		\node [style=dot] (7) at (-0.5, 1.5) {};
%%		\node [style=oplus] (8) at (-0.5, 1) {};
%%		\node [style=oplus] (9) at (0, 1.5) {};
%%		\node [style=nothing] (10) at (-0.5, 2.5) {};
%%		\node [style=nothing] (11) at (-1.5, 2.5) {};
%%		\node [style=nothing] (12) at (-1, 2.5) {};
%%		\node [style=oplus] (13) at (-0.5, 2) {};
%%		\node [style=dot] (14) at (-1, 2) {};
%%		\node [style=dot] (15) at (-1.5, 2) {};
%%	\end{pgfonlayer}
%%	\begin{pgfonlayer}{edgelayer}
%%		\draw (5) to (3);
%%		\draw (6) to (1);
%%		\draw (2) to (8);
%%		\draw (8) to (7);
%%		\draw (4) to (9);
%%		\draw (9) to (0);
%%		\draw (8) to (6);
%%		\draw (6) to (5);
%%		\draw (9) to (7);
%%		\draw (5) to (15);
%%		\draw (15) to (11);
%%		\draw (12) to (14);
%%		\draw (14) to (6);
%%		\draw (7) to (13);
%%		\draw (13) to (10);
%%		\draw (13) to (14);
%%		\draw (14) to (15);
%%	\end{pgfonlayer}
%%\end{tikzpicture}
%%=
%%\begin{tikzpicture}
%%	\begin{pgfonlayer}{nodelayer}
%%		\node [style=nothing] (0) at (2, 0.25) {};
%%		\node [style=nothing] (1) at (1, 0.25) {};
%%		\node [style=nothing] (2) at (1.5, 0.25) {};
%%		\node [style=nothing] (3) at (0.5, 0.25) {};
%%		\node [style=dot] (4) at (0.5, 1) {};
%%		\node [style=dot] (5) at (1, 1) {};
%%		\node [style=oplus] (6) at (2, 1) {};
%%		\node [style=nothing] (7) at (1.5, 2.25) {};
%%		\node [style=nothing] (8) at (1, 2.25) {};
%%		\node [style=nothing] (9) at (2, 2.25) {};
%%		\node [style=nothing] (10) at (0.5, 2.25) {};
%%		\node [style=dot] (11) at (1.5, 1.5) {};
%%		\node [style=oplus] (12) at (2, 1.5) {};
%%	\end{pgfonlayer}
%%	\begin{pgfonlayer}{edgelayer}
%%		\draw (0) to (6);
%%		\draw (1) to (5);
%%		\draw (4) to (3);
%%		\draw (4) to (5);
%%		\draw (5) to (6);
%%		\draw (11) to (12);
%%		\draw (12) to (9);
%%		\draw (12) to (6);
%%		\draw (2) to (11);
%%		\draw (4) to (10);
%%		\draw (8) to (5);
%%		\draw (11) to (7);
%%	\end{pgfonlayer}
%%\end{tikzpicture}
%%$}
%%
%%\item
%%\label{TOF.14}
%%{\hfil
%%$
%%\begin{tikzpicture}
%%	\begin{pgfonlayer}{nodelayer}
%%		\node [style=nothing] (0) at (0, 0.5) {};
%%		\node [style=nothing] (1) at (-0.5, 0.5) {};
%%		\node [style=nothing] (2) at (-0.5, 2.5) {};
%%		\node [style=nothing] (3) at (0, 2.5) {};
%%		\node [style=oplus] (4) at (0, 1) {};
%%		\node [style=oplus] (5) at (0, 2) {};
%%		\node [style=oplus] (6) at (-0.5, 1.5) {};
%%		\node [style=dot] (7) at (-0.5, 2) {};
%%		\node [style=dot] (8) at (0, 1.5) {};
%%		\node [style=dot] (9) at (-0.5, 1) {};
%%	\end{pgfonlayer}
%%	\begin{pgfonlayer}{edgelayer}
%%		\draw (1) to (9);
%%		\draw (9) to (6);
%%		\draw (6) to (7);
%%		\draw (7) to (2);
%%		\draw (3) to (5);
%%		\draw (5) to (8);
%%		\draw (8) to (4);
%%		\draw (4) to (0);
%%		\draw (4) to (9);
%%		\draw (8) to (6);
%%		\draw (5) to (7);
%%	\end{pgfonlayer}
%%\end{tikzpicture}
%%=
%%\begin{tikzpicture}
%%	\begin{pgfonlayer}{nodelayer}
%%		\node [style=nothing] (0) at (1, 0.5) {};
%%		\node [style=nothing] (1) at (0.5, 0.5) {};
%%		\node [style=nothing] (2) at (0.5, 2.5) {};
%%		\node [style=nothing] (3) at (1, 2.5) {};
%%	\end{pgfonlayer}
%%	\begin{pgfonlayer}{edgelayer}
%%		\draw [in=-90, out=90, looseness=1.25] (1) to (3);
%%		\draw [in=-90, out=90, looseness=1.25] (0) to (2);
%%	\end{pgfonlayer}
%%\end{tikzpicture}
%%$}
%%
%%\item
%%\label{TOF.15}
%%{\hfil
%%$
%%\begin{tikzpicture}
%%	\begin{pgfonlayer}{nodelayer}
%%		\node [style=nothing] (0) at (-1.75, 0.5) {};
%%		\node [style=nothing] (1) at (-1.25, 0.5) {};
%%		\node [style=nothing] (2) at (-0.75, 0.5) {};
%%		\node [style=nothing] (3) at (-1.75, 2.5) {};
%%		\node [style=nothing] (4) at (-1.25, 2.5) {};
%%		\node [style=nothing] (5) at (-0.75, 2.5) {};
%%		\node [style=dot] (6) at (-1.75, 1.5) {};
%%		\node [style=dot] (7) at (-1.25, 1.5) {};
%%		\node [style=oplus] (8) at (-0.75, 1.5) {};
%%	\end{pgfonlayer}
%%	\begin{pgfonlayer}{edgelayer}
%%		\draw (0) to (6);
%%		\draw (6) to (3);
%%		\draw (4) to (7);
%%		\draw (7) to (1);
%%		\draw (2) to (8);
%%		\draw (8) to (5);
%%		\draw (8) to (7);
%%		\draw (7) to (6);
%%	\end{pgfonlayer}
%%\end{tikzpicture}
%%=
%%\begin{tikzpicture}
%%	\begin{pgfonlayer}{nodelayer}
%%		\node [style=nothing] (0) at (-1.75, 0.5) {};
%%		\node [style=nothing] (1) at (-1.25, 0.5) {};
%%		\node [style=nothing] (2) at (-0.75, 0.5) {};
%%		\node [style=dot] (3) at (-1.75, 1.5) {};
%%		\node [style=dot] (4) at (-1.25, 1.5) {};
%%		\node [style=oplus] (5) at (-0.75, 1.5) {};
%%		\node [style=nothing] (6) at (-1.75, 2.5) {};
%%		\node [style=nothing] (7) at (-1.25, 2.5) {};
%%		\node [style=nothing] (8) at (-0.75, 2.5) {};
%%	\end{pgfonlayer}
%%	\begin{pgfonlayer}{edgelayer}
%%		\draw [in=-90, out=90, looseness=1.25] (0) to (4);
%%		\draw [in=-90, out=90, looseness=1.25] (4) to (6);
%%		\draw [in=-90, out=90, looseness=1.25] (3) to (7);
%%		\draw [in=90, out=-90, looseness=1.25] (3) to (1);
%%		\draw (2) to (5);
%%		\draw (5) to (8);
%%		\draw (3) to (4);
%%		\draw (4) to (5);
%%	\end{pgfonlayer}
%%\end{tikzpicture}
%%$}
%%
%%\item
%%\label{TOF.16}
%%{\hfil
%%$
%%\begin{tikzpicture}
%%	\begin{pgfonlayer}{nodelayer}
%%		\node [style=nothing] (0) at (2.5, 0.5) {};
%%		\node [style=nothing] (1) at (1, 0.5) {};
%%		\node [style=nothing] (2) at (2, 0.5) {};
%%		\node [style=nothing] (3) at (0.5, 0.5) {};
%%		\node [style=X] (4) at (1.5, 0.75) {};
%%		\node [style=oplus] (5) at (1.5, 1.75) {};
%%		\node [style=oplus] (6) at (1.5, 2.75) {};
%%		\node [style=dot] (7) at (1.5, 2.25) {};
%%		\node [style=dot] (8) at (2, 2.25) {};
%%		\node [style=dot] (9) at (1, 1.75) {};
%%		\node [style=dot] (10) at (0.5, 1.75) {};
%%		\node [style=dot] (11) at (1, 2.75) {};
%%		\node [style=dot] (12) at (0.5, 2.75) {};
%%		\node [style=oplus] (13) at (2.5, 2.25) {};
%%		\node [style=X] (14) at (1.5, 3.75) {};
%%		\node [style=nothing] (15) at (2.5, 4) {};
%%		\node [style=nothing] (16) at (0.5, 4) {};
%%		\node [style=nothing] (17) at (1, 4) {};
%%		\node [style=nothing] (18) at (2, 4) {};
%%	\end{pgfonlayer}
%%	\begin{pgfonlayer}{edgelayer}
%%		\draw (3) to (10);
%%		\draw (10) to (12);
%%		\draw (12) to (16);
%%		\draw (11) to (9);
%%		\draw (15) to (13);
%%		\draw (13) to (0);
%%		\draw (13) to (8);
%%		\draw (8) to (7);
%%		\draw (9) to (5);
%%		\draw (9) to (10);
%%		\draw (12) to (11);
%%		\draw (6) to (11);
%%		\draw (4) to (5);
%%		\draw (5) to (7);
%%		\draw (7) to (6);
%%		\draw (14) to (6);
%%		\draw [style=simple, in=90, out=-90, looseness=1.25] (17) to (8);
%%		\draw [style=simple, in=90, out=-90, looseness=1.25] (8) to (1);
%%		\draw [style=simple, in=270, out=90] (2) to (9);
%%		\draw [style=simple, in=270, out=90] (11) to (18);
%%	\end{pgfonlayer}
%%\end{tikzpicture}
%%=
%%\begin{tikzpicture}
%%	\begin{pgfonlayer}{nodelayer}
%%		\node [style=nothing] (0) at (2.5, 0.5) {};
%%		\node [style=nothing] (1) at (2, 0.5) {};
%%		\node [style=nothing] (2) at (1, 0.5) {};
%%		\node [style=nothing] (3) at (0.5, 0.5) {};
%%		\node [style=X] (4) at (1.5, 1.25) {};
%%		\node [style=oplus] (5) at (1.5, 1.75) {};
%%		\node [style=oplus] (6) at (1.5, 2.75) {};
%%		\node [style=dot] (7) at (1.5, 2.25) {};
%%		\node [style=dot] (8) at (2, 2.25) {};
%%		\node [style=dot] (9) at (1, 1.75) {};
%%		\node [style=dot] (10) at (0.5, 1.75) {};
%%		\node [style=dot] (11) at (1, 2.75) {};
%%		\node [style=dot] (12) at (0.5, 2.75) {};
%%		\node [style=oplus] (13) at (2.5, 2.25) {};
%%		\node [style=X] (14) at (1.5, 3.25) {};
%%		\node [style=nothing] (15) at (2.5, 4) {};
%%		\node [style=nothing] (16) at (0.5, 4) {};
%%		\node [style=nothing] (17) at (2, 4) {};
%%		\node [style=nothing] (18) at (1, 4) {};
%%	\end{pgfonlayer}
%%	\begin{pgfonlayer}{edgelayer}
%%		\draw (3) to (10);
%%		\draw (10) to (12);
%%		\draw (12) to (16);
%%		\draw (11) to (9);
%%		\draw (15) to (13);
%%		\draw (13) to (0);
%%		\draw (13) to (8);
%%		\draw (8) to (7);
%%		\draw (9) to (5);
%%		\draw (9) to (10);
%%		\draw (12) to (11);
%%		\draw (6) to (11);
%%		\draw (4) to (5);
%%		\draw (5) to (7);
%%		\draw (7) to (6);
%%		\draw (14) to (6);
%%		\draw [style=simple] (17) to (8);
%%		\draw [style=simple] (8) to (1);
%%		\draw [style=simple] (2) to (9);
%%		\draw [style=simple] (11) to (18);
%%	\end{pgfonlayer}
%%\end{tikzpicture}
%%$}
%%\end{enumerate}
%%\end{multicols}
%%
%%Where the controlled-not gate is derived:
%%$$
%%\begin{tikzpicture}
%%	\begin{pgfonlayer}{nodelayer}
%%		\node [style=dot] (1) at (1.5, 1) {};
%%		\node [style=oplus] (2) at (2, 1) {};
%%		\node [style=none] (5) at (1.5, 0.25) {};
%%		\node [style=none] (6) at (2, 0.25) {};
%%		\node [style=none] (7) at (2, 1.75) {};
%%		\node [style=none] (8) at (1.5, 1.75) {};
%%	\end{pgfonlayer}
%%	\begin{pgfonlayer}{edgelayer}
%%		\draw (5.center) to (1);
%%		\draw (1) to (8.center);
%%		\draw (7.center) to (2);
%%		\draw (2) to (6.center);
%%		\draw (2) to (1);
%%	\end{pgfonlayer}
%%\end{tikzpicture}
%%:=
%%\begin{tikzpicture}
%%	\begin{pgfonlayer}{nodelayer}
%%		\node [style=dot] (0) at (1, 1) {};
%%		\node [style=dot] (1) at (1.5, 1) {};
%%		\node [style=oplus] (2) at (2, 1) {};
%%		\node [style=X] (3) at (1, 1.5) {$1$};
%%		\node [style=X] (4) at (1, 0.5) {$1$};
%%		\node [style=none] (5) at (1.5, 0.25) {};
%%		\node [style=none] (6) at (2, 0.25) {};
%%		\node [style=none] (7) at (2, 1.75) {};
%%		\node [style=none] (8) at (1.5, 1.75) {};
%%	\end{pgfonlayer}
%%	\begin{pgfonlayer}{edgelayer}
%%		\draw (5.center) to (1);
%%		\draw (1) to (8.center);
%%		\draw (7.center) to (2);
%%		\draw (2) to (6.center);
%%		\draw (2) to (1);
%%		\draw (1) to (0);
%%		\draw (0) to (3);
%%		\draw (4) to (0);
%%	\end{pgfonlayer}
%%\end{tikzpicture}
%%$$
%%
%%\subsubsection{$(\FPar_2,\times)$}
%%\label{subsubsec:presentations:three:par}
%%$(\FPar_2,\times)$ is presented by the generators and equations in \S \ref{subsubsec:presentations:two:par} as well as the additional generator $
%%\begin{tikzpicture}
%%	\begin{pgfonlayer}{nodelayer}
%%		\node [style=none] (0) at (-3.75, 0.5) {};
%%		\node [style=none] (1) at (-3.75, -0.25) {};
%%		\node [style=andin] (2) at (-3.75, -0.25) {};
%%		\node [style=none] (3) at (-4, -1) {};
%%		\node [style=none] (4) at (-3.5, -1) {};
%%	\end{pgfonlayer}
%%	\begin{pgfonlayer}{edgelayer}
%%		\draw (0.center) to (1.center);
%%		\draw [in=-60, out=90, looseness=1.00] (4.center) to (1.center);
%%		\draw [in=90, out=-120, looseness=1.00] (1.center) to (3.center);
%%	\end{pgfonlayer}
%%\end{tikzpicture}
%%$ (the and gate),  so that 
%%$
%%\left(
%%\begin{tikzpicture}
%%	\begin{pgfonlayer}{nodelayer}
%%		\node [style=none] (0) at (-3.75, 0.5) {};
%%		\node [style=none] (1) at (-3.75, -0.25) {};
%%		\node [style=andin] (2) at (-3.75, -0.25) {};
%%		\node [style=none] (3) at (-4, -1) {};
%%		\node [style=none] (4) at (-3.5, -1) {};
%%	\end{pgfonlayer}
%%	\begin{pgfonlayer}{edgelayer}
%%		\draw (0.center) to (1.center);
%%		\draw [in=-60, out=90, looseness=1.00] (4.center) to (1.center);
%%		\draw [in=90, out=-120, looseness=1.00] (1.center) to (3.center);
%%	\end{pgfonlayer}
%%\end{tikzpicture},
%%\begin{tikzpicture}
%%	\begin{pgfonlayer}{nodelayer}
%%		\node [style=none] (0) at (0.75, 0.5) {};
%%		\node [style=none] (1) at (0.75, -0.25) {};
%%		\node [style=X] (2) at (0.75, -0.25) {$1$};
%%	\end{pgfonlayer}
%%	\begin{pgfonlayer}{edgelayer}
%%		\draw (0.center) to (1.center);
%%	\end{pgfonlayer}
%%\end{tikzpicture},
%%\begin{tikzpicture}
%%	\begin{pgfonlayer}{nodelayer}
%%		\node [style=none] (0) at (0.75, -1) {};
%%		\node [style=none] (1) at (0.75, -0.25) {};
%%		\node [style=Z] (2) at (0.75, -0.25) {};
%%		\node [style=none] (3) at (0.5, 0.5) {};
%%		\node [style=none] (4) at (1, 0.5) {};
%%	\end{pgfonlayer}
%%	\begin{pgfonlayer}{edgelayer}
%%		\draw (0.center) to (1.center);
%%		\draw [in=60, out=-90] (4.center) to (1.center);
%%		\draw [in=-90, out=120] (1.center) to (3.center);
%%	\end{pgfonlayer}
%%\end{tikzpicture},
%%\begin{tikzpicture}
%%	\begin{pgfonlayer}{nodelayer}
%%		\node [style=none] (0) at (0.75, -0.25) {};
%%		\node [style=none] (1) at (0.75, 0.5) {};
%%		\node [style=Z] (2) at (0.75, 0.5) {};
%%	\end{pgfonlayer}
%%	\begin{pgfonlayer}{edgelayer}
%%		\draw (0.center) to (1.center);
%%	\end{pgfonlayer}
%%\end{tikzpicture}
%%\right)
%%$ forms a bicommutative bialgebra; and additionally:
%%$$
%%\begin{tikzpicture}
%%	\begin{pgfonlayer}{nodelayer}
%%		\node [style=none] (0) at (-7, 1) {};
%%		\node [style=none] (1) at (-7, 0.5) {};
%%		\node [style=Z] (2) at (-7, -0.25) {};
%%		\node [style=none] (3) at (-7, -0.75) {};
%%		\node [style=andin] (4) at (-7, 0.5) {};
%%	\end{pgfonlayer}
%%	\begin{pgfonlayer}{edgelayer}
%%		\draw (3.center) to (2.center);
%%		\draw [in=-60, out=60, looseness=1.25] (2.center) to (1);
%%		\draw [in=120, out=-120, looseness=1.25] (1) to (2.center);
%%		\draw (1) to (0.center);
%%	\end{pgfonlayer}
%%\end{tikzpicture}
%%\eref{antispecial}
%%\begin{tikzpicture}
%%	\begin{pgfonlayer}{nodelayer}
%%		\node [style=none] (0) at (-7, 1) {};
%%		\node [style=none] (1) at (-7, -0.75) {};
%%	\end{pgfonlayer}
%%	\begin{pgfonlayer}{edgelayer}
%%		\draw (1.center) to (0.center);
%%	\end{pgfonlayer}
%%\end{tikzpicture},
%%\hspace*{.5cm}
%%\begin{tikzpicture}
%%	\begin{pgfonlayer}{nodelayer}
%%		\node [style=andin] (4) at (1.25, 0.5) {};
%%		\node [style=X] (5) at (0.75, -0.5) {};
%%		\node [style=none] (6) at (0.5, -1) {};
%%		\node [style=none] (7) at (1, -1) {};
%%		\node [style=none] (8) at (1.75, -1) {};
%%		\node [style=none] (9) at (1.25, 0.5) {};
%%		\node [style=none] (10) at (1.25, 1.5) {};
%%	\end{pgfonlayer}
%%	\begin{pgfonlayer}{edgelayer}
%%		\draw [in=-30, out=90] (8.center) to (9.center);
%%		\draw [in=90, out=-150] (9.center) to (5);
%%		\draw [in=90, out=-45] (5) to (7.center);
%%		\draw [in=-135, out=90] (6.center) to (5);
%%		\draw (9.center) to (10.center);
%%	\end{pgfonlayer}
%%\end{tikzpicture}
%%  \eref{ring.mul}
%%\begin{tikzpicture}
%%	\begin{pgfonlayer}{nodelayer}
%%		\node [style=none] (0) at (1, 0) {};
%%		\node [style=none] (1) at (0.5, -1.25) {};
%%		\node [style=none] (2) at (1.75, -0.75) {};
%%		\node [style=none] (3) at (1.33, 0.75) {};
%%		\node [style=andin] (4) at (1, 0) {};
%%		\node [style=none] (5) at (1.75, 0) {};
%%		\node [style=none] (6) at (1, -1.25) {};
%%		\node [style=none] (7) at (1.75, -0.75) {};
%%		\node [style=none] (8) at (1.33, 0.75) {};
%%		\node [style=andin] (9) at (1.75, 0) {};
%%		\node [style=X] (10) at (1.33, 0.75) {};
%%		\node [style=none] (11) at (1.33, 1.25) {};
%%		\node [style=none] (12) at (1.75, -1.25) {};
%%		\node [style=Z] (13) at (1.75, -0.75) {};
%%	\end{pgfonlayer}
%%	\begin{pgfonlayer}{edgelayer}
%%		\draw [in=-135, out=90] (0.center) to (3.center);
%%		\draw [in=165, out=-30, looseness=1.25] (0.center) to (2.center);
%%		\draw [in=-45, out=90] (5.center) to (8.center);
%%		\draw [in=45, out=-45, looseness=1.25] (5.center) to (7.center);
%%		\draw (10) to (11.center);
%%		\draw [in=90, out=-150] (4) to (1.center);
%%		\draw [in=-150, out=90] (6.center) to (9);
%%		\draw (12.center) to (13);
%%	\end{pgfonlayer}
%%\end{tikzpicture},
%%\hspace*{.5cm}
%%\begin{tikzpicture}
%%	\begin{pgfonlayer}{nodelayer}
%%		\node [style=none] (0) at (2, 0) {};
%%		\node [style=none] (1) at (1.75, -0.75) {};
%%		\node [style=none] (2) at (2.25, -0.75) {};
%%		\node [style=none] (3) at (2, 0.5) {};
%%		\node [style=none] (4) at (2.25, -1) {};
%%		\node [style=X] (5) at (1.75, -0.75) {};
%%		\node [style=andin] (6) at (2, 0) {};
%%	\end{pgfonlayer}
%%	\begin{pgfonlayer}{edgelayer}
%%		\draw (0.center) to (3.center);
%%		\draw [in=90, out=-45] (0.center) to (2.center);
%%		\draw (4.center) to (2.center);
%%		\draw [in=-135, out=90] (1.center) to (0.center);
%%	\end{pgfonlayer}
%%\end{tikzpicture}
%%\eref{ring.unit}
%%\begin{tikzpicture}
%%	\begin{pgfonlayer}{nodelayer}
%%		\node [style=none] (12) at (2, 0.5) {};
%%		\node [style=none] (14) at (2, -1) {};
%%		\node [style=X] (15) at (2, 0) {};
%%		\node [style=Z] (16) at (2, -0.5) {};
%%	\end{pgfonlayer}
%%	\begin{pgfonlayer}{edgelayer}
%%		\draw (15) to (12.center);
%%		\draw (16) to (14.center);
%%	\end{pgfonlayer}
%%\end{tikzpicture},
%%\hspace*{.5cm}
%%\begin{tikzpicture}
%%	\begin{pgfonlayer}{nodelayer}
%%		\node [style=none] (0) at (0.75, 0.5) {};
%%		\node [style=none] (1) at (0.75, -0.25) {};
%%		\node [style=andin] (2) at (0.75, -0.25) {};
%%		\node [style=none] (3) at (0.5, -1) {};
%%		\node [style=none] (4) at (1, -1) {};
%%		\node [style=X] (5) at (0.75, 0.5) {$1$};
%%	\end{pgfonlayer}
%%	\begin{pgfonlayer}{edgelayer}
%%		\draw (0.center) to (1.center);
%%		\draw [in=-60, out=90] (4.center) to (1.center);
%%		\draw [in=90, out=-120] (1.center) to (3.center);
%%	\end{pgfonlayer}
%%\end{tikzpicture}
%%  \eref{bi.two}
%%\begin{tikzpicture}
%%	\begin{pgfonlayer}{nodelayer}
%%		\node [style=none] (3) at (0.5, -1) {};
%%		\node [style=none] (4) at (1, -1) {};
%%		\node [style=X] (5) at (0.5, 0.5) {$1$};
%%		\node [style=X] (6) at (1, 0.5) {$1$};
%%	\end{pgfonlayer}
%%	\begin{pgfonlayer}{edgelayer}
%%		\draw (3.center) to (5);
%%		\draw (6) to (4.center);
%%	\end{pgfonlayer}
%%\end{tikzpicture}
%%$$
%\subsubsection{$(\FSpan_2,\times)$}
%\label{subsubsec:presentations:three:span}
%
%$(\FSpan_2,\times)$ is presented by the generators and equations of \S \ref{subsubsec:presentations:three:span} as well as the generator 
%$\begin{tikzpicture}
%	\begin{pgfonlayer}{nodelayer}
%		\node [style=none] (0) at (0.75, 0.5) {};
%		\node [style=none] (1) at (0.75, -0.25) {};
%		\node [style=Z] (2) at (0.75, -0.25) {};
%	\end{pgfonlayer}
%	\begin{pgfonlayer}{edgelayer}
%		\draw (0.center) to (1.center);
%	\end{pgfonlayer}
%\end{tikzpicture}$ 
%and the equation making the codiagonal map counital:
%$$
%  \begin{tikzpicture}[rotate=90,yscale=-1]
%	\begin{pgfonlayer}{nodelayer}
%		\node [style=Z] (0) at (-9, -0) {};
%		\node [style=none] (1) at (-8.25, -0) {};
%		\node [style=Z] (2) at (-9.75, 0.25) {};
%		\node [style=none] (3) at (-10, -0.25) {};
%	\end{pgfonlayer}
%	\begin{pgfonlayer}{edgelayer}
%		\draw [in=-150, out=0, looseness=1.00] (3.center) to (0);
%		\draw [in=150, out=0, looseness=1.00] (2.center) to (0);
%		\draw (0) to (1.center);
%	\end{pgfonlayer}
%  \end{tikzpicture}
%  \eref{unit}
%  \begin{tikzpicture}[rotate=90]
%	\begin{pgfonlayer}{nodelayer}
%		\node [style=none] (0) at (-9, 0.25) {};
%		\node [style=none] (1) at (-9.75, 0.25) {};
%	\end{pgfonlayer}
%	\begin{pgfonlayer}{edgelayer}
%		\draw (1) to (0.center);
%	\end{pgfonlayer}
%  \end{tikzpicture}
%$$
%%
%%
%%\end{comment}




%
%This prop is equivalently presented in terms of the 
%
%\begin{figure}[H]
%	\noindent
%	\scalebox{1.0}{%
%		\vbox{%
%			\begin{mdframed}
%				\begin{multicols}{2}
%					\begin{enumerate}[label={\bf [CNOT.\arabic*]}, ref={\bf [CNOT.\arabic*]}, wide = 0pt, leftmargin = 2em]
%						\item
%						\label{CNOT.1}
%						{\hfil
%							$
%			\begin{tikzpicture}
%	\begin{pgfonlayer}{nodelayer}
%		\node [style=nothing] (26) at (0, 6) {};
%		\node [style=nothing] (27) at (-0.5, 6) {};
%		\node [style=oplus] (28) at (0, 6.5) {};
%		\node [style=dot] (29) at (-0.5, 6.5) {};
%		\node [style=dot] (30) at (0, 7) {};
%		\node [style=oplus] (31) at (-0.5, 7) {};
%		\node [style=oplus] (32) at (0, 7.5) {};
%		\node [style=dot] (33) at (-0.5, 7.5) {};
%		\node [style=nothing] (34) at (0, 8) {};
%		\node [style=nothing] (35) at (-0.5, 8) {};
%	\end{pgfonlayer}
%	\begin{pgfonlayer}{edgelayer}
%		\draw [style=simple] (26) to (34);
%		\draw [style=simple] (27) to (35);
%		\draw [style=simple] (28) to (29);
%		\draw [style=simple] (30) to (31);
%		\draw [style=simple] (32) to (33);
%	\end{pgfonlayer}
%\end{tikzpicture}
%							=
%							\begin{tikzpicture}
%	\begin{pgfonlayer}{nodelayer}
%		\node [style=nothing] (0) at (0, 0.5) {};
%		\node [style=nothing] (1) at (-0.5, 0.5) {};
%		\node [style=nothing] (2) at (-0.5, 1.5) {};
%		\node [style=nothing] (3) at (0, 1.5) {};
%	\end{pgfonlayer}
%	\begin{pgfonlayer}{edgelayer}
%		\draw [in=-90, out=90, looseness=1.25] (1) to (3);
%		\draw [in=-90, out=90, looseness=1.25] (0) to (2);
%	\end{pgfonlayer}
%\end{tikzpicture}
%$}
%						
%						
%						\item
%						\label{CNOT.2}
%						\hfil{
%							$
%							\begin{tikzpicture}
%	\begin{pgfonlayer}{nodelayer}
%		\node [style=nothing] (1) at (0, 0) {};
%		\node [style=nothing] (2) at (-0.5, 0) {};
%		\node [style=oplus] (3) at (0, 0.5) {};
%		\node [style=dot] (4) at (-0.5, 0.5) {};
%		\node [style=oplus] (5) at (0, 1) {};
%		\node [style=dot] (6) at (-0.5, 1) {};
%		\node [style=nothing] (7) at (0, 1.5) {};
%		\node [style=nothing] (8) at (-0.5, 1.5) {};
%	\end{pgfonlayer}
%	\begin{pgfonlayer}{edgelayer}
%		\draw [style=simple] (1) to (7);
%		\draw [style=simple] (2) to (8);
%		\draw [style=simple] (3) to (4);
%		\draw [style=simple] (5) to (6);
%	\end{pgfonlayer}
%\end{tikzpicture}
%							=
%							\begin{tikzpicture}
%	\begin{pgfonlayer}{nodelayer}
%		\node [style=nothing] (2) at (0, 0) {};
%		\node [style=nothing] (3) at (-0.5, 0) {};
%		\node [style=nothing] (4) at (0, 1.5) {};
%		\node [style=nothing] (5) at (-0.5, 1.5) {};
%	\end{pgfonlayer}
%	\begin{pgfonlayer}{edgelayer}
%		\draw [style=simple] (2) to (4);
%		\draw [style=simple] (3) to (5);
%	\end{pgfonlayer}
%\end{tikzpicture}
%							$}
%						
%						\item
%						\label{CNOT.3}
%						\hfil{
%							$
%							\begin{tikzpicture}
%	\begin{pgfonlayer}{nodelayer}
%		\node [style=nothing] (3) at (-1, 0) {};
%		\node [style=nothing] (4) at (-0.5, 0) {};
%		\node [style=nothing] (5) at (0, 0) {};
%		\node [style=oplus] (6) at (-1, 0.75) {};
%		\node [style=dot] (7) at (-0.5, 0.75) {};
%		\node [style=dot] (8) at (-0.5, 1.25) {};
%		\node [style=oplus] (9) at (0, 1.25) {};
%		\node [style=nothing] (10) at (-1, 2) {};
%		\node [style=nothing] (11) at (-0.5, 2) {};
%		\node [style=nothing] (12) at (0, 2) {};
%	\end{pgfonlayer}
%	\begin{pgfonlayer}{edgelayer}
%		\draw [style=simple] (3) to (10);
%		\draw [style=simple] (4) to (11);
%		\draw [style=simple] (5) to (12);
%		\draw [style=simple] (6) to (7);
%		\draw [style=simple] (8) to (9);
%	\end{pgfonlayer}
%\end{tikzpicture}
%							=
%							\begin{tikzpicture}
%	\begin{pgfonlayer}{nodelayer}
%		\node [style=nothing] (4) at (-1, 2.75) {};
%		\node [style=nothing] (5) at (-0.5, 2.75) {};
%		\node [style=nothing] (6) at (0, 2.75) {};
%		\node [style=oplus] (7) at (-1, 4) {};
%		\node [style=dot] (8) at (-0.5, 4) {};
%		\node [style=dot] (9) at (-0.5, 3.5) {};
%		\node [style=oplus] (10) at (0, 3.5) {};
%		\node [style=nothing] (11) at (-1, 4.75) {};
%		\node [style=nothing] (12) at (-0.5, 4.75) {};
%		\node [style=nothing] (13) at (0, 4.75) {};
%	\end{pgfonlayer}
%	\begin{pgfonlayer}{edgelayer}
%		\draw [style=simple] (4) to (11);
%		\draw [style=simple] (5) to (12);
%		\draw [style=simple] (6) to (13);
%		\draw [style=simple] (7) to (8);
%		\draw [style=simple] (9) to (10);
%	\end{pgfonlayer}
%\end{tikzpicture}
%							$}
%						
%						\item 
%						\label{CNOT.4}
%						\hfil{
%							\begin{tabular}{c}
%							$
%							\begin{tikzpicture}
%	\begin{pgfonlayer}{nodelayer}
%		\node [style=onein] (5) at (-0.5, 2.75) {};
%		\node [style=nothing] (6) at (0, 2.75) {};
%		\node [style=dot] (7) at (-0.5, 3.25) {};
%		\node [style=oplus] (8) at (0, 3.25) {};
%		\node [style=nothing] (9) at (-0.5, 3.75) {};
%		\node [style=nothing] (10) at (0, 3.75) {};
%	\end{pgfonlayer}
%	\begin{pgfonlayer}{edgelayer}
%		\draw [style=simple] (5) to (9);
%		\draw [style=simple] (6) to (10);
%		\draw [style=simple] (7) to (8);
%	\end{pgfonlayer}
%\end{tikzpicture}
%							=
%							\begin{tikzpicture}
%	\begin{pgfonlayer}{nodelayer}
%		\node [style=onein] (6) at (-0.5, 2.75) {};
%		\node [style=nothing] (7) at (0, 2.75) {};
%		\node [style=dot] (8) at (-0.5, 3.25) {};
%		\node [style=oplus] (9) at (0, 3.25) {};
%		\node [style=oneout] (10) at (-0.5, 3.75) {};
%		\node [style=nothing] (11) at (0, 4.75) {};
%		\node [style=onein] (12) at (-0.5, 4.25) {};
%		\node [style=nothing] (13) at (-0.5, 4.75) {};
%	\end{pgfonlayer}
%	\begin{pgfonlayer}{edgelayer}
%		\draw [style=simple] (6) to (10);
%		\draw [style=simple] (7) to (11);
%		\draw [style=simple] (8) to (9);
%		\draw [style=simple] (12) to (13);
%	\end{pgfonlayer}
%\end{tikzpicture}$\\
%							$ $\\
%							$\begin{tikzpicture}[tikzfig]
%	\begin{pgfonlayer}{nodelayer}
%		\node [style=nothing] (0) at (-0.5, 0.5) {};
%		\node [style=nothing] (1) at (0, 0.5) {};
%		\node [style=dot] (2) at (-0.5, 1) {};
%		\node [style=oplus] (3) at (0, 1) {};
%		\node [style=oneout] (4) at (-0.5, 1.5) {};
%		\node [style=nothing] (5) at (0, 1.5) {};
%	\end{pgfonlayer}
%	\begin{pgfonlayer}{edgelayer}
%		\draw [style=simple] (0) to (4);
%		\draw [style=simple] (1) to (5);
%		\draw [style=simple] (2) to (3);
%	\end{pgfonlayer}
%\end{tikzpicture}
%							=
%							\begin{tikzpicture}
%	\begin{pgfonlayer}{nodelayer}
%		\node [style=oneout] (8) at (-0.5, 7.25) {};
%		\node [style=nothing] (9) at (0, 7.25) {};
%		\node [style=dot] (10) at (-0.5, 6.75) {};
%		\node [style=oplus] (11) at (0, 6.75) {};
%		\node [style=onein] (12) at (-0.5, 6.25) {};
%		\node [style=nothing] (13) at (0, 5.25) {};
%		\node [style=oneout] (14) at (-0.5, 5.75) {};
%		\node [style=nothing] (15) at (-0.5, 5.25) {};
%	\end{pgfonlayer}
%	\begin{pgfonlayer}{edgelayer}
%		\draw [style=simple] (8) to (12);
%		\draw [style=simple] (9) to (13);
%		\draw [style=simple] (10) to (11);
%		\draw [style=simple] (14) to (15);
%	\end{pgfonlayer}
%\end{tikzpicture}$
%							\end{tabular}
%							}
%						
%						\item 
%						\label{CNOT.5}
%						\hfil{
%							$
%							\begin{tikzpicture}
%	\begin{pgfonlayer}{nodelayer}
%		\node [style=nothing] (9) at (-1, 5.25) {};
%		\node [style=nothing] (10) at (-0.5, 5.25) {};
%		\node [style=nothing] (11) at (0, 5.25) {};
%		\node [style=dot] (12) at (-1, 6) {};
%		\node [style=oplus] (13) at (-0.5, 6) {};
%		\node [style=oplus] (14) at (-0.5, 6.5) {};
%		\node [style=dot] (15) at (0, 6.5) {};
%		\node [style=nothing] (16) at (-1, 7.25) {};
%		\node [style=nothing] (17) at (-0.5, 7.25) {};
%		\node [style=nothing] (18) at (0, 7.25) {};
%	\end{pgfonlayer}
%	\begin{pgfonlayer}{edgelayer}
%		\draw [style=simple] (9) to (16);
%		\draw [style=simple] (10) to (17);
%		\draw [style=simple] (11) to (18);
%		\draw [style=simple] (12) to (13);
%		\draw [style=simple] (14) to (15);
%	\end{pgfonlayer}
%\end{tikzpicture}
%							=
%							\begin{tikzpicture}
%	\begin{pgfonlayer}{nodelayer}
%		\node [style=nothing] (10) at (-1, 5.25) {};
%		\node [style=nothing] (11) at (-0.5, 5.25) {};
%		\node [style=nothing] (12) at (0, 5.25) {};
%		\node [style=dot] (13) at (-1, 6.5) {};
%		\node [style=oplus] (14) at (-0.5, 6.5) {};
%		\node [style=oplus] (15) at (-0.5, 6) {};
%		\node [style=dot] (16) at (0, 6) {};
%		\node [style=nothing] (17) at (-1, 7.25) {};
%		\node [style=nothing] (18) at (-0.5, 7.25) {};
%		\node [style=nothing] (19) at (0, 7.25) {};
%	\end{pgfonlayer}
%	\begin{pgfonlayer}{edgelayer}
%		\draw [style=simple] (10) to (17);
%		\draw [style=simple] (11) to (18);
%		\draw [style=simple] (12) to (19);
%		\draw [style=simple] (13) to (14);
%		\draw [style=simple] (15) to (16);
%	\end{pgfonlayer}
%\end{tikzpicture}
%							$}
%						
%						\item 
%						\label{CNOT.6}
%						\hfil{
%							$
%							\begin{tikzpicture}
%	\begin{pgfonlayer}{nodelayer}
%		\node [style=onein] (11) at (0, 5.25) {};
%		\node [style=oneout] (12) at (0, 6.25) {};
%	\end{pgfonlayer}
%	\begin{pgfonlayer}{edgelayer}
%		\draw [style=simple] (11) to (12);
%	\end{pgfonlayer}
%\end{tikzpicture}
%							=
%							\begin{tikzpicture}
%	\begin{pgfonlayer}{nodelayer}
%		\node [style=rn] (12) at (0, 5.25) {};
%		\node [style=rn] (13) at (0, 6.25) {};
%	\end{pgfonlayer}
%\end{tikzpicture}
%							$}
%						
%						\item 
%						\label{CNOT.7}
%						\hfil{
%							\begin{tabular}{c}
%							$\begin{tikzpicture}
%	\begin{pgfonlayer}{nodelayer}
%		\node [style=onein] (13) at (-1, 5.25) {};
%		\node [style=onein] (14) at (-0.5, 5.25) {};
%		\node [style=nothing] (15) at (0, 5.25) {};
%		\node [style=dot] (16) at (-1, 5.75) {};
%		\node [style=oplus] (17) at (-0.5, 5.75) {};
%		\node [style=dot] (18) at (-0.5, 6.25) {};
%		\node [style=oplus] (19) at (0, 6.25) {};
%		\node [style=oneout] (20) at (-1, 6.25) {};
%		\node [style=nothing] (21) at (-0.5, 6.75) {};
%		\node [style=nothing] (22) at (0, 6.75) {};
%	\end{pgfonlayer}
%	\begin{pgfonlayer}{edgelayer}
%		\draw [style=simple] (13) to (20);
%		\draw [style=simple] (14) to (21);
%		\draw [style=simple] (15) to (22);
%		\draw [style=simple] (16) to (17);
%		\draw [style=simple] (18) to (19);
%	\end{pgfonlayer}
%\end{tikzpicture}
%							=
%							\begin{tikzpicture}
%	\begin{pgfonlayer}{nodelayer}
%		\node [style=onein] (14) at (-1, 5.25) {};
%		\node [style=onein] (15) at (-0.5, 5.25) {};
%		\node [style=nothing] (16) at (0, 5.25) {};
%		\node [style=dot] (17) at (-1, 5.75) {};
%		\node [style=oplus] (18) at (-0.5, 5.75) {};
%		\node [style=oneout] (19) at (-1, 6.25) {};
%		\node [style=nothing] (20) at (-0.5, 6.75) {};
%		\node [style=nothing] (21) at (0, 6.75) {};
%	\end{pgfonlayer}
%	\begin{pgfonlayer}{edgelayer}
%		\draw [style=simple] (14) to (19);
%		\draw [style=simple] (15) to (20);
%		\draw [style=simple] (16) to (21);
%		\draw [style=simple] (17) to (18);
%	\end{pgfonlayer}
%\end{tikzpicture}$\\
%							$ $\\
%							$\begin{tikzpicture}
%	\begin{pgfonlayer}{nodelayer}
%		\node [style=oneout] (15) at (-1, 6.75) {};
%		\node [style=oneout] (16) at (-0.5, 6.75) {};
%		\node [style=nothing] (17) at (0, 6.75) {};
%		\node [style=dot] (18) at (-1, 6.25) {};
%		\node [style=oplus] (19) at (-0.5, 6.25) {};
%		\node [style=dot] (20) at (-0.5, 5.75) {};
%		\node [style=oplus] (21) at (0, 5.75) {};
%		\node [style=onein] (22) at (-1, 5.75) {};
%		\node [style=nothing] (23) at (-0.5, 5.25) {};
%		\node [style=nothing] (24) at (0, 5.25) {};
%	\end{pgfonlayer}
%	\begin{pgfonlayer}{edgelayer}
%		\draw [style=simple] (15) to (22);
%		\draw [style=simple] (16) to (23);
%		\draw [style=simple] (17) to (24);
%		\draw [style=simple] (18) to (19);
%		\draw [style=simple] (20) to (21);
%	\end{pgfonlayer}
%\end{tikzpicture}
%							=
%							\begin{tikzpicture}
%	\begin{pgfonlayer}{nodelayer}
%		\node [style=oneout] (16) at (-1, 6.75) {};
%		\node [style=oneout] (17) at (-0.5, 6.75) {};
%		\node [style=nothing] (18) at (0, 6.75) {};
%		\node [style=dot] (19) at (-1, 6.25) {};
%		\node [style=oplus] (20) at (-0.5, 6.25) {};
%		\node [style=onein] (21) at (-1, 5.75) {};
%		\node [style=nothing] (22) at (-0.5, 5.25) {};
%		\node [style=nothing] (23) at (0, 5.25) {};
%	\end{pgfonlayer}
%	\begin{pgfonlayer}{edgelayer}
%		\draw [style=simple] (16) to (21);
%		\draw [style=simple] (17) to (22);
%		\draw [style=simple] (18) to (23);
%		\draw [style=simple] (19) to (20);
%	\end{pgfonlayer}
%\end{tikzpicture}$
%							\end{tabular}
%							}
%						
%						\item 
%						\label{CNOT.8}
%						\hfil{
%							$
%							\begin{tikzpicture}
%	\begin{pgfonlayer}{nodelayer}
%		\node [style=nothing] (17) at (-1, 5.25) {};
%		\node [style=nothing] (18) at (-0.5, 5.25) {};
%		\node [style=nothing] (19) at (0, 5.25) {};
%		\node [style=dot] (20) at (-1, 5.75) {};
%		\node [style=oplus] (21) at (-0.5, 5.75) {};
%		\node [style=dot] (22) at (-0.5, 6.25) {};
%		\node [style=oplus] (23) at (0, 6.25) {};
%		\node [style=dot] (24) at (-1, 6.75) {};
%		\node [style=oplus] (25) at (-0.5, 6.75) {};
%		\node [style=nothing] (26) at (-1, 7.25) {};
%		\node [style=nothing] (27) at (-0.5, 7.25) {};
%		\node [style=nothing] (28) at (0, 7.25) {};
%	\end{pgfonlayer}
%	\begin{pgfonlayer}{edgelayer}
%		\draw [style=simple] (17) to (26);
%		\draw [style=simple] (18) to (27);
%		\draw [style=simple] (19) to (28);
%		\draw [style=simple] (20) to (21);
%		\draw [style=simple] (22) to (23);
%		\draw [style=simple] (24) to (25);
%	\end{pgfonlayer}
%\end{tikzpicture}
%							=
%							\begin{tikzpicture}
%	\begin{pgfonlayer}{nodelayer}
%		\node [style=nothing] (18) at (-1, 5.25) {};
%		\node [style=nothing] (19) at (-0.5, 5.25) {};
%		\node [style=nothing] (20) at (0, 5.25) {};
%		\node [style=dot] (21) at (-0.5, 5.75) {};
%		\node [style=oplus] (22) at (0, 5.75) {};
%		\node [style=dot] (23) at (-1, 6.25) {};
%		\node [style=oplus] (24) at (0, 6.25) {};
%		\node [style=nothing] (25) at (-1, 6.75) {};
%		\node [style=nothing] (26) at (-0.5, 6.75) {};
%		\node [style=nothing] (27) at (0, 6.75) {};
%	\end{pgfonlayer}
%	\begin{pgfonlayer}{edgelayer}
%		\draw [style=simple] (18) to (25);
%		\draw [style=simple] (19) to (26);
%		\draw [style=simple] (20) to (27);
%		\draw [style=simple] (21) to (22);
%		\draw [style=simple] (23) to (24);
%	\end{pgfonlayer}
%\end{tikzpicture}
%							$}
%						
%						\item 
%						\label{CNOT.9}
%						\hfil{
%							$
%							\begin{tikzpicture}
%	\begin{pgfonlayer}{nodelayer}
%		\node [style=onein] (19) at (-1, 5.25) {};
%		\node [style=onein] (20) at (-0.5, 5.25) {};
%		\node [style=nothing] (21) at (0, 5.25) {};
%		\node [style=dot] (22) at (-1, 5.75) {};
%		\node [style=oplus] (23) at (-0.5, 5.75) {};
%		\node [style=oneout] (24) at (-1, 6.25) {};
%		\node [style=oneout] (25) at (-0.5, 6.25) {};
%		\node [style=nothing] (26) at (0, 6.25) {};
%	\end{pgfonlayer}
%	\begin{pgfonlayer}{edgelayer}
%		\draw [style=simple] (19) to (24);
%		\draw [style=simple] (20) to (25);
%		\draw [style=simple] (21) to (26);
%		\draw [style=simple] (22) to (23);
%	\end{pgfonlayer}
%\end{tikzpicture}
%							=
%							\begin{tikzpicture}
%	\begin{pgfonlayer}{nodelayer}
%		\node [style=onein] (20) at (-1, 5.25) {};
%		\node [style=onein] (21) at (-0.5, 5.25) {};
%		\node [style=nothing] (22) at (0, 5.25) {};
%		\node [style=dot] (23) at (-1, 6.25) {};
%		\node [style=oplus] (24) at (-0.5, 6.25) {};
%		\node [style=oneout] (25) at (-1, 7.25) {};
%		\node [style=oneout] (26) at (-0.5, 7.25) {};
%		\node [style=nothing] (27) at (0, 7.25) {};
%		\node [style=oneout] (28) at (0, 6) {};
%		\node [style=onein] (29) at (0, 6.5) {};
%	\end{pgfonlayer}
%	\begin{pgfonlayer}{edgelayer}
%		\draw [style=simple] (20) to (25);
%		\draw [style=simple] (21) to (26);
%		\draw [style=simple] (22) to (28);
%		\draw [style=simple] (29) to (27);
%		\draw [style=simple] (23) to (24);
%	\end{pgfonlayer}
%\end{tikzpicture}
%							$}
%					\end{enumerate}
%				\end{multicols}
%				\
%			\end{mdframed}
%	}}
%	\caption{The identities of \texorpdfstring{$\CNOT$}{CNOT}}
%	\label{fig:CNOT}
%\end{figure}



\chapter{Relational semantics for stabilizer circuits}

In this chapter, we will give a relational account of stabilizer codes (also known as mixed stabilizer circuits) using linear and affine symplectic geometry.
Unlike the previous section where the state spaces were finite sets, here the objects of study are symplectic vector space.  Symplectic vector spaces capture the possible configurations of position and momentum, along with a symplectic form which determines their commutation. The morphisms of study  are categories of (affine) (co)isotropic relations: capturing the nondeterministic evolution of the mechanical system in a way that preserves the commutation relation between position and momentum.

In Section \ref{sec:sym}, we give an overview of the theory of linear Lagrangian relations using the language of graphical linear algebra. In Section \ref{sec:univ}, we give generators for Lagrangian relations; showing that for prime fields, Lagrangian relations can be presented as a CPM construction over linear relations with respect to the orthogonal complement.
In Section \ref{sec:aff} we show that only one more generator is needed to obtain {\em affine} Lagrangian relations.  In the case of odd prime fields, we show in Theorem \ref{theorem:spekkens} that affine Lagrangian relations are quopit stabilizer circuits, modulo invertible scalars.  This gives  a graphical calculus which extends the previous work on Spekkens'  qubit toy model \cite{backensspek}, and the qutrit stabilizer ZX-calculus \cite{qutrit}.  We also discuss the relation to electrical circuits.
In Section \ref{sec:coisot} we add discarding to (affine) Lagrangian relations.  We show that this gives a semantics for stabilizer codes.  By splitting decoherence maps, we get another category of affine/linear relations.  We show that this gives semantics for state preparation and measurement of stabilizer codes, as well as electrical circuits with controlled voltage and current sources.  In Section \ref{sec:qec}, we discuss how the connection to error correction.

\section{Linear symplectic geometry}
\label{sec:sym}

In this section, we give an brief overview of finite-dimensional linear symplectic geometry, as well as categories of linear coisotropic/isotropic and Lagrangian relations.

See \cite{Weinstein2017} or  \cite{weinsteinsymplectic} for generalizations of to the infinite-dimensional linear and the nonlinear settings, respectively.

\begin{definition}
  Given a field  $k$ and a $k$-vector space $V$, a {\bf symplectic form} on $V$ is a bilinear map $\omega:V\times V\to k$ which is:
\begin{description}
 \item[\ \ Alternating:] $\forall v \in V$, $\omega(v,v)=0$.
 \item[\ \ Non-degenerate:] if $\exists v \in V: \forall w \in V: \omega(v,w)=0$, then $v=0$.
\end{description}
  A {\bf symplectic vector space} is a vector space equipped with a symplectic form. A (linear) {\bf symplectomorphism} is a linear isomorphism between symplectic vector spaces that preserves the symplectic form.
\end{definition}


\begin{lemma}[Darboux theorem]
\label{lemma:sform}
Every vector space $k^{2n}$ with a chosen basis is equipped with a symplectic form,given by the following block matrix:
$$
\omega:=
\begin{bmatrix}
0_n & I_n\\
-I_n & 0_n
\end{bmatrix}
$$
so that $\omega(v,w) := v \omega w^T$.

Moreover, every finite dimensional symplectic vector space over $k$ is symplectomorphic to one of the form $k^{2n}$ with such a symplectic form.
\end{lemma}


\begin{definition}

Let $W \subseteq V$ be a linear subspace of a symplectic space $V$.
The {\bf symplectic dual} of the subspace $W$ is defined to be
$
W^\omega:= \{v \in V : \forall w \in W, \omega(v,w)=0 \}
$.
A linear subspace  $W$ of a symplectic vector space $V$ is {\bf isotropic} when $W^\omega \supseteq W$, {\bf coisotropic} when $W^\omega \subseteq W$ and {\bf Lagrangian} when $W^\omega=W$.
\end{definition}

Notice that the symplectic complement reverses the order of inclusion, so that coisotropic subspaces are turned into isotropic subspaces and vice versa.  In particular, we can see how this acts on the dimension of these subspaces:


\begin{lemma}
Given a symplectic vector space $k^{2n}$, an isotropic vector space has dimension less than or equal to $n$; and a coisotropic subspace has dimension greater or equal to $n$.  Therefore, a Lagrangian subspace has dimension $n$
\end{lemma}


\begin{definition}
A {\bf symplectic basis} for a $n-m$ dimensional isotropic subspace $W \subseteq k^2n$ of dimension $n-m$ is a basis $\{b_0,\ldots, b_{n-m-1}\}$ such that for all $b_i$ and $b_j$, $\omega(b_i,b_j)$.
\end{definition}

Every isotropic subspace admits a (non-unique) symplectic basis.  Take an isotropic subspace $W \subseteq k^{2n}$ of dimension $n-m$.  We will denote such a basis by the block matrix $[Z|X]$ where $Z$ and $X$ are  $n-k\times n$ matrices.  The rows of this matrix are the basis elements and the two gradings reflect the grading of the symplectic vector space.


All of these subspaces generalize the notion of symplectomorphism:

\begin{lemma}
Every symplectomorphism $f:V\to V$ induces a Lagrangian subspace $\Gamma_f:=\{ (fv, v) | v \in V \}$.
\end{lemma}



In linear mechanical systems, symplectic vector spaces over $\R$ are interpreted as  the phase space: ie the space of all allowable configurations of position and momentum.  The Lagrangian subspaces are interpreted as the initial configurations of the mechanical system. The symplectic form measures the volume between points, in the phase space; where symplectomorphisms are the time-reversible, volume preserving transformations of the phase space.

The following  categories of isotropic/coisotropic/Lagrangian relations generalizes symplectomorphisms in a way that allows nondeterministic evolution:

\begin{definition}
Given a field $k$, the prop of {\bf Lagrangian relations},  $\Lag\Rel_k$ has morphisms $n\to m$ being Lagrangian subspaces of the symplectic vector space $k^{n+m} \oplus k^{n+m}$ with respect to the symplectic form given above.  Composition is given by relational composition. The tensor product is given by the direct sum, where the $Z$ and $X$ gradients are grouped together as follows:

$$
\begin{tikzpicture}
	\begin{pgfonlayer}{nodelayer}
		\node [style=map] (616) at (272, 0) {$V$};
		\node [style=none] (617) at (271.75, 1) {};
		\node [style=none] (618) at (272.25, 1) {};
	\end{pgfonlayer}
	\begin{pgfonlayer}{edgelayer}
		\draw [in=-90, out=60] (616) to (618.center);
		\draw [in=-90, out=120] (616) to (617.center);
	\end{pgfonlayer}
\end{tikzpicture}
+
\begin{tikzpicture}
	\begin{pgfonlayer}{nodelayer}
		\node [style=map] (616) at (272, 0) {$W$};
		\node [style=none] (617) at (271.75, 1) {};
		\node [style=none] (618) at (272.25, 1) {};
	\end{pgfonlayer}
	\begin{pgfonlayer}{edgelayer}
		\draw [in=-90, out=60] (616) to (618.center);
		\draw [in=-90, out=120] (616) to (617.center);
	\end{pgfonlayer}
\end{tikzpicture}
:=
\begin{tikzpicture}
	\begin{pgfonlayer}{nodelayer}
		\node [style=map] (99) at (192.5, -3.5) {$V$};
		\node [style=none] (100) at (192.5, -2.25) {};
		\node [style=none] (101) at (193.5, -2.25) {};
		\node [style=map] (102) at (193.5, -3.5) {$W$};
		\node [style=none] (103) at (192.5, -2.25) {};
		\node [style=none] (104) at (193.5, -2.25) {};
		\node [style=otimes] (105) at (192.5, -2.25) {};
		\node [style=none] (106) at (192.5, -1.75) {};
		\node [style=none] (107) at (193.5, -1.75) {};
		\node [style=none] (108) at (192.5, -1.75) {};
		\node [style=none] (109) at (193.5, -1.75) {};
		\node [style=otimes] (110) at (193.5, -2.25) {};
	\end{pgfonlayer}
	\begin{pgfonlayer}{edgelayer}
		\draw [in=-150, out=45, looseness=0.75] (99) to (101.center);
		\draw [in=-135, out=120] (99) to (100.center);
		\draw [in=-45, out=60] (102) to (104.center);
		\draw [in=-30, out=135, looseness=0.75] (102) to (103.center);
		\draw (109.center) to (104.center);
		\draw (103.center) to (108.center);
	\end{pgfonlayer}
\end{tikzpicture}
$$

The props of {\bf isotropic relations} and {\bf coisotropic relations}, $\Isot\Rel_{k}$ and $\Co\Isot\Rel_{k}$ are defined in the obvious analogous ways.
\end{definition}


Such structures appear in the literature in the nonlinear setting as ``canonical relations,'' first appearing in \cite{Guillemin} (although composition is only partially defined in this more general setting).  However, this construction is  apparenly originally attributed to the, at that time, unpublished work of Weinsten, thus, it is also called the ``Weinstein sympectic category''.


%For the purposes of this paper, because it is so much easier to work in a prop, we will draw string diagram in the skeleton of $\Lag\Rel_k$ whose objects are all of the form $k^{2n}$ equipped with the symplectic form of Lemma \ref{lemma:sform}.

%Where we are grouping the $X$ gradings together on the left and the $Z$ gradings together on the right. Note that this means the embedding of $\Lag\Rel_k$ into $\LinRel_k$ preserves the monoidal product only up to isomorphism. More precisely, we have the following fact.

It is easy to see that there is an embedding of these categories into $\LinRel_k$:

\begin{lemma}
\label{lemma:strong}
The forgetful functors fom Lagrangian/isotropic/cosisotropic relations to linear relations  are faithful, strong symmetric monoidal.
\end{lemma}



Due to the above lemma, we will regard $\Lag\Rel_k$, $\Isot\Rel_k$, $\Co\Isot\Rel_k$ as symmetric monoidal subcategories of $\LinRel_k$.
As such, we can ask what the generators of $\Lag\Rel_k$, $\Isot\Rel_k$ and $\Co\Isot\Rel_k$ look like in terms of string diagrams of linear relations. We first describe what it means to be a Lagrangian relation in pictures.  In Section \ref{sec:coisot}, we will return to the question of (co)isotropic relations.

Concretely, the symplectic dual of a linear subspace $W \subseteq V$ is:
\begin{align*}
W^\omega :&= \{(v_1,v_2) \in V : \forall (w_1,w_2) \in W, \omega((v_1,v_2),(w_1,w_2))=0 \}\\
                    &= \{(v_1,v_2) \in V : \forall (w_1,w_2) \in W,  \langle (v_2,-v_1) ,(w_1,w_2)\rangle =0 \}\\
                    &= \{(v_2,-v_1) \in V : \forall (w_1,w_2) \in W,  \langle (v_1,v_2) ,(w_1,w_2)\rangle =0 \}
\end{align*}

Therefore, the condition asking that $W=W^\omega$ is graphically:

\begin{equation}
\label{eq:lag}
\begin{tikzpicture}
	\begin{pgfonlayer}{nodelayer}
		\node [style=map] (0) at (0.75, -1) {$W$};
		\node [style=none] (1) at (0.5, 0) {};
		\node [style=none] (2) at (1, 0) {};
	\end{pgfonlayer}
	\begin{pgfonlayer}{edgelayer}
		\draw [in=120, out=-90] (1.center) to (0);
		\draw [in=-90, out=60] (0) to (2.center);
	\end{pgfonlayer}
\end{tikzpicture}
=
\begin{tikzpicture}
	\begin{pgfonlayer}{nodelayer}
		\node [style=map] (0) at (0.75, -1.75) {$W^\perp$};
		\node [style=none] (1) at (0.5, -1) {};
		\node [style=none] (2) at (1, -1) {};
		\node [style=none] (3) at (1, 0) {};
		\node [style=none] (4) at (0.5, 0) {};
		\node [style=s] (5) at (1, -1) {};
	\end{pgfonlayer}
	\begin{pgfonlayer}{edgelayer}
		\draw [in=120, out=-90] (1.center) to (0);
		\draw [in=-90, out=60] (0) to (2.center);
		\draw [in=-90, out=90] (2.center) to (4.center);
		\draw [in=-270, out=-90] (3.center) to (1.center);
	\end{pgfonlayer}
\end{tikzpicture}
\end{equation}


The category of Lagrangian relations is compact closed.  Given a relation $V$ between symplectic vector spaces, we can curry it into a state $\lfloor V \rfloor$; and similarily, we can uncurry a states back into processes:
$$
\begin{tikzpicture}
	\begin{pgfonlayer}{nodelayer}
		\node [style=map] (0) at (0.75, -1.75) {$V$};
		\node [style=none] (1) at (0.5, -1) {};
		\node [style=none] (2) at (1, -1) {};
		\node [style=none] (3) at (0.5, -2.5) {};
		\node [style=none] (4) at (1, -2.5) {};
	\end{pgfonlayer}
	\begin{pgfonlayer}{edgelayer}
		\draw [in=120, out=-90] (1.center) to (0);
		\draw [in=-90, out=60] (0) to (2.center);
		\draw [in=-60, out=90] (4.center) to (0);
		\draw [in=90, out=-120] (0) to (3.center);
	\end{pgfonlayer}
\end{tikzpicture}
\xmapsto{\lfloor{\_}\rfloor }
\begin{tikzpicture}
	\begin{pgfonlayer}{nodelayer}
		\node [style=map] (0) at (0.75, -1.75) {$V$};
		\node [style=none] (1) at (0, -1) {};
		\node [style=none] (2) at (1.25, -1) {};
		\node [style=none] (4) at (1.25, -2.5) {};
		\node [style=X] (5) at (0, -3) {};
		\node [style=Z] (6) at (0.75, -3) {};
		\node [style=none] (7) at (0.75, -1) {};
		\node [style=none] (8) at (-0.5, -1) {};
		\node [style=none] (9) at (0, -2) {};
	\end{pgfonlayer}
	\begin{pgfonlayer}{edgelayer}
		\draw [in=120, out=-90] (1.center) to (0);
		\draw [in=-90, out=60] (0) to (2.center);
		\draw [in=-45, out=90] (4.center) to (0);
		\draw [in=-90, out=135, looseness=0.75] (5) to (8.center);
		\draw [in=30, out=-90] (4.center) to (6);
		\draw [in=90, out=-90] (7.center) to (9.center);
		\draw [in=150, out=-90] (9.center) to (6);
		\draw [in=45, out=-135] (0) to (5);
	\end{pgfonlayer}
\end{tikzpicture}
\hspace*{1cm}
\begin{tikzpicture}
	\begin{pgfonlayer}{nodelayer}
		\node [style=map] (0) at (1.5, -2) {$W$};
		\node [style=none] (1) at (1.25, -1) {};
		\node [style=none] (2) at (2.25, -1) {};
		\node [style=none] (6) at (1.75, -1) {};
		\node [style=none] (7) at (0.75, -1) {};
	\end{pgfonlayer}
	\begin{pgfonlayer}{edgelayer}
		\draw [in=105, out=-90] (1.center) to (0);
		\draw [in=-90, out=60] (0) to (2.center);
		\draw [in=120, out=-90] (7.center) to (0);
		\draw [in=-90, out=75] (0) to (6.center);
	\end{pgfonlayer}
\end{tikzpicture}
\mapsto
\begin{tikzpicture}
	\begin{pgfonlayer}{nodelayer}
		\node [style=map] (0) at (1.75, -2.25) {$W$};
		\node [style=none] (1) at (1, -0.75) {};
		\node [style=none] (2) at (2, -0.75) {};
		\node [style=none] (6) at (1.5, -1.25) {};
		\node [style=none] (7) at (0.75, -1.25) {};
		\node [style=Z] (8) at (1.5, -1.25) {};
		\node [style=X] (9) at (0.75, -1.25) {};
		\node [style=none] (10) at (0.25, -2.5) {};
		\node [style=none] (11) at (1, -2.5) {};
	\end{pgfonlayer}
	\begin{pgfonlayer}{edgelayer}
		\draw [in=105, out=-90, looseness=1.25] (1.center) to (0);
		\draw [in=-90, out=45, looseness=0.75] (0) to (2.center);
		\draw [in=135, out=-30] (7.center) to (0);
		\draw [in=-45, out=75] (0) to (6.center);
		\draw [in=-135, out=90] (10.center) to (9);
		\draw [in=-135, out=90] (11.center) to (8);
	\end{pgfonlayer}
\end{tikzpicture}
$$
It is easy to see that these two constructions are inverse to each other.
This allows us to derive a graphical criteria for abitrary Lagrangian relations, generalizing Equation \ref{eq:lag}:
$$
\begin{tikzpicture}
	\begin{pgfonlayer}{nodelayer}
		\node [style=map] (0) at (0.75, -1.75) {$V$};
		\node [style=none] (1) at (0, -0.75) {};
		\node [style=none] (2) at (1.25, -0.75) {};
		\node [style=none] (4) at (1.25, -2.5) {};
		\node [style=X] (5) at (0, -3) {};
		\node [style=Z] (6) at (0.75, -3) {};
		\node [style=none] (7) at (0.75, -0.75) {};
		\node [style=none] (8) at (-0.5, -0.75) {};
		\node [style=none] (9) at (0, -2) {};
	\end{pgfonlayer}
	\begin{pgfonlayer}{edgelayer}
		\draw [in=120, out=-90] (1.center) to (0);
		\draw [in=-90, out=60] (0) to (2.center);
		\draw [in=-45, out=90] (4.center) to (0);
		\draw [in=-90, out=135, looseness=0.75] (5) to (8.center);
		\draw [in=30, out=-90] (4.center) to (6);
		\draw [in=90, out=-90] (7.center) to (9.center);
		\draw [in=150, out=-90] (9.center) to (6);
		\draw [in=45, out=-135] (0) to (5);
	\end{pgfonlayer}
\end{tikzpicture}
=
\begin{tikzpicture}
	\begin{pgfonlayer}{nodelayer}
		\node [style=map] (43) at (13.5, -1.75) {$V^\perp$};
		\node [style=none] (44) at (12.75, -0.75) {};
		\node [style=none] (45) at (14, -0.75) {};
		\node [style=none] (46) at (14, -2.5) {};
		\node [style=Z] (47) at (12.75, -3) {};
		\node [style=X] (48) at (13.5, -3) {};
		\node [style=none] (49) at (13.5, -0.75) {};
		\node [style=none] (50) at (12.25, -0.75) {};
		\node [style=none] (51) at (12.75, -2) {};
		\node [style=none] (52) at (13.5, 0.75) {};
		\node [style=none] (53) at (14, 0.75) {};
		\node [style=none] (54) at (12.25, 0.75) {};
		\node [style=none] (55) at (12.75, 0.75) {};
		\node [style=s] (56) at (14, -0.75) {};
		\node [style=s] (57) at (13.5, -0.75) {};
		\node [style=none] (58) at (13.25, -3.5) {};
	\end{pgfonlayer}
	\begin{pgfonlayer}{edgelayer}
		\draw [in=120, out=-90] (44.center) to (43);
		\draw [in=-90, out=60] (43) to (45.center);
		\draw [in=-45, out=90] (46.center) to (43);
		\draw [in=-90, out=135, looseness=0.75] (47) to (50.center);
		\draw [in=30, out=-90] (46.center) to (48);
		\draw [in=90, out=-90] (49.center) to (51.center);
		\draw [in=150, out=-90] (51.center) to (48);
		\draw [in=45, out=-135] (43) to (47);
		\draw [in=270, out=90] (45.center) to (55.center);
		\draw [in=270, out=90] (49.center) to (54.center);
		\draw [in=270, out=90] (44.center) to (53.center);
		\draw [in=270, out=90] (50.center) to (52.center);
	\end{pgfonlayer}
\end{tikzpicture}
\iff
\begin{tikzpicture}
	\begin{pgfonlayer}{nodelayer}
		\node [style=map] (0) at (2, -2) {$V$};
		\node [style=none] (1) at (1.75, -1.25) {};
		\node [style=none] (2) at (2.25, -1.25) {};
		\node [style=none] (3) at (1.75, -2.75) {};
		\node [style=none] (4) at (2.25, -2.75) {};
	\end{pgfonlayer}
	\begin{pgfonlayer}{edgelayer}
		\draw [in=120, out=-90] (1.center) to (0);
		\draw [in=-90, out=60] (0) to (2.center);
		\draw [in=-60, out=90] (4.center) to (0);
		\draw [in=90, out=-120] (0) to (3.center);
	\end{pgfonlayer}
\end{tikzpicture}
=
\begin{tikzpicture}
	\begin{pgfonlayer}{nodelayer}
		\node [style=map] (0) at (2.5, -1.75) {$V$};
		\node [style=none] (1) at (2, -0.5) {};
		\node [style=none] (2) at (3, -0.5) {};
		\node [style=none] (3) at (3, -2.5) {};
		\node [style=X] (4) at (1.75, -3) {};
		\node [style=Z] (5) at (2.5, -3) {};
		\node [style=none] (6) at (2.5, -0.75) {};
		\node [style=none] (7) at (1.5, -0.75) {};
		\node [style=none] (8) at (1.75, -2) {};
		\node [style=X] (9) at (1.5, -0.75) {};
		\node [style=Z] (10) at (2.5, -0.75) {};
		\node [style=none] (11) at (0.5, -3.25) {};
		\node [style=none] (12) at (1, -3.25) {};
	\end{pgfonlayer}
	\begin{pgfonlayer}{edgelayer}
		\draw [in=120, out=-90] (1.center) to (0);
		\draw [in=-90, out=60] (0) to (2.center);
		\draw [in=-45, out=90] (3.center) to (0);
		\draw [in=-45, out=150] (4) to (7.center);
		\draw [in=30, out=-90] (3.center) to (5);
		\draw [in=90, out=-45, looseness=1.25] (6.center) to (8.center);
		\draw [in=150, out=-90] (8.center) to (5);
		\draw [in=45, out=-135] (0) to (4);
		\draw [in=90, out=-150] (10) to (12.center);
		\draw [in=90, out=-120] (9) to (11.center);
	\end{pgfonlayer}
\end{tikzpicture}
=
\begin{tikzpicture}
	\begin{pgfonlayer}{nodelayer}
		\node [style=map] (59) at (17, -1.75) {$V^\perp$};
		\node [style=none] (60) at (16.25, -0.75) {};
		\node [style=none] (61) at (17.5, -0.75) {};
		\node [style=none] (62) at (17.5, -2.5) {};
		\node [style=Z] (63) at (16.25, -3) {};
		\node [style=X] (64) at (17, -3) {};
		\node [style=none] (65) at (17, -0.75) {};
		\node [style=none] (66) at (15.75, -0.75) {};
		\node [style=none] (67) at (16.25, -2) {};
		\node [style=none] (68) at (16.5, 0.75) {};
		\node [style=none] (69) at (17.5, 1) {};
		\node [style=none] (70) at (15.5, 0.75) {};
		\node [style=none] (71) at (16, 1) {};
		\node [style=s] (72) at (17.5, -0.75) {};
		\node [style=s] (73) at (17, -0.75) {};
		\node [style=X] (74) at (15.5, 0.75) {};
		\node [style=Z] (75) at (16.5, 0.75) {};
		\node [style=none] (76) at (15, -3.25) {};
		\node [style=none] (77) at (15.5, -3.25) {};
		\node [style=none] (78) at (16.5, 1.5) {};
	\end{pgfonlayer}
	\begin{pgfonlayer}{edgelayer}
		\draw [in=120, out=-90] (60.center) to (59);
		\draw [in=-90, out=60] (59) to (61.center);
		\draw [in=-45, out=90] (62.center) to (59);
		\draw [in=-90, out=135, looseness=0.75] (63) to (66.center);
		\draw [in=30, out=-90] (62.center) to (64);
		\draw [in=90, out=-90] (65.center) to (67.center);
		\draw [in=150, out=-90] (67.center) to (64);
		\draw [in=45, out=-135] (59) to (63);
		\draw [in=-90, out=90, looseness=1.25] (61.center) to (71.center);
		\draw [in=-60, out=90] (65.center) to (70.center);
		\draw [in=270, out=90] (60.center) to (69.center);
		\draw [in=-45, out=90, looseness=1.25] (66.center) to (68.center);
		\draw [in=-135, out=90, looseness=0.50] (76.center) to (74);
		\draw [in=-150, out=90] (77.center) to (75);
	\end{pgfonlayer}
\end{tikzpicture}
=
\begin{tikzpicture}
	\begin{pgfonlayer}{nodelayer}
		\node [style=map] (0) at (4.5, -1.75) {$V^\perp$};
		\node [style=none] (1) at (4, -1) {};
		\node [style=none] (2) at (5, -1) {};
		\node [style=none] (3) at (5, -2.5) {};
		\node [style=X] (4) at (4.75, -3) {};
		\node [style=none] (5) at (3.75, -2.25) {};
		\node [style=none] (6) at (5, 0) {};
		\node [style=none] (7) at (4, 0) {};
		\node [style=s] (8) at (5, -1) {};
		\node [style=none] (9) at (3.25, -3.25) {};
		\node [style=none] (10) at (4, -3.25) {};
		\node [style=Z] (11) at (3.75, -2.25) {};
	\end{pgfonlayer}
	\begin{pgfonlayer}{edgelayer}
		\draw [in=135, out=-90] (1.center) to (0);
		\draw [in=-90, out=60] (0) to (2.center);
		\draw [in=-45, out=90] (3.center) to (0);
		\draw [in=30, out=-90] (3.center) to (4);
		\draw [in=165, out=-15, looseness=1.25] (5.center) to (4);
		\draw [in=-90, out=90, looseness=1.25] (2.center) to (7.center);
		\draw [in=270, out=90] (1.center) to (6.center);
		\draw [in=240, out=90] (10.center) to (0);
		\draw [in=90, out=-150] (11) to (9.center);
	\end{pgfonlayer}
\end{tikzpicture}
=
\begin{tikzpicture}
	\begin{pgfonlayer}{nodelayer}
		\node [style=map] (0) at (2, -2) {$V^\perp$};
		\node [style=none] (1) at (1.75, -1.25) {};
		\node [style=none] (2) at (2.25, -1.25) {};
		\node [style=none] (3) at (1.75, -2.75) {};
		\node [style=none] (4) at (2.25, -2.75) {};
		\node [style=none] (5) at (2.25, -0.5) {};
		\node [style=none] (6) at (1.75, -0.5) {};
		\node [style=none] (7) at (2.25, -3.5) {};
		\node [style=none] (8) at (1.75, -3.5) {};
		\node [style=s] (9) at (2.25, -1.25) {};
		\node [style=s] (10) at (2.25, -2.75) {};
	\end{pgfonlayer}
	\begin{pgfonlayer}{edgelayer}
		\draw [in=120, out=-90] (1.center) to (0);
		\draw [in=-90, out=60] (0) to (2.center);
		\draw [in=-60, out=90] (4.center) to (0);
		\draw [in=90, out=-120] (0) to (3.center);
		\draw [in=90, out=-90] (6.center) to (2.center);
		\draw [in=270, out=90] (1.center) to (5.center);
		\draw [in=270, out=90] (7.center) to (3.center);
		\draw [in=270, out=90] (8.center) to (4.center);
	\end{pgfonlayer}
\end{tikzpicture}
$$
For this reason, we will depict Lagrangian relations as processes, where the input wires are on the bottom and output wires on on the top.  There is a functor in the other direction

\begin{lemma}
There is a faithful, strong symmetric monoidal functor $L:\LinRel_k\to\Lag\Rel_k$ given by the following action on the generators of $\ih_k$; doubling, and then changing the colours of one of the copies:
$$
\begin{tikzpicture}
	\begin{pgfonlayer}{nodelayer}
		\node [style=map] (0) at (-3, -1) {$V$};
		\node [style=none] (1) at (-3, -0.25) {};
		\node [style=none] (2) at (-3, -1.75) {};
	\end{pgfonlayer}
	\begin{pgfonlayer}{edgelayer}
		\draw (1.center) to (0);
		\draw (0) to (2.center);
	\end{pgfonlayer}
\end{tikzpicture}
\mapsto
\begin{tikzpicture}
	\begin{pgfonlayer}{nodelayer}
		\node [style=map] (0) at (0.5, -1) {$V^\perp$};
		\node [style=none] (1) at (0.5, -0.25) {};
		\node [style=none] (2) at (0.5, -1.75) {};
		\node [style=map] (3) at (1.5, -1) {$V$};
		\node [style=none] (4) at (1.5, -0.25) {};
		\node [style=none] (5) at (1.5, -1.75) {};
	\end{pgfonlayer}
	\begin{pgfonlayer}{edgelayer}
		\draw (1.center) to (0);
		\draw (0) to (2.center);
		\draw (4.center) to (3);
		\draw (3) to (5.center);
	\end{pgfonlayer}
\end{tikzpicture}
$$
%
%$$
%\begin{tikzpicture}
%	\begin{pgfonlayer}{nodelayer}
%		\node [style=map] (0) at (-3, -1) {$V^\perp$};
%		\node [style=none] (1) at (-3, -0.25) {};
%		\node [style=none] (2) at (-3, -1.75) {};
%		\node [style=map] (3) at (-2.25, -1) {$V$};
%		\node [style=none] (4) at (-2.25, -0.25) {};
%		\node [style=none] (5) at (-2.25, -1.75) {};
%	\end{pgfonlayer}
%	\begin{pgfonlayer}{edgelayer}
%		\draw (1.center) to (0);
%		\draw (0) to (2.center);
%		\draw (4.center) to (3);
%		\draw (3) to (5.center);
%	\end{pgfonlayer}
%\end{tikzpicture}
%=
%\begin{tikzpicture}
%	\begin{pgfonlayer}{nodelayer}
%		\node [style=map] (0) at (-3, -1) {$V$};
%		\node [style=none] (1) at (-3, -0.25) {};
%		\node [style=none] (2) at (-3, -1.75) {};
%		\node [style=map] (3) at (-2.25, -1) {$V^\perp$};
%		\node [style=none] (4) at (-2.25, -0.25) {};
%		\node [style=none] (5) at (-2.25, -1.75) {};
%		\node [style=none] (6) at (-2.25, 0.75) {};
%		\node [style=none] (7) at (-3, 0.75) {};
%		\node [style=none] (8) at (-2.25, -2.75) {};
%		\node [style=none] (9) at (-3, -2.75) {};
%	\end{pgfonlayer}
%	\begin{pgfonlayer}{edgelayer}
%		\draw (1.center) to (0);
%		\draw (0) to (2.center);
%		\draw (4.center) to (3);
%		\draw (3) to (5.center);
%		\draw [in=270, out=90] (1.center) to (6.center);
%		\draw [in=270, out=90] (4.center) to (7.center);
%		\draw [in=270, out=90] (8.center) to (2.center);
%		\draw [in=270, out=90] (9.center) to (5.center);
%	\end{pgfonlayer}
%\end{tikzpicture}
%=
%\begin{tikzpicture}
%	\begin{pgfonlayer}{nodelayer}
%		\node [style=map] (0) at (0.5, -1) {$V$};
%		\node [style=none] (1) at (0.5, 0.25) {};
%		\node [style=none] (2) at (0.5, -1.75) {};
%		\node [style=map] (3) at (1.25, -1) {$V^\perp$};
%		\node [style=none] (4) at (1.25, 0.25) {};
%		\node [style=none] (5) at (1.25, -1.75) {};
%		\node [style=none] (6) at (1.25, 1.25) {};
%		\node [style=none] (7) at (0.5, 1.25) {};
%		\node [style=none] (8) at (1.25, -2.75) {};
%		\node [style=none] (9) at (0.5, -2.75) {};
%		\node [style=s] (10) at (1.25, 0.25) {};
%		\node [style=s] (11) at (1.25, -0.25) {};
%	\end{pgfonlayer}
%	\begin{pgfonlayer}{edgelayer}
%		\draw (1.center) to (0);
%		\draw (0) to (2.center);
%		\draw (4.center) to (3);
%		\draw (3) to (5.center);
%		\draw [in=270, out=90] (1.center) to (6.center);
%		\draw [in=270, out=90] (4.center) to (7.center);
%		\draw [in=270, out=90] (8.center) to (2.center);
%		\draw [in=270, out=90] (9.center) to (5.center);
%	\end{pgfonlayer}
%\end{tikzpicture}
%=
%\begin{tikzpicture}
%	\begin{pgfonlayer}{nodelayer}
%		\node [style=map] (0) at (-3, -1) {$V$};
%		\node [style=none] (1) at (-3, -0.25) {};
%		\node [style=none] (2) at (-3, -1.75) {};
%		\node [style=map] (3) at (-2.25, -1) {$V^\perp$};
%		\node [style=none] (4) at (-2.25, -0.25) {};
%		\node [style=none] (5) at (-2.25, -1.75) {};
%		\node [style=none] (6) at (-2.25, 0.75) {};
%		\node [style=none] (7) at (-3, 0.75) {};
%		\node [style=none] (8) at (-2.25, -2.75) {};
%		\node [style=none] (9) at (-3, -2.75) {};
%		\node [style=s] (10) at (-2.25, -0.25) {};
%		\node [style=s] (11) at (-2.25, -1.75) {};
%	\end{pgfonlayer}
%	\begin{pgfonlayer}{edgelayer}
%		\draw (1.center) to (0);
%		\draw (0) to (2.center);
%		\draw (4.center) to (3);
%		\draw (3) to (5.center);
%		\draw [in=270, out=90] (1.center) to (6.center);
%		\draw [in=270, out=90] (4.center) to (7.center);
%		\draw [in=270, out=90] (8.center) to (2.center);
%		\draw [in=270, out=90] (9.center) to (5.center);
%	\end{pgfonlayer}
%\end{tikzpicture}
%$$
\end{lemma}


To check this is a functor, all we have to show is that it produces Lagrangian relations. This follows immediately from the naturality of $-1$.
Indeed, because Lagrangian subspaces are isotropic and coisotropic, this extends to functors $\LinRel_k\to\Isot\Rel_k$ and $\LinRel_k\to\Co\Isot\Rel_k$.


This functor is symmetric monoidal and faithful but not full, as for example, the following Lagrangian relation is not in the image of $L$:
$$
\left(
\begin{tikzpicture}
	\begin{pgfonlayer}{nodelayer}
		\node [style=X] (127) at (71, 0.25) {};
		\node [style=none] (128) at (71, 1) {};
		\node [style=none] (129) at (71, -1) {};
		\node [style=Z] (130) at (72, -0.25) {};
		\node [style=none] (132) at (72, 1) {};
		\node [style=none] (133) at (72, -1) {};
	\end{pgfonlayer}
	\begin{pgfonlayer}{edgelayer}
		\draw (129.center) to (127);
		\draw (127) to (128.center);
		\draw (133.center) to (130);
		\draw (130) to (132.center);
		\draw (130) to (127);
	\end{pgfonlayer}
\end{tikzpicture}\right)^\omega
=
\begin{tikzpicture}
	\begin{pgfonlayer}{nodelayer}
		\node [style=X] (134) at (74, -0.25) {};
		\node [style=Z] (137) at (73, 0.25) {};
		\node [style=none] (138) at (73, -1) {};
		\node [style=none] (139) at (73, 0.5) {};
		\node [style=s] (140) at (74, 0.5) {};
		\node [style=s] (141) at (74, -1) {};
		\node [style=none] (142) at (74, 1.5) {};
		\node [style=none] (143) at (73, 1.5) {};
		\node [style=none] (144) at (74, -2) {};
		\node [style=none] (145) at (73, -2) {};
	\end{pgfonlayer}
	\begin{pgfonlayer}{edgelayer}
		\draw (139.center) to (137);
		\draw (137) to (138.center);
		\draw (137) to (134);
		\draw (141) to (134);
		\draw (134) to (140);
		\draw [in=270, out=90] (139.center) to (142.center);
		\draw [in=90, out=-90] (143.center) to (140);
		\draw [in=90, out=-90] (138.center) to (144.center);
		\draw [in=270, out=90] (145.center) to (141);
	\end{pgfonlayer}
\end{tikzpicture}
=
\begin{tikzpicture}
	\begin{pgfonlayer}{nodelayer}
		\node [style=X] (146) at (75, -0.25) {};
		\node [style=Z] (147) at (76, 0.25) {};
		\node [style=none] (148) at (76, -1) {};
		\node [style=none] (149) at (76, 0.5) {};
		\node [style=s] (150) at (75, 0.5) {};
		\node [style=s] (151) at (75, -1) {};
		\node [style=none] (152) at (76, 1.5) {};
		\node [style=none] (153) at (75, 1.5) {};
		\node [style=none] (154) at (76, -2) {};
		\node [style=none] (155) at (75, -2) {};
	\end{pgfonlayer}
	\begin{pgfonlayer}{edgelayer}
		\draw (149.center) to (147);
		\draw (147) to (148.center);
		\draw (147) to (146);
		\draw (151) to (146);
		\draw (146) to (150);
		\draw [in=270, out=90] (149.center) to (152.center);
		\draw [in=90, out=-90] (153.center) to (150);
		\draw [in=90, out=-90] (148.center) to (154.center);
		\draw [in=270, out=90] (155.center) to (151);
	\end{pgfonlayer}
\end{tikzpicture}
=
\begin{tikzpicture}
	\begin{pgfonlayer}{nodelayer}
		\node [style=X] (156) at (77, -0.25) {};
		\node [style=Z] (157) at (78, 0.25) {};
		\node [style=none] (158) at (78, -1) {};
		\node [style=none] (159) at (78, 0.75) {};
		\node [style=none] (162) at (78, 1.5) {};
		\node [style=none] (163) at (77, 1.5) {};
		\node [style=none] (164) at (78, -2) {};
		\node [style=none] (165) at (77, -2) {};
		\node [style=s] (166) at (77.5, 0) {};
	\end{pgfonlayer}
	\begin{pgfonlayer}{edgelayer}
		\draw (159.center) to (157);
		\draw (157) to (158.center);
		\draw [in=270, out=90] (159.center) to (162.center);
		\draw [in=90, out=-90] (158.center) to (164.center);
		\draw (163.center) to (156);
		\draw (156) to (165.center);
		\draw (156) to (166);
		\draw (166) to (157);
	\end{pgfonlayer}
\end{tikzpicture}
=
\begin{tikzpicture}
	\begin{pgfonlayer}{nodelayer}
		\node [style=X] (127) at (71, 0.25) {};
		\node [style=none] (128) at (71, 1) {};
		\node [style=none] (129) at (71, -1) {};
		\node [style=Z] (130) at (72, -0.25) {};
		\node [style=none] (132) at (72, 1) {};
		\node [style=none] (133) at (72, -1) {};
	\end{pgfonlayer}
	\begin{pgfonlayer}{edgelayer}
		\draw (129.center) to (127);
		\draw (127) to (128.center);
		\draw (133.center) to (130);
		\draw (130) to (132.center);
		\draw (130) to (127);
	\end{pgfonlayer}
\end{tikzpicture}
$$



Unlike $\LinRel_k$, these are no longer a bicategories of relations:

\begin{remark}
\label{rem:lagrelbicatrel}
$\Lag\Rel_k$ is not a bicategory of relations.  No matter which Frobenius algebra we chose, it is not laxly natural with respect to both the phase shifts and the Fourier transform of the phase shifts:

$$
\begin{tikzpicture}
	\begin{pgfonlayer}{nodelayer}
		\node [style=X] (0) at (-1.25, 0) {};
		\node [style=Z] (1) at (0.25, 0) {};
		\node [style=none] (2) at (-1.75, 1) {};
		\node [style=none] (3) at (-0.75, 1) {};
		\node [style=none] (4) at (-0.25, 1) {};
		\node [style=none] (5) at (0.75, 1) {};
		\node [style=X] (13) at (-1.25, -1) {};
		\node [style=Z] (14) at (0.25, -0.5) {};
		\node [style=X] (15) at (-1.25, -1.5) {};
		\node [style=Z] (16) at (0.25, -1.5) {};
	\end{pgfonlayer}
	\begin{pgfonlayer}{edgelayer}
		\draw [in=-90, out=150] (0) to (2.center);
		\draw [in=270, out=30] (0) to (3.center);
		\draw [bend right] (4.center) to (1);
		\draw [bend right] (1) to (5.center);
		\draw (15) to (13);
		\draw (13) to (0);
		\draw (13) to (14);
		\draw (14) to (1);
		\draw (14) to (16);
	\end{pgfonlayer}
\end{tikzpicture}
=
\begin{tikzpicture}
	\begin{pgfonlayer}{nodelayer}
		\node [style=X] (51) at (2, -0.25) {};
		\node [style=Z] (52) at (3.5, 0.25) {};
		\node [style=none] (53) at (1.5, 0.75) {};
		\node [style=none] (54) at (2.5, 0.75) {};
		\node [style=none] (55) at (3, 1.25) {};
		\node [style=none] (56) at (4, 1.25) {};
		\node [style=none] (57) at (2.5, 1.25) {};
		\node [style=none] (58) at (1.5, 1.25) {};
	\end{pgfonlayer}
	\begin{pgfonlayer}{edgelayer}
		\draw [in=-90, out=150] (51) to (53.center);
		\draw [in=270, out=30] (51) to (54.center);
		\draw [bend right] (55.center) to (52);
		\draw [bend right] (52) to (56.center);
		\draw (51) to (52);
		\draw (53.center) to (58.center);
		\draw (57.center) to (54.center);
	\end{pgfonlayer}
\end{tikzpicture}
\not\subseteq
\begin{tikzpicture}
	\begin{pgfonlayer}{nodelayer}
		\node [style=X] (80) at (7.75, -0.75) {};
		\node [style=none] (81) at (7.25, 0.25) {};
		\node [style=none] (82) at (8.25, 0.25) {};
		\node [style=Z] (83) at (8.25, 0.75) {};
		\node [style=none] (84) at (7.25, 1.25) {};
		\node [style=none] (85) at (8.25, 1.25) {};
		\node [style=X] (86) at (9, 0.75) {};
		\node [style=none] (87) at (9, 0.75) {};
		\node [style=Z] (89) at (9.75, 0.75) {};
		\node [style=none] (90) at (9, 1.25) {};
		\node [style=none] (91) at (9.75, 1.25) {};
	\end{pgfonlayer}
	\begin{pgfonlayer}{edgelayer}
		\draw [in=-90, out=150] (80) to (81.center);
		\draw [in=270, out=30] (80) to (82.center);
		\draw (82.center) to (83);
		\draw (83) to (85.center);
		\draw (81.center) to (84.center);
		\draw [in=90, out=-150] (83) to (80);
		\draw (89) to (91.center);
		\draw (87.center) to (90.center);
	\end{pgfonlayer}
\end{tikzpicture}
=
\begin{tikzpicture}
	\begin{pgfonlayer}{nodelayer}
		\node [style=X] (59) at (11.25, -0.75) {};
		\node [style=none] (61) at (10.75, 0.25) {};
		\node [style=none] (62) at (11.75, 0.25) {};
		\node [style=Z] (68) at (11.75, 0.75) {};
		\node [style=none] (71) at (10.75, 1.25) {};
		\node [style=none] (72) at (11.75, 1.25) {};
		\node [style=X] (73) at (12.75, -0.75) {};
		\node [style=none] (74) at (12.25, 0.25) {};
		\node [style=none] (75) at (13.25, 0) {};
		\node [style=Z] (76) at (13.25, 0.75) {};
		\node [style=none] (77) at (12.25, 1.25) {};
		\node [style=none] (78) at (13.25, 1.25) {};
		\node [style=s] (79) at (13.25, 0) {};
	\end{pgfonlayer}
	\begin{pgfonlayer}{edgelayer}
		\draw [in=-90, out=150] (59) to (61.center);
		\draw [in=270, out=30] (59) to (62.center);
		\draw (62.center) to (68);
		\draw (68) to (72.center);
		\draw (61.center) to (71.center);
		\draw [in=90, out=-150] (68) to (59);
		\draw [in=-90, out=150] (73) to (74.center);
		\draw [in=270, out=30] (73) to (75.center);
		\draw (75.center) to (76);
		\draw (76) to (78.center);
		\draw (74.center) to (77.center);
		\draw [in=90, out=-150] (76) to (73);
	\end{pgfonlayer}
\end{tikzpicture}
=
\begin{tikzpicture}
	\begin{pgfonlayer}{nodelayer}
		\node [style=X] (35) at (14.75, -0.5) {};
		\node [style=Z] (36) at (16.25, -0.5) {};
		\node [style=none] (37) at (14.25, 0.5) {};
		\node [style=none] (38) at (15.25, 0.5) {};
		\node [style=none] (39) at (15.75, 0.5) {};
		\node [style=none] (40) at (16.75, 0.5) {};
		\node [style=X] (41) at (14.75, -1.25) {};
		\node [style=Z] (42) at (16.25, -1.25) {};
		\node [style=X] (43) at (14.25, 0.5) {};
		\node [style=Z] (44) at (15.25, 1) {};
		\node [style=X] (45) at (15.75, 0.5) {};
		\node [style=Z] (46) at (16.75, 1) {};
		\node [style=none] (47) at (14.25, 1.5) {};
		\node [style=none] (48) at (15.25, 1.5) {};
		\node [style=none] (49) at (15.75, 1.5) {};
		\node [style=none] (50) at (16.75, 1.5) {};
	\end{pgfonlayer}
	\begin{pgfonlayer}{edgelayer}
		\draw [in=-90, out=150] (35) to (37.center);
		\draw [in=270, out=30] (35) to (38.center);
		\draw [bend right] (39.center) to (36);
		\draw [bend right] (36) to (40.center);
		\draw (43) to (44);
		\draw (45) to (46);
		\draw (38.center) to (44);
		\draw (44) to (48.center);
		\draw (49.center) to (45);
		\draw (40.center) to (46);
		\draw (46) to (50.center);
		\draw (47.center) to (43);
		\draw (41) to (35);
		\draw (36) to (42);
	\end{pgfonlayer}
\end{tikzpicture}
$$
\end{remark}

\section{Generators for Lagrangian relations}
\label{sec:univ}

%
%Linear subspaces can be represented in terms of the row space of a matrix.
%In particular, an $n$-dimensional Lagrangian subspace is represented by a block diagonal matrix of the form $[X|Z]$ for $X,Z$ both $n\times n$ matrices.
%This correspondance is made one to one, when the symplectic form is required to vanish, so that $[X|Z] w [X|Z]^n =0$ as well as  $[X|Z]$  having dimension $n$.

In this section, we give a universal set of generators for $\Lag\Rel_k$; although, we do not directly give a complete set of identities.  Instead we defer to the completeness of the monoidal presentation of $\LinRel_k$.


Consider the following symplectomorphisms; the symplectic Fourier transform $F$,  the $a$-shift gate $S_a$ and the controlled-$a$ gate $C_a$:
$$
\left\llbracket
\begin{tikzpicture}
	\begin{pgfonlayer}{nodelayer}
		\node [style=none] (0) at (0.5, 1) {};
		\node [style=none] (1) at (0.5, -0.25) {};
		\node [style=none] (2) at (1, -0.25) {};
		\node [style=none] (3) at (1, 1) {};
		\node [style=s] (4) at (1, 0.5) {};
		\node [style=none] (5) at (0.5, 0.5) {};
	\end{pgfonlayer}
	\begin{pgfonlayer}{edgelayer}
		\draw (4) to (3.center);
		\draw [in=90, out=-90] (4) to (1.center);
		\draw [in=-90, out=90] (2.center) to (5.center);
		\draw (5.center) to (0.center);
	\end{pgfonlayer}
\end{tikzpicture}
\right\rrbracket
=
\begin{bmatrix}
0   & 1 \\
-1  & 0
\end{bmatrix}
=:F
$$

$$
\left\llbracket
\begin{tikzpicture}
	\begin{pgfonlayer}{nodelayer}
		\node [style=X] (0) at (0.5, 1.25) {};
		\node [style=Z] (1) at (1.5, -0.25) {};
		\node [style=scalar] (2) at (1, 0.5) {$a$};
		\node [style=none] (3) at (0.5, 1.75) {};
		\node [style=none] (4) at (1.5, 1.75) {};
		\node [style=none] (5) at (1.5, -0.75) {};
		\node [style=none] (6) at (0.5, -0.75) {};
	\end{pgfonlayer}
	\begin{pgfonlayer}{edgelayer}
		\draw (5.center) to (1);
		\draw (1) to (4.center);
		\draw [in=-90, out=135] (1) to (2);
		\draw [in=-45, out=90] (2) to (0);
		\draw (3.center) to (0);
		\draw (0) to (6.center);
	\end{pgfonlayer}
\end{tikzpicture}
\right\rrbracket
=
\begin{bmatrix}
1 &a\\
0 & 1
\end{bmatrix}=:S_a
$$

$$
\left\llbracket
\begin{tikzpicture}
	\begin{pgfonlayer}{nodelayer}
		\node [style=Z] (430) at (219.75, 0) {};
		\node [style=X] (431) at (220.75, 1.5) {};
		\node [style=none] (432) at (219.75, 0) {};
		\node [style=none] (433) at (221.25, -0.75) {};
		\node [style=none] (434) at (219.25, 2.25) {};
		\node [style=none] (435) at (220.75, 2.25) {};
		\node [style=scalar] (436) at (220.25, 0.75) {$a$};
		\node [style=X] (437) at (217.5, 0) {};
		\node [style=Z] (438) at (218.5, 1.5) {};
		\node [style=none] (439) at (217.5, -0.75) {};
		\node [style=none] (440) at (219, -0.75) {};
		\node [style=none] (441) at (217.25, 2.25) {};
		\node [style=none] (442) at (218.5, 1.5) {};
		\node [style=scalarop] (443) at (218, 0.75) {$a$};
		\node [style=none] (445) at (219.75, -0.75) {};
		\node [style=none] (446) at (218.5, 2.25) {};
	\end{pgfonlayer}
	\begin{pgfonlayer}{edgelayer}
		\draw [in=-105, out=30] (430) to (436);
		\draw [in=-150, out=90] (436) to (431);
		\draw [in=90, out=-60] (431) to (433.center);
		\draw (431) to (435.center);
		\draw [in=135, out=-90, looseness=0.75] (434.center) to (430);
		\draw (439.center) to (437);
		\draw [in=-105, out=30] (437) to (443);
		\draw [in=-150, out=90] (443) to (438);
		\draw [in=90, out=-45, looseness=0.75] (438) to (440.center);
		\draw [in=120, out=-90] (441.center) to (437);
		\draw [in=270, out=90] (442.center) to (446.center);
		\draw [in=270, out=90] (445.center) to (432.center);
	\end{pgfonlayer}
\end{tikzpicture}
\right\rrbracket
=
\begin{bmatrix}
1 & -a & 0 & 0 \\
0 & 1 & 0 & 0 \\
0 & 0 & 1 & 0 \\
0 & 0 & a & 1
\end{bmatrix}
=:C_a
%\begin{bmatrix}
%1 & -a & 0 & 0 \\
%0 & 1 & 0 & 0 \\
%0 & 0 & 1 & 0 \\
%0 & 0 & a & 0
%\end{bmatrix}
$$


%When composed with identities, these have having the following action on Lagrangian subspaces when applied to certain wires:

Use the notation $G^{(j)}$ to denote a gate $G$ being applied to wire $j$; and the notation $C_a^{(i,j)}$ to denote the controlled-$a$ gate controlling on wire $i$ targetting wire $j$.

Note the right action of these gates in terms of matrix multiplication of Lagrangian subspaces for any nonzero $a \in k$ (as observed in \cite[p. 4]{aaronson}):

\begin{itemize}
\item
$F^{(i)}$ sets columns $x_i$ to $-z_i$ and $z_i$ to $x_i$.

\item
$S_a^{(i)}$ sets $z_i$ to $z_i+a\cdot x_i$.

%\item
%$M_a^{(i)}$ sets $x_i$ to $a\cdot x_i$ and $z_i$ to $a^{-1}\cdot x_i$.

\item
$C_a^{(i,j)}$ sets $x_j$ to $x_j- a \cdot x_i$ and $z_i$ to $z_i+a\cdot z_j$.

\end{itemize}

Using these symplectomorphisms regarded as Lagrangian relations, we have:


%
%
%
%\begin{align*}
%&\left[
%\begin{array}{*{7}c|*{7}c}
%x_{1,1} & \cdots & x_{1,a} & \cdots & x_{1,b} & \cdots & x_{1,n} &  z_{1,1} & \cdots & z_{1,a} & \cdots & z_{1,b} & \cdots & z_{1,n} \\-
%\vdots   & \ddots  & \vdots  & \ddots  & \vdots   &  \ddots & \vdots   & \vdots    & \ddots  & \vdots  & \ddots  & \vdots  & \ddots  & \vdots   \\
%x_{n,1} & \cdots & x_{n,a} & \cdots & x_{n,b} & \cdots & x_{n,n} &  z_{n,1} & \cdots & z_{n,a} & \cdots & z_{n,b} & \cdots & z_{n,n}
%\end{array}
%\right]\\
%&\mapsto\\
%&\left[
%\begin{array}{*{7}c|*{7}c}
%x_{1,1} & \cdots & x_{1,a} & \cdots & x_{1,b} - kx_{1,a} & \cdots & x_{1,n} &  z_{1,1} & \cdots & z_{1,a} +k z_{1,b} & \cdots & z_{1,b} & \cdots & z_{1,n} \\
%\vdots   & \ddots  & \vdots  & \ddots  & \vdots                    &  \ddots & \vdots   & \vdots    & \ddots  & \vdots                  & \ddots  & \vdots  & \ddots  & \vdots   \\
%x_{n,1} & \cdots & x_{n,a} & \cdots & x_{n,b} -kx_{n,a} & \cdots & x_{n,n} &  z_{n,1} & \cdots & z_{n,a} +k  z_{n,b}& \cdots & z_{n,b} & \cdots & z_{n,n}
%\end{array}
%\right]
%\end{align*}
%
%
%
%\begin{align*}
%&\left[
%\begin{array}{*{5}c|*{5}c}
%x_{1,1} & \cdots & x_{1,a}  & \cdots & x_{1,n} &  z_{1,1} & \cdots & z_{1,a} & \cdots  & z_{1,n} \\
%\vdots   & \ddots  & \vdots   &  \ddots & \vdots   & \vdots    & \ddots  & \vdots  & \ddots   & \vdots   \\  
%x_{n,1} & \cdots & x_{n,a} & \cdots & x_{n,n} &  z_{n,1} & \cdots & z_{n,a} & \cdots   & z_{n,n}
%\end{array}
%\right]\\
%&\mapsto\\
%&\left[
%\begin{array}{*{7}c|*{7}c}
%x_{1,1} & \cdots & x_{1,a-1} & -z_{1,a} &  x_{1,a+1}   & \cdots & x_{1,n} &  z_{1,1} & \cdots & z_{1,a-1} & x_{1,a} &  z_{1,a+1} & \cdots  & z_{1,n} \\
%\vdots   & \ddots  & \vdots      & \vdots   & \vdots           &  \ddots & \vdots   & \vdots    & \ddots  & \vdots      & \vdots    & \vdots        & \ddots   & \vdots   \\  
%x_{n,1} & \cdots & x_{n,a-1} & -z_{n,a} &  x_{n,a+1} & \cdots & x_{n,n} &  z_{n,1} & \cdots &  z_{n,a-1} & x_{n,a} &  z_{n,a+1} & \cdots   & z_{n,n}
%\end{array}
%\right]\\
%\end{align*}
%
%
%



\begin{theorem}
\label{theorem:generators}
For any field $k$ the maps in $L(\LinRel_k)$ as well as $F$ and $S_a$ for all $a \in k$ generate $\Lag\Rel_k$.
\end{theorem}

\begin{proof}
The following proof is very similar to that of  \cite[Lem. 6]{aaronson}.
Consider the matrix $[Z|X]$ of an arbitrary Lagrangian relation over field $k$ regarded as a state.
%
%$$
%[X | Z]
%=
%\left[
%\begin{array}{ccc|ccc}
%x_{1,1} & \cdots & x_{n,n} & z_{1,1} & \cdots & z_{1,n}\\
%\vdots    & \ddots & \vdots    & \vdots   & \ddots & \vdots \\
%x_{n,1} & \cdots & x_{n,n} & z_{n,1} & \cdots & z_{n,n}\\
%\end{array}
%\right]
%$$
We show how one can reduce $[Z|X]$ to the block matrix $[I|0]$ by right multiplication with the aforementioned symplectomorphisms.
To do so, we first reduce it to a matrix $[I|X']$.
This involves repeatedly applying Gaussian elimination and then applying the Fourier transform to wires when the pivot is in the $X$ block.
We are guaranteed to obtain a matrix $[I|X']$ because the dimension of Lagrangian subspace of $k^{2n}$ is $n$. Moreover, because  $[I|X']$ spans a Lagrangian subspace, we have: 

$$
0
=
\begin{bmatrix}
I | X'
\end{bmatrix}
\omega
\begin{bmatrix}
I | X'
\end{bmatrix}^T
$$
which holds if and only if 
$$
0=
\begin{bmatrix}
I | X'
\end{bmatrix}
\begin{bmatrix}
X' | -I 
\end{bmatrix}^T
=
{X'}^T-X'
$$

That is to say $X'$ is symmetric, meaning that $X'$ describes the adjacency matrix of a graph coloured by the elements of $k$.
In the language of stabilizer circuits, this is called a {\em graph state}.  In the case of prime fields, this observation was made in \cite[Eq. 18]{gross}.  Graph states were originally discussed in \cite{hein2006entanglement}.

We prove that graph states can be reduced to the subspace $[I|0]$ by right multiplication of  symplectomorphisms.
The proof is by induction on the dimension of the subspace.
This base case is trivial.

Suppose we have a $(n+1)$-dimensional Lagrangian subspaces described by a graph state, then:


\begin{align*}
&\hspace*{-2cm}\left[
\begin{array}{*{5}c|*{6}c}
1         & 0        & 0         & \cdots & 0         & x_{1,1} & x_{1,2} & x_{1,3} & \cdots & x_{1,n}\\
0         & 1        & 0         & \cdots & 0         & x_{1,2} & x_{2,2} & x_{2,3} & \cdots & x_{2,n}\\
0         & 0        & 1         & \ddots & \vdots & x_{1,3} & x_{2,3} & x_{3,3} & \cdots & x_{3,n}\\
\vdots & \vdots & \ddots & \ddots & 0         & \vdots   & \vdots    & \vdots    & \ddots &  \vdots \\
0         & 0         & \cdots & 0        & 1          & x_{1,n} & x_{2,n} & x_{3,n} & \cdots & x_{n,n}\\
\end{array}
\right]\\
\xmapsto{(F^{(1)})^{-1}}&
\left[
\begin{array}{*{5}c|*{6}c}
x_{1,1}         & 0        & 0         & \cdots & 0         & -1 & x_{1,2} & x_{1,3} & \cdots & x_{1,n}\\
x_{1,2}         & 1        & 0         & \cdots & 0         & 0 & x_{2,2} & x_{2,3} & \cdots & x_{2,n}\\
x_{1,3}         & 0        & 1         & \ddots & \vdots & 0 & x_{2,3} & x_{3,3} & \cdots & x_{3,n}\\
\vdots & \vdots & \ddots & \ddots & 0         & \vdots   & \vdots    & \vdots    & \ddots &  \vdots \\
x_{1,n}         & 0         & \cdots & 0        & 1          & 0 & x_{2,n} & x_{3,n} & \cdots & x_{n,n}\\
\end{array}
\right]\\
\xmapsto{C_{x_{1,2}}^{(2,1)}}&
\left[
\begin{array}{*{5}c|*{6}c}
x_{1,1}-0           & 0         & 0         & \cdots & 0         & -1 & x_{1,2}-x_{1,2} & x_{1,3} & \cdots & x_{1,n}\\
x_{1,2}-x_{1,2} & 1         & 0         & \cdots & 0         & 0 & x_{2,2}-0            & x_{2,3} & \cdots & x_{2,n}\\
x_{1,3}-0           & 0         & 1         & \ddots & \vdots & 0 & x_{2,3}-0            & x_{3,3} & \cdots & x_{3,n}\\
\vdots                 & \vdots & \ddots & \ddots & 0         & \vdots   & \vdots        & \vdots    & \ddots &  \vdots \\
x_{1,n}-0           & 0         & \cdots & 0        & 1          & 0 & x_{2,n}-0            & x_{3,n} & \cdots & x_{n,n}\\
\end{array}
\right]\\
=&
\left[
\begin{array}{*{5}c|*{6}c}
x_{1,1}             & 0         & 0         & \cdots & 0         & -1 & 0                         & x_{1,3} & \cdots & x_{1,n}\\
0                       & 1         & 0         & \cdots & 0         & 0 & x_{2,2}                & x_{2,3} & \cdots & x_{2,n}\\
x_{1,3}             & 0         & 1         & \ddots & \vdots & 0 & x_{2,3}                & x_{3,3} & \cdots & x_{3,n}\\
\vdots                & \vdots & \ddots & \ddots & 0         & \vdots   & \vdots        & \vdots    & \ddots &  \vdots \\
x_{1,n}             & 0         & \cdots & 0        & 1          & 0 & x_{2,n}                & x_{3,n} & \cdots & x_{n,n}\\
\end{array}
\right]\\
\xmapsto{\prod_{i>1}^n C_{x_{1,i}}^{(i,1)}}&
\left[
\begin{array}{*{5}c|*{6}c}
x_{1,1}             & 0         & 0         & \cdots & 0         & -1 & 0                         & 0           & \cdots & 0\\
0                       & 1         & 0         & \cdots & 0         & 0 & x_{2,2}                & x_{2,3} & \cdots & x_{2,n}\\
0                       & 0         & 1         & \ddots & \vdots & 0 & x_{2,3}                & x_{3,3} & \cdots & x_{3,n}\\
\vdots               & \vdots & \ddots & \ddots & 0         & \vdots   & \vdots        & \vdots    & \ddots &  \vdots \\
0                       & 0         & \cdots & 0        & 1          & 0 & x_{2,n}                & x_{3,n} & \cdots & x_{n,n}\\
\end{array}
\right]\\
\xmapsto{F^{(1)}}&
\left[
\begin{array}{*{5}c|*{6}c}
1                       & 0         & 0         & \cdots & 0         & x_{1,1} & 0                          & 0           & \cdots & 0\\
0                       & 1         & 0         & \cdots & 0         & 0           & x_{2,2}                & x_{2,3} & \cdots & x_{2,n}\\
0                       & 0         & 1         & \ddots & \vdots & 0           & x_{2,3}                & x_{3,3} & \cdots & x_{3,n}\\
\vdots               & \vdots & \ddots & \ddots & 0         & \vdots   & \vdots                   & \vdots    & \ddots &  \vdots \\
0                       & 0         & \cdots & 0        & 1          & 0           & x_{2,n}                & x_{3,n} & \cdots & x_{n,n}\\
\end{array}
\right]\\
\xmapsto{S_{-x_{1,1}}^{(1)}  }&
\left[
\begin{array}{*{5}c|*{6}c}
1                       & 0         & 0         & \cdots & 0         & 0 & 0                          & 0           & \cdots & 0\\
0                       & 1         & 0         & \cdots & 0         & 0           & x_{2,2}                & x_{2,3} & \cdots & x_{2,n}\\
0                       & 0         & 1         & \ddots & \vdots & 0           & x_{2,3}                & x_{3,3} & \cdots & x_{3,n}\\
\vdots               & \vdots & \ddots & \ddots & 0         & \vdots   & \vdots                   & \vdots    & \ddots &  \vdots \\
0                       & 0         & \cdots & 0        & 1          & 0           & x_{2,n}                & x_{3,n} & \cdots & x_{n,n}\\
\end{array}
\right]
\end{align*}

Therefore all Lagrangian relations can be reduced to the subspace $[I|0]$ by right multiplication by symplectomorphisms.
In the $n$-dimensional case, this subspace is given by the circuit
$L(
\begin{tikzpicture}[scale=.5]
	\begin{pgfonlayer}{nodelayer}
		\node [style=X] (0) at (6, 3) {};
		\node [style=none] (1) at (6, 3.5) {};
	\end{pgfonlayer}
	\begin{pgfonlayer}{edgelayer}
		\draw (0) to (1.center);
	\end{pgfonlayer}
\end{tikzpicture}
^{\otimes n}$).

\end{proof} 


%
%This is proved by using these symplectomorphisms to reduce a Lagrangian relation by Guassian elimination to the state
%$L(
%\begin{tikzpicture}[scale=.5]
%	\begin{pgfonlayer}{nodelayer}
%		\node [style=X] (0) at (6, 3) {};
%		\node [style=none] (1) at (6, 3.5) {};
%	\end{pgfonlayer}
%	\begin{pgfonlayer}{edgelayer}
%		\draw (0) to (1.center);
%	\end{pgfonlayer}
%\end{tikzpicture}
%^{\otimes n}$).
%

By decomposing the Fourier transform we obtain a more symmetric, equivalent set of generators:

\begin{corollary}
\label{theorem:unbiased}
$\Lag\Rel_k$ is presented by $L(\LinRel_k)$ as well as the generators:
$$
d_a :=
\begin{tikzpicture}
	\begin{pgfonlayer}{nodelayer}
		\node [style=Z] (0) at (0, 0.75) {};
		\node [style=scalar] (1) at (0.5, 0) {$a$};
		\node [style=none] (2) at (0.5, -0.75) {};
		\node [style=none] (3) at (-0.5, 0) {};
		\node [style=none] (4) at (-0.5, -0.75) {};
	\end{pgfonlayer}
	\begin{pgfonlayer}{edgelayer}
		\draw [in=-30, out=90] (1) to (0);
		\draw [in=90, out=-150] (0) to (3.center);
		\draw (4.center) to (3.center);
		\draw (2.center) to (1);
	\end{pgfonlayer}
\end{tikzpicture}:1\to 0
$$
\end{corollary}


\begin{proof}
We show that $F$ and $S_a$ can be constructed using these generators. The $S_a$ gate and it's colour-reversed version $V_a$ can be obtained by composing a pure morphism with $d_{-a}$ and $d_{a}$, respectively:
$$
\begin{tikzpicture}
	\begin{pgfonlayer}{nodelayer}
		\node [style=none] (0) at (2.25, -1.25) {};
		\node [style=Z] (1) at (2.25, -0.75) {};
		\node [style=X] (2) at (0.5, -0.75) {};
		\node [style=none] (3) at (0.5, -1.25) {};
		\node [style=none] (4) at (0.5, 1.25) {};
		\node [style=none] (5) at (2.25, 1.25) {};
		\node [style=scalar] (6) at (1.75, 0) {$-a$};
		\node [style=none] (7) at (0, 0) {};
		\node [style=Z] (8) at (0.875, 0.75) {};
	\end{pgfonlayer}
	\begin{pgfonlayer}{edgelayer}
		\draw (1) to (0.center);
		\draw (1) to (5.center);
		\draw (2) to (4.center);
		\draw (2) to (3.center);
		\draw [in=-90, out=165] (1) to (6);
		\draw [in=-90, out=165] (2) to (7.center);
		\draw [in=-15, out=90] (6) to (8);
		\draw [in=90, out=-165] (8) to (7.center);
	\end{pgfonlayer}
\end{tikzpicture}
=
\begin{tikzpicture}
	\begin{pgfonlayer}{nodelayer}
		\node [style=none] (9) at (5.5, -1.25) {};
		\node [style=Z] (10) at (5.5, -0.75) {};
		\node [style=X] (11) at (3.75, -0.75) {};
		\node [style=none] (12) at (3.75, -1.25) {};
		\node [style=none] (13) at (3.75, 1.25) {};
		\node [style=none] (14) at (5.5, 1.25) {};
		\node [style=scalar] (15) at (5, 0) {$-a$};
		\node [style=Z] (16) at (4.625, 0.75) {};
		\node [style=X] (17) at (4.25, -0.25) {};
		\node [style=X] (18) at (3.4, 0.5) {};
	\end{pgfonlayer}
	\begin{pgfonlayer}{edgelayer}
		\draw (10) to (9.center);
		\draw (10) to (14.center);
		\draw (11) to (13.center);
		\draw (11) to (12.center);
		\draw [in=-90, out=165] (10) to (15);
		\draw [in=-15, out=90] (15) to (16);
		\draw [in=-120, out=150, looseness=0.75] (11) to (18);
		\draw [in=150, out=-15, looseness=0.75] (18) to (17);
		\draw [in=-120, out=60] (17) to (16);
	\end{pgfonlayer}
\end{tikzpicture}
=
\begin{tikzpicture}
	\begin{pgfonlayer}{nodelayer}
		\node [style=none] (19) at (8.75, -1.75) {};
		\node [style=Z] (20) at (8.75, -1.25) {};
		\node [style=X] (21) at (7, 0) {};
		\node [style=none] (22) at (7, -1.75) {};
		\node [style=none] (23) at (7, 1) {};
		\node [style=none] (24) at (8.75, 1) {};
		\node [style=scalar] (25) at (8.25, -0.7) {$-a$};
		\node [style=X] (26) at (7.75, 0.75) {};
		\node [style=s] (27) at (8.25, 0) {};
	\end{pgfonlayer}
	\begin{pgfonlayer}{edgelayer}
		\draw (20) to (19.center);
		\draw (20) to (24.center);
		\draw (21) to (23.center);
		\draw (21) to (22.center);
		\draw [in=-90, out=150] (20) to (25);
		\draw [in=-150, out=30] (21) to (26);
		\draw (25) to (27);
		\draw [in=-15, out=90] (27) to (26);
	\end{pgfonlayer}
\end{tikzpicture}
=
\begin{tikzpicture}
	\begin{pgfonlayer}{nodelayer}
		\node [style=none] (10) at (10.25, -1) {};
		\node [style=Z] (11) at (10.25, -0.5) {};
		\node [style=X] (12) at (9.25, 0.5) {};
		\node [style=none] (13) at (9.25, -1) {};
		\node [style=scalar] (14) at (9.75, 0) {$a$};
		\node [style=none] (15) at (9.25, 1) {};
		\node [style=none] (16) at (10.25, 1) {};
	\end{pgfonlayer}
	\begin{pgfonlayer}{edgelayer}
		\draw (11) to (10.center);
		\draw [in=-90, out=165] (11) to (14);
		\draw [in=-15, out=90] (14) to (12);
		\draw (13.center) to (12);
		\draw (11) to (16.center);
		\draw (12) to (15.center);
	\end{pgfonlayer}
\end{tikzpicture}
= S_a
$$
$$
\begin{tikzpicture}
	\begin{pgfonlayer}{nodelayer}
		\node [style=none] (28) at (12.5, -1.25) {};
		\node [style=X] (29) at (12.5, -0.75) {};
		\node [style=Z] (30) at (10.75, -0.75) {};
		\node [style=none] (31) at (10.75, -1.25) {};
		\node [style=none] (32) at (10.75, 1.25) {};
		\node [style=none] (33) at (12.5, 1.25) {};
		\node [style=scalar] (34) at (11.925, 0) {$a$};
		\node [style=none] (35) at (10.25, 0) {};
		\node [style=Z] (36) at (11.125, 0.75) {};
	\end{pgfonlayer}
	\begin{pgfonlayer}{edgelayer}
		\draw (29) to (28.center);
		\draw (29) to (33.center);
		\draw (30) to (32.center);
		\draw (30) to (31.center);
		\draw [in=-90, out=150] (29) to (34);
		\draw [in=-90, out=150] (30) to (35.center);
		\draw [in=-15, out=90] (34) to (36);
		\draw [in=90, out=-165] (36) to (35.center);
	\end{pgfonlayer}
\end{tikzpicture}
=
\begin{tikzpicture}
	\begin{pgfonlayer}{nodelayer}
		\node [style=none] (50) at (19.75, -1) {};
		\node [style=X] (51) at (19.75, -0.5) {};
		\node [style=Z] (52) at (18.75, 0.5) {};
		\node [style=none] (53) at (18.75, -1) {};
		\node [style=scalar] (54) at (19.25, 0) {$a$};
		\node [style=none] (55) at (18.75, 1) {};
		\node [style=none] (56) at (19.75, 1) {};
	\end{pgfonlayer}
	\begin{pgfonlayer}{edgelayer}
		\draw (51) to (50.center);
		\draw [in=-90, out=165] (51) to (54);
		\draw [in=-15, out=90] (54) to (52);
		\draw (53.center) to (52);
		\draw (51) to (56.center);
		\draw (52) to (55.center);
	\end{pgfonlayer}
\end{tikzpicture}
 =: V_a
$$



We can then obtain $F$ as $S_1 \circ V_1 \circ S_1$, which can be proven as a variation of the familiar `3 CNOT' rule for quantum circuits (see e.g.~\cite[\S 3.2.1]{coecke2008interacting}):
\begin{align*}
S_1 \circ V_1 \circ S_1
&=
\begin{tikzpicture}[xscale=-1]
	\begin{pgfonlayer}{nodelayer}
		\node [style=Z] (0) at (3, 1.25) {};
		\node [style=X] (1) at (4, 1.75) {};
		\node [style=Z] (2) at (4, 2.5) {};
		\node [style=X] (3) at (3, 2) {};
		\node [style=Z] (4) at (3, 2.75) {};
		\node [style=X] (5) at (4, 3.25) {};
		\node [style=none] (6) at (3, 4.5) {};
		\node [style=none] (7) at (4, 4.5) {};
		\node [style=none] (8) at (3, 0) {};
		\node [style=none] (9) at (4, 0) {};
	\end{pgfonlayer}
	\begin{pgfonlayer}{edgelayer}
		\draw (6.center) to (4);
		\draw (4) to (3);
		\draw (0) to (3);
		\draw (8.center) to (0);
		\draw (9.center) to (1);
		\draw (1) to (2);
		\draw (2) to (5);
		\draw (5) to (7.center);
		\draw (3) to (2);
		\draw (4) to (5);
		\draw (0) to (1);
	\end{pgfonlayer}
\end{tikzpicture}
=
\begin{tikzpicture}[xscale=-1]
	\begin{pgfonlayer}{nodelayer}
		\node [style=Z] (0) at (3, 1.25) {};
		\node [style=X] (1) at (4, 0.75) {};
		\node [style=Z] (2) at (4, 2.5) {};
		\node [style=X] (3) at (3, 2) {};
		\node [style=Z] (4) at (3, 3.75) {};
		\node [style=X] (5) at (4, 3.25) {};
		\node [style=none] (6) at (3, 4.5) {};
		\node [style=none] (7) at (4, 4.5) {};
		\node [style=none] (8) at (3, 0) {};
		\node [style=none] (9) at (4, 0) {};
		\node [style=s] (10) at (3.5, 3.5) {};
		\node [style=s] (11) at (3.5, 1) {};
	\end{pgfonlayer}
	\begin{pgfonlayer}{edgelayer}
		\draw (6.center) to (4);
		\draw (4) to (3);
		\draw (0) to (3);
		\draw (8.center) to (0);
		\draw (9.center) to (1);
		\draw (1) to (2);
		\draw (2) to (5);
		\draw (5) to (7.center);
		\draw (3) to (2);
		\draw (1) to (11);
		\draw (11) to (0);
		\draw (5) to (10);
		\draw (10) to (4);
	\end{pgfonlayer}
\end{tikzpicture}
=
\begin{tikzpicture}[xscale=-1]
	\begin{pgfonlayer}{nodelayer}
		\node [style=X] (28) at (6.5, 0) {};
		\node [style=Z] (31) at (5, 4) {};
		\node [style=none] (33) at (5, 4.5) {};
		\node [style=none] (34) at (6.5, 4.5) {};
		\node [style=none] (35) at (5, -0.5) {};
		\node [style=none] (36) at (6.5, -0.5) {};
		\node [style=s] (37) at (6, 0.5) {};
		\node [style=s] (38) at (5.5, 3.5) {};
		\node [style=X] (39) at (6, 2.25) {};
		\node [style=Z] (40) at (6, 3) {};
		\node [style=X] (41) at (6.5, 2.25) {};
		\node [style=Z] (42) at (6.5, 3) {};
		\node [style=X] (43) at (5, 1) {};
		\node [style=Z] (44) at (5, 1.75) {};
		\node [style=X] (45) at (5.5, 1) {};
		\node [style=Z] (46) at (5.5, 1.75) {};
	\end{pgfonlayer}
	\begin{pgfonlayer}{edgelayer}
		\draw (33.center) to (31);
		\draw (36.center) to (28);
		\draw (28) to (37);
		\draw (31) to (38);
		\draw (40) to (41);
		\draw (41) to (42);
		\draw (40) to (39);
		\draw (39) to (42);
		\draw (44) to (45);
		\draw (45) to (46);
		\draw (44) to (43);
		\draw (43) to (46);
		\draw (34.center) to (42);
		\draw (40) to (38);
		\draw (46) to (39);
		\draw (41) to (28);
		\draw (37) to (45);
		\draw (35.center) to (43);
		\draw (31) to (44);
	\end{pgfonlayer}
\end{tikzpicture}
=
\begin{tikzpicture}[xscale=-1]
	\begin{pgfonlayer}{nodelayer}
		\node [style=X] (67) at (11.5, 0) {};
		\node [style=Z] (68) at (10, 4.25) {};
		\node [style=none] (69) at (10, 4.75) {};
		\node [style=none] (70) at (11.5, 4.75) {};
		\node [style=none] (71) at (10, -0.5) {};
		\node [style=none] (72) at (11.5, -0.5) {};
		\node [style=s] (73) at (11, 0.5) {};
		\node [style=s] (74) at (10.5, 3.75) {};
		\node [style=Z] (75) at (11, 3.25) {};
		\node [style=X] (76) at (11.5, 2.5) {};
		\node [style=Z] (77) at (11.5, 3.25) {};
		\node [style=X] (78) at (10, 1) {};
		\node [style=Z] (79) at (10, 1.75) {};
		\node [style=X] (80) at (10.5, 1) {};
		\node [style=X] (81) at (10.5, 1.75) {};
		\node [style=Z] (82) at (10.5, 2.5) {};
		\node [style=X] (83) at (11, 1.75) {};
		\node [style=Z] (84) at (11, 2.5) {};
	\end{pgfonlayer}
	\begin{pgfonlayer}{edgelayer}
		\draw (69.center) to (68);
		\draw (72.center) to (67);
		\draw (67) to (73);
		\draw (68) to (74);
		\draw (75) to (76);
		\draw (76) to (77);
		\draw (79) to (80);
		\draw (79) to (78);
		\draw (70.center) to (77);
		\draw (75) to (74);
		\draw (76) to (67);
		\draw (73) to (80);
		\draw (71.center) to (78);
		\draw (68) to (79);
		\draw (82) to (83);
		\draw (83) to (84);
		\draw (82) to (81);
		\draw (81) to (84);
		\draw (78) to (81);
		\draw (80) to (83);
		\draw (84) to (77);
		\draw (75) to (82);
	\end{pgfonlayer}
\end{tikzpicture}\\
&=
\begin{tikzpicture}[xscale=-1]
	\begin{pgfonlayer}{nodelayer}
		\node [style=X] (0) at (11, 0.75) {};
		\node [style=Z] (1) at (10.25, 4.75) {};
		\node [style=none] (2) at (10.25, 5.25) {};
		\node [style=none] (3) at (11, 5.25) {};
		\node [style=none] (4) at (10.25, 0.25) {};
		\node [style=none] (5) at (11, 0.25) {};
		\node [style=s] (6) at (11, 1.5) {};
		\node [style=s] (7) at (10.25, 4) {};
		\node [style=Z] (8) at (10.25, 3.25) {};
		\node [style=X] (9) at (11, 0.75) {};
		\node [style=Z] (10) at (11, 3.25) {};
		\node [style=X] (11) at (10.25, 2.25) {};
		\node [style=Z] (12) at (10.25, 4.75) {};
		\node [style=X] (13) at (11, 2.25) {};
		\node [style=X] (14) at (10.25, 2.25) {};
		\node [style=Z] (15) at (10.25, 3.25) {};
		\node [style=X] (16) at (11, 2.25) {};
		\node [style=Z] (17) at (11, 3.25) {};
	\end{pgfonlayer}
	\begin{pgfonlayer}{edgelayer}
		\draw (2.center) to (1);
		\draw (5.center) to (0);
		\draw (0) to (6);
		\draw (1) to (7);
		\draw [in=135, out=-60] (8) to (9);
		\draw [bend right, looseness=1.25] (9) to (10);
		\draw [in=120, out=-45] (12) to (13);
		\draw [bend right, looseness=1.25] (12) to (11);
		\draw (3.center) to (10);
		\draw (8) to (7);
		\draw (6) to (13);
		\draw (4.center) to (11);
		\draw [in=150, out=-30, looseness=0.75] (15) to (16);
		\draw (16) to (17);
		\draw (15) to (14);
		\draw (14) to (17);
	\end{pgfonlayer}
\end{tikzpicture}
=
\begin{tikzpicture}[xscale=-1]
	\begin{pgfonlayer}{nodelayer}
		\node [style=X] (0) at (19.5, 1.25) {};
		\node [style=Z] (1) at (18, 4.25) {};
		\node [style=none] (2) at (18, 4.75) {};
		\node [style=none] (3) at (19.5, 4.75) {};
		\node [style=none] (4) at (18, 0.75) {};
		\node [style=none] (5) at (19.5, 0.75) {};
		\node [style=X] (6) at (19.5, 1.25) {};
		\node [style=Z] (7) at (19.5, 4.25) {};
		\node [style=X] (8) at (18, 1.25) {};
		\node [style=Z] (9) at (18, 4.25) {};
		\node [style=X] (10) at (19.5, 1.25) {};
		\node [style=X] (11) at (18, 1.25) {};
		\node [style=X] (12) at (19.5, 1.25) {};
		\node [style=Z] (13) at (19.5, 4.25) {};
		\node [style=s] (14) at (17.75, 3) {};
		\node [style=s] (15) at (18.75, 3) {};
		\node [style=s] (16) at (19.75, 3) {};
		\node [style=s] (17) at (18.25, 3) {};
	\end{pgfonlayer}
	\begin{pgfonlayer}{edgelayer}
		\draw (2.center) to (1);
		\draw (5.center) to (0);
		\draw [bend right=45] (6) to (7);
		\draw [in=120, out=-135, looseness=1.25] (9) to (8);
		\draw (3.center) to (7);
		\draw (4.center) to (8);
		\draw [in=-120, out=15, looseness=0.75] (11) to (13);
		\draw [in=90, out=-105] (9) to (14);
		\draw [in=90, out=-45, looseness=0.75] (9) to (15);
		\draw [in=90, out=-90] (14) to (11);
		\draw [in=-90, out=120, looseness=0.75] (6) to (15);
		\draw [in=-15, out=90] (12) to (9);
		\draw [in=-90, out=150, looseness=0.75] (12) to (17);
		\draw [in=285, out=90] (17) to (9);
		\draw [in=-90, out=75, looseness=0.75] (12) to (16);
		\draw [in=-75, out=90] (16) to (13);
	\end{pgfonlayer}
\end{tikzpicture}
=
\begin{tikzpicture}
	\begin{pgfonlayer}{nodelayer}
		\node [style=none] (0) at (24, 4.5) {};
		\node [style=none] (1) at (23.5, 4) {};
		\node [style=none] (2) at (24, 2.75) {};
		\node [style=none] (3) at (23.5, 2.75) {};
		\node [style=s] (4) at (24, 4) {};
		\node [style=none] (5) at (23.5, 4.5) {};
	\end{pgfonlayer}
	\begin{pgfonlayer}{edgelayer}
		\draw [in=-90, out=90, looseness=1.25] (2.center) to (1.center);
		\draw (1.center) to (5.center);
		\draw (0.center) to (4);
		\draw [in=90, out=-90, looseness=1.25] (4) to (3.center);
	\end{pgfonlayer}
\end{tikzpicture}
=
F
\end{align*}
\end{proof}

In the ZX-calculus literature, this decomposition of the Fourier transform is known as {\it Euler decomposition} \cite{duncan2009graph}.
A variant of this decomposition is given in \cite[p.6]{control}; although in the context of plain old linear relations instead of Lagrangian relations, so an antipode is missing in their case.  A similar observation was made in \cite[Eq. 34]{ranchin2014depicting} in terms of qudit controlled-{\cal X} gates; however, the connection to phase-shift gates and Euler decomposition was not made.

From Corollary \ref{theorem:unbiased}, we know that we can build any Lagrangian relation using pure Lagrangian relations and discard maps. Since the former is closed under composition and monoidal product, the following can be shown immediately from string diagram deformation.

\begin{corollary}[Phase purification]\label{cor:pure}
Any linear Lagrangian relation can be written in the following form, for $V$ a linear relation:
$$
\begin{tikzpicture}
	\begin{pgfonlayer}{nodelayer}
		\node [style=Z] (37) at (17, 2) {};
		\node [style=scalar] (38) at (18.25, 1) {$a_1$};
		\node [style=none] (39) at (18.25, 0.25) {};
		\node [style=none] (40) at (15, 0.75) {};
		\node [style=none] (41) at (15, 0.25) {};
		\node [style=map, minimum width=2cm, minimum height=1cm] (42) at (15.75, -0.25) {$V^\perp$};
		\node [style=map, minimum width=2cm, minimum height=1cm] (43) at (19, -0.25) {$V$};
		\node [style=Z] (44) at (18.25, 2) {};
		\node [style=scalar] (45) at (19.25, 1) {$a_k$};
		\node [style=none] (46) at (19.25, 0.25) {};
		\node [style=none] (47) at (16, 0.75) {};
		\node [style=none] (48) at (16, 0.25) {};
		\node [style=none] (49) at (18.75, 0.5) {...};
		\node [style=none] (50) at (16.5, 0.25) {};
		\node [style=none] (51) at (16.5, 2.5) {};
		\node [style=none] (52) at (19.75, 0.25) {};
		\node [style=none] (53) at (19.75, 2.5) {};
		\node [style=none] (54) at (15.5, 0.5) {...};
		\node [style=none] (55) at (15.75, -1.25) {};
		\node [style=none] (56) at (15.75, -0.5) {};
		\node [style=none] (57) at (19, -1.25) {};
		\node [style=none] (58) at (19, -0.5) {};
	\end{pgfonlayer}
	\begin{pgfonlayer}{edgelayer}
		\draw [in=-30, out=90, looseness=0.75] (38) to (37);
		\draw [in=90, out=-165, looseness=0.50] (37) to (40.center);
		\draw (41.center) to (40.center);
		\draw (39.center) to (38);
		\draw [in=-30, out=90, looseness=0.75] (45) to (44);
		\draw [in=90, out=-165, looseness=0.50] (44) to (47.center);
		\draw (48.center) to (47.center);
		\draw (46.center) to (45);
		\draw [in=270, out=90] (50.center) to (51.center);
		\draw [in=270, out=90] (52.center) to (53.center);
		\draw [in=270, out=90] (55.center) to (56.center);
		\draw [in=270, out=90] (57.center) to (58.center);
	\end{pgfonlayer}
\end{tikzpicture}
$$
\end{corollary}


%Recall that given a compact closed prop with an identity on objects conjugation functor, applying the CPM construction produces a compact closed prop.
%
%We have already shown that there is a doubling functor $\LinRel_k\to\Lag\Rel_k$, with respect to the orthogonal complement.  To get the $\CPM$ construction one has to add the "discard map" coming from the distinguished closed structure of $\LinRel_k$.  In the case when $k=\F_p$ or $k=\mathbb Q$, this produces  $\Lag\Rel_k$:

Recall that given a compact closed prop with an identity on objects monoidal conjugation functor, we can apply the CPM construction to obtain a compact closed prop.

The phase-purification is so-named because of the similarity to the purification of maps in the CPM-construction.  Indeed one might hope that $\Lag\Rel_k$ is isomorphic to $\CPM(\LinRel_k,\perp)$, however, in the example of phase-purification, we are tracing out both sides with respect to multiple compact closed structures:  one for each $a_j \in k$.  However prime fields are special:


\begin{corollary}
\label{cor}
Given $k=\F_p$ for $p$ prime or $k=\mathbb{Q}$:

$$\CPM(\LinRel_{k}, \perp) \cong \Lag\Rel_k$$
\end{corollary}
\begin{proof}
Remark that:
$$
\begin{tikzpicture}
	\begin{pgfonlayer}{nodelayer}
		\node [style=Z] (95) at (27.35, 2) {};
		\node [style=scalar] (96) at (27.85, 1.5) {$n$};
		\node [style=none] (97) at (26.85, 1.5) {};
		\node [style=none] (98) at (26.85, 1) {};
		\node [style=none] (99) at (27.85, 1) {};
	\end{pgfonlayer}
	\begin{pgfonlayer}{edgelayer}
		\draw [in=-15, out=90] (96) to (95);
		\draw [in=-165, out=90] (97.center) to (95);
		\draw (98.center) to (97.center);
		\draw (99.center) to (96);
	\end{pgfonlayer}
\end{tikzpicture}
=
\begin{tikzpicture}
	\begin{pgfonlayer}{nodelayer}
		\node [style=X] (100) at (29.85, 1.75) {};
		\node [style=Z] (101) at (30.85, 0.5) {};
		\node [style=none] (102) at (28.85, 0) {};
		\node [style=none] (103) at (30.85, 0) {};
		\node [style=none] (104) at (30.35, 1.25) {$\iddots$};
		\node [style=none] (105) at (30.5, 1) {$n$};
		\node [style=Z] (106) at (29.35, 2.25) {};
	\end{pgfonlayer}
	\begin{pgfonlayer}{edgelayer}
		\draw [in=0, out=90, looseness=1.25] (101) to (100);
		\draw [in=-90, out=165, looseness=1.25] (101) to (100);
		\draw [in=-120, out=90] (102.center) to (106);
		\draw [in=105, out=-15] (106) to (100);
		\draw (103.center) to (101);
	\end{pgfonlayer}
\end{tikzpicture}
=
\begin{tikzpicture}
	\begin{pgfonlayer}{nodelayer}
		\node [style=Z] (125) at (34.5, 1) {};
		\node [style=X] (126) at (33, 1) {};
		\node [style=Z] (127) at (33.75, 1.75) {};
		\node [style=Z] (130) at (33.75, 3) {};
		\node [style=none] (131) at (33.75, 2.34) {$\vdots$};
		\node [style=none] (132) at (34, 2.25) {$n$};
		\node [style=none] (133) at (33, 0.25) {};
		\node [style=none] (134) at (34.5, 0.25) {};
	\end{pgfonlayer}
	\begin{pgfonlayer}{edgelayer}
		\draw [in=345, out=120, looseness=0.75] (125) to (127);
		\draw [in=60, out=-165, looseness=0.75] (127) to (126);
		\draw [in=225, out=105] (126) to (130);
		\draw [in=75, out=-45] (130) to (125);
		\draw (133.center) to (126);
		\draw (134.center) to (125);
	\end{pgfonlayer}
\end{tikzpicture}
=
\begin{tikzpicture}
	\begin{pgfonlayer}{nodelayer}
		\node [style=Z] (59) at (24, -0.25) {};
		\node [style=X] (60) at (22.75, -0.25) {};
		\node [style=Z] (61) at (23.25, 0.75) {};
		\node [style=none] (62) at (22.75, -1) {};
		\node [style=none] (63) at (24, -1) {};
		\node [style=Z] (64) at (24, 1) {};
		\node [style=X] (65) at (22.75, 1) {};
		\node [style=Z] (66) at (23.25, 2) {};
		\node [style=Z] (67) at (24, 2.25) {};
		\node [style=X] (68) at (22.75, 2.25) {};
		\node [style=none] (69) at (23.25, 1.515) {$\vdots$};
		\node [style=none] (70) at (23.5, 1.425) {$n$};
	\end{pgfonlayer}
	\begin{pgfonlayer}{edgelayer}
		\draw [in=-15, out=135] (59) to (61);
		\draw [in=150, out=-165, looseness=1.25] (61) to (60);
		\draw (62.center) to (60);
		\draw (63.center) to (59);
		\draw [in=-15, out=135] (64) to (66);
		\draw [in=135, out=-165, looseness=1.25] (66) to (65);
		\draw (60) to (65);
		\draw (65) to (68);
		\draw (67) to (64);
		\draw (64) to (59);
	\end{pgfonlayer}
\end{tikzpicture}
$$

Therefore, every map in $\Lag\Rel_{\F_p}$ can be produced by tracing out a map in $L(\LinRel_{\F_p})$.  For the rationals, observe that for every $n \in \N$:

$$
\begin{tikzpicture}
	\begin{pgfonlayer}{nodelayer}
		\node [style=Z] (95) at (27.35, 2) {};
		\node [style=scalarop] (96) at (27.85, 1.5) {$n$};
		\node [style=none] (97) at (26.85, 1.5) {};
		\node [style=none] (98) at (26.85, 1) {};
		\node [style=none] (99) at (27.85, 1) {};
	\end{pgfonlayer}
	\begin{pgfonlayer}{edgelayer}
		\draw [in=-15, out=90] (96) to (95);
		\draw [in=-165, out=90] (97.center) to (95);
		\draw (98.center) to (97.center);
		\draw (99.center) to (96);
	\end{pgfonlayer}
\end{tikzpicture}
=
\begin{tikzpicture}
	\begin{pgfonlayer}{nodelayer}
		\node [style=Z] (100) at (29.85, 1.75) {};
		\node [style=X] (101) at (30.85, 0.5) {};
		\node [style=none] (102) at (28.85, 0) {};
		\node [style=none] (103) at (30.85, 0) {};
		\node [style=none] (104) at (30.35, 1.25) {$\iddots$};
		\node [style=none] (105) at (30.5, 1) {$n$};
		\node [style=Z] (106) at (29.35, 2.25) {};
	\end{pgfonlayer}
	\begin{pgfonlayer}{edgelayer}
		\draw [in=0, out=90, looseness=1.25] (101) to (100);
		\draw [in=-90, out=165, looseness=1.25] (101) to (100);
		\draw [in=-120, out=90] (102.center) to (106);
		\draw [in=105, out=-15] (106) to (100);
		\draw (103.center) to (101);
	\end{pgfonlayer}
\end{tikzpicture}
=
\begin{tikzpicture}
	\begin{pgfonlayer}{nodelayer}
		\node [style=X] (125) at (34.5, 1) {};
		\node [style=Z] (126) at (33, 1) {};
		\node [style=Z] (127) at (33.75, 1.75) {};
		\node [style=Z] (130) at (33.75, 3) {};
		\node [style=none] (131) at (33.75, 2.34) {$\vdots$};
		\node [style=none] (132) at (34, 2.25) {$n$};
		\node [style=none] (133) at (33, 0.25) {};
		\node [style=none] (134) at (34.5, 0.25) {};
	\end{pgfonlayer}
	\begin{pgfonlayer}{edgelayer}
		\draw [in=345, out=120, looseness=0.75] (125) to (127);
		\draw [in=60, out=-165, looseness=0.75] (127) to (126);
		\draw [in=225, out=105] (126) to (130);
		\draw [in=75, out=-45] (130) to (125);
		\draw (133.center) to (126);
		\draw (134.center) to (125);
	\end{pgfonlayer}
\end{tikzpicture}
=
\begin{tikzpicture}
	\begin{pgfonlayer}{nodelayer}
		\node [style=X] (71) at (26.75, -0.25) {};
		\node [style=Z] (72) at (25.5, -0.25) {};
		\node [style=Z] (73) at (26, 0.75) {};
		\node [style=none] (74) at (25.5, -1) {};
		\node [style=none] (75) at (26.75, -1) {};
		\node [style=X] (76) at (26.75, 1) {};
		\node [style=Z] (77) at (25.5, 1) {};
		\node [style=Z] (78) at (26, 2) {};
		\node [style=X] (79) at (26.75, 2.25) {};
		\node [style=Z] (80) at (25.5, 2.25) {};
		\node [style=none] (81) at (26, 1.515) {$\vdots$};
		\node [style=none] (82) at (26.25, 1.425) {$n$};
	\end{pgfonlayer}
	\begin{pgfonlayer}{edgelayer}
		\draw [in=-30, out=135] (71) to (73);
		\draw [in=135, out=-165] (73) to (72);
		\draw (74.center) to (72);
		\draw (75.center) to (71);
		\draw [in=-30, out=135] (76) to (78);
		\draw [in=135, out=-165] (78) to (77);
		\draw (72) to (77);
		\draw (77) to (80);
		\draw (79) to (76);
		\draw (76) to (71);
	\end{pgfonlayer}
\end{tikzpicture}
$$

So that given any rational $n/m$:

$$
\begin{tikzpicture}
	\begin{pgfonlayer}{nodelayer}
		\node [style=Z] (95) at (27.35, 2.5) {};
		\node [style=scalar] (96) at (27.85, 1.65) {$n/m$};
		\node [style=none] (97) at (26.85, 1.5) {};
		\node [style=none] (98) at (26.85, 1) {};
		\node [style=none] (99) at (27.85, 1) {};
	\end{pgfonlayer}
	\begin{pgfonlayer}{edgelayer}
		\draw [in=-15, out=90] (96) to (95);
		\draw [in=-165, out=90] (97.center) to (95);
		\draw (98.center) to (97.center);
		\draw (99.center) to (96);
	\end{pgfonlayer}
\end{tikzpicture}
=
\begin{tikzpicture}
	\begin{pgfonlayer}{nodelayer}
		\node [style=Z] (208) at (54.25, 2.45) {};
		\node [style=scalar] (209) at (54.75, 1.75) {$n$};
		\node [style=none] (210) at (53.75, 1.75) {};
		\node [style=none] (211) at (53.75, 0.5) {};
		\node [style=none] (212) at (54.75, 0.5) {};
		\node [style=scalarop,fill=white] (213) at (54.75, 1) {$m$};
	\end{pgfonlayer}
	\begin{pgfonlayer}{edgelayer}
		\draw [in=-15, out=90] (209) to (208);
		\draw [in=-165, out=90] (210.center) to (208);
		\draw (211.center) to (210.center);
		\draw (212.center) to (209);
	\end{pgfonlayer}
\end{tikzpicture}
=
\begin{tikzpicture}
	\begin{pgfonlayer}{nodelayer}
		\node [style=Z] (97) at (62.5, 1) {};
		\node [style=X] (98) at (61.25, 1) {};
		\node [style=Z] (99) at (61.75, 1.75) {};
		\node [style=none] (100) at (61.25, 0) {};
		\node [style=none] (101) at (62.5, 0) {};
		\node [style=Z] (102) at (62.5, 2.25) {};
		\node [style=X] (103) at (61.25, 2.25) {};
		\node [style=Z] (104) at (61.75, 3) {};
		\node [style=Z] (105) at (62.5, 3.5) {};
		\node [style=X] (106) at (61.25, 3.5) {};
		\node [style=none] (107) at (61.75, 2.315) {$\vdots$};
		\node [style=none] (108) at (62, 2.225) {$n$};
		\node [style=scalarop, fill=white] (109) at (62.5, 0.5) {$m$};
	\end{pgfonlayer}
	\begin{pgfonlayer}{edgelayer}
		\draw [in=-30, out=150] (97) to (99);
		\draw [in=135, out=-165, looseness=1.25] (99) to (98);
		\draw (100.center) to (98);
		\draw (101.center) to (97);
		\draw [in=-15, out=135] (102) to (104);
		\draw [in=135, out=-165, looseness=1.25] (104) to (103);
		\draw (98) to (103);
		\draw (103) to (106);
		\draw (105) to (102);
		\draw (102) to (97);
	\end{pgfonlayer}
\end{tikzpicture}
=
\begin{tikzpicture}
	\begin{pgfonlayer}{nodelayer}
		\node [style=Z] (83) at (59.75, 0.75) {};
		\node [style=X] (84) at (58.25, 0.75) {};
		\node [style=Z] (85) at (58.75, 2) {};
		\node [style=none] (86) at (58.25, -0.25) {};
		\node [style=none] (87) at (59.75, -0.25) {};
		\node [style=Z] (88) at (59.75, 2.25) {};
		\node [style=X] (89) at (58.25, 2.25) {};
		\node [style=Z] (90) at (58.75, 3.5) {};
		\node [style=Z] (91) at (59.75, 3.75) {};
		\node [style=X] (92) at (58.25, 3.75) {};
		\node [style=none] (93) at (58.75, 2.59) {$\vdots$};
		\node [style=none] (94) at (59, 2.5) {$n$};
		\node [style=scalarop] (95) at (59.3, 1.5) {$m$};
		\node [style=scalarop] (96) at (59.3, 3) {$m$};
	\end{pgfonlayer}
	\begin{pgfonlayer}{edgelayer}
		\draw [in=135, out=-165, looseness=1.25] (85) to (84);
		\draw (86.center) to (84);
		\draw (87.center) to (83);
		\draw [in=135, out=-165, looseness=1.25] (90) to (89);
		\draw (84) to (89);
		\draw (89) to (92);
		\draw (91) to (88);
		\draw (88) to (83);
		\draw [in=-90, out=150] (88) to (96);
		\draw [in=-15, out=90] (96) to (90);
		\draw [in=-90, out=150] (83) to (95);
		\draw [in=90, out=-15] (85) to (95);
	\end{pgfonlayer}
\end{tikzpicture}
$$
Thus, by applying the previous equation $m$ times, this can be rewritten into an circuit in the image of $L(\LinRel_{\mathbb Q})$ followed by $n\times m$ traces.
\end{proof}

\section{Affine Lagrangian relations}
\label{sec:aff}

Affine Lagrangian relations are perhaps of more practical interest than linear Lagrangian relations.  As we will discuss in this section, these give a semantics for qudit stabilizer circuits as well as certain electrical circuits with current and voltage sources.


\begin{definition}
An {\bf affine Lagrangian subspace} of a symplectic vector space $k^{2n}$ is an affine subspace of $k^{2n}$ with linear part $L$ and affine shift $a$, $L+a \subseteq k^{2n}$, where $L$ is a Lagrangian subspace.

Let $\Aff\Lag\Rel_k$ denote the prop category whose morphisms $n\to m$ are affine Lagrangian subspaces of $k^{2(n+m)}$ or the empty affine space.  The monoidal structure and composition is the same as for linear Lagrangian relations.
\end{definition}
Because the tensor product is defined in the same way as in $\Lag\Rel_k$, as in Lemma \ref{lemma:strong}, the forgetful functor  $\Aff\Lag\Rel_k\to \Aff\Rel_k$ is faithful, strong monoidal.



\begin{definition}
Let $\alr_k$ denote the monoidal subcategory of $\aih_k$ with objects $2n$, generated by the morphisms in the image of $\Lag\Rel_k\xrightarrow{E} \LinRel_k \cong \ih_k \to \aih_k$ as well as the following generator:
$$
\begin{tikzpicture}
	\begin{pgfonlayer}{nodelayer}
		\node [style=X] (0) at (0, 0) {$1$};
		\node [style=Z] (1) at (0.5, 0) {};
		\node [style=none] (2) at (0, 0.75) {};
		\node [style=none] (3) at (0.5, 0.75) {};
	\end{pgfonlayer}
	\begin{pgfonlayer}{edgelayer}
		\draw (1) to (3.center);
		\draw (2.center) to (0);
	\end{pgfonlayer}
\end{tikzpicture}
$$
\end{definition}

\begin{lemma}
\label{lem:alr}
$\alr_k$ is a presentation of $\Aff\Lag\Rel_k$.
\end{lemma}
\begin{proof}
All the affine shifts can be produced from tensoring and composing these two maps on the right:
$$
\begin{tikzpicture}
	\begin{pgfonlayer}{nodelayer}
		\node [style=X] (0) at (0, 0) {$1$};
		\node [style=Z] (1) at (0.5, 0) {};
		\node [style=none] (2) at (0, 0.75) {};
		\node [style=none] (3) at (0.5, 0.75) {};
		\node [style=none] (4) at (0.5, 1.5) {};
		\node [style=none] (5) at (0, 1.5) {};
		\node [style=s] (6) at (0.5, 0.75) {};
	\end{pgfonlayer}
	\begin{pgfonlayer}{edgelayer}
		\draw (1) to (3.center);
		\draw (2.center) to (0);
		\draw [in=270, out=90] (3.center) to (5.center);
		\draw [in=270, out=90] (2.center) to (4.center);
	\end{pgfonlayer}
\end{tikzpicture}
=
\begin{tikzpicture}
	\begin{pgfonlayer}{nodelayer}
		\node [style=X] (0) at (0.5, 0) {$1$};
		\node [style=Z] (1) at (0, 0) {};
		\node [style=none] (2) at (0.5, 0.75) {};
		\node [style=none] (3) at (0, 0.75) {};
	\end{pgfonlayer}
	\begin{pgfonlayer}{edgelayer}
		\draw (1) to (3.center);
		\draw (2.center) to (0);
	\end{pgfonlayer}
\end{tikzpicture} \hspace*{.1cm} \in  \alr_k
\hspace*{.5cm}
\implies
\hspace*{.5cm}
\begin{tikzpicture}
	\begin{pgfonlayer}{nodelayer}
		\node [style=Z] (447) at (224.25, 0.75) {};
		\node [style=X] (448) at (222.75, 0.75) {};
		\node [style=none] (449) at (223.75, -0.25) {};
		\node [style=none] (450) at (222.25, -0.25) {};
		\node [style=none] (451) at (224.25, 1.5) {};
		\node [style=none] (452) at (222.75, 1.5) {};
		\node [style=X] (453) at (223.25, -1) {$1$};
		\node [style=Z] (454) at (224.75, -1) {};
		\node [style=scalar] (455) at (223.25, -0.25) {$a$};
		\node [style=scalarop] (456) at (224.75, -0.25) {$a$};
		\node [style=none] (457) at (223.75, -1.5) {};
		\node [style=none] (458) at (222.25, -1.5) {};
	\end{pgfonlayer}
	\begin{pgfonlayer}{edgelayer}
		\draw [in=90, out=-150] (447) to (449.center);
		\draw [in=-150, out=90] (450.center) to (448);
		\draw (447) to (451.center);
		\draw (448) to (452.center);
		\draw (457.center) to (449.center);
		\draw (458.center) to (450.center);
		\draw (453) to (455);
		\draw (454) to (456);
		\draw [in=-30, out=90] (456) to (447);
		\draw [in=-30, out=90] (455) to (448);
	\end{pgfonlayer}
\end{tikzpicture}
=
\begin{tikzpicture}
	\begin{pgfonlayer}{nodelayer}
		\node [style=X] (17) at (5, 0) {$a$};
		\node [style=none] (18) at (5.5, -1.5) {};
		\node [style=none] (19) at (5, -1.5) {};
		\node [style=none] (20) at (5.5, 1.5) {};
		\node [style=none] (21) at (5, 1.5) {};
	\end{pgfonlayer}
	\begin{pgfonlayer}{edgelayer}
		\draw (19.center) to (17);
		\draw (17) to (21.center);
		\draw (18.center) to (20.center);
	\end{pgfonlayer}
\end{tikzpicture},
\hspace*{.5cm}
\begin{tikzpicture}
	\begin{pgfonlayer}{nodelayer}
		\node [style=Z] (459) at (226.25, 0.75) {};
		\node [style=X] (460) at (227.75, 0.75) {};
		\node [style=none] (461) at (226.75, -0.25) {};
		\node [style=none] (462) at (228.25, -0.25) {};
		\node [style=none] (463) at (226.25, 1.5) {};
		\node [style=none] (464) at (227.75, 1.5) {};
		\node [style=X] (465) at (227.25, -1) {$1$};
		\node [style=Z] (466) at (225.75, -1) {};
		\node [style=scalar] (467) at (227.25, -0.25) {$a$};
		\node [style=scalarop] (468) at (225.75, -0.25) {$a$};
		\node [style=none] (469) at (226.75, -1.5) {};
		\node [style=none] (470) at (228.25, -1.5) {};
	\end{pgfonlayer}
	\begin{pgfonlayer}{edgelayer}
		\draw [in=90, out=-30] (459) to (461.center);
		\draw [in=-30, out=90] (462.center) to (460);
		\draw (459) to (463.center);
		\draw (460) to (464.center);
		\draw (469.center) to (461.center);
		\draw (470.center) to (462.center);
		\draw (465) to (467);
		\draw (466) to (468);
		\draw [in=-150, out=90] (468) to (459);
		\draw [in=-150, out=90] (467) to (460);
	\end{pgfonlayer}
\end{tikzpicture}
=
\begin{tikzpicture}
	\begin{pgfonlayer}{nodelayer}
		\node [style=X] (0) at (1.5, 0) {$a$};
		\node [style=none] (1) at (1, -1.5) {};
		\node [style=none] (2) at (1.5, -1.5) {};
		\node [style=none] (3) at (1, 1.5) {};
		\node [style=none] (4) at (1.5, 1.5) {};
	\end{pgfonlayer}
	\begin{pgfonlayer}{edgelayer}
		\draw (2.center) to (0);
		\draw (0) to (4.center);
		\draw (1.center) to (3.center);
	\end{pgfonlayer}
\end{tikzpicture}
\hspace*{.1cm}\in\alr_k
$$
\end{proof}
Therefore, we are justified in using string diagrams for affine relations to reason about affine Lagrangian relations.

%We will restate the interpretations given in \cite{affine} of some components for electrical circuits in terms  affine relations  in terms of the generators for graphical calculus for Lagrangian relations.  This interpretation is also explored in \cite{passive,network}; albeit, not enjoying the graphical calculus for affine relations.

\subsection{Stabilizer circuits and Spekkens' toy model}
The connection between the stabilizer formalism and symplectic geometry has been known for quite a while, at least as early as the papers of Calderbank, Rains, Shor and Sloane
\cite{css,cssone} in the qubit case.
This developped in great detail in the quopit case by Gross \cite{gross}. However, its role in the stabilizer formalism is often underplayed, for example, it is not explicitly mentioned in Gottesman's highly influential PhD thesis \cite{gottesman}.  Perhaps a reason for this is that despite their dominance in the quantum computer literature, qubit stabilizer circuits do not conform so nicely to the symplectic geometric framework as do all other prime qudit dimensions

In this subsection, we with build on the work of Gross and show that, when $p$ is an odd prime, the prop of affine Lagrangian relations over $\F_p$  is isomorphic to quopit stabilizer circuits modulo invertible scalars. 


To show this, we must first recall some very important results of Gross, relating the stabilizer formalism to symplectic geometry.  First we need the following convention to represent Weyl operators:

\begin{definition}
Given $a \in \F_p$ and $(z,x )\in \F_p^{2n}$, define:
$$
\chi(a) = e^{2\pi\cdot i \cdot a/p}, \hspace*{1cm} {\mathcal W}(z,x)=\chi(-zx^T/2)\bigotimes_{j=0}^{n-1}\mathcal{Z}_{(j)}^{z_j}\mathcal{X}_{(j)}^{x_j}
$$
Where the subscript $(j)$ denotes that we are applying a gate on wire $(j)$ tensored on both sides with identities.

\end{definition}


This makes the following result more easy to prove:

\begin{lemma}\cite[Thm. 3]{gross}
For every odd prime $p$ the prop  of affine symplectomorphisms over  $\F_p$ is isomorphic to the quopit Clifford groupoid modulo scalars.
\end{lemma}



\begin{proof}
We know that the Clifford group is defined as the normalizer of the Pauli group so that a Clifford operator is defined by its action on Weyl operators. Given an $n$ quopit Clifford operator $C$  and $(z,x)\in \F_p^{2n}$, there exists  an isomorphism $C_L:\F_p^{2n}\to \F_p^{2n}$ and a vector  $C_a\in \F_p^{2n}$ such that:

$$
C {\mathcal W}(z,x) C^\dag
=
\chi(C_a(z,x)){\mathcal W}(C_L(z,x))
$$


We seek to show that $C_L$ is the the symplectomorphism and $C_a$ is the affine shift.  $C_L$ is clearly linear.  To see that it is a symplectomorphism, first observe:

$$
\mathcal{Z}\mathcal{X}  = \chi(1)\mathcal{X} \mathcal{Z}
$$

Therefore, 


\begin{align*}
(\mathcal{Z}^{z_0}\mathcal{X}^{x_0})(\mathcal{Z}^{z_1}\mathcal{X}^{x_1} )
&=\chi(-x_0z_1) \mathcal{Z}^{z_0}\mathcal{Z}^{z_1}\mathcal{X}^{x_0}\mathcal{X}^{x_1}  \\
&=\chi(-x_0z_1)\mathcal{Z}^{z_1}  \mathcal{Z}^{z_0}\mathcal{X}^{x_1} \mathcal{X}^{x_0}  \\
&=\chi(z_0x_1-x_0z_1)(\mathcal{Z}^{z_1} \mathcal{X}^{x_1})(  \mathcal{Z}^{z_0}\mathcal{X}^{x_0} ) \\
&=\chi(\omega((z_0,x_0),(z_1,x_1)))(\mathcal{Z}^{z_1} \mathcal{X}^{x_1})(  \mathcal{Z}^{z_0}\mathcal{X}^{x_0} ) 
\end{align*}

So that Weyl operators commute with each other up to the symplectic form:

$$
{\mathcal W}(z,x) {\mathcal W}(z',x') = \chi(\omega(z,x),(z',x')) {\mathcal W}(z',x') {\mathcal W}(z,x) $$

Moreover, we can combine Weyl operators together as follows:

\begin{align*}
{\mathcal W}(z,x) {\mathcal W}(z',x')
&=\chi(\omega((z,x),(z',x'))/2){\mathcal W}(z+z',x+x') 
\end{align*}


Therefore:
\begin{align*}
&C\mathcal{W}(z,x)\mathcal{W}(z',x')C^\dag\\
&=C(\mathcal{W}(z,x) C^\dag C \mathcal{W}(z',x'))C^\dag\\
&=\chi(C_a(z,x)+C_a(z',x')) \mathcal{W}(C_L(z,x))\mathcal{W}(C_L(z',x')) \\
&=\chi(C_a(z,x)+C_a(z',x')+\omega(C_L(z,x)),C_L(z',x'))/2)  \mathcal{W}(C_L(z,x)+C_L(z',x')) \\
\end{align*}

Similarly:

\begin{align*}
&C\mathcal{W}(z,x)\mathcal{W}(z',x')C^\dag\\
&= \chi(\omega((z,x),(z',x'))/2) C \mathcal{W}(z+z',x+x') C^\dag\\
&= \chi(\omega((z,x),(z',x'))/2+C_a(z+z',x+x')) \mathcal{W}(C_{L}(z+z',x+x'))\\
&= \chi(\omega((z,x),(z',x'))/2+C_a(z,x)+ C_a(z',x')) \mathcal{W}(C_L(z,x)+C_L(z',x'))\\
\end{align*}


So that $\omega(C_L(z,x),C_L(z',x')) = \omega((z,x),(z',x'))$; meaning that $C_L$ is a symplectomorphism. Moreover for a Clifford operator $D$:

\begin{align*}
DC {\mathcal W}(z,x) C^\dag D^\dag
&=D\chi(C_a(z,x)){\mathcal W}(C_L(z,x))D^\dag\\
&=\chi(C_a(z,x)+D_a(C_L(z,x))){\mathcal W}(D_L(C_L(z,x)))\\
\end{align*}

So that $C_L$ and $C_a$ determine an affine transformation.

\end{proof}


The reason this fails for qubits is because one can not represent all Weyl operators in the following way:

$$
\chi(C_a(z,x)){\mathcal W}(z,x)
$$

 For example, the qubit phase shift gate $\mathcal{S}\mathcal{X} \mathcal{S}^{\dag} = i \mathcal{Z}\mathcal{X}$; however $i=e^{2\pi \cdot i/4}$, so there is no value of $a\in \F_2$ for which $\chi(a)=e^{2\pi\cdot i\cdot a/2} =e^{\pi\cdot i a} = i$.


%
%\begin{proof}
%As a matter of notation let
%$$
%\chi(a) = e^{2\pi\cdot i \cdot a/p}, \hspace*{1cm} {\mathcal W}(z,x)=\chi(-zx^T/2)\bigotimes_{j=0}^{n-1}\mathcal{Z}_{(j)}^{z_j}\mathcal{X}_{(j)}^{x_j}
%$$
%Where the subscript $(j)$ denotes that we are applying a gate on wire $(j)$ tensored on both sides with identities.
%
%
%We know that the Clifford group is defined as the normalizer of the Pauli group so that a Clifford operator is defined by its action on Weyl operators. Given an $n$ qudit Clifford operator $C$  and $(z,x)\in \F_p^{2n}$, there exists  an isomorphism $C_L:\F_p^{2n}\to \F_p^{2n}$ and a vector  $C_a\in \F_p^{2n}$ such that:
%
%$$
%{\mathcal W}(z,x) C {\mathcal W}(z,x)^\dag
%=
%\chi(C_a(z,x)){\mathcal W}(C_L(z,x))
%$$
%
%Note that for qubits the phase is not $\chi(C_a(z,x))$, so this is not true.  For example, the qubit phase shift gate $\mathcal{X}\mathcal{S}\mathcal{X} = i \cdot I_2$; however $i=e^{2\pi \cdot i/4}$, so there is no value of $a\in \F_2$ for which $\chi(a)=e^{2\pi\cdot i\cdot a/2} =e^{\pi\cdot i a} = i$.
%
%We seek to show that $C_L$ is the the symplectomorphism and $C_a$ is the affine shift.  $C_L$ is clearly linear.  To see that it is a symplectomorphism, first observe:
%
%$$
%\mathcal{Z}\mathcal{X}  = \chi(1)\mathcal{X} \mathcal{Z}
%$$
%
%Therefore, 
%
%
%\begin{align*}
%(\mathcal{Z}^{z_0}\mathcal{X}^{x_0})(\mathcal{Z}^{z_1}\mathcal{X}^{x_1} )
%&=\chi(-x_0z_1) \mathcal{Z}^{z_0}\mathcal{Z}^{z_1}\mathcal{X}^{x_0}\mathcal{X}^{x_1}  \\
%&=\chi(-x_0z_1)\mathcal{Z}^{z_1}  \mathcal{Z}^{z_0}\mathcal{X}^{x_1} \mathcal{X}^{x_0}  \\
%&=\chi(z_0x_1-x_0z_1)(\mathcal{Z}^{z_1} \mathcal{X}^{x_1})(  \mathcal{Z}^{z_0}\mathcal{X}^{x_0} ) \\
%&=\chi(\omega((z_0,x_0),(z_1,x_1)))(\mathcal{Z}^{z_1} \mathcal{X}^{x_1})(  \mathcal{Z}^{z_0}\mathcal{X}^{x_0} ) 
%\end{align*}
%
%So that:
%
%$$
%{\mathcal W}(z,x) {\mathcal W}(z',x') = \chi(\omega(z,x),(z',x')) {\mathcal W}(z',x') {\mathcal W}(z,x) $$
%
%
%Take $P=\mathcal{W}(z,x)$ and  $P'=\mathcal{W}(z',x')$.
%There are two ways to compute  $(P'P)C(P'P)^\dag$;  we could compute the conjugation of $C$ by $P'P$ via $C_a$ and $C_L$; or we could first compute the conjugation of $C$ by $P$ and then conjugate by $P'$.  In either case, to obtain a Weyl operator in normal form (so that tensors of  powers of $\cal Z$ are followed by tensors of powers of $\cal X$), we must commute the Weyl operators past each other.  This means that:
%
%$$
%\omega(C_L(z,x),C_L(z',x')) = C_L(\omega((z_0,x_0),(z_0',x_0')),\cdots, \omega((z_{n-1},x_{n-1}),(z_{n-1}',x_{n-1}')) )
%$$
%
%So that $C_L$ is a symplectomorphism.   Given another Clifford operator $D$:
%$$
%\chi(C_a(z,x)){\cal W}(C_L(z,x)) D {\cal W}(C_L(z,x))^\dag
%=
% \chi(C_a(z,x)+D_a(C_L(z,x)))    {\mathcal W}( D_L(C_L(z,x)))
%$$
%
%So that  $C_L$ and $C_a$ determine an affine transformation $(z,x) \mapsto C_L(z,x)+C_a$ with respect to composition of Clifford operators.
%Indeed, the affine symplectormorphisms $\F_p^{2n}\to \F_p^{2n}$ are the only functions which determine Clifford operators in this way.
%\end{proof}



Recall that, up to scalars, the $n$-quopit Clifford group is generated by the Pauli $\mathcal X$-gate, the $\mathcal{C}_\mathcal{X}$ gate, the Fourier transform $\mathcal F$ and the phase gate $\mathcal S$ and scaling gates $\mathcal{M}_a$ for all $a \in \F_p^*$.
In the odd prime case, these correspond to the affine symplectormorphisms:

$$
\begin{tikzpicture}
	\begin{pgfonlayer}{nodelayer}
		\node [style=none] (295) at (232.5, 0) {};
		\node [style=none] (296) at (232.5, 1) {};
		\node [style=none] (297) at (233, 1) {};
		\node [style=none] (298) at (233, 0) {};
		\node [style=X] (299) at (233, 0.5) {$1$};
	\end{pgfonlayer}
	\begin{pgfonlayer}{edgelayer}
		\draw (298.center) to (299);
		\draw (299) to (297.center);
		\draw (296.center) to (295.center);
	\end{pgfonlayer}
\end{tikzpicture}
 \leftrightarrow \mathcal{X},\hspace*{.2cm}
\begin{tikzpicture}[yscale=-1]
	\begin{pgfonlayer}{nodelayer}
		\node [style=none] (300) at (234.5, 1) {};
		\node [style=none] (301) at (234, -0.25) {};
		\node [style=none] (302) at (234.5, -0.25) {};
		\node [style=none] (303) at (234, 1) {};
		\node [style=none] (304) at (234, 0.5) {};
		\node [style=none] (305) at (234.5, 0.5) {};
		\node [style=s] (306) at (234.5, 0.25) {};
	\end{pgfonlayer}
	\begin{pgfonlayer}{edgelayer}
		\draw (301.center) to (304.center);
		\draw [in=-90, out=90] (304.center) to (300.center);
		\draw [in=-270, out=-90] (303.center) to (305.center);
		\draw (302.center) to (305.center);
	\end{pgfonlayer}
\end{tikzpicture}
 \leftrightarrow {\mathcal F},\hspace*{.2cm}
\begin{tikzpicture}
	\begin{pgfonlayer}{nodelayer}
		\node [style=none] (307) at (235.5, 0) {};
		\node [style=none] (308) at (235.5, 1.25) {};
		\node [style=none] (309) at (236, 1.25) {};
		\node [style=none] (310) at (236, 0) {};
		\node [style=Z] (311) at (235.5, 0.5) {};
		\node [style=X] (312) at (236, 0.75) {};
	\end{pgfonlayer}
	\begin{pgfonlayer}{edgelayer}
		\draw (308.center) to (311);
		\draw (311) to (307.center);
		\draw (311) to (312);
		\draw (309.center) to (312);
		\draw (312) to (310.center);
	\end{pgfonlayer}
\end{tikzpicture}
 \leftrightarrow \ {\mathcal S},\hspace*{.2cm}
\begin{tikzpicture}
	\begin{pgfonlayer}{nodelayer}
		\node [style=X] (313) at (237, 1) {};
		\node [style=Z] (314) at (237.5, 1.25) {};
		\node [style=X] (315) at (238.75, 1.25) {};
		\node [style=Z] (316) at (238.25, 1) {};
		\node [style=none] (317) at (237, 0.5) {};
		\node [style=none] (318) at (237.5, 0.5) {};
		\node [style=none] (319) at (238.25, 0.5) {};
		\node [style=none] (320) at (238.75, 0.5) {};
		\node [style=none] (321) at (238.25, 1.75) {};
		\node [style=none] (322) at (238.75, 1.75) {};
		\node [style=none] (323) at (237, 1.75) {};
		\node [style=none] (324) at (237.5, 1.75) {};
	\end{pgfonlayer}
	\begin{pgfonlayer}{edgelayer}
		\draw (317.center) to (313);
		\draw (313) to (323.center);
		\draw (324.center) to (314);
		\draw (314) to (318.center);
		\draw (313) to (314);
		\draw (321.center) to (316);
		\draw (319.center) to (316);
		\draw (316) to (315);
		\draw (322.center) to (315);
		\draw (315) to (320.center);
	\end{pgfonlayer}
\end{tikzpicture}
 \leftrightarrow \mathcal{C}_{\mathcal X},\hspace*{.2cm}
\begin{tikzpicture}
	\begin{pgfonlayer}{nodelayer}
		\node [style=none] (329) at (239.75, 0.25) {};
		\node [style=none] (330) at (240.5, 0.25) {};
		\node [style=none] (335) at (239.75, 1.75) {};
		\node [style=none] (336) at (240.5, 1.75) {};
		\node [style=scalar] (337) at (240.5, 1) {$a$};
		\node [style=scalarop] (338) at (239.75, 1) {$a$};
	\end{pgfonlayer}
	\begin{pgfonlayer}{edgelayer}
		\draw (330.center) to (337);
		\draw (337) to (336.center);
		\draw (335.center) to (338);
		\draw (329.center) to (338);
	\end{pgfonlayer}
\end{tikzpicture}
 \leftrightarrow \mathcal{M}_{a}
$$


The following result gets us even closer to where we need to be:

\begin{lemma}\cite[Lem. 8]{gross}
For every odd prime $p$ and $n \in \N$, there is a bijection between (nonempty) affine Lagrangian subspaces of $\F_p^{2n}$ and $n$-quopit stabilizer states modulo nonzero scalars.
\end{lemma}


\begin{proof}
Given any affine Lagrangian subspace $L+a\subseteq \F_p^{2n}$; then up to global phase there is a stabilizer state $C | 0\rangle^{\otimes n}$ determined by the rank 1 projector:
$$
C | 0\rangle^{\otimes n} \langle 0| ^{\otimes n}C^{\dag}:=\dfrac{1}{p^n}\sum_{v \in L}\chi(\omega(a,v)) {\cal W}(v)
$$
as for any $v' \in L$:
\begin{align*}
\chi(\omega(a,v')) {\cal W}(v') &\dfrac{1}{p^n}\sum_{v \in L}\chi(\omega(a,v)) {\cal W}(v) \chi(-\omega(a,v')) {\cal W}(v')^\dag\\
=&
 \dfrac{1}{p^n}\sum_{v \in L}\chi(\omega(a,v)+\omega(v,v')) {\cal W}(v)
{\cal W}(v') {\cal W}(v')^\dag\\
=&
 \dfrac{1}{p^n}\sum_{v \in L}\chi(\omega(a,v) ){\cal W}(v)
{\cal W}(v') {\cal W}(v')^\dag\\
=&
 \dfrac{1}{p^n}\sum_{v \in L}\chi(\omega(a,v)) {\cal W}(v)
\end{align*}

Moreover, every stabilizer state is of this form.  Recall that stabilizer groups are Abelian.
If two stabilizers $\chi(a){\cal W}(u)$ and $\chi(b){\cal W}(v)$ stabilize the same state, they must commute so that $\omega(u,v)=0$.
Moreover a stabilizer state is stabilized by exactly $p^n$ stabilizers,  making the space of stabilizers into an affine Lagrangian subspace of $\F_p$. 
\end{proof}

In other words, a symplectic basis for the affine Lagrangian subspace over $\F_p$ corresponds to the {\bf stabilizer tableau} for a pure quopit stabilizer state (ie a stabilizer tableau on $n$ qudits with dimension $n$).

Explicitly, the state $|0\rangle$ is identified with the following Afffine Lagrangian subspace:

$$
\begin{tikzpicture}
	\begin{pgfonlayer}{nodelayer}
		\node [style=X] (389) at (205, -2) {};
		\node [style=none] (390) at (205, -1.5) {};
		\node [style=Z] (391) at (204.5, -2) {};
		\node [style=none] (392) at (204.5, -1.5) {};
	\end{pgfonlayer}
	\begin{pgfonlayer}{edgelayer}
		\draw (389) to (390.center);
		\draw (391) to (392.center);
	\end{pgfonlayer}
\end{tikzpicture}
 \leftrightarrow |0\rangle
$$


We seek to recast these results in a more compositional light; to prove that for odd prime $p$, $\Aff\Lag\Rel_{\F_p}$ is isomorphic to quopit stabilizer circuits, modulo scalars.  Let us give the latter category a name:


\begin{definition}
Let $\Stab_p$ denote the prop of quopit stabilizer circuits modulo nonzero scalars, regarded as a \dag-compact closed category with respect to the standard basis.
\end{definition}


%The following isomorphism is described in \cite{gross}, when restricted to the nonempty case.  This comes from the projective representation of the quopit Clifford group in terms of the affine symplectomorphisms over $\F_p^n$.  However, since there is only one empty relation and one zero matrix of every type, we get the following result immediately:



We extend this isomorphism of states to an isomorphism of props using a symplectic notion of Weyl groups and stabilizers:


\begin{definition}
Given a field $k$  the  $1$-fold symplectic Weyl operators are the symplectomorphisms of the following form,  for $a,b \in k$:


$$
W(a,b):=
\begin{tikzpicture}
	\begin{pgfonlayer}{nodelayer}
		\node [style=X] (0) at (0, 1) {$a$};
		\node [style=none] (1) at (0, 2) {};
		\node [style=none] (2) at (0, 0) {};
		\node [style=none] (3) at (1, 2) {};
		\node [style=none] (4) at (1, 0) {};
		\node [style=X] (5) at (1, 1) {$b$};
	\end{pgfonlayer}
	\begin{pgfonlayer}{edgelayer}
		\draw (4.center) to (5.center);
		\draw (5.center) to (3.center);
		\draw (1.center) to (0.center);
		\draw (0.center) to (2.center);
	\end{pgfonlayer}
\end{tikzpicture}
$$

Given a tuple $$((z_1,\cdots, z_n),(x_1,\cdots, x_n)) \in k^{2n}$$ define the $n$-fold  symplectic Weyl operator $$W((z_1,\cdots, z_n),(x_1,\cdots, x_n) )= W(z_1,x_1)+ \cdots + W(z_n,x_n)$$




The $n$-fold symplectic Weyl operators form the $n$-fold {\bf symplectic Weyl group}, $P_k^n$ under composition.  And altogether, they form a prop under tensor product and composition. 

Given some affine Lagrangian subspace $f$ of $\F_k^{2n}$, the {\bf symplectic stabilizer group} of $f$ is the subgroup  $S\subseteq P_k^n$ so that for all $a\in S$, $f; a =f$.
\end{definition}

Notice that unlike the qudit Pauli group, there is no phase-factor.

\begin{lemma}
Two states in $\Aff\Lag\Rel_k$ are equal if and only if they have the same symplectic stabilizer group.
\end{lemma}

\begin{proof}
Take a state $f:0\to n$ in $\Aff\Lag\Rel_k$.  Then $W(z,x)$ is a stabilizer of $f$ if and only if  $(z,x) \in f$; therefore two subspaces are equal if and only if they have the same elements if and only if they have the same stabilizers.
\end{proof}




\begin{theorem}
\label{theorem:spekkens}
When $p$ is an odd prime, the mapping $\Lag\Rel_{\F_p} \to \Stab_p$ defined by:
$$
\begin{tikzpicture}
	\begin{pgfonlayer}{nodelayer}
		\node [style=map] (21) at (2, -2) {$f$};
		\node [style=none] (22) at (1.75, -1.25) {};
		\node [style=none] (23) at (2.25, -1.25) {};
		\node [style=none] (24) at (1.75, -2.75) {};
		\node [style=none] (25) at (2.25, -2.75) {};
	\end{pgfonlayer}
	\begin{pgfonlayer}{edgelayer}
		\draw [in=-90, out=120] (21) to (22.center);
		\draw [in=90, out=-120] (21) to (24.center);
		\draw [in=-60, out=90] (25.center) to (21);
		\draw [in=-90, out=60] (21) to (23.center);
	\end{pgfonlayer}
\end{tikzpicture}
\mapsto
\begin{cases}
{\bf 0} & \text{if $f=\emptyset$}\\
\begin{tikzpicture}
	\begin{pgfonlayer}{nodelayer}
		\node [style=map] (7) at (12, 0) {$G\left(\lfloor f\rfloor\right)$};
		\node [style=none] (8) at (11.25, 1.5) {};
		\node [style=Z] (10) at (13, 1) {};
		\node [style=none] (12) at (13.5, -0.5) {};
	\end{pgfonlayer}
	\begin{pgfonlayer}{edgelayer}
		\draw [in=135, out=-90] (8.center) to (7);
		\draw [in=-150, out=60, looseness=0.75] (7) to (10);
		\draw [in=-45, out=90] (12.center) to (10);
	\end{pgfonlayer}
\end{tikzpicture} & \text{otherwise}
\end{cases}
$$
is a symmetric monoidal isomorphism, where $G$ is the bijection  between affine Lagrangian subspaces and stabilizer states modulo scalars. And $\bf 0$ is the unique stabilizer circuit of appropriate dimension which is multiplied by the scalar 0.
\end{theorem}

%
%%See Appendix \ref{proof:theorem:spekkens} for the proof.  
%The main difficulty in proving this is to show that $H$ is a functor.  This is shown by observing that stabilizer states with the same stabilizer group only differ by a global scalar.


\begin{proof}
We already know that there is a bijection between the states of both of these props. Because these props are both compact closed, it only remains to show that this isomorphism is monoidal and functorial.  It clearly is monoidal and preserves the identity; the nontrivial part is to show that it preserves composition.

Consider some composable pair in $\Aff\Lag\Rel_{\F_p}$:
$$
\F_p^n \xrightarrow{f} \F_p^m \xrightarrow{g} \F_p^\ell
$$
If the composite is empty, then the result follows immediately.  Suppose otherwise.


First, observe that in $\Stab_p$

$$
\begin{tikzpicture}
	\begin{pgfonlayer}{nodelayer}
		\node [style=Z] (111) at (197.5, -2) {};
		\node [style=none] (112) at (196.75, -2.75) {};
		\node [style=none] (113) at (198.25, -2.75) {};
		\node [style=none] (114) at (196.75, -3.5) {};
		\node [style=none] (115) at (198.25, -3.5) {};
		\node [style=map] (116) at (196.75, -2.75) {${\mathcal W}(a,b)$};
	\end{pgfonlayer}
	\begin{pgfonlayer}{edgelayer}
		\draw (115.center) to (113.center);
		\draw [in=-15, out=90] (113.center) to (111);
		\draw [in=90, out=-165] (111) to (112.center);
		\draw (112.center) to (114.center);
	\end{pgfonlayer}
\end{tikzpicture}
=
\begin{tikzpicture}
	\begin{pgfonlayer}{nodelayer}
		\node [style=Z] (117) at (200, -2) {};
		\node [style=none] (118) at (199.25, -2.75) {};
		\node [style=none] (119) at (200.75, -2.75) {};
		\node [style=none] (120) at (199.25, -3.5) {};
		\node [style=none] (121) at (200.75, -3.5) {};
		\node [style=map] (122) at (199.25, -2.75) {${\cal Z}^a{\cal X}^b$};
	\end{pgfonlayer}
	\begin{pgfonlayer}{edgelayer}
		\draw (121.center) to (119.center);
		\draw [in=-15, out=90] (119.center) to (117);
		\draw [in=90, out=-165] (117) to (118.center);
		\draw (118.center) to (120.center);
	\end{pgfonlayer}
\end{tikzpicture}
=
\begin{tikzpicture}
	\begin{pgfonlayer}{nodelayer}
		\node [style=Z] (123) at (202.5, -2) {};
		\node [style=none] (124) at (203.25, -2.75) {};
		\node [style=none] (125) at (201.75, -2.75) {};
		\node [style=none] (126) at (203.25, -3.5) {};
		\node [style=none] (127) at (201.75, -3.5) {};
		\node [style=map] (128) at (203.25, -2.75) {${\cal Z}^a{\cal X}^{-b}$};
	\end{pgfonlayer}
	\begin{pgfonlayer}{edgelayer}
		\draw (127.center) to (125.center);
		\draw [in=-165, out=90] (125.center) to (123);
		\draw [in=90, out=-15] (123) to (124.center);
		\draw (124.center) to (126.center);
	\end{pgfonlayer}
\end{tikzpicture}
=
\begin{tikzpicture}
	\begin{pgfonlayer}{nodelayer}
		\node [style=Z] (129) at (205, -2) {};
		\node [style=none] (130) at (205.75, -2.75) {};
		\node [style=none] (131) at (204.25, -2.75) {};
		\node [style=none] (132) at (205.75, -3.5) {};
		\node [style=none] (133) at (204.25, -3.5) {};
		\node [style=map] (134) at (205.75, -2.75) {$\mathcal W(a,-b)$};
	\end{pgfonlayer}
	\begin{pgfonlayer}{edgelayer}
		\draw (133.center) to (131.center);
		\draw [in=-165, out=90] (131.center) to (129);
		\draw [in=90, out=-15] (129) to (130.center);
		\draw (130.center) to (132.center);
	\end{pgfonlayer}
\end{tikzpicture}
$$


Moreover, in $\Aff\Lag\Rel_{\F_p}$

\begin{align*}
\begin{tikzpicture}
	\begin{pgfonlayer}{nodelayer}
		\node [style=X] (271) at (226.75, -6) {};
		\node [style=Z] (272) at (227.75, -6) {};
		\node [style=none] (275) at (227.75, -7) {};
		\node [style=none] (276) at (228.5, -7) {};
		\node [style=none] (277) at (226, -8) {};
		\node [style=none] (278) at (226.75, -8) {};
		\node [style=none] (279) at (227.75, -8) {};
		\node [style=none] (280) at (228.5, -8) {};
		\node [style=none] (281) at (226, -7) {};
		\node [style=none] (282) at (226.75, -7) {};
		\node [style=map] (283) at (226.425, -7.25) {$W(a,b)$};
	\end{pgfonlayer}
	\begin{pgfonlayer}{edgelayer}
		\draw [in=90, out=-15] (271) to (275.center);
		\draw [in=-15, out=90] (276.center) to (272);
		\draw [in=-165, out=90] (282.center) to (272);
		\draw [in=90, out=-165] (271) to (281.center);
		\draw (281.center) to (277.center);
		\draw (278.center) to (282.center);
		\draw (275.center) to (279.center);
		\draw (280.center) to (276.center);
	\end{pgfonlayer}
\end{tikzpicture}
&=
\begin{tikzpicture}
	\begin{pgfonlayer}{nodelayer}
		\node [style=X] (135) at (198, -6) {};
		\node [style=Z] (136) at (199, -6) {};
		\node [style=X] (137) at (197.25, -7) {$a$};
		\node [style=X] (138) at (198, -7) {$b$};
		\node [style=none] (139) at (199, -7) {};
		\node [style=none] (140) at (199.75, -7) {};
		\node [style=none] (141) at (197.25, -8) {};
		\node [style=none] (142) at (198, -8) {};
		\node [style=none] (143) at (199, -8) {};
		\node [style=none] (144) at (199.75, -8) {};
	\end{pgfonlayer}
	\begin{pgfonlayer}{edgelayer}
		\draw (141.center) to (137);
		\draw [in=-165, out=90] (137) to (135);
		\draw [in=-165, out=90] (138) to (136);
		\draw (142.center) to (138);
		\draw [in=90, out=-15] (135) to (139.center);
		\draw (139.center) to (143.center);
		\draw (144.center) to (140.center);
		\draw [in=-15, out=90] (140.center) to (136);
	\end{pgfonlayer}
\end{tikzpicture}
=
\begin{tikzpicture}
	\begin{pgfonlayer}{nodelayer}
		\node [style=X] (145) at (201.25, -6) {};
		\node [style=Z] (146) at (202.25, -6) {};
		\node [style=X] (148) at (201.25, -7) {$b$};
		\node [style=none] (149) at (202.25, -7) {};
		\node [style=none] (150) at (203, -7) {};
		\node [style=none] (151) at (200.5, -8) {};
		\node [style=none] (152) at (201.25, -8) {};
		\node [style=none] (153) at (202.25, -8) {};
		\node [style=none] (154) at (203, -8) {};
		\node [style=X] (155) at (202.25, -7) {$a$};
		\node [style=none] (156) at (200.5, -7) {};
	\end{pgfonlayer}
	\begin{pgfonlayer}{edgelayer}
		\draw [in=-165, out=90] (148) to (146);
		\draw (152.center) to (148);
		\draw [in=90, out=-15] (145) to (149.center);
		\draw (149.center) to (153.center);
		\draw (154.center) to (150.center);
		\draw [in=-15, out=90] (150.center) to (146);
		\draw (156.center) to (151.center);
		\draw [in=-165, out=90] (156.center) to (145);
	\end{pgfonlayer}
\end{tikzpicture}
=
\begin{tikzpicture}
	\begin{pgfonlayer}{nodelayer}
		\node [style=X] (157) at (204.75, -6) {};
		\node [style=X] (159) at (204.75, -7) {$b$};
		\node [style=none] (160) at (205.75, -7) {};
		\node [style=none] (161) at (206.5, -7) {};
		\node [style=none] (162) at (204, -8) {};
		\node [style=none] (163) at (204.75, -8) {};
		\node [style=none] (164) at (205.75, -8) {};
		\node [style=none] (165) at (206.5, -8) {};
		\node [style=X] (166) at (205.75, -7) {$a$};
		\node [style=none] (167) at (204, -7) {};
		\node [style=X] (168) at (205.75, -6) {};
		\node [style=s] (169) at (206.5, -7) {};
	\end{pgfonlayer}
	\begin{pgfonlayer}{edgelayer}
		\draw (163.center) to (159);
		\draw [in=90, out=-15] (157) to (160.center);
		\draw (160.center) to (164.center);
		\draw (165.center) to (161.center);
		\draw (167.center) to (162.center);
		\draw [in=-165, out=90] (167.center) to (157);
		\draw [in=-165, out=90] (159) to (168);
		\draw [in=90, out=-15] (168) to (169);
	\end{pgfonlayer}
\end{tikzpicture}
=
\begin{tikzpicture}
	\begin{pgfonlayer}{nodelayer}
		\node [style=X] (170) at (208.25, -6) {};
		\node [style=none] (172) at (209.25, -7) {};
		\node [style=none] (173) at (210, -7) {};
		\node [style=none] (174) at (207.5, -8) {};
		\node [style=none] (175) at (208.25, -8) {};
		\node [style=none] (176) at (209.25, -8) {};
		\node [style=none] (177) at (210, -8) {};
		\node [style=X] (178) at (209.25, -7) {$a$};
		\node [style=none] (179) at (207.5, -7) {};
		\node [style=X] (180) at (209.25, -6) {};
		\node [style=none] (182) at (208.25, -7) {};
		\node [style=s] (183) at (210, -7.5) {};
		\node [style=X] (184) at (210, -7) {$b$};
	\end{pgfonlayer}
	\begin{pgfonlayer}{edgelayer}
		\draw [in=90, out=-15] (170) to (172.center);
		\draw (172.center) to (176.center);
		\draw (177.center) to (173.center);
		\draw (179.center) to (174.center);
		\draw [in=-165, out=90] (179.center) to (170);
		\draw (175.center) to (182.center);
		\draw [in=-165, out=90] (182.center) to (180);
		\draw [in=-15, out=90] (184) to (180);
	\end{pgfonlayer}
\end{tikzpicture}\\
&=
\begin{tikzpicture}
	\begin{pgfonlayer}{nodelayer}
		\node [style=X] (220) at (211.75, -6) {};
		\node [style=none] (221) at (212.75, -7) {};
		\node [style=none] (222) at (213.5, -7) {};
		\node [style=none] (223) at (211, -8) {};
		\node [style=none] (224) at (211.75, -8) {};
		\node [style=none] (225) at (212.75, -8) {};
		\node [style=none] (226) at (213.5, -8) {};
		\node [style=X] (227) at (212.75, -7) {$a$};
		\node [style=none] (228) at (211, -7) {};
		\node [style=X] (229) at (212.75, -6) {};
		\node [style=none] (230) at (211.75, -7) {};
		\node [style=s] (231) at (213.5, -6.8) {};
		\node [style=X] (232) at (213.5, -7.5) {$-b$};
	\end{pgfonlayer}
	\begin{pgfonlayer}{edgelayer}
		\draw [in=90, out=-15] (220) to (221.center);
		\draw (221.center) to (225.center);
		\draw (226.center) to (222.center);
		\draw (228.center) to (223.center);
		\draw [in=-165, out=90] (228.center) to (220);
		\draw (224.center) to (230.center);
		\draw [in=-165, out=90] (230.center) to (229);
		\draw [in=-15, out=90] (222.center) to (229);
	\end{pgfonlayer}
\end{tikzpicture}
=
\begin{tikzpicture}
	\begin{pgfonlayer}{nodelayer}
		\node [style=X] (233) at (215.25, -6) {};
		\node [style=none] (234) at (216.25, -7) {};
		\node [style=none] (235) at (217, -7) {};
		\node [style=none] (236) at (214.5, -8) {};
		\node [style=none] (237) at (215.25, -8) {};
		\node [style=none] (238) at (216.25, -8) {};
		\node [style=none] (239) at (217, -8) {};
		\node [style=X] (240) at (216.25, -7) {$a$};
		\node [style=none] (241) at (214.5, -7) {};
		\node [style=none] (243) at (215.25, -7) {};
		\node [style=X] (245) at (217, -7) {$-b$};
		\node [style=Z] (246) at (216.25, -6) {};
	\end{pgfonlayer}
	\begin{pgfonlayer}{edgelayer}
		\draw [in=90, out=-15] (233) to (234.center);
		\draw (234.center) to (238.center);
		\draw (239.center) to (235.center);
		\draw (241.center) to (236.center);
		\draw [in=-165, out=90] (241.center) to (233);
		\draw (237.center) to (243.center);
		\draw [in=-165, out=90] (243.center) to (246);
		\draw [in=90, out=-15] (246) to (235.center);
	\end{pgfonlayer}
\end{tikzpicture}
=
\begin{tikzpicture}
	\begin{pgfonlayer}{nodelayer}
		\node [style=X] (284) at (230.25, -6) {};
		\node [style=Z] (285) at (231.25, -6) {};
		\node [style=none] (286) at (231.25, -7) {};
		\node [style=none] (287) at (232, -7) {};
		\node [style=none] (288) at (229.5, -8) {};
		\node [style=none] (289) at (230.25, -8) {};
		\node [style=none] (290) at (231.25, -8) {};
		\node [style=none] (291) at (232, -8) {};
		\node [style=none] (292) at (229.5, -7) {};
		\node [style=none] (293) at (230.25, -7) {};
		\node [style=map] (294) at (231.675, -7.25) {$W(a,-b)$};
	\end{pgfonlayer}
	\begin{pgfonlayer}{edgelayer}
		\draw [in=90, out=-15] (284) to (286.center);
		\draw [in=-15, out=90] (287.center) to (285);
		\draw [in=-165, out=90] (293.center) to (285);
		\draw [in=90, out=-165] (284) to (292.center);
		\draw (292.center) to (288.center);
		\draw (289.center) to (293.center);
		\draw (286.center) to (290.center);
		\draw (291.center) to (287.center);
	\end{pgfonlayer}
\end{tikzpicture}
\end{align*}

Therefore, the symplectic Weyl operators commute with the symplectic $Z$ spider in $\Aff\Lag\Rel_{\F_p}$ in the same way that the Weyl operators commute with the $Z$ spiders in $\Stab_p$.
%We also know that:
%$$
%\begin{tikzpicture}
%	\begin{pgfonlayer}{nodelayer}
%		\node [style=map] (189) at (103, 0) {$\hat {f;g}$};
%		\node [style=none] (190) at (102.25, 1.5) {};
%		\node [style=none] (191) at (103.25, 1.5) {};
%		\node [style=none] (192) at (102.75, 1.5) {};
%		\node [style=none] (193) at (103.75, 1.5) {};
%	\end{pgfonlayer}
%	\begin{pgfonlayer}{edgelayer}
%		\draw [in=135, out=-90] (190.center) to (189);
%		\draw [in=-90, out=75] (189) to (191.center);
%		\draw [in=-90, out=45] (189) to (193.center);
%		\draw [in=105, out=-90] (192.center) to (189);
%	\end{pgfonlayer}
%\end{tikzpicture}
%=
%\begin{tikzpicture}
%	\begin{pgfonlayer}{nodelayer}
%		\node [style=map] (163) at (96.5, -2.25) {$f;g$};
%		\node [style=none] (164) at (96.75, -1.5) {};
%		\node [style=none] (165) at (95.75, -1.5) {};
%		\node [style=Z] (166) at (96.25, -3) {};
%		\node [style=X] (167) at (95.75, -3) {};
%		\node [style=none] (168) at (95.75, -2.25) {};
%		\node [style=none] (169) at (95.25, -2.25) {};
%		\node [style=none] (170) at (96.25, -1.5) {};
%		\node [style=none] (171) at (95.25, -1.5) {};
%	\end{pgfonlayer}
%	\begin{pgfonlayer}{edgelayer}
%		\draw [in=-90, out=60] (163) to (164.center);
%		\draw [in=-90, out=120] (163) to (165.center);
%		\draw [in=-120, out=30] (167) to (163);
%		\draw [in=30, out=-60, looseness=1.25] (163) to (166);
%		\draw [in=-90, out=150] (166) to (168.center);
%		\draw [in=-90, out=135] (167) to (169.center);
%		\draw (169.center) to (171.center);
%		\draw [in=-90, out=90] (168.center) to (170.center);
%	\end{pgfonlayer}
%\end{tikzpicture}
%=
%\begin{tikzpicture}
%	\begin{pgfonlayer}{nodelayer}
%		\node [style=map] (211) at (110.5, -2.25) {$g$};
%		\node [style=none] (212) at (110.75, -1.5) {};
%		\node [style=none] (213) at (109.75, -1.5) {};
%		\node [style=Z] (214) at (110.25, -3) {};
%		\node [style=X] (215) at (109.75, -3) {};
%		\node [style=none] (216) at (109.75, -2.25) {};
%		\node [style=none] (217) at (109.25, -2.25) {};
%		\node [style=none] (218) at (110.25, -1.5) {};
%		\node [style=none] (219) at (109.25, -1.5) {};
%		\node [style=map] (220) at (112.5, -2.25) {$f$};
%		\node [style=none] (221) at (112.75, -1.5) {};
%		\node [style=none] (222) at (111.75, -1.5) {};
%		\node [style=Z] (223) at (112.25, -3) {};
%		\node [style=X] (224) at (111.75, -3) {};
%		\node [style=none] (225) at (111.75, -2.25) {};
%		\node [style=none] (226) at (111.25, -2.25) {};
%		\node [style=none] (227) at (112.25, -1.5) {};
%		\node [style=none] (228) at (111.25, -1.5) {};
%		\node [style=Z] (229) at (111.5, -0.5) {};
%		\node [style=X] (230) at (110.5, -0.5) {};
%		\node [style=none] (231) at (110.5, 0.25) {};
%		\node [style=none] (232) at (111.5, 0.25) {};
%		\node [style=none] (233) at (109.25, 0.25) {};
%		\node [style=none] (234) at (112.75, 0.25) {};
%	\end{pgfonlayer}
%	\begin{pgfonlayer}{edgelayer}
%		\draw [in=-90, out=60] (211) to (212.center);
%		\draw [in=-90, out=120] (211) to (213.center);
%		\draw [in=-120, out=30] (215) to (211);
%		\draw [in=30, out=-60, looseness=1.25] (211) to (214);
%		\draw [in=-90, out=150] (214) to (216.center);
%		\draw [in=-90, out=135] (215) to (217.center);
%		\draw (217.center) to (219.center);
%		\draw [in=-90, out=90] (216.center) to (218.center);
%		\draw [in=-90, out=60] (220) to (221.center);
%		\draw [in=-90, out=120] (220) to (222.center);
%		\draw [in=-120, out=30] (224) to (220);
%		\draw [in=30, out=-60, looseness=1.25] (220) to (223);
%		\draw [in=-90, out=150] (223) to (225.center);
%		\draw [in=-90, out=135] (224) to (226.center);
%		\draw (226.center) to (228.center);
%		\draw [in=-90, out=90] (225.center) to (227.center);
%		\draw [in=-30, out=90] (227.center) to (229);
%		\draw [in=90, out=-150] (229) to (212.center);
%		\draw [in=-30, out=90] (228.center) to (230);
%		\draw [in=90, out=-150] (230) to (213.center);
%		\draw (219.center) to (233.center);
%		\draw [in=-90, out=90] (218.center) to (232.center);
%		\draw [in=-90, out=90, looseness=0.75] (222.center) to (231.center);
%		\draw (221.center) to (234.center);
%	\end{pgfonlayer}
%\end{tikzpicture}
%=
%\begin{tikzpicture}
%	\begin{pgfonlayer}{nodelayer}
%		\node [style=map] (64) at (28.75, 0) {$\hat g$};
%		\node [style=none] (65) at (28, 1.25) {};
%		\node [style=X] (66) at (29.25, 1) {};
%		\node [style=Z] (67) at (29.75, 1) {};
%		\node [style=none] (68) at (28.75, 1.25) {};
%		\node [style=map] (69) at (30.25, 0) {$\hat f$};
%		\node [style=none] (70) at (30.25, 1.25) {};
%		\node [style=none] (71) at (31, 1.25) {};
%		\node [style=none] (72) at (30.25, 2.5) {};
%		\node [style=none] (73) at (28.75, 2.5) {};
%		\node [style=none] (74) at (31, 2.5) {};
%		\node [style=none] (75) at (28, 2.5) {};
%	\end{pgfonlayer}
%	\begin{pgfonlayer}{edgelayer}
%		\draw [in=135, out=-90] (65.center) to (64);
%		\draw [in=-150, out=45] (64) to (67);
%		\draw [in=-165, out=105, looseness=1.25] (64) to (66);
%		\draw [in=-90, out=60] (64) to (68.center);
%		\draw [in=-30, out=60, looseness=1.25] (69) to (67);
%		\draw [in=135, out=-30] (66) to (69);
%		\draw [in=-90, out=120] (69) to (70.center);
%		\draw [in=45, out=-90] (71.center) to (69);
%		\draw (65.center) to (75.center);
%		\draw [in=270, out=90] (68.center) to (72.center);
%		\draw (71.center) to (74.center);
%		\draw [in=270, out=90] (70.center) to (73.center);
%	\end{pgfonlayer}
%\end{tikzpicture}
%$$
Therefore, the following two states in $\Stab_p$ have the same stabilizers
$$
\begin{tikzpicture}
	\begin{pgfonlayer}{nodelayer}
		\node [style=map] (29) at (23, 0) {$G\left(\lfloor {f;g}\rfloor\right)$};
		\node [style=none] (30) at (22.25, 1.5) {};
		\node [style=none] (33) at (23.75, 1.5) {};
	\end{pgfonlayer}
	\begin{pgfonlayer}{edgelayer}
		\draw [in=135, out=-90] (30.center) to (29);
		\draw [in=-90, out=45] (29) to (33.center);
	\end{pgfonlayer}
\end{tikzpicture}
=
\begin{tikzpicture}
	\begin{pgfonlayer}{nodelayer}
		\node [style=map] (21) at (175.25, -4.5) {$\lfloor g\rfloor$};
		\node [style=none] (22) at (174.5, -3.25) {};
		\node [style=X] (23) at (175.75, -3.5) {};
		\node [style=Z] (24) at (176.25, -3.5) {};
		\node [style=map] (25) at (176.75, -4.5) {$\lfloor f\rfloor$};
		\node [style=none] (26) at (177.5, -3.25) {};
		\node [style=none] (27) at (174.85, -2.5) {};
		\node [style=none] (28) at (177.1, -2.5) {};
		\node [style=none] (29) at (177.1, -2.5) {};
		\node [style=none] (30) at (174.85, -2.5) {};
		\node [style=none] (31) at (174.25, -2) {};
		\node [style=none] (32) at (177.75, -2) {};
		\node [style=none] (33) at (177.75, -5) {};
		\node [style=none] (34) at (174.25, -5) {};
		\node [style=none] (35) at (174.5, -4.75) {$G$};
		\node [style=none] (36) at (177.1, -1.5) {};
		\node [style=none] (37) at (174.85, -1.5) {};
		\node [style=otimes] (38) at (174.85, -2.5) {};
		\node [style=otimes] (39) at (177.1, -2.5) {};
	\end{pgfonlayer}
	\begin{pgfonlayer}{edgelayer}
		\draw [in=135, out=-90] (22.center) to (21);
		\draw [in=-150, out=45] (21) to (24);
		\draw [in=-165, out=105, looseness=1.25] (21) to (23);
		\draw [in=-30, out=60, looseness=1.25] (25) to (24);
		\draw [in=135, out=-30] (23) to (25);
		\draw [in=45, out=-90] (26.center) to (25);
		\draw [in=-150, out=90] (22.center) to (30.center);
		\draw [in=-30, out=90] (26.center) to (29.center);
		\draw (32.center) to (31.center);
		\draw (31.center) to (34.center);
		\draw (34.center) to (33.center);
		\draw (33.center) to (32.center);
		\draw (37.center) to (30.center);
		\draw (36.center) to (29.center);
		\draw [in=75, out=-45] (30.center) to (21);
		\draw [in=105, out=-135] (29.center) to (25);
	\end{pgfonlayer}
\end{tikzpicture}\ ,
\hspace*{.5cm}
\begin{tikzpicture}
	\begin{pgfonlayer}{nodelayer}
		\node [style=map] (40) at (179.75, -4) {$\lfloor g\rfloor$};
		\node [style=map] (44) at (181, -4) {$\lfloor f \rfloor$};
		\node [style=none] (46) at (179.35, -2.5) {};
		\node [style=none] (47) at (181.35, -2.5) {};
		\node [style=none] (48) at (181.35, -2.5) {};
		\node [style=none] (49) at (179.35, -2.5) {};
		\node [style=none] (50) at (178.75, -2) {};
		\node [style=none] (51) at (182, -2) {};
		\node [style=none] (52) at (182, -4.5) {};
		\node [style=none] (53) at (178.75, -4.5) {};
		\node [style=none] (54) at (179, -4.25) {$G$};
		\node [style=none] (55) at (181.35, -1.5) {};
		\node [style=none] (56) at (179.35, -1.5) {};
		\node [style=otimes] (57) at (179.35, -2.5) {};
		\node [style=otimes] (58) at (181.35, -2.5) {};
		\node [style=none] (59) at (180, -2.5) {};
		\node [style=none] (60) at (180.75, -2.5) {};
		\node [style=Z] (61) at (180.375, -1.5) {};
		\node [style=otimes] (62) at (180, -2.5) {};
		\node [style=otimes] (63) at (180.75, -2.5) {};
	\end{pgfonlayer}
	\begin{pgfonlayer}{edgelayer}
		\draw (51.center) to (50.center);
		\draw (50.center) to (53.center);
		\draw (53.center) to (52.center);
		\draw (52.center) to (51.center);
		\draw (56.center) to (49.center);
		\draw (55.center) to (48.center);
		\draw [in=75, out=-45] (49.center) to (40);
		\draw [in=105, out=-135] (48.center) to (44);
		\draw [bend right] (59.center) to (40);
		\draw [bend left] (60.center) to (44);
		\draw [bend right] (44) to (48.center);
		\draw [bend right] (49.center) to (40);
		\draw [in=-45, out=90] (60.center) to (61);
		\draw [in=90, out=-135] (61) to (59.center);
		\draw [bend left] (59.center) to (40);
		\draw [bend left] (44) to (60.center);
	\end{pgfonlayer}
\end{tikzpicture}
=
\begin{tikzpicture}
	\begin{pgfonlayer}{nodelayer}
		\node [style=map] (64) at (204.5, 0) {$G\left(\lfloor g\rfloor \right)$};
		\node [style=none] (65) at (204, 1.25) {};
		\node [style=Z] (66) at (205.5, 1) {};
		\node [style=map] (67) at (206.5, 0) {$G\left(\lfloor f\rfloor\right)$};
		\node [style=none] (68) at (207, 1.25) {};
		\node [style=none] (69) at (207, 2.5) {};
		\node [style=none] (70) at (204, 2.5) {};
		\node [style=none] (71) at (206.25, 0) {};
		\node [style=none] (72) at (204.75, 0) {};
	\end{pgfonlayer}
	\begin{pgfonlayer}{edgelayer}
		\draw [in=135, out=-90] (65.center) to (64);
		\draw [in=45, out=-90] (68.center) to (67);
		\draw (65.center) to (70.center);
		\draw (68.center) to (69.center);
		\draw [in=-15, out=90] (71.center) to (66);
		\draw [in=90, out=-165] (66) to (72.center);
	\end{pgfonlayer}
\end{tikzpicture}
$$

And thus they are equal.
Therefore:
$$
\begin{tikzpicture}
	\begin{pgfonlayer}{nodelayer}
		\node [style=map] (295) at (233.75, 0) {$G\left(\lfloor {f;g}\rfloor \right)$};
		\node [style=none] (296) at (233, 1.5) {};
		\node [style=none] (297) at (234.5, 1) {};
		\node [style=Z] (298) at (234.5, 1) {};
		\node [style=none] (299) at (235, -1) {};
	\end{pgfonlayer}
	\begin{pgfonlayer}{edgelayer}
		\draw [in=135, out=-90] (296.center) to (295);
		\draw [in=-135, out=45, looseness=0.75] (295) to (297.center);
		\draw [in=315, out=90, looseness=0.75] (299.center) to (298);
	\end{pgfonlayer}
\end{tikzpicture}
=
\begin{tikzpicture}
	\begin{pgfonlayer}{nodelayer}
		\node [style=map] (259) at (222, 0) {$G\left(\lfloor g\rfloor \right)$};
		\node [style=map] (260) at (224, 0) {$G\left(\lfloor f\rfloor\right)$};
		\node [style=Z] (263) at (223, 1) {};
		\node [style=Z] (264) at (224.5, 1) {};
		\node [style=none] (265) at (223.75, 0) {};
		\node [style=none] (266) at (222.25, 0) {};
		\node [style=none] (267) at (221.75, 1) {};
		\node [style=none] (268) at (225, 0) {};
		\node [style=none] (269) at (225, -1) {};
		\node [style=none] (270) at (221.75, 2) {};
	\end{pgfonlayer}
	\begin{pgfonlayer}{edgelayer}
		\draw (270.center) to (267.center);
		\draw [in=150, out=-90, looseness=0.75] (267.center) to (266.center);
		\draw [in=-150, out=30] (266.center) to (263);
		\draw [in=150, out=-30] (263) to (265.center);
		\draw [in=-150, out=30] (265.center) to (264);
		\draw [in=90, out=-30] (264) to (268.center);
		\draw (268.center) to (269.center);
	\end{pgfonlayer}
\end{tikzpicture}
$$
\end{proof}





The novelty in interpreting stabilizer states in this categorical framework is that it reveals that the {\em relational composition} of tableaus is the composition of stabilizer circuits.


If we drop the affine shifts, we get a smaller fragment of stabilizer circuits:


\begin{corollary}
For odd prime $p$, $\Lag\Rel_{\F_p}$ is a presentation for Weyl-free quopit stabilizer circuits.
\end{corollary}

Bevcause of the reasons we mentioned earlier $\Aff\Lag\Rel_{\F_2}$ is not isomorphic to $\Stab_2$.  
However, only some generators cause this problem:

\begin{corollary}
\label{cor:nophase}
For any prime $p$, the image of $L(\LinRel_{\F_p}) \hookrightarrow \Aff\Lag\Rel_{\F_p}$ in addition to the symplectic Weyl operators can be interpreted in $\Stab_p$.
\end{corollary}

On the other hand $\Aff\Lag\Rel_{\F_2}$ has already been studied in other terms. Spekkens first introduced his toy model as a noncontextual hidden variable alternative to qubit stabilizer quantum mechanics, satisfying the so-called "knowledge-balance" principle \cite{spekkens}.
Later, this was formalized in terms of symplectic geometry, where he showed that the pure states in the toy model are affine Lagrangian relations over $\F_2$ and the reversible evolution is given by affine symplectomorphisms: the knowledge balance principle corresponding to the preservation of the symplectic form \cite{spekkens2016quasi}.  Therefore:


\begin{corollary}
$\Aff\Lag\Rel_{\F_2}$ is a presentation for the pure states in Spekkens' qubit toy model.
\end{corollary}

In fact, this prop  already has a complete axiomatization in the literature in terms of ``the ZX-calculus'' for Spekkens' toy model \cite{backensspek}.  In fact, Spekkens' toy model has been worked out in all dimensions, thus subsuming the work of Gross for finite fields \cite{catani}. However, affine relations are only defined on fields (or at least principal ideal domains) not for example the rings $\Z/d\Z$ for $d$ not a prime.  Therefore, we can not give a relational account for nonprime dimensional stabilizer circuits/Spekkens' toy model.

There is another way to present $\Aff\Lag\Rel_{k}$ which makes it clear that $\Aff\Lag\Rel_{\F_p}$ can be regarded as a fragment of the ZX-calculus:

\begin{theorem}
$\Aff\Lag\Rel_{k}$ is generated by two spiders both decorated by the additive group of $k^2$:
$$
\left\llbracket
\begin{tikzpicture}
	\begin{pgfonlayer}{nodelayer}
		\node [style=none] (0) at (21, 5) {};
		\node [style=none] (1) at (22, 5) {};
		\node [style=none] (2) at (21, 2.5) {};
		\node [style=none] (3) at (22, 2.5) {};
		\node [style=Z] (4) at (21.5, 3.75) {$\hspace*{.05cm}n,m\hspace*{.05cm}$};
		\node [style=none] (5) at (21.5, 4.5) {$\cdots$};
		\node [style=none] (6) at (21.5, 3) {$\cdots$};
		\node [style=none] (7) at (21.5, 4.75) {};
		\node [style=none] (8) at (21.5, 2.75) {};
	\end{pgfonlayer}
	\begin{pgfonlayer}{edgelayer}
		\draw [in=150, out=-90, looseness=0.75] (0.center) to (4);
		\draw [in=90, out=-150, looseness=0.75] (4) to (2.center);
		\draw [in=-30, out=90, looseness=0.75] (3.center) to (4);
		\draw [in=-90, out=30, looseness=0.75] (4) to (1.center);
	\end{pgfonlayer}
\end{tikzpicture}
\right\rrbracket
=
\begin{tikzpicture}
	\begin{pgfonlayer}{nodelayer}
		\node [style=none] (9) at (231.75, 0.5) {};
		\node [style=none] (10) at (231.75, -3) {};
		\node [style=Z] (11) at (231, -2) {};
		\node [style=none] (12) at (231.32, -1.25) {$\cdots$};
		\node [style=none] (13) at (231, -2.5) {$\cdots$};
		\node [style=none] (14) at (230.75, 0.5) {};
		\node [style=none] (15) at (229.25, -3) {};
		\node [style=X] (16) at (230, -0.5) {$n$};
		\node [style=none] (17) at (230, 0) {$\cdots$};
		\node [style=none] (18) at (229.72, -1.25) {$\cdots$};
		\node [style=none] (19) at (230, -3) {};
		\node [style=none] (20) at (230.25, -3) {};
		\node [style=none] (21) at (229.25, 0.5) {};
		\node [style=none] (22) at (231, 0.5) {};
		\node [style=scalar] (23) at (230.5, -1.25) {$m$};
		\node [style=none] (24) at (230, 0.25) {};
		\node [style=none] (25) at (231, -2.75) {};
		\node [style=none] (26) at (229.7, -1.5) {};
		\node [style=none] (27) at (231.3, -1) {};
	\end{pgfonlayer}
	\begin{pgfonlayer}{edgelayer}
		\draw [in=-30, out=90] (10.center) to (11);
		\draw [in=-90, out=30, looseness=0.75] (11) to (9.center);
		\draw [in=90, out=-150] (16) to (15.center);
		\draw [in=-90, out=30] (16) to (14.center);
		\draw (19.center) to (16);
		\draw [in=-150, out=90] (20.center) to (11);
		\draw [in=-90, out=150] (16) to (21.center);
		\draw (11) to (22.center);
		\draw [in=-75, out=150] (11) to (23);
		\draw [in=330, out=90] (23) to (16);
	\end{pgfonlayer}
\end{tikzpicture}
\hspace*{.5cm}
\left\llbracket
\begin{tikzpicture}
	\begin{pgfonlayer}{nodelayer}
		\node [style=none] (0) at (21, 5) {};
		\node [style=none] (1) at (22, 5) {};
		\node [style=none] (2) at (21, 2.5) {};
		\node [style=none] (3) at (22, 2.5) {};
		\node [style=X] (4) at (21.5, 3.75) {$\hspace*{.05cm}n,m\hspace*{.05cm}$};
		\node [style=none] (5) at (21.5, 4.5) {$\cdots$};
		\node [style=none] (6) at (21.5, 3) {$\cdots$};
		\node [style=none] (7) at (21.5, 4.75) {};
		\node [style=none] (8) at (21.5, 2.75) {};
	\end{pgfonlayer}
	\begin{pgfonlayer}{edgelayer}
		\draw [in=150, out=-90, looseness=0.75] (0.center) to (4);
		\draw [in=90, out=-150, looseness=0.75] (4) to (2.center);
		\draw [in=-30, out=90, looseness=0.75] (3.center) to (4);
		\draw [in=-90, out=30, looseness=0.75] (4) to (1.center);
	\end{pgfonlayer}
\end{tikzpicture}
\right\rrbracket
:=
\begin{tikzpicture}
	\begin{pgfonlayer}{nodelayer}
		\node [style=none] (0) at (232.75, 0.5) {};
		\node [style=none] (1) at (232.75, -3) {};
		\node [style=Z] (2) at (233.5, -2) {};
		\node [style=none] (3) at (233.25, -1.25) {$\cdots$};
		\node [style=none] (4) at (233.5, -2.5) {$\cdots$};
		\node [style=none] (5) at (233.75, 0.5) {};
		\node [style=none] (6) at (235.25, -3) {};
		\node [style=X] (7) at (234.5, -0.5) {$n$};
		\node [style=none] (8) at (234.5, 0) {$\cdots$};
		\node [style=none] (9) at (234.82, -1.25) {$\cdots$};
		\node [style=none] (10) at (234.5, -3) {};
		\node [style=none] (11) at (234.25, -3) {};
		\node [style=none] (12) at (235.25, 0.5) {};
		\node [style=none] (13) at (233.5, 0.5) {};
		\node [style=scalar] (14) at (234, -1.25) {$m$};
		\node [style=none] (15) at (233.25, -1) {};
		\node [style=none] (16) at (234.5, 0.25) {};
		\node [style=none] (17) at (233.5, -2.75) {};
		\node [style=none] (18) at (234.78, -1.5) {};
	\end{pgfonlayer}
	\begin{pgfonlayer}{edgelayer}
		\draw [in=-150, out=90] (1.center) to (2);
		\draw [in=-90, out=150, looseness=0.75] (2) to (0.center);
		\draw [in=90, out=-30] (7) to (6.center);
		\draw [in=-90, out=150] (7) to (5.center);
		\draw (10.center) to (7);
		\draw [in=-30, out=90] (11.center) to (2);
		\draw [in=-90, out=30] (7) to (12.center);
		\draw (2) to (13.center);
		\draw [in=-105, out=30] (2) to (14);
		\draw [in=-150, out=90] (14) to (7);
	\end{pgfonlayer}
\end{tikzpicture}
$$

Where we recall that Fourier transform is derived by Euler composition:
$$
\left\llbracket
\begin{tikzpicture}
	\begin{pgfonlayer}{nodelayer}
		\node [style=none] (0) at (1.25, -1) {};
		\node [style=map] (1) at (1.25, -1.5) {$F$};
		\node [style=none] (2) at (1.25, -2) {};
	\end{pgfonlayer}
	\begin{pgfonlayer}{edgelayer}
		\draw (2.center) to (1);
		\draw (1) to (0.center);
	\end{pgfonlayer}
\end{tikzpicture}\
\right\rrbracket
=
\begin{tikzpicture}
	\begin{pgfonlayer}{nodelayer}
		\node [style=none] (0) at (0.5, 1) {};
		\node [style=none] (1) at (0.5, -0.25) {};
		\node [style=none] (2) at (1, -0.25) {};
		\node [style=none] (3) at (1, 1) {};
		\node [style=s] (4) at (1, 0.5) {};
		\node [style=none] (5) at (0.5, 0.5) {};
	\end{pgfonlayer}
	\begin{pgfonlayer}{edgelayer}
		\draw (4) to (3.center);
		\draw [in=90, out=-90] (4) to (1.center);
		\draw [in=-90, out=90] (2.center) to (5.center);
		\draw (5.center) to (0.center);
	\end{pgfonlayer}
\end{tikzpicture}
=
\begin{tikzpicture}[xscale=-1]
	\begin{pgfonlayer}{nodelayer}
		\node [style=X] (0) at (19.5, 1.25) {};
		\node [style=Z] (1) at (18, 4.25) {};
		\node [style=none] (2) at (18, 4.75) {};
		\node [style=none] (3) at (19.5, 4.75) {};
		\node [style=none] (4) at (18, 0.75) {};
		\node [style=none] (5) at (19.5, 0.75) {};
		\node [style=X] (6) at (19.5, 1.25) {};
		\node [style=Z] (7) at (19.5, 4.25) {};
		\node [style=X] (8) at (18, 1.25) {};
		\node [style=Z] (9) at (18, 4.25) {};
		\node [style=X] (10) at (19.5, 1.25) {};
		\node [style=X] (11) at (18, 1.25) {};
		\node [style=X] (12) at (19.5, 1.25) {};
		\node [style=Z] (13) at (19.5, 4.25) {};
		\node [style=s] (14) at (17.75, 3) {};
		\node [style=s] (15) at (18.75, 3) {};
		\node [style=s] (16) at (19.75, 3) {};
		\node [style=s] (17) at (18.25, 3) {};
	\end{pgfonlayer}
	\begin{pgfonlayer}{edgelayer}
		\draw (2.center) to (1);
		\draw (5.center) to (0);
		\draw [bend right=45] (6) to (7);
		\draw [in=135, out=-135, looseness=1.25] (9) to (8);
		\draw (3.center) to (7);
		\draw (4.center) to (8);
		\draw [in=-120, out=15, looseness=0.75] (11) to (13);
		\draw [in=90, out=-105] (9) to (14);
		\draw [in=90, out=-45, looseness=0.75] (9) to (15);
		\draw [in=90, out=-90] (14) to (11);
		\draw [in=-90, out=120, looseness=0.75] (6) to (15);
		\draw [in=-15, out=90] (12) to (9);
		\draw [in=-90, out=150, looseness=0.75] (12) to (17);
		\draw [in=285, out=90] (17) to (9);
		\draw [in=-90, out=75, looseness=0.75] (12) to (16);
		\draw [in=-75, out=90] (16) to (13);
	\end{pgfonlayer}
\end{tikzpicture}
=
\begin{tikzpicture}
	\begin{pgfonlayer}{nodelayer}
		\node [style=none] (0) at (0, 0.75) {};
		\node [style=none] (1) at (0.75, 0.75) {};
		\node [style=none] (2) at (0, 3.25) {};
		\node [style=none] (3) at (0.75, 3.25) {};
		\node [style=Z] (4) at (0.75, 1.25) {};
		\node [style=X] (5) at (0, 1.75) {};
		\node [style=Z] (6) at (0.75, 2.25) {};
		\node [style=X] (7) at (0, 2.75) {};
		\node [style=Z] (8) at (0, 2.25) {};
		\node [style=X] (9) at (0.75, 1.75) {};
	\end{pgfonlayer}
	\begin{pgfonlayer}{edgelayer}
		\draw (4) to (5);
		\draw (6) to (7);
		\draw (8) to (9);
		\draw (1.center) to (4);
		\draw (4) to (9);
		\draw (9) to (6);
		\draw (6) to (3.center);
		\draw (2.center) to (7);
		\draw (7) to (8);
		\draw (8) to (5);
		\draw (5) to (0.center);
	\end{pgfonlayer}
\end{tikzpicture}
=
\left\llbracket
\begin{tikzpicture}
	\begin{pgfonlayer}{nodelayer}
		\node [style=none] (0) at (1.25, 0) {};
		\node [style=none] (1) at (1.25, -3.5) {};
		\node [style=Z] (2) at (1.25, -2.75) {$\hspace*{.05cm}0,1\hspace*{.05cm}$};
		\node [style=Z] (3) at (1.25, -0.75) {$\hspace*{.05cm}0,1\hspace*{.05cm}$};
		\node [style=X] (4) at (1.25, -1.75) {$\hspace*{.05cm}0,-1\hspace*{.05cm}$};
	\end{pgfonlayer}
	\begin{pgfonlayer}{edgelayer}
		\draw (1.center) to (2);
		\draw (2) to (4);
		\draw (4) to (3);
		\draw (3) to (0.center);
	\end{pgfonlayer}
\end{tikzpicture}
\right\rrbracket
$$


%Transporting the complex conjugation along the isomorphism $\Aff\Lag\Rel_k \cong \Stab_p$ yields a conjugation functor $\Aff\Lag\Rel_k\to \Aff\Lag\Rel_k$ such that:

%In $\Stab_p$ the spiders are linear maps:
%
%$$
%\left\llbracket
%\begin{tikzpicture}
%	\begin{pgfonlayer}{nodelayer}
%		\node [style=none] (0) at (21, 5) {};
%		\node [style=none] (1) at (22, 5) {};
%		\node [style=none] (2) at (21, 2.5) {};
%		\node [style=none] (3) at (22, 2.5) {};
%		\node [style=Z] (4) at (21.5, 3.75) {$\hspace*{.05cm}n,m\hspace*{.05cm}$};
%		\node [style=none] (5) at (21.5, 4.5) {$\cdots$};
%		\node [style=none] (6) at (21.5, 3) {$\cdots$};
%		\node [style=none] (7) at (21.5, 4.75) {};
%		\node [style=none] (8) at (21.5, 2.75) {};
%	\end{pgfonlayer}
%	\begin{pgfonlayer}{edgelayer}
%		\draw [in=150, out=-90, looseness=0.75] (0.center) to (4);
%		\draw [in=90, out=-150, looseness=0.75] (4) to (2.center);
%		\draw [in=-30, out=90, looseness=0.75] (3.center) to (4);
%		\draw [in=-90, out=30, looseness=0.75] (4) to (1.center);
%	\end{pgfonlayer}
%\end{tikzpicture}
%\right\rrbracket
%=
%\sum{}
%$$
%
%
%
%GIVE ACTION OF COMPLEX CONJUGATION
\end{theorem}






The spider fusion is pointwise  where $(a,b)=(n+k,m+\ell)$:

$$
\begin{tikzpicture}
	\begin{pgfonlayer}{nodelayer}
		\node [style=none] (0) at (1.5, -0.5) {};
		\node [style=none] (1) at (0.5, -0.5) {};
		\node [style=none] (2) at (1, -0.5) {$\cdots$};
		\node [style=none] (3) at (0.5, -2.75) {};
		\node [style=Z] (4) at (1, -1.25) {$n,m$};
		\node [style=none] (5) at (2, -0.5) {};
		\node [style=none] (6) at (1.5, -2.75) {$\cdots$};
		\node [style=none] (7) at (1, -2.75) {};
		\node [style=Z] (8) at (1.5, -2) {$k,\ell$};
		\node [style=none] (9) at (2, -2.75) {};
		\node [style=none] (10) at (1.25, -1.5) {\reflectbox{$\ddots$}};
	\end{pgfonlayer}
	\begin{pgfonlayer}{edgelayer}
		\draw [in=-124, out=90] (3.center) to (4);
		\draw [in=-90, out=56] (4) to (0.center);
		\draw [in=124, out=-90] (1.center) to (4);
		\draw [in=-124, out=90] (7.center) to (8);
		\draw [in=90, out=-56] (8) to (9.center);
		\draw [in=-90, out=56] (8) to (5.center);
		\draw [bend left=45, looseness=1.25] (8) to (4);
		\draw [bend left=45, looseness=1.25] (4) to (8);
	\end{pgfonlayer}
\end{tikzpicture}
=
\begin{tikzpicture}
	\begin{pgfonlayer}{nodelayer}
		\node [style=none] (11) at (4, -0.5) {};
		\node [style=none] (12) at (3, -0.5) {};
		\node [style=none] (13) at (3.5, -0.5) {$\cdots$};
		\node [style=none] (14) at (2.5, -2) {};
		\node [style=none] (15) at (3.5, -1.25) {};
		\node [style=none] (16) at (4.5, -0.5) {};
		\node [style=none] (17) at (3.5, -2) {$\cdots$};
		\node [style=none] (18) at (3, -2) {};
		\node [style=Z] (19) at (3.5, -1.25) {$a,b$};
		\node [style=none] (20) at (4, -2) {};
	\end{pgfonlayer}
	\begin{pgfonlayer}{edgelayer}
		\draw [in=-150, out=90] (14.center) to (15);
		\draw [in=-90, out=56] (15) to (11.center);
		\draw [in=124, out=-90] (12.center) to (15);
		\draw [in=-124, out=90] (18.center) to (19);
		\draw [in=90, out=-56] (19) to (20.center);
		\draw [in=-90, out=30] (19) to (16.center);
	\end{pgfonlayer}
\end{tikzpicture}
\hspace*{1cm}
\begin{tikzpicture}
	\begin{pgfonlayer}{nodelayer}
		\node [style=none] (0) at (1.5, -0.5) {};
		\node [style=none] (1) at (0.5, -0.5) {};
		\node [style=none] (2) at (1, -0.5) {$\cdots$};
		\node [style=none] (3) at (0.5, -2.75) {};
		\node [style=X] (4) at (1, -1.25) {$n,m$};
		\node [style=none] (5) at (2, -0.5) {};
		\node [style=none] (6) at (1.5, -2.75) {$\cdots$};
		\node [style=none] (7) at (1, -2.75) {};
		\node [style=X] (8) at (1.5, -2) {$k,\ell$};
		\node [style=none] (9) at (2, -2.75) {};
		\node [style=none] (10) at (1.25, -1.5) {\reflectbox{$\ddots$}};
	\end{pgfonlayer}
	\begin{pgfonlayer}{edgelayer}
		\draw [in=-124, out=90] (3.center) to (4);
		\draw [in=-90, out=56] (4) to (0.center);
		\draw [in=124, out=-90] (1.center) to (4);
		\draw [in=-124, out=90] (7.center) to (8);
		\draw [in=90, out=-56] (8) to (9.center);
		\draw [in=-90, out=56] (8) to (5.center);
		\draw [bend left=45, looseness=1.25] (8) to (4);
		\draw [bend left=45, looseness=1.25] (4) to (8);
	\end{pgfonlayer}
\end{tikzpicture}
=
\begin{tikzpicture}
	\begin{pgfonlayer}{nodelayer}
		\node [style=none] (11) at (4, -0.5) {};
		\node [style=none] (12) at (3, -0.5) {};
		\node [style=none] (13) at (3.5, -0.5) {$\cdots$};
		\node [style=none] (14) at (2.5, -2) {};
		\node [style=none] (15) at (3.5, -1.25) {};
		\node [style=none] (16) at (4.5, -0.5) {};
		\node [style=none] (17) at (3.5, -2) {$\cdots$};
		\node [style=none] (18) at (3, -2) {};
		\node [style=X] (19) at (3.5, -1.25) {$a,b$};
		\node [style=none] (20) at (4, -2) {};
	\end{pgfonlayer}
	\begin{pgfonlayer}{edgelayer}
		\draw [in=-150, out=90] (14.center) to (15);
		\draw [in=-90, out=56] (15) to (11.center);
		\draw [in=124, out=-90] (12.center) to (15);
		\draw [in=-124, out=90] (18.center) to (19);
		\draw [in=90, out=-56] (19) to (20.center);
		\draw [in=-90, out=30] (19) to (16.center);
	\end{pgfonlayer}
\end{tikzpicture}
$$

Call the first component of the phase group the {\bf affine phase} and the second component the {\bf linear phase}.  The white spider corresponds to the $Z$ basis and the grey spider corresponds to the $X$ basis.



In Hilbert spaces, the spiders are interpreted as follows:

\begin{align*}
\left\llbracket\
\begin{tikzpicture}
	\begin{pgfonlayer}{nodelayer}
		\node [style=none] (0) at (21, 5) {};
		\node [style=none] (1) at (22, 5) {};
		\node [style=none] (2) at (21, 2.5) {};
		\node [style=none] (3) at (22, 2.5) {};
		\node [style=Z] (4) at (21.5, 3.75) {$\hspace*{.05cm}n,m\hspace*{.05cm}$};
		\node [style=none] (5) at (21.5, 4.5) {$\cdots$};
		\node [style=none] (6) at (21.5, 3) {$\cdots$};
		\node [style=none] (7) at (21.5, 4.75) {};
		\node [style=none] (8) at (21.5, 2.75) {};
	\end{pgfonlayer}
	\begin{pgfonlayer}{edgelayer}
		\draw [in=150, out=-90, looseness=0.75] (0.center) to (4);
		\draw [in=90, out=-150, looseness=0.75] (4) to (2.center);
		\draw [in=-30, out=90, looseness=0.75] (3.center) to (4);
		\draw [in=-90, out=30, looseness=0.75] (4) to (1.center);
	\end{pgfonlayer}
\end{tikzpicture}\
\right\rrbracket
\propto &
\sum_{a=0}^{p-1}  e^{\pi\cdot i /p (n\cdot a+m\cdot a^2)} |a, \ldots, a \rangle \langle a, \ldots, a|\\
\left\llbracket\
\begin{tikzpicture}
	\begin{pgfonlayer}{nodelayer}
		\node [style=none] (0) at (21, 5) {};
		\node [style=none] (1) at (22, 5) {};
		\node [style=none] (2) at (21, 2.5) {};
		\node [style=none] (3) at (22, 2.5) {};
		\node [style=X] (4) at (21.5, 3.75) {$\hspace*{.05cm}n,m\hspace*{.05cm}$};
		\node [style=none] (5) at (21.5, 4.5) {$\cdots$};
		\node [style=none] (6) at (21.5, 3) {$\cdots$};
		\node [style=none] (7) at (21.5, 4.75) {};
		\node [style=none] (8) at (21.5, 2.75) {};
	\end{pgfonlayer}
	\begin{pgfonlayer}{edgelayer}
		\draw [in=150, out=-90, looseness=0.75] (0.center) to (4);
		\draw [in=90, out=-150, looseness=0.75] (4) to (2.center);
		\draw [in=-30, out=90, looseness=0.75] (3.center) to (4);
		\draw [in=-90, out=30, looseness=0.75] (4) to (1.center);
	\end{pgfonlayer}
\end{tikzpicture}\
\right\rrbracket
\propto &
\sum_{a,b,c,d,e=0}^{p-1} 
e^{- \pi\cdot i/ p \cdot d} |e, \ldots,e \rangle \langle d, \ldots, d|\\
&\hspace*{.25cm} \cdot e^{\pi\cdot i /p (n\cdot c-m\cdot c^2)} |c, \ldots, c \rangle \langle c, \ldots, c|\\
&\hspace*{.25cm}  e^{\pi\cdot i/ p \cdot b} |b, \ldots, b \rangle \langle a, \ldots, a| \\
=&
\sum_{a,e=0}^{p-1} 
e^{\pi\cdot i/ p \cdot (-b+n\cdot b-m\cdot b^2+b)} |e, \ldots,e \rangle \langle a, \ldots, a| \\
=&
\sum_{a,e=0}^{p-1} 
e^{\pi\cdot i/ p \cdot (n\cdot b-m\cdot b^2)} |e, \ldots,e \rangle \langle a, \ldots, a| \\
\end{align*}

Notice that this implies that the phase groups for quopit stabilizer spiders are now the torus $(\Z/p\Z)^2$  as noted \cite[p. 166]{ranchin2016alternative}.  This is in contrast to the qubit case where the phase groups are $\Z/4\Z$; which \cite{coecke2011phase} point out as a crucial difference between the toy model and qubit stabilizer theory.

However, the restricted class of circuits described in Corollary \ref{cor:nophase} is the fragment of the stabilizer ZX-calculus where in the qubit case, the phases are in the subgroup $\Z/2\Z\subseteq \Z/4\Z$ sending $a\mapsto 2a$; and in the quopit case, they are in the subgroup $\Z/p\Z \subseteq (\Z/p\Z)^2$ which sends $a\mapsto (a,0)$.  These subgroups both agree for the qubits and quopits, so we can think of them both as Affine Lagrangian spiders with trivial linear phases.


Up until now, we have also neglected to relate the symplectic picture to the \dag-structure of $\FHilb$:


\begin{definition}
There is a monoidal conjugation functor $\bar{(\_)}:\Aff\Lag\Rel_k\to \Aff\Lag\Rel_k$ given by:
$$
\begin{tikzpicture}
	\begin{pgfonlayer}{nodelayer}
		\node [style=none] (0) at (21, 5) {};
		\node [style=none] (1) at (22, 5) {};
		\node [style=none] (2) at (21, 2.5) {};
		\node [style=none] (3) at (22, 2.5) {};
		\node [style=Z] (4) at (21.5, 3.75) {$\hspace*{.05cm}n,m\hspace*{.05cm}$};
		\node [style=none] (5) at (21.5, 4.5) {$\cdots$};
		\node [style=none] (6) at (21.5, 3) {$\cdots$};
		\node [style=none] (7) at (21.5, 4.75) {};
		\node [style=none] (8) at (21.5, 2.75) {};
	\end{pgfonlayer}
	\begin{pgfonlayer}{edgelayer}
		\draw [in=150, out=-90, looseness=0.75] (0.center) to (4);
		\draw [in=90, out=-150, looseness=0.75] (4) to (2.center);
		\draw [in=-30, out=90, looseness=0.75] (3.center) to (4);
		\draw [in=-90, out=30, looseness=0.75] (4) to (1.center);
	\end{pgfonlayer}
\end{tikzpicture}
\mapsto
\begin{tikzpicture}
	\begin{pgfonlayer}{nodelayer}
		\node [style=none] (0) at (20.75, 5) {};
		\node [style=none] (1) at (22.25, 5) {};
		\node [style=none] (2) at (20.75, 2.5) {};
		\node [style=none] (3) at (22.25, 2.5) {};
		\node [style=Z] (4) at (21.5, 3.75) {};
		\node [style=none] (5) at (21.5, 4.75) {$\cdots$};
		\node [style=none] (6) at (21.5, 2.75) {$\cdots$};
		\node [style=Z] (9) at (21.5, 3.75) {$\hspace*{.05cm}-n,-m\hspace*{.05cm}$};
	\end{pgfonlayer}
	\begin{pgfonlayer}{edgelayer}
		\draw [in=150, out=-90, looseness=0.75] (0.center) to (4);
		\draw [in=90, out=-150, looseness=0.75] (4) to (2.center);
		\draw [in=-30, out=90, looseness=0.75] (3.center) to (4);
		\draw [in=-90, out=30, looseness=0.75] (4) to (1.center);
	\end{pgfonlayer}
\end{tikzpicture}
\hspace*{1cm}
\begin{tikzpicture}
	\begin{pgfonlayer}{nodelayer}
		\node [style=none] (0) at (21, 5) {};
		\node [style=none] (1) at (22, 5) {};
		\node [style=none] (2) at (21, 2.5) {};
		\node [style=none] (3) at (22, 2.5) {};
		\node [style=X] (4) at (21.5, 3.75) {$\hspace*{.05cm}n,m\hspace*{.05cm}$};
		\node [style=none] (5) at (21.5, 4.5) {$\cdots$};
		\node [style=none] (6) at (21.5, 3) {$\cdots$};
		\node [style=none] (7) at (21.5, 4.75) {};
		\node [style=none] (8) at (21.5, 2.75) {};
	\end{pgfonlayer}
	\begin{pgfonlayer}{edgelayer}
		\draw [in=150, out=-90, looseness=0.75] (0.center) to (4);
		\draw [in=90, out=-150, looseness=0.75] (4) to (2.center);
		\draw [in=-30, out=90, looseness=0.75] (3.center) to (4);
		\draw [in=-90, out=30, looseness=0.75] (4) to (1.center);
	\end{pgfonlayer}
\end{tikzpicture}
\mapsto
\begin{tikzpicture}
	\begin{pgfonlayer}{nodelayer}
		\node [style=none] (0) at (20.75, 5) {};
		\node [style=none] (1) at (22.25, 5) {};
		\node [style=none] (2) at (20.75, 2.5) {};
		\node [style=none] (3) at (22.25, 2.5) {};
		\node [style=X] (4) at (21.5, 3.75) {};
		\node [style=none] (5) at (21.5, 4.75) {$\cdots$};
		\node [style=none] (6) at (21.5, 2.75) {$\cdots$};
		\node [style=X] (9) at (21.5, 3.75) {$\hspace*{.05cm}n,-m\hspace*{.05cm}$};
	\end{pgfonlayer}
	\begin{pgfonlayer}{edgelayer}
		\draw [in=150, out=-90, looseness=0.75] (0.center) to (4);
		\draw [in=90, out=-150, looseness=0.75] (4) to (2.center);
		\draw [in=-30, out=90, looseness=0.75] (3.center) to (4);
		\draw [in=-90, out=30, looseness=0.75] (4) to (1.center);
	\end{pgfonlayer}
\end{tikzpicture}
$$
\end{definition}

In the case of $k=\F_p$ for $p$ an odd prime, this is transported to the complex conjugation along $\Aff\Lag\Rel_{\F_p} \cong \Stab_p$:



\begin{lemma}
For odd prime $p$, $\Aff\Lag\Rel_{\F_p}$ and  $\Stab_p$ are isomorphic as \dag-compact closed categories.
\end{lemma}

The proof follows by checking the action of the complex conjugation on the generators.




\subsection{Electrical circuits}

Symplectic geometry, and indeed the early work of Weinsten and his ``symplectic category'' was motivated by the goal to formalize classical mechanics in a synthetic setting.  


In \cite{passive} used Lagrangian relations to give a semantics for passive linear electrical circuits:



Ohms law


Kirkoffs voltage law



derivative and real rational functions following \cite{control}




Affine Lagrangian relations and power sources \cite{network}


%
%
%Affine Lagrangian relations have been used as a semantics for certain classes of electrical circuits:
%\begin{example}
%\label{ex:circuits}
%TODO: MUCH MORE EXPOSITION, CITATIONS
%
%For any non-negative real $a$, wires, $a$-weighted resistors, inductors and capacitors have the following interpretations in $\Aff\Lag\Rel_{\mathbb{R}(x)}$:
%$$
%\left\llbracket
%\begin{tikzpicture}
%	\begin{pgfonlayer}{nodelayer}
%		\node [style=none] (0) at (21, 4.25) {};
%		\node [style=none] (1) at (22, 4.25) {};
%		\node [style=none] (2) at (21, 2.75) {};
%		\node [style=none] (3) at (22, 2.75) {};
%		\node [style=dot] (4) at (21.5, 3.5) {};
%		\node [style=none] (5) at (21.5, 4) {$\cdots$};
%		\node [style=none] (6) at (21.5, 3) {$\cdots$};
%	\end{pgfonlayer}
%	\begin{pgfonlayer}{edgelayer}
%		\draw [in=150, out=-90] (0.center) to (4);
%		\draw [in=90, out=-150] (4) to (2.center);
%		\draw [in=-30, out=90] (3.center) to (4);
%		\draw [in=-90, out=30] (4) to (1.center);
%	\end{pgfonlayer}
%\end{tikzpicture}
%\right\rrbracket
%=
%\begin{tikzpicture}
%	\begin{pgfonlayer}{nodelayer}
%		\node [style=none] (501) at (237.5, 4.25) {};
%		\node [style=none] (502) at (238.5, 4.25) {};
%		\node [style=none] (503) at (237.5, 2.75) {};
%		\node [style=none] (504) at (238.5, 2.75) {};
%		\node [style=X] (505) at (238, 3.5) {};
%		\node [style=none] (506) at (238, 4) {$\cdots$};
%		\node [style=none] (507) at (238, 3) {$\cdots$};
%		\node [style=none] (508) at (236.25, 4.25) {};
%		\node [style=none] (509) at (237.25, 4.25) {};
%		\node [style=none] (510) at (236.25, 2.75) {};
%		\node [style=none] (511) at (237.25, 2.75) {};
%		\node [style=Z] (512) at (236.75, 3.5) {};
%		\node [style=none] (513) at (236.75, 4) {$\cdots$};
%		\node [style=none] (514) at (236.75, 3) {$\cdots$};
%	\end{pgfonlayer}
%	\begin{pgfonlayer}{edgelayer}
%		\draw [in=150, out=-90] (501.center) to (505);
%		\draw [in=90, out=-150] (505) to (503.center);
%		\draw [in=-30, out=90] (504.center) to (505);
%		\draw [in=-90, out=30] (505) to (502.center);
%		\draw [in=150, out=-90] (508.center) to (512);
%		\draw [in=90, out=-150] (512) to (510.center);
%		\draw [in=-30, out=90] (511.center) to (512);
%		\draw [in=-90, out=30] (512) to (509.center);
%	\end{pgfonlayer}
%\end{tikzpicture}
%\hspace*{.5cm}
%\left\llbracket
%\tikz \draw (0,0) to[R=$a$] (0,2);
%\hspace*{,3cm}
%\right\rrbracket
%=
%\begin{tikzpicture}
%	\begin{pgfonlayer}{nodelayer}
%		\node [style=Z] (0) at (23, 3.75) {};
%		\node [style=X] (1) at (22, 5.25) {};
%		\node [style=scalar] (2) at (22.5, 4.5) {$a$};
%		\node [style=none] (3) at (22, 5.75) {};
%		\node [style=none] (4) at (23, 5.75) {};
%		\node [style=none] (5) at (23, 3.25) {};
%		\node [style=none] (6) at (22, 3.25) {};
%	\end{pgfonlayer}
%	\begin{pgfonlayer}{edgelayer}
%		\draw [in=-90, out=135] (0) to (2);
%		\draw [in=315, out=90] (2) to (1);
%		\draw (1) to (3.center);
%		\draw (1) to (6.center);
%		\draw (5.center) to (0);
%		\draw (4.center) to (0);
%	\end{pgfonlayer}
%\end{tikzpicture}
%$$
%$$
%\left\llbracket
%\tikz \draw (0,0) to[L=$a$] (0,2);
%\right\rrbracket
%=
%\begin{tikzpicture}
%	\begin{pgfonlayer}{nodelayer}
%		\node [style=Z] (0) at (23, 3.75) {};
%		\node [style=X] (1) at (22, 5.25) {};
%		\node [style=scalar] (2) at (22.5, 4.5) {$ax$};
%		\node [style=none] (3) at (22, 5.75) {};
%		\node [style=none] (4) at (23, 5.75) {};
%		\node [style=none] (5) at (23, 3.25) {};
%		\node [style=none] (6) at (22, 3.25) {};
%	\end{pgfonlayer}
%	\begin{pgfonlayer}{edgelayer}
%		\draw [in=-90, out=135] (0) to (2);
%		\draw [in=315, out=90] (2) to (1);
%		\draw (1) to (3.center);
%		\draw (1) to (6.center);
%		\draw (5.center) to (0);
%		\draw (4.center) to (0);
%	\end{pgfonlayer}
%\end{tikzpicture}
%\hspace*{.5cm}
%\left\llbracket
%\tikz \draw (0,0) to[C=$a$] (0,2);
%\hspace*{,3cm}
%\right\rrbracket
%=
%\begin{tikzpicture}
%	\begin{pgfonlayer}{nodelayer}
%		\node [style=Z] (0) at (23.25, 3.75) {};
%		\node [style=X] (1) at (21.75, 5.25) {};
%		\node [style=scalar] (2) at (22.5, 4.5) {$-ax$};
%		\node [style=none] (3) at (21.75, 5.75) {};
%		\node [style=none] (4) at (23.25, 5.75) {};
%		\node [style=none] (5) at (23.25, 3.25) {};
%		\node [style=none] (6) at (21.75, 3.25) {};
%	\end{pgfonlayer}
%	\begin{pgfonlayer}{edgelayer}
%		\draw [in=-90, out=135] (0) to (2);
%		\draw [in=315, out=90] (2) to (1);
%		\draw (1) to (3.center);
%		\draw (1) to (6.center);
%		\draw (5.center) to (0);
%		\draw (4.center) to (0);
%	\end{pgfonlayer}
%\end{tikzpicture}
%$$
%Similarly for $a$-valued voltage and current sources (again, for $a$ a non-negative real number):
%$$
%\left\llbracket
%\begin{tikzpicture}
%	\begin{pgfonlayer}{nodelayer}
%		\node [style=none] (0) at (0, 2) {};
%		\node [style=isourceAMshape,rotate=90] (1) at (0, 1) {};
%		\node [style=none] (2) at (0, 0) {};
%	\end{pgfonlayer}
%	\begin{pgfonlayer}{edgelayer}
%		\draw (2.center) to (1);
%		\draw (1) to (0.center);
%		\node [style=none] (3) at (-.7, 1) {$a$};
%	\end{pgfonlayer}
%\end{tikzpicture}
%\hspace*{,3cm}
%\right\rrbracket
%=
%\begin{tikzpicture}
%	\begin{pgfonlayer}{nodelayer}
%		\node [style=X] (1) at (22, 5.25) {};
%		\node [style=scalar] (2) at (22.5, 4.5) {$ax$};
%		\node [style=none] (3) at (22, 5.75) {};
%		\node [style=none] (4) at (23, 5.75) {};
%		\node [style=none] (5) at (23, 3.25) {};
%		\node [style=none] (6) at (22, 3.25) {};
%		\node [style=X] (7) at (22.5, 3.75) {$1$};
%	\end{pgfonlayer}
%	\begin{pgfonlayer}{edgelayer}
%		\draw [in=315, out=90] (2) to (1);
%		\draw (1) to (3.center);
%		\draw (1) to (6.center);
%		\draw (7) to (2);
%		\draw (5.center) to (4.center);
%	\end{pgfonlayer}
%\end{tikzpicture}
%=
%\begin{tikzpicture}
%	\begin{pgfonlayer}{nodelayer}
%		\node [style=X] (32) at (129.25, -0.25) {$ax$};
%		\node [style=none] (33) at (129.25, 1) {};
%		\node [style=none] (34) at (129.75, 1) {};
%		\node [style=none] (35) at (129.75, -1.5) {};
%		\node [style=none] (36) at (129.25, -1.5) {};
%	\end{pgfonlayer}
%	\begin{pgfonlayer}{edgelayer}
%		\draw (32) to (33.center);
%		\draw (32) to (36.center);
%		\draw (35.center) to (34.center);
%	\end{pgfonlayer}
%\end{tikzpicture}
%$$
%$$
%\left\llbracket
%\begin{tikzpicture}
%	\begin{pgfonlayer}{nodelayer}
%		\node [style=none] (0) at (0, 2) {};
%		\node [style=vsourceAMshape,rotate=-90] (1) at (0, 1) {};
%		\node [style=none] (2) at (0, 0) {};
%		\node [style=none] (3) at (-.7, 1) {$a$};
%	\end{pgfonlayer}
%	\begin{pgfonlayer}{edgelayer}
%		\draw (2.center) to (1);
%		\draw (1) to (0.center);
%	\end{pgfonlayer}
%\end{tikzpicture}
%\hspace*{,3cm}
%\right\rrbracket
%=
%\begin{tikzpicture}
%	\begin{pgfonlayer}{nodelayer}
%		\node [style=none] (28) at (9, 1) {};
%		\node [style=none] (29) at (10, 1) {};
%		\node [style=none] (30) at (10, -1.5) {};
%		\node [style=none] (31) at (9, -1.5) {};
%		\node [style=Z] (32) at (9, 0.25) {};
%		\node [style=Z] (33) at (9, -0.75) {};
%		\node [style=X] (34) at (9.5, -1.25) {$1$};
%		\node [style=scalar] (35) at (9.5, -0.5) {$a$};
%		\node [style=Z] (36) at (10, 0.5) {};
%	\end{pgfonlayer}
%	\begin{pgfonlayer}{edgelayer}
%		\draw (31.center) to (33);
%		\draw (32) to (28.center);
%		\draw (36) to (29.center);
%		\draw [in=90, out=-150] (36) to (35);
%		\draw (35) to (34);
%		\draw (30.center) to (36);
%	\end{pgfonlayer}
%\end{tikzpicture}
%=
%\begin{tikzpicture}
%	\begin{pgfonlayer}{nodelayer}
%		\node [style=none] (37) at (11, 1) {};
%		\node [style=none] (38) at (12, 1) {};
%		\node [style=none] (39) at (12, -1.5) {};
%		\node [style=none] (40) at (11, -1.5) {};
%		\node [style=Z] (41) at (11, 0.25) {};
%		\node [style=Z] (42) at (11, -0.75) {};
%		\node [style=X] (43) at (11.5, -1.25) {$1$};
%		\node [style=Z] (45) at (12, -0.25) {};
%		\node [style=scalar] (46) at (12, 0.5) {$a$};
%		\node [style=scalarop] (47) at (12, -1) {$a$};
%	\end{pgfonlayer}
%	\begin{pgfonlayer}{edgelayer}
%		\draw (40.center) to (42);
%		\draw (41) to (37.center);
%		\draw (45) to (46);
%		\draw (46) to (38.center);
%		\draw [in=-150, out=90] (43) to (45);
%		\draw (39.center) to (47);
%		\draw (47) to (45);
%	\end{pgfonlayer}
%\end{tikzpicture}
%=
%\begin{tikzpicture}
%	\begin{pgfonlayer}{nodelayer}
%		\node [style=none] (73) at (17, 0.75) {};
%		\node [style=none] (74) at (17.75, 0.75) {};
%		\node [style=none] (75) at (17.75, -2) {};
%		\node [style=none] (76) at (17, -2) {};
%		\node [style=Z] (77) at (17, 0.25) {};
%		\node [style=Z] (78) at (17, -0.5) {};
%		\node [style=X] (79) at (17.75, 0.25) {$a$};
%		\node [style=scalarop] (80) at (17.75, -1.5) {$a$};
%		\node [style=X] (83) at (17.75, -0.5) {$1$};
%		\node [style=s] (84) at (17.75, -1) {};
%	\end{pgfonlayer}
%	\begin{pgfonlayer}{edgelayer}
%		\draw (76.center) to (78);
%		\draw (77) to (73.center);
%		\draw [in=-90, out=90] (75.center) to (80);
%		\draw (79) to (74.center);
%		\draw (84) to (83);
%		\draw [in=-90, out=90] (80) to (84);
%	\end{pgfonlayer}
%\end{tikzpicture}
%=
%\begin{tikzpicture}
%	\begin{pgfonlayer}{nodelayer}
%		\node [style=none] (16) at (19.25, 1.25) {};
%		\node [style=none] (17) at (20.25, 1.25) {};
%		\node [style=none] (18) at (20.25, -1.5) {};
%		\node [style=none] (19) at (19.25, -1.5) {};
%		\node [style=Z] (20) at (19.25, 0.25) {};
%		\node [style=Z] (21) at (19.25, -0.5) {};
%		\node [style=X] (22) at (20.25, 0.25) {$1$};
%		\node [style=X] (23) at (20.25, -0.5) {$1$};
%		\node [style=scalarop] (24) at (20.25, -1) {$-a$};
%		\node [style=scalar] (25) at (19.25, -1) {$-a$};
%		\node [style=scalar] (26) at (20.25, 0.75) {$a$};
%		\node [style=scalarop] (27) at (19.25, 0.75) {$a$};
%	\end{pgfonlayer}
%	\begin{pgfonlayer}{edgelayer}
%		\draw (18.center) to (24);
%		\draw (24) to (23);
%		\draw (21) to (25);
%		\draw (19.center) to (25);
%		\draw (20) to (27);
%		\draw (27) to (16.center);
%		\draw (22) to (26);
%		\draw (26) to (17.center);
%	\end{pgfonlayer}
%\end{tikzpicture}
%=
%\begin{tikzpicture}
%	\begin{pgfonlayer}{nodelayer}
%		\node [style=none] (602) at (268.25, 1.75) {};
%		\node [style=none] (603) at (269, 1.75) {};
%		\node [style=none] (604) at (269, -1) {};
%		\node [style=none] (605) at (267.25, -1) {};
%		\node [style=Z] (606) at (268.25, 0.75) {};
%		\node [style=Z] (607) at (267.75, 0.25) {};
%		\node [style=X] (608) at (269, 0.75) {$1$};
%		\node [style=X] (609) at (269.5, 0.25) {};
%		\node [style=scalarop] (610) at (269, -0.5) {$-a$};
%		\node [style=scalar] (611) at (267.25, -0.5) {$-a$};
%		\node [style=scalar] (612) at (269, 1.25) {$a$};
%		\node [style=scalarop] (613) at (268.25, 1.25) {$a$};
%		\node [style=Z] (614) at (268.25, -0.5) {};
%		\node [style=X] (615) at (270, -0.5) {$1$};
%	\end{pgfonlayer}
%	\begin{pgfonlayer}{edgelayer}
%		\draw (604.center) to (610);
%		\draw [in=-150, out=90] (610) to (609);
%		\draw [in=90, out=-150] (607) to (611);
%		\draw (605.center) to (611);
%		\draw (606) to (613);
%		\draw (613) to (602.center);
%		\draw (608) to (612);
%		\draw (612) to (603.center);
%		\draw [in=-30, out=90] (615) to (609);
%		\draw [in=-30, out=90] (614) to (607);
%	\end{pgfonlayer}
%\end{tikzpicture}
%$$
%\end{example}
%Note that these generators do not generate the whole category of Lagrangian relations; for instance, the coefficients are required to be non-negative.




\section{Affine coisotropic relations and stabilizer codes}
\label{sec:coisotrel}
\label{sec:coisot}


In this section we show that by only requiring that the morphisms are affine {\em coisotropic} subspaces   (subspaces $V$ so that $V^\omega \subseteq V$) instead of affine Lagrangian subspaces (where $V^\omega= V$), we can capture the maximally mixed state/discarding; with which we can recover state preparation and measurement compositionally.


\begin{theorem}[Relational purification]~\\
The prop $\Isot\Rel_k$ of isotropic relations is generated by adding the doubled zero relation to the image of the forgetful functor $\Lag\Rel_k\to\LinRel_k$, ie. the following generator in $\LinRel_k$:
$
\begin{tikzpicture}[scale=-1]
	\begin{pgfonlayer}{nodelayer}
		\node [style=X] (0) at (0.5, 0.5) {};
		\node [style=X] (1) at (1, 0.5) {};
		\node [style=none] (2) at (0.5, 0) {};
		\node [style=none] (3) at (1, 0) {};
	\end{pgfonlayer}
	\begin{pgfonlayer}{edgelayer}
		\draw (1) to (3.center);
		\draw (0) to (2.center);
	\end{pgfonlayer}
\end{tikzpicture}
$
\end{theorem}
\begin{proof}
This generator is an isotropic subspace of $(k^{2n},\omega)$ since:
$$
\left(
\begin{tikzpicture}
	\begin{pgfonlayer}{nodelayer}
		\node [style=X] (0) at (0.5, 0.5) {};
		\node [style=X] (1) at (1, 0.5) {};
		\node [style=none] (2) at (0.5, 0) {};
		\node [style=none] (3) at (1, 0) {};
	\end{pgfonlayer}
	\begin{pgfonlayer}{edgelayer}
		\draw (1) to (3.center);
		\draw (0) to (2.center);
	\end{pgfonlayer}
\end{tikzpicture}
\right)^\omega
=
\begin{tikzpicture}
	\begin{pgfonlayer}{nodelayer}
		\node [style=Z] (0) at (0.5, 0.5) {};
		\node [style=Z] (1) at (1, 0.5) {};
		\node [style=none] (2) at (0.5, 0) {};
		\node [style=none] (3) at (1, 0) {};
		\node [style=s] (4) at (1, 0) {};
		\node [style=none] (5) at (1, -0.75) {};
		\node [style=none] (7) at (0.5, -0.75) {};
	\end{pgfonlayer}
	\begin{pgfonlayer}{edgelayer}
		\draw (1) to (3.center);
		\draw (0) to (2.center);
		\draw [in=270, out=90] (7.center) to (4.center);
		\draw [in=90, out=-90] (2.center) to (5.center);
	\end{pgfonlayer}
\end{tikzpicture}
=
\begin{tikzpicture}
	\begin{pgfonlayer}{nodelayer}
		\node [style=Z] (0) at (0.5, 0.5) {};
		\node [style=Z] (1) at (1, 0.5) {};
		\node [style=none] (2) at (0.5, 0) {};
		\node [style=none] (3) at (1, 0) {};
	\end{pgfonlayer}
	\begin{pgfonlayer}{edgelayer}
		\draw (1) to (3.center);
		\draw (0) to (2.center);
	\end{pgfonlayer}
\end{tikzpicture}
\supset
\begin{tikzpicture}
	\begin{pgfonlayer}{nodelayer}
		\node [style=X] (0) at (0.5, 0.5) {};
		\node [style=X] (1) at (1, 0.5) {};
		\node [style=none] (2) at (0.5, 0) {};
		\node [style=none] (3) at (1, 0) {};
	\end{pgfonlayer}
	\begin{pgfonlayer}{edgelayer}
		\draw (1) to (3.center);
		\draw (0) to (2.center);
	\end{pgfonlayer}
\end{tikzpicture}
$$



Suppose that we have an isotropic subspace $V$ of $(k^{2n},\omega)$ with dimension $n-1$. 

By applying Fourier transforms, we obtain a symplectomorphic subspace generated by a matrix whose pivots are all in the $Z$ block.  Therefore, we can row reduce this matrix to obtain one of the following form:

$$
\left[\begin{array}{cc|cc}
I_{n-1} & Z_B & X_A & X_B 
\end{array}\right]
$$

By applying controlled-{\cal X} gates from the first $n-1$ wires to the last wire, we obtain an isotropic subspace $V'\cong V$ generated by a matrix of the following form:


$$
\left[\begin{array}{cc|cc}
I_{n-1} & 0 & X_A' & X_B' 
\end{array}\right]
$$

Since all of the rows of this subspace are orthogonal with respect to the symplectic form, we have:

\begin{align*}
0 &=
\left[\begin{array}{cc|cc}
I_{n-1} & 0 & X_A' & X_B' 
\end{array}\right]
\omega
\left[\begin{array}{cc|cc}
I_{n-1} & 0 & X_A' & X_B' 
\end{array}\right]^T\\
&=
\left[\begin{array}{cc|cc}
I_{n-1} & 0 & X_A' & X_B' 
\end{array}\right]
\left[\begin{array}{cc|cc}
 -X_A' & -X_B'  & I_{n-1} & 0
\end{array}\right]^T\\
&=
I_{n-1}(-X_A')^T +  0( -X_B' )^T +X_A'I_{n-1} + X_B' 0 \\
&=
(-X_A')^T +X_A'
\end{align*}
So that  $X_A'=(X_A')^T$ is a symmetric matrix.

Therefore, the following matrix generates a graph state, and thus a Lagrangian subspace of $k^{2(n+1)}$:
$$
\left[\begin{array}{ccc|ccc}
I_{n-1} & 0    & 0 & X_A'       & X_B' & 0\\
0           & 1 & 0 & (X_B')^T & 0     & 1 \\
0           & 0    & 1  & 0            & 1 & 0
\end{array}\right]
$$
Let $W$ be the Lagrangian subspace generated by this matrix.  Then
$$
\begin{tikzpicture}
	\begin{pgfonlayer}{nodelayer}
		\node [style=X] (0) at (0.5, 0.5) {};
		\node [style=X] (1) at (1.5, 0.5) {};
		\node [style=map] (8) at (0.75, -0.5) {$W$};
		\node [style=none] (9) at (1, 0.5) {};
		\node [style=none] (10) at (0, 0.5) {};
		\node [style=none] (11) at (0, 1) {};
		\node [style=none] (12) at (1, 1) {};
	\end{pgfonlayer}
	\begin{pgfonlayer}{edgelayer}
		\draw [bend left, looseness=0.75] (8) to (10.center);
		\draw [bend left=15] (8) to (0);
		\draw [bend right=15] (8) to (9.center);
		\draw [bend right, looseness=0.75] (8) to (1);
		\draw (11.center) to (10.center);
		\draw (9.center) to (12.center);
	\end{pgfonlayer}
\end{tikzpicture}
=
\begin{tikzpicture}
	\begin{pgfonlayer}{nodelayer}
		\node [style=map] (15) at (3.25, -0.5) {$V'$};
		\node [style=none] (16) at (3.5, 0.25) {};
		\node [style=none] (17) at (3, 0.25) {};
	\end{pgfonlayer}
	\begin{pgfonlayer}{edgelayer}
		\draw [bend left=15, looseness=0.75] (15) to (17.center);
		\draw [bend right=15, looseness=0.75] (15) to (16.center);
	\end{pgfonlayer}
\end{tikzpicture}
$$
This follows because composing $W$ with the cozero maps on the last wire of the $X$ and $Z$ blocks picks out the rows where the last entries of the  $Z$ and $X$ blocks are both postselected to be $0$; that is, those of the generator matrix of $V'$. Then by applying the inverse controlled-X and inverse Fourier transform to $V'$ we get back $V'$ again.  This yields a Lagrangian dilation of $V$.

Suppose that we have an isotropic subspace $V$ of $(k^{2n},\omega)$ with dimension $n-k$; by induction, $V$ can be purified to a  Lagrangian subspace of $k^{2(n+k)}$.
\end{proof}

Since the symplectic complement reverses the order of inclusion, it extends to an isomorphism $\Co\Isot\Rel_k\cong \Isot\Rel_k$ so that we get a dual purification result:


\begin{corollary}
The prop $\Co\Isot\Rel_k$ of affine coisotropic relations is generated by adding the doubled discard relation to the image of the embedding $\Lag\Rel_k\to\LinRel_k$, ie. the linear relation
$
\begin{tikzpicture}[yscale=-1]
	\begin{pgfonlayer}{nodelayer}
		\node [style=Z] (0) at (0, 0) {};
		\node [style=Z] (1) at (0.5, 0) {};
		\node [style=none] (2) at (0, 0.5) {};
		\node [style=none] (3) at (0.5, 0.5) {};
	\end{pgfonlayer}
	\begin{pgfonlayer}{edgelayer}
		\draw (1.center) to (3.center);
		\draw (0.center) to (2.center);
	\end{pgfonlayer}
\end{tikzpicture}
$

\end{corollary}


From the same argument in Lemma \ref{lem:alr} that yields $\Aff\Lag\Rel_k$ from $\Lag\Rel_k$:

\begin{lemma}
The props $\Aff\Isot\Rel_k$ and $\Aff\Co\Isot\Rel_k$ are generated by adding the generator $X$ to $\Isot\Rel_k$ and $\Co\Isot\Rel_k$ respectively seen as categories of affine relations with trivial affine shift.
\end{lemma}


\begin{remark}
\label{rem:xdisc}
Unlike in the linear case, these two props are not isomorphic, as the doubled discard and  doubled cozero maps interact differently with the $X$ gate.  For example:

$$
\begin{tikzpicture}
	\begin{pgfonlayer}{nodelayer}
		\node [style=X] (0) at (1.75, -0.75) {$1$};
		\node [style=Z] (3) at (1.75, 0) {};
		\node [style=Z] (4) at (0.75, 0) {};
		\node [style=none] (9) at (1.75, -1.5) {};
		\node [style=none] (10) at (0.75, -1.5) {};
	\end{pgfonlayer}
	\begin{pgfonlayer}{edgelayer}
		\draw (0) to (9.center);
		\draw (0) to (3);
		\draw (10.center) to (4);
	\end{pgfonlayer}
\end{tikzpicture}
=
\begin{tikzpicture}
	\begin{pgfonlayer}{nodelayer}
		\node [style=Z] (3) at (1.75, 0) {};
		\node [style=Z] (4) at (0.75, 0) {};
		\node [style=none] (9) at (1.75, -1.5) {};
		\node [style=none] (10) at (0.75, -1.5) {};
	\end{pgfonlayer}
	\begin{pgfonlayer}{edgelayer}
		\draw (10.center) to (4);
		\draw (9.center) to (3);
	\end{pgfonlayer}
\end{tikzpicture}
\hspace*{.5cm}
\text{but}
\hspace*{.5cm}
\begin{tikzpicture}
	\begin{pgfonlayer}{nodelayer}
		\node [style=X] (0) at (1.75, -0.75) {$1$};
		\node [style=X] (3) at (1.75, 0) {};
		\node [style=X] (4) at (0.75, 0) {};
		\node [style=none] (9) at (1.75, -1.5) {};
		\node [style=none] (10) at (0.75, -1.5) {};
	\end{pgfonlayer}
	\begin{pgfonlayer}{edgelayer}
		\draw (0) to (9.center);
		\draw (0) to (3);
		\draw (10.center) to (4);
	\end{pgfonlayer}
\end{tikzpicture}
\neq
\begin{tikzpicture}
	\begin{pgfonlayer}{nodelayer}
		\node [style=X] (3) at (1.75, 0) {};
		\node [style=X] (4) at (0.75, 0) {};
		\node [style=none] (9) at (1.75, -1.5) {};
		\node [style=none] (10) at (0.75, -1.5) {};
	\end{pgfonlayer}
	\begin{pgfonlayer}{edgelayer}
		\draw (10.center) to (4);
		\draw (9.center) to (3);
	\end{pgfonlayer}
\end{tikzpicture}
$$

\end{remark}

We extend the notion of a symplectic stabilizer group to the mixed setting:

\begin{definition}
Given some state $f:0\to n$ in $\CPM(\Aff\Lag\Rel_k)$ the (mixed)  {\bf symplectic stabilizer group} of $f$ is the (not necessarily maximal) subgroup of the symplectic Weyl group group generated by the Weyl operators $a \in P_k^n$ so that:

$$
\begin{tikzpicture}
	\begin{pgfonlayer}{nodelayer}
		\node [style=none] (0) at (2.25, 3.25) {};
		\node [style=none] (1) at (3.75, 3.25) {};
		\node [style=map] (2) at (3, 2.25) {$f$};
		\node [style=map] (3) at (3.75, 3.25) {$a$};
		\node [style=none] (4) at (3.75, 3.75) {};
		\node [style=none] (5) at (2.25, 3.75) {};
	\end{pgfonlayer}
	\begin{pgfonlayer}{edgelayer}
		\draw [in=-90, out=45] (2) to (1.center);
		\draw [in=-90, out=135] (2) to (0.center);
		\draw (0.center) to (5.center);
		\draw (3) to (4.center);
	\end{pgfonlayer}
\end{tikzpicture}
=
\begin{tikzpicture}
	\begin{pgfonlayer}{nodelayer}
		\node [style=map] (0) at (4, 3.25) {$a$};
		\node [style=none] (1) at (2.5, 3.25) {};
		\node [style=Z] (2) at (3, 2.75) {};
		\node [style=map] (3) at (3.75, 2) {$g$};
		\node [style=map] (4) at (2.25, 2) {$\bar g$};
		\node [style=none] (5) at (2.5, 3.25) {};
		\node [style=none] (6) at (4, 3.75) {};
		\node [style=none] (7) at (2.5, 3.75) {};
	\end{pgfonlayer}
	\begin{pgfonlayer}{edgelayer}
		\draw [bend right] (3) to (2);
		\draw [in=120, out=180, looseness=1.50] (2) to (4);
		\draw [in=-90, out=60] (4) to (1.center);
		\draw [in=-90, out=60] (3) to (0);
		\draw (5.center) to (7.center);
		\draw (0) to (6.center);
	\end{pgfonlayer}
\end{tikzpicture}
=
\begin{tikzpicture}
	\begin{pgfonlayer}{nodelayer}
		\node [style=none] (0) at (4, 3.25) {};
		\node [style=none] (1) at (2.5, 3.25) {};
		\node [style=Z] (2) at (3, 2.75) {};
		\node [style=map] (3) at (3.75, 2) {$g$};
		\node [style=map] (4) at (2.25, 2) {$\bar g$};
	\end{pgfonlayer}
	\begin{pgfonlayer}{edgelayer}
		\draw [bend right] (3) to (2);
		\draw [in=120, out=180, looseness=1.50] (2) to (4);
		\draw [in=-90, out=60] (4) to (1.center);
		\draw [in=-90, out=60] (3) to (0.center);
	\end{pgfonlayer}
\end{tikzpicture}
=
\begin{tikzpicture}
	\begin{pgfonlayer}{nodelayer}
		\node [style=none] (19) at (11.75, 3.25) {};
		\node [style=none] (20) at (10.25, 3.25) {};
		\node [style=map] (21) at (11, 2.25) {$f$};
	\end{pgfonlayer}
	\begin{pgfonlayer}{edgelayer}
		\draw [in=-90, out=135] (21) to (20.center);
		\draw [in=-90, out=45] (21) to (19.center);
	\end{pgfonlayer}
\end{tikzpicture}
$$


\end{definition}


This is needed to prove the essential uniqueness of purification for $\CPM(\Aff\Lag\Rel_k)$, generalizing the case for stabilizers:

\begin{proposition}[Essential uniqueness of quantum purification]~\\
\label{prop:uniqueness}
States in $\CPM(\Aff\Lag\Rel_k)$ are uniquely determined by their stabilizer groups.
\end{proposition}
\begin{proof}

Take dilations of two parallel maps in $\CPM(\Aff\Lag\Rel_k)$:

$$
(m,f:0\to n+m) \hspace*{.2cm}  
\text{of} \hspace*{.2cm} \hat f:0\to m 
\hspace*{.5cm}
\text{and}
\hspace*{.5cm}
(\ell, g:0\to n+\ell)
\hspace*{.2cm}  \text{of} \hspace*{.2cm}
\hat g:0\to m
$$ 

Take  $\hat f$ and $\hat g$ to have the same stabilizer groups, and without loss of generality take $m\geq \ell$. 

Consider the unitary  $u:m\to m$  and isometry $v:\ell\to m$, which perform Gaussian elimination on the ancillary space $m$ and $\ell$ of $f$ and $g$.  Compose $u$ and $v$ on the ancillary spaces of $f$ and $g$, respectively; and then regard them as pure states.  These pure states both have the same stabilizer groups.  Therefore they span the same affine subspace:

$$
\begin{tikzpicture}
	\begin{pgfonlayer}{nodelayer}
		\node [style=map] (110) at (64.25, 2.25) {$f$};
		\node [style=none] (111) at (65, 4.25) {};
		\node [style=none] (112) at (64, 4.25) {};
		\node [style=map] (113) at (64.25, 3.25) {$u$};
		\node [style=none] (114) at (64.5, 4.25) {};
		\node [style=none] (115) at (63.5, 4.25) {};
	\end{pgfonlayer}
	\begin{pgfonlayer}{edgelayer}
		\draw [in=45, out=-90] (111.center) to (110);
		\draw [in=-90, out=120] (110) to (112.center);
		\draw [in=-75, out=75] (110) to (113);
		\draw [in=-90, out=60] (113) to (114.center);
		\draw [in=-135, out=150, looseness=1.25] (110) to (113);
		\draw [in=105, out=-90] (115.center) to (113);
	\end{pgfonlayer}
\end{tikzpicture}
=
\begin{tikzpicture}
	\begin{pgfonlayer}{nodelayer}
		\node [style=map] (116) at (66.75, 2.25) {$g$};
		\node [style=none] (117) at (67.5, 4.25) {};
		\node [style=none] (118) at (66.5, 4.25) {};
		\node [style=map] (119) at (66.75, 3.25) {$v$};
		\node [style=none] (120) at (67, 4.25) {};
		\node [style=none] (121) at (66, 4.25) {};
	\end{pgfonlayer}
	\begin{pgfonlayer}{edgelayer}
		\draw [in=45, out=-90] (117.center) to (116);
		\draw [in=-90, out=120] (116) to (118.center);
		\draw [in=-75, out=75] (116) to (119);
		\draw [in=-90, out=60] (119) to (120.center);
		\draw [in=-135, out=150, looseness=1.25] (116) to (119);
		\draw [in=105, out=-90] (121.center) to (119);
	\end{pgfonlayer}
\end{tikzpicture}
$$


Therefore:


$$
\begin{tikzpicture}
	\begin{pgfonlayer}{nodelayer}
		\node [style=map] (0) at (3.75, 2.25) {$f$};
		\node [style=none] (1) at (4.75, 4.25) {};
		\node [style=none] (2) at (4, 4.25) {};
		\node [style=map] (3) at (2, 2.25) {$\bar f$};
		\node [style=Z] (4) at (3.5, 3.75) {};
		\node [style=X] (5) at (1.75, 3.75) {};
		\node [style=none] (6) at (4.75, 4.25) {};
		\node [style=none] (7) at (4, 4.25) {};
		\node [style=none] (8) at (3, 4.25) {};
		\node [style=none] (9) at (2.25, 4.25) {};
	\end{pgfonlayer}
	\begin{pgfonlayer}{edgelayer}
		\draw [in=45, out=-90] (1.center) to (0);
		\draw [in=-90, out=120] (0) to (2.center);
		\draw [in=-15, out=75, looseness=1.25] (0) to (4);
		\draw [in=-135, out=150, looseness=1.50] (3) to (5);
		\draw [in=-90, out=105] (3) to (9.center);
		\draw [in=-90, out=30, looseness=0.75] (3) to (8.center);
		\draw [in=-165, out=75] (3) to (4);
		\draw [in=-15, out=150, looseness=1.25] (0) to (5);
	\end{pgfonlayer}
\end{tikzpicture}
=
\begin{tikzpicture}
	\begin{pgfonlayer}{nodelayer}
		\node [style=map] (0) at (7.75, 2.25) {$f$};
		\node [style=none] (1) at (9, 3.5) {};
		\node [style=none] (2) at (8.5, 3.5) {};
		\node [style=map] (3) at (6, 2.25) {$\bar f$};
		\node [style=Z] (4) at (8, 4.25) {};
		\node [style=X] (5) at (5.75, 4.25) {};
		\node [style=none] (6) at (6.75, 4.75) {};
		\node [style=none] (7) at (6.25, 4.75) {};
		\node [style=map] (8) at (8, 3.5) {$u$};
		\node [style=map] (9) at (5.5, 3.25) {$\bar u$};
		\node [style=none] (10) at (9, 4.75) {};
		\node [style=none] (11) at (8.5, 4.75) {};
	\end{pgfonlayer}
	\begin{pgfonlayer}{edgelayer}
		\draw [in=45, out=-90] (1.center) to (0);
		\draw [in=-90, out=120] (0) to (2.center);
		\draw [in=-90, out=105] (3) to (7.center);
		\draw [in=-90, out=30, looseness=0.75] (3) to (6.center);
		\draw [in=-15, out=75, looseness=1.25] (3) to (9);
		\draw [bend right, looseness=0.75] (9) to (3);
		\draw [bend left=45] (9) to (5);
		\draw [in=180, out=60, looseness=0.75] (9) to (4);
		\draw [in=120, out=0, looseness=0.50] (5) to (8);
		\draw [bend left=15] (8) to (0);
		\draw [in=-150, out=135, looseness=1.25] (0) to (8);
		\draw [bend right] (8) to (4);
		\draw (2.center) to (11.center);
		\draw (1.center) to (10.center);
	\end{pgfonlayer}
\end{tikzpicture}
=
\begin{tikzpicture}
	\begin{pgfonlayer}{nodelayer}
		\node [style=map] (0) at (7.75, 2.25) {$g$};
		\node [style=none] (1) at (9, 3.5) {};
		\node [style=none] (2) at (8.5, 3.5) {};
		\node [style=map] (3) at (6, 2.25) {$\bar g$};
		\node [style=Z] (4) at (8, 4.25) {};
		\node [style=X] (5) at (5.75, 4.25) {};
		\node [style=none] (6) at (6.75, 4.75) {};
		\node [style=none] (7) at (6.25, 4.75) {};
		\node [style=map] (8) at (8, 3.5) {$v$};
		\node [style=map] (9) at (5.5, 3.25) {$\bar v$};
		\node [style=none] (10) at (9, 4.75) {};
		\node [style=none] (11) at (8.5, 4.75) {};
	\end{pgfonlayer}
	\begin{pgfonlayer}{edgelayer}
		\draw [in=45, out=-90] (1.center) to (0);
		\draw [in=-90, out=120] (0) to (2.center);
		\draw [in=-90, out=105] (3) to (7.center);
		\draw [in=-90, out=30, looseness=0.75] (3) to (6.center);
		\draw [in=-15, out=75, looseness=1.25] (3) to (9);
		\draw [bend right, looseness=0.75] (9) to (3);
		\draw [bend left=45] (9) to (5);
		\draw [in=180, out=60, looseness=0.75] (9) to (4);
		\draw [in=120, out=0, looseness=0.50] (5) to (8);
		\draw [bend left=15] (8) to (0);
		\draw [in=-150, out=135, looseness=1.25] (0) to (8);
		\draw [bend right] (8) to (4);
		\draw (2.center) to (11.center);
		\draw (1.center) to (10.center);
	\end{pgfonlayer}
\end{tikzpicture}
=
\begin{tikzpicture}
	\begin{pgfonlayer}{nodelayer}
		\node [style=map] (0) at (3.75, 2.25) {$g$};
		\node [style=none] (1) at (4.75, 4.25) {};
		\node [style=none] (2) at (4, 4.25) {};
		\node [style=map] (3) at (2, 2.25) {$\bar g$};
		\node [style=Z] (4) at (3.5, 3.75) {};
		\node [style=X] (5) at (1.75, 3.75) {};
		\node [style=none] (6) at (4.75, 4.25) {};
		\node [style=none] (7) at (4, 4.25) {};
		\node [style=none] (8) at (3, 4.25) {};
		\node [style=none] (9) at (2.25, 4.25) {};
	\end{pgfonlayer}
	\begin{pgfonlayer}{edgelayer}
		\draw [in=45, out=-90] (1.center) to (0);
		\draw [in=-90, out=120] (0) to (2.center);
		\draw [in=-15, out=75, looseness=1.25] (0) to (4);
		\draw [in=-135, out=150, looseness=1.50] (3) to (5);
		\draw [in=-90, out=105] (3) to (9.center);
		\draw [in=-90, out=30, looseness=0.75] (3) to (8.center);
		\draw [in=-165, out=75] (3) to (4);
		\draw [in=-15, out=150, looseness=1.25] (0) to (5);
	\end{pgfonlayer}
\end{tikzpicture}
$$

\end{proof}




\begin{theorem}[Essential uniqueness of relational purification]~\\
\label{them:dilation}
 $\CPM(\Aff\Lag\Rel_k) \cong \Aff\Co\Isot\Rel_k$
\end{theorem}


\begin{proof}

Because both categories are compact closed, it suffices to exhibit a functorial bijection between the states of both categories.
Consider the map $\CPM(\Aff\Lag\Rel_k) \cong \Aff\Co\Isot\Rel_k$, sending:
$$
\begin{tikzpicture}
	\begin{pgfonlayer}{nodelayer}
		\node [style=map] (0) at (3.75, 2.25) {$f$};
		\node [style=none] (1) at (4.75, 4.25) {};
		\node [style=none] (2) at (4, 4.25) {};
		\node [style=map] (3) at (2, 2.25) {$\bar f$};
		\node [style=Z] (4) at (3.5, 3.75) {};
		\node [style=X] (5) at (1.75, 3.75) {};
		\node [style=none] (6) at (4.75, 4.25) {};
		\node [style=none] (7) at (4, 4.25) {};
		\node [style=none] (8) at (3, 4.25) {};
		\node [style=none] (9) at (2.25, 4.25) {};
	\end{pgfonlayer}
	\begin{pgfonlayer}{edgelayer}
		\draw [in=45, out=-90] (1.center) to (0);
		\draw [in=-90, out=120] (0) to (2.center);
		\draw [in=-15, out=75, looseness=1.25] (0) to (4);
		\draw [in=-135, out=150, looseness=1.50] (3) to (5);
		\draw [in=-90, out=105] (3) to (9.center);
		\draw [in=-90, out=30, looseness=0.75] (3) to (8.center);
		\draw [in=-165, out=75] (3) to (4);
		\draw [in=-15, out=150, looseness=1.25] (0) to (5);
	\end{pgfonlayer}
\end{tikzpicture}
\mapsto
\begin{tikzpicture}
	\begin{pgfonlayer}{nodelayer}
		\node [style=map] (122) at (69.25, 2.25) {$f$};
		\node [style=none] (123) at (70, 4.25) {};
		\node [style=none] (124) at (69, 4.25) {};
		\node [style=Z] (125) at (69.5, 3.5) {};
		\node [style=Z] (126) at (68.5, 3.5) {};
	\end{pgfonlayer}
	\begin{pgfonlayer}{edgelayer}
		\draw [in=45, out=-90] (123.center) to (122);
		\draw [in=-90, out=120] (122) to (124.center);
		\draw [in=-90, out=75] (122) to (125);
		\draw [in=-90, out=135] (122) to (126);
	\end{pgfonlayer}
\end{tikzpicture}
$$
By Proposition \ref{prop:uniqueness}, this mapping is a full functor. For faithfulness, take maps $\hat f$ and $\hat g$ sent to the same affine coisotropic relation with dilations $(m,f:0\to n+ m)$ and  $(\ell,g:0\to n+ \ell )$ where without loss of generality $m\geq \ell$.
Then there is a unitary  $u:m\to m$  and isometry $v:\ell\to m$, which perform Gaussian elimination on the ancillary systems of $f$ and $g$ so that:

$$
\begin{tikzpicture}
	\begin{pgfonlayer}{nodelayer}
		\node [style=map] (110) at (64.25, 2.25) {$f$};
		\node [style=none] (111) at (65, 4.25) {};
		\node [style=none] (112) at (64, 4.25) {};
		\node [style=map] (113) at (64.25, 3.25) {$u$};
		\node [style=none] (114) at (64.5, 4.25) {};
		\node [style=none] (115) at (63.5, 4.25) {};
	\end{pgfonlayer}
	\begin{pgfonlayer}{edgelayer}
		\draw [in=45, out=-90] (111.center) to (110);
		\draw [in=-90, out=120] (110) to (112.center);
		\draw [in=-75, out=75] (110) to (113);
		\draw [in=-90, out=60] (113) to (114.center);
		\draw [in=-135, out=150, looseness=1.25] (110) to (113);
		\draw [in=105, out=-90] (115.center) to (113);
	\end{pgfonlayer}
\end{tikzpicture}
=
\begin{tikzpicture}
	\begin{pgfonlayer}{nodelayer}
		\node [style=map] (116) at (66.75, 2.25) {$g$};
		\node [style=none] (117) at (67.5, 4.25) {};
		\node [style=none] (118) at (66.5, 4.25) {};
		\node [style=map] (119) at (66.75, 3.25) {$v$};
		\node [style=none] (120) at (67, 4.25) {};
		\node [style=none] (121) at (66, 4.25) {};
	\end{pgfonlayer}
	\begin{pgfonlayer}{edgelayer}
		\draw [in=45, out=-90] (117.center) to (116);
		\draw [in=-90, out=120] (116) to (118.center);
		\draw [in=-75, out=75] (116) to (119);
		\draw [in=-90, out=60] (119) to (120.center);
		\draw [in=-135, out=150, looseness=1.25] (116) to (119);
		\draw [in=105, out=-90] (121.center) to (119);
	\end{pgfonlayer}
\end{tikzpicture}
$$

So that, as before, they are both dilations of $\hat f = \hat g$.
\end{proof}

\begin{corollary}
\label{cor:stabcode}
For odd prime $p$, $\Aff\Co\Isot\Rel_{\F_p}\cong \CPM(\Aff\Lag\Rel_{\F_p})\cong \CPM(\Stab_p)$.  That is, adding the discard relation to $\Aff\Lag\Rel_{\F_p}$ gives a semantics for {\bf mixed stabilizer circuits/stabilizer codes}, graphically:
$$
\left\llbracket \
\begin{tikzpicture}[yscale=-1]
	\begin{pgfonlayer}{nodelayer}
		\node [style=none] (0) at (0.25, 0) {};
		\node [ground] (1) at (0.25, -0.5) {};
	\end{pgfonlayer}
	\begin{pgfonlayer}{edgelayer}
		\draw (1) to (0.center);
	\end{pgfonlayer}
\end{tikzpicture}\
\right\rrbracket
=
\begin{tikzpicture}
	\begin{pgfonlayer}{nodelayer}
		\node [style=Z] (2) at (4, 0) {};
		\node [style=Z] (3) at (4.5, 0) {};
		\node [style=none] (4) at (4, -1) {};
		\node [style=none] (5) at (4.5, -1) {};
	\end{pgfonlayer}
	\begin{pgfonlayer}{edgelayer}
		\draw (4.center) to (2);
		\draw (3) to (5.center);
	\end{pgfonlayer}
\end{tikzpicture}
=
\left\{ 
\left(
\begin{pmatrix}
z\\x
\end{pmatrix},
*
\right)
:\forall z,x \in \F_p
\right\}
$$
\end{corollary}

This formalizes the relationship between mixed stabilizer circuits and stabilizer tableaux with not-necessarily-full rank in  a compositional way. This presentation is similar in spirit to the way in which adding quantum discarding can often be presented by adding a generator which freely discarding the isometries, formalized by the discard construction \cite{disc}. Although in our case the quantum discarding is interpreted as the literal discard relation, therefore our semantics is still in affine relations.



It was already known that stabilizer codes are in bijection with  affine isotropic subspaces, for example \cite[Sec. A]{gross}.  Indeed any affine coisotropic subspace is canonically associated to an affine isotropic subspace by  taking the symplectic complement of the linear component of the affine subspace. However, as noted in Remark  \ref{rem:xdisc}, $\Aff\Isot\Rel_{\F_p} \not \cong \Aff\Co\Isot\Rel_{\F_p}$, so their compositions as affine relations are different. The interpretation of the doubled zero postselection as the quantum discard map is not sound with respect to relational composition.   



We also can do the same for Spekkens' toy model:

\begin{corollary}
$\Aff\Co\Isot\Rel_{\F_2}\cong \CPM(\Aff\Lag\Rel_{\F_2})$ is Spekkens' toy model with mixed states.
\end{corollary}

Just as Spekkens' toy model is an epistemic restricted toy theory of quantum circuits; by dualizing things, we also have that $\Aff\Co\Isot\Rel_{\F_p}$ is an empistemically restricted toy theory of quantum circuits; albeit the epistemic restriction is going in the wrong direction to capture stabilizer quantum mechanics.


{\em Absolutely remarkably}, and seemingly out of nowhere, the quopit stabilizer codes with trivial affine phase (no Pauli gates) can be expressed, modulo invertible phase, in terms of an iterated $\CPM$ construction with respect to the orthogonal complement at the inner level, and the complex conjugation at the outer level:
\begin{corollary}
Given a prime $p$ and $k=\F_p$ or $k=\mathbb{Q}$
$$\Isot\Rel_{k}\cong\Co\Isot\Rel_{k}\cong \CPM(\CPM(\LinRel_{k},\perp),\bar{(\_)})$$
\end{corollary}
The astounding symmetry involved here begs the question if iterating the $\CPM$ construction more times yields anything physically interesting. Perhaps the work of \cite{CPMho} can shed some light on this question. 

In order to add measurement and state preparation in the symplectic setting, we split the projectors for the $Z$ and $X$ bases:

\begin{definition}
The $X$ and $Z$ projectors are defined as follows in $\Aff\Co\Isot\Rel_{\F_p}$:
$$
p_X:=
\begin{tikzpicture}
	\begin{pgfonlayer}{nodelayer}
		\node [style=X] (0) at (0.5, -0.75) {};
		\node [style=none] (2) at (0.25, 0) {};
		\node [style=none] (4) at (1.25, 0.5) {};
		\node [style=Z] (5) at (0.75, 0) {};
		\node [style=Z] (6) at (1.75, 0) {};
		\node [style=Z] (7) at (1.5, -0.75) {};
		\node [style=none] (8) at (0.75, 0) {};
		\node [style=none] (9) at (1.25, 0) {};
		\node [style=none] (10) at (1.75, 0) {};
		\node [style=none] (11) at (0.5, -1.5) {};
		\node [style=none] (13) at (1.5, -1.5) {};
		\node [style=none] (14) at (0.25, 0.5) {};
	\end{pgfonlayer}
	\begin{pgfonlayer}{edgelayer}
		\draw [in=-90, out=120] (7) to (9.center);
		\draw (7) to (13.center);
		\draw [in=60, out=-90] (10.center) to (7);
		\draw [in=60, out=-90] (8.center) to (0);
		\draw [in=-90, out=120] (0) to (2.center);
		\draw (0) to (11.center);
		\draw (9.center) to (4.center);
		\draw (2.center) to (14.center);
	\end{pgfonlayer}
\end{tikzpicture}
=
\begin{tikzpicture}
	\begin{pgfonlayer}{nodelayer}
		\node [style=none] (4) at (1, 0.5) {};
		\node [style=Z] (5) at (0.25, 0) {};
		\node [style=none] (9) at (1, -1.25) {};
		\node [style=none] (11) at (0.25, -1.25) {};
		\node [style=none] (14) at (0.25, 0.5) {};
		\node [style=Z] (15) at (0.25, -0.75) {};
	\end{pgfonlayer}
	\begin{pgfonlayer}{edgelayer}
		\draw (9.center) to (4.center);
		\draw (14.center) to (5);
		\draw (11.center) to (15);
	\end{pgfonlayer}
\end{tikzpicture}
\hspace*{.5cm}
p_Z:=
\begin{tikzpicture}
	\begin{pgfonlayer}{nodelayer}
		\node [style=Z] (0) at (0.5, -0.75) {};
		\node [style=none] (2) at (0.25, 0) {};
		\node [style=none] (4) at (1.25, 0.5) {};
		\node [style=Z] (5) at (0.75, 0) {};
		\node [style=Z] (6) at (1.75, 0) {};
		\node [style=X] (7) at (1.5, -0.75) {};
		\node [style=none] (8) at (0.75, 0) {};
		\node [style=none] (9) at (1.25, 0) {};
		\node [style=none] (10) at (1.75, 0) {};
		\node [style=none] (11) at (0.5, -1.5) {};
		\node [style=none] (13) at (1.5, -1.5) {};
		\node [style=none] (14) at (0.25, 0.5) {};
	\end{pgfonlayer}
	\begin{pgfonlayer}{edgelayer}
		\draw [in=-90, out=120] (7) to (9.center);
		\draw (7) to (13.center);
		\draw [in=60, out=-90] (10.center) to (7);
		\draw [in=60, out=-90] (8.center) to (0);
		\draw [in=-90, out=120] (0) to (2.center);
		\draw (0) to (11.center);
		\draw (9.center) to (4.center);
		\draw (2.center) to (14.center);
	\end{pgfonlayer}
\end{tikzpicture}
=
\begin{tikzpicture}[scale=-1]
	\begin{pgfonlayer}{nodelayer}
		\node [style=none] (4) at (1, 0.5) {};
		\node [style=Z] (5) at (0.25, 0) {};
		\node [style=none] (9) at (1, -1.25) {};
		\node [style=none] (11) at (0.25, -1.25) {};
		\node [style=none] (14) at (0.25, 0.5) {};
		\node [style=Z] (15) at (0.25, -0.75) {};
	\end{pgfonlayer}
	\begin{pgfonlayer}{edgelayer}
		\draw (9.center) to (4.center);
		\draw (14.center) to (5);
		\draw (11.center) to (15);
	\end{pgfonlayer}
\end{tikzpicture}
$$
\end{definition}


The $X$ projector discards and then codiscards the $Z$-gradient: cutting the $Z$ gradient in two so that no information is preserved, while acting trivially on the $X$ gradient.  Dually for the $Z$ projector. 

We will split only one of these projectors for simplicity:

\begin{definition}
Let $\Aff\Co\Isot\Rel_k^M$ denote the two-coloured prop generated by splitting $p_Z$ in $\Aff\Co\Isot\Rel_k$; that is the ${\sf Split}_{\{p_Z^{\otimes n}, 1_n | n \in \N \}}(\Aff\Co\Isot\Rel_k)$.


Let $Q=(1_1,1_1)$ denote the original object and $C=(1_1,p_Z)$ the object obtained by splitting $p_Z$.
\end{definition}

We could have instead split $p_X$, or split both $p_X$ and $p_Z$; however, all three of these multicoloured props are equivalent.  This equivalence is witnessed via the Fourier transform. Indeed this suffices to split all nonzero projectors up to isomorphism because all projectors of the same dimension are isomorphic as affine coisotropic subspaces.  Therefore we can construct a nonzero projector of each possible dimension by composition with $p_X$ and affine symplectomorphisms.  
It is important to remark that the choice of projectors which are split effects the code-distance, because code-distance is basis dependent, and not invariant under equivalence.
We chose not to split the zero projector for the same reason why we did not chose to have it as an object in $\Aff\Rel_k$: for ease of notation.

\begin{remark}
The object $Q$ can be interpreted as a quantum channel and the object $C$ as a classical channel. Or equivalently, $C^{\otimes n}$ is interpreted as the space of logical qubits and  $Q^{\otimes m}$  as the space of physical qubits.
\end{remark}

This category has a nice presentation;  adding the affine relations to  $\Aff\Co\Isot\Rel_k$ obtained by cutting /splitting the $Z$ projector in two:


\begin{theorem}
The full subcategory of $\Aff\Co\Isot\Rel_k^M$ generated by tensor powers of $C$ is isomorphic to $\Aff\Rel_k$.
Therefore $\Aff\Co\Isot\Rel_k^M$ is isomorphic to adding the following linear relations to the image of $\Aff\Co\Isot\Rel_k^M\to \Aff\Co\Isot\Rel_k$ in the way which makes this into a two-coloured prop:
$$
\begin{tikzpicture}[xscale=-1]
	\begin{pgfonlayer}{nodelayer}
		\node [style=none] (3) at (24, 0.5) {};
		\node [style=Z] (4) at (24.75, 0) {};
		\node [style=none] (5) at (24, 0) {};
		\node [style=none] (6) at (24.75, 0.5) {};
		\node [style=none] (7) at (24, -0.5) {};
	\end{pgfonlayer}
	\begin{pgfonlayer}{edgelayer}
		\draw (5.center) to (3.center);
		\draw (6.center) to (4);
		\draw [style=red] (7.center) to (5.center);
	\end{pgfonlayer}
\end{tikzpicture}
\hspace*{.5cm}\text{and}\hspace*{.5cm}
\begin{tikzpicture}[scale=-1]
	\begin{pgfonlayer}{nodelayer}
		\node [style=none] (3) at (24, 0.5) {};
		\node [style=Z] (4) at (24.75, 0) {};
		\node [style=none] (5) at (24, 0) {};
		\node [style=none] (6) at (24.75, 0.5) {};
		\node [style=none] (7) at (24, -0.5) {};
	\end{pgfonlayer}
	\begin{pgfonlayer}{edgelayer}
		\draw (5.center) to (3.center);
		\draw (6.center) to (4);
		\draw [style=red] (7.center) to (5.center);
	\end{pgfonlayer}
\end{tikzpicture}
$$

\end{theorem}

We draw the classical wire in red to indicate the type, (although the colour is just syntactic sugar).

 The classical state ``lives'' on a single wire and the stabilizer state ``lives'' on the doubled wires.
Because of this, the  aforementioned circuits are interpreted in terms of state preparation (which we denote by the box labelled ``$a\mapsto|a\rangle$'') and measurement in the $Z$ basis:

$$
\left\llbracket \
\begin{tikzpicture}
	\begin{pgfonlayer}{nodelayer}
		\node[style=map,fill=white] at (0,0) {$ a \mapsto |a \rangle$};
	\end{pgfonlayer}
	\begin{pgfonlayer}{edgelayer}
		\draw (.1,0) to (.1,-1);
		\draw (-.1,0) to (-.1,-1);
		\draw (0,1) to (0,0);
	\end{pgfonlayer}
\end{tikzpicture}\
\right\rrbracket 
=
\begin{tikzpicture}[xscale=-1]
	\begin{pgfonlayer}{nodelayer}
		\node [style=none] (3) at (24, 0.5) {};
		\node [style=Z] (4) at (24.75, 0) {};
		\node [style=none] (5) at (24, 0) {};
		\node [style=none] (6) at (24.75, 0.5) {};
		\node [style=none] (7) at (24, -0.5) {};
	\end{pgfonlayer}
	\begin{pgfonlayer}{edgelayer}
		\draw (5.center) to (3.center);
		\draw (6.center) to (4);
		\draw [style=red] (7.center) to (5.center);
	\end{pgfonlayer}
\end{tikzpicture}\ ,\hspace*{1cm}
\left\llbracket \
\begin{circuitikz}
\node[meter] (meter) at (0,0) {};
\draw (.1,.5) to (.1,1);
\draw (-.1,.5) to (-.1,1);
\draw (0,-1) to (0,-.5);
\end{circuitikz} \ 
\right\rrbracket 
=
\begin{tikzpicture}[scale=-1]
	\begin{pgfonlayer}{nodelayer}
		\node [style=none] (3) at (24, 0.5) {};
		\node [style=Z] (4) at (24.75, 0) {};
		\node [style=none] (5) at (24, 0) {};
		\node [style=none] (6) at (24.75, 0.5) {};
		\node [style=none] (7) at (24, -0.5) {};
	\end{pgfonlayer}
	\begin{pgfonlayer}{edgelayer}
		\draw (5.center) to (3.center);
		\draw (6.center) to (4);
		\draw [style=red] (7.center) to (5.center);
	\end{pgfonlayer}
\end{tikzpicture}
$$


For example, given any classical dit $x \in \F_p$, to prepare the state $|x\rangle$ is to take the composite:

$$
\left\llbracket \
\begin{tikzpicture}
	\begin{pgfonlayer}{nodelayer}
		\node[style=map,fill=white] at (0,0) {$ a \mapsto |a \rangle$};
		\node at (0,-1.2) {$a$};
	\end{pgfonlayer}
	\begin{pgfonlayer}{edgelayer}
		\draw (.1,0) to (.1,-1);
		\draw (-.1,0) to (-.1,-1);
		\draw (0,1) to (0,0);
	\end{pgfonlayer}
\end{tikzpicture}\
\right\rrbracket 
=
\begin{tikzpicture}
	\begin{pgfonlayer}{nodelayer}
		\node [style=none] (0) at (1.25, 0.75) {};
		\node [style=Z] (1) at (0.5, 0) {};
		\node [style=none] (3) at (0.5, 0.75) {};
		\node [style=X] (4) at (1.25, -0.75) {$x$};
		\node [style=none] (5) at (1.25, 0) {};
	\end{pgfonlayer}
	\begin{pgfonlayer}{edgelayer}
		\draw (3.center) to (1);
		\draw [style=red] (4) to (5.center);
		\draw (5.center) to (0.center);
	\end{pgfonlayer}
\end{tikzpicture}
=
\begin{tikzpicture}
	\begin{pgfonlayer}{nodelayer}
		\node [style=none] (0) at (1.25, 0.5) {};
		\node [style=Z] (1) at (0.5, 0) {};
		\node [style=none] (2) at (1.25, 0) {};
		\node [style=none] (3) at (0.5, 0.5) {};
		\node [style=X] (4) at (1.25, 0) {$x$};
	\end{pgfonlayer}
	\begin{pgfonlayer}{edgelayer}
		\draw (2.center) to (0.center);
		\draw (3.center) to (1);
	\end{pgfonlayer}
\end{tikzpicture}
=
\left\llbracket \
\begin{tikzpicture}
	\begin{pgfonlayer}{nodelayer}
		\node [style=none] (0) at (23, 2) {};
		\node [style=none] (1) at (23, 3) {};
		\node [style=none] (2) at (23, 1.75) {$|a\rangle$};
	\end{pgfonlayer}
	\begin{pgfonlayer}{edgelayer}
		\draw (0.center) to (1.center);
	\end{pgfonlayer}
\end{tikzpicture}\
\right\rrbracket 
$$

The state preparation and discarding in the $Z$ basis are obtained by composition of these morphisms with the Fourier transform; yielding morphisms which discard the $X$ wire instead of the $Z$ wire:

$$
\left\llbracket \
\begin{tikzpicture}
	\begin{pgfonlayer}{nodelayer}
		\node[meter] (meter) at (0,0) {};
		\node[style=map,fill=white] at (0,-1) {$\mathcal{F}^\dag$};
	\end{pgfonlayer}
	\begin{pgfonlayer}{edgelayer}
		\draw (.1,.5) to (.1,1);
		\draw (-.1,.5) to (-.1,1);
		\draw (0,-2) to (0,-.5);
	\end{pgfonlayer} 
\end{tikzpicture}\
\right\rrbracket 
=
\begin{tikzpicture}[yscale=-1]
	\begin{pgfonlayer}{nodelayer}
		\node [style=none] (3) at (24, 0.5) {};
		\node [style=Z] (4) at (24.75, 0) {};
		\node [style=none] (5) at (24, 0) {};
		\node [style=none] (6) at (24.75, 0.5) {};
		\node [style=none] (7) at (24, -0.5) {};
	\end{pgfonlayer}
	\begin{pgfonlayer}{edgelayer}
		\draw (5.center) to (3.center);
		\draw (6.center) to (4);
		\draw [style=red] (7.center) to (5.center);
	\end{pgfonlayer}
\end{tikzpicture}\ ,
\hspace*{.8cm}
\left\llbracket \
\begin{tikzpicture}
	\begin{pgfonlayer}{nodelayer}
		\node[style=map,fill=white] at (0,0) {$ a \mapsto \mathcal{F}|a \rangle$};
	\end{pgfonlayer}
	\begin{pgfonlayer}{edgelayer}
		\draw (.1,0) to (.1,-1);
		\draw (-.1,0) to (-.1,-1);
		\draw (0,1) to (0,0);
	\end{pgfonlayer}
\end{tikzpicture}\
\right\rrbracket 
=
\begin{tikzpicture}
	\begin{pgfonlayer}{nodelayer}
		\node [style=none] (3) at (24, 0.5) {};
		\node [style=Z] (4) at (24.75, 0) {};
		\node [style=none] (5) at (24, 0) {};
		\node [style=none] (6) at (24.75, 0.5) {};
		\node [style=none] (7) at (24, -0.5) {};
	\end{pgfonlayer}
	\begin{pgfonlayer}{edgelayer}
		\draw (5.center) to (3.center);
		\draw (6.center) to (4);
		\draw [style=red] (7.center) to (5.center);
	\end{pgfonlayer}
\end{tikzpicture}
$$

\begin{remark}
Recall that in Lemma \ref{lem:strongcomp}, we used the Hopf rule to show that preparing X basis and then measuring in the Z basis preserves no information.  In the symplectic picture, this result becomes purely topological:

$$
\left\llbracket\
\begin{tikzpicture}
	\begin{pgfonlayer}{nodelayer}
		\node[meter] (meter) at (0,0) {};
		\node[style=map,fill=white] at (0,-1.1) {$\mathcal{F}^\dag$};
		\node[style=map,fill=white] at (0,-2) {$ a \mapsto |a \rangle$};
	\end{pgfonlayer}
	\begin{pgfonlayer}{edgelayer}
		\draw (.1,.5) to (.1,1);
		\draw (-.1,.5) to (-.1,1);
		\draw (0,-.5) to (0,-2);
		\draw (.1,-2) to (.1,-3);
		\draw (-.1,-2) to (-.1,-3);
	\end{pgfonlayer} 
\end{tikzpicture}\
\right\rrbracket
=
\begin{tikzpicture}
	\begin{pgfonlayer}{nodelayer}
		\node [style=Z] (0) at (59, 6.25) {};
		\node [style=none] (1) at (59, 7) {};
		\node [style=Z] (2) at (59.5, 7) {};
		\node [style=none] (3) at (59.5, 6.25) {};
		\node [style=none] (4) at (59, 7.5) {};
		\node [style=none] (5) at (59.5, 5.75) {};
	\end{pgfonlayer}
	\begin{pgfonlayer}{edgelayer}
		\draw (1.center) to (0);
		\draw (3.center) to (2);
		\draw [style=red] (1.center) to (4.center);
		\draw [style=red] (5.center) to (3.center);
	\end{pgfonlayer}
\end{tikzpicture}
=
\begin{tikzpicture}
	\begin{pgfonlayer}{nodelayer}
		\node [style=Z] (219) at (59, 6.25) {};
		\node [style=none] (220) at (59, 6.75) {};
		\node [style=Z] (221) at (59, 5.75) {};
		\node [style=none] (222) at (59, 5.25) {};
	\end{pgfonlayer}
	\begin{pgfonlayer}{edgelayer}
		\draw[style=red]  (220.center) to (219);
		\draw[style=red]  (222.center) to (221);
	\end{pgfonlayer}
\end{tikzpicture}
$$
\end{remark}


For this reason, we can prove the correctness of the quantum teleportation algorithm using only spider fusion (compare to the graphical proofs given in \cite{abramsky,pqp}):


\begin{example}
\label{ex:teleportation}
Given any prime $p$, the following string diagram in $\Aff\Co\Isot\Rel_{\F_p}^M$ depicts the quopit quantum teleportation protocol where Alice on the left teleports a qudit to Bob, on the right. They share an EPR pair (on the bottom of the diagram)  and two classical dits (drawn in red).  


$$
\begin{tikzpicture}
	\begin{pgfonlayer}{nodelayer}
		\node [style=none] (569) at (366.75, 1.75) {};
		\node [style=none] (570) at (367.25, 3.25) {};
		\node [style=none] (571) at (366.25, 3.25) {};
		\node [style=none] (572) at (365.75, 1.75) {};
		\node [style=Z] (573) at (367.25, 4.75) {};
		\node [style=Z] (574) at (365.75, 4.75) {};
		\node [style=none] (575) at (369.5, 3.75) {};
		\node [style=none] (576) at (370.25, 3.75) {};
		\node [style=none] (577) at (369.5, 9) {};
		\node [style=none] (578) at (370.25, 9) {};
		\node [style=Z] (579) at (365.75, 3.25) {};
		\node [style=X] (580) at (366.25, 3.75) {};
		\node [style=Z] (581) at (367.25, 3.75) {};
		\node [style=X] (582) at (366.75, 3.25) {};
		\node [style=none] (583) at (366.25, 4.75) {};
		\node [style=none] (584) at (366.75, 4.75) {};
		\node [style=X] (585) at (367.5, 2.25) {};
		\node [style=Z] (586) at (369.25, 2.25) {};
		\node [style=Z] (587) at (369.5, 7.75) {};
		\node [style=X] (588) at (370.25, 7.75) {};
		\node [style=X] (589) at (369.5, 8.5) {};
		\node [style=Z] (590) at (370.25, 8.5) {};
		\node [style=Z] (591) at (368.25, 6.75) {};
		\node [style=Z] (592) at (368.75, 6.75) {};
		\node [style=none] (593) at (369.25, 6.75) {};
		\node [style=none] (594) at (367.75, 6.75) {};
		\node [style=none] (595) at (366.75, 9) {};
		\node [style=none] (596) at (368.5, 1.75) {};
		\node [style=none] (597) at (365.75, 9) {Alice};
		\node [style=none] (598) at (367.5, 9) {Bob};
		\node [style=none] (599) at (365.25, 4.75) {};
		\node [style=none] (600) at (370.75, 4.75) {};
		\node [style=none] (601) at (365.25, 6.75) {};
		\node [style=none] (602) at (370.75, 6.75) {};
		\node [style=none] (603) at (363.25, 6.75) {Phase correction};
		\node [style=none] (604) at (363.25, 4.75) {Measurement};
	\end{pgfonlayer}
	\begin{pgfonlayer}{edgelayer}
		\draw (580) to (579);
		\draw (582) to (581);
		\draw (570.center) to (581);
		\draw (581) to (573);
		\draw (572.center) to (579);
		\draw (579) to (574);
		\draw (571.center) to (580);
		\draw (569.center) to (582);
		\draw (582) to (584.center);
		\draw (580) to (583.center);
		\draw [in=0, out=-90, looseness=0.75] (575.center) to (585);
		\draw [in=-90, out=165, looseness=0.50] (585) to (571.center);
		\draw [in=165, out=-90, looseness=0.50] (570.center) to (586);
		\draw [in=-90, out=30] (586) to (576.center);
		\draw (590) to (588);
		\draw (589) to (587);
		\draw (575.center) to (587);
		\draw (589) to (577.center);
		\draw (578.center) to (590);
		\draw (588) to (576.center);
		\draw [in=-150, out=90, looseness=0.75] (591) to (590);
		\draw [in=-150, out=90] (592) to (587);
		\draw [color=red, in=-90, out=90, looseness=0.50] (584.center) to (593.center);
		\draw [color=red, in=-90, out=90] (583.center) to (594.center);
		\draw [in=210, out=90, looseness=0.75] (594.center) to (589);
		\draw [in=-150, out=90] (593.center) to (588);
		\draw [style=dotted, in=270, out=90] (596.center) to (595.center);
		\draw [style=dotted] (600.center) to (599.center);
		\draw [style=dotted] (602.center) to (601.center);
	\end{pgfonlayer}
\end{tikzpicture}
=
\begin{tikzpicture}
	\begin{pgfonlayer}{nodelayer}
		\node [style=none] (301) at (81.75, 1.5) {};
		\node [style=none] (302) at (81.25, 3) {};
		\node [style=none] (303) at (80.75, 1.5) {};
		\node [style=none] (304) at (84.5, 3) {};
		\node [style=none] (305) at (85, 3) {};
		\node [style=none] (306) at (84.5, 8.5) {};
		\node [style=none] (307) at (85, 8.5) {};
		\node [style=X] (308) at (81.25, 4.5) {};
		\node [style=X] (309) at (82.75, 1.75) {};
		\node [style=Z] (310) at (83.75, 1.75) {};
		\node [style=X] (311) at (85, 7.25) {};
		\node [style=X] (312) at (84.5, 7.25) {};
		\node [style=Z] (313) at (83, 5) {};
		\node [style=X] (314) at (82, 4.5) {};
	\end{pgfonlayer}
	\begin{pgfonlayer}{edgelayer}
		\draw (302.center) to (308);
		\draw [in=15, out=-90, looseness=0.75] (304.center) to (309);
		\draw [in=-90, out=165, looseness=0.50] (309) to (302.center);
		\draw [in=-90, out=15, looseness=0.75] (310) to (305.center);
		\draw (312) to (306.center);
		\draw (311) to (305.center);
		\draw (311) to (307.center);
		\draw (304.center) to (312);
		\draw [in=90, out=-120] (308) to (303.center);
		\draw [color=red, in=225, out=90, looseness=0.75] (308) to (312);
		\draw [in=225, out=120, looseness=0.50, color=red] (314) to (311);
		\draw [in=15, out=-165] (313) to (314);
		\draw [in=270, out=90] (301.center) to (314);
		\draw [in=-60, out=135, looseness=0.75] (310) to (313);
	\end{pgfonlayer}
\end{tikzpicture}
=
\begin{tikzpicture}
	\begin{pgfonlayer}{nodelayer}
		\node [style=none] (166) at (42.5, 1.5) {};
		\node [style=none] (167) at (42, 1.5) {};
		\node [style=none] (168) at (43.5, 8.5) {};
		\node [style=none] (169) at (44, 8.5) {};
		\node [style=X] (170) at (42, 4.5) {};
		\node [style=X] (171) at (42.5, 3.5) {};
		\node [style=X] (172) at (44, 6.25) {};
		\node [style=X] (173) at (43.5, 7.25) {};
	\end{pgfonlayer}
	\begin{pgfonlayer}{edgelayer}
		\draw [in=-120, out=90, looseness=0.75] (166.center) to (171);
		\draw [in=270, out=60] (173) to (168.center);
		\draw [in=270, out=60, looseness=0.75] (172) to (169.center);
		\draw [color=red, bend left=15, looseness=0.50] (171) to (172);
		\draw [in=90, out=-120, looseness=0.50] (170) to (167.center);
		\draw [bend right=15, looseness=0.50] (171) to (172);
		\draw [bend left=15, looseness=0.50] (173) to (170);
		\draw [color=red, bend left=15, looseness=0.50] (170) to (173);
	\end{pgfonlayer}
\end{tikzpicture}
=
\begin{tikzpicture}
	\begin{pgfonlayer}{nodelayer}
		\node [style=none] (174) at (45.5, 1.5) {};
		\node [style=none] (175) at (45, 1.5) {};
		\node [style=none] (176) at (45, 8.5) {};
		\node [style=none] (177) at (45.5, 8.5) {};
	\end{pgfonlayer}
	\begin{pgfonlayer}{edgelayer}
		\draw (177.center) to (174.center);
		\draw (175.center) to (176.center);
	\end{pgfonlayer}
\end{tikzpicture}
$$

\end{example}

Because $\Aff\Co\Isot\Rel_{\F_p}^M$ is a subcategory of relations, composable maps are ordered by subspace inclusion (ie, it is poset-enriched). Moreover, since all possible outcomes are equally likely we can identify when the measurement statistics of one process arise from the marginalization of the measurement statistics of another process:

\begin{remark}
Take two quopit stabilizer circuits with state preparations and measurement $f,g$ interpreted as parallel maps in  $\Aff\Co\Isot\Rel_{\F_p}^M$.
Then $f$ is a coarse-graining of $g$ when $f \subset g$ is a (strict)  affine subspace.
\end{remark}

\begin{example}
For an extreme example, the identity circuit on a classical wire is contained within  the circuit obtained by preparing in the $Z$ basis and measuring in the $X$:

$$
\begin{tikzpicture}
	\begin{pgfonlayer}{nodelayer}
		\node [style=none] (220) at (59, 6.75) {};
		\node [style=none] (222) at (59, 5.25) {};
	\end{pgfonlayer}
	\begin{pgfonlayer}{edgelayer}
		\draw [style=red] (220.center) to (222);
	\end{pgfonlayer}
\end{tikzpicture}
\subset
\begin{tikzpicture}
	\begin{pgfonlayer}{nodelayer}
		\node [style=Z] (219) at (59, 6.25) {};
		\node [style=none] (220) at (59, 6.75) {};
		\node [style=Z] (221) at (59, 5.75) {};
		\node [style=none] (222) at (59, 5.25) {};
	\end{pgfonlayer}
	\begin{pgfonlayer}{edgelayer}
		\draw [style=red] (220.center) to (219);
		\draw [style=red] (222.center) to (221);
	\end{pgfonlayer}
\end{tikzpicture}
=
\begin{tikzpicture}
	\begin{pgfonlayer}{nodelayer}
		\node [style=Z] (0) at (59, 6.25) {};
		\node [style=none] (1) at (59, 7) {};
		\node [style=Z] (2) at (59.5, 7) {};
		\node [style=none] (3) at (59.5, 6.25) {};
		\node [style=none] (4) at (59, 7.75) {};
		\node [style=none] (5) at (59.5, 5.5) {};
	\end{pgfonlayer}
	\begin{pgfonlayer}{edgelayer}
		\draw (1.center) to (0);
		\draw (3.center) to (2);
		\draw [style=red] (5.center) to (3.center);
		\draw [style=red] (1.center) to (4.center);
	\end{pgfonlayer}
\end{tikzpicture}
$$

This is because, given any input state, the circuit on the right hand side can produce any output state; however, the identity circuit forces the inputs to be the same as the outputs.
\end{example}


\begin{example}
Similarly, the identity on a quantum wire is a coarse graining of the decoherence map:
$$
\begin{tikzpicture}
	\begin{pgfonlayer}{nodelayer}
		\node [style=none] (261) at (69.25, 6.75) {};
		\node [style=none] (263) at (69.25, 5.25) {};
		\node [style=none] (264) at (69.75, 6.75) {};
		\node [style=none] (265) at (69.75, 5.25) {};
	\end{pgfonlayer}
	\begin{pgfonlayer}{edgelayer}
		\draw (265.center) to (264.center);
		\draw (263.center) to (261.center);
	\end{pgfonlayer}
\end{tikzpicture}
\subset
\begin{tikzpicture}
	\begin{pgfonlayer}{nodelayer}
		\node [style=Z] (0) at (69.25, 6.25) {};
		\node [style=none] (1) at (69.25, 6.75) {};
		\node [style=Z] (2) at (69.25, 5.75) {};
		\node [style=none] (3) at (69.25, 5.25) {};
		\node [style=none] (4) at (69.75, 6.75) {};
		\node [style=none] (5) at (69.75, 5.25) {};
		\node [style=none] (6) at (69.75, 6.25) {};
		\node [style=none] (7) at (69.75, 5.75) {};
	\end{pgfonlayer}
	\begin{pgfonlayer}{edgelayer}
		\draw (1.center) to (0);
		\draw (3.center) to (2);
		\draw [style=red] (7.center) to (6.center);
		\draw (5.center) to (7.center);
		\draw (6.center) to (4.center);
	\end{pgfonlayer}
\end{tikzpicture}
$$
\end{example}

This two coloured prop also has a semantics in terms of electrical circuits, when changing the field from $\F_p$ to the field of fractions of real polynomials $\R(x)$.  It gives a well-structured semantics for a large fragment (although not all) of the impedance calculus \cite{impedence}:

\begin{remark}
\label{rem:electrical}
$\Aff\Co\Isot\Rel_{\R(x)}^M$ is a semantics for the fragment of the impedance calculus generated by resistors, inductors, capacitors, controlled and uncontrolled voltage/current sources.
\end{remark}

This can be verified by factoring the generators in \cite{impedence} into the appropriate form. DO THIS!!!!!


\section{Error correction}
\label{sec:qec}

%Recall the observation from Corollary \ref{cor:stabcode} that nonempty states in $\Aff\Co\Isot\Rel_{\F_p}$ correspond to stabilizer codes. Moreover an affine coisotropic subspace of $\F_p^{2n}$ with dimension $m$  interpreted as a state in $\CPM(\Stab_p)$ encodes $m-n$ logical qubits in $n$ physical qubits.  In the literature, these are the $[n,m-n]$ odd prime dimensional qudit stabilizer codes \cite{????}.  In this section we show how to model error correction protocols within this framework.  Starting with stabilizer codes and then extending this interpretation to topological stabilizer codes by looking at the quantized Weinstein category.

Quantum channels are inherently noisy, and it is an active area of study to protect quantum channels against errors.  Quantum error detection/correction protocols are often performed using stabilizer circuits, which is amenable to the symplectic formalism.  We will the categorical semantics for stabilier codes to describe quantum error correction protocols. 
In this section, we show how to implement quantum error correction protocols for stabilizer codes (see \cite{gottesman}) using the string diagrams we developed in the previous section.   This generalizes the graphical $\F_2$-linear subspace semantics of qubit CSS codes in \cite{grok} to quopit stabilizer codes as well as qubit CSS codes plus Weyl operators.


Fix an odd prime $p$.
Consider an affine coisotropic subspace $S=L+a \subseteq \F_p^{2n}$ where $L$ has dimension $n+k$.  Then the associated projector on $n$-qudits is called a $[n,k]$-stabilizer code, as it encodes $k$ logical qudits into $n$ physical qudits. 
The relationship between logical and physical qubits can be understood in terms of pictures.
Fix a unitary purification $U$ of $S$:
$$
\begin{tikzpicture}
	\begin{pgfonlayer}{nodelayer}
		\node [style=map] (249) at (105.1, 3.25) {$S$};
		\node [style=none] (250) at (104.85, 4) {};
		\node [style=none] (251) at (105.35, 4) {};
		\node [style=none] (256) at (104.85, 4.25) {$n$};
		\node [style=none] (257) at (105.35, 4.25) {$n$};
	\end{pgfonlayer}
	\begin{pgfonlayer}{edgelayer}
		\draw [in=-90, out=120] (249) to (250.center);
		\draw [in=-90, out=60] (249) to (251.center);
	\end{pgfonlayer}
\end{tikzpicture}
=
\begin{tikzpicture}
	\begin{pgfonlayer}{nodelayer}
		\node [style=map] (249) at (105.1, 3.25) {$U$};
		\node [style=none] (250) at (104.85, 4) {};
		\node [style=none] (251) at (105.35, 4) {};
		\node [style=none] (252) at (103.85, 2) {};
		\node [style=none] (253) at (105.6, 2) {};
		\node [style=Z] (254) at (103.85, 2) {};
		\node [style=none] (255) at (106.35, 1.55) {$n-k$};
		\node [style=none] (256) at (104.85, 4.25) {$n$};
		\node [style=none] (257) at (105.35, 4.25) {$n$};
		\node [style=X] (258) at (106.35, 2) {};
		\node [style=Z] (259) at (104.6, 2) {};
		\node [style=none] (260) at (105.6, 1.5) {$k$};
		\node [style=none] (261) at (103.75, 1.5) {$k$};
		\node [style=none] (262) at (104.6, 1.5) {$n-k$};
		\node [style=Z] (263) at (105.6, 2) {};
	\end{pgfonlayer}
	\begin{pgfonlayer}{edgelayer}
		\draw [in=-90, out=120] (249) to (250.center);
		\draw [in=-90, out=60] (249) to (251.center);
		\draw [in=-60, out=90] (253.center) to (249);
		\draw [in=90, out=-150] (249) to (252.center);
		\draw [in=-30, out=90] (258) to (249);
		\draw [in=90, out=-120] (249) to (259);
	\end{pgfonlayer}
\end{tikzpicture}
$$
The Lagrangian dilation of $S$, the {\bf encoder}, embeds $k$ logical qudits into $n$ physical qudits:
$$
\begin{tikzpicture}
	\begin{pgfonlayer}{nodelayer}
		\node [style=map] (249) at (105.1, 3.25) {$U$};
		\node [style=none] (250) at (104.85, 4) {};
		\node [style=none] (251) at (105.35, 4) {};
		\node [style=none] (252) at (103.85, 1.75) {};
		\node [style=none] (253) at (105.6, 1.75) {};
		\node [style=none] (255) at (106.35, 1.55) {$n-k$};
		\node [style=none] (256) at (104.85, 4.25) {$n$};
		\node [style=none] (257) at (105.35, 4.25) {$n$};
		\node [style=X] (258) at (106.35, 2) {};
		\node [style=Z] (259) at (104.6, 2) {};
		\node [style=none] (260) at (105.6, 1.25) {$k$};
		\node [style=none] (262) at (104.6, 1.5) {$n-k$};
		\node [style=none] (265) at (103.85, 1.25) {$k$};
	\end{pgfonlayer}
	\begin{pgfonlayer}{edgelayer}
		\draw [in=-90, out=120] (249) to (250.center);
		\draw [in=-90, out=60] (249) to (251.center);
		\draw [in=-60, out=90] (253.center) to (249);
		\draw [in=90, out=-150] (249) to (252.center);
		\draw [in=-30, out=90] (258) to (249);
		\draw [in=90, out=-120] (249) to (259);
	\end{pgfonlayer}
\end{tikzpicture}
$$
Splitting this projector fixes a basis $\{b_1,\ldots, b_{n-k}\}$ for $L^\omega$, where the elements of the basis are pairwise orthogonal with respect to the symplectic form.  That is to say, they commute, which fixes the possible measurement outcomes:
$$
\begin{tikzpicture}
	\begin{pgfonlayer}{nodelayer}
		\node [style=map] (266) at (108.7, 3.25) {$U$};
		\node [style=none] (267) at (108.45, 4) {};
		\node [style=none] (268) at (108.95, 4) {};
		\node [style=none] (269) at (107.45, 2) {};
		\node [style=none] (270) at (109.2, 2) {};
		\node [style=Z] (271) at (107.45, 2) {};
		\node [style=none] (272) at (109.95, 1.05) {$n-k$};
		\node [style=none] (273) at (108.45, 4.25) {$n$};
		\node [style=none] (274) at (108.95, 4.25) {$n$};
		\node [style=Z] (276) at (108.2, 2) {};
		\node [style=none] (277) at (109.2, 1.5) {$k$};
		\node [style=none] (278) at (107.35, 1.5) {$k$};
		\node [style=none] (279) at (108.2, 1.5) {$n-k$};
		\node [style=none] (280) at (109.95, 1.5) {};
		\node [style=none] (281) at (109.95, 2) {};
		\node [style=Z] (282) at (109.2, 2) {};
	\end{pgfonlayer}
	\begin{pgfonlayer}{edgelayer}
		\draw [in=-90, out=120] (266) to (267.center);
		\draw [in=-90, out=60] (266) to (268.center);
		\draw [in=-60, out=90] (270.center) to (266);
		\draw [in=90, out=-150] (266) to (269.center);
		\draw [in=90, out=-120] (266) to (276);
		\draw [in=-30, out=90] (281.center) to (266);
		\draw [style=red] (280.center) to (281.center);
	\end{pgfonlayer}
\end{tikzpicture}
$$
Suppose that Alice encodes a state and sends it to Bob on a noisy quantum channel with Pauli error $W(e)$.  To detect the error apply the non-destructive measurement with respect to our chosen basis on the last $n-k$ wires in the $Z$ basis conjugated by $U$:
\begin{align*}
\begin{tikzpicture}
	\begin{pgfonlayer}{nodelayer}
		\node [style=map] (0) at (95.75, 2.75) {$U$};
		\node [style=none] (1) at (94.75, 1.25) {};
		\node [style=none] (2) at (96, 1.25) {};
		\node [style=X] (3) at (96.5, 1.5) {};
		\node [style=Z] (4) at (95.25, 1.5) {};
		\node [style=none] (5) at (95.5, 6) {};
		\node [style=none] (6) at (96, 6) {};
		\node [style=none] (7) at (97.5, 6) {};
		\node [style=none] (8) at (98, 6) {};
		\node [style=Z] (9) at (96.5, 6.25) {};
		\node [style=X] (10) at (96, 6) {};
		\node [style=Z] (11) at (96.5, 5.5) {};
		\node [style=none] (12) at (96.75, 8) {};
		\node [style=none] (13) at (96.75, 8) {};
		\node [style=Z] (14) at (98, 6) {};
		\node [style=X] (15) at (98.5, 6.25) {};
		\node [style=X] (16) at (98.5, 5.5) {};
		\node [style=none] (17) at (96.75, 8) {};
		\node [style=none] (18) at (96.75, 8) {};
		\node [style=Z] (19) at (96.5, 7) {};
		\node [style=none] (20) at (98.5, 9) {};
		\node [style=none] (21) at (98.5, 7) {};
		\node [style=map] (22) at (96.25, 3.75) {$W(e)$};
		\node [style=map] (23) at (96.75, 4.75) {$U^\dag$};
		\node [style=map] (24) at (96.75, 8) {$U$};
		\node [style=none] (25) at (95.75, 9) {};
		\node [style=none] (26) at (96.25, 9) {};
		\node [style=none] (27) at (97.25, 9) {};
		\node [style=none] (28) at (97.75, 9) {};
		\node [style=none] (29) at (94.25, 9) {};
		\node [style=none] (30) at (97, 1) {};
		\node [style=none] (31) at (98.75, 7) {};
		\node [style=none] (32) at (94.25, 7) {};
		\node [style=none] (33) at (98.75, 1.5) {};
		\node [style=none] (34) at (94.5, 1.5) {};
		\node [style=none] (35) at (93, 7.25) {Syndrome};
		\node [style=none] (36) at (93.25, 1.5) {Encoding};
		\node [style=none] (37) at (93.75, 2.75) {Alice};
		\node [style=none] (38) at (97.25, 2.75) {Bob};
		\node [style=none] (39) at (98.75, 3.75) {};
		\node [style=none] (40) at (94.25, 3.75) {};
		\node [style=none] (41) at (93.5, 3.75) {Error};
		\node [style=none] (42) at (93, 6.75) {measurement};
	\end{pgfonlayer}
	\begin{pgfonlayer}{edgelayer}
		\draw [in=-60, out=90, looseness=0.75] (2.center) to (0);
		\draw [in=90, out=-165] (0) to (1.center);
		\draw [in=-30, out=90] (3) to (0);
		\draw [in=90, out=-135] (0) to (4);
		\draw (9) to (10);
		\draw (11) to (9);
		\draw [in=-150, out=90] (5.center) to (13.center);
		\draw [in=-120, out=90, looseness=1.25] (10) to (12.center);
		\draw [in=-60, out=90, looseness=0.75] (7.center) to (18.center);
		\draw [in=90, out=-30, looseness=0.75] (17.center) to (14);
		\draw (14) to (15);
		\draw (15) to (16);
		\draw [style=red] (21.center) to (20.center);
		\draw (9) to (19);
		\draw (15) to (21.center);
		\draw [in=270, out=75, looseness=0.75] (0) to (22);
		\draw [in=-105, out=90, looseness=0.50] (22) to (23);
		\draw [in=-90, out=150] (23) to (5.center);
		\draw [in=135, out=-90] (6.center) to (23);
		\draw [in=-90, out=60] (23) to (7.center);
		\draw [in=30, out=-90, looseness=0.75] (8.center) to (23);
		\draw [in=-90, out=135] (24) to (25.center);
		\draw [in=-90, out=105] (24) to (26.center);
		\draw [in=-90, out=75] (24) to (27.center);
		\draw [in=-90, out=45] (24) to (28.center);
		\draw [style=dotted, in=270, out=90] (30.center) to (29.center);
		\draw [style=dotted] (32.center) to (31.center);
		\draw [style=dotted] (34.center) to (33.center);
		\draw [style=dotted] (40.center) to (39.center);
	\end{pgfonlayer}
\end{tikzpicture}
&=
\begin{tikzpicture}
	\begin{pgfonlayer}{nodelayer}
		\node [style=none] (43) at (100.25, -39.25) {};
		\node [style=none] (44) at (101.25, -39.25) {};
		\node [style=X] (45) at (101.75, -39) {};
		\node [style=Z] (46) at (100.75, -39) {};
		\node [style=none] (47) at (100, -36) {};
		\node [style=none] (48) at (100.5, -36) {};
		\node [style=none] (49) at (101.5, -36) {};
		\node [style=none] (50) at (102, -36) {};
		\node [style=Z] (51) at (101, -35.75) {};
		\node [style=X] (52) at (100.5, -36) {};
		\node [style=Z] (53) at (101, -36.5) {};
		\node [style=none] (54) at (101, -34) {};
		\node [style=none] (55) at (101, -34) {};
		\node [style=Z] (56) at (102, -36) {};
		\node [style=X] (57) at (102.5, -35.75) {};
		\node [style=X] (58) at (102.5, -36.5) {};
		\node [style=none] (59) at (101, -34) {};
		\node [style=none] (60) at (101, -34) {};
		\node [style=Z] (61) at (101, -35) {};
		\node [style=none] (62) at (102.5, -33) {};
		\node [style=none] (63) at (102.5, -35) {};
		\node [style=map] (64) at (101, -37.75) {$U;W(e);U^\dag$};
		\node [style=map] (65) at (101, -34) {$U$};
		\node [style=none] (66) at (100.25, -33) {};
		\node [style=none] (67) at (100.75, -33) {};
		\node [style=none] (68) at (101.25, -33) {};
		\node [style=none] (69) at (101.75, -33) {};
	\end{pgfonlayer}
	\begin{pgfonlayer}{edgelayer}
		\draw (51) to (52);
		\draw (53) to (51);
		\draw [in=-150, out=90] (47.center) to (55.center);
		\draw [in=-120, out=90, looseness=1.25] (52) to (54.center);
		\draw [in=-60, out=90, looseness=0.75] (49.center) to (60.center);
		\draw [in=90, out=-30, looseness=0.75] (59.center) to (56);
		\draw (56) to (57);
		\draw (57) to (58);
		\draw [style=red] (63.center) to (62.center);
		\draw (51) to (61);
		\draw (57) to (63.center);
		\draw [in=-90, out=135] (65) to (66.center);
		\draw [in=-90, out=105] (65) to (67.center);
		\draw [in=-90, out=75] (65) to (68.center);
		\draw [in=-90, out=45] (65) to (69.center);
		\draw [in=-135, out=90] (43.center) to (64);
		\draw [in=90, out=-105] (64) to (46);
		\draw [in=-75, out=90] (44.center) to (64);
		\draw [in=-45, out=90] (45) to (64);
		\draw [in=135, out=-90] (47.center) to (64);
		\draw [in=-90, out=105] (64) to (52);
		\draw [in=75, out=-90] (49.center) to (64);
		\draw [in=-90, out=45] (64) to (56);
	\end{pgfonlayer}
\end{tikzpicture}
=
\begin{tikzpicture}
	\begin{pgfonlayer}{nodelayer}
		\node [style=none] (70) at (103.5, -37.75) {};
		\node [style=none] (71) at (105, -37.75) {};
		\node [style=X] (72) at (105.5, -37.5) {};
		\node [style=Z] (73) at (104, -37.5) {};
		\node [style=none] (74) at (103.5, -36) {};
		\node [style=none] (75) at (104, -36) {};
		\node [style=none] (76) at (105, -36) {};
		\node [style=none] (77) at (105.5, -36) {};
		\node [style=Z] (78) at (104.5, -35.75) {};
		\node [style=X] (79) at (104, -36) {};
		\node [style=Z] (80) at (104.5, -36.5) {};
		\node [style=none] (81) at (104.5, -34) {};
		\node [style=none] (82) at (104.5, -34) {};
		\node [style=Z] (83) at (105.5, -36) {};
		\node [style=X] (84) at (106, -35.75) {};
		\node [style=X] (85) at (106, -36.5) {};
		\node [style=none] (86) at (104.5, -34) {};
		\node [style=none] (87) at (104.5, -34) {};
		\node [style=Z] (88) at (104.5, -35) {};
		\node [style=none] (89) at (106, -33) {};
		\node [style=none] (90) at (106, -35) {};
		\node [style=map] (91) at (104.5, -34) {$U$};
		\node [style=none] (92) at (103.75, -33) {};
		\node [style=none] (93) at (104.25, -33) {};
		\node [style=none] (94) at (104.75, -33) {};
		\node [style=none] (95) at (105.25, -33) {};
		\node [style=X] (96) at (103.5, -36.75) {$a$};
		\node [style=X] (97) at (104, -36.75) {$b$};
		\node [style=X] (98) at (105, -36.75) {$c$};
		\node [style=X] (99) at (105.5, -36.75) {$d$};
	\end{pgfonlayer}
	\begin{pgfonlayer}{edgelayer}
		\draw (78) to (79);
		\draw (80) to (78);
		\draw [in=-150, out=90] (74.center) to (82.center);
		\draw [in=-120, out=90, looseness=1.25] (79) to (81.center);
		\draw [in=-60, out=90, looseness=0.75] (76.center) to (87.center);
		\draw [in=90, out=-30, looseness=0.75] (86.center) to (83);
		\draw (83) to (84);
		\draw (84) to (85);
		\draw [style=red] (90.center) to (89.center);
		\draw (78) to (88);
		\draw (84) to (90.center);
		\draw [in=-90, out=135] (91) to (92.center);
		\draw [in=-90, out=105] (91) to (93.center);
		\draw [in=-90, out=75] (91) to (94.center);
		\draw [in=-90, out=45] (91) to (95.center);
		\draw (70.center) to (96);
		\draw (96) to (74.center);
		\draw (79) to (97);
		\draw (97) to (73);
		\draw (71.center) to (98);
		\draw (98) to (76.center);
		\draw (83) to (99);
		\draw (99) to (72);
	\end{pgfonlayer}
\end{tikzpicture}
=
\begin{tikzpicture}
	\begin{pgfonlayer}{nodelayer}
		\node [style=none] (100) at (107, -37.75) {};
		\node [style=none] (101) at (108.5, -37.75) {};
		\node [style=X] (102) at (109, -37.5) {};
		\node [style=Z] (103) at (107.5, -37.5) {};
		\node [style=none] (104) at (107, -36.5) {};
		\node [style=none] (105) at (107.5, -36.5) {};
		\node [style=none] (106) at (108.5, -36.5) {};
		\node [style=none] (107) at (109, -36.5) {};
		\node [style=Z] (108) at (108, -36.25) {};
		\node [style=X] (109) at (107.5, -36.5) {};
		\node [style=Z] (110) at (108, -37) {};
		\node [style=none] (111) at (108, -34) {};
		\node [style=none] (112) at (108, -34) {};
		\node [style=Z] (113) at (109, -36.5) {};
		\node [style=X] (114) at (109.5, -36.25) {};
		\node [style=X] (115) at (109.5, -37) {};
		\node [style=none] (116) at (108, -34) {};
		\node [style=none] (117) at (108, -34) {};
		\node [style=Z] (118) at (108, -35.5) {};
		\node [style=none] (119) at (109.5, -33) {};
		\node [style=none] (120) at (109.5, -35) {};
		\node [style=map] (121) at (108, -34) {$U$};
		\node [style=none] (122) at (107.25, -33) {};
		\node [style=none] (123) at (107.75, -33) {};
		\node [style=none] (124) at (108.25, -33) {};
		\node [style=none] (125) at (108.75, -33) {};
		\node [style=X] (126) at (107, -35.5) {$a$};
		\node [style=X] (127) at (107.5, -35.5) {$b$};
		\node [style=X] (128) at (108.5, -35.5) {$c$};
		\node [style=X] (129) at (109, -35.5) {$d$};
		\node [style=X] (130) at (109.5, -35.5) {$d$};
	\end{pgfonlayer}
	\begin{pgfonlayer}{edgelayer}
		\draw (108) to (109);
		\draw (110) to (108);
		\draw (113) to (114);
		\draw (114) to (115);
		\draw [style=red] (120.center) to (119.center);
		\draw (108) to (118);
		\draw (114) to (120.center);
		\draw [in=-90, out=135] (121) to (122.center);
		\draw [in=-90, out=105] (121) to (123.center);
		\draw [in=-90, out=75] (121) to (124.center);
		\draw [in=-90, out=45] (121) to (125.center);
		\draw (100.center) to (104.center);
		\draw (104.center) to (126);
		\draw [in=225, out=90] (126) to (121);
		\draw [in=90, out=-120] (121) to (127);
		\draw (127) to (109);
		\draw (109) to (103);
		\draw (101.center) to (106.center);
		\draw (102) to (113);
		\draw (113) to (129);
		\draw [in=90, out=-45] (121) to (129);
		\draw [in=300, out=90] (128) to (121);
		\draw (106.center) to (128);
	\end{pgfonlayer}
\end{tikzpicture}\\
&=
\begin{tikzpicture}
	\begin{pgfonlayer}{nodelayer}
		\node [style=X] (131) at (298, -36.5) {};
		\node [style=Z] (132) at (297, -36.5) {};
		\node [style=none] (133) at (296.5, -36.75) {};
		\node [style=none] (134) at (297.5, -36.75) {};
		\node [style=none] (135) at (297.25, -34) {};
		\node [style=none] (136) at (297.25, -34) {};
		\node [style=none] (137) at (297.25, -34) {};
		\node [style=none] (138) at (297.25, -34) {};
		\node [style=none] (139) at (298.5, -33) {};
		\node [style=none] (140) at (298.5, -35) {};
		\node [style=map] (141) at (297.25, -34) {$U$};
		\node [style=none] (142) at (296.5, -33) {};
		\node [style=none] (143) at (297, -33) {};
		\node [style=none] (144) at (297.5, -33) {};
		\node [style=none] (145) at (298, -33) {};
		\node [style=X] (146) at (296.5, -35.5) {$a$};
		\node [style=X] (147) at (297, -35.5) {$b$};
		\node [style=X] (148) at (297.5, -35.5) {$c$};
		\node [style=X] (149) at (298, -35.5) {$d$};
		\node [style=X] (150) at (298.5, -34.75) {$d$};
	\end{pgfonlayer}
	\begin{pgfonlayer}{edgelayer}
		\draw [style=red] (140.center) to (139.center);
		\draw [in=-90, out=135] (141) to (142.center);
		\draw [in=-90, out=105] (141) to (143.center);
		\draw [in=-90, out=75] (141) to (144.center);
		\draw [in=-90, out=45] (141) to (145.center);
		\draw (133.center) to (146);
		\draw [in=225, out=90] (146) to (141);
		\draw [in=90, out=-120] (141) to (147);
		\draw [in=90, out=-45] (141) to (149);
		\draw [in=300, out=90] (148) to (141);
		\draw (134.center) to (148);
		\draw (132) to (147);
		\draw (149) to (131);
	\end{pgfonlayer}
\end{tikzpicture}
=
\begin{tikzpicture}
	\begin{pgfonlayer}{nodelayer}
		\node [style=X] (456) at (347, -36.5) {};
		\node [style=Z] (457) at (346, -36.5) {};
		\node [style=none] (458) at (345.5, -36.5) {};
		\node [style=none] (459) at (346.5, -36.5) {};
		\node [style=none] (460) at (346.25, -34) {};
		\node [style=none] (461) at (346.25, -34) {};
		\node [style=none] (462) at (346.25, -34) {};
		\node [style=none] (463) at (346.25, -34) {};
		\node [style=none] (464) at (347.5, -33) {};
		\node [style=none] (465) at (347.5, -34.25) {};
		\node [style=map] (466) at (346.25, -34) {$U$};
		\node [style=none] (467) at (345.5, -33) {};
		\node [style=none] (468) at (346, -33) {};
		\node [style=none] (469) at (346.5, -33) {};
		\node [style=none] (470) at (347, -33) {};
		\node [style=X] (471) at (347.5, -34.25) {$d$};
		\node [style=map] (472) at (346.25, -35) {$U;W(e);U^\dag$};
		\node [style=none] (473) at (345.5, -37) {};
		\node [style=none] (474) at (346.5, -37) {};
	\end{pgfonlayer}
	\begin{pgfonlayer}{edgelayer}
		\draw [style=red] (465.center) to (464.center);
		\draw [in=-90, out=135] (466) to (467.center);
		\draw [in=-90, out=105] (466) to (468.center);
		\draw [in=-90, out=75] (466) to (469.center);
		\draw [in=-90, out=45] (466) to (470.center);
		\draw [in=-75, out=90] (459.center) to (472);
		\draw [in=90, out=-135] (472) to (458.center);
		\draw [in=-105, out=90] (457) to (472);
		\draw [in=90, out=-45] (472) to (456);
		\draw (472) to (466);
		\draw (473.center) to (458.center);
		\draw (474.center) to (459.center);
	\end{pgfonlayer}
\end{tikzpicture}
=
\begin{tikzpicture}
	\begin{pgfonlayer}{nodelayer}
		\node [style=none] (170) at (304.75, -36.75) {};
		\node [style=none] (171) at (305.75, -36.75) {};
		\node [style=X] (172) at (306.25, -36.5) {};
		\node [style=Z] (173) at (305.25, -36.5) {};
		\node [style=none] (174) at (304.75, -36.5) {};
		\node [style=none] (175) at (305.75, -36.5) {};
		\node [style=none] (176) at (305.5, -34) {};
		\node [style=none] (177) at (305.5, -34) {};
		\node [style=none] (178) at (305.5, -34) {};
		\node [style=none] (179) at (305.5, -34) {};
		\node [style=none] (180) at (306.75, -33) {};
		\node [style=none] (181) at (306.75, -34.5) {};
		\node [style=map] (182) at (305.5, -34) {$W(e)$};
		\node [style=none] (183) at (304.75, -33) {};
		\node [style=none] (184) at (305.25, -33) {};
		\node [style=none] (185) at (305.75, -33) {};
		\node [style=none] (186) at (306.25, -33) {};
		\node [style=X] (187) at (306.75, -34.5) {$d$};
		\node [style=map] (188) at (305.5, -35) {$U$};
	\end{pgfonlayer}
	\begin{pgfonlayer}{edgelayer}
		\draw [style=red] (181.center) to (180.center);
		\draw [in=-90, out=135] (182) to (183.center);
		\draw [in=-90, out=105] (182) to (184.center);
		\draw [in=-90, out=75] (182) to (185.center);
		\draw [in=-90, out=45] (182) to (186.center);
		\draw (170.center) to (174.center);
		\draw (171.center) to (175.center);
		\draw [in=-75, out=90] (175.center) to (188);
		\draw [in=90, out=-135] (188) to (174.center);
		\draw [in=-105, out=90] (173) to (188);
		\draw [in=90, out=-45] (188) to (172);
		\draw (188) to (182);
	\end{pgfonlayer}
\end{tikzpicture}
\end{align*}
Where $W((a,b),(c,d)) = U;W(e);U^\dag$.

The tuple $d \in \F_p^{n-k}$ is called the {\bf syndrome}. The syndrome measures the displacement of the basis elements $b_i$ by errors.
An error $W(e)$ is {\bf undetectable} if and only if the syndrome is the zero vector; this is because $e$ commutes with everything in $L+a$ meaning that $e \in L^\omega+a$.  In particular, the trivial error is undetectable; so undetectable errors are indistinguishable from having no errors at all.

To correct errors, we construct am operation classically controlled from the syndrome register.
Given any syndrome measurement $d\in \F_p^{n-k}$, pick an error $W(e)$ which one wishes to correct, where additionally, the trivial syndrome corrects nothing.  This determines a function $f:\F_p^{n-k}\to\F_p^{2n}$ sending $d\mapsto e$.


If $f$ is an affine transformation, then we can construct the classically controlled error correction operation $c_f$ so that given a syndrome $d$, it applies the operation $W(-f(d))$ to $n$ qudits.  This restriction requiring the function to affine comes from the fact that within the model we have constructed, only affine classical processing is allowed. Finally, we perform $U^\dag$ and then discard the ancillary qudits. The final error correction protocol is as follows:
$$
\begin{tikzpicture}
	\begin{pgfonlayer}{nodelayer}
		\node [style=map] (475) at (351.75, 3) {$U$};
		\node [style=none] (476) at (350.75, 1.5) {};
		\node [style=none] (477) at (352, 1.5) {};
		\node [style=X] (478) at (352.5, 1.75) {};
		\node [style=Z] (479) at (351.25, 1.75) {};
		\node [style=none] (481) at (353, 6.25) {};
		\node [style=none] (483) at (354.25, 6.25) {};
		\node [style=Z] (484) at (353.5, 6.5) {};
		\node [style=X] (485) at (353, 6.25) {};
		\node [style=Z] (486) at (353.5, 5.75) {};
		\node [style=none] (487) at (353.5, 8.25) {};
		\node [style=none] (488) at (353.5, 8.25) {};
		\node [style=Z] (489) at (354.25, 6.25) {};
		\node [style=X] (490) at (354.75, 6.5) {};
		\node [style=X] (491) at (354.75, 5.75) {};
		\node [style=none] (492) at (353.5, 8.25) {};
		\node [style=none] (493) at (353.5, 8.25) {};
		\node [style=Z] (494) at (353.5, 7.25) {};
		\node [style=none] (495) at (354.5, 9) {};
		\node [style=none] (496) at (354.75, 7.25) {};
		\node [style=map] (497) at (352.75, 4) {$W(e)$};
		\node [style=map] (498) at (353.5, 5) {$U^\dag$};
		\node [style=map] (499) at (353.5, 8.25) {$U$};
		\node [style=none] (500) at (350.25, 13.25) {};
		\node [style=none] (501) at (353, 1.5) {};
		\node [style=none] (502) at (355.25, 7.25) {};
		\node [style=none] (503) at (350, 7.25) {};
		\node [style=none] (504) at (355.25, 1.75) {};
		\node [style=none] (505) at (350, 1.75) {};
		\node [style=none] (506) at (348.75, 7.5) {Syndrome};
		\node [style=none] (507) at (349, 1.75) {Encoding};
		\node [style=none] (508) at (349.5, 3) {Alice};
		\node [style=none] (509) at (353.75, 3) {Bob};
		\node [style=none] (510) at (355.25, 4) {};
		\node [style=none] (511) at (350, 4) {};
		\node [style=none] (512) at (349.25, 4) {Error};
		\node [style=none] (513) at (348.75, 7) {measurement};
		\node [style=Z] (514) at (353.5, 9) {};
		\node [style=map] (515) at (353.5, 10.25) {$c_f$};
		\node [style=none] (516) at (355.25, 10.25) {};
		\node [style=none] (517) at (350, 10.25) {};
		\node [style=none] (518) at (348.5, 10.25) {Error correction};
		\node [style=Z] (519) at (354.75, 8) {};
		\node [style=none] (520) at (355, 13.25) {};
		\node [style=map] (521) at (353.5, 11.75) {$U^\dag$};
		\node [style=none] (522) at (355.25, 11.75) {};
		\node [style=none] (523) at (350, 11.75) {};
		\node [style=none] (524) at (348.75, 11.75) {Decoding};
		\node [style=Z] (525) at (354.25, 12.75) {};
		\node [style=Z] (526) at (353.25, 12.75) {};
		\node [style=none] (527) at (353.75, 12.75) {};
		\node [style=none] (528) at (352.75, 12.75) {};
		\node [style=none] (529) at (352.75, 13.25) {};
		\node [style=none] (530) at (353.75, 13.25) {};
	\end{pgfonlayer}
	\begin{pgfonlayer}{edgelayer}
		\draw [in=-60, out=90, looseness=0.75] (477.center) to (475);
		\draw [in=90, out=-165] (475) to (476.center);
		\draw [in=-30, out=90] (478) to (475);
		\draw [in=90, out=-135] (475) to (479);
		\draw (484) to (485);
		\draw (486) to (484);
		\draw [in=-120, out=90, looseness=1.25] (485) to (487.center);
		\draw [in=90, out=-30, looseness=0.75] (492.center) to (489);
		\draw (489) to (490);
		\draw (490) to (491);
		\draw (484) to (494);
		\draw (490) to (496.center);
		\draw [in=-90, out=90] (475) to (497);
		\draw [in=-90, out=90] (497) to (498);
		\draw [in=135, out=-90] (481.center) to (498);
		\draw [in=30, out=-90, looseness=0.75] (483.center) to (498);
		\draw [style=dotted, in=270, out=90] (501.center) to (500.center);
		\draw [style=dotted] (503.center) to (502.center);
		\draw [style=dotted] (505.center) to (504.center);
		\draw [style=dotted] (511.center) to (510.center);
		\draw [in=-165, out=150] (499) to (515);
		\draw [in=30, out=-30, looseness=1.25] (515) to (499);
		\draw [bend right=45] (499) to (515);
		\draw [bend right=45, looseness=0.75] (515) to (499);
		\draw [in=-15, out=90, looseness=0.75] (495.center) to (515);
		\draw [in=-105, out=90, looseness=0.75] (514) to (515);
		\draw [style=dotted] (517.center) to (516.center);
		\draw [style=red, in=-90, out=120] (519) to (495.center);
		\draw [style=red, in=-90, out=60, looseness=0.25] (519) to (520.center);
		\draw [style=dotted] (523.center) to (522.center);
		\draw [in=-90, out=60] (521) to (527.center);
		\draw [in=-90, out=45] (521) to (525);
		\draw [in=-90, out=120] (521) to (526);
		\draw [in=-90, out=135, looseness=0.75] (521) to (528.center);
		\draw (528.center) to (529.center);
		\draw (527.center) to (530.center);
		\draw [bend right=60, looseness=1.50] (515) to (521);
		\draw [bend right=60, looseness=1.50] (521) to (515);
		\draw [bend left, looseness=1.25] (515) to (521);
		\draw [bend left, looseness=1.25] (521) to (515);
		\draw [style=red] (496.center) to (519);
		\draw [bend right, looseness=0.75] (498) to (499);
		\draw [bend right=60] (499) to (498);
	\end{pgfonlayer}
\end{tikzpicture}
=
\begin{tikzpicture}
	\begin{pgfonlayer}{nodelayer}
		\node [style=none] (605) at (371.75, 6.25) {};
		\node [style=none] (606) at (372.75, 6.25) {};
		\node [style=X] (607) at (373.25, 6.5) {};
		\node [style=Z] (608) at (372.25, 6.5) {};
		\node [style=map] (609) at (372.5, 7.75) {$U$};
		\node [style=X] (610) at (373.75, 9.25) {$d$};
		\node [style=map] (611) at (372.5, 8.5) {$W(e)$};
		\node [style=none] (612) at (372.5, 10.25) {};
		\node [style=none] (613) at (372.5, 10.25) {};
		\node [style=none] (614) at (372.5, 10.25) {};
		\node [style=none] (615) at (372.5, 10.25) {};
		\node [style=none] (616) at (373.75, 9.25) {};
		\node [style=Z] (617) at (372.5, 9.25) {};
		\node [style=map] (618) at (372.5, 10.25) {$c_f$};
		\node [style=X] (619) at (373.75, 11.25) {$d$};
		\node [style=none] (620) at (373.75, 13.75) {};
		\node [style=map] (621) at (372.5, 12.25) {$U^\dag$};
		\node [style=Z] (622) at (373.25, 13.25) {};
		\node [style=Z] (623) at (372.25, 13.25) {};
		\node [style=none] (624) at (372.75, 13.25) {};
		\node [style=none] (625) at (371.75, 13.25) {};
		\node [style=none] (626) at (371.75, 13.75) {};
		\node [style=none] (627) at (372.75, 13.75) {};
	\end{pgfonlayer}
	\begin{pgfonlayer}{edgelayer}
		\draw [in=90, out=-30] (609) to (607);
		\draw [in=-60, out=90] (606.center) to (609);
		\draw [in=-120, out=90] (608) to (609);
		\draw [in=-150, out=90] (605.center) to (609);
		\draw (609) to (611);
		\draw [bend left=60, looseness=1.50] (611) to (612.center);
		\draw [in=-30, out=30, looseness=1.50] (611) to (615.center);
		\draw [in=60, out=-60, looseness=1.25] (614.center) to (611);
		\draw [in=-120, out=120, looseness=1.25] (611) to (613.center);
		\draw [in=-15, out=90] (616.center) to (618);
		\draw (617) to (618);
		\draw [style=red] (619) to (620.center);
		\draw [in=-90, out=60] (621) to (624.center);
		\draw [in=-90, out=45] (621) to (622);
		\draw [in=-90, out=120] (621) to (623);
		\draw [in=-90, out=135, looseness=0.75] (621) to (625.center);
		\draw (625.center) to (626.center);
		\draw (624.center) to (627.center);
		\draw [bend right] (621) to (618);
		\draw [bend right] (618) to (621);
		\draw [bend right=60, looseness=1.25] (621) to (618);
		\draw [bend right=60, looseness=1.25] (618) to (621);
	\end{pgfonlayer}
\end{tikzpicture}
=
\begin{tikzpicture}
	\begin{pgfonlayer}{nodelayer}
		\node [style=none] (425) at (339.5, 6) {};
		\node [style=none] (426) at (340.5, 6) {};
		\node [style=X] (427) at (341, 6.25) {};
		\node [style=Z] (428) at (340, 6.25) {};
		\node [style=map] (429) at (340.25, 7.5) {$U$};
		\node [style=map] (430) at (340.25, 8.25) {$W(e)$};
		\node [style=map] (431) at (340.25, 9) {$W(-f(d))$};
		\node [style=X] (432) at (341.5, 9.5) {$d$};
		\node [style=none] (433) at (341.5, 11.25) {};
		\node [style=map] (434) at (340.25, 9.75) {$U^\dag$};
		\node [style=Z] (435) at (341, 10.75) {};
		\node [style=Z] (436) at (340, 10.75) {};
		\node [style=none] (437) at (340.5, 10.75) {};
		\node [style=none] (438) at (339.5, 10.75) {};
		\node [style=none] (439) at (339.5, 11.25) {};
		\node [style=none] (440) at (340.5, 11.25) {};
	\end{pgfonlayer}
	\begin{pgfonlayer}{edgelayer}
		\draw [in=90, out=-30] (429) to (427);
		\draw [in=-60, out=90] (426.center) to (429);
		\draw [in=-120, out=90] (428) to (429);
		\draw [in=-150, out=90] (425.center) to (429);
		\draw (429) to (430);
		\draw (430) to (431);
		\draw [style=red] (432) to (433.center);
		\draw [in=-90, out=60] (434) to (437.center);
		\draw [in=-90, out=45] (434) to (435);
		\draw [in=-90, out=120] (434) to (436);
		\draw [in=-90, out=135, looseness=0.75] (434) to (438.center);
		\draw (438.center) to (439.center);
		\draw (437.center) to (440.center);
		\draw (431) to (434);
	\end{pgfonlayer}
\end{tikzpicture}
=
\begin{tikzpicture}
	\begin{pgfonlayer}{nodelayer}
		\node [style=none] (554) at (359.25, 6.5) {};
		\node [style=none] (555) at (360.25, 6.5) {};
		\node [style=X] (556) at (360.75, 6.75) {};
		\node [style=Z] (557) at (359.75, 6.75) {};
		\node [style=map] (558) at (360, 8) {$U$};
		\node [style=map] (559) at (360, 8.75) {$W(e-f(d))$};
		\node [style=X] (560) at (361.25, 9.5) {$d$};
		\node [style=none] (561) at (361.25, 11) {};
		\node [style=map] (562) at (360, 9.5) {$U^\dag$};
		\node [style=Z] (563) at (360.75, 10.5) {};
		\node [style=Z] (564) at (359.75, 10.5) {};
		\node [style=none] (565) at (360.25, 10.5) {};
		\node [style=none] (566) at (359.25, 10.5) {};
		\node [style=none] (567) at (359.25, 11) {};
		\node [style=none] (568) at (360.25, 11) {};
	\end{pgfonlayer}
	\begin{pgfonlayer}{edgelayer}
		\draw [in=90, out=-30] (558) to (556);
		\draw [in=-60, out=90] (555.center) to (558);
		\draw [in=-120, out=90] (557) to (558);
		\draw [in=-150, out=90] (554.center) to (558);
		\draw (558) to (559);
		\draw [style=red] (560) to (561.center);
		\draw [in=-90, out=60] (562) to (565.center);
		\draw [in=-90, out=45] (562) to (563);
		\draw [in=-90, out=120] (562) to (564);
		\draw [in=-90, out=135, looseness=0.75] (562) to (566.center);
		\draw (566.center) to (567.center);
		\draw (565.center) to (568.center);
		\draw (559) to (562);
	\end{pgfonlayer}
\end{tikzpicture}
$$


So that if $e=f(d)$ then this reduces to the identity channel:

$$
\begin{tikzpicture}
	\begin{pgfonlayer}{nodelayer}
		\node [style=none] (1079) at (162.5, 6.5) {};
		\node [style=none] (1080) at (164, 6.5) {};
		\node [style=X] (1081) at (164.5, 6.75) {};
		\node [style=Z] (1082) at (163, 6.75) {};
		\node [style=map] (1083) at (163.5, 8) {$U$};
		\node [style=map] (1088) at (163.5, 8.75) {$W(e-f(d))$};
		\node [style=X] (1089) at (165.25, 8.75) {$d$};
		\node [style=none] (1090) at (165.25, 11) {};
		\node [style=map] (1103) at (163.5, 9.5) {$U^\dag$};
		\node [style=Z] (1104) at (164.5, 10.5) {};
		\node [style=Z] (1105) at (163, 10.5) {};
		\node [style=none] (1106) at (164, 10.5) {};
		\node [style=none] (1107) at (162.5, 10.5) {};
		\node [style=none] (1108) at (162.5, 11) {};
		\node [style=none] (1109) at (164, 11) {};
	\end{pgfonlayer}
	\begin{pgfonlayer}{edgelayer}
		\draw [in=90, out=-30] (1083) to (1081);
		\draw [in=-60, out=90] (1080.center) to (1083);
		\draw [in=-120, out=90] (1082) to (1083);
		\draw [in=-150, out=90] (1079.center) to (1083);
		\draw (1083) to (1088);
		\draw [style=red] (1089) to (1090.center);
		\draw [in=-90, out=60] (1103) to (1106.center);
		\draw [in=-90, out=45] (1103) to (1104);
		\draw [in=-90, out=120] (1103) to (1105);
		\draw [in=-90, out=135, looseness=0.75] (1103) to (1107.center);
		\draw (1107.center) to (1108.center);
		\draw (1106.center) to (1109.center);
		\draw (1088) to (1103);
	\end{pgfonlayer}
\end{tikzpicture}
=
\begin{tikzpicture}
	\begin{pgfonlayer}{nodelayer}
		\node [style=none] (1110) at (166.25, 6.5) {};
		\node [style=none] (1111) at (167.75, 6.5) {};
		\node [style=X] (1112) at (168.25, 6.75) {};
		\node [style=Z] (1113) at (166.75, 6.75) {};
		\node [style=map] (1114) at (167.25, 8) {$U$};
		\node [style=X] (1116) at (169, 8.75) {$d$};
		\node [style=none] (1117) at (169, 10.25) {};
		\node [style=map] (1118) at (167.25, 8.75) {$U^\dag$};
		\node [style=Z] (1119) at (168.25, 9.75) {};
		\node [style=Z] (1120) at (166.75, 9.75) {};
		\node [style=none] (1121) at (167.75, 9.75) {};
		\node [style=none] (1122) at (166.25, 9.75) {};
		\node [style=none] (1123) at (166.25, 10.25) {};
		\node [style=none] (1124) at (167.75, 10.25) {};
	\end{pgfonlayer}
	\begin{pgfonlayer}{edgelayer}
		\draw [in=90, out=-30] (1114) to (1112);
		\draw [in=-60, out=90] (1111.center) to (1114);
		\draw [in=-120, out=90] (1113) to (1114);
		\draw [in=-150, out=90] (1110.center) to (1114);
		\draw [style=red] (1116) to (1117.center);
		\draw [in=-90, out=60] (1118) to (1121.center);
		\draw [in=-90, out=45] (1118) to (1119);
		\draw [in=-90, out=120] (1118) to (1120);
		\draw [in=-90, out=135, looseness=0.75] (1118) to (1122.center);
		\draw (1122.center) to (1123.center);
		\draw (1121.center) to (1124.center);
		\draw (1114) to (1118);
	\end{pgfonlayer}
\end{tikzpicture}
=
\begin{tikzpicture}
	\begin{pgfonlayer}{nodelayer}
		\node [style=none] (1124) at (170, 6.5) {};
		\node [style=none] (1125) at (171.5, 6.5) {};
		\node [style=X] (1126) at (172, 6.75) {};
		\node [style=Z] (1127) at (170.5, 6.75) {};
		\node [style=X] (1129) at (172.75, 8.75) {$d$};
		\node [style=none] (1130) at (172.75, 10.25) {};
		\node [style=Z] (1132) at (172, 9.75) {};
		\node [style=Z] (1133) at (170.5, 9.75) {};
		\node [style=none] (1134) at (171.5, 10.25) {};
		\node [style=none] (1135) at (170, 10.25) {};
	\end{pgfonlayer}
	\begin{pgfonlayer}{edgelayer}
		\draw [style=red] (1129) to (1130.center);
		\draw (1126) to (1132);
		\draw (1125.center) to (1134.center);
		\draw (1124.center) to (1135.center);
		\draw (1127) to (1133);
	\end{pgfonlayer}
\end{tikzpicture}
=
\begin{tikzpicture}
	\begin{pgfonlayer}{nodelayer}
		\node [style=none] (1136) at (174.5, 6.5) {};
		\node [style=none] (1137) at (175.25, 6.5) {};
		\node [style=X] (1140) at (176, 8.75) {$d$};
		\node [style=none] (1141) at (176, 10.25) {};
		\node [style=none] (1144) at (175.25, 10.25) {};
		\node [style=none] (1145) at (174.5, 10.25) {};
	\end{pgfonlayer}
	\begin{pgfonlayer}{edgelayer}
		\draw [style=red] (1140) to (1141.center);
		\draw (1137.center) to (1144.center);
		\draw (1136.center) to (1145.center);
	\end{pgfonlayer}
\end{tikzpicture}
$$

%Because the encoder is an affine symplectomorphism,  it an affine transformation ${\sf enc}: \F_p^{2k}\to \F_p^{2n}$.  Given a logical Weyl operator $W(z,x)$ on $k$ qudits the corresponding physical operator is $W({\sf enc}(z,x))$ on $n$ qudits.  One may wish to apply logical operations to the physically encoded state before the decoding proceedure.


All of the things we have discussed in the section also works for qubits when the stabilizer code has trivial phases linear: ie, it is a CSS code, possibly with Weyl operators.  Indeed qubit CSS codes are very common in the literature.  We use the threefold qubit repetition code (see \cite[\S 10.1.1]{nielsen}) as an example:
\begin{example}
\label{ex:rep}
Consider the Linear subspace:
$$
S = \{ ((z_1,z_2,z_3),(x_1,x_2,x_3)) : x_1=x_2=x_3 \} \subseteq \F_2^{6}
$$
Which can be written in the form of a circuit:
$$
\begin{tikzpicture}
	\begin{pgfonlayer}{nodelayer}
		\node [style=none] (853) at (88.75, 2) {};
		\node [style=none] (854) at (89.25, 2) {};
		\node [style=none] (855) at (89.75, 2) {};
		\node [style=none] (856) at (86.75, 2) {};
		\node [style=none] (857) at (87.25, 2) {};
		\node [style=none] (858) at (87.75, 2) {};
		\node [style=map] (859) at (88.25, 0.75) {$S$};
	\end{pgfonlayer}
	\begin{pgfonlayer}{edgelayer}
		\draw [in=-90, out=105] (859) to (858.center);
		\draw [in=135, out=-90, looseness=0.75] (857.center) to (859);
		\draw [in=-90, out=150, looseness=0.75] (859) to (856.center);
		\draw [in=75, out=-90] (853.center) to (859);
		\draw [in=-90, out=45, looseness=0.75] (859) to (854.center);
		\draw [in=30, out=-90, looseness=0.75] (855.center) to (859);
	\end{pgfonlayer}
\end{tikzpicture}
=
\begin{tikzpicture}
	\begin{pgfonlayer}{nodelayer}
		\node [style=Z] (1149) at (180.25, 1) {};
		\node [style=none] (1156) at (177.75, 1) {};
		\node [style=none] (1157) at (178.25, 1) {};
		\node [style=none] (1158) at (178.75, 1) {};
		\node [style=none] (1163) at (179.75, 2) {};
		\node [style=none] (1164) at (180.25, 2) {};
		\node [style=none] (1165) at (180.75, 2) {};
		\node [style=none] (1166) at (177.75, 2) {};
		\node [style=none] (1167) at (178.25, 2) {};
		\node [style=none] (1168) at (178.75, 2) {};
		\node [style=Z] (1169) at (177.75, 1) {};
		\node [style=Z] (1170) at (178.25, 1) {};
		\node [style=Z] (1171) at (178.75, 1) {};
	\end{pgfonlayer}
	\begin{pgfonlayer}{edgelayer}
		\draw [in=-90, out=30] (1149) to (1165.center);
		\draw (1164.center) to (1149);
		\draw [in=-90, out=150] (1149) to (1163.center);
		\draw (1171) to (1168.center);
		\draw (1170) to (1167.center);
		\draw (1169) to (1166.center);
	\end{pgfonlayer}
\end{tikzpicture}
=
\begin{tikzpicture}
	\begin{pgfonlayer}{nodelayer}
		\node [style=none] (0) at (3, 0) {};
		\node [style=none] (1) at (3.5, 0) {};
		\node [style=none] (2) at (4, 0) {};
		\node [style=Z] (3) at (3, 0.5) {};
		\node [style=Z] (4) at (3, 1) {};
		\node [style=X] (5) at (3.5, 0.75) {};
		\node [style=X] (6) at (4, 1.5) {};
		\node [style=X] (7) at (4, 0) {};
		\node [style=X] (8) at (3.5, 0) {};
		\node [style=Z] (9) at (3, 0) {};
		\node [style=none] (10) at (1, 0) {};
		\node [style=none] (11) at (1.5, 0) {};
		\node [style=none] (12) at (2, 0) {};
		\node [style=Z] (17) at (1.5, 0.75) {};
		\node [style=X] (18) at (1, 0.5) {};
		\node [style=X] (19) at (1, 1) {};
		\node [style=Z] (20) at (2, 1.5) {};
		\node [style=none] (21) at (3, 2) {};
		\node [style=none] (22) at (3.5, 2) {};
		\node [style=none] (23) at (4, 2) {};
		\node [style=none] (24) at (1, 2) {};
		\node [style=none] (25) at (1.5, 2) {};
		\node [style=none] (26) at (2, 2) {};
		\node [style=Z] (27) at (1, 0) {};
		\node [style=Z] (28) at (1.5, 0) {};
		\node [style=Z] (29) at (2, 0) {};
	\end{pgfonlayer}
	\begin{pgfonlayer}{edgelayer}
		\draw (0.center) to (3);
		\draw (1.center) to (5);
		\draw (2.center) to (6);
		\draw (3) to (5);
		\draw (6) to (4);
		\draw (3) to (4);
		\draw (19) to (20);
		\draw (18) to (17);
		\draw (17) to (11.center);
		\draw (10.center) to (18);
		\draw (18) to (19);
		\draw (12.center) to (20);
		\draw (6) to (23.center);
		\draw (22.center) to (5);
		\draw (4) to (21.center);
		\draw (20) to (26.center);
		\draw (25.center) to (17);
		\draw (19) to (24.center);
	\end{pgfonlayer}
\end{tikzpicture}
$$
$S$ is coisotropic because:
$$
\begin{tikzpicture}
	\begin{pgfonlayer}{nodelayer}
		\node [style=none] (853) at (88.75, 2) {};
		\node [style=none] (854) at (89.25, 2) {};
		\node [style=none] (855) at (89.75, 2) {};
		\node [style=none] (856) at (86.75, 2) {};
		\node [style=none] (857) at (87.25, 2) {};
		\node [style=none] (858) at (87.75, 2) {};
		\node [style=map] (859) at (88.25, 0.75) {$S$};
	\end{pgfonlayer}
	\begin{pgfonlayer}{edgelayer}
		\draw [in=-90, out=105] (859) to (858.center);
		\draw [in=135, out=-90, looseness=0.75] (857.center) to (859);
		\draw [in=-90, out=150, looseness=0.75] (859) to (856.center);
		\draw [in=75, out=-90] (853.center) to (859);
		\draw [in=-90, out=45, looseness=0.75] (859) to (854.center);
		\draw [in=30, out=-90, looseness=0.75] (855.center) to (859);
	\end{pgfonlayer}
\end{tikzpicture}
=
\begin{tikzpicture}
	\begin{pgfonlayer}{nodelayer}
		\node [style=none] (0) at (3, 0) {};
		\node [style=none] (1) at (3.5, 0) {};
		\node [style=none] (2) at (4, 0) {};
		\node [style=Z] (3) at (3, 0.5) {};
		\node [style=Z] (4) at (3, 1) {};
		\node [style=X] (5) at (3.5, 0.75) {};
		\node [style=X] (6) at (4, 1.5) {};
		\node [style=X] (7) at (4, 0) {};
		\node [style=X] (8) at (3.5, 0) {};
		\node [style=Z] (9) at (3, 0) {};
		\node [style=none] (10) at (1, 0) {};
		\node [style=none] (11) at (1.5, 0) {};
		\node [style=none] (12) at (2, 0) {};
		\node [style=Z] (17) at (1.5, 0.75) {};
		\node [style=X] (18) at (1, 0.5) {};
		\node [style=X] (19) at (1, 1) {};
		\node [style=Z] (20) at (2, 1.5) {};
		\node [style=none] (21) at (3, 2) {};
		\node [style=none] (22) at (3.5, 2) {};
		\node [style=none] (23) at (4, 2) {};
		\node [style=none] (24) at (1, 2) {};
		\node [style=none] (25) at (1.5, 2) {};
		\node [style=none] (26) at (2, 2) {};
		\node [style=Z] (27) at (1, 0) {};
		\node [style=Z] (28) at (1.5, 0) {};
		\node [style=Z] (29) at (2, 0) {};
	\end{pgfonlayer}
	\begin{pgfonlayer}{edgelayer}
		\draw (0.center) to (3);
		\draw (1.center) to (5);
		\draw (2.center) to (6);
		\draw (3) to (5);
		\draw (6) to (4);
		\draw (3) to (4);
		\draw (19) to (20);
		\draw (18) to (17);
		\draw (17) to (11.center);
		\draw (10.center) to (18);
		\draw (18) to (19);
		\draw (12.center) to (20);
		\draw (6) to (23.center);
		\draw (22.center) to (5);
		\draw (4) to (21.center);
		\draw (20) to (26.center);
		\draw (25.center) to (17);
		\draw (19) to (24.center);
	\end{pgfonlayer}
\end{tikzpicture}
\supseteq
\begin{tikzpicture}
	\begin{pgfonlayer}{nodelayer}
		\node [style=none] (934) at (102, 0) {};
		\node [style=none] (935) at (102.5, 0) {};
		\node [style=none] (936) at (103, 0) {};
		\node [style=Z] (937) at (102, 0.5) {};
		\node [style=Z] (938) at (102, 1) {};
		\node [style=X] (939) at (102.5, 0.75) {};
		\node [style=X] (940) at (103, 1.5) {};
		\node [style=X] (941) at (103, 0) {};
		\node [style=X] (942) at (102.5, 0) {};
		\node [style=none] (943) at (100.25, 0) {};
		\node [style=none] (944) at (100.75, 0) {};
		\node [style=none] (945) at (101.25, 0) {};
		\node [style=Z] (946) at (100.75, 0.75) {};
		\node [style=X] (947) at (100.25, 0.5) {};
		\node [style=X] (948) at (100.25, 1) {};
		\node [style=Z] (949) at (101.25, 1.5) {};
		\node [style=none] (950) at (102, 2) {};
		\node [style=none] (951) at (102.5, 2) {};
		\node [style=none] (952) at (103, 2) {};
		\node [style=none] (953) at (100.25, 2) {};
		\node [style=none] (954) at (100.75, 2) {};
		\node [style=none] (955) at (101.25, 2) {};
		\node [style=Z] (956) at (100.75, 0) {};
		\node [style=Z] (957) at (101.25, 0) {};
		\node [style=X] (958) at (100.25, 0) {};
		\node [style=X] (959) at (102, 0) {};
	\end{pgfonlayer}
	\begin{pgfonlayer}{edgelayer}
		\draw (934.center) to (937);
		\draw (935.center) to (939);
		\draw (936.center) to (940);
		\draw (937) to (939);
		\draw (940) to (938);
		\draw (937) to (938);
		\draw (948) to (949);
		\draw (947) to (946);
		\draw (946) to (944.center);
		\draw (943.center) to (947);
		\draw (947) to (948);
		\draw (945.center) to (949);
		\draw (940) to (952.center);
		\draw (951.center) to (939);
		\draw (938) to (950.center);
		\draw (949) to (955.center);
		\draw (954.center) to (946);
		\draw (948) to (953.center);
	\end{pgfonlayer}
\end{tikzpicture}
=
\begin{tikzpicture}
	\begin{pgfonlayer}{nodelayer}
		\node [style=none] (862) at (92, 0) {};
		\node [style=none] (863) at (92.5, 0) {};
		\node [style=none] (864) at (93, 0) {};
		\node [style=Z] (865) at (92, 0.5) {};
		\node [style=Z] (866) at (92, 1) {};
		\node [style=X] (867) at (92.5, 0.75) {};
		\node [style=X] (868) at (93, 1.5) {};
		\node [style=X] (869) at (93, 0) {};
		\node [style=X] (870) at (92.5, 0) {};
		\node [style=none] (872) at (93.75, 0) {};
		\node [style=none] (873) at (94.25, 0) {};
		\node [style=none] (874) at (94.75, 0) {};
		\node [style=Z] (875) at (94.25, 0.75) {};
		\node [style=X] (876) at (93.75, 0.5) {};
		\node [style=X] (877) at (93.75, 1) {};
		\node [style=Z] (878) at (94.75, 1.5) {};
		\node [style=none] (879) at (92, 2) {};
		\node [style=none] (880) at (92.5, 2) {};
		\node [style=none] (881) at (93, 2) {};
		\node [style=none] (882) at (93.75, 2) {};
		\node [style=none] (883) at (94.25, 2) {};
		\node [style=none] (884) at (94.75, 2) {};
		\node [style=Z] (886) at (94.25, 0) {};
		\node [style=Z] (887) at (94.75, 0) {};
		\node [style=X] (888) at (93.75, 0) {};
		\node [style=X] (889) at (92, 0) {};
		\node [style=s] (890) at (92, 2) {};
		\node [style=s] (891) at (92.5, 2) {};
		\node [style=s] (892) at (93, 2) {};
		\node [style=none] (893) at (93.75, 3.75) {};
		\node [style=none] (894) at (94.25, 3.75) {};
		\node [style=none] (895) at (94.75, 3.75) {};
		\node [style=none] (896) at (92, 3.75) {};
		\node [style=none] (897) at (92.5, 3.75) {};
		\node [style=none] (898) at (93, 3.75) {};
	\end{pgfonlayer}
	\begin{pgfonlayer}{edgelayer}
		\draw (862.center) to (865);
		\draw (863.center) to (867);
		\draw (864.center) to (868);
		\draw (865) to (867);
		\draw (868) to (866);
		\draw (865) to (866);
		\draw (877) to (878);
		\draw (876) to (875);
		\draw (875) to (873.center);
		\draw (872.center) to (876);
		\draw (876) to (877);
		\draw (874.center) to (878);
		\draw (868) to (881.center);
		\draw (880.center) to (867);
		\draw (866) to (879.center);
		\draw (878) to (884.center);
		\draw (883.center) to (875);
		\draw (877) to (882.center);
		\draw [in=270, out=90] (892) to (895.center);
		\draw [in=270, out=90] (891) to (894.center);
		\draw [in=270, out=90] (890) to (893.center);
		\draw [in=270, out=90] (882.center) to (896.center);
		\draw [in=90, out=-90] (897.center) to (883.center);
		\draw [in=270, out=90] (884.center) to (898.center);
	\end{pgfonlayer}
\end{tikzpicture}
=
\begin{tikzpicture}
	\begin{pgfonlayer}{nodelayer}
		\node [style=none] (853) at (88.75, 2) {};
		\node [style=none] (854) at (89.25, 2) {};
		\node [style=none] (855) at (89.75, 2) {};
		\node [style=none] (856) at (86.75, 2) {};
		\node [style=none] (857) at (87.25, 2) {};
		\node [style=none] (858) at (87.75, 2) {};
		\node [style=map] (859) at (88.25, 0.75) {$S^\omega$};
	\end{pgfonlayer}
	\begin{pgfonlayer}{edgelayer}
		\draw [in=-90, out=105] (859) to (858.center);
		\draw [in=135, out=-90, looseness=0.75] (857.center) to (859);
		\draw [in=-90, out=150, looseness=0.75] (859) to (856.center);
		\draw [in=75, out=-90] (853.center) to (859);
		\draw [in=-90, out=45, looseness=0.75] (859) to (854.center);
		\draw [in=30, out=-90, looseness=0.75] (855.center) to (859);
	\end{pgfonlayer}
\end{tikzpicture}
$$
Chopping off the maximally mixed state gives us an encoding map:
$$
\begin{tikzpicture}
	\begin{pgfonlayer}{nodelayer}
		\node [style=none] (2085) at (279.75, -3.25) {};
		\node [style=none] (2086) at (280.25, -3) {};
		\node [style=none] (2087) at (280.75, -3) {};
		\node [style=Z] (2088) at (279.75, -2.5) {};
		\node [style=Z] (2089) at (279.75, -2) {};
		\node [style=X] (2090) at (280.25, -2.25) {};
		\node [style=X] (2091) at (280.75, -1.5) {};
		\node [style=X] (2092) at (280.75, -3) {};
		\node [style=X] (2093) at (280.25, -3) {};
		\node [style=none] (2094) at (277.75, -3.25) {};
		\node [style=none] (2095) at (278.25, -3) {};
		\node [style=none] (2096) at (278.75, -3) {};
		\node [style=Z] (2097) at (278.25, -2.25) {};
		\node [style=X] (2098) at (277.75, -2.5) {};
		\node [style=X] (2099) at (277.75, -2) {};
		\node [style=Z] (2100) at (278.75, -1.5) {};
		\node [style=none] (2101) at (279.75, -1) {};
		\node [style=none] (2102) at (280.25, -1) {};
		\node [style=none] (2103) at (280.75, -1) {};
		\node [style=none] (2104) at (277.75, -1) {};
		\node [style=none] (2105) at (278.25, -1) {};
		\node [style=none] (2106) at (278.75, -1) {};
		\node [style=Z] (2107) at (278.25, -3) {};
		\node [style=Z] (2108) at (278.75, -3) {};
	\end{pgfonlayer}
	\begin{pgfonlayer}{edgelayer}
		\draw (2085.center) to (2088);
		\draw (2086.center) to (2090);
		\draw (2087.center) to (2091);
		\draw (2088) to (2090);
		\draw (2091) to (2089);
		\draw (2088) to (2089);
		\draw (2099) to (2100);
		\draw (2098) to (2097);
		\draw (2097) to (2095.center);
		\draw (2094.center) to (2098);
		\draw (2098) to (2099);
		\draw (2096.center) to (2100);
		\draw (2091) to (2103.center);
		\draw (2102.center) to (2090);
		\draw (2089) to (2101.center);
		\draw (2100) to (2106.center);
		\draw (2105.center) to (2097);
		\draw (2099) to (2104.center);
	\end{pgfonlayer}
\end{tikzpicture}
$$
Also, we will chose to measure in the $Z$ basis:
$$
\begin{tikzpicture}
	\begin{pgfonlayer}{nodelayer}
		\node [style=none] (2109) at (283.75, -6.25) {};
		\node [style=none] (2110) at (284.25, -6.5) {};
		\node [style=none] (2111) at (284.75, -6.5) {};
		\node [style=Z] (2112) at (283.75, -5.75) {};
		\node [style=Z] (2113) at (283.75, -5.25) {};
		\node [style=X] (2114) at (284.25, -5.5) {};
		\node [style=X] (2115) at (284.75, -4.75) {};
		\node [style=none] (2116) at (281.75, -6.25) {};
		\node [style=none] (2117) at (282.25, -6.25) {};
		\node [style=none] (2118) at (282.75, -6.25) {};
		\node [style=Z] (2119) at (282.25, -5.5) {};
		\node [style=X] (2120) at (281.75, -5.75) {};
		\node [style=X] (2121) at (281.75, -5.25) {};
		\node [style=Z] (2122) at (282.75, -4.75) {};
		\node [style=none] (2123) at (283.75, -4.25) {};
		\node [style=none] (2124) at (284.25, -4.25) {};
		\node [style=none] (2125) at (284.75, -4.25) {};
		\node [style=none] (2126) at (281.75, -4.25) {};
		\node [style=none] (2127) at (282.25, -4.25) {};
		\node [style=none] (2128) at (282.75, -4.25) {};
		\node [style=Z] (2129) at (282.25, -6.25) {};
		\node [style=Z] (2130) at (282.75, -6.25) {};
		\node [style=Z] (2131) at (283.75, -6.25) {};
		\node [style=Z] (2132) at (281.75, -6.25) {};
	\end{pgfonlayer}
	\begin{pgfonlayer}{edgelayer}
		\draw (2109.center) to (2112);
		\draw (2110.center) to (2114);
		\draw (2111.center) to (2115);
		\draw (2112) to (2114);
		\draw (2115) to (2113);
		\draw (2112) to (2113);
		\draw (2121) to (2122);
		\draw (2120) to (2119);
		\draw (2119) to (2117.center);
		\draw (2116.center) to (2120);
		\draw (2120) to (2121);
		\draw (2118.center) to (2122);
		\draw (2115) to (2125.center);
		\draw (2124.center) to (2114);
		\draw (2113) to (2123.center);
		\draw (2122) to (2128.center);
		\draw (2127.center) to (2119);
		\draw (2121) to (2126.center);
	\end{pgfonlayer}
\end{tikzpicture}
$$
Suppose there is a Pauli error $W((a,b,c),(d,e,f))$, then we have the following error detection circuit:
$$
\begin{tikzpicture}
	\begin{pgfonlayer}{nodelayer}
		\node [style=none] (1843) at (254.5, -25.5) {};
		\node [style=none] (1844) at (255, -25.25) {};
		\node [style=none] (1845) at (255.5, -25.25) {};
		\node [style=Z] (1846) at (254.5, -24.75) {};
		\node [style=Z] (1847) at (254.5, -24.25) {};
		\node [style=X] (1848) at (255, -24.5) {};
		\node [style=X] (1849) at (255.5, -23.75) {};
		\node [style=X] (1850) at (255.5, -25.25) {};
		\node [style=X] (1851) at (255, -25.25) {};
		\node [style=none] (1852) at (252.75, -25.5) {};
		\node [style=none] (1853) at (253.25, -25.25) {};
		\node [style=none] (1854) at (253.75, -25.25) {};
		\node [style=Z] (1855) at (253.25, -24.5) {};
		\node [style=X] (1856) at (252.75, -24.75) {};
		\node [style=X] (1857) at (252.75, -24.25) {};
		\node [style=Z] (1858) at (253.75, -23.75) {};
		\node [style=none] (1859) at (254.5, -23.75) {};
		\node [style=none] (1860) at (255, -23.75) {};
		\node [style=none] (1861) at (255.5, -23.75) {};
		\node [style=none] (1862) at (252.75, -23.75) {};
		\node [style=none] (1863) at (253.25, -23.75) {};
		\node [style=none] (1864) at (253.75, -23.75) {};
		\node [style=Z] (1865) at (253.25, -25.25) {};
		\node [style=Z] (1866) at (253.75, -25.25) {};
		\node [style=none] (1867) at (258, -20.75) {};
		\node [style=none] (1868) at (258.5, -20.75) {};
		\node [style=none] (1869) at (259, -20.75) {};
		\node [style=none] (1870) at (255.5, -20.75) {};
		\node [style=none] (1871) at (256, -20.75) {};
		\node [style=none] (1872) at (256.5, -20.75) {};
		\node [style=X] (1873) at (254.5, -22) {$a$};
		\node [style=X] (1874) at (255, -22) {$b$};
		\node [style=X] (1875) at (255.5, -22) {$c$};
		\node [style=X] (1876) at (256.5, -22) {$d$};
		\node [style=X] (1877) at (257, -22) {$e$};
		\node [style=X] (1878) at (257.5, -22) {$f$};
		\node [style=none] (1879) at (258, -20.75) {};
		\node [style=none] (1880) at (258.5, -20.75) {};
		\node [style=none] (1881) at (259, -20.75) {};
		\node [style=none] (1882) at (255.5, -20.75) {};
		\node [style=none] (1883) at (256, -20.75) {};
		\node [style=none] (1884) at (256.5, -20.75) {};
		\node [style=Z] (1885) at (258, -18.5) {};
		\node [style=Z] (1886) at (258, -18) {};
		\node [style=X] (1887) at (258.5, -18.25) {};
		\node [style=X] (1888) at (259, -17.5) {};
		\node [style=Z] (1889) at (256, -18.25) {};
		\node [style=X] (1890) at (255.5, -18.5) {};
		\node [style=X] (1891) at (255.5, -18) {};
		\node [style=Z] (1892) at (256.5, -17.5) {};
		\node [style=none] (1893) at (258, -16.25) {};
		\node [style=none] (1894) at (258.5, -16.25) {};
		\node [style=none] (1895) at (259, -16.25) {};
		\node [style=none] (1896) at (255.5, -16.25) {};
		\node [style=none] (1897) at (256, -16.25) {};
		\node [style=none] (1898) at (256.5, -16.25) {};
		\node [style=Z] (1899) at (258, -19.75) {};
		\node [style=Z] (1900) at (258, -20.25) {};
		\node [style=X] (1901) at (258.5, -20) {};
		\node [style=X] (1902) at (259, -20.75) {};
		\node [style=Z] (1903) at (256, -20) {};
		\node [style=X] (1904) at (255.5, -19.75) {};
		\node [style=X] (1905) at (255.5, -20.25) {};
		\node [style=Z] (1906) at (256.5, -20.75) {};
		\node [style=none] (1907) at (258, -20.75) {};
		\node [style=none] (1908) at (258.5, -20.75) {};
		\node [style=none] (1909) at (259, -20.75) {};
		\node [style=none] (1910) at (255.5, -20.75) {};
		\node [style=none] (1911) at (256, -20.75) {};
		\node [style=none] (1912) at (256.5, -20.75) {};
		\node [style=X] (1913) at (259.5, -18.5) {};
		\node [style=X] (1914) at (260, -18.75) {};
		\node [style=Z] (1915) at (257, -18.25) {};
		\node [style=Z] (1916) at (257.5, -18.5) {};
		\node [style=Z] (1917) at (257, -19.75) {};
		\node [style=Z] (1918) at (257.5, -19.75) {};
		\node [style=X] (1919) at (259.5, -19.75) {};
		\node [style=X] (1920) at (260, -19.75) {};
		\node [style=Z] (1921) at (257, -17.25) {};
		\node [style=Z] (1922) at (257.5, -17.25) {};
		\node [style=none] (1923) at (259.5, -17.25) {};
		\node [style=none] (1924) at (260, -17.25) {};
		\node [style=Z] (1925) at (258.5, -19) {};
		\node [style=Z] (1926) at (259, -19.25) {};
		\node [style=X] (1927) at (256, -18.75) {};
		\node [style=X] (1928) at (256.5, -19) {};
		\node [style=none] (1929) at (260, -16.25) {};
		\node [style=none] (1930) at (260.5, -16.25) {};
		\node [style=none] (1931) at (254, -16.25) {};
		\node [style=none] (1932) at (256.75, -25.5) {};
		\node [style=none] (1933) at (253.5, -20.75) {Alice};
		\node [style=none] (1934) at (257.75, -23.25) {Bob};
	\end{pgfonlayer}
	\begin{pgfonlayer}{edgelayer}
		\draw (1843.center) to (1846);
		\draw (1844.center) to (1848);
		\draw (1845.center) to (1849);
		\draw (1846) to (1848);
		\draw (1849) to (1847);
		\draw (1846) to (1847);
		\draw (1857) to (1858);
		\draw (1856) to (1855);
		\draw (1855) to (1853.center);
		\draw (1852.center) to (1856);
		\draw (1856) to (1857);
		\draw (1854.center) to (1858);
		\draw (1860.center) to (1848);
		\draw (1847) to (1859.center);
		\draw (1863.center) to (1855);
		\draw (1857) to (1862.center);
		\draw [in=-135, out=90, looseness=1.25] (1862.center) to (1873);
		\draw [in=45, out=-90] (1870.center) to (1873);
		\draw [in=-135, out=90, looseness=1.25] (1863.center) to (1874);
		\draw [in=270, out=45] (1874) to (1871.center);
		\draw [in=45, out=-90] (1872.center) to (1875);
		\draw [in=90, out=-135, looseness=1.25] (1875) to (1864.center);
		\draw [in=45, out=-90] (1867.center) to (1876);
		\draw [in=90, out=-135, looseness=1.25] (1876) to (1859.center);
		\draw [in=-135, out=90, looseness=1.25] (1860.center) to (1877);
		\draw [in=270, out=45] (1877) to (1868.center);
		\draw [in=45, out=-90] (1869.center) to (1878);
		\draw [in=90, out=-135, looseness=1.25] (1878) to (1861.center);
		\draw (1885) to (1887);
		\draw (1888) to (1886);
		\draw (1885) to (1886);
		\draw (1891) to (1892);
		\draw (1890) to (1889);
		\draw (1890) to (1891);
		\draw (1888) to (1895.center);
		\draw (1894.center) to (1887);
		\draw (1886) to (1893.center);
		\draw (1892) to (1898.center);
		\draw (1897.center) to (1889);
		\draw (1891) to (1896.center);
		\draw (1899) to (1901);
		\draw (1902) to (1900);
		\draw (1899) to (1900);
		\draw (1905) to (1906);
		\draw (1904) to (1903);
		\draw (1904) to (1905);
		\draw (1908.center) to (1901);
		\draw (1900) to (1907.center);
		\draw (1911.center) to (1903);
		\draw (1905) to (1910.center);
		\draw (1899) to (1885);
		\draw (1918) to (1916);
		\draw (1916) to (1922);
		\draw (1921) to (1915);
		\draw (1915) to (1917);
		\draw (1919) to (1913);
		\draw (1914) to (1920);
		\draw (1913) to (1923.center);
		\draw (1914) to (1924.center);
		\draw (1906) to (1928);
		\draw (1928) to (1892);
		\draw (1927) to (1889);
		\draw (1903) to (1927);
		\draw (1904) to (1890);
		\draw (1925) to (1901);
		\draw (1888) to (1926);
		\draw (1926) to (1902);
		\draw (1925) to (1887);
		\draw (1928) to (1916);
		\draw (1927) to (1915);
		\draw (1925) to (1913);
		\draw (1926) to (1914);
		\draw [style=red, in=270, out=90] (1923.center) to (1929.center);
		\draw [style=red, in=270, out=90] (1924.center) to (1930.center);
		\draw [style=dotted, in=90, out=-90, looseness=1.25] (1931.center) to (1932.center);
	\end{pgfonlayer}
\end{tikzpicture}
=
\begin{tikzpicture}
	\begin{pgfonlayer}{nodelayer}
		\node [style=none] (1265) at (192.5, -21.25) {};
		\node [style=none] (1266) at (193, -21) {};
		\node [style=none] (1267) at (193.5, -21) {};
		\node [style=Z] (1268) at (192.5, -20.5) {};
		\node [style=Z] (1269) at (192.5, -20) {};
		\node [style=X] (1270) at (193, -20.25) {};
		\node [style=X] (1271) at (193.5, -19.5) {};
		\node [style=X] (1272) at (193.5, -21) {};
		\node [style=X] (1273) at (193, -21) {};
		\node [style=none] (1274) at (190.5, -21.25) {};
		\node [style=none] (1275) at (191, -21) {};
		\node [style=none] (1276) at (191.5, -21) {};
		\node [style=Z] (1277) at (191, -20.25) {};
		\node [style=X] (1278) at (190.5, -20.5) {};
		\node [style=X] (1279) at (190.5, -20) {};
		\node [style=Z] (1280) at (191.5, -19.5) {};
		\node [style=Z] (1281) at (191, -21) {};
		\node [style=Z] (1282) at (191.5, -21) {};
		\node [style=none] (1283) at (192.5, -18) {};
		\node [style=none] (1284) at (193, -18) {};
		\node [style=none] (1285) at (193.5, -18) {};
		\node [style=none] (1286) at (190.5, -18) {};
		\node [style=none] (1287) at (191, -18) {};
		\node [style=none] (1288) at (191.5, -18) {};
		\node [style=X] (1289) at (193, -18.75) {$e$};
		\node [style=X] (1290) at (193.5, -18.75) {$f$};
		\node [style=X] (1291) at (192.5, -18.75) {$d$};
		\node [style=X] (1294) at (194.5, -18.75) {$e+d$};
		\node [style=X] (1295) at (195.5, -18.75) {$f+d$};
		\node [style=X] (1296) at (190.5, -18.75) {$a$};
		\node [style=X] (1297) at (191, -18.75) {$b$};
		\node [style=X] (1298) at (191.5, -18.75) {$c$};
		\node [style=none] (1299) at (194.5, -18) {};
		\node [style=none] (1300) at (195.5, -18) {};
	\end{pgfonlayer}
	\begin{pgfonlayer}{edgelayer}
		\draw (1265.center) to (1268);
		\draw (1266.center) to (1270);
		\draw (1267.center) to (1271);
		\draw (1268) to (1270);
		\draw (1271) to (1269);
		\draw (1268) to (1269);
		\draw (1279) to (1280);
		\draw (1278) to (1277);
		\draw (1277) to (1275.center);
		\draw (1274.center) to (1278);
		\draw (1278) to (1279);
		\draw (1276.center) to (1280);
		\draw (1277) to (1287.center);
		\draw (1280) to (1288.center);
		\draw (1269) to (1291);
		\draw (1291) to (1283.center);
		\draw (1284.center) to (1289);
		\draw (1289) to (1270);
		\draw (1271) to (1290);
		\draw (1290) to (1285.center);
		\draw (1279) to (1296);
		\draw (1296) to (1286.center);
		\draw [style=red] (1294) to (1299.center);
		\draw [style=red] (1295) to (1300.center);
	\end{pgfonlayer}
\end{tikzpicture}
$$
Suppose we want to correct for single Pauli $X$ errors, then we find that:
\begin{itemize}
\item Pauli $X$ error $(d,e,f) = (1,0,0)$ yields syndrome $(e+d,f+d) = (1,1)$
\item Pauli $X$ error $(d,e,f) = (0,1,0)$ yields syndrome $(e+d,f+d) = (1,0)$
\item Pauli $X$ error$(d,e,f) = (0,0,1)$ yields syndrome $(e+d,f+d) = (0,1)$
\item Pauli $X$ error $(d,e,f) = (0,0,0)$ yields syndrome $(e+d,f+d) = (0,0)$
\end{itemize}
Therefore, we want to apply the correction $(s,t) \mapsto (s t, s (t+1),(s+1) t)$. The error correction protocol then has the following form:
$$
\begin{tikzpicture}
	\begin{pgfonlayer}{nodelayer}
		\node [style=none] (2375) at (319, -25.5) {};
		\node [style=none] (2376) at (319.5, -25.25) {};
		\node [style=none] (2377) at (320, -25.25) {};
		\node [style=Z] (2378) at (319, -24.75) {};
		\node [style=Z] (2379) at (319, -24.25) {};
		\node [style=X] (2380) at (319.5, -24.5) {};
		\node [style=X] (2381) at (320, -23.75) {};
		\node [style=X] (2382) at (320, -25.25) {};
		\node [style=X] (2383) at (319.5, -25.25) {};
		\node [style=none] (2384) at (316.5, -25.5) {};
		\node [style=none] (2385) at (317, -25.25) {};
		\node [style=none] (2386) at (317.5, -25.25) {};
		\node [style=Z] (2387) at (317, -24.5) {};
		\node [style=X] (2388) at (316.5, -24.75) {};
		\node [style=X] (2389) at (316.5, -24.25) {};
		\node [style=Z] (2390) at (317.5, -23.75) {};
		\node [style=none] (2391) at (319, -23.75) {};
		\node [style=none] (2392) at (319.5, -23.75) {};
		\node [style=none] (2393) at (320, -23.75) {};
		\node [style=none] (2394) at (316.5, -23.75) {};
		\node [style=none] (2395) at (317, -23.75) {};
		\node [style=none] (2396) at (317.5, -23.75) {};
		\node [style=Z] (2397) at (317, -25.25) {};
		\node [style=Z] (2398) at (317.5, -25.25) {};
		\node [style=none] (2399) at (321.75, -20.75) {};
		\node [style=none] (2400) at (322.25, -20.75) {};
		\node [style=none] (2401) at (322.75, -20.75) {};
		\node [style=none] (2402) at (319.25, -20.75) {};
		\node [style=none] (2403) at (319.75, -20.75) {};
		\node [style=none] (2404) at (320.25, -20.75) {};
		\node [style=X] (2405) at (318, -22) {$a$};
		\node [style=X] (2406) at (318.5, -22) {$b$};
		\node [style=X] (2407) at (319, -22) {$c$};
		\node [style=X] (2408) at (320.5, -22) {$d$};
		\node [style=X] (2409) at (321.25, -22) {$e$};
		\node [style=X] (2410) at (321.75, -22) {$f$};
		\node [style=none] (2411) at (321.75, -20.75) {};
		\node [style=none] (2412) at (322.25, -20.75) {};
		\node [style=none] (2413) at (322.75, -20.75) {};
		\node [style=none] (2414) at (319.25, -20.75) {};
		\node [style=none] (2415) at (319.75, -20.75) {};
		\node [style=none] (2416) at (320.25, -20.75) {};
		\node [style=Z] (2417) at (321.75, -18.5) {};
		\node [style=Z] (2418) at (321.75, -18) {};
		\node [style=X] (2419) at (322.25, -18.25) {};
		\node [style=X] (2420) at (322.75, -17.5) {};
		\node [style=Z] (2421) at (319.75, -18.25) {};
		\node [style=X] (2422) at (319.25, -18.5) {};
		\node [style=X] (2423) at (319.25, -18) {};
		\node [style=Z] (2424) at (320.25, -17.5) {};
		\node [style=none] (2425) at (321.75, -11.5) {};
		\node [style=none] (2426) at (322.25, -11.75) {};
		\node [style=none] (2427) at (322.75, -11.75) {};
		\node [style=none] (2428) at (319.25, -11.5) {};
		\node [style=none] (2429) at (319.75, -11.75) {};
		\node [style=none] (2430) at (320.25, -11.75) {};
		\node [style=Z] (2431) at (321.75, -19.75) {};
		\node [style=Z] (2432) at (321.75, -20.25) {};
		\node [style=X] (2433) at (322.25, -20) {};
		\node [style=X] (2434) at (322.75, -20.75) {};
		\node [style=Z] (2435) at (319.75, -20) {};
		\node [style=X] (2436) at (319.25, -19.75) {};
		\node [style=X] (2437) at (319.25, -20.25) {};
		\node [style=Z] (2438) at (320.25, -20.75) {};
		\node [style=none] (2439) at (321.75, -20.75) {};
		\node [style=none] (2440) at (322.25, -20.75) {};
		\node [style=none] (2441) at (322.75, -20.75) {};
		\node [style=none] (2442) at (319.25, -20.75) {};
		\node [style=none] (2443) at (319.75, -20.75) {};
		\node [style=none] (2444) at (320.25, -20.75) {};
		\node [style=X] (2445) at (324.25, -17.75) {};
		\node [style=X] (2446) at (324.75, -18) {};
		\node [style=Z] (2447) at (320.75, -18.25) {};
		\node [style=Z] (2448) at (321.25, -18.5) {};
		\node [style=Z] (2449) at (320.75, -19.25) {};
		\node [style=Z] (2450) at (321.25, -19.25) {};
		\node [style=X] (2451) at (324.25, -19.25) {};
		\node [style=X] (2452) at (324.75, -19.25) {};
		\node [style=Z] (2453) at (320.75, -17.25) {};
		\node [style=Z] (2454) at (321.25, -17.25) {};
		\node [style=none] (2455) at (324.25, -17.25) {};
		\node [style=none] (2456) at (324.75, -17.25) {};
		\node [style=Z] (2457) at (322.25, -18.75) {};
		\node [style=Z] (2458) at (322.75, -19) {};
		\node [style=X] (2459) at (319.75, -18.75) {};
		\node [style=X] (2460) at (320.25, -19) {};
		\node [style=none] (2461) at (324.75, -16.75) {};
		\node [style=none] (2462) at (317.75, -16.75) {};
		\node [style=none] (2463) at (320.75, -25.5) {};
		\node [style=none] (2464) at (317.5, -20.75) {Alice};
		\node [style=none] (2465) at (324, -21) {Bob};
		\node [style=Z] (2466) at (324.25, -16.75) {};
		\node [style=Z] (2467) at (324.75, -16.75) {};
		\node [style=X] (2468) at (322.75, -13.75) {};
		\node [style=X] (2469) at (322.25, -14) {};
		\node [style=X] (2470) at (321.75, -14.25) {};
		\node [style=none] (2471) at (323.25, -15) {};
		\node [style=none] (2472) at (325.75, -11.5) {};
		\node [style=none] (2473) at (326.25, -11.5) {};
		\node [style=none] (2474) at (317.75, -11.5) {};
		\node [style=andin] (2475) at (325, -14.75) {};
		\node [style=Z] (2476) at (321.75, -12.25) {};
		\node [style=Z] (2477) at (321.75, -12.75) {};
		\node [style=X] (2478) at (322.25, -12.5) {};
		\node [style=X] (2479) at (322.75, -13.25) {};
		\node [style=Z] (2480) at (319.75, -12.5) {};
		\node [style=X] (2481) at (319.25, -12.25) {};
		\node [style=X] (2482) at (319.25, -12.75) {};
		\node [style=Z] (2483) at (320.25, -13.25) {};
		\node [style=Z] (2484) at (319.75, -11.75) {};
		\node [style=Z] (2485) at (322.25, -11.75) {};
		\node [style=Z] (2486) at (322.75, -11.75) {};
		\node [style=Z] (2487) at (320.25, -11.75) {};
		\node [style=none] (2488) at (324, -14.75) {};
		\node [style=andin] (2489) at (323.25, -15) {};
		\node [style=none] (2490) at (325, -14.75) {};
		\node [style=andin] (2491) at (324, -14.75) {};
		\node [style=X] (2492) at (324.25, -15.5) {$1$};
		\node [style=X] (2493) at (324.75, -15.5) {$1$};
	\end{pgfonlayer}
	\begin{pgfonlayer}{edgelayer}
		\draw (2375.center) to (2378);
		\draw (2376.center) to (2380);
		\draw (2377.center) to (2381);
		\draw (2378) to (2380);
		\draw (2381) to (2379);
		\draw (2378) to (2379);
		\draw (2389) to (2390);
		\draw (2388) to (2387);
		\draw (2387) to (2385.center);
		\draw (2384.center) to (2388);
		\draw (2388) to (2389);
		\draw (2386.center) to (2390);
		\draw (2392.center) to (2380);
		\draw (2379) to (2391.center);
		\draw (2395.center) to (2387);
		\draw (2389) to (2394.center);
		\draw [in=-135, out=90, looseness=1.25] (2394.center) to (2405);
		\draw [in=45, out=-90] (2402.center) to (2405);
		\draw [in=-135, out=90, looseness=1.25] (2395.center) to (2406);
		\draw [in=270, out=45] (2406) to (2403.center);
		\draw [in=45, out=-90] (2404.center) to (2407);
		\draw [in=90, out=-135, looseness=1.25] (2407) to (2396.center);
		\draw [in=45, out=-90] (2399.center) to (2408);
		\draw [in=90, out=-135, looseness=1.25] (2408) to (2391.center);
		\draw [in=-135, out=90, looseness=1.25] (2392.center) to (2409);
		\draw [in=270, out=45] (2409) to (2400.center);
		\draw [in=45, out=-90] (2401.center) to (2410);
		\draw [in=90, out=-135, looseness=1.25] (2410) to (2393.center);
		\draw (2417) to (2419);
		\draw (2420) to (2418);
		\draw (2417) to (2418);
		\draw (2423) to (2424);
		\draw (2422) to (2421);
		\draw (2422) to (2423);
		\draw (2420) to (2427.center);
		\draw (2426.center) to (2419);
		\draw (2418) to (2425.center);
		\draw (2424) to (2430.center);
		\draw (2429.center) to (2421);
		\draw (2423) to (2428.center);
		\draw (2431) to (2433);
		\draw (2434) to (2432);
		\draw (2431) to (2432);
		\draw (2437) to (2438);
		\draw (2436) to (2435);
		\draw (2436) to (2437);
		\draw (2440.center) to (2433);
		\draw (2432) to (2439.center);
		\draw (2443.center) to (2435);
		\draw (2437) to (2442.center);
		\draw (2431) to (2417);
		\draw (2450) to (2448);
		\draw (2448) to (2454);
		\draw (2453) to (2447);
		\draw (2447) to (2449);
		\draw (2451) to (2445);
		\draw (2446) to (2452);
		\draw (2445) to (2455.center);
		\draw (2446) to (2456.center);
		\draw (2438) to (2460);
		\draw (2460) to (2424);
		\draw (2459) to (2421);
		\draw (2435) to (2459);
		\draw (2436) to (2422);
		\draw (2457) to (2433);
		\draw (2420) to (2458);
		\draw (2458) to (2434);
		\draw (2457) to (2419);
		\draw (2460) to (2448);
		\draw (2459) to (2447);
		\draw (2457) to (2445);
		\draw (2458) to (2446);
		\draw [style=red] (2456.center) to (2461.center);
		\draw [style=dotted, in=90, out=-90, looseness=1.25] (2462.center) to (2463.center);
		\draw (2425.center) to (2470);
		\draw [style=red, in=-60, out=135] (2467) to (2471.center);
		\draw [style=red, in=-105, out=165] (2466) to (2471.center);
		\draw [style=red, in=90, out=-45] (2470) to (2471.center);
		\draw [style=red, in=-90, out=45] (2466) to (2472.center);
		\draw [style=red, in=270, out=15, looseness=0.75] (2467) to (2473.center);
		\draw (2476) to (2478);
		\draw (2479) to (2477);
		\draw (2482) to (2483);
		\draw (2481) to (2480);
		\draw [style=red] (2455.center) to (2466);
		\draw [style=dotted] (2462.center) to (2474.center);
		\draw [style=red, in=-135, out=135] (2466) to (2488.center);
		\draw [style=red, in=-90, out=90] (2467) to (2492);
		\draw [style=red, in=-45, out=90] (2492) to (2488.center);
		\draw [style=red, in=-105, out=90] (2466) to (2493);
		\draw [style=red, in=-135, out=90] (2493) to (2490.center);
		\draw [style=red, in=45, out=-30] (2490.center) to (2467);
		\draw [style=red, in=-30, out=90] (2490.center) to (2468);
		\draw [style=red, in=-30, out=90] (2488.center) to (2469);
	\end{pgfonlayer}
\end{tikzpicture}
=
\begin{tikzpicture}
	\begin{pgfonlayer}{nodelayer}
		\node [style=none] (1741) at (242.5, -21.25) {};
		\node [style=none] (1742) at (243, -21) {};
		\node [style=none] (1743) at (243.5, -21) {};
		\node [style=Z] (1744) at (242.5, -20.5) {};
		\node [style=Z] (1745) at (242.5, -20) {};
		\node [style=X] (1746) at (243, -20.25) {};
		\node [style=X] (1747) at (243.5, -19.5) {};
		\node [style=X] (1748) at (243.5, -21) {};
		\node [style=X] (1749) at (243, -21) {};
		\node [style=none] (1750) at (240.5, -21.25) {};
		\node [style=none] (1751) at (241, -21) {};
		\node [style=none] (1752) at (241.5, -21) {};
		\node [style=Z] (1753) at (241, -20.25) {};
		\node [style=X] (1754) at (240.5, -20.5) {};
		\node [style=X] (1755) at (240.5, -20) {};
		\node [style=Z] (1756) at (241.5, -19.5) {};
		\node [style=Z] (1757) at (241, -21) {};
		\node [style=Z] (1758) at (241.5, -21) {};
		\node [style=none] (1759) at (242.5, -18) {};
		\node [style=none] (1760) at (243, -18) {};
		\node [style=none] (1761) at (243.5, -18) {};
		\node [style=none] (1762) at (240.5, -18) {};
		\node [style=none] (1763) at (241, -18) {};
		\node [style=none] (1764) at (241.5, -18) {};
		\node [style=X] (1765) at (243, -18.75) {$g$};
		\node [style=X] (1766) at (243.5, -18.75) {$g$};
		\node [style=X] (1767) at (242.5, -18.75) {$g$};
		\node [style=X] (1768) at (244.5, -17.5) {$e+d$};
		\node [style=X] (1769) at (245.5, -17.5) {$f+d$};
		\node [style=X] (1770) at (240.5, -18.75) {$a$};
		\node [style=X] (1771) at (241, -18.75) {$b$};
		\node [style=X] (1772) at (241.5, -18.75) {$c$};
		\node [style=none] (1773) at (244.5, -16.25) {};
		\node [style=none] (1774) at (245.5, -16.25) {};
		\node [style=Z] (1775) at (242.5, -17) {};
		\node [style=Z] (1776) at (242.5, -17.5) {};
		\node [style=X] (1777) at (243, -17.25) {};
		\node [style=X] (1778) at (243.5, -18) {};
		\node [style=Z] (1779) at (241, -17.25) {};
		\node [style=X] (1780) at (240.5, -17) {};
		\node [style=X] (1781) at (240.5, -17.5) {};
		\node [style=Z] (1782) at (241.5, -18) {};
		\node [style=Z] (1789) at (241, -16.5) {};
		\node [style=Z] (1790) at (241.5, -16.5) {};
		\node [style=Z] (1791) at (243, -16.5) {};
		\node [style=Z] (1792) at (243.5, -16.5) {};
		\node [style=none] (1793) at (242.5, -16.25) {};
		\node [style=none] (1794) at (240.5, -16.25) {};
	\end{pgfonlayer}
	\begin{pgfonlayer}{edgelayer}
		\draw (1741.center) to (1744);
		\draw (1742.center) to (1746);
		\draw (1743.center) to (1747);
		\draw (1744) to (1746);
		\draw (1747) to (1745);
		\draw (1744) to (1745);
		\draw (1755) to (1756);
		\draw (1754) to (1753);
		\draw (1753) to (1751.center);
		\draw (1750.center) to (1754);
		\draw (1754) to (1755);
		\draw (1752.center) to (1756);
		\draw (1753) to (1763.center);
		\draw (1756) to (1764.center);
		\draw (1745) to (1767);
		\draw (1767) to (1759.center);
		\draw (1760.center) to (1765);
		\draw (1765) to (1746);
		\draw (1747) to (1766);
		\draw (1766) to (1761.center);
		\draw (1755) to (1770);
		\draw (1770) to (1762.center);
		\draw [style=red] (1768) to (1773.center);
		\draw [style=red] (1769) to (1774.center);
		\draw (1775) to (1777);
		\draw (1778) to (1776);
		\draw (1775) to (1776);
		\draw (1781) to (1782);
		\draw (1780) to (1779);
		\draw (1780) to (1781);
		\draw (1780) to (1794.center);
		\draw (1763.center) to (1789);
		\draw (1790) to (1782);
		\draw (1775) to (1793.center);
		\draw (1791) to (1777);
		\draw (1778) to (1792);
		\draw (1777) to (1760.center);
		\draw (1759.center) to (1776);
		\draw (1781) to (1762.center);
	\end{pgfonlayer}
\end{tikzpicture}
$$
Where 
$$
g:=d+(e+d)(f+d) = e+(e+d)(f+d+1) =f+(e+d+1)(f+d) = de+ef+fd
\mod 2$$
If no more than one of $d,e,f$ is $1$ then $g=0$.  If furthermore $a=b=c=0$, then:
$$
\begin{tikzpicture}
	\begin{pgfonlayer}{nodelayer}
		\node [style=none] (1741) at (242.5, -21.25) {};
		\node [style=none] (1742) at (243, -21) {};
		\node [style=none] (1743) at (243.5, -21) {};
		\node [style=Z] (1744) at (242.5, -20.5) {};
		\node [style=Z] (1745) at (242.5, -20) {};
		\node [style=X] (1746) at (243, -20.25) {};
		\node [style=X] (1747) at (243.5, -19.5) {};
		\node [style=X] (1748) at (243.5, -21) {};
		\node [style=X] (1749) at (243, -21) {};
		\node [style=none] (1750) at (240.5, -21.25) {};
		\node [style=none] (1751) at (241, -21) {};
		\node [style=none] (1752) at (241.5, -21) {};
		\node [style=Z] (1753) at (241, -20.25) {};
		\node [style=X] (1754) at (240.5, -20.5) {};
		\node [style=X] (1755) at (240.5, -20) {};
		\node [style=Z] (1756) at (241.5, -19.5) {};
		\node [style=Z] (1757) at (241, -21) {};
		\node [style=Z] (1758) at (241.5, -21) {};
		\node [style=none] (1759) at (242.5, -18) {};
		\node [style=none] (1760) at (243, -18) {};
		\node [style=none] (1761) at (243.5, -18) {};
		\node [style=none] (1762) at (240.5, -18) {};
		\node [style=none] (1763) at (241, -18) {};
		\node [style=none] (1764) at (241.5, -18) {};
		\node [style=X] (1765) at (243, -18.75) {$g$};
		\node [style=X] (1766) at (243.5, -18.75) {$g$};
		\node [style=X] (1767) at (242.5, -18.75) {$g$};
		\node [style=X] (1768) at (244.5, -17.5) {$e+d$};
		\node [style=X] (1769) at (245.5, -17.5) {$f+d$};
		\node [style=X] (1770) at (240.5, -18.75) {$a$};
		\node [style=X] (1771) at (241, -18.75) {$b$};
		\node [style=X] (1772) at (241.5, -18.75) {$c$};
		\node [style=none] (1773) at (244.5, -16.25) {};
		\node [style=none] (1774) at (245.5, -16.25) {};
		\node [style=Z] (1775) at (242.5, -17) {};
		\node [style=Z] (1776) at (242.5, -17.5) {};
		\node [style=X] (1777) at (243, -17.25) {};
		\node [style=X] (1778) at (243.5, -18) {};
		\node [style=Z] (1779) at (241, -17.25) {};
		\node [style=X] (1780) at (240.5, -17) {};
		\node [style=X] (1781) at (240.5, -17.5) {};
		\node [style=Z] (1782) at (241.5, -18) {};
		\node [style=Z] (1789) at (241, -16.5) {};
		\node [style=Z] (1790) at (241.5, -16.5) {};
		\node [style=Z] (1791) at (243, -16.5) {};
		\node [style=Z] (1792) at (243.5, -16.5) {};
		\node [style=none] (1793) at (242.5, -16.25) {};
		\node [style=none] (1794) at (240.5, -16.25) {};
	\end{pgfonlayer}
	\begin{pgfonlayer}{edgelayer}
		\draw (1741.center) to (1744);
		\draw (1742.center) to (1746);
		\draw (1743.center) to (1747);
		\draw (1744) to (1746);
		\draw (1747) to (1745);
		\draw (1744) to (1745);
		\draw (1755) to (1756);
		\draw (1754) to (1753);
		\draw (1753) to (1751.center);
		\draw (1750.center) to (1754);
		\draw (1754) to (1755);
		\draw (1752.center) to (1756);
		\draw (1753) to (1763.center);
		\draw (1756) to (1764.center);
		\draw (1745) to (1767);
		\draw (1767) to (1759.center);
		\draw (1760.center) to (1765);
		\draw (1765) to (1746);
		\draw (1747) to (1766);
		\draw (1766) to (1761.center);
		\draw (1755) to (1770);
		\draw (1770) to (1762.center);
		\draw [style=red] (1768) to (1773.center);
		\draw [style=red] (1769) to (1774.center);
		\draw (1775) to (1777);
		\draw (1778) to (1776);
		\draw (1775) to (1776);
		\draw (1781) to (1782);
		\draw (1780) to (1779);
		\draw (1780) to (1781);
		\draw (1780) to (1794.center);
		\draw (1763.center) to (1789);
		\draw (1790) to (1782);
		\draw (1775) to (1793.center);
		\draw (1791) to (1777);
		\draw (1778) to (1792);
		\draw (1777) to (1760.center);
		\draw (1759.center) to (1776);
		\draw (1781) to (1762.center);
	\end{pgfonlayer}
\end{tikzpicture}
=
\begin{tikzpicture}
	\begin{pgfonlayer}{nodelayer}
		\node [style=none] (1795) at (248.5, -21.25) {};
		\node [style=none] (1796) at (249, -21) {};
		\node [style=none] (1797) at (249.5, -21) {};
		\node [style=Z] (1798) at (248.5, -20.5) {};
		\node [style=Z] (1799) at (248.5, -20) {};
		\node [style=X] (1800) at (249, -20.25) {};
		\node [style=X] (1801) at (249.5, -19.5) {};
		\node [style=X] (1802) at (249.5, -21) {};
		\node [style=X] (1803) at (249, -21) {};
		\node [style=none] (1804) at (246.5, -21.25) {};
		\node [style=none] (1805) at (247, -21) {};
		\node [style=none] (1806) at (247.5, -21) {};
		\node [style=Z] (1807) at (247, -20.25) {};
		\node [style=X] (1808) at (246.5, -20.5) {};
		\node [style=X] (1809) at (246.5, -20) {};
		\node [style=Z] (1810) at (247.5, -19.5) {};
		\node [style=Z] (1811) at (247, -21) {};
		\node [style=Z] (1812) at (247.5, -21) {};
		\node [style=none] (1813) at (248.5, -18) {};
		\node [style=none] (1814) at (249, -18) {};
		\node [style=none] (1815) at (249.5, -18) {};
		\node [style=none] (1816) at (246.5, -18) {};
		\node [style=none] (1817) at (247, -18) {};
		\node [style=none] (1818) at (247.5, -18) {};
		\node [style=X] (1822) at (250.5, -17.5) {$e+d$};
		\node [style=X] (1823) at (251.5, -17.5) {$f+d$};
		\node [style=none] (1827) at (250.5, -16.25) {};
		\node [style=none] (1828) at (251.5, -16.25) {};
		\node [style=Z] (1829) at (248.5, -17) {};
		\node [style=Z] (1830) at (248.5, -17.5) {};
		\node [style=X] (1831) at (249, -17.25) {};
		\node [style=X] (1832) at (249.5, -18) {};
		\node [style=Z] (1833) at (247, -17.25) {};
		\node [style=X] (1834) at (246.5, -17) {};
		\node [style=X] (1835) at (246.5, -17.5) {};
		\node [style=Z] (1836) at (247.5, -18) {};
		\node [style=Z] (1837) at (247, -16.5) {};
		\node [style=Z] (1838) at (247.5, -16.5) {};
		\node [style=Z] (1839) at (249, -16.5) {};
		\node [style=Z] (1840) at (249.5, -16.5) {};
		\node [style=none] (1841) at (248.5, -16.25) {};
		\node [style=none] (1842) at (246.5, -16.25) {};
	\end{pgfonlayer}
	\begin{pgfonlayer}{edgelayer}
		\draw (1795.center) to (1798);
		\draw (1796.center) to (1800);
		\draw (1797.center) to (1801);
		\draw (1798) to (1800);
		\draw (1801) to (1799);
		\draw (1798) to (1799);
		\draw (1809) to (1810);
		\draw (1808) to (1807);
		\draw (1807) to (1805.center);
		\draw (1804.center) to (1808);
		\draw (1808) to (1809);
		\draw (1806.center) to (1810);
		\draw (1807) to (1817.center);
		\draw (1810) to (1818.center);
		\draw [style=red] (1822) to (1827.center);
		\draw [style=red] (1823) to (1828.center);
		\draw (1829) to (1831);
		\draw (1832) to (1830);
		\draw (1829) to (1830);
		\draw (1835) to (1836);
		\draw (1834) to (1833);
		\draw (1834) to (1835);
		\draw (1834) to (1842.center);
		\draw (1817.center) to (1837);
		\draw (1838) to (1836);
		\draw (1829) to (1841.center);
		\draw (1839) to (1831);
		\draw (1832) to (1840);
		\draw (1831) to (1814.center);
		\draw (1813.center) to (1830);
		\draw (1835) to (1816.center);
		\draw (1809) to (1816.center);
		\draw (1800) to (1814.center);
		\draw (1799) to (1813.center);
		\draw (1801) to (1832);
	\end{pgfonlayer}
\end{tikzpicture}
=
\begin{tikzpicture}
	\begin{pgfonlayer}{nodelayer}
		\node [style=none] (1795) at (248.5, -21.25) {};
		\node [style=none] (1804) at (248, -21.25) {};
		\node [style=X] (1822) at (249.5, -17.5) {$e+d$};
		\node [style=X] (1823) at (250.5, -17.5) {$f+d$};
		\node [style=none] (1827) at (249.5, -16.25) {};
		\node [style=none] (1828) at (250.5, -16.25) {};
		\node [style=none] (1841) at (248.5, -16.25) {};
		\node [style=none] (1842) at (248, -16.25) {};
	\end{pgfonlayer}
	\begin{pgfonlayer}{edgelayer}
		\draw [style=red] (1822) to (1827.center);
		\draw [style=red] (1823) to (1828.center);
		\draw (1804.center) to (1842.center);
		\draw (1841.center) to (1795.center);
	\end{pgfonlayer}
\end{tikzpicture}
$$
\end{example}


\section{Discussion}
\label{sec:conclagrel}
It is important to note that the controlled operation we applied for the error correction step requires nonlinear classical processing power, and therefore the diagram we drew doesn't ``live'' within the calculus we have constructed throughout this paper. 

Indeed, in this example, we have secretly been working in the pushout:

$$\ZXA/\sim \leftarrow \Aff\Rel_{\F_2} \rightarrow \Aff\Co\Isot\Rel_{\F_2}^M$$


Where $\ZXA/\sim$ is the prop of Boolean {\em relations} which we presented by taking a quotient of the prop $\ZXA$.
Where moreover  $\Aff\Rel_{\F_2}\to \ZXA/\sim$ sends linear subspaces to subsets; and  $\Aff\Rel_{\F_2} \to \Aff\Co\Isot\Rel_{\F_2}^M$  sends 
$$(n\xrightarrow{f} m) \mapsto (C^{\otimes n}\xrightarrow{(p_Z^{\otimes n}, f, p_Z^{\otimes m})} C^{\otimes n})$$

So that we can regard the {\em classical wires} as being subsets rather than affine subspaces.
We can afford to do this because of the presentation of $\ZXA/\sim$ in the previous chapter. However, this is not very satisfying, because we only have a presentation for qubit nonlinear post-processing; and the symplectic formalism only gives a semantics for quantum circuits without phase-shift gates, as discussed in Corollary \ref{cor:nophase}.   This gives more motivation to actually work out a presentation for the quopit version of $\ZXA/\sim$ where there is no such subtlely.


This also begs the following question.  If quopit stabilizer circuits correspond to affine coisotropic subspaces of symplectic vector spaces, is there a succinct characterization of the structure of the subsets of  $\F_p^{2n}$ obtained by applying nonlinear classical processing to stabilizer states?
The nonlinear processing imposes equations which mean that the subset is no longer affine, for example consider the following stabilizer state which is measured, the measurement outcomes are multiplied and then a phase correction is performed using this value:
$$
\begin{tikzpicture}
	\begin{pgfonlayer}{nodelayer}
		\node [style=X] (358) at (248.5, 10.5) {};
		\node [style=Z] (359) at (249.75, 10.5) {};
		\node [style=none] (360) at (248.75, 11) {};
		\node [style=none] (362) at (248.25, 11) {};
		\node [style=none] (363) at (249.5, 11) {};
		\node [style=none] (365) at (250, 11) {};
		\node [style=X] (369) at (247.5, 10.5) {};
		\node [style=Z] (370) at (250.75, 10.5) {};
		\node [style=none] (371) at (247.75, 11) {};
		\node [style=none] (372) at (247.25, 11) {};
		\node [style=none] (373) at (250.5, 11) {};
		\node [style=none] (374) at (251, 11) {};
		\node [style=Z] (375) at (248.75, 11) {};
		\node [style=Z] (376) at (247.75, 11) {};
		\node [style=none] (377) at (251, 11) {};
		\node [style=none] (378) at (250, 11) {};
		\node [style=mult] (37wer9) at (250.75, 12) {};
		\node [style=none] (379) at (250.75, 12) {};
		\node [style=X] (380) at (250, 13) {};
		\node [style=Z] (381) at (247.75, 12) {};
		\node [style=X] (382) at (250, 12) {};
		\node [style=Z] (383) at (247.75, 13) {};
		\node [style=Z] (384) at (248.5, 12) {};
		\node [style=none] (385) at (250, 13.75) {};
		\node [style=none] (386) at (247.75, 13.75) {};
	\end{pgfonlayer}
	\begin{pgfonlayer}{edgelayer}
		\draw [in=45, out=-90] (360.center) to (358);
		\draw [in=135, out=-90] (362.center) to (358);
		\draw [in=45, out=-90] (365.center) to (359);
		\draw [in=135, out=-90] (363.center) to (359);
		\draw [in=45, out=-90] (371.center) to (369);
		\draw [in=135, out=-90] (372.center) to (369);
		\draw [in=45, out=-90] (374.center) to (370);
		\draw [in=135, out=-90] (373.center) to (370);
		\draw [style=red, in=-120, out=90] (378.center) to (379);
		\draw [style=red, in=-15, out=90, looseness=0.75] (379) to (380);
		\draw [style=red, in=90, out=-60] (379) to (377.center);
		\draw [in=225, out=90] (372.center) to (381);
		\draw [in=90, out=-45] (381) to (362.center);
		\draw [in=-45, out=90] (373.center) to (382);
		\draw [in=90, out=-135] (382) to (363.center);
		\draw (383) to (381);
		\draw [in=-15, out=90, looseness=1.25] (384) to (383);
		\draw (382) to (380);
		\draw (383) to (386.center);
		\draw (380) to (385.center);
	\end{pgfonlayer}
\end{tikzpicture}
$$

This mixed quantum state is stabilized by Weyl operators of the form:
$$\{W(z,x_1+x_2+x_1\cdot x_2)\ |\ \forall z,x_1,x_2\in \F_p\}$$
However, the stabilizers do {\em not} correspond to an affine subspace of $\F_p^2$ because of the nonlinear factor $x_1\cdot x_2$.



We can't construct circuits stabilized by Weyl operators corresponding to arbitrary {\em subsets} of $\F_p^{2n}$; for example, we can't copy the $Z$ and $X$ variables, because they can not be simultaneously measured.


Similarly, we can't produce the qudit generalization of the gate which multiplies standard basis elements in $\FHilb$, as we did with the $\AND$ gate in $\ZXA$.  This is because we can only multiply the {\em classical} outcomes.


What is the structure of this intermediate notion of subspace?
 To the knowledge of the author, this question has not yet been addressed, presumably, because classical processing and stabilizer codes are usually presented on different footings, as opposed to both being regarded themeselves as subspaces.

 

We have not given an undoubled completeness proof for $\Aff\Co\Isot\Rel_k$, although the completeness of the quopit stabilizer ZX-calculus \cite{poor} gives a completness result for $\Aff\Co\Isot\Rel_{\F_p}$.  A completness result for arbitrary fields should closely follow their techniques.

Even though completeness for the quopit stabilizer fragment of the ZX-calculus is known, there is motivation for generalizing their result.
Because everything we have developed in this chapter works for all fields, rather than just odd prime fields; this begs the question if this semantics could be used for continuous variable quantum mechanics/quantum photontics, by working over fields with characteristic 0; so that a complete axiomatization of $\Aff\Lag\Rel_k$ would yield continous variable analogues of the ZX-calculus.

 For example, in \cite[Thm. 2]{gross} Gross proves how  Gaussian states on the Hilbert space $\ell^2(\R^n)$ are in bijection with the affine Lagrangian subspaces of the symplectic vector space $\C^{2n}$ given by complex valued graph states, acted on by real valued Weyl operators. 
Gaussian states are the analogue of stabilizer states for continuous variables, allowing for coherent control.
However,  $\Aff\Lag\Rel_{\C}$ contains not only these Gaussian states, but also the dirac deltas for the position and momentum observables, providing a path towards exploring continuous variable ZX-calculi.

These   dirac deltas are ``nonphysical'' in the sense that they can not be interpreted as states in $\Hilb$; which is one way to see why $\Hilb$ not compact closed.  However, they are often used  to perform calculation, as they capture the behaviour of Gaussian quantum mechanics in the limit.  For example, squeezed states are used for continuous variable quantum teleportation \cite{Milburn1999}. 

Nonstandard analysis has been used to try to solve the problem of dirac deltas not existing in $\Hilb$/the lack of compact closure in infinite dimensions \cite{Gogioso2017}; however, this approach is considerably more powerful, as well as complicated.  The symplectic approach could potentially provide a simpler path towards expanding categorical quantum mechanics towards continous variables.





%
%There is also a practical motivation for providing this graphical calculus.  Just as increasing the characteristic of the prime field increases the quopit dimension; if one transitions to working in $\Aff\Co\Isot\Rel_k$ where $k$ has characteristic 0, it appears that this will give a semantics for continous variable quantum mechanics.
%
%
%For example, given a linear subspace of $\R^n$; then $\ell^2(\R^n)$ has the structure of a countably infinite Hilbert space; the dirac deltas being picked out by the indicator functions on the elements of $\R^n$. 
%
%Such a Hilbert space is the natural semantics for Guassian quantum mechanics; where a state is determined not by a finite set of stabilizers, but rather, by a probability distribution over $\R^{2n}$.  Such a probabilty distribution has the propery 
%
%
%
%
% interpretatble in  $\Aff\Co\Isot\Rel_\C$, following the 
%
%



\chapter{Categorical semantics for the scalable ZX-calculus}

\chapter{Conclusion}


\bibliography{bibliography} 
\bibliographystyle{plain}
\end{document}
