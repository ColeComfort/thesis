\label{chap:cat}

%In order to understand this thesis, we will assume only basic knowledge of category theory and quantum computing.  I intend for most of this thesis to be understandable by an average theoretical computer scientist  (given that they read the most of this chapter); except for some parts to be understood by category theorists.


Although we formally state the various categorical constructions which are used throughout this thesis, in almost all cases the accompanying string diagrams also help give intuition to the reader.  The exception to this rule is the somewhat more technical  material on internal category theory and distributive laws of monoidal theories reviewed in Subsection \ref{subsec:internal} and used in Section \ref{sec:dist}.  For this, we assume some basic understanding of bicategories.  %The most category theory heavy material is contained in Chapter \ref{chap:grothendieck} and assumes understanding of monoidal bicategories.  However, both of these parts stand on their own and are not needed to understand the rest of this thesis.

%In Section \ref{sec:cqm} we will review the  traditional algebraic paradigm of quantum computing to establish some basic notation; quickly moving to string diagrams and reviewing essential notions in categorical quantum mechanics,

As a matter of convention, call pairs of maps $f:X\to Y$ and $g:X\to Y$ with the same domain and codomain {\bf parallel}.   Similarly, call a pair of maps $f:X\to Y$ and $g:Y\to Z$ where the codomain of the first map is the domain of the second map {\bf composable}.
Given two composable maps $f:X\to Y$ and $g:Y \to Z$, we will denote their {\bf diagrammatic composition} using a semicolon as follows $f;g:X\to Z$. This notation for composition will be preferred throughout this thesis; except when talking about quantum circuits.  In this setting, to agree with the conventional notation we will denote their {\bf contravariant composition} by concatenation as follows $gf:X\to Z$.



%\section{Category theory}
%\label{sec:cat}





