
\section{Category theory}



\subsection{Basic category theory}

Limits, colimits, monoids, functor category monads, adjunctions, kleisli category.

\subsection{Monoidal categories}

\subsubsection{Monoidal categories and string diagrams}
\begin{definition}
A {\bf monoidal category} is a category $\X$ equipped with a functor $\X\times\X\to \X$ called the {\bf tensor product}, equipped with a distiguished object $I$ of $\X$ called the {\bf tensor unit}; along with the following natural isomorphisms (given by components):

\begin{description}
\item[Left unitor:]
$$
u_X^L:I\otimes X \to X
$$
\item[Right unitor:]
$$
u_X^R: X\otimes I \to X
$$
\item[Associator:]
$$
\alpha_{X,Y,Z}:(X\otimes Y)\otimes Z \to X\otimes(Y\otimes Z)
$$
\end{description}

Satisfying the following coherence equations:

\begin{description}
\item[MacLane pentagon:]


$$
\xymatrix{
  ((W\otimes X )\otimes Y)\otimes Z \ar[rr]^{\alpha_{W\otimes X,Y,Z}} \ar[d]_{\alpha_{W,X,Y}\otimes 1_Z}
    &
    & (W\otimes X )\otimes (Y\otimes Z) \ar[d]^{\alpha_{W, X,Y\otimes Z}}
  \\  (W\otimes(X\otimes Y))\otimes Z \ar[dr]_{\alpha_{W,\otimes Y,Z}}
    & 
    & W\otimes (X\otimes(Y\otimes Z)) 
  \\&
    W\otimes ((X\otimes Y)\otimes Z) \ar[ur]_{1_W\otimes \alpha_{X,Y,Z}}
}
$$

\item[Unit triangle:]

$$
\xymatrix{
  (X\otimes I)\otimes Y \ar[rr]^{\alpha_{X,I,Y}} \ar[dr]_{u_X^{R}\otimes 1_Y}
    &
    & X\otimes (I\otimes Y) \ar[dl]^{1_X\otimes u_Y^{L}}\\
  & X\otimes Y
}
$$

\end{description}

If all of the components of the natural transformations are equalities, then the monoidal category is {\bf strict}.


TODO Strong monoidal and strict monoidal functors.
\end{definition}

\begin{example}
Sets and functions is a monoidal category under both the product and under the coproduct.
Finite ordinals and functions are a strict monoidal under the same tensor product.

Sets and monotone functions is a monoidal category under the coproduct.
Finite ordinals and order preserving maps is also a monoidal category under the tensor product.


Given a field $k$, vector spaces over $k$ are monoidal with respect to the bilinear tensor product and the direct sum.
Matrices over $k$ are monoidal categories under both the bilinear tensor product and the direct sum.
\end{example}

Strict monoidal categories have a particularly concise graphical calculus, called {\bf string diagrams}.  A map $f:X_1\otimes \cdots X_n\to Y_1\otimes \cdots \otimes Y_n$ is drawn as a box with $n$ wires coming out of the bottom and $m$ wires coming out of the top, all being labelled by their respective objects, as follows:


$$
\begin{tikzpicture}
	\begin{pgfonlayer}{nodelayer}
		\node [style=map] (0) at (0, 3) {$f$};
		\node [style=none] (1) at (-0.75, 2.25) {};
		\node [style=none] (2) at (-0.25, 2.25) {};
		\node [style=none] (3) at (0.75, 2.25) {};
		\node [style=none] (4) at (-0.75, 2) {$X_1$};
		\node [style=none] (5) at (-0.25, 2) {$X_2$};
		\node [style=none] (6) at (0.75, 2) {$X_n$};
		\node [style=none] (7) at (0.25, 2.25) {$\cdots$};
		\node [style=none] (8) at (-0.75, 3.75) {};
		\node [style=none] (9) at (-0.25, 3.75) {};
		\node [style=none] (10) at (0.75, 3.75) {};
		\node [style=none] (11) at (-0.75, 4) {$Y_1$};
		\node [style=none] (12) at (-0.25, 4) {$Y_2$};
		\node [style=none] (13) at (0.75, 4) {$Y_m$};
		\node [style=none] (14) at (0.25, 3.75) {$\cdots$};
	\end{pgfonlayer}
	\begin{pgfonlayer}{edgelayer}
		\draw [in=90, out=-135, looseness=0.75] (0) to (1.center);
		\draw [in=255, out=90] (2.center) to (0);
		\draw [in=90, out=-45, looseness=0.75] (0) to (3.center);
		\draw [in=105, out=-90] (9.center) to (0);
		\draw [in=-90, out=135, looseness=0.75] (0) to (8.center);
		\draw [in=-90, out=45, looseness=0.75] (0) to (10.center);
	\end{pgfonlayer}
\end{tikzpicture}
$$


We often omit the objects when it is clear from context.  The tensor product of two maps $f:W_1\otimes W_n\to X_1\otimes \cdots \otimes X_m$ and $g:Y_1\otimes Y_k\to Z_1\otimes \cdots \otimes Z_\ell$,  $f\otimes g$ is given by pasting them side-by-side:


$$
\begin{tikzpicture}
	\begin{pgfonlayer}{nodelayer}
		\node [style=none] (79) at (20.25, 2.25) {};
		\node [style=none] (80) at (20.75, 2.25) {};
		\node [style=none] (81) at (21.75, 2.25) {};
		\node [style=none] (82) at (21.25, 2.25) {$\cdots$};
		\node [style=none] (83) at (20.25, 3.75) {};
		\node [style=none] (84) at (20.75, 3.75) {};
		\node [style=none] (85) at (21.75, 3.75) {};
		\node [style=none] (86) at (21.25, 3.75) {$\cdots$};
		\node [style=none] (87) at (22.25, 2.25) {};
		\node [style=none] (88) at (22.75, 2.25) {};
		\node [style=none] (89) at (23.75, 2.25) {};
		\node [style=none] (90) at (23.25, 2.25) {$\cdots$};
		\node [style=none] (91) at (22.25, 3.75) {};
		\node [style=none] (92) at (22.75, 3.75) {};
		\node [style=none] (93) at (23.75, 3.75) {};
		\node [style=none] (94) at (23.25, 3.75) {$\cdots$};
		\node [style=map] (95) at (22.1, 3) {\ $f\otimes g$\ \ };
		\node [style=none] (96) at (21.5, 3) {};
		\node [style=none] (97) at (21.75, 3) {};
		\node [style=none] (98) at (22, 3) {};
		\node [style=none] (99) at (22.25, 3) {};
		\node [style=none] (100) at (22.5, 3) {};
		\node [style=none] (101) at (22.75, 3) {};
		\node [style=none] (102) at (20.25, 2) {$W_1$};
		\node [style=none] (103) at (20.75, 2) {$W_2$};
		\node [style=none] (104) at (21.75, 2) {$W_m$};
		\node [style=none] (105) at (22.25, 2) {$Y_1$};
		\node [style=none] (106) at (22.75, 2) {$Y_2$};
		\node [style=none] (107) at (23.75, 2) {$Y_k$};
		\node [style=none] (108) at (20.25, 4) {$X_1$};
		\node [style=none] (109) at (20.75, 4) {$X_2$};
		\node [style=none] (110) at (21.75, 4) {$X_m$};
		\node [style=none] (111) at (22.25, 4) {$Z_1$};
		\node [style=none] (112) at (22.75, 4) {$Z_2$};
		\node [style=none] (113) at (23.75, 4) {$Z_\ell$};
	\end{pgfonlayer}
	\begin{pgfonlayer}{edgelayer}
		\draw [in=-90, out=90, looseness=0.50] (79.center) to (96.center);
		\draw [in=-90, out=90, looseness=0.50] (80.center) to (97.center);
		\draw [in=90, out=-90] (98.center) to (81.center);
		\draw (87.center) to (99.center);
		\draw [in=90, out=-90] (100.center) to (88.center);
		\draw [in=-90, out=90, looseness=0.75] (89.center) to (101.center);
		\draw [in=-90, out=90, looseness=0.75] (101.center) to (93.center);
		\draw [in=90, out=-90] (92.center) to (100.center);
		\draw (99.center) to (91.center);
		\draw [in=90, out=-90] (85.center) to (98.center);
		\draw [in=-90, out=90, looseness=0.50] (97.center) to (84.center);
		\draw [in=-90, out=90, looseness=0.50] (96.center) to (83.center);
	\end{pgfonlayer}
\end{tikzpicture}
:=
\begin{tikzpicture}
	\begin{pgfonlayer}{nodelayer}
		\node [style=map] (49) at (16.5, 3) {$f$};
		\node [style=none] (50) at (15.75, 2.25) {};
		\node [style=none] (51) at (16.25, 2.25) {};
		\node [style=none] (52) at (17.25, 2.25) {};
		\node [style=none] (53) at (16.75, 2.25) {$\cdots$};
		\node [style=none] (54) at (15.75, 3.75) {};
		\node [style=none] (55) at (16.25, 3.75) {};
		\node [style=none] (56) at (17.25, 3.75) {};
		\node [style=none] (57) at (16.75, 3.75) {$\cdots$};
		\node [style=map] (58) at (18.5, 3) {$g$};
		\node [style=none] (59) at (17.75, 2.25) {};
		\node [style=none] (60) at (18.25, 2.25) {};
		\node [style=none] (61) at (19.25, 2.25) {};
		\node [style=none] (62) at (18.75, 2.25) {$\cdots$};
		\node [style=none] (63) at (17.75, 3.75) {};
		\node [style=none] (64) at (18.25, 3.75) {};
		\node [style=none] (65) at (19.25, 3.75) {};
		\node [style=none] (66) at (18.75, 3.75) {$\cdots$};
		\node [style=none] (67) at (15.75, 4) {$X_1$};
		\node [style=none] (68) at (16.25, 4) {$X_2$};
		\node [style=none] (69) at (17.25, 4) {$X_m$};
		\node [style=none] (70) at (15.75, 2) {$W_1$};
		\node [style=none] (71) at (16.25, 2) {$W_2$};
		\node [style=none] (72) at (17.25, 2) {$W_m$};
		\node [style=none] (73) at (17.75, 2) {$Y_1$};
		\node [style=none] (74) at (18.25, 2) {$Y_2$};
		\node [style=none] (75) at (19.25, 2) {$Y_k$};
		\node [style=none] (76) at (17.75, 4) {$Z_1$};
		\node [style=none] (77) at (18.25, 4) {$Z_2$};
		\node [style=none] (78) at (19.25, 4) {$Z_\ell$};
	\end{pgfonlayer}
	\begin{pgfonlayer}{edgelayer}
		\draw [in=90, out=-135, looseness=0.75] (49) to (50.center);
		\draw [in=255, out=90] (51.center) to (49);
		\draw [in=90, out=-45, looseness=0.75] (49) to (52.center);
		\draw [in=105, out=-90] (55.center) to (49);
		\draw [in=-90, out=135, looseness=0.75] (49) to (54.center);
		\draw [in=-90, out=45, looseness=0.75] (49) to (56.center);
		\draw [in=90, out=-135, looseness=0.75] (58) to (59.center);
		\draw [in=255, out=90] (60.center) to (58);
		\draw [in=90, out=-45, looseness=0.75] (58) to (61.center);
		\draw [in=105, out=-90] (64.center) to (58);
		\draw [in=-90, out=135, looseness=0.75] (58) to (63.center);
		\draw [in=-90, out=45, looseness=0.75] (58) to (65.center);
	\end{pgfonlayer}
\end{tikzpicture}
$$

And a map $f:X_1\otimes X_n:\to Y_1\otimes \cdots \otimes Y_m$ is composed with a map $f:Y_1\otimes Y_m:\to Z_1\otimes \cdots \otimes Z_k$ by connecting the $Z_i$ togther:

$$
\begin{tikzpicture}
	\begin{pgfonlayer}{nodelayer}
		\node [style=none] (34) at (13.25, 2) {$X_1$};
		\node [style=none] (35) at (13.75, 2) {$X_2$};
		\node [style=none] (36) at (14.75, 2) {$X_n$};
		\node [style=none] (37) at (13.25, 4) {$Z_1$};
		\node [style=none] (38) at (13.75, 4) {$Z_2$};
		\node [style=none] (39) at (14.75, 4) {$Z_k$};
		\node [style=map] (40) at (14, 3) {$f;g$};
		\node [style=none] (41) at (13.25, 2.25) {};
		\node [style=none] (42) at (13.75, 2.25) {};
		\node [style=none] (43) at (14.75, 2.25) {};
		\node [style=none] (44) at (14.25, 2.25) {$\cdots$};
		\node [style=none] (45) at (13.25, 3.75) {};
		\node [style=none] (46) at (13.75, 3.75) {};
		\node [style=none] (47) at (14.75, 3.75) {};
		\node [style=none] (48) at (14.25, 3.75) {$\cdots$};
	\end{pgfonlayer}
	\begin{pgfonlayer}{edgelayer}
		\draw [in=90, out=-135, looseness=0.75] (40) to (41.center);
		\draw [in=255, out=90] (42.center) to (40);
		\draw [in=90, out=-45, looseness=0.75] (40) to (43.center);
		\draw [in=105, out=-90] (46.center) to (40);
		\draw [in=-90, out=135, looseness=0.75] (40) to (45.center);
		\draw [in=-90, out=45, looseness=0.75] (40) to (47.center);
	\end{pgfonlayer}
\end{tikzpicture}
=
\begin{tikzpicture}
	\begin{pgfonlayer}{nodelayer}
		\node [style=map] (23) at (11.5, 1.75) {$f$};
		\node [style=none] (24) at (10.75, 1) {};
		\node [style=none] (25) at (11.25, 1) {};
		\node [style=none] (26) at (12.25, 1) {};
		\node [style=none] (27) at (11.75, 1) {$\cdots$};
		\node [style=map] (28) at (11.5, 2.75) {$g$};
		\node [style=none] (29) at (10.75, 3.5) {};
		\node [style=none] (30) at (11.25, 3.5) {};
		\node [style=none] (31) at (12.25, 3.5) {};
		\node [style=none] (32) at (11.75, 3.5) {$\cdots$};
		\node [style=none] (33) at (11.7, 2.25) {$\cdots$};
		\node [style=none] (34) at (10.75, 0.75) {$X_1$};
		\node [style=none] (35) at (11.25, 0.75) {$X_2$};
		\node [style=none] (36) at (12.25, 0.75) {$X_n$};
		\node [style=none] (37) at (10.75, 3.75) {$Z_1$};
		\node [style=none] (38) at (11.25, 3.75) {$Z_2$};
		\node [style=none] (39) at (12.25, 3.75) {$Z_k$};
	\end{pgfonlayer}
	\begin{pgfonlayer}{edgelayer}
		\draw [in=90, out=-135, looseness=0.75] (23) to (24.center);
		\draw [in=255, out=90] (25.center) to (23);
		\draw [in=90, out=-45, looseness=0.75] (23) to (26.center);
		\draw [in=105, out=-90] (30.center) to (28);
		\draw [in=-90, out=135, looseness=0.75] (28) to (29.center);
		\draw [in=-90, out=45, looseness=0.75] (28) to (31.center);
		\draw [in=105, out=-105] (28) to (23);
		\draw [in=225, out=135, looseness=1.25] (23) to (28);
		\draw [in=30, out=-30, looseness=1.50] (28) to (23);
	\end{pgfonlayer}
\end{tikzpicture}
$$

The axioms of a strict monoidal category are equivalent to planar isotopy of their string diagrams, for example, one can exchange two disconnected maps:

$$
\begin{tikzpicture}
	\begin{pgfonlayer}{nodelayer}
		\node [style=none] (22) at (1, 5) {};
		\node [style=none] (23) at (0, 5) {};
		\node [style=none] (37) at (0, 3) {};
		\node [style=none] (38) at (1, 3) {};
		\node [style=map] (39) at (0, 3.75) {$f$};
		\node [style=map] (40) at (1, 4.25) {$g$};
	\end{pgfonlayer}
	\begin{pgfonlayer}{edgelayer}
		\draw (37.center) to (39);
		\draw (39) to (23.center);
		\draw (38.center) to (40);
		\draw (40) to (22.center);
	\end{pgfonlayer}
\end{tikzpicture}
=
\begin{tikzpicture}
	\begin{pgfonlayer}{nodelayer}
		\node [style=none] (41) at (3, 5) {};
		\node [style=none] (42) at (2, 5) {};
		\node [style=none] (43) at (2, 3) {};
		\node [style=none] (44) at (3, 3) {};
		\node [style=map] (45) at (2, 4.25) {$f$};
		\node [style=map] (46) at (3, 3.75) {$g$};
	\end{pgfonlayer}
	\begin{pgfonlayer}{edgelayer}
		\draw (43.center) to (45);
		\draw (45) to (42.center);
		\draw (44.center) to (46);
		\draw (46) to (41.center);
	\end{pgfonlayer}
\end{tikzpicture}
$$

FUNCTOR BOX 



\begin{theorem}
Every monoidal category is strong-monoidally isomorphic to a strict monoidal category. 
\end{theorem}

\begin{example}
Finite ordinals and order preserving maps is the strictification of finite sets and monotone functions under the coproduct.


Finite ordinals and functions is the strictification of finite sets and functions under both the product and coproduct.


Matrices over a field is the strictification of finite dimensional vector spaces under the direct sum and bilinear tensor product.
\end{example}


Non-strict monoidal categories have a graphical calculus, often called proof nets.  These extend string diagrams for strict monoidal categories by adding the following generators for all objects $A,B$ called (co)tensors and unit introduction/removal.


$$
\begin{tikzpicture}
	\begin{pgfonlayer}{nodelayer}
		\node [style=none] (0) at (1.5, 3.5) {};
		\node [style=none] (1) at (0.5, 3.5) {};
		\node [style=none] (2) at (1, 4.25) {};
		\node [style=none] (3) at (1, 5) {};
		\node [style=none] (4) at (0.5, 3.25) {$X$};
		\node [style=none] (5) at (1.5, 3.25) {$Y$};
		\node [style=none] (6) at (1, 5.25) {$X\otimes Y$};
		\node [style=otimes] (20) at (1, 4.25) {};
	\end{pgfonlayer}
	\begin{pgfonlayer}{edgelayer}
		\draw (3.center) to (2);
		\draw [in=90, out=-30] (2) to (0.center);
		\draw [in=90, out=-150] (2) to (1.center);
	\end{pgfonlayer}
\end{tikzpicture}
\,\hspace{.5cm}
\begin{tikzpicture}
	\begin{pgfonlayer}{nodelayer}
		\node [style=none] (0) at (1.5, 5) {};
		\node [style=none] (1) at (0.5, 5) {};
		\node [style=none] (2) at (1, 4.25) {};
		\node [style=none] (3) at (1, 3.5) {};
		\node [style=none] (4) at (0.5, 5.25) {$X$};
		\node [style=none] (5) at (1.5, 5.25) {$Y$};
		\node [style=none] (6) at (1, 3.25) {$X\otimes Y$};
		\node [style=otimes] (20) at (1, 4.25) {};
	\end{pgfonlayer}
	\begin{pgfonlayer}{edgelayer}
		\draw (3.center) to (2);
		\draw [in=-90, out=30] (2) to (0.center);
		\draw [in=-90, out=150] (2) to (1.center);
	\end{pgfonlayer}
\end{tikzpicture}
\,\hspace{.5cm}
\begin{tikzpicture}
	\begin{pgfonlayer}{nodelayer}
		\node [style=unit] (2) at (1, 3.25) {};
		\node [style=none] (3) at (1, 4) {};
		\node [style=none] (6) at (1, 4.25) {$I$};
	\end{pgfonlayer}
	\begin{pgfonlayer}{edgelayer}
		\draw (3.center) to (2);
	\end{pgfonlayer}
\end{tikzpicture}
\,\hspace{.5cm}
\begin{tikzpicture}
	\begin{pgfonlayer}{nodelayer}
		\node [style=unit] (2) at (1, 4.25) {};
		\node [style=none] (3) at (1, 3.5) {};
		\node [style=none] (6) at (1, 3.25) {$I$};
	\end{pgfonlayer}
	\begin{pgfonlayer}{edgelayer}
		\draw (3.center) to (2);
	\end{pgfonlayer}
\end{tikzpicture}
$$


Modulo the following equations making the counit inverse to the unit and the comultiplications inverse to the multiplications:

$$
\begin{tikzpicture}
	\begin{pgfonlayer}{nodelayer}
		\node [style=otimes] (13) at (4, 5.25) {};
		\node [style=none] (14) at (4, 6) {};
		\node [style=otimes] (17) at (4, 4.25) {};
		\node [style=none] (18) at (4, 3.5) {};
		\node [style=none] (19) at (4, 5.25) {};
		\node [style=none] (20) at (4, 4.25) {};
	\end{pgfonlayer}
	\begin{pgfonlayer}{edgelayer}
		\draw (14.center) to (13);
		\draw (18.center) to (17);
		\draw [bend left=60, looseness=1.25] (20.center) to (19.center);
		\draw [bend left=60, looseness=1.25] (19.center) to (20.center);
	\end{pgfonlayer}
\end{tikzpicture}
=
\begin{tikzpicture}
	\begin{pgfonlayer}{nodelayer}
		\node [style=none] (28) at (4, 2.75) {};
		\node [style=none] (30) at (4, 0.25) {};
	\end{pgfonlayer}
	\begin{pgfonlayer}{edgelayer}
		\draw (30.center) to (28.center);
	\end{pgfonlayer}
\end{tikzpicture}
\ ,\hspace{.5cm}
\begin{tikzpicture}
	\begin{pgfonlayer}{nodelayer}
		\node [style=none] (0) at (1.5, 3.5) {};
		\node [style=none] (1) at (0.5, 3.5) {};
		\node [style=none] (2) at (1, 4.25) {};
		\node [style=none] (3) at (1, 5) {};
		\node [style=none] (7) at (1.5, 5.75) {};
		\node [style=none] (8) at (0.5, 5.75) {};
		\node [style=none] (9) at (1, 5) {};
		\node [style=otimes] (20) at (1, 4.25) {};
		\node [style=otimes] (90) at (1, 5) {};
	\end{pgfonlayer}
	\begin{pgfonlayer}{edgelayer}
		\draw (3.center) to (2);
		\draw [in=90, out=-30] (2) to (0.center);
		\draw [in=90, out=-150] (2) to (1.center);
		\draw [in=-90, out=30] (9) to (7.center);
		\draw [in=-90, out=150] (9) to (8.center);
	\end{pgfonlayer}
\end{tikzpicture}
=
\begin{tikzpicture}
	\begin{pgfonlayer}{nodelayer}
		\node [style=none] (21) at (1.5, 0.5) {};
		\node [style=none] (22) at (0.5, 0.5) {};
		\node [style=none] (25) at (1.5, 2.75) {};
		\node [style=none] (26) at (0.5, 2.75) {};
	\end{pgfonlayer}
	\begin{pgfonlayer}{edgelayer}
		\draw (21.center) to (25.center);
		\draw (22.center) to (26.center);
	\end{pgfonlayer}
\end{tikzpicture}
\ ,\hspace{.5cm}
\begin{tikzpicture}
	\begin{pgfonlayer}{nodelayer}
		\node [style=unit] (13) at (4, 5.25) {};
		\node [style=none] (14) at (4, 6) {};
		\node [style=unit] (17) at (4, 4.25) {};
		\node [style=none] (18) at (4, 3.5) {};
		\node [style=none] (19) at (4, 5.25) {};
		\node [style=none] (20) at (4, 4.25) {};
	\end{pgfonlayer}
	\begin{pgfonlayer}{edgelayer}
		\draw (14.center) to (13);
		\draw (18.center) to (17);
	\end{pgfonlayer}
\end{tikzpicture}
=
\begin{tikzpicture}
	\begin{pgfonlayer}{nodelayer}
		\node [style=none] (28) at (4, 2.75) {};
		\node [style=none] (30) at (4, 0.25) {};
	\end{pgfonlayer}
	\begin{pgfonlayer}{edgelayer}
		\draw (30.center) to (28.center);
	\end{pgfonlayer}
\end{tikzpicture}
\ ,\hspace{.5cm}
\begin{tikzpicture}
	\begin{pgfonlayer}{nodelayer}
		\node [style=unit] (2) at (1, 4.25) {};
		\node [style=none] (3) at (1, 5) {};
		\node [style=unit] (9) at (1, 5) {};
	\end{pgfonlayer}
	\begin{pgfonlayer}{edgelayer}
		\draw (3.center) to (2);
	\end{pgfonlayer}
\end{tikzpicture}
=
\begin{tikzpicture}
	\begin{pgfonlayer}{nodelayer}
		\node [style=none] (0) at (2, 0) {};
		\node [style=none] (1) at (2, -1) {};
		\node [style=none] (2) at (3, -1) {};
		\node [style=none] (3) at (3, 0) {};
	\end{pgfonlayer}
	\begin{pgfonlayer}{edgelayer}
		\draw[style=dashed] (3.center) to (0.center);
		\draw[style=dashed] (0.center) to (1.center);
		\draw[style=dashed] (1.center) to (2.center);
		\draw[style=dashed] (2.center) to (3.center);
	\end{pgfonlayer}
\end{tikzpicture}
$$

The components of the unitors and associators are given by the following diagrams:

$$
\alpha_{X,Y,Z}=
\begin{tikzpicture}
	\begin{pgfonlayer}{nodelayer}
		\node [style=none] (11) at (8.5, 5) {};
		\node [style=none] (12) at (7.5, 5) {};
		\node [style=none] (13) at (8, 4.25) {};
		\node [style=none] (14) at (8, 3.5) {};
		\node [style=none] (17) at (8, 3.25) {$(X\otimes Y)\otimes Z$};
		\node [style=otimes] (18) at (8, 4.25) {};
		\node [style=none] (19) at (8, 5.75) {};
		\node [style=none] (20) at (7, 5.75) {};
		\node [style=none] (21) at (7.5, 5) {};
		\node [style=none] (23) at (7.5, 7.5) {$X\otimes (Y\otimes Z)$};
		\node [style=otimes] (24) at (7.5, 5) {};
		\node [style=none] (25) at (7, 5.75) {};
		\node [style=none] (26) at (8, 5.75) {};
		\node [style=none] (27) at (7.5, 6.5) {};
		\node [style=none] (28) at (7.5, 7.25) {};
		\node [style=otimes] (29) at (7.5, 6.5) {};
		\node [style=none] (30) at (7.5, 5) {};
		\node [style=none] (31) at (8.5, 5) {};
		\node [style=none] (32) at (8, 5.75) {};
		\node [style=otimes] (33) at (8, 5.75) {};
	\end{pgfonlayer}
	\begin{pgfonlayer}{edgelayer}
		\draw (14.center) to (13.center);
		\draw [in=-90, out=30] (13.center) to (11.center);
		\draw [in=-90, out=150] (13.center) to (12.center);
		\draw [in=-90, out=150] (21.center) to (20.center);
		\draw (28.center) to (27.center);
		\draw [in=90, out=-150] (27.center) to (25.center);
		\draw [in=90, out=-30] (27.center) to (26.center);
		\draw (32.center) to (30.center);
		\draw [in=90, out=-30] (32.center) to (31.center);
	\end{pgfonlayer}
\end{tikzpicture}
\ ,
\hspace*{.5cm}
u_X^L
=
\begin{tikzpicture}
	\begin{pgfonlayer}{nodelayer}
		\node [style=none] (63) at (7.5, -2.5) {};
		\node [style=none] (64) at (8.5, -2.5) {};
		\node [style=none] (65) at (8, -1.75) {};
		\node [style=none] (66) at (8, -1) {};
		\node [style=none] (67) at (8.5, -2.75) {$X$};
		\node [style=none] (68) at (8, -0.75) {$I\otimes X$};
		\node [style=otimes] (69) at (8, -1.75) {};
		\node [style=unit] (70) at (7.5, -2.5) {};
	\end{pgfonlayer}
	\begin{pgfonlayer}{edgelayer}
		\draw (66.center) to (65.center);
		\draw [in=90, out=-150] (65.center) to (63.center);
		\draw [in=90, out=-30] (65.center) to (64.center);
	\end{pgfonlayer}
\end{tikzpicture}
\ ,
\hspace*{.5cm}
u_X^R
=
\begin{tikzpicture}
	\begin{pgfonlayer}{nodelayer}
		\node [style=none] (54) at (8.5, 0.25) {};
		\node [style=none] (55) at (7.5, 0.25) {};
		\node [style=none] (56) at (8, 1) {};
		\node [style=none] (57) at (8, 1.75) {};
		\node [style=none] (58) at (7.5, 0) {$X$};
		\node [style=none] (60) at (8, 2) {$X\otimes I$};
		\node [style=otimes] (61) at (8, 1) {};
		\node [style=unit] (62) at (8.5, 0.25) {};
	\end{pgfonlayer}
	\begin{pgfonlayer}{edgelayer}
		\draw (57.center) to (56.center);
		\draw [in=90, out=-30] (56.center) to (54.center);
		\draw [in=90, out=-150] (56.center) to (55.center);
	\end{pgfonlayer}
\end{tikzpicture}
$$

And their inverses are given by flipping the diagrams vertically:

$$
\alpha_{X,Y,Z}^{-1}=
\begin{tikzpicture}
	\begin{pgfonlayer}{nodelayer}
		\node [style=none] (34) at (12.75, 5.75) {};
		\node [style=none] (35) at (11.75, 5.75) {};
		\node [style=none] (36) at (12.25, 6.5) {};
		\node [style=none] (37) at (12.25, 7.25) {};
		\node [style=none] (38) at (12.25, 7.5) {$(X\otimes Y)\otimes Z$};
		\node [style=otimes] (39) at (12.25, 6.5) {};
		\node [style=none] (40) at (12.25, 5) {};
		\node [style=none] (41) at (11.25, 5) {};
		\node [style=none] (42) at (11.75, 5.75) {};
		\node [style=none] (43) at (11.75, 3.25) {$X\otimes (Y\otimes Z)$};
		\node [style=otimes] (44) at (11.75, 5.75) {};
		\node [style=none] (45) at (11.25, 5) {};
		\node [style=none] (46) at (12.25, 5) {};
		\node [style=none] (47) at (11.75, 4.25) {};
		\node [style=none] (48) at (11.75, 3.5) {};
		\node [style=otimes] (49) at (11.75, 4.25) {};
		\node [style=none] (50) at (11.75, 5.75) {};
		\node [style=none] (51) at (12.75, 5.75) {};
		\node [style=none] (52) at (12.25, 5) {};
		\node [style=otimes] (53) at (12.25, 5) {};
	\end{pgfonlayer}
	\begin{pgfonlayer}{edgelayer}
		\draw (37.center) to (36.center);
		\draw [in=90, out=-30] (36.center) to (34.center);
		\draw [in=90, out=-150] (36.center) to (35.center);
		\draw [in=90, out=-150] (42.center) to (41.center);
		\draw (48.center) to (47.center);
		\draw [in=-90, out=150] (47.center) to (45.center);
		\draw [in=-90, out=30] (47.center) to (46.center);
		\draw (52.center) to (50.center);
		\draw [in=-90, out=30] (52.center) to (51.center);
	\end{pgfonlayer}
\end{tikzpicture}
\ ,
\hspace*{.5cm}
(u_X^L)^{-1}
=
\begin{tikzpicture}
	\begin{pgfonlayer}{nodelayer}
		\node [style=none] (79) at (11.25, 2) {};
		\node [style=none] (80) at (12.25, 2) {};
		\node [style=none] (81) at (11.75, 1.25) {};
		\node [style=none] (82) at (11.75, 0.5) {};
		\node [style=none] (83) at (12.25, 2.25) {$X$};
		\node [style=none] (84) at (11.75, 0.25) {$I\otimes X$};
		\node [style=otimes] (85) at (11.75, 1.25) {};
		\node [style=unit] (86) at (11.25, 2) {};
	\end{pgfonlayer}
	\begin{pgfonlayer}{edgelayer}
		\draw (82.center) to (81.center);
		\draw [in=-90, out=150] (81.center) to (79.center);
		\draw [in=-90, out=30] (81.center) to (80.center);
	\end{pgfonlayer}
\end{tikzpicture}
\ ,
\hspace*{.5cm}
(u_X^R)^{-1}
=
\begin{tikzpicture}
	\begin{pgfonlayer}{nodelayer}
		\node [style=none] (71) at (12.25, -0.75) {};
		\node [style=none] (72) at (11.25, -0.75) {};
		\node [style=none] (73) at (11.75, -1.5) {};
		\node [style=none] (74) at (11.75, -2.25) {};
		\node [style=none] (75) at (11.25, -0.5) {$X$};
		\node [style=none] (76) at (11.75, -2.5) {$X\otimes I$};
		\node [style=otimes] (77) at (11.75, -1.5) {};
		\node [style=unit] (78) at (12.25, -0.75) {};
	\end{pgfonlayer}
	\begin{pgfonlayer}{edgelayer}
		\draw (74.center) to (73.center);
		\draw [in=-90, out=30] (73.center) to (71.center);
		\draw [in=-90, out=150] (73.center) to (72.center);
	\end{pgfonlayer}
\end{tikzpicture}
$$


Where the tensor product of two diagrams is given by convolution with the tensor and cotensor:

$$
f\otimes g =
\begin{tikzpicture}
	\begin{pgfonlayer}{nodelayer}
		\node [style=none] (87) at (16.75, 3.5) {};
		\node [style=none] (88) at (15.75, 3.5) {};
		\node [style=none] (89) at (16.25, 2.75) {};
		\node [style=none] (90) at (16.25, 2) {};
		\node [style=otimes] (94) at (16.25, 2.75) {};
		\node [style=none] (95) at (16.75, 3.5) {};
		\node [style=none] (96) at (15.75, 3.5) {};
		\node [style=none] (97) at (16.25, 4.25) {};
		\node [style=none] (98) at (16.25, 5) {};
		\node [style=otimes] (102) at (16.25, 4.25) {};
		\node [style=map] (103) at (15.75, 3.5) {$f$};
		\node [style=map] (104) at (16.75, 3.5) {$g$};
	\end{pgfonlayer}
	\begin{pgfonlayer}{edgelayer}
		\draw (90.center) to (89.center);
		\draw [in=-90, out=30] (89.center) to (87.center);
		\draw [in=-90, out=150] (89.center) to (88.center);
		\draw (98.center) to (97.center);
		\draw [in=90, out=-30] (97.center) to (95.center);
		\draw [in=90, out=-150] (97.center) to (96.center);
	\end{pgfonlayer}
\end{tikzpicture}
$$

The composition is given by attaching the wires as before.

These are not quite string diagrams, because the only proof nets that correspond to maps are those with a single wire coming in on the bottom and a single wire leaving on top: specifying the domain and codomain of the map.  However, as mathematical shorthand, we will often draw proof nets where we don't tensor all the top or bottom wires, in which case we will implicitly assume that the top wires are left-associatively tensored together and the bottom wires are left-coassociatively tensored together.


There is a more refined notion of monoidal category where one can pass wires through each other:

\begin{definition}
A {\bf symmetric monoidal category} is a monoidal category equipped with an extra natural isomorphism called the symmetry:

$$
c_{X,Y}:X\otimes Y \to Y\otimes X
$$

satisfying the following coherence equations:

\begin{itemize}
\item
$$
\xymatrix{
I \otimes X \ar[rr]^{c_{I,X}} \ar[rd]_{u^L_X} && X \otimes I \ar[ld]^{u^R_X} \\
& X &
}
$$

\item
$$
\xymatrix{
  (X\otimes Y)\otimes Z \ar[rr]^{c_{X,Y}\otimes 1_Z} \ar[d]_{\alpha_{X,Y,Z}}
    &
    &  (Y\otimes X)\otimes Z \ar[d]^{\alpha_{Y,X,Z}}
  \\X\otimes(Y\otimes Z) \ar[d]_{c_{X,Y\otimes Z}}
    &
    &  Y\otimes(X\otimes Z) \ar[d]^{1_Y\otimes c_{X,Z}}
  \\ (Y\otimes Z)\otimes X \ar[rr]_{\alpha_{Y,Z,X}}
    &
    & Y\otimes (Z\otimes X)
}
$$

\item
$$c_{X,Y}^{-1}= c_{Y,X}$$
\end{itemize}

A {\bf strict symmetric monoidal category} is a symmetric monoidal category, whose underlying monoidal category is strict. That is to say, all the coherence isomorphisms, except for the symmetry map are indentities.

TODO: Strict symmetric and strong symmetric monoidal functors.
\end{definition}


\begin{example}
Sets and functions, finite sets and functions, finite ordinals and functions, vector spaces and matrices are all symmetric monoidal categories under the aforementioned tensor products.
\end{example}

Strict monoidal categories also have a notion of string diagrams, except the symmetry allows wires to pass over each other:

$$
c_{X,Y}=
\begin{tikzpicture}
	\begin{pgfonlayer}{nodelayer}
		\node [style=none] (22) at (1, 5) {};
		\node [style=none] (23) at (0, 5) {};
		\node [style=none] (24) at (0, 4) {};
		\node [style=none] (25) at (1, 4) {};
		\node [style=none] (26) at (0, 3.75) {$X$};
		\node [style=none] (27) at (1, 3.75) {$Y$};
		\node [style=none] (28) at (0, 5.25) {$Y$};
		\node [style=none] (29) at (1, 5.25) {$X$};
	\end{pgfonlayer}
	\begin{pgfonlayer}{edgelayer}
		\draw [in=270, out=90] (24.center) to (22.center);
		\draw [in=270, out=90] (25.center) to (23.center);
	\end{pgfonlayer}
\end{tikzpicture}
$$

The naturality means that maps can be pulled through wires:

$$
\begin{tikzpicture}
	\begin{pgfonlayer}{nodelayer}
		\node [style=none] (22) at (1, 5) {};
		\node [style=none] (23) at (0, 5) {};
		\node [style=none] (24) at (0, 4) {};
		\node [style=none] (25) at (1, 4) {};
		\node [style=map] (26) at (1, 4) {$g$};
		\node [style=map] (27) at (0, 4) {$f$};
		\node [style=none] (28) at (1, 5.75) {};
		\node [style=none] (29) at (0, 5.75) {};
		\node [style=none] (30) at (1, 3.25) {};
		\node [style=none] (31) at (0, 3.25) {};
	\end{pgfonlayer}
	\begin{pgfonlayer}{edgelayer}
		\draw [in=270, out=90] (24.center) to (22.center);
		\draw [in=270, out=90] (25.center) to (23.center);
		\draw (23.center) to (29.center);
		\draw (22.center) to (28.center);
		\draw (30.center) to (25.center);
		\draw (31.center) to (24.center);
	\end{pgfonlayer}
\end{tikzpicture}
=
\begin{tikzpicture}
	\begin{pgfonlayer}{nodelayer}
		\node [style=none] (32) at (3, 5) {};
		\node [style=none] (33) at (2, 5) {};
		\node [style=none] (34) at (2, 4) {};
		\node [style=none] (35) at (3, 4) {};
		\node [style=map] (36) at (2, 5) {$g$};
		\node [style=map] (37) at (3, 5) {$f$};
		\node [style=none] (38) at (3, 5.75) {};
		\node [style=none] (39) at (2, 5.75) {};
		\node [style=none] (40) at (3, 3.25) {};
		\node [style=none] (41) at (2, 3.25) {};
	\end{pgfonlayer}
	\begin{pgfonlayer}{edgelayer}
		\draw [in=270, out=90] (34.center) to (32.center);
		\draw [in=270, out=90] (35.center) to (33.center);
		\draw (33.center) to (39.center);
		\draw (32.center) to (38.center);
		\draw (40.center) to (35.center);
		\draw (41.center) to (34.center);
	\end{pgfonlayer}
\end{tikzpicture}
$$

The self inverse of the braid means that the wires untangle:

$$
\begin{tikzpicture}
	\begin{pgfonlayer}{nodelayer}
		\node [style=none] (22) at (1, 5) {};
		\node [style=none] (23) at (0, 5) {};
		\node [style=none] (24) at (0, 4) {};
		\node [style=none] (25) at (1, 4) {};
		\node [style=none] (30) at (1, 3) {};
		\node [style=none] (31) at (0, 3) {};
	\end{pgfonlayer}
	\begin{pgfonlayer}{edgelayer}
		\draw [in=270, out=90] (24.center) to (22.center);
		\draw [in=270, out=90] (25.center) to (23.center);
		\draw [in=270, out=90] (31.center) to (25.center);
		\draw [in=270, out=90] (30.center) to (24.center);
	\end{pgfonlayer}
\end{tikzpicture}
=
\begin{tikzpicture}
	\begin{pgfonlayer}{nodelayer}
		\node [style=none] (32) at (3, 5) {};
		\node [style=none] (33) at (2, 5) {};
		\node [style=none] (36) at (3, 3) {};
		\node [style=none] (37) at (2, 3) {};
	\end{pgfonlayer}
	\begin{pgfonlayer}{edgelayer}
		\draw (37.center) to (33.center);
		\draw (36.center) to (32.center);
	\end{pgfonlayer}
\end{tikzpicture}
$$


And the other coherence equation becomes trivial:

$$
\begin{tikzpicture}
	\begin{pgfonlayer}{nodelayer}
		\node [style=none] (22) at (1, 5) {};
		\node [style=none] (23) at (0, 5) {};
		\node [style=none] (33) at (2, 5) {};
		\node [style=none] (34) at (2, 4) {};
		\node [style=none] (35) at (0, 4) {};
		\node [style=none] (36) at (1, 4) {};
		\node [style=none] (37) at (0, 3) {};
		\node [style=none] (38) at (1, 3) {};
		\node [style=none] (39) at (2, 3) {};
	\end{pgfonlayer}
	\begin{pgfonlayer}{edgelayer}
		\draw [in=270, out=90] (34.center) to (22.center);
		\draw (23.center) to (35.center);
		\draw [in=270, out=90] (36.center) to (33.center);
		\draw (39.center) to (34.center);
		\draw [in=270, out=90] (37.center) to (36.center);
		\draw [in=270, out=90] (38.center) to (35.center);
	\end{pgfonlayer}
\end{tikzpicture}
=
\begin{tikzpicture}
	\begin{pgfonlayer}{nodelayer}
		\node [style=none] (40) at (4, 5) {};
		\node [style=none] (41) at (3, 5) {};
		\node [style=none] (42) at (5, 5) {};
		\node [style=none] (46) at (3, 3) {};
		\node [style=none] (47) at (4, 3) {};
		\node [style=none] (48) at (5, 3) {};
	\end{pgfonlayer}
	\begin{pgfonlayer}{edgelayer}
		\draw [in=-90, out=90] (48.center) to (40.center);
		\draw [in=-90, out=90] (47.center) to (41.center);
		\draw [in=-90, out=90] (46.center) to (42.center);
	\end{pgfonlayer}
\end{tikzpicture}
$$



\begin{theorem}
Every symmetric monoidal category is strong-monoidally isomorphic to a strict symetric monoidal category. 
\end{theorem}

Non-strict symmetric monoidal categories have essentially the same notion of proof nets as non-strict monoidal categories, except now the symmetry is drawn as follows:

$$
c_{X,Y}=
\begin{tikzpicture}
	\begin{pgfonlayer}{nodelayer}
		\node [style=none] (2) at (16.25, 2.75) {};
		\node [style=none] (3) at (16.25, 2) {};
		\node [style=otimes] (4) at (16.25, 2.75) {};
		\node [style=none] (7) at (16.25, 4.25) {};
		\node [style=none] (8) at (16.25, 5) {};
		\node [style=otimes] (9) at (16.25, 4.25) {};
		\node [style=none] (10) at (16.25, 1.75) {$X\otimes Y$};
		\node [style=none] (11) at (16.25, 5.25) {$Y\otimes X$};
	\end{pgfonlayer}
	\begin{pgfonlayer}{edgelayer}
		\draw (3.center) to (2.center);
		\draw (8.center) to (7.center);
		\draw [in=-150, out=30, looseness=1.50] (2.center) to (7.center);
		\draw [in=150, out=-30, looseness=1.50] (7.center) to (2.center);
	\end{pgfonlayer}
\end{tikzpicture}
$$


\begin{definition}
A {\bf compact closed category} is a symmetric monoidal category such that for every object $X$, there is a chosen object $X^*$, called the {\bf dual} of $X$ along with the following maps, unit and counit:

$$
\eta_X:I\to X^* \otimes X\to I \hspace*{.5cm}\text{and}\hspace*{.5cm} \epsilon_X:X\otimes X^*\to I
$$

Satisfying the following coherence equations

\begin{itemize}
\item
$$
\xymatrix{
  (X\otimes X^*)\otimes X \ar[rr]^{\alpha_{X,X^*,X}}  \ar[d]_{\epsilon_X\otimes 1_X}
    & 
    & X\otimes(X^*\otimes X)
  \\I\otimes X \ar[rr]_{c_{I,X}}
    &
    & X\otimes I \ar[u]_{a_X\otimes \eta_X}
}
$$

\item

$$
\xymatrix{
  X^*\otimes ( X\otimes X^*) \ar[rr]^{\alpha_{X^*,X,X^*}}  \ar[d]_{1_{X^*}\otimes \epsilon_X}
    & 
    & (X^* \otimes X)\otimes X^*
  \\X^*\otimes I \ar[rr]_{c_{X^*,I}}
    &
    & I\otimes X^* \ar[u]_{\eta_X\otimes 1_X{X^*}}
}
$$
\end{itemize}


A strict compact closed category is a compact closed category where the underlying symmetric monoidal category is strict.

Strict symmetric monoidal functors and strong symmetric monoidal functors are the appropriate notion of map between strict/non-strict compact closed categories, as they preserve the duals stricly/strongly.
\end{definition}

The graphical calculus for strict compact closed categories extends string diagrams for symmetric monoidal categories, where  the units and counits are drawn as cups and caps:

$$
\eta_X=
\begin{tikzpicture}
	\begin{pgfonlayer}{nodelayer}
		\node [style=none] (1) at (0.5, -0.25) {};
		\node [style=none] (4) at (1.5, -0.25) {};
		\node [style=none] (5) at (0.5, 0) {$X^*$};
		\node [style=none] (6) at (1.5, 0) {$X$};
	\end{pgfonlayer}
	\begin{pgfonlayer}{edgelayer}
		\draw [in=270, out=-90, looseness=1.75] (1.center) to (4.center);
	\end{pgfonlayer}
\end{tikzpicture}
\hspace*{.5cm}\text{and}\hspace*{.5cm}
\epsilon_X=
\begin{tikzpicture}
	\begin{pgfonlayer}{nodelayer}
		\node [style=none] (7) at (3.25, 0.5) {};
		\node [style=none] (8) at (2.25, 0.5) {};
		\node [style=none] (9) at (3.25, 0.25) {$X^*$};
		\node [style=none] (10) at (2.25, 0.25) {$X$};
	\end{pgfonlayer}
	\begin{pgfonlayer}{edgelayer}
		\draw [in=90, out=90, looseness=1.75] (7.center) to (8.center);
	\end{pgfonlayer}
\end{tikzpicture}
$$


The coherence conditions are the yanking/zig-zag/snake equations:

$$
\begin{tikzpicture}
	\begin{pgfonlayer}{nodelayer}
		\node [style=none] (1) at (1.25, 0.5) {};
		\node [style=none] (4) at (2.25, 0.5) {};
		\node [style=none] (7) at (3.25, 0.5) {};
		\node [style=none] (8) at (3.25, -0.5) {};
		\node [style=none] (9) at (1.25, 1.5) {};
	\end{pgfonlayer}
	\begin{pgfonlayer}{edgelayer}
		\draw [in=270, out=-90, looseness=1.75] (1.center) to (4.center);
		\draw [in=90, out=90, looseness=1.75] (4.center) to (7.center);
		\draw (8.center) to (7.center);
		\draw (1.center) to (9.center);
	\end{pgfonlayer}
\end{tikzpicture}
=
\begin{tikzpicture}
	\begin{pgfonlayer}{nodelayer}
		\node [style=none] (13) at (6.25, -0.5) {};
		\node [style=none] (14) at (4.25, 1.5) {};
	\end{pgfonlayer}
	\begin{pgfonlayer}{edgelayer}
		\draw [in=90, out=-90] (14.center) to (13.center);
	\end{pgfonlayer}
\end{tikzpicture}
\hspace*{1cm}
\begin{tikzpicture}
	\begin{pgfonlayer}{nodelayer}
		\node [style=none] (15) at (7.25, 0.5) {};
		\node [style=none] (16) at (8.25, 0.5) {};
		\node [style=none] (17) at (9.25, 0.5) {};
		\node [style=none] (18) at (9.25, 1.5) {};
		\node [style=none] (19) at (7.25, -0.5) {};
	\end{pgfonlayer}
	\begin{pgfonlayer}{edgelayer}
		\draw [in=90, out=90, looseness=1.75] (15.center) to (16.center);
		\draw [in=-90, out=-90, looseness=1.75] (16.center) to (17.center);
		\draw (18.center) to (17.center);
		\draw (15.center) to (19.center);
	\end{pgfonlayer}
\end{tikzpicture}
=
\begin{tikzpicture}
	\begin{pgfonlayer}{nodelayer}
		\node [style=none] (20) at (12.25, 1.5) {};
		\node [style=none] (21) at (10.25, -0.5) {};
	\end{pgfonlayer}
	\begin{pgfonlayer}{edgelayer}
		\draw [in=-90, out=90] (21.center) to (20.center);
	\end{pgfonlayer}
\end{tikzpicture}
$$

Compact-closed categories axiomatize the kinds of processes where inputs can be turned into outputs, and vice-versa.  In other words, they axiomatize a particular notion of feedback.


\begin{example}
Out of all the examples we have discussed so far, only matrices/finite dimensional vector spaces under the bilinear tensor product are compact closed.

Given a basis $\{ | i \rangle \}_{i=0,\ldots, n-1}$ of a finite dimensional vector space $X$, with dual basis $\{\langle i| = |i \rangle^* \}_{i=0,\ldots, n-1}$ of $X^*$, the unit and counit are given by the following linear maps:

$$
\eta_X = 1 \mapsto \sum_{i=0}^{n-1} \langle i| \otimes | i \rangle \hspace*{1cm}\epsilon_X = \sum_{i=0}^{n-1} | i \rangle\otimes \langle i| \mapsto \delta_{i,j}
$$

Where $\delta_{i,j}$ is $1$ if $i=j$ and $0$ otherwise.

In the finite dimensional case, we don't even need to be working with a field, Given a semiring $R$, $\Mat_R$ is compact closed with respect to the bilinear tensor product.
\end{example}

\begin{theorem}
Every compact closed category is strong-monoidally isomorphic to a strict compact closed category. 
\end{theorem}

Similarly, their proof nets are essentially the same as for symmetric monoidal categories, where now the units and counits of the compact closed structure are drawn as follows:

$$
\eta_X=
\begin{tikzpicture}
	\begin{pgfonlayer}{nodelayer}
		\node [style=none] (7) at (18, 2.75) {};
		\node [style=none] (8) at (18, 3.75) {};
		\node [style=none] (9) at (18, 4.25) {};
		\node [style=otimes] (10) at (18, 3.75) {};
		\node [style=none] (11) at (18, 4.5) {$X^*\otimes X$};
	\end{pgfonlayer}
	\begin{pgfonlayer}{edgelayer}
		\draw [in=180, out=-150, looseness=1.50] (8.center) to (7.center);
		\draw [in=-30, out=0, looseness=1.50] (7.center) to (8.center);
		\draw (9.center) to (8.center);
	\end{pgfonlayer}
\end{tikzpicture}
\hspace*{.5cm}
\text{and}
\hspace*{.5cm}
\epsilon_X
=
\begin{tikzpicture}
	\begin{pgfonlayer}{nodelayer}
		\node [style=none] (0) at (16.5, 4.5) {};
		\node [style=none] (1) at (16.5, 3.5) {};
		\node [style=none] (4) at (16.5, 3) {};
		\node [style=otimes] (5) at (16.5, 3.5) {};
		\node [style=none] (6) at (16.5, 2.75) {$X\otimes X^*$};
	\end{pgfonlayer}
	\begin{pgfonlayer}{edgelayer}
		\draw [in=-180, out=150, looseness=1.50] (1.center) to (0.center);
		\draw [in=30, out=0, looseness=1.50] (0.center) to (1.center);
		\draw (4.center) to (1.center);
	\end{pgfonlayer}
\end{tikzpicture}
$$



\subsubsection{A note on proof nets}

In most contexts we will use the string diagrams for strict monoidal/symmetric monoidal/compact closed categories.  However, at some points, the careful treatment of units is needed, necessitating this exposition of proof nets. 

As a historical note, proof nets were originally invented by Girard as a graphical calculus for multiplicative linear logic \cite{???}, a resource-senstitive refinement of logic where the antecendents and consequents of sequents can not be freely copied and deleted.

For example, in multiplicative linear logic, two proofs $X \vdash Y$ and $Y \vdash Z$ are cut together in a way that consumes $Y$:

$$
\dfrac{\Gamma, X\vdash \Delta,Y \hspace{.5cm}  Y,\Gamma'\vdash Z,\Delta'}{\Gamma,X,\Gamma' \vdash \Delta, Z, \Delta'}
$$

The symbols $\Gamma,\Gamma'$ and $\Delta,\Delta'$ represent the contexts in the antecendents and sequents of the proof.

Later, Cockett Blute and Seely formalized  multiplicative linear logic in the representable setting; that is the setting where the antecendents and consequents can be tensored together.
They formalize the notion of linearly distributive categories by having two separate tensor product functors for the antencedents and consequents, called tensor (drawn as $\otimes$) and par (drawn as $\otimes$) respectively which distribute over each other ``linearly'' according to left and right linear distributors:
$$
\delta_{X,Y,Z}^L: X \otimes (Y\oplus Z) \to (X\otimes Y)\oplus Z
\hspace*{.5cm}
\delta_{X,Y,Z}^R: (X\oplus Y) \otimes Z \to X\oplus (Y \otimes Z)
$$



They with a formal calculus of proof nets with a completeness proof \cite{ldc}.  In Proof nets there are now tensoring and untensoring operations for both the tensor and par:

$$
\begin{tikzpicture}
	\begin{pgfonlayer}{nodelayer}
		\node [style=none] (105) at (18.75, 5) {};
		\node [style=none] (106) at (17.75, 5) {};
		\node [style=none] (107) at (18.25, 4.25) {};
		\node [style=none] (108) at (18.25, 3.5) {};
		\node [style=none] (109) at (17.75, 5.25) {$X$};
		\node [style=none] (110) at (18.75, 5.25) {$Y$};
		\node [style=none] (111) at (18.25, 3.25) {$X\otimes Y$};
		\node [style=otimes] (112) at (18.25, 4.25) {};
	\end{pgfonlayer}
	\begin{pgfonlayer}{edgelayer}
		\draw (108.center) to (107.center);
		\draw [in=-90, out=30] (107.center) to (105.center);
		\draw [in=-90, out=150] (107.center) to (106.center);
	\end{pgfonlayer}
\end{tikzpicture}
\ ,\hspace*{.5cm}
\begin{tikzpicture}
	\begin{pgfonlayer}{nodelayer}
		\node [style=none] (113) at (20.75, 3.5) {};
		\node [style=none] (114) at (19.75, 3.5) {};
		\node [style=none] (115) at (20.25, 4.25) {};
		\node [style=none] (116) at (20.25, 5) {};
		\node [style=none] (117) at (19.75, 3.25) {$X$};
		\node [style=none] (118) at (20.75, 3.25) {$Y$};
		\node [style=none] (119) at (20.25, 5.25) {$X\otimes Y$};
		\node [style=otimes] (120) at (20.25, 4.25) {};
	\end{pgfonlayer}
	\begin{pgfonlayer}{edgelayer}
		\draw (116.center) to (115.center);
		\draw [in=90, out=-30] (115.center) to (113.center);
		\draw [in=90, out=-150] (115.center) to (114.center);
	\end{pgfonlayer}
\end{tikzpicture}
\ ,\hspace*{.5cm}
\begin{tikzpicture}
	\begin{pgfonlayer}{nodelayer}
		\node [style=none] (105) at (18.75, 5) {};
		\node [style=none] (106) at (17.75, 5) {};
		\node [style=none] (107) at (18.25, 4.25) {};
		\node [style=none] (108) at (18.25, 3.5) {};
		\node [style=none] (109) at (17.75, 5.25) {$X$};
		\node [style=none] (110) at (18.75, 5.25) {$Y$};
		\node [style=none] (111) at (18.25, 3.25) {$X\oplus Y$};
		\node [style=oplus, fill=white] (112) at (18.25, 4.25) {};
	\end{pgfonlayer}
	\begin{pgfonlayer}{edgelayer}
		\draw (108.center) to (107.center);
		\draw [in=-90, out=30] (107.center) to (105.center);
		\draw [in=-90, out=150] (107.center) to (106.center);
	\end{pgfonlayer}
\end{tikzpicture}
\ ,\hspace*{.5cm}
\begin{tikzpicture}
	\begin{pgfonlayer}{nodelayer}
		\node [style=none] (113) at (20.75, 3.5) {};
		\node [style=none] (114) at (19.75, 3.5) {};
		\node [style=none] (115) at (20.25, 4.25) {};
		\node [style=none] (116) at (20.25, 5) {};
		\node [style=none] (117) at (19.75, 3.25) {$X$};
		\node [style=none] (118) at (20.75, 3.25) {$Y$};
		\node [style=none] (119) at (20.25, 5.25) {$X\oplus Y$};
		\node [style=oplus, fill=white] (120) at (20.25, 4.25) {};
	\end{pgfonlayer}
	\begin{pgfonlayer}{edgelayer}
		\draw (116.center) to (115.center);
		\draw [in=90, out=-30] (115.center) to (113.center);
		\draw [in=90, out=-150] (115.center) to (114.center);
	\end{pgfonlayer}
\end{tikzpicture}
$$




In the proof net notation, the linear distributors then are drawn as follows:


$$
\delta_{X,Y,Z}^L
=
\begin{tikzpicture}
	\begin{pgfonlayer}{nodelayer}
		\node [style=none] (141) at (25, 5) {};
		\node [style=none] (142) at (26, 5) {};
		\node [style=none] (143) at (25.5, 4.25) {};
		\node [style=none] (144) at (25.5, 3.5) {};
		\node [style=none] (145) at (25.5, 3.25) {$X\otimes (Y\oplus Z)$};
		\node [style=otimes] (146) at (25.5, 4.25) {};
		\node [style=none] (147) at (25.5, 5.75) {};
		\node [style=none] (148) at (26.5, 5.75) {};
		\node [style=none] (149) at (26, 5) {};
		\node [style=oplus, fill=white] (151) at (26, 5) {};
		\node [style=none] (152) at (26.5, 5.75) {};
		\node [style=none] (153) at (25.5, 5.75) {};
		\node [style=none] (154) at (26, 6.5) {};
		\node [style=none] (155) at (26, 7.25) {};
		\node [style=oplus, fill=white] (156) at (26, 6.5) {};
		\node [style=none] (157) at (26, 5) {};
		\node [style=none] (158) at (25, 5) {};
		\node [style=none] (159) at (25.5, 5.75) {};
		\node [style=otimes] (160) at (25.5, 5.75) {};
		\node [style=none] (161) at (26, 7.5) {$(X\otimes Y)\oplus Z$};
	\end{pgfonlayer}
	\begin{pgfonlayer}{edgelayer}
		\draw (144.center) to (143.center);
		\draw [in=-90, out=150] (143.center) to (141.center);
		\draw [in=-90, out=30] (143.center) to (142.center);
		\draw [in=-90, out=30] (149.center) to (148.center);
		\draw (155.center) to (154.center);
		\draw [in=90, out=-30] (154.center) to (152.center);
		\draw [in=90, out=-150] (154.center) to (153.center);
		\draw (159.center) to (157.center);
		\draw [in=90, out=-150] (159.center) to (158.center);
	\end{pgfonlayer}
\end{tikzpicture}
\ , \hspace*{.5cm}
\delta_{X,Y,Z}^R 
=
\begin{tikzpicture}
	\begin{pgfonlayer}{nodelayer}
		\node [style=none] (121) at (23.25, 5) {};
		\node [style=none] (122) at (22.25, 5) {};
		\node [style=none] (123) at (22.75, 4.25) {};
		\node [style=none] (124) at (22.75, 3.5) {};
		\node [style=none] (125) at (22.75, 3.25) {$(X\oplus Y)\otimes Z$};
		\node [style=otimes] (126) at (22.75, 4.25) {};
		\node [style=none] (127) at (22.75, 5.75) {};
		\node [style=none] (128) at (21.75, 5.75) {};
		\node [style=none] (129) at (22.25, 5) {};
		\node [style=none] (130) at (22.25, 7.5) {$X\oplus (Y\otimes Z)$};
		\node [style=oplus, fill=white] (131) at (22.25, 5) {};
		\node [style=none] (132) at (21.75, 5.75) {};
		\node [style=none] (133) at (22.75, 5.75) {};
		\node [style=none] (134) at (22.25, 6.5) {};
		\node [style=none] (135) at (22.25, 7.25) {};
		\node [style=oplus, fill=white] (136) at (22.25, 6.5) {};
		\node [style=none] (137) at (22.25, 5) {};
		\node [style=none] (138) at (23.25, 5) {};
		\node [style=none] (139) at (22.75, 5.75) {};
		\node [style=otimes] (140) at (22.75, 5.75) {};
	\end{pgfonlayer}
	\begin{pgfonlayer}{edgelayer}
		\draw (124.center) to (123.center);
		\draw [in=-90, out=30] (123.center) to (121.center);
		\draw [in=-90, out=150] (123.center) to (122.center);
		\draw [in=-90, out=150] (129.center) to (128.center);
		\draw (135.center) to (134.center);
		\draw [in=90, out=-150] (134.center) to (132.center);
		\draw [in=90, out=-30] (134.center) to (133.center);
		\draw (139.center) to (137.center);
		\draw [in=90, out=-30] (139.center) to (138.center);
	\end{pgfonlayer}
\end{tikzpicture}
$$

Therefore, one can see monoidal categories as special cases of linearly distributive categories where the distributors are the associators, ie where the distributors are isomorphisms.

The proof nets of \cite{ldc} do not explicitly focus on the degenerate case of monoidal categories. However, in the case of nondegenerate linearly distributive categories the appropriate notion of isotopy is inherently 3-dimensional as the linear distribution adds another dimension.  A graphical calculus of 3-dimensional dimensional natural was later explored in  \cite{dunn}; see for example the following translation between a proof net and a 3d-surface:

\begin{figure}
Insert picture from Jamie et als paper
\end{figure}




However, because proof nets are drawn on a 2-dimensional sheet of paper, the projection loses information: in particular the unit introduction and removal are degeneracies under the projection, making their connectivity  under the projection ambiguous. In \cite{ldc} the units are accounted for by using so-called ``thinning links,'' which we won't expose here as we shall not use it, and its combinatorial nature doesn't lead to a particularly elegant exposition.  This can be ignored in the case of monoidal categories, because they are inherently lower-dimensional objects; thus these problems with projections are assuaged. A direct construction of proof nets was later presented by \cite{wilson}, giving an inductive proof of MacLane's coherence theorem.  This greatly simplified the exposition of proof nets in the monoidal case.

Proof nets have also been rediscovered in the compact closed case in the form of the scalable ZX-calculus \cite{szx} although they have not discovered the units and counits.  The scalable ZX-calculus has been since used extensively as a graphical calculus for quantum protocols where one seeks to index a register of wires \cite{?????}.  We will motivate this more in Section \cite{????} when we introduce categorical quantum mechanics and the ZX-calculus.

On a separate note, linearly distributive categories have also been used to explore quantum causality \cite{sander} as well as to give toy models for infinite dimensional quantum processes  \cite{muc}.  Therefore, there is also motivation for understanding proof nets in the non-monoidal setting in the context of exploring quantum mechanics from a categorical perpsective, although it is not the focus of this thesis.


\subsubsection{Monoidal presentations}
In this section, we review how monoidal categories can be presented in terms of generators and equations. .





\begin{definition}
\label{def:monoidaltheory}

%A {\bf symmetric monoidal theory} is a triple $T=({\sf Ob},\Sigma ,E \)$. $\sf Ob$ is a set of {\em objects}. $\Sigma\in [{\sf Ob}]\times [{\sf Ob}]^{G}$ is a set of {\bf generators} $g$ with associated arities $(X,Y) \in [{\sf Ob}]\times [{\sf Ob}]$,  denoted $g:X\to Y$, where $[\_]$ is the finite list monad. Let $\Sigma'$ denote the set $\Sigma\sqcup\{c_X: [X,X]\to[X,X] | \forall X \in {\sf Ob} \}$, where the $c_X$ are regarded as the braiding maps.  Moreover, let $\Sigma^*$ denote the 
%$E \subseteq \{(f:X\to Y,g:X\to Y) \in \Sigma^2\}$

A {\bf monoidal theory} is a triple $T=({\sf Ob},\Sigma ,E )$ where: ${\sf Ob}$ is regarded as the set of generating objects; $\Sigma$, the set of generating morphisms with arities in $[{\sf Ob}]\times [{\sf Ob}]$, denoted $f:X\to Y$;  and $E$ the set of generating equations between parallel maps generated by $\Sigma$.

Every monoidal theory uniquely defines a strict monoidal category $\bar T$ given by quotienting the strict monoidal category freely generated by the objects in $\sf Ob$ and maps in $\Sigma$ modulo the equations in $E$.  Call such a monoidal category a {\bf multicoloured pro}, or merely a {\bf pro} when $|{\sf Ob}|=1$.  We will say that $T$ is a presentation of $\bar T$.


Similarly, a symmetric monoidal theory $T$ consists of the same data as a monoidal theory except the equations are now defined by parallel maps generated by $\Sigma\sqcup C$, where  $C=\{c_X:[X,X]\to [X,X]|\forall X \in {\sf Ob}\}$ is the set of distinguished braiding maps.


The corresponding strict symmetric monoidal category $\bar T$ is given by quotienting the symmetric monoidal category freely generated by the objects $\sf Ob$ and morphisms $\Sigma$ by the equations in $E$.  These symmetric monoidal categories are called {\bf multicoloured props}, or merely {\bf props} when $|{\sf Ob}|=1$.
\end{definition}

In practice, we won't explicitly regard (symmetric) monoidal theory as a triple; rather, we will present multicoloured pro(p)s by first drawing the string diagrams for all of the generators and then equations string diagrams for the equations. For example:




\begin{example}
Let $\cm$ denote the prop generated by the free commutative monoid on one object; that is, the prop with generators:


and equations:

\end{example}

We see that this is a formal way to talk about the graph of a function between finite sets:

\begin{lemma}
$\cm$ is equivalent as a symmetric monoidal category to the symmetric monoidal category of finite sets and functions with respect to the coproduct as the monoidal struture.
\end{lemma}

This elegant description of the symmetric monoidal category of finite sets motivates finding presentations for other well-known mathematical structures.

\begin{example}
Prop of matrices
\end{example}


We will use this presentation extensively throughout the thesis.  We will also use the following:

\begin{example}
Multicoloured prop of affine matrices (with zero)
\end{example}





%
%There are various ways to combine symmetric monoidal theories theories, by coproduct, pushout and distributive laws:
%
%\begin{lemma}
%Given two symmetric  monoidal theories $({\sf Ob}_1, \Sigma_1,E_1)$  and $({\sf Ob}_1, \Sigma_2,E_2)$  the coproduct  $\bar{(\Sigma_1,E_1)}+\bar{(\Sigma_2,E_2)}$ is generated by the (symmetric) monoidal theory $(\Sigma_1+\Sigma_2,E_1+E_2)$.
%\end{lemma}





The following two lemmatae are out of place
Operations on monoidal theory:  pushout, coproduct



\begin{lemma}
coproduct of symmetric monoidal theories
\end{lemma}

\begin{lemma}
pushout of symmetric monoidal theories
\end{lemma}



There is a third quite important way to compose presentations of monoidal and symmetric monoidal categories which we will introduce slightly later, because it requires a considerable amount of mathematical machinery to expose.

%Notice that the coproduct is a special case of the pushout.

\subsubsection{Dagger-monoidal categories}
In this thesis, we will usually work with monoidal categories with extra structure called the dagger which allows one to "run maps in reverse:"


\begin{definition}
A {\bf \dag-category} ({\em read dagger-category}) is a category $\X$ equipped with a functor ${(\_)}\dag:\X^\op\to\X$ (the dagger) which is:

\begin{description}
\item[Identity on objects:] so that for all objects $X$ of $\X$, $X^\dag = X$.
\item[Involutive:] so that for all maps $f$ of $\X$, $(f^\dag)^\dag = f$.
\end{description}

A map $f$ in a dagger category is an:

\begin{description}
\item[Isometry] when $f^\dag; f = 1$.
\item[Coisometry] when $f; f^\dag = 1$.
\item[Unitary] when $f^\dag = f^{-1}$.
\item[Projector] such that $f;f=f$ and $f^\dag=f$ (also known as a $\dag$-idempotent).
\end{description}

\end{definition}


\begin{example}
 $\Mat_\C$ is a \dag-category with respect to both the transpose and the complex conjugate transpose.
\end{example}

\begin{example}

The category $\Hilb$ of complex Hilbert spaces and bounded linear maps is a dagger category with respect to the Hermetian adjoint 
$$
\langle x;A|y\rangle = \langle x | A^\dag; y \rangle
$$


Let $\FHilb$ denote the full subcategory of finite dimensional Hilbert spaces of $\Hilb$.

There is an equivalence of categories $\FHilb \cong \Mat_\C$ preserving and reflecting the dagger structure.
The Hermetian conjugate of finite dimensional Hilbert spaces corresponds to the complex conjugate transpose along the equivalence $\Mat_\C \cong \FHilb$.


\end{example}



This leads naturally to the following definition:

\begin{definition}
A {\bf  (strict) \dag-(symmetric) monoidal category} is a (strict) (symmetric) monoidal category equipped with a strict (symmetric) dagger functor with respect to which all the components of the  coherence isomorphisms of the (symmetric) monoidal structure are unitary.
\end{definition}



\begin{example}
The dagger category and symmetric monoidal structures of $\FHilb$, $\Hilb$ and $\Mat_\C$ are all compatible making them \dag-symmetric monoidal categories.
\end{example}


\begin{definition}
A {(strict) \dag-compact closed category} is a (strict) compact closed category which is (strict) \dag-symmetric monoidal and for all objects $\X$:

$$
\xymatrix{
I \ar[r]^{\epsilon_X^\dag} \ar[dr]_{\eta_X}   &  X\otimes X^* \ar[d]^{c_{X,X^*}}\\
                                                                        &  X^* \otimes X 
}
$$
\end{definition}

\begin{example}
$\Mat_\C$ is a strict \dag-compact closed category with respect to both the transpose and the complex conjugate transpose.  
$\FHilb$ is \dag-compact closed with respect to the  Hermetian adjoint.

\end{example}




\subsection{Spans and relations in computation}

Categories are defined in a manner which distinguishes the inputs and outputs of morphisms.  \dag-categories are one approach to moving beyond this bias; however, they are evil in the sense are the \dag-structure is not always preserved/reflected by categorical equivalence.  Spans and relations provide a categorically well-behaved, flexible setting with which to interpret processes without elevating inputs over outputs.
If functions are interpreted as producing unique outputs from inputs, spans and relations possibilistically associate several inputs with several outputs.  To introduce these mathematical constructions, we first need to recall some basic facts about limits: 



\begin{definition}
The {\bf product} of two objects $X$ and $Y$ (if it exists) is an object $X\times Y$ equipped with maps $\pi_0:X\times Y\to X $ and $\pi_1:X\times Y \to Y$ called the {\bf projections},  such that for any object $A$ and diagram  $X \xleftarrow{f} A \xrightarrow{g} Y$ there exists a unique map $\langle  f, g \rangle :A \to X\times Y$ called {\bf the pairing map} making the following diagram commute:


$$
\xymatrix{
    &
    & A \ar[lld]_f \ar[rrd]^g \ar@{-->}[d]^{\langle f,g\rangle}
    &
    &
  \\X 
    &
    & X\times Y \ar[ll]^{\pi_0} \ar[rr]_{\pi_1}
    &
    & Y
}
$$ 

Given two maps $f:W\to X$ and $G:Y\to Z$, their product is defined to be the universal map $f\times g:W\times Y \to X\times Z$: 

$$
\xymatrix{
    W \ar[d]_f
    &
    & W\times Y \ar@{-->}[d]^{f\times g} \ar[ll]_{\pi_0} \ar[rr]^{\pi_1}
    &
    & Y \ar[d]^g
  \\X
    &
    & Y\times Z  \ar[ll]_{\pi^0} \ar[rr]_{\pi_1}
    &
    &Z
}
$$


A terminal object (if it exists) is an object $1$ equipped with a unique map $1:X\to 1$ for every object $1$ called the {\bf discard map}.



A category is  {\bf Cartesian } when it has all finite products and a terminal object. In a Cartesian category, the discard maps and {\bf diagonals} $\langle 1, 1 \rangle$ yield a natural family of commutative monoids.


The products/projections/pairing maps/terminal objects/cartesian categories/diagonal maps in $\X$ are respectively {\bf coproducts}/{\bf injections}/{\bf copairing maps}/{\bf initial objects}/{\bf cocartesian categories}/{\bf codiagonal maps} in $\X^\op$.

\end{definition}

The basic idea is that Cartesian categories axiomatize copying, and coCartesian categories axiomatize comparison.


The following Lemma justfies thinking of our running  examples of (finite) sets under the Cartesian product and matrices, vector spaces, hilbert spaces under the direct sum as symmetric monoidal categories.  Similarly, for (finite) sets as a symmetric monoidal categories under the coproduct.

\begin{lemma}
(Co)cartesian categories are symmetric monoidal with respect to the (co)product as the tensor and the initial/terminal object as the unit.
\end{lemma}




The Cartesian notion of copying is biases inputs over outputs; however, this is fundementally incompatible with dagger structure.  We are interested in a more permissive, symmetric notion of copying. The following notion allows us to develop such a structure:


\begin{definition}
The {\bf pullback} of a diagram  $X \xrightarrow{f} A \xleftarrow{g} Y$ (if it exists) is an object $X\ {}_f \times_{g} Y$ called and maps $\pi_0:X\ {}_f \times_{g} Y\to X$ and  $\pi_1:X\ {}_f \times_{g} Y\to Y$ called {\bf the  projections}, such that for any diagram $X \xleftarrow{p_0} P \xrightarrow{p_1} Y$ making the the following diagram commute:

$$
\xymatrix{
    &
    & B   \ar[dll]_{p_0} \ar[drr]^{p^1}
    &
    &
  \\X \ar[drr]_f 
    &
    & 
    &
    & Y  \ar[dll]^{g}
  \\
    &
    & A
    &
    & 
}
$$

There exists a unique map $\langle p_0, p^1 \rangle: P\to X\ {}_f\times_g Y $ called the {\bf pairing map} making the following diagram commute:

$$
\xymatrix{
    &
    & P \ar@/_/[ddll]_{p_0}  \ar@/^/[ddrr]^{p_1} \ar@{-->}[d]^{u}
    &
    &
  \\
    &
    & X\ {}_f \times_{g} Y  \ar[dll]_{\pi_0} \ar[drr]^{\pi_1}
    &
    &
  \\X \ar[drr]_f 
    &
    & 
    &
    & Y \ar[dll]^g 
  \\
    &
    & A
    &
    & 
}
$$

A category is {\bf finitely complete} if it has a terminal object and all pullbacks exist. Notice that product $X\times Y$ is the pullback of the diagram $X \rightarrow 1 \leftarrow Y$.
\end{definition}

\begin{example} 
In Sets the pullback of a cospan $X \xrightarrow{f} A \xleftarrow{g} Y$ is (up to unique isomorphism) the set $\{(x,y) \in X\times Y : f(x) = g(y)\}$.
The concrete pullback of matrices is essientially the same.
\end{example}

Spans form a 2-category under pullback:
\begin{definition}
Given a finitely complete category $\X$, the 2-category of spans $\Span(\X)$ has:

\begin{description}
\item[0-cells:] Objects of $\X$.
\item[1-cells:] 1-cells $(A,f,g):X\to Y$ are spans in $\X$ from $A$:

$$
\xymatrix{
    & A \ar[dl]_{f} \ar[dr]^{g}
    &
  \\X 
    &
    & Y
}
$$

Composition is given by the span induced by pullback:
$$
\xymatrix{
    & A\ar[dl]_{f} \ar[dr]^{g}
    &
  \\X 
    &
    & Y
}\ ;\
\xymatrix{
    & B\ar[dl]_{h} \ar[dr]^{k}
    &
  \\Y 
    &
    & Z
}
:=
\xymatrix{
    &
    & A {}_g\times_k B \ar[dl]_{\pi_0} \ar[dr]^{\pi_1}
    &
    &
  \\
    & A \ar[dl]_{f} \ar[dr]^{g}
    &
    & B \ar[dl]_{h} \ar[dr]^{k}
    &
  \\X
    &
    & Y
    &
    & Z
}
$$

The identity on $X$ is given by the span:

$$
\xymatrix{
    & X \ar@{=}[dl] \ar@{=}[dr] 
    &
  \\X 
    &
    & X
}
$$
%ooPoo
%oAoBo
%XoYoZ

\item[2-cells:] A 2-cell $\phi:(A,f,g)\Rightarrow (B,h,k)$ between parallel spans is a map $f:A\to B$ in $\X$ such that the following diagram commute

$$
\xymatrix{
    & A \ar[dl]_{f} \ar[dr]^{g} \ar[dd]^{\phi}
    &
  \\X 
    &
    & Y
  \\
    & B \ar[ul]^{h} \ar[ur]_{k}
    &
}
$$

The composition and identity of  2-cells is given by the compostition and identity in $\X$.
\end{description}

Note that composition of 1-cells is not strict, so that the associativity and unitality of composition hold up to coherent isomorphism.  The coherence isomorphisms are the natural 2-cells induced by the universal property of the pullback.


The 1-category of spans of $\X$, $\Span^\sim(\X)$, has maps being equivalence classes of isomorphic spans. 
\end{definition}


\begin{example}
Concrete description of spans of sets.
\end{example}


\begin{lemma}
Spans of finite sets are matrices over the natural numbers
\end{lemma}

In a sense, one can interpret the element of a matrix at coordinates $i,j$ as the number of possible ways to transition from index $j$ to index $i$.  We seek moreover, to quotient by multiplicity, to obtain a semantics for merely possibilistic processes.  To do so, we need more assumptions about the category with which we seek to work internal to.


\begin{definition}
The {\bf equalizer}, of two parallel maps $f,g:X\to Y$, if it exists, is an object $E_{f,g}$ equipped with a map $m:E_{f,g}\to X$ such that for all objects $F$ and maps $h:F\to E_{f,g}$, there exists a unique map $u:F\to A$ such that the following diagram commutes:

$$
\xymatrix{
    F \ar@{-->}[dr]^{u} \ar[d]_h
  \\ E_{f,g} \ar[r]_e
    & X \ar@<-.5ex>[r]_g \ar@<.5ex>[r]^f
    & Y
}
$$

The maps $e$ arising from equalizers are monomorphisms.  Monomorphisms arising this way are called {\bf regular monomorphisms}.


The dual notion to an equalizer is a {\bf coequalizer}, and the epimorphisms arizing in this way are called {\bf regular epimorphisms}.
\end{definition}



\begin{lemma}
Epi mono factorization TODO
\end{lemma}


\begin{example}
Sets and matrices both have equalizers.

In sets, the equalizer of two functions $g,f:X\to Y$ is (up to unique isomorphism) the set $\{x \in X:f(x)=g(x)\} \subseteq X$.

The coequalizer is the quotient$Y/\sim$   of the set $Y$ by the equivalence relation $f(x)\sim f(y)$.

The situation is basically the same for matrices.
\end{example}


\begin{definition}
A {\bf regular category} is a finitely complete category where moreover:

\begin{itemize}
\item For any map $f:X\to Y$, the pullback of $f$ along itself $\xymatrix{X\ {}_f \times_f X  \ar@<-.5ex>[r]_{\ \ \ \ \pi_0;f} \ar@<.5ex>[r]^{\ \ \ \ \pi_0;f} & Y}$ admits a coequalizer.

\item Pullbacks of arbitrary maps along regular epimorphisms are regular epimorphisms.
\end{itemize}

\end{definition}


\begin{example}
Sets, finite sets and matrices over a field are all regular categories.
\end{example}


\begin{definition}
Given a regular category $\X$, define the 2-category of {\bf relations} internal to $\X$, $\Rel(\X)$ has:

\begin{description}
\item[0-cells:] Objects of $\X$.
\item[1-cells:] 1-cells $(A,f,g):X\to Y$ are jointly monic spans in $\X$ from $A$:


$$
\xymatrix{
    & A \ar[dl]_{f} \ar[dr]^{g}
    &
  \\X 
    &
    & Y
}
$$
This span being {\bf jointly monic} means that for any object $B$ and morphisms $h,k:B\to A$ if $h;f=k;f$ and $h;g=k;g$, then $h=k$.

To compose jointly monic spans $(A,f,g):X\to Y$ and $(B,h,k):Y\to Z$,  first compute the pullback
$$
\xymatrix{
    &
    & A\ {}_g\times_k B \ar[dl]_{\pi_0} \ar[dr]^{\pi_1}
    &
    &
  \\
    & A \ar[dl]_{f} \ar[dr]^{g}
    &
    & B \ar[dl]_{h} \ar[dr]^{k}
    &
  \\X
    &
    & Y
    &
    & Z
}
$$

Composing with the pairing map we get a map $\langle \pi_0;f,\pi_1;k\rangle :A {}_g\times_k B \to X\times Z$.
Because $\X$ is a regular category, there is a factorization 

$$
\xymatrix{
  A\ {}_g\times_k B \ar[dr]^{\langle \pi_0;f,\pi_1;k\rangle}  \ar@{->>}[d]
  \\ E \ar@{>->}[r]_{e}
    &  X\times Z
}
$$

Which induces a jointly monic span, which we take to be the composite:

$$
\xymatrix{
    & A \ar[dl]_{f} \ar[dr]_{g}
    &
  \\X 
    &
    & Y
};
\xymatrix{
    & B \ar[dl]_{f} \ar[dr]_{g}
    &
  \\Y 
    &
    & Z
}
:=
\xymatrix{
    & E \ar[dl]_{e\pi_0} \ar[dr]^{e\pi_1}
    &
  \\X 
    &
    & Y
}
$$

The identity for composition is the same as for spans.

\item[2-cells:] The 2-cells and their composition and identity is the same as for spans.

\end{description}

\end{definition}

Internal relations have the special property that they are poset-enriched; that is to say, either there exists a single 2-cell between 1-cells or there doesn't. This makes things much simpler than the spans picture, because one never has to deal with coherence equations.  This also justifies the interpretation of possibilistic processes in this setting: possibility amounts to the mere existence of a 2-cell.


We give a concrete description of the canonical example of internal relations.


\begin{example}
$\Rel=\Rel(\Set)$ has:

\begin{description}
\item[0-cells:] Natural numbers.

\item[1-cells:] A relation from $n\to m$ is a a subset $X \times Y$.

The composition of relations $R \subseteq X \times Y$  and $S \subseteq Y \times Z$ is given by:
$$
R;S := \{  (x,z) \in X\times Z: \exists y \in Y, (x,y) \in R \wedge (y,z) \in S \} \subseteq X\times Z
$$ 

\item[2-cells:] 
A 2-cell $R\Rightarrow S$ is a subset $R\subseteq S$.
\end{description}

Finite relations $\FRel=\Rel(\FSet)$, linear relations $\LinRel=\Rel(\Mat_k)$ are defined in the same way.
\end{example}



Relations of finite sets have a familiar presentation:

\begin{lemma}
Relations of finite sets is isomorphic to matrices over the boolean semiring.
\end{lemma}



Relations are a possibilistic semantics for processes and spans are a possibilistic semantics, where the number of possible paths are also counted.  This process with which we factorize the pullback to obtain a jointly monic pair is essentially discarding the multiplicity of paths.


In the dual picure, corelations can be interpreted as the algebra for  partitions, and cospans as partitions with counting.  These interpretations are elucidated by looking at the monoidal presentations when these constructions are applied to finite sets.

\begin{lemma} %GIVE DAGGERS
Buni et als paper, separate into different lemmas


Rel/Span under coproduct is monoidal.  Give presentation for finite sets, hint at the later connection to distributive laws
  spans finite sets under the coproduct is strong monoidally isomorphic to natural number matrices under the direct sum
  relations of finite sets under the coproduct is strong monoidally isomorphic to boolean matrices under the direct sum

Rel/Span under product is monoidal.  Note that it is not a prop, so presenations are harder.  Hint at how we will take a shot at this in the ZXA section
  spans finite sets under the product is strong monoidally isomorphic to natural number matrices under the bilinear tensor product
  relations of finite sets under the product is strong monoidally isomorphic to boolean matrices under the bilinear tensor product


CoRel/Cospan under coproduct is monoidal.  Give presentation for finite sets, hint at the later connection to distributive laws



CoRel/Cospan under product is not monoidal.  It is only premonoidal.  Say this is an open problem.  Cite recent papers on premonoidal categories
\end{lemma}


The following category of relations is very important for this thesis:


\begin{definition}
Given a field $k$, the $\dag$-compact closed prop of {\bf linear relations} over $k$, $\LinRel_{k}$ is defined to be $\Rel(\Mat_k)$ with respect to the direct sum.

Explicity, $\LinRel_{k}$ has:

\begin{description}
\item[Objects:] Natural numbers.

\item[Maps:] A linear relation $n\to m$ is a linear subspace of $k^n \oplus k^m$.

\item[Composition:] Relational composition, so that for $R \subseteq k^n \oplus k^m$  and $S \subseteq k^m \oplus k^\ell$:
$$
R;S := \{  (x,z) \in k^{n} \oplus k^{\ell} : \exists y \in k^{m}, (x,y) \in R \wedge (y,z) \in S \} \subseteq k^n \oplus k^\ell
$$ 

\item[Tensor product:] Direct sum, so that for $R \subseteq k^n \oplus k^m$ and $S \subseteq k^\ell \oplus k^q$:

$$R\oplus S : =
\left\{
\left(
\begin{pmatrix}
a_1\\a_2
\end{pmatrix},
\begin{pmatrix}
b_1\\b_2
\end{pmatrix}
:
\forall (a_1,b_1) \in R, (a_2,b_2) \in S
\right)
\right\} \subseteq k^{n+\ell}\oplus k^{m+q}
$$

\item[Dagger:] Relational converse, so that for $R \subseteq k^{n}\oplus k^m$:

$$
R^T := \{ (b,a) : \forall (a,b) \in R \} \subseteq k^{m} \oplus k^n
$$
\end{description}
\end{definition}



\begin{lemma}[\cite{ihpub}]
Given a field $k$, $\LinRel_{k}$ is generated by the generators and equations of the presentation of $\Mat_k$ as well as those of $\Mat_k^{\op}$ (drawn as the vertically flipped generators of $\Mat_k$) modulo the equations for all $a \in k$, $a\neq 0$:

$$
\begin{tikzpicture}
	\begin{pgfonlayer}{nodelayer}
		\node [style=none] (0) at (1.5, -0.5) {};
		\node [style=none] (1) at (0.5, -0.5) {};
		\node [style=none] (2) at (1, -0.5) {$\cdots$};
		\node [style=none] (3) at (0.5, -2.75) {};
		\node [style=Z] (4) at (1, -1.25) {};
		\node [style=none] (5) at (2, -0.5) {};
		\node [style=none] (6) at (1.5, -2.75) {$\cdots$};
		\node [style=none] (7) at (1, -2.75) {};
		\node [style=Z] (8) at (1.5, -2) {};
		\node [style=none] (9) at (2, -2.75) {};
		\node [style=none] (10) at (1.25, -1.5) {\reflectbox{$\ddots$}};
	\end{pgfonlayer}
	\begin{pgfonlayer}{edgelayer}
		\draw [in=-124, out=90] (3.center) to (4);
		\draw [in=-90, out=56] (4) to (0.center);
		\draw [in=124, out=-90] (1.center) to (4);
		\draw [in=-124, out=90] (7.center) to (8);
		\draw [in=90, out=-56] (8) to (9.center);
		\draw [in=-90, out=56] (8) to (5.center);
		\draw [bend left=45, looseness=1.25] (8) to (4);
		\draw [bend left=45, looseness=1.25] (4) to (8);
	\end{pgfonlayer}
\end{tikzpicture}
=
\begin{tikzpicture}
	\begin{pgfonlayer}{nodelayer}
		\node [style=none] (11) at (4, -0.5) {};
		\node [style=none] (12) at (3, -0.5) {};
		\node [style=none] (13) at (3.5, -0.5) {$\cdots$};
		\node [style=none] (14) at (2.5, -2) {};
		\node [style=Z] (15) at (3.5, -1.25) {};
		\node [style=none] (16) at (4.5, -0.5) {};
		\node [style=none] (17) at (3.5, -2) {$\cdots$};
		\node [style=none] (18) at (3, -2) {};
		\node [style=Z] (19) at (3.5, -1.25) {};
		\node [style=none] (20) at (4, -2) {};
	\end{pgfonlayer}
	\begin{pgfonlayer}{edgelayer}
		\draw [in=-150, out=90] (14.center) to (15);
		\draw [in=-90, out=56] (15) to (11.center);
		\draw [in=124, out=-90] (12.center) to (15);
		\draw [in=-124, out=90] (18.center) to (19);
		\draw [in=90, out=-56] (19) to (20.center);
		\draw [in=-90, out=30] (19) to (16.center);
	\end{pgfonlayer}
\end{tikzpicture},
\hspace*{1cm}
\begin{tikzpicture}
	\begin{pgfonlayer}{nodelayer}
		\node [style=none] (0) at (1.5, -0.5) {};
		\node [style=none] (1) at (0.5, -0.5) {};
		\node [style=none] (2) at (1, -0.5) {$\cdots$};
		\node [style=none] (3) at (0.5, -2.75) {};
		\node [style=X] (4) at (1, -1.25) {};
		\node [style=none] (5) at (2, -0.5) {};
		\node [style=none] (6) at (1.5, -2.75) {$\cdots$};
		\node [style=none] (7) at (1, -2.75) {};
		\node [style=X] (8) at (1.5, -2) {};
		\node [style=none] (9) at (2, -2.75) {};
		\node [style=none] (10) at (1.25, -1.5) {\reflectbox{$\ddots$}};
	\end{pgfonlayer}
	\begin{pgfonlayer}{edgelayer}
		\draw [in=-124, out=90] (3.center) to (4);
		\draw [in=-90, out=56] (4) to (0.center);
		\draw [in=124, out=-90] (1.center) to (4);
		\draw [in=-124, out=90] (7.center) to (8);
		\draw [in=90, out=-56] (8) to (9.center);
		\draw [in=-90, out=56] (8) to (5.center);
		\draw [bend left=45, looseness=1.25] (8) to (4);
		\draw [bend left=45, looseness=1.25] (4) to (8);
	\end{pgfonlayer}
\end{tikzpicture}
=
\begin{tikzpicture}
	\begin{pgfonlayer}{nodelayer}
		\node [style=none] (11) at (4, -0.5) {};
		\node [style=none] (12) at (3, -0.5) {};
		\node [style=none] (13) at (3.5, -0.5) {$\cdots$};
		\node [style=none] (14) at (2.5, -2) {};
		\node [style=X] (15) at (3.5, -1.25) {};
		\node [style=none] (16) at (4.5, -0.5) {};
		\node [style=none] (17) at (3.5, -2) {$\cdots$};
		\node [style=none] (18) at (3, -2) {};
		\node [style=X] (19) at (3.5, -1.25) {};
		\node [style=none] (20) at (4, -2) {};
	\end{pgfonlayer}
	\begin{pgfonlayer}{edgelayer}
		\draw [in=-150, out=90] (14.center) to (15);
		\draw [in=-90, out=56] (15) to (11.center);
		\draw [in=124, out=-90] (12.center) to (15);
		\draw [in=-124, out=90] (18.center) to (19);
		\draw [in=90, out=-56] (19) to (20.center);
		\draw [in=-90, out=30] (19) to (16.center);
	\end{pgfonlayer}
\end{tikzpicture},
\hspace*{1cm}
\begin{tikzpicture}
	\begin{pgfonlayer}{nodelayer}
		\node [style=Z] (0) at (3.75, -1) {};
	\end{pgfonlayer}
\end{tikzpicture}
=
\begin{tikzpicture}
	\begin{pgfonlayer}{nodelayer}
		\node [style=X] (0) at (3.75, -1) {};
	\end{pgfonlayer}
\end{tikzpicture}
=
\begin{tikzpicture}
	\begin{pgfonlayer}{nodelayer}
		\node [style=none] (0) at (2, 0) {};
		\node [style=none] (1) at (2, -1) {};
		\node [style=none] (2) at (3, -1) {};
		\node [style=none] (3) at (3, 0) {};
	\end{pgfonlayer}
	\begin{pgfonlayer}{edgelayer}
		\draw[style=dashed] (3.center) to (0.center);
		\draw[style=dashed] (0.center) to (1.center);
		\draw[style=dashed] (1.center) to (2.center);
		\draw[style=dashed] (2.center) to (3.center);
	\end{pgfonlayer}
\end{tikzpicture},
\hspace*{1cm}
\begin{tikzpicture}
	\begin{pgfonlayer}{nodelayer}
		\node [style=none] (3) at (17, 1.5) {};
		\node [style=none] (4) at (17, -0.75) {};
		\node [style=scalarop] (5) at (17, 0.75) {$a$};
		\node [style=scalar] (6) at (17, 0) {$a$};
	\end{pgfonlayer}
	\begin{pgfonlayer}{edgelayer}
		\draw (4.center) to (6);
		\draw (6) to (5);
		\draw (5) to (3.center);
	\end{pgfonlayer}
\end{tikzpicture}
=
\begin{tikzpicture}
	\begin{pgfonlayer}{nodelayer}
		\node [style=none] (3) at (17, 1.5) {};
		\node [style=none] (4) at (17, -0.75) {};
		\node [style=scalar] (5) at (17, 0.75) {$a$};
		\node [style=scalarop] (6) at (17, 0) {$a$};
	\end{pgfonlayer}
	\begin{pgfonlayer}{edgelayer}
		\draw (4.center) to (6);
		\draw (6) to (5);
		\draw (5) to (3.center);
	\end{pgfonlayer}
\end{tikzpicture}
=
\begin{tikzpicture}
	\begin{pgfonlayer}{nodelayer}
		\node [style=none] (3) at (17, 1.5) {};
		\node [style=none] (4) at (17, -0.75) {};
	\end{pgfonlayer}
	\begin{pgfonlayer}{edgelayer}
		\draw (4.center) to (3.center);
	\end{pgfonlayer}
\end{tikzpicture}
$$
\end{lemma}

We can also capture affine relations as a prop, by regarding the empty relation as a subobject:


\begin{definition}
The prop of affine relations over $k$, $\Aff\Rel_{k}$ is the full subcategory of $\Rel(\Aff\Mat_k)$ of nonempty affine subspaces. Concretely, this is constructed in the same way as $\LinRel_k$ except a map $n\to m$ is instead a (possibly empty) affine subspace of $k^n\oplus k^m$.
\end{definition}


\begin{lemma}[\cite{affine}]
$\Aff\Rel_{\F_p}$ is presented by the prop $\aih_k$ given by $\ih_k$ in addition to the following generator:

$$
\begin{tikzpicture}
	\begin{pgfonlayer}{nodelayer}
		\node [style=none] (0) at (4, -0.5) {};
		\node [style=none] (1) at (3, -0.5) {};
		\node [style=none] (2) at (3.5, -0.5) {$\cdots$};
		\node [style=X] (4) at (3.5, -1.25) {};
		\node [style=none] (6) at (3.5, -2) {$\cdots$};
		\node [style=none] (7) at (3, -2) {};
		\node [style=none] (8) at (3.5, -1.25) {};
		\node [style=none] (9) at (4, -2) {};
	\end{pgfonlayer}
	\begin{pgfonlayer}{edgelayer}
		\draw [in=-90, out=56] (4) to (0.center);
		\draw [in=124, out=-90] (1.center) to (4);
		\draw [in=-124, out=90] (7.center) to (8.center);
		\draw [in=90, out=-56] (8.center) to (9.center);
	\end{pgfonlayer}
\end{tikzpicture}
:=
\begin{tikzpicture}
	\begin{pgfonlayer}{nodelayer}
		\node [style=none] (0) at (4, -0.5) {};
		\node [style=none] (1) at (3, -0.5) {};
		\node [style=none] (2) at (3.5, -0.5) {$\cdots$};
		\node [style=X] (4) at (3.5, -1.25) {$0$};
		\node [style=none] (6) at (3.5, -2) {$\cdots$};
		\node [style=none] (7) at (3, -2) {};
		\node [style=none] (8) at (3.5, -1.25) {};
		\node [style=none] (9) at (4, -2) {};
	\end{pgfonlayer}
	\begin{pgfonlayer}{edgelayer}
		\draw [in=-90, out=56] (4) to (0.center);
		\draw [in=124, out=-90] (1.center) to (4);
		\draw [in=-124, out=90] (7.center) to (8.center);
		\draw [in=90, out=-56] (8.center) to (9.center);
	\end{pgfonlayer}
\end{tikzpicture},
\hspace*{1cm}
\begin{tikzpicture}
	\begin{pgfonlayer}{nodelayer}
		\node [style=none] (0) at (1.5, -0.5) {};
		\node [style=none] (1) at (0.5, -0.5) {};
		\node [style=none] (2) at (1, -0.5) {$\cdots$};
		\node [style=none] (3) at (0.5, -2.75) {};
		\node [style=X] (4) at (1, -1.25) {$a$};
		\node [style=none] (5) at (2, -0.5) {};
		\node [style=none] (6) at (1.5, -2.75) {$\cdots$};
		\node [style=none] (7) at (1, -2.75) {};
		\node [style=X] (8) at (1.5, -2) {$b$};
		\node [style=none] (9) at (2, -2.75) {};
		\node [style=none] (10) at (1.25, -1.5) {\reflectbox{$\ddots$}};
	\end{pgfonlayer}
	\begin{pgfonlayer}{edgelayer}
		\draw [in=-124, out=90] (3.center) to (4);
		\draw [in=-90, out=56] (4) to (0.center);
		\draw [in=124, out=-90] (1.center) to (4);
		\draw [in=-124, out=90] (7.center) to (8);
		\draw [in=90, out=-56] (8) to (9.center);
		\draw [in=-90, out=56] (8) to (5.center);
		\draw [bend left=45, looseness=1.25] (8) to (4);
		\draw [bend left=45, looseness=1.25] (4) to (8);
	\end{pgfonlayer}
\end{tikzpicture}
=
\begin{tikzpicture}
	\begin{pgfonlayer}{nodelayer}
		\node [style=none] (11) at (4, -0.5) {};
		\node [style=none] (12) at (3, -0.5) {};
		\node [style=none] (13) at (3.5, -0.5) {$\cdots$};
		\node [style=none] (14) at (2.5, -2) {};
		\node [style=X] (15) at (3.5, -1.25) {$a+b$};
		\node [style=none] (16) at (4.5, -0.5) {};
		\node [style=none] (17) at (3.5, -2) {$\cdots$};
		\node [style=none] (18) at (3, -2) {};
		\node [style=none] (19) at (3.5, -1.25) {};
		\node [style=none] (20) at (4, -2) {};
	\end{pgfonlayer}
	\begin{pgfonlayer}{edgelayer}
		\draw [in=-150, out=90] (14.center) to (15);
		\draw [in=-90, out=56] (15) to (11.center);
		\draw [in=124, out=-90] (12.center) to (15);
		\draw [in=-124, out=90] (18.center) to (19);
		\draw [in=90, out=-56] (19) to (20.center);
		\draw [in=-90, out=30] (19) to (16.center);
	\end{pgfonlayer}
\end{tikzpicture},
\hspace*{1cm}
\begin{tikzpicture}
	\begin{pgfonlayer}{nodelayer}
		\node [style=X] (0) at (3.5, -1.25) {$1$};
		\node [style=none] (1) at (4, -0.5) {};
		\node [style=none] (2) at (4, -2) {};
	\end{pgfonlayer}
	\begin{pgfonlayer}{edgelayer}
		\draw (2.center) to (1.center);
	\end{pgfonlayer}
\end{tikzpicture}
=
\begin{tikzpicture}
	\begin{pgfonlayer}{nodelayer}
		\node [style=X] (4) at (3.5, -1.25) {$1$};
		\node [style=none] (9) at (4, -0.5) {};
		\node [style=none] (10) at (4, -2) {};
		\node [style=Z] (11) at (4, -1.5) {};
		\node [style=X] (12) at (4, -1) {};
	\end{pgfonlayer}
	\begin{pgfonlayer}{edgelayer}
		\draw (10.center) to (11);
		\draw (12) to (9.center);
	\end{pgfonlayer}
\end{tikzpicture}
$$


\end{lemma}


Categories of relations regarded as monoidal categories under the cartesian product have a succinct algebraic generalization.


\begin{definition}
A {\bf cartesian bicategory of relations} is a monoidal category $\X$ enriched in posests,  equipped with a lax natural commutative comonoid structure $(\Delta, !)$ so that the comultiplication and counit both have right adjoints $(\Delta^*,!^*)$.

We also require that the monoid and comonoid structure interact to form a special Frobenius algebra.


The category of comonoid homorphisms of a Cartesian bicategory of relations is Cartesian category $\Map(\X)$.

Note that the underlying monoidal category of a Cartesian bicategory of relations is a $\dag$-compact closed category due to each object being equipped with a Frobenius algebra.
\end{definition}

Span is not a cartesian bicategory of relations because it is poset enriched.  One could have given a similar axiomatization for non-poset enriched categories, as done in \cite{?}, however this is much more difficult to work with because this notion requires coherence conditions.

\begin{example}
$\Rel(\X)$ is a cartesian bicategory of relations under the Cartesian product and $\Map(\Rel(\X))=\X$ is Cartesian.
\end{example}


The Frobenius algebra structure of a Cartesian bicategory of relations can be interpreted as a nondeterministic copying/comparison structure. However, there are classes of categories in between Cartesian categories and Cartesian bicategories of relations which capture partially invertible (like $\Pinj$) and partial notions of copying (like $\Par$).  In this section, we review these notions and give examples which will serve to motivate their usage in quantum computing later in this thesis.



Partial maps of sets are spans with left leg monic. Give span diagram with domain and function


\label{sec:rest}




%Restriction and inverse categories provide a categorical semantics for partial computing and reversible computing, respectively.  We review how weakened products can be constructed in both settings; relating one to the other.

\begin{definition}\cite[\S 2.1.1]{cockett}
A {\bf restriction category} is a category along with a restriction operator:

\hfil
$
(A \xrightarrow{f} B )\mapsto (A \xrightarrow{\bar f} A)
$\\
such that:

\begin{multicols}{4}
\begin{enumerate}[label={\bf [R.\arabic*]}, ref={\bf [R.\arabic*]}]
\item $\bar f f  = f$
\label{R.1}
\item $\bar f \bar g = \bar g \bar f$
\label{R.2}
\item $\bar f \bar g = \bar{\bar f g}$
\label{R.3}
\item $f \bar g = \bar{fg} f$
\label{R.4}
\end{enumerate}
\end{multicols}

Maps of the form $\bar f$ are called restriction idempotents.
The canonical example of a restriction category is $\Par$, sets and partial maps.  The restriction in this case, just restricts partial functions to their domain of definition.


Restriction categories are poset enriched where $f \leq g \iff \bar f g = f$.


A map $f$ in a restriction category is called a {\bf partial isomorphism}, in case there exists a map $g$ called the partial inverse of $f$ so that $fg=\bar f$ and $gf = \bar g$.  Similarly, a map $f$ in a restriction category is {\bf total} if $\bar f =1$.  Denote the subcategories of partial isomorphisms and total maps of a restriction category $\X$, respectively by $\ParIso(\X)$ and $\Total(\X)$.



%A {\bf split restriction category} is a restriction category in which all restriction idempotents split.
\end{definition}



\begin{example} \cite[p. 101]{pcat} \cite[\S 5]{restiii}
A {\bf counital copy category} (or a p-category with a one element object) is a monoidal category with a family of commutative comonoids on every object compatible with the monoidal structure, with a natural comultiplication.  This gives a restriction via copying and then discarding:
$$
\begin{tikzpicture}
	\begin{pgfonlayer}{nodelayer}
		\node [style=none] (0) at (0.75, -2.5) {};
		\node [style=none] (1) at (0.75, -0.5) {};
		\node [style=map] (2) at (0.75, -1.5) {$\bar f$};
	\end{pgfonlayer}
	\begin{pgfonlayer}{edgelayer}
		\draw [style=simple] (0.center) to (2);
		\draw [style=simple] (2) to (1.center);
	\end{pgfonlayer}
\end{tikzpicture}
:=
\begin{tikzpicture}
	\begin{pgfonlayer}{nodelayer}
		\node [style=map] (0) at (0, 2.5) {$f$};
		\node [style=Z] (1) at (0, 3.5) {};
		\node [style=Z] (2) at (0.5, 1.5) {};
		\node [style=none] (3) at (1, 3.5) {};
		\node [style=none] (4) at (0.5, 0.5) {};
	\end{pgfonlayer}
	\begin{pgfonlayer}{edgelayer}
		\draw [style=simple] (1) to (0);
		\draw [style=simple, in=117, out=-90] (0) to (2);
		\draw [style=simple] (2) to (4.center);
		\draw [style=simple, in=-90, out=60] (2) to (3.center);
	\end{pgfonlayer}
\end{tikzpicture}
$$
\end{example}


\begin{definition}\cite[\S 3.1]{cockett}
A {\bf stable system of monics} $\M$ of $\X$ is a collection of monics in $\X$ containing all isomorphisms; where for any cospan $ X\xrightarrow{f} Z \xleftarrowtail{m} Y$  in $\X$, where $m'$ is in $\M$, the following pullback exists:

%\hfil$
%\xymatrixrowsep{.005in}
%\xymatrixcolsep{.13in}
%  \xymatrix{
%    W \ar[r]^{f'} \ar@{>->}[d]_{m'} & Y  \ar@{>->}[d]^m \\
%    X \ar[r]_{f} & Z
%  }
%$\\

$$
\xymatrixrowsep{.005in}
\xymatrixcolsep{.13in}
  \xymatrix{
  	& W \ar@{>->}[dl]_{m'} \ar[dr]^{f'}\\
  	X \ar[dr]_f &  & Y \ar@{>->}[dl]^m\\
  	& Z
  }
$$

Where $m'$ is in $\M$.

\end{definition}

Stable systems of monics allow one to represent the domains of definition of a partial functions as a subobjects:

\begin{definition}\cite[\S 3.1]{cockett}
Given a stable system of monics $\M$ in a category $\X$, the {\bf partial map category} $\Par(\X,\M)$ is given by the same objects as in $\X$ where morphisms $X\to Y$, given by isomorphism classes of spans $X\xleftarrowtail{m} Z \xrightarrow{f} Y$ where $f$ is a map in $\X$ and $m$ is a map in $\M$.  Composition is given by pullback and the identity is given by the trivial span.


Partial map categories have a restriction structure given by:  $(X\xleftarrowtail{m} Z \xrightarrow{f} Y) \mapsto (X\xleftarrowtail{m} Z \xrightarrowtail{m} X)$.  Moreover, a partial isomorphism is a span $X\xleftarrowtail{e} Z \xrightarrowtail{m} Y$ where $e,m \in \M$; the partial inverse given by  $Y\xleftarrowtail{m} Z \xrightarrowtail{e} X$.
\end{definition}


$\Par$ is equivalently the partial map category $\Par(\Sets,\M)$ where $\M$ is all monics in $\Sets$.



In the case of restriction categories, one must weaken the notion of the product to lax products using the partial order enrichment:


\begin{definition}\cite{restiii}
A restriction category has {\bf binary restriction products}, when for all objects  $X,Y$, there exists an object $X\times Y$ and total maps $X \xleftarrow{\pi_0}  X\times Y \xrightarrow{\pi_1} Y$, so that for all objects $Z$ and all maps $X \xleftarrow{f} Z \xrightarrow{g} Y$, there exists a unique $Z\xrightarrow{\langle f,g \rangle} X\times Y$ making the diagram commute:
$$
\xymatrixrowsep{0.2cm}
\xymatrixcolsep{0.4cm}
\xymatrix{
&& Z\ar@{..>}[dd]|-{\langle f, g\rangle} \ar@/_/[ddll]_f \ar@/^/[ddrr]^g &&\\
& \ar@{}[dr]|-{\geq} && \ar@{}[dl] |-{\leq} &\\
X &&  X\times Y \ar[rr]_{\pi_1} \ar[ll]^{\pi_0}  && Y
}
$$

so that $\bar{\langle f, g\rangle \pi_0} f = \langle f, g\rangle \pi_0$ and $\bar{\langle f, g\rangle \pi_1} g = \langle f, g\rangle \pi_1$;
where additionally $\bar{\langle f, g\rangle} =  \bar f \bar g$.

%%DRAW DIAGRAM
%\begin{center}
%\begin{tabular}{ccc}
%  $\langle f, g\rangle \pi_0 \leq f$ &
%  $\langle f, g\rangle \pi_1 \leq g$ &
%  $\bar{\langle f, g\rangle} =  \bar f \bar g$
%\end{tabular}
%\end{center}

A restriction category has a {\bf restriction terminal object} $\top$ when for all objects $X$, there exists a unique total map $!_X:X\to\top$ such that $f !_Y = \bar f !_X$.

A restriction category with a restriction terminal object and binary restriction products is a {\bf Cartesian restriction category}.


An object $A$ in a restriction category with restriction products is {\bf discrete} when the diagonal map $\Delta_X:=\langle 1_X, 1_X\rangle$ is a partial isomorphism. A restriction category is discrete when all objects are discrete. 
\end{definition}



$\Par$ is a canonical example of a Cartesian restriction category; the restriction product is given by the Cartesian product on underlying sets and the terminal object is  the singleton set. In fact it is also discrete, because the converse of the diagonal relation is a partial function.




\begin{theorem}\cite[Thm. 5.2]{restiii}
Counital copy categories are in bijection with Cartesian restriction categories.
\end{theorem}


\begin{proposition} 
\label{prop:cartesian}
If $\X$ is a  Cartesian restriction category, then $\Total(\X)$ is Cartesian.
\end{proposition}

Discrete cartesian restriction categories are a deterministic, but still partial version of relations:

\begin{lemma}
The category of comultiplication homorphisms of a Cartesian bicategory of relations is a discrete Cartesian restriction category.
\end{lemma}

\begin{definition}\cite[\S 2.3.2]{cockett}
An {\bf inverse category} is a restriction category in which all maps are partial isomorphisms.
\end{definition}

\begin{example}
The subcategory of partial isomorphisms of $\Par$ is  $\Pinj$.
\end{example}

Inverse categories are particular kinds of \dag-categories:

\begin{theorem}\cite[Thm. 2.20]{cockett}
A restriction category $\X$ is an inverse category if and only if there is a dagger functor $(\_)^\circ:\X^\op\to\X$ such that for all $X\xleftarrow{f} Z \xrightarrow{g} Y$:
\begin{center}
\begin{tabular}{cc}
 $f f^\circ f = f$ & 
 $f f ^\circ gg^\circ = gg^\circ f f ^\circ $
\end{tabular}
\end{center}
\end{theorem}

The unitary maps in an inverse category are the total maps.


Sets and partial injections are intimately related to Hilbert spaces

\begin{definition}
There is a \dag-symmetric monoidal embedding $\ell^2:\Pinj\to\Hilb$ taking:

\begin{description}
\item[Objects:] Sets $X$ are taken to the Hilbert space of square-integrable functions on $X$:

$$
\ell^2(X):=\left\{\phi:X\to \C  : \sum_{x\in X} \left|\phi(x)\right|^2 < \infty \right\}
$$

\item[Maps:] Given a partial injection $f=X \xleftarrowtail{f_0} A \xrightarrowtail{f_1} Z$ and some $\phi:X\to \C$ in $\ell^2(X)$

$$
(\ell^2(f)(\phi))(y) = \sum_{x\in f_1^{-1}(y)} \phi(f_0(x))
$$ 

\end{description}

The partial inverse is mapped to the adjoint
\end{definition}


Partial injections is special in the sense that there is an embedding into $\ell^2$. This does not generalize to sets and functions.  However, it can be modified to yield a functor from finite sets and functions/finites sets and spans into $\FHilb$, where all maps are bounded.  We will extensively use such embeddings thoughout this thesis.


The weakened notion of products in restriction categories is the right generalization of products for inverse categories because it does not impose enough equations governing the interaction between the diagonal map and its partial inverse.  In order to mix the dagger structure of an inverse category with symmetric monoidal structure, we first need the following definition:



\begin{definition}\cite[Def. 4.3.1]{giles}
A symmetric monoidal inverse category $\X$ is a {\bf discrete inverse category} when it is a dagger symmetric monoidal  category, 
equipped with a commutative semigroup and cocommutative cosemigroup on every object  compatible with the tensor product:

$$
\begin{tikzpicture}
	\begin{pgfonlayer}{nodelayer}
		\node [style=none] (0) at (0, 2.5) {};
		\node [style=none] (1) at (1, 2.5) {};
		\node [style=Z] (2) at (0.5, 1.5) {};
		\node [style=none] (3) at (0.5, 0.5) {};
	\end{pgfonlayer}
	\begin{pgfonlayer}{edgelayer}
		\draw [style=simple] (3.center) to (2);
		\draw [style=simple, in=-90, out=117] (2) to (0.center);
		\draw [style=simple, in=63, out=-90] (1.center) to (2);
	\end{pgfonlayer}
\end{tikzpicture}
=
\begin{tikzpicture}
	\begin{pgfonlayer}{nodelayer}
		\node [style=Z] (0) at (0, 2.5) {};
		\node [style=Z] (1) at (1, 2.5) {};
		\node [style=none] (2) at (0.5, 1.5) {};
		\node [style=none] (3) at (0.5, 0.5) {};
		\node [style=none] (4) at (0, 3.5) {};
		\node [style=none] (5) at (1, 3.5) {};
		\node [style=none] (6) at (0, 4.5) {};
		\node [style=none] (7) at (1, 4.5) {};
		\node [style=otimes] (8) at (0.5, 1.5) {};
		\node [style=otimes] (9) at (1, 3.5) {};
		\node [style=otimes] (10) at (0, 3.5) {};
	\end{pgfonlayer}
	\begin{pgfonlayer}{edgelayer}
		\draw [style=simple] (3.center) to (2.center);
		\draw [style=simple, in=-90, out=135] (2.center) to (0);
		\draw [style=simple] (0) to (5.center);
		\draw [style=simple, in=120, out=-120, looseness=1.25] (4.center) to (0);
		\draw [style=simple, in=-60, out=60, looseness=1.25] (1) to (5.center);
		\draw [style=simple] (1) to (4.center);
		\draw [style=simple, in=45, out=-90] (1) to (2.center);
		\draw [style=simple] (4.center) to (6.center);
		\draw [style=simple] (5.center) to (7.center);
	\end{pgfonlayer}
\end{tikzpicture}
$$

Where the semigroup and cosemigroup are daggers of each other and interact interact to form a (co)unitless special commutative Frobenius algebra.

GIVE STRING DIAGRAMS


Where the diagonal is natural so that for arbitrary $f$:
$$
\begin{tikzpicture}
	\begin{pgfonlayer}{nodelayer}
		\node [style=Z] (0) at (0, 12) {};
		\node [style=map] (1) at (0, 11.25) {$f$};
		\node [style=none] (2) at (-0.5, 12.75) {};
		\node [style=none] (3) at (0.5, 12.75) {};
		\node [style=none] (4) at (0, 10.5) {};
	\end{pgfonlayer}
	\begin{pgfonlayer}{edgelayer}
		\draw (4.center) to (1);
		\draw (1) to (0);
		\draw [in=-90, out=150] (0) to (2.center);
		\draw [in=-90, out=30] (0) to (3.center);
	\end{pgfonlayer}
\end{tikzpicture}
=
\begin{tikzpicture}
	\begin{pgfonlayer}{nodelayer}
		\node [style=Z] (5) at (2, 11.25) {};
		\node [style=none] (7) at (1.5, 12) {};
		\node [style=none] (8) at (2.5, 12) {};
		\node [style=none] (9) at (2, 10.5) {};
		\node [style=map] (10) at (2.5, 12) {$f$};
		\node [style=map] (11) at (1.5, 12) {$f$};
		\node [style=none] (12) at (1.5, 12.75) {};
		\node [style=none] (13) at (2.5, 12.75) {};
	\end{pgfonlayer}
	\begin{pgfonlayer}{edgelayer}
		\draw [in=-90, out=150] (5) to (7.center);
		\draw [in=-90, out=30] (5) to (8.center);
		\draw (9.center) to (5);
		\draw (11) to (12.center);
		\draw (10) to (13.center);
	\end{pgfonlayer}
\end{tikzpicture}
$$
\end{definition}

In a discrete inverse category, restriction idempotents are strengths for the multiplication and comultiplication so that:
$$
\begin{tikzpicture}
	\begin{pgfonlayer}{nodelayer}
		\node [style=Z] (0) at (3, 1.75) {};
		\node [style=map] (1) at (3, 1) {$\bar f$};
		\node [style=none] (2) at (3, 0.5) {};
		\node [style=none] (3) at (2.5, 2.5) {};
		\node [style=none] (4) at (3.5, 2.5) {};
	\end{pgfonlayer}
	\begin{pgfonlayer}{edgelayer}
		\draw [style=simple, in=63, out=-90] (4.center) to (0);
		\draw [style=simple, in=-90, out=117] (0) to (3.center);
		\draw [style=simple] (1) to (0);
		\draw [style=simple] (1) to (2.center);
	\end{pgfonlayer}
\end{tikzpicture}
=
\begin{tikzpicture}
	\begin{pgfonlayer}{nodelayer}
		\node [style=Z] (0) at (3, 2) {};
		\node [style=none] (1) at (3, 1.5) {};
		\node [style=none] (2) at (2.5, 3) {};
		\node [style=none] (3) at (3.5, 3) {};
		\node [style=map] (4) at (2.5, 3) {$\bar f$};
		\node [style=none] (5) at (3.5, 3.5) {};
		\node [style=none] (6) at (2.5, 3.5) {};
	\end{pgfonlayer}
	\begin{pgfonlayer}{edgelayer}
		\draw [style=simple, in=63, out=-90] (3.center) to (0);
		\draw [style=simple, in=-90, out=117] (0) to (2.center);
		\draw [style=simple] (6.center) to (2.center);
		\draw [style=simple] (5.center) to (3.center);
		\draw [style=simple] (0) to (1.center);
	\end{pgfonlayer}
\end{tikzpicture}
=
\begin{tikzpicture}
	\begin{pgfonlayer}{nodelayer}
		\node [style=Z] (0) at (3, 2) {};
		\node [style=none] (1) at (3, 1.5) {};
		\node [style=none] (2) at (3.5, 3) {};
		\node [style=none] (3) at (2.5, 3) {};
		\node [style=map] (4) at (3.5, 3) {$\bar f$};
		\node [style=none] (5) at (2.5, 3.5) {};
		\node [style=none] (6) at (3.5, 3.5) {};
	\end{pgfonlayer}
	\begin{pgfonlayer}{edgelayer}
		\draw [style=simple, in=117, out=-90] (3.center) to (0);
		\draw [style=simple, in=-90, out=63] (0) to (2.center);
		\draw [style=simple] (6.center) to (2.center);
		\draw [style=simple] (5.center) to (3.center);
		\draw [style=simple] (0) to (1.center);
	\end{pgfonlayer}
\end{tikzpicture}
\hspace*{.6cm}
\begin{tikzpicture}
	\begin{pgfonlayer}{nodelayer}
		\node [style=Z] (0) at (3, 3) {};
		\node [style=none] (1) at (3, 3.5) {};
		\node [style=none] (2) at (3.5, 2) {};
		\node [style=none] (3) at (2.5, 2) {};
		\node [style=map] (4) at (3.5, 2) {$\bar f$};
		\node [style=none] (5) at (2.5, 1.5) {};
		\node [style=none] (6) at (3.5, 1.5) {};
	\end{pgfonlayer}
	\begin{pgfonlayer}{edgelayer}
		\draw [style=simple, in=-117, out=90] (3.center) to (0);
		\draw [style=simple, in=90, out=-63] (0) to (2.center);
		\draw [style=simple] (6.center) to (2.center);
		\draw [style=simple] (5.center) to (3.center);
		\draw [style=simple] (0) to (1.center);
	\end{pgfonlayer}
\end{tikzpicture}
=
\begin{tikzpicture}
	\begin{pgfonlayer}{nodelayer}
		\node [style=Z] (0) at (3, 3) {};
		\node [style=none] (1) at (3, 3.5) {};
		\node [style=none] (2) at (2.5, 2) {};
		\node [style=none] (3) at (3.5, 2) {};
		\node [style=map] (4) at (2.5, 2) {$\bar f$};
		\node [style=none] (5) at (3.5, 1.5) {};
		\node [style=none] (6) at (2.5, 1.5) {};
	\end{pgfonlayer}
	\begin{pgfonlayer}{edgelayer}
		\draw [style=simple, in=-63, out=90] (3.center) to (0);
		\draw [style=simple, in=90, out=-117] (0) to (2.center);
		\draw [style=simple] (6.center) to (2.center);
		\draw [style=simple] (5.center) to (3.center);
		\draw [style=simple] (0) to (1.center);
	\end{pgfonlayer}
\end{tikzpicture}
=
\begin{tikzpicture}
	\begin{pgfonlayer}{nodelayer}
		\node [style=Z] (0) at (3, 1.25) {};
		\node [style=map] (1) at (3, 2) {$\bar f$};
		\node [style=none] (2) at (3, 2.5) {};
		\node [style=none] (3) at (2.5, 0.5) {};
		\node [style=none] (4) at (3.5, 0.5) {};
	\end{pgfonlayer}
	\begin{pgfonlayer}{edgelayer}
		\draw [style=simple, in=-63, out=90] (4.center) to (0);
		\draw [style=simple, in=90, out=-117] (0) to (3.center);
		\draw [style=simple] (1) to (0);
		\draw [style=simple] (1) to (2.center);
	\end{pgfonlayer}
\end{tikzpicture}
$$



Discrete inverse categories are the canonical notion of weakened products for monoidal inverse categories:

\begin{lemma}
The subcategory of semi-Frobenius algebra homorphisms of a Cartesian bicategory of relations is a discrete inverse category.

Similarly the subcategory of  semi-Frobenius algebra homorphisms of a discrete Cartesian restriction category (the partial isomorphisms) is a discrete inverse category.
\end{lemma}



The other direction is less trivial; in particular getting a Cartesian restriction category from a discrete inverse category involves adding a restriction terminal object via the following construction which ``adds a history'' to a partial isomorphisms.  We will first introduce the more general copara construction, we will need this machinery later.


\begin{definition}
Given a monoidal category $\X$, the copara construction, $\CoPara(\X)$ is the monoidal category with:

\begin{description}
\item[Objects:] Same as in $\X$.

\item[Maps:]  
\hfil $
\dfrac{ X\xrightarrow{f} Y \otimes S \in \X           }
         { X\xrightarrow{(f,S)} Y \in  \CoPara(\X) }
$

\item[Composition]:  
\hfil $
\dfrac{X\xrightarrow{(f,S)} Y , \hspace*{.5cm} Y\xrightarrow{(g,T)} Z }
         {(f,S);(g;T) := (f;(g\otimes 1_S);\alpha^{-1}_{Z,S,T} ,S\otimes T) } 
$

\hfil Or in proof net notation:
\hspace*{.5cm}
$
\begin{tikzpicture}
	\begin{pgfonlayer}{nodelayer}
		\node [style=map] (0) at (0, 1.5) {$f$};
		\node [style=none] (1) at (-0.5, 2.5) {};
		\node [style=none] (2) at (0.5, 2.5) {};
		\node [style=none] (3) at (0, 0.5) {};
	\end{pgfonlayer}
	\begin{pgfonlayer}{edgelayer}
		\draw [in=117, out=-90] (1.center) to (0);
		\draw [in=-90, out=63] (0) to (2.center);
		\draw (0) to (3.center);
	\end{pgfonlayer}
\end{tikzpicture}
;
\begin{tikzpicture}
	\begin{pgfonlayer}{nodelayer}
		\node [style=map] (0) at (0, 1.5) {$g$};
		\node [style=none] (1) at (-0.5, 2.5) {};
		\node [style=none] (2) at (0.5, 2.5) {};
		\node [style=none] (3) at (0, 0.5) {};
	\end{pgfonlayer}
	\begin{pgfonlayer}{edgelayer}
		\draw [in=117, out=-90] (1.center) to (0);
		\draw [in=-90, out=63] (0) to (2.center);
		\draw (0) to (3.center);
	\end{pgfonlayer}
\end{tikzpicture}
:=
\begin{tikzpicture}
	\begin{pgfonlayer}{nodelayer}
		\node [style=map] (0) at (0, 1.5) {$f$};
		\node [style=none] (1) at (0.5, 2.5) {};
		\node [style=none] (2) at (0, 0.5) {};
		\node [style=map] (3) at (-0.5, 2.5) {$g$};
		\node [style=none] (4) at (-1, 3.5) {};
		\node [style=otimes] (5) at (0, 3.5) {};
		\node [style=none] (6) at (-0.5, 2.5) {};
		\node [style=none] (7) at (-1, 4.5) {};
		\node [style=none] (8) at (0, 4.5) {};
	\end{pgfonlayer}
	\begin{pgfonlayer}{edgelayer}
		\draw [in=-90, out=63] (0) to (1.center);
		\draw (0) to (2.center);
		\draw [in=117, out=-90] (4.center) to (3);
		\draw (3) to (5);
		\draw [in=117, out=-90] (6.center) to (0);
		\draw [in=-63, out=90] (1.center) to (5);
		\draw (5) to (8.center);
		\draw (4.center) to (7.center);
	\end{pgfonlayer}
\end{tikzpicture}
$

\item[Identity:]

$$
\dfrac{ 1_X \in  \CoPara(\X)}{(u^R_X)^{-1} \in \X}
$$
\hfil Or in proof net notation:
$$
\begin{tikzpicture}
	\begin{pgfonlayer}{nodelayer}
		\node [style=none] (71) at (12.25, -0.75) {};
		\node [style=none] (72) at (11.25, -0.75) {};
		\node [style=none] (73) at (11.75, -1.5) {};
		\node [style=none] (74) at (11.75, -2.25) {};
		\node [style=none] (75) at (11.25, -0.5) {$X$};
		\node [style=none] (76) at (11.75, -2.5) {$X\otimes I$};
		\node [style=otimes] (77) at (11.75, -1.5) {};
		\node [style=unit] (78) at (12.25, -0.75) {};
	\end{pgfonlayer}
	\begin{pgfonlayer}{edgelayer}
		\draw (74.center) to (73.center);
		\draw [in=-90, out=30] (73.center) to (71.center);
		\draw [in=-90, out=150] (73.center) to (72.center);
	\end{pgfonlayer}
\end{tikzpicture}
$$

\item[Tensor product:]\

\hspace*{-2cm}
$
\dfrac{X\xrightarrow{(f,S)} Y, \hspace*{.5cm} Z\xrightarrow{(g,T)} W}
{(f,S)\otimes (g;T) := ((f\otimes g);(1_{X\otimes S} \otimes c_{W,T});\alpha_{X,S,T\otimes W};(1_X\otimes \alpha_{S,T,W}^{-1};(c_{S,T}));\alpha_{Y,W,S\otimes T}^{-1} ,S\otimes T)} 
$

\hfil Or in proof net notation:
$
\begin{tikzpicture}
	\begin{pgfonlayer}{nodelayer}
		\node [style=map] (0) at (0, 1.5) {$f$};
		\node [style=none] (1) at (-0.5, 2.5) {};
		\node [style=none] (2) at (0.5, 2.5) {};
		\node [style=none] (3) at (0, 0.5) {};
	\end{pgfonlayer}
	\begin{pgfonlayer}{edgelayer}
		\draw [in=117, out=-90] (1.center) to (0);
		\draw [in=-90, out=63] (0) to (2.center);
		\draw (0) to (3.center);
	\end{pgfonlayer}
\end{tikzpicture}
\otimes
\begin{tikzpicture}
	\begin{pgfonlayer}{nodelayer}
		\node [style=map] (0) at (0, 1.5) {$g$};
		\node [style=none] (1) at (-0.5, 2.5) {};
		\node [style=none] (2) at (0.5, 2.5) {};
		\node [style=none] (3) at (0, 0.5) {};
	\end{pgfonlayer}
	\begin{pgfonlayer}{edgelayer}
		\draw [in=117, out=-90] (1.center) to (0);
		\draw [in=-90, out=63] (0) to (2.center);
		\draw (0) to (3.center);
	\end{pgfonlayer}
\end{tikzpicture}
:=
\begin{tikzpicture}
	\begin{pgfonlayer}{nodelayer}
		\node [style=map] (9) at (3.5, 1.5) {$f$};
		\node [style=map] (13) at (4.5, 1.5) {$g$};
		\node [style=otimes] (17) at (4.5, 2.5) {};
		\node [style=otimes] (18) at (3.5, 2.5) {};
		\node [style=otimes] (190) at (4, 0.75) {};
		\node  (19) at (4, 0.75) {};
		\node [style=none] (20) at (3.5, 3) {};
		\node [style=none] (21) at (4.5, 3) {};
		\node [style=none] (22) at (4, 0.25) {};
	\end{pgfonlayer}
	\begin{pgfonlayer}{edgelayer}
		\draw (13) to (18);
		\draw [bend right] (18) to (9);
		\draw (9) to (17);
		\draw [bend left] (17) to (13);
		\draw [in=45, out=-90] (13) to (19);
		\draw [in=-90, out=135] (19) to (9);
		\draw (21.center) to (17);
		\draw (22.center) to (19);
		\draw (18) to (20.center);
	\end{pgfonlayer}
\end{tikzpicture}
$


\item[Tensor unit:]\

$$
\dfrac{ I \in  \CoPara(\X)}{ 1_{I\otimes I}\in \X}
$$

\hfil Or in proof net notation:

$$
\begin{tikzpicture}
	\begin{pgfonlayer}{nodelayer}
		\node [style=none] (0) at (11.25, 2) {};
		\node [style=none] (1) at (12.25, 2) {};
		\node [style=none] (2) at (11.75, 1.25) {};
		\node [style=none] (3) at (11.75, 0.5) {};
		\node [style=none] (4) at (12.25, 2.25) {$I$};
		\node [style=none] (5) at (11.75, 0.25) {$I\otimes I$};
		\node [style=otimes] (6) at (11.75, 1.25) {};
		\node [style=none] (7) at (11.25, 2.25) {$I$};
	\end{pgfonlayer}
	\begin{pgfonlayer}{edgelayer}
		\draw (3.center) to (2.center);
		\draw [in=-90, out=150] (2.center) to (0.center);
		\draw [in=-90, out=30] (2.center) to (1.center);
	\end{pgfonlayer}
\end{tikzpicture}
$$

\end{description}
\end{definition}

The coherence data for the monoidal structure is inhereted in a straightforward way from $\X$. Moreover, if $\X$ is symmetric monoidal, then it is easy to see how ${\CoPara}(\X)$ is as well.

\begin{definition}\cite[Def. 5.1.1]{giles}
Given a discrete inverse category $\X$,  its {\bf Cartesian completion} $\tilde \X$ is the quotient of ${\CoPara}(\X)$ by either of the following equivalent congruence relations:
$$
\begin{tikzpicture}
	\begin{pgfonlayer}{nodelayer}
		\node [style=map] (0) at (0, 1.5) {$f$};
		\node [style=none] (1) at (0, 0.5) {};
		\node [style=map] (2) at (0, 3) {$f^\circ$};
		\node [style=map] (3) at (0, 4) {$g$};
		\node [style=Z] (4) at (-0.5, 2.25) {};
		\node [style=Z] (5) at (-0.5, 5) {};
		\node [style=none] (6) at (-0.5, 6) {};
		\node [style=none] (7) at (0.25, 6) {};
	\end{pgfonlayer}
	\begin{pgfonlayer}{edgelayer}
		\draw (0) to (1.center);
		\draw [in=75, out=-90] (7.center) to (3);
		\draw (6.center) to (5);
		\draw [in=120, out=-120] (5) to (4);
		\draw (4) to (2);
		\draw [in=60, out=-60, looseness=1.25] (2) to (0);
		\draw (0) to (4);
		\draw (3) to (2);
		\draw (3) to (5);
	\end{pgfonlayer}
\end{tikzpicture}
=
\begin{tikzpicture}
	\begin{pgfonlayer}{nodelayer}
		\node [style=map] (0) at (0, 1.5) {$g$};
		\node [style=none] (1) at (-0.5, 2.5) {};
		\node [style=none] (2) at (0.5, 2.5) {};
		\node [style=none] (3) at (0, 0.5) {};
	\end{pgfonlayer}
	\begin{pgfonlayer}{edgelayer}
		\draw [in=117, out=-90] (1.center) to (0);
		\draw [in=-90, out=63] (0) to (2.center);
		\draw (0) to (3.center);
	\end{pgfonlayer}
\end{tikzpicture}
\hspace*{.3cm}
or
\hspace*{.3cm}
\begin{tikzpicture}
	\begin{pgfonlayer}{nodelayer}
		\node [style=map] (0) at (0, 1.5) {$g$};
		\node [style=none] (1) at (0, 0.5) {};
		\node [style=map] (2) at (0, 3) {$g^\circ$};
		\node [style=map] (3) at (0, 4) {$f$};
		\node [style=Z] (4) at (-0.5, 2.25) {};
		\node [style=Z] (5) at (-0.5, 5) {};
		\node [style=none] (6) at (-0.5, 6) {};
		\node [style=none] (7) at (0.25, 6) {};
	\end{pgfonlayer}
	\begin{pgfonlayer}{edgelayer}
		\draw (0) to (1.center);
		\draw [in=75, out=-90] (7.center) to (3);
		\draw (6.center) to (5);
		\draw [in=120, out=-120] (5) to (4);
		\draw (4) to (2);
		\draw [in=60, out=-60, looseness=1.25] (2) to (0);
		\draw (0) to (4);
		\draw (3) to (2);
		\draw (3) to (5);
	\end{pgfonlayer}
\end{tikzpicture}
=
\begin{tikzpicture}
	\begin{pgfonlayer}{nodelayer}
		\node [style=map] (0) at (0, 1.5) {$f$};
		\node [style=none] (1) at (-0.5, 2.5) {};
		\node [style=none] (2) at (0.5, 2.5) {};
		\node [style=none] (3) at (0, 0.5) {};
	\end{pgfonlayer}
	\begin{pgfonlayer}{edgelayer}
		\draw [in=117, out=-90] (1.center) to (0);
		\draw [in=-90, out=63] (0) to (2.center);
		\draw (0) to (3.center);
	\end{pgfonlayer}
\end{tikzpicture}
$$

$\tilde \X$ is regarded as a discrete cartesian restriction category with:
\begin{description}

\item[Restriction product:]
\hfil
$
\langle f,g \rangle:=
\begin{tikzpicture}
	\begin{pgfonlayer}{nodelayer}
		\node [style=map] (0) at (-0.25, 2.5) {$f$};
		\node [style=none] (1) at (-0.25, 3.5) {};
		\node [style=none] (2) at (0.75, 3.5) {};
		\node [style=none] (3) at (-0.25, 3.5) {};
		\node [style=map] (4) at (0.75, 2.5) {$g$};
		\node [style=none] (5) at (0.75, 3.5) {};
		\node [style=otimes] (6) at (0.75, 3.5) {};
		\node [style=otimes] (7) at (-0.25, 3.5) {};
		\node [style=Z] (8) at (0.25, 1.5) {};
		\node [style=none] (9) at (-0.25, 4.5) {};
		\node [style=none] (10) at (0.75, 4.5) {};
		\node [style=none] (11) at (0.25, 0.5) {};
	\end{pgfonlayer}
	\begin{pgfonlayer}{edgelayer}
		\draw [style=simple, in=117, out=-120] (1.center) to (0);
		\draw [style=simple] (2.center) to (0);
		\draw [style=simple] (3.center) to (4);
		\draw [style=simple, in=63, out=-60] (5.center) to (4);
		\draw [style=simple, in=56, out=-90] (4) to (8);
		\draw [style=simple, in=-90, out=124] (8) to (0);
		\draw [style=simple] (9.center) to (1.center);
		\draw [style=simple] (2.center) to (10.center);
		\draw [style=simple] (8) to (11.center);
	\end{pgfonlayer}
\end{tikzpicture}
$

\item[Restriction terminal map:]
\hfil
$
\begin{tikzpicture}
	\begin{pgfonlayer}{nodelayer}
		\node [style=none] (2) at (11.75, -0.25) {};
		\node [style=none] (3) at (11.75, -1.25) {};
		\node [style=none] (4) at (11.75, 0) {$X$};
		\node [style=none] (5) at (11.75, -1.5) {$X$};
		\node [style=none] (7) at (12.75, -0.75) {};
		\node [style=unit] (8) at (12.75, -0.75) {};
		\node [style=none] (9) at (12.75, -0.25) {};
		\node [style=none] (10) at (12.75, 0) {$I$};
	\end{pgfonlayer}
	\begin{pgfonlayer}{edgelayer}
		\draw (3.center) to (2.center);
		\draw (7.center) to (9.center);
	\end{pgfonlayer}
\end{tikzpicture}
$
\end{description}

\end{definition}



\begin{theorem}\cite[Thm. 5.2.6]{giles}
There is an equivalence of categories between the category of discrete inverse categories and the category of discrete Cartesian categories.
\end{theorem}


\begin{example}\cite[Ex. 5.3.3]{giles}
$\tilde \Pinj$ is $\Par$.
\end{example}
\begin{proof}
For a partial function $f:X\to Y$, $\{(x,(y,x)) | (x,y) \in f \}/\sim$ is a partial isomorphism.
\end{proof}



\begin{lemma}
\label{lemma:xtildefaithful}
The canonical functor $\iota:\X\to \tilde \X$ is faithful.
\end{lemma}

\begin{proof}
Suppose that $\iota(f)\sim\iota(g)$, Then:


\begin{align*}
\begin{tikzpicture}
	\begin{pgfonlayer}{nodelayer}
		\node [style=map] (0) at (-0.5, 1.5) {$g$};
		\node [style=none] (1) at (-0.5, 2.5) {};
		\node [style=none] (2) at (-0.5, 0.5) {};
	\end{pgfonlayer}
	\begin{pgfonlayer}{edgelayer}
		\draw (1.center) to (0);
		\draw (0) to (2.center);
	\end{pgfonlayer}
\end{tikzpicture}
&=
\begin{tikzpicture}
	\begin{pgfonlayer}{nodelayer}
		\node [style=map] (0) at (-0.5, 1.25) {$f$};
		\node [style=none] (1) at (-0.5, 0.5) {};
		\node [style=map] (2) at (-0.25, 3) {$f^\circ$};
		\node [style=map] (3) at (-0.25, 3.75) {$g$};
		\node [style=Z] (4) at (-0.5, 2) {};
		\node [style=Z] (5) at (-0.5, 4.75) {};
		\node [style=none] (6) at (-0.5, 5.75) {};
	\end{pgfonlayer}
	\begin{pgfonlayer}{edgelayer}
		\draw (0) to (1.center);
		\draw (6.center) to (5);
		\draw [in=120, out=-120, looseness=0.75] (5) to (4);
		\draw [in=-90, out=56] (4) to (2);
		\draw (0) to (4);
		\draw (3) to (2);
		\draw [in=-63, out=90] (3) to (5);
	\end{pgfonlayer}
\end{tikzpicture}
=
\begin{tikzpicture}
	\begin{pgfonlayer}{nodelayer}
		\node [style=none] (0) at (-0.5, 0.5) {};
		\node [style=map] (1) at (-0.25, 4.5) {$g$};
		\node [style=Z] (2) at (-0.5, 2.75) {};
		\node [style=Z] (3) at (-0.5, 5.5) {};
		\node [style=none] (4) at (-0.5, 6.5) {};
		\node [style=map] (5) at (-0.25, 3.75) {$f^\circ$};
		\node [style=map] (6) at (-0.5, 1.25) {$f$};
		\node [style=map] (7) at (-0.5, 2) {$f^\circ f$};
	\end{pgfonlayer}
	\begin{pgfonlayer}{edgelayer}
		\draw (4.center) to (3);
		\draw [in=120, out=-120, looseness=0.75] (3) to (2);
		\draw [in=-63, out=90] (1) to (3);
		\draw [in=-90, out=56] (2) to (5);
		\draw (1) to (5);
		\draw [style=simple] (2) to (7);
		\draw [style=simple] (7) to (6);
		\draw [style=simple] (6) to (0.center);
	\end{pgfonlayer}
\end{tikzpicture}
=
\begin{tikzpicture}
	\begin{pgfonlayer}{nodelayer}
		\node [style=none] (0) at (-0.5, 0.5) {};
		\node [style=map] (1) at (0, 4) {$g$};
		\node [style=Z] (2) at (-0.5, 2.25) {};
		\node [style=Z] (3) at (-0.5, 5) {};
		\node [style=none] (4) at (-0.5, 6) {};
		\node [style=map] (5) at (0, 3.25) {$f^\circ$};
		\node [style=map] (6) at (-0.5, 1.25) {$f$};
		\node [style=map] (7) at (-1, 3.25) {$f^\circ f$};
		\node [style=none] (8) at (-1, 4) {};
	\end{pgfonlayer}
	\begin{pgfonlayer}{edgelayer}
		\draw (4.center) to (3);
		\draw [in=-60, out=90] (1) to (3);
		\draw [in=-90, out=56] (2) to (5);
		\draw (1) to (5);
		\draw [style=simple] (6) to (0.center);
		\draw [style=simple, in=-90, out=120] (2) to (7);
		\draw [style=simple] (2) to (6);
		\draw [style=simple, in=90, out=-120] (3) to (8.center);
		\draw [style=simple] (8.center) to (7);
	\end{pgfonlayer}
\end{tikzpicture}
=
\begin{tikzpicture}
	\begin{pgfonlayer}{nodelayer}
		\node [style=none] (0) at (-0.5, 0.5) {};
		\node [style=map] (1) at (0, 3.25) {$g$};
		\node [style=Z] (2) at (-0.5, 2.25) {};
		\node [style=Z] (3) at (-0.5, 4.25) {};
		\node [style=none] (4) at (-0.5, 5.25) {};
		\node [style=map] (5) at (-0.5, 1.25) {$ff^\circ$};
		\node [style=map] (6) at (-1, 3.25) {$f$};
	\end{pgfonlayer}
	\begin{pgfonlayer}{edgelayer}
		\draw (4.center) to (3);
		\draw [in=-60, out=90] (1) to (3);
		\draw [style=simple] (5) to (0.center);
		\draw [style=simple, in=-90, out=120] (2) to (6);
		\draw [style=simple] (2) to (5);
		\draw [style=simple, in=60, out=-90] (1) to (2);
		\draw [style=simple, in=90, out=-120] (3) to (6);
	\end{pgfonlayer}
\end{tikzpicture}
=
\begin{tikzpicture}
	\begin{pgfonlayer}{nodelayer}
		\node [style=none] (0) at (-0.5, 0.5) {};
		\node [style=map] (1) at (0, 3.25) {$g$};
		\node [style=Z] (2) at (-0.5, 1.5) {};
		\node [style=Z] (3) at (-0.5, 4.25) {};
		\node [style=none] (4) at (-0.5, 5.25) {};
		\node [style=map] (5) at (-1, 3.25) {$f$};
		\node [style=map] (6) at (-1, 2.5) {$ff^\circ$};
		\node [style=none] (7) at (0, 2.5) {};
	\end{pgfonlayer}
	\begin{pgfonlayer}{edgelayer}
		\draw (4.center) to (3);
		\draw [in=-60, out=90] (1) to (3);
		\draw [style=simple, in=90, out=-120] (3) to (5);
		\draw (1) to (7.center);
		\draw [in=60, out=-90] (7.center) to (2);
		\draw (2) to (0.center);
		\draw [in=-90, out=120] (2) to (6);
		\draw (6) to (5);
	\end{pgfonlayer}
\end{tikzpicture}\\
&=
\begin{tikzpicture}
	\begin{pgfonlayer}{nodelayer}
		\node [style=none] (0) at (-0.5, 0.5) {};
		\node [style=map] (1) at (0, 2.5) {$g$};
		\node [style=Z] (2) at (-0.5, 1.5) {};
		\node [style=Z] (3) at (-0.5, 3.5) {};
		\node [style=none] (4) at (-0.5, 4.5) {};
		\node [style=map] (5) at (-1, 2.5) {$f$};
	\end{pgfonlayer}
	\begin{pgfonlayer}{edgelayer}
		\draw (4.center) to (3);
		\draw [in=-60, out=90] (1) to (3);
		\draw [style=simple, in=90, out=-120] (3) to (5);
		\draw (2) to (0.center);
		\draw [in=60, out=-90] (1) to (2);
		\draw [in=-90, out=120] (2) to (5);
	\end{pgfonlayer}
\end{tikzpicture}
=
\begin{tikzpicture}
	\begin{pgfonlayer}{nodelayer}
		\node [style=none] (0) at (-0.5, 0.5) {};
		\node [style=map] (1) at (-0.2, 2.5) {$g$};
		\node [style=Z] (2) at (-0.5, 1.5) {};
		\node [style=Z] (3) at (-0.5, 3.5) {};
		\node [style=none] (4) at (-0.5, 4.5) {};
		\node [style=map] (5) at (-0.8, 2.5) {$f$};
	\end{pgfonlayer}
	\begin{pgfonlayer}{edgelayer}
		\draw (4.center) to (3);
		\draw [in=-120, out=90, looseness=1.25] (1) to (3);
		\draw [style=simple, in=90, out=-60, looseness=1.25] (3) to (5);
		\draw (2) to (0.center);
		\draw [in=120, out=-90, looseness=1.25] (1) to (2);
		\draw [in=-90, out=60, looseness=1.25] (2) to (5);
	\end{pgfonlayer}
\end{tikzpicture}
=
\begin{tikzpicture}
	\begin{pgfonlayer}{nodelayer}
		\node [style=none] (0) at (-0.5, 0.5) {};
		\node [style=map] (1) at (0, 2.5) {$f$};
		\node [style=Z] (2) at (-0.5, 1.5) {};
		\node [style=Z] (3) at (-0.5, 3.5) {};
		\node [style=none] (4) at (-0.5, 4.5) {};
		\node [style=map] (5) at (-1, 2.5) {$g$};
	\end{pgfonlayer}
	\begin{pgfonlayer}{edgelayer}
		\draw (4.center) to (3);
		\draw [in=-60, out=90] (1) to (3);
		\draw [style=simple, in=90, out=-120] (3) to (5);
		\draw (2) to (0.center);
		\draw [in=60, out=-90] (1) to (2);
		\draw [in=-90, out=120] (2) to (5);
	\end{pgfonlayer}
\end{tikzpicture}
=
\begin{tikzpicture}
	\begin{pgfonlayer}{nodelayer}
		\node [style=map] (0) at (-0.5, 1.25) {$g$};
		\node [style=none] (1) at (-0.5, 0.5) {};
		\node [style=map] (2) at (-0.25, 3) {$g^\circ$};
		\node [style=map] (3) at (-0.25, 3.75) {$f$};
		\node [style=Z] (4) at (-0.5, 2) {};
		\node [style=Z] (5) at (-0.5, 4.75) {};
		\node [style=none] (6) at (-0.5, 5.75) {};
	\end{pgfonlayer}
	\begin{pgfonlayer}{edgelayer}
		\draw (0) to (1.center);
		\draw (6.center) to (5);
		\draw [in=120, out=-120, looseness=0.75] (5) to (4);
		\draw [in=-90, out=56] (4) to (2);
		\draw (0) to (4);
		\draw (3) to (2);
		\draw [in=-63, out=90] (3) to (5);
	\end{pgfonlayer}
\end{tikzpicture}
=
\begin{tikzpicture}
	\begin{pgfonlayer}{nodelayer}
		\node [style=map] (0) at (-0.5, 1.5) {$f$};
		\node [style=none] (1) at (-0.5, 2.5) {};
		\node [style=none] (2) at (-0.5, 0.5) {};
	\end{pgfonlayer}
	\begin{pgfonlayer}{edgelayer}
		\draw (1.center) to (0);
		\draw (0) to (2.center);
	\end{pgfonlayer}
\end{tikzpicture}
\end{align*}



\end{proof}

Obtaining Cartesian bicategories of relations from discrete inverse categories is more difficult.  We will discuss this later in the thesis.


Let us summarize the various notions of weakenings of cartesian bicategories of relations in a table. Given a posetally enriched monoidal category with (if it exists) a left adjoint special Frobenius algebra structure:


\hfil
\begin{tabular}{l|cccc}
                                                     & $\Delta$          & $!$             & $\Delta^*$         & $!^*$\\
\hline
Discrete inverse category            & nat &  & nat  & \\
Cartesian restriction category      & nat &  lax  & lax \\
Cartesian                                      & nat & nat &   \\
Cartesian bicategory of relations & lax  & lax & lax & lax \\
\end{tabular}


The unit can be interpreted as providing the existence of a process, the counit as the discarding; the comultiplication as copying and the multiplication as comparison.

A process being a comultiplication homomorphism means it is deterministic and a counit homorphism means it is total.


\subsection{Internal category theory}

%\begin{definition}
%\label{def:monad}
%%monad
%\end{definition}
%
%
%
%\begin{definition}
%\label{def:span}
%
%%2-category of spans, cospans
%\end{definition}
%
%
%\begin{definition}
%\label{def:rel}
%
%%2-category of relations, corelations
%\end{definition}




This allows us to regard categories as monads:
\begin{definition}
\label{def:monad}
%monad
\end{definition}


\begin{definition}
\label{def:internalcat}

%Internal category
Given a category $\mathcal V$ with finite pullbacks $\mathcal V$, a $\mathcal V$-{\bf internal category} is a monad in $\Span(\mathcal V)$.
\end{definition}

Internal categories are indeed categories.  The collection of objects is given by the feet of the span, the set of morphisms by the apex, the domain and codomain by the left and right legs respectively.  The components of the unit of the comonad give the identity morphisms and the multiplication of the monad gives the composition.

\begin{lemma}
\label{lem:internalcat}

Monads internal to $\Set$ are in bijection with small categories.
\end{lemma}


It is not the case that monad maps correspond to functors between internal categories.  A canonical way to obtain such a notion requires the machinery of double categories, which is outside of the purview of this thesis.  However, (globular) 2-categories suffice to construct internal profunctors which is what we need the machinery for.


This allows us to compose internal categories with the same set of objects via distributive law via the composite monad induced by a distributive law on monads in $\Span(\mathcal{V})$.


Just as we can regard categories as certain monoids; by changing the setting with which we are working internal to, we can do the same for {\em monoidal} categories.

\begin{definition}
\label{def:monoid}
Let $\Mon$ denote the category with set-monoids as objects and monoid homorphisms as morphisms.
\end{definition}



\begin{lemma}
\label{def:internalmonoidalcat}

Monads internal to $\Mon$ are in bijection with small monoidal categories.
\end{lemma}

We want to be able to take distributive laws of two categories which both share some structure.  For example, distributive laws of symmetric monoidal categories where the braiding of both categories are identified with each other.  For this, we need to work in bimodules of internal categories:

\begin{definition}
Given a category $\mathcal V$ with finite pullbacks and coequalizers preserving them, let $\mathcal V-\Prof:=\Mod(\Mnd(\mathcal V))$ denote the 2-category of $\mathcal V$-{\bf internal profunctors}.  
The 1-cells of $\Prof$ are called (internal) {\bf  profunctors}.
The tensor product of bimodules of internal categories is the (internal) {\bf coend}.
\end{definition}




%
%String diagrams are the canonical graphical calculi for {\em strict monoidal categories}. Objects are drawn as wires, morphisms are drawn as boxes; the tensor product is giving by connecting the wires together and the tensor product is given by monoidal pasting.  The coherence for strict monoidal categories is equivalent to planar isotopy of these diagrams. As a matter of convention, we will draw the order of composition from bottom to top and the tensor from left to right.
%
%
%GIVE EXAMPLE
%
%String diagrams for monoidal categories can be augmented to describe morphisms in non-strict monoidal categories by adding four connectives and equations:
%
%\begin{definition}
%Given a (non-strict) monoidal category the monoidal category of proof nets in $\X$ is generated by the string diagrams for $\X$ addition to the following generators for all objects $X,Y$
%
%modulo the equations
%\end{definition}
%
%
%\begin{lemma}
%There is a fully faithful monoidal functor from $\C$ to proof nets over $\C$ given by:
%
%Draw action
%
%
%give coherence rules
%\end{lemma}
%
%Although this has long been known, the idea of proof nets for monoidal categories has recently been rediscovered \cite{wilson}, where the coauthors exhibit proof nets as the residue of a novel algebraic proof of MacLane's coherence theorem for monoidal categories.In the ZX-calculus literature, proof nets for strict monoidal categories have also been rediscovered as the scalable ZX calculus.  In the scalable ZX-calculus, the nets for the units have been ommited, and they use the proof nets to index wires when specifying quantum protocols diagrammatically.
%
% we give a novel conceptual proof in Section \ref{??} which constructs proof nets from string diagrams in a canonical way.  This way of viewing things can be generalized to other settings that monoidal categories.



\begin{definition}
distributive law of symmetric monoidal theories
\end{definition}


Spans and cospans arise from distributive laws





\begin{definition}
Consider the following two distributive laws: 
\begin{align*}
\cm^\op  \otimes_\P \cm;&
  \begin{tikzpicture}
	\begin{pgfonlayer}{nodelayer}
		\node [style=X] (0) at (-3.75, -1) {};
		\node [style=none] (1) at (-4, -1.75) {};
		\node [style=none] (2) at (-3.5, -1.75) {};
		\node [style=Z] (3) at (-3.75, -0.25) {};
		\node [style=none] (4) at (-4, 0.5) {};
		\node [style=none] (5) at (-3.5, 0.5) {};
	\end{pgfonlayer}
	\begin{pgfonlayer}{edgelayer}
		\draw [in=90, out=-60, looseness=1.00] (0) to (2.center);
		\draw [in=-120, out=90, looseness=1.00] (1.center) to (0);
		\draw (0) to (3);
		\draw [in=60, out=-90, looseness=1.00] (5.center) to (3);
		\draw [in=-90, out=120, looseness=1.00] (3) to (4.center);
	\end{pgfonlayer}
  \end{tikzpicture}
  \eqzxa{bi.one}
  \begin{tikzpicture}
	\begin{pgfonlayer}{nodelayer}
		\node [style=X] (0) at (-4, 0.5) {};
		\node [style=Z] (1) at (-4, -0.25) {};
		\node [style=X] (2) at (-4.5, 0.5) {};
		\node [style=Z] (3) at (-4.5, -0.25) {};
		\node [style=none] (4) at (-4, -1) {};
		\node [style=none] (5) at (-4.5, -1) {};
		\node [style=none] (6) at (-4.5, 1.25) {};
		\node [style=none] (7) at (-4, 1.25) {};
	\end{pgfonlayer}
	\begin{pgfonlayer}{edgelayer}
		\draw [bend left, looseness=1.25] (0) to (1);
		\draw [bend right, looseness=1.25] (2) to (3);
		\draw (1) to (2);
		\draw (3) to (0);
		\draw (0) to (7.center);
		\draw (6.center) to (2);
		\draw (3) to (5.center);
		\draw (4.center) to (1);
	\end{pgfonlayer}
\end{tikzpicture},
\hspace*{.5cm}
  \begin{tikzpicture}
	\begin{pgfonlayer}{nodelayer}
		\node [style=Z] (0) at (-4, -0) {};
		\node [style=X] (1) at (-4, -0.75) {};
		\node [style=none] (2) at (-4.25, -1.5) {};
		\node [style=none] (3) at (-3.75, -1.5) {};
	\end{pgfonlayer}
	\begin{pgfonlayer}{edgelayer}
		\draw [in=-60, out=90, looseness=1.00] (3.center) to (1);
		\draw (1) to (0);
		\draw [in=90, out=-120, looseness=1.00] (1) to (2.center);
	\end{pgfonlayer}
  \end{tikzpicture}
  \eqzxa{bi.two}
  \begin{tikzpicture}
	\begin{pgfonlayer}{nodelayer}
		\node [style=Z] (0) at (-4.25, -0.75) {};
		\node [style=none] (1) at (-4.25, -1.5) {};
		\node [style=none] (2) at (-3.5, -1.5) {};
		\node [style=Z] (3) at (-3.5, -0.75) {};
	\end{pgfonlayer}
	\begin{pgfonlayer}{edgelayer}
		\draw (2.center) to (3);
		\draw (0) to (1.center);
	\end{pgfonlayer}
  \end{tikzpicture},
  \hspace*{.5cm}
   \begin{tikzpicture}[yscale=-1]
	\begin{pgfonlayer}{nodelayer}
		\node [style=X] (0) at (-4, -0) {};
		\node [style=Z] (1) at (-4, -0.75) {};
		\node [style=none] (2) at (-4.25, -1.5) {};
		\node [style=none] (3) at (-3.75, -1.5) {};
	\end{pgfonlayer}
	\begin{pgfonlayer}{edgelayer}
		\draw [in=-60, out=90, looseness=1.00] (3.center) to (1);
		\draw (1) to (0);
		\draw [in=90, out=-120, looseness=1.00] (1) to (2.center);
	\end{pgfonlayer}
  \end{tikzpicture}
  \erefop{bi.two}
   \begin{tikzpicture}[yscale=-1]
	\begin{pgfonlayer}{nodelayer}
		\node [style=X] (0) at (-4.25, -0.75) {};
		\node [style=none] (1) at (-4.25, -1.5) {};
		\node [style=none] (2) at (-3.5, -1.5) {};
		\node [style=X] (3) at (-3.5, -0.75) {};
	\end{pgfonlayer}
	\begin{pgfonlayer}{edgelayer}
		\draw (2.center) to (3);
		\draw (0) to (1.center);
	\end{pgfonlayer}
  \end{tikzpicture},
\hspace*{.5cm}
  \begin{tikzpicture}[rotate=90]
	\begin{pgfonlayer}{nodelayer}
		\node [style=Z] (0) at (-8.25, -0) {};
		\node [style=X] (1) at (-9.25, -0) {};
	\end{pgfonlayer}
	\begin{pgfonlayer}{edgelayer}
		\draw (0) to (1);
	\end{pgfonlayer}
\end{tikzpicture}
  \eqzxa{extra}
\\
 \cm \otimes_\P \cm^\op;&
    \begin{tikzpicture}[rotate=90]
	\begin{pgfonlayer}{nodelayer}
		\node [style=X] (0) at (-6.25, 0.25) {};
		\node [style=none] (1) at (-7, 0.25) {};
		\node [style=none] (2) at (-4.75, 0.25) {};
		\node [style=X] (3) at (-5.5, 0.25) {};
	\end{pgfonlayer}
	\begin{pgfonlayer}{edgelayer}
		\draw (0) to (1.center);
		\draw (3) to (2.center);
		\draw [bend right, looseness=1.25] (3) to (0);
		\draw [bend right, looseness=1.25] (0) to (3);
	\end{pgfonlayer}
  \end{tikzpicture}
  \eqzxa{special}
  \begin{tikzpicture}[rotate=90]
	\begin{pgfonlayer}{nodelayer}
		\node [style=none] (0) at (-7, 0.25) {};
		\node [style=none] (1) at (-6, 0.25) {};
	\end{pgfonlayer}
	\begin{pgfonlayer}{edgelayer}
		\draw (1.center) to (0.center);
	\end{pgfonlayer}
  \end{tikzpicture},
  \hspace*{.5cm}
  \begin{tikzpicture}[rotate=90]
	\begin{pgfonlayer}{nodelayer}
		\node [style=X] (0) at (-7, -0) {};
		\node [style=X] (1) at (-6.25, 0.5) {};
		\node [style=none] (2) at (-7, 0.75) {};
		\node [style=none] (3) at (-7.75, 0.75) {};
		\node [style=none] (4) at (-7.75, -0) {};
		\node [style=none] (5) at (-6.25, -0.25) {};
		\node [style=none] (6) at (-5.5, -0.25) {};
		\node [style=none] (7) at (-5.5, 0.5) {};
	\end{pgfonlayer}
	\begin{pgfonlayer}{edgelayer}
		\draw (6.center) to (5.center);
		\draw [in=-30, out=180, looseness=1.00] (5.center) to (0);
		\draw (1) to (0);
		\draw [in=0, out=150, looseness=1.00] (1) to (2.center);
		\draw (2.center) to (3.center);
		\draw (0) to (4.center);
		\draw (1) to (7.center);
	\end{pgfonlayer}
  \end{tikzpicture}
 =
  \begin{tikzpicture}[rotate=90,xscale=-1]
	\begin{pgfonlayer}{nodelayer}
		\node [style=X] (0) at (-7, -0) {};
		\node [style=X] (1) at (-6.25, 0.5) {};
		\node [style=none] (2) at (-7, 0.75) {};
		\node [style=none] (3) at (-7.75, 0.75) {};
		\node [style=none] (4) at (-7.75, -0) {};
		\node [style=none] (5) at (-6.25, -0.25) {};
		\node [style=none] (6) at (-5.5, -0.25) {};
		\node [style=none] (7) at (-5.5, 0.5) {};
	\end{pgfonlayer}
	\begin{pgfonlayer}{edgelayer}
		\draw (6.center) to (5.center);
		\draw [in=-30, out=180, looseness=1.00] (5.center) to (0);
		\draw (1) to (0);
		\draw [in=0, out=150, looseness=1.00] (1) to (2.center);
		\draw (2.center) to (3.center);
		\draw (0) to (4.center);
		\draw (1) to (7.center);
	\end{pgfonlayer}
  \end{tikzpicture}
  \eqzxa{frob}
  \begin{tikzpicture}[rotate=90]
	\begin{pgfonlayer}{nodelayer}
		\node [style=none] (0) at (-4.75, -0.25) {};
		\node [style=X] (1) at (-5.5, -0) {};
		\node [style=none] (2) at (-7, -0.25) {};
		\node [style=X] (3) at (-6.25, 0) {};
		\node [style=none] (4) at (-4.75, 0.25) {};
		\node [style=none] (5) at (-7, 0.25) {};
	\end{pgfonlayer}
	\begin{pgfonlayer}{edgelayer}
		\draw [in=-30, out=180, looseness=1.25] (0.center) to (1);
		\draw (3) to (1);
		\draw [in=180, out=30, looseness=1.25] (1) to (4.center);
		\draw [in=0, out=-150, looseness=1.25] (3) to (2.center);
		\draw [in=0, out=150, looseness=1.25] (3) to (5.center);
	\end{pgfonlayer}
\end{tikzpicture}
  \end{align*}

The former yields, {\sf cb}, the prop for the free {\bf bicommutative bialgebra} and the latter yields, {\sf scfa}, the prop for the free {\bf special commutative Frobenius algebra}.

\end{definition}




\begin{lemma} \cite[\S 5.3, 5.4]{lack}
{\sf cb} is a presentation for $(\Span^{\sim}(\FSets),+)$ and {\sf scfa} is a presentation for $(\Csp^\sim(\FSets),+)$.

\end{lemma}

There is an equivalent way of thinking of $(\Span^{\sim}(\FSets),+)$, which we will also use throughout this thesis:

\begin{lemma}
{\sf cb} is a presentation for $\Mat_\N$ under the direct sum.
\end{lemma}


Because $\N$ is the initial commutative semiring ring:
\begin{lemma}
Given any semiring $S$, $\Mat_S$ is presented by the prop $\ch_S$ given by $\ch$ as well as generators $r$ for each $r \in S$ modulo the equations of the ring $S$:

TODO
\end{lemma}
This way of looking at the prop for the free bicommutative bialgebra, naturally leads to a concrete discription of the free bicommutative hopf algebra.

\begin{definition}
A  {\bf bicommutative  Hopf algebra} is a bicommutative bialgebra with an antipode map $s:1\to1$ satisfying the following equation:

$$
\begin{tikzpicture}
	\begin{pgfonlayer}{nodelayer}
		\node [style=none] (0) at (2.5, 5) {};
		\node [style=X] (1) at (2.5, 4.25) {};
		\node [style=Z] (2) at (2.5, 2.75) {};
		\node [style=none] (3) at (2.5, 2) {};
		\node [style=map] (4) at (2, 3.5) {$s$};
	\end{pgfonlayer}
	\begin{pgfonlayer}{edgelayer}
		\draw [in=150, out=-90] (4) to (2);
		\draw [bend right=60, looseness=1.25] (2) to (1);
		\draw [in=90, out=-150] (1) to (4);
		\draw (2) to (3.center);
		\draw (1) to (0.center);
	\end{pgfonlayer}
\end{tikzpicture}
=
\begin{tikzpicture}
	\begin{pgfonlayer}{nodelayer}
		\node [style=none] (0) at (2.5, 2) {};
		\node [style=Z] (1) at (2.5, 2.75) {};
		\node [style=X] (2) at (2.5, 4.25) {};
		\node [style=none] (3) at (2.5, 5) {};
	\end{pgfonlayer}
	\begin{pgfonlayer}{edgelayer}
		\draw (2) to (3.center);
		\draw (1) to (0.center);
	\end{pgfonlayer}
\end{tikzpicture}
%=
%\begin{tikzpicture}
%	\begin{pgfonlayer}{nodelayer}
%		\node [style=none] (0) at (2, 5) {};
%		\node [style=X] (1) at (2, 4.25) {};
%		\node [style=Z] (2) at (2, 2.75) {};
%		\node [style=none] (3) at (2, 2) {};
%		\node [style=map] (4) at (2.5, 3.5) {$s$};
%	\end{pgfonlayer}
%	\begin{pgfonlayer}{edgelayer}
%		\draw [in=30, out=-90] (4) to (2);
%		\draw [bend left=60, looseness=1.25] (2) to (1);
%		\draw [in=90, out=-30] (1) to (4);
%		\draw (2) to (3.center);
%		\draw (1) to (0.center);
%	\end{pgfonlayer}
%\end{tikzpicture}
$$



Let $\ch$ denote the prop for the free commutative hopf algebra.
\end{definition}

\begin{lemma}
$\ch$ is a presentation for $\Mat_\Z$ under the direct sum.
\end{lemma}

Because $\Z$ is the initial commutative ring:
\begin{lemma}
Given any ring $R$, $\Mat_R$ is presented by the prop $\ch_R$ given by $\ch$ as well as generators $r$ for each $r \in R$ modulo the equations of the ring $R$:

TODO
\end{lemma}

Distributive laws are particularly nice because they allow us to factorize maps into normal forms:
\begin{lemma}
\label{lem:distfact}
TODO: Distributive laws and factorization theorems
\end{lemma}


For example:

\begin{lemma}[Spider theorem]
Given parallel string diagrams generated by the components of a Frobenius algebra, then they are equal if and only if they have the same connected components.  A connected component with $n$ inputs and $m$ outputs has the normal form where the $n$ inputs are left associated, and plugged into the left coassociated $m$ outputs.


Graphically, the connected components are normalized to the following shape which we contract using the spider notation:

$$
\begin{tikzpicture}
	\begin{pgfonlayer}{nodelayer}
		\node [style=Z] (0) at (1.25, 3) {};
		\node [style=Z] (1) at (0.5, 4) {};
		\node [style=Z] (2) at (1.25, 2.25) {};
		\node [style=Z] (3) at (0.5, 1.25) {};
		\node [style=none] (4) at (1.5, 4) {};
		\node [style=none] (5) at (1.5, 1.25) {};
		\node [style=none] (6) at (0.25, 0.5) {};
		\node [style=none] (7) at (1.5, 4.75) {};
		\node [style=none] (8) at (1.5, 0.5) {};
		\node [style=none] (9) at (0.75, 4.75) {};
		\node [style=none] (10) at (0.25, 4.75) {};
		\node [style=none] (11) at (0.75, 0.5) {};
		\node [style=none] (12) at (1, 3.25) {};
		\node [style=none] (13) at (0.5, 3.75) {};
		\node [style=none] (14) at (0.5, 1.5) {};
		\node [style=none] (15) at (1, 2) {};
		\node [style=none] (16) at (0.75, 3.5) {$\ddots$};
		\node [style=none] (17) at (0.75, 1.75) {$\reflectbox{$\ddots$}$};
		\node [style=none] (18) at (1.2, 0.5) {$\cdots$};
		\node [style=none] (19) at (1.2, 4.75) {$\cdots$};
	\end{pgfonlayer}
	\begin{pgfonlayer}{edgelayer}
		\draw (7.center) to (4.center);
		\draw [in=105, out=-90] (10.center) to (1);
		\draw [in=60, out=-90, looseness=0.75] (4.center) to (0);
		\draw [in=-90, out=75] (1) to (9.center);
		\draw [in=300, out=90] (5.center) to (2);
		\draw [in=90, out=-120] (3) to (6.center);
		\draw [in=90, out=-60] (3) to (11.center);
		\draw (8.center) to (5.center);
		\draw (0) to (2);
		\draw (3) to (14.center);
		\draw (15.center) to (2);
		\draw (13.center) to (1);
		\draw (0) to (12.center);
	\end{pgfonlayer}
\end{tikzpicture}
=:
\begin{tikzpicture}
	\begin{pgfonlayer}{nodelayer}
		\node [style=none] (0) at (1.5, 1.75) {};
		\node [style=none] (1) at (2.75, 1.75) {};
		\node [style=none] (2) at (2, 1.75) {};
		\node [style=none] (3) at (2.45, 1.75) {$\cdots$};
		\node [style=none] (4) at (2.75, 3.25) {};
		\node [style=none] (5) at (2, 3.25) {};
		\node [style=none] (6) at (1.5, 3.25) {};
		\node [style=none] (7) at (2.45, 3.25) {$\cdots$};
		\node [style=Z] (8) at (2, 2.5) {};
	\end{pgfonlayer}
	\begin{pgfonlayer}{edgelayer}
		\draw [in=-90, out=45] (8) to (4.center);
		\draw (8) to (5.center);
		\draw [in=135, out=-90] (6.center) to (8);
		\draw [in=90, out=-150] (8) to (0.center);
		\draw (2.center) to (8);
		\draw [in=90, out=-30] (8) to (1.center);
	\end{pgfonlayer}
\end{tikzpicture}
$$

\end{lemma}


%
%\subsection{Enriched category theory}
%In this chapter, we develop the theory of enriched profunctors.  Recall in Definition \ref{def:internalprof}, in order to define distributive laws of props, we briefly described internal profunctors.  This imposes size restrictions which are sometimes undesirable. The cousin of internal category theory is enriched category theory:  in the $\mathcal V$-enriched setting, there is only the requirement that between two objects there is a $\mathcal V$-category.  
%
%
%The following data gives enough structure to develop the theory of enriched categories: 
%\begin{definition}
%A {\bf Benabou cosmos } is a complete, cocomplete symmetric monoidal closed category.
%\end{definition}
%
%
%
%\begin{definition}
%Given a Benabou cosmos ${\mathcal V}$,  a $\mathcal V$-{\bf profunctor} $\X \proarrow \Y$ is a functor  $\X^\op \times \Y \to \mathcal V$.
%\end{definition}
%
%
%\begin{definition}
%Given an endo $\mathcal V$-{\bf profunctor} $P:\X \proarrow \X$, the  {\bf coend} $\int^{X} P(X,X) $ is given by the coequalizer:
%
%$
%  \xymatrix{
%      \coprod_{X\to X'} P(X',X)  \ar@<-0.5ex>[r]\ar@<0.5ex>[r] &
%      \coprod_{X \in \X_0} P(X,X) \ar[r] &
%  	\int^{X} P(X,X) \\
%  }
%$
%
%\end{definition}
%
%
%
%\begin{lemma}
%Monoidal bicategory
%\end{lemma}
%
%
%\begin{definition}
%The monoidal bicategory $\Prof$ has:
%
%%embeddings and adjoints
%\end{definition}
%
%
%
%\begin{definition}
%The monoidal bicategory $\Prof^*$ has:
%
%
%\end{definition}
%
%
%\begin{theorem}
%Quasitrictification theorem for monoidal 2-category
%\end{theorem}
%
%\begin{corollary}
%Graphical calculus for quasitrict monoidal 2-category
%\end{corollary}
%
%
%\begin{lemma}
%Graphical calculus for pointed profunctors
%
%%
%%monoidal functors
%\end{lemma}





\section{Categorical quantum mechanics}
Historically, quantum computing literature has described quantum processes in terms of sums, products bras, kets and so on.
However, graphical perspectives on quantum mechanics have existed for quite a while.  Notably Penrose used string diagrams in several of his papers \cite{penrosei,??}. It wasn't until relatively recently, that string diagrams for quantum processes started be be taken seriously in their own right.  In their seminal paper, \cite{abramsky} used the theory of monoidal categories to interpret quantum protocols as well as their string diagrams.  This has spawned a myriad of research analyzing quantum computing from graphical and categorical perspectives.  We will will give a brief review of quantum compting from this categorical, graphical perspective; ie categorical quantum mechanics.  A mathematical introduction to the subject can be found in \cite{heunen}, with a more broadly accessible introduction being found in \cite{pqp}.

There is a very important algebraic structure on $\dag$-compact closed categories:

\begin{definition}
\label{def:specialdagfa}
%special dag-Frobenius algebras
A {\bf dagger Frobenius algebra}  is a Frobenius algebra in a dagger category whose monoid structure is the dagger of the comonoid structure.
\end{definition}

These allow us to treat bases diagramatically:
\begin{lemma}\cite{???}
\label{lem:specialdagfa}
%special dag-Frobenius algebras in FHilb are orthonomal bases/quantum observables

Special commutative dagger Frobenius algebras in $\FHilb$ are in bijection with orthonormal bases, where a basis $\{ |i\rangle \}_{i=0,\ldots, n-1}$ yields the Frobenius algebra:
$$
\left\llbracket
\begin{tikzpicture}
	\begin{pgfonlayer}{nodelayer}
		\node [style=none] (0) at (1.5, 1.75) {};
		\node [style=none] (1) at (2.75, 1.75) {};
		\node [style=none] (2) at (2, 1.75) {};
		\node [style=none] (3) at (2.45, 1.75) {$\cdots$};
		\node [style=none] (4) at (2.75, 3.25) {};
		\node [style=none] (5) at (2, 3.25) {};
		\node [style=none] (6) at (1.5, 3.25) {};
		\node [style=none] (7) at (2.45, 3.25) {$\cdots$};
		\node [style=Z] (8) at (2, 2.5) {};
	\end{pgfonlayer}
	\begin{pgfonlayer}{edgelayer}
		\draw [in=-90, out=45] (8) to (4.center);
		\draw (8) to (5.center);
		\draw [in=135, out=-90] (6.center) to (8);
		\draw [in=90, out=-150] (8) to (0.center);
		\draw (2.center) to (8);
		\draw [in=90, out=-30] (8) to (1.center);
	\end{pgfonlayer}
\end{tikzpicture}
\right\rrbracket
= 
\sum_{i=0}^{n-1} |i, \ldots, i \rangle i ,\ldots, i\langle
$$
\end{lemma}
The basic idea is that these structures are in bijection with dirac deltas for orthonormal bases.

Orthonormal bases have an important intepretation in quantum mechanics.



\begin{definition}
A quantum observable $A$ on a Hilbert space $\mathcal H$ is a self adjoint bounded linear map on $\mathcal H$, so that $A^\dag =A$.
\end{definition}


In quantum computing, quantum observables are Hermetian matrices and spectrum of measurement outcomes the orthonormal eigenbasis of the observable.  Therefore, $\dag$-Frobenius algebras play a prominent role in the diagrammatic analysis of quantum computing.

This naturally leads to the following:

\begin{definition}
\label{def:phases}
Given a $\dag$-Frobenius algebra on an object $X$, a {\bf phase} for the Frobenius algebra is a unitary endomorphism $\theta:X\to X$ which commutes with the multiplication and comultiplication, so that:
$$
\begin{tikzpicture}
	\begin{pgfonlayer}{nodelayer}
		\node [style=Z] (14) at (-1.75, 12) {};
		\node [style=none] (15) at (-1.25, 11.25) {};
		\node [style=none] (16) at (-2.25, 11.25) {};
		\node [style=none] (17) at (-1.75, 12.75) {};
		\node [style=map] (18) at (-1.25, 11.25) {$\theta$};
		\node [style=none] (19) at (-1.25, 10.5) {};
		\node [style=none] (20) at (-2.25, 10.5) {};
	\end{pgfonlayer}
	\begin{pgfonlayer}{edgelayer}
		\draw [in=90, out=-30] (14) to (15.center);
		\draw [in=90, out=-150] (14) to (16.center);
		\draw (17.center) to (14);
		\draw (18) to (19.center);
		\draw (20.center) to (16.center);
	\end{pgfonlayer}
\end{tikzpicture}
=
\begin{tikzpicture}
	\begin{pgfonlayer}{nodelayer}
		\node [style=Z] (0) at (0, 11.25) {};
		\node [style=map] (1) at (0, 12) {$\theta$};
		\node [style=none] (2) at (-0.5, 10.5) {};
		\node [style=none] (3) at (0.5, 10.5) {};
		\node [style=none] (4) at (0, 12.75) {};
	\end{pgfonlayer}
	\begin{pgfonlayer}{edgelayer}
		\draw (4.center) to (1);
		\draw (1) to (0);
		\draw [in=90, out=-150] (0) to (2.center);
		\draw [in=90, out=-30] (0) to (3.center);
	\end{pgfonlayer}
\end{tikzpicture}
=
\begin{tikzpicture}
	\begin{pgfonlayer}{nodelayer}
		\node [style=Z] (5) at (2, 12) {};
		\node [style=none] (7) at (1.5, 11.25) {};
		\node [style=none] (8) at (2.5, 11.25) {};
		\node [style=none] (9) at (2, 12.75) {};
		\node [style=map] (11) at (1.5, 11.25) {$\theta$};
		\node [style=none] (12) at (1.5, 10.5) {};
		\node [style=none] (13) at (2.5, 10.5) {};
	\end{pgfonlayer}
	\begin{pgfonlayer}{edgelayer}
		\draw [in=90, out=-150] (5) to (7.center);
		\draw [in=90, out=-30] (5) to (8.center);
		\draw (9.center) to (5);
		\draw (11) to (12.center);
		\draw (13.center) to (8.center);
	\end{pgfonlayer}
\end{tikzpicture}
\hspace*{1cm}
\begin{tikzpicture}
	\begin{pgfonlayer}{nodelayer}
		\node [style=Z] (33) at (4, 11.25) {};
		\node [style=none] (34) at (4.5, 12) {};
		\node [style=none] (35) at (3.5, 12) {};
		\node [style=none] (36) at (4, 10.5) {};
		\node [style=map] (37) at (4.5, 12) {$\theta$};
		\node [style=none] (38) at (4.5, 12.75) {};
		\node [style=none] (39) at (3.5, 12.75) {};
	\end{pgfonlayer}
	\begin{pgfonlayer}{edgelayer}
		\draw [in=-90, out=30] (33) to (34.center);
		\draw [in=-90, out=150] (33) to (35.center);
		\draw (36.center) to (33);
		\draw (37) to (38.center);
		\draw (39.center) to (35.center);
	\end{pgfonlayer}
\end{tikzpicture}
=
\begin{tikzpicture}
	\begin{pgfonlayer}{nodelayer}
		\node [style=Z] (21) at (5.75, 12) {};
		\node [style=map] (22) at (5.75, 11.25) {$\theta$};
		\node [style=none] (23) at (5.25, 12.75) {};
		\node [style=none] (24) at (6.25, 12.75) {};
		\node [style=none] (25) at (5.75, 10.5) {};
	\end{pgfonlayer}
	\begin{pgfonlayer}{edgelayer}
		\draw (25.center) to (22);
		\draw (22) to (21);
		\draw [in=-90, out=150] (21) to (23.center);
		\draw [in=-90, out=30] (21) to (24.center);
	\end{pgfonlayer}
\end{tikzpicture}
=
\begin{tikzpicture}
	\begin{pgfonlayer}{nodelayer}
		\node [style=Z] (26) at (7.75, 11.25) {};
		\node [style=none] (27) at (7.25, 12) {};
		\node [style=none] (28) at (8.25, 12) {};
		\node [style=none] (29) at (7.75, 10.5) {};
		\node [style=map] (30) at (7.25, 12) {$\theta$};
		\node [style=none] (31) at (7.25, 12.75) {};
		\node [style=none] (32) at (8.25, 12.75) {};
	\end{pgfonlayer}
	\begin{pgfonlayer}{edgelayer}
		\draw [in=-90, out=150] (26) to (27.center);
		\draw [in=-90, out=30] (26) to (28.center);
		\draw (29.center) to (26);
		\draw (30) to (31.center);
		\draw (32.center) to (28.center);
	\end{pgfonlayer}
\end{tikzpicture}
$$

Phases for Frobenius algebras are preserved by composition; and they form a group called the {\bf phase group} for the Frobenius algebra.  The phase group associated with a commutative Frobenius algebra is therefore Abelian.
\end{definition}

The motivating example is in $\FHilb$
\begin{example}
Given an orthonormal basis $\{| j \rangle \}_{j \in I}$ in $\FHilb$, the phases are generated by unitaries $\sum_{j} e^{ \theta \pi i}|  j \rangle\langle j|$ for all $\theta \in [0, 2\pi)$. 

The curve  $e^{ \theta \pi i}$ parameterized by  $\theta \in [0, 2\pi)$ carves out the unit circle in the complex plane so the phase group is isomorphic to the circle (hence the name).
\end{example}

\begin{lemma}[Phased spider theorem]
There is a normal form for the string diagrams generated by the components of a Frobenius algebra and its phase group.
\end{lemma}

The normal form is the same as the vanilla spider theorem, except decorated with a single element of the phase group in the middle, between where the monoid and comonoid meet:

$$
\begin{tikzpicture}
	\begin{pgfonlayer}{nodelayer}
		\node [style=Z] (9) at (4.75, 3) {};
		\node [style=Z] (10) at (4, 4) {};
		\node [style=Z] (11) at (4.75, 2) {};
		\node [style=Z] (12) at (4, 1) {};
		\node [style=none] (13) at (5, 4) {};
		\node [style=none] (14) at (5, 1) {};
		\node [style=none] (15) at (3.75, 0.25) {};
		\node [style=none] (16) at (5, 4.75) {};
		\node [style=none] (17) at (5, 0.25) {};
		\node [style=none] (18) at (4.25, 4.75) {};
		\node [style=none] (19) at (3.75, 4.75) {};
		\node [style=none] (20) at (4.25, 0.25) {};
		\node [style=none] (21) at (4.5, 3.25) {};
		\node [style=none] (22) at (4, 3.75) {};
		\node [style=none] (23) at (4, 1.25) {};
		\node [style=none] (24) at (4.5, 1.75) {};
		\node [style=none] (25) at (4.25, 3.5) {$\ddots$};
		\node [style=none] (26) at (4.25, 1.5) {$\reflectbox{$\ddots$}$};
		\node [style=none] (27) at (4.7, 0.25) {$\cdots$};
		\node [style=none] (28) at (4.7, 4.75) {$\cdots$};
		\node [style=map] (29) at (4.75, 2.5) {$\theta$};
	\end{pgfonlayer}
	\begin{pgfonlayer}{edgelayer}
		\draw (16.center) to (13.center);
		\draw [in=105, out=-90] (19.center) to (10);
		\draw [in=60, out=-90, looseness=0.75] (13.center) to (9);
		\draw [in=-90, out=75] (10) to (18.center);
		\draw [in=300, out=90] (14.center) to (11);
		\draw [in=90, out=-120] (12) to (15.center);
		\draw [in=90, out=-60] (12) to (20.center);
		\draw (17.center) to (14.center);
		\draw (9) to (11);
		\draw (12) to (23.center);
		\draw (24.center) to (11);
		\draw (22.center) to (10);
		\draw (9) to (21.center);
	\end{pgfonlayer}
\end{tikzpicture}
=:
\begin{tikzpicture}
	\begin{pgfonlayer}{nodelayer}
		\node [style=none] (0) at (1.5, 1.75) {};
		\node [style=none] (1) at (2.75, 1.75) {};
		\node [style=none] (2) at (2, 1.75) {};
		\node [style=none] (3) at (2.45, 1.75) {$\cdots$};
		\node [style=none] (4) at (2.75, 3.25) {};
		\node [style=none] (5) at (2, 3.25) {};
		\node [style=none] (6) at (1.5, 3.25) {};
		\node [style=none] (7) at (2.45, 3.25) {$\cdots$};
		\node [style=Z] (8) at (2, 2.5) {$\theta$};
	\end{pgfonlayer}
	\begin{pgfonlayer}{edgelayer}
		\draw [in=-90, out=45] (8) to (4.center);
		\draw (8) to (5.center);
		\draw [in=135, out=-90] (6.center) to (8);
		\draw [in=90, out=-150] (8) to (0.center);
		\draw (2.center) to (8);
		\draw [in=90, out=-30] (8) to (1.center);
	\end{pgfonlayer}
\end{tikzpicture}
$$


The normal form induces a phased spider fusion rule:

$$
\begin{tikzpicture}
	\begin{pgfonlayer}{nodelayer}
		\node [style=none] (0) at (1.5, -0.5) {};
		\node [style=none] (1) at (0.5, -0.5) {};
		\node [style=none] (2) at (1, -0.5) {$\cdots$};
		\node [style=none] (3) at (0.5, -2.75) {};
		\node [style=Z] (4) at (1, -1.25) {$\theta$};
		\node [style=none] (5) at (2, -0.5) {};
		\node [style=none] (6) at (1.5, -2.75) {$\cdots$};
		\node [style=none] (7) at (1, -2.75) {};
		\node [style=Z] (8) at (1.5, -2) {$\phi$};
		\node [style=none] (9) at (2, -2.75) {};
		\node [style=none] (10) at (1.25, -1.5) {\reflectbox{$\ddots$}};
	\end{pgfonlayer}
	\begin{pgfonlayer}{edgelayer}
		\draw [in=-124, out=90] (3.center) to (4);
		\draw [in=-90, out=56] (4) to (0.center);
		\draw [in=124, out=-90] (1.center) to (4);
		\draw [in=-124, out=90] (7.center) to (8);
		\draw [in=90, out=-56] (8) to (9.center);
		\draw [in=-90, out=56] (8) to (5.center);
		\draw (8) to (4);
	\end{pgfonlayer}
\end{tikzpicture}
=
\begin{tikzpicture}
	\begin{pgfonlayer}{nodelayer}
		\node [style=none] (11) at (4, -0.5) {};
		\node [style=none] (12) at (3, -0.5) {};
		\node [style=none] (13) at (3.5, -0.5) {$\cdots$};
		\node [style=none] (14) at (2.5, -2) {};
		\node [style=none] (15) at (3.5, -1.25) {};
		\node [style=none] (16) at (4.5, -0.5) {};
		\node [style=none] (17) at (3.5, -2) {$\cdots$};
		\node [style=none] (18) at (3, -2) {};
		\node [style=Z] (19) at (3.5, -1.25) {$\theta;\phi$};
		\node [style=none] (20) at (4, -2) {};
	\end{pgfonlayer}
	\begin{pgfonlayer}{edgelayer}
		\draw [in=-150, out=90] (14.center) to (15);
		\draw [in=-90, out=56] (15) to (11.center);
		\draw [in=124, out=-90] (12.center) to (15);
		\draw [in=-124, out=90] (18.center) to (19);
		\draw [in=90, out=-56] (19) to (20.center);
		\draw [in=-90, out=30] (19) to (16.center);
	\end{pgfonlayer}
\end{tikzpicture}
$$

This notation is compatible with the non-phased spider notation, where a spider drawn with no phase corresponds to a phased spider whose phase is the identity:

$$
\begin{tikzpicture}
	\begin{pgfonlayer}{nodelayer}
		\node [style=none] (0) at (4, -0.5) {};
		\node [style=none] (1) at (3, -0.5) {};
		\node [style=none] (2) at (3.5, -0.5) {$\cdots$};
		\node [style=none] (4) at (3.5, -1.25) {};
		\node [style=none] (6) at (3.5, -2) {$\cdots$};
		\node [style=none] (7) at (3, -2) {};
		\node [style=Z] (8) at (3.5, -1.25) {};
		\node [style=none] (9) at (4, -2) {};
	\end{pgfonlayer}
	\begin{pgfonlayer}{edgelayer}
		\draw [in=-90, out=56] (4.center) to (0.center);
		\draw [in=124, out=-90] (1.center) to (4.center);
		\draw [in=-124, out=90] (7.center) to (8);
		\draw [in=90, out=-56] (8) to (9.center);
	\end{pgfonlayer}
\end{tikzpicture}
=
\begin{tikzpicture}
	\begin{pgfonlayer}{nodelayer}
		\node [style=none] (0) at (4, -0.5) {};
		\node [style=none] (1) at (3, -0.5) {};
		\node [style=none] (2) at (3.5, -0.5) {$\cdots$};
		\node [style=none] (4) at (3.5, -1.25) {};
		\node [style=none] (6) at (3.5, -2) {$\cdots$};
		\node [style=none] (7) at (3, -2) {};
		\node [style=Z] (8) at (3.5, -1.25) {$1$};
		\node [style=none] (9) at (4, -2) {};
	\end{pgfonlayer}
	\begin{pgfonlayer}{edgelayer}
		\draw [in=-90, out=56] (4.center) to (0.center);
		\draw [in=124, out=-90] (1.center) to (4.center);
		\draw [in=-124, out=90] (7.center) to (8);
		\draw [in=90, out=-56] (8) to (9.center);
	\end{pgfonlayer}
\end{tikzpicture}
$$



\begin{definition}
\label{def:complementary}
%Interacting Hopf-Frobenius algebras/ strongly complementary observables

Two bases in $\FHilb$ are {\bf strongly complementary} when their corresponding Frobenius algebras interact to form a Hopf algebra whose antipode is equivalently any of the following maps:
$$
\begin{tikzpicture}
	\begin{pgfonlayer}{nodelayer}
		\node [style=Z] (0) at (0.5, 0) {};
		\node [style=X] (1) at (1, 0.5) {};
		\node [style=none] (2) at (0, 1) {};
		\node [style=none] (3) at (1.5, -0.5) {};
	\end{pgfonlayer}
	\begin{pgfonlayer}{edgelayer}
		\draw [in=-90, out=135] (0) to (2.center);
		\draw (0) to (1);
		\draw [in=90, out=-45] (1) to (3.center);
	\end{pgfonlayer}
\end{tikzpicture}=
\begin{tikzpicture}
	\begin{pgfonlayer}{nodelayer}
		\node [style=X] (0) at (0.5, 0) {};
		\node [style=Z] (1) at (1, 0.5) {};
		\node [style=none] (2) at (0, 1) {};
		\node [style=none] (3) at (1.5, -0.5) {};
	\end{pgfonlayer}
	\begin{pgfonlayer}{edgelayer}
		\draw [in=-90, out=135] (0) to (2.center);
		\draw (0) to (1);
		\draw [in=90, out=-45] (1) to (3.center);
	\end{pgfonlayer}
\end{tikzpicture}=
\begin{tikzpicture}
	\begin{pgfonlayer}{nodelayer}
		\node [style=Z] (0) at (1, 0) {};
		\node [style=X] (1) at (0.5, 0.5) {};
		\node [style=none] (2) at (1.5, 1) {};
		\node [style=none] (3) at (0, -0.5) {};
	\end{pgfonlayer}
	\begin{pgfonlayer}{edgelayer}
		\draw [in=-90, out=45] (0) to (2.center);
		\draw (0) to (1);
		\draw [in=90, out=-135] (1) to (3.center);
	\end{pgfonlayer}
\end{tikzpicture}=
\begin{tikzpicture}
	\begin{pgfonlayer}{nodelayer}
		\node [style=X] (0) at (1, 0) {};
		\node [style=Z] (1) at (0.5, 0.5) {};
		\node [style=none] (2) at (1.5, 1) {};
		\node [style=none] (3) at (0, -0.5) {};
	\end{pgfonlayer}
	\begin{pgfonlayer}{edgelayer}
		\draw [in=-90, out=45] (0) to (2.center);
		\draw (0) to (1);
		\draw [in=90, out=-135] (1) to (3.center);
	\end{pgfonlayer}
\end{tikzpicture}
$$

\end{definition}
%example: this can be used to construct the CNOT gate





\begin{definition}
\label{def:zx}
Given some fixed dimension $d$, the qudit ZX-calculus is a collection of related graphical calculi with faithful interpretations into $\FHilb$ generated by the phased Frobenius algebras for the standard basis and Fourier bases.

A symmetric monoidal theory is a {\bf fragment of the ZX-calculus} when it is a symmetric monoidal subtheory of the $\ZX$-calculus with a faithful interpretation in $\FHilb$.
\end{definition}


\begin{definition}
The {\bf phase-free} qudit $\ZX$-calculus
is the fragment of the ZX-calculus generated by both Frobenius algebras with trivial phases interpreted as:

$$
\left\llbracket\ 
\begin{tikzpicture}[scale=.84]
	\begin{pgfonlayer}{nodelayer}
		\node [style=none] (0) at (4, -0.5) {};
		\node [style=none] (1) at (3, -0.5) {};
		\node [style=none] (2) at (3.5, -0.75) {$\cdots$};
		\node [style=Z] (4) at (3.5, -1.25) {};
		\node [style=none] (6) at (3.5, -1.75) {$\cdots$};
		\node [style=none] (7) at (3, -2) {};
		\node [style=Z] (8) at (3.5, -1.25) {};
		\node [style=none] (9) at (4, -2) {};
		\node [style=none] (10) at (3.5, -2) {$n$};
		\node [style=none] (11) at (3.5, -0.5) {$m$};
	\end{pgfonlayer}
	\begin{pgfonlayer}{edgelayer}
		\draw [in=-90, out=56] (4) to (0.center);
		\draw [in=124, out=-90] (1.center) to (4);
		\draw [in=-124, out=90] (7.center) to (8);
		\draw [in=90, out=-56] (8) to (9.center);
	\end{pgfonlayer}
\end{tikzpicture}
\ \right\rrbracket
\propto
\sum_{i=0}^{p-1} | i, \ldots, i\rangle \langle i,\ldots, i|
\hspace*{1cm} 
\left\llbracket\ 
\begin{tikzpicture}[scale=.84]
	\begin{pgfonlayer}{nodelayer}
		\node [style=none] (0) at (4, -0.5) {};
		\node [style=none] (1) at (3, -0.5) {};
		\node [style=none] (2) at (3.5, -0.75) {$\cdots$};
		\node [style=X] (4) at (3.5, -1.25) {};
		\node [style=none] (6) at (3.5, -1.75) {$\cdots$};
		\node [style=none] (7) at (3, -2) {};
		\node [style=none] (8) at (3.5, -1.25) {};
		\node [style=none] (9) at (4, -2) {};
		\node [style=none] (10) at (3.5, -2) {$n$};
		\node [style=none] (11) at (3.5, -0.5) {$m$};
	\end{pgfonlayer}
	\begin{pgfonlayer}{edgelayer}
		\draw [in=-90, out=56] (4) to (0.center);
		\draw [in=124, out=-90] (1.center) to (4);
		\draw [in=-124, out=90] (7.center) to (8);
		\draw [in=90, out=-56] (8) to (9.center);
	\end{pgfonlayer}
\end{tikzpicture}
\ \right\rrbracket
\propto
\sum_{\sum  x_i = \sum y _j \mod p} | y_1 ,\ldots, y_n \rangle \langle  x_1,\ldots, x_n|
$$
\end{definition}


\begin{lemma}[\cite{cole}]
Given an odd prime $p$, $\LinRel_{\F_p}$ is isomorphic to the $p$-dimensional qudit phase-free ZX-calculus modulo invertible scalars.
\end{lemma}

\begin{proof}
Recall that for qudits $X^a$ is the gate that shifts the computational basis vectors by $a$ modulo $p$:
$$X^a := \sum_{b=0}^{p-1} | b+a\rangle \langle b|$$


A phase free ZX-diagram $D$ is characterized exactly by  its $X$ stabilizers, so that:
$$
\left\llbracket
D
\right\rrbracket_X
:=
\left\{ 
\left(
\begin{pmatrix}
           a_{1} \\
           \vdots \\
           a_{n}
\end{pmatrix}
,
\begin{pmatrix}
           b_{1} \\
           \vdots \\
           b_{m}
\end{pmatrix}
\right) \in \F_p^{n}\oplus\F_p^m
\ : \
\begin{tikzpicture}
	\begin{pgfonlayer}{nodelayer}
		\node [style=map] (0) at (0, 0) {$D$};
		\node [style=map] (1) at (-0.5, 0.75) {$X^{b_1}$};
		\node [style=map] (2) at (0.5, 0.75) {$X^{b_m}$};
		\node [style=map] (3) at (0.5, -0.75) {$X^{a_n}$};
		\node [style=map] (4) at (-0.5, -0.75) {$X^{a_1}$};
		\node [style=none] (5) at (-0.5, 1.25) {};
		\node [style=none] (6) at (0.5, 1.25) {};
		\node [style=none] (7) at (-0.5, -1.25) {};
		\node [style=none] (8) at (0.5, -1.25) {};
		\node [style=none] (9) at (0, 1.2) {$\cdots$};
		\node [style=none] (10) at (0, -1.2) {$\cdots$};
	\end{pgfonlayer}
	\begin{pgfonlayer}{edgelayer}
		\draw [in=-90, out=45] (0) to (2);
		\draw [in=-45, out=90] (3) to (0);
		\draw [in=-90, out=135] (0) to (1);
		\draw [in=-135, out=90] (4) to (0);
		\draw (1) to (5.center);
		\draw (7.center) to (4);
		\draw (8.center) to (3);
		\draw (2) to (6.center);
	\end{pgfonlayer}
\end{tikzpicture}
=
\begin{tikzpicture}
	\begin{pgfonlayer}{nodelayer}
		\node [style=map] (0) at (0, 0) {$D$};
		\node [style=none] (1) at (-0.5, 0.75) {};
		\node [style=none] (2) at (0.5, 0.75) {};
		\node [style=none] (3) at (0.5, -0.75) {};
		\node [style=none] (4) at (-0.5, -0.75) {};
		\node [style=none] (9) at (0, 0.5) {$\cdots$};
		\node [style=none] (10) at (0, -0.5) {$\cdots$};
	\end{pgfonlayer}
	\begin{pgfonlayer}{edgelayer}
		\draw [in=-90, out=45] (0) to (2);
		\draw [in=-45, out=90] (3) to (0);
		\draw [in=-90, out=135] (0) to (1);
		\draw [in=-135, out=90] (4) to (0);
	\end{pgfonlayer}
\end{tikzpicture}
 \right\}
$$

Conversely, given an $\F_p$-linear subspace, we obtain a phase-free ZX-diagram which is the joint $+1$-eigenstate of the corresponding $X$ stabilizers.  If we partition the codomain of the state into an input and output, bending the input wires down with the grey spider yields an inverse to the previous mapping.
\end{proof}

\begin{example}
Consider the following phase-free ZX-diagram: \ \ 
$
\begin{tikzpicture}
	\begin{pgfonlayer}{nodelayer}
		\node [style=Z] (0) at (42.75, 0.25) {};
		\node [style=X] (1) at (43.25, 0.75) {};
		\node [style=none] (2) at (42.5, -0.25) {};
		\node [style=none] (3) at (43, -0.25) {};
		\node [style=none] (4) at (43.5, -0.25) {};
		\node [style=none] (5) at (42.5, 1.25) {};
		\node [style=none] (6) at (43, 1.25) {};
		\node [style=none] (7) at (43.5, 1.25) {};
	\end{pgfonlayer}
	\begin{pgfonlayer}{edgelayer}
		\draw [in=-135, out=90] (2.center) to (0);
		\draw [in=90, out=-45] (0) to (3.center);
		\draw [in=285, out=90] (4.center) to (1);
		\draw [in=-90, out=135] (1) to (6.center);
		\draw [in=-90, out=45] (1) to (7.center);
		\draw (0) to (1);
		\draw [in=-90, out=105] (0) to (5.center);
	\end{pgfonlayer}
\end{tikzpicture}
$

Its $X$ stabilizers are parameterized by all the  $a_1,a_2,a_3,b_1,b_2,b_3 \in \F_p$ such that:
$$
\begin{tikzpicture}
	\begin{pgfonlayer}{nodelayer}
		\node [style=Z] (0) at (42.75, 0.5) {};
		\node [style=X] (1) at (43.75, 0.75) {};
		\node [style=none] (2) at (42.25, -0.25) {};
		\node [style=none] (3) at (43.25, -0.25) {};
		\node [style=none] (4) at (44.25, -0.25) {};
		\node [style=none] (5) at (42.25, 1.5) {};
		\node [style=none] (6) at (43.25, 1.5) {};
		\node [style=none] (7) at (44.25, 1.5) {};
		\node [style=none] (8) at (42.25, 2.25) {};
		\node [style=none] (9) at (43.25, 2.25) {};
		\node [style=none] (10) at (44.25, 2.25) {};
		\node [style=none] (11) at (42.25, -1) {};
		\node [style=none] (12) at (43.25, -1) {};
		\node [style=none] (13) at (44.25, -1) {};
		\node [style=map] (14) at (42.25, -0.25) {$X^{a_1}$};
		\node [style=map] (15) at (43.25, -0.25) {$X^{a_2}$};
		\node [style=map] (16) at (44.25, -0.25) {$X^{a_3}$};
		\node [style=map] (17) at (42.25, 1.5) {$X^{b_1}$};
		\node [style=map] (18) at (43.25, 1.5) {$X^{b_2}$};
		\node [style=map] (19) at (44.25, 1.5) {$X^{b_3}$};
	\end{pgfonlayer}
	\begin{pgfonlayer}{edgelayer}
		\draw [in=-135, out=90] (2.center) to (0);
		\draw [in=90, out=-45] (0) to (3.center);
		\draw [in=285, out=90] (4.center) to (1);
		\draw [in=-90, out=135] (1) to (6.center);
		\draw [in=-90, out=45] (1) to (7.center);
		\draw (0) to (1);
		\draw [in=-90, out=105] (0) to (5.center);
		\draw (5.center) to (8.center);
		\draw (6.center) to (9.center);
		\draw (7.center) to (10.center);
		\draw (13.center) to (4.center);
		\draw (12.center) to (3.center);
		\draw (11.center) to (2.center);
	\end{pgfonlayer}
\end{tikzpicture}
=
\begin{tikzpicture}
	\begin{pgfonlayer}{nodelayer}
		\node [style=Z] (0) at (42.75, 0.5) {};
		\node [style=X] (1) at (43.75, 0.75) {};
		\node [style=none] (2) at (42.25, -0.25) {};
		\node [style=none] (3) at (43.25, -0.25) {};
		\node [style=none] (4) at (44.25, -0.25) {};
		\node [style=none] (5) at (42.25, 1.5) {};
		\node [style=none] (6) at (43.25, 1.5) {};
		\node [style=none] (7) at (44.25, 1.5) {};
		\node [style=none] (8) at (42.25, 2.25) {};
		\node [style=none] (9) at (43.25, 2.25) {};
		\node [style=none] (10) at (44.25, 2.25) {};
		\node [style=none] (11) at (42.25, -1) {};
		\node [style=none] (12) at (43.25, -1) {};
		\node [style=none] (13) at (44.25, -1) {};
	\end{pgfonlayer}
	\begin{pgfonlayer}{edgelayer}
		\draw [in=-135, out=90] (2.center) to (0);
		\draw [in=90, out=-45] (0) to (3.center);
		\draw [in=285, out=90] (4.center) to (1);
		\draw [in=-90, out=135] (1) to (6.center);
		\draw [in=-90, out=45] (1) to (7.center);
		\draw (0) to (1);
		\draw [in=-90, out=105] (0) to (5.center);
		\draw (5.center) to (8.center);
		\draw (6.center) to (9.center);
		\draw (7.center) to (10.center);
		\draw (13.center) to (4.center);
		\draw (12.center) to (3.center);
		\draw (11.center) to (2.center);
	\end{pgfonlayer}
\end{tikzpicture}
$$
By labeling the wires with linear equations over $\F_p$, we can calculate these stabilizers:
$$
\begin{tikzpicture}
	\begin{pgfonlayer}{nodelayer}
		\node [style=Z] (0) at (42.75, 0.5) {};
		\node [style=X] (1) at (43.75, 0.75) {};
		\node [style=none] (2) at (42.25, -0.25) {};
		\node [style=none] (3) at (43.25, -0.25) {};
		\node [style=none] (4) at (44.25, -0.25) {};
		\node [style=none] (5) at (42.25, 1.5) {};
		\node [style=none] (6) at (43.25, 1.5) {};
		\node [style=none] (7) at (44.25, 1.5) {};
		\node [style=none] (8) at (42.25, 2.25) {};
		\node [style=none] (9) at (43.25, 2.25) {};
		\node [style=none] (10) at (44.25, 2.25) {};
		\node [style=none] (11) at (42.25, -1) {};
		\node [style=none] (12) at (43.25, -1) {};
		\node [style=none] (13) at (44.25, -1) {};
		\node [style=none] (14) at (42, -0.5) {$a_1$};
		\node [style=none] (15) at (43, -0.5) {$a_2$};
		\node [style=none] (16) at (44, -0.5) {$a_3$};
		\node [style=none] (17) at (42, 1.75) {$b_1$};
		\node [style=none] (18) at (43, 1.75) {$b_2$};
		\node [style=none] (19) at (44, 1.75) {$b_3$};
		\node [style=none] (20) at (40.75, 0.5) {$a_1=a_2=b_1$};
		\node [style=none] (21) at (45.75, 0.75) {$a_1+a_3=b_2+b_3$};
	\end{pgfonlayer}
	\begin{pgfonlayer}{edgelayer}
		\draw [in=-135, out=90] (2.center) to (0);
		\draw [in=90, out=-45] (0) to (3.center);
		\draw [in=285, out=90] (4.center) to (1);
		\draw [in=-90, out=135] (1) to (6.center);
		\draw [in=-90, out=45] (1) to (7.center);
		\draw (0) to (1);
		\draw [in=-90, out=105] (0) to (5.center);
		\draw (5.center) to (8.center);
		\draw (6.center) to (9.center);
		\draw (7.center) to (10.center);
		\draw (13.center) to (4.center);
		\draw (12.center) to (3.center);
		\draw (11.center) to (2.center);
	\end{pgfonlayer}
\end{tikzpicture}
$$
Which gives us a linear subspace of $\F_p^{3} \oplus \F_p^3$:
$$
\left\llbracket
\begin{tikzpicture}
	\begin{pgfonlayer}{nodelayer}
		\node [style=Z] (0) at (42.75, 0.25) {};
		\node [style=X] (1) at (43.25, 0.75) {};
		\node [style=none] (2) at (42.5, -0.25) {};
		\node [style=none] (3) at (43, -0.25) {};
		\node [style=none] (4) at (43.5, -0.25) {};
		\node [style=none] (5) at (42.5, 1.25) {};
		\node [style=none] (6) at (43, 1.25) {};
		\node [style=none] (7) at (43.5, 1.25) {};
	\end{pgfonlayer}
	\begin{pgfonlayer}{edgelayer}
		\draw [in=-135, out=90] (2.center) to (0);
		\draw [in=90, out=-45] (0) to (3.center);
		\draw [in=285, out=90] (4.center) to (1);
		\draw [in=-90, out=135] (1) to (6.center);
		\draw [in=-90, out=45] (1) to (7.center);
		\draw (0) to (1);
		\draw [in=-90, out=105] (0) to (5.center);
	\end{pgfonlayer}
\end{tikzpicture}
\right\rrbracket_X
=
\left\{
\left(
\begin{pmatrix}
           a_{1} \\
           a_{2} \\
           a_{3}
\end{pmatrix}
,
\begin{pmatrix}
           b_{1} \\
           b_{2} \\
           b_{3}
\end{pmatrix}
\right)
: a_1,a_2,a_3,b_1,b_2,b_3 \in \F_p,
a_1=a_2=b_1\wedge
a_1+a_3 = b_2+b_3
\right\}
$$
\end{example}



We can go add some phases to get a bit more expressiveness:
\begin{definition}
The $X$-gate fragment of the ZX-calculus is given by adjoining the $X$-gate as a generator to the phase free ZX-calculus.
\end{definition}
\begin{lemma}
$\Aff\Rel_{\F_p}$ is isomorphic to the $p$-dimensional qudit ZX-calculus with Pauli $X$ phases modulo invertible scalars.
\end{lemma}
This is given by the interpretation:
$$
\left\llbracket\ 
\begin{tikzpicture}[scale=.84]
	\begin{pgfonlayer}{nodelayer}
		\node [style=none] (0) at (4, -0.5) {};
		\node [style=none] (1) at (3, -0.5) {};
		\node [style=none] (2) at (3.5, -0.75) {$\cdots$};
		\node [style=Z] (4) at (3.5, -1.25) {};
		\node [style=none] (6) at (3.5, -1.75) {$\cdots$};
		\node [style=none] (7) at (3, -2) {};
		\node [style=Z] (8) at (3.5, -1.25) {};
		\node [style=none] (9) at (4, -2) {};
		\node [style=none] (10) at (3.5, -2) {$n$};
		\node [style=none] (11) at (3.5, -0.5) {$m$};
	\end{pgfonlayer}
	\begin{pgfonlayer}{edgelayer}
		\draw [in=-90, out=56] (4) to (0.center);
		\draw [in=124, out=-90] (1.center) to (4);
		\draw [in=-124, out=90] (7.center) to (8);
		\draw [in=90, out=-56] (8) to (9.center);
	\end{pgfonlayer}
\end{tikzpicture}
\ \right\rrbracket
\propto
\sum_{i=0}^{p-1} | i, \ldots, i\rangle \langle i,\ldots, i|
\hspace*{1cm} 
\left\llbracket\ 
\begin{tikzpicture}[scale=.84]
	\begin{pgfonlayer}{nodelayer}
		\node [style=none] (0) at (4, -0.5) {};
		\node [style=none] (1) at (3, -0.5) {};
		\node [style=none] (2) at (3.5, -0.75) {$\cdots$};
		\node [style=X] (4) at (3.5, -1.25) {$a$};
		\node [style=none] (6) at (3.5, -1.75) {$\cdots$};
		\node [style=none] (7) at (3, -2) {};
		\node [style=none] (8) at (3.5, -1.25) {};
		\node [style=none] (9) at (4, -2) {};
		\node [style=none] (10) at (3.5, -2) {$n$};
		\node [style=none] (11) at (3.5, -0.5) {$m$};
	\end{pgfonlayer}
	\begin{pgfonlayer}{edgelayer}
		\draw [in=-90, out=56] (4) to (0.center);
		\draw [in=124, out=-90] (1.center) to (4);
		\draw [in=-124, out=90] (7.center) to (8);
		\draw [in=90, out=-56] (8) to (9.center);
	\end{pgfonlayer}
\end{tikzpicture}
\ \right\rrbracket
\propto
\sum_{\sum  x_i = \sum y _j +a \mod p} | y_1 ,\ldots, y_n \rangle \langle  x_1,\ldots, x_n|
$$



The proof is almost identical to that for linear relations and phase-free ZX-diagrams.


\begin{example}
Consider the following phase-free+$X$ gate ZX-diagram:


$$
\begin{tikzpicture}
	\begin{pgfonlayer}{nodelayer}
		\node [style=Z] (0) at (42.75, 0.5) {};
		\node [style=X] (1) at (43.75, 0.75) {};
		\node [style=none] (2) at (42.25, -0.25) {};
		\node [style=none] (3) at (43.25, -0.25) {};
		\node [style=none] (4) at (44.25, -0.25) {};
		\node [style=none] (5) at (42.25, 2.25) {};
		\node [style=none] (6) at (43.25, 2.25) {};
		\node [style=none] (7) at (44.25, 2.25) {};
		\node [style=map] (8) at (43.25, 1.5) {$X^c$};
	\end{pgfonlayer}
	\begin{pgfonlayer}{edgelayer}
		\draw [in=-135, out=90] (2.center) to (0);
		\draw [in=90, out=-45] (0) to (3.center);
		\draw [in=285, out=90] (4.center) to (1);
		\draw [in=-90, out=45] (1) to (7.center);
		\draw (0) to (1);
		\draw [in=-90, out=105] (0) to (5.center);
		\draw [in=-90, out=150] (1) to (8);
		\draw (8) to (6.center);
	\end{pgfonlayer}
\end{tikzpicture}
$$

To compute the $X$ stabilizers is to find the $a_1,a_2,a_3,b_1,b_2,b_3 \in \F_p$ such that

$$
\begin{tikzpicture}
	\begin{pgfonlayer}{nodelayer}
		\node [style=Z] (0) at (42.75, 0) {};
		\node [style=X] (1) at (43.75, 0.25) {};
		\node [style=none] (2) at (42.25, -0.75) {};
		\node [style=none] (3) at (43.25, -0.75) {};
		\node [style=none] (4) at (44.25, -0.75) {};
		\node [style=none] (5) at (42.25, 1.75) {};
		\node [style=none] (6) at (43.25, 1.75) {};
		\node [style=none] (7) at (44.25, 1.75) {};
		\node [style=none] (8) at (42.25, 2.5) {};
		\node [style=none] (9) at (43.25, 2.5) {};
		\node [style=none] (10) at (44.25, 2.5) {};
		\node [style=none] (11) at (42.25, -1.5) {};
		\node [style=none] (12) at (43.25, -1.5) {};
		\node [style=none] (13) at (44.25, -1.5) {};
		\node [style=map] (14) at (42.25, -0.75) {$X^{a_1}$};
		\node [style=map] (15) at (43.25, -0.75) {$X^{a_2}$};
		\node [style=map] (16) at (44.25, -0.75) {$X^{a_3}$};
		\node [style=map] (17) at (42.25, 1.75) {$X^{b_1}$};
		\node [style=map] (18) at (43.25, 1.75) {$X^{b_2}$};
		\node [style=map] (19) at (44.25, 1.75) {$X^{b_3}$};
		\node [style=map] (20) at (43.25, 1) {$X^c$};
	\end{pgfonlayer}
	\begin{pgfonlayer}{edgelayer}
		\draw [in=-135, out=90] (2.center) to (0);
		\draw [in=90, out=-45] (0) to (3.center);
		\draw [in=285, out=90] (4.center) to (1);
		\draw [in=-90, out=45] (1) to (7.center);
		\draw (0) to (1);
		\draw [in=-90, out=105] (0) to (5.center);
		\draw (5.center) to (8.center);
		\draw (6.center) to (9.center);
		\draw (7.center) to (10.center);
		\draw (13.center) to (4.center);
		\draw (12.center) to (3.center);
		\draw (11.center) to (2.center);
		\draw [in=-90, out=150] (1) to (20);
		\draw (20) to (18);
	\end{pgfonlayer}
\end{tikzpicture}
=
\begin{tikzpicture}
	\begin{pgfonlayer}{nodelayer}
		\node [style=Z] (0) at (42.75, 0.5) {};
		\node [style=X] (1) at (43.75, 0.75) {};
		\node [style=none] (2) at (42.25, -0.25) {};
		\node [style=none] (3) at (43.25, -0.25) {};
		\node [style=none] (4) at (44.25, -0.25) {};
		\node [style=none] (5) at (42.25, 1.5) {};
		\node [style=none] (6) at (43.25, 1.5) {};
		\node [style=none] (7) at (44.25, 1.5) {};
		\node [style=none] (8) at (42.25, 2.25) {};
		\node [style=none] (9) at (43.25, 2.25) {};
		\node [style=none] (10) at (44.25, 2.25) {};
		\node [style=none] (11) at (42.25, -1) {};
		\node [style=none] (12) at (43.25, -1) {};
		\node [style=none] (13) at (44.25, -1) {};
		\node [style=map] (14) at (43.25, 1.5) {$X^c$};
	\end{pgfonlayer}
	\begin{pgfonlayer}{edgelayer}
		\draw [in=-135, out=90] (2.center) to (0);
		\draw [in=90, out=-45] (0) to (3.center);
		\draw [in=285, out=90] (4.center) to (1);
		\draw [in=-90, out=135] (1) to (6.center);
		\draw [in=-90, out=45] (1) to (7.center);
		\draw (0) to (1);
		\draw [in=-90, out=105] (0) to (5.center);
		\draw (5.center) to (8.center);
		\draw (6.center) to (9.center);
		\draw (7.center) to (10.center);
		\draw (13.center) to (4.center);
		\draw (12.center) to (3.center);
		\draw (11.center) to (2.center);
	\end{pgfonlayer}
\end{tikzpicture}
$$

In $\Aff\Rel_{\F_p}$, this equation looks like:

$$
\begin{tikzpicture}
	\begin{pgfonlayer}{nodelayer}
		\node [style=Z] (0) at (42.75, 0) {};
		\node [style=X] (1) at (43.75, 0.25) {$c$};
		\node [style=none] (2) at (42.25, -0.75) {};
		\node [style=none] (3) at (43.25, -0.75) {};
		\node [style=none] (4) at (44.25, -0.75) {};
		\node [style=none] (5) at (42.25, 1) {};
		\node [style=none] (6) at (43.25, 1) {};
		\node [style=none] (7) at (44.25, 1) {};
		\node [style=none] (8) at (42.25, 1.75) {};
		\node [style=none] (9) at (43.25, 1.75) {};
		\node [style=none] (10) at (44.25, 1.75) {};
		\node [style=none] (11) at (42.25, -1.5) {};
		\node [style=none] (12) at (43.25, -1.5) {};
		\node [style=none] (13) at (44.25, -1.5) {};
		\node [style=X] (14) at (42.25, -0.75) {$a_1$};
		\node [style=X] (15) at (43.25, -0.75) {$a_2$};
		\node [style=X] (16) at (44.25, -0.75) {$a_3$};
		\node [style=X] (17) at (42.25, 1) {$b_1$};
		\node [style=X] (18) at (43.25, 1) {$b_2$};
		\node [style=X] (19) at (44.25, 1) {$b_3$};
	\end{pgfonlayer}
	\begin{pgfonlayer}{edgelayer}
		\draw [in=-135, out=90] (2.center) to (0);
		\draw [in=90, out=-45] (0) to (3.center);
		\draw [in=-60, out=90] (4.center) to (1);
		\draw [in=-90, out=135] (1) to (6.center);
		\draw [in=-90, out=45] (1) to (7.center);
		\draw (0) to (1);
		\draw [in=-90, out=135] (0) to (5.center);
		\draw (5.center) to (8.center);
		\draw (6.center) to (9.center);
		\draw (7.center) to (10.center);
		\draw (13.center) to (4.center);
		\draw (12.center) to (3.center);
		\draw (11.center) to (2.center);
	\end{pgfonlayer}
\end{tikzpicture}
=
\begin{tikzpicture}
	\begin{pgfonlayer}{nodelayer}
		\node [style=Z] (0) at (42.75, 0) {};
		\node [style=X] (1) at (43.75, 0.25) {$c$};
		\node [style=none] (2) at (42.25, -0.75) {};
		\node [style=none] (3) at (43.25, -0.75) {};
		\node [style=none] (4) at (44.25, -0.75) {};
		\node [style=none] (5) at (42.25, 1) {};
		\node [style=none] (6) at (43.25, 1) {};
		\node [style=none] (7) at (44.25, 1) {};
		\node [style=none] (8) at (42.25, 1.75) {};
		\node [style=none] (9) at (43.25, 1.75) {};
		\node [style=none] (10) at (44.25, 1.75) {};
		\node [style=none] (11) at (42.25, -1.5) {};
		\node [style=none] (12) at (43.25, -1.5) {};
		\node [style=none] (13) at (44.25, -1.5) {};
	\end{pgfonlayer}
	\begin{pgfonlayer}{edgelayer}
		\draw [in=-135, out=90] (2.center) to (0);
		\draw [in=90, out=-45] (0) to (3.center);
		\draw [in=-45, out=90] (4.center) to (1);
		\draw [in=-90, out=135] (1) to (6.center);
		\draw [in=-90, out=45] (1) to (7.center);
		\draw (0) to (1);
		\draw [in=-90, out=135] (0) to (5.center);
		\draw (5.center) to (8.center);
		\draw (6.center) to (9.center);
		\draw (7.center) to (10.center);
		\draw (13.center) to (4.center);
		\draw (12.center) to (3.center);
		\draw (11.center) to (2.center);
	\end{pgfonlayer}
\end{tikzpicture}
$$


These $a_1,a_2,a_3,b_1,b_2,b_3$ are parameterized by the elements of the affine subspace:

$$
\begin{tikzpicture}
	\begin{pgfonlayer}{nodelayer}
		\node [style=Z] (14) at (48.75, 0.5) {};
		\node [style=X] (15) at (49.75, 0.75) {$c$};
		\node [style=none] (16) at (48.25, -0.25) {};
		\node [style=none] (17) at (49.25, -0.25) {};
		\node [style=none] (18) at (50.25, -0.25) {};
		\node [style=none] (19) at (48.25, 1.5) {};
		\node [style=none] (20) at (49.25, 1.5) {};
		\node [style=none] (21) at (50.25, 1.5) {};
		\node [style=none] (22) at (48.25, 2.25) {};
		\node [style=none] (23) at (49.25, 2.25) {};
		\node [style=none] (24) at (50.25, 2.25) {};
		\node [style=none] (25) at (48.25, -1) {};
		\node [style=none] (26) at (49.25, -1) {};
		\node [style=none] (27) at (50.25, -1) {};
		\node [style=none] (28) at (48, -0.5) {$a_1$};
		\node [style=none] (29) at (49, -0.5) {$a_2$};
		\node [style=none] (30) at (50, -0.5) {$a_3$};
		\node [style=none] (31) at (48, 1.75) {$b_1$};
		\node [style=none] (32) at (49, 1.75) {$b_2$};
		\node [style=none] (33) at (50, 1.75) {$b_3$};
		\node [style=none] (34) at (46.75, 0.75) {$a_1=a_2=b_1$};
		\node [style=none] (35) at (52, 0.75) {$a_1+a_3+c=b_2+b_3$};
	\end{pgfonlayer}
	\begin{pgfonlayer}{edgelayer}
		\draw [in=-135, out=90] (16.center) to (14);
		\draw [in=90, out=-45] (14) to (17.center);
		\draw [in=285, out=90] (18.center) to (15);
		\draw [in=-90, out=135] (15) to (20.center);
		\draw [in=-90, out=45] (15) to (21.center);
		\draw (14) to (15);
		\draw [in=-90, out=105] (14) to (19.center);
		\draw (19.center) to (22.center);
		\draw (20.center) to (23.center);
		\draw (21.center) to (24.center);
		\draw (27.center) to (18.center);
		\draw (26.center) to (17.center);
		\draw (25.center) to (16.center);
	\end{pgfonlayer}
\end{tikzpicture}
$$


So that:

$$
\left\llbracket
\begin{tikzpicture}
	\begin{pgfonlayer}{nodelayer}
		\node [style=Z] (0) at (42.75, 0.25) {};
		\node [style=X] (1) at (43.25, 0.75) {$c$};
		\node [style=none] (2) at (42.5, -0.25) {};
		\node [style=none] (3) at (43, -0.25) {};
		\node [style=none] (4) at (43.5, -0.25) {};
		\node [style=none] (5) at (42.5, 1.25) {};
		\node [style=none] (6) at (43, 1.25) {};
		\node [style=none] (7) at (43.5, 1.25) {};
	\end{pgfonlayer}
	\begin{pgfonlayer}{edgelayer}
		\draw [in=-135, out=90] (2.center) to (0);
		\draw [in=90, out=-45] (0) to (3.center);
		\draw [in=285, out=90] (4.center) to (1);
		\draw [in=-90, out=135] (1) to (6.center);
		\draw [in=-90, out=45] (1) to (7.center);
		\draw (0) to (1);
		\draw [in=-90, out=105] (0) to (5.center);
	\end{pgfonlayer}
\end{tikzpicture}
\right\rrbracket
=
\left\{
\left(
\begin{pmatrix}
           a_{1} \\
           a_{2} \\
           a_{3}
\end{pmatrix}
,
\begin{pmatrix}
           b_{1} \\
           b_{2} \\
           b_{3}
\end{pmatrix}
\right)
:
a_1=a_2=a_3\wedge
a_1+a_3+c = b_2+b_3
\right\}
$$

\end{example}







In \cite{cpm}, Selinger gives a construction to produce categories of quantum channels from general \dag-compact closed categories.  We give a definition in terms of the $\CoPara$ construction for the sake of uniformity.
\begin{definition}
\label{def:cpm}

%Dagger category... equivalent to ioo compact closed conjugation 

Given a $\dag$-compact closed category $\X$, then  $\CPM(\X)$ is the quotient of ${\CoPara}(\X)$ by the congruence relation:
$$
(f,S) \sim (g,T) \iff
\begin{tikzpicture}
	\begin{pgfonlayer}{nodelayer}
		\node [style=none] (0) at (0.75, 11.75) {};
		\node [style=none] (1) at (0.75, 10.75) {};
		\node [style=map] (3) at (0.75, 10.75) {$f$};
		\node [style=map] (4) at (0.75, 11.75) {$f^\dag$};
		\node [style=none] (5) at (0.75, 10) {};
		\node [style=none] (6) at (0.75, 12.5) {};
		\node [style=none] (7) at (0, 12.5) {};
		\node [style=none] (8) at (0, 10) {};
	\end{pgfonlayer}
	\begin{pgfonlayer}{edgelayer}
		\draw (6.center) to (4);
		\draw (4) to (3);
		\draw (3) to (5.center);
		\draw [in=-90, out=135, looseness=0.75] (3) to (7.center);
		\draw [in=-135, out=90] (8.center) to (4);
	\end{pgfonlayer}
\end{tikzpicture}
=
\begin{tikzpicture}
	\begin{pgfonlayer}{nodelayer}
		\node [style=none] (0) at (0.75, 11.75) {};
		\node [style=none] (1) at (0.75, 10.75) {};
		\node [style=map] (3) at (0.75, 10.75) {$g$};
		\node [style=map] (4) at (0.75, 11.75) {$g^\dag$};
		\node [style=none] (5) at (0.75, 10) {};
		\node [style=none] (6) at (0.75, 12.5) {};
		\node [style=none] (7) at (0, 12.5) {};
		\node [style=none] (8) at (0, 10) {};
	\end{pgfonlayer}
	\begin{pgfonlayer}{edgelayer}
		\draw (6.center) to (4);
		\draw (4) to (3);
		\draw (3) to (5.center);
		\draw [in=-90, out=135, looseness=0.75] (3) to (7.center);
		\draw [in=-135, out=90] (8.center) to (4);
	\end{pgfonlayer}
\end{tikzpicture}
$$

This category has a $\dag$-compact closed structure which is inherited from $\X$ in the obvious way.

The image of doubling functor $\X\to \CPM(\X)$ sending $f \mapsto (f,I)$ takes maps in $\X$ to {\bf pure maps}. The maps which are not pure are called {\bf mixed}. The map $d_X=((u^L_X)^{-1}, X)$ is called the {\bf discarding map} on $X$.  

All maps can be obtained by composing pure maps with discard maps.  Any such factorization is called a {\bf dilation}.
\end{definition}



\begin{example}
$\CPM(\FHilb)$ is the dagger compact closed category of density matrices between finite dimensional Hilbert spaces.
\end{example}

Density matrices model mixed quantum circuits.  The circuits in the image of the doubling functor are interpreted as the rays of pure quantum processes, unexposed to a classical system.

Oftentimes, we will refer to the maps in  $\CPM(\X)$, not in terms of the individual representatives of the equivalence class; rather, we think of them in the ``doubled picture'' where the congruence lives so that a map $(f,S)$ is drawn as:
$$
\begin{tikzpicture}
	\begin{pgfonlayer}{nodelayer}
		\node [style=none] (1) at (1.5, 10.5) {};
		\node [style=map] (3) at (1.5, 10.5) {$f$};
		\node [style=map] (4) at (2.5, 10.5) {$(f^\dag)^*$};
		\node [style=none] (5) at (1.5, 9.75) {};
		\node [style=none] (6) at (2, 11.75) {};
		\node [style=none] (7) at (1, 11.75) {};
		\node [style=none] (8) at (2.5, 9.75) {};
	\end{pgfonlayer}
	\begin{pgfonlayer}{edgelayer}
		\draw [in=135, out=-90, looseness=1.25] (6.center) to (4);
		\draw [in=75, out=405, looseness=2.00] (4) to (3);
		\draw (3) to (5.center);
		\draw [in=-90, out=135, looseness=0.75] (3) to (7.center);
		\draw (8.center) to (4);
	\end{pgfonlayer}
\end{tikzpicture}
$$
Then composition of equivalence classes is composition in $\X$.  

There are several variations on the $\CPM$ construction which are defined using different congurence relations on ${\CoPara}(\X)$, where this ``doubled picture'' doesn't make sense. 

For example, thre is an infinite dimensional version which uses a different congruence. as the quotient we have given is not a congruence relation for general $\dag$-symmetric monoidal categories \cite{cpinf}.  There is a seperate equivalence relation which is defined in terms of universally quantifying over all of the maps which fit where we have drawn the braiding.


Similarly the discard construction \cite{discard} quotients by the congruence which freely discards isometries.  All three of these constructions are isomorphic when $\X=\FHilb$.

%
%
%\begin{remark}
%We can present $\CPM(\X)$ in terms of adding generators and relations to $\X$, regarding the morphisms of $\X$ as the pure maps: adding a mixing map for each object $X$ in $\X$, with represetative $d_X=((u^L_X)^{-1}, X)$.
%
%In the case of $\X=\FHilb$, $d_X$ is interpreted as quantum discarding.
%\end{remark}


\begin{definition}
Given an orthonormal basis $B$ in $\FHilb$, the projector $p_B$ onto this basis, is first given by copying in the basis and then discarding.  In the doubled picture:
$$
\begin{tikzpicture}
	\begin{pgfonlayer}{nodelayer}
		\node [style=none] (0) at (0, 0.25) {};
		\node [style=none] (1) at (0.75, 0.25) {};
		\node [style=Z] (2) at (0, 1) {};
		\node [style=Z] (3) at (0.75, 1) {};
		\node [style=none] (4) at (-0.5, 2) {};
		\node [style=none] (5) at (0.25, 2) {};
		\node [style=Z] (12) at (0.75, 1.75) {};
	\end{pgfonlayer}
	\begin{pgfonlayer}{edgelayer}
		\draw (0.center) to (2);
		\draw [in=-90, out=150, looseness=0.75] (2) to (4.center);
		\draw (1.center) to (3);
		\draw [in=270, out=90] (3) to (5.center);
		\draw [bend right=45, looseness=1.25] (3) to (12);
		\draw [in=30, out=-150] (12) to (2);
	\end{pgfonlayer}
\end{tikzpicture}
=
\begin{tikzpicture}
	\begin{pgfonlayer}{nodelayer}
		\node [style=none] (6) at (2.25, 0.5) {};
		\node [style=Z] (8) at (2.5, 1.25) {};
		\node [style=none] (10) at (2.25, 2) {};
		\node [style=none] (13) at (2.75, 0.5) {};
		\node [style=Z] (14) at (2.5, 1.25) {};
		\node [style=none] (15) at (2.75, 2) {};
	\end{pgfonlayer}
	\begin{pgfonlayer}{edgelayer}
		\draw [in=-120, out=90] (6.center) to (8);
		\draw [in=-90, out=135] (8) to (10.center);
		\draw [in=-60, out=90] (13.center) to (14);
		\draw [in=-90, out=45] (14) to (15.center);
	\end{pgfonlayer}
\end{tikzpicture}
$$
\end{definition}

This notion of projectors leads allows us to define state preparation and measurement categorically:

\begin{definition}{\cite{idempotent}}
Take a $\dag$-compact closed category $\X$ and class of projectors $\mathcal I$ in $\X$.  Then the {\bf $\dag$-Karoubi envelope} at $\mathcal I$, ${\sf Split}_{\mathcal I}^{\dag}(\X)$, is a $\dag$-compact closed category with:
\begin{description}
\item[\ \ Objects:] Pairs $(X,e)$ where $X$ is an object of $\X$ and $e$ is a projector on $X$.
\item[\ \ Maps:] A map $(e,f,m):(X,e)\to (Y,m)$ is a map $f:X\to Y$ in $\X$ such that $e;f;m=f$.
\item[\ \ Composition:] $(e,f,m);(m,g,\ell) = (e,f;g,\ell)$.
\item[\ \ Identities:] $1_{(X,e)}=(1_X,e,1_X)$.
\item[\ \ Dagger compact structure:] Pointwise.
\end{description}



Given any class of projectors $\mathcal I$, there is a $\dag$-compact closed embedding 
$\X\to{\sf Split}_{\mathcal I}^{\dag}(\X)$.

Let ${\sf Split}^{\dag}(\X):={\sf Split}_{\mathcal I}^{\dag}(\X)$ where $I$ is the class of all projectors in $\X$.
\end{definition}

Given  any projector $e:X\to X$, then the map $(e,1_X, 1_x) :(X,e)\to (X,1_X)$ is an isometry with adjoint $(1_X,1_X, e) :(X,1_X)\to (X,e)$ .
Because the projectors now factor through the identity, they are said to be {\bf split}. Moreover, 
${\sf Split}_{{\mathcal I }\cup \{ 1_X | X \in \X\}}^{\dag}(\X)$ is said to be the category obtained by {\bf splitting the projectors in $\mathcal I$}.





\begin{remark}
In \cite{idempotent}, they show that splitting projectors in $\CPM(\FHilb)$ yields a category where the the split projectors can be interpreted as classical types: the isometry  $(e,1_X, 1_x) :(X,e)\to (X,1_X)$ is regarded as the state preparation map and its adjoint $(1_X,1_X, e) :(X,1_X)\to (X,e)$  as measurement.
\end{remark}

SINGLE PICTURE AND DOUBLED PICTURE INTERACTION

\begin{lemma}
Meauring and preparing strongly complementary observables preserves no information (actually we only need the hopf and not the bialgebra).
\end{lemma}

We can prove the correctness of quantum teleportation.

\begin{example}

\end{example}


