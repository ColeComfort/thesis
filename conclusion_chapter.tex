
In this thesis, we have given nondeterministic semantics for two classes of circuits. 


First, we gave a presentation $\ZXA$ for the  class of qubit circuits generated by the Toffoli gate, the not gate as well as the states $|0\rangle, \sqrt 2 |+\rangle$ and effects $\langle 0 |, \langle +|\sqrt{2}$.  We showed that this is isomorphic to the full subcategory of spans of finite sets (or equivalently matrices over the natural numbers) where the objects are powers of the 2 element set.  We also imposed a quotient to give a presentation for the full subcategory of relations of finite sets (or equivalently matrices over the Boolean semiring) where the objects are powers of the 2 element set.  In order to prove this, we gave a simpler characterization of the Cartesian completion of a discrete inverse category: which is one half of the equivalence between partially reversible and partial computing with copying.  We restated this construction in terms of freely adding counits to the diagonal maps; and showed how this can be related to the construction of unnormalized stochastic systems from quantum systems.  We analyzed the interaction of all of these generators and showed how larger and larger fragments of $\ZXA$ can be constructed incrementally using pushouts and distributive laws; revealing the different nondeterministic/partial structures which occur along the way.


Secondly, we investigated the structure of stabilizer circuits; and showed how quopit stabilizer circuits are isomorphic to the prop of affine coisotropic relations over $\F_p$.  To perform this, we studied  the props of (affine) (co)isotropic and Lagrangian relations using graphical linear/affine algebra and the tools of categorical quantum mechanics.  We gave generators for the each of these props and showed how over prime fields, Lagrangian relations can be presented as the CPM construction applied to linear relations with respect to the orthogonal complement.  We showed that doubling the prop of (affine) Lagrangian relations again with respect to the symplectic generalization of complex conjugation, given any base field, yields the prop of (affine) coisotropic relations.  We showed how by splitting the idempotents for the  $Z$ or $X$ projectors one obtains a two-coloured prop where the state preparation and measurement maps have elegant relational interpretations.  We showed how this gives a graphical semantics for mixed stabilizer circuits/stabilizer codes which are broadly used for quantum error correction.  Also we related this semantics of mixed stabilizer circuits to relational semantics for electrical circuits.

In both of these two examples, these props are not only monoidal categories, but there are 2-cells between the maps themselves specifying when circuits with constrained behaviour can be coherently transformed into less constrained circuits.  In the case of $\ZXA$, this comes from the fact that it is a Cartesian bicategory. On the other hand, for qupit mixed stabilizer circuits, we showed how this embeds into the Cartesian bicategory of relations $\Aff\Rel_{\F_p}$.


%There are several threads in the set forth b direction which are still open.  

In this thesis, we have regarded  circuits as subspaces respecting certain structures.  By changing the structure which is respected, and thus the notion of subspace, one therefore yields different classes of circuits.  This line of thinking leaves several threads open.

First, is to generalize $\ZXA$ to qudits; providing semantics for other full subcategories of spans and relations of finite sets/matrices over the natural numbers and Booleans. In some sense, we have given the first model of  ``graphical algebraic geometry,'' to be compared with graphical linear algebra \cite{ihpub}, graphical affine algebra \cite{affine}, graphical polyhedral algebra \cite{dpa}, graphical piecewise-linear algebra \cite{dpla}.  Note that in some sense, our graphical analysis (affine) (co)isotropic relations is also the first step towards graphical symplectic algebra.
Of course by splitting idempotents  in $\ZXA$ we would obtain presentations for the full categories  matrices over the naturals and Booleans.  The challenge is to find satisfying well-structured presentations of these categories, not an encoding of base $n$ arithmetic in base $2$.  Doing so would potentially be the first step in proving completeness for qudit fragments of the ZH-calculus.  We hinted at a potential way to solve this by computing Gr\"obner bases string-diagrammatically.  

%This way of viewing the natural-number fragments of the ZH-calculus hints at the span/relational semantics of non-prime dimensional stabilizer fragments of the  ZX-calculus.  When we presented stabilizer circuits in terms of affine Lagrangian relations, we required that the dimension be prime (or odd prime when we wanted to express phase-gates and Fourier transforms).  The requirement that the dimension be prime is needed in order for the categories of matrices and its possibly empty affine counterpart to be regular categories.  Indeed,  when a ring is not a principal ideal domain, it is not clear whether these categories are even finitely complete.  
%
%owever, just as in the qubit case $\ZXA$ is the full subcategory of spans of finite sets/matrices over the natural whose objects are powers of 2; one could do the same for qudits for any $d$.  This raises the question: what are the {\em linear} and {\em affine} full subcategories of matrices over the naturals and Booleans where the objects are powers of some natural number $d$.  There are both covariant and contravariant monoidal embeddings of $(\Aff\Mat_{\Z/d\Z},\oplus)$ / $(\Mat_{\Z/d\Z},\oplus)$ into $(\Mat_\N,\otimes)$ / $(\Mat_\B,\otimes)$.  What are presentations for the pushout of these two embeddings?  Does this give any insight in how to construct something close to linear/affine spans/relations over arbitrary rings?

We also did not give a completely satisfying investigation into the connection between the two-sided (co)unital completion of the inverse products of a discrete inverse category and Cartesian bicategories (of the span-variety, not of relations).  In general, it is natural to ask: when one takes the subcategory of partial isomorphisms of a Cartesian bicategory, when will the original Cartesian bicategory be recovered by adding units and counits to the inverse products.  Is it enough to impose that the unit and counits be adjoint to each other, or will information about the original Cartesian bicategory be lost?


In terms of stabilizer circuits there are also many unanswered questions.  First, is giving a presentation for affine coisotropic relations.  This will not be too hard to do, given the recent developments on presenting quopit stabilizer circuits, as discussed in Section \ref{sec:conclagrel}.  This is a very important thing to work out, because of the close connection to Gaussian quantum mechanics.   Even outside of the realm of quantum mechanics, giving  a relational semantics thereof will possibly shed more light on the classical connection between Gaussian probability and nondeterminism as in the work of \cite{stein}.


It would also be really interesting to augment our two-coloured semantics for stabilizer circuits and affine classical processing with stronger classical processing.  If the relational semantics for $p$-mode nonlinear classical nondeterministic circuits were combined with quopit stabilizer circuits, what sorts of quantum states could be constructed.  Is this the convex hull of stabilizer states?

Also in our categorical analysis of stabilizer codes, we have not given a graphical/categorical account of code distance, which of great practical importance in quantum error correction. 

%Finally, we have made some steps into giving universal construction of  proof nets for monoidal categories/the scalable ZX-calculus.  %The goal of this work was to attempt to find ways to glue different monoidal categories together so that one can work with string diagrams for multiple monoidal categories within the same setting.  This would be interesting not only from a quantum perspective, but from the broader perspective of a computer scientists who uses string diagrams as a tool for their work.  Although this work is far from finished, we hope that in the future it could lead to the development of more rich graphical calculi where the interfaces between two different systems, regarded as monoidal functors between monoidal categories can induce presentations for richer monoidal categories.




For the interested reader, the author has also coauthored work which was not included in this thesis, all in some broad sense studying the connection between linear logic and quantum computing \cite{2010.13361,Cockett2021,Hefford2023}.

