
\usepackage{hhline}
\usepackage{bookmark}
\usepackage[table]{xcolor}% http://ctan.org/pkg/xcolor
\usepackage[all,cmtip]{xy} 
\usepackage{float}

\usepackage{tikzit}
\input{thesis.tikzstyles}
\input{thesis.tikzdefs}

\usepackage{comment}


\usepackage{mdframed}
\usepackage{arydshln}
\usepackage{multicol}
\renewcommand{\tilde}{\widetilde}
\usepackage{everypage}
\usepackage{lipsum}
\usepackage{amsthm}
\usepackage[inline]{enumitem}   
\usepackage{scalerel,stackengine}
\stackMath
\renewcommand\hat[1]{%
\savestack{\tmpbox}{\stretchto{%
  \scaleto{%
    \scalerel*[\widthof{\ensuremath{#1}}]{\kern-.6pt\bigwedge\kern-.6pt}%
    {\rule[-\textheight/2]{1ex}{\textheight}}%WIDTH-LIMITED BIG WEDGE
  }{\textheight}% 
}{0.5ex}}%
\stackon[1pt]{#1}{\tmpbox}%
}
\parskip 1ex



\newcommand{\bcell}{\cellcolor{black!10}}

\makeatletter
% This command ignores the optional argument for itemize and enumerate lists
\newcommand{\inlineitem}[1][]{%
\ifnum\enit@type=\tw@
    {\descriptionlabel{#1}}
  \hspace{\labelsep}%
\else
  \ifnum\enit@type=\z@
       \refstepcounter{\@listctr}\fi
    \quad\@itemlabel\hspace{\labelsep}%
\fi}
\makeatother
\parindent=0pt
 

%\usepackage{extpfeil}
%\newextarrow{\xleftarrowtail}{500(40)}{\leftarrow\relbar<}
%\newextarrow{\xrightarrowtail}{500(40)}{>\relbar\rightarrow}



\makeatletter
\def\proarrowfill@#1#2#3#4#5{%
  $\m@th\thickmuskip0mu\medmuskip\thickmuskip\thinmuskip\thickmuskip
   \relax#5#1\mkern-7mu%
   \cleaders\hbox{$#5\mkern-2mu#2\mkern-2mu$}\hfill
   \mathclap{#3}\mathclap{#2}%
   \cleaders\hbox{$#5\mkern-2mu#2\mkern-2mu$}\hfill
   \mkern-7mu#4$%
}
\def\rightproarrowfill@{%
  \proarrowfill@\relbar\relbar\mapstochar\rightarrow}
\newcommand\xproarrow[2][]{%
  \ext@arrow 0055{\rightproarrowfill@}{#1}{#2}}
\makeatother

\newcommand{\proarrow}{\xproarrow{}}



%\newcommand\xrightarrowtail[2][]{\ensurestackMath{\mathrel{%
%  \stackengine{1pt}{%
 %   \stackengine{0pt}{\rightarrowtail}{\scriptstyle#2}{O}{c}{F}{F}{S}%
%  }{\scriptstyle#1}{U}{c}{F}{F}{S}%
%}}}



%\newcommand\xleftarrowtail[2][]{\ensurestackMath{\mathrel{%
%  \stackengine{1pt}{%
%    \stackengine{0pt}{\leftarrowtail}{\scriptstyle#2}{O}{c}{F}{F}{S}%
%  }{\scriptstyle#1}{U}{c}{F}{F}{S}%
%}}}


%\newcommand{\xrightarrowtail}[1]{\!\!\stackrel{#1}{\xymatrix@C=0.78em{\ar@{>->}[r]&}}\!\!\!}
%\newcommand{\xleftarrowtail}[1]{\!\!\!\stackrel{#1}{\xymatrix@C=0.78em{&\ar@{>->}[l]}}\!\!}



\newcommand{\alr}{{\sf alr}}
\newcommand{\lr}{{\sf lr}}
\newcommand{\rel}{{\sf r}}
\newcommand{\aih}{{\sf aih}}
\newcommand{\ih}{{\sf ih}}


\newcommand{\xrightarrowtail}[1]{\!\!{\xymatrix@C=1em{\ar@{>->}[r]^{#1}&}}\!\!\!}
\newcommand{\xleftarrowtail}[1]{\!\!\!{\xymatrix@C=1em{&\ar@{>->}[l]_{#1}}}\!\!}


\newcommand{\xrightarrowiso}[1]{\!\!{\xymatrix@C=1em{\ar@{->}[r]^{#1}_\cong&}}\!\!\!}
\newcommand{\xleftarrowiso}[1]{\!\!\!{\xymatrix@C=1em{&\ar@{->}[l]_{#1}^\cong}}\!\!}




%\theoremstyle{theorem} 
  \newtheorem{theorem}{Theorem}[section]
  \newtheorem{corollary}[theorem]{Corollary}
  \newtheorem{lemma}[theorem]{Lemma}
  \newtheorem{proposition}[theorem]{Proposition}
  
%\theoremstyle{definition}    conjecture
  \newtheorem{definition}[theorem]{Definition}
  \newtheorem{example}[theorem]{Example}
  \newtheorem{conjecture}[theorem]{Conjecture}
  \newtheorem{remark}[theorem]{Remark}
  
  
\newcommand{\Mat}{\mathsf{Mat}}



\newcommand{\Set}{\Sets}
\newcommand{\Mnd}{{\sf Mnd}}
\newcommand{\Map}{{\sf Map}}
\newcommand{\Monot}{{\sf Monot}}

\newcommand{\dom}{{\sf dom}}
\newcommand{\cod}{{\sf cod}}

\newcommand{\cnot}{\mathsf{cnot}}
\newcommand{\tof}{\mathsf{tof}}
\newcommand{\Not}{\mathsf{not}}
\newcommand{\zeroin}{|0\rangle}
\newcommand{\zeroout}{\langle 0|}
\newcommand{\CNOT}{\mathsf{CNOT}}
\newcommand{\Sets}{\mathsf{Set}}
\newcommand{\FSets}{\mathsf{FSet}}
\newcommand{\FinOrdMonot}{\mathsf{FinOrdMonot}}
%\newcommand{\FSet}{\mathsf{FinOrd}}
\newcommand{\FinOrd}{\mathsf{FinOrd}}
\newcommand{\FinMonot}{\mathsf{FinMonot}}
\newcommand{\Fin}{\mathsf{Fd}}
\newcommand{\TOF}{\mathsf{TOF}}
\newcommand{\Span}{\mathsf{Span}}
\newcommand{\dec}{\mathsf{dec}}
\newcommand{\Rel}{\mathsf{Rel}}
\newcommand{\FRel}{\mathsf{FRel}}
\newcommand{\op}{\mathsf{op}}
\newcommand{\co}{\mathsf{co}}
\newcommand{\Hilb}{\mathsf{Hilb}}
\newcommand{\FdHilb}{\mathsf{FHilb}}
\newcommand{\FHilb}{\mathsf{FHilb}}
\newcommand{\CPM}{\mathsf{CPM}}
\newcommand{\CP}{\mathsf{CP}}
\newcommand{\FPinj}{\mathsf{FPinj}}
\newcommand{\FPar}{\mathsf{FPar}}
\newcommand{\FSpan}{\mathsf{FSpan}}
\newcommand{\Pinj}{\mathsf{Pinj}}
\newcommand{\Par}{\mathsf{Par}}
\newcommand{\Aff}{\mathsf{Aff}}
\newcommand{\ParIso}{\mathsf{ParIso}}

\newcommand{\Total}{\mathsf{Total}}
%\newcommand{\CFrob}{\mathsf{CFrob}}
\newcommand{\tr}{\mathsf{Tr}}
\newcommand{\ox}{\otimes}
\newcommand{\Csp}{{\sf Cospan}}
\newcommand{\Corel}{{\sf Corel}}
\newcommand{\Bool}{\mathbb{B}}
\newcommand{\Iso}{{\sf Iso}}
\renewcommand{\P}{{\sf p}}
\newcommand{\pmul}{{\sf pmul}}

\newcommand{\Prof}{{\sf Prof}}
\newcommand{\Mod}{{\sf Mod}}

\newcommand{\unit}{{\sf unit}}
\newcommand{\comm}{{\sf comm}}
\newcommand{\assoc}{{\sf assoc}}
\newcommand{\inj}{{\sf Inj}}
\newcommand{\surj}{{\sf Surj}}
\newcommand{\PSurj}{{\sf PSurj}}

\newcommand{\pre}{{\sf pre}}
\newcommand{\poly}{{\sf poly}}
\newcommand{\sub}{{\sf sub}}

\newcommand{\C}{\mathbb{C}}
\newcommand{\R}{\mathbb{R}}
\newcommand{\CoPara}{{\sf CoPara}}

\newcommand{\ch}{{\sf ch}}
\newcommand{\m}{{\sf m}}
\newcommand{\cm}{{\sf cm}}
\newcommand{\cb}{{\sf cb}}
\newcommand{\pcm}{{\sf pcm}}
\renewcommand{\r}{{\sf r}}
%\newcommand{\scfrob}{{\sf scfrob}}

\newcommand{\bi}{{\sf b1}}
\newcommand{\bii}{{\sf b2}}
\newcommand{\biii}{{\sf b3}}
\newcommand{\biv}{{\sf b4}}

\newcommand{\Kl}{{\sf Kl}}
\newcommand{\Mon}{{\sf Mon}}

\newcommand{\ev}{{\sf ev}}

\renewcommand{\P}{{\sf p}}
\newcommand{\f}{\mathsf{f}}

\newcommand{\F}{\mathbb{F}}
\newcommand{\X}{\mathbb{X}}
\newcommand{\Y}{\mathbb{Y}}
\newcommand{\Z}{\mathbb{Z}}
\newcommand{\N}{\mathbb{N}}
\newcommand{\T}{\mathbb{T}}
\newcommand{\s}{\mathbb{S}}
\newcommand{\U}{\mathbb{U}}

\newcommand{\IH}{\mathbb{IH}}


\newcommand{\M}{\mathcal{M}}
\newcommand{\E}{\mathcal{E}}



\renewcommand{\sp}{\mathsf{sp}}
\newcommand{\pr}{\mathsf{p}}
\newcommand{\iso}{\mathsf{i}}







\newcounter{eq}

\makeatletter
\newcommand{\ltxlabel}{\ltx@label}
\makeatother

\newcommand{\eqzxa}[1]{%
\refstepcounter{eq}%
\ltxlabel{#1}%
\eqstack{#1}%
}




\newcommand{\eqstack}[1]{%
\stackrel{\scalebox{.6}{(\ref{#1})}}{=}%
}

\newcommand{\eq}[1]{\stackrel{\scalebox{.6}{#1}}{=}}

\newcommand{\defeq}[1]{\stackrel{\scalebox{.6}{#1}}{:=}}


\newcommand{\eref}{\eqstack}

\newcommand{\erefop}[1]{%
\stackrel{\scalebox{.6}{(\ref{#1})${}^\op$}}{=}%
}

\newcommand{\ZXA}{\mathsf{ZX}\textit{\&}}


\newcommand{\Vect}{\mathsf{Vect}}
\newcommand{\FVect}{\mathsf{FVect}}
\newcommand{\Lag}{\mathsf{Lag}}
\newcommand{\im}{\mathsf{im}}
\renewcommand{\ker}{\mathsf{ker}}
\newcommand{\ZX}{\mathsf{ZX}}
\newcommand{\ZH}{\mathsf{ZH}}
\DeclareMathSymbol{\bot}{\mathord}{symbols}{"3F}


\newcommand{\pullbackcorner}[1][dl]{\save*!/#1-1pc/#1:(-1,1)@^{|-}\restore}

\renewcommand{\epsilon}{\varepsilon}
\renewcommand{\phi}{\varphi}
\renewcommand{\bar}[1]{\overline{#1}\hspace*{.01cm}}

\newcommand{\Stab}{{\sf Stab}}
\newcommand{\LinRel}{\sf LinRel}


\newcommand{\Isot}{{\sf Isot}}
\newcommand{\Co}{{\sf Co}}

\newcommand{\B}{\mathbb{B}}

\newcommand{\STOCH}{\mathsf{STOCH}}

%\renewcommand\floatpagefraction{.9}
%\renewcommand\topfraction{.9}
%\renewcommand\bottomfraction{.9}
%\renewcommand\textfraction{.1}   
%\Setcounter{totalnumber}{50}
%\Setcounter{topnumber}{50}
%\Setcounter{bottomnumber}{50}


\usepackage{amsmath}
\usepackage{pict2e}

\newcommand{\lbparen}{\{
}

\newcommand{\rbparen}{ \}
}



\newcommand{\Cat}{{\sf Cat}}
\newcommand{\D}{\mathcal{D}}
%\newcommand{\sfa}{{\sf SFA}}
\newcommand{\one}{\mathbbm{1}}






\newdir{|>}{-<5pt,0pt>{
\begin{tikzpicture}[scale=.7]
	\begin{pgfonlayer}{nodelayer}
		\node [style=none] (0) at (0, 0) {};
		\node [style=none] (1) at (1, 0) {};
		\node [style=none] (2) at (-1, -0.25) {};
	\end{pgfonlayer}
	\begin{pgfonlayer}{edgelayer}
		\draw (2.center) to (0.center);
		\draw (0.center) to (1.center);
	\end{pgfonlayer}
\end{tikzpicture}
}}
\newdir{|<}{-<5pt,0pt>{
\begin{tikzpicture}[scale=.9]
	\begin{pgfonlayer}{nodelayer}
		\node [style=none] (0) at (0, -0.25) {};
		\node [style=none] (1) at (-1, -0.25) {};
		\node [style=none] (2) at (1, 0) {};
	\end{pgfonlayer}
	\begin{pgfonlayer}{edgelayer}
		\draw (2.center) to (0.center);
		\draw (0.center) to (1.center);
	\end{pgfonlayer}
\end{tikzpicture}
}}




\newcommand{\zcirc}{\begin{tikzpicture}
	\begin{pgfonlayer}{nodelayer}
		\node [style=Z] (0) at (0, 0) {};
	\end{pgfonlayer}
\end{tikzpicture}}

\newcommand{\xcirc}{\begin{tikzpicture}
	\begin{pgfonlayer}{nodelayer}
		\node [style=X] (0) at (0, 0) {};
	\end{pgfonlayer}
\end{tikzpicture}}



\newcommand{\skewpullbackcorner}[1][dl]{\save*!/#1-1.1pc/#1:(-.5,1)@^{|>}\restore}
\newcommand{\skewpushoutcorner}[1][dl]{\save*!/#1-1pc/#1:(-1,1)@^{|<}\restore}


\DeclareFontFamily{U}{mathx}{\hyphenchar\font45}
\DeclareFontShape{U}{mathx}{m}{n}{
      <5> <6> <7> <8> <9> <10>
      <10.95> <12> <14.4> <17.28> <20.74> <24.88>
      mathx10
      }{}
\DeclareSymbolFont{mathx}{U}{mathx}{m}{n}
\DeclareFontSubstitution{U}{mathx}{m}{n}
\DeclareMathAccent{\widecheck}{0}{mathx}{"71}
\DeclareMathAccent{\wideparen}{0}{mathx}{"75}

\def\cs#1{\texttt{\char`\\#1}}


\usepackage{amsmath}


\usepackage{hyperref}

\newcommand\numeq[2]%
  {\label{#2}\stackrel{\scriptscriptstyle(\mkern-1.5mu#1\mkern-1.5mu)}{=}}

\newcommand{\cubetopbl}{A}
\newcommand{\cubetopbr}{B}
\newcommand{\cubetopfl}{C}
\newcommand{\cubetopfr}{D}
\newcommand{\cubebotbl}{E}
\newcommand{\cubebotbr}{F}
\newcommand{\cubebotfl}{G}
\newcommand{\cubebotfr}{H}

\xymatrixrowsep{.5cm}
\xymatrixcolsep{.65cm}


\newcommand{\fa}{{\sf fa}}
\newcommand{\sfa}{{\sf sfa}}
\newcommand{\scfa}{{\sf scfa}}
\newcommand{\cfa}{{\sf cfa}}


\tikzset{meter/.append style={draw, inner sep=10, rectangle, font=\vphantom{A}, minimum width=30, line width=.8,
 path picture={\draw[black] ([shift={(.1,.3)}]path picture bounding box.south west) to[bend left=50] ([shift={(-.1,.3)}]path picture bounding box.south east);\draw[black,-latex] ([shift={(0,.1)}]path picture bounding box.south) -- ([shift={(.3,-.1)}]path picture bounding box.north);}}}



\tikzset{
  unit/.style={shape=rectangle, rounded corners,inner sep=0.3em,draw=black,fill=white, font={$I$}}
}


\usepackage{bbm}

\usepackage{Cobordism}
\pgfsetlayers{bottom,background,main,internal,foreground,label,cobordism,top,selectionbox,edgelayer,nodelayer}
%\input{styles.tikzdefs}
%\input{styles.tikzstyles}
