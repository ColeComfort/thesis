
\subsection{Monoidal categories and string diagrams}
\label{subsec:moncat}

In this subsection we review the theory of monoidal categories as well as their string diagrams.   The material for which a reference is not provided can be found in an introductory reference to category theory (eg  \cite{maclane}).  The reader uninterested in technical details is invited to skip the commutative diagrams and go straight to the pictures.

\begin{definition}
A {\bf monoidal category} is a category $\X$ equipped with a functor $\X\times\X\to \X$ called the {\bf tensor product}, equipped with a distinguished object $I$ of $\X$ called the {\bf tensor unit}; along with the following natural isomorphisms (given by components):

\begin{description}
\item[Left unitor:]
\hfil$
u_X^L:I\otimes X \to X
$
\item[Right unitor:]
\hfil$
u_X^R: X\otimes I \to X
$
\item[Associator:]
\hfil$
\alpha_{X,Y,Z}:(X\otimes Y)\otimes Z \to X\otimes(Y\otimes Z)
$
\end{description}

Satisfying the following coherence equations:

\begin{description}
\item[Mac Lane pentagon:]
$$
\xymatrix{
  ((W\otimes X )\otimes Y)\otimes Z \ar[rr]^{\alpha_{W\otimes X,Y,Z}} \ar[d]_{\alpha_{W,X,Y}\otimes 1_Z}
    &
    & (W\otimes X )\otimes (Y\otimes Z) \ar[d]^{\alpha_{W, X,Y\otimes Z}}
  \\  (W\otimes(X\otimes Y))\otimes Z \ar[dr]_{\alpha_{W,\otimes Y,Z}}
    & 
    & W\otimes (X\otimes(Y\otimes Z)) 
  \\&
    W\otimes ((X\otimes Y)\otimes Z) \ar[ur]_{1_W\otimes \alpha_{X,Y,Z}}
}
$$
\item[Unit triangle:]
$$
\xymatrix{
  (X\otimes I)\otimes Y \ar[rr]^{\alpha_{X,I,Y}} \ar[dr]_{u_X^{R}\otimes 1_Y}
    &
    & X\otimes (I\otimes Y) \ar[dl]^{1_X\otimes u_Y^{L}}\\
  & X\otimes Y
}
$$
\end{description}
We will call a map out of the tensor unit a {\bf state}, a map into the tensor unit an {\bf effect}, and an endomorphism on a the tensor unit a {\bf scalar}.
\end{definition}
\begin{example}
Both the category $\FSets$ of finite sets and functions and the category $\Sets$ of sets and functions are monoidal categories under the product and coproduct.

Given a field $k$, the category $\Vect_k$ of vector spaces over $k$  and the category $\FVect_k$ of finite-dimensional vector spaces over $k$ are monoidal with respect to the bilinear tensor product and the direct sum.

The category $\Hilb$ of Hilbert spaces and the category $\FHilb$ of finite dimensional Hilbert spaces are both monoidal categories with respect to the bilinear tensor product and direct sum.
\end{example}
\begin{definition}
Given two monoidal categories $\X$ and $\Y$ a (strong) {\bf monoidal functor} from $\X$ to $\Y$ is a functor $F:\X\to \Y$ together with an isomorphism $\epsilon:I^\X \to F(I^\X)$ and natural isomorphism with components $\mu_{X,Y}:F(X)\otimes^\Y F(Y) \to F(X \otimes^\X Y)$ such that the following coherence equations hold:
\begin{description}
\item[Interaction with associator:]
$$
\xymatrix{
 (F(X)\otimes^\Y F(Y))\otimes^\Y F(Z) \ar[rrr]^{\alpha_{F(X),F(Y),F(Z)}^{\Y}} \ar[d]_{\mu_{X,Y}\otimes^\Y F(Z)}
   &&& F(X) \otimes^\Y (F(Y)\otimes^\Y F(Z)) \ar[d]^{F(X)\otimes^\Y \mu_{Y,Z}}
 \\F(X\otimes^\X Y)\otimes^\Y F(Z) \ar[d]_{\mu_{X\otimes^\X Y,Z}}
   &&& F(X)\otimes^\Y F(Y\otimes^\X Z) \ar[d]^{\mu_{X,Y\otimes^\Y Z}}
 \\ F((X\otimes^\X Y) \otimes^\X Z) \ar[rrr]_{F(\alpha_{X,Y,Z}^\Y)}
   &&& F(X\otimes^\X (Y\otimes^\X Z))
}
$$
\item[Interaction with unitors:]
$$
\xymatrix{
 I^\Y \otimes^\Y F(X) \ar[rr]^{\epsilon \otimes^\Y F(X)} \ar[d]_{(u^L)^\Y_{F(X)}}
  &&  F(I^\X) \otimes^\Y F(X) \ar[d]^{\mu_{1^\X,X}}
\\F(X)
 & & F(I^\X\otimes^\X X) \ar[ll]^{F((u^L)^\X_{X} )}
} \hspace*{.2cm}
\xymatrix{
  F(X)  \otimes^\Y I^\Y \ar[rr]^{ F(X)  \otimes^\Y \epsilon} \ar[d]_{(u^R)^\Y_{F(X)}}
  && F(X) \otimes^\Y    F(I^\X)  \ar[d]^{\mu_{X,1^\X}}
\\F(X)
 & & F(X \otimes^\X I^\X ) \ar[ll]^{F((u^R)^\X_{X} )}
}
$$
\end{description}
A {\bf monoidal natural transformation} between parallel monoidal functors $F,G:\X\to \Y$ is a natural transformation $\phi:F\to G$ such that the following coherence equations hold:
$$
\xymatrix{
  F(X)\otimes^\Y F(Y) \ar[rr]^{\phi_X\otimes^\Y \phi_Y} \ar[d]_{\mu^F_{X,Y}}
   && G(X)\otimes^\Y G(Y) \ar[d]^{\mu^G_{X,Y}}
 \\G(X\otimes^\X Y) \ar[rr]_{\phi_{X\otimes^\X Y}}
   && G(X\otimes^\X Y)
}
\hspace*{.5cm}
\xymatrix{
I^\Y \ar[dr]^{\eta^G} \ar[d]_{\eta^F}
\\ F(I^\X) \ar[r]_{\phi_{I^\X}}
  &G(I^\X)
}
$$
Monoidal categories, monoidal functors and monoidal natural transformations arrange themselves into the strict 2-category of monoidal categories. 
\end{definition}
If all of the components of the natural transformations are equalities, then the monoidal category is {\bf strict}.  Therefore, we can forget the bracketing when we tensor things, and regard the tensor product of multiple objects as a list.
A {\bf strict monoidal functor} is a monoidal functor where isomorphisms $\epsilon$ and $\mu$ are the identity.
Likewise, strict monoidal categories, strict monoidal functors and monoidal natural transformations arrange themselves into the strict 2-category of strict monoidal categories.
\begin{example}
The category $\FinOrd$ of finite ordinals and functions is the category where:
\begin{description}
\item[Objects:] The objects are the natural numbers.
\item[Maps:]
For each natural number $n$, the finite ordinal $[n]$ is a distinguished $n$-element set with a chosen total order.

A map from $n\to m$ is a function from $[n]\to [m]$.

The composition and identity is given by the composition and identity of sets and functions.
\end{description}
This is a strict monoidal category with respect to the disjoint union:
\begin{description}
\item[Tensor unit:] The tensor unit is the natural number 0.
\item[Monoidal product:]
On objects this acts as addition.  On maps, $f:n\to m$ and $g:k\to \ell$, the map $f+g$ is corresponds to the chosen disjoint union $f\sqcup g$ $[n+k]\to [k+\ell]$ respecting the chosen order.
\end{description}
%
%The second comes from the cartesian product:
%
%\begin{description}
%\item[Tensor unit:] The tensor unit is the natural number 1.
%\item[Monoidal product:]
%On objects this acts as multiplication.  On maps, $f:n\to m$ and $g:k\to \ell$, the map $f\times g$ is corresponds to the chosen product $f\times g:[n\cdot k]\to [k\cdot \ell]$ respecting the chosen order.
%\end{description}
Restricting $\FinOrd$ to functions which preserve the chosen order of the ordinals yields the strict monoidal category $\FinOrdMonot$ of finite ordinals and monotone functions.
\end{example}

The following example comes up quite a lot, so we spell it out it detail:
\begin{example}
\label{def:mat}
Given a commutative semiring $S$, the category $\Mat_S$, of matrices over $S$ has:

\begin{description}
\item[Objects:] Natural numbers.
\item[Maps:] A map from $n\to m$ is an $m\times n$ matrix. In other words, a matrix is an element $A=(a_{i,j})_{0\leq i< m, 0\leq j \leq n}\in S^{m\times n}$.  Matrices are denoted as follows:

$$
A =
\begin{bmatrix}
a_{0,0}     & a_{0,1}     & \cdots & a_{0,n-1}\\
a_{1,0}     & a_{1,1}     & \cdots & a_{1,n-1}\\
  \vdots      & \vdots        & \ddots & \vdots \\
a_{m-1,0} & a_{m-1,1} & \cdots  & a_{m-1,n-1}
\end{bmatrix}
$$

\item[Identity:] Given by the dirac delta $I_n=(\delta_{i,j})_{0\leq i,j< n}$

\item[Composition:] Given two matrices $n\xrightarrow{A} m \xrightarrow{B} \ell$ their composite $BA$ has elements given my matrix multiplication:

$$
(AB)_{i,j}=\sum_{k=0}^{m-1} a_{i,k} b_{k,j}
$$

\end{description}


$\Mat_S$ is strict monoidal with respect to two monoidal structures.  The first one is given by:

\begin{description}
\item[Monoidal product:] Given by the direct sum, so that given any two matrices  $A,B$:
$$
A \oplus B:=
\begin{bmatrix}
A & 0\\
0 & B
\end{bmatrix}
$$
\item[Tensor unit:] 0.

\end{description}



The first one is given by:

\begin{description}
\item[Monoidal product:] Given by the Kronecker product so that given any two matrices  $A,B$:
$$
A \otimes  B:=
\begin{bmatrix}
a_{0,0}B     & a_{0,1}B     & \cdots & a_{0,n-1} B\\
a_{1,0}B     & a_{1,1}B     & \cdots & a_{1,n-1} B\\
  \vdots      & \vdots            & \ddots & \vdots \\
a_{m-1,0}B & a_{m-1,1}B & \cdots  & a_{m-1,n-1} B
\end{bmatrix}
$$

Where the notation $a_{i,j}B$ is the pointwise multiplication of $B$ by $a_{i,j}$.

\item[Tensor unit:] 1.

\end{description}

\end{example}
Strict monoidal categories are very nice to work with because they have a particularly concise graphical calculus, called {\bf string diagrams}.  A map $f:X_1\otimes \cdots \otimes X_n\to Y_1\otimes \cdots \otimes Y_m$ is drawn as a box with $n$ wires coming out of the bottom and $m$ wires coming out of the top, all being labeled by their respective objects, as follows:
$$
\begin{tikzpicture}
	\begin{pgfonlayer}{nodelayer}
		\node [style=map] (0) at (0, 3) {$f$};
		\node [style=none] (1) at (-0.75, 2.25) {};
		\node [style=none] (2) at (-0.25, 2.25) {};
		\node [style=none] (3) at (0.75, 2.25) {};
		\node [style=none] (4) at (-0.75, 2) {$X_1$};
		\node [style=none] (5) at (-0.25, 2) {$X_2$};
		\node [style=none] (6) at (0.75, 2) {$X_n$};
		\node [style=none] (7) at (0.25, 2.25) {$\cdots$};
		\node [style=none] (8) at (-0.75, 3.75) {};
		\node [style=none] (9) at (-0.25, 3.75) {};
		\node [style=none] (10) at (0.75, 3.75) {};
		\node [style=none] (11) at (-0.75, 4) {$Y_1$};
		\node [style=none] (12) at (-0.25, 4) {$Y_2$};
		\node [style=none] (13) at (0.75, 4) {$Y_m$};
		\node [style=none] (14) at (0.25, 3.75) {$\cdots$};
	\end{pgfonlayer}
	\begin{pgfonlayer}{edgelayer}
		\draw [in=90, out=-135, looseness=0.75] (0) to (1.center);
		\draw [in=255, out=90] (2.center) to (0);
		\draw [in=90, out=-45, looseness=0.75] (0) to (3.center);
		\draw [in=105, out=-90] (9.center) to (0);
		\draw [in=-90, out=135, looseness=0.75] (0) to (8.center);
		\draw [in=-90, out=45, looseness=0.75] (0) to (10.center);
	\end{pgfonlayer}
\end{tikzpicture}
$$

The identity on an object $X$ is drawn as a line:
$$
\begin{tikzpicture}
	\begin{pgfonlayer}{nodelayer}
		\node [style=none] (1) at (0.5, 0.5) {};
		\node [style=none] (3) at (0.5, 3) {};
		\node [style=none] (4) at (0.5, 0.25) {$X$};
		\node [style=none] (6) at (0.5, 3.25) {$X$};
		\node [style=map] (13) at (0.5, 1.75) {$1_X$};
	\end{pgfonlayer}
	\begin{pgfonlayer}{edgelayer}
		\draw (3.center) to (13);
		\draw (13) to (1.center);
	\end{pgfonlayer}
\end{tikzpicture}
=
\begin{tikzpicture}
	\begin{pgfonlayer}{nodelayer}
		\node [style=none] (9) at (1.5, 0.5) {};
		\node [style=none] (10) at (1.5, 3) {};
		\node [style=none] (11) at (1.5, 0.25) {$X$};
		\node [style=none] (14) at (1.5, 3.25) {$X$};
	\end{pgfonlayer}
	\begin{pgfonlayer}{edgelayer}
		\draw (10.center) to (9.center);
	\end{pgfonlayer}
\end{tikzpicture}
$$

We often omit the object labels when it is clear from context.  The tensor product of two maps
$$
\dfrac{f:W_1\otimes\cdots \otimes W_n\to X_1\otimes \cdots \otimes X_m, \hspace*{.5cm} g:Y_1\otimes\cdots \otimes Y_k\to Z_1\otimes \cdots \otimes Z_\ell}{f\otimes g:W_1\otimes\cdots \otimes W_n\otimes Y_1\otimes\cdots \otimes Y_k\to  X_1\otimes \cdots \otimes X_m\otimes Z_1\otimes \cdots \otimes Z_\ell}
$$
is drawn by pasting them side-by-side:

$$
\begin{tikzpicture}
	\begin{pgfonlayer}{nodelayer}
		\node [style=none] (79) at (20.25, 2.25) {};
		\node [style=none] (80) at (20.75, 2.25) {};
		\node [style=none] (81) at (21.75, 2.25) {};
		\node [style=none] (82) at (21.25, 2.25) {$\cdots$};
		\node [style=none] (83) at (20.25, 3.75) {};
		\node [style=none] (84) at (20.75, 3.75) {};
		\node [style=none] (85) at (21.75, 3.75) {};
		\node [style=none] (86) at (21.25, 3.75) {$\cdots$};
		\node [style=none] (87) at (22.25, 2.25) {};
		\node [style=none] (88) at (22.75, 2.25) {};
		\node [style=none] (89) at (23.75, 2.25) {};
		\node [style=none] (90) at (23.25, 2.25) {$\cdots$};
		\node [style=none] (91) at (22.25, 3.75) {};
		\node [style=none] (92) at (22.75, 3.75) {};
		\node [style=none] (93) at (23.75, 3.75) {};
		\node [style=none] (94) at (23.25, 3.75) {$\cdots$};
		\node [style=map] (95) at (22.1, 3) {\ $f\otimes g$\ \ };
		\node [style=none] (96) at (21.5, 3) {};
		\node [style=none] (97) at (21.75, 3) {};
		\node [style=none] (98) at (22, 3) {};
		\node [style=none] (99) at (22.25, 3) {};
		\node [style=none] (100) at (22.5, 3) {};
		\node [style=none] (101) at (22.75, 3) {};
		\node [style=none] (102) at (20.25, 2) {$W_1$};
		\node [style=none] (103) at (20.75, 2) {$W_2$};
		\node [style=none] (104) at (21.75, 2) {$W_m$};
		\node [style=none] (105) at (22.25, 2) {$Y_1$};
		\node [style=none] (106) at (22.75, 2) {$Y_2$};
		\node [style=none] (107) at (23.75, 2) {$Y_k$};
		\node [style=none] (108) at (20.25, 4) {$X_1$};
		\node [style=none] (109) at (20.75, 4) {$X_2$};
		\node [style=none] (110) at (21.75, 4) {$X_m$};
		\node [style=none] (111) at (22.25, 4) {$Z_1$};
		\node [style=none] (112) at (22.75, 4) {$Z_2$};
		\node [style=none] (113) at (23.75, 4) {$Z_\ell$};
	\end{pgfonlayer}
	\begin{pgfonlayer}{edgelayer}
		\draw [in=-90, out=90, looseness=0.50] (79.center) to (96.center);
		\draw [in=-90, out=90, looseness=0.50] (80.center) to (97.center);
		\draw [in=90, out=-90] (98.center) to (81.center);
		\draw (87.center) to (99.center);
		\draw [in=90, out=-90] (100.center) to (88.center);
		\draw [in=-90, out=90, looseness=0.75] (89.center) to (101.center);
		\draw [in=-90, out=90, looseness=0.75] (101.center) to (93.center);
		\draw [in=90, out=-90] (92.center) to (100.center);
		\draw (99.center) to (91.center);
		\draw [in=90, out=-90] (85.center) to (98.center);
		\draw [in=-90, out=90, looseness=0.50] (97.center) to (84.center);
		\draw [in=-90, out=90, looseness=0.50] (96.center) to (83.center);
	\end{pgfonlayer}
\end{tikzpicture}
:=
\begin{tikzpicture}
	\begin{pgfonlayer}{nodelayer}
		\node [style=map] (49) at (16.5, 3) {$f$};
		\node [style=none] (50) at (15.75, 2.25) {};
		\node [style=none] (51) at (16.25, 2.25) {};
		\node [style=none] (52) at (17.25, 2.25) {};
		\node [style=none] (53) at (16.75, 2.25) {$\cdots$};
		\node [style=none] (54) at (15.75, 3.75) {};
		\node [style=none] (55) at (16.25, 3.75) {};
		\node [style=none] (56) at (17.25, 3.75) {};
		\node [style=none] (57) at (16.75, 3.75) {$\cdots$};
		\node [style=map] (58) at (18.5, 3) {$g$};
		\node [style=none] (59) at (17.75, 2.25) {};
		\node [style=none] (60) at (18.25, 2.25) {};
		\node [style=none] (61) at (19.25, 2.25) {};
		\node [style=none] (62) at (18.75, 2.25) {$\cdots$};
		\node [style=none] (63) at (17.75, 3.75) {};
		\node [style=none] (64) at (18.25, 3.75) {};
		\node [style=none] (65) at (19.25, 3.75) {};
		\node [style=none] (66) at (18.75, 3.75) {$\cdots$};
		\node [style=none] (67) at (15.75, 4) {$X_1$};
		\node [style=none] (68) at (16.25, 4) {$X_2$};
		\node [style=none] (69) at (17.25, 4) {$X_m$};
		\node [style=none] (70) at (15.75, 2) {$W_1$};
		\node [style=none] (71) at (16.25, 2) {$W_2$};
		\node [style=none] (72) at (17.25, 2) {$W_m$};
		\node [style=none] (73) at (17.75, 2) {$Y_1$};
		\node [style=none] (74) at (18.25, 2) {$Y_2$};
		\node [style=none] (75) at (19.25, 2) {$Y_k$};
		\node [style=none] (76) at (17.75, 4) {$Z_1$};
		\node [style=none] (77) at (18.25, 4) {$Z_2$};
		\node [style=none] (78) at (19.25, 4) {$Z_\ell$};
	\end{pgfonlayer}
	\begin{pgfonlayer}{edgelayer}
		\draw [in=90, out=-135, looseness=0.75] (49) to (50.center);
		\draw [in=255, out=90] (51.center) to (49);
		\draw [in=90, out=-45, looseness=0.75] (49) to (52.center);
		\draw [in=105, out=-90] (55.center) to (49);
		\draw [in=-90, out=135, looseness=0.75] (49) to (54.center);
		\draw [in=-90, out=45, looseness=0.75] (49) to (56.center);
		\draw [in=90, out=-135, looseness=0.75] (58) to (59.center);
		\draw [in=255, out=90] (60.center) to (58);
		\draw [in=90, out=-45, looseness=0.75] (58) to (61.center);
		\draw [in=105, out=-90] (64.center) to (58);
		\draw [in=-90, out=135, looseness=0.75] (58) to (63.center);
		\draw [in=-90, out=45, looseness=0.75] (58) to (65.center);
	\end{pgfonlayer}
\end{tikzpicture}
$$

And the composite of two  composable maps
$$
\dfrac{
f:X_1\otimes \cdots \otimes X_n\to Y_1\otimes \cdots \otimes Y_m,\hspace*{.5cm}g:Y_1\otimes\cdots \otimes Y_m\to Z_1\otimes \cdots \otimes Z_k}
{f;g:X_1\otimes \cdots \otimes X_n\to Z_1\otimes \cdots \otimes Z_k}
$$ 
is drawn by connecting each of  the $Z_i$ wires together:
$$
\begin{tikzpicture}
	\begin{pgfonlayer}{nodelayer}
		\node [style=none] (34) at (13.25, 2) {$X_1$};
		\node [style=none] (35) at (13.75, 2) {$X_2$};
		\node [style=none] (36) at (14.75, 2) {$X_n$};
		\node [style=none] (37) at (13.25, 4) {$Z_1$};
		\node [style=none] (38) at (13.75, 4) {$Z_2$};
		\node [style=none] (39) at (14.75, 4) {$Z_k$};
		\node [style=map] (40) at (14, 3) {$f;g$};
		\node [style=none] (41) at (13.25, 2.25) {};
		\node [style=none] (42) at (13.75, 2.25) {};
		\node [style=none] (43) at (14.75, 2.25) {};
		\node [style=none] (44) at (14.25, 2.25) {$\cdots$};
		\node [style=none] (45) at (13.25, 3.75) {};
		\node [style=none] (46) at (13.75, 3.75) {};
		\node [style=none] (47) at (14.75, 3.75) {};
		\node [style=none] (48) at (14.25, 3.75) {$\cdots$};
	\end{pgfonlayer}
	\begin{pgfonlayer}{edgelayer}
		\draw [in=90, out=-135, looseness=0.75] (40) to (41.center);
		\draw [in=255, out=90] (42.center) to (40);
		\draw [in=90, out=-45, looseness=0.75] (40) to (43.center);
		\draw [in=105, out=-90] (46.center) to (40);
		\draw [in=-90, out=135, looseness=0.75] (40) to (45.center);
		\draw [in=-90, out=45, looseness=0.75] (40) to (47.center);
	\end{pgfonlayer}
\end{tikzpicture}
=
\begin{tikzpicture}
	\begin{pgfonlayer}{nodelayer}
		\node [style=map] (23) at (11.5, 1.75) {$f$};
		\node [style=none] (24) at (10.75, 1) {};
		\node [style=none] (25) at (11.25, 1) {};
		\node [style=none] (26) at (12.25, 1) {};
		\node [style=none] (27) at (11.75, 1) {$\cdots$};
		\node [style=map] (28) at (11.5, 2.75) {$g$};
		\node [style=none] (29) at (10.75, 3.5) {};
		\node [style=none] (30) at (11.25, 3.5) {};
		\node [style=none] (31) at (12.25, 3.5) {};
		\node [style=none] (32) at (11.75, 3.5) {$\cdots$};
		\node [style=none] (33) at (11.7, 2.25) {$\cdots$};
		\node [style=none] (34) at (10.75, 0.75) {$X_1$};
		\node [style=none] (35) at (11.25, 0.75) {$X_2$};
		\node [style=none] (36) at (12.25, 0.75) {$X_n$};
		\node [style=none] (37) at (10.75, 3.75) {$Z_1$};
		\node [style=none] (38) at (11.25, 3.75) {$Z_2$};
		\node [style=none] (39) at (12.25, 3.75) {$Z_k$};
	\end{pgfonlayer}
	\begin{pgfonlayer}{edgelayer}
		\draw [in=90, out=-135, looseness=0.75] (23) to (24.center);
		\draw [in=255, out=90] (25.center) to (23);
		\draw [in=90, out=-45, looseness=0.75] (23) to (26.center);
		\draw [in=105, out=-90] (30.center) to (28);
		\draw [in=-90, out=135, looseness=0.75] (28) to (29.center);
		\draw [in=-90, out=45, looseness=0.75] (28) to (31.center);
		\draw [in=105, out=-105] (28) to (23);
		\draw [in=225, out=135, looseness=1.25] (23) to (28);
		\draw [in=30, out=-30, looseness=1.50] (28) to (23);
	\end{pgfonlayer}
\end{tikzpicture}
$$

The axioms of a strict monoidal category are equivalent to planar isotopy of their string diagrams.  In other words, the string diagrams can be continuously deformed as long as they don't cross over each other. For example, the functoriality of the tensor product allows one to exchange two disconnected maps:
$$
\begin{tikzpicture}
	\begin{pgfonlayer}{nodelayer}
		\node [style=none] (22) at (1, 5) {};
		\node [style=none] (23) at (0, 5) {};
		\node [style=none] (37) at (0, 3) {};
		\node [style=none] (38) at (1, 3) {};
		\node [style=map] (39) at (0, 3.75) {$f$};
		\node [style=map] (40) at (1, 4.25) {$g$};
	\end{pgfonlayer}
	\begin{pgfonlayer}{edgelayer}
		\draw (37.center) to (39);
		\draw (39) to (23.center);
		\draw (38.center) to (40);
		\draw (40) to (22.center);
	\end{pgfonlayer}
\end{tikzpicture}
=
\begin{tikzpicture}
	\begin{pgfonlayer}{nodelayer}
		\node [style=none] (41) at (3, 5) {};
		\node [style=none] (42) at (2, 5) {};
		\node [style=none] (43) at (2, 3) {};
		\node [style=none] (44) at (3, 3) {};
		\node [style=map] (45) at (2, 4.25) {$f$};
		\node [style=map] (46) at (3, 3.75) {$g$};
	\end{pgfonlayer}
	\begin{pgfonlayer}{edgelayer}
		\draw (43.center) to (45);
		\draw (45) to (42.center);
		\draw (44.center) to (46);
		\draw (46) to (41.center);
	\end{pgfonlayer}
\end{tikzpicture}
$$

We can always chose to work with strict monoidal categories if we want to:
\begin{theorem}
Every monoidal category is monoidally equivalent to a strict monoidal category. 
\end{theorem}
The strictification of a monoidal category has a particularly succinct presentation due to \cite{wilson}, so that we can use string diagrams for strict monoidal categories to reason about nonstrict monoidal categories:
\begin{definition}
\label{def:proofnets}
Given a monoidal category $(\X,\otimes,I,\alpha,u^L,u^R)$, there is a monoidally equivalent strict monoidal category $\bar \X$ with:
\begin{description}
\item[Objects:] Finite lists of objects in $\X$, $\List(\Ob_\X)$.
\item[Maps:] The maps are generated by a map $f:[X]\to [Y]$ for every map $f:X\to Y$ in $\X$. As well as the four following generators (referred to as tensor, cotensor, unit introduction and unit removal):
$$
\begin{tikzpicture}
	\begin{pgfonlayer}{nodelayer}
		\node [style=none] (0) at (1.5, 3.5) {};
		\node [style=none] (1) at (0.5, 3.5) {};
		\node [style=none] (2) at (1, 4.25) {};
		\node [style=none] (3) at (1, 5) {};
		\node [style=none] (4) at (0.5, 3.25) {$X$};
		\node [style=none] (5) at (1.5, 3.25) {$Y$};
		\node [style=none] (6) at (1, 5.25) {$X\otimes Y$};
		\node [style=otimes] (20) at (1, 4.25) {};
	\end{pgfonlayer}
	\begin{pgfonlayer}{edgelayer}
		\draw (3.center) to (2);
		\draw [in=90, out=-30] (2) to (0.center);
		\draw [in=90, out=-150] (2) to (1.center);
	\end{pgfonlayer}
\end{tikzpicture}
\,\hspace{.5cm}
\begin{tikzpicture}
	\begin{pgfonlayer}{nodelayer}
		\node [style=none] (0) at (1.5, 5) {};
		\node [style=none] (1) at (0.5, 5) {};
		\node [style=none] (2) at (1, 4.25) {};
		\node [style=none] (3) at (1, 3.5) {};
		\node [style=none] (4) at (0.5, 5.25) {$X$};
		\node [style=none] (5) at (1.5, 5.25) {$Y$};
		\node [style=none] (6) at (1, 3.25) {$X\otimes Y$};
		\node [style=otimes] (20) at (1, 4.25) {};
	\end{pgfonlayer}
	\begin{pgfonlayer}{edgelayer}
		\draw (3.center) to (2);
		\draw [in=-90, out=30] (2) to (0.center);
		\draw [in=-90, out=150] (2) to (1.center);
	\end{pgfonlayer}
\end{tikzpicture}
\,\hspace{.5cm}
\begin{tikzpicture}
	\begin{pgfonlayer}{nodelayer}
		\node [style=unit] (2) at (1, 3.25) {};
		\node [style=none] (3) at (1, 4) {};
		\node [style=none] (6) at (1, 4.25) {$I$};
	\end{pgfonlayer}
	\begin{pgfonlayer}{edgelayer}
		\draw (3.center) to (2);
	\end{pgfonlayer}
\end{tikzpicture}
\,\hspace{.5cm}
\begin{tikzpicture}
	\begin{pgfonlayer}{nodelayer}
		\node [style=unit] (2) at (1, 4.25) {};
		\node [style=none] (3) at (1, 3.5) {};
		\node [style=none] (6) at (1, 3.25) {$I$};
	\end{pgfonlayer}
	\begin{pgfonlayer}{edgelayer}
		\draw (3.center) to (2);
	\end{pgfonlayer}
\end{tikzpicture}
$$
\item[Modulo the equations:] 


For all $f:X\to Y$ and $g:Y\to Z$ in $\X$:
$$
\begin{tikzpicture}
	\begin{pgfonlayer}{nodelayer}
		\node [style=none] (1) at (0.5, 0.5) {};
		\node [style=none] (3) at (0.5, 3) {};
		\node [style=none] (4) at (0.5, 0.25) {$X$};
		\node [style=none] (6) at (0.5, 3.25) {$Z$};
		\node [style=map] (7) at (0.5, 1.25) {$f$};
		\node [style=map] (8) at (0.5, 2.25) {$g$};
	\end{pgfonlayer}
	\begin{pgfonlayer}{edgelayer}
		\draw (3.center) to (8);
		\draw (8) to (7);
		\draw (7) to (1.center);
	\end{pgfonlayer}
\end{tikzpicture}
=
\begin{tikzpicture}
	\begin{pgfonlayer}{nodelayer}
		\node [style=none] (9) at (1.5, 0.5) {};
		\node [style=none] (10) at (1.5, 3) {};
		\node [style=none] (11) at (1.5, 0.25) {$X$};
		\node [style=none] (12) at (1.5, 3.25) {$Z$};
		\node [style=map] (13) at (1.5, 1.75) {$f;g$};
	\end{pgfonlayer}
	\begin{pgfonlayer}{edgelayer}
		\draw (13) to (9.center);
		\draw (13) to (10.center);
	\end{pgfonlayer}
\end{tikzpicture}\, 
\hspace*{1cm}
\begin{tikzpicture}
	\begin{pgfonlayer}{nodelayer}
		\node [style=none] (1) at (0.5, 0.5) {};
		\node [style=none] (3) at (0.5, 3) {};
		\node [style=none] (4) at (0.5, 0.25) {$X$};
		\node [style=none] (6) at (0.5, 3.25) {$X$};
		\node [style=map] (13) at (0.5, 1.75) {$1_X$};
	\end{pgfonlayer}
	\begin{pgfonlayer}{edgelayer}
		\draw (3.center) to (13);
		\draw (13) to (1.center);
	\end{pgfonlayer}
\end{tikzpicture}
=
\begin{tikzpicture}
	\begin{pgfonlayer}{nodelayer}
		\node [style=none] (9) at (1.5, 0.5) {};
		\node [style=none] (10) at (1.5, 3) {};
		\node [style=none] (11) at (1.5, 0.25) {$X$};
		\node [style=none] (14) at (1.5, 3.25) {$X$};
	\end{pgfonlayer}
	\begin{pgfonlayer}{edgelayer}
		\draw (10.center) to (9.center);
	\end{pgfonlayer}
\end{tikzpicture}
$$
And for all $f:W\to X$ and $g:Y\to Z$ in $\X$:
$$
\begin{tikzpicture}
	\begin{pgfonlayer}{nodelayer}
		\node [style=otimes] (0) at (4, 5.25) {};
		\node [style=none] (1) at (4, 6) {};
		\node [style=otimes] (2) at (4, 3.75) {};
		\node [style=none] (3) at (4, 3) {};
		\node [style=none] (4) at (4, 5.25) {};
		\node [style=none] (5) at (4, 3.75) {};
		\node [style=map] (6) at (3.5, 4.5) {$ f$};
		\node [style=map] (7) at (4.5, 4.5) {$ g$};
		\node [style=none] (8) at (4, 6.25) {$Y\otimes Z$};
		\node [style=none] (9) at (4, 2.75) {$X\otimes Y$};
	\end{pgfonlayer}
	\begin{pgfonlayer}{edgelayer}
		\draw (1.center) to (0);
		\draw (3.center) to (2);
		\draw [in=90, out=-150] (4.center) to (6);
		\draw [in=150, out=-90] (6) to (5.center);
		\draw [in=-90, out=30] (5.center) to (7);
		\draw [in=-30, out=90] (7) to (4.center);
	\end{pgfonlayer}
\end{tikzpicture}
=
\begin{tikzpicture}
	\begin{pgfonlayer}{nodelayer}
		\node [style=none] (1) at (4, 6) {};
		\node [style=none] (3) at (4, 3) {};
		\node [style=map] (6) at (4, 4.5) {$ {f\otimes g}$};
		\node [style=none] (8) at (4, 6.25) {$Y\otimes Z$};
		\node [style=none] (9) at (4, 2.75) {$X\otimes Y$};
	\end{pgfonlayer}
	\begin{pgfonlayer}{edgelayer}
		\draw (1.center) to (6);
		\draw (6) to (3.center);
	\end{pgfonlayer}
\end{tikzpicture}
\ ,\hspace{.5cm}
\begin{tikzpicture}
	\begin{pgfonlayer}{nodelayer}
		\node [style=none] (0) at (1.5, 3.5) {};
		\node [style=none] (1) at (0.5, 3.5) {};
		\node [style=none] (2) at (1, 4.25) {};
		\node [style=none] (3) at (1, 5) {};
		\node [style=none] (4) at (1.5, 5.75) {};
		\node [style=none] (5) at (0.5, 5.75) {};
		\node [style=none] (6) at (1, 5) {};
		\node [style=otimes] (7) at (1, 4.25) {};
		\node [style=otimes] (8) at (1, 5) {};
		\node [style=none] (9) at (0.5, 3.25) {$X$};
		\node [style=none] (10) at (1.5, 3.25) {$Y$};
		\node [style=none] (11) at (0.5, 6) {$X$};
		\node [style=none] (12) at (1.5, 6) {$Y$};
	\end{pgfonlayer}
	\begin{pgfonlayer}{edgelayer}
		\draw (3.center) to (2.center);
		\draw [in=90, out=-30] (2.center) to (0.center);
		\draw [in=90, out=-150] (2.center) to (1.center);
		\draw [in=-90, out=30] (6.center) to (4.center);
		\draw [in=-90, out=150] (6.center) to (5.center);
	\end{pgfonlayer}
\end{tikzpicture}
=
\begin{tikzpicture}
	\begin{pgfonlayer}{nodelayer}
		\node [style=none] (0) at (1.5, 0.5) {};
		\node [style=none] (1) at (0.5, 0.5) {};
		\node [style=none] (2) at (1.5, 2.75) {};
		\node [style=none] (3) at (0.5, 2.75) {};
		\node [style=none] (4) at (0.5, 0.25) {$X$};
		\node [style=none] (5) at (1.5, 0.25) {$Y$};
		\node [style=none] (6) at (0.5, 3) {$X$};
		\node [style=none] (7) at (1.5, 3) {$Y$};
	\end{pgfonlayer}
	\begin{pgfonlayer}{edgelayer}
		\draw (0.center) to (2.center);
		\draw (1.center) to (3.center);
	\end{pgfonlayer}
\end{tikzpicture}
\ ,\hspace{.5cm}
\begin{tikzpicture}
	\begin{pgfonlayer}{nodelayer}
		\node [style=unit] (0) at (4, 5.25) {};
		\node [style=none] (1) at (4, 6) {};
		\node [style=unit] (2) at (4, 4.25) {};
		\node [style=none] (3) at (4, 3.5) {};
		\node [style=none] (4) at (4, 5.25) {};
		\node [style=none] (5) at (4, 4.25) {};
		\node [style=none] (6) at (4, 6.25) {$I$};
		\node [style=none] (7) at (4, 3.25) {$I$};
	\end{pgfonlayer}
	\begin{pgfonlayer}{edgelayer}
		\draw (1.center) to (0);
		\draw (3.center) to (2);
	\end{pgfonlayer}
\end{tikzpicture}
=
\begin{tikzpicture}
	\begin{pgfonlayer}{nodelayer}
		\node [style=none] (8) at (4, 2.75) {};
		\node [style=none] (9) at (4, 0.25) {};
		\node [style=none] (10) at (4, 3) {$I$};
		\node [style=none] (11) at (4, 0) {$I$};
	\end{pgfonlayer}
	\begin{pgfonlayer}{edgelayer}
		\draw (9.center) to (8.center);
	\end{pgfonlayer}
\end{tikzpicture}
\ ,\hspace{.5cm}
\begin{tikzpicture}
	\begin{pgfonlayer}{nodelayer}
		\node [style=unit] (2) at (1, 4.25) {};
		\node [style=none] (3) at (1, 5) {};
		\node [style=unit] (9) at (1, 5) {};
	\end{pgfonlayer}
	\begin{pgfonlayer}{edgelayer}
		\draw (3.center) to (2);
	\end{pgfonlayer}
\end{tikzpicture}
=
\begin{tikzpicture}
	\begin{pgfonlayer}{nodelayer}
		\node [style=none] (0) at (2, 0) {};
		\node [style=none] (1) at (2, -1) {};
		\node [style=none] (2) at (3, -1) {};
		\node [style=none] (3) at (3, 0) {};
	\end{pgfonlayer}
	\begin{pgfonlayer}{edgelayer}
		\draw[style=dashed] (3.center) to (0.center) to (1.center) to (2.center) to cycle;
	\end{pgfonlayer}
\end{tikzpicture}
$$
$$
\begin{tikzpicture}
	\begin{pgfonlayer}{nodelayer}
		\node [style=none] (3) at (8, 3.5) {};
		\node [style=none] (4) at (8, 3.25) {$(X\otimes Y)\otimes Z$};
		\node [style=none] (9) at (8, 7.5) {$X\otimes (Y\otimes Z)$};
		\node [style=none] (14) at (8, 7.25) {};
		\node [style=map] (15) at (8, 5.25) {$\alpha_{X,Y,Z}$};
	\end{pgfonlayer}
	\begin{pgfonlayer}{edgelayer}
		\draw (15) to (14.center);
		\draw (15) to (3.center);
	\end{pgfonlayer}
\end{tikzpicture}
=
\begin{tikzpicture}
	\begin{pgfonlayer}{nodelayer}
		\node [style=none] (11) at (8.5, 5) {};
		\node [style=none] (12) at (7.5, 5) {};
		\node [style=none] (13) at (8, 4.25) {};
		\node [style=none] (14) at (8, 3.5) {};
		\node [style=none] (17) at (8, 3.25) {$(X\otimes Y)\otimes Z$};
		\node [style=otimes] (18) at (8, 4.25) {};
		\node [style=none] (19) at (8, 5.75) {};
		\node [style=none] (20) at (7, 5.75) {};
		\node [style=none] (21) at (7.5, 5) {};
		\node [style=none] (23) at (7.5, 7.5) {$X\otimes (Y\otimes Z)$};
		\node [style=otimes] (24) at (7.5, 5) {};
		\node [style=none] (25) at (7, 5.75) {};
		\node [style=none] (26) at (8, 5.75) {};
		\node [style=none] (27) at (7.5, 6.5) {};
		\node [style=none] (28) at (7.5, 7.25) {};
		\node [style=otimes] (29) at (7.5, 6.5) {};
		\node [style=none] (30) at (7.5, 5) {};
		\node [style=none] (31) at (8.5, 5) {};
		\node [style=none] (32) at (8, 5.75) {};
		\node [style=otimes] (33) at (8, 5.75) {};
	\end{pgfonlayer}
	\begin{pgfonlayer}{edgelayer}
		\draw (14.center) to (13.center);
		\draw [in=-90, out=30] (13.center) to (11.center);
		\draw [in=-90, out=150] (13.center) to (12.center);
		\draw [in=-90, out=150] (21.center) to (20.center);
		\draw (28.center) to (27.center);
		\draw [in=90, out=-150] (27.center) to (25.center);
		\draw [in=90, out=-30] (27.center) to (26.center);
		\draw (32.center) to (30.center);
		\draw [in=90, out=-30] (32.center) to (31.center);
	\end{pgfonlayer}
\end{tikzpicture}
\ ,
\hspace*{.5cm}
\begin{tikzpicture}
	\begin{pgfonlayer}{nodelayer}
		\node [style=none] (17) at (11.75, 2) {};
		\node [style=none] (19) at (11.75, 0.5) {};
		\node [style=none] (20) at (11.75, 2.25) {$X$};
		\node [style=none] (21) at (11.75, 0.25) {$I\otimes X$};
		\node [style=map] (22) at (11.75, 1.25) {$u_X^L$};
	\end{pgfonlayer}
	\begin{pgfonlayer}{edgelayer}
		\draw (17.center) to (22);
		\draw (19.center) to (22);
	\end{pgfonlayer}
\end{tikzpicture}
=
 \begin{tikzpicture}
	\begin{pgfonlayer}{nodelayer}
		\node [style=none] (79) at (11.25, 2) {};
		\node [style=none] (80) at (12.25, 2) {};
		\node [style=none] (81) at (11.75, 1.25) {};
		\node [style=none] (82) at (11.75, 0.5) {};
		\node [style=none] (83) at (12.25, 2.25) {$X$};
		\node [style=none] (84) at (11.75, 0.25) {$I\otimes X$};
		\node [style=otimes] (85) at (11.75, 1.25) {};
		\node [style=unit] (86) at (11.25, 2) {};
	\end{pgfonlayer}
	\begin{pgfonlayer}{edgelayer}
		\draw (82.center) to (81.center);
		\draw [in=-90, out=150] (81.center) to (79.center);
		\draw [in=-90, out=30] (81.center) to (80.center);
	\end{pgfonlayer}
\end{tikzpicture}
\ ,
\hspace*{.5cm}
\begin{tikzpicture}
	\begin{pgfonlayer}{nodelayer}
		\node [style=none] (17) at (11.75, 2) {};
		\node [style=none] (19) at (11.75, 0.5) {};
		\node [style=none] (20) at (11.75, 2.25) {$X$};
		\node [style=none] (21) at (11.75, 0.25) {$ X\otimes I$};
		\node [style=map] (22) at (11.75, 1.25) {$u_X^R$};
	\end{pgfonlayer}
	\begin{pgfonlayer}{edgelayer}
		\draw (17.center) to (22);
		\draw (19.center) to (22);
	\end{pgfonlayer}
\end{tikzpicture}
=
\begin{tikzpicture}
	\begin{pgfonlayer}{nodelayer}
		\node [style=none] (71) at (12.25, -0.75) {};
		\node [style=none] (72) at (11.25, -0.75) {};
		\node [style=none] (73) at (11.75, -1.5) {};
		\node [style=none] (74) at (11.75, -2.25) {};
		\node [style=none] (75) at (11.25, -0.5) {$X$};
		\node [style=none] (76) at (11.75, -2.5) {$X\otimes I$};
		\node [style=otimes] (77) at (11.75, -1.5) {};
		\node [style=unit] (78) at (12.25, -0.75) {};
	\end{pgfonlayer}
	\begin{pgfonlayer}{edgelayer}
		\draw (74.center) to (73.center);
		\draw [in=-90, out=30] (73.center) to (71.center);
		\draw [in=-90, out=150] (73.center) to (72.center);
	\end{pgfonlayer}
\end{tikzpicture}
$$
\item[Composition:] Vertical pasting:
$$
\begin{tikzpicture}
	\begin{pgfonlayer}{nodelayer}
		\node [style=none] (0) at (4, 5.25) {};
		\node [style=none] (1) at (4, 3.75) {};
		\node [style=map] (2) at (4, 4.5) {$f$};
		\node [style=none] (3) at (4, 5.5) {$Y$};
		\node [style=none] (4) at (4, 3.5) {$X$};
	\end{pgfonlayer}
	\begin{pgfonlayer}{edgelayer}
		\draw (0.center) to (2);
		\draw (2) to (1.center);
	\end{pgfonlayer}
\end{tikzpicture}
;
\begin{tikzpicture}
	\begin{pgfonlayer}{nodelayer}
		\node [style=none] (0) at (4, 5.25) {};
		\node [style=none] (1) at (4, 3.75) {};
		\node [style=map] (2) at (4, 4.5) {$g$};
		\node [style=none] (3) at (4, 5.5) {$Z$};
		\node [style=none] (4) at (4, 3.5) {$Y$};
	\end{pgfonlayer}
	\begin{pgfonlayer}{edgelayer}
		\draw (0.center) to (2);
		\draw (2) to (1.center);
	\end{pgfonlayer}
\end{tikzpicture}
:=
\begin{tikzpicture}
	\begin{pgfonlayer}{nodelayer}
		\node [style=none] (1) at (4, 3.75) {};
		\node [style=map] (2) at (4, 4.5) {$f$};
		\node [style=none] (4) at (4, 3.5) {$X$};
		\node [style=none] (5) at (4, 6.25) {};
		\node [style=map] (7) at (4, 5.5) {$g$};
		\node [style=none] (8) at (4, 6.5) {$Z$};
	\end{pgfonlayer}
	\begin{pgfonlayer}{edgelayer}
		\draw (2) to (1.center);
		\draw (5.center) to (7);
		\draw (2) to (7);
	\end{pgfonlayer}
\end{tikzpicture}
$$

\item[Tensor product:] Horizontal pasting:
$$
\begin{tikzpicture}
	\begin{pgfonlayer}{nodelayer}
		\node [style=none] (0) at (4, 5.25) {};
		\node [style=none] (1) at (4, 3.75) {};
		\node [style=map] (2) at (4, 4.5) {$f$};
		\node [style=none] (3) at (4, 5.5) {$Y$};
		\node [style=none] (4) at (4, 3.5) {$X$};
	\end{pgfonlayer}
	\begin{pgfonlayer}{edgelayer}
		\draw (0.center) to (2);
		\draw (2) to (1.center);
	\end{pgfonlayer}
\end{tikzpicture}
\otimes
\begin{tikzpicture}
	\begin{pgfonlayer}{nodelayer}
		\node [style=none] (0) at (4, 5.25) {};
		\node [style=none] (1) at (4, 3.75) {};
		\node [style=map] (2) at (4, 4.5) {$g$};
		\node [style=none] (3) at (4, 5.5) {$Z$};
		\node [style=none] (4) at (4, 3.5) {$W$};
	\end{pgfonlayer}
	\begin{pgfonlayer}{edgelayer}
		\draw (0.center) to (2);
		\draw (2) to (1.center);
	\end{pgfonlayer}
\end{tikzpicture}
:=
\begin{tikzpicture}
	\begin{pgfonlayer}{nodelayer}
		\node [style=none] (0) at (4, 5.25) {};
		\node [style=none] (1) at (4, 3.75) {};
		\node [style=map] (2) at (4, 4.5) {$f$};
		\node [style=none] (3) at (4, 5.5) {$Y$};
		\node [style=none] (4) at (4, 3.5) {$X$};
		\node [style=none] (5) at (5, 5.25) {};
		\node [style=none] (6) at (5, 3.75) {};
		\node [style=map] (7) at (5, 4.5) {$g$};
		\node [style=none] (8) at (5, 5.5) {$Z$};
		\node [style=none] (9) at (5, 3.5) {$W$};
	\end{pgfonlayer}
	\begin{pgfonlayer}{edgelayer}
		\draw (0.center) to (2);
		\draw (2) to (1.center);
		\draw (5.center) to (7);
		\draw (7) to (6.center);
	\end{pgfonlayer}
\end{tikzpicture}
$$
\item[Tensor unit:] The empty list $[]$ (drawn as blank space).
\end{description}
\end{definition}
By flipping around the diagrams for the unitors and associators we get their inverses: 

$$
\begin{tikzpicture}
	\begin{pgfonlayer}{nodelayer}
		\node [style=none] (3) at (8, 3.5) {};
		\node [style=none] (4) at (8, 3.25) {$X\otimes (Y\otimes Z)$};
		\node [style=none] (9) at (8, 7.5)  {$(X\otimes Y)\otimes Z$};
		\node [style=none] (14) at (8, 7.25) {};
		\node [style=map] (15) at (8, 5.25) {$\alpha_{X,Y,Z}^{-1}$};
	\end{pgfonlayer}
	\begin{pgfonlayer}{edgelayer}
		\draw (15) to (14.center);
		\draw (15) to (3.center);
	\end{pgfonlayer}
\end{tikzpicture}
=
\begin{tikzpicture}
	\begin{pgfonlayer}{nodelayer}
		\node [style=none] (34) at (12.75, 5.75) {};
		\node [style=none] (35) at (11.75, 5.75) {};
		\node [style=none] (36) at (12.25, 6.5) {};
		\node [style=none] (37) at (12.25, 7.25) {};
		\node [style=none] (38) at (12.25, 7.5) {$(X\otimes Y)\otimes Z$};
		\node [style=otimes] (39) at (12.25, 6.5) {};
		\node [style=none] (40) at (12.25, 5) {};
		\node [style=none] (41) at (11.25, 5) {};
		\node [style=none] (42) at (11.75, 5.75) {};
		\node [style=none] (43) at (11.75, 3.25) {$X\otimes (Y\otimes Z)$};
		\node [style=otimes] (44) at (11.75, 5.75) {};
		\node [style=none] (45) at (11.25, 5) {};
		\node [style=none] (46) at (12.25, 5) {};
		\node [style=none] (47) at (11.75, 4.25) {};
		\node [style=none] (48) at (11.75, 3.5) {};
		\node [style=otimes] (49) at (11.75, 4.25) {};
		\node [style=none] (50) at (11.75, 5.75) {};
		\node [style=none] (51) at (12.75, 5.75) {};
		\node [style=none] (52) at (12.25, 5) {};
		\node [style=otimes] (53) at (12.25, 5) {};
	\end{pgfonlayer}
	\begin{pgfonlayer}{edgelayer}
		\draw (37.center) to (36.center);
		\draw [in=90, out=-30] (36.center) to (34.center);
		\draw [in=90, out=-150] (36.center) to (35.center);
		\draw [in=90, out=-150] (42.center) to (41.center);
		\draw (48.center) to (47.center);
		\draw [in=-90, out=150] (47.center) to (45.center);
		\draw [in=-90, out=30] (47.center) to (46.center);
		\draw (52.center) to (50.center);
		\draw [in=-90, out=30] (52.center) to (51.center);
	\end{pgfonlayer}
\end{tikzpicture}
\ ,
\hspace*{.5cm}
\begin{tikzpicture}
	\begin{pgfonlayer}{nodelayer}
		\node [style=none] (17) at (11.75, 2) {};
		\node [style=none] (19) at (11.75, 0.5) {};
		\node [style=none] (20) at (11.75, 2.25) {$I\otimes X$};
		\node [style=none] (21) at (11.75, 0.25) {$X$};
		\node [style=map] (22) at (11.75, 1.25) {$(u_X^L)^{-1}$};
	\end{pgfonlayer}
	\begin{pgfonlayer}{edgelayer}
		\draw (17.center) to (22);
		\draw (19.center) to (22);
	\end{pgfonlayer}
\end{tikzpicture}
=
\begin{tikzpicture}
	\begin{pgfonlayer}{nodelayer}
		\node [style=none] (63) at (7.5, -2.5) {};
		\node [style=none] (64) at (8.5, -2.5) {};
		\node [style=none] (65) at (8, -1.75) {};
		\node [style=none] (66) at (8, -1) {};
		\node [style=none] (67) at (8.5, -2.75) {$X$};
		\node [style=none] (68) at (8, -0.75) {$I\otimes X$};
		\node [style=otimes] (69) at (8, -1.75) {};
		\node [style=unit] (70) at (7.5, -2.5) {};
	\end{pgfonlayer}
	\begin{pgfonlayer}{edgelayer}
		\draw (66.center) to (65.center);
		\draw [in=90, out=-150] (65.center) to (63.center);
		\draw [in=90, out=-30] (65.center) to (64.center);
	\end{pgfonlayer}
\end{tikzpicture}
\ ,
\hspace*{.5cm}
\begin{tikzpicture}
	\begin{pgfonlayer}{nodelayer}
		\node [style=none] (17) at (11.75, 2) {};
		\node [style=none] (19) at (11.75, 0.5) {};
		\node [style=none] (20) at (11.75, 2.25) {$ X\otimes I$};
		\node [style=none] (21) at (11.75, 0.25) {$X$};
		\node [style=map] (22) at (11.75, 1.25) {$(u_X^R)^{-1}$};
	\end{pgfonlayer}
	\begin{pgfonlayer}{edgelayer}
		\draw (17.center) to (22);
		\draw (19.center) to (22);
	\end{pgfonlayer}
\end{tikzpicture}
=
\begin{tikzpicture}
	\begin{pgfonlayer}{nodelayer}
		\node [style=none] (54) at (8.5, 0.25) {};
		\node [style=none] (55) at (7.5, 0.25) {};
		\node [style=none] (56) at (8, 1) {};
		\node [style=none] (57) at (8, 1.75) {};
		\node [style=none] (58) at (7.5, 0) {$X$};
		\node [style=none] (60) at (8, 2) {$X\otimes I$};
		\node [style=otimes] (61) at (8, 1) {};
		\node [style=unit] (62) at (8.5, 0.25) {};
	\end{pgfonlayer}
	\begin{pgfonlayer}{edgelayer}
		\draw (57.center) to (56.center);
		\draw [in=90, out=-30] (56.center) to (54.center);
		\draw [in=90, out=-150] (56.center) to (55.center);
	\end{pgfonlayer}
\end{tikzpicture}
$$
Even in the case when we are already working in a strict monoidal category, it will still often be useful to use string diagrams for its strictification; for example, we can bundle up wires together so that we can make inductive arguments using pictures.  Indeed, these string diagrams have been rediscovered in the setting of quantum circuits as the scalable ZX-calculus for precisely this reason \cite{szx}.  They have not made use of the units and counits in this setting; nevertheless it has found rich applications \cite{szxi,szxii}.  We will discuss this further in Section \ref{sec:cqm}.
\begin{aside}
These string diagrams are closely related to proof nets for linearly-distributive categories; so much so, that this monoidal counterpart was considered folklore by some.  % I will discuss the relation to proof nets in linear logic in Chapter \ref{chap:grothendieck}, as well as an attempt to present them as a universal construction.  This proved to be much more difficult than expected.
\end{aside}
Some monoidal categories are monoidally equivalent to {\bf skeletal}, strict monoidal categories, where the adjective skeletal means that every two isomorphic objects are equal.  These are very nice to work with because if we want, we can forgo having to use the tensoring/untensoring and unit introduction and removal:
\begin{example}
\label{ex:skeleton}
$\FinOrd$ is a skeletal and strict monoidal under both tensor prodcuts. It is monoidally equivalent to $\FSets$ under both thensor products.


Given a commutative semiring $R$, $\Mat_R$ is a skeletal category and is strict monoidal. Moreover for a field $k$, by chosing a basis for each dimension,  $\FVect_k$   is monoidally equivalent to $\Mat_k$ under the bilinear tensor product and the direct sum.

In particular, because $\FVect_\C$ and $\FHilb$ are monoidally equivalent under both the bilinear tensor product and the direct sum then $\FHilb$ is monoidally equivalent to the skeletal strict monoidal category $\Mat_\C$ under both tensor products.
\end{example}
The strictification of a monoidal category need not be skeletal, for example there is no skeletal strict monoidal  category which is monoidally equivalent to $\Set$.  Indeed, the strictification which we described when applied to $\FSets$ is not skeletal and thus not $\FinOrd$ on the nose.


There is a more refined notion of monoidal category where one can pass wires through each other:
\begin{definition}
A {\bf symmetric monoidal category} is a monoidal category equipped with an extra natural isomorphism called the symmetry
$$
\sigma_{X,Y}:X\otimes Y \to Y\otimes X
$$
satisfying the following coherence equations:
\begin{description}
\item[Interaction with unitors:]
$$
\xymatrix{
I \otimes X \ar[rr]^{\sigma_{I,X}} \ar[rd]_{u^L_X} && X \otimes I \ar[ld]^{u^R_X} \\
& X &
}
$$
\item[Interaction with associator:]
$$
\xymatrix{
  (X\otimes Y)\otimes Z \ar[rr]^{\sigma_{X,Y}\otimes 1_Z} \ar[d]_{\alpha_{X,Y,Z}}
    &
    &  (Y\otimes X)\otimes Z \ar[d]^{\alpha_{Y,X,Z}}
  \\X\otimes(Y\otimes Z) \ar[d]_{\sigma_{X,Y\otimes Z}}
    &
    &  Y\otimes(X\otimes Z) \ar[d]^{1_Y\otimes \sigma_{X,Z}}
  \\ (Y\otimes Z)\otimes X \ar[rr]_{\alpha_{Y,Z,X}}
    &
    & Y\otimes (Z\otimes X)
}
$$
\item[Symmetry map is self-inverse:]
$$
\xymatrix{
   X\otimes Y \ar[r]^{\sigma_{X,Y}} \ar@{=}[dr]
   &  Y\otimes X \ar[d]^{\sigma_{Y,X}}\\
   & X \otimes Y
}
$$
\end{description}
\end{definition}
\begin{example}
$\Sets$, $\FSets$, $\FinOrd$, $\Mat_R$, $\Vect_k$, $\Hilb$, $\FHilb$ are all symmetric monoidal categories with respect to the aforementioned monoidal structures; and the corresponding equivalences between these categories are also symmetric monoidal.
\end{example}
\begin{definition}
A (strong) {\bf symmetric monoidal functor} between symmetric monoidal categories $\X$ and $\Y$ is a monoidal functor where the following coherence equation holds:
$$
\xymatrix{
  F(X)\otimes^\Y F(Y) \ar[rr]^{\sigma_{F(X),F(Y)}^\Y} \ar[d]_{\mu_{X,Y}}
   && F(Y) \otimes^\Y F(X) \ar[d]^{\mu_{Y,X}}
\\F(X\otimes^\X Y) \ar[rr]_{F(\sigma_{X,Y}^\X)}
 && F(Y \otimes^\X X)
}
$$
A {\bf symmetric monoidal natural transformation} is a monoidal natural transformation between symmetric monoidal functors.  
A {\bf strict symmetric monoidal category} is a symmetric monoidal category, whose underlying monoidal category is strict. That is to say, all the coherence isomorphisms except for the symmetry maps are identities.
A {\bf strict symmetric monoidal functor} is a symmetric monoidal functor which is simultaneously a strict monoidal functor.
Just as in the monoidal case, there are strict 2-categories of strict symmetric monoidal and symmetric monoidal categories.
\end{definition}
Strict monoidal categories also have a notion of string diagrams, except the symmetry allows wires to pass over each other:
$$
\sigma_{X,Y}=
\begin{tikzpicture}
	\begin{pgfonlayer}{nodelayer}
		\node [style=none] (22) at (1, 5) {};
		\node [style=none] (23) at (0, 5) {};
		\node [style=none] (24) at (0, 4) {};
		\node [style=none] (25) at (1, 4) {};
		\node [style=none] (26) at (0, 3.75) {$X$};
		\node [style=none] (27) at (1, 3.75) {$Y$};
		\node [style=none] (28) at (0, 5.25) {$Y$};
		\node [style=none] (29) at (1, 5.25) {$X$};
	\end{pgfonlayer}
	\begin{pgfonlayer}{edgelayer}
		\draw [in=270, out=90] (24.center) to (22.center);
		\draw [in=270, out=90] (25.center) to (23.center);
	\end{pgfonlayer}
\end{tikzpicture}
$$

The naturality means that maps can be pulled through the symmetry:
$$
\begin{tikzpicture}
	\begin{pgfonlayer}{nodelayer}
		\node [style=none] (22) at (1, 5) {};
		\node [style=none] (23) at (0, 5) {};
		\node [style=none] (24) at (0, 4) {};
		\node [style=none] (25) at (1, 4) {};
		\node [style=map] (26) at (1, 4) {$g$};
		\node [style=map] (27) at (0, 4) {$f$};
		\node [style=none] (28) at (1, 5.75) {};
		\node [style=none] (29) at (0, 5.75) {};
		\node [style=none] (30) at (1, 3.25) {};
		\node [style=none] (31) at (0, 3.25) {};
	\end{pgfonlayer}
	\begin{pgfonlayer}{edgelayer}
		\draw [in=270, out=90] (24.center) to (22.center);
		\draw [in=270, out=90] (25.center) to (23.center);
		\draw (23.center) to (29.center);
		\draw (22.center) to (28.center);
		\draw (30.center) to (25.center);
		\draw (31.center) to (24.center);
	\end{pgfonlayer}
\end{tikzpicture}
=
\begin{tikzpicture}
	\begin{pgfonlayer}{nodelayer}
		\node [style=none] (32) at (3, 5) {};
		\node [style=none] (33) at (2, 5) {};
		\node [style=none] (34) at (2, 4) {};
		\node [style=none] (35) at (3, 4) {};
		\node [style=map] (36) at (2, 5) {$g$};
		\node [style=map] (37) at (3, 5) {$f$};
		\node [style=none] (38) at (3, 5.75) {};
		\node [style=none] (39) at (2, 5.75) {};
		\node [style=none] (40) at (3, 3.25) {};
		\node [style=none] (41) at (2, 3.25) {};
	\end{pgfonlayer}
	\begin{pgfonlayer}{edgelayer}
		\draw [in=270, out=90] (34.center) to (32.center);
		\draw [in=270, out=90] (35.center) to (33.center);
		\draw (33.center) to (39.center);
		\draw (32.center) to (38.center);
		\draw (40.center) to (35.center);
		\draw (41.center) to (34.center);
	\end{pgfonlayer}
\end{tikzpicture}
$$

The interaction with the unitor and associator becomes completely absorbed into the graphical calculus.
%The interaction with the associator also becomes trivial:
%
%$$
%\begin{tikzpicture}
%	\begin{pgfonlayer}{nodelayer}
%		\node [style=none] (22) at (1, 5) {};
%		\node [style=none] (23) at (0, 5) {};
%		\node [style=none] (33) at (2, 5) {};
%		\node [style=none] (34) at (2, 4) {};
%		\node [style=none] (35) at (0, 4) {};
%		\node [style=none] (36) at (1, 4) {};
%		\node [style=none] (37) at (0, 3) {};
%		\node [style=none] (38) at (1, 3) {};
%		\node [style=none] (39) at (2, 3) {};
%	\end{pgfonlayer}
%	\begin{pgfonlayer}{edgelayer}
%		\draw [in=270, out=90] (34.center) to (22.center);
%		\draw (23.center) to (35.center);
%		\draw [in=270, out=90] (36.center) to (33.center);
%		\draw (39.center) to (34.center);
%		\draw [in=270, out=90] (37.center) to (36.center);
%		\draw [in=270, out=90] (38.center) to (35.center);
%	\end{pgfonlayer}
%\end{tikzpicture}
%=
%\begin{tikzpicture}
%	\begin{pgfonlayer}{nodelayer}
%		\node [style=none] (40) at (4, 5) {};
%		\node [style=none] (41) at (3, 5) {};
%		\node [style=none] (42) at (5, 5) {};
%		\node [style=none] (46) at (3, 3) {};
%		\node [style=none] (47) at (4, 3) {};
%		\node [style=none] (48) at (5, 3) {};
%	\end{pgfonlayer}
%	\begin{pgfonlayer}{edgelayer}
%		\draw [in=-90, out=90] (48.center) to (40.center);
%		\draw [in=-90, out=90] (47.center) to (41.center);
%		\draw [in=-90, out=90] (46.center) to (42.center);
%	\end{pgfonlayer}
%\end{tikzpicture}
%$$
The self inverse of the symmetry map means that the wires untangle:
$$
\begin{tikzpicture}
	\begin{pgfonlayer}{nodelayer}
		\node [style=none] (22) at (1, 5) {};
		\node [style=none] (23) at (0, 5) {};
		\node [style=none] (24) at (0, 4) {};
		\node [style=none] (25) at (1, 4) {};
		\node [style=none] (30) at (1, 3) {};
		\node [style=none] (31) at (0, 3) {};
	\end{pgfonlayer}
	\begin{pgfonlayer}{edgelayer}
		\draw [in=270, out=90] (24.center) to (22.center);
		\draw [in=270, out=90] (25.center) to (23.center);
		\draw [in=270, out=90] (31.center) to (25.center);
		\draw [in=270, out=90] (30.center) to (24.center);
	\end{pgfonlayer}
\end{tikzpicture}
=
\begin{tikzpicture}
	\begin{pgfonlayer}{nodelayer}
		\node [style=none] (32) at (3, 5) {};
		\node [style=none] (33) at (2, 5) {};
		\node [style=none] (36) at (3, 3) {};
		\node [style=none] (37) at (2, 3) {};
	\end{pgfonlayer}
	\begin{pgfonlayer}{edgelayer}
		\draw (37.center) to (33.center);
		\draw (36.center) to (32.center);
	\end{pgfonlayer}
\end{tikzpicture}
$$
\begin{theorem}
Every symmetric monoidal category is symmetric monoidally equivalent to a strict symmetric monoidal category. 
\end{theorem}
Non-strict symmetric monoidal categories have essentially the same notion of proof nets as non-strict monoidal categories, except where the symmetry map is internalized to untensoring, exchanging the wires and then tensoring:
$$
\begin{tikzpicture}
	\begin{pgfonlayer}{nodelayer}
		\node [style=none] (1) at (20.75, 4.75) {};
		\node [style=none] (4) at (20.75, 5) {$Y\otimes X$};
		\node [style=none] (8) at (20.75, 1.25) {};
		\node [style=none] (13) at (20.75, 1) {$X\otimes Y$};
		\node [style=map] (14) at (20.75, 3) {$\sigma_{X,Y}$};
	\end{pgfonlayer}
	\begin{pgfonlayer}{edgelayer}
		\draw (14) to (1.center);
		\draw (14) to (8.center);
	\end{pgfonlayer}
\end{tikzpicture}
=
\begin{tikzpicture}
	\begin{pgfonlayer}{nodelayer}
		\node [style=none] (25) at (20.75, 4) {};
		\node [style=none] (26) at (20.75, 4.75) {};
		\node [style=none] (27) at (20.75, 4) {};
		\node [style=otimes] (28) at (20.75, 4) {};
		\node [style=none] (29) at (20.75, 5) {$Y\otimes X$};
		\node [style=none] (30) at (20.25, 3.5) {};
		\node [style=none] (31) at (21.25, 3.5) {};
		\node [style=none] (32) at (20.75, 2) {};
		\node [style=none] (33) at (20.75, 1.25) {};
		\node [style=none] (34) at (20.75, 2) {};
		\node [style=otimes] (35) at (20.75, 2) {};
		\node [style=none] (36) at (21.25, 2.5) {};
		\node [style=none] (37) at (20.25, 2.5) {};
		\node [style=none] (38) at (20.75, 1) {$X\otimes Y$};
	\end{pgfonlayer}
	\begin{pgfonlayer}{edgelayer}
		\draw (26.center) to (25.center);
		\draw [in=90, out=-165] (27.center) to (30.center);
		\draw [in=-15, out=90] (31.center) to (27.center);
		\draw (33.center) to (32.center);
		\draw [in=-90, out=15] (34.center) to (36.center);
		\draw [in=165, out=-90] (37.center) to (34.center);
		\draw [in=270, out=90] (36.center) to (30.center);
		\draw [in=270, out=90] (37.center) to (31.center);
	\end{pgfonlayer}
\end{tikzpicture}
$$

This notion of string diagrams for (non-strict) symmetric monoidal categories is not contained in \cite{wilson}; however, it is folklore, in analogy to the case for symmetric linearly distributive categories \cite{ldc}.
\begin{example}
The category $\Mat_k$ is strict symmetric monoidal  and symmetric monoidally equivalent to $\FVect_k$ under both aforementioned tensor products.  The same with $\FinOrd$ and $\FSets$.
\end{example}
\begin{definition}
A {\bf compact closed category} is a symmetric monoidal category such that for every object $X$, there is a chosen object $X^*$, called the {\bf dual} of $X$.
For all objects $X$, there are maps called the {\bf unit} and {\bf counit}:
$$
\eta_X:I\to X^* \otimes X\hspace*{.5cm}\text{and}\hspace*{.5cm} \epsilon_X:X\otimes X^*\to I 
$$
Satisfying the following coherence equations:
\begin{description}
\item[Zig-zag equations:]
$$
\xymatrix{
  (X\otimes X^*)\otimes X \ar[rr]^{\alpha_{X,X^*,X}}  \ar[d]_{\epsilon_X\otimes 1_X}
    & 
    & X\otimes(X^*\otimes X)
  \\I\otimes X \ar[rr]_{\sigma_{I,X}}
    &
    & X\otimes I \ar[u]_{1_X\otimes \eta_X}
}
\hspace*{.1cm}
\xymatrix{
  X^*\otimes ( X\otimes X^*) \ar[rr]^{\alpha_{X^*,X,X^*}}  \ar[d]_{1_{X^*}\otimes \epsilon_X}
    & 
    & (X^* \otimes X)\otimes X^*
  \\X^*\otimes I \ar[rr]_{\sigma_{X^*,I}}
    &
    & I\otimes X^* \ar[u]_{\eta_X\otimes 1_X{X^*}}
}
$$
\item[Compatibility with the tensor product:]\

\hspace*{-1.15cm}
\scalebox{1}{$
\xymatrix{
I  \ar[rr]^{(u_I^L)^{-1}} \ar[rrrddddd]!<-.5ex,-.5ex>_{\eta_{X\otimes Y}}
 &
 & I\otimes I \ar[r]^{\hspace*{-1.3cm}\eta_X\otimes \eta_Y}
 & (X\otimes X^*)\otimes(Y\otimes Y^*)  \ar[d]^{\alpha_{X,X^*,Y\otimes Y^*}} 
\\
 &
 &
 &  X\otimes (X^*\otimes(Y\otimes Y^*)) \ar[d]^{X\otimes \alpha_{X^*,Y,Y^*}^{-1}}
\\
I&
&
& X\otimes ((X^*\otimes Y)\otimes Y^*) \ar[d]^{1_X\otimes (\sigma_{X^*,Y}\otimes 1_Y^*)}
\\
 I\otimes I \ar[u]^{u^R_I}&
&
& X\otimes ((Y\otimes X^*)\otimes Y^*) \ar[d]^{1_X\otimes \alpha_{Y,X^*,Y^*}} 
& 
\\
(X^*\otimes X) \otimes (Y^*\otimes Y)  \ar[u]^{\epsilon_X\otimes \epsilon_Y} &
&
&X\otimes (Y\otimes (X^*\otimes Y^*)) \ar[d]^{\alpha_{X,Y,X^*\otimes Y^*}^{-1}}
\\
X^*\otimes( X\otimes (Y^*\otimes Y))  \ar[u]^{\alpha_{X^*,X,X^*\otimes 1_Y}^{-1}} &
&
& (X\otimes Y)\otimes (X^*\otimes Y^*)
\\
X^*\otimes( (X\otimes Y^*)\otimes Y)   \ar[u]^{1_{X^*}\otimes \alpha_{X,Y^*,Y}} &
&
& 
\\
X^*\otimes( (Y^*\otimes X)\otimes Y) \ar[u]^{1_{X^*}\otimes (\sigma_{Y^*,X}\otimes 1_Y)} &
&
X^*\otimes( Y^*\otimes (X\otimes Y))   \ar[ll]^{1_{X^*} \otimes \alpha^{-1}_{Y^*,X,Y}} & (X^*\otimes Y^*)\otimes (X\otimes Y)  \ar[l]^{\ \ \ \ \ \alpha^{-1}_{X^*,Y^*,X\otimes Y}} \ar[llluuuuu]!<0ex,.0ex>_{\epsilon_{X\otimes Y}}
}
$}
\end{description}

A strict compact closed category is a compact closed category where the underlying symmetric monoidal category is strict.
A self-dual compact closed category is one where $X^*=X$ for all objects $X$.
Strict symmetric monoidal functors and strong symmetric monoidal functors are the appropriate notion of map between strict/non-strict compact closed categories, as they preserve the duals strictly/strongly.
\end{definition}
The proceeding result follows immediately from the coherence theorem for symmetric monoidal categories because compact closed structure is preserved by symmetric monoidal functors:
\begin{theorem}
Every compact closed category is  symmetric monoidally equivalent to a strict compact closed category. 
\end{theorem}
Compact closed categories axiomatize the kinds of processes where inputs can be turned into outputs, and vice-versa.  In other words, they axiomatize a particular notion of feedback.  This is illuminated by looking at the string diagrams.
%In proof net notation the unit and counit are drawn as follows:
We will draw the unit and counit for the compact closed structure as follows in the strict case:
%
%
%$$
%\begin{tikzpicture}
%	\begin{pgfonlayer}{nodelayer}
%		\node [style=none] (2) at (18, 4.25) {};
%		\node [style=none] (4) at (18, 4.5) {$X^*\otimes X$};
%		\node [style=map] (5) at (18, 2.5) {$\eta_X$};
%		\node [style=none] (6) at (18, 1) {$I$};
%		\node [style=none] (7) at (18, 1.25) {};
%	\end{pgfonlayer}
%	\begin{pgfonlayer}{edgelayer}
%		\draw (7.center) to (5);
%		\draw (2.center) to (5);
%	\end{pgfonlayer}
%\end{tikzpicture}
%=
%\begin{tikzpicture}
%	\begin{pgfonlayer}{nodelayer}
%		\node [style=none] (8) at (20.75, 2.75) {};
%		\node [style=none] (9) at (20.75, 3.75) {};
%		\node [style=none] (10) at (20.75, 4.25) {};
%		\node [style=otimes] (11) at (20.75, 3.75) {};
%		\node [style=none] (12) at (20.75, 4.5) {$X^*\otimes X$};
%		\node [style=unit] (13) at (20.75, 2) {};
%		\node [style=none] (14) at (20.75, 1) {};
%		\node [style=none] (15) at (20.75, 1.25) {};
%	\end{pgfonlayer}
%	\begin{pgfonlayer}{edgelayer}
%		\draw [in=180, out=-150, looseness=1.50] (9.center) to (8.center);
%		\draw [in=-30, out=0, looseness=1.50] (8.center) to (9.center);
%		\draw (10.center) to (9.center);
%		\draw (15.center) to (13);
%	\end{pgfonlayer}
%\end{tikzpicture}
%\hspace*{.5cm}
%\text{and}
%\hspace*{.5cm}
%\begin{tikzpicture}
%	\begin{pgfonlayer}{nodelayer}
%		\node [style=none] (24) at (23, 1.25) {};
%		\node [style=none] (25) at (23, 1) {$X \otimes X^*$};
%		\node [style=map] (26) at (23, 3) {$\epsilon_X$};
%		\node [style=none] (27) at (23, 4.5) {$I$};
%		\node [style=none] (28) at (23, 4.25) {};
%	\end{pgfonlayer}
%	\begin{pgfonlayer}{edgelayer}
%		\draw (28.center) to (26);
%		\draw (24.center) to (26);
%	\end{pgfonlayer}
%\end{tikzpicture}
%=
%\begin{tikzpicture}
%	\begin{pgfonlayer}{nodelayer}
%		\node [style=none] (16) at (25.5, 2.75) {};
%		\node [style=none] (17) at (25.5, 1.75) {};
%		\node [style=none] (18) at (25.5, 1.25) {};
%		\node [style=otimes] (19) at (25.5, 1.75) {};
%		\node [style=none] (20) at (25.5, 1) {$X\otimes X^*$};
%		\node [style=unit] (21) at (25.5, 3.5) {$$};
%		\node [style=none] (22) at (25.5, 4.5) {$I$};
%		\node [style=none] (23) at (25.5, 4.25) {};
%	\end{pgfonlayer}
%	\begin{pgfonlayer}{edgelayer}
%		\draw [in=-180, out=150, looseness=1.50] (17.center) to (16.center);
%		\draw [in=30, out=0, looseness=1.50] (16.center) to (17.center);
%		\draw (18.center) to (17.center);
%		\draw (23.center) to (21);
%	\end{pgfonlayer}
%\end{tikzpicture}
%$$
%
$$
\eta_X=
\begin{tikzpicture}
	\begin{pgfonlayer}{nodelayer}
		\node [style=none] (1) at (0.5, -0.25) {};
		\node [style=none] (4) at (1.5, -0.25) {};
		\node [style=none] (5) at (0.5, 0) {$X^*$};
		\node [style=none] (6) at (1.5, 0) {$X$};
	\end{pgfonlayer}
	\begin{pgfonlayer}{edgelayer}
		\draw [in=270, out=-90, looseness=1.75] (1.center) to (4.center);
	\end{pgfonlayer}
\end{tikzpicture}
\hspace*{.5cm}\text{and}\hspace*{.5cm}
\epsilon_X=
\begin{tikzpicture}
	\begin{pgfonlayer}{nodelayer}
		\node [style=none] (7) at (3.25, 0.5) {};
		\node [style=none] (8) at (2.25, 0.5) {};
		\node [style=none] (9) at (3.25, 0.25) {$X^*$};
		\node [style=none] (10) at (2.25, 0.25) {$X$};
	\end{pgfonlayer}
	\begin{pgfonlayer}{edgelayer}
		\draw [in=90, out=90, looseness=1.75] (7.center) to (8.center);
	\end{pgfonlayer}
\end{tikzpicture}
$$
The zig-zag equations are drawn as follows;
$$
\begin{tikzpicture}
	\begin{pgfonlayer}{nodelayer}
		\node [style=none] (1) at (1.25, 0.5) {};
		\node [style=none] (4) at (2.25, 0.5) {};
		\node [style=none] (7) at (3.25, 0.5) {};
		\node [style=none] (8) at (3.25, -0.5) {};
		\node [style=none] (9) at (1.25, 1.5) {};
	\end{pgfonlayer}
	\begin{pgfonlayer}{edgelayer}
		\draw [in=270, out=-90, looseness=1.75] (1.center) to (4.center);
		\draw [in=90, out=90, looseness=1.75] (4.center) to (7.center);
		\draw (8.center) to (7.center);
		\draw (1.center) to (9.center);
	\end{pgfonlayer}
\end{tikzpicture}
=
\begin{tikzpicture}
	\begin{pgfonlayer}{nodelayer}
		\node [style=none] (13) at (6.25, -0.5) {};
		\node [style=none] (14) at (4.25, 1.5) {};
	\end{pgfonlayer}
	\begin{pgfonlayer}{edgelayer}
		\draw [in=90, out=-90] (14.center) to (13.center);
	\end{pgfonlayer}
\end{tikzpicture}
\hspace*{1cm}
\begin{tikzpicture}
	\begin{pgfonlayer}{nodelayer}
		\node [style=none] (15) at (7.25, 0.5) {};
		\node [style=none] (16) at (8.25, 0.5) {};
		\node [style=none] (17) at (9.25, 0.5) {};
		\node [style=none] (18) at (9.25, 1.5) {};
		\node [style=none] (19) at (7.25, -0.5) {};
	\end{pgfonlayer}
	\begin{pgfonlayer}{edgelayer}
		\draw [in=90, out=90, looseness=1.75] (15.center) to (16.center);
		\draw [in=-90, out=-90, looseness=1.75] (16.center) to (17.center);
		\draw (18.center) to (17.center);
		\draw (15.center) to (19.center);
	\end{pgfonlayer}
\end{tikzpicture}
=
\begin{tikzpicture}
	\begin{pgfonlayer}{nodelayer}
		\node [style=none] (20) at (12.25, 1.5) {};
		\node [style=none] (21) at (10.25, -0.5) {};
	\end{pgfonlayer}
	\begin{pgfonlayer}{edgelayer}
		\draw [in=-90, out=90] (21.center) to (20.center);
	\end{pgfonlayer}
\end{tikzpicture}
$$
And the second two equations correspond to the requirement that:
$$
\begin{tikzpicture}
	\begin{pgfonlayer}{nodelayer}
		\node [style=otimes] (0) at (25.5, 5.75) {};
		\node [style=none] (1) at (25.5, 5) {};
		\node [style=none] (4) at (26.25, 5) {};
		\node [style=none] (5) at (25.5, 5.75) {};
		\node [style=none] (6) at (26.25, 5.75) {};
		\node [style=none] (7) at (25.5, 6.5) {};
		\node [style=none] (8) at (26.25, 6.5) {};
		\node [style=otimes] (9) at (26.25, 5.75) {};
	\end{pgfonlayer}
	\begin{pgfonlayer}{edgelayer}
		\draw (5.center) to (7.center);
		\draw (6.center) to (8.center);
		\draw [in=180, out=-45] (5.center) to (4.center);
		\draw [in=-45, out=0, looseness=1.50] (4.center) to (6.center);
		\draw [in=0, out=-135] (6.center) to (1.center);
		\draw [in=-135, out=-180, looseness=1.50] (1.center) to (5.center);
	\end{pgfonlayer}
\end{tikzpicture}
=
\begin{tikzpicture}
	\begin{pgfonlayer}{nodelayer}
		\node [style=none] (27) at (27.5, 5.25) {};
		\node [style=none] (28) at (28.25, 5.25) {};
		\node [style=none] (29) at (27.5, 6.5) {};
		\node [style=none] (30) at (28.25, 6.5) {};
	\end{pgfonlayer}
	\begin{pgfonlayer}{edgelayer}
		\draw [bend right=90, looseness=1.25] (27.center) to (28.center);
		\draw (29.center) to (27.center);
		\draw (28.center) to (30.center);
	\end{pgfonlayer}
\end{tikzpicture}\ ,\hspace*{.5cm}
\begin{tikzpicture}
	\begin{pgfonlayer}{nodelayer}
		\node [style=otimes] (0) at (25.5, 5.75) {};
		\node [style=none] (1) at (25.5, 6.5) {};
		\node [style=none] (4) at (26.25, 6.5) {};
		\node [style=none] (5) at (25.5, 5.75) {};
		\node [style=none] (6) at (26.25, 5.75) {};
		\node [style=none] (7) at (25.5, 5) {};
		\node [style=none] (8) at (26.25, 5) {};
		\node [style=otimes] (9) at (26.25, 5.75) {};
	\end{pgfonlayer}
	\begin{pgfonlayer}{edgelayer}
		\draw (5.center) to (7.center);
		\draw (6.center) to (8.center);
		\draw [in=-180, out=45] (5.center) to (4.center);
		\draw [in=45, out=0, looseness=1.50] (4.center) to (6.center);
		\draw [in=0, out=135] (6.center) to (1.center);
		\draw [in=135, out=180, looseness=1.50] (1.center) to (5.center);
	\end{pgfonlayer}
\end{tikzpicture}
=
\begin{tikzpicture}
	\begin{pgfonlayer}{nodelayer}
		\node [style=none] (41) at (31.5, 6.25) {};
		\node [style=none] (42) at (32.25, 6.25) {};
		\node [style=none] (43) at (31.5, 5) {};
		\node [style=none] (44) at (32.25, 5) {};
	\end{pgfonlayer}
	\begin{pgfonlayer}{edgelayer}
		\draw [bend left=90, looseness=1.25] (41.center) to (42.center);
		\draw (43.center) to (41.center);
		\draw (42.center) to (44.center);
	\end{pgfonlayer}
\end{tikzpicture}
$$
This fixes the dualizing objects on tensor products $(X\otimes Y)^* = X^* \otimes Y^*$.
%The graphical calculus for strict compact closed categories extends string diagrams for symmetric monoidal categories, where  the units and counits are drawn without the tensor/untensor and unit removal/introduction.




One thing that is nice about compact closed categories is that we can treat all maps as either states or effects:
\begin{definition}
In a compact closed category, every map $f:X\to Y$ canonically induces a state $\lfloor f \rfloor:I\to X^* \otimes Y$ and an effect
$\lceil f \rceil: X \otimes Y^* \to I$ given by bending the wires of $f$ as follows:
$$
\begin{tikzpicture}
	\begin{pgfonlayer}{nodelayer}
		\node [style=map] (3) at (266.75, 5.5) {$\lfloor f \rfloor$};
		\node [style=none] (4) at (266.25, 6.5) {};
		\node [style=none] (5) at (267.25, 6.5) {};
	\end{pgfonlayer}
	\begin{pgfonlayer}{edgelayer}
		\draw [in=45, out=-90] (5.center) to (3);
		\draw [in=270, out=135] (3) to (4.center);
	\end{pgfonlayer}
\end{tikzpicture}
:=
\begin{tikzpicture}
	\begin{pgfonlayer}{nodelayer}
		\node [style=none] (436) at (267.25, 8.25) {};
		\node [style=none] (437) at (267.75, 8.25) {};
		\node [style=map] (438) at (267.75, 7.5) {$f$};
		\node [style=none] (439) at (267.25, 7) {};
		\node [style=none] (440) at (267.75, 7) {};
	\end{pgfonlayer}
	\begin{pgfonlayer}{edgelayer}
		\draw (436.center) to (439.center);
		\draw [in=270, out=-90, looseness=1.75] (439.center) to (440.center);
		\draw (440.center) to (437.center);
	\end{pgfonlayer}
\end{tikzpicture}
\ , \hspace*{.5cm}
\begin{tikzpicture}
	\begin{pgfonlayer}{nodelayer}
		\node [style=map] (0) at (264.25, 6.25) {$\lceil f \rceil$};
		\node [style=none] (1) at (263.75, 5.25) {};
		\node [style=none] (2) at (264.75, 5.25) {};
	\end{pgfonlayer}
	\begin{pgfonlayer}{edgelayer}
		\draw [in=-45, out=90] (2.center) to (0);
		\draw [in=-270, out=-135] (0) to (1.center);
	\end{pgfonlayer}
\end{tikzpicture}
:=
\begin{tikzpicture}
	\begin{pgfonlayer}{nodelayer}
		\node [style=none] (444) at (268, 5) {};
		\node [style=none] (445) at (267.5, 5) {};
		\node [style=map] (446) at (267.5, 5.75) {$f$};
		\node [style=none] (447) at (268, 6.25) {};
		\node [style=none] (448) at (267.5, 6.25) {};
	\end{pgfonlayer}
	\begin{pgfonlayer}{edgelayer}
		\draw (444.center) to (447.center);
		\draw [in=90, out=90, looseness=1.75] (447.center) to (448.center);
		\draw (448.center) to (445.center);
	\end{pgfonlayer}
\end{tikzpicture}
$$
\end{definition}
This abstract wire-bending induces a functor:
\begin{definition}
If $\X$ is a compact closed category, there is a  symmetric monoidal functor, $(-)^*:\X^\op\to\X$, called {\bf the transpose}, which sends:
\begin{description}
\item[Objects]\hfil $X \mapsto X^*$
\item[Maps:] \hfil
$
\begin{tikzpicture}
	\begin{pgfonlayer}{nodelayer}
		\node [style=map] (288) at (56, 0) {$f$};
		\node [style=none] (289) at (56, 1) {};
		\node [style=none] (290) at (56, -1) {};
	\end{pgfonlayer}
	\begin{pgfonlayer}{edgelayer}
		\draw (290.center) to (288);
		\draw (288) to (289.center);
	\end{pgfonlayer}
\end{tikzpicture}
\mapsto
\begin{tikzpicture}
	\begin{pgfonlayer}{nodelayer}
		\node [style=map] (288) at (56, 0) {$f$};
		\node [style=none] (289) at (56, 0.5) {};
		\node [style=none] (290) at (56, -0.5) {};
		\node [style=none] (291) at (56.5, 0.5) {};
		\node [style=none] (292) at (55.5, -0.5) {};
		\node [style=none] (293) at (55.5, 0) {};
		\node [style=none] (294) at (56.5, 0) {};
		\node [style=none] (295) at (55.8, 1) {};
		\node [style=none] (296) at (56.2, -1) {};
	\end{pgfonlayer}
	\begin{pgfonlayer}{edgelayer}
		\draw (290.center) to (288);
		\draw (288) to (289.center);
		\draw [in=90, out=90, looseness=1.25] (289.center) to (291.center);
		\draw [in=90, out=-90] (291.center) to (294.center);
		\draw [in=270, out=270, looseness=1.25] (290.center) to (292.center);
		\draw [in=270, out=90] (292.center) to (293.center);
		\draw [in=90, out=-90] (295.center) to (293.center);
		\draw [in=-90, out=90] (296.center) to (294.center);
	\end{pgfonlayer}
\end{tikzpicture}
$
\end{description}
\end{definition}
\begin{example}
Out of all the examples we have discussed so far, only $\Mat_R$, $\FVect_k$ and $\FHilb$ are compact closed when regarded as symmetric monoidal categories with respect to the bilinear tensor product.  They are not compact closed with respect to the direct sum.

For $\FHilb$ and $\FVect_k$, the compact closed structure is the same.  The dual object is given by the internal hom into $\C$/$k$.  Given an orthonormal basis $\{ b_i \}_{i=0,\ldots, n-1}$ of a finite dimensional vector space $X$, with dual basis $\{b_i^* \}_{i=0,\ldots, n-1}$ of $X^*$, the unit and counit are given by the following linear maps:
$$
\eta_X = 1 \mapsto \sum_{i=0}^{n-1} b_i^* \otimes b_i \hspace*{1cm}\epsilon_X = b_i\otimes b_j^* \mapsto 
\begin{cases}
1 & \text{If $i=j$}\\
0 & \text {Otherwise}
\end{cases}
$$

The situation for $\Mat_R$ is essentially the same.
Because $\Mat_R$ is is skeletal, every object is equal to it's dual, so that $n^*=n$.
The unit and counit are $\epsilon_n=(1,\cdots, 1)$ and  $\eta_n=\epsilon_n^T$.  In this case the transpose functor is exactly the transpose of matrices. 
\end{example}






\subsection{Dagger-monoidal categories}
\label{subsec:dag}

In this thesis, we will usually work with monoidal categories with extra structure called the dagger which allows one in some sense to``run maps in reverse.'' See \cite{cpm,abramsky} for further reference.
\begin{definition}
A {\bf \dag-category} ({\em read dagger-category}) is a category $\X$ equipped with a functor ${(-)}^\dag:\X^\op\to\X$ ({\em read the dagger}) that is:
\begin{description}
\item[Identity on objects:] so that for all objects $X$ of $\X$, $X^\dag = X$.
\item[Involutive:] so that for all maps $f$ of $\X$, $(f^\dag)^\dag = f$.
\end{description}
A map $f$ in a dagger category is:
\begin{description}
\item[an isometry] when $f^\dag; f = 1$.
\item[a coisometry] when $f; f^\dag = 1$.
\item[unitary] when $f^\dag = f^{-1}$.
\item[self-adjoint] when $f^\dag=f$.
\item[a projector]  when $f;f=f$ and $f^\dag=f$ (also known as a $\dag$-idempotent).
\end{description}
\end{definition}
\begin{example}
 $\Mat_\C$ is a \dag-category with respect to both the transpose and the complex conjugate transpose.
\end{example}
\begin{example}
The category $\Hilb$ of complex Hilbert spaces and bounded linear maps is a dagger category with respect to the Hermitian adjoint.  The Hermitian adjoint of a map $A$ is the unique map $A^\dag$ satisfying the following equation:
$$
\langle x;A|y\rangle = \langle x | A^\dag; y \rangle
$$
$\FHilb$  is also a $\dag$-category with respect to the Hermitian adjoint.
\end{example}
\begin{lemma}
There is an equivalence of categories $\FHilb \cong \Mat_\C$ preserving and reflecting the dagger structure.
The Hermitian adjoint corresponds to the complex conjugate transpose along the equivalence $\Mat_\C \cong \FHilb$.
\end{lemma}
This example is actually a bit tricky; while the dagger in $\Mat_\C$ is given by the complex conjugate transpose; the complex conjugate transpose in $\FHilb$ is {\em not } the Hermitian adjoint, because $A^*$ is only isomorphic to $A$.




There is a natural way to combined monoidal and dagger structure:
\begin{definition}
A {\bf  (strict) \dag-(symmetric) monoidal category} is a (strict) (symmetric) monoidal category equipped with a strict (symmetric) monoidal \dag-functor with respect to which all the components of the  coherence isomorphisms of the (symmetric) monoidal category are unitary.
\end{definition}
\begin{example}
The \dag-category and symmetric monoidal structures of $\FHilb$, $\Hilb$ and $\Mat_\C$ are all compatible making them \dag-symmetric monoidal categories.

Moreover, $\FHilb$ and $\Mat_\C$  are equivalent as $\dag$-symmetric monoidal categories.
\end{example}
We capture more of monoidal category theory within the framework of dagger categories:
\begin{definition}
A {(strict) \dag-compact closed category} is a (strict) compact closed category which is (strict) \dag-symmetric monoidal and for all objects $\X$:
$$
\xymatrix{
I \ar[r]^{\epsilon_X^\dag} \ar[dr]_{\eta_X}   &  X\otimes X^* \ar[d]^{\sigma_{X,X^*}}\\
 &  X^* \otimes X 
}
\hspace*{.5cm}
\text{or equivalently}
\hspace*{.5cm}
\xymatrix{
X\otimes X^* \ar[dr]_{\eta_X} \ar[r]^{\sigma_{X,X*}}
 & X^* \otimes X  \ar[d]^{\epsilon_X^\dag}\\
& I
}
$$
\end{definition}
\begin{example}
The compact closed and \dag-symmetric monoidal structures of $\Mat_\C$ and $\FHilb$ are both compatible, making them \dag-compact closed.
\end{example}
\subsection{Monoidal presentations}
\label{subsec:monpres}
In this subsection, we review how monoidal categories can be presented in terms of generators and equations. A more detailed reference can be found in \cite{ih}:
\begin{definition}
\label{def:monoidaltheory}
%A {\bf symmetric monoidal theory} is a triple $T=({\sf Ob},\Sigma ,E \)$. $\sf Ob$ is a set of {\em objects}. $\Sigma\in [{\sf Ob}]\times [{\sf Ob}]^{G}$ is a set of {\bf generators} $g$ with associated arities $(X,Y) \in [{\sf Ob}]\times [{\sf Ob}]$,  denoted $g:X\to Y$, where $[\_]$ is the finite list monad. Let $\Sigma'$ denote the set $\Sigma\sqcup\{\sigma_X: [X,X]\to[X,X] | \forall X \in {\sf Ob} \}$, where the $\sigma_X$ are regarded as the braiding maps.  Moreover, let $\Sigma^*$ denote the 
%$E \subseteq \{(f:X\to Y,g:X\to Y) \in \Sigma^2\}$
A {\bf monoidal theory} is a triple $T=({\sf Ob},\Sigma ,E )$. $\Ob$  is the set of {\bf colours}. The set The set of {\bf signatures} $\Sigma$ contains generators of the form $f:[X_1,\cdots, X_n]\to [Y_1,\cdots, Y_m]$, where  the {\bf arity} $[X_1,\cdots, X_n]$ and {\bf coarity} $[Y_1,\cdots, Y_m]$ are in $[\Ob]$ and the {\bf name} is $f$ indexed from some fixed set. For every object $X$, there is a distinguished generator ${\sf id}_X:X\to X$ called the {\bf unit}.  The set of $(\Ob,\Sigma)$-{\bf terms} is given by induction.  For the base case, all generators are fomal generators.  For the first inductive case, given composable terms $f:X\to Y$ and $g:Y\to Z$, there is a formal composite $f;g:X\to Z$.  Second, given two parallel terms $f:X\to Y$ and $g:Z\to W$ there is a formal tensor product $f\otimes g:[X,Z]\to [Y,W]$. 

The set of ${\bf equations}$ $E$ consists of pairs of parallel formal terms $f:X\to Y$ and $g:X\to Y$.

Every monoidal theory defines a strict monoidal category $\bar T$. This is the free strict monoidal category with objects in  $[\sf Ob]$ and maps formal composites of $(\Ob,\Sigma)$-terms modulo the equations in $E$.  $\bar T$ has the structure of a category as the identity on an object $[X_1,\cdot, X_n]$ is given by the formal composite ${\sf id}_{X_1}\otimes \cdots {\sf id}_{X_n}$ and the composition is given by formal composition.  For the strict monoidal structure,  the tensor unit is given by the empty list and the tensor product is given by the formal tensor product.  Call such a monoidal category a {\bf coloured pro}, or merely a {\bf pro} when $|{\sf Ob}|=1$.  We will say that $T$ is a {\bf presentation} of a monoidal category $\X$ when $\bar T$ is monoidally equivalent to $\X$.  

The coloured pro in a presentation is regarded as the {\em syntax}, and the monoidal category which it is equivalent to is regarded as the {\bf semantics}.  Throughout this thesis, the semantics will usually be concrete mathematical objects which are easy to define; whereas, finding the appropriate equations is substantially harder.  Therefore, even if both monoidal categories are equivalent, they feel much different.
\end{definition}



Throughout this thesis, when we impose an equation between generators, if it comes up later then we will put a label above the axiom.  When ever the same axiom is imposed again we will refer to the first time it is referenced.

 In practice, we won't explicitly regard a  monoidal theory as a triple; rather, we will present coloured pros by drawing a list of generating equations between string diagrams.
For example, the way in which string diagrams for nonstrict monoidal categories were constructed in Definition \ref{def:proofnets} is secretly a monoidal theory.
For a more elementary example:
\begin{example}
Consider the monoidal theory for the prop $\m$ generated by a monoid on one object:
$$
\begin{tikzpicture}[yscale=-1]
	\begin{pgfonlayer}{nodelayer}
		\node [style=X] (10) at (8, 2) {};
		\node [style=none] (11) at (8.5, 2.75) {};
		\node [style=none] (12) at (7.5, 2.75) {};
		\node [style=none] (13) at (8, 1.25) {};
		\node [style=none] (14) at (8.5, 3.25) {};
		\node [style=X] (15) at (7.5, 2.75) {};
	\end{pgfonlayer}
	\begin{pgfonlayer}{edgelayer}
		\draw [in=-90, out=30] (10) to (11.center);
		\draw (13.center) to (10);
		\draw [in=-90, out=150] (10) to (12.center);
		\draw (11.center) to (14.center);
	\end{pgfonlayer}
\end{tikzpicture}
 \eqzxa{unitl}
\begin{tikzpicture}[yscale=-1]
	\begin{pgfonlayer}{nodelayer}
		\node [style=none] (16) at (6.5, 1.25) {};
		\node [style=none] (17) at (6.5, 3.25) {};
	\end{pgfonlayer}
	\begin{pgfonlayer}{edgelayer}
		\draw (16.center) to (17.center);
	\end{pgfonlayer}
\end{tikzpicture}
 \eqzxa{unitr}
\begin{tikzpicture}[yscale=-1]
	\begin{pgfonlayer}{nodelayer}
		\node [style=X] (4) at (5, 2) {};
		\node [style=none] (5) at (4.5, 2.75) {};
		\node [style=none] (6) at (5.5, 2.75) {};
		\node [style=none] (7) at (5, 1.25) {};
		\node [style=none] (8) at (4.5, 3.25) {};
		\node [style=X] (9) at (5.5, 2.75) {};
	\end{pgfonlayer}
	\begin{pgfonlayer}{edgelayer}
		\draw [in=-90, out=150] (4) to (5.center);
		\draw (7.center) to (4);
		\draw [in=-90, out=30] (4) to (6.center);
		\draw (5.center) to (8.center);
	\end{pgfonlayer}
\end{tikzpicture}
\ ,
\hspace*{.2cm}
\begin{tikzpicture}
	\begin{pgfonlayer}{nodelayer}
		\node [style=X] (0) at (12, 2) {};
		\node [style=none] (1) at (12.5, 1.25) {};
		\node [style=none] (2) at (11.5, 1.25) {};
		\node [style=none] (3) at (12, 2.75) {};
		\node [style=X] (4) at (12.5, 1.25) {};
		\node [style=none] (5) at (13, 0.5) {};
		\node [style=none] (6) at (12, 0.5) {};
		\node [style=none] (7) at (11.5, 0.5) {};
	\end{pgfonlayer}
	\begin{pgfonlayer}{edgelayer}
		\draw [in=90, out=-30] (0) to (1.center);
		\draw (3.center) to (0);
		\draw [in=90, out=-150] (0) to (2.center);
		\draw [in=90, out=-30] (4) to (5.center);
		\draw [in=90, out=-150] (4) to (6.center);
		\draw (7.center) to (2.center);
	\end{pgfonlayer}
\end{tikzpicture}
 \eqzxa{assoc}
\begin{tikzpicture}[xscale=-1]
	\begin{pgfonlayer}{nodelayer}
		\node [style=X] (0) at (12, 2) {};
		\node [style=none] (1) at (12.5, 1.25) {};
		\node [style=none] (2) at (11.5, 1.25) {};
		\node [style=none] (3) at (12, 2.75) {};
		\node [style=X] (4) at (12.5, 1.25) {};
		\node [style=none] (5) at (13, 0.5) {};
		\node [style=none] (6) at (12, 0.5) {};
		\node [style=none] (7) at (11.5, 0.5) {};
	\end{pgfonlayer}
	\begin{pgfonlayer}{edgelayer}
		\draw [in=90, out=-30] (0) to (1.center);
		\draw (3.center) to (0);
		\draw [in=90, out=-150] (0) to (2.center);
		\draw [in=90, out=-30] (4) to (5.center);
		\draw [in=90, out=-150] (4) to (6.center);
		\draw (7.center) to (2.center);
	\end{pgfonlayer}
\end{tikzpicture}
$$
This is a presentation for the pro $\FinOrdMonot$, of finite ordinals and monotone maps \cite[\S 3.1]{Lafont1995}.
\end{example}
\begin{definition}
A {\bf symmetric monoidal theory} $T$ consists of the same data as a monoidal theory except the equations are now defined by parallel maps generated by $\Sigma\sqcup C$, where  $C=\{\sigma_X:[X,X]\to [X,X]\ |\ \forall X \in {\sf Ob}\}$ is the set of distinguished symmetry maps.


The corresponding strict symmetric monoidal category $\bar T$ is given by quotienting the symmetric monoidal category freely generated by the objects $\sf Ob$ and maps $\Sigma$ by the equations in $E$.  These symmetric monoidal categories are called {\bf coloured props}, or merely {\bf props} when $|{\sf Ob}|=1$.
\end{definition}
\begin{example}
\label{example:monoid}
Consider the symmetric monoidal theory $\cm$ generated by a monoid $\xcirc$, which is also commutative:
$$
\begin{tikzpicture}[yscale=-1]
	\begin{pgfonlayer}{nodelayer}
		\node [style=X] (18) at (10, 2) {};
		\node [style=none] (19) at (10.5, 2.75) {};
		\node [style=none] (20) at (9.5, 2.75) {};
		\node [style=none] (21) at (10, 1.25) {};
		\node [style=none] (22) at (9.5, 3.75) {};
		\node [style=none] (23) at (10.5, 3.75) {};
	\end{pgfonlayer}
	\begin{pgfonlayer}{edgelayer}
		\draw [in=-90, out=30] (18) to (19.center);
		\draw (21.center) to (18);
		\draw [in=-90, out=150] (18) to (20.center);
		\draw [in=270, out=90] (19.center) to (22.center);
		\draw [in=270, out=90] (20.center) to (23.center);
	\end{pgfonlayer}
\end{tikzpicture}
\eqzxa{comm}
\begin{tikzpicture}[yscale=-1]
	\begin{pgfonlayer}{nodelayer}
		\node [style=X] (24) at (12, 2) {};
		\node [style=none] (25) at (12.5, 2.75) {};
		\node [style=none] (26) at (11.5, 2.75) {};
		\node [style=none] (27) at (12, 1.25) {};
		\node [style=none] (28) at (12.5, 3.75) {};
		\node [style=none] (29) at (11.5, 3.75) {};
	\end{pgfonlayer}
	\begin{pgfonlayer}{edgelayer}
		\draw [in=-90, out=30] (24) to (25.center);
		\draw (27.center) to (24);
		\draw [in=-90, out=150] (24) to (26.center);
		\draw (25.center) to (28.center);
		\draw (26.center) to (29.center);
	\end{pgfonlayer}
\end{tikzpicture}
$$
This is a presentation for the prop of finite ordinals and functions $\FinOrd$ under the disjoint union \cite[\S 3.3]{Lafont1995}.
This is a formal way to talk about the graph of a function between finite sets.
\end{example}
This elegant presentation of the symmetric monoidal category of finite sets motivates finding presentations for other well-known mathematical structures.  


The following result has probably been known for quite some time.  The earliest reference I could find considers only the Boolean semiring \cite[Figure 3]{Lafont1995}; and later, by the same author he proves the analogous result for arbitrary fields \cite[Figure 26]{lafont}:
\begin{example}
\label{ex:bialg}
Take a commutative semiring $S$.
Consider the prop ${\sf cb}_S$ generated by a commutative monoid $\xcirc$ and comonoid $\zcirc$ interacting to form a {\bf bicommutative bialgebra}:
$$
  \begin{tikzpicture}
	\begin{pgfonlayer}{nodelayer}
		\node [style=X] (0) at (-3.75, -1) {};
		\node [style=none] (1) at (-4, -1.75) {};
		\node [style=none] (2) at (-3.5, -1.75) {};
		\node [style=Z] (3) at (-3.75, -0.25) {};
		\node [style=none] (4) at (-4, 0.5) {};
		\node [style=none] (5) at (-3.5, 0.5) {};
	\end{pgfonlayer}
	\begin{pgfonlayer}{edgelayer}
		\draw [in=90, out=-60, looseness=1.00] (0) to (2.center);
		\draw [in=-120, out=90, looseness=1.00] (1.center) to (0);
		\draw (0) to (3);
		\draw [in=60, out=-90, looseness=1.00] (5.center) to (3);
		\draw [in=-90, out=120, looseness=1.00] (3) to (4.center);
	\end{pgfonlayer}
  \end{tikzpicture}
  \eqzxa{bi.one}
  \begin{tikzpicture}
	\begin{pgfonlayer}{nodelayer}
		\node [style=X] (0) at (-4, 0.5) {};
		\node [style=Z] (1) at (-4, -0.25) {};
		\node [style=X] (2) at (-4.5, 0.5) {};
		\node [style=Z] (3) at (-4.5, -0.25) {};
		\node [style=none] (4) at (-4, -1) {};
		\node [style=none] (5) at (-4.5, -1) {};
		\node [style=none] (6) at (-4.5, 1.25) {};
		\node [style=none] (7) at (-4, 1.25) {};
	\end{pgfonlayer}
	\begin{pgfonlayer}{edgelayer}
		\draw [bend left, looseness=1.25] (0) to (1);
		\draw [bend right, looseness=1.25] (2) to (3);
		\draw (1) to (2);
		\draw (3) to (0);
		\draw (0) to (7.center);
		\draw (6.center) to (2);
		\draw (3) to (5.center);
		\draw (4.center) to (1);
	\end{pgfonlayer}
\end{tikzpicture},
\hspace*{.5cm}
  \begin{tikzpicture}
	\begin{pgfonlayer}{nodelayer}
		\node [style=Z] (0) at (-4, -0) {};
		\node [style=X] (1) at (-4, -0.75) {};
		\node [style=none] (2) at (-4.25, -1.5) {};
		\node [style=none] (3) at (-3.75, -1.5) {};
	\end{pgfonlayer}
	\begin{pgfonlayer}{edgelayer}
		\draw [in=-60, out=90, looseness=1.00] (3.center) to (1);
		\draw (1) to (0);
		\draw [in=90, out=-120, looseness=1.00] (1) to (2.center);
	\end{pgfonlayer}
  \end{tikzpicture}
  \eqzxa{bi.two}
  \begin{tikzpicture}
	\begin{pgfonlayer}{nodelayer}
		\node [style=Z] (0) at (-4.25, -0.75) {};
		\node [style=none] (1) at (-4.25, -1.5) {};
		\node [style=none] (2) at (-3.5, -1.5) {};
		\node [style=Z] (3) at (-3.5, -0.75) {};
	\end{pgfonlayer}
	\begin{pgfonlayer}{edgelayer}
		\draw (2.center) to (3);
		\draw (0) to (1.center);
	\end{pgfonlayer}
  \end{tikzpicture},
  \hspace*{.5cm}
   \begin{tikzpicture}[yscale=-1]
	\begin{pgfonlayer}{nodelayer}
		\node [style=X] (0) at (-4, -0) {};
		\node [style=Z] (1) at (-4, -0.75) {};
		\node [style=none] (2) at (-4.25, -1.5) {};
		\node [style=none] (3) at (-3.75, -1.5) {};
	\end{pgfonlayer}
	\begin{pgfonlayer}{edgelayer}
		\draw [in=-60, out=90, looseness=1.00] (3.center) to (1);
		\draw (1) to (0);
		\draw [in=90, out=-120, looseness=1.00] (1) to (2.center);
	\end{pgfonlayer}
  \end{tikzpicture}
  \erefop{bi.two}
   \begin{tikzpicture}[yscale=-1]
	\begin{pgfonlayer}{nodelayer}
		\node [style=X] (0) at (-4.25, -0.75) {};
		\node [style=none] (1) at (-4.25, -1.5) {};
		\node [style=none] (2) at (-3.5, -1.5) {};
		\node [style=X] (3) at (-3.5, -0.75) {};
	\end{pgfonlayer}
	\begin{pgfonlayer}{edgelayer}
		\draw (2.center) to (3);
		\draw (0) to (1.center);
	\end{pgfonlayer}
  \end{tikzpicture},
\hspace*{.5cm}
  \begin{tikzpicture}[rotate=90]
	\begin{pgfonlayer}{nodelayer}
		\node [style=Z] (0) at (-8.25, -0) {};
		\node [style=X] (1) at (-9.25, -0) {};
	\end{pgfonlayer}
	\begin{pgfonlayer}{edgelayer}
		\draw (0) to (1);
	\end{pgfonlayer}
\end{tikzpicture}
  \eqzxa{extra}
\begin{tikzpicture}
	\begin{pgfonlayer}{nodelayer}
		\node [style=none] (0) at (2, 0) {};
		\node [style=none] (1) at (2, -1) {};
		\node [style=none] (2) at (3, -1) {};
		\node [style=none] (3) at (3, 0) {};
	\end{pgfonlayer}
	\begin{pgfonlayer}{edgelayer}
		\draw[style=dashed] (3.center) to (0.center) to (1.center) to (2.center) to cycle;
	\end{pgfonlayer}
\end{tikzpicture}
$$

And generators for all elements $a,b \in S$ such that the structure of the commutative semiring $S$ is reflected in the convolution of the bialgebra
$$
\begin{tikzpicture}
	\begin{pgfonlayer}{nodelayer}
		\node [style=none] (0) at (0, 0.25) {};
		\node [style=none] (1) at (0, -2.25) {};
		\node [style=X] (2) at (0, -0.25) {};
		\node [style=Z] (3) at (0, -1.75) {};
		\node [style=none] (4) at (-0.5, -1) {};
		\node [style=none] (5) at (0.5, -1) {};
		\node [style=scalar,fill=white] (6) at (-0.5, -1) {$a$};
		\node [style=scalar,fill=white] (7) at (0.5, -1) {$b$};
	\end{pgfonlayer}
	\begin{pgfonlayer}{edgelayer}
		\draw (1.center) to (3);
		\draw (2) to (0.center);
		\draw [in=90, out=-30] (2) to (5.center);
		\draw [in=30, out=-90] (5.center) to (3);
		\draw [in=-90, out=150] (3) to (4.center);
		\draw [in=-150, out=90] (4.center) to (2);
	\end{pgfonlayer}
\end{tikzpicture}
=
\begin{tikzpicture}
	\begin{pgfonlayer}{nodelayer}
		\node [style=none] (0) at (0.5, 0.25) {};
		\node [style=none] (1) at (0.5, -2.25) {};
		\node [style=scalar] (2) at (0.5, -1) {$a+b$};
	\end{pgfonlayer}
	\begin{pgfonlayer}{edgelayer}
		\draw (2) to (0.center);
		\draw (1.center) to (2);
	\end{pgfonlayer}
\end{tikzpicture}\ ,
\hspace*{.5cm}
\begin{tikzpicture}
	\begin{pgfonlayer}{nodelayer}
		\node [style=none] (0) at (0.5, 0.25) {};
		\node [style=none] (1) at (0.5, -2.25) {};
		\node [style=scalar] (2) at (0.5, -0.5) {$b$};
		\node [style=scalar] (3) at (0.5, -1.5) {$a$};
	\end{pgfonlayer}
	\begin{pgfonlayer}{edgelayer}
		\draw (2) to (0.center);
		\draw (1.center) to (3);
		\draw (3) to (2);
	\end{pgfonlayer}
\end{tikzpicture}
=
\begin{tikzpicture}
	\begin{pgfonlayer}{nodelayer}
		\node [style=none] (0) at (0.5, 0.25) {};
		\node [style=none] (1) at (0.5, -2.25) {};
		\node [style=scalar] (2) at (0.5, -1) {$ab$};
	\end{pgfonlayer}
	\begin{pgfonlayer}{edgelayer}
		\draw (2) to (0.center);
		\draw (1.center) to (2);
	\end{pgfonlayer}
\end{tikzpicture}\ ,
\hspace*{.5cm}
\begin{tikzpicture}
	\begin{pgfonlayer}{nodelayer}
		\node [style=none] (0) at (0.5, 0.25) {};
		\node [style=none] (1) at (0.5, -1.25) {};
		\node [style=scalar] (2) at (0.5, -0.5) {$1$};
	\end{pgfonlayer}
	\begin{pgfonlayer}{edgelayer}
		\draw (2) to (0.center);
		\draw (1.center) to (2);
	\end{pgfonlayer}
\end{tikzpicture}
=
\begin{tikzpicture}
	\begin{pgfonlayer}{nodelayer}
		\node [style=none] (0) at (0.5, 0.25) {};
		\node [style=none] (1) at (0.5, -1.25) {};
	\end{pgfonlayer}
	\begin{pgfonlayer}{edgelayer}
		\draw (1.center) to (0.center);
	\end{pgfonlayer}
\end{tikzpicture}\ ,
\hspace*{.5cm}
\begin{tikzpicture}
	\begin{pgfonlayer}{nodelayer}
		\node [style=none] (0) at (0.5, 0.25) {};
		\node [style=none] (1) at (0.5, -1.25) {};
		\node [style=scalar] (2) at (0.5, -0.5) {$0$};
	\end{pgfonlayer}
	\begin{pgfonlayer}{edgelayer}
		\draw (2) to (0.center);
		\draw (1.center) to (2);
	\end{pgfonlayer}
\end{tikzpicture}
=
\begin{tikzpicture}
	\begin{pgfonlayer}{nodelayer}
		\node [style=none] (0) at (0.5, 0.25) {};
		\node [style=none] (1) at (0.5, -1.25) {};
		\node [style=X] (2) at (0.5, -0.25) {};
		\node [style=Z] (3) at (0.5, -0.75) {};
	\end{pgfonlayer}
	\begin{pgfonlayer}{edgelayer}
		\draw (2) to (0.center);
		\draw (3) to (1.center);
	\end{pgfonlayer}
\end{tikzpicture}
$$
where the commutative monoid and cocommutative comonoid are both natural with respect to the scalars:
$$
\begin{tikzpicture}
	\begin{pgfonlayer}{nodelayer}
		\node [style=Z] (12) at (2, 0) {};
		\node [style=none] (15) at (2.5, 0.75) {};
		\node [style=none] (17) at (2, -0.75) {};
		\node [style=none] (18) at (1.5, 0.75) {};
		\node [style=scalar,fill=white] (19) at (2, -0.75) {$a$};
		\node [style=none] (20) at (2, -1.5) {};
	\end{pgfonlayer}
	\begin{pgfonlayer}{edgelayer}
		\draw (17.center) to (12);
		\draw [in=-90, out=30] (12) to (15.center);
		\draw [in=150, out=-90] (18.center) to (12);
		\draw (20.center) to (17.center);
	\end{pgfonlayer}
\end{tikzpicture}
=
\begin{tikzpicture}
	\begin{pgfonlayer}{nodelayer}
		\node [style=Z] (21) at (3.75, -0.75) {};
		\node [style=none] (22) at (4.25, 0) {};
		\node [style=none] (23) at (3.75, -1.5) {};
		\node [style=none] (24) at (3.25, 0) {};
		\node [style=scalar,fill=white] (25) at (3.25, 0) {$a$};
		\node [style=scalar,fill=white] (27) at (4.25, 0) {$a$};
		\node [style=none] (28) at (3.25, 0.75) {};
		\node [style=none] (29) at (4.25, 0.75) {};
	\end{pgfonlayer}
	\begin{pgfonlayer}{edgelayer}
		\draw (23.center) to (21);
		\draw [in=-90, out=30] (21) to (22.center);
		\draw [in=150, out=-90] (24.center) to (21);
		\draw (28.center) to (25);
		\draw (29.center) to (27);
	\end{pgfonlayer}
\end{tikzpicture}
\ ,
\hspace*{.2cm}
\begin{tikzpicture}
	\begin{pgfonlayer}{nodelayer}
		\node [style=Z] (0) at (2, 0) {};
		\node [style=none] (2) at (2, -0.75) {};
		\node [style=scalar,fill=white] (4) at (2, -0.75) {$a$};
		\node [style=none] (5) at (2, -1.5) {};
	\end{pgfonlayer}
	\begin{pgfonlayer}{edgelayer}
		\draw (2.center) to (0);
		\draw (5.center) to (2.center);
	\end{pgfonlayer}
\end{tikzpicture}
=
\begin{tikzpicture}
	\begin{pgfonlayer}{nodelayer}
		\node [style=Z] (6) at (3.75, -0.75) {};
		\node [style=none] (8) at (3.75, -1.5) {};
	\end{pgfonlayer}
	\begin{pgfonlayer}{edgelayer}
		\draw (8.center) to (6);
	\end{pgfonlayer}
\end{tikzpicture}
\ ,
\hspace*{.2cm}
\begin{tikzpicture}
	\begin{pgfonlayer}{nodelayer}
		\node [style=X] (36) at (7.5, 0) {};
		\node [style=none] (37) at (8, -0.75) {};
		\node [style=none] (38) at (7.5, 0.75) {};
		\node [style=none] (39) at (7, -0.75) {};
		\node [style=scalar,fill=white] (40) at (7, -0.75) {$a$};
		\node [style=scalar,fill=white] (41) at (8, -0.75) {$a$};
		\node [style=none] (42) at (7, -1.5) {};
		\node [style=none] (43) at (8, -1.5) {};
	\end{pgfonlayer}
	\begin{pgfonlayer}{edgelayer}
		\draw (38.center) to (36);
		\draw [in=90, out=-30] (36) to (37.center);
		\draw [in=-150, out=90] (39.center) to (36);
		\draw (42.center) to (40);
		\draw (43.center) to (41);
	\end{pgfonlayer}
\end{tikzpicture}
=
\begin{tikzpicture}
	\begin{pgfonlayer}{nodelayer}
		\node [style=X] (30) at (5.75, -0.75) {};
		\node [style=none] (31) at (6.25, -1.5) {};
		\node [style=none] (32) at (5.75, 0) {};
		\node [style=none] (33) at (5.25, -1.5) {};
		\node [style=scalar,fill=white] (34) at (5.75, 0) {$a$};
		\node [style=none] (35) at (5.75, 0.75) {};
	\end{pgfonlayer}
	\begin{pgfonlayer}{edgelayer}
		\draw (32.center) to (30);
		\draw [in=90, out=-30] (30) to (31.center);
		\draw [in=-150, out=90] (33.center) to (30);
		\draw (35.center) to (32.center);
	\end{pgfonlayer}
\end{tikzpicture}
\ ,
\hspace*{.2cm}
\begin{tikzpicture}
	\begin{pgfonlayer}{nodelayer}
		\node [style=X] (15) at (7.5, 0) {};
		\node [style=none] (17) at (7.5, 0.75) {};
	\end{pgfonlayer}
	\begin{pgfonlayer}{edgelayer}
		\draw (17.center) to (15);
	\end{pgfonlayer}
\end{tikzpicture}
=
\begin{tikzpicture}
	\begin{pgfonlayer}{nodelayer}
		\node [style=X] (9) at (5.75, -0.75) {};
		\node [style=none] (11) at (5.75, 0) {};
		\node [style=scalar,fill=white] (13) at (5.75, 0) {$a$};
		\node [style=none] (14) at (5.75, 0.75) {};
	\end{pgfonlayer}
	\begin{pgfonlayer}{edgelayer}
		\draw (11.center) to (9);
		\draw (14.center) to (11.center);
	\end{pgfonlayer}
\end{tikzpicture}
$$

This monoidal theory is equivalent to the prop of matrices over $S$, $\Mat_S$, under the direct sum.  Recall that the direct sum of matrices $R$ and $S$ is given by the following block diagonal matrix:
$$
R\oplus S :=
\begin{bmatrix}
R & 0\\
0 & S
\end{bmatrix}
$$

The generators of this presentation are interpreted in $\Mat_S$ as 
follows:

$$
\left\llbracket\
\begin{tikzpicture}
	\begin{pgfonlayer}{nodelayer}
		\node [style=none] (9) at (3, -2.25) {};
		\node [style=Z] (11) at (3, -1.75) {};
		\node [style=none] (12) at (2.5, -1) {};
		\node [style=none] (13) at (3.5, -1) {};
	\end{pgfonlayer}
	\begin{pgfonlayer}{edgelayer}
		\draw (9.center) to (11);
		\draw [in=30, out=-90] (13.center) to (11);
		\draw [in=-90, out=150] (11) to (12.center);
	\end{pgfonlayer}
\end{tikzpicture}
\ \right\rrbracket
=
\begin{pmatrix}
1 & 1
\end{pmatrix} , \hspace*{.2cm}
\left\llbracket\
\begin{tikzpicture}
	\begin{pgfonlayer}{nodelayer}
		\node [style=none] (14) at (7, -1) {};
		\node [style=X] (16) at (7, -1.5) {};
		\node [style=none] (18) at (6.5, -2.25) {};
		\node [style=none] (19) at (7.5, -2.25) {};
	\end{pgfonlayer}
	\begin{pgfonlayer}{edgelayer}
		\draw (16) to (14.center);
		\draw [in=90, out=-30] (16) to (19.center);
		\draw [in=-150, out=90] (18.center) to (16);
	\end{pgfonlayer}
\end{tikzpicture}
\ \right\rrbracket
=
\begin{pmatrix}
1 \\ 1
\end{pmatrix} , \hspace*{.2cm}
\begin{tikzpicture}
	\begin{pgfonlayer}{nodelayer}
		\node [style=none] (9) at (2.5, -2.25) {};
		\node [style=none] (12) at (2.5, -0.75) {};
		\node [style=scalar] (24) at (2.5, -1.5) {$a$};
	\end{pgfonlayer}
	\begin{pgfonlayer}{edgelayer}
		\draw (9.center) to (24);
		\draw (24) to (12.center);
	\end{pgfonlayer}
\end{tikzpicture}
=a
$$
The unit and counit are interpreted as the unique matrices from $0 \to 1$ and $1\to 0$, respectively. Because $\N$ is the initial commutative semiring, $\Mat_\N$ can be presented in terms of the prop for the free bicommutative bialgebra, where the generators and equations for scalars are derivable.



In particular, when $S$ is a ring, then the bialgebra is promoted to a {\bf Hopf algebra}, so that there exists an {\bf antipode}
$\begin{tikzpicture}
	\begin{pgfonlayer}{nodelayer}
		\node [style=none] (0) at (1, 0.25) {};
		\node [style=none] (4) at (1, -0.75) {};
		\node [style=s] (6) at (1, -0.25) {};
	\end{pgfonlayer}
	\begin{pgfonlayer}{edgelayer}
		\draw (0.center) to (6);
		\draw (6) to (4.center);
	\end{pgfonlayer}
\end{tikzpicture}
$
such that:
$$
\begin{tikzpicture}
	\begin{pgfonlayer}{nodelayer}
		\node [style=none] (0) at (1, 0.25) {};
		\node [style=none] (1) at (1, -2.25) {};
		\node [style=X] (2) at (1, -0.25) {};
		\node [style=Z] (3) at (1, -1.75) {};
		\node [style=none] (4) at (1.5, -1) {};
		\node [style=none] (5) at (0.5, -1) {};
		\node [style=s] (6) at (0.5, -1) {};
	\end{pgfonlayer}
	\begin{pgfonlayer}{edgelayer}
		\draw (1.center) to (3);
		\draw (2) to (0.center);
		\draw [in=90, out=-150] (2) to (5.center);
		\draw [in=150, out=-90] (5.center) to (3);
		\draw [in=-90, out=30] (3) to (4.center);
		\draw [in=-30, out=90] (4.center) to (2);
	\end{pgfonlayer}
\end{tikzpicture}
=
\begin{tikzpicture}
	\begin{pgfonlayer}{nodelayer}
		\node [style=none] (0) at (1, 0.25) {};
		\node [style=none] (1) at (1, -2.25) {};
		\node [style=X] (2) at (1, -0.25) {};
		\node [style=Z] (3) at (1, -1.75) {};
	\end{pgfonlayer}
	\begin{pgfonlayer}{edgelayer}
		\draw (1.center) to (3);
		\draw (2) to (0.center);
	\end{pgfonlayer}
\end{tikzpicture}
=
\begin{tikzpicture}[xscale=-1]
	\begin{pgfonlayer}{nodelayer}
		\node [style=none] (0) at (1, 0.25) {};
		\node [style=none] (1) at (1, -2.25) {};
		\node [style=X] (2) at (1, -0.25) {};
		\node [style=Z] (3) at (1, -1.75) {};
		\node [style=none] (4) at (1.5, -1) {};
		\node [style=none] (5) at (0.5, -1) {};
		\node [style=s] (6) at (0.5, -1) {};
	\end{pgfonlayer}
	\begin{pgfonlayer}{edgelayer}
		\draw (1.center) to (3);
		\draw (2) to (0.center);
		\draw [in=90, out=-150] (2) to (5.center);
		\draw [in=150, out=-90] (5.center) to (3);
		\draw [in=-90, out=30] (3) to (4.center);
		\draw [in=-30, out=90] (4.center) to (2);
	\end{pgfonlayer}
\end{tikzpicture}
$$
where the antipode of the Hopf algebra is given by the scalar $-1$:
$$
\begin{tikzpicture}
	\begin{pgfonlayer}{nodelayer}
		\node [style=none] (0) at (1, 0.25) {};
		\node [style=none] (1) at (1, -2.25) {};
		\node [style=X] (2) at (1, -0.25) {};
		\node [style=Z] (3) at (1, -1.75) {};
		\node [style=none] (4) at (1.5, -1) {};
		\node [style=none] (5) at (0.5, -1) {};
		\node [style=scalar, fill=white] (6) at (0.5, -1) {$-1$};
	\end{pgfonlayer}
	\begin{pgfonlayer}{edgelayer}
		\draw (1.center) to (3);
		\draw (2) to (0.center);
		\draw [in=90, out=-150] (2) to (5.center);
		\draw [in=150, out=-90] (5.center) to (3);
		\draw [in=-90, out=30] (3) to (4.center);
		\draw [in=-30, out=90] (4.center) to (2);
	\end{pgfonlayer}
\end{tikzpicture}
=
\begin{tikzpicture}
	\begin{pgfonlayer}{nodelayer}
		\node [style=none] (20) at (4.85, 0.25) {};
		\node [style=none] (21) at (4.85, -3.25) {};
		\node [style=X] (22) at (4.85, -0.25) {};
		\node [style=Z] (23) at (4.85, -2.75) {};
		\node [style=scalar, fill=white] (25) at (4.425, -1.5) {$-1$};
		\node [style=none] (26) at (4.5, -0.75) {};
		\node [style=none] (27) at (5.2, -0.75) {};
		\node [style=none] (28) at (5.2, -2.25) {};
		\node [style=none] (29) at (4.5, -2.25) {};
		\node [style=none] (30) at (5.2, -1.25) {};
		\node [style=none] (31) at (5.2, -1.75) {};
		\node [style=none] (32) at (4.45, -1.25) {};
		\node [style=none] (33) at (4.45, -1.75) {};
	\end{pgfonlayer}
	\begin{pgfonlayer}{edgelayer}
		\draw (21.center) to (23);
		\draw (22) to (20.center);
		\draw [in=-30, out=90] (27.center) to (22);
		\draw [in=30, out=-90] (28.center) to (23);
		\draw [in=-90, out=150] (23) to (29.center);
		\draw [in=-150, out=90] (26.center) to (22);
		\draw (31.center) to (30.center);
		\draw (33.center) to (32.center);
		\draw [in=-90, out=90, looseness=0.75] (32.center) to (27.center);
		\draw [in=-90, out=90] (30.center) to (26.center);
		\draw [in=90, out=-90, looseness=0.75] (31.center) to (29.center);
		\draw [in=90, out=-90, looseness=0.75] (33.center) to (28.center);
	\end{pgfonlayer}
\end{tikzpicture}
=
\begin{tikzpicture}
	\begin{pgfonlayer}{nodelayer}
		\node [style=none] (0) at (1, 0.25) {};
		\node [style=none] (1) at (1, -2.25) {};
		\node [style=X] (2) at (1, -0.25) {};
		\node [style=Z] (3) at (1, -1.75) {};
		\node [style=none] (4) at (0.5, -1) {};
		\node [style=none] (5) at (1.5, -1) {};
		\node [style=scalar, fill=white] (7) at (1.5, -1) {$-1$};
	\end{pgfonlayer}
	\begin{pgfonlayer}{edgelayer}
		\draw (1.center) to (3);
		\draw (2) to (0.center);
		\draw [in=90, out=-30] (2) to (5.center);
		\draw [in=30, out=-90] (5.center) to (3);
		\draw [in=-90, out=150] (3) to (4.center);
		\draw [in=-150, out=90] (4.center) to (2);
	\end{pgfonlayer}
\end{tikzpicture}
=
\begin{tikzpicture}
	\begin{pgfonlayer}{nodelayer}
		\node [style=none] (8) at (3, 0.25) {};
		\node [style=none] (9) at (3, -2.25) {};
		\node [style=X] (10) at (3, -0.25) {};
		\node [style=Z] (11) at (3, -1.75) {};
		\node [style=none] (12) at (2.5, -1) {};
		\node [style=none] (13) at (3.5, -1) {};
		\node [style=scalar, fill=white] (14) at (3.5, -1) {$-1$};
		\node [style=scalar, fill=white] (15) at (2.5, -1) {$1$};
	\end{pgfonlayer}
	\begin{pgfonlayer}{edgelayer}
		\draw (9.center) to (11);
		\draw (10) to (8.center);
		\draw [in=90, out=-30] (10) to (13.center);
		\draw [in=30, out=-90] (13.center) to (11);
		\draw [in=-90, out=150] (11) to (12.center);
		\draw [in=-150, out=90] (12.center) to (10);
	\end{pgfonlayer}
\end{tikzpicture}
=
\begin{tikzpicture}
	\begin{pgfonlayer}{nodelayer}
		\node [style=none] (16) at (5, 0.25) {};
		\node [style=none] (17) at (5, -2.25) {};
		\node [style=scalar, fill=white] (23) at (5, -1) {$1-1$};
	\end{pgfonlayer}
	\begin{pgfonlayer}{edgelayer}
		\draw (17.center) to (23);
		\draw (23) to (16.center);
	\end{pgfonlayer}
\end{tikzpicture}
=
\begin{tikzpicture}
	\begin{pgfonlayer}{nodelayer}
		\node [style=none] (24) at (6.25, 0.25) {};
		\node [style=none] (25) at (6.25, -2.25) {};
		\node [style=scalar, fill=white] (26) at (6.25, -1) {$0$};
	\end{pgfonlayer}
	\begin{pgfonlayer}{edgelayer}
		\draw (25.center) to (26);
		\draw (26) to (24.center);
	\end{pgfonlayer}
\end{tikzpicture}
=
\begin{tikzpicture}
	\begin{pgfonlayer}{nodelayer}
		\node [style=none] (27) at (7.75, 0.25) {};
		\node [style=none] (28) at (7.75, -2.25) {};
		\node [style=X] (29) at (7.75, -0.25) {};
		\node [style=Z] (30) at (7.75, -1.75) {};
	\end{pgfonlayer}
	\begin{pgfonlayer}{edgelayer}
		\draw (28.center) to (30);
		\draw (29) to (27.center);
	\end{pgfonlayer}
\end{tikzpicture}
$$
\end{example}
If we define monoids and comonoids on composite systems as follows:
$$
\begin{tikzpicture}
	\begin{pgfonlayer}{nodelayer}
		\node [style=Z] (100) at (9.5, 1) {};
		\node [style=none] (101) at (9, 2) {};
		\node [style=none] (102) at (10, 2) {};
		\node [style=none] (103) at (9.5, 0) {};
	\end{pgfonlayer}
	\begin{pgfonlayer}{edgelayer}
		\draw [in=-90, out=150] (100) to (101.center);
		\draw [in=270, out=30] (100) to (102.center);
		\draw (100) to (103.center);
	\end{pgfonlayer}
\end{tikzpicture}
:=
\begin{tikzpicture}
	\begin{pgfonlayer}{nodelayer}
		\node [style=Z] (79) at (7, 1) {};
		\node [style=none] (84) at (7.5, 0.25) {};
		\node [style=Z] (91) at (8, 1) {};
		\node [style=none] (92) at (7, 2) {};
		\node [style=none] (93) at (8, 2) {};
		\node [style=none] (94) at (7, 2.5) {};
		\node [style=none] (95) at (8, 2.5) {};
		\node [style=none] (96) at (7.5, -0.25) {};
		\node [style=otimes] (97) at (7, 2) {};
		\node [style=otimes] (98) at (8, 2) {};
		\node [style=otimes] (99) at (7.5, 0.25) {};
	\end{pgfonlayer}
	\begin{pgfonlayer}{edgelayer}
		\draw [in=-90, out=150] (84.center) to (79);
		\draw (91) to (92.center);
		\draw (79) to (93.center);
		\draw [bend left=45, looseness=1.25] (93.center) to (91);
		\draw [bend left=45, looseness=1.25] (79) to (92.center);
		\draw (92.center) to (94.center);
		\draw (93.center) to (95.center);
		\draw [in=30, out=-90] (91) to (84.center);
		\draw (84.center) to (96.center);
	\end{pgfonlayer}
\end{tikzpicture}
\ ,\hspace*{.2cm}
\begin{tikzpicture}
	\begin{pgfonlayer}{nodelayer}
		\node [style=Z] (100) at (9.5, 1) {};
		\node [style=none] (103) at (9.5, 0) {};
	\end{pgfonlayer}
	\begin{pgfonlayer}{edgelayer}
		\draw (100) to (103.center);
	\end{pgfonlayer}
\end{tikzpicture}
:=
\begin{tikzpicture}
	\begin{pgfonlayer}{nodelayer}
		\node [style=Z] (79) at (7, 1) {};
		\node [style=none] (84) at (7.5, 0.25) {};
		\node [style=Z] (91) at (8, 1) {};
		\node [style=none] (96) at (7.5, -0.25) {};
		\node [style=otimes] (99) at (7.5, 0.25) {};
	\end{pgfonlayer}
	\begin{pgfonlayer}{edgelayer}
		\draw [in=-90, out=150] (84.center) to (79);
		\draw [in=30, out=-90] (91) to (84.center);
		\draw (84.center) to (96.center);
	\end{pgfonlayer}
\end{tikzpicture}
\ ,\hspace*{.2cm}
\begin{tikzpicture}[yscale=-1]
	\begin{pgfonlayer}{nodelayer}
		\node [style=X] (100) at (9.5, 1) {};
		\node [style=none] (101) at (9, 2) {};
		\node [style=none] (102) at (10, 2) {};
		\node [style=none] (103) at (9.5, 0) {};
	\end{pgfonlayer}
	\begin{pgfonlayer}{edgelayer}
		\draw [in=-90, out=150] (100) to (101.center);
		\draw [in=270, out=30] (100) to (102.center);
		\draw (100) to (103.center);
	\end{pgfonlayer}
\end{tikzpicture}
:=
\begin{tikzpicture}[yscale=-1]
	\begin{pgfonlayer}{nodelayer}
		\node [style=X] (79) at (7, 1) {};
		\node [style=none] (84) at (7.5, 0.25) {};
		\node [style=X] (91) at (8, 1) {};
		\node [style=none] (92) at (7, 2) {};
		\node [style=none] (93) at (8, 2) {};
		\node [style=none] (94) at (7, 2.5) {};
		\node [style=none] (95) at (8, 2.5) {};
		\node [style=none] (96) at (7.5, -0.25) {};
		\node [style=otimes] (97) at (7, 2) {};
		\node [style=otimes] (98) at (8, 2) {};
		\node [style=otimes] (99) at (7.5, 0.25) {};
	\end{pgfonlayer}
	\begin{pgfonlayer}{edgelayer}
		\draw [in=-90, out=150] (84.center) to (79);
		\draw (91) to (92.center);
		\draw (79) to (93.center);
		\draw [bend left=45, looseness=1.25] (93.center) to (91);
		\draw [bend left=45, looseness=1.25] (79) to (92.center);
		\draw (92.center) to (94.center);
		\draw (93.center) to (95.center);
		\draw [in=30, out=-90] (91) to (84.center);
		\draw (84.center) to (96.center);
	\end{pgfonlayer}
\end{tikzpicture}
\ ,\hspace*{.2cm}
\begin{tikzpicture}[yscale=-1]
	\begin{pgfonlayer}{nodelayer}
		\node [style=X] (100) at (9.5, 1) {};
		\node [style=none] (103) at (9.5, 0) {};
	\end{pgfonlayer}
	\begin{pgfonlayer}{edgelayer}
		\draw (100) to (103.center);
	\end{pgfonlayer}
\end{tikzpicture}
:=
\begin{tikzpicture}[yscale=-1]
	\begin{pgfonlayer}{nodelayer}
		\node [style=X] (79) at (7, 1) {};
		\node [style=none] (84) at (7.5, 0.25) {};
		\node [style=X] (91) at (8, 1) {};
		\node [style=none] (96) at (7.5, -0.25) {};
		\node [style=otimes] (99) at (7.5, 0.25) {};
	\end{pgfonlayer}
	\begin{pgfonlayer}{edgelayer}
		\draw [in=-90, out=150] (84.center) to (79);
		\draw [in=30, out=-90] (91) to (84.center);
		\draw (84.center) to (96.center);
	\end{pgfonlayer}
\end{tikzpicture}
$$
Then arbitrary  matrices $M:n\to m$ are natural with respect to these families of maps:
$$
\begin{tikzpicture}
	\begin{pgfonlayer}{nodelayer}
		\node [style=Z] (12) at (2, 0) {};
		\node [style=none] (15) at (2.5, 0.75) {};
		\node [style=none] (17) at (2, -0.75) {};
		\node [style=none] (18) at (1.5, 0.75) {};
		\node [style=map](19) at (2, -0.75) {$M$};
		\node [style=none] (20) at (2, -1.5) {};
	\end{pgfonlayer}
	\begin{pgfonlayer}{edgelayer}
		\draw (17.center) to (12);
		\draw [in=-90, out=30] (12) to (15.center);
		\draw [in=150, out=-90] (18.center) to (12);
		\draw (20.center) to (17.center);
	\end{pgfonlayer}
\end{tikzpicture}
=
\begin{tikzpicture}
	\begin{pgfonlayer}{nodelayer}
		\node [style=Z] (21) at (3.75, -0.75) {};
		\node [style=none] (22) at (4.25, 0) {};
		\node [style=none] (23) at (3.75, -1.5) {};
		\node [style=none] (24) at (3.25, 0) {};
		\node [style=map](25) at (3.25, 0) {$M$};
		\node [style=map](27) at (4.25, 0) {$M$};
		\node [style=none] (28) at (3.25, 0.75) {};
		\node [style=none] (29) at (4.25, 0.75) {};
	\end{pgfonlayer}
	\begin{pgfonlayer}{edgelayer}
		\draw (23.center) to (21);
		\draw [in=-90, out=30] (21) to (22.center);
		\draw [in=150, out=-90] (24.center) to (21);
		\draw (28.center) to (25);
		\draw (29.center) to (27);
	\end{pgfonlayer}
\end{tikzpicture}
\ ,
\hspace*{.2cm}
\begin{tikzpicture}
	\begin{pgfonlayer}{nodelayer}
		\node [style=Z] (0) at (2, 0) {};
		\node [style=none] (2) at (2, -0.75) {};
		\node [style=map](4) at (2, -0.75) {$M$};
		\node [style=none] (5) at (2, -1.5) {};
	\end{pgfonlayer}
	\begin{pgfonlayer}{edgelayer}
		\draw (2.center) to (0);
		\draw (5.center) to (2.center);
	\end{pgfonlayer}
\end{tikzpicture}
=
\begin{tikzpicture}
	\begin{pgfonlayer}{nodelayer}
		\node [style=Z] (6) at (3.75, -0.75) {};
		\node [style=none] (8) at (3.75, -1.5) {};
	\end{pgfonlayer}
	\begin{pgfonlayer}{edgelayer}
		\draw (8.center) to (6);
	\end{pgfonlayer}
\end{tikzpicture}
\ ,
\hspace*{.2cm}
\begin{tikzpicture}
	\begin{pgfonlayer}{nodelayer}
		\node [style=X] (36) at (7.5, 0) {};
		\node [style=none] (37) at (8, -0.75) {};
		\node [style=none] (38) at (7.5, 0.75) {};
		\node [style=none] (39) at (7, -0.75) {};
		\node [style=map](40) at (7, -0.75) {$M$};
		\node [style=map](41) at (8, -0.75) {$M$};
		\node [style=none] (42) at (7, -1.5) {};
		\node [style=none] (43) at (8, -1.5) {};
	\end{pgfonlayer}
	\begin{pgfonlayer}{edgelayer}
		\draw (38.center) to (36);
		\draw [in=90, out=-30] (36) to (37.center);
		\draw [in=-150, out=90] (39.center) to (36);
		\draw (42.center) to (40);
		\draw (43.center) to (41);
	\end{pgfonlayer}
\end{tikzpicture}
=
\begin{tikzpicture}
	\begin{pgfonlayer}{nodelayer}
		\node [style=X] (30) at (5.75, -0.75) {};
		\node [style=none] (31) at (6.25, -1.5) {};
		\node [style=none] (32) at (5.75, 0) {};
		\node [style=none] (33) at (5.25, -1.5) {};
		\node [style=map](34) at (5.75, 0) {$M$};
		\node [style=none] (35) at (5.75, 0.75) {};
	\end{pgfonlayer}
	\begin{pgfonlayer}{edgelayer}
		\draw (32.center) to (30);
		\draw [in=90, out=-30] (30) to (31.center);
		\draw [in=-150, out=90] (33.center) to (30);
		\draw (35.center) to (32.center);
	\end{pgfonlayer}
\end{tikzpicture}
\ ,
\hspace*{.2cm}
\begin{tikzpicture}
	\begin{pgfonlayer}{nodelayer}
		\node [style=X] (15) at (7.5, 0) {};
		\node [style=none] (17) at (7.5, 0.75) {};
	\end{pgfonlayer}
	\begin{pgfonlayer}{edgelayer}
		\draw (17.center) to (15);
	\end{pgfonlayer}
\end{tikzpicture}
=
\begin{tikzpicture}
	\begin{pgfonlayer}{nodelayer}
		\node [style=X] (9) at (5.75, -0.75) {};
		\node [style=none] (11) at (5.75, 0) {};
		\node [style=map](13) at (5.75, 0) {$M$};
		\node [style=none] (14) at (5.75, 0.75) {};
	\end{pgfonlayer}
	\begin{pgfonlayer}{edgelayer}
		\draw (11.center) to (9);
		\draw (14.center) to (11.center);
	\end{pgfonlayer}
\end{tikzpicture}
$$
Therefore, given two parallel matrices  $M$ and $N$, their sum is given by convolution with the bialgebra:
$$
\begin{tikzpicture}
	\begin{pgfonlayer}{nodelayer}
		\node [style=Z] (79) at (7, 1) {};
		\node [style=X] (80) at (7, 3) {};
		\node [style=map] (81) at (6.5, 2) {$M$};
		\node [style=map] (82) at (7.5, 2) {$N$};
		\node [style=none] (83) at (7, 3.75) {};
		\node [style=none] (84) at (7, 0.25) {};
	\end{pgfonlayer}
	\begin{pgfonlayer}{edgelayer}
		\draw [in=-30, out=90] (82) to (80);
		\draw (80) to (83.center);
		\draw [in=90, out=-150] (80) to (81);
		\draw [in=150, out=-90] (81) to (79);
		\draw [in=-90, out=30] (79) to (82);
		\draw (84.center) to (79);
	\end{pgfonlayer}
\end{tikzpicture}
=
\begin{tikzpicture}
	\begin{pgfonlayer}{nodelayer}
		\node [style=map] (87) at (9, 2) {$M+N$};
		\node [style=none] (89) at (9, 3.75) {};
		\node [style=none] (90) at (9, 0.25) {};
	\end{pgfonlayer}
	\begin{pgfonlayer}{edgelayer}
		\draw (90.center) to (87);
		\draw (87) to (89.center);
	\end{pgfonlayer}
\end{tikzpicture}
$$

One can perform matrix multiplication by pulling all of the white generators to the bottom and grey generators to the top so that the elements of the commutative semiring live in the middle.  Consider the following example:
$$
\begin{tikzpicture}
	\begin{pgfonlayer}{nodelayer}
		\node [style=Z] (12) at (11.5, 0.5) {};
		\node [style=Z] (13) at (12.25, -2.25) {};
		\node [style=X] (14) at (11.5, -1) {};
		\node [style=scalar,fill=white] (15) at (11.5, -0.25) {$a$};
		\node [style=none] (16) at (11, -3) {};
		\node [style=none] (17) at (12.25, -3) {};
		\node [style=none] (18) at (11, 2) {};
		\node [style=X] (19) at (12.25, 1.25) {};
		\node [style=none] (20) at (12.25, 2) {};
		\node [style=scalar,fill=white] (21) at (12.5, -0.25) {$b$};
		\node [style=Z] (22) at (13, -1.25) {};
		\node [style=Z] (23) at (13.5, -0.25) {};
	\end{pgfonlayer}
	\begin{pgfonlayer}{edgelayer}
		\draw (12) to (15);
		\draw (15) to (14);
		\draw [in=-120, out=90] (16.center) to (14);
		\draw (14) to (13);
		\draw (13) to (17.center);
		\draw [in=-90, out=120] (12) to (18.center);
		\draw (12) to (19);
		\draw (19) to (20.center);
		\draw [in=90, out=-60] (19) to (21);
		\draw [in=135, out=-90] (21) to (22);
		\draw [in=45, out=-90] (22) to (13);
		\draw [in=-90, out=30] (22) to (23);
	\end{pgfonlayer}
\end{tikzpicture}
=
\begin{tikzpicture}
	\begin{pgfonlayer}{nodelayer}
		\node [style=Z] (24) at (15, -0.25) {};
		\node [style=Z] (25) at (16.25, -1.5) {};
		\node [style=X] (26) at (15, -1) {};
		\node [style=none] (28) at (14.5, -2.25) {};
		\node [style=none] (29) at (16.25, -2.25) {};
		\node [style=none] (30) at (14.5, 0.5) {};
		\node [style=X] (31) at (16.25, 2) {};
		\node [style=none] (32) at (16.25, 2.75) {};
		\node [style=scalar,fill=white] (34) at (14.5, 0.5) {$a$};
		\node [style=scalar,fill=white] (35) at (16.25, 2.75) {$b$};
		\node [style=scalar,fill=white] (36) at (15.5, 0.5) {$a$};
		\node [style=scalar,fill=white] (37) at (15.5, 1.25) {$b^{-1}$};
		\node [style=none] (38) at (14.5, 3.5) {};
		\node [style=none] (39) at (16.25, 3.5) {};
	\end{pgfonlayer}
	\begin{pgfonlayer}{edgelayer}
		\draw [in=-120, out=90] (28.center) to (26);
		\draw (26) to (25);
		\draw (25) to (29.center);
		\draw [in=-90, out=135] (24) to (30.center);
		\draw (31) to (32.center);
		\draw [in=-90, out=45] (24) to (36);
		\draw (38.center) to (34);
		\draw (36) to (37);
		\draw [in=-150, out=90] (37) to (31);
		\draw [in=60, out=-60] (31) to (25);
		\draw (26) to (24);
		\draw (39.center) to (35);
	\end{pgfonlayer}
\end{tikzpicture}
=
\begin{tikzpicture}
	\begin{pgfonlayer}{nodelayer}
		\node [style=Z] (41) at (19.5, -2) {};
		\node [style=none] (43) at (18, -3) {};
		\node [style=none] (44) at (19.5, -2.75) {};
		\node [style=none] (45) at (18, 0.5) {};
		\node [style=X] (46) at (19.75, 2) {};
		\node [style=none] (47) at (19.75, 2.75) {};
		\node [style=scalar,fill=white] (48) at (18, 0.5) {$a$};
		\node [style=scalar,fill=white] (49) at (19.75, 2.75) {$b$};
		\node [style=scalar,fill=white] (50) at (19, 0.5) {$a$};
		\node [style=scalar,fill=white] (51) at (19, 1.25) {$b^{-1}$};
		\node [style=none] (52) at (18, 3.5) {};
		\node [style=none] (53) at (19.75, 3.5) {};
		\node [style=Z] (54) at (18, -1.5) {};
		\node [style=X] (55) at (18, -0.25) {};
		\node [style=Z] (56) at (19, -1.5) {};
		\node [style=X] (57) at (19, -0.25) {};
	\end{pgfonlayer}
	\begin{pgfonlayer}{edgelayer}
		\draw (41) to (44.center);
		\draw (46) to (47.center);
		\draw (52.center) to (48);
		\draw (50) to (51);
		\draw [in=-150, out=90] (51) to (46);
		\draw [in=45, out=-60, looseness=0.75] (46) to (41);
		\draw (53.center) to (49);
		\draw (56) to (55);
		\draw [bend right] (55) to (54);
		\draw (57) to (54);
		\draw (55) to (48);
		\draw (50) to (57);
		\draw (54) to (43.center);
		\draw [in=150, out=-90] (56) to (41);
		\draw [bend left] (57) to (56);
	\end{pgfonlayer}
\end{tikzpicture}
=
\begin{tikzpicture}
	\begin{pgfonlayer}{nodelayer}
		\node [style=Z] (58) at (22.75, -2.5) {};
		\node [style=none] (59) at (21.25, -3) {};
		\node [style=none] (60) at (22.75, -3) {};
		\node [style=none] (61) at (21.15, 3.5) {};
		\node [style=X] (62) at (23.125, 1) {};
		\node [style=scalar,fill=white] (65) at (23.125, 2.5) {$b$};
		\node [style=none] (69) at (23.125, 3.5) {};
		\node [style=Z] (70) at (21.25, -1.75) {};
		\node [style=X] (71) at (21.15, 0.25) {};
		\node [style=Z] (72) at (22.25, -1.75) {};
		\node [style=X] (73) at (22.6, 0.25) {};
		\node [style=scalar,fill=white] (75) at (23.125, 1.75) {$b^{-1}$};
		\node [style=scalar,fill=white] (76) at (23.75, -0.75) {$b$};
		\node [style=scalar,fill=white] (77) at (22.25, -0.75) {$a$};
		\node [style=scalar,fill=white] (78) at (23, -0.75) {$a$};
		\node [style=scalar,fill=white] (79) at (20.75, -0.75) {$a$};
		\node [style=scalar,fill=white] (80) at (21.5, -0.75) {$a$};
	\end{pgfonlayer}
	\begin{pgfonlayer}{edgelayer}
		\draw (58) to (60.center);
		\draw (69.center) to (65);
		\draw (70) to (59.center);
		\draw [in=150, out=-90] (72) to (58);
		\draw [in=-90, out=30] (70) to (77);
		\draw [in=-135, out=90] (77) to (73);
		\draw [in=90, out=-45] (73) to (78);
		\draw [in=30, out=-90] (78) to (72);
		\draw [in=-165, out=90] (73) to (62);
		\draw (62) to (75);
		\draw (75) to (65);
		\draw [in=90, out=-30, looseness=0.75] (62) to (76);
		\draw [in=30, out=-90, looseness=0.75] (76) to (58);
		\draw [in=-90, out=150] (72) to (80);
		\draw [in=-45, out=90] (80) to (71);
		\draw [in=90, out=-150] (71) to (79);
		\draw [in=135, out=-90] (79) to (70);
		\draw (71) to (61.center);
	\end{pgfonlayer}
\end{tikzpicture}
=
\begin{tikzpicture}
	\begin{pgfonlayer}{nodelayer}
		\node [style=Z] (81) at (26.75, -2.5) {};
		\node [style=none] (82) at (25.25, -3) {};
		\node [style=none] (83) at (26.75, -3) {};
		\node [style=none] (84) at (25.15, 1.75) {};
		\node [style=X] (85) at (27.125, 1) {};
		\node [style=none] (87) at (27.125, 1.75) {};
		\node [style=Z] (88) at (25.25, -1.75) {};
		\node [style=X] (89) at (25.15, 0.25) {};
		\node [style=Z] (90) at (26.25, -1.75) {};
		\node [style=X] (91) at (26.6, 0.25) {};
		\node [style=scalar,fill=white] (93) at (27.75, -0.75) {$b$};
		\node [style=scalar,fill=white] (94) at (26.25, -0.75) {$a$};
		\node [style=scalar,fill=white] (95) at (27, -0.75) {$a$};
		\node [style=scalar,fill=white] (96) at (24.75, -0.75) {$a$};
		\node [style=scalar,fill=white] (97) at (25.5, -0.75) {$a$};
	\end{pgfonlayer}
	\begin{pgfonlayer}{edgelayer}
		\draw (81) to (83.center);
		\draw (88) to (82.center);
		\draw [in=150, out=-90] (90) to (81);
		\draw [in=-90, out=30] (88) to (94);
		\draw [in=-135, out=90] (94) to (91);
		\draw [in=90, out=-45] (91) to (95);
		\draw [in=30, out=-90] (95) to (90);
		\draw [in=-165, out=90] (91) to (85);
		\draw [in=90, out=-30, looseness=0.75] (85) to (93);
		\draw [in=30, out=-90, looseness=0.75] (93) to (81);
		\draw [in=-90, out=150] (90) to (97);
		\draw [in=-45, out=90] (97) to (89);
		\draw [in=90, out=-150] (89) to (96);
		\draw [in=135, out=-90] (96) to (88);
		\draw (89) to (84.center);
		\draw (85) to (87.center);
	\end{pgfonlayer}
\end{tikzpicture}
$$
This makes it clear how to interpret this as a matrix. Follow the wires from the bottom and chase their paths to the top, copying them when they meet white nodes, adding them when they meet grey ones, and multiplying them when they meet scalars:

\begin{example}
$$
\left\llbracket \
\begin{tikzpicture}
	\begin{pgfonlayer}{nodelayer}
		\node [style=Z] (10) at (35.75, -2.125) {};
		\node [style=none] (11) at (31, -2.625) {};
		\node [style=none] (12) at (35.75, -2.625) {};
		\node [style=none] (13) at (31.15, 1.75) {};
		\node [style=X] (14) at (35.625, 1.125) {};
		\node [style=none] (15) at (35.625, 1.75) {};
		\node [style=Z] (16) at (31, -1.9) {};
		\node [style=X] (17) at (31.15, 0.5) {};
		\node [style=Z] (18) at (34.75, -1.65) {};
		\node [style=X] (19) at (34.85, 0.5) {};
		\node [style=scalar] (20) at (36.75, -0.5) {$b$};
		\node [style=scalar] (21) at (34.25, -0.5) {$a$};
		\node [style=scalar] (22) at (35.5, -0.5) {$a$};
		\node [style=scalar] (23) at (30.5, -0.5) {$a$};
		\node [style=scalar] (24) at (31.75, -0.5) {$a$};
		\node [style=none, color=blue] (25) at (31, -2.875) {$x_1$};
		\node [style=none, color=blue] (26) at (35.75, -2.875) {$x_2$};
		\node [style=none, color=blue] (27) at (37.075, -1.025) {$x_2$};
		\node [style=none, color=blue] (28) at (35.75, -1.025) {$x_2$};
		\node [style=none, color=blue] (29) at (34.5, -1.025) {$x_1$};
		\node [style=none, color=blue] (30) at (30.1, -0.975) {$x_1$};
		\node [style=none, color=blue] (31) at (32.35, -0.95) {$x_2$};
		\node [style=none, color=blue] (32) at (37.275, 0) {$x_2\cdot b$};
		\node [style=none, color=blue] (33) at (35.925, 0) {$x_2\cdot a$};
		\node [style=none, color=blue] (34) at (33.8, 0.05) {$x_1\cdot a$};
		\node [style=none, color=blue] (35) at (32.2, 0.05) {$x_2\cdot a$};
		\node [style=none, color=blue] (36) at (30, 0.05) {$x_1\cdot a$};
		\node [style=none, color=blue] (37) at (33.725, 0.925) {$x_1\cdot a+x_2\cdot a$};
		\node [style=none, color=blue] (38) at (35.875, 2.55) {$x_1\cdot a+x_2\cdot a+x_2\cdot b$};
		\node [style=none, color=blue] (39) at (31.25, 2.125) {$x_1\cdot a+x_2\cdot a$};
		\node [style=none, color=blue] (40) at (36.025, 2.125) {$=x_1\cdot a+x_2\cdot (a+b)$};
	\end{pgfonlayer}
	\begin{pgfonlayer}{edgelayer}
		\draw (10) to (12.center);
		\draw (16) to (11.center);
		\draw (18) to (10);
		\draw [in=-90, out=15, looseness=0.50] (16) to (21);
		\draw [in=-135, out=90] (21) to (19);
		\draw [in=90, out=-45] (19) to (22);
		\draw [in=30, out=-90] (22) to (18);
		\draw [in=-165, out=90] (19) to (14);
		\draw [in=90, out=-30, looseness=0.75] (14) to (20);
		\draw [in=30, out=-90, looseness=0.75] (20) to (10);
		\draw [in=-90, out=165, looseness=0.75] (18) to (24);
		\draw [in=-30, out=90] (24) to (17);
		\draw [in=90, out=-165] (17) to (23);
		\draw [in=135, out=-90] (23) to (16);
		\draw (17) to (13.center);
		\draw (14) to (15.center);
	\end{pgfonlayer}
\end{tikzpicture}
\ \right\rrbracket
=
\begin{bmatrix}
a & a\\
a & a+b
\end{bmatrix}
$$
Where
$$
\begin{bmatrix}
a & a\\
a & a+b
\end{bmatrix}
\begin{bmatrix}
x_1\\
x_2
\end{bmatrix}
=
\begin{bmatrix}
x_1 \cdot a+x_2\cdot a\\
x_1\cdot a+x_2\cdot (a+b)
\end{bmatrix}
$$
\end{example}

Matrices can be generalized to have no fixed origin:
\begin{definition}
\label{def:affmat}
Given a commutative semiring $R$, the prop of affine matrices over $R$, $\Aff\Mat_R$ has:
\begin{description}
\item[Objects:] Natural numbers.
\item[Maps:] A map $(M,a):n\to m$ is a pair of a matrix $M:n\to m$ and a vector $1\to m$.
\item[Identity:] The identity on an object $n$ is the pair
$$(I_n,{\bf 0}:1\to n)$$
 where $I_n$ is the identity matrix and $\bf 0$ is the zero vector.
\item[Composition]
$$
\dfrac{
n \xrightarrow{(M,a)} m\ , \hspace*{.5cm} m \xrightarrow {(N,b)} k
}{
n \xrightarrow{(M,a);(N,b):=(NM, Na+b)} k
}
$$
\item[Monoidal structure:]  The tensor product is given pointwise:
 $$(M,a)\oplus (N,b):=(M\oplus n, a\oplus b)$$ 
The tensor unit is the identity on $0$.
\end{description}
\end{definition}
Note that  $\Mat_R$ faithfully embeds into $\Aff\Mat_R$:
$$
\left(n \xrightarrow{M} m\right)
\mapsto 
\left(n \xrightarrow{(M,{\bf 0})} m\right)
$$
In the other direction, there is also a fathful embedding $\Aff\Mat_R\to \Mat_R$ taking an affine matrix to its {\bf augmented matrix}:
$$
\left(n\xrightarrow{(M,a)} m\right)
\mapsto
\left(
\vcenter{\hbox{$\displaystyle{
n+1
\xrightarrow[ . ]{
\begin{bmatrix}
M & a\\
\mathbf{0} & 1
\end{bmatrix}}
m+1}$}}
\right)
$$
I can not find a reference for the following result, but it is is an immediate consequence of the analysis of affine relations in \cite{affine}:
\begin{example}
\label{ex:affmat}
Given a commutative semiring $R$, the prop $\acb_R$ is presented by adding the following generators and relations to $\cb_R$:
$$
\begin{tikzpicture}
	\begin{pgfonlayer}{nodelayer}
		\node [style=none] (0) at (0.5, 0.5) {};
		\node [style=X] (2) at (0.5, -0.25) {$1$};
		\node [style=Z] (3) at (0.5, 0.5) {};
		\node [style=none] (4) at (0, 1.25) {};
		\node [style=none] (5) at (1, 1.25) {};
	\end{pgfonlayer}
	\begin{pgfonlayer}{edgelayer}
		\draw (2) to (0.center);
		\draw [in=270, out=150] (3) to (4.center);
		\draw [in=270, out=30] (3) to (5.center);
	\end{pgfonlayer}
\end{tikzpicture}
\erefop{bi.two}
\begin{tikzpicture}
	\begin{pgfonlayer}{nodelayer}
		\node [style=X] (7) at (2, -0.25) {$1$};
		\node [style=none] (9) at (2, 1.25) {};
		\node [style=none] (10) at (3, 1.25) {};
		\node [style=X] (11) at (3, -0.25) {$1$};
	\end{pgfonlayer}
	\begin{pgfonlayer}{edgelayer}
		\draw (7) to (9.center);
		\draw (11) to (10.center);
	\end{pgfonlayer}
\end{tikzpicture},
\hspace*{.5cm}
\begin{tikzpicture}
	\begin{pgfonlayer}{nodelayer}
		\node [style=none] (0) at (0.5, 0.5) {};
		\node [style=X] (2) at (0.5, -0.25) {$1$};
		\node [style=Z] (3) at (0.5, 0.5) {};
	\end{pgfonlayer}
	\begin{pgfonlayer}{edgelayer}
		\draw (2) to (0.center);
	\end{pgfonlayer}
\end{tikzpicture}
\eref{extra}
\begin{tikzpicture}
	\begin{pgfonlayer}{nodelayer}
		\node [style=none] (0) at (2, 0) {};
		\node [style=none] (1) at (2, -1) {};
		\node [style=none] (2) at (3, -1) {};
		\node [style=none] (3) at (3, 0) {};
	\end{pgfonlayer}
	\begin{pgfonlayer}{edgelayer}
		\draw[style=dashed] (3.center) to (0.center) to (1.center) to (2.center) to cycle;
	\end{pgfonlayer}
\end{tikzpicture}
$$
This is a presentation for the prop of affine matrices over $S$.  This new generator is interpreted as the affine shift.
\end{example}
%The way we have described adding a new generator to $\cb_R$ in terms of the interaction with the monoid $\zcirc$ is secrely a pushout over $\cm^\op$.  We have specified the interaction of the new generator with the commutative monoid structure, embedded it in $\cb_R$ and then taken the free prop given by these generators and equations.
Given an affine matrix $(M,a)$, it is represented by  the following string diagram:
$$
\begin{tikzpicture}
	\begin{pgfonlayer}{nodelayer}
		\node [style=map] (0) at (0, 0) {$M$};
		\node [style=map] (1) at (1, 0) {$a$};
		\node [style=X] (2) at (0.5, 1) {};
		\node [style=X] (3) at (1, -0.75) {$1$};
		\node [style=none] (4) at (0, -1) {};
		\node [style=none] (5) at (0.5, 1.75) {};
	\end{pgfonlayer}
	\begin{pgfonlayer}{edgelayer}
		\draw (3) to (1);
		\draw [in=-30, out=90] (1) to (2);
		\draw (2) to (5.center);
		\draw [in=90, out=-150] (2) to (0);
		\draw (0) to (4.center);
	\end{pgfonlayer}
\end{tikzpicture}
$$
So that the composite of two affine matrices can be computed diagrammatically:
$$
\begin{tikzpicture}
	\begin{pgfonlayer}{nodelayer}
		\node [style=map] (0) at (0.5, 0) {$M$};
		\node [style=map] (1) at (1.5, 0) {$a$};
		\node [style=X] (2) at (1, 1) {};
		\node [style=X] (3) at (1.5, -0.75) {$1$};
		\node [style=none] (4) at (0.5, -1) {};
		\node [style=map] (6) at (1, 2) {$N$};
		\node [style=map] (7) at (2, 2) {$b$};
		\node [style=X] (8) at (1.5, 3) {};
		\node [style=X] (9) at (2, 1.25) {$1$};
		\node [style=none] (10) at (1, 1) {};
		\node [style=none] (11) at (1.5, 3.75) {};
	\end{pgfonlayer}
	\begin{pgfonlayer}{edgelayer}
		\draw (3) to (1);
		\draw [in=-30, out=90] (1) to (2);
		\draw [in=90, out=-150] (2) to (0);
		\draw (0) to (4.center);
		\draw (9) to (7);
		\draw [in=-30, out=90] (7) to (8);
		\draw (8) to (11.center);
		\draw [in=90, out=-150] (8) to (6);
		\draw (6) to (10.center);
	\end{pgfonlayer}
\end{tikzpicture}
=
\begin{tikzpicture}
	\begin{pgfonlayer}{nodelayer}
		\node [style=map] (12) at (3.5, 0) {$N$};
		\node [style=map] (13) at (4.5, 0) {$N$};
		\node [style=X] (14) at (4, 1) {};
		\node [style=map] (18) at (5, 2) {$b$};
		\node [style=X] (19) at (4.5, 3) {};
		\node [style=X] (20) at (5, 1.25) {$1$};
		\node [style=none] (21) at (4, 2) {};
		\node [style=none] (22) at (4.5, 3.75) {};
		\node [style=X] (23) at (4.5, -1.5) {$1$};
		\node [style=none] (24) at (3.5, -1.75) {};
		\node [style=map] (25) at (3.5, -0.75) {$M$};
		\node [style=map] (26) at (4.5, -0.75) {$a$};
	\end{pgfonlayer}
	\begin{pgfonlayer}{edgelayer}
		\draw [in=-30, out=90] (13) to (14);
		\draw [in=90, out=-150] (14) to (12);
		\draw (20) to (18);
		\draw [in=-30, out=90] (18) to (19);
		\draw (19) to (22.center);
		\draw [in=-150, out=90] (21.center) to (19);
		\draw (21.center) to (14);
		\draw (24.center) to (25);
		\draw (25) to (12);
		\draw (26) to (13);
		\draw (23) to (26);
	\end{pgfonlayer}
\end{tikzpicture}
=
\begin{tikzpicture}
	\begin{pgfonlayer}{nodelayer}
		\node [style=map] (27) at (6, 1) {$N$};
		\node [style=map] (28) at (7, 1) {$N$};
		\node [style=X] (29) at (6.5, 2) {};
		\node [style=X] (31) at (7.25, 3) {};
		\node [style=none] (33) at (6.5, 2) {};
		\node [style=none] (34) at (7.25, 3.75) {};
		\node [style=none] (36) at (6, -1.5) {};
		\node [style=map] (37) at (6, 0.25) {$M$};
		\node [style=map] (38) at (7, 0.25) {$a$};
		\node [style=X] (39) at (7.5, -1.25) {$1$};
		\node [style=none] (40) at (8, 2) {};
		\node [style=map] (41) at (8, 0.25) {$b$};
		\node [style=Z] (42) at (7.5, -0.5) {};
	\end{pgfonlayer}
	\begin{pgfonlayer}{edgelayer}
		\draw [in=-30, out=90] (28) to (29);
		\draw [in=90, out=-150] (29) to (27);
		\draw (31) to (34.center);
		\draw [in=-150, out=90] (33.center) to (31);
		\draw (36.center) to (37);
		\draw (37) to (27);
		\draw (38) to (28);
		\draw [in=90, out=-30] (31) to (40.center);
		\draw (41) to (40.center);
		\draw (39) to (42);
		\draw [in=-90, out=30] (42) to (41);
		\draw [in=-90, out=150] (42) to (38);
	\end{pgfonlayer}
\end{tikzpicture}
=
\begin{tikzpicture}
	\begin{pgfonlayer}{nodelayer}
		\node [style=map] (43) at (9, 1.5) {$N$};
		\node [style=map] (44) at (9.75, 1.5) {$N$};
		\node [style=none] (48) at (9.75, 3.75) {};
		\node [style=none] (49) at (9, -1) {};
		\node [style=map] (50) at (9, 0.75) {$M$};
		\node [style=map] (51) at (9.75, 0.75) {$a$};
		\node [style=X] (52) at (10.25, -0.75) {$1$};
		\node [style=map] (54) at (10.75, 0.75) {$b$};
		\node [style=Z] (55) at (10.25, 0) {};
		\node [style=X] (56) at (10.25, 2.25) {};
		\node [style=X] (57) at (9.75, 3) {};
		\node [style=none] (58) at (10.75, 1.5) {};
	\end{pgfonlayer}
	\begin{pgfonlayer}{edgelayer}
		\draw (49.center) to (50);
		\draw (50) to (43);
		\draw (51) to (44);
		\draw (52) to (55);
		\draw [in=-90, out=30] (55) to (54);
		\draw [in=-90, out=150] (55) to (51);
		\draw [in=-150, out=90] (43) to (57);
		\draw (57) to (48.center);
		\draw [in=90, out=-30] (57) to (56);
		\draw (54) to (58.center);
		\draw [in=-30, out=90] (58.center) to (56);
		\draw [in=90, out=-150] (56) to (44);
	\end{pgfonlayer}
\end{tikzpicture}
=
\begin{tikzpicture}
	\begin{pgfonlayer}{nodelayer}
		\node [style=map] (79) at (14, 1.75) {$NM$};
		\node [style=none] (80) at (14.7, 3.75) {};
		\node [style=none] (81) at (14, 0.75) {};
		\node [style=X] (82) at (15.4, 1) {$1$};
		\node [style=X] (83) at (14.7, 3) {};
		\node [style=map] (84) at (15.4, 1.75) {$Na+b$};
	\end{pgfonlayer}
	\begin{pgfonlayer}{edgelayer}
		\draw [in=-150, out=90] (79) to (83);
		\draw (83) to (80.center);
		\draw (82) to (84);
		\draw [in=-30, out=90] (84) to (83);
		\draw (81.center) to (79);
	\end{pgfonlayer}
\end{tikzpicture}
$$
The other axiom is needed for the identity law:
$$
\begin{tikzpicture}
	\begin{pgfonlayer}{nodelayer}
		\node [style=X] (8) at (1.5, 3) {};
		\node [style=X] (9) at (2, 0.75) {$1$};
		\node [style=none] (10) at (1, 2.25) {};
		\node [style=none] (11) at (1.5, 3.75) {};
		\node [style=Z] (69) at (2, 1.5) {};
		\node [style=X] (70) at (2, 2.25) {};
		\node [style=none] (71) at (1, 0.5) {};
	\end{pgfonlayer}
	\begin{pgfonlayer}{edgelayer}
		\draw (8) to (11.center);
		\draw (9) to (69);
		\draw [in=-30, out=90] (70) to (8);
		\draw [in=90, out=-150] (8) to (10.center);
		\draw (71.center) to (10.center);
	\end{pgfonlayer}
\end{tikzpicture}
=
\begin{tikzpicture}
	\begin{pgfonlayer}{nodelayer}
		\node [style=none] (74) at (3.25, 3.75) {};
		\node [style=none] (78) at (3.25, 0.5) {};
	\end{pgfonlayer}
	\begin{pgfonlayer}{edgelayer}
		\draw (78.center) to (74.center);
	\end{pgfonlayer}
\end{tikzpicture}
$$

The following structure will show up quite a lot throughout this thesis:
\begin{example}
\label{ex:frob}
A {\bf Frobenius} algebra is a monoid $\xcirc$ and comonoid $\xcirc$ interacting to satisfy the Frobenius laws:
$$
  \begin{tikzpicture}[rotate=90]
	\begin{pgfonlayer}{nodelayer}
		\node [style=X] (0) at (-7, -0) {};
		\node [style=X] (1) at (-6.25, 0.5) {};
		\node [style=none] (2) at (-7, 0.75) {};
		\node [style=none] (3) at (-7.75, 0.75) {};
		\node [style=none] (4) at (-7.75, -0) {};
		\node [style=none] (5) at (-6.25, -0.25) {};
		\node [style=none] (6) at (-5.5, -0.25) {};
		\node [style=none] (7) at (-5.5, 0.5) {};
	\end{pgfonlayer}
	\begin{pgfonlayer}{edgelayer}
		\draw (6.center) to (5.center);
		\draw [in=-30, out=180, looseness=1.00] (5.center) to (0);
		\draw (1) to (0);
		\draw [in=0, out=150, looseness=1.00] (1) to (2.center);
		\draw (2.center) to (3.center);
		\draw (0) to (4.center);
		\draw (1) to (7.center);
	\end{pgfonlayer}
  \end{tikzpicture}
 \eqzxa{frobl}
  \begin{tikzpicture}[rotate=90]
	\begin{pgfonlayer}{nodelayer}
		\node [style=none] (0) at (-4.75, -0.25) {};
		\node [style=X] (1) at (-5.5, -0) {};
		\node [style=none] (2) at (-7, -0.25) {};
		\node [style=X] (3) at (-6.25, 0) {};
		\node [style=none] (4) at (-4.75, 0.25) {};
		\node [style=none] (5) at (-7, 0.25) {};
	\end{pgfonlayer}
	\begin{pgfonlayer}{edgelayer}
		\draw [in=-30, out=180, looseness=1.25] (0.center) to (1);
		\draw (3) to (1);
		\draw [in=180, out=30, looseness=1.25] (1) to (4.center);
		\draw [in=0, out=-150, looseness=1.25] (3) to (2.center);
		\draw [in=0, out=150, looseness=1.25] (3) to (5.center);
	\end{pgfonlayer}
\end{tikzpicture}
  \eqzxa{frobr}
 \begin{tikzpicture}[rotate=90,xscale=-1]
	\begin{pgfonlayer}{nodelayer}
		\node [style=X] (0) at (-7, -0) {};
		\node [style=X] (1) at (-6.25, 0.5) {};
		\node [style=none] (2) at (-7, 0.75) {};
		\node [style=none] (3) at (-7.75, 0.75) {};
		\node [style=none] (4) at (-7.75, -0) {};
		\node [style=none] (5) at (-6.25, -0.25) {};
		\node [style=none] (6) at (-5.5, -0.25) {};
		\node [style=none] (7) at (-5.5, 0.5) {};
	\end{pgfonlayer}
	\begin{pgfonlayer}{edgelayer}
		\draw (6.center) to (5.center);
		\draw [in=-30, out=180, looseness=1.00] (5.center) to (0);
		\draw (1) to (0);
		\draw [in=0, out=150, looseness=1.00] (1) to (2.center);
		\draw (2.center) to (3.center);
		\draw (0) to (4.center);
		\draw (1) to (7.center);
	\end{pgfonlayer}
  \end{tikzpicture}
$$
This is moreover a {\bf special } Frobenius algebra when:
$$
  \begin{tikzpicture}[rotate=90]
	\begin{pgfonlayer}{nodelayer}
		\node [style=X] (0) at (-6.25, 0.25) {};
		\node [style=none] (1) at (-7, 0.25) {};
		\node [style=none] (2) at (-4.75, 0.25) {};
		\node [style=X] (3) at (-5.5, 0.25) {};
	\end{pgfonlayer}
	\begin{pgfonlayer}{edgelayer}
		\draw (0) to (1.center);
		\draw (3) to (2.center);
		\draw [bend right, looseness=1.25] (3) to (0);
		\draw [bend right, looseness=1.25] (0) to (3);
	\end{pgfonlayer}
  \end{tikzpicture}
  \eqzxa{special}
  \begin{tikzpicture}[rotate=90]
	\begin{pgfonlayer}{nodelayer}
		\node [style=none] (0) at (-7, 0.25) {};
		\node [style=none] (1) at (-6, 0.25) {};
	\end{pgfonlayer}
	\begin{pgfonlayer}{edgelayer}
		\draw (1.center) to (0.center);
	\end{pgfonlayer}
  \end{tikzpicture}
$$
A commutative (special) Frobenius algebra is a (special) Frobenius algebra between a commutative monoid and a cocommutative comonoid.

Denote the pro generated by a Frobenius algebra by $\fa$, and a special Frobenius algebra by $\sfa$.
Denote the prop generated by a commutative Frobenius algebra by $\cfa$, and a special commutative Frobenius algebra by $\scfa$.
\end{example}
Frobenius algebras make objects self dual:
$$
\begin{tikzpicture}
	\begin{pgfonlayer}{nodelayer}
		\node [style=X] (0) at (0, 0) {};
		\node [style=X] (1) at (0.5, -1) {};
		\node [style=X] (2) at (0, 0.5) {};
		\node [style=X] (3) at (0.5, -1.5) {};
		\node [style=none] (4) at (1, 0) {};
		\node [style=none] (5) at (-0.5, -1) {};
		\node [style=none] (6) at (1, 1) {};
		\node [style=none] (7) at (-0.5, -2) {};
	\end{pgfonlayer}
	\begin{pgfonlayer}{edgelayer}
		\draw [in=-135, out=90] (5.center) to (0);
		\draw (0) to (1);
		\draw (1) to (3);
		\draw [in=45, out=-90] (4.center) to (1);
		\draw (0) to (2);
		\draw (7.center) to (5.center);
		\draw (4.center) to (6.center);
	\end{pgfonlayer}
\end{tikzpicture}
\eref{frobl}
\begin{tikzpicture}
	\begin{pgfonlayer}{nodelayer}
		\node [style=X] (10) at (2.25, 0.5) {};
		\node [style=X] (11) at (3.25, -1.5) {};
		\node [style=none] (12) at (3.25, 0.5) {};
		\node [style=none] (13) at (2.25, -1.5) {};
		\node [style=none] (14) at (3.25, 1) {};
		\node [style=none] (15) at (2.25, -2) {};
		\node [style=X] (16) at (2.75, -0.25) {};
		\node [style=X] (17) at (2.75, -0.75) {};
	\end{pgfonlayer}
	\begin{pgfonlayer}{edgelayer}
		\draw (15.center) to (13.center);
		\draw (12.center) to (14.center);
		\draw [in=-90, out=30] (16) to (12.center);
		\draw (16) to (17);
		\draw [in=90, out=-30] (17) to (11);
		\draw [in=90, out=-150] (17) to (13.center);
		\draw [in=-90, out=150] (16) to (10);
	\end{pgfonlayer}
\end{tikzpicture}
\eq{$\ref{unitl},\ref{unitr}^\op$}
\begin{tikzpicture}
	\begin{pgfonlayer}{nodelayer}
		\node [style=none] (20) at (5.25, -2) {};
		\node [style=none] (22) at (5.25, 1) {};
	\end{pgfonlayer}
	\begin{pgfonlayer}{edgelayer}
		\draw (20.center) to (22.center);
	\end{pgfonlayer}
\end{tikzpicture}
\eq{$\ref{unitr},\ref{unitl}^\op$}
\begin{tikzpicture}
	\begin{pgfonlayer}{nodelayer}
		\node [style=X] (31) at (9.75, 0.5) {};
		\node [style=X] (32) at (8.75, -1.5) {};
		\node [style=none] (33) at (8.75, 0.5) {};
		\node [style=none] (34) at (9.75, -1.5) {};
		\node [style=none] (35) at (8.75, 1) {};
		\node [style=none] (36) at (9.75, -2) {};
		\node [style=X] (37) at (9.25, -0.25) {};
		\node [style=X] (38) at (9.25, -0.75) {};
	\end{pgfonlayer}
	\begin{pgfonlayer}{edgelayer}
		\draw (36.center) to (34.center);
		\draw (33.center) to (35.center);
		\draw [in=-90, out=150] (37) to (33.center);
		\draw (37) to (38);
		\draw [in=90, out=-150] (38) to (32);
		\draw [in=90, out=-30] (38) to (34.center);
		\draw [in=-90, out=30] (37) to (31);
	\end{pgfonlayer}
\end{tikzpicture}
\eref{frobr}
\begin{tikzpicture}
	\begin{pgfonlayer}{nodelayer}
		\node [style=X] (23) at (7.25, 0) {};
		\node [style=X] (24) at (6.75, -1) {};
		\node [style=X] (25) at (7.25, 0.5) {};
		\node [style=X] (26) at (6.75, -1.5) {};
		\node [style=none] (27) at (6.25, 0) {};
		\node [style=none] (28) at (7.75, -1) {};
		\node [style=none] (29) at (6.25, 1) {};
		\node [style=none] (30) at (7.75, -2) {};
	\end{pgfonlayer}
	\begin{pgfonlayer}{edgelayer}
		\draw [in=-45, out=90] (28.center) to (23);
		\draw (23) to (24);
		\draw (24) to (26);
		\draw [in=135, out=-90] (27.center) to (24);
		\draw (23) to (25);
		\draw (30.center) to (28.center);
		\draw (27.center) to (29.center);
	\end{pgfonlayer}
\end{tikzpicture}
$$
So that symmetric monoidal categories equipped with a supply of commutative (\dag-)Frobenius alegbras compatible with the monoidal structure is self dual (\dag-)compact closed.


