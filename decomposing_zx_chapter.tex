In this section, we modularly build up to the presentation of $\ZXA$ by taking distributive laws and pushouts of smaller symmetric monoidal theories.  Along the way, we obtain various fragments of quantum circuits with interesting semantics.


%%We first review some basic theory involving the presentation of props.  These results are mostly folklore, however, I will refer the reader to \cite[\S 2]{ih} for a more comprehensive introduction.
%
%\begin{definition}
%A {\bf pro} is a strict monoidal category generated by one object under the tensor product, and a {\bf prop} is a  strict {\em symmetric} monoidal category generated by one object under the tensor product
%
%\end{definition}

%
%\begin{definition}
%A {\bf monoidal theory} is a pair $(\Sigma,E)$ of {\bf generators} $\Sigma$ and {\bf equations} $E$.
%Each generator $f \in \Sigma$ has a chosen domain $\dom (f) \in \N$  and codomain $\cod (f) \in \N$, so that $f$ can be seen as a map from $\dom(f)$ to $\cod (f)$.
%
%The free pro with signature $\Sigma$ has maps in $\Sigma^*$ obtained by inductively  tensoring all the generators and composing all appropriately typed generators in $\Sigma$,
%The equations in $E$ are pairs of parallel maps in $\Sigma^*$.
%Any monoidal theory $(\Sigma,E)$  generates a pro $\bar{(\Sigma,E)}$ given by the free pro with signature $\Sigma$ modulo the equations in $E$.
%
%A {\bf symmetric monoidal theory} is the symmetric version of a monoidal theory, which generates a prop.  Here the set $\Sigma^*$ is obtained by composing and tensoring maps with symmetries, and then quotienting by the axioms of a prop.
%\end{definition}
%
%\begin{lemma}
%Given two (symmetric) monoidal theories $(\Sigma_1,E_1)$  and $(\Sigma_2,E_2)$  the coproduct of pro(p)s  $\bar{(\Sigma_1,E_1)}+\bar{(\Sigma_2,E_2)}$ is generated by the (symmetric) monoidal theory $(\Sigma_1+\Sigma_2,E_1+E_2)$.
%\end{lemma}
%
%
%\begin{lemma}
%Given three  (symmetric) monoidal theories $(\Sigma_1,E_1)$, $(\Sigma_2,E_2)$ and $(\Sigma_3,E_3)$ where $\bar{(\Sigma_3,E_3)}$ is a sub-pro(p) of both $\bar{(\Sigma_1,E_1)}$ and $\bar{(\Sigma_2,E_2)}$.  The pushout of the diagram of pro(p)s
%$$
%\bar{(\Sigma_1,E_1)} \leftarrow \bar{(\Sigma_3,E_3)}\rightarrow \bar{(\Sigma_2,E_2)}
%$$
%is generated by the (symmetric) monoidal theory $(\Sigma_1^* +_{\Sigma_3} \Sigma_2^*, E_1 + E_2)$.
%\end{lemma}

%
%\begin{definition}
%Informally, given two small categories $\X,\Y$ with the same objects; a distributive law on $\Y \otimes \X$ is a way to turn formal composites of maps in $\Y$ followed by $\X$ into a category.  That is to say, a quotient which gives us a way to turn composites of the form $\xrightarrow{f \in \X} \xrightarrow{g \in \Y}$  into ones of the form  $\xrightarrow{g' \in \Y} \xrightarrow{f' \in \X}$.
%
%If $\X$ and $\Y$ have some shared structure witnessed by a subcategory $\Z$ with the same objects, a relaxed distributive law $\Y \otimes_{\Z} \X$ is like a distributive law $\Y \otimes \X$, except where the maps in $\Y$ can be identified as either being in $\X$ or in $\Y$.
%\footnote{These are called distributive laws because small categories are monads in the bicategory $\Span(\Sets)$; where distributive laws in the first sense are distributive laws of the corresponding monads in $\Span(\Sets)$.  In the latter case, these are given by distributive laws of bimodules of spans of sets (ie. distributive laws in small profunctors).}
%\end{definition}
%
%This second notion of distributive law will be needed when $\X$ and $\Y$ are props; because they (minimally) share a subcategory $\P$ of permutations.

%We recall the novel way to compose pro(p), first described in \cite{lack}:
%\begin{definition}
%Suppose there three  (symmetric) monoidal theories $(\Sigma_1,E_1)$, $(\Sigma_2,E_2)$ and $(\Sigma_3,E_3)$ where $\bar{(\Sigma_3,E_3)}$ is a sub-pro(p) of both $\bar{(\Sigma_1,E_1)}$ and $\bar{(\Sigma_2,E_2)}$. A {\bf distributive law of pro(p)s} is a distributive law $\lambda:\bar{(\Sigma_2,E_2)} \otimes_{\bar{(\Sigma_3,E_3)}} \bar{ (\Sigma_1,E_1)}$   in $\Mon$-$\Prof$.  Informally, this is a way to push all the maps in $\Sigma_1^*$ past those of  $\Sigma_2^*$ modulo $\Sigma_3$ and the equations $E_1+E_2$ and the axioms of a pro(p).
%\end{definition}
%
%In \cite{lack} it is required that $\bar{(\Sigma_3,E_3)}$ is a groupoid; however, we must loosen this requirement (note that when this is not a groupoid, there is no correspondence to factorization systems as in \cite{rosebrugh}).  

%\begin{lemma}
%Suppose that we have three (symmetric) monoidal theories and a distribtuive law $\lambda:\bar{(\Sigma_2,E_2)} \otimes_{\bar{(\Sigma_3,E_3)}} \bar{ (\Sigma_1,E_1)}$ as above.
%
%Then the induced pro(p) $\bar{(\Sigma_2,E_2)} \otimes_{\bar{(\Sigma_3,E_3)}} \bar{ (\Sigma_1,E_1)}$ is presented by the monoidal theory\\ ${(\Sigma_1^* +_{\Sigma_3} \Sigma_2^*, E_1 + E_2+E_\lambda)}$, where $E_\lambda$ are all the equations needed to push elements of $\Sigma_1^*$ past those of $ \Sigma_2^*$ up to  $\Sigma_3^*$, dictated by $\lambda$.
%\end{lemma}






\section{The phase-free fragment}
\label{sec:one}

In this section we build up to giving a presentation for $(\Span(\Mat(\F_2)),+)$ in a modular way. This category is shown to be the same as the phase-free Hadamard free fragment of the ZX-calculus. Although this presentation of linear spans has already been discussed in great detail  for arbitrary PIDs \cite{ih}, our particular method of exposition is necessary to motivate the affine and full cases. 


\begin{definition}
Consider the prop $\Iso(\cb_2)$ generated by the controlled not gate:
\begin{align*}
&\left\llbracket
\begin{tikzpicture}
	\begin{pgfonlayer}{nodelayer}
		\node [style=oplus] (5) at (0.5, 2.75) {};
		\node [style=dot] (6) at (0, 2.75) {};
		\node [style=none] (7) at (0.5, 3.5) {};
		\node [style=none] (8) at (0.5, 2) {};
		\node [style=none] (9) at (0, 2) {};
		\node [style=none] (10) at (0, 3.5) {};
	\end{pgfonlayer}
	\begin{pgfonlayer}{edgelayer}
		\draw (8.center) to (5);
		\draw (5) to (7.center);
		\draw (10.center) to (6);
		\draw (6) to (5);
		\draw (6) to (9.center);
	\end{pgfonlayer}
\end{tikzpicture}
\right\rrbracket
=
\begin{tikzpicture}
	\begin{pgfonlayer}{nodelayer}
		\node [style=X] (0) at (-0.25, -1) {};
		\node [style=Z] (1) at (-0.75, -1.75) {};
		\node [style=none] (2) at (0, -2.25) {};
		\node [style=none] (3) at (-0.25, -0.5) {};
		\node [style=none] (4) at (-1, -0.5) {};
		\node [style=none] (5) at (-0.75, -2.25) {};
	\end{pgfonlayer}
	\begin{pgfonlayer}{edgelayer}
		\draw (3.center) to (0);
		\draw [in=90, out=-75] (0) to (2.center);
		\draw (5.center) to (1);
		\draw (1) to (0);
		\draw [in=-90, out=105] (1) to (4.center);
	\end{pgfonlayer}
\end{tikzpicture} 
\hspace*{1cm}
\text{modulo the following relations:}\\
\begin{tikzpicture}
	\begin{pgfonlayer}{nodelayer}
		\node [style=dot] (0) at (7, 0) {};
		\node [style=oplus] (1) at (7.5, 0) {};
		\node [style=oplus] (2) at (7.5, 0.5) {};
		\node [style=dot] (3) at (7, 0.5) {};
		\node [style=none] (4) at (7.5, 1) {};
		\node [style=none] (5) at (7, 1) {};
		\node [style=none] (6) at (7, -0.5) {};
		\node [style=none] (7) at (7.5, -0.5) {};
	\end{pgfonlayer}
	\begin{pgfonlayer}{edgelayer}
		\draw (5.center) to (3);
		\draw (3) to (0);
		\draw (0) to (6.center);
		\draw (7.center) to (1);
		\draw (1) to (2);
		\draw (2) to (4.center);
		\draw (2) to (3);
		\draw (0) to (1);
	\end{pgfonlayer}
\end{tikzpicture}
\eqzxa{cnot.one}
\begin{tikzpicture}
	\begin{pgfonlayer}{nodelayer}
		\node [style=none] (4) at (7.5, 1) {};
		\node [style=none] (5) at (7, 1) {};
		\node [style=none] (6) at (7, -0.5) {};
		\node [style=none] (7) at (7.5, -0.5) {};
	\end{pgfonlayer}
	\begin{pgfonlayer}{edgelayer}
		\draw (7.center) to (4.center);
		\draw (5.center) to (6.center);
	\end{pgfonlayer}
\end{tikzpicture}
&\hspace*{.5cm}
\begin{tikzpicture}
	\begin{pgfonlayer}{nodelayer}
		\node [style=none] (4) at (7.5, 1) {};
		\node [style=none] (5) at (7, 1) {};
		\node [style=none] (6) at (7, -1) {};
		\node [style=none] (7) at (7.5, -1) {};
		\node [style=dot] (8) at (7.5, 0.5) {};
		\node [style=dot] (9) at (7.5, -0.5) {};
		\node [style=dot] (10) at (7, 0) {};
		\node [style=oplus] (11) at (7.5, 0) {};
		\node [style=oplus] (12) at (7, 0.5) {};
		\node [style=oplus] (13) at (7, -0.5) {};
	\end{pgfonlayer}
	\begin{pgfonlayer}{edgelayer}
		\draw (7.center) to (4.center);
		\draw (5.center) to (6.center);
		\draw (8) to (12);
		\draw (10) to (11);
		\draw (9) to (13);
	\end{pgfonlayer}
\end{tikzpicture}
\eqzxa{cnot.two}
\begin{tikzpicture}
	\begin{pgfonlayer}{nodelayer}
		\node [style=none] (4) at (7, 1) {};
		\node [style=none] (5) at (7.5, 1) {};
		\node [style=none] (6) at (7, -1) {};
		\node [style=none] (7) at (7.5, -1) {};
	\end{pgfonlayer}
	\begin{pgfonlayer}{edgelayer}
		\draw [in=270, out=90] (7.center) to (4.center);
		\draw [in=90, out=-90] (5.center) to (6.center);
	\end{pgfonlayer}
\end{tikzpicture}
\hspace*{.5cm}
\begin{tikzpicture}
	\begin{pgfonlayer}{nodelayer}
		\node [style=dot] (0) at (8, 0) {};
		\node [style=dot] (1) at (8.5, 0.5) {};
		\node [style=dot] (2) at (8.5, -0.5) {};
		\node [style=oplus] (3) at (8.5, 0) {};
		\node [style=oplus] (4) at (9, 0.5) {};
		\node [style=oplus] (5) at (9, -0.5) {};
		\node [style=none] (6) at (9, -1) {};
		\node [style=none] (7) at (8.5, -1) {};
		\node [style=none] (8) at (8, -1) {};
		\node [style=none] (9) at (8, 1) {};
		\node [style=none] (10) at (8.5, 1) {};
		\node [style=none] (11) at (9, 1) {};
	\end{pgfonlayer}
	\begin{pgfonlayer}{edgelayer}
		\draw (9.center) to (8.center);
		\draw (7.center) to (10.center);
		\draw (11.center) to (6.center);
		\draw (5) to (2);
		\draw (3) to (0);
		\draw (1) to (4);
	\end{pgfonlayer}
\end{tikzpicture}
\eqzxa{cnot.three}
\begin{tikzpicture}
	\begin{pgfonlayer}{nodelayer}
		\node [style=dot] (0) at (8, -0.25) {};
		\node [style=dot] (1) at (8, 0.25) {};
		\node [style=oplus] (3) at (8.5, -0.25) {};
		\node [style=oplus] (4) at (9, 0.25) {};
		\node [style=none] (6) at (9, -1) {};
		\node [style=none] (7) at (8.5, -1) {};
		\node [style=none] (8) at (8, -1) {};
		\node [style=none] (9) at (8, 1) {};
		\node [style=none] (10) at (8.5, 1) {};
		\node [style=none] (11) at (9, 1) {};
	\end{pgfonlayer}
	\begin{pgfonlayer}{edgelayer}
		\draw (9.center) to (8.center);
		\draw (7.center) to (10.center);
		\draw (11.center) to (6.center);
		\draw (3) to (0);
		\draw (1) to (4);
	\end{pgfonlayer}
\end{tikzpicture}
\hspace*{.5cm}
\begin{tikzpicture}
	\begin{pgfonlayer}{nodelayer}
		\node [style=dot] (0) at (2, 1.5) {};
		\node [style=oplus] (1) at (2.5, 1.5) {};
		\node [style=none] (3) at (2, 1) {};
		\node [style=none] (4) at (2.5, 1) {};
		\node [style=none] (5) at (2, 2.5) {};
		\node [style=none] (6) at (2.5, 2.5) {};
		\node [style=none] (7) at (3, 2.5) {};
		\node [style=none] (8) at (3, 1) {};
		\node [style=dot] (9) at (3, 2) {};
		\node [style=oplus] (10) at (2.5, 2) {};
	\end{pgfonlayer}
	\begin{pgfonlayer}{edgelayer}
		\draw (0) to (5.center);
		\draw (6.center) to (1);
		\draw (1) to (0);
		\draw (4.center) to (1);
		\draw (3.center) to (0);
		\draw (10) to (9);
		\draw (8.center) to (7.center);
	\end{pgfonlayer}
\end{tikzpicture}
\eqzxa{cnot.four}
\begin{tikzpicture}
	\begin{pgfonlayer}{nodelayer}
		\node [style=dot] (0) at (2, 2) {};
		\node [style=oplus] (1) at (2.5, 2) {};
		\node [style=none] (3) at (2, 2.5) {};
		\node [style=none] (4) at (2.5, 2.5) {};
		\node [style=none] (5) at (2, 1) {};
		\node [style=none] (6) at (2.5, 1) {};
		\node [style=none] (7) at (3, 1) {};
		\node [style=none] (8) at (3, 2.5) {};
		\node [style=dot] (9) at (3, 1.5) {};
		\node [style=oplus] (10) at (2.5, 1.5) {};
	\end{pgfonlayer}
	\begin{pgfonlayer}{edgelayer}
		\draw (0) to (5.center);
		\draw (6.center) to (1);
		\draw (1) to (0);
		\draw (4.center) to (1);
		\draw (3.center) to (0);
		\draw (10) to (9);
		\draw (8.center) to (7.center);
	\end{pgfonlayer}
\end{tikzpicture}
\hspace*{.5cm}
\begin{tikzpicture}
	\begin{pgfonlayer}{nodelayer}
		\node [style=oplus] (0) at (2, 1.5) {};
		\node [style=dot] (1) at (2.5, 1.5) {};
		\node [style=none] (3) at (2, 1) {};
		\node [style=none] (4) at (2.5, 1) {};
		\node [style=none] (5) at (2, 2.5) {};
		\node [style=none] (6) at (2.5, 2.5) {};
		\node [style=none] (7) at (3, 2.5) {};
		\node [style=none] (8) at (3, 1) {};
		\node [style=oplus] (9) at (3, 2) {};
		\node [style=dot] (10) at (2.5, 2) {};
	\end{pgfonlayer}
	\begin{pgfonlayer}{edgelayer}
		\draw (0) to (5.center);
		\draw (6.center) to (1);
		\draw (1) to (0);
		\draw (4.center) to (1);
		\draw (3.center) to (0);
		\draw (10) to (9);
		\draw (8.center) to (7.center);
	\end{pgfonlayer}
\end{tikzpicture}
\eqzxa{cnot.five}
\begin{tikzpicture}
	\begin{pgfonlayer}{nodelayer}
		\node [style=oplus] (0) at (2, 2) {};
		\node [style=dot] (1) at (2.5, 2) {};
		\node [style=none] (3) at (2, 2.5) {};
		\node [style=none] (4) at (2.5, 2.5) {};
		\node [style=none] (5) at (2, 1) {};
		\node [style=none] (6) at (2.5, 1) {};
		\node [style=none] (7) at (3, 1) {};
		\node [style=none] (8) at (3, 2.5) {};
		\node [style=oplus] (9) at (3, 1.5) {};
		\node [style=dot] (10) at (2.5, 1.5) {};
	\end{pgfonlayer}
	\begin{pgfonlayer}{edgelayer}
		\draw (0) to (5.center);
		\draw (6.center) to (1);
		\draw (1) to (0);
		\draw (4.center) to (1);
		\draw (3.center) to (0);
		\draw (10) to (9);
		\draw (8.center) to (7.center);
	\end{pgfonlayer}
\end{tikzpicture}
\end{align*}
\end{definition}




\begin{lemma} \cite[Thm. 6]{lafont}
$\Iso(\cb_2)$ is a presentation for the prop $(\Iso(\Mat(\F_2),+))$
\end{lemma}



\begin{definition}
Consider the prop $\inj(\cb_2)$ generated by the coproduct of props $\Iso(\cb_2)+\inj$ modulo the equation:
\hspace*{1cm}
$
\begin{tikzpicture}
	\begin{pgfonlayer}{nodelayer}
		\node [style=X] (0) at (0, 0) {};
		\node [style=dot] (1) at (0, 0.5) {};
		\node [style=oplus] (2) at (0.5, 0.5) {};
		\node [style=none] (3) at (0.5, -0.25) {};
		\node [style=none] (4) at (0.5, 1) {};
		\node [style=none] (5) at (0, 1) {};
	\end{pgfonlayer}
	\begin{pgfonlayer}{edgelayer}
		\draw (0) to (1);
		\draw (1) to (5.center);
		\draw (1) to (2);
		\draw (2) to (4.center);
		\draw (3.center) to (2);
	\end{pgfonlayer}
\end{tikzpicture}
\eqzxa{cnot.six}
\begin{tikzpicture}
	\begin{pgfonlayer}{nodelayer}
		\node [style=X] (0) at (0, 0) {};
		\node [style=none] (3) at (0.5, -0.25) {};
		\node [style=none] (4) at (0.5, 0.5) {};
		\node [style=none] (5) at (0, 0.5) {};
	\end{pgfonlayer}
	\begin{pgfonlayer}{edgelayer}
		\draw (0) to (5.center);
		\draw (3.center) to (4.center);
	\end{pgfonlayer}
\end{tikzpicture}
$

\end{definition}


\begin{lemma} \cite[Thm. 7]{lafont}
$\inj(\cb_2)$ is a presentation for the prop $(\inj(\Mat(\F_2)),+)$
\end{lemma}


The white comultiplication can be derived in this fragment:
\hspace*{1cm}$
\left\llbracket
\begin{tikzpicture}
	\begin{pgfonlayer}{nodelayer}
		\node [style=oplus] (0) at (1, 2.75) {};
		\node [style=dot] (1) at (0.5, 2.75) {};
		\node [style=none] (2) at (1, 3.5) {};
		\node [style=none] (3) at (1, 2.25) {};
		\node [style=none] (4) at (0.5, 2) {};
		\node [style=none] (5) at (0.5, 3.5) {};
		\node [style=X] (6) at (1, 2.25) {};
	\end{pgfonlayer}
	\begin{pgfonlayer}{edgelayer}
		\draw (3.center) to (0);
		\draw (0) to (2.center);
		\draw (5.center) to (1);
		\draw (1) to (0);
		\draw (1) to (4.center);
	\end{pgfonlayer}
\end{tikzpicture}
\right\rrbracket
=
\begin{tikzpicture}
	\begin{pgfonlayer}{nodelayer}
		\node [style=X] (0) at (1.25, -1) {};
		\node [style=Z] (1) at (0.75, -1.75) {};
		\node [style=none] (2) at (1.5, -2) {};
		\node [style=none] (3) at (1.25, -0.5) {};
		\node [style=none] (4) at (0.5, -0.5) {};
		\node [style=none] (5) at (0.75, -2.25) {};
		\node [style=X] (6) at (1.5, -2) {};
	\end{pgfonlayer}
	\begin{pgfonlayer}{edgelayer}
		\draw (3.center) to (0);
		\draw [in=90, out=-75] (0) to (2.center);
		\draw (5.center) to (1);
		\draw (1) to (0);
		\draw [in=-90, out=105] (1) to (4.center);
	\end{pgfonlayer}
\end{tikzpicture}
=
\begin{tikzpicture}
	\begin{pgfonlayer}{nodelayer}
		\node [style=Z] (1) at (0.75, -1.75) {};
		\node [style=none] (3) at (1, -1) {};
		\node [style=none] (4) at (0.5, -1) {};
		\node [style=none] (5) at (0.75, -2.25) {};
	\end{pgfonlayer}
	\begin{pgfonlayer}{edgelayer}
		\draw (5.center) to (1);
		\draw [in=-90, out=120] (1) to (4.center);
		\draw [in=-90, out=60] (1) to (3.center);
	\end{pgfonlayer}
\end{tikzpicture}
$

%This is to be expected because there is a faithful ``'graph functor' monoidal functor $(\inj(\FinSet),+) \to (\Inj(\Span(\FinSet)),+)$.




As a matter of notation, given a category $\X$ with finite limits, we refer to the subcategory of $\Span(\X)$ where the left leg is monic as $\Par(\X)$, and the subcategory of spans where all legs are monic by $\Par\Iso(\X)$.  These two categories, respectively, give semantics for partial maps and partially invertible maps in $\X$ (see \cite{cockett} for more details).


\begin{definition}
\label{def:pariso:cb}
Consider the prop $\ParIso(\cb_2)$ generated by the distributive law of props:
$$
\inj(\cb_2)^\op \otimes_{\Iso(\cb_2)} \inj(\cb_2);
\begin{tikzpicture}
	\begin{pgfonlayer}{nodelayer}
		\node [style=X] (0) at (0, 0) {};
		\node [style=X] (1) at (0, 0.75) {};
	\end{pgfonlayer}
	\begin{pgfonlayer}{edgelayer}
		\draw (0) to (1);
	\end{pgfonlayer}
\end{tikzpicture}
\eref{extra}
$$
\end{definition}


\begin{remark}
\label{rem:pariso:cb}
This is actually a distributive law because the only only seemingly nontrivial situation arises when controlled not gates are sandwiched by black units/counits on their target wires.  However the following identity holds by induction on the number of controlled not gates.   For the base case of $n=0$, this follows from the bone law which we added to the distributive law.  For $n>1$, we have the following situation:
$$
\begin{tikzpicture}
	\begin{pgfonlayer}{nodelayer}
		\node [style=X] (0) at (1.5, -0.25) {};
		\node [style=oplus] (1) at (1.5, -0.75) {};
		\node [style=oplus] (2) at (1.5, -1.5) {};
		\node [style=dot] (3) at (1, -0.75) {};
		\node [style=dot] (4) at (0.5, -1.5) {};
		\node [style=none] (5) at (1, 0) {};
		\node [style=none] (6) at (0.5, 0) {};
		\node [style=none] (7) at (0.75, -1) {$\iddots$};
		\node [style=none] (8) at (1.5, -1) {$\vdots$};
		\node [style=X] (9) at (1.5, -2.5) {};
		\node [style=none] (10) at (0.5, -2.75) {};
		\node [style=none] (11) at (1, -2.75) {};
		\node [style=none] (12) at (0.75, -0.25) {$\cdots$};
		\node [style=oplus] (13) at (1.5, -2) {};
		\node [style=dot] (14) at (0, -2) {};
		\node [style=none] (15) at (0, -2.75) {};
		\node [style=none] (16) at (0, 0) {};
		\node [style=none] (18) at (1.25, -0.75) {};
	\end{pgfonlayer}
	\begin{pgfonlayer}{edgelayer}
		\draw (0) to (1);
		\draw (1) to (3);
		\draw (5.center) to (3);
		\draw (6.center) to (4);
		\draw (4) to (2);
		\draw (4) to (10.center);
		\draw (11.center) to (3);
		\draw (9) to (2);
		\draw (14) to (13);
		\draw (15.center) to (14);
		\draw (14) to (16.center);
	\end{pgfonlayer}
\end{tikzpicture}
=
\begin{tikzpicture}
	\begin{pgfonlayer}{nodelayer}
		\node [style=X] (19) at (3.5, -1.25) {};
		\node [style=none] (24) at (3.5, 0.5) {};
		\node [style=none] (25) at (3, 0.5) {};
		\node [style=X] (28) at (3.5, -1.75) {};
		\node [style=none] (29) at (3, -3.5) {};
		\node [style=none] (30) at (3.5, -3.5) {};
		\node [style=none] (34) at (2.5, -3.5) {};
		\node [style=none] (35) at (2.5, 0.5) {};
		\node [style=oplus] (36) at (3.5, -0.75) {};
		\node [style=oplus] (37) at (3.5, 0) {};
		\node [style=oplus] (38) at (3.5, -3) {};
		\node [style=oplus] (39) at (3.5, -2.25) {};
		\node [style=dot] (40) at (2.5, -3) {};
		\node [style=dot] (41) at (3, -2.25) {};
		\node [style=dot] (42) at (3, -0.75) {};
		\node [style=dot] (43) at (2.5, 0) {};
		\node [style=none] (44) at (3.5, -0.25) {$\vdots$};
		\node [style=none] (45) at (3.5, -2.5) {$\vdots$};
		\node [style=none] (46) at (2.75, -1.5) {$\cdots$};
		\node [style=none] (47) at (2.75, -2.5) {$\iddots$};
		\node [style=none] (48) at (2.75, -0.25) {$\ddots$};
	\end{pgfonlayer}
	\begin{pgfonlayer}{edgelayer}
		\draw (24.center) to (37);
		\draw (36) to (19);
		\draw (28) to (39);
		\draw (38) to (30.center);
		\draw (29.center) to (41);
		\draw (41) to (42);
		\draw (42) to (25.center);
		\draw (35.center) to (43);
		\draw (43) to (40);
		\draw (40) to (38);
		\draw (37) to (43);
		\draw (42) to (36);
		\draw (39) to (41);
	\end{pgfonlayer}
\end{tikzpicture}
$$
For $n=1$, that is:
$$
\begin{tikzpicture}
	\begin{pgfonlayer}{nodelayer}
		\node [style=X] (0) at (4, 2) {};
		\node [style=X] (1) at (4, 1) {};
		\node [style=oplus] (2) at (4, 1.5) {};
		\node [style=dot] (3) at (3.5, 1.5) {};
		\node [style=none] (4) at (3.5, 2.25) {};
		\node [style=none] (5) at (3.5, 0.75) {};
	\end{pgfonlayer}
	\begin{pgfonlayer}{edgelayer}
		\draw (1) to (0);
		\draw (4.center) to (5.center);
		\draw (3) to (2);
	\end{pgfonlayer}
\end{tikzpicture}
=
\begin{tikzpicture}
	\begin{pgfonlayer}{nodelayer}
		\node [style=X] (0) at (4, 2.5) {};
		\node [style=X] (1) at (4, 0.5) {};
		\node [style=oplus] (2) at (4, 1.5) {};
		\node [style=dot] (3) at (3.5, 1.5) {};
		\node [style=none] (4) at (3.5, 2.75) {};
		\node [style=none] (5) at (3.5, 0.25) {};
		\node [style=oplus] (6) at (3.5, 2) {};
		\node [style=dot] (7) at (4, 2) {};
		\node [style=oplus] (8) at (3.5, 1) {};
		\node [style=dot] (9) at (4, 1) {};
	\end{pgfonlayer}
	\begin{pgfonlayer}{edgelayer}
		\draw (1) to (0);
		\draw (4.center) to (5.center);
		\draw (3) to (2);
		\draw (7) to (6);
		\draw (9) to (8);
	\end{pgfonlayer}
\end{tikzpicture}
=
\begin{tikzpicture}
	\begin{pgfonlayer}{nodelayer}
		\node [style=X] (0) at (4, 2.5) {};
		\node [style=X] (1) at (4, 1.5) {};
		\node [style=none] (4) at (3.5, 2.75) {};
		\node [style=none] (5) at (3.5, 1.25) {};
	\end{pgfonlayer}
	\begin{pgfonlayer}{edgelayer}
		\draw [in=-90, out=90] (1) to (4.center);
		\draw [in=90, out=-90] (0) to (5.center);
	\end{pgfonlayer}
\end{tikzpicture}
=
\begin{tikzpicture}
	\begin{pgfonlayer}{nodelayer}
		\node [style=X] (0) at (3.5, 1.75) {};
		\node [style=X] (1) at (3.5, 2.25) {};
		\node [style=none] (4) at (3.5, 2.75) {};
		\node [style=none] (5) at (3.5, 1.25) {};
	\end{pgfonlayer}
	\begin{pgfonlayer}{edgelayer}
		\draw [in=-90, out=90] (1) to (4.center);
		\draw [in=90, out=-90] (0) to (5.center);
	\end{pgfonlayer}
\end{tikzpicture}
$$
And for the base case for $n=2$:
$$
\begin{tikzpicture}
	\begin{pgfonlayer}{nodelayer}
		\node [style=X] (0) at (4.5, 2) {};
		\node [style=X] (1) at (4.5, 0.5) {};
		\node [style=oplus] (2) at (4.5, 1) {};
		\node [style=dot] (3) at (3.5, 1) {};
		\node [style=none] (4) at (3.5, 2.25) {};
		\node [style=none] (5) at (3.5, 0.25) {};
		\node [style=oplus] (6) at (4.5, 1.5) {};
		\node [style=dot] (7) at (4, 1.5) {};
		\node [style=none] (8) at (4, 2.25) {};
		\node [style=none] (9) at (4, 0.25) {};
	\end{pgfonlayer}
	\begin{pgfonlayer}{edgelayer}
		\draw (1) to (0);
		\draw (4.center) to (5.center);
		\draw (3) to (2);
		\draw (8.center) to (9.center);
		\draw (7) to (6);
	\end{pgfonlayer}
\end{tikzpicture}
=
\begin{tikzpicture}
	\begin{pgfonlayer}{nodelayer}
		\node [style=X] (0) at (4.5, 2.5) {};
		\node [style=X] (1) at (4.5, 0.5) {};
		\node [style=oplus] (2) at (4.5, 1) {};
		\node [style=dot] (3) at (3.5, 1) {};
		\node [style=none] (4) at (3.5, 2.75) {};
		\node [style=none] (5) at (3.5, 0.25) {};
		\node [style=oplus] (6) at (4.5, 1.5) {};
		\node [style=dot] (7) at (4, 1.5) {};
		\node [style=none] (8) at (4, 2.75) {};
		\node [style=none] (9) at (4, 0.25) {};
		\node [style=oplus] (10) at (4, 2) {};
		\node [style=dot] (11) at (4.5, 2) {};
	\end{pgfonlayer}
	\begin{pgfonlayer}{edgelayer}
		\draw (1) to (0);
		\draw (4.center) to (5.center);
		\draw (3) to (2);
		\draw (8.center) to (9.center);
		\draw (7) to (6);
		\draw (11) to (10);
	\end{pgfonlayer}
\end{tikzpicture}
=
\begin{tikzpicture}
	\begin{pgfonlayer}{nodelayer}
		\node [style=X] (0) at (4.5, 3.5) {};
		\node [style=X] (1) at (4.5, 0.5) {};
		\node [style=oplus] (2) at (4.5, 1) {};
		\node [style=dot] (3) at (3.5, 1) {};
		\node [style=none] (4) at (3.5, 3.75) {};
		\node [style=none] (5) at (3.5, 0.25) {};
		\node [style=oplus] (6) at (4.5, 2.5) {};
		\node [style=dot] (7) at (4, 2.5) {};
		\node [style=none] (8) at (4, 3.75) {};
		\node [style=none] (9) at (4, 0.25) {};
		\node [style=oplus] (10) at (4, 3) {};
		\node [style=dot] (11) at (4.5, 3) {};
		\node [style=oplus] (12) at (4, 2) {};
		\node [style=dot] (13) at (4.5, 2) {};
		\node [style=oplus] (14) at (4, 1.5) {};
		\node [style=dot] (15) at (4.5, 1.5) {};
	\end{pgfonlayer}
	\begin{pgfonlayer}{edgelayer}
		\draw (1) to (0);
		\draw (4.center) to (5.center);
		\draw (3) to (2);
		\draw (8.center) to (9.center);
		\draw (7) to (6);
		\draw (11) to (10);
		\draw (13) to (12);
		\draw (15) to (14);
	\end{pgfonlayer}
\end{tikzpicture}
=
\begin{tikzpicture}
	\begin{pgfonlayer}{nodelayer}
		\node [style=X] (0) at (4.5, 2.5) {};
		\node [style=X] (1) at (4.5, 0.5) {};
		\node [style=oplus] (2) at (4.5, 1) {};
		\node [style=dot] (3) at (3.5, 1) {};
		\node [style=none] (4) at (3.5, 2.75) {};
		\node [style=none] (5) at (3.5, 0.25) {};
		\node [style=none] (8) at (4, 2.75) {};
		\node [style=none] (9) at (4, 0.25) {};
		\node [style=oplus] (14) at (4, 1.5) {};
		\node [style=dot] (15) at (4.5, 1.5) {};
		\node [style=none] (16) at (4, 2.5) {};
	\end{pgfonlayer}
	\begin{pgfonlayer}{edgelayer}
		\draw (4.center) to (5.center);
		\draw (3) to (2);
		\draw (15) to (14);
		\draw (9.center) to (14);
		\draw (1) to (15);
		\draw [in=-90, out=90] (15) to (16.center);
		\draw [in=-90, out=90] (14) to (0);
		\draw (8.center) to (16.center);
	\end{pgfonlayer}
\end{tikzpicture}
=
\begin{tikzpicture}
	\begin{pgfonlayer}{nodelayer}
		\node [style=X] (0) at (4, 2) {};
		\node [style=X] (1) at (4.5, 0.5) {};
		\node [style=oplus] (2) at (4.5, 1) {};
		\node [style=dot] (3) at (3.5, 1) {};
		\node [style=none] (4) at (3.5, 2.75) {};
		\node [style=none] (5) at (3.5, 0.25) {};
		\node [style=none] (8) at (4, 2.75) {};
		\node [style=none] (9) at (4, 0.25) {};
		\node [style=oplus] (14) at (4, 1.5) {};
		\node [style=dot] (15) at (4.5, 1.5) {};
		\node [style=none] (16) at (4.5, 2) {};
	\end{pgfonlayer}
	\begin{pgfonlayer}{edgelayer}
		\draw (4.center) to (5.center);
		\draw (3) to (2);
		\draw (15) to (14);
		\draw (9.center) to (14);
		\draw (1) to (15);
		\draw [in=-90, out=90] (14) to (0);
		\draw (15) to (16.center);
		\draw [in=-90, out=90, looseness=0.75] (16.center) to (8.center);
	\end{pgfonlayer}
\end{tikzpicture}
=
\begin{tikzpicture}
	\begin{pgfonlayer}{nodelayer}
		\node [style=X] (17) at (6, 2.5) {};
		\node [style=X] (18) at (6.5, 1) {};
		\node [style=oplus] (19) at (6.5, 2) {};
		\node [style=dot] (20) at (5.5, 2) {};
		\node [style=none] (21) at (5.5, 3.25) {};
		\node [style=none] (22) at (5.5, 0.25) {};
		\node [style=none] (23) at (6, 3.25) {};
		\node [style=none] (24) at (6, 0.25) {};
		\node [style=oplus] (25) at (6, 1.5) {};
		\node [style=dot] (26) at (6.5, 1.5) {};
		\node [style=none] (27) at (6.5, 2.5) {};
		\node [style=oplus] (28) at (6, 1) {};
		\node [style=dot] (29) at (5.5, 1) {};
	\end{pgfonlayer}
	\begin{pgfonlayer}{edgelayer}
		\draw (21.center) to (22.center);
		\draw (20) to (19);
		\draw (26) to (25);
		\draw (24.center) to (25);
		\draw (18) to (26);
		\draw [in=-90, out=90] (25) to (17);
		\draw (26) to (27.center);
		\draw [in=-90, out=90, looseness=0.75] (27.center) to (23.center);
		\draw (29) to (28);
	\end{pgfonlayer}
\end{tikzpicture}
=
\begin{tikzpicture}
	\begin{pgfonlayer}{nodelayer}
		\node [style=X] (17) at (6, 2) {};
		\node [style=X] (18) at (6.5, 1) {};
		\node [style=oplus] (19) at (6.5, 1.5) {};
		\node [style=dot] (20) at (5.5, 1.5) {};
		\node [style=none] (21) at (5.5, 2.75) {};
		\node [style=none] (22) at (5.5, 0.25) {};
		\node [style=none] (23) at (6, 2.75) {};
		\node [style=none] (24) at (6, 0.25) {};
		\node [style=none] (27) at (6.5, 2) {};
		\node [style=oplus] (28) at (6, 1) {};
		\node [style=dot] (29) at (5.5, 1) {};
	\end{pgfonlayer}
	\begin{pgfonlayer}{edgelayer}
		\draw (21.center) to (22.center);
		\draw (20) to (19);
		\draw [in=-90, out=90, looseness=0.75] (27.center) to (23.center);
		\draw (29) to (28);
		\draw (24.center) to (28);
		\draw (28) to (17);
		\draw (27.center) to (18);
	\end{pgfonlayer}
\end{tikzpicture}
=
\begin{tikzpicture}
	\begin{pgfonlayer}{nodelayer}
		\node [style=X] (17) at (6, 1.5) {};
		\node [style=X] (18) at (6, 2) {};
		\node [style=oplus] (19) at (6, 2.5) {};
		\node [style=dot] (20) at (5.5, 2.5) {};
		\node [style=none] (21) at (5.5, 3) {};
		\node [style=none] (22) at (5.5, 0.5) {};
		\node [style=none] (24) at (6, 0.5) {};
		\node [style=none] (27) at (6, 3) {};
		\node [style=oplus] (28) at (6, 1) {};
		\node [style=dot] (29) at (5.5, 1) {};
	\end{pgfonlayer}
	\begin{pgfonlayer}{edgelayer}
		\draw (21.center) to (22.center);
		\draw (20) to (19);
		\draw (29) to (28);
		\draw (24.center) to (28);
		\draw (28) to (17);
		\draw (27.center) to (18);
	\end{pgfonlayer}
\end{tikzpicture}
$$


The inductive case is essentially the same as the base case for 2.
%
%Note that the distributive law is actually a partially reversible formulation of the bialgebra law in disguise since:
%
%
%$$
%\left\llbracket
%\begin{tikzpicture}
%	\begin{pgfonlayer}{nodelayer}
%		\node [style=X] (0) at (1, -0.25) {};
%		\node [style=oplus] (1) at (1, -0.75) {};
%		\node [style=oplus] (2) at (1, -1.5) {};
%		\node [style=dot] (3) at (0.5, -0.75) {};
%		\node [style=dot] (4) at (0, -1.5) {};
%		\node [style=none] (5) at (0.5, 0) {};
%		\node [style=none] (6) at (0, 0) {};
%		\node [style=none] (7) at (0.25, -1) {$\iddots$};
%		\node [style=none] (8) at (1, -1) {$\vdots$};
%		\node [style=X] (9) at (1, -2) {};
%		\node [style=none] (18) at (0, -2.25) {};
%		\node [style=none] (19) at (0.5, -2.25) {};
%		\node [style=none] (22) at (0.25, -0.25) {$\cdots$};
%	\end{pgfonlayer}
%	\begin{pgfonlayer}{edgelayer}
%		\draw (0) to (1);
%		\draw (1) to (3);
%		\draw (5.center) to (3);
%		\draw (6.center) to (4);
%		\draw (4) to (2);
%		\draw (4) to (18.center);
%		\draw (19.center) to (3);
%		\draw (9) to (2);
%	\end{pgfonlayer}
%\end{tikzpicture}
%\right\rrbracket
%=
%\begin{tikzpicture}
%	\begin{pgfonlayer}{nodelayer}
%		\node [style=none] (0) at (9.25, 0.5) {};
%		\node [style=none] (1) at (8.75, 0.5) {};
%		\node [style=none] (2) at (8.75, -1.75) {};
%		\node [style=none] (3) at (9.25, -1.75) {};
%		\node [style=Z] (4) at (9.25, 0) {};
%		\node [style=Z] (5) at (8.75, -0.75) {};
%		\node [style=none] (6) at (9.5, -1.25) {};
%		\node [style=none] (7) at (10.5, -1.25) {};
%		\node [style=none] (8) at (10.5, 0.25) {};
%		\node [style=none] (9) at (9.5, 0.25) {};
%		\node [style=none] (10) at (10.25, -1) {$!$};
%		\node [style=X] (11) at (10, 0) {};
%		\node [style=X] (12) at (10, -0.75) {};
%	\end{pgfonlayer}
%	\begin{pgfonlayer}{edgelayer}
%		\draw (1.center) to (5);
%		\draw (5) to (2.center);
%		\draw (3.center) to (4);
%		\draw (4) to (0.center);
%		\draw [style=dotted] (6.center) to (7.center);
%		\draw [style=dotted] (7.center) to (8.center);
%		\draw [style=dotted] (8.center) to (9.center);
%		\draw [style=dotted] (9.center) to (6.center);
%		\draw (5) to (12);
%		\draw (12) to (11);
%		\draw (11) to (4);
%	\end{pgfonlayer}
%\end{tikzpicture}
%=
%\begin{tikzpicture}
%	\begin{pgfonlayer}{nodelayer}
%		\node [style=none] (12) at (9.25, 1) {};
%		\node [style=none] (13) at (8.75, 1) {};
%		\node [style=none] (15) at (8.75, -1.75) {};
%		\node [style=none] (16) at (9.25, -1.75) {};
%		\node [style=Z] (18) at (8.75, -1.25) {};
%		\node [style=none] (19) at (9.5, -1.25) {};
%		\node [style=none] (20) at (10.5, -1.25) {};
%		\node [style=none] (21) at (10.5, 0.75) {};
%		\node [style=none] (22) at (9.5, 0.75) {};
%		\node [style=none] (23) at (10.25, -1) {$!$};
%		\node [style=X] (25) at (10, -0.75) {};
%		\node [style=Z] (28) at (10, 0.25) {};
%		\node [style=X] (29) at (9.25, 0.5) {};
%		\node [style=X] (30) at (9.25, -0.25) {};
%	\end{pgfonlayer}
%	\begin{pgfonlayer}{edgelayer}
%		\draw (13.center) to (18);
%		\draw (18) to (15.center);
%		\draw [style=dotted] (19.center) to (20.center);
%		\draw [style=dotted] (20.center) to (21.center);
%		\draw [style=dotted] (21.center) to (22.center);
%		\draw [style=dotted] (22.center) to (19.center);
%		\draw (18) to (25);
%		\draw (12.center) to (29);
%		\draw (29) to (28);
%		\draw (28) to (30);
%		\draw (25) to (28);
%		\draw (16.center) to (30);
%	\end{pgfonlayer}
%\end{tikzpicture}
%=
%\begin{tikzpicture}
%	\begin{pgfonlayer}{nodelayer}
%		\node [style=none] (0) at (9.25, 1.25) {};
%		\node [style=none] (1) at (8.75, 1.25) {};
%		\node [style=none] (2) at (8.75, -1.75) {};
%		\node [style=none] (3) at (9.25, -1.75) {};
%		\node [style=none] (5) at (9.5, -1.25) {};
%		\node [style=none] (6) at (10.5, -1.25) {};
%		\node [style=none] (7) at (10.5, 1) {};
%		\node [style=none] (8) at (9.5, 1) {};
%		\node [style=none] (9) at (10.25, -1) {$!$};
%		\node [style=Z] (11) at (10, 0.5) {};
%		\node [style=X] (12) at (9.25, 0.75) {};
%		\node [style=X] (13) at (9.25, 0) {};
%		\node [style=Z] (14) at (10, -0.75) {};
%		\node [style=X] (15) at (8.75, -0.25) {};
%		\node [style=X] (16) at (8.75, -1.25) {};
%	\end{pgfonlayer}
%	\begin{pgfonlayer}{edgelayer}
%		\draw [style=dotted] (5.center) to (6.center);
%		\draw [style=dotted] (6.center) to (7.center);
%		\draw [style=dotted] (7.center) to (8.center);
%		\draw [style=dotted] (8.center) to (5.center);
%		\draw (0.center) to (12);
%		\draw (12) to (11);
%		\draw (11) to (13);
%		\draw (3.center) to (13);
%		\draw (16) to (14);
%		\draw (14) to (11);
%		\draw (14) to (15);
%		\draw (15) to (1.center);
%		\draw (16) to (2.center);
%	\end{pgfonlayer}
%\end{tikzpicture}
%=
%\begin{tikzpicture}
%	\begin{pgfonlayer}{nodelayer}
%		\node [style=none] (0) at (9.25, 0.75) {};
%		\node [style=none] (1) at (8.75, 0.75) {};
%		\node [style=none] (2) at (8.75, -2) {};
%		\node [style=none] (3) at (9.25, -2) {};
%		\node [style=none] (4) at (9.5, -1.75) {};
%		\node [style=none] (5) at (10.5, -1.75) {};
%		\node [style=none] (6) at (10.5, 0.5) {};
%		\node [style=none] (7) at (9.5, 0.5) {};
%		\node [style=none] (8) at (10.25, -1.5) {$!$};
%		\node [style=Z] (9) at (10, 0.25) {};
%		\node [style=X] (10) at (9.25, 0.25) {};
%		\node [style=X] (11) at (9.25, -0.25) {};
%		\node [style=Z] (12) at (10, -0.75) {};
%		\node [style=X] (13) at (8.75, -0.75) {};
%		\node [style=X] (14) at (8.75, -1.25) {};
%		\node [style=Z] (15) at (10, -0.25) {};
%		\node [style=Z] (16) at (10, -1.25) {};
%	\end{pgfonlayer}
%	\begin{pgfonlayer}{edgelayer}
%		\draw [style=dotted] (4.center) to (5.center);
%		\draw [style=dotted] (5.center) to (6.center);
%		\draw [style=dotted] (6.center) to (7.center);
%		\draw [style=dotted] (7.center) to (4.center);
%		\draw (0.center) to (10);
%		\draw (10) to (9);
%		\draw (3.center) to (11);
%		\draw (13) to (12);
%		\draw (13) to (1.center);
%		\draw (14) to (2.center);
%		\draw (9) to (12);
%		\draw (16) to (14);
%		\draw (11) to (15);
%		\draw (16) to (12);
%	\end{pgfonlayer}
%\end{tikzpicture}
%=
%\left\llbracket
%\begin{tikzpicture}
%	\begin{pgfonlayer}{nodelayer}
%		\node [style=none] (0) at (7, 2.25) {};
%		\node [style=none] (1) at (6.5, 2.25) {};
%		\node [style=none] (2) at (6.75, -.2) {$\iddots$};
%		\node [style=none] (3) at (6.5, -3) {};
%		\node [style=none] (4) at (7, -3) {};
%		\node [style=oplus] (6) at (6.5, -2.25) {};
%		\node [style=X] (7) at (6.5, -1.75) {};
%		\node [style=X] (8) at (6.5, -1.25) {};
%		\node [style=dot] (9) at (7.5, -2.25) {};
%		\node [style=dot] (10) at (7.5, -0.75) {};
%		\node [style=Z] (12) at (7.5, -2.75) {};
%		\node [style=oplus] (13) at (7, 0) {};
%		\node [style=X] (14) at (7, 0.5) {};
%		\node [style=X] (15) at (7, 1) {};
%		\node [style=dot] (16) at (7.5, 0) {};
%		\node [style=dot] (17) at (7.5, 1.5) {};
%		\node [style=none] (18) at (7.5, -0.25) {$\vdots$};
%		\node [style=Z] (19) at (7.5, 2) {};
%		\node [style=oplus] (20) at (7, 1.5) {};
%		\node [style=oplus] (21) at (6.5, -0.75) {};
%	\end{pgfonlayer}
%	\begin{pgfonlayer}{edgelayer}
%		\draw (3.center) to (6);
%		\draw (6) to (7);
%		\draw (6) to (9);
%		\draw (12) to (9);
%		\draw (9) to (10);
%		\draw (13) to (14);
%		\draw (13) to (16);
%		\draw (16) to (17);
%		\draw (4.center) to (13);
%		\draw (19) to (17);
%		\draw (17) to (20);
%		\draw (20) to (15);
%		\draw (0.center) to (20);
%		\draw (10) to (21);
%		\draw (21) to (8);
%		\draw (21) to (1.center);
%	\end{pgfonlayer}
%\end{tikzpicture}
%\right\rrbracket
%=
%\left\llbracket
%\begin{tikzpicture}
%	\begin{pgfonlayer}{nodelayer}
%		\node [style=none] (18) at (7.5, 1.25) {};
%		\node [style=none] (19) at (6.5, 0.5) {};
%		\node [style=none] (20) at (6.7, -0.25) {$\iddots$};
%		\node [style=none] (23) at (7, -2) {};
%		\node [style=none] (24) at (7, -1.25) {};
%		\node [style=X] (30) at (6.5, -1.25) {};
%		\node [style=dot] (32) at (7.5, -0.75) {};
%		\node [style=oplus] (35) at (7, 0) {};
%		\node [style=X] (36) at (7, 0.5) {};
%		\node [style=dot] (38) at (7.5, 0) {};
%		\node [style=none] (41) at (7.5, -0.25) {$\vdots$};
%		\node [style=none] (42) at (7.5, -1.25) {};
%		\node [style=oplus] (43) at (6.5, -0.75) {};
%		\node [style=none] (44) at (7.5, -2) {};
%		\node [style=none] (45) at (7, 1.25) {};
%	\end{pgfonlayer}
%	\begin{pgfonlayer}{edgelayer}
%		\draw (35) to (36);
%		\draw (35) to (38);
%		\draw [in=270, out=90] (24.center) to (35);
%		\draw (38) to (18.center);
%		\draw [in=-90, out=90, looseness=0.75] (23.center) to (42.center);
%		\draw (42.center) to (32);
%		\draw (32) to (43);
%		\draw (43) to (19.center);
%		\draw (43) to (30);
%		\draw [in=-90, out=90] (44.center) to (24.center);
%		\draw [in=-90, out=90] (19.center) to (45.center);
%	\end{pgfonlayer}
%\end{tikzpicture}
%\right\rrbracket
%=
%\left\llbracket
%\begin{tikzpicture}
%	\begin{pgfonlayer}{nodelayer}
%		\node [style=none] (15) at (9.5, 1.5) {};
%		\node [style=none] (19) at (10, -1) {};
%		\node [style=X] (20) at (9, 0.5) {};
%		\node [style=dot] (21) at (9.5, 1) {};
%		\node [style=oplus] (22) at (10, -0.5) {};
%		\node [style=X] (23) at (10, 0) {};
%		\node [style=dot] (24) at (9.5, -0.5) {};
%		\node [style=none] (26) at (9.5, -1) {};
%		\node [style=oplus] (27) at (9, 1) {};
%		\node [style=none] (29) at (9, 1.5) {};
%		\node [style=none] (30) at (9.5, 0.25) {$\ddots$};
%	\end{pgfonlayer}
%	\begin{pgfonlayer}{edgelayer}
%		\draw (22) to (23);
%		\draw (22) to (24);
%		\draw [in=270, out=90] (19.center) to (22);
%		\draw (21) to (27);
%		\draw (27) to (20);
%		\draw (21) to (15.center);
%		\draw (24) to (26.center);
%		\draw (29.center) to (27);
%		\draw (24) to (21);
%	\end{pgfonlayer}
%\end{tikzpicture}
%\right\rrbracket
%$$
%\end{definition}

%Notice that the choice of which wires to straighten out the zig-zag is arbitrary.

\end{remark}


\begin{lemma}
\label{lem:parisocb}
$\Par\Iso(\cb_2)$ is a presentation for the prop $(\Par\Iso(\Mat(\F_2),+))$.
\end{lemma}
%This follows from \cite[??]{ih}.


We can get partial maps by freely adding a counit to the nonunital, noncounital special commutative Frobenius algebra:

\begin{definition}

Let $\Par(\cb_2)$ denote the pushout of the diagram of props:
$$
\Par\Iso(\cb_2)  \leftarrow  \surj^\op \rightarrow   \cm^\op
$$

\end{definition}



%The following Lemma follows after meticulous calculation and the application of \cite[Lem. 3.5]{zxa}:

\begin{lemma}
\label{lem:parcb}

$\Par(\cb_2)$ is a presentation for the prop $(\Par(\Mat(\F_2),+))$.
\end{lemma}



\begin{proof}
One must show that the following diagram commutes:

%
%\renewcommand{\cubetopbl}{$\inj(\cb_2)$}
%\renewcommand{\cubetopbr}{$\inj(\cb_2)^\op \otimes_{\Iso(\cb_2)} \inj(\cb_2)$}
%\renewcommand{\cubetopfl}{$\inj(\cb_2)\otimes_{\Iso(\cb_2)} \surj(\cb_2)^\op$}
%\renewcommand{\cubetopfr}{$\Par(\cb_2)$}\ParIso(\cb_2)
%\renewcommand{\cubebotbl}{$(\inj(\Mat(\F_2)), +)$}
%\renewcommand{\cubebotbr}{$(\Par\Iso(\Mat(\F_2)), +)$}
%\renewcommand{\cubebotfl}{$(\Par\surj(\Mat(\F_2)), +)$}
%\renewcommand{\cubebotfr}{}
%
%$$
%\xymatrixrowsep{3mm}\xymatrixcolsep{1mm}
%\xymatrix{
%                                       & \mbox{\cubetopbl} \ar[rr] \ar[dl] \ar[dd]^(.7){\cong}      &                                                  & \mbox{\cubetopbr}  \ar[dd]^{\cong} \ar[dl] \\
%\mbox{\cubetopfl} \ar[rr]  \ar[dd]_{\cong}           &                                                                                              &\mbox{\cubetopfr} \ar@{-->}[dd]^(.35){\cong}   \skewpullbackcorner[ul]              \\
%                                       &  \mbox{\cubebotbl} \ar[dl] \ar[rr]                    &                                                  & \mbox{\cubebotbr} \ar@/^1pc/[ddl] \ar[dl] \\
%\mbox{\cubebotfl} \ar@/_1pc/[drr] \ar[rr]  &                                                                                             & \mbox{\cubebotfr} \skewpullbackcorner[ul]    \ar@{-->}[d]^{\cong}  \\
%                                                   &                                                                                             & (\Par(\Mat(\F_2)),+)
%}
%$$
%

\renewcommand{\cubetopbl}{$\surj^\op$}
\renewcommand{\cubetopbr}{$\cm^\op$}
\renewcommand{\cubetopfl}{$\ParIso(\cb_2)$}
\renewcommand{\cubetopfr}{$\Par(\cb_2)$}
\renewcommand{\cubebotbl}{$\surj^\op$ }
\renewcommand{\cubebotbr}{$\cm^\op$ }
\renewcommand{\cubebotfl}{$\ParIso(\Mat(\F_2)),+)$ }
\renewcommand{\cubebotfr}{}

$$
\xymatrixrowsep{2mm}\xymatrixcolsep{2mm}
\xymatrix{
                                       & \mbox{\cubetopbl} \ar[rr] \ar[dl] \ar@{=}[dd]     &                                                  & \mbox{\cubetopbr} \ar@{=}[dd] \ar[dl] \\
\mbox{\cubetopfl} \ar[rr]  \ar[dd]_{\cong}           &                                                                                              &\mbox{\cubetopfr} \ar@{-->}[dd]^(.35){\cong}   \skewpullbackcorner[ul]              \\
                                       &  \mbox{\cubebotbl} \ar[dl] \ar[rr]                    &                                                  & \mbox{\cubebotbr} \ar@/^1pc/[ddl] \ar[dl] \\
\mbox{\cubebotfl} \ar@/_1pc/[drr] \ar[rr]  &                                                                                             & \mbox{\cubebotfr} \skewpullbackcorner[ul]    \ar@{-->}[d]^{\cong}  \\
                                                   &                                                                                             & (\Par(\Mat(\F_2)),+)
}
$$

It doesn't take to much work to show that $\ParIso(\cb_2)\cong\ParIso(\Mat(\F_2))$ is a discrete inverse category (defined in \cite[\S 4.3]{giles}).
We know that the counital completion of a discrete inverse category is the same as its Cartesian completion from \cite[Lem. 3.5]{zxa}; moreover, the Cartesian completion of  $\ParIso(\Mat(\F_2))$ is $\Par(\Mat(\F_2))$.  So this diagram commutes as a consequence.

\end{proof}



%\begin{comment}
This props has a particularly elegant presentation which is given in \S \ref{subsubsec:presentations:one:par}.
%\end{comment}



%
%\begin{corollary}
%Equivalently, $(\Par(\Mat(\F_2),+))$ is presented by the coproduct
%$ T_{\eta_X} + \cb_2 $
%modulo:
%
%?????
%
%\end{corollary}


%
%For $R$ a PID, $(\Span(\Mat(R)),+)$ TODO
%
%
%
%
%For $R$ a PID, $(\Rel(\Mat(R)),+)$ TODO
%
%
%  Kernel, Image, orthogonal complement
%  Pushout of TODO
%
%
%Because are self-dual with respect to the transpose functor, we omit the discussion of cospans and corelations.
%


%I will omit the discussion of spans and relations; however, the details are contained in \cite{ih}.




\begin{definition}
Let $\Span(\cb_2)$ denote the pushout of the diagram of props:
$$
\Par(\cb_2)^\op \leftarrow  \ParIso(\cb_2) \rightarrow \Par(\cb_2)
$$
\end{definition}

The following lemma holds because of \cite[Lem. 4.3]{zxa}:


\begin{lemma}
\label{lem:spancb}

$\Span(\cb_2)$ is a presentation for the prop $(\Span(\Mat(\F_2)), +)$.
\end{lemma}


\begin{proof}


\renewcommand{\cubetopbl}{$\inj(\cb_2)^\op \otimes_{\Iso(\cb_2)} \inj(\cb_2)$}
\renewcommand{\cubetopbr}{$\Par(\cb_2)$}
\renewcommand{\cubetopfl}{$\Par(\cb_2)^\op$}
\renewcommand{\cubetopfr}{$\Span(\cb_2)$}
\renewcommand{\cubebotbl}{$(\Par\Iso(\Mat(\F_2)),+)$ }
\renewcommand{\cubebotbr}{$(\Par(\Mat(\F_2)),+)$ }
\renewcommand{\cubebotfl}{$(\Par(\Mat(\F_2)),+)^\op$ }
\renewcommand{\cubebotfr}{}

$$
\xymatrixrowsep{2mm}\xymatrixcolsep{1mm}
\xymatrix{
                                       & \mbox{\cubetopbl} \ar[rr] \ar[dl] \ar[dd]^(.7){\cong}      &                                                  & \mbox{\cubetopbr}  \ar[dd]^{\cong} \ar[dl] \\
\mbox{\cubetopfl} \ar[rr]  \ar[dd]_{\cong}           &                                                                                              &\mbox{\cubetopfr} \ar@{-->}[dd]^(.35){\cong}   \skewpullbackcorner[ul]              \\
                                       &  \mbox{\cubebotbl} \ar[dl] \ar[rr]                    &                                                  & \mbox{\cubebotbr} \ar@/^1pc/[ddl] \ar[dl] \\
\mbox{\cubebotfl} \ar@/_1pc/[drr] \ar[rr]  &                                                                                             & \mbox{\cubebotfr} \skewpullbackcorner[ul]    \ar@{-->}[d]^{\cong}_F \\
                                                   &                                                                                             & (\Span(\Mat(\F_2)),+)
}
$$


The cube easily commutes.  What remains to be shown is that the universal map $F$ is an isomorphism of props.  It is clearly the identity on objects, so we just need to show it is full and faithful.

It is clearly full as any span $ n \xleftarrow{ f}  k \xrightarrow{g } m$, we have:
$$
F\left( (n \xleftarrow{f} k = k);(k = k \xrightarrow{g} m) \right)=n \xleftarrow{ f}  k \xrightarrow{g } m
$$ 
For faithfulness, we must observe given for any two isomorphic maps in $\Span(\Mat(\F_2))$:
$$
\xymatrixrowsep{2mm}\xymatrixcolsep{6mm}
\xymatrix{
          & k \ar[dl]_{f'} \ar[dd]_{\cong}^{h} \ar[dr]^{g'}\\
n  &                                                                                                    & m\\
         & k \ar[ul]^{f} \ar[ur]_{g}\\
}
$$
Then in the domain of $F$, we have:
{
\xymatrixrowsep{0mm}\xymatrixcolsep{1.7mm}
\begin{align*}
&
\xymatrix{
   & k \ar[dl]_f \ar@{=}[dr]\\
n &                                      &k
};
\xymatrix{
   & k \ar[dr]^g \ar@{=}[dl]\\
k &                                      &m
}
%&
 =
\xymatrix{
   & k \ar[dl]_f \ar@{=}[dr]\\
n &                                      &k
};
\xymatrix{
   & k \ar@{=}[dl] \ar@{=}[dr]\\
k &                                             & k\\
   & k \ar[ul]^h \ar[ur]_h \ar[uu]^\cong_h
};
\xymatrix{
   & k \ar[dr]^g \ar@{=}[dl]\\
k &                                      &m
}\\
 &=
\xymatrix{
   & k \ar[dl]_f \ar@{=}[dr]\\
n &                                      &k
};
\xymatrix{
   & k \ar[dl]_h \ar@{=}[dr]\\
k &                                         & k
};
\xymatrix{
   & k \ar[dr]^h \ar@{=}[dl]\\
k &                                         & k
};
\xymatrix{
   & k \ar[dr]^g \ar@{=}[dl]\\
k &                                      &m
}
%\\&
=
\xymatrix{
            &                                                        &k \ar[dl]_{h} \ar@{=}[dr] \ar@/_1.2pc/[ddll]_{f'}\\
            & k \ar@{=}[dr] \ar[dl]^{f}&                                                          & k \ar@{=}[dr] \ar[dl]_{h}\\
n &                                                         & k                                             &                                                         &k
};
\xymatrix{
            &                                                        & k \ar[dr]^{h} \ar@{=}[dl] \ar@/^1.2pc/[ddrr]^{g'}  \\
            & k \ar[dr]^{h}   \ar@{=}[dl] &                                                          & k \ar@{=}[dl] \ar[dr]_{g}\\
k &                                                         & k                                             &                                                         &m
}
\end{align*}
}
 
\end{proof}


Given a PID $k$, the prop $(\Span(\Mat(k)), +)$ is already known to have a presentation given in terms of ``interacting Hopf algebras" \cite[Definition 3.13]{ih}.  This is also the way in which the phase-free fragment of the ZX-calculus would be presented, in terms of two Frobenius algebras  corresponding to the $Z$ and $X$ observables, interacting to form Hopf algebras in addition to a few more equations.
%\begin{comment}
 We have included this presentation in \S \ref{subsubsec:presentations:one:span}.
%\end{comment}



%
%\begin{corollary}
%Because  prop $(\Mat(\F_2),+)$ is self dual, the prop $\Csp(\cb_2)$ obtained by swapping colours is a presentation for the prop $(\Csp(\Mat(\F_2)),+)$
%\end{corollary}
%
%
%
%
%\begin{definition} 
%
%Consider the prop $\Rel(\cb_2)$ given by the following pushout of the following diagram of props:
%
%$$
%\Csp(\cb_2) \leftarrow \inj(\cb_2)^\op + \inj(\cb_2)  \rightarrow \ParIso(\cb_2)
%$$
%
%\end{definition}
%
%This particular construction of the following presentation is due to \cite[Thm. 3.6]{universal}; however, it was constructed in a slightly different manner before in \cite[Thm. 3.49]{ih}:
%
%\begin{lemma}
%$\Rel(\cb_2)$ is a presentation for the prop $(\Rel(\Mat(\F_2)), +)$
%\end{lemma}
%
%
%
%\begin{proof}
%
%
%
%\renewcommand{\cubetopbl}{$\inj(\cb_2)^\op + \inj(\cb_2)$}
%\renewcommand{\cubetopbr}{$\inj(\cb_2)\otimes_{\Iso(\cb_2)} \inj(\cb_2)^\op$}
%\renewcommand{\cubetopfl}{$\cb_2^\op \otimes_{\Iso(\cb_2)}  \cb_2 $}
%\renewcommand{\cubetopfr}{$\Rel(\cb_2)$}
%\renewcommand{\cubebotbl}{$(\inj(\Mat(\F_2)),+)^\op+(\inj(\Mat(\F_2)),+)$ }
%\renewcommand{\cubebotbr}{$(\Par\Iso(\Mat(\F_2)),+)$ }
%\renewcommand{\cubebotfl}{$(\Csp(\Mat(\F_2)),+)$ }
%\renewcommand{\cubebotfr}{$$}
%
%$$
%\xymatrixrowsep{3mm}\xymatrixcolsep{1mm}
%\xymatrix{
%                                       & \mbox{\cubetopbl} \ar[rr] \ar[dl] \ar[dd]^(.7){\cong}      &                                                  & \mbox{\cubetopbr}  \ar[dd]^{\cong} \ar[dl] \\
%\mbox{\cubetopfl} \ar[rr]  \ar[dd]_{\cong}           &                                                                                              &\mbox{\cubetopfr} \ar@{-->}[dd]^(.35){\cong}   \skewpullbackcorner[ul]              \\
%                                       &  \mbox{\cubebotbl} \ar[dl] \ar[rr]                    &                                                  & \mbox{\cubebotbr} \ar@/^1pc/[ddl] \ar[dl] \\
%\mbox{\cubebotfl} \ar@/_1pc/[drr] \ar[rr]  &                                                                                             & \mbox{\cubebotfr} \skewpullbackcorner[ul]    \ar@{-->}[d]^{\cong}  \\
%                                                   &                                                                                             & (\Rel(\Mat(\F_2)),+)
%}
%$$
%
%
%\end{proof}
%
%


\section{Additive affine models}
\label{sec:two}


\begin{definition}
Consider the prop  $\Aff\cb_2$ given by adjoining the following generator to $\cb_2$
\hfil
$
\begin{tikzpicture}
	\begin{pgfonlayer}{nodelayer}
		\node [style=none] (0) at (-3.75, -0.25) {};
		\node [style=X] (1) at (-3.75, -1) {$1$};
	\end{pgfonlayer}
	\begin{pgfonlayer}{edgelayer}
		\draw (0.center) to (1);
	\end{pgfonlayer}
\end{tikzpicture}
$

modulo the equations:\hspace*{2.2cm}
$
\begin{tikzpicture}
	\begin{pgfonlayer}{nodelayer}
		\node [style=X] (0) at (0.75, 0.25) {$1$};
		\node [style=Z] (1) at (0.75, 0.75) {};
		\node [style=none] (2) at (0.5, 1.5) {};
		\node [style=none] (3) at (1, 1.5) {};
	\end{pgfonlayer}
	\begin{pgfonlayer}{edgelayer}
		\draw (1) to (0);
		\draw [in=-90, out=60] (1) to (3.center);
		\draw [in=120, out=-90] (2.center) to (1);
	\end{pgfonlayer}
\end{tikzpicture}
\erefop{bi.two}
\begin{tikzpicture}
	\begin{pgfonlayer}{nodelayer}
		\node [style=X] (0) at (0.5, 0.5) {$1$};
		\node [style=none] (1) at (0.5, 1.75) {};
		\node [style=none] (2) at (1, 1.75) {};
		\node [style=X] (3) at (1, 0.5) {$1$};
	\end{pgfonlayer}
	\begin{pgfonlayer}{edgelayer}
		\draw (0) to (1.center);
		\draw (2.center) to (3);
	\end{pgfonlayer}
\end{tikzpicture}
\hspace*{1cm}
\begin{tikzpicture}
	\begin{pgfonlayer}{nodelayer}
		\node [style=X] (0) at (0, 0) {$1$};
		\node [style=Z] (1) at (0, 0.75) {};
	\end{pgfonlayer}
	\begin{pgfonlayer}{edgelayer}
		\draw (1) to (0);
	\end{pgfonlayer}
\end{tikzpicture}
\eref{extra}
$



\end{definition}



\begin{lemma} \cite[\S 4]{lafont}
 $\Aff\cb_2$ is a presentation for the prop $(\Aff\Mat(\F_2),+)$.
\end{lemma}


Note that this assumes that affine matrices are non-empty, as this is a prop.  This will become a problem later, when we wish to pull back affine spaces.






\begin{definition}
Consider the prop $\Iso(\Aff\cb_2)$ generated by the controlled not gate, and the not gate (interpreted as matrices):
\hspace*{.2cm}
$
\left\llbracket
\begin{tikzpicture}
	\begin{pgfonlayer}{nodelayer}
		\node [style=oplus] (5) at (0.5, 2.75) {};
		\node [style=dot] (6) at (0, 2.75) {};
		\node [style=none] (7) at (0.5, 3.5) {};
		\node [style=none] (8) at (0.5, 2) {};
		\node [style=none] (9) at (0, 2) {};
		\node [style=none] (10) at (0, 3.5) {};
	\end{pgfonlayer}
	\begin{pgfonlayer}{edgelayer}
		\draw (8.center) to (5);
		\draw (5) to (7.center);
		\draw (10.center) to (6);
		\draw (6) to (5);
		\draw (6) to (9.center);
	\end{pgfonlayer}
\end{tikzpicture}
\right\rrbracket
=
\begin{tikzpicture}
	\begin{pgfonlayer}{nodelayer}
		\node [style=X] (0) at (-0.25, -1) {};
		\node [style=Z] (1) at (-0.75, -1.75) {};
		\node [style=none] (2) at (0, -2.25) {};
		\node [style=none] (3) at (-0.25, -0.5) {};
		\node [style=none] (4) at (-1, -0.5) {};
		\node [style=none] (5) at (-0.75, -2.25) {};
	\end{pgfonlayer}
	\begin{pgfonlayer}{edgelayer}
		\draw (3.center) to (0);
		\draw [in=90, out=-75] (0) to (2.center);
		\draw (5.center) to (1);
		\draw (1) to (0);
		\draw [in=-90, out=105] (1) to (4.center);
	\end{pgfonlayer}
\end{tikzpicture}
\hspace*{1cm}
\left\llbracket
\begin{tikzpicture}
	\begin{pgfonlayer}{nodelayer}
		\node [style=none] (0) at (0, -0.5) {};
		\node [style=none] (1) at (0, -1.5) {};
		\node [style=oplus] (2) at (0, -1) {};
	\end{pgfonlayer}
	\begin{pgfonlayer}{edgelayer}
		\draw (1.center) to (2);
		\draw (2) to (0.center);
	\end{pgfonlayer}
\end{tikzpicture}
\right\rrbracket
=
\begin{tikzpicture}
	\begin{pgfonlayer}{nodelayer}
		\node [style=none] (0) at (1, 1) {};
		\node [style=none] (1) at (0.75, 2) {};
		\node [style=X] (2) at (0.75, 1.5) {};
		\node [style=X] (3) at (0.5, 1) {$1$};
		\node [style=none] (4) at (1, 0.5) {};
	\end{pgfonlayer}
	\begin{pgfonlayer}{edgelayer}
		\draw (1.center) to (2);
		\draw [in=90, out=-45, looseness=0.75] (2) to (0.center);
		\draw [in=90, out=-135, looseness=0.75] (2) to (3);
		\draw (0.center) to (4.center);
	\end{pgfonlayer}
\end{tikzpicture}
$

Modulo the relations of $\Iso(\cb_2)$ as well as the additional relations:
$$
\begin{tikzpicture}
	\begin{pgfonlayer}{nodelayer}
		\node [style=oplus] (0) at (0, 0) {};
		\node [style=oplus] (1) at (0, 0.5) {};
		\node [style=none] (2) at (0, 1) {};
		\node [style=none] (3) at (0, -0.5) {};
	\end{pgfonlayer}
	\begin{pgfonlayer}{edgelayer}
		\draw (2.center) to (3.center);
	\end{pgfonlayer}
\end{tikzpicture}
\eqzxa{cnot.seven}
\begin{tikzpicture}
	\begin{pgfonlayer}{nodelayer}
		\node [style=none] (2) at (0, 1) {};
		\node [style=none] (3) at (0, -0.5) {};
	\end{pgfonlayer}
	\begin{pgfonlayer}{edgelayer}
		\draw (2.center) to (3.center);
	\end{pgfonlayer}
\end{tikzpicture}
\hspace*{.5cm}
\begin{tikzpicture}
	\begin{pgfonlayer}{nodelayer}
		\node [style=dot] (0) at (2, 2) {};
		\node [style=oplus] (1) at (2.5, 2) {};
		\node [style=oplus] (2) at (2, 1.5) {};
		\node [style=none] (3) at (2, 1) {};
		\node [style=none] (4) at (2.5, 1) {};
		\node [style=none] (5) at (2, 2.5) {};
		\node [style=none] (6) at (2.5, 2.5) {};
	\end{pgfonlayer}
	\begin{pgfonlayer}{edgelayer}
		\draw (3.center) to (2);
		\draw (2) to (0);
		\draw (0) to (5.center);
		\draw (6.center) to (1);
		\draw (1) to (0);
		\draw (4.center) to (1);
	\end{pgfonlayer}
\end{tikzpicture}
\eqzxa{cnot.eight}
\begin{tikzpicture}
	\begin{pgfonlayer}{nodelayer}
		\node [style=dot] (0) at (2, 1.5) {};
		\node [style=oplus] (1) at (2.5, 1.5) {};
		\node [style=none] (3) at (2, 1) {};
		\node [style=none] (4) at (2.5, 1) {};
		\node [style=none] (5) at (2, 2.5) {};
		\node [style=none] (6) at (2.5, 2.5) {};
		\node [style=oplus] (7) at (2, 2) {};
		\node [style=oplus] (8) at (2.5, 2) {};
	\end{pgfonlayer}
	\begin{pgfonlayer}{edgelayer}
		\draw (0) to (5.center);
		\draw (6.center) to (1);
		\draw (1) to (0);
		\draw (4.center) to (1);
		\draw (3.center) to (0);
	\end{pgfonlayer}
\end{tikzpicture}
\hspace*{.5cm}
\begin{tikzpicture}
	\begin{pgfonlayer}{nodelayer}
		\node [style=dot] (0) at (2, 2) {};
		\node [style=oplus] (1) at (2.5, 2) {};
		\node [style=none] (3) at (2, 1) {};
		\node [style=none] (4) at (2.5, 1) {};
		\node [style=none] (5) at (2, 2.5) {};
		\node [style=none] (6) at (2.5, 2.5) {};
		\node [style=oplus] (8) at (2.5, 1.5) {};
	\end{pgfonlayer}
	\begin{pgfonlayer}{edgelayer}
		\draw (0) to (5.center);
		\draw (6.center) to (1);
		\draw (1) to (0);
		\draw (4.center) to (1);
		\draw (3.center) to (0);
	\end{pgfonlayer}
\end{tikzpicture}
\eqzxa{cnot.nine}
\begin{tikzpicture}
	\begin{pgfonlayer}{nodelayer}
		\node [style=dot] (0) at (2, 1.5) {};
		\node [style=oplus] (1) at (2.5, 1.5) {};
		\node [style=none] (3) at (2, 1) {};
		\node [style=none] (4) at (2.5, 1) {};
		\node [style=none] (5) at (2, 2.5) {};
		\node [style=none] (6) at (2.5, 2.5) {};
		\node [style=oplus] (8) at (2.5, 2) {};
	\end{pgfonlayer}
	\begin{pgfonlayer}{edgelayer}
		\draw (0) to (5.center);
		\draw (6.center) to (1);
		\draw (1) to (0);
		\draw (4.center) to (1);
		\draw (3.center) to (0);
	\end{pgfonlayer}
\end{tikzpicture}
$$

\end{definition}


\begin{lemma}\cite[Thm. 11]{lafont}
$\Iso(\Aff\cb_2)$ is a presentation for the prop $(\Iso(\Aff\Mat(\F_2),+))$.
\end{lemma}






\begin{definition}
Let $\inj(\Aff\cb_2)$ denote the pushout of the diagram of props:
$$
 \inj(\cb_2) \leftarrow  \Iso(\cb_2)\rightarrow  \Iso(\Aff\cb_2)
$$
\end{definition}




\begin{lemma}
\label{lem:injaffcb}
$\inj(\Aff\cb_2)$ is a presentation for the prop $(\inj(\Aff\Mat(\F_2)),+)$.
\end{lemma}


\begin{proof}
Consider the following diagram:


\renewcommand{\cubetopbl}{$\Iso(\cb_2)$}
\renewcommand{\cubetopbr}{$\Iso(\Aff\cb_2)$}
\renewcommand{\cubetopfl}{$\inj(\cb_2)$}
\renewcommand{\cubetopfr}{$\inj(\Aff\cb_2)$}
\renewcommand{\cubebotbl}{$(\Iso(\Mat(\F_2)),+)$ }
\renewcommand{\cubebotbr}{$(\Iso(\Aff\Mat(\F_2)),+)$ }
\renewcommand{\cubebotfl}{$(\inj(\Mat(\F_2)),+)$ }
\renewcommand{\cubebotfr}{}

$$
\xymatrixrowsep{2mm}\xymatrixcolsep{1.5mm}
\xymatrix{
                                       & \mbox{\cubetopbl} \ar[rr] \ar[dl] \ar[dd]^(.7){\cong}      &                                                  & \mbox{\cubetopbr}  \ar[dd]^{\cong} \ar[dl] \\
\mbox{\cubetopfl} \ar[rr]  \ar[dd]_{\cong}           &                                                                                              &\mbox{\cubetopfr} \ar@{-->}[dd]^(.35){\cong}   \skewpullbackcorner[ul]              \\
                                       &  \mbox{\cubebotbl} \ar[dl] \ar[rr]                    &                                                  & \mbox{\cubebotbr} \ar@/^1pc/[ddl] \ar[dl] \\
\mbox{\cubebotfl} \ar@/_1pc/[drr] \ar[rr]  &                                                                                             & \mbox{\cubebotfr} \skewpullbackcorner[ul]    \ar@{-->}[d]^{\cong}_F  \\
                                                   &                                                                                             & (\inj(\Aff\Mat(\F_2)),+)
}
$$

 The rear and left faces of the cube commute and the vertical maps are all isomorphisms. Therefore, the whole cube commutes via universal property of the pushout, with the upper universal map being an isomorphism.

We seek to show that the lower universal map  $F$ is also an isomorphism.  It is clearly the identity on objects, so we just have to show fullness and faithfulness.

For fullness, consider any map $n\xrightarrowtail{(A,x)} m$ in $(\inj(\Aff\Mat(\F_2)),+)$.  Note that this can be factored into:
$$
n\xrightarrowtail{(A,0)} m \xrightarrowiso{(1,x)}  m
$$
Which lies in the image of $F$ as $m \xrightarrowiso{(1,x)} m$ is an isomorphism.

For faithfulness, we show that every map in $(\Iso(\Aff\Mat(\F_2)),+)+_{(\Iso(\Mat(\F_2)),+)} (\inj(\Mat(\F_2)),+)$ can be factored uniquely in this way. 
There are two cases:
$$
\left( n \xrightarrowtail{ A} m ; m \xrightarrowiso{(B, x)} m \right)
= \left( n \xrightarrowtail{ A} m ; m \xrightarrowiso{(B, 0)} m; m \xrightarrowiso{(1, x)}  m \right)
= \left( n \xrightarrowtail{ A;B}  m\xrightarrowiso{(1, x)}  m \right)
$$
$$
\left(n \xrightarrowtail{ (A,x)} m ; m \xrightarrowiso{B} m \right)
= \left( n \xrightarrowtail{ (A,0)}m; m \xrightarrowiso{(1,x)} m ; m \xrightarrowiso{B} m \right)
= \left( n \xrightarrowtail{ A }m; m \xrightarrowiso{(B,B(x))} m  \right)
= \left( n \xrightarrowtail{ A;B }m; m \xrightarrowiso{(1,B(x))} m  \right)
$$
So every map in this pushout has the correct form, which is unique by construction.
\end{proof}


To define partial isomorphisms, we must add a point to the constituent props of the desired distributive law, because the empty set can arise as a subobject by pullback (where the empty set is not properly an object in the prop).

%\begin{definition}
%Given a prop $\X$, let $\X!$ denote the prop generated by adding a scalar $0$,  quotiented by the equation, for all parallel $f,g$: 
%$
%f \otimes 0  =  g \otimes 0
%$ 
%\end{definition}
%
%
%
%That is to say, $\X!$ is the prop with zero maps formally added.  In affine matrices, there is no proper zero object: the one element space is the terminal object and the empty set is the initial object.  By taking spans of affine matrices, the initial object becomes a zero object; however, seeing as we are working with props, the empty set can not be represented using this formalism.  Thus we just add the zero object as a subjobject.





\begin{definition}
\label{def:isoaffcbzero}
Let $\Iso(\Aff\cb_2)^{+1}$ denote the prop obtained by adjoining the following generator to $\Iso(\Aff\cb_2)$ 
$
\begin{tikzpicture}
	\begin{pgfonlayer}{nodelayer}
		\node [style=X] (0) at (0, 0) {$1$};
	\end{pgfonlayer}
\end{tikzpicture}
$
modulo the equations:
$$
\begin{tikzpicture}
	\begin{pgfonlayer}{nodelayer}
		\node [style=X] (0) at (0, 0) {$1$};
		\node [style=X] (3) at (0.5, 0) {$1$};
	\end{pgfonlayer}
\end{tikzpicture}
\eqzxa{zero.one}
\begin{tikzpicture}
	\begin{pgfonlayer}{nodelayer}
		\node [style=X] (0) at (0, 0) {$1$};
	\end{pgfonlayer}
\end{tikzpicture},
\hspace*{.5cm}
\begin{tikzpicture}
	\begin{pgfonlayer}{nodelayer}
		\node [style=X] (0) at (0, 1) {$1$};
		\node [style=none] (1) at (0.5, 0.5) {};
		\node [style=none] (2) at (0.5, 1.5) {};
		\node [style=none] (3) at (1, 1.5) {};
		\node [style=none] (4) at (1, 0.5) {};
		\node [style=dot] (5) at (0.5, 1) {};
		\node [style=oplus] (6) at (1, 1) {};
	\end{pgfonlayer}
	\begin{pgfonlayer}{edgelayer}
		\draw [in=90, out=-90] (2.center) to (1.center);
		\draw [in=-90, out=90] (4.center) to (3.center);
		\draw (6) to (5);
	\end{pgfonlayer}
\end{tikzpicture}
\eqzxa{zero.two}
\begin{tikzpicture}
	\begin{pgfonlayer}{nodelayer}
		\node [style=X] (0) at (0, 1) {$1$};
		\node [style=none] (1) at (0.5, 0.5) {};
		\node [style=none] (2) at (0.5, 1.5) {};
		\node [style=none] (3) at (1, 1.5) {};
		\node [style=none] (4) at (1, 0.5) {};
	\end{pgfonlayer}
	\begin{pgfonlayer}{edgelayer}
		\draw [in=90, out=-90] (2.center) to (1.center);
		\draw [in=-90, out=90] (4.center) to (3.center);
	\end{pgfonlayer}
\end{tikzpicture},
\hspace*{.5cm}
\begin{tikzpicture}
	\begin{pgfonlayer}{nodelayer}
		\node [style=X] (0) at (0, 1) {$1$};
		\node [style=none] (1) at (0.5, 0.5) {};
		\node [style=none] (2) at (1, 1.5) {};
		\node [style=none] (3) at (0.5, 1.5) {};
		\node [style=none] (4) at (1, 0.5) {};
	\end{pgfonlayer}
	\begin{pgfonlayer}{edgelayer}
		\draw [in=90, out=-90] (2.center) to (1.center);
		\draw [in=-90, out=90] (4.center) to (3.center);
	\end{pgfonlayer}
\end{tikzpicture}
\eqzxa{zero.three}
\begin{tikzpicture}
	\begin{pgfonlayer}{nodelayer}
		\node [style=X] (0) at (0, 1) {$1$};
		\node [style=none] (1) at (0.5, 0.5) {};
		\node [style=none] (2) at (0.5, 1.5) {};
		\node [style=none] (3) at (1, 1.5) {};
		\node [style=none] (4) at (1, 0.5) {};
	\end{pgfonlayer}
	\begin{pgfonlayer}{edgelayer}
		\draw [in=90, out=-90] (2.center) to (1.center);
		\draw [in=-90, out=90] (4.center) to (3.center);
	\end{pgfonlayer}
\end{tikzpicture},
\hspace*{.5cm}
\begin{tikzpicture}
	\begin{pgfonlayer}{nodelayer}
		\node [style=X] (0) at (0, 1) {$1$};
		\node [style=none] (1) at (0.5, 0.5) {};
		\node [style=none] (2) at (0.5, 1.5) {};
		\node [style=oplus] (3) at (0.5, 1) {};
	\end{pgfonlayer}
	\begin{pgfonlayer}{edgelayer}
		\draw (2.center) to (1.center);
	\end{pgfonlayer}
\end{tikzpicture}
\eqzxa{zero.four}
\begin{tikzpicture}
	\begin{pgfonlayer}{nodelayer}
		\node [style=X] (0) at (0, 1) {$1$};
		\node [style=none] (1) at (0.5, 0.5) {};
		\node [style=none] (2) at (0.5, 1.5) {};
	\end{pgfonlayer}
	\begin{pgfonlayer}{edgelayer}
		\draw (2.center) to (1.center);
	\end{pgfonlayer}
\end{tikzpicture}
$$
\end{definition}


\begin{lemma}
$\Iso(\Aff\cb_2)^{+1}$ is a presentation for the subcategory of $(\Span(\Aff\Fin\Vect(\F_2)), +)$ generated by spans $\F_2^n = \F_2^n \xrightarrow[\cong]{f} \F_2^n$ and $\F_2^n \xleftarrowtail {?} \emptyset \xrightarrowtail{?}  \F_2^n$, for all $n \in \N$ and isomorphisms $f$. 
\end{lemma}


%
%\begin{lemma}
%
%The props $\Iso(\Aff\cb_2)^{+1}$ and $\Iso(\Aff\cb_2)!$ are isomorphic.
%
%\end{lemma}

\begin{proof}
Identify this new generator with the span $\F_2^0 \leftarrow \emptyset \rightarrow \F_2^0$.  If there is a factor of 
$
\begin{tikzpicture}
	\begin{pgfonlayer}{nodelayer}
		\node [style=X] (0) at (0, 0) {$1$};
	\end{pgfonlayer}
\end{tikzpicture}
$,   repeatedly apply these identities from left to right until the diagram corresponding to the identity tensored by $
\begin{tikzpicture}
	\begin{pgfonlayer}{nodelayer}
		\node [style=X] (0) at (0, 0) {$1$};
	\end{pgfonlayer}
\end{tikzpicture}
$ is obtained, which is as a normal form.
\end{proof}

\begin{definition}
Let $\inj(\Aff\cb_2)^{+1}$ denote the pushout of the diagram of props:


$$
\inj(\Aff\cb_2) \leftarrow \Iso(\Aff\cb_2) \rightarrow \Iso(\Aff\cb_2)^{+1}
$$


\end{definition}

\begin{lemma}
$\inj(\Aff\cb_2)^{+1}$ is a presentation for the subcategory of $(\Span(\Aff\Fin\Vect(\F_2)), +)$ generated by spans $\F_2^n = \F_2^n \xrightarrowtail{e} \F_2^m$ and $\F_2^n \xleftarrowtail{?} \emptyset \xrightarrowtail{?}  \F_2^n$, for all $n,m \in \N$ and monics $e$. 
\end{lemma}
%
%\begin{lemma}
%The props $\inj(\Aff\cb_2)^{+1}$ and $\inj(\Aff\cb_2)!$ are isomorphic.
%\end{lemma}

The proof of this lemma is essentially the same for $\Iso(\Aff\cb_2)^{+1}$, although diagrams with a factor of
$
\begin{tikzpicture}
	\begin{pgfonlayer}{nodelayer}
		\node [style=X] (0) at (0, 0) {$1$};
	\end{pgfonlayer}
\end{tikzpicture}
$ are reduced to the following normal form:
\hspace*{3cm}
$
\begin{tikzpicture}
	\begin{pgfonlayer}{nodelayer}
		\node [style=X] (0) at (0, 1.25) {$1$};
		\node [style=none] (1) at (0.5, 0.5) {};
		\node [style=none] (2) at (0.5, 1.75) {};
		\node [style=none] (3) at (1, 0.5) {};
		\node [style=none] (4) at (1, 1.75) {};
		\node [style=X] (5) at (1.5, 0.75) {};
		\node [style=X] (6) at (2, 0.75) {};
		\node [style=none] (7) at (1.5, 1.75) {};
		\node [style=none] (8) at (2, 1.75) {};
		\node [style=none] (9) at (0.75, 1.5) {$n$};
		\node [style=none] (10) at (1.75, 1.5) {$m$};
		\node [style=none] (11) at (1.77, 1.25) {$\cdots$};
		\node [style=none] (12) at (0.77, 1.25) {$\cdots$};
	\end{pgfonlayer}
	\begin{pgfonlayer}{edgelayer}
		\draw (2.center) to (1.center);
		\draw (4.center) to (3.center);
		\draw (5) to (7.center);
		\draw (8.center) to (6);
	\end{pgfonlayer}
\end{tikzpicture}
$

Unlike in the linear case, now we must consider a distributive law over a prop which is not a groupoid: we add a single idempotent corresponding to the empty set to the isomorphisms.  To satisfy the requirement that this prop is a sub-prop of the left and right components of the  distributive law, we also add this idempotent to the injections and the co-injections:


\begin{definition}
\label{def:parisoaffcb}
Consider the prop $\pr\iso\Aff\cb_2$ generated by the distributive law of props:

$$
 (\inj(\Aff\cb_2)^{+1})^\op \otimes_{\Iso(\Aff\cb_2)^{+1}}  \inj(\Aff\cb_2)^{+1}
$$
Given by the equations of $\pr\iso\Aff\cb_2$ as well as:
\!
$
\begin{tikzpicture}
	\begin{pgfonlayer}{nodelayer}
		\node [style=X] (0) at (0.5, 0.75) {$1$};
		\node [style=X] (1) at (0.5, 0) {};
	\end{pgfonlayer}
	\begin{pgfonlayer}{edgelayer}
		\draw (0) to (1);
	\end{pgfonlayer}
\end{tikzpicture}
=
\begin{tikzpicture}
	\begin{pgfonlayer}{nodelayer}
		\node [style=X] (0) at (0, 0) {$1$};
		\node [style=X] (1) at (0, 0.75) {};
	\end{pgfonlayer}
	\begin{pgfonlayer}{edgelayer}
		\draw (0) to (1);
	\end{pgfonlayer}
\end{tikzpicture}
\eqzxa{zero.five}
\begin{tikzpicture}
	\begin{pgfonlayer}{nodelayer}
		\node [style=X] (0) at (0, 0) {$1$};
	\end{pgfonlayer}
	\begin{pgfonlayer}{edgelayer}
	\end{pgfonlayer}
\end{tikzpicture}
$

\end{definition}


\begin{remark}
\label{rem:parisoaffcb}
$ (\inj(\Aff\cb_2)^{+1})^\op \otimes_{\Iso(\Aff\cb_2)^{+1}}  \inj(\Aff\cb_2)^{+1}$ is actually a distributive law because the only only nontrivial situation arises when controlled-not gates are sandwiched between black, or black $1$ units/counits on their target wires.  The case where there are no controlled not gates in between is resolved by the new axiom we have added.  When there are more controlled-not gates, they can be pushed past each other as follows:
$$
\begin{tikzpicture}
	\begin{pgfonlayer}{nodelayer}
		\node [style=X] (0) at (2.5, 0.25) {};
		\node [style=oplus] (1) at (2.5, -0.75) {};
		\node [style=oplus] (2) at (2.5, -1.5) {};
		\node [style=dot] (3) at (1.5, -0.75) {};
		\node [style=dot] (4) at (1, -1.5) {};
		\node [style=none] (5) at (1.5, 0.5) {};
		\node [style=none] (6) at (1, 0.5) {};
		\node [style=none] (7) at (1.25, -1) {$\iddots$};
		\node [style=none] (8) at (2.5, -1) {$\vdots$};
		\node [style=X] (9) at (2.5, -2) {$1$};
		\node [style=none] (10) at (1, -2.25) {};
		\node [style=none] (11) at (1.5, -2.25) {};
		\node [style=none] (12) at (1.25, -0.25) {$\cdots$};
		\node [style=none] (17) at (2, -2.25) {};
		\node [style=none] (18) at (2, 0.5) {};
		\node [style=oplus] (19) at (2.5, -0.25) {};
		\node [style=dot] (20) at (2, -0.25) {};
	\end{pgfonlayer}
	\begin{pgfonlayer}{edgelayer}
		\draw (0) to (1);
		\draw (1) to (3);
		\draw (5.center) to (3);
		\draw (6.center) to (4);
		\draw (4) to (2);
		\draw (4) to (10.center);
		\draw (11.center) to (3);
		\draw (9) to (2);
		\draw (17.center) to (20);
		\draw (20) to (18.center);
		\draw (19) to (20);
	\end{pgfonlayer}
\end{tikzpicture}
=
\begin{tikzpicture}
	\begin{pgfonlayer}{nodelayer}
		\node [style=X] (0) at (2.5, 0.25) {};
		\node [style=oplus] (1) at (2.5, -0.75) {};
		\node [style=oplus] (2) at (2.5, -1.5) {};
		\node [style=dot] (3) at (1.5, -0.75) {};
		\node [style=dot] (4) at (1, -1.5) {};
		\node [style=none] (5) at (1.5, 0.5) {};
		\node [style=none] (6) at (1, 0.5) {};
		\node [style=none] (7) at (1.25, -1) {$\iddots$};
		\node [style=none] (8) at (2.5, -1) {$\vdots$};
		\node [style=X] (9) at (2.5, -2.5) {};
		\node [style=none] (10) at (1, -2.75) {};
		\node [style=none] (11) at (1.5, -2.75) {};
		\node [style=none] (12) at (1.25, -0.25) {$\cdots$};
		\node [style=none] (17) at (2, -2.75) {};
		\node [style=none] (18) at (2, 0.5) {};
		\node [style=oplus] (19) at (2.5, -0.25) {};
		\node [style=dot] (20) at (2, -0.25) {};
		\node [style=oplus] (21) at (2.5, -2) {};
	\end{pgfonlayer}
	\begin{pgfonlayer}{edgelayer}
		\draw (0) to (1);
		\draw (1) to (3);
		\draw (5.center) to (3);
		\draw (6.center) to (4);
		\draw (4) to (2);
		\draw (4) to (10.center);
		\draw (11.center) to (3);
		\draw (9) to (2);
		\draw (17.center) to (20);
		\draw (20) to (18.center);
		\draw (19) to (20);
	\end{pgfonlayer}
\end{tikzpicture}
=
\begin{tikzpicture}
	\begin{pgfonlayer}{nodelayer}
		\node [style=X] (0) at (2.5, 0.75) {};
		\node [style=oplus] (1) at (2.5, -0.75) {};
		\node [style=oplus] (2) at (2.5, -1.5) {};
		\node [style=dot] (3) at (1.5, -0.75) {};
		\node [style=dot] (4) at (1, -1.5) {};
		\node [style=none] (5) at (1.5, 1.25) {};
		\node [style=none] (6) at (1, 1.25) {};
		\node [style=none] (7) at (1.25, -1) {$\iddots$};
		\node [style=none] (8) at (2.5, -1) {$\vdots$};
		\node [style=X] (9) at (2.5, -2) {};
		\node [style=none] (10) at (1, -2.25) {};
		\node [style=none] (11) at (1.5, -2.25) {};
		\node [style=none] (12) at (1.25, 0.25) {$\cdots$};
		\node [style=none] (17) at (2, -2.25) {};
		\node [style=none] (18) at (2, 1.25) {};
		\node [style=oplus] (19) at (2.5, 0.25) {};
		\node [style=dot] (20) at (2, 0.25) {};
		\node [style=oplus] (21) at (2, -0.25) {};
		\node [style=oplus] (22) at (2, 0.75) {};
	\end{pgfonlayer}
	\begin{pgfonlayer}{edgelayer}
		\draw (0) to (1);
		\draw (1) to (3);
		\draw (5.center) to (3);
		\draw (6.center) to (4);
		\draw (4) to (2);
		\draw (4) to (10.center);
		\draw (11.center) to (3);
		\draw (9) to (2);
		\draw (17.center) to (20);
		\draw (20) to (18.center);
		\draw (19) to (20);
	\end{pgfonlayer}
\end{tikzpicture}
=
\begin{tikzpicture}
	\begin{pgfonlayer}{nodelayer}
		\node [style=X] (0) at (2.5, 0.25) {};
		\node [style=oplus] (1) at (2.5, -0.75) {};
		\node [style=oplus] (2) at (2.5, -1.5) {};
		\node [style=dot] (3) at (1.5, -0.75) {};
		\node [style=dot] (4) at (1, -1.5) {};
		\node [style=none] (5) at (1.5, 1) {};
		\node [style=none] (6) at (1, 1) {};
		\node [style=none] (7) at (1.25, -1) {$\iddots$};
		\node [style=none] (8) at (2.5, -1) {$\vdots$};
		\node [style=X] (9) at (2.5, -2) {};
		\node [style=none] (10) at (1, -2.75) {};
		\node [style=none] (11) at (1.5, -2.75) {};
		\node [style=none] (12) at (1.25, -0.25) {$\cdots$};
		\node [style=none] (17) at (2, -2.75) {};
		\node [style=none] (18) at (2, 1) {};
		\node [style=oplus] (19) at (2.5, -0.25) {};
		\node [style=dot] (20) at (2, -0.25) {};
		\node [style=oplus] (21) at (2, -2.25) {};
		\node [style=oplus] (22) at (2, 0.5) {};
	\end{pgfonlayer}
	\begin{pgfonlayer}{edgelayer}
		\draw (0) to (1);
		\draw (1) to (3);
		\draw (5.center) to (3);
		\draw (6.center) to (4);
		\draw (4) to (2);
		\draw (4) to (10.center);
		\draw (11.center) to (3);
		\draw (9) to (2);
		\draw (17.center) to (20);
		\draw (20) to (18.center);
		\draw (19) to (20);
	\end{pgfonlayer}
\end{tikzpicture}
=
\begin{tikzpicture}
	\begin{pgfonlayer}{nodelayer}
		\node [style=none] (24) at (4.5, 1.75) {};
		\node [style=none] (25) at (4, 1.75) {};
		\node [style=X] (28) at (5, -1) {};
		\node [style=none] (29) at (4, -3.25) {};
		\node [style=none] (30) at (4.5, -3.25) {};
		\node [style=none] (32) at (5, -3.25) {};
		\node [style=none] (33) at (5, 1.75) {};
		\node [style=oplus] (37) at (5, 1.25) {};
		\node [style=oplus] (38) at (5, -1.5) {};
		\node [style=dot] (39) at (4.5, -1.5) {};
		\node [style=oplus] (40) at (5, -2.25) {};
		\node [style=dot] (41) at (4, -2.25) {};
		\node [style=oplus] (42) at (5, 0) {};
		\node [style=dot] (43) at (4.5, 0) {};
		\node [style=oplus] (44) at (5, 0.75) {};
		\node [style=dot] (45) at (4, 0.75) {};
		\node [style=oplus] (46) at (5, -2.75) {};
		\node [style=X] (47) at (5, -0.5) {};
		\node [style=none] (48) at (4.25, -0.75) {$\cdots$};
		\node [style=none] (49) at (4.25, -1.75) {$\iddots$};
		\node [style=none] (50) at (4.25, 0.5) {$\ddots$};
		\node [style=none] (51) at (5, 0.5) {$\vdots$};
		\node [style=none] (52) at (5, -1.75) {$\vdots$};
	\end{pgfonlayer}
	\begin{pgfonlayer}{edgelayer}
		\draw (38) to (39);
		\draw (40) to (41);
		\draw (42) to (43);
		\draw (44) to (45);
		\draw (29.center) to (25.center);
		\draw (24.center) to (30.center);
		\draw (38) to (28);
		\draw (40) to (32.center);
		\draw (47) to (42);
		\draw (44) to (33.center);
	\end{pgfonlayer}
\end{tikzpicture}
=
\begin{tikzpicture}
	\begin{pgfonlayer}{nodelayer}
		\node [style=none] (24) at (4.5, 1.25) {};
		\node [style=none] (25) at (4, 1.25) {};
		\node [style=X] (28) at (5, -1) {$1$};
		\node [style=none] (29) at (4, -2.75) {};
		\node [style=none] (30) at (4.5, -2.75) {};
		\node [style=none] (32) at (5, -2.75) {};
		\node [style=none] (33) at (5, 1.25) {};
		\node [style=oplus] (38) at (5, -1.5) {};
		\node [style=dot] (39) at (4.5, -1.5) {};
		\node [style=oplus] (40) at (5, -2.25) {};
		\node [style=dot] (41) at (4, -2.25) {};
		\node [style=oplus] (42) at (5, 0) {};
		\node [style=dot] (43) at (4.5, 0) {};
		\node [style=oplus] (44) at (5, 0.75) {};
		\node [style=dot] (45) at (4, 0.75) {};
		\node [style=X] (47) at (5, -0.5) {$1$};
		\node [style=none] (48) at (4.25, -0.75) {$\cdots$};
		\node [style=none] (49) at (4.25, -1.75) {$\iddots$};
		\node [style=none] (50) at (4.25, 0.5) {$\ddots$};
		\node [style=none] (51) at (5, 0.5) {$\vdots$};
		\node [style=none] (52) at (5, -1.75) {$\vdots$};
	\end{pgfonlayer}
	\begin{pgfonlayer}{edgelayer}
		\draw (38) to (39);
		\draw (40) to (41);
		\draw (42) to (43);
		\draw (44) to (45);
		\draw (29.center) to (25.center);
		\draw (24.center) to (30.center);
		\draw (38) to (28);
		\draw (40) to (32.center);
		\draw (47) to (42);
		\draw (44) to (33.center);
	\end{pgfonlayer}
\end{tikzpicture}
$$
%Notice that the choice of which wires to straighten out the zig-zag is arbitrary.
\end{remark}







\begin{lemma}
\label{lem:parisoaffcb}
$\ParIso(\Aff\cb_2)$ is a presentation for the full subcategory $\Par\Iso(\Aff\Fin\Vect(\F_2))^*$ of $\Par\Iso(\Aff\Fin\Vect(\F_2))$ where the objects are nonempty affine vector spaces.
\end{lemma}


\begin{proof}
The obvious functor $\ParIso(\Aff\cb_2)\to \Par\Iso(\Aff\Fin\Vect(\F_2))^*$ is clearly full,  as well as an isomorphism on objects.
It remains to show it is faihful.  It is faithful on maps which are taken to spans with nonempty apex by the same argument as Lemma \ref{lem:parisocb}. For empty case, there is exactly one diagram of each type with a factor of $0$; and similarly, there is exactly one span with an empty apex.
\end{proof}

By \cite{cnot}  in this the identities of Definition \ref{def:isoaffcbzero}
 can be replaced by the following identity, while maintaining completeness:
\hspace*{.4cm}
$
\begin{tikzpicture}
	\begin{pgfonlayer}{nodelayer}
		\node [style=X] (0) at (0, 5) {$1$};
		\node [style=none] (1) at (0.5, 5.75) {};
		\node [style=none] (2) at (0.5, 4.25) {};
	\end{pgfonlayer}
	\begin{pgfonlayer}{edgelayer}
		\draw (2.center) to (1.center);
	\end{pgfonlayer}
\end{tikzpicture}
\eqzxa{zero.six}
\begin{tikzpicture}
	\begin{pgfonlayer}{nodelayer}
		\node [style=X] (0) at (0, 5) {$1$};
		\node [style=none] (1) at (0.5, 5.75) {};
		\node [style=none] (2) at (0.5, 4.25) {};
		\node [style=X] (3) at (0.5, 5.25) {$1$};
		\node [style=X] (4) at (0.5, 4.75) {$1$};
	\end{pgfonlayer}
	\begin{pgfonlayer}{edgelayer}
		\draw (3) to (1.center);
		\draw (4) to (2.center);
	\end{pgfonlayer}
\end{tikzpicture}
$

%Using the identities presented in \cite{cnot}, this has a more compact presentation given in Appendix \ref{subsubsec:presentations:three:pinj}.
%This alternative form is much more in the aesthetic vein of the ZX-calculus,

%\begin{corollary}
%The prop $\ParIso(\Aff\cb_2)$ has a finite presentation with axioms where the axioms are the union of the axioms for $\inj(\Aff\cb_2),\inj(\Aff\cb_2)^\op$, the law $\eta_X\eta_X^\op=1_I$ as well as the $1$-law, replacing the universally quantified law for the 0 scalar:
%$$
%\begin{tikzpicture}
%	\begin{pgfonlayer}{nodelayer}
%		\node [style=none] (0) at (-1, -7) {};
%		\node [style=X] (1) at (-0.5, -7.75) {$1$};
%		\node [style=none] (2) at (-1, -8.5) {};
%	\end{pgfonlayer}
%	\begin{pgfonlayer}{edgelayer}
%		\draw (0.center) to (2.center);
%	\end{pgfonlayer}
%\end{tikzpicture}
%=
%\begin{tikzpicture}
%	\begin{pgfonlayer}{nodelayer}
%		\node [style=none] (0) at (-1, -7) {};
%		\node [style=X] (1) at (-0.5, -7.75) {$1$};
%		\node [style=X] (2) at (-1, -7.5) {};
%		\node [style=X] (3) at (-1, -8) {};
%		\node [style=none] (4) at (-1, -8.5) {};
%	\end{pgfonlayer}
%	\begin{pgfonlayer}{edgelayer}
%		\draw (0.center) to (2);
%		\draw (3) to (4.center);
%	\end{pgfonlayer}
%\end{tikzpicture}
%$$
% 
%
%\end{corollary}


\begin{definition}

Let $\pr\Aff\cb_2$ denote the pushout of the diagram of props:
$$
\pr\iso\Aff\cb_2 \leftarrow \surj^\op \rightarrow \cm^\op
$$

\end{definition}





\begin{lemma}
\label{lem:paraffcb}

$\pr\Aff\cb_2$ is a presentation for the prop $(\Par(\Aff\Fin\Vect(\F_2))^*,+)$.
\end{lemma}

\begin{proof}
%\renewcommand{\cubetopbl}{$\inj(\Aff\cb_2)$}
%\renewcommand{\cubetopbr}{$\inj(\Aff\cb_2)+1\otimes_{\Iso(\Aff\cb_2)+1} \inj(\Aff\cb_2)+1^\op$}
%\renewcommand{\cubetopfl}{$\inj(\Aff\cb_2)\otimes_{\Iso(\Aff\cb_2)} \surj(\Aff\cb_2)^\op$}
%\renewcommand{\cubetopfr}{$\Par(\Aff\cb_2)$}
%\renewcommand{\cubebotbl}{$(\inj(\Aff\Mat(\F_2)),+)$ }
%\renewcommand{\cubebotbr}{$(\ParIso(\Aff\Vect(\F_2))^*,+)$ }
%\renewcommand{\cubebotfl}{$(\Par\surj(\Aff\Mat(\F_2)),+)^\op$ }
%\renewcommand{\cubebotfr}{}
%
%$$
%\xymatrixrowsep{3mm}\xymatrixcolsep{-10mm}
%\xymatrix{
%                                       & \mbox{\cubetopbl} \ar[rr] \ar[dl] \ar[dd]^(.7){\cong}      &                                                  & \mbox{\cubetopbr}  \ar[dd]^{\cong} \ar[dl] \\
%\mbox{\cubetopfl} \ar[rr]  \ar[dd]_{\cong}           &                                                                                              &\mbox{\cubetopfr} \ar@{-->}[dd]^(.35){\cong}   \skewpullbackcorner[ul]              \\
%                                       &  \mbox{\cubebotbl} \ar[dl] \ar[rr]                    &                                                  & \mbox{\cubebotbr} \ar@/^1pc/[ddl] \ar[dl] \\
%\mbox{\cubebotfl} \ar@/_1pc/[drr] \ar[rr]  &                                                                                             & \mbox{\cubebotfr} \skewpullbackcorner[ul]    \ar@{-->}[d]^{\cong}  \\
%                                                   &                                                                                             & (\Par(\Aff\Mat(\F_2)),+)
%}
%$$



\renewcommand{\cubetopbl}{$\surj^\op$}
\renewcommand{\cubetopbr}{$\cm^\op$}
\renewcommand{\cubetopfl}{$\pr\iso\Aff\cb_2$}
\renewcommand{\cubetopfr}{$\pr\Aff\cb_2$}
\renewcommand{\cubebotbl}{$\surj^\op$ }
\renewcommand{\cubebotbr}{$\cm^\op$ }
\renewcommand{\cubebotfl}{$(\ParIso(\Aff\Vect(\F_2))^*,+)$ }
\renewcommand{\cubebotfr}{}

$$
\xymatrixrowsep{2mm}\xymatrixcolsep{1mm}
\xymatrix{
                                       & \mbox{\cubetopbl} \ar[rr] \ar[dl] \ar@{=}[dd]     &                                                  & \mbox{\cubetopbr} \ar@{=}[dd] \ar[dl] \\
\mbox{\cubetopfl} \ar[rr]  \ar[dd]_{\cong}           &                                                                                              &\mbox{\cubetopfr} \ar@{-->}[dd]^(.35){\cong}   \skewpullbackcorner[ul]              \\
                                       &  \mbox{\cubebotbl} \ar[dl] \ar[rr]                    &                                                  & \mbox{\cubebotbr} \ar@/^1pc/[ddl] \ar[dl] \\
\mbox{\cubebotfl} \ar@/_1pc/[drr] \ar[rr]  &                                                                                             & \mbox{\cubebotfr} \skewpullbackcorner[ul]    \ar@{-->}[d]_F^{\cong}  \\
                                                   &                                                                                             & (\Par(\Aff\Vect(\F_2))^*,+) 
}
$$


We know that $\pr\iso\Aff\cb_2\cong \ParIso(\Aff\Vect(\F_2))^*,+)$ is a discrete inverse category by \cite[Prop. 3.4]{cnot}.

The cube commutes by the universal property of the pushout, as before.

We just have to show that the universal map $F$ is an isomorphism.  It is clearly the identity on objects, so we just have to show it is full and faithful.
This follows from essentially the same argument as in the linear case.


\end{proof}


%\begin{comment}
$\pr\Aff\cb_2$ has a particularly elegant presentation given in \S \ref{subsubsec:presentations:two:par}, which is much more in the spirit of the ZX-calculus.
%\end{comment}


\begin{definition}
Let $\sp\Aff\cb_2$ denote the pushout of the diagram of props:
$$
 \pr\Aff\cb_2^\op \leftarrow \pr\iso\Aff\cb_2 \rightarrow \pr\Aff\cb_2
$$

\end{definition}



\begin{lemma}
\label{lem:spanaffcb}
$\sp\Aff\cb_2$ is a presentation for the prop $(\Span(\Aff\Fin\Vect(\F_2))^*,+)$.
\end{lemma}

\begin{proof}
\renewcommand{\cubetopbl}{$\pr\iso\Aff\cb_2$}
\renewcommand{\cubetopbr}{$\pr\Aff\cb_2$}
\renewcommand{\cubetopfl}{$\pr\Aff\cb_2^\op$}
\renewcommand{\cubetopfr}{$\sp\Aff\cb_2$}
\renewcommand{\cubebotbl}{$(\ParIso(\Aff\Vect(\F_2))^*,+)$ }
\renewcommand{\cubebotbr}{$(\Par(\Aff\Vect(\F_2))^*,+)$ }
\renewcommand{\cubebotfl}{$(\Par(\Aff\Vect(\F_2))^*,+)^\op$ }
\renewcommand{\cubebotfr}{}

$$
\hspace*{-1cm}
\xymatrixrowsep{2mm}\xymatrixcolsep{.5mm}
\xymatrix{
                                       & \mbox{\cubetopbl} \ar[rr] \ar[dl] \ar[dd]^(.7){\cong}      &                                                  & \mbox{\cubetopbr}  \ar[dd]^{\cong} \ar[dl] \\
\mbox{\cubetopfl} \ar[rr]  \ar[dd]_{\cong}           &                                                                                              &\mbox{\cubetopfr} \ar@{-->}[dd]^(.35){\cong}   \skewpullbackcorner[ul]              \\
                                       &  \mbox{\cubebotbl} \ar[dl] \ar[rr]                    &                                                  & \mbox{\cubebotbr} \ar@/^1pc/[ddl] \ar[dl] \\
\mbox{\cubebotfl} \ar@/_1pc/[drr] \ar[rr]  &                                                                                             & \mbox{\cubebotfr} \skewpullbackcorner[ul]    \ar@{-->}[d]_F^{\cong}  \\
                                                   &                                                                                             & (\Span(\Aff\Vect(\F_2))^*,+)
}
$$

 The rear and left faces of the cube commute and the vertical maps are all isomorphisms. Therefore, the whole cube commutes by the universal property of the pushout, with the upper universal map being an isomorphism.

We seek to show that the lower universal map  $F$ is also an isomorphism.  It is clearly the identity on objects, so we just have to show fullness and faithfulness.

For fullness, let us first consider the nonempty case; that is a map $\F_2^n \xleftarrow{(A,x)} \F_2^k \xrightarrow{(B,y)}\F^m$ in $(\Span(\Aff\Vect(\F_2))^*,+)$.  This is in the image of the following diagram under $F$:
$$
(\F_2^n \xleftarrow{(A,x)} \F_2^k  = \F_2^k); (\F_2^k = \F_2^k  \xrightarrow{(B,y)}\F^m)
$$ 
Otherwise, consider a map of the form  $\F_2^n \xleftarrow{?} \emptyset  \xrightarrow{?}\F^m$.  This the image of the following diagram:
$$
(\F_2^n \xleftarrow{?} \emptyset \xrightarrow {?} \F_2^0  );(\F_2^0 \xleftarrow{?} \emptyset  \xrightarrow{?}\F^m)
$$
For faithfulness, again, we separate the proof into two cases.  The functor is faithful on diagrams in $(\Span(\Aff\Vect(\F_2))^*,+)$ with nonempty apex by the same argument as in Lemma \ref{lem:spancb}.
%$$
%\xymatrix{
%          & \F_2^k \ar[dl]_{(A',x')} \ar[dd]_{\cong}^{(C,z)} \ar[dr]^{(B',y')}\\
%\F_2^n  &                                                                                                    & \F_2^m\\
%         & \F_2^k \ar[ul]^{(A,x)} \ar[ur]_{(B,y)}\\
%}
%$$
%We have the following equation in $\Span(\Aff\cb_2)$:
%{
%\xymatrixrowsep{1mm}\xymatrixcolsep{3.5mm}
%\begin{align*}
%\xymatrix{
%          & \F_2^k \ar[dl]_{(A,x)}  \ar@{=}[dr]\\
%\F_2^n  &                                                                                                    & \F_2^k\\
%};
%\xymatrix{
%          & \F_2^k \ar[dr]^{(B,y)}  \ar@{=}[dl]\\
%\F_2^k  &                                                                                                    & \F_2^m\\
%} &=
%\xymatrix{
%          & \F_2^k \ar[dl]_{(A,x)}  \ar@{=}[dr]\\
%\F_2^n  &                                                                                                    & \F_2^k\\
%};
%\xymatrix{
%          & \F_2^k \ar@{=}[dr] \ar@{=}[dl] \\
%\F_2^k  &                                                                                                    & \F_2^k\\
%         & \F_2^k \ar[ul]^{(C,z)} \ar[ur] _{(C,z)} \ar[uu]^{\cong}_{(C,z)}
%};
%\xymatrix{
%          & \F_2^k \ar[dr]^{(B,y)}  \ar@{=}[dl]\\
%\F_2^k  &                                                                                                    & \F_2^m\\
%}\\
%&=
%\xymatrix{
%          & \F_2^k \ar[dl]_{(A,x)}  \ar@{=}[dr]\\
%\F_2^n  &                                                                                                    & \F_2^k\\
%};
%\xymatrix{
%        & \F_2^k \ar[dl]_{(C,z)} \ar@{=}[dr]\\
%\F_2^k  &                                                     & \F_2^k
%};
%\xymatrix{
%        & \F_2^k \ar[dr]^{(C,z)} \ar@{=}[dl]\\
%\F_2^k  &                                                     & \F_2^k
%};
%\xymatrix{
%          & \F_2^k \ar[dr]^{(B,y)}  \ar@{=}[dl]\\
%\F_2^k  &                                                                                                    & \F_2^m\\
%}\\
%&=
%\xymatrix{
%            &                                                        & \F_2^k \ar[dl]_{(C,z)} \ar@{=}[dr] \ar@/_2.0pc/[ddll]_{(A',x')}\\
%            & \F_2^k \ar@{=}[dr] \ar[dl]^{(A,x)}&                                                          & \F_2^k \ar@{=}[dr] \ar[dl]_{(C,z)}\\
%\F_2^k &                                                         & \F_2^k                                             &                                                         &\F_2^k
%};
%\xymatrix{
%            &                                                        & \F_2^k \ar[dr]^{(C,z)} \ar@{=}[dl] \ar@/^2.0pc/[ddrr]^{(B',y')}  \\
%            & \F_2^k \ar[dr]^{(C,z)}   \ar@{=}[dl] &                                                          & \F_2^k \ar@{=}[dl] \ar[dr]_{(B,y)}\\
%\F_2^k &                                                         & \F_2^k                                             &                                                         &\F_2^k
%}
%\end{align*}
%}
The case for spans with empty apex follows immediately as the only endomorphism on the empty set is the identity; thus,  isomorphic spans must be equal on the nose.

\end{proof}

%\begin{comment}
There is a particularly elegant equivalent presentation given in \S \ref{subsubsec:presentations:two:span}.
%\end{comment}
This is almost equivalent to the presentation given in \cite{affine} which gives a presentation for the full subcategory of relations of finite dimensional affine vector spaces where the objects are given by the nonempty vector spaces, and is much more in the spirit of the ZX-calculus.


\section{The and gate}
\label{sec:three}

Recall that unlike when the tensor product is the coproduct; when the tensor product is induced by the multiplication, to obtain a prop, one must consider the subcategory generated by tensoring a fixed object with itsef.
%
%Because $\sum_n 1=n$ grows linearly in $n$ and $\prod_n k = k^n$ grows exponentially, giving presentations for multiplicative models will be much more involved because there are more points to deal with, and in particular, more subobjects arise by pullback.


\begin{definition}
Let $L_{\F_2^\times}$ be the prop generated by quotienting $\cb$ by the equation:

$$
\begin{tikzpicture}
	\begin{pgfonlayer}{nodelayer}
		\node [style=none] (0) at (-7, 1) {};
		\node [style=none] (1) at (-7, 0.5) {};
		\node [style=Z] (2) at (-7, -0.25) {};
		\node [style=none] (3) at (-7, -0.75) {};
		\node [style=andin] (4) at (-7, 0.5) {};
	\end{pgfonlayer}
	\begin{pgfonlayer}{edgelayer}
		\draw (3.center) to (2.center);
		\draw [in=-60, out=60, looseness=1.25] (2.center) to (1);
		\draw [in=120, out=-120, looseness=1.25] (1) to (2.center);
		\draw (1) to (0.center);
	\end{pgfonlayer}
\end{tikzpicture}
\eqzxa{antispecial}
\begin{tikzpicture}
	\begin{pgfonlayer}{nodelayer}
		\node [style=none] (0) at (-7, 1) {};
		\node [style=none] (1) at (-7, -0.75) {};
	\end{pgfonlayer}
	\begin{pgfonlayer}{edgelayer}
		\draw (1.center) to (0.center);
	\end{pgfonlayer}
\end{tikzpicture}
$$

Where the components of the  monoid are relabled as follows:
\hspace*{.5cm}
$
\left(
\begin{tikzpicture}
	\begin{pgfonlayer}{nodelayer}
		\node [style=none] (0) at (-3.75, 0.5) {};
		\node [style=none] (1) at (-3.75, -0.25) {};
		\node [style=andin] (2) at (-3.75, -0.25) {};
		\node [style=none] (3) at (-4, -1) {};
		\node [style=none] (4) at (-3.5, -1) {};
	\end{pgfonlayer}
	\begin{pgfonlayer}{edgelayer}
		\draw (0.center) to (1.center);
		\draw [in=-60, out=90, looseness=1.00] (4.center) to (1.center);
		\draw [in=90, out=-120, looseness=1.00] (1.center) to (3.center);
	\end{pgfonlayer}
\end{tikzpicture},
\begin{tikzpicture}
	\begin{pgfonlayer}{nodelayer}
		\node [style=none] (0) at (-3.75, -0.25) {};
		\node [style=X] (1) at (-3.75, -1) {$1$};
	\end{pgfonlayer}
	\begin{pgfonlayer}{edgelayer}
		\draw (0.center) to (1);
	\end{pgfonlayer}
\end{tikzpicture}
\right)
$


\end{definition}


\begin{lemma}
$L_{\F_2}^\times$ is a presentation for the Lawvere theory for the group of units of the field $\F_2$.
\end{lemma}

\begin{definition}
Consider the prop $\f_2$, generated by the distributive law:
$$
L_{\F_2^\times} \otimes_{\cm^\op} \cb_2;
\begin{tikzpicture}
	\begin{pgfonlayer}{nodelayer}
		\node [style=andin] (4) at (1.25, 0.5) {};
		\node [style=X] (5) at (0.75, -0.5) {};
		\node [style=none] (6) at (0.5, -1) {};
		\node [style=none] (7) at (1, -1) {};
		\node [style=none] (8) at (1.75, -1) {};
		\node [style=none] (9) at (1.25, 0.5) {};
		\node [style=none] (10) at (1.25, 1.5) {};
	\end{pgfonlayer}
	\begin{pgfonlayer}{edgelayer}
		\draw [in=-30, out=90] (8.center) to (9.center);
		\draw [in=90, out=-150] (9.center) to (5);
		\draw [in=90, out=-45] (5) to (7.center);
		\draw [in=-135, out=90] (6.center) to (5);
		\draw (9.center) to (10.center);
	\end{pgfonlayer}
\end{tikzpicture}
\eqzxa{ring.mul}
\begin{tikzpicture}
	\begin{pgfonlayer}{nodelayer}
		\node [style=none] (0) at (1, 0) {};
		\node [style=none] (1) at (0.5, -1.25) {};
		\node [style=none] (2) at (1.75, -0.75) {};
		\node [style=none] (3) at (1.33, 0.75) {};
		\node [style=andin] (4) at (1, 0) {};
		\node [style=none] (5) at (1.75, 0) {};
		\node [style=none] (6) at (1, -1.25) {};
		\node [style=none] (7) at (1.75, -0.75) {};
		\node [style=none] (8) at (1.33, 0.75) {};
		\node [style=andin] (9) at (1.75, 0) {};
		\node [style=X] (10) at (1.33, 0.75) {};
		\node [style=none] (11) at (1.33, 1.25) {};
		\node [style=none] (12) at (1.75, -1.25) {};
		\node [style=Z] (13) at (1.75, -0.75) {};
	\end{pgfonlayer}
	\begin{pgfonlayer}{edgelayer}
		\draw [in=-135, out=90] (0.center) to (3.center);
		\draw [in=165, out=-30, looseness=1.25] (0.center) to (2.center);
		\draw [in=-45, out=90] (5.center) to (8.center);
		\draw [in=45, out=-45, looseness=1.25] (5.center) to (7.center);
		\draw (10) to (11.center);
		\draw [in=90, out=-150] (4) to (1.center);
		\draw [in=-150, out=90] (6.center) to (9);
		\draw (12.center) to (13);
	\end{pgfonlayer}
\end{tikzpicture},
\hspace*{.5cm}
\begin{tikzpicture}
	\begin{pgfonlayer}{nodelayer}
		\node [style=none] (0) at (2, 0) {};
		\node [style=none] (1) at (1.75, -0.75) {};
		\node [style=none] (2) at (2.25, -0.75) {};
		\node [style=none] (3) at (2, 0.5) {};
		\node [style=none] (4) at (2.25, -1) {};
		\node [style=X] (5) at (1.75, -0.75) {};
		\node [style=andin] (6) at (2, 0) {};
	\end{pgfonlayer}
	\begin{pgfonlayer}{edgelayer}
		\draw (0.center) to (3.center);
		\draw [in=90, out=-45] (0.center) to (2.center);
		\draw (4.center) to (2.center);
		\draw [in=-135, out=90] (1.center) to (0.center);
	\end{pgfonlayer}
\end{tikzpicture}
\eqzxa{ring.unit}
\begin{tikzpicture}
	\begin{pgfonlayer}{nodelayer}
		\node [style=none] (12) at (2, 0.5) {};
		\node [style=none] (14) at (2, -1) {};
		\node [style=X] (15) at (2, 0) {};
		\node [style=Z] (16) at (2, -0.5) {};
	\end{pgfonlayer}
	\begin{pgfonlayer}{edgelayer}
		\draw (15) to (12.center);
		\draw (16) to (14.center);
	\end{pgfonlayer}
\end{tikzpicture}
$$

\end{definition}


\begin{lemma} \cite[Thm. 10]{lafont}
$\f_2$ is a presentation for the prop $(\FSets_2,\times)$.
\end{lemma}


Therefore in some sense, we are justified in thinking of this prop $(\FSets_2,\times)$ as a sort of categorification of boolean polynomials.

%\begin{proof}
%The generators have the following interpretations in $\Sets_2$;  $X$ corresponds to addition:
%
%$$
%|a,b\rangle \xmapsto{\llbracket \mu_X \rrbracket} | a+b\rangle
%\hspace*{.5cm}
%* \xmapsto{\llbracket \mu_X \rrbracket} | 0 \rangle
%$$
%
%$\&$ corresponds to multiplication:
%
%$$
%|a,b\rangle \xmapsto{\llbracket \mu_\& \rrbracket} | a\cdot b\rangle
%\hspace*{.5cm}
%* \xmapsto{\llbracket\mu_X \rrbracket} | 1 \rangle
%$$
%
%
%The two bicommutative bialgebra laws correspond to the commutation of multiplication and addition with copying; and the monad map corresponds to the fact that multiplication distributes over addition.  Therefore, this is just the presentation of the ring $\Z_2$ as a Lawvere theory. 
%\end{proof}


To find larger fragments, it will be useful to first identify the isomorphisms and the monics of $\f_2$.


\begin{definition}
Given a map $f$ in  $\f_2$, the {\bf oracle} for $f$, ${\mathcal O}_f$ is defined as follows:
$$
\begin{tikzpicture}
	\begin{pgfonlayer}{nodelayer}
		\node [style=Z] (0) at (0.75, 0.25) {};
		\node [style=X] (1) at (1.5, 2.25) {};
		\node [style=map] (2) at (1, 1.5) {$f$};
		\node [style=none] (3) at (0.5, 2.75) {};
		\node [style=none] (4) at (1.5, 2.75) {};
		\node [style=none] (5) at (1.5, -0.25) {};
		\node [style=none] (6) at (0.75, -0.25) {};
		\node [style=Z] (7) at (-0.25, 0.25) {};
		\node [style=none] (8) at (-0.5, 2.75) {};
		\node [style=none] (9) at (-0.25, -0.25) {};
		\node [style=none] (10) at (0.25, 0) {$\cdots$};
		\node [style=none] (11) at (0, 2.5) {$\cdots$};
	\end{pgfonlayer}
	\begin{pgfonlayer}{edgelayer}
		\draw (6.center) to (0);
		\draw [in=-60, out=60] (0) to (2);
		\draw [in=-120, out=90] (2) to (1);
		\draw (1) to (4.center);
		\draw [in=90, out=-60] (1) to (5.center);
		\draw [in=-90, out=120] (0) to (3.center);
		\draw (9.center) to (7);
		\draw [in=-90, out=120] (7) to (8.center);
		\draw [in=45, out=-120] (2) to (7);
	\end{pgfonlayer}
\end{tikzpicture}
$$

\end{definition}

\begin{lemma}
The oracles in $f_2$ are generated by the generalized controlled-not gates:
$$
\begin{tikzpicture}
	\begin{pgfonlayer}{nodelayer}
		\node [style=none] (0) at (1, -0.75) {};
		\node [style=X] (1) at (0.5, -0.75) {$1$};
		\node [style=none] (2) at (0.75, 0.75) {};
		\node [style=X] (3) at (0.75, 0) {};
		\node [style=none] (4) at (1, -1.25) {};
	\end{pgfonlayer}
	\begin{pgfonlayer}{edgelayer}
		\draw [in=-45, out=90, looseness=0.75] (0.center) to (3);
		\draw [in=90, out=-135, looseness=0.75] (3) to (1);
		\draw (3) to (2.center);
		\draw (4.center) to (0.center);
	\end{pgfonlayer}
\end{tikzpicture},
\hspace*{.5cm}
\begin{tikzpicture}[xscale=-1]
	\begin{pgfonlayer}{nodelayer}
		\node [style=X] (0) at (0.75, 0) {};
		\node [style=Z] (1) at (1.25, -0.5) {};
		\node [style=none] (2) at (0.5, -1) {};
		\node [style=none] (3) at (1.25, -1) {};
		\node [style=none] (4) at (1.5, 0.5) {};
		\node [style=none] (5) at (0.75, 0.5) {};
	\end{pgfonlayer}
	\begin{pgfonlayer}{edgelayer}
		\draw (5.center) to (0);
		\draw [in=150, out=-30] (0) to (1);
		\draw [in=-90, out=60, looseness=0.75] (1) to (4.center);
		\draw (1) to (3.center);
		\draw [in=90, out=-120, looseness=0.75] (0) to (2.center);
	\end{pgfonlayer}
\end{tikzpicture},
\hspace*{.5cm}
\begin{tikzpicture}
	\begin{pgfonlayer}{nodelayer}
		\node [style=Z] (0) at (-10.25, 0.25) {};
		\node [style=Z] (1) at (-11.25, 0.25) {};
		\node [style=none] (2) at (-10.75, 1) {};
		\node [style=X] (3) at (-9.75, 1.75) {};
		\node [style=none] (4) at (-11.25, -0.5) {};
		\node [style=none] (5) at (-10.25, -0.5) {};
		\node [style=none] (6) at (-9.75, -0.5) {};
		\node [style=none] (7) at (-9.75, 2.25) {};
		\node [style=none] (8) at (-10.25, 2.25) {};
		\node [style=none] (9) at (-11.25, 2.25) {};
		\node [style=andin] (10) at (-10.75, 1) {};
		\node [style=none] (11) at (-10.75, 2.25) {$n$};
		\node [style=none] (12) at (-10.75, 0.25) {$n$};
		\node [style=none] (13) at (-10.75, 2) {$\cdots$};
		\node [style=none] (14) at (-10.75, 0.5) {$\cdots$};
	\end{pgfonlayer}
	\begin{pgfonlayer}{edgelayer}
		\draw (4.center) to (1);
		\draw (1) to (2.center);
		\draw (2.center) to (0);
		\draw (0) to (5.center);
		\draw (6.center) to (3);
		\draw [in=90, out=-146, looseness=1.50] (3) to (2.center);
		\draw [in=-90, out=120, looseness=1.00] (1) to (9.center);
		\draw [in=-90, out=60, looseness=0.75] (0) to (8.center);
		\draw (3) to (7.center);
	\end{pgfonlayer}
\end{tikzpicture}
$$




%and the equations of Iwama et al, where $[n,X]$ denotes an $|X|$-controlled not gate controlled by the wires indexed by the set X, and targetting the wire $n \notin X$  \cite{iwama} generalized b

%https://web.eecs.umich.edu/~imarkov/pubs/jour/tcad03-iwls.pdf

%\begin{description}
%\item $[x,X][x,X]=1$
%\item  When the target wire are the same $[x,X][x,Y] = [x,Y] [x,X]$
%\item  When $x \not\in Y$ and $y \not\in X$ then $[x,X][y,Y] = [y,Y] [x,X]$
%\item  $[x,X] [y,{\{ x\} \sqcup Y}] = [y,{X \cup Y}][y,{\{ x\} \sqcup Y}] [x,X]$
%\item For $x \neq y$, $[x,] $
%\end{description}
%
%
%TODO


\end{lemma}


\begin{proof}
Any  Boolean function of $n$ arguments can be represented by a polynomial in\\
 $\F_2[x_1,\ldots, x_n]/\langle x_1^2-x_1,\ldots x_1^2-x_1\rangle$.  Every polynomial in this quotient ring has a unique normal form given by sums of products (which is not true for arbitrary finite fields).  Each product corresponds to a generalized controlled-not gate, and the sum corresponds to composing these generalized controlled-not gates in sequence.
\end{proof}

In the quantum circuit notation the generalized controlled-not gates are drawn as follows (the first being the not gate, and the second being the controlled-not gate):
$$
\begin{tikzpicture}
	\begin{pgfonlayer}{nodelayer}
		\node [style=oplus] (0) at (0, 1.5) {};
		\node [style=none] (1) at (0, 2) {};
		\node [style=none] (2) at (0, 1) {};
	\end{pgfonlayer}
	\begin{pgfonlayer}{edgelayer}
		\draw (0) to (1.center);
		\draw (0) to (2.center);
	\end{pgfonlayer}
\end{tikzpicture}
,
\hspace*{.5cm}
\begin{tikzpicture}
	\begin{pgfonlayer}{nodelayer}
		\node [style=oplus] (0) at (0, 1.5) {};
		\node [style=none] (1) at (0, 2) {};
		\node [style=none] (2) at (0, 1) {};
		\node [style=none] (4) at (-0.5, 2) {};
		\node [style=none] (5) at (-0.5, 1) {};
		\node [style=dot] (6) at (-0.5, 1.5) {};
	\end{pgfonlayer}
	\begin{pgfonlayer}{edgelayer}
		\draw (0) to (1.center);
		\draw (0) to (2.center);
		\draw (0) to (6);
		\draw (6) to (4.center);
		\draw (6) to (5.center);
	\end{pgfonlayer}
\end{tikzpicture}
,
\hspace*{.5cm}
\begin{tikzpicture}
	\begin{pgfonlayer}{nodelayer}
		\node [style=oplus] (0) at (-0.25, 1.5) {};
		\node [style=none] (1) at (-0.25, 2) {};
		\node [style=none] (2) at (-0.25, 1) {};
		\node [style=none] (4) at (-0.75, 2) {};
		\node [style=none] (5) at (-0.75, 1) {};
		\node [style=dot] (6) at (-0.75, 1.5) {};
		\node [style=none] (7) at (-1.75, 2) {};
		\node [style=none] (8) at (-1.75, 1) {};
		\node [style=dot] (9) at (-1.75, 1.5) {};
		\node [style=none] (10) at (-1, 1.5) {};
		\node [style=none] (11) at (-1.5, 1.5) {};
		\node [style=none] (12) at (-1.25, 1.5) {$\cdots$};
		\node [style=none] (13) at (-1.25, 1.75) {$n$};
	\end{pgfonlayer}
	\begin{pgfonlayer}{edgelayer}
		\draw (0) to (1.center);
		\draw (0) to (2.center);
		\draw (0) to (6);
		\draw (6) to (4.center);
		\draw (6) to (5.center);
		\draw (9) to (7.center);
		\draw (9) to (8.center);
		\draw (11.center) to (9);
		\draw (10.center) to (6);
	\end{pgfonlayer}
\end{tikzpicture}
$$


%We will also allow generalized controlled not gates controlled and targetting arbitrary wires, possibly with gaps in the middle.




\begin{lemma}\cite[Thm. 5.1]{toffolireversible}
The prop generated by the oracles in $\f_2$ generate $\Iso(\f_2)$.
\end{lemma}

%This actually follows from https://arxiv.org/pdf/quant-ph/0207001.pdf
%Scratch space is used to construct n-bit cnot gate
%http://theory.caltech.edu/~preskill/ph229/notes/chap6.pdf

Denote a generalized controlled not gate controlled by wires indexed by $X$, operating on $x$ by $\lbparen X,x\rbparen$


%
%
%Iwama et al,  give a set of identities which are complete for oracles, not general isomorphisms \cite{iwama}.
%Shende restated these identities using the commutator \cite{shende}.






Iwama et al \cite{iwama} originally gave a complete set of identities for circuits generated by generalized controlled not gates where the value of all-but-one output wires are fixed.  It is worth mentioning that Shende et al. later used the commutator to generalize some of these identities \cite[Cor. 26]{shende}.  We conjecture that a very similar set of identities is complete for Boolean isomorphisms: 
%Recall that the {\bf commutator} of two group elements $f,g$ is the element $[f,g]:=fgf^{-1}g^{-1}$; therefore $fg=[f,g]gf$.


%Shende gives a way to compute commutators of controlled isomorphsisms in $\Iso(\FSets_2)$.  In particular, take isomorphisms $f,g$  which only change bits in sets indexed by $X,Y$, respectively.  Then if we control these gates by $Z$ and $W$ respectively, denoted by $f^Z,g^W$, we have :
%$$
%[f,g] = [f^{Y * Z}, g^{X*W}]^{\bar{(X\#Y)} * (W\#Z) }
%$$
%Where $X\# Y $ is the defined as the pointwise xor, and $X * Y$ is defined as the pointwise and.

\begin{conjecture}
%Denote a generalized controlled-not gate controlled from wires indexed by $X$ and operating on the wire $x$ by $\lbparen X,x \rbparen$.

Let  $\Iso(\FSets_2)$ denote the prop generated by all generalized controlled-not gates modulo the following identities:

\begin{itemize}
\item $\lbparen X,x\rbparen \lbparen X,x \rbparen= 1$ 
%So the first identity is that  in this case $\lbparen X,x\rbparen\lbparen Y,y\rbparen=\lbparen Y,y\rbparen\lbparen X,x\rbparen$


\item
If  $x \notin Y $ and $ y \notin X$ then $\lbparen X,x\rbparen \lbparen Y,y\rbparen =\lbparen Y,y\rbparen \lbparen X,x\rbparen $.


%
%So the second identity is that in this case:
%$$
%\lbparen  X,x\rbparen  \lbparen  Y ,y\rbparen = \lbparen \lbparen  X \cup Y -y,x\rbparen \rbparen  \lbparen  Y ,y\rbparen  \lbparen  X,x\rbparen  
%$$

\item
If $x \notin Y$, then $\lbparen X,x\rbparen \lbparen \{x\} \sqcup Y, y\rbparen = \lbparen X\cup Y,y\rbparen  \lbparen \{x\} \sqcup Y, y\rbparen  \lbparen X,x\rbparen $.

\item
If $x \notin Y$, then $ \lbparen \{x\} \sqcup Y, y\rbparen \lbparen X,x\rbparen = \lbparen X,x\rbparen   \lbparen \{x\} \sqcup Y, y\rbparen  \lbparen X\cup Y,y\rbparen $.


%So the third identity is that $\lbparen  X,x\rbparen ^2= 1 $.

\item
If $x \in Y$ and $y \in X$, then
$
\lbparen  X,x \rbparen  \lbparen  Y,y \rbparen   \lbparen  X,x \rbparen 
=
\lbparen  Y,y \rbparen  \lbparen  X,x \rbparen   \lbparen  Y,y \rbparen 
$.



\end{itemize}

\end{conjecture}



Note that the symmetry is derived in this fragment by composing 3 controlled not gates, as in Definition \label{def:isoaff}.  The axioms of a prop are derived, so we are justified in calling $\Iso(\f_2)$ a prop.


Although we aren't sure if these identities are complete, it doesn't matter in the end.  With each generator we add, we add new enough identities to give a complete presentation, given that there is a complete presentation for $\Iso(\f_2)$.  However, eventually once we add enough generators and identities, we get a finite, complete presentation.

\begin{definition}
Let $\inj(\f_2)$ be the prop given by adjoining the black unit to $\Iso(\f_2)$ modulo:

$$
\begin{tikzpicture}
	\begin{pgfonlayer}{nodelayer}
		\node [style=oplus] (0) at (2, 1.5) {};
		\node [style=none] (1) at (2, 2.25) {};
		\node [style=none] (2) at (2, 0.75) {};
		\node [style=none] (3) at (1.5, 2.25) {};
		\node [style=none] (4) at (1.5, 0.75) {};
		\node [style=dot] (5) at (1.5, 1.5) {};
		\node [style=none] (6) at (0.5, 2.25) {};
		\node [style=dot] (7) at (0.5, 1.5) {};
		\node [style=none] (8) at (1.25, 1.5) {};
		\node [style=none] (9) at (0.75, 1.5) {};
		\node [style=none] (10) at (1, 1.5) {$\cdots$};
		\node [style=none] (11) at (1, 1.75) {$n$};
		\node [style=none] (13) at (0, 2.25) {};
		\node [style=dot] (14) at (0, 1.5) {};
		\node [style=X] (15) at (0, 1) {};
		\node [style=none] (16) at (0.5, 0.75) {};
	\end{pgfonlayer}
	\begin{pgfonlayer}{edgelayer}
		\draw (0) to (1.center);
		\draw (0) to (2.center);
		\draw (0) to (5);
		\draw (5) to (3.center);
		\draw (5) to (4.center);
		\draw (7) to (6.center);
		\draw (9.center) to (7);
		\draw (8.center) to (5);
		\draw (14) to (13.center);
		\draw (15) to (14);
		\draw (16.center) to (7);
		\draw (7) to (14);
	\end{pgfonlayer}
\end{tikzpicture}
\eqzxa{mono.ftwo}
\begin{tikzpicture}
	\begin{pgfonlayer}{nodelayer}
		\node [style=none] (1) at (2, 2.25) {};
		\node [style=none] (2) at (2, 0.75) {};
		\node [style=none] (3) at (1.5, 2.25) {};
		\node [style=none] (4) at (1.5, 0.75) {};
		\node [style=none] (6) at (0.5, 2.25) {};
		\node [style=none] (10) at (1, 1.5) {$\cdots$};
		\node [style=none] (11) at (1, 1.75) {$n$};
		\node [style=none] (13) at (0, 2.25) {};
		\node [style=X] (15) at (0, 1) {};
		\node [style=none] (16) at (0.5, 0.75) {};
	\end{pgfonlayer}
	\begin{pgfonlayer}{edgelayer}
		\draw (16.center) to (6.center);
		\draw (13.center) to (15);
		\draw (4.center) to (3.center);
		\draw (1.center) to (2.center);
	\end{pgfonlayer}
\end{tikzpicture}
$$
\end{definition}

\begin{lemma}
\label{lem:injand}
$\inj(\f_2)$ is a presentation for the prop $(\inj(\FSets_2),\times)$.
\end{lemma}

%Note, however, that the monoidal theory for $\inj(\f_2)$ is dependant on that of $\iso(\f_2)$ the identities of which are conjectured, even though we only need one extra identity to get injections.

The pushout of a diagram of sets and functions $2^n \xleftarrowtail{} 2^k \xrightarrowtail{} 2^m$ is not always a power of 2.  Therefore, one should not expect to construct categories of partial isomorphisms via a distributive law of on $\inj(\f_2)\otimes_{\Iso(\f_2)} \inj(\f_2)^\op$. Instead one must add all of the nontrivial subobjects to the constituent props forming the distributive law; as opposed to the affine case, there are more than one such subobjects which arise in this way.

\begin{definition}
Consider the pro $\sub_2$ generated by endomorphisms such that for any $n$, $\sub_2(n,n)$ is the set described by all $n$-variable polynomials over $\F_2$.  Denote such a generator by a box with $n$ inputs and $n$ outputs labelled by the corresponding polynomial.

We require that the following equations hold so that
$$\forall n,m \in \N, p,r \in \F_2[x_1,\ldots, x_n],  q \in \F_2[x_{n+1},\ldots, x_{n+m}]:\hspace*{1cm}
$$
$$
\begin{tikzpicture}
	\begin{pgfonlayer}{nodelayer}
		\node [style=none] (0) at (3, 2.25) {};
		\node [style=none] (1) at (3, 3.25) {};
		\node [style=map] (2) at (3, 2.75) {$1$};
		\node [style=none] (6) at (3, 3.5) {$n$};
		\node [style=none] (8) at (3, 2) {$n$};
	\end{pgfonlayer}
	\begin{pgfonlayer}{edgelayer}
		\draw (0.center) to (1.center);
	\end{pgfonlayer}
\end{tikzpicture}
\eqzxa{sub.one}
\begin{tikzpicture}
	\begin{pgfonlayer}{nodelayer}
		\node [style=none] (0) at (3, 2.25) {};
		\node [style=none] (1) at (3, 3.25) {};
		\node [style=none] (6) at (3, 3.5) {$n$};
		\node [style=none] (8) at (3, 2) {$n$};
	\end{pgfonlayer}
	\begin{pgfonlayer}{edgelayer}
		\draw (0.center) to (1.center);
	\end{pgfonlayer}
\end{tikzpicture}
\hspace*{,5cm}
\begin{tikzpicture}
	\begin{pgfonlayer}{nodelayer}
		\node [style=none] (0) at (5, 2) {};
		\node [style=none] (1) at (5, 4) {};
		\node [style=map] (2) at (5, 2.5) {$r$};
		\node [style=map] (3) at (5, 3.5) {$p$};
		\node [style=none] (4) at (5, 4.25) {$n$};
		\node [style=none] (5) at (5, 1.75) {$n$};
	\end{pgfonlayer}
	\begin{pgfonlayer}{edgelayer}
		\draw (0.center) to (1.center);
	\end{pgfonlayer}
\end{tikzpicture}
\eqzxa{sub.two}
\begin{tikzpicture}
	\begin{pgfonlayer}{nodelayer}
		\node [style=none] (0) at (5, 2.25) {};
		\node [style=none] (1) at (5, 4.25) {};
		\node [style=map] (2) at (5, 3.25) {$p+r+pr$};
		\node [style=none] (4) at (5, 4.5) {$n$};
		\node [style=none] (5) at (5, 2) {$n$};
	\end{pgfonlayer}
	\begin{pgfonlayer}{edgelayer}
		\draw (0.center) to (1.center);
	\end{pgfonlayer}
\end{tikzpicture}
\hspace*{.5cm}
\begin{tikzpicture}
	\begin{pgfonlayer}{nodelayer}
		\node [style=none] (0) at (2.4, 2.25) {};
		\node [style=none] (1) at (2.4, 3.25) {};
		\node [style=map] (2) at (2.4, 2.75) {$p$};
		\node [style=none] (3) at (3, 3.25) {};
		\node [style=none] (4) at (3, 2.25) {};
		\node [style=map] (5) at (3, 2.75) {$q$};
		\node [style=none] (6) at (2.4, 3.5) {$n$};
		\node [style=none] (7) at (3, 3.5) {$m$};
		\node [style=none] (8) at (2.4, 2) {$n$};
		\node [style=none] (9) at (3, 2) {$m$};
	\end{pgfonlayer}
	\begin{pgfonlayer}{edgelayer}
		\draw (0.center) to (1.center);
		\draw (3.center) to (4.center);
	\end{pgfonlayer}
\end{tikzpicture}
\eqzxa{sub.three}
\begin{tikzpicture}
	\begin{pgfonlayer}{nodelayer}
		\node [style=none] (0) at (2.5, 2.25) {};
		\node [style=none] (1) at (2.5, 3.25) {};
		\node [style=map] (2) at (2.75, 2.75) {$p\cdot q$};
		\node [style=none] (3) at (3, 3.25) {};
		\node [style=none] (4) at (3, 2.25) {};
		\node [style=none] (6) at (2.5, 3.5) {$n$};
		\node [style=none] (7) at (3, 3.5) {$m$};
		\node [style=none] (8) at (2.5, 2) {$n$};
		\node [style=none] (9) at (3, 2) {$m$};
	\end{pgfonlayer}
	\begin{pgfonlayer}{edgelayer}
		\draw (0.center) to (1.center);
		\draw (3.center) to (4.center);
	\end{pgfonlayer}
\end{tikzpicture}
$$
As well as, for all $n$, the equations of the quotient rings  $\F_2[x_1,\ldots, x_n]/\langle x_1^2-x_1,\ldots, x_n^2-x_n \rangle$.

\end{definition}


\begin{lemma}
\label{lem:sub}
$\sub_2$ is a presentation for the pro of symmetric spans of monic functions, ie spans of the following form $2^n \xleftarrow{e} k \xrightarrow{e}2^n$, for all $n,k \in \N$ and monics $e$.
\end{lemma}



\begin{proof}
Each polynomial  $p \in \F_2[x_1,\ldots, x_n]/\langle x_1^2-x_1,\ldots, x_n^2-x_n \rangle$ corresponds to a function $\ev_p:\Z_2^n \to \Z_2$ given by evaluation.  Let $k = |\ev^{-1}(1)|$, then there chose a function $f_p:k \rightarrowtail 2^n$ picking out all the solutions which evaluate to $1$. The functor from $\sub_2$ to this subcategory spans takes polynomials $p \mapsto (2^n \xleftarrowtail {f_p} k \xrightarrowtail {f_p} 2^n)$.  Any two two spans induced by the same polynomial are isomorphic, so this is actually well defined.  It is clearly an isomorphism on objects, and it can easily be shown to be a monoidal functor.

The fullness is easy and the faithfulness comes from the fact that we can reduce every map to a polynomial and then reduce the polynomial to algebraic normal form.

\end{proof}


%
%The first axiom enforces that the trivial polynomial is the identity.
%The second axiom allows one to perform elementary row operations on polynomials.
%The third axiom allows one to multiply all polynomials.
%Because every map $f$ in $\sub_2(n,n)$ has precisely one representative polynomial, and polynomials have a normal form via their reduction, and row reduction is confluent, completeness is immediate.




\begin{definition}
Let $\sub\Iso\f_2$ be the prop generated by a distributive law of pros:
$$
\sub_2 \otimes \Iso(\f_2);
$$
$$
 \forall n,m,k \in \N, \forall p \in \F_2[x_1,\ldots, x_{n+2+m}],
q \in \F_2[x_1,\ldots,x_{n+m+1+k}],
r \in \F_2[x_1,\ldots, x_n]:
$$
$$
\begin{tikzpicture}
	\begin{pgfonlayer}{nodelayer}
		\node [style=map] (0) at (5, 3.5) {$p(x_1,\ldots, x_n, x_{n+1}, x_{n+2}, x_{n+3},\ldots, x_{n+2+m})$};
		\node [style=none] (1) at (4.5, 4.25) {};
		\node [style=none] (2) at (5.5, 4.25) {};
		\node [style=none] (3) at (4.75, 4.25) {};
		\node [style=none] (4) at (5.25, 4.25) {};
		\node [style=none] (5) at (4.5, 2.25) {};
		\node [style=none] (6) at (5.5, 2.25) {};
		\node [style=none] (7) at (4.75, 2.75) {};
		\node [style=none] (8) at (5.25, 2.75) {};
		\node [style=none] (9) at (4.5, 4.5) {$n$};
		\node [style=none] (10) at (4.5, 2) {$n$};
		\node [style=none] (11) at (5.5, 4.5) {$m$};
		\node [style=none] (12) at (5.5, 2) {$m$};
		\node [style=none] (13) at (5.25, 2.25) {};
		\node [style=none] (14) at (4.75, 2.25) {};
	\end{pgfonlayer}
	\begin{pgfonlayer}{edgelayer}
		\draw [in=120, out=-90] (1.center) to (0);
		\draw [in=-90, out=60] (0) to (2.center);
		\draw [in=75, out=-90, looseness=0.75] (4.center) to (0);
		\draw [in=-90, out=105, looseness=0.75] (0) to (3.center);
		\draw [in=300, out=90] (6.center) to (0);
		\draw [in=90, out=-75, looseness=0.75] (0) to (8.center);
		\draw [in=255, out=90, looseness=0.75] (7.center) to (0);
		\draw [in=90, out=-120] (0) to (5.center);
		\draw [in=270, out=90] (13.center) to (7.center);
		\draw [in=270, out=90] (14.center) to (8.center);
	\end{pgfonlayer}
\end{tikzpicture}
\eqzxa{subiso.one}
\begin{tikzpicture}
	\begin{pgfonlayer}{nodelayer}
		\node [style=map] (0) at (5, 3.25) {$p(x_1,\ldots, x_n, x_{n+2}, x_{n+1}, x_{n+3},\ldots, x_{n+2+m})$};
		\node [style=none] (1) at (4.5, 2.5) {};
		\node [style=none] (2) at (5.5, 2.5) {};
		\node [style=none] (3) at (4.75, 2.5) {};
		\node [style=none] (4) at (5.25, 2.5) {};
		\node [style=none] (5) at (4.5, 4.5) {};
		\node [style=none] (6) at (5.5, 4.5) {};
		\node [style=none] (7) at (4.75, 4) {};
		\node [style=none] (8) at (5.25, 4) {};
		\node [style=none] (9) at (4.5, 2.25) {$n$};
		\node [style=none] (10) at (4.5, 4.75) {$n$};
		\node [style=none] (11) at (5.5, 2.25) {$m$};
		\node [style=none] (12) at (5.5, 4.75) {$m$};
		\node [style=none] (13) at (5.25, 4.5) {};
		\node [style=none] (14) at (4.75, 4.5) {};
	\end{pgfonlayer}
	\begin{pgfonlayer}{edgelayer}
		\draw [in=-120, out=90] (1.center) to (0);
		\draw [in=90, out=-60] (0) to (2.center);
		\draw [in=-75, out=90, looseness=0.75] (4.center) to (0);
		\draw [in=90, out=-105, looseness=0.75] (0) to (3.center);
		\draw [in=-300, out=-90] (6.center) to (0);
		\draw [in=-90, out=75, looseness=0.75] (0) to (8.center);
		\draw [in=-255, out=-90, looseness=0.75] (7.center) to (0);
		\draw [in=-90, out=120] (0) to (5.center);
		\draw [in=-270, out=-90] (13.center) to (7.center);
		\draw [in=-270, out=-90] (14.center) to (8.center);
	\end{pgfonlayer}
\end{tikzpicture}
$$
$$
\begin{tikzpicture}
	\begin{pgfonlayer}{nodelayer}
		\node [style=map] (0) at (5, 3.5) {$q(x_1,\ldots, x_{n+m+1+k})$};
		\node [style=none] (1) at (3.75, 4) {};
		\node [style=none] (2) at (6.25, 4) {};
		\node [style=none] (3) at (4.25, 4) {};
		\node [style=none] (4) at (5.75, 4) {};
		\node [style=none] (5) at (3.75, 3) {};
		\node [style=none] (6) at (6.25, 3) {};
		\node [style=none] (7) at (4.25, 3) {};
		\node [style=none] (8) at (5.75, 3) {};
		\node [style=none] (9) at (3.75, 2.25) {$n$};
		\node [style=none] (10) at (4.25, 2.5) {};
		\node [style=none] (11) at (5.75, 2.5) {};
		\node [style=dot] (12) at (4.25, 2.75) {};
		\node [style=oplus] (13) at (5.75, 2.75) {};
		\node [style=none] (14) at (3.75, 2.5) {};
		\node [style=none] (15) at (6.25, 2.5) {};
		\node [style=none] (16) at (5.25, 4) {};
		\node [style=dot] (17) at (5.25, 2.75) {};
		\node [style=none] (18) at (5.25, 2.5) {};
		\node [style=none] (19) at (6.25, 2.25) {$k$};
		\node [style=none] (20) at (4.75, 2.5) {$m$};
		\node [style=none] (21) at (3.75, 5) {$n$};
		\node [style=none] (22) at (6.25, 5) {$k$};
		\node [style=none] (23) at (4.75, 4) {$m$};
		\node [style=none] (24) at (4.25, 4.5) {};
		\node [style=none] (25) at (5.75, 4.5) {};
		\node [style=none] (26) at (3.75, 4.5) {};
		\node [style=none] (27) at (6.25, 4.5) {};
		\node [style=none] (28) at (5.25, 4.5) {};
	\end{pgfonlayer}
	\begin{pgfonlayer}{edgelayer}
		\draw [in=270, out=90] (10.center) to (7.center);
		\draw [in=270, out=90] (11.center) to (8.center);
		\draw (15.center) to (6.center);
		\draw (14.center) to (5.center);
		\draw (13) to (17);
		\draw [style=dotted] (17) to (12);
		\draw (18.center) to (17);
		\draw (1.center) to (26.center);
		\draw (3.center) to (24.center);
		\draw (16.center) to (28.center);
		\draw (4.center) to (25.center);
		\draw (2.center) to (27.center);
		\draw (1.center) to (5.center);
		\draw (12) to (3.center);
		\draw (16.center) to (17);
		\draw (13) to (4.center);
		\draw (2.center) to (6.center);
	\end{pgfonlayer}
\end{tikzpicture}
\eqzxa{subiso.two}
\begin{tikzpicture}
	\begin{pgfonlayer}{nodelayer}
		\node [style=map] (0) at (5, 3) {$q(x_1,\ldots, x_{n+m}, (x_{n+1}\ldots x_{n+m-1})+x_{n+m+1}, x_{n+m+2}, \ldots, x_{n+m+1+k})$};
		\node [style=none] (1) at (3.75, 2.5) {};
		\node [style=none] (2) at (6.25, 2.5) {};
		\node [style=none] (3) at (4.25, 2.5) {};
		\node [style=none] (4) at (5.75, 2.5) {};
		\node [style=none] (5) at (3.75, 3.5) {};
		\node [style=none] (6) at (6.25, 3.5) {};
		\node [style=none] (7) at (4.25, 3.5) {};
		\node [style=none] (8) at (5.75, 3.5) {};
		\node [style=none] (9) at (3.75, 4.25) {$n$};
		\node [style=none] (10) at (4.25, 4) {};
		\node [style=none] (11) at (5.75, 4) {};
		\node [style=dot] (12) at (4.25, 3.75) {};
		\node [style=oplus] (13) at (5.75, 3.75) {};
		\node [style=none] (14) at (3.75, 4) {};
		\node [style=none] (15) at (6.25, 4) {};
		\node [style=none] (16) at (5.25, 2.5) {};
		\node [style=dot] (17) at (5.25, 3.75) {};
		\node [style=none] (18) at (5.25, 4) {};
		\node [style=none] (19) at (6.25, 4.25) {$k$};
		\node [style=none] (20) at (4.75, 4) {$m$};
		\node [style=none] (21) at (3.75, 1.75) {$n$};
		\node [style=none] (22) at (6.25, 1.75) {$k$};
		\node [style=none] (23) at (4.75, 2.5) {$m$};
		\node [style=none] (24) at (4.25, 2) {};
		\node [style=none] (25) at (5.75, 2) {};
		\node [style=none] (26) at (3.75, 2) {};
		\node [style=none] (27) at (6.25, 2) {};
		\node [style=none] (28) at (5.25, 2) {};
	\end{pgfonlayer}
	\begin{pgfonlayer}{edgelayer}
		\draw [in=-270, out=-90] (10.center) to (7.center);
		\draw [in=-270, out=-90] (11.center) to (8.center);
		\draw (15.center) to (6.center);
		\draw (14.center) to (5.center);
		\draw (13) to (17);
		\draw [style=dotted] (17) to (12);
		\draw (18.center) to (17);
		\draw (26.center) to (1.center);
		\draw (24.center) to (3.center);
		\draw (28.center) to (16.center);
		\draw (25.center) to (4.center);
		\draw (27.center) to (2.center);
		\draw (1.center) to (5.center);
		\draw (3.center) to (12);
		\draw (16.center) to (17);
		\draw (4.center) to (13);
		\draw (2.center) to (6.center);
	\end{pgfonlayer}
\end{tikzpicture}\hspace*{.5cm}
\begin{tikzpicture}
	\begin{pgfonlayer}{nodelayer}
		\node [style=none] (5) at (3.25, 4.25) {};
		\node [style=none] (6) at (3.25, 3.25) {};
		\node [style=map] (9) at (3.25, 3.75) {$r$};
		\node [style=none] (10) at (3.25, 2.25) {};
		\node [style=none] (11) at (3.75, 4.25) {};
		\node [style=none] (12) at (3.75, 2.25) {};
		\node [style=none] (13) at (3.75, 3.25) {};
	\end{pgfonlayer}
	\begin{pgfonlayer}{edgelayer}
		\draw (5.center) to (6.center);
		\draw [in=90, out=-90] (13.center) to (10.center);
		\draw [in=-90, out=90] (12.center) to (6.center);
		\draw (13.center) to (11.center);
	\end{pgfonlayer}
\end{tikzpicture}
\eqzxa{subiso.three}
\begin{tikzpicture}
	\begin{pgfonlayer}{nodelayer}
		\node [style=none] (5) at (3.75, 2.25) {};
		\node [style=none] (6) at (3.75, 3.25) {};
		\node [style=map] (9) at (3.75, 2.75) {$r$};
		\node [style=none] (10) at (3.75, 4.25) {};
		\node [style=none] (11) at (3.25, 2.25) {};
		\node [style=none] (12) at (3.25, 4.25) {};
		\node [style=none] (13) at (3.25, 3.25) {};
	\end{pgfonlayer}
	\begin{pgfonlayer}{edgelayer}
		\draw (5.center) to (6.center);
		\draw [in=270, out=90] (13.center) to (10.center);
		\draw [in=90, out=-90] (12.center) to (6.center);
		\draw (13.center) to (11.center);
	\end{pgfonlayer}
\end{tikzpicture}
$$


\end{definition}

\begin{lemma}
\label{lem:subiso}
$\sub\Iso\f_2$ is a presentation for the subcategory of $(\Span(\FSets),\times)$ generated by spans of the form $2^n \xleftarrowtail{e} k \xrightarrowtail {e} 2^m \xrightarrow[\cong]{f} 2^m$, for all $n,m k \in \N$ and all isomorphisms $f$ and monics $e$.
\end{lemma}


\begin{proof}
The obvious functor is clearly monoidal. Moreover, it is full by construction.
For the faithfulness, take two maps $f$ and $g$ in $\sub\Iso\f_2$.  Then one can just push everything to the end and then use the decidability of equality on both factors of the distributive law to show that they are equal.
\end{proof}

% $2^n \xleftarrow{e} k \xrightarrow{e} 2^n$, for all monics $e$ and isomorphisms of the form $2^n = 2^n \xrightarrow{\sim} 2^n$. 




\begin{definition}
Consider the prop $\sub\inj\f_2$ generated by a distributive law of props:
$$
 \sub\Iso\f_2 \otimes_{\Iso(\f_2)} \inj(\f_2);
 \forall n,m \in \N, p \in \F_2[x_1,\ldots, x_{n+1+m}]:
$$
$$
\begin{tikzpicture}
	\begin{pgfonlayer}{nodelayer}
		\node [style=none] (0) at (1.5, 3.5) {};
		\node [style=map] (1) at (2.5, 2.75) {$p(x_1,\ldots,x_{n+1+m})$};
		\node [style=none] (2) at (1.5, 3.75) {$n$};
		\node [style=none] (3) at (3.5, 3.5) {};
		\node [style=none] (4) at (3.5, 3.75) {$m$};
		\node [style=none] (5) at (1.5, 1.75) {};
		\node [style=none] (6) at (3.5, 1.75) {};
		\node [style=none] (7) at (1.5, 1.5) {$n$};
		\node [style=none] (8) at (3.5, 1.5) {$m$};
		\node [style=X] (9) at (2.5, 2) {};
		\node [style=none] (10) at (2.5, 3.5) {};
	\end{pgfonlayer}
	\begin{pgfonlayer}{edgelayer}
		\draw (0.center) to (5.center);
		\draw (3.center) to (6.center);
		\draw (10.center) to (9);
	\end{pgfonlayer}
\end{tikzpicture}
\eqzxa{subinj}
\begin{tikzpicture}
	\begin{pgfonlayer}{nodelayer}
		\node [style=none] (0) at (1.5, 3.75) {};
		\node [style=map] (1) at (2.5, 2.75) {$p(x_1,\ldots,x_n,0,x_{n+2},\ldots,x_{n+1+m})$};
		\node [style=none] (2) at (1.5, 4) {$n$};
		\node [style=none] (3) at (3.5, 3.75) {};
		\node [style=none] (4) at (3.5, 4) {$m$};
		\node [style=none] (5) at (1.5, 2) {};
		\node [style=none] (6) at (3.5, 2) {};
		\node [style=none] (7) at (1.5, 1.75) {$n$};
		\node [style=none] (8) at (3.5, 1.75) {$m$};
		\node [style=X] (9) at (2.5, 3.4) {};
		\node [style=none] (10) at (2.5, 3.75) {};
	\end{pgfonlayer}
	\begin{pgfonlayer}{edgelayer}
		\draw (0.center) to (5.center);
		\draw (3.center) to (6.center);
		\draw (10.center) to (9);
	\end{pgfonlayer}
\end{tikzpicture}
$$
\end{definition}



\begin{lemma}
\label{lem:subinj}
$\sub\inj\f_2$ is a presentation for the subcategory of $(\Span(\FSets),\times)$ generated by spans of the form $2^n \xleftarrowtail{e} k \xrightarrowtail{e} 2^n \xrightarrowtail{e'} 2^{m}$ for all $n,m,k \in \N$ and all monics $e,e'$.
\end{lemma}

The proof is completely analogous to as in the case of $\sub\iso\f_2$.

Any $n$ variable polynomial $p$ can be interpreted as a span of monics via the oracle $\mathcal{O}_p$, where the value of the target wire is restricted to have the value $0$.  Each such polynomial corresponds to a subobject, which complicates the matter further than in the affine case.


\begin{definition}
Consider the prop $\pr\iso\f_2$ given by the distributive law of props:

$$
\sub\inj\f_2^\op \otimes_{\sub\Iso\f_2} \sub\inj\f_2;
\begin{tikzpicture}
	\begin{pgfonlayer}{nodelayer}
		\node [style=map] (0) at (0, 0) {$\mathcal{O}_p$};
		\node [style=none] (1) at (-0.25, -0.75) {};
		\node [style=none] (2) at (-0.25, 0.75) {};
		\node [style=none] (3) at (-0.25, 1) {};
		\node [style=none] (4) at (-0.25, -1) {};
		\node [style=X] (5) at (0.25, 0.75) {};
		\node [style=X] (6) at (0.25, -0.75) {};
	\end{pgfonlayer}
	\begin{pgfonlayer}{edgelayer}
		\draw (3.center) to (2.center);
		\draw (4.center) to (1.center);
		\draw [in=-60, out=90] (6) to (0);
		\draw [in=-90, out=60] (0) to (5);
		\draw [in=120, out=-90] (2.center) to (0);
		\draw [in=90, out=-120, looseness=1.25] (0) to (1.center);
	\end{pgfonlayer}
\end{tikzpicture}
\eqzxa{oracle}
\begin{tikzpicture}
	\begin{pgfonlayer}{nodelayer}
		\node [style=map] (0) at (-0.25, 0) {$p$};
		\node [style=none] (1) at (-0.25, -0.75) {};
		\node [style=none] (2) at (-0.25, 0.75) {};
		\node [style=none] (3) at (-0.25, 1) {};
		\node [style=none] (4) at (-0.25, -1) {};
	\end{pgfonlayer}
	\begin{pgfonlayer}{edgelayer}
		\draw (3.center) to (2.center);
		\draw (4.center) to (1.center);
		\draw (2.center) to (0);
		\draw (0) to (1.center);
	\end{pgfonlayer}
\end{tikzpicture}
$$

\end{definition}

%This is actually a distributive law, because it need only be witnessed by pushing $\eta_X$ past $\eta_X^\op$.  The only time this can't be done is when there is the target  of a generalized controlled-not not gate--or those of several generalized controlled-not gates is in between both of these generators.  In which case, the obstructing generalized controlled-not gates form an oracle for which we can apply this equation.  This is computing the apex of the span when performing a pullback.

%Note that $\pr\iso\f_2$ is not actually partial isomorphisms over $\f_2$, per se, but rather a full subcategory of partial isomorphisms of sets and functions, which is not iself a prop with respect to the Cartesian product.  It is because of such a complication that the aforementioned distributive law isn't quite a pullback.

\begin{lemma}
\label{lem:parisof}
$\pr\iso \f_2$ is a presentation for the full subcategory $(\FPinj_2,\times)$ of $(\ParIso(\FSets),\times)$ with objects powers of two.
\end{lemma}

Unlike the previous lemmas, this is not dependant on a complete presentation for the isomorphisms.
The proof is a consequence of \cite[Thm 7.6.14]{cole} where they give a finite, complete presentation for this category.  The identities up to this point are equivalent to this finite presentation, whether or not the conjectured presentation for the isomorphisms is complete.

%\begin{comment}
$\pr\iso \f_2$ can be presented in terms of finitely many generators and relations.  The identities are contained in \S \ref{subsubsec:presentations:three:pinj}.
%\end{comment}

\begin{definition}

Consider the prop $\pr\f_2$ given by the  pushout of the following diagram of props, given by adding a counit to the diagonal map:
$$\pr\iso\f_2 \leftarrow \surj^\op \rightarrow \cm^\op$$

\end{definition}


\begin{lemma}
\label{lem:parand}
$\pr\f_2$ is a presentation for the the full subcategory $(\FPar_2,\times)$ of $(\Par(\FSets),\times)$ with objects powers of two.
\end{lemma}

\begin{proof}
One has to show that the following diagram commutes:

\renewcommand{\cubetopbl}{$\surj^\op$}
\renewcommand{\cubetopbr}{$\cm^\op$}
\renewcommand{\cubetopfl}{$\pr\iso\f_2$}
\renewcommand{\cubetopfr}{$\pr\f_2$}
\renewcommand{\cubebotbl}{$\surj^\op$ }
\renewcommand{\cubebotbr}{$\cm^\op$ }
\renewcommand{\cubebotfl}{$(\FPinj_2,\times)$ }
\renewcommand{\cubebotfr}{}

$$
\xymatrixrowsep{2mm}\xymatrixcolsep{2mm}
\xymatrix{
                                       & \mbox{\cubetopbl} \ar[rr] \ar[dl] \ar@{=}[dd]     &                                                  & \mbox{\cubetopbr} \ar@{=}[dd] \ar[dl] \\
\mbox{\cubetopfl} \ar[rr]  \ar[dd]_{\cong}           &                                                                                              &\mbox{\cubetopfr} \ar@{-->}[dd]^(.35){\cong}   \skewpullbackcorner[ul]              \\
                                       &  \mbox{\cubebotbl} \ar[dl] \ar[rr]                    &                                                  & \mbox{\cubebotbr} \ar@/^1pc/[ddl] \ar[dl] \\
\mbox{\cubebotfl} \ar@/_1pc/[drr] \ar[rr]  &                                                                                             & \mbox{\cubebotfr} \skewpullbackcorner[ul]    \ar@{-->}[d]^{\cong}  \\
                                                   &                                                                                             & (\FPar_2,\times)
}
$$

Again, the proof is essentially the same as for the linear and affine cases; the only difference being that the Cartesian completion of $\FPinj_2$ is $\FPar_2$.


\end{proof}


%\begin{comment}
There is a particularly elegant finite presentation  contained in \S \ref{subsubsec:presentations:three:par}, which  is much more ZX-flavoured.
%\end{comment}


%
%\begin{corollary}
%Give easier chacterization TODO
%\end{corollary}


\begin{definition}
Let $\sp\f_2$ denote the pushout of the diagram of props:
$$
\pr\f_2^\op\leftarrow \pr\iso\f_2 \rightarrow \pr\f_2
$$
\end{definition}

\begin{lemma}\cite{zxa}
\label{lem:spanand}
$\sp\f_2$ is a presentation for  the full subcategory $(\FSpan_2,\times)$ of $(\Span(\FSets),\times)$ with objects powers of two.
\end{lemma}

\begin{proof}
One has to show that the following diagram commutes:

\renewcommand{\cubetopbl}{$\pr\iso\f_2$}
\renewcommand{\cubetopbr}{$ \pr\f_2$}
\renewcommand{\cubetopfl}{$\pr\f_2^\op$}
\renewcommand{\cubetopfr}{$\sp\f_2$}
\renewcommand{\cubebotbl}{$(\FPinj_2,\times)$ }
\renewcommand{\cubebotbr}{$(\FPar_2,\times)$ }
\renewcommand{\cubebotfl}{$(\FPar_2,\times)^\op$ }
\renewcommand{\cubebotfr}{}

$$
\xymatrixrowsep{2mm}\xymatrixcolsep{0mm}
\xymatrix{
                                       & \mbox{\cubetopbl} \ar[rr] \ar[dl] \ar[dd]^(.7){\cong}      &                                                  & \mbox{\cubetopbr}  \ar[dd]^{\cong} \ar[dl] \\
\mbox{\cubetopfl} \ar[rr]  \ar[dd]_{\cong}           &                                                                                              &\mbox{\cubetopfr} \ar@{-->}[dd]^(.35){\cong}   \skewpullbackcorner[ul]              \\
                                       &  \mbox{\cubebotbl} \ar[dl] \ar[rr]                    &                                                  & \mbox{\cubebotbr} \ar@/^1pc/[ddl] \ar[dl] \\
\mbox{\cubebotfl} \ar@/_1pc/[drr] \ar[rr]  &                                                                                             & \mbox{\cubebotfr} \skewpullbackcorner[ul]    \ar@{-->}[d]^{\cong}  \\
                                                   &                                                                                             &( \FSpan_2,\times)
}
$$


This follows from \cite[Lem. 4.3]{zxa}.

\end{proof}


\begin{remark}
$\sp\f_2$ is isomorphic to $\ZXA$.
\end{remark}






%
%
%That is to say $\Span(\f_2)$ is a presentation for the prop of ``qubit matrices'' over $\N$.  There is an alternative presentation of this category due to \cite{zxa}:
%\begin{corollary}
%$\Span(\f_2)$ can equivalently can be presented in terms of the coproduct of the monoids $X,Z,\&$ and comonoids $Z^\dag,X^\dag$
%\begin{itemize}
%\item An extra-Frobenius algebra between $X^\op$ and $Y$.
%
%\item A special-Frobenius algebra between $Z^\op$ and $Z$.
%
%\item The Lawvere theory for $\F_2^+$ represented by $X$ and $Z^\op$.
%
%\item The Lawvere theory for $\F_2^\times$  represented by  $\&$ and $Z^\op$.
%
%\item A distributive law $L_{X} \otimes_L L_{\&}  \Rightarrow  L_{\&} \otimes_L L_{X}$  in $\Kl(\T_{\Mon-\Prof}^\times)$, where $Z^\op$ is identified with prop for the diagonal monoid.
%
%\item The naturality of $\eta_{\&}$ with respect to $\mu_{\&}^\op$.
%
%\end{itemize}
%
%
%\end{corollary}


%
%The proof follows from realizing that this equation makes the triangle gate idempotent, which allows one to reduce the value of non-scalar positive natural number H-boxes to $1$, alike to the quotient described in \cite{niel} (H-boxes are first described in the paper \cite{zh}).    The other law forces all nonzero scalar H-boxes, and thus all nonzero scalars to be 0. So this is complete for qubit boolean matrices, and thus, qubit relations.
%


\nocite{ih}
\nocite{coecke2008interacting}
\nocite{zh}
\nocite{tof}

%\appendix 



%
%\begin{lemma}
%
%The phase-free fragment of the ZH calculus is presented by the pushout of the following diagram of props:
%
%
%TODO
%
%modulo the quotient:
%
%\end{lemma}



%\begin{comment}
%






%
%\section{Alternative presentations}
%\label{sec:presentations}
%
%In this section, we give alternative presentations of the props presented in the main body of this paper.  With the exception of the alternative presentation of $(\FPinj_2,\times)$, these are presented in terms of a bunch of (co)monoids modulo equations.  This is more in the aesthetic tradition of the ZX-calculus, for example.  
%
%\subsection{Section \ref{sec:one}}
%\label{subsec:presentations:one}
%
%\subsubsection{$(\Par(\Mat(\F_2)),+)$}
%\label{subsubsec:presentations:one:par}
%
%
%$(\Par(\Mat(\F_2)),+)$ is presented by the symmetric monoidal theory with the following generators:
%$$
%\begin{tikzpicture}
%	\begin{pgfonlayer}{nodelayer}
%		\node [style=Z] (0) at (0.75, 5) {};
%		\node [style=none] (1) at (0.5, 4.5) {};
%		\node [style=none] (2) at (1, 4.5) {};
%		\node [style=none] (3) at (0.75, 5.5) {};
%	\end{pgfonlayer}
%	\begin{pgfonlayer}{edgelayer}
%		\draw [in=-135, out=90] (1.center) to (0);
%		\draw [in=90, out=-45] (0) to (2.center);
%		\draw (0) to (3.center);
%	\end{pgfonlayer}
%\end{tikzpicture}
%\hspace*{.5cm}
%\begin{tikzpicture}
%	\begin{pgfonlayer}{nodelayer}
%		\node [style=Z] (0) at (0, 5) {};
%		\node [style=none] (1) at (-0.25, 5.5) {};
%		\node [style=none] (2) at (0.25, 5.5) {};
%		\node [style=none] (3) at (0, 4.5) {};
%	\end{pgfonlayer}
%	\begin{pgfonlayer}{edgelayer}
%		\draw [in=135, out=-90] (1.center) to (0);
%		\draw [in=-90, out=45] (0) to (2.center);
%		\draw (0) to (3.center);
%	\end{pgfonlayer}
%\end{tikzpicture}
%\hspace*{.5cm}
%\begin{tikzpicture}
%	\begin{pgfonlayer}{nodelayer}
%		\node [style=Z] (0) at (0, 5) {};
%		\node [style=none] (3) at (0, 4.5) {};
%	\end{pgfonlayer}
%	\begin{pgfonlayer}{edgelayer}
%		\draw (0) to (3.center);
%	\end{pgfonlayer}
%\end{tikzpicture}
%\hspace*{.5cm}
%\begin{tikzpicture}
%	\begin{pgfonlayer}{nodelayer}
%		\node [style=X] (0) at (0.75, 5) {};
%		\node [style=none] (1) at (0.5, 4.5) {};
%		\node [style=none] (2) at (1, 4.5) {};
%		\node [style=none] (3) at (0.75, 5.5) {};
%	\end{pgfonlayer}
%	\begin{pgfonlayer}{edgelayer}
%		\draw [in=-135, out=90] (1.center) to (0);
%		\draw [in=90, out=-45] (0) to (2.center);
%		\draw (0) to (3.center);
%	\end{pgfonlayer}
%\end{tikzpicture}
%\hspace*{.5cm}
%\begin{tikzpicture}
%	\begin{pgfonlayer}{nodelayer}
%		\node [style=X] (0) at (0, 4.5) {};
%		\node [style=none] (3) at (0, 5) {};
%	\end{pgfonlayer}
%	\begin{pgfonlayer}{edgelayer}
%		\draw (0) to (3.center);
%	\end{pgfonlayer}
%\end{tikzpicture}
%\hspace*{,5cm}
%\begin{tikzpicture}
%	\begin{pgfonlayer}{nodelayer}
%		\node [style=X] (0) at (0.5, 5) {};
%		\node [style=none] (1) at (0.5, 4.5) {};
%	\end{pgfonlayer}
%	\begin{pgfonlayer}{edgelayer}
%		\draw (0) to (1.center);
%	\end{pgfonlayer}
%\end{tikzpicture}
%$$
%Modulo the all the equations of $\cb_2$ in addition to the identity and its transpose:
%$$
%\begin{tikzpicture}
%	\begin{pgfonlayer}{nodelayer}
%		\node [style=X] (0) at (0.5, 5) {};
%		\node [style=none] (1) at (0.5, 4.5) {};
%		\node [style=Z] (2) at (0.5, 4.5) {};
%		\node [style=none] (3) at (0.25, 4) {};
%		\node [style=none] (4) at (0.75, 4) {};
%	\end{pgfonlayer}
%	\begin{pgfonlayer}{edgelayer}
%		\draw (0) to (1.center);
%		\draw [in=-135, out=90] (3.center) to (2);
%		\draw [in=-45, out=90] (4.center) to (2);
%	\end{pgfonlayer}
%\end{tikzpicture}
%  \eref{bi.two}
%\begin{tikzpicture}
%	\begin{pgfonlayer}{nodelayer}
%		\node [style=X] (0) at (0.25, 5) {};
%		\node [style=none] (3) at (0.25, 4) {};
%		\node [style=none] (4) at (0.75, 4) {};
%		\node [style=X] (5) at (0.75, 5) {};
%	\end{pgfonlayer}
%	\begin{pgfonlayer}{edgelayer}
%		\draw (4.center) to (5);
%		\draw (0) to (3.center);
%	\end{pgfonlayer}
%\end{tikzpicture}
%$$
%
%
%\subsubsection{$(\Span(\Mat(\F_2)),+)$}
%\label{subsubsec:presentations:one:span}
%
%$(\Span(\Mat(\F_2)),+)$ is presented by the symmetric monoidal theory with the same generators and equations as \S \ref{subsubsec:presentations:one:par} and their transposes as well as the following equations making the white monoid/comnoid pair into a special commutative Frobenius algebra and the black monoid/comonoid pair into a commutative Frobenius algebra.
%
%
%
%\subsection{Section \ref{sec:two}}
%\label{subsec:presentations:two}
%
%
%\subsubsection{$(\Par(\Aff\Fin\Vect(\F_2))^*,+)$}
%\label{subsubsec:presentations:two:par}
%
%$(\Par(\Aff\Fin\Vect(\F_2))^*,+)$ is presented by the symmetric monoidal theory with the same generators and equations as in \S \ref{subsubsec:presentations:one:par} in addition to the following generator:
%$$
%\begin{tikzpicture}
%	\begin{pgfonlayer}{nodelayer}
%		\node [style=X] (0) at (0, 4) {$1$};
%		\node [style=none] (1) at (0, 4.5) {};
%	\end{pgfonlayer}
%	\begin{pgfonlayer}{edgelayer}
%		\draw (0) to (1.center);
%	\end{pgfonlayer}
%\end{tikzpicture}
%$$
%and the following equations:
%$$
%\begin{tikzpicture}
%	\begin{pgfonlayer}{nodelayer}
%		\node [style=X] (0) at (-4, 0.75) {$1$};
%		\node [style=none] (1) at (-3.5, 0) {};
%		\node [style=none] (2) at (-3.5, 1.5) {};
%	\end{pgfonlayer}
%	\begin{pgfonlayer}{edgelayer}
%		\draw (1.center) to (2.center);
%	\end{pgfonlayer}
%\end{tikzpicture}
%\eqzxa{zero.new}
%\begin{tikzpicture}
%	\begin{pgfonlayer}{nodelayer}
%		\node [style=X] (0) at (-4, 0.75) {$1$};
%		\node [style=none] (1) at (-3.5, 0) {};
%		\node [style=none] (2) at (-3.5, 1.5) {};
%		\node [style=Z] (3) at (-3.5, 0.5) {};
%		\node [style=X] (4) at (-3.5, 1) {};
%	\end{pgfonlayer}
%	\begin{pgfonlayer}{edgelayer}
%		\draw (2.center) to (4);
%		\draw (3) to (1.center);
%	\end{pgfonlayer}
%\end{tikzpicture},
%\hspace*{,5cm}
%\begin{tikzpicture}
%	\begin{pgfonlayer}{nodelayer}
%		\node [style=X] (0) at (0.75, 4) {$1$};
%		\node [style=none] (1) at (0.75, 4.5) {};
%		\node [style=Z] (2) at (0.75, 4.5) {};
%		\node [style=none] (3) at (0.5, 5) {};
%		\node [style=none] (4) at (1, 5) {};
%	\end{pgfonlayer}
%	\begin{pgfonlayer}{edgelayer}
%		\draw (0) to (1.center);
%		\draw [in=135, out=-90] (3.center) to (2);
%		\draw [in=45, out=-90] (4.center) to (2);
%	\end{pgfonlayer}
%\end{tikzpicture}
%  \erefop{bi.two}
%\begin{tikzpicture}
%	\begin{pgfonlayer}{nodelayer}
%		\node [style=X] (0) at (0.5, 4) {$1$};
%		\node [style=none] (1) at (0.5, 5) {};
%		\node [style=none] (2) at (1, 5) {};
%		\node [style=X] (3) at (1, 4) {$1$};
%	\end{pgfonlayer}
%	\begin{pgfonlayer}{edgelayer}
%		\draw (2.center) to (3);
%		\draw (0) to (1.center);
%	\end{pgfonlayer}
%\end{tikzpicture}
%\hspace*{,5cm}
%\begin{tikzpicture}
%	\begin{pgfonlayer}{nodelayer}
%		\node [style=X] (0) at (0.75, 4) {$1$};
%		\node [style=none] (1) at (0.75, 4.5) {};
%		\node [style=Z] (2) at (0.75, 4.5) {};
%	\end{pgfonlayer}
%	\begin{pgfonlayer}{edgelayer}
%		\draw (0) to (1.center);
%	\end{pgfonlayer}
%\end{tikzpicture}
%  \eref{extra}
%$$
%
%
%\subsubsection{$(\Span(\Aff\Fin\Vect(\F_2))^*,+)$}
%\label{subsubsec:presentations:two:span}
%
%$(\Span(\Aff\Fin\Vect(\F_2))^*,+)$ is presented by the generators and identities of  \ref{subsubsec:presentations:two:par} as well as as well as the generator 
%$\begin{tikzpicture}
%	\begin{pgfonlayer}{nodelayer}
%		\node [style=none] (0) at (0.75, 0.5) {};
%		\node [style=none] (1) at (0.75, -0.25) {};
%		\node [style=Z] (2) at (0.75, -0.25) {};
%	\end{pgfonlayer}
%	\begin{pgfonlayer}{edgelayer}
%		\draw (0.center) to (1.center);
%	\end{pgfonlayer}
%\end{tikzpicture}$ 
%and the equation making the codiagonal map counital:
%$$
%  \begin{tikzpicture}[rotate=90,yscale=-1]
%	\begin{pgfonlayer}{nodelayer}
%		\node [style=Z] (0) at (-9, -0) {};
%		\node [style=none] (1) at (-8.25, -0) {};
%		\node [style=Z] (2) at (-9.75, 0.25) {};
%		\node [style=none] (3) at (-10, -0.25) {};
%	\end{pgfonlayer}
%	\begin{pgfonlayer}{edgelayer}
%		\draw [in=-150, out=0, looseness=1.00] (3.center) to (0);
%		\draw [in=150, out=0, looseness=1.00] (2.center) to (0);
%		\draw (0) to (1.center);
%	\end{pgfonlayer}
%  \end{tikzpicture}
%  \eref{unit}
%  \begin{tikzpicture}[rotate=90]
%	\begin{pgfonlayer}{nodelayer}
%		\node [style=none] (0) at (-9, 0.25) {};
%		\node [style=none] (1) at (-9.75, 0.25) {};
%	\end{pgfonlayer}
%	\begin{pgfonlayer}{edgelayer}
%		\draw (1) to (0.center);
%	\end{pgfonlayer}
%  \end{tikzpicture}
%$$
%
%\subsection{Section \ref{sec:three}}
%\label{subsec:presentations:three}
%
%
%
%%\subsubsection{$(\FPinj_2,\times)$}
%%\label{subsubsec:presentations:three:pinj}
%
%%By \cite[\S 7]{cole} $(\FPinj_2,\times)$ is presented by the prop generated by the Toffoli gate (the triple-controlled-not gate) as well as unit for the and gate and its transpose modulo the following equations:
%%
%%\begin{multicols}{2}
%%\begin{enumerate}[label={\bf [TOF.\arabic*]}, ref={\bf [TOF.\arabic*]}, wide = 0pt, leftmargin = 2em]
%%\item
%%\label{TOF.1}
%%{\hfil
%%$
%%\begin{tabular}{cc}
%%\begin{tikzpicture}
%%	\begin{pgfonlayer}{nodelayer}
%%		\node [style=nothing] (0) at (1.5, 0) {};
%%		\node [style=nothing] (1) at (1, 0) {};
%%		\node [style=oplus] (2) at (1.5, 1) {};
%%		\node [style=dot] (3) at (1, 1) {};
%%		\node [style=dot] (4) at (0.5, 1) {};
%%		\node [style=X] (5) at (0.5, 0.5) {$1$};
%%		\node [style=nothing] (6) at (0.5, 1.5) {};
%%		\node [style=nothing] (7) at (1, 1.5) {};
%%		\node [style=nothing] (8) at (1.5, 1.5) {};
%%	\end{pgfonlayer}
%%	\begin{pgfonlayer}{edgelayer}
%%		\draw (5) to (4);
%%		\draw (4) to (6);
%%		\draw (7) to (3);
%%		\draw (1) to (3);
%%		\draw (0) to (2);
%%		\draw (2) to (8);
%%		\draw (2) to (3);
%%		\draw (3) to (4);
%%	\end{pgfonlayer}
%%\end{tikzpicture}
%%=
%%\begin{tikzpicture}
%%	\begin{pgfonlayer}{nodelayer}
%%		\node [style=nothing] (0) at (1.5, 0) {};
%%		\node [style=nothing] (1) at (1, 0) {};
%%		\node [style=oplus] (2) at (1.5, 1) {};
%%		\node [style=dot] (3) at (1, 1) {};
%%		\node [style=X] (4) at (0.5, 1) {$1$};
%%		\node [style=nothing] (5) at (0.5, 1.5) {};
%%		\node [style=nothing] (6) at (1, 1.5) {};
%%		\node [style=nothing] (7) at (1.5, 1.5) {};
%%	\end{pgfonlayer}
%%	\begin{pgfonlayer}{edgelayer}
%%		\draw (1) to (3);
%%		\draw (0) to (2);
%%		\draw (2) to (3);
%%		\draw (6) to (3);
%%		\draw (4) to (5);
%%		\draw (2) to (7);
%%	\end{pgfonlayer}
%%\end{tikzpicture} &
%%\begin{tikzpicture}
%%	\begin{pgfonlayer}{nodelayer}
%%		\node [style=nothing] (0) at (1.5, 2) {};
%%		\node [style=nothing] (1) at (1, 2) {};
%%		\node [style=oplus] (2) at (1.5, 1) {};
%%		\node [style=dot] (3) at (1, 1) {};
%%		\node [style=dot] (4) at (0.5, 1) {};
%%		\node [style=X] (5) at (0.5, 1.5) {$1$};
%%		\node [style=nothing] (6) at (0.5, 0.5) {};
%%		\node [style=nothing] (7) at (1, 0.5) {};
%%		\node [style=nothing] (8) at (1.5, 0.5) {};
%%	\end{pgfonlayer}
%%	\begin{pgfonlayer}{edgelayer}
%%		\draw (5) to (4);
%%		\draw (4) to (6);
%%		\draw (7) to (3);
%%		\draw (1) to (3);
%%		\draw (0) to (2);
%%		\draw (2) to (8);
%%		\draw (2) to (3);
%%		\draw (3) to (4);
%%	\end{pgfonlayer}
%%\end{tikzpicture}
%%=
%%\begin{tikzpicture}
%%	\begin{pgfonlayer}{nodelayer}
%%		\node [style=nothing] (0) at (1.5, 2) {};
%%		\node [style=nothing] (1) at (1, 2) {};
%%		\node [style=oplus] (2) at (1.5, 1) {};
%%		\node [style=dot] (3) at (1, 1) {};
%%		\node [style=X] (4) at (0.5, 1) {$1$};
%%		\node [style=nothing] (5) at (0.5, 0.5) {};
%%		\node [style=nothing] (6) at (1, 0.5) {};
%%		\node [style=nothing] (7) at (1.5, 0.5) {};
%%	\end{pgfonlayer}
%%	\begin{pgfonlayer}{edgelayer}
%%		\draw (1) to (3);
%%		\draw (0) to (2);
%%		\draw (2) to (3);
%%		\draw (6) to (3);
%%		\draw (4) to (5);
%%		\draw (2) to (7);
%%	\end{pgfonlayer}
%%\end{tikzpicture}
%%\end{tabular}
%%$}
%%
%%
%%\item
%%\label{TOF.2}
%%{\hfil
%%$
%%\begin{tabular}{cc}
%%\begin{tikzpicture}
%%	\begin{pgfonlayer}{nodelayer}
%%		\node [style=nothing] (0) at (1, 0.5) {};
%%		\node [style=nothing] (1) at (1.5, 0.5) {};
%%		\node [style=nothing] (2) at (0.5, 2) {};
%%		\node [style=nothing] (3) at (1, 2) {};
%%		\node [style=nothing] (4) at (1.5, 2) {};
%%		\node [style=dot] (5) at (0.5, 1.5) {};
%%		\node [style=dot] (6) at (1, 1.5) {};
%%		\node [style=oplus] (7) at (1.5, 1.5) {};
%%		\node [style=X] (8) at (0.5, 1) {};
%%	\end{pgfonlayer}
%%	\begin{pgfonlayer}{edgelayer}
%%		\draw (5) to (2);
%%		\draw (3) to (6);
%%		\draw (6) to (0);
%%		\draw (1) to (7);
%%		\draw (7) to (4);
%%		\draw (7) to (6);
%%		\draw (6) to (5);
%%		\draw (8) to (5);
%%	\end{pgfonlayer}
%%\end{tikzpicture}
%%=
%%\begin{tikzpicture}
%%	\begin{pgfonlayer}{nodelayer}
%%		\node [style=nothing] (0) at (1, 0.75) {};
%%		\node [style=nothing] (1) at (1.5, 0.75) {};
%%		\node [style=nothing] (2) at (0.5, 2) {};
%%		\node [style=nothing] (3) at (1, 2) {};
%%		\node [style=nothing] (4) at (1.5, 2) {};
%%		\node [style=X] (5) at (0.5, 1.25) {};
%%	\end{pgfonlayer}
%%	\begin{pgfonlayer}{edgelayer}
%%		\draw (5) to (2);
%%		\draw (0) to (3);
%%		\draw (1) to (4);
%%	\end{pgfonlayer}
%%\end{tikzpicture} & 
%%\begin{tikzpicture}
%%	\begin{pgfonlayer}{nodelayer}
%%		\node [style=nothing] (0) at (1, 2.5) {};
%%		\node [style=nothing] (1) at (1.5, 2.5) {};
%%		\node [style=nothing] (2) at (0.5, 1) {};
%%		\node [style=nothing] (3) at (1, 1) {};
%%		\node [style=nothing] (4) at (1.5, 1) {};
%%		\node [style=dot] (5) at (0.5, 1.5) {};
%%		\node [style=dot] (6) at (1, 1.5) {};
%%		\node [style=oplus] (7) at (1.5, 1.5) {};
%%		\node [style=X] (8) at (0.5, 2) {};
%%	\end{pgfonlayer}
%%	\begin{pgfonlayer}{edgelayer}
%%		\draw (5) to (2);
%%		\draw (3) to (6);
%%		\draw (6) to (0);
%%		\draw (1) to (7);
%%		\draw (7) to (4);
%%		\draw (7) to (6);
%%		\draw (6) to (5);
%%		\draw (8) to (5);
%%	\end{pgfonlayer}
%%\end{tikzpicture}
%%=
%%\begin{tikzpicture}
%%	\begin{pgfonlayer}{nodelayer}
%%		\node [style=nothing] (0) at (1, 2.5) {};
%%		\node [style=nothing] (1) at (1.5, 2.5) {};
%%		\node [style=nothing] (2) at (0.5, 1.25) {};
%%		\node [style=nothing] (3) at (1, 1.25) {};
%%		\node [style=nothing] (4) at (1.5, 1.25) {};
%%		\node [style=X] (5) at (0.5, 2) {};
%%	\end{pgfonlayer}
%%	\begin{pgfonlayer}{edgelayer}
%%		\draw (5) to (2);
%%		\draw (0) to (3);
%%		\draw (1) to (4);
%%	\end{pgfonlayer}
%%\end{tikzpicture}
%%\end{tabular}
%%$}
%%
%%\item
%%\label{TOF.3}
%%{\hfil
%%$
%%\begin{tikzpicture}
%%	\begin{pgfonlayer}{nodelayer}
%%		\node [style=nothing] (0) at (-0.5, 0.5) {};
%%		\node [style=nothing] (1) at (0, 0.5) {};
%%		\node [style=nothing] (2) at (-1, 0.5) {};
%%		\node [style=nothing] (3) at (-1.5, 0.5) {};
%%		\node [style=nothing] (4) at (-2, 0.5) {};
%%		\node [style=dot] (5) at (-1.5, 1) {};
%%		\node [style=oplus] (6) at (-1, 1) {};
%%		\node [style=oplus] (7) at (-1, 1.5) {};
%%		\node [style=dot] (8) at (-0.5, 1.5) {};
%%		\node [style=dot] (9) at (-2, 1) {};
%%		\node [style=dot] (10) at (0, 1.5) {};
%%		\node [style=nothing] (11) at (-0.5, 2) {};
%%		\node [style=nothing] (12) at (-1.5, 2) {};
%%		\node [style=nothing] (13) at (-2, 2) {};
%%		\node [style=nothing] (14) at (0, 2) {};
%%		\node [style=nothing] (15) at (-1, 2) {};
%%	\end{pgfonlayer}
%%	\begin{pgfonlayer}{edgelayer}
%%		\draw (4) to (9);
%%		\draw (9) to (13);
%%		\draw (3) to (5);
%%		\draw (5) to (12);
%%		\draw (2) to (6);
%%		\draw (6) to (7);
%%		\draw (7) to (15);
%%		\draw (0) to (8);
%%		\draw (8) to (11);
%%		\draw (1) to (10);
%%		\draw (10) to (14);
%%		\draw (10) to (8);
%%		\draw (8) to (7);
%%		\draw (6) to (5);
%%		\draw (5) to (9);
%%	\end{pgfonlayer}
%%\end{tikzpicture}
%%=
%%\begin{tikzpicture}
%%	\begin{pgfonlayer}{nodelayer}
%%		\node [style=nothing] (0) at (-0.5, 0.5) {};
%%		\node [style=nothing] (1) at (0, 0.5) {};
%%		\node [style=nothing] (2) at (-1, 0.5) {};
%%		\node [style=nothing] (3) at (-1.5, 0.5) {};
%%		\node [style=nothing] (4) at (-2, 0.5) {};
%%		\node [style=dot] (5) at (-1.5, 1.5) {};
%%		\node [style=dot] (6) at (-0.5, 1) {};
%%		\node [style=dot] (7) at (-2, 1.5) {};
%%		\node [style=dot] (8) at (0, 1) {};
%%		\node [style=nothing] (9) at (-0.5, 2) {};
%%		\node [style=nothing] (10) at (-1.5, 2) {};
%%		\node [style=nothing] (11) at (-2, 2) {};
%%		\node [style=nothing] (12) at (0, 2) {};
%%		\node [style=nothing] (13) at (-1, 2) {};
%%		\node [style=oplus] (14) at (-1, 1.5) {};
%%		\node [style=oplus] (15) at (-1, 1) {};
%%	\end{pgfonlayer}
%%	\begin{pgfonlayer}{edgelayer}
%%		\draw (4) to (7);
%%		\draw (7) to (11);
%%		\draw (3) to (5);
%%		\draw (5) to (10);
%%		\draw (0) to (6);
%%		\draw (6) to (9);
%%		\draw (1) to (8);
%%		\draw (8) to (12);
%%		\draw (8) to (6);
%%		\draw (5) to (7);
%%		\draw (2) to (15);
%%		\draw (15) to (14);
%%		\draw (14) to (13);
%%		\draw (14) to (5);
%%		\draw (6) to (15);
%%	\end{pgfonlayer}
%%\end{tikzpicture}
%%$}
%%
%%
%%\item
%%\label{TOF.4}
%%{\hfil
%%$
%%\begin{tikzpicture}
%%	\begin{pgfonlayer}{nodelayer}
%%		\node [style=nothing] (0) at (-0.5, 0.5) {};
%%		\node [style=nothing] (1) at (0, 0.5) {};
%%		\node [style=nothing] (2) at (-1, 0.5) {};
%%		\node [style=nothing] (3) at (-1.5, 0.5) {};
%%		\node [style=nothing] (4) at (-2, 0.5) {};
%%		\node [style=dot] (5) at (-1.5, 1) {};
%%		\node [style=dot] (6) at (-1, 1) {};
%%		\node [style=dot] (7) at (-1, 1.5) {};
%%		\node [style=dot] (8) at (-0.5, 1.5) {};
%%		\node [style=oplus] (9) at (-2, 1) {};
%%		\node [style=oplus] (10) at (0, 1.5) {};
%%		\node [style=nothing] (11) at (-0.5, 2) {};
%%		\node [style=nothing] (12) at (-1.5, 2) {};
%%		\node [style=nothing] (13) at (-2, 2) {};
%%		\node [style=nothing] (14) at (0, 2) {};
%%		\node [style=nothing] (15) at (-1, 2) {};
%%	\end{pgfonlayer}
%%	\begin{pgfonlayer}{edgelayer}
%%		\draw (4) to (9);
%%		\draw (9) to (13);
%%		\draw (3) to (5);
%%		\draw (5) to (12);
%%		\draw (2) to (6);
%%		\draw (6) to (7);
%%		\draw (7) to (15);
%%		\draw (0) to (8);
%%		\draw (8) to (11);
%%		\draw (1) to (10);
%%		\draw (10) to (14);
%%		\draw (10) to (8);
%%		\draw (8) to (7);
%%		\draw (6) to (5);
%%		\draw (5) to (9);
%%	\end{pgfonlayer}
%%\end{tikzpicture}
%%=
%%\begin{tikzpicture}
%%	\begin{pgfonlayer}{nodelayer}
%%		\node [style=nothing] (0) at (-0.5, 0.5) {};
%%		\node [style=nothing] (1) at (0, 0.5) {};
%%		\node [style=nothing] (2) at (-1, 0.5) {};
%%		\node [style=nothing] (3) at (-1.5, 0.5) {};
%%		\node [style=nothing] (4) at (-2, 0.5) {};
%%		\node [style=dot] (5) at (-1.5, 1.5) {};
%%		\node [style=dot] (6) at (-0.5, 1) {};
%%		\node [style=oplus] (7) at (-2, 1.5) {};
%%		\node [style=oplus] (8) at (0, 1) {};
%%		\node [style=nothing] (9) at (-0.5, 2) {};
%%		\node [style=nothing] (10) at (-1.5, 2) {};
%%		\node [style=nothing] (11) at (-2, 2) {};
%%		\node [style=nothing] (12) at (0, 2) {};
%%		\node [style=nothing] (13) at (-1, 2) {};
%%		\node [style=dot] (14) at (-1, 1.5) {};
%%		\node [style=dot] (15) at (-1, 1) {};
%%	\end{pgfonlayer}
%%	\begin{pgfonlayer}{edgelayer}
%%		\draw (4) to (7);
%%		\draw (7) to (11);
%%		\draw (3) to (5);
%%		\draw (5) to (10);
%%		\draw (0) to (6);
%%		\draw (6) to (9);
%%		\draw (1) to (8);
%%		\draw (8) to (12);
%%		\draw (8) to (6);
%%		\draw (5) to (7);
%%		\draw (2) to (15);
%%		\draw (15) to (14);
%%		\draw (14) to (13);
%%		\draw (14) to (5);
%%		\draw (6) to (15);
%%	\end{pgfonlayer}
%%\end{tikzpicture}
%%$}
%%
%%\item
%%\label{TOF.5}
%%{\hfil
%%$
%%\begin{tikzpicture}
%%	\begin{pgfonlayer}{nodelayer}
%%		\node [style=nothing] (0) at (-1, 0.5) {};
%%		\node [style=nothing] (1) at (-0.5, 0.5) {};
%%		\node [style=nothing] (2) at (-1.5, 0.5) {};
%%		\node [style=nothing] (3) at (-2, 0.5) {};
%%		\node [style=nothing] (4) at (-1, 2) {};
%%		\node [style=nothing] (5) at (-1.5, 2) {};
%%		\node [style=nothing] (6) at (-2, 2) {};
%%		\node [style=nothing] (7) at (-0.5, 2) {};
%%		\node [style=oplus] (8) at (-2, 1) {};
%%		\node [style=oplus] (9) at (-0.5, 1.5) {};
%%		\node [style=dot] (10) at (-1.5, 1) {};
%%		\node [style=dot] (11) at (-1, 1) {};
%%		\node [style=dot] (12) at (-1.5, 1.5) {};
%%		\node [style=dot] (13) at (-1, 1.5) {};
%%	\end{pgfonlayer}
%%	\begin{pgfonlayer}{edgelayer}
%%		\draw (3) to (8);
%%		\draw (8) to (6);
%%		\draw (5) to (12);
%%		\draw (12) to (10);
%%		\draw (10) to (2);
%%		\draw (0) to (11);
%%		\draw (11) to (13);
%%		\draw (13) to (4);
%%		\draw (7) to (9);
%%		\draw (9) to (1);
%%		\draw (10) to (11);
%%		\draw (10) to (8);
%%		\draw (12) to (13);
%%		\draw (13) to (9);
%%	\end{pgfonlayer}
%%\end{tikzpicture}
%%=
%%\begin{tikzpicture}
%%	\begin{pgfonlayer}{nodelayer}
%%		\node [style=nothing] (0) at (-1, 0.5) {};
%%		\node [style=nothing] (1) at (-0.5, 0.5) {};
%%		\node [style=nothing] (2) at (-1.5, 0.5) {};
%%		\node [style=nothing] (3) at (-2, 0.5) {};
%%		\node [style=nothing] (4) at (-1, 2) {};
%%		\node [style=nothing] (5) at (-1.5, 2) {};
%%		\node [style=nothing] (6) at (-2, 2) {};
%%		\node [style=nothing] (7) at (-0.5, 2) {};
%%		\node [style=oplus] (8) at (-2, 1.5) {};
%%		\node [style=dot] (9) at (-1.5, 1.5) {};
%%		\node [style=dot] (10) at (-1, 1.5) {};
%%		\node [style=oplus] (11) at (-0.5, 1) {};
%%		\node [style=dot] (12) at (-1, 1) {};
%%		\node [style=dot] (13) at (-1.5, 1) {};
%%	\end{pgfonlayer}
%%	\begin{pgfonlayer}{edgelayer}
%%		\draw (9) to (10);
%%		\draw (9) to (8);
%%		\draw (13) to (12);
%%		\draw (12) to (11);
%%		\draw (3) to (8);
%%		\draw (8) to (6);
%%		\draw (5) to (9);
%%		\draw (9) to (13);
%%		\draw (13) to (2);
%%		\draw (0) to (12);
%%		\draw (12) to (10);
%%		\draw (10) to (4);
%%		\draw (7) to (11);
%%		\draw (11) to (1);
%%	\end{pgfonlayer}
%%\end{tikzpicture}
%%$}
%%
%%
%%\item
%%\label{TOF.6}
%%{\hfil
%%$
%%\begin{tikzpicture}
%%	\begin{pgfonlayer}{nodelayer}
%%		\node [style=nothing] (0) at (-1, 0.5) {};
%%		\node [style=nothing] (1) at (-1.5, 0.5) {};
%%		\node [style=nothing] (2) at (-2, 0.5) {};
%%		\node [style=nothing] (3) at (-1, 2) {};
%%		\node [style=nothing] (4) at (-1.5, 2) {};
%%		\node [style=nothing] (5) at (-2, 2) {};
%%		\node [style=nothing] (6) at (-0.5, 2) {};
%%		\node [style=oplus] (7) at (-0.5, 1) {};
%%		\node [style=dot] (8) at (-1.5, 1.5) {};
%%		\node [style=dot] (9) at (-1, 1.5) {};
%%		\node [style=dot] (10) at (-1, 1) {};
%%		\node [style=oplus] (11) at (-0.5, 1.5) {};
%%		\node [style=nothing] (12) at (-0.5, 0.5) {};
%%		\node [style=dot] (13) at (-2, 1) {};
%%	\end{pgfonlayer}
%%	\begin{pgfonlayer}{edgelayer}
%%		\draw (8) to (1);
%%		\draw (0) to (9);
%%		\draw (9) to (10);
%%		\draw (10) to (3);
%%		\draw (6) to (7);
%%		\draw (8) to (9);
%%		\draw (10) to (7);
%%		\draw (12) to (11);
%%		\draw (11) to (7);
%%		\draw (8) to (4);
%%		\draw (9) to (11);
%%		\draw (10) to (13);
%%		\draw (13) to (5);
%%		\draw (13) to (2);
%%	\end{pgfonlayer}
%%\end{tikzpicture}
%%=
%%\begin{tikzpicture}
%%	\begin{pgfonlayer}{nodelayer}
%%		\node [style=nothing] (0) at (-1, 0.5) {};
%%		\node [style=nothing] (1) at (-1.5, 0.5) {};
%%		\node [style=nothing] (2) at (-2, 0.5) {};
%%		\node [style=nothing] (3) at (-1, 2) {};
%%		\node [style=nothing] (4) at (-1.5, 2) {};
%%		\node [style=nothing] (5) at (-2, 2) {};
%%		\node [style=nothing] (6) at (-0.5, 2) {};
%%		\node [style=oplus] (7) at (-0.5, 1.5) {};
%%		\node [style=dot] (8) at (-1.5, 1) {};
%%		\node [style=dot] (9) at (-1, 1) {};
%%		\node [style=dot] (10) at (-1, 1.5) {};
%%		\node [style=oplus] (11) at (-0.5, 1) {};
%%		\node [style=nothing] (12) at (-0.5, 0.5) {};
%%		\node [style=dot] (13) at (-2, 1.5) {};
%%	\end{pgfonlayer}
%%	\begin{pgfonlayer}{edgelayer}
%%		\draw (8) to (1);
%%		\draw (0) to (9);
%%		\draw (9) to (10);
%%		\draw (10) to (3);
%%		\draw (6) to (7);
%%		\draw (8) to (9);
%%		\draw (10) to (7);
%%		\draw (12) to (11);
%%		\draw (11) to (7);
%%		\draw (8) to (4);
%%		\draw (9) to (11);
%%		\draw (10) to (13);
%%		\draw (13) to (5);
%%		\draw (13) to (2);
%%	\end{pgfonlayer}
%%\end{tikzpicture}
%%$}
%%
%%\item
%%\label{TOF.7}
%%{\hfil
%%$
%%\begin{tikzpicture}
%%	\begin{pgfonlayer}{nodelayer}
%%		\node [style=nothing] (0) at (1, 0) {};
%%		\node [style=nothing] (1) at (0.5, 0) {};
%%		\node [style=nothing] (2) at (0.5, 3.5) {};
%%		\node [style=nothing] (3) at (1, 3.5) {};
%%		\node [style=X] (4) at (1.5, 3) {};
%%		\node [style=oplus] (5) at (1.5, 2.5) {};
%%		\node [style=dot] (6) at (1, 2.5) {};
%%		\node [style=dot] (7) at (0.5, 1) {};
%%		\node [style=oplus] (8) at (1.5, 1) {};
%%		\node [style=X] (9) at (1.5, 1.5) {};
%%		\node [style=X] (10) at (1.5, 0.5) {$1$};
%%		\node [style=X] (11) at (1.5, 2) {$1$};
%%	\end{pgfonlayer}
%%	\begin{pgfonlayer}{edgelayer}
%%		\draw (1) to (7);
%%		\draw (7) to (2);
%%		\draw (3) to (6);
%%		\draw (6) to (0);
%%		\draw (8) to (9);
%%		\draw (8) to (7);
%%		\draw (5) to (4);
%%		\draw (5) to (6);
%%		\draw (10) to (8);
%%		\draw (11) to (5);
%%	\end{pgfonlayer}
%%\end{tikzpicture}
%%=
%%\begin{tikzpicture}
%%	\begin{pgfonlayer}{nodelayer}
%%		\node [style=nothing] (0) at (3, 0) {};
%%		\node [style=nothing] (1) at (2.5, 0) {};
%%		\node [style=nothing] (2) at (2.5, 3.5) {};
%%		\node [style=nothing] (3) at (3, 3.5) {};
%%		\node [style=dot] (4) at (2.5, 1.75) {};
%%		\node [style=dot] (5) at (3, 1.75) {};
%%		\node [style=X] (6) at (3.5, 1.25) {$1$};
%%		\node [style=X] (7) at (3.5, 2.25) {};
%%		\node [style=oplus] (8) at (3.5, 1.75) {};
%%	\end{pgfonlayer}
%%	\begin{pgfonlayer}{edgelayer}
%%		\draw (1) to (4);
%%		\draw (4) to (2);
%%		\draw (3) to (5);
%%		\draw (5) to (0);
%%		\draw (6) to (8);
%%		\draw (8) to (7);
%%		\draw (8) to (5);
%%		\draw (5) to (4);
%%	\end{pgfonlayer}
%%\end{tikzpicture}
%%$}
%%%
%%%\item
%%%\label{TOF.7}
%%%{\hfil
%%%$
%%%\begin{tikzpicture}
%%%	\begin{pgfonlayer}{nodelayer}
%%%		\node [style=nothing] (0) at (0, -0) {};
%%%		\node [style=nothing] (1) at (1.5, -0) {};
%%%		\node [style=X] (2) at (0.4, 0.5) {};
%%%		\node [style=X] (3) at (1.1, 0.5) {};
%%%	\end{pgfonlayer}
%%%	\begin{pgfonlayer}{edgelayer}
%%%		\draw (2) to (3);
%%%		\draw (0) to (1);
%%%	\end{pgfonlayer}
%%%\end{tikzpicture}
%%%=
%%%\begin{tikzpicture}
%%%	\begin{pgfonlayer}{nodelayer}
%%%		\node [style=nothing] (0) at (0, -0) {};
%%%		\node [style=nothing] (1) at (1.5, -0) {};
%%%		\node [style=X] (2) at (0.4, 0.5) {};
%%%		\node [style=X] (3) at (1.1, 0.5) {};
%%%		\node [style=X] (4) at (1, -0) {};
%%%		\node [style=X] (5) at (0.5000002, -0) {};
%%%	\end{pgfonlayer}
%%%	\begin{pgfonlayer}{edgelayer}
%%%		\draw (2) to (3);
%%%		\draw (5) to (0);
%%%		\draw (4) to (1);
%%%	\end{pgfonlayer}
%%%\end{tikzpicture}
%%%$}
%%
%%\item
%%\label{TOF.8}
%%{\hfil
%%$
%%\begin{tikzpicture}
%%	\begin{pgfonlayer}{nodelayer}
%%		\node [style=X] (0) at (0, 0.5) {$1$};
%%		\node [style=X] (1) at (0, 1.5) {$1$};
%%	\end{pgfonlayer}
%%	\begin{pgfonlayer}{edgelayer}
%%		\draw (0) to (1);
%%	\end{pgfonlayer}
%%\end{tikzpicture}
%%=
%%\begin{tikzpicture}
%%	\begin{pgfonlayer}{nodelayer}
%%		\node [style=rn] (0) at (0, 0.5) {};
%%		\node [style=rn] (1) at (0, 1.5) {};
%%	\end{pgfonlayer}
%%\end{tikzpicture}
%%%\hspace*{-.8cm}
%%%\begin{tikzpicture}[scale=.5]
%%%\begin{pgfonlayer}{nodelayer}
%%%\begin{tikzpicture}
%%%\node[cloud, cloud puffs=15.7,minimum width=3cm, draw,] (cloud) at (0,0) {$1_0$};
%%%\end{tikzpicture}
%%%\end{pgfonlayer}
%%%\begin{pgfonlayer}{edgelayer}
%%%\end{pgfonlayer}
%%%\end{tikzpicture}
%%$}
%%
%%\item
%%\label{TOF.9}
%%{\hfil
%%$
%%\begin{tikzpicture}
%%	\begin{pgfonlayer}{nodelayer}
%%		\node [style=nothing] (0) at (-1.75, 0.5) {};
%%		\node [style=nothing] (1) at (-1.25, 0.5) {};
%%		\node [style=nothing] (2) at (-0.75, 0.5) {};
%%		\node [style=dot] (3) at (-1.75, 1) {};
%%		\node [style=dot] (4) at (-1.25, 1) {};
%%		\node [style=oplus] (5) at (-0.75, 1) {};
%%		\node [style=dot] (6) at (-1.75, 1.5) {};
%%		\node [style=oplus] (7) at (-0.75, 1.5) {};
%%		\node [style=dot] (8) at (-1.25, 1.5) {};
%%		\node [style=nothing] (9) at (-1.25, 2) {};
%%		\node [style=nothing] (10) at (-0.75, 2) {};
%%		\node [style=nothing] (11) at (-1.75, 2) {};
%%	\end{pgfonlayer}
%%	\begin{pgfonlayer}{edgelayer}
%%		\draw (0) to (3);
%%		\draw (1) to (4);
%%		\draw (2) to (5);
%%		\draw (3) to (4);
%%		\draw (4) to (5);
%%		\draw (6) to (8);
%%		\draw (8) to (7);
%%		\draw (3) to (6);
%%		\draw (6) to (11);
%%		\draw (4) to (8);
%%		\draw (8) to (9);
%%		\draw (5) to (7);
%%		\draw (7) to (10);
%%	\end{pgfonlayer}
%%\end{tikzpicture}
%%=
%%\begin{tikzpicture}
%%	\begin{pgfonlayer}{nodelayer}
%%		\node [style=nothing] (0) at (-1.75, 0.5) {};
%%		\node [style=nothing] (1) at (-1.25, 0.5) {};
%%		\node [style=nothing] (2) at (-0.75, 0.5) {};
%%		\node [style=nothing] (3) at (-1.25, 2) {};
%%		\node [style=nothing] (4) at (-0.75, 2) {};
%%		\node [style=nothing] (5) at (-1.75, 2) {};
%%	\end{pgfonlayer}
%%	\begin{pgfonlayer}{edgelayer}
%%		\draw (0) to (5);
%%		\draw (1) to (3);
%%		\draw (2) to (4);
%%	\end{pgfonlayer}
%%\end{tikzpicture}
%%$}
%%
%%\item
%%\label{TOF.10}
%%{\hfil
%%$
%%\begin{tikzpicture}
%%	\begin{pgfonlayer}{nodelayer}
%%		\node [style=nothing] (0) at (0, 0.5) {};
%%		\node [style=nothing] (1) at (-0.5, 0.5) {};
%%		\node [style=nothing] (2) at (-1, 0.5) {};
%%		\node [style=nothing] (3) at (-1.5, 0.5) {};
%%		\node [style=dot] (4) at (-1, 1) {};
%%		\node [style=dot] (5) at (-0.5, 1) {};
%%		\node [style=oplus] (6) at (0, 1) {};
%%		\node [style=dot] (7) at (-1.5, 1.5) {};
%%		\node [style=oplus] (8) at (-0.5, 1.5) {};
%%		\node [style=dot] (9) at (-1, 1.5) {};
%%		\node [style=dot] (10) at (-1, 2) {};
%%		\node [style=oplus] (11) at (0, 2) {};
%%		\node [style=dot] (12) at (-0.5, 2) {};
%%		\node [style=nothing] (13) at (-1.5, 2.5) {};
%%		\node [style=nothing] (14) at (-0.5, 2.5) {};
%%		\node [style=nothing] (15) at (-1, 2.5) {};
%%		\node [style=nothing] (16) at (0, 2.5) {};
%%	\end{pgfonlayer}
%%	\begin{pgfonlayer}{edgelayer}
%%		\draw (4) to (5);
%%		\draw (5) to (6);
%%		\draw (7) to (9);
%%		\draw (9) to (8);
%%		\draw (10) to (12);
%%		\draw (12) to (11);
%%		\draw (3) to (7);
%%		\draw (7) to (13);
%%		\draw (15) to (10);
%%		\draw (10) to (9);
%%		\draw (9) to (4);
%%		\draw (4) to (2);
%%		\draw (1) to (5);
%%		\draw (5) to (8);
%%		\draw (8) to (12);
%%		\draw (12) to (14);
%%		\draw (16) to (11);
%%		\draw (11) to (6);
%%		\draw (6) to (0);
%%	\end{pgfonlayer}
%%\end{tikzpicture}
%%=
%%\begin{tikzpicture}
%%	\begin{pgfonlayer}{nodelayer}
%%		\node [style=nothing] (17) at (4.5, 0.5) {};
%%		\node [style=nothing] (18) at (4, 0.5) {};
%%		\node [style=nothing] (19) at (3.5, 0.5) {};
%%		\node [style=nothing] (20) at (3, 0.5) {};
%%		\node [style=nothing] (21) at (3, 2.5) {};
%%		\node [style=nothing] (22) at (4, 2.5) {};
%%		\node [style=nothing] (23) at (3.5, 2.5) {};
%%		\node [style=nothing] (24) at (4.5, 2.5) {};
%%		\node [style=dot] (25) at (3, 1.25) {};
%%		\node [style=dot] (26) at (3.5, 1.25) {};
%%		\node [style=dot] (27) at (3, 1.75) {};
%%		\node [style=dot] (28) at (3.5, 1.75) {};
%%		\node [style=oplus] (29) at (4, 1.75) {};
%%		\node [style=oplus] (30) at (4.5, 1.25) {};
%%	\end{pgfonlayer}
%%	\begin{pgfonlayer}{edgelayer}
%%		\draw (20) to (25);
%%		\draw (25) to (27);
%%		\draw (27) to (21);
%%		\draw (23) to (28);
%%		\draw (28) to (26);
%%		\draw (26) to (19);
%%		\draw (18) to (29);
%%		\draw (29) to (22);
%%		\draw (24) to (30);
%%		\draw (30) to (17);
%%		\draw (30) to (26);
%%		\draw (26) to (25);
%%		\draw (27) to (28);
%%		\draw (28) to (29);
%%	\end{pgfonlayer}
%%\end{tikzpicture}
%%$}
%%
%%\item
%%\label{TOF.11}
%%{\hfil
%%$
%%\begin{tikzpicture}
%%	\begin{pgfonlayer}{nodelayer}
%%		\node [style=nothing] (0) at (0, 0.5) {};
%%		\node [style=nothing] (1) at (-0.5, 0.5) {};
%%		\node [style=nothing] (2) at (-1, 0.5) {};
%%		\node [style=nothing] (3) at (-1.5, 0.5) {};
%%		\node [style=nothing] (4) at (-0.5, 2.5) {};
%%		\node [style=nothing] (5) at (0, 2.5) {};
%%		\node [style=dot] (6) at (-1.5, 1) {};
%%		\node [style=dot] (7) at (-1, 1.5) {};
%%		\node [style=dot] (8) at (-0.5, 1.5) {};
%%		\node [style=oplus] (9) at (-1, 1) {};
%%		\node [style=oplus] (10) at (0, 1.5) {};
%%		\node [style=nothing] (11) at (-1.5, 2.5) {};
%%		\node [style=nothing] (12) at (-1, 2.5) {};
%%		\node [style=oplus] (13) at (-1, 2) {};
%%		\node [style=dot] (14) at (-1.5, 2) {};
%%	\end{pgfonlayer}
%%	\begin{pgfonlayer}{edgelayer}
%%		\draw (6) to (9);
%%		\draw (7) to (8);
%%		\draw (8) to (10);
%%		\draw (0) to (10);
%%		\draw (10) to (5);
%%		\draw (4) to (8);
%%		\draw (8) to (1);
%%		\draw (2) to (9);
%%		\draw (9) to (7);
%%		\draw (6) to (3);
%%		\draw (6) to (14);
%%		\draw (14) to (11);
%%		\draw (12) to (13);
%%		\draw (13) to (7);
%%		\draw (13) to (14);
%%	\end{pgfonlayer}
%%\end{tikzpicture}
%%=
%%\begin{tikzpicture}
%%	\begin{pgfonlayer}{nodelayer}
%%		\node [style=nothing] (0) at (2, 0.5) {};
%%		\node [style=nothing] (1) at (1, 0.5) {};
%%		\node [style=nothing] (2) at (1.5, 0.5) {};
%%		\node [style=nothing] (3) at (0.5, 0.5) {};
%%		\node [style=dot] (4) at (0.5, 1.25) {};
%%		\node [style=dot] (5) at (1.5, 1.25) {};
%%		\node [style=oplus] (6) at (2, 1.25) {};
%%		\node [style=nothing] (7) at (1.5, 2.5) {};
%%		\node [style=nothing] (8) at (1, 2.5) {};
%%		\node [style=nothing] (9) at (0.5, 2.5) {};
%%		\node [style=nothing] (10) at (2, 2.5) {};
%%		\node [style=dot] (11) at (1, 1.75) {};
%%		\node [style=dot] (12) at (1.5, 1.75) {};
%%		\node [style=oplus] (13) at (2, 1.75) {};
%%	\end{pgfonlayer}
%%	\begin{pgfonlayer}{edgelayer}
%%		\draw (3) to (4);
%%		\draw (2) to (5);
%%		\draw (6) to (0);
%%		\draw (6) to (5);
%%		\draw (5) to (4);
%%		\draw (11) to (1);
%%		\draw (5) to (12);
%%		\draw (12) to (7);
%%		\draw (10) to (13);
%%		\draw (13) to (6);
%%		\draw (13) to (12);
%%		\draw (12) to (11);
%%		\draw (4) to (9);
%%		\draw (11) to (8);
%%	\end{pgfonlayer}
%%\end{tikzpicture}
%%$}
%%
%%\item
%%\label{TOF.12}
%%{\hfil
%%$
%%\begin{tikzpicture}
%%	\begin{pgfonlayer}{nodelayer}
%%		\node [style=nothing] (0) at (-0.5, 0.5) {};
%%		\node [style=nothing] (1) at (0, 0.5) {};
%%		\node [style=nothing] (2) at (-1, 0.5) {};
%%		\node [style=nothing] (3) at (-1.5, 0.5) {};
%%		\node [style=nothing] (4) at (-0.5, 2.5) {};
%%		\node [style=nothing] (5) at (-1.5, 2.5) {};
%%		\node [style=nothing] (6) at (0, 2.5) {};
%%		\node [style=nothing] (7) at (-1, 2.5) {};
%%		\node [style=dot] (8) at (-1.5, 1) {};
%%		\node [style=dot] (9) at (-1, 1) {};
%%		\node [style=oplus] (10) at (-0.5, 1) {};
%%		\node [style=oplus] (11) at (0, 1.5) {};
%%		\node [style=dot] (12) at (-1, 1.5) {};
%%		\node [style=dot] (13) at (-0.5, 1.5) {};
%%		\node [style=oplus] (14) at (-0.5, 2) {};
%%		\node [style=dot] (15) at (-1.5, 2) {};
%%		\node [style=dot] (16) at (-1, 2) {};
%%	\end{pgfonlayer}
%%	\begin{pgfonlayer}{edgelayer}
%%		\draw (8) to (9);
%%		\draw (9) to (10);
%%		\draw (12) to (13);
%%		\draw (13) to (11);
%%		\draw (15) to (16);
%%		\draw (16) to (14);
%%		\draw (3) to (8);
%%		\draw (8) to (15);
%%		\draw (15) to (5);
%%		\draw (7) to (16);
%%		\draw (16) to (12);
%%		\draw (12) to (9);
%%		\draw (9) to (2);
%%		\draw (0) to (10);
%%		\draw (10) to (13);
%%		\draw (13) to (14);
%%		\draw (14) to (4);
%%		\draw (6) to (11);
%%		\draw (11) to (1);
%%	\end{pgfonlayer}
%%\end{tikzpicture}
%%=
%%\begin{tikzpicture}
%%	\begin{pgfonlayer}{nodelayer}
%%		\node [style=nothing] (0) at (1.5, 0.25) {};
%%		\node [style=nothing] (1) at (2, 0.25) {};
%%		\node [style=nothing] (2) at (1, 0.25) {};
%%		\node [style=nothing] (3) at (0.5, 0.25) {};
%%		\node [style=nothing] (4) at (1.5, 2.25) {};
%%		\node [style=nothing] (5) at (0.5, 2.25) {};
%%		\node [style=nothing] (6) at (2, 2.25) {};
%%		\node [style=nothing] (7) at (1, 2.25) {};
%%		\node [style=dot] (8) at (1, 1.5) {};
%%		\node [style=dot] (9) at (1.5, 1.5) {};
%%		\node [style=dot] (10) at (0.5, 1) {};
%%		\node [style=dot] (11) at (1, 1) {};
%%		\node [style=oplus] (12) at (2, 1) {};
%%		\node [style=oplus] (13) at (2, 1.5) {};
%%	\end{pgfonlayer}
%%	\begin{pgfonlayer}{edgelayer}
%%		\draw (8) to (9);
%%		\draw (3) to (10);
%%		\draw (10) to (5);
%%		\draw (2) to (11);
%%		\draw (11) to (8);
%%		\draw (8) to (7);
%%		\draw (0) to (9);
%%		\draw (9) to (4);
%%		\draw (1) to (12);
%%		\draw (12) to (13);
%%		\draw (13) to (6);
%%		\draw (13) to (9);
%%		\draw (12) to (11);
%%		\draw (11) to (10);
%%	\end{pgfonlayer}
%%\end{tikzpicture}
%%$}
%%
%%\item
%%\label{TOF.13}
%%{\hfil
%%$
%%\begin{tikzpicture}
%%	\begin{pgfonlayer}{nodelayer}
%%		\node [style=nothing] (0) at (0, 0.5) {};
%%		\node [style=nothing] (1) at (-1, 0.5) {};
%%		\node [style=nothing] (2) at (-0.5, 0.5) {};
%%		\node [style=nothing] (3) at (-1.5, 0.5) {};
%%		\node [style=nothing] (4) at (0, 2.5) {};
%%		\node [style=dot] (5) at (-1.5, 1) {};
%%		\node [style=dot] (6) at (-1, 1) {};
%%		\node [style=dot] (7) at (-0.5, 1.5) {};
%%		\node [style=oplus] (8) at (-0.5, 1) {};
%%		\node [style=oplus] (9) at (0, 1.5) {};
%%		\node [style=nothing] (10) at (-0.5, 2.5) {};
%%		\node [style=nothing] (11) at (-1.5, 2.5) {};
%%		\node [style=nothing] (12) at (-1, 2.5) {};
%%		\node [style=oplus] (13) at (-0.5, 2) {};
%%		\node [style=dot] (14) at (-1, 2) {};
%%		\node [style=dot] (15) at (-1.5, 2) {};
%%	\end{pgfonlayer}
%%	\begin{pgfonlayer}{edgelayer}
%%		\draw (5) to (3);
%%		\draw (6) to (1);
%%		\draw (2) to (8);
%%		\draw (8) to (7);
%%		\draw (4) to (9);
%%		\draw (9) to (0);
%%		\draw (8) to (6);
%%		\draw (6) to (5);
%%		\draw (9) to (7);
%%		\draw (5) to (15);
%%		\draw (15) to (11);
%%		\draw (12) to (14);
%%		\draw (14) to (6);
%%		\draw (7) to (13);
%%		\draw (13) to (10);
%%		\draw (13) to (14);
%%		\draw (14) to (15);
%%	\end{pgfonlayer}
%%\end{tikzpicture}
%%=
%%\begin{tikzpicture}
%%	\begin{pgfonlayer}{nodelayer}
%%		\node [style=nothing] (0) at (2, 0.25) {};
%%		\node [style=nothing] (1) at (1, 0.25) {};
%%		\node [style=nothing] (2) at (1.5, 0.25) {};
%%		\node [style=nothing] (3) at (0.5, 0.25) {};
%%		\node [style=dot] (4) at (0.5, 1) {};
%%		\node [style=dot] (5) at (1, 1) {};
%%		\node [style=oplus] (6) at (2, 1) {};
%%		\node [style=nothing] (7) at (1.5, 2.25) {};
%%		\node [style=nothing] (8) at (1, 2.25) {};
%%		\node [style=nothing] (9) at (2, 2.25) {};
%%		\node [style=nothing] (10) at (0.5, 2.25) {};
%%		\node [style=dot] (11) at (1.5, 1.5) {};
%%		\node [style=oplus] (12) at (2, 1.5) {};
%%	\end{pgfonlayer}
%%	\begin{pgfonlayer}{edgelayer}
%%		\draw (0) to (6);
%%		\draw (1) to (5);
%%		\draw (4) to (3);
%%		\draw (4) to (5);
%%		\draw (5) to (6);
%%		\draw (11) to (12);
%%		\draw (12) to (9);
%%		\draw (12) to (6);
%%		\draw (2) to (11);
%%		\draw (4) to (10);
%%		\draw (8) to (5);
%%		\draw (11) to (7);
%%	\end{pgfonlayer}
%%\end{tikzpicture}
%%$}
%%
%%\item
%%\label{TOF.14}
%%{\hfil
%%$
%%\begin{tikzpicture}
%%	\begin{pgfonlayer}{nodelayer}
%%		\node [style=nothing] (0) at (0, 0.5) {};
%%		\node [style=nothing] (1) at (-0.5, 0.5) {};
%%		\node [style=nothing] (2) at (-0.5, 2.5) {};
%%		\node [style=nothing] (3) at (0, 2.5) {};
%%		\node [style=oplus] (4) at (0, 1) {};
%%		\node [style=oplus] (5) at (0, 2) {};
%%		\node [style=oplus] (6) at (-0.5, 1.5) {};
%%		\node [style=dot] (7) at (-0.5, 2) {};
%%		\node [style=dot] (8) at (0, 1.5) {};
%%		\node [style=dot] (9) at (-0.5, 1) {};
%%	\end{pgfonlayer}
%%	\begin{pgfonlayer}{edgelayer}
%%		\draw (1) to (9);
%%		\draw (9) to (6);
%%		\draw (6) to (7);
%%		\draw (7) to (2);
%%		\draw (3) to (5);
%%		\draw (5) to (8);
%%		\draw (8) to (4);
%%		\draw (4) to (0);
%%		\draw (4) to (9);
%%		\draw (8) to (6);
%%		\draw (5) to (7);
%%	\end{pgfonlayer}
%%\end{tikzpicture}
%%=
%%\begin{tikzpicture}
%%	\begin{pgfonlayer}{nodelayer}
%%		\node [style=nothing] (0) at (1, 0.5) {};
%%		\node [style=nothing] (1) at (0.5, 0.5) {};
%%		\node [style=nothing] (2) at (0.5, 2.5) {};
%%		\node [style=nothing] (3) at (1, 2.5) {};
%%	\end{pgfonlayer}
%%	\begin{pgfonlayer}{edgelayer}
%%		\draw [in=-90, out=90, looseness=1.25] (1) to (3);
%%		\draw [in=-90, out=90, looseness=1.25] (0) to (2);
%%	\end{pgfonlayer}
%%\end{tikzpicture}
%%$}
%%
%%\item
%%\label{TOF.15}
%%{\hfil
%%$
%%\begin{tikzpicture}
%%	\begin{pgfonlayer}{nodelayer}
%%		\node [style=nothing] (0) at (-1.75, 0.5) {};
%%		\node [style=nothing] (1) at (-1.25, 0.5) {};
%%		\node [style=nothing] (2) at (-0.75, 0.5) {};
%%		\node [style=nothing] (3) at (-1.75, 2.5) {};
%%		\node [style=nothing] (4) at (-1.25, 2.5) {};
%%		\node [style=nothing] (5) at (-0.75, 2.5) {};
%%		\node [style=dot] (6) at (-1.75, 1.5) {};
%%		\node [style=dot] (7) at (-1.25, 1.5) {};
%%		\node [style=oplus] (8) at (-0.75, 1.5) {};
%%	\end{pgfonlayer}
%%	\begin{pgfonlayer}{edgelayer}
%%		\draw (0) to (6);
%%		\draw (6) to (3);
%%		\draw (4) to (7);
%%		\draw (7) to (1);
%%		\draw (2) to (8);
%%		\draw (8) to (5);
%%		\draw (8) to (7);
%%		\draw (7) to (6);
%%	\end{pgfonlayer}
%%\end{tikzpicture}
%%=
%%\begin{tikzpicture}
%%	\begin{pgfonlayer}{nodelayer}
%%		\node [style=nothing] (0) at (-1.75, 0.5) {};
%%		\node [style=nothing] (1) at (-1.25, 0.5) {};
%%		\node [style=nothing] (2) at (-0.75, 0.5) {};
%%		\node [style=dot] (3) at (-1.75, 1.5) {};
%%		\node [style=dot] (4) at (-1.25, 1.5) {};
%%		\node [style=oplus] (5) at (-0.75, 1.5) {};
%%		\node [style=nothing] (6) at (-1.75, 2.5) {};
%%		\node [style=nothing] (7) at (-1.25, 2.5) {};
%%		\node [style=nothing] (8) at (-0.75, 2.5) {};
%%	\end{pgfonlayer}
%%	\begin{pgfonlayer}{edgelayer}
%%		\draw [in=-90, out=90, looseness=1.25] (0) to (4);
%%		\draw [in=-90, out=90, looseness=1.25] (4) to (6);
%%		\draw [in=-90, out=90, looseness=1.25] (3) to (7);
%%		\draw [in=90, out=-90, looseness=1.25] (3) to (1);
%%		\draw (2) to (5);
%%		\draw (5) to (8);
%%		\draw (3) to (4);
%%		\draw (4) to (5);
%%	\end{pgfonlayer}
%%\end{tikzpicture}
%%$}
%%
%%\item
%%\label{TOF.16}
%%{\hfil
%%$
%%\begin{tikzpicture}
%%	\begin{pgfonlayer}{nodelayer}
%%		\node [style=nothing] (0) at (2.5, 0.5) {};
%%		\node [style=nothing] (1) at (1, 0.5) {};
%%		\node [style=nothing] (2) at (2, 0.5) {};
%%		\node [style=nothing] (3) at (0.5, 0.5) {};
%%		\node [style=X] (4) at (1.5, 0.75) {};
%%		\node [style=oplus] (5) at (1.5, 1.75) {};
%%		\node [style=oplus] (6) at (1.5, 2.75) {};
%%		\node [style=dot] (7) at (1.5, 2.25) {};
%%		\node [style=dot] (8) at (2, 2.25) {};
%%		\node [style=dot] (9) at (1, 1.75) {};
%%		\node [style=dot] (10) at (0.5, 1.75) {};
%%		\node [style=dot] (11) at (1, 2.75) {};
%%		\node [style=dot] (12) at (0.5, 2.75) {};
%%		\node [style=oplus] (13) at (2.5, 2.25) {};
%%		\node [style=X] (14) at (1.5, 3.75) {};
%%		\node [style=nothing] (15) at (2.5, 4) {};
%%		\node [style=nothing] (16) at (0.5, 4) {};
%%		\node [style=nothing] (17) at (1, 4) {};
%%		\node [style=nothing] (18) at (2, 4) {};
%%	\end{pgfonlayer}
%%	\begin{pgfonlayer}{edgelayer}
%%		\draw (3) to (10);
%%		\draw (10) to (12);
%%		\draw (12) to (16);
%%		\draw (11) to (9);
%%		\draw (15) to (13);
%%		\draw (13) to (0);
%%		\draw (13) to (8);
%%		\draw (8) to (7);
%%		\draw (9) to (5);
%%		\draw (9) to (10);
%%		\draw (12) to (11);
%%		\draw (6) to (11);
%%		\draw (4) to (5);
%%		\draw (5) to (7);
%%		\draw (7) to (6);
%%		\draw (14) to (6);
%%		\draw [style=simple, in=90, out=-90, looseness=1.25] (17) to (8);
%%		\draw [style=simple, in=90, out=-90, looseness=1.25] (8) to (1);
%%		\draw [style=simple, in=270, out=90] (2) to (9);
%%		\draw [style=simple, in=270, out=90] (11) to (18);
%%	\end{pgfonlayer}
%%\end{tikzpicture}
%%=
%%\begin{tikzpicture}
%%	\begin{pgfonlayer}{nodelayer}
%%		\node [style=nothing] (0) at (2.5, 0.5) {};
%%		\node [style=nothing] (1) at (2, 0.5) {};
%%		\node [style=nothing] (2) at (1, 0.5) {};
%%		\node [style=nothing] (3) at (0.5, 0.5) {};
%%		\node [style=X] (4) at (1.5, 1.25) {};
%%		\node [style=oplus] (5) at (1.5, 1.75) {};
%%		\node [style=oplus] (6) at (1.5, 2.75) {};
%%		\node [style=dot] (7) at (1.5, 2.25) {};
%%		\node [style=dot] (8) at (2, 2.25) {};
%%		\node [style=dot] (9) at (1, 1.75) {};
%%		\node [style=dot] (10) at (0.5, 1.75) {};
%%		\node [style=dot] (11) at (1, 2.75) {};
%%		\node [style=dot] (12) at (0.5, 2.75) {};
%%		\node [style=oplus] (13) at (2.5, 2.25) {};
%%		\node [style=X] (14) at (1.5, 3.25) {};
%%		\node [style=nothing] (15) at (2.5, 4) {};
%%		\node [style=nothing] (16) at (0.5, 4) {};
%%		\node [style=nothing] (17) at (2, 4) {};
%%		\node [style=nothing] (18) at (1, 4) {};
%%	\end{pgfonlayer}
%%	\begin{pgfonlayer}{edgelayer}
%%		\draw (3) to (10);
%%		\draw (10) to (12);
%%		\draw (12) to (16);
%%		\draw (11) to (9);
%%		\draw (15) to (13);
%%		\draw (13) to (0);
%%		\draw (13) to (8);
%%		\draw (8) to (7);
%%		\draw (9) to (5);
%%		\draw (9) to (10);
%%		\draw (12) to (11);
%%		\draw (6) to (11);
%%		\draw (4) to (5);
%%		\draw (5) to (7);
%%		\draw (7) to (6);
%%		\draw (14) to (6);
%%		\draw [style=simple] (17) to (8);
%%		\draw [style=simple] (8) to (1);
%%		\draw [style=simple] (2) to (9);
%%		\draw [style=simple] (11) to (18);
%%	\end{pgfonlayer}
%%\end{tikzpicture}
%%$}
%%\end{enumerate}
%%\end{multicols}
%%
%%Where the controlled-not gate is derived:
%%$$
%%\begin{tikzpicture}
%%	\begin{pgfonlayer}{nodelayer}
%%		\node [style=dot] (1) at (1.5, 1) {};
%%		\node [style=oplus] (2) at (2, 1) {};
%%		\node [style=none] (5) at (1.5, 0.25) {};
%%		\node [style=none] (6) at (2, 0.25) {};
%%		\node [style=none] (7) at (2, 1.75) {};
%%		\node [style=none] (8) at (1.5, 1.75) {};
%%	\end{pgfonlayer}
%%	\begin{pgfonlayer}{edgelayer}
%%		\draw (5.center) to (1);
%%		\draw (1) to (8.center);
%%		\draw (7.center) to (2);
%%		\draw (2) to (6.center);
%%		\draw (2) to (1);
%%	\end{pgfonlayer}
%%\end{tikzpicture}
%%:=
%%\begin{tikzpicture}
%%	\begin{pgfonlayer}{nodelayer}
%%		\node [style=dot] (0) at (1, 1) {};
%%		\node [style=dot] (1) at (1.5, 1) {};
%%		\node [style=oplus] (2) at (2, 1) {};
%%		\node [style=X] (3) at (1, 1.5) {$1$};
%%		\node [style=X] (4) at (1, 0.5) {$1$};
%%		\node [style=none] (5) at (1.5, 0.25) {};
%%		\node [style=none] (6) at (2, 0.25) {};
%%		\node [style=none] (7) at (2, 1.75) {};
%%		\node [style=none] (8) at (1.5, 1.75) {};
%%	\end{pgfonlayer}
%%	\begin{pgfonlayer}{edgelayer}
%%		\draw (5.center) to (1);
%%		\draw (1) to (8.center);
%%		\draw (7.center) to (2);
%%		\draw (2) to (6.center);
%%		\draw (2) to (1);
%%		\draw (1) to (0);
%%		\draw (0) to (3);
%%		\draw (4) to (0);
%%	\end{pgfonlayer}
%%\end{tikzpicture}
%%$$
%%
%%\subsubsection{$(\FPar_2,\times)$}
%%\label{subsubsec:presentations:three:par}
%%$(\FPar_2,\times)$ is presented by the generators and equations in \S \ref{subsubsec:presentations:two:par} as well as the additional generator $
%%\begin{tikzpicture}
%%	\begin{pgfonlayer}{nodelayer}
%%		\node [style=none] (0) at (-3.75, 0.5) {};
%%		\node [style=none] (1) at (-3.75, -0.25) {};
%%		\node [style=andin] (2) at (-3.75, -0.25) {};
%%		\node [style=none] (3) at (-4, -1) {};
%%		\node [style=none] (4) at (-3.5, -1) {};
%%	\end{pgfonlayer}
%%	\begin{pgfonlayer}{edgelayer}
%%		\draw (0.center) to (1.center);
%%		\draw [in=-60, out=90, looseness=1.00] (4.center) to (1.center);
%%		\draw [in=90, out=-120, looseness=1.00] (1.center) to (3.center);
%%	\end{pgfonlayer}
%%\end{tikzpicture}
%%$ (the and gate),  so that 
%%$
%%\left(
%%\begin{tikzpicture}
%%	\begin{pgfonlayer}{nodelayer}
%%		\node [style=none] (0) at (-3.75, 0.5) {};
%%		\node [style=none] (1) at (-3.75, -0.25) {};
%%		\node [style=andin] (2) at (-3.75, -0.25) {};
%%		\node [style=none] (3) at (-4, -1) {};
%%		\node [style=none] (4) at (-3.5, -1) {};
%%	\end{pgfonlayer}
%%	\begin{pgfonlayer}{edgelayer}
%%		\draw (0.center) to (1.center);
%%		\draw [in=-60, out=90, looseness=1.00] (4.center) to (1.center);
%%		\draw [in=90, out=-120, looseness=1.00] (1.center) to (3.center);
%%	\end{pgfonlayer}
%%\end{tikzpicture},
%%\begin{tikzpicture}
%%	\begin{pgfonlayer}{nodelayer}
%%		\node [style=none] (0) at (0.75, 0.5) {};
%%		\node [style=none] (1) at (0.75, -0.25) {};
%%		\node [style=X] (2) at (0.75, -0.25) {$1$};
%%	\end{pgfonlayer}
%%	\begin{pgfonlayer}{edgelayer}
%%		\draw (0.center) to (1.center);
%%	\end{pgfonlayer}
%%\end{tikzpicture},
%%\begin{tikzpicture}
%%	\begin{pgfonlayer}{nodelayer}
%%		\node [style=none] (0) at (0.75, -1) {};
%%		\node [style=none] (1) at (0.75, -0.25) {};
%%		\node [style=Z] (2) at (0.75, -0.25) {};
%%		\node [style=none] (3) at (0.5, 0.5) {};
%%		\node [style=none] (4) at (1, 0.5) {};
%%	\end{pgfonlayer}
%%	\begin{pgfonlayer}{edgelayer}
%%		\draw (0.center) to (1.center);
%%		\draw [in=60, out=-90] (4.center) to (1.center);
%%		\draw [in=-90, out=120] (1.center) to (3.center);
%%	\end{pgfonlayer}
%%\end{tikzpicture},
%%\begin{tikzpicture}
%%	\begin{pgfonlayer}{nodelayer}
%%		\node [style=none] (0) at (0.75, -0.25) {};
%%		\node [style=none] (1) at (0.75, 0.5) {};
%%		\node [style=Z] (2) at (0.75, 0.5) {};
%%	\end{pgfonlayer}
%%	\begin{pgfonlayer}{edgelayer}
%%		\draw (0.center) to (1.center);
%%	\end{pgfonlayer}
%%\end{tikzpicture}
%%\right)
%%$ forms a bicommutative bialgebra; and additionally:
%%$$
%%\begin{tikzpicture}
%%	\begin{pgfonlayer}{nodelayer}
%%		\node [style=none] (0) at (-7, 1) {};
%%		\node [style=none] (1) at (-7, 0.5) {};
%%		\node [style=Z] (2) at (-7, -0.25) {};
%%		\node [style=none] (3) at (-7, -0.75) {};
%%		\node [style=andin] (4) at (-7, 0.5) {};
%%	\end{pgfonlayer}
%%	\begin{pgfonlayer}{edgelayer}
%%		\draw (3.center) to (2.center);
%%		\draw [in=-60, out=60, looseness=1.25] (2.center) to (1);
%%		\draw [in=120, out=-120, looseness=1.25] (1) to (2.center);
%%		\draw (1) to (0.center);
%%	\end{pgfonlayer}
%%\end{tikzpicture}
%%\eref{antispecial}
%%\begin{tikzpicture}
%%	\begin{pgfonlayer}{nodelayer}
%%		\node [style=none] (0) at (-7, 1) {};
%%		\node [style=none] (1) at (-7, -0.75) {};
%%	\end{pgfonlayer}
%%	\begin{pgfonlayer}{edgelayer}
%%		\draw (1.center) to (0.center);
%%	\end{pgfonlayer}
%%\end{tikzpicture},
%%\hspace*{.5cm}
%%\begin{tikzpicture}
%%	\begin{pgfonlayer}{nodelayer}
%%		\node [style=andin] (4) at (1.25, 0.5) {};
%%		\node [style=X] (5) at (0.75, -0.5) {};
%%		\node [style=none] (6) at (0.5, -1) {};
%%		\node [style=none] (7) at (1, -1) {};
%%		\node [style=none] (8) at (1.75, -1) {};
%%		\node [style=none] (9) at (1.25, 0.5) {};
%%		\node [style=none] (10) at (1.25, 1.5) {};
%%	\end{pgfonlayer}
%%	\begin{pgfonlayer}{edgelayer}
%%		\draw [in=-30, out=90] (8.center) to (9.center);
%%		\draw [in=90, out=-150] (9.center) to (5);
%%		\draw [in=90, out=-45] (5) to (7.center);
%%		\draw [in=-135, out=90] (6.center) to (5);
%%		\draw (9.center) to (10.center);
%%	\end{pgfonlayer}
%%\end{tikzpicture}
%%  \eref{ring.mul}
%%\begin{tikzpicture}
%%	\begin{pgfonlayer}{nodelayer}
%%		\node [style=none] (0) at (1, 0) {};
%%		\node [style=none] (1) at (0.5, -1.25) {};
%%		\node [style=none] (2) at (1.75, -0.75) {};
%%		\node [style=none] (3) at (1.33, 0.75) {};
%%		\node [style=andin] (4) at (1, 0) {};
%%		\node [style=none] (5) at (1.75, 0) {};
%%		\node [style=none] (6) at (1, -1.25) {};
%%		\node [style=none] (7) at (1.75, -0.75) {};
%%		\node [style=none] (8) at (1.33, 0.75) {};
%%		\node [style=andin] (9) at (1.75, 0) {};
%%		\node [style=X] (10) at (1.33, 0.75) {};
%%		\node [style=none] (11) at (1.33, 1.25) {};
%%		\node [style=none] (12) at (1.75, -1.25) {};
%%		\node [style=Z] (13) at (1.75, -0.75) {};
%%	\end{pgfonlayer}
%%	\begin{pgfonlayer}{edgelayer}
%%		\draw [in=-135, out=90] (0.center) to (3.center);
%%		\draw [in=165, out=-30, looseness=1.25] (0.center) to (2.center);
%%		\draw [in=-45, out=90] (5.center) to (8.center);
%%		\draw [in=45, out=-45, looseness=1.25] (5.center) to (7.center);
%%		\draw (10) to (11.center);
%%		\draw [in=90, out=-150] (4) to (1.center);
%%		\draw [in=-150, out=90] (6.center) to (9);
%%		\draw (12.center) to (13);
%%	\end{pgfonlayer}
%%\end{tikzpicture},
%%\hspace*{.5cm}
%%\begin{tikzpicture}
%%	\begin{pgfonlayer}{nodelayer}
%%		\node [style=none] (0) at (2, 0) {};
%%		\node [style=none] (1) at (1.75, -0.75) {};
%%		\node [style=none] (2) at (2.25, -0.75) {};
%%		\node [style=none] (3) at (2, 0.5) {};
%%		\node [style=none] (4) at (2.25, -1) {};
%%		\node [style=X] (5) at (1.75, -0.75) {};
%%		\node [style=andin] (6) at (2, 0) {};
%%	\end{pgfonlayer}
%%	\begin{pgfonlayer}{edgelayer}
%%		\draw (0.center) to (3.center);
%%		\draw [in=90, out=-45] (0.center) to (2.center);
%%		\draw (4.center) to (2.center);
%%		\draw [in=-135, out=90] (1.center) to (0.center);
%%	\end{pgfonlayer}
%%\end{tikzpicture}
%%\eref{ring.unit}
%%\begin{tikzpicture}
%%	\begin{pgfonlayer}{nodelayer}
%%		\node [style=none] (12) at (2, 0.5) {};
%%		\node [style=none] (14) at (2, -1) {};
%%		\node [style=X] (15) at (2, 0) {};
%%		\node [style=Z] (16) at (2, -0.5) {};
%%	\end{pgfonlayer}
%%	\begin{pgfonlayer}{edgelayer}
%%		\draw (15) to (12.center);
%%		\draw (16) to (14.center);
%%	\end{pgfonlayer}
%%\end{tikzpicture},
%%\hspace*{.5cm}
%%\begin{tikzpicture}
%%	\begin{pgfonlayer}{nodelayer}
%%		\node [style=none] (0) at (0.75, 0.5) {};
%%		\node [style=none] (1) at (0.75, -0.25) {};
%%		\node [style=andin] (2) at (0.75, -0.25) {};
%%		\node [style=none] (3) at (0.5, -1) {};
%%		\node [style=none] (4) at (1, -1) {};
%%		\node [style=X] (5) at (0.75, 0.5) {$1$};
%%	\end{pgfonlayer}
%%	\begin{pgfonlayer}{edgelayer}
%%		\draw (0.center) to (1.center);
%%		\draw [in=-60, out=90] (4.center) to (1.center);
%%		\draw [in=90, out=-120] (1.center) to (3.center);
%%	\end{pgfonlayer}
%%\end{tikzpicture}
%%  \eref{bi.two}
%%\begin{tikzpicture}
%%	\begin{pgfonlayer}{nodelayer}
%%		\node [style=none] (3) at (0.5, -1) {};
%%		\node [style=none] (4) at (1, -1) {};
%%		\node [style=X] (5) at (0.5, 0.5) {$1$};
%%		\node [style=X] (6) at (1, 0.5) {$1$};
%%	\end{pgfonlayer}
%%	\begin{pgfonlayer}{edgelayer}
%%		\draw (3.center) to (5);
%%		\draw (6) to (4.center);
%%	\end{pgfonlayer}
%%\end{tikzpicture}
%%$$
%\subsubsection{$(\FSpan_2,\times)$}
%\label{subsubsec:presentations:three:span}
%
%$(\FSpan_2,\times)$ is presented by the generators and equations of \S \ref{subsubsec:presentations:three:span} as well as the generator 
%$\begin{tikzpicture}
%	\begin{pgfonlayer}{nodelayer}
%		\node [style=none] (0) at (0.75, 0.5) {};
%		\node [style=none] (1) at (0.75, -0.25) {};
%		\node [style=Z] (2) at (0.75, -0.25) {};
%	\end{pgfonlayer}
%	\begin{pgfonlayer}{edgelayer}
%		\draw (0.center) to (1.center);
%	\end{pgfonlayer}
%\end{tikzpicture}$ 
%and the equation making the codiagonal map counital:
%$$
%  \begin{tikzpicture}[rotate=90,yscale=-1]
%	\begin{pgfonlayer}{nodelayer}
%		\node [style=Z] (0) at (-9, -0) {};
%		\node [style=none] (1) at (-8.25, -0) {};
%		\node [style=Z] (2) at (-9.75, 0.25) {};
%		\node [style=none] (3) at (-10, -0.25) {};
%	\end{pgfonlayer}
%	\begin{pgfonlayer}{edgelayer}
%		\draw [in=-150, out=0, looseness=1.00] (3.center) to (0);
%		\draw [in=150, out=0, looseness=1.00] (2.center) to (0);
%		\draw (0) to (1.center);
%	\end{pgfonlayer}
%  \end{tikzpicture}
%  \eref{unit}
%  \begin{tikzpicture}[rotate=90]
%	\begin{pgfonlayer}{nodelayer}
%		\node [style=none] (0) at (-9, 0.25) {};
%		\node [style=none] (1) at (-9.75, 0.25) {};
%	\end{pgfonlayer}
%	\begin{pgfonlayer}{edgelayer}
%		\draw (1) to (0.center);
%	\end{pgfonlayer}
%  \end{tikzpicture}
%$$
%%
%%
%%\end{comment}




%
%This prop is equivalently presented in terms of the 
%
%\begin{figure}[H]
%	\noindent
%	\scalebox{1.0}{%
%		\vbox{%
%			\begin{mdframed}
%				\begin{multicols}{2}
%					\begin{enumerate}[label={\bf [CNOT.\arabic*]}, ref={\bf [CNOT.\arabic*]}, wide = 0pt, leftmargin = 2em]
%						\item
%						\label{CNOT.1}
%						{\hfil
%							$
%			\begin{tikzpicture}
%	\begin{pgfonlayer}{nodelayer}
%		\node [style=nothing] (26) at (0, 6) {};
%		\node [style=nothing] (27) at (-0.5, 6) {};
%		\node [style=oplus] (28) at (0, 6.5) {};
%		\node [style=dot] (29) at (-0.5, 6.5) {};
%		\node [style=dot] (30) at (0, 7) {};
%		\node [style=oplus] (31) at (-0.5, 7) {};
%		\node [style=oplus] (32) at (0, 7.5) {};
%		\node [style=dot] (33) at (-0.5, 7.5) {};
%		\node [style=nothing] (34) at (0, 8) {};
%		\node [style=nothing] (35) at (-0.5, 8) {};
%	\end{pgfonlayer}
%	\begin{pgfonlayer}{edgelayer}
%		\draw [style=simple] (26) to (34);
%		\draw [style=simple] (27) to (35);
%		\draw [style=simple] (28) to (29);
%		\draw [style=simple] (30) to (31);
%		\draw [style=simple] (32) to (33);
%	\end{pgfonlayer}
%\end{tikzpicture}
%							=
%							\begin{tikzpicture}
%	\begin{pgfonlayer}{nodelayer}
%		\node [style=nothing] (0) at (0, 0.5) {};
%		\node [style=nothing] (1) at (-0.5, 0.5) {};
%		\node [style=nothing] (2) at (-0.5, 1.5) {};
%		\node [style=nothing] (3) at (0, 1.5) {};
%	\end{pgfonlayer}
%	\begin{pgfonlayer}{edgelayer}
%		\draw [in=-90, out=90, looseness=1.25] (1) to (3);
%		\draw [in=-90, out=90, looseness=1.25] (0) to (2);
%	\end{pgfonlayer}
%\end{tikzpicture}
%$}
%						
%						
%						\item
%						\label{CNOT.2}
%						\hfil{
%							$
%							\begin{tikzpicture}
%	\begin{pgfonlayer}{nodelayer}
%		\node [style=nothing] (1) at (0, 0) {};
%		\node [style=nothing] (2) at (-0.5, 0) {};
%		\node [style=oplus] (3) at (0, 0.5) {};
%		\node [style=dot] (4) at (-0.5, 0.5) {};
%		\node [style=oplus] (5) at (0, 1) {};
%		\node [style=dot] (6) at (-0.5, 1) {};
%		\node [style=nothing] (7) at (0, 1.5) {};
%		\node [style=nothing] (8) at (-0.5, 1.5) {};
%	\end{pgfonlayer}
%	\begin{pgfonlayer}{edgelayer}
%		\draw [style=simple] (1) to (7);
%		\draw [style=simple] (2) to (8);
%		\draw [style=simple] (3) to (4);
%		\draw [style=simple] (5) to (6);
%	\end{pgfonlayer}
%\end{tikzpicture}
%							=
%							\begin{tikzpicture}
%	\begin{pgfonlayer}{nodelayer}
%		\node [style=nothing] (2) at (0, 0) {};
%		\node [style=nothing] (3) at (-0.5, 0) {};
%		\node [style=nothing] (4) at (0, 1.5) {};
%		\node [style=nothing] (5) at (-0.5, 1.5) {};
%	\end{pgfonlayer}
%	\begin{pgfonlayer}{edgelayer}
%		\draw [style=simple] (2) to (4);
%		\draw [style=simple] (3) to (5);
%	\end{pgfonlayer}
%\end{tikzpicture}
%							$}
%						
%						\item
%						\label{CNOT.3}
%						\hfil{
%							$
%							\begin{tikzpicture}
%	\begin{pgfonlayer}{nodelayer}
%		\node [style=nothing] (3) at (-1, 0) {};
%		\node [style=nothing] (4) at (-0.5, 0) {};
%		\node [style=nothing] (5) at (0, 0) {};
%		\node [style=oplus] (6) at (-1, 0.75) {};
%		\node [style=dot] (7) at (-0.5, 0.75) {};
%		\node [style=dot] (8) at (-0.5, 1.25) {};
%		\node [style=oplus] (9) at (0, 1.25) {};
%		\node [style=nothing] (10) at (-1, 2) {};
%		\node [style=nothing] (11) at (-0.5, 2) {};
%		\node [style=nothing] (12) at (0, 2) {};
%	\end{pgfonlayer}
%	\begin{pgfonlayer}{edgelayer}
%		\draw [style=simple] (3) to (10);
%		\draw [style=simple] (4) to (11);
%		\draw [style=simple] (5) to (12);
%		\draw [style=simple] (6) to (7);
%		\draw [style=simple] (8) to (9);
%	\end{pgfonlayer}
%\end{tikzpicture}
%							=
%							\begin{tikzpicture}
%	\begin{pgfonlayer}{nodelayer}
%		\node [style=nothing] (4) at (-1, 2.75) {};
%		\node [style=nothing] (5) at (-0.5, 2.75) {};
%		\node [style=nothing] (6) at (0, 2.75) {};
%		\node [style=oplus] (7) at (-1, 4) {};
%		\node [style=dot] (8) at (-0.5, 4) {};
%		\node [style=dot] (9) at (-0.5, 3.5) {};
%		\node [style=oplus] (10) at (0, 3.5) {};
%		\node [style=nothing] (11) at (-1, 4.75) {};
%		\node [style=nothing] (12) at (-0.5, 4.75) {};
%		\node [style=nothing] (13) at (0, 4.75) {};
%	\end{pgfonlayer}
%	\begin{pgfonlayer}{edgelayer}
%		\draw [style=simple] (4) to (11);
%		\draw [style=simple] (5) to (12);
%		\draw [style=simple] (6) to (13);
%		\draw [style=simple] (7) to (8);
%		\draw [style=simple] (9) to (10);
%	\end{pgfonlayer}
%\end{tikzpicture}
%							$}
%						
%						\item 
%						\label{CNOT.4}
%						\hfil{
%							\begin{tabular}{c}
%							$
%							\begin{tikzpicture}
%	\begin{pgfonlayer}{nodelayer}
%		\node [style=onein] (5) at (-0.5, 2.75) {};
%		\node [style=nothing] (6) at (0, 2.75) {};
%		\node [style=dot] (7) at (-0.5, 3.25) {};
%		\node [style=oplus] (8) at (0, 3.25) {};
%		\node [style=nothing] (9) at (-0.5, 3.75) {};
%		\node [style=nothing] (10) at (0, 3.75) {};
%	\end{pgfonlayer}
%	\begin{pgfonlayer}{edgelayer}
%		\draw [style=simple] (5) to (9);
%		\draw [style=simple] (6) to (10);
%		\draw [style=simple] (7) to (8);
%	\end{pgfonlayer}
%\end{tikzpicture}
%							=
%							\begin{tikzpicture}
%	\begin{pgfonlayer}{nodelayer}
%		\node [style=onein] (6) at (-0.5, 2.75) {};
%		\node [style=nothing] (7) at (0, 2.75) {};
%		\node [style=dot] (8) at (-0.5, 3.25) {};
%		\node [style=oplus] (9) at (0, 3.25) {};
%		\node [style=oneout] (10) at (-0.5, 3.75) {};
%		\node [style=nothing] (11) at (0, 4.75) {};
%		\node [style=onein] (12) at (-0.5, 4.25) {};
%		\node [style=nothing] (13) at (-0.5, 4.75) {};
%	\end{pgfonlayer}
%	\begin{pgfonlayer}{edgelayer}
%		\draw [style=simple] (6) to (10);
%		\draw [style=simple] (7) to (11);
%		\draw [style=simple] (8) to (9);
%		\draw [style=simple] (12) to (13);
%	\end{pgfonlayer}
%\end{tikzpicture}$\\
%							$ $\\
%							$\begin{tikzpicture}[tikzfig]
%	\begin{pgfonlayer}{nodelayer}
%		\node [style=nothing] (0) at (-0.5, 0.5) {};
%		\node [style=nothing] (1) at (0, 0.5) {};
%		\node [style=dot] (2) at (-0.5, 1) {};
%		\node [style=oplus] (3) at (0, 1) {};
%		\node [style=oneout] (4) at (-0.5, 1.5) {};
%		\node [style=nothing] (5) at (0, 1.5) {};
%	\end{pgfonlayer}
%	\begin{pgfonlayer}{edgelayer}
%		\draw [style=simple] (0) to (4);
%		\draw [style=simple] (1) to (5);
%		\draw [style=simple] (2) to (3);
%	\end{pgfonlayer}
%\end{tikzpicture}
%							=
%							\begin{tikzpicture}
%	\begin{pgfonlayer}{nodelayer}
%		\node [style=oneout] (8) at (-0.5, 7.25) {};
%		\node [style=nothing] (9) at (0, 7.25) {};
%		\node [style=dot] (10) at (-0.5, 6.75) {};
%		\node [style=oplus] (11) at (0, 6.75) {};
%		\node [style=onein] (12) at (-0.5, 6.25) {};
%		\node [style=nothing] (13) at (0, 5.25) {};
%		\node [style=oneout] (14) at (-0.5, 5.75) {};
%		\node [style=nothing] (15) at (-0.5, 5.25) {};
%	\end{pgfonlayer}
%	\begin{pgfonlayer}{edgelayer}
%		\draw [style=simple] (8) to (12);
%		\draw [style=simple] (9) to (13);
%		\draw [style=simple] (10) to (11);
%		\draw [style=simple] (14) to (15);
%	\end{pgfonlayer}
%\end{tikzpicture}$
%							\end{tabular}
%							}
%						
%						\item 
%						\label{CNOT.5}
%						\hfil{
%							$
%							\begin{tikzpicture}
%	\begin{pgfonlayer}{nodelayer}
%		\node [style=nothing] (9) at (-1, 5.25) {};
%		\node [style=nothing] (10) at (-0.5, 5.25) {};
%		\node [style=nothing] (11) at (0, 5.25) {};
%		\node [style=dot] (12) at (-1, 6) {};
%		\node [style=oplus] (13) at (-0.5, 6) {};
%		\node [style=oplus] (14) at (-0.5, 6.5) {};
%		\node [style=dot] (15) at (0, 6.5) {};
%		\node [style=nothing] (16) at (-1, 7.25) {};
%		\node [style=nothing] (17) at (-0.5, 7.25) {};
%		\node [style=nothing] (18) at (0, 7.25) {};
%	\end{pgfonlayer}
%	\begin{pgfonlayer}{edgelayer}
%		\draw [style=simple] (9) to (16);
%		\draw [style=simple] (10) to (17);
%		\draw [style=simple] (11) to (18);
%		\draw [style=simple] (12) to (13);
%		\draw [style=simple] (14) to (15);
%	\end{pgfonlayer}
%\end{tikzpicture}
%							=
%							\begin{tikzpicture}
%	\begin{pgfonlayer}{nodelayer}
%		\node [style=nothing] (10) at (-1, 5.25) {};
%		\node [style=nothing] (11) at (-0.5, 5.25) {};
%		\node [style=nothing] (12) at (0, 5.25) {};
%		\node [style=dot] (13) at (-1, 6.5) {};
%		\node [style=oplus] (14) at (-0.5, 6.5) {};
%		\node [style=oplus] (15) at (-0.5, 6) {};
%		\node [style=dot] (16) at (0, 6) {};
%		\node [style=nothing] (17) at (-1, 7.25) {};
%		\node [style=nothing] (18) at (-0.5, 7.25) {};
%		\node [style=nothing] (19) at (0, 7.25) {};
%	\end{pgfonlayer}
%	\begin{pgfonlayer}{edgelayer}
%		\draw [style=simple] (10) to (17);
%		\draw [style=simple] (11) to (18);
%		\draw [style=simple] (12) to (19);
%		\draw [style=simple] (13) to (14);
%		\draw [style=simple] (15) to (16);
%	\end{pgfonlayer}
%\end{tikzpicture}
%							$}
%						
%						\item 
%						\label{CNOT.6}
%						\hfil{
%							$
%							\begin{tikzpicture}
%	\begin{pgfonlayer}{nodelayer}
%		\node [style=onein] (11) at (0, 5.25) {};
%		\node [style=oneout] (12) at (0, 6.25) {};
%	\end{pgfonlayer}
%	\begin{pgfonlayer}{edgelayer}
%		\draw [style=simple] (11) to (12);
%	\end{pgfonlayer}
%\end{tikzpicture}
%							=
%							\begin{tikzpicture}
%	\begin{pgfonlayer}{nodelayer}
%		\node [style=rn] (12) at (0, 5.25) {};
%		\node [style=rn] (13) at (0, 6.25) {};
%	\end{pgfonlayer}
%\end{tikzpicture}
%							$}
%						
%						\item 
%						\label{CNOT.7}
%						\hfil{
%							\begin{tabular}{c}
%							$\begin{tikzpicture}
%	\begin{pgfonlayer}{nodelayer}
%		\node [style=onein] (13) at (-1, 5.25) {};
%		\node [style=onein] (14) at (-0.5, 5.25) {};
%		\node [style=nothing] (15) at (0, 5.25) {};
%		\node [style=dot] (16) at (-1, 5.75) {};
%		\node [style=oplus] (17) at (-0.5, 5.75) {};
%		\node [style=dot] (18) at (-0.5, 6.25) {};
%		\node [style=oplus] (19) at (0, 6.25) {};
%		\node [style=oneout] (20) at (-1, 6.25) {};
%		\node [style=nothing] (21) at (-0.5, 6.75) {};
%		\node [style=nothing] (22) at (0, 6.75) {};
%	\end{pgfonlayer}
%	\begin{pgfonlayer}{edgelayer}
%		\draw [style=simple] (13) to (20);
%		\draw [style=simple] (14) to (21);
%		\draw [style=simple] (15) to (22);
%		\draw [style=simple] (16) to (17);
%		\draw [style=simple] (18) to (19);
%	\end{pgfonlayer}
%\end{tikzpicture}
%							=
%							\begin{tikzpicture}
%	\begin{pgfonlayer}{nodelayer}
%		\node [style=onein] (14) at (-1, 5.25) {};
%		\node [style=onein] (15) at (-0.5, 5.25) {};
%		\node [style=nothing] (16) at (0, 5.25) {};
%		\node [style=dot] (17) at (-1, 5.75) {};
%		\node [style=oplus] (18) at (-0.5, 5.75) {};
%		\node [style=oneout] (19) at (-1, 6.25) {};
%		\node [style=nothing] (20) at (-0.5, 6.75) {};
%		\node [style=nothing] (21) at (0, 6.75) {};
%	\end{pgfonlayer}
%	\begin{pgfonlayer}{edgelayer}
%		\draw [style=simple] (14) to (19);
%		\draw [style=simple] (15) to (20);
%		\draw [style=simple] (16) to (21);
%		\draw [style=simple] (17) to (18);
%	\end{pgfonlayer}
%\end{tikzpicture}$\\
%							$ $\\
%							$\begin{tikzpicture}
%	\begin{pgfonlayer}{nodelayer}
%		\node [style=oneout] (15) at (-1, 6.75) {};
%		\node [style=oneout] (16) at (-0.5, 6.75) {};
%		\node [style=nothing] (17) at (0, 6.75) {};
%		\node [style=dot] (18) at (-1, 6.25) {};
%		\node [style=oplus] (19) at (-0.5, 6.25) {};
%		\node [style=dot] (20) at (-0.5, 5.75) {};
%		\node [style=oplus] (21) at (0, 5.75) {};
%		\node [style=onein] (22) at (-1, 5.75) {};
%		\node [style=nothing] (23) at (-0.5, 5.25) {};
%		\node [style=nothing] (24) at (0, 5.25) {};
%	\end{pgfonlayer}
%	\begin{pgfonlayer}{edgelayer}
%		\draw [style=simple] (15) to (22);
%		\draw [style=simple] (16) to (23);
%		\draw [style=simple] (17) to (24);
%		\draw [style=simple] (18) to (19);
%		\draw [style=simple] (20) to (21);
%	\end{pgfonlayer}
%\end{tikzpicture}
%							=
%							\begin{tikzpicture}
%	\begin{pgfonlayer}{nodelayer}
%		\node [style=oneout] (16) at (-1, 6.75) {};
%		\node [style=oneout] (17) at (-0.5, 6.75) {};
%		\node [style=nothing] (18) at (0, 6.75) {};
%		\node [style=dot] (19) at (-1, 6.25) {};
%		\node [style=oplus] (20) at (-0.5, 6.25) {};
%		\node [style=onein] (21) at (-1, 5.75) {};
%		\node [style=nothing] (22) at (-0.5, 5.25) {};
%		\node [style=nothing] (23) at (0, 5.25) {};
%	\end{pgfonlayer}
%	\begin{pgfonlayer}{edgelayer}
%		\draw [style=simple] (16) to (21);
%		\draw [style=simple] (17) to (22);
%		\draw [style=simple] (18) to (23);
%		\draw [style=simple] (19) to (20);
%	\end{pgfonlayer}
%\end{tikzpicture}$
%							\end{tabular}
%							}
%						
%						\item 
%						\label{CNOT.8}
%						\hfil{
%							$
%							\begin{tikzpicture}
%	\begin{pgfonlayer}{nodelayer}
%		\node [style=nothing] (17) at (-1, 5.25) {};
%		\node [style=nothing] (18) at (-0.5, 5.25) {};
%		\node [style=nothing] (19) at (0, 5.25) {};
%		\node [style=dot] (20) at (-1, 5.75) {};
%		\node [style=oplus] (21) at (-0.5, 5.75) {};
%		\node [style=dot] (22) at (-0.5, 6.25) {};
%		\node [style=oplus] (23) at (0, 6.25) {};
%		\node [style=dot] (24) at (-1, 6.75) {};
%		\node [style=oplus] (25) at (-0.5, 6.75) {};
%		\node [style=nothing] (26) at (-1, 7.25) {};
%		\node [style=nothing] (27) at (-0.5, 7.25) {};
%		\node [style=nothing] (28) at (0, 7.25) {};
%	\end{pgfonlayer}
%	\begin{pgfonlayer}{edgelayer}
%		\draw [style=simple] (17) to (26);
%		\draw [style=simple] (18) to (27);
%		\draw [style=simple] (19) to (28);
%		\draw [style=simple] (20) to (21);
%		\draw [style=simple] (22) to (23);
%		\draw [style=simple] (24) to (25);
%	\end{pgfonlayer}
%\end{tikzpicture}
%							=
%							\begin{tikzpicture}
%	\begin{pgfonlayer}{nodelayer}
%		\node [style=nothing] (18) at (-1, 5.25) {};
%		\node [style=nothing] (19) at (-0.5, 5.25) {};
%		\node [style=nothing] (20) at (0, 5.25) {};
%		\node [style=dot] (21) at (-0.5, 5.75) {};
%		\node [style=oplus] (22) at (0, 5.75) {};
%		\node [style=dot] (23) at (-1, 6.25) {};
%		\node [style=oplus] (24) at (0, 6.25) {};
%		\node [style=nothing] (25) at (-1, 6.75) {};
%		\node [style=nothing] (26) at (-0.5, 6.75) {};
%		\node [style=nothing] (27) at (0, 6.75) {};
%	\end{pgfonlayer}
%	\begin{pgfonlayer}{edgelayer}
%		\draw [style=simple] (18) to (25);
%		\draw [style=simple] (19) to (26);
%		\draw [style=simple] (20) to (27);
%		\draw [style=simple] (21) to (22);
%		\draw [style=simple] (23) to (24);
%	\end{pgfonlayer}
%\end{tikzpicture}
%							$}
%						
%						\item 
%						\label{CNOT.9}
%						\hfil{
%							$
%							\begin{tikzpicture}
%	\begin{pgfonlayer}{nodelayer}
%		\node [style=onein] (19) at (-1, 5.25) {};
%		\node [style=onein] (20) at (-0.5, 5.25) {};
%		\node [style=nothing] (21) at (0, 5.25) {};
%		\node [style=dot] (22) at (-1, 5.75) {};
%		\node [style=oplus] (23) at (-0.5, 5.75) {};
%		\node [style=oneout] (24) at (-1, 6.25) {};
%		\node [style=oneout] (25) at (-0.5, 6.25) {};
%		\node [style=nothing] (26) at (0, 6.25) {};
%	\end{pgfonlayer}
%	\begin{pgfonlayer}{edgelayer}
%		\draw [style=simple] (19) to (24);
%		\draw [style=simple] (20) to (25);
%		\draw [style=simple] (21) to (26);
%		\draw [style=simple] (22) to (23);
%	\end{pgfonlayer}
%\end{tikzpicture}
%							=
%							\begin{tikzpicture}
%	\begin{pgfonlayer}{nodelayer}
%		\node [style=onein] (20) at (-1, 5.25) {};
%		\node [style=onein] (21) at (-0.5, 5.25) {};
%		\node [style=nothing] (22) at (0, 5.25) {};
%		\node [style=dot] (23) at (-1, 6.25) {};
%		\node [style=oplus] (24) at (-0.5, 6.25) {};
%		\node [style=oneout] (25) at (-1, 7.25) {};
%		\node [style=oneout] (26) at (-0.5, 7.25) {};
%		\node [style=nothing] (27) at (0, 7.25) {};
%		\node [style=oneout] (28) at (0, 6) {};
%		\node [style=onein] (29) at (0, 6.5) {};
%	\end{pgfonlayer}
%	\begin{pgfonlayer}{edgelayer}
%		\draw [style=simple] (20) to (25);
%		\draw [style=simple] (21) to (26);
%		\draw [style=simple] (22) to (28);
%		\draw [style=simple] (29) to (27);
%		\draw [style=simple] (23) to (24);
%	\end{pgfonlayer}
%\end{tikzpicture}
%							$}
%					\end{enumerate}
%				\end{multicols}
%				\
%			\end{mdframed}
%	}}
%	\caption{The identities of \texorpdfstring{$\CNOT$}{CNOT}}
%	\label{fig:CNOT}
%\end{figure}

