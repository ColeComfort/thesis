In this section, we modularly build up to the prop $\ZXA$ by taking distributive laws and pushouts of smaller symmetric monoidal theories.  Along the way, we obtain various fragments of quantum circuits with partial, reversible and partially reversible semantics.


We use the machinery of distributive laws of monads in $\Prof(\Mon)^\op$ discussed in Subsection \ref{subsec:internal}.  In the literature, props are usually decomposed according to orthogonal factorization systems, or more generally factorization systems over subgroupoids.  However, we have to work in the more general setting of a factorization system over subcategories of subobjects.  As discussed in Definition \ref{def:zigza} and Lemma \ref{def:zigza}; the mathematical machinery already exists to do this.  However, the original motivation was to capture distributive law of Lawveret theories and had nothing to do with subobjects \cite{lawvere}.

We hope that the techniques used in this section can lead to presentations of other full subcategories of $\Mat_\N$ and $\Mat_\B$.  We hope that this can help eventually prove the completeness of qudit fragments of the ZH-calculus.

Because we want to compose everything using distributive laws and pushouts, note that the counital completion of a discrete inverse prop $\X$ is the following pushout of props:
$$
\X \leftarrow  \surj^\op \rightarrow \cm^\op
$$
That is to say, it picks out the diagonal map of $\X$ and adds a counit.  As a matter of notation, as we have been doing throughout this thesis, we will colour this comonoid and its components $\zcirc$.

We recall some notation all the way back from Subsection \ref{subsec:spanrel}.  $\Inv(\X)$ denotes  the subcategory of partial isomorphisms of a restriction category.  $\Iso(\X)$ denotes the category of isomorphims in a category $\X$.  $\Par(\X)$ denotes the partial map category of $\X$.  $\ParIso(\X)$ denotes the inverse category of spans of monorphisms in $\X$.  Moreover, denote the  category of monomorphisms in $\X$ by $\sfmono(\X)$.
\subsection{The phase-free fragment}
\label{sec:one}
In this subsection we build up to giving a presentation for $(\Span^\sim(\Mat_{\F_2}),\oplus)$ in a modular way. This category is shown to be the same as the phase-free of the ZX-calculus on the nose (not just up to invertible scalars). Note that a presentation for the full category of linear spans has already been discussed in great detail  for arbitrary PIDs \cite{ih}.


First, we construct the linear isomorphisms:
\begin{definition}
Consider the prop $\sfiso\cb_{\F_2}$ generated by the controlled not gate modulo the following relations:
$$
\begin{tikzpicture}
	\begin{pgfonlayer}{nodelayer}
		\node [style=dot] (0) at (7, 0) {};
		\node [style=oplus] (1) at (7.5, 0) {};
		\node [style=oplus] (2) at (7.5, 0.5) {};
		\node [style=dot] (3) at (7, 0.5) {};
		\node [style=none] (4) at (7.5, 1) {};
		\node [style=none] (5) at (7, 1) {};
		\node [style=none] (6) at (7, -0.5) {};
		\node [style=none] (7) at (7.5, -0.5) {};
	\end{pgfonlayer}
	\begin{pgfonlayer}{edgelayer}
		\draw (5.center) to (3);
		\draw (3) to (0);
		\draw (0) to (6.center);
		\draw (7.center) to (1);
		\draw (1) to (2);
		\draw (2) to (4.center);
		\draw (2) to (3);
		\draw (0) to (1);
	\end{pgfonlayer}
\end{tikzpicture}
\eqzxa{cnot.one}
\begin{tikzpicture}
	\begin{pgfonlayer}{nodelayer}
		\node [style=none] (4) at (7.5, 1) {};
		\node [style=none] (5) at (7, 1) {};
		\node [style=none] (6) at (7, -0.5) {};
		\node [style=none] (7) at (7.5, -0.5) {};
	\end{pgfonlayer}
	\begin{pgfonlayer}{edgelayer}
		\draw (7.center) to (4.center);
		\draw (5.center) to (6.center);
	\end{pgfonlayer}
\end{tikzpicture}
\hspace*{.5cm}
\begin{tikzpicture}
	\begin{pgfonlayer}{nodelayer}
		\node [style=none] (4) at (7.5, 1) {};
		\node [style=none] (5) at (7, 1) {};
		\node [style=none] (6) at (7, -1) {};
		\node [style=none] (7) at (7.5, -1) {};
		\node [style=dot] (8) at (7.5, 0.5) {};
		\node [style=dot] (9) at (7.5, -0.5) {};
		\node [style=dot] (10) at (7, 0) {};
		\node [style=oplus] (11) at (7.5, 0) {};
		\node [style=oplus] (12) at (7, 0.5) {};
		\node [style=oplus] (13) at (7, -0.5) {};
	\end{pgfonlayer}
	\begin{pgfonlayer}{edgelayer}
		\draw (7.center) to (4.center);
		\draw (5.center) to (6.center);
		\draw (8) to (12);
		\draw (10) to (11);
		\draw (9) to (13);
	\end{pgfonlayer}
\end{tikzpicture}
\eqzxa{cnot.two}
\begin{tikzpicture}
	\begin{pgfonlayer}{nodelayer}
		\node [style=none] (4) at (7, 1) {};
		\node [style=none] (5) at (7.5, 1) {};
		\node [style=none] (6) at (7, -1) {};
		\node [style=none] (7) at (7.5, -1) {};
	\end{pgfonlayer}
	\begin{pgfonlayer}{edgelayer}
		\draw [in=270, out=90] (7.center) to (4.center);
		\draw [in=90, out=-90] (5.center) to (6.center);
	\end{pgfonlayer}
\end{tikzpicture}
\hspace*{.5cm}
\begin{tikzpicture}
	\begin{pgfonlayer}{nodelayer}
		\node [style=dot] (0) at (8, 0) {};
		\node [style=dot] (1) at (8.5, 0.5) {};
		\node [style=dot] (2) at (8.5, -0.5) {};
		\node [style=oplus] (3) at (8.5, 0) {};
		\node [style=oplus] (4) at (9, 0.5) {};
		\node [style=oplus] (5) at (9, -0.5) {};
		\node [style=none] (6) at (9, -1) {};
		\node [style=none] (7) at (8.5, -1) {};
		\node [style=none] (8) at (8, -1) {};
		\node [style=none] (9) at (8, 1) {};
		\node [style=none] (10) at (8.5, 1) {};
		\node [style=none] (11) at (9, 1) {};
	\end{pgfonlayer}
	\begin{pgfonlayer}{edgelayer}
		\draw (9.center) to (8.center);
		\draw (7.center) to (10.center);
		\draw (11.center) to (6.center);
		\draw (5) to (2);
		\draw (3) to (0);
		\draw (1) to (4);
	\end{pgfonlayer}
\end{tikzpicture}
\eqzxa{cnot.three}
\begin{tikzpicture}
	\begin{pgfonlayer}{nodelayer}
		\node [style=dot] (0) at (8, -0.25) {};
		\node [style=dot] (1) at (8, 0.25) {};
		\node [style=oplus] (3) at (8.5, -0.25) {};
		\node [style=oplus] (4) at (9, 0.25) {};
		\node [style=none] (6) at (9, -1) {};
		\node [style=none] (7) at (8.5, -1) {};
		\node [style=none] (8) at (8, -1) {};
		\node [style=none] (9) at (8, 1) {};
		\node [style=none] (10) at (8.5, 1) {};
		\node [style=none] (11) at (9, 1) {};
	\end{pgfonlayer}
	\begin{pgfonlayer}{edgelayer}
		\draw (9.center) to (8.center);
		\draw (7.center) to (10.center);
		\draw (11.center) to (6.center);
		\draw (3) to (0);
		\draw (1) to (4);
	\end{pgfonlayer}
\end{tikzpicture}
\hspace*{.5cm}
\begin{tikzpicture}
	\begin{pgfonlayer}{nodelayer}
		\node [style=dot] (0) at (2, 1.5) {};
		\node [style=oplus] (1) at (2.5, 1.5) {};
		\node [style=none] (3) at (2, 1) {};
		\node [style=none] (4) at (2.5, 1) {};
		\node [style=none] (5) at (2, 2.5) {};
		\node [style=none] (6) at (2.5, 2.5) {};
		\node [style=none] (7) at (3, 2.5) {};
		\node [style=none] (8) at (3, 1) {};
		\node [style=dot] (9) at (3, 2) {};
		\node [style=oplus] (10) at (2.5, 2) {};
	\end{pgfonlayer}
	\begin{pgfonlayer}{edgelayer}
		\draw (0) to (5.center);
		\draw (6.center) to (1);
		\draw (1) to (0);
		\draw (4.center) to (1);
		\draw (3.center) to (0);
		\draw (10) to (9);
		\draw (8.center) to (7.center);
	\end{pgfonlayer}
\end{tikzpicture}
\eqzxa{cnot.four}
\begin{tikzpicture}
	\begin{pgfonlayer}{nodelayer}
		\node [style=dot] (0) at (2, 2) {};
		\node [style=oplus] (1) at (2.5, 2) {};
		\node [style=none] (3) at (2, 2.5) {};
		\node [style=none] (4) at (2.5, 2.5) {};
		\node [style=none] (5) at (2, 1) {};
		\node [style=none] (6) at (2.5, 1) {};
		\node [style=none] (7) at (3, 1) {};
		\node [style=none] (8) at (3, 2.5) {};
		\node [style=dot] (9) at (3, 1.5) {};
		\node [style=oplus] (10) at (2.5, 1.5) {};
	\end{pgfonlayer}
	\begin{pgfonlayer}{edgelayer}
		\draw (0) to (5.center);
		\draw (6.center) to (1);
		\draw (1) to (0);
		\draw (4.center) to (1);
		\draw (3.center) to (0);
		\draw (10) to (9);
		\draw (8.center) to (7.center);
	\end{pgfonlayer}
\end{tikzpicture}
\hspace*{.5cm}
\begin{tikzpicture}
	\begin{pgfonlayer}{nodelayer}
		\node [style=oplus] (0) at (2, 1.5) {};
		\node [style=dot] (1) at (2.5, 1.5) {};
		\node [style=none] (3) at (2, 1) {};
		\node [style=none] (4) at (2.5, 1) {};
		\node [style=none] (5) at (2, 2.5) {};
		\node [style=none] (6) at (2.5, 2.5) {};
		\node [style=none] (7) at (3, 2.5) {};
		\node [style=none] (8) at (3, 1) {};
		\node [style=oplus] (9) at (3, 2) {};
		\node [style=dot] (10) at (2.5, 2) {};
	\end{pgfonlayer}
	\begin{pgfonlayer}{edgelayer}
		\draw (0) to (5.center);
		\draw (6.center) to (1);
		\draw (1) to (0);
		\draw (4.center) to (1);
		\draw (3.center) to (0);
		\draw (10) to (9);
		\draw (8.center) to (7.center);
	\end{pgfonlayer}
\end{tikzpicture}
\eqzxa{cnot.five}
\begin{tikzpicture}
	\begin{pgfonlayer}{nodelayer}
		\node [style=oplus] (0) at (2, 2) {};
		\node [style=dot] (1) at (2.5, 2) {};
		\node [style=none] (3) at (2, 2.5) {};
		\node [style=none] (4) at (2.5, 2.5) {};
		\node [style=none] (5) at (2, 1) {};
		\node [style=none] (6) at (2.5, 1) {};
		\node [style=none] (7) at (3, 1) {};
		\node [style=none] (8) at (3, 2.5) {};
		\node [style=oplus] (9) at (3, 1.5) {};
		\node [style=dot] (10) at (2.5, 1.5) {};
	\end{pgfonlayer}
	\begin{pgfonlayer}{edgelayer}
		\draw (0) to (5.center);
		\draw (6.center) to (1);
		\draw (1) to (0);
		\draw (4.center) to (1);
		\draw (3.center) to (0);
		\draw (10) to (9);
		\draw (8.center) to (7.center);
	\end{pgfonlayer}
\end{tikzpicture}
$$
\end{definition}
\begin{lemma}[{\cite[Thm. 6]{lafont}}]
$\sfiso\cb_{\F_2}$ is a presentation for the prop $(\Iso(\Mat_{\F_2}),\oplus)$ with respect to the interpretation:
$$
\left\llbracket
\begin{tikzpicture}
	\begin{pgfonlayer}{nodelayer}
		\node [style=oplus] (5) at (0.5, 2.75) {};
		\node [style=dot] (6) at (0, 2.75) {};
		\node [style=none] (7) at (0.5, 3.5) {};
		\node [style=none] (8) at (0.5, 2) {};
		\node [style=none] (9) at (0, 2) {};
		\node [style=none] (10) at (0, 3.5) {};
	\end{pgfonlayer}
	\begin{pgfonlayer}{edgelayer}
		\draw (8.center) to (5);
		\draw (5) to (7.center);
		\draw (10.center) to (6);
		\draw (6) to (5);
		\draw (6) to (9.center);
	\end{pgfonlayer}
\end{tikzpicture}
\right\rrbracket
=
\begin{tikzpicture}
	\begin{pgfonlayer}{nodelayer}
		\node [style=X] (0) at (-0.25, -1) {};
		\node [style=Z] (1) at (-0.75, -1.75) {};
		\node [style=none] (2) at (0, -2.25) {};
		\node [style=none] (3) at (-0.25, -0.5) {};
		\node [style=none] (4) at (-1, -0.5) {};
		\node [style=none] (5) at (-0.75, -2.25) {};
	\end{pgfonlayer}
	\begin{pgfonlayer}{edgelayer}
		\draw (3.center) to (0);
		\draw [in=90, out=-75] (0) to (2.center);
		\draw (5.center) to (1);
		\draw (1) to (0);
		\draw [in=-90, out=105] (1) to (4.center);
	\end{pgfonlayer}
\end{tikzpicture} 
$$
\end{lemma}
In fact, this result can be generalized to an arbitrary field \cite[Figure 37]{lafont}.


Adding the $|0\rangle$ state yields linear injections:
\begin{definition}
Consider the prop $\sfinj\cb_{\F_2}$ generated by the coproduct of props $\sfiso\cb_{\F_2}+\inj$ modulo the equation:
\hspace*{1cm}
$
\begin{tikzpicture}
	\begin{pgfonlayer}{nodelayer}
		\node [style=X] (0) at (0, 0) {};
		\node [style=dot] (1) at (0, 0.5) {};
		\node [style=oplus] (2) at (0.5, 0.5) {};
		\node [style=none] (3) at (0.5, -0.25) {};
		\node [style=none] (4) at (0.5, 1) {};
		\node [style=none] (5) at (0, 1) {};
	\end{pgfonlayer}
	\begin{pgfonlayer}{edgelayer}
		\draw (0) to (1);
		\draw (1) to (5.center);
		\draw (1) to (2);
		\draw (2) to (4.center);
		\draw (3.center) to (2);
	\end{pgfonlayer}
\end{tikzpicture}
\eqzxa{cnot.six}
\begin{tikzpicture}
	\begin{pgfonlayer}{nodelayer}
		\node [style=X] (0) at (0, 0) {};
		\node [style=none] (3) at (0.5, -0.25) {};
		\node [style=none] (4) at (0.5, 0.5) {};
		\node [style=none] (5) at (0, 0.5) {};
	\end{pgfonlayer}
	\begin{pgfonlayer}{edgelayer}
		\draw (0) to (5.center);
		\draw (3.center) to (4.center);
	\end{pgfonlayer}
\end{tikzpicture}
$
\end{definition}
\begin{lemma}[{\cite[Thm. 7]{lafont}}]
$\sfinj\cb_{\F_2}$ is a presentation for the prop $(\sfmono(\Mat_{\F_2}),\oplus)$
\end{lemma}
The white comultiplication can be derived in this fragment:
$$
\left\llbracket
\begin{tikzpicture}
	\begin{pgfonlayer}{nodelayer}
		\node [style=oplus] (0) at (1, 2.75) {};
		\node [style=dot] (1) at (0.5, 2.75) {};
		\node [style=none] (2) at (1, 3.5) {};
		\node [style=none] (3) at (1, 2.25) {};
		\node [style=none] (4) at (0.5, 2) {};
		\node [style=none] (5) at (0.5, 3.5) {};
		\node [style=X] (6) at (1, 2.25) {};
	\end{pgfonlayer}
	\begin{pgfonlayer}{edgelayer}
		\draw (3.center) to (0);
		\draw (0) to (2.center);
		\draw (5.center) to (1);
		\draw (1) to (0);
		\draw (1) to (4.center);
	\end{pgfonlayer}
\end{tikzpicture}
\right\rrbracket
=
\begin{tikzpicture}
	\begin{pgfonlayer}{nodelayer}
		\node [style=X] (0) at (1.25, -1) {};
		\node [style=Z] (1) at (0.75, -1.75) {};
		\node [style=none] (2) at (1.5, -2) {};
		\node [style=none] (3) at (1.25, -0.5) {};
		\node [style=none] (4) at (0.5, -0.5) {};
		\node [style=none] (5) at (0.75, -2.25) {};
		\node [style=X] (6) at (1.5, -2) {};
	\end{pgfonlayer}
	\begin{pgfonlayer}{edgelayer}
		\draw (3.center) to (0);
		\draw [in=90, out=-75] (0) to (2.center);
		\draw (5.center) to (1);
		\draw (1) to (0);
		\draw [in=-90, out=105] (1) to (4.center);
	\end{pgfonlayer}
\end{tikzpicture}
=
\begin{tikzpicture}
	\begin{pgfonlayer}{nodelayer}
		\node [style=Z] (1) at (0.75, -1.75) {};
		\node [style=none] (3) at (1, -1) {};
		\node [style=none] (4) at (0.5, -1) {};
		\node [style=none] (5) at (0.75, -2.25) {};
	\end{pgfonlayer}
	\begin{pgfonlayer}{edgelayer}
		\draw (5.center) to (1);
		\draw [in=-90, out=120] (1) to (4.center);
		\draw [in=-90, out=60] (1) to (3.center);
	\end{pgfonlayer}
\end{tikzpicture}
$$
%This is to be expected because there is a faithful ``'graph functor' monoidal functor $(\sfmono(\FinSet),+) \to (\Inj(\Span(\FinSet)),+)$.
By adding the effect $\langle 0 |$ we get linear partially reversible semantics:
\begin{lemma}
\label{def:pariso:cb}
There is a distributive law of props:
$$
\sfpariso\cb_{\F_2}:=\sfinj\cb_{\F_2}^\op \otimes_{\sfiso\cb_{\F_2}} \sfinj\cb_{\F_2};
\begin{tikzpicture}
	\begin{pgfonlayer}{nodelayer}
		\node [style=X] (0) at (0, 0) {};
		\node [style=X] (1) at (0, 0.75) {};
	\end{pgfonlayer}
	\begin{pgfonlayer}{edgelayer}
		\draw (0) to (1);
	\end{pgfonlayer}
\end{tikzpicture}
\eref{extra}
\begin{tikzpicture}
	\begin{pgfonlayer}{nodelayer}
		\node [style=none] (0) at (2, 0) {};
		\node [style=none] (1) at (2, -1) {};
		\node [style=none] (2) at (3, -1) {};
		\node [style=none] (3) at (3, 0) {};
	\end{pgfonlayer}
	\begin{pgfonlayer}{edgelayer}
		\draw [style=dashed] (3.center) to (0.center) to (1.center) to (2.center) to cycle;
	\end{pgfonlayer}
\end{tikzpicture}
$$
\end{lemma}
\begin{proof}
We can always slide things past each other, except for the critical pair  when controlled not gates are sandwiched by grey unit and counit on their target wires.

Take $n$ to be the number of controlled not gates targeting the same wire, sandwiched by a grey unit and counit

For the base case of $n=0$, this follows from the bone law which we imposed.

Suppose that the claim holds for some $n \in \N$. If $n+1$-cnot gates have their targets sandwiched between a grey unit and counit without loss of generality, we can assume they are controlled from different wires.  This is because  $\cnot$ gates are self-inverse, so that they can be slid together and cancel out.

So given $n+1$ $\cnot$ gates controlled on different wires and targeting the final wire:
$$
\begin{tikzpicture}
	\begin{pgfonlayer}{nodelayer}
		\node [style=X] (0) at (2, 0) {};
		\node [style=oplus] (1) at (2, -0.5) {};
		\node [style=oplus] (2) at (2, -1) {};
		\node [style=dot] (3) at (1.5, -0.5) {};
		\node [style=dot] (4) at (1, -1) {};
		\node [style=none] (5) at (1.5, 0.25) {};
		\node [style=none] (6) at (1, 0.25) {};
		\node [style=X] (9) at (2, -2.5) {};
		\node [style=none] (10) at (1, -2.75) {};
		\node [style=none] (11) at (1.5, -2.75) {};
		\node [style=none] (12) at (1.25, 0) {$\cdots$};
		\node [style=oplus] (13) at (2, -2) {};
		\node [style=dot] (14) at (0.5, -2) {};
		\node [style=none] (15) at (0.5, -2.75) {};
		\node [style=none] (16) at (0.5, 0.25) {};
		\node [style=none] (18) at (0.75, -1.5) {$\iddots$};
		\node [style=none] (19) at (2, -1.5) {$\vdots$};
		\node [style=none] (20) at (2.25, -1.5) {$n$};
	\end{pgfonlayer}
	\begin{pgfonlayer}{edgelayer}
		\draw (0) to (1);
		\draw (1) to (3);
		\draw (5.center) to (3);
		\draw (6.center) to (4);
		\draw (4) to (2);
		\draw (4) to (10.center);
		\draw (11.center) to (3);
		\draw (14) to (13);
		\draw (15.center) to (14);
		\draw (14) to (16.center);
		\draw (2) to (1);
		\draw (9) to (13);
	\end{pgfonlayer}
\end{tikzpicture}
=
\begin{tikzpicture}
	\begin{pgfonlayer}{nodelayer}
		\node [style=X] (21) at (4.75, 0.5) {};
		\node [style=oplus] (22) at (4.75, -0.5) {};
		\node [style=oplus] (23) at (4.75, -1) {};
		\node [style=dot] (24) at (4.25, -0.5) {};
		\node [style=dot] (25) at (3.75, -1) {};
		\node [style=none] (26) at (4.25, 0.75) {};
		\node [style=none] (27) at (3.75, 0.75) {};
		\node [style=X] (28) at (4.75, -3) {};
		\node [style=none] (29) at (3.75, -3.25) {};
		\node [style=none] (30) at (4.25, -3.25) {};
		\node [style=none] (31) at (4, 0.5) {$\cdots$};
		\node [style=oplus] (32) at (4.75, -2) {};
		\node [style=dot] (33) at (3.25, -2) {};
		\node [style=none] (34) at (3.25, -3.25) {};
		\node [style=none] (35) at (3.25, 0.75) {};
		\node [style=none] (37) at (3.5, -1.5) {$\iddots$};
		\node [style=none] (38) at (4.75, -1.5) {$\vdots$};
		\node [style=none] (39) at (5, -1.5) {$n$};
		\node [style=oplus] (40) at (4.25, 0) {};
		\node [style=dot] (41) at (4.75, 0) {};
		\node [style=oplus] (42) at (4.25, -2.5) {};
		\node [style=dot] (43) at (4.75, -2.5) {};
	\end{pgfonlayer}
	\begin{pgfonlayer}{edgelayer}
		\draw (21) to (22);
		\draw (22) to (24);
		\draw (26.center) to (24);
		\draw (27.center) to (25);
		\draw (25) to (23);
		\draw (25) to (29.center);
		\draw (30.center) to (24);
		\draw (33) to (32);
		\draw (34.center) to (33);
		\draw (33) to (35.center);
		\draw (23) to (22);
		\draw (28) to (32);
		\draw (40) to (41);
		\draw (42) to (43);
	\end{pgfonlayer}
\end{tikzpicture}
=
\begin{tikzpicture}
	\begin{pgfonlayer}{nodelayer}
		\node [style=X] (44) at (7.5, 0.5) {};
		\node [style=oplus] (45) at (7.5, -0.5) {};
		\node [style=oplus] (46) at (7.5, -1.5) {};
		\node [style=dot] (47) at (7, -0.5) {};
		\node [style=dot] (48) at (6.5, -1.5) {};
		\node [style=none] (49) at (7, 0.75) {};
		\node [style=none] (50) at (6.5, 0.75) {};
		\node [style=X] (51) at (7.5, -3) {};
		\node [style=none] (52) at (6.5, -3.25) {};
		\node [style=none] (53) at (7, -3.25) {};
		\node [style=none] (54) at (6.75, 0.5) {$\cdots$};
		\node [style=oplus] (55) at (7.5, -2.5) {};
		\node [style=dot] (56) at (6, -2.5) {};
		\node [style=none] (57) at (6, -3.25) {};
		\node [style=none] (58) at (6, 0.75) {};
		\node [style=none] (60) at (6.25, -2) {$\iddots$};
		\node [style=none] (61) at (7.5, -2) {$\vdots$};
		\node [style=none] (62) at (7.75, -2) {$n$};
		\node [style=oplus] (63) at (7, 0) {};
		\node [style=dot] (64) at (7.5, 0) {};
		\node [style=oplus] (65) at (7, -1) {};
		\node [style=dot] (66) at (7.5, -1) {};
	\end{pgfonlayer}
	\begin{pgfonlayer}{edgelayer}
		\draw (44) to (45);
		\draw (45) to (47);
		\draw (49.center) to (47);
		\draw (50.center) to (48);
		\draw (48) to (46);
		\draw (48) to (52.center);
		\draw (53.center) to (47);
		\draw (56) to (55);
		\draw (57.center) to (56);
		\draw (56) to (58.center);
		\draw (46) to (45);
		\draw (51) to (55);
		\draw (63) to (64);
		\draw (65) to (66);
	\end{pgfonlayer}
\end{tikzpicture}
=
\begin{tikzpicture}
	\begin{pgfonlayer}{nodelayer}
		\node [style=X] (67) at (10.25, 0.5) {};
		\node [style=oplus] (69) at (10.25, -1.5) {};
		\node [style=dot] (71) at (9.25, -1.5) {};
		\node [style=none] (72) at (9.75, 0.75) {};
		\node [style=none] (73) at (9.25, 0.75) {};
		\node [style=X] (74) at (10.25, -3) {};
		\node [style=none] (75) at (9.25, -3.25) {};
		\node [style=none] (76) at (9.75, -3.25) {};
		\node [style=none] (77) at (9.5, 0.5) {$\cdots$};
		\node [style=oplus] (78) at (10.25, -2.5) {};
		\node [style=dot] (79) at (8.75, -2.5) {};
		\node [style=none] (80) at (8.75, -3.25) {};
		\node [style=none] (81) at (8.75, 0.75) {};
		\node [style=none] (82) at (9, -2) {$\iddots$};
		\node [style=none] (83) at (10.25, -2) {$\vdots$};
		\node [style=none] (84) at (10.5, -2) {$n$};
		\node [style=none] (85) at (9.75, 0.5) {};
		\node [style=none] (86) at (10.25, -1) {};
		\node [style=none] (87) at (9.75, -1) {};
	\end{pgfonlayer}
	\begin{pgfonlayer}{edgelayer}
		\draw (73.center) to (71);
		\draw (71) to (69);
		\draw (71) to (75.center);
		\draw (79) to (78);
		\draw (80.center) to (79);
		\draw (79) to (81.center);
		\draw (74) to (78);
		\draw (72.center) to (85.center);
		\draw [in=90, out=-90] (85.center) to (86.center);
		\draw [in=270, out=90] (87.center) to (67);
		\draw (87.center) to (76.center);
		\draw (86.center) to (69);
	\end{pgfonlayer}
\end{tikzpicture}
=
\begin{tikzpicture}
	\begin{pgfonlayer}{nodelayer}
		\node [style=X] (88) at (12.5, -2.75) {};
		\node [style=oplus] (89) at (12.5, -0.75) {};
		\node [style=dot] (90) at (12, -0.75) {};
		\node [style=none] (92) at (12, 0) {};
		\node [style=X] (93) at (12.5, -2.25) {};
		\node [style=none] (94) at (12, -3.25) {};
		\node [style=none] (96) at (12.25, -0.25) {$\cdots$};
		\node [style=oplus] (97) at (12.5, -1.75) {};
		\node [style=dot] (98) at (11.5, -1.75) {};
		\node [style=none] (99) at (11.5, -3.25) {};
		\node [style=none] (100) at (11.5, 0) {};
		\node [style=none] (101) at (11.75, -1.25) {$\iddots$};
		\node [style=none] (102) at (12.5, -1.25) {$\vdots$};
		\node [style=none] (103) at (12.75, -1.25) {$n$};
		\node [style=none] (105) at (12.5, 0) {};
		\node [style=none] (106) at (12.5, -3.25) {};
	\end{pgfonlayer}
	\begin{pgfonlayer}{edgelayer}
		\draw (92.center) to (90);
		\draw (90) to (89);
		\draw (90) to (94.center);
		\draw (98) to (97);
		\draw (99.center) to (98);
		\draw (98) to (100.center);
		\draw (93) to (97);
		\draw [in=270, out=90] (106.center) to (88);
		\draw (105.center) to (89);
	\end{pgfonlayer}
\end{tikzpicture}
$$
\end{proof}
\begin{lemma}
\label{lem:parisocb}
$\sfpariso\cb_{\F_2}$ is a presentation for the prop $(\Par\Iso(\Mat_{\F_2}),\oplus)$.
\end{lemma}
\begin{proof}
$\Par\Iso(\Mat_{\F_2})$ is the category of spans of monorphisms in $\Mat_{\F_2}$.  This equation is precisely the one needed to compute the pullback.
\end{proof}
%This follows from \cite[??]{ih}.
We can get partial linear maps by adding the effect $\sqrt{2}|+\rangle$:
\begin{definition}
Let $\sfpar\cb_{\F_2}$ denote the pushout of the diagram of props:
$$
\sfpariso\cb_{\F_2}  \leftarrow  \surj^\op \rightarrow   \cm^\op
$$
Adding a counit to the white comultiplication.
\end{definition}
%The following Lemma follows after meticulous calculation and the application of \cite[Lem. 3.5]{zxa}:
\begin{lemma}
\label{lem:parcb}
$\sfpar\cb_{\F_2}$ is a presentation for the prop $(\Par(\Mat_{\F_2}),\oplus)$.
\end{lemma}
\begin{proof}
We show that the following diagram commutes and that the vertical maps are isomorphisms:
%
%\renewcommand{\cubetopbl}{$\sfmono(\cb_{\F_2})$}
%\renewcommand{\cubetopbr}{$\sfmono(\cb_{\F_2})^\op \otimes_{\Iso(\cb_{\F_2})} \sfmono(\cb_{\F_2})$}
%\renewcommand{\cubetopfl}{$\sfmono(\cb_{\F_2})\otimes_{\Iso(\cb_{\F_2})} \sfepi(\cb_{\F_2})^\op$}
%\renewcommand{\cubetopfr}{$\Par(\cb_{\F_2})$}\ParIso(\cb_{\F_2})
%\renewcommand{\cubebotbl}{$(\sfmono(\Mat_{\F_2}), +)$}
%\renewcommand{\cubebotbr}{$(\Par\Iso(\Mat_{\F_2}), +)$}
%\renewcommand{\cubebotfl}{$(\Par\sfepi(\Mat_{\F_2}), +)$}
%\renewcommand{\cubebotfr}{}
%
%$$
%\xymatrixrowsep{3mm}\xymatrixcolsep{1mm}
%\xymatrix{
%                                       & \mbox{\cubetopbl} \ar[rr] \ar[dl] \ar[dd]^(.7){\cong}      &                                                  & \mbox{\cubetopbr}  \ar[dd]^{\cong} \ar[dl] \\
%\mbox{\cubetopfl} \ar[rr]  \ar[dd]_{\cong}           &                                                                                              &\mbox{\cubetopfr} \ar@{-->}[dd]^(.35){\cong}   \skewpullbackcorner[ul]              \\
%                                       &  \mbox{\cubebotbl} \ar[dl] \ar[rr]                    &                                                  & \mbox{\cubebotbr} \ar@/^1pc/[ddl] \ar[dl] \\
%\mbox{\cubebotfl} \ar@/_1pc/[drr] \ar[rr]  &                                                                                             & \mbox{\cubebotfr} \skewpullbackcorner[ul]    \ar@{-->}[d]^{\cong}  \\
%                                                   &                                                                                             & (\Par(\Mat_{\F_2}),+)
%}
%$$
%
\renewcommand{\cubetopbl}{$\surj^\op$}
\renewcommand{\cubetopbr}{$\cm^\op$}
\renewcommand{\cubetopfl}{$\sfpariso\cb_{\F_2}$}
\renewcommand{\cubetopfr}{$\sfpar\cb_{\F_2}$}
\renewcommand{\cubebotbl}{$\surj^\op$ }
\renewcommand{\cubebotbr}{$\cm^\op$ }
\renewcommand{\cubebotfl}{$(\ParIso(\Mat_{\F_2}),\oplus)$ }
\renewcommand{\cubebotfr}{}
$$
\xymatrixrowsep{2mm}\xymatrixcolsep{2mm}
\xymatrix{
                                       & \mbox{\cubetopbl} \ar[rr] \ar[dl] \ar@{=}[dd]     &                                                  & \mbox{\cubetopbr} \ar@{=}[dd] \ar[dl] \\
\mbox{\cubetopfl} \ar[rr]  \ar[dd]_{\cong}           &                                                                                              &\mbox{\cubetopfr} \ar@{-->}[dd]    \skewpullbackcorner[ul]              \\
                                       &  \mbox{\cubebotbl} \ar[dl] \ar[rr]                    &                                                  & \mbox{\cubebotbr} \ar@/^1pc/[ddl] \ar[dl] \\
\mbox{\cubebotfl} \ar@/_1pc/[drr] \ar[rr]  &                                                                                             & \mbox{\cubebotfr} \skewpullbackcorner[ul]    \ar@{-->}[d] \\
                                                   &                                                                                             & (\Par(\Mat_{\F_2}),\oplus)
}
$$
Because $\Mat_{\F_2}$ is Cartesian,  $\sfpariso\cb_{\F_2}\cong\ParIso(\Mat_{\F_2})$ is a discrete inverse category.
We know that the counital completion of a discrete inverse category is the same as its Cartesian completion from  Proposition \ref{cor:copy} moreover, the Cartesian completion of  $\ParIso(\Mat_{\F_2})$ is $\Par(\Mat_{\F_2})$.  So this diagram commutes as a consequence.
\end{proof}
By adding the state $\sqrt 2 |+ \rangle$ we obtain the prop of linear spans:
\begin{definition}
Let $\sfspan\cb_{\F_2}$ denote the pushout of the diagram of props:
$$
\sfpar\cb_{\F_2}^\op \leftarrow  \sfpariso\cb_{\F_2} \rightarrow \sfpar\cb_{\F_2}
$$
\end{definition}
\begin{lemma}
\label{lem:spancb}
$\sfspan\cb_{\F_2}$ is a presentation for the prop $(\Span^\sim(\Mat_{\F_2}), \oplus)$.
\end{lemma}
\begin{proof}
We show that the following diagram commutes and that the vertical maps are isomorphisms:

\renewcommand{\cubetopbl}{$\sfpariso\cb_{\F_2}$}
\renewcommand{\cubetopbr}{$\sfpar\cb_{\F_2}$}
\renewcommand{\cubetopfl}{$\sfpar\cb_{\F_2}^\op$}
\renewcommand{\cubetopfr}{$\sfspan\cb_{\F_2}$}
\renewcommand{\cubebotbl}{$(\Par\Iso(\Mat_{\F_2}),\oplus)$ }
\renewcommand{\cubebotbr}{$(\Par(\Mat_{\F_2}),\oplus)$ }
\renewcommand{\cubebotfl}{$(\Par(\Mat_{\F_2}),\oplus)^\op$ }
\renewcommand{\cubebotfr}{}

$$
\xymatrixrowsep{2mm}\xymatrixcolsep{1mm}
\xymatrix{
                                       & \mbox{\cubetopbl} \ar[rr] \ar[dl] \ar[dd]^(.7){\cong}      &                                                  & \mbox{\cubetopbr}  \ar[dd]^{\cong} \ar[dl] \\
\mbox{\cubetopfl} \ar[rr]  \ar[dd]_{\cong}           &                                                                                              &\mbox{\cubetopfr} \ar@{-->}[dd]   \skewpullbackcorner[ul]              \\
                                       &  \mbox{\cubebotbl} \ar[dl] \ar[rr]                    &                                                  & \mbox{\cubebotbr} \ar@/^1pc/[ddl] \ar[dl] \\
\mbox{\cubebotfl} \ar@/_1pc/[drr] \ar[rr]  &                                                                                             & \mbox{\cubebotfr} \skewpullbackcorner[ul]    \ar@{-->}[d]_F \\
                                                   &                                                                                             & (\Span^\sim(\Mat_{\F_2}),\oplus)
}
$$

The cube easily commutes.  What remains to be shown is that the universal map $F$ is an isomorphism of props.  It is clearly the identity on objects, so we just need to show it is full and faithful.

It is full because given any span $ n \xleftarrow{ f}  k \xrightarrow{g } m$, we have:
$$
F\left( (n \xleftarrow{f} k = k);(k = k \xrightarrow{g} m) \right)=n \xleftarrow{ f}  k \xrightarrow{g } m
$$ 
For faithfulness,  given for any two isomorphic maps in $\Span(\Mat_{\F_2})$:
$$
\xymatrixrowsep{2mm}\xymatrixcolsep{6mm}
\xymatrix{
          & k \ar[dl]_{f'} \ar[dd]_{\cong}^{h} \ar[dr]^{g'}\\
n  &                                                                                                    & m\\
         & k \ar[ul]^{f} \ar[ur]_{g}\\
}
$$
Then in the domain of $F$:

{
\xymatrixrowsep{0mm}\xymatrixcolsep{1.7mm}
\begin{align*}
&
\xymatrix{
   & k \ar[dl]_f \ar@{=}[dr]\\
n &                                      &k
};
\xymatrix{
   & k \ar[dr]^g \ar@{=}[dl]\\
k &                                      &m
}
%&
 =
\xymatrix{
   & k \ar[dl]_f \ar@{=}[dr]\\
n &                                      &k
};
\xymatrix{
   & k \ar@{=}[dl] \ar@{=}[dr]\\
k &                                             & k\\
   & k \ar[ul]^h \ar[ur]_h \ar[uu]^\cong_h
};
\xymatrix{
   & k \ar[dr]^g \ar@{=}[dl]\\
k &                                      &m
}\\
 &=
\xymatrix{
   & k \ar[dl]_f \ar@{=}[dr]\\
n &                                      &k
};
\xymatrix{
   & k \ar[dl]_h \ar@{=}[dr]\\
k &                                         & k
};
\xymatrix{
   & k \ar[dr]^h \ar@{=}[dl]\\
k &                                         & k
};
\xymatrix{
   & k \ar[dr]^g \ar@{=}[dl]\\
k &                                      &m
}
%\\&
=
\xymatrix{
            &                                                        &k \ar[dl]_{h} \ar@{=}[dr] \ar@/_1.2pc/[ddll]_{f'}\\
            & k \ar@{=}[dr] \ar[dl]^{f}&                                                          & k \ar@{=}[dr] \ar[dl]_{h}\\
n &                                                         & k                                             &                                                         &k
};
\xymatrix{
            &                                                        & k \ar[dr]^{h} \ar@{=}[dl] \ar@/^1.2pc/[ddrr]^{g'}  \\
            & k \ar[dr]^{h}   \ar@{=}[dl] &                                                          & k \ar@{=}[dl] \ar[dr]_{g}\\
k &                                                         & k                                             &                                                         &m
}
\end{align*}
}
\end{proof}
Given a PID $k$, the prop $(\Span^\sim(\Mat_k), \oplus)$ is already known to have a much nicer given in terms of ``interacting Hopf algebras" \cite[Definition 3.13]{ih}. 
\subsection{Adding the \texorpdfstring{$\Not$}{not}-gate}
\label{sec:two}

In this subsection we perform the same analysis for the affine fragment as we did in the previous subsection for the linear fragment.  First, consider the affine isomorphisms:

First, for the $\Not$ gate:
\begin{definition}
Let ${\sf N}_2$ denote the prop generated by the not gate modulo the following equation:
$$
\begin{tikzpicture}
	\begin{pgfonlayer}{nodelayer}
		\node [style=oplus] (0) at (0, 0) {};
		\node [style=oplus] (1) at (0, 0.5) {};
		\node [style=none] (2) at (0, 1) {};
		\node [style=none] (3) at (0, -0.5) {};
	\end{pgfonlayer}
	\begin{pgfonlayer}{edgelayer}
		\draw (2.center) to (3.center);
	\end{pgfonlayer}
\end{tikzpicture}
\eqzxa{cnot.seven}
\begin{tikzpicture}
	\begin{pgfonlayer}{nodelayer}
		\node [style=none] (2) at (0, 1) {};
		\node [style=none] (3) at (0, -0.5) {};
	\end{pgfonlayer}
	\begin{pgfonlayer}{edgelayer}
		\draw (2.center) to (3.center);
	\end{pgfonlayer}
\end{tikzpicture}
$$
\end{definition}
The $\Not$ gate and $\cnot$ gate interact via distributive law:
\begin{definition}
There is a distributive law of props:
$$
\sfiso\acb_{\F_2}:=\sfiso\cb_{\F_2} \otimes_\P {\sf N}_2;\
\begin{tikzpicture}
	\begin{pgfonlayer}{nodelayer}
		\node [style=dot] (0) at (2, 2) {};
		\node [style=oplus] (1) at (2.5, 2) {};
		\node [style=oplus] (2) at (2, 1.5) {};
		\node [style=none] (3) at (2, 1) {};
		\node [style=none] (4) at (2.5, 1) {};
		\node [style=none] (5) at (2, 2.5) {};
		\node [style=none] (6) at (2.5, 2.5) {};
	\end{pgfonlayer}
	\begin{pgfonlayer}{edgelayer}
		\draw (3.center) to (2);
		\draw (2) to (0);
		\draw (0) to (5.center);
		\draw (6.center) to (1);
		\draw (1) to (0);
		\draw (4.center) to (1);
	\end{pgfonlayer}
\end{tikzpicture}
\eqzxa{cnot.eight}
\begin{tikzpicture}
	\begin{pgfonlayer}{nodelayer}
		\node [style=dot] (0) at (2, 1.5) {};
		\node [style=oplus] (1) at (2.5, 1.5) {};
		\node [style=none] (3) at (2, 1) {};
		\node [style=none] (4) at (2.5, 1) {};
		\node [style=none] (5) at (2, 2.5) {};
		\node [style=none] (6) at (2.5, 2.5) {};
		\node [style=oplus] (7) at (2, 2) {};
		\node [style=oplus] (8) at (2.5, 2) {};
	\end{pgfonlayer}
	\begin{pgfonlayer}{edgelayer}
		\draw (0) to (5.center);
		\draw (6.center) to (1);
		\draw (1) to (0);
		\draw (4.center) to (1);
		\draw (3.center) to (0);
	\end{pgfonlayer}
\end{tikzpicture}
\hspace*{.5cm}
\begin{tikzpicture}
	\begin{pgfonlayer}{nodelayer}
		\node [style=dot] (0) at (2, 2) {};
		\node [style=oplus] (1) at (2.5, 2) {};
		\node [style=none] (3) at (2, 1) {};
		\node [style=none] (4) at (2.5, 1) {};
		\node [style=none] (5) at (2, 2.5) {};
		\node [style=none] (6) at (2.5, 2.5) {};
		\node [style=oplus] (8) at (2.5, 1.5) {};
	\end{pgfonlayer}
	\begin{pgfonlayer}{edgelayer}
		\draw (0) to (5.center);
		\draw (6.center) to (1);
		\draw (1) to (0);
		\draw (4.center) to (1);
		\draw (3.center) to (0);
	\end{pgfonlayer}
\end{tikzpicture}
\eqzxa{cnot.nine}
\begin{tikzpicture}
	\begin{pgfonlayer}{nodelayer}
		\node [style=dot] (0) at (2, 1.5) {};
		\node [style=oplus] (1) at (2.5, 1.5) {};
		\node [style=none] (3) at (2, 1) {};
		\node [style=none] (4) at (2.5, 1) {};
		\node [style=none] (5) at (2, 2.5) {};
		\node [style=none] (6) at (2.5, 2.5) {};
		\node [style=oplus] (8) at (2.5, 2) {};
	\end{pgfonlayer}
	\begin{pgfonlayer}{edgelayer}
		\draw (0) to (5.center);
		\draw (6.center) to (1);
		\draw (1) to (0);
		\draw (4.center) to (1);
		\draw (3.center) to (0);
	\end{pgfonlayer}
\end{tikzpicture}
$$
\end{definition}
\begin{lemma}[{\cite[Thm. 11]{lafont}}]
$\sfiso\acb_{\F_2}$ is a presentation for the prop $(\Iso(\Aff\Mat_{\F_2}),\oplus)$ with respect to the interpretation:
$$
\left\llbracket
\begin{tikzpicture}
	\begin{pgfonlayer}{nodelayer}
		\node [style=oplus] (5) at (0.5, 2.75) {};
		\node [style=dot] (6) at (0, 2.75) {};
		\node [style=none] (7) at (0.5, 3.5) {};
		\node [style=none] (8) at (0.5, 2) {};
		\node [style=none] (9) at (0, 2) {};
		\node [style=none] (10) at (0, 3.5) {};
	\end{pgfonlayer}
	\begin{pgfonlayer}{edgelayer}
		\draw (8.center) to (5);
		\draw (5) to (7.center);
		\draw (10.center) to (6);
		\draw (6) to (5);
		\draw (6) to (9.center);
	\end{pgfonlayer}
\end{tikzpicture}
\right\rrbracket
=
\begin{tikzpicture}
	\begin{pgfonlayer}{nodelayer}
		\node [style=X] (0) at (-0.25, -1) {};
		\node [style=Z] (1) at (-0.75, -1.75) {};
		\node [style=none] (2) at (0, -2.25) {};
		\node [style=none] (3) at (-0.25, -0.5) {};
		\node [style=none] (4) at (-1, -0.5) {};
		\node [style=none] (5) at (-0.75, -2.25) {};
	\end{pgfonlayer}
	\begin{pgfonlayer}{edgelayer}
		\draw (3.center) to (0);
		\draw [in=90, out=-75] (0) to (2.center);
		\draw (5.center) to (1);
		\draw (1) to (0);
		\draw [in=-90, out=105] (1) to (4.center);
	\end{pgfonlayer}
\end{tikzpicture}
\hspace*{1cm}
\left\llbracket
\begin{tikzpicture}
	\begin{pgfonlayer}{nodelayer}
		\node [style=none] (0) at (0, -0.5) {};
		\node [style=none] (1) at (0, -1.5) {};
		\node [style=oplus] (2) at (0, -1) {};
	\end{pgfonlayer}
	\begin{pgfonlayer}{edgelayer}
		\draw (1.center) to (2);
		\draw (2) to (0.center);
	\end{pgfonlayer}
\end{tikzpicture}
\right\rrbracket
=
\begin{tikzpicture}
	\begin{pgfonlayer}{nodelayer}
		\node [style=none] (7) at (5.25, 2) {};
		\node [style=X] (9) at (5.25, 1.375) {$1$};
		\node [style=none] (10) at (5.25, 0.75) {};
	\end{pgfonlayer}
	\begin{pgfonlayer}{edgelayer}
		\draw (10.center) to (9);
		\draw (9) to (7.center);
	\end{pgfonlayer}
\end{tikzpicture}
$$
\end{lemma}
We get affine injections by adding the $|0\rangle$ state:
\begin{definition}
Let $\sfinj\acb_{\F_2}$ denote the pushout of the diagram of props:
$$
 \sfinj\cb_{\F_2} \leftarrow  \sfiso\cb_{\F_2} \rightarrow  \sfiso\acb_{\F_2}
$$
\end{definition}
\begin{lemma}
\label{lem:injaffcb}
$\sfinj\acb_{\F_2}$ is a presentation for the prop $(\sfmono(\Aff\Mat_{\F_2}),\oplus)$.
\end{lemma}
\begin{proof}
We show that the following diagram commutes and that the vertical maps are isomorphisms:
\renewcommand{\cubetopbl}{$\sfiso\cb_{\F_2}$}
\renewcommand{\cubetopbr}{$\sfiso\acb_{\F_2}$}
\renewcommand{\cubetopfl}{$\sfinj\cb_{\F_2}$}
\renewcommand{\cubetopfr}{$\sfinj\acb_{\F_2}$}
\renewcommand{\cubebotbl}{$(\Iso(\Mat_{\F_2}),\oplus)$ }
\renewcommand{\cubebotbr}{$(\Iso(\Aff\Mat_{\F_2}),\oplus)$ }
\renewcommand{\cubebotfl}{$(\sfmono(\Mat_{\F_2}),\oplus)$ }
\renewcommand{\cubebotfr}{}
$$
\xymatrixrowsep{2mm}\xymatrixcolsep{1.5mm}
\xymatrix{
                                       & \mbox{\cubetopbl} \ar[rr] \ar[dl] \ar[dd]^(.7){\cong}      &                                                  & \mbox{\cubetopbr}  \ar[dd]^{\cong} \ar[dl] \\
\mbox{\cubetopfl} \ar[rr]  \ar[dd]_{\cong}           &                                                                                              &\mbox{\cubetopfr} \ar@{-->}[dd]    \skewpullbackcorner[ul]              \\
                                       &  \mbox{\cubebotbl} \ar[dl] \ar[rr]                    &                                                  & \mbox{\cubebotbr} \ar@/^1pc/[ddl] \ar[dl] \\
\mbox{\cubebotfl} \ar@/_1pc/[drr] \ar[rr]  &                                                                                             & \mbox{\cubebotfr} \skewpullbackcorner[ul]    \ar@{-->}[d]_F  \\
                                                   &                                                                                             & (\sfmono(\Aff\Mat_{\F_2}),\oplus)
}
$$
 The rear and left faces of the cube commute and their vertical maps are all isomorphisms. Therefore, the whole cube commutes via universal property of the pushout, with the upper universal map of the cube also being an isomorphism.

We seek to show that the lower universal map  $F$ is also an isomorphism.  It is clearly the identity on objects, so we just have to show fullness and faithfulness.

For fullness, consider any map $n\ \xrightarrowtail{(A,x)} m$ in $(\sfmono(\Aff\Mat_{\F_2}),\oplus)$.  Note that this can be factored into:
$$
n\ \xrightarrowtail{(A,0)} m \xrightarrowiso{(1,x)}  m
$$
Which lies in the image of $F$ as $m \xrightarrowiso{(1,x)} m$ is an isomorphism.

For faithfulness, we show that there is a unique normal form for maps in 
$$(\Iso(\Aff\Mat_{\F_2}),\oplus)+_{(\Iso(\Mat_{\F_2}))} (\sfmono(\Mat_{\F_2}),\oplus) $$
There are two cases:
$$
\left( n \ \xrightarrowtail{ A} m ; m \xrightarrowiso{(B, x)} m \right)
= \left( n \ \xrightarrowtail{ A} m ; m \xrightarrowiso{(B, 0)} m; m \xrightarrowiso{(1, x)}  m \right)
= \left( n \ \xrightarrowtail{ A;B}  m\xrightarrowiso{(1, x)}  m \right)
$$
and
\begin{align*}
\left(n \ \xrightarrowtail{ (A,x)} m ; m \xrightarrowiso{B} m \right)
&= \left( n \ \xrightarrowtail{ (A,0)}m; m \xrightarrowiso{(1,x)} m ; m \xrightarrowiso{B} m \right)
= \left( n\  \xrightarrowtail{ A }m; m \xrightarrowiso{(B,B(x))} m  \right)\\
&= \left( n \ \xrightarrowtail{ A;B }m; m \xrightarrowiso{(1,B(x))} m  \right)
\end{align*}
\end{proof}
To define partial isomorphisms, we add a generator to the constituent props corresponding to the zero subobject/zero scalar:
%\begin{definition}
%Given a prop $\X$, let $\X!$ denote the prop generated by adding a scalar $0$,  quotiented by the equation, for all parallel $f,g$: 
%$
%f \otimes 0  =  g \otimes 0
%$ 
%\end{definition}
%
%
%
%That is to say, $\X!$ is the prop with zero maps formally added.  In affine matrices, there is no proper zero object: the one element space is the terminal object and the empty set is the initial object.  By taking spans of affine matrices, the initial object becomes a zero object; however, seeing as we are working with props, the empty set can not be represented using this formalism.  Thus we just add the zero object as a subjobject.
\begin{definition}
\label{def:isoaffcbzero}
Let $\sfiso\acb_{\F_2}^{+1}$ denote the prop obtained by adjoining the following generator to $\sfiso\acb_{\F_2}$ 
$
\begin{tikzpicture}
	\begin{pgfonlayer}{nodelayer}
		\node [style=X] (0) at (0, 0) {$1$};
	\end{pgfonlayer}
\end{tikzpicture}
$
modulo the equations:
$$
\begin{tikzpicture}
	\begin{pgfonlayer}{nodelayer}
		\node [style=X] (0) at (0, 0) {$1$};
		\node [style=X] (3) at (0.5, 0) {$1$};
	\end{pgfonlayer}
\end{tikzpicture}
\eqzxa{zero.one}
\begin{tikzpicture}
	\begin{pgfonlayer}{nodelayer}
		\node [style=X] (0) at (0, 0) {$1$};
	\end{pgfonlayer}
\end{tikzpicture},
\hspace*{.5cm}
\begin{tikzpicture}
	\begin{pgfonlayer}{nodelayer}
		\node [style=X] (0) at (0, 1) {$1$};
		\node [style=none] (1) at (0.5, 0.5) {};
		\node [style=none] (2) at (0.5, 1.5) {};
		\node [style=none] (3) at (1, 1.5) {};
		\node [style=none] (4) at (1, 0.5) {};
		\node [style=dot] (5) at (0.5, 1) {};
		\node [style=oplus] (6) at (1, 1) {};
	\end{pgfonlayer}
	\begin{pgfonlayer}{edgelayer}
		\draw [in=90, out=-90] (2.center) to (1.center);
		\draw [in=-90, out=90] (4.center) to (3.center);
		\draw (6) to (5);
	\end{pgfonlayer}
\end{tikzpicture}
\eqzxa{zero.two}
\begin{tikzpicture}
	\begin{pgfonlayer}{nodelayer}
		\node [style=X] (0) at (0, 1) {$1$};
		\node [style=none] (1) at (0.5, 0.5) {};
		\node [style=none] (2) at (0.5, 1.5) {};
		\node [style=none] (3) at (1, 1.5) {};
		\node [style=none] (4) at (1, 0.5) {};
	\end{pgfonlayer}
	\begin{pgfonlayer}{edgelayer}
		\draw [in=90, out=-90] (2.center) to (1.center);
		\draw [in=-90, out=90] (4.center) to (3.center);
	\end{pgfonlayer}
\end{tikzpicture},
\hspace*{.5cm}
\begin{tikzpicture}
	\begin{pgfonlayer}{nodelayer}
		\node [style=X] (0) at (0, 1) {$1$};
		\node [style=none] (1) at (0.5, 0.5) {};
		\node [style=none] (2) at (1, 1.5) {};
		\node [style=none] (3) at (0.5, 1.5) {};
		\node [style=none] (4) at (1, 0.5) {};
	\end{pgfonlayer}
	\begin{pgfonlayer}{edgelayer}
		\draw [in=90, out=-90] (2.center) to (1.center);
		\draw [in=-90, out=90] (4.center) to (3.center);
	\end{pgfonlayer}
\end{tikzpicture}
\eqzxa{zero.three}
\begin{tikzpicture}
	\begin{pgfonlayer}{nodelayer}
		\node [style=X] (0) at (0, 1) {$1$};
		\node [style=none] (1) at (0.5, 0.5) {};
		\node [style=none] (2) at (0.5, 1.5) {};
		\node [style=none] (3) at (1, 1.5) {};
		\node [style=none] (4) at (1, 0.5) {};
	\end{pgfonlayer}
	\begin{pgfonlayer}{edgelayer}
		\draw [in=90, out=-90] (2.center) to (1.center);
		\draw [in=-90, out=90] (4.center) to (3.center);
	\end{pgfonlayer}
\end{tikzpicture},
\hspace*{.5cm}
\begin{tikzpicture}
	\begin{pgfonlayer}{nodelayer}
		\node [style=X] (0) at (0, 1) {$1$};
		\node [style=none] (1) at (0.5, 0.5) {};
		\node [style=none] (2) at (0.5, 1.5) {};
		\node [style=oplus] (3) at (0.5, 1) {};
	\end{pgfonlayer}
	\begin{pgfonlayer}{edgelayer}
		\draw (2.center) to (1.center);
	\end{pgfonlayer}
\end{tikzpicture}
\eqzxa{zero.four}
\begin{tikzpicture}
	\begin{pgfonlayer}{nodelayer}
		\node [style=X] (0) at (0, 1) {$1$};
		\node [style=none] (1) at (0.5, 0.5) {};
		\node [style=none] (2) at (0.5, 1.5) {};
	\end{pgfonlayer}
	\begin{pgfonlayer}{edgelayer}
		\draw (2.center) to (1.center);
	\end{pgfonlayer}
\end{tikzpicture}
$$
\end{definition}
\begin{lemma}
$\sfiso\acb_{\F_2}^{+1}$ is a presentation for the subcategory of $$(\Span^\sim(\Aff\Mat_{\F_2}+1), \oplus)$$ generated by spans 
$$\F_2^n = \F_2^n \xrightarrow[\cong]{f} \F_2^n\hspace*{.5cm}\text{and}\hspace*{.5cm}\F_2^n \xleftarrowtail {?}\  \emptyset \ \xrightarrowtail{?}  \F_2^n$$
for all $n \in \N$ and isomorphisms $f$. 
\end{lemma}
%
%\begin{lemma}
%
%The props $\Iso(\acb_{\F_2})^{+1}$ and $\Iso(\acb_{\F_2})!$ are isomorphic.
%
%\end{lemma}
\begin{proof}
Identify the generator 
$
\begin{tikzpicture}
	\begin{pgfonlayer}{nodelayer}
		\node [style=X] (0) at (0, 0) {$1$};
	\end{pgfonlayer}
\end{tikzpicture}
$
 with the span $\F_2^0 \xleftarrowtail {?}\  \emptyset \ \xrightarrowtail{?}  \F_2^0$.  If there is a factor of 
$
\begin{tikzpicture}
	\begin{pgfonlayer}{nodelayer}
		\node [style=X] (0) at (0, 0) {$1$};
	\end{pgfonlayer}
\end{tikzpicture}
$,   repeatedly apply these identities from left to right until the diagram corresponding to the identity tensored by $
\begin{tikzpicture}
	\begin{pgfonlayer}{nodelayer}
		\node [style=X] (0) at (0, 0) {$1$};
	\end{pgfonlayer}
\end{tikzpicture}
$ is obtained, which is as a normal form.
\end{proof}
By adding the state $|0\rangle$ to our previous presentation, we get a presentation for injections which can also be zero:
\begin{definition}
Let $\sfinj\acb_{\F_2}^{+1}$ denote the pushout of the diagram of props:
$$
\sfinj\acb_{\F_2} \leftarrow \sfiso\acb_{\F_2} \rightarrow \sfiso\acb_{\F_2}^{+1}
$$
\end{definition}
\begin{lemma}
$\sfinj\acb_{\F_2}^{+1}$ is a presentation for the subcategory of $(\Span^\sim(\Aff\Mat_{\F_2}+1), \oplus)$ generated by spans $\F_2^n = \F_2^n \ \xrightarrowtail{e} \F_2^m$ and $\F_2^n \xleftarrowtail{?} \ \emptyset \ \xrightarrowtail{?}  \F_2^n$, for all $n,m \in \N$ and monics $e$. 
\end{lemma}
%
%\begin{lemma}
%The props $\sfmono(\acb_{\F_2})^{+1}$ and $\sfmono(\acb_{\F_2})!$ are isomorphic.
%\end{lemma}
The proof of this lemma is essentially the same for $\sfiso\acb_{\F_2}^{+1}$, although diagrams with a factor of
$
\begin{tikzpicture}
	\begin{pgfonlayer}{nodelayer}
		\node [style=X] (0) at (0, 0) {$1$};
	\end{pgfonlayer}
\end{tikzpicture}
$ are reduced to the following normal form:
$$
\begin{tikzpicture}
	\begin{pgfonlayer}{nodelayer}
		\node [style=X] (0) at (0, 1.25) {$1$};
		\node [style=none] (1) at (0.5, 0.5) {};
		\node [style=none] (2) at (0.5, 1.75) {};
		\node [style=none] (3) at (1, 0.5) {};
		\node [style=none] (4) at (1, 1.75) {};
		\node [style=X] (5) at (1.5, 0.75) {};
		\node [style=X] (6) at (2, 0.75) {};
		\node [style=none] (7) at (1.5, 1.75) {};
		\node [style=none] (8) at (2, 1.75) {};
		\node [style=none] (9) at (0.75, 1.5) {$n$};
		\node [style=none] (10) at (1.75, 1.5) {$m$};
		\node [style=none] (11) at (1.77, 1.25) {$\cdots$};
		\node [style=none] (12) at (0.77, 1.25) {$\cdots$};
	\end{pgfonlayer}
	\begin{pgfonlayer}{edgelayer}
		\draw (2.center) to (1.center);
		\draw (4.center) to (3.center);
		\draw (5) to (7.center);
		\draw (8.center) to (6);
	\end{pgfonlayer}
\end{tikzpicture}
$$
%Unlike in the linear case, now we must consider a distributive law over a prop which is not a groupoid: we add a single idempotent corresponding to the empty set to the isomorphisms.  To satisfy the requirement that this prop is a sub-prop of the left and right components of the  distributive law, we also add this idempotent to the injections and the co-injections:
To get partial injections, we add the effect $\langle 0|$:
\begin{lemma}
\label{def:parisoaffcb}
There is a distributive law of props:
$$
\sfpariso\acb_{\F_2}:=
 (\sfinj\acb_{\F_2}^{+1})^\op \otimes_{\sfiso\acb_{\F_2}^{+1}}  \sfinj\acb_{\F_2}^{+1}
$$
Given by the equations of $\sfpariso\cb_{\F_2}$ as well as:
$$
\begin{tikzpicture}
	\begin{pgfonlayer}{nodelayer}
		\node [style=X] (0) at (0.5, 0.75) {$1$};
		\node [style=X] (1) at (0.5, 0) {};
	\end{pgfonlayer}
	\begin{pgfonlayer}{edgelayer}
		\draw (0) to (1);
	\end{pgfonlayer}
\end{tikzpicture}
=
\begin{tikzpicture}
	\begin{pgfonlayer}{nodelayer}
		\node [style=X] (0) at (0, 0) {$1$};
		\node [style=X] (1) at (0, 0.75) {};
	\end{pgfonlayer}
	\begin{pgfonlayer}{edgelayer}
		\draw (0) to (1);
	\end{pgfonlayer}
\end{tikzpicture}
\eqzxa{zero.five}
\begin{tikzpicture}
	\begin{pgfonlayer}{nodelayer}
		\node [style=X] (0) at (0, 0) {$1$};
	\end{pgfonlayer}
	\begin{pgfonlayer}{edgelayer}
	\end{pgfonlayer}
\end{tikzpicture}
$$
\end{lemma}
\begin{proof}
\label{rem:parisoaffcb}
The only nontrivial critical pair arises when controlled-not gates are sandwiched between grey, or grey $1$ units/counits on their target wires.  The case where there are no controlled not gates in between is resolved by axiom \ref{zero.five}.  When there are more controlled-not gates, they can be pushed past each other as follows:
$$
\begin{tikzpicture}
	\begin{pgfonlayer}{nodelayer}
		\node [style=X] (0) at (2.5, 0.25) {};
		\node [style=oplus] (1) at (2.5, -0.75) {};
		\node [style=oplus] (2) at (2.5, -1.5) {};
		\node [style=dot] (3) at (1.5, -0.75) {};
		\node [style=dot] (4) at (1, -1.5) {};
		\node [style=none] (5) at (1.5, 0.5) {};
		\node [style=none] (6) at (1, 0.5) {};
		\node [style=none] (7) at (1.25, -1) {$\iddots$};
		\node [style=none] (8) at (2.5, -1) {$\vdots$};
		\node [style=X] (9) at (2.5, -2) {$1$};
		\node [style=none] (10) at (1, -2.25) {};
		\node [style=none] (11) at (1.5, -2.25) {};
		\node [style=none] (12) at (1.25, -0.25) {$\cdots$};
		\node [style=none] (17) at (2, -2.25) {};
		\node [style=none] (18) at (2, 0.5) {};
		\node [style=oplus] (19) at (2.5, -0.25) {};
		\node [style=dot] (20) at (2, -0.25) {};
	\end{pgfonlayer}
	\begin{pgfonlayer}{edgelayer}
		\draw (0) to (1);
		\draw (1) to (3);
		\draw (5.center) to (3);
		\draw (6.center) to (4);
		\draw (4) to (2);
		\draw (4) to (10.center);
		\draw (11.center) to (3);
		\draw (9) to (2);
		\draw (17.center) to (20);
		\draw (20) to (18.center);
		\draw (19) to (20);
	\end{pgfonlayer}
\end{tikzpicture}
=
\begin{tikzpicture}
	\begin{pgfonlayer}{nodelayer}
		\node [style=X] (0) at (2.5, 0.25) {};
		\node [style=oplus] (1) at (2.5, -0.75) {};
		\node [style=oplus] (2) at (2.5, -1.5) {};
		\node [style=dot] (3) at (1.5, -0.75) {};
		\node [style=dot] (4) at (1, -1.5) {};
		\node [style=none] (5) at (1.5, 0.5) {};
		\node [style=none] (6) at (1, 0.5) {};
		\node [style=none] (7) at (1.25, -1) {$\iddots$};
		\node [style=none] (8) at (2.5, -1) {$\vdots$};
		\node [style=X] (9) at (2.5, -2.5) {};
		\node [style=none] (10) at (1, -2.75) {};
		\node [style=none] (11) at (1.5, -2.75) {};
		\node [style=none] (12) at (1.25, -0.25) {$\cdots$};
		\node [style=none] (17) at (2, -2.75) {};
		\node [style=none] (18) at (2, 0.5) {};
		\node [style=oplus] (19) at (2.5, -0.25) {};
		\node [style=dot] (20) at (2, -0.25) {};
		\node [style=oplus] (21) at (2.5, -2) {};
	\end{pgfonlayer}
	\begin{pgfonlayer}{edgelayer}
		\draw (0) to (1);
		\draw (1) to (3);
		\draw (5.center) to (3);
		\draw (6.center) to (4);
		\draw (4) to (2);
		\draw (4) to (10.center);
		\draw (11.center) to (3);
		\draw (9) to (2);
		\draw (17.center) to (20);
		\draw (20) to (18.center);
		\draw (19) to (20);
	\end{pgfonlayer}
\end{tikzpicture}
=
\begin{tikzpicture}
	\begin{pgfonlayer}{nodelayer}
		\node [style=X] (0) at (2.5, 0.75) {};
		\node [style=oplus] (1) at (2.5, -0.75) {};
		\node [style=oplus] (2) at (2.5, -1.5) {};
		\node [style=dot] (3) at (1.5, -0.75) {};
		\node [style=dot] (4) at (1, -1.5) {};
		\node [style=none] (5) at (1.5, 1.25) {};
		\node [style=none] (6) at (1, 1.25) {};
		\node [style=none] (7) at (1.25, -1) {$\iddots$};
		\node [style=none] (8) at (2.5, -1) {$\vdots$};
		\node [style=X] (9) at (2.5, -2) {};
		\node [style=none] (10) at (1, -2.25) {};
		\node [style=none] (11) at (1.5, -2.25) {};
		\node [style=none] (12) at (1.25, 0.25) {$\cdots$};
		\node [style=none] (17) at (2, -2.25) {};
		\node [style=none] (18) at (2, 1.25) {};
		\node [style=oplus] (19) at (2.5, 0.25) {};
		\node [style=dot] (20) at (2, 0.25) {};
		\node [style=oplus] (21) at (2, -0.25) {};
		\node [style=oplus] (22) at (2, 0.75) {};
	\end{pgfonlayer}
	\begin{pgfonlayer}{edgelayer}
		\draw (0) to (1);
		\draw (1) to (3);
		\draw (5.center) to (3);
		\draw (6.center) to (4);
		\draw (4) to (2);
		\draw (4) to (10.center);
		\draw (11.center) to (3);
		\draw (9) to (2);
		\draw (17.center) to (20);
		\draw (20) to (18.center);
		\draw (19) to (20);
	\end{pgfonlayer}
\end{tikzpicture}
=
\begin{tikzpicture}
	\begin{pgfonlayer}{nodelayer}
		\node [style=X] (0) at (2.5, 0.25) {};
		\node [style=oplus] (1) at (2.5, -0.75) {};
		\node [style=oplus] (2) at (2.5, -1.5) {};
		\node [style=dot] (3) at (1.5, -0.75) {};
		\node [style=dot] (4) at (1, -1.5) {};
		\node [style=none] (5) at (1.5, 1) {};
		\node [style=none] (6) at (1, 1) {};
		\node [style=none] (7) at (1.25, -1) {$\iddots$};
		\node [style=none] (8) at (2.5, -1) {$\vdots$};
		\node [style=X] (9) at (2.5, -2) {};
		\node [style=none] (10) at (1, -2.75) {};
		\node [style=none] (11) at (1.5, -2.75) {};
		\node [style=none] (12) at (1.25, -0.25) {$\cdots$};
		\node [style=none] (17) at (2, -2.75) {};
		\node [style=none] (18) at (2, 1) {};
		\node [style=oplus] (19) at (2.5, -0.25) {};
		\node [style=dot] (20) at (2, -0.25) {};
		\node [style=oplus] (21) at (2, -2.25) {};
		\node [style=oplus] (22) at (2, 0.5) {};
	\end{pgfonlayer}
	\begin{pgfonlayer}{edgelayer}
		\draw (0) to (1);
		\draw (1) to (3);
		\draw (5.center) to (3);
		\draw (6.center) to (4);
		\draw (4) to (2);
		\draw (4) to (10.center);
		\draw (11.center) to (3);
		\draw (9) to (2);
		\draw (17.center) to (20);
		\draw (20) to (18.center);
		\draw (19) to (20);
	\end{pgfonlayer}
\end{tikzpicture}
=
\begin{tikzpicture}
	\begin{pgfonlayer}{nodelayer}
		\node [style=none] (24) at (4.5, 1.75) {};
		\node [style=none] (25) at (4, 1.75) {};
		\node [style=X] (28) at (5, -1) {};
		\node [style=none] (29) at (4, -3.25) {};
		\node [style=none] (30) at (4.5, -3.25) {};
		\node [style=none] (32) at (5, -3.25) {};
		\node [style=none] (33) at (5, 1.75) {};
		\node [style=oplus] (37) at (5, 1.25) {};
		\node [style=oplus] (38) at (5, -1.5) {};
		\node [style=dot] (39) at (4.5, -1.5) {};
		\node [style=oplus] (40) at (5, -2.25) {};
		\node [style=dot] (41) at (4, -2.25) {};
		\node [style=oplus] (42) at (5, 0) {};
		\node [style=dot] (43) at (4.5, 0) {};
		\node [style=oplus] (44) at (5, 0.75) {};
		\node [style=dot] (45) at (4, 0.75) {};
		\node [style=oplus] (46) at (5, -2.75) {};
		\node [style=X] (47) at (5, -0.5) {};
		\node [style=none] (48) at (4.25, -0.75) {$\cdots$};
		\node [style=none] (49) at (4.25, -1.75) {$\iddots$};
		\node [style=none] (50) at (4.25, 0.5) {$\ddots$};
		\node [style=none] (51) at (5, 0.5) {$\vdots$};
		\node [style=none] (52) at (5, -1.75) {$\vdots$};
	\end{pgfonlayer}
	\begin{pgfonlayer}{edgelayer}
		\draw (38) to (39);
		\draw (40) to (41);
		\draw (42) to (43);
		\draw (44) to (45);
		\draw (29.center) to (25.center);
		\draw (24.center) to (30.center);
		\draw (38) to (28);
		\draw (40) to (32.center);
		\draw (47) to (42);
		\draw (44) to (33.center);
	\end{pgfonlayer}
\end{tikzpicture}
=
\begin{tikzpicture}
	\begin{pgfonlayer}{nodelayer}
		\node [style=none] (24) at (4.5, 1.25) {};
		\node [style=none] (25) at (4, 1.25) {};
		\node [style=X] (28) at (5, -1) {$1$};
		\node [style=none] (29) at (4, -2.75) {};
		\node [style=none] (30) at (4.5, -2.75) {};
		\node [style=none] (32) at (5, -2.75) {};
		\node [style=none] (33) at (5, 1.25) {};
		\node [style=oplus] (38) at (5, -1.5) {};
		\node [style=dot] (39) at (4.5, -1.5) {};
		\node [style=oplus] (40) at (5, -2.25) {};
		\node [style=dot] (41) at (4, -2.25) {};
		\node [style=oplus] (42) at (5, 0) {};
		\node [style=dot] (43) at (4.5, 0) {};
		\node [style=oplus] (44) at (5, 0.75) {};
		\node [style=dot] (45) at (4, 0.75) {};
		\node [style=X] (47) at (5, -0.5) {$1$};
		\node [style=none] (48) at (4.25, -0.75) {$\cdots$};
		\node [style=none] (49) at (4.25, -1.75) {$\iddots$};
		\node [style=none] (50) at (4.25, 0.5) {$\ddots$};
		\node [style=none] (51) at (5, 0.5) {$\vdots$};
		\node [style=none] (52) at (5, -1.75) {$\vdots$};
	\end{pgfonlayer}
	\begin{pgfonlayer}{edgelayer}
		\draw (38) to (39);
		\draw (40) to (41);
		\draw (42) to (43);
		\draw (44) to (45);
		\draw (29.center) to (25.center);
		\draw (24.center) to (30.center);
		\draw (38) to (28);
		\draw (40) to (32.center);
		\draw (47) to (42);
		\draw (44) to (33.center);
	\end{pgfonlayer}
\end{tikzpicture}
$$
%Notice that the choice of which wires to straighten out the zig-zag is arbitrary.
\end{proof}
\begin{lemma}
\label{lem:parisoaffcb}
$\sfpariso\acb_{\F_2}$ is a presentation for the full subcategory $(\Par\Iso(\Aff\Mat_{\F_2}+1)^*, \oplus)$ of $(\Par\Iso(\Aff\Mat_{\F_2}+1),\oplus)$ where the objects are nonempty affine vector spaces.
\end{lemma}
\begin{proof}
The obvious functor $\sfpariso\acb_{\F_2}\to (\Par\Iso(\Aff\Mat_{\F_2}+1)^*,\oplus)$ is clearly full,  as well as an isomorphism on ojects.
It remains to show it is faihful.  It is faithful on maps which are taken to spans with nonempty apex by the same argument as Lemma \ref{lem:parisocb}. For empty case, there is exactly one diagram of each type with a factor of $0$; and similarly, there is exactly one span with an empty apex.
\end{proof}
This is equivalent to the prop $\CNOT$ presented in \cite{cnot}.   Therefore the presentation we have given for $(\Par\Iso(\Aff\Mat_{\F_2}+1)^*,\oplus)$  can be simplified so that the identities given in Definition \ref{def:isoaffcbzero} can be replaced by the following equation:
$$
\begin{tikzpicture}
	\begin{pgfonlayer}{nodelayer}
		\node [style=X] (0) at (0, 5) {$1$};
		\node [style=none] (1) at (0.5, 5.75) {};
		\node [style=none] (2) at (0.5, 4.25) {};
	\end{pgfonlayer}
	\begin{pgfonlayer}{edgelayer}
		\draw (2.center) to (1.center);
	\end{pgfonlayer}
\end{tikzpicture}
\eqzxa{zero.six}
\begin{tikzpicture}
	\begin{pgfonlayer}{nodelayer}
		\node [style=X] (53) at (46, 5) {$1$};
		\node [style=none] (54) at (46.625, 5.825) {};
		\node [style=none] (55) at (46.625, 4.175) {};
		\node [style=X] (56) at (46.625, 5.325) {$1$};
		\node [style=X] (57) at (46.625, 4.675) {$1$};
	\end{pgfonlayer}
	\begin{pgfonlayer}{edgelayer}
		\draw (56) to (54.center);
		\draw (57) to (55.center);
	\end{pgfonlayer}
\end{tikzpicture}
$$
To get to partial maps we add the effect $\langle +|$:
\begin{definition}
Let $\sfpar\acb_{\F_2}$ denote the pushout of the diagram of props:
$$
\sfpariso\acb_{\F_2} \leftarrow \surj^\op \rightarrow \cm^\op
$$
\end{definition}
\begin{lemma}
\label{lem:paraffcb}
$\sfpar\acb_{\F_2}$ is a presentation for the prop $(\Par(\Aff\Mat_{\F_2}+1)^*,\oplus)$.
\end{lemma}
The proof is essentially the same as for Lemma \ref{lem:parcb}.
%\begin{proof}
%%\renewcommand{\cubetopbl}{$\sfmono(\acb_{\F_2})$}
%%\renewcommand{\cubetopbr}{$\sfmono(\acb_{\F_2})+1\otimes_{\Iso(\acb_{\F_2})+1} \sfmono(\acb_{\F_2})+1^\op$}
%%\renewcommand{\cubetopfl}{$\sfmono(\acb_{\F_2})\otimes_{\Iso(\acb_{\F_2})} \sfepi(\acb_{\F_2})^\op$}
%%\renewcommand{\cubetopfr}{$\Par(\acb_{\F_2})$}
%%\renewcommand{\cubebotbl}{$(\sfmono(\Aff\Mat_{\F_2}),+)$ }
%%\renewcommand{\cubebotbr}{$(\ParIso(\Aff\Mat_{\F_2}+1)^*,+)$ }
%%\renewcommand{\cubebotfl}{$(\Par\sfepi(\Aff\Mat_{\F_2}),+)^\op$ }
%%\renewcommand{\cubebotfr}{}
%%
%%$$
%%\xymatrixrowsep{3mm}\xymatrixcolsep{-10mm}
%%\xymatrix{
%%                                       & \mbox{\cubetopbl} \ar[rr] \ar[dl] \ar[dd]^(.7){\cong}      &                                                  & \mbox{\cubetopbr}  \ar[dd]^{\cong} \ar[dl] \\
%%\mbox{\cubetopfl} \ar[rr]  \ar[dd]_{\cong}           &                                                                                              &\mbox{\cubetopfr} \ar@{-->}[dd]^(.35){\cong}   \skewpullbackcorner[ul]              \\
%%                                       &  \mbox{\cubebotbl} \ar[dl] \ar[rr]                    &                                                  & \mbox{\cubebotbr} \ar@/^1pc/[ddl] \ar[dl] \\
%%\mbox{\cubebotfl} \ar@/_1pc/[drr] \ar[rr]  &                                                                                             & \mbox{\cubebotfr} \skewpullbackcorner[ul]    \ar@{-->}[d]^{\cong}  \\
%%                                                   &                                                                                             & (\Par(\Aff\Mat_{\F_2}),+)
%%}
%%$$
%\renewcommand{\cubetopbl}{$\surj^\op$}
%\renewcommand{\cubetopbr}{$\cm^\op$}
%\renewcommand{\cubetopfl}{$\sfpariso\acb_{\F_2}$}
%\renewcommand{\cubetopfr}{$\sfpar\acb_{\F_2}$}
%\renewcommand{\cubebotbl}{$\surj^\op$ }
%\renewcommand{\cubebotbr}{$\cm^\op$ }
%\renewcommand{\cubebotfl}{$(\ParIso(\Aff\Mat_{\F_2}+1)^*,\oplus)$ }
%\renewcommand{\cubebotfr}{}
%$$
%\xymatrixrowsep{2mm}\xymatrixcolsep{1mm}
%\xymatrix{
%                                       & \mbox{\cubetopbl} \ar[rr] \ar[dl] \ar@{=}[dd]     &                                                  & \mbox{\cubetopbr} \ar@{=}[dd] \ar[dl] \\
%\mbox{\cubetopfl} \ar[rr]  \ar[dd]_{\cong}           &                                                                                              &\mbox{\cubetopfr} \ar@{-->}[dd]^(.35){\cong}   \skewpullbackcorner[ul]              \\
%                                       &  \mbox{\cubebotbl} \ar[dl] \ar[rr]                    &                                                  & \mbox{\cubebotbr} \ar@/^1pc/[ddl] \ar[dl] \\
%\mbox{\cubebotfl} \ar@/_1pc/[drr] \ar[rr]  &                                                                                             & \mbox{\cubebotfr} \skewpullbackcorner[ul]    \ar@{-->}[d]_F^{\cong}  \\
%                                                   &                                                                                             & (\Par(\Aff\Mat_{\F_2}+1)^*,\oplus) 
%}
%$$
%We know that $\sfpariso\acb_{\F_2}\cong( \ParIso(\Aff\Mat_{\F_2}+1)^*,\oplus)$ is a discrete inverse category by \cite[Prop. 3.4]{cnot}.
%
%The cube commutes by the universal property of the pushout, as before.
%
%We just have to show that the universal map $F$ is an isomorphism.  It is clearly the identity on objects, so we just have to show it is full and faithful.
%This follows from essentially the same argument as in the linear case.
%\end{proof}
By adding the state $|+\rangle$ we get a presentation for affine spans:
\begin{definition}
Let $\sfspan\acb_{\F_2}$ denote the pushout of the diagram of props:
$$
 \sfpar\acb_{\F_2}^\op \leftarrow \sfpariso\acb_{\F_2} \rightarrow \sfpar\acb_{\F_2}
$$
\end{definition}
\begin{lemma}
\label{lem:spanaffcb}
$\sfspan\acb_{\F_2}$ is a presentation for the prop $(\Span^\sim(\Aff\Mat_{\F_2}+1)^*,\oplus)$.
\end{lemma}
\begin{proof} 
We show that the following diagram commutes and that the vertical maps are isomorphisms:

\renewcommand{\cubetopbl}{$\sfpariso\acb_{\F_2}$}
\renewcommand{\cubetopbr}{$\sfpar\acb_{\F_2}$}
\renewcommand{\cubetopfl}{$\sfpar\acb_{\F_2}^\op$}
\renewcommand{\cubetopfr}{$\sfspan\acb_{\F_2}$}
\renewcommand{\cubebotbl}{$(\ParIso(\Aff\Mat_{\F_2}+1)^*,\oplus)$ }
\renewcommand{\cubebotbr}{$(\Par(\Aff\Mat_{\F_2}+1)^*,\oplus)$ }
\renewcommand{\cubebotfl}{$((\Par(\Aff\Mat_{\F_2}+1)^*)^\op,\oplus)$ }
\renewcommand{\cubebotfr}{}

\scalebox{.8}{$
\xymatrixrowsep{1.7mm}\xymatrixcolsep{.5mm}
\xymatrix{
                                       & \mbox{\cubetopbl} \ar[rr] \ar[dl] \ar[dd]^(.7){\cong}      &                                                  & \mbox{\cubetopbr}  \ar[dd]^{\cong} \ar[dl] \\
\mbox{\cubetopfl} \ar[rr]  \ar[dd]_{\cong}           &                                                                                              &\mbox{\cubetopfr} \ar@{-->}[dd]    \skewpullbackcorner[ul]              \\
                                       &  \mbox{\cubebotbl} \ar[dl] \ar[rr]                    &                                                  & \mbox{\cubebotbr} \ar@/^1pc/[ddl] \ar[dl] \\
\mbox{\cubebotfl} \ar@/_1pc/[drr] \ar[rr]  &                                                                                             & \mbox{\cubebotfr} \skewpullbackcorner[ul]    \ar@{-->}[d]_F  \\
                                                   &                                                                                             & (\Span^\sim(\Aff\Mat_{\F_2}+1)^*,\oplus)
}
$}

 The rear and left faces of the cube commute and the vertical maps are all isomorphisms. Therefore, the whole cube commutes by the universal property of the pushout, with the upper universal map being an isomorphism.

We seek to show that the lower universal map  $F$ is also an isomorphism.  It is clearly the identity on objects, so we just have to show fullness and faithfulness.

For fullness, let us first consider the nonempty case; that is a map $\F_2^n \xleftarrow{(A,x)} \F_2^k \xrightarrow{(B,y)}\F^m$ in $(\Span^\sim (\Aff\Mat(\F_2)+1)^*,\oplus)$.  This is in the image of the following diagram under $F$:
$$
(\F_2^n \xleftarrow{(A,x)} \F_2^k  = \F_2^k); (\F_2^k = \F_2^k  \xrightarrow{(B,y)}\F^m)
$$ 
Otherwise, consider a map of the form  $\F_2^n \xleftarrow{?} \emptyset  \xrightarrow{?}\F^m$.  This is in the image of the following diagram:
$$
(\F_2^n \xleftarrow{?} \emptyset \xrightarrow {?} \F_2^0  );(\F_2^0 \xleftarrow{?} \emptyset  \xrightarrow{?}\F^m)
$$
For faithfulness, again, we separate the proof into two cases.  The functor is faithful on diagrams in $(\Span^\sim(\Aff\Mat_{\F_2}+1)^*,\oplus)$ with nonempty apex by the same argument as in Lemma \ref{lem:spancb}.
%$$
%\xymatrix{
%          & \F_2^k \ar[dl]_{(A',x')} \ar[dd]_{\cong}^{(C,z)} \ar[dr]^{(B',y')}\\
%\F_2^n  &                                                                                                    & \F_2^m\\
%         & \F_2^k \ar[ul]^{(A,x)} \ar[ur]_{(B,y)}\\
%}
%$$
%We have the following equation in $\Span(\acb_{\F_2})$:
%{
%\xymatrixrowsep{1mm}\xymatrixcolsep{3.5mm}
%\begin{align*}
%\xymatrix{
%          & \F_2^k \ar[dl]_{(A,x)}  \ar@{=}[dr]\\
%\F_2^n  &                                                                                                    & \F_2^k\\
%};
%\xymatrix{
%          & \F_2^k \ar[dr]^{(B,y)}  \ar@{=}[dl]\\
%\F_2^k  &                                                                                                    & \F_2^m\\
%} &=
%\xymatrix{
%          & \F_2^k \ar[dl]_{(A,x)}  \ar@{=}[dr]\\
%\F_2^n  &                                                                                                    & \F_2^k\\
%};
%\xymatrix{
%          & \F_2^k \ar@{=}[dr] \ar@{=}[dl] \\
%\F_2^k  &                                                                                                    & \F_2^k\\
%         & \F_2^k \ar[ul]^{(C,z)} \ar[ur] _{(C,z)} \ar[uu]^{\cong}_{(C,z)}
%};
%\xymatrix{
%          & \F_2^k \ar[dr]^{(B,y)}  \ar@{=}[dl]\\
%\F_2^k  &                                                                                                    & \F_2^m\\
%}\\
%&=
%\xymatrix{
%          & \F_2^k \ar[dl]_{(A,x)}  \ar@{=}[dr]\\
%\F_2^n  &                                                                                                    & \F_2^k\\
%};
%\xymatrix{
%        & \F_2^k \ar[dl]_{(C,z)} \ar@{=}[dr]\\
%\F_2^k  &                                                     & \F_2^k
%};
%\xymatrix{
%        & \F_2^k \ar[dr]^{(C,z)} \ar@{=}[dl]\\
%\F_2^k  &                                                     & \F_2^k
%};
%\xymatrix{
%          & \F_2^k \ar[dr]^{(B,y)}  \ar@{=}[dl]\\
%\F_2^k  &                                                                                                    & \F_2^m\\
%}\\
%&=
%\xymatrix{
%            &                                                        & \F_2^k \ar[dl]_{(C,z)} \ar@{=}[dr] \ar@/_2.0pc/[ddll]_{(A',x')}\\
%            & \F_2^k \ar@{=}[dr] \ar[dl]^{(A,x)}&                                                          & \F_2^k \ar@{=}[dr] \ar[dl]_{(C,z)}\\
%\F_2^k &                                                         & \F_2^k                                             &                                                         &\F_2^k
%};
%\xymatrix{
%            &                                                        & \F_2^k \ar[dr]^{(C,z)} \ar@{=}[dl] \ar@/^2.0pc/[ddrr]^{(B',y')}  \\
%            & \F_2^k \ar[dr]^{(C,z)}   \ar@{=}[dl] &                                                          & \F_2^k \ar@{=}[dl] \ar[dr]_{(B,y)}\\
%\F_2^k &                                                         & \F_2^k                                             &                                                         &\F_2^k
%}
%\end{align*}
%}
The case for spans with empty apex follows immediately as the only endomorphism on the empty set is the identity; thus,  isomorphic spans must be equal on the nose.
\end{proof}
This gives a recipe for constructing the props of affine spans over arbitrary fields.  To my knowledge there is no presentation for affine isomorphisms, affine monororphisms for arbitrary fields; however, one doesn't need to have presentations for all the intermediary categories in order to build a presentation for affine spans.  In \cite[\S 3.3]{ih}, the category of linear spans over a principle ideal domain is constructed similarly without ever giving a presentation for the linear isomorphisms, so the situation should be completely analogous.  We won't give the presentation here, because it doesn't bring much insight to this subject.  
\subsection{Adding the  \texorpdfstring{$\AND$}{and}-gate}
\label{sec:three}
In this subsection we do the same thing as in the previous two subsections but  in the nonlinear setting.
\begin{definition}
Consider the prop of bicommutative bialgebras $\cb_{\B}$, where the comonoid is drawn as $\zcirc$ and the monoid is drawn as follows:
$$
\left(
\begin{tikzpicture}
	\begin{pgfonlayer}{nodelayer}
		\node [style=none] (0) at (-3.75, 0.5) {};
		\node [style=none] (1) at (-3.75, -0.25) {};
		\node [style=andin] (2) at (-3.75, -0.25) {};
		\node [style=none] (3) at (-4, -1) {};
		\node [style=none] (4) at (-3.5, -1) {};
	\end{pgfonlayer}
	\begin{pgfonlayer}{edgelayer}
		\draw (0.center) to (1.center);
		\draw [in=-60, out=90, looseness=1.00] (4.center) to (1.center);
		\draw [in=90, out=-120, looseness=1.00] (1.center) to (3.center);
	\end{pgfonlayer}
\end{tikzpicture},
\begin{tikzpicture}
	\begin{pgfonlayer}{nodelayer}
		\node [style=none] (0) at (-3.75, -0.25) {};
		\node [style=X] (1) at (-3.75, -1) {$1$};
	\end{pgfonlayer}
	\begin{pgfonlayer}{edgelayer}
		\draw (0.center) to (1);
	\end{pgfonlayer}
\end{tikzpicture}
\right)
$$
There is a distributive law of Lawvere theories:
$$
\f_2:=\cb_\B \otimes_{\cm^\op} \cb_{\F_2};
\begin{tikzpicture}
	\begin{pgfonlayer}{nodelayer}
		\node [style=andin] (4) at (1.25, 0.5) {};
		\node [style=X] (5) at (0.75, -0.5) {};
		\node [style=none] (6) at (0.5, -1) {};
		\node [style=none] (7) at (1, -1) {};
		\node [style=none] (8) at (1.75, -1) {};
		\node [style=none] (9) at (1.25, 0.5) {};
		\node [style=none] (10) at (1.25, 1.5) {};
	\end{pgfonlayer}
	\begin{pgfonlayer}{edgelayer}
		\draw [in=-30, out=90] (8.center) to (9.center);
		\draw [in=90, out=-150] (9.center) to (5);
		\draw [in=90, out=-45] (5) to (7.center);
		\draw [in=-135, out=90] (6.center) to (5);
		\draw (9.center) to (10.center);
	\end{pgfonlayer}
\end{tikzpicture}
\eref{semiring.mult.r}
\begin{tikzpicture}
	\begin{pgfonlayer}{nodelayer}
		\node [style=none] (0) at (1, 0) {};
		\node [style=none] (1) at (0.5, -1.25) {};
		\node [style=none] (2) at (1.75, -0.75) {};
		\node [style=none] (3) at (1.33, 0.75) {};
		\node [style=andin] (4) at (1, 0) {};
		\node [style=none] (5) at (1.75, 0) {};
		\node [style=none] (6) at (1, -1.25) {};
		\node [style=none] (7) at (1.75, -0.75) {};
		\node [style=none] (8) at (1.33, 0.75) {};
		\node [style=andin] (9) at (1.75, 0) {};
		\node [style=X] (10) at (1.33, 0.75) {};
		\node [style=none] (11) at (1.33, 1.25) {};
		\node [style=none] (12) at (1.75, -1.25) {};
		\node [style=Z] (13) at (1.75, -0.75) {};
	\end{pgfonlayer}
	\begin{pgfonlayer}{edgelayer}
		\draw [in=-135, out=90] (0.center) to (3.center);
		\draw [in=165, out=-30, looseness=1.25] (0.center) to (2.center);
		\draw [in=-45, out=90] (5.center) to (8.center);
		\draw [in=45, out=-45, looseness=1.25] (5.center) to (7.center);
		\draw (10) to (11.center);
		\draw [in=90, out=-150] (4) to (1.center);
		\draw [in=-150, out=90] (6.center) to (9);
		\draw (12.center) to (13);
	\end{pgfonlayer}
\end{tikzpicture},
\hspace*{.5cm}
\begin{tikzpicture}
	\begin{pgfonlayer}{nodelayer}
		\node [style=none] (0) at (2, 0) {};
		\node [style=none] (1) at (1.75, -0.75) {};
		\node [style=none] (2) at (2.25, -0.75) {};
		\node [style=none] (3) at (2, 0.5) {};
		\node [style=none] (4) at (2.25, -1) {};
		\node [style=X] (5) at (1.75, -0.75) {};
		\node [style=andin] (6) at (2, 0) {};
	\end{pgfonlayer}
	\begin{pgfonlayer}{edgelayer}
		\draw (0.center) to (3.center);
		\draw [in=90, out=-45] (0.center) to (2.center);
		\draw (4.center) to (2.center);
		\draw [in=-135, out=90] (1.center) to (0.center);
	\end{pgfonlayer}
\end{tikzpicture}
\eref{semiring.unit.l}
\begin{tikzpicture}
	\begin{pgfonlayer}{nodelayer}
		\node [style=none] (12) at (2, 0.5) {};
		\node [style=none] (14) at (2, -1) {};
		\node [style=X] (15) at (2, 0) {};
		\node [style=Z] (16) at (2, -0.5) {};
	\end{pgfonlayer}
	\begin{pgfonlayer}{edgelayer}
		\draw (15) to (12.center);
		\draw (16) to (14.center);
	\end{pgfonlayer}
\end{tikzpicture}
$$
Where $\cm^\op$ picks out the comonoid $\zcirc$ of $\cb_\B$ and $\cb_{\F_2}$.
\end{definition}
\begin{lemma}[{\cite[Theorem 10]{lafont}}]
$\f_2$ is a presentation for full subcategory of finite ordinals and functions whose objects are powers of 2, $(\FinOrd_2,\times)$, regarded as a prop with respect to the product.
\end{lemma}
By Stone duality, we see that this is also a presentation for the Lawvere theory of polynomial functions over $\F_2$.  We will come back to this alternative view slightly later.

To find larger fragments, it will be useful to first identify the isomorphisms and the monics of $\f_2$.
\begin{definition}
Given a map $f:n\to 1$ in  $\f_2$, the {\bf oracle} for $f$, ${\mathcal O}_f$ is defined as follows:
$$
\begin{tikzpicture}
	\begin{pgfonlayer}{nodelayer}
		\node [style=Z] (11) at (7.5, 0.25) {};
		\node [style=X] (12) at (8.25, 2.25) {};
		\node [style=map] (13) at (7.75, 1.5) {$f$};
		\node [style=none] (14) at (7.25, 2.75) {};
		\node [style=none] (15) at (8.25, 2.75) {};
		\node [style=none] (16) at (8.25, -0.25) {};
		\node [style=none] (17) at (7.5, -0.25) {};
		\node [style=Z] (18) at (6.5, 0.25) {};
		\node [style=none] (19) at (6.25, 2.75) {};
		\node [style=none] (20) at (6.5, -0.25) {};
		\node [style=none] (21) at (7, 0) {$\cdots$};
		\node [style=none] (22) at (6.75, 2.5) {$\cdots$};
		\node [style=none] (23) at (6.75, 2.75) {$n$};
		\node [style=none] (24) at (7, -0.25) {$n$};
	\end{pgfonlayer}
	\begin{pgfonlayer}{edgelayer}
		\draw (17.center) to (11);
		\draw [in=-60, out=60] (11) to (13);
		\draw [in=-120, out=90] (13) to (12);
		\draw (12) to (15.center);
		\draw [in=90, out=-60] (12) to (16.center);
		\draw [in=-90, out=120] (11) to (14.center);
		\draw (20.center) to (18);
		\draw [in=-90, out=120] (18) to (19.center);
		\draw [in=45, out=-120] (13) to (18);
	\end{pgfonlayer}
\end{tikzpicture}
$$
These are called oracles, because these correspond to the reversible implementations of Boolean functions which are queried  by quantum circuits.
\end{definition}
\begin{lemma}
The oracles in $f_2$ are generated by the generalized controlled-not gates:
$$
\left\llbracket\
\begin{tikzpicture}
	\begin{pgfonlayer}{nodelayer}
		\node [style=oplus] (0) at (0, 1.5) {};
		\node [style=none] (1) at (0, 2) {};
		\node [style=none] (2) at (0, 1) {};
	\end{pgfonlayer}
	\begin{pgfonlayer}{edgelayer}
		\draw (0) to (1.center);
		\draw (0) to (2.center);
	\end{pgfonlayer}
\end{tikzpicture}
\ \right\rrbracket
=
\begin{tikzpicture}
	\begin{pgfonlayer}{nodelayer}
		\node [style=none] (7) at (5.25, 2) {};
		\node [style=X] (9) at (5.25, 1.375) {$1$};
		\node [style=none] (10) at (5.25, 0.75) {};
	\end{pgfonlayer}
	\begin{pgfonlayer}{edgelayer}
		\draw (10.center) to (9);
		\draw (9) to (7.center);
	\end{pgfonlayer}
\end{tikzpicture},
\hspace*{.5cm}
\left\llbracket\
\begin{tikzpicture}
	\begin{pgfonlayer}{nodelayer}
		\node [style=oplus] (0) at (0, 1.5) {};
		\node [style=none] (1) at (0, 2) {};
		\node [style=none] (2) at (0, 1) {};
		\node [style=none] (4) at (-0.5, 2) {};
		\node [style=none] (5) at (-0.5, 1) {};
		\node [style=dot] (6) at (-0.5, 1.5) {};
	\end{pgfonlayer}
	\begin{pgfonlayer}{edgelayer}
		\draw (0) to (1.center);
		\draw (0) to (2.center);
		\draw (0) to (6);
		\draw (6) to (4.center);
		\draw (6) to (5.center);
	\end{pgfonlayer}
\end{tikzpicture}
\ \right\rrbracket
=
\begin{tikzpicture}[xscale=-1]
	\begin{pgfonlayer}{nodelayer}
		\node [style=X] (0) at (0.75, 0) {};
		\node [style=Z] (1) at (1.25, -0.5) {};
		\node [style=none] (2) at (0.5, -1) {};
		\node [style=none] (3) at (1.25, -1) {};
		\node [style=none] (4) at (1.5, 0.5) {};
		\node [style=none] (5) at (0.75, 0.5) {};
	\end{pgfonlayer}
	\begin{pgfonlayer}{edgelayer}
		\draw (5.center) to (0);
		\draw [in=150, out=-30] (0) to (1);
		\draw [in=-90, out=60, looseness=0.75] (1) to (4.center);
		\draw (1) to (3.center);
		\draw [in=90, out=-120, looseness=0.75] (0) to (2.center);
	\end{pgfonlayer}
\end{tikzpicture},
\hspace*{.5cm}
\left\llbracket\
\begin{tikzpicture}
	\begin{pgfonlayer}{nodelayer}
		\node [style=oplus] (0) at (-0.25, 1.5) {};
		\node [style=none] (1) at (-0.25, 2) {};
		\node [style=none] (2) at (-0.25, 1) {};
		\node [style=none] (4) at (-0.75, 2) {};
		\node [style=none] (5) at (-0.75, 1) {};
		\node [style=dot] (6) at (-0.75, 1.5) {};
		\node [style=none] (7) at (-1.75, 2) {};
		\node [style=none] (8) at (-1.75, 1) {};
		\node [style=dot] (9) at (-1.75, 1.5) {};
		\node [style=none] (10) at (-1, 1.5) {};
		\node [style=none] (11) at (-1.5, 1.5) {};
		\node [style=none] (12) at (-1.25, 1.5) {$\cdots$};
		\node [style=none] (13) at (-1.25, 1.75) {$n$};
	\end{pgfonlayer}
	\begin{pgfonlayer}{edgelayer}
		\draw (0) to (1.center);
		\draw (0) to (2.center);
		\draw (0) to (6);
		\draw (6) to (4.center);
		\draw (6) to (5.center);
		\draw (9) to (7.center);
		\draw (9) to (8.center);
		\draw (11.center) to (9);
		\draw (10.center) to (6);
	\end{pgfonlayer}
\end{tikzpicture}
\ \right\rrbracket
=
\begin{tikzpicture}
	\begin{pgfonlayer}{nodelayer}
		\node [style=Z] (0) at (-10.25, 0.25) {};
		\node [style=Z] (1) at (-11.25, 0.25) {};
		\node [style=none] (2) at (-10.75, 1) {};
		\node [style=X] (3) at (-9.75, 1.75) {};
		\node [style=none] (4) at (-11.25, -0.5) {};
		\node [style=none] (5) at (-10.25, -0.5) {};
		\node [style=none] (6) at (-9.75, -0.5) {};
		\node [style=none] (7) at (-9.75, 2.25) {};
		\node [style=none] (8) at (-10.25, 2.25) {};
		\node [style=none] (9) at (-11.25, 2.25) {};
		\node [style=andin] (10) at (-10.75, 1) {};
		\node [style=none] (11) at (-10.75, 2.25) {$n$};
		\node [style=none] (12) at (-10.75, 0.25) {$n$};
		\node [style=none] (13) at (-10.75, 2) {$\cdots$};
		\node [style=none] (14) at (-10.75, 0.5) {$\cdots$};
	\end{pgfonlayer}
	\begin{pgfonlayer}{edgelayer}
		\draw (4.center) to (1);
		\draw (1) to (2.center);
		\draw (2.center) to (0);
		\draw (0) to (5.center);
		\draw (6.center) to (3);
		\draw [in=90, out=-146, looseness=1.50] (3) to (2.center);
		\draw [in=-90, out=120, looseness=1.00] (1) to (9.center);
		\draw [in=-90, out=60, looseness=0.75] (0) to (8.center);
		\draw (3) to (7.center);
	\end{pgfonlayer}
\end{tikzpicture}
$$
%and the equations of Iwama et al, where $[n,X]$ denotes an $|X|$-controlled not gate controlled by the wires indexed by the set X, and targetting the wire $n \notin X$  \cite{iwama} generalized b
%https://web.eecs.umich.edu/~imarkov/pubs/jour/tcad03-iwls.pdf
%\begin{description}
%\item $[x,X][x,X]=1$
%\item  When the target wire are the same $[x,X][x,Y] = [x,Y] [x,X]$
%\item  When $x \not\in Y$ and $y \not\in X$ then $[x,X][y,Y] = [y,Y] [x,X]$
%\item  $[x,X] [y,{\{ x\} \sqcup Y}] = [y,{X \cup Y}][y,{\{ x\} \sqcup Y}] [x,X]$
%\item For $x \neq y$, $[x,] $
%\end{description}
%
%
%TODO
\end{lemma}
\begin{proof}
Any  Boolean function of $n$ arguments can be represented by a polynomial in\\
 $\F_2[x_1,\ldots, x_n]/\langle x_1^2-x_1,\ldots x_n^2-x_n\rangle$.  Every polynomial in this quotient ring has a unique normal form given by sums of products (which is not true for arbitrary finite fields).  Each product corresponds to a generalized controlled-not gate, and the sum corresponds to composing these generalized controlled-not gates in sequence.
\end{proof}
%We will also allow generalized controlled not gates controlled and targetting arbitrary wires, possibly with gaps in the middle.
These generate all reversible Boolean circuits according to this classical result:
\begin{lemma}[{\cite[Theorem 5.1]{toffolireversible}}]
The oracles in $\f_2$ in addition to permutations generate all of $\Iso(\f_2)$.
\end{lemma}
%This actually follows from https://arxiv.org/pdf/quant-ph/0207001.pdf
%Scratch space is used to construct n-bit cnot gate
%http://theory.caltech.edu/~preskill/ph229/notes/chap6.pdf
%
%In particular, the braid is derived from the 3 $\cnot$ gates:
%
%$$
%\begin{tikzpicture}
%	\begin{pgfonlayer}{nodelayer}
%		\node [style=none] (147) at (41.5, -1.5) {};
%		\node [style=none] (148) at (42.25, -1.5) {};
%		\node [style=none] (149) at (42.25, 0.5) {};
%		\node [style=none] (150) at (41.5, 0.5) {};
%	\end{pgfonlayer}
%	\begin{pgfonlayer}{edgelayer}
%		\draw [in=90, out=-90] (149.center) to (147.center);
%		\draw [in=270, out=90] (148.center) to (150.center);
%	\end{pgfonlayer}
%\end{tikzpicture}
%:=
%\begin{tikzpicture}
%	\begin{pgfonlayer}{nodelayer}
%		\node [style=none] (137) at (43, -1.5) {};
%		\node [style=none] (138) at (43.75, -1.5) {};
%		\node [style=none] (139) at (43, 0.5) {};
%		\node [style=none] (140) at (43.75, 0.5) {};
%		\node [style=oplus] (141) at (43, 0) {};
%		\node [style=oplus] (142) at (43, -1) {};
%		\node [style=oplus] (143) at (43.75, -0.5) {};
%		\node [style=dot] (144) at (43, -0.5) {};
%		\node [style=dot] (145) at (43.75, 0) {};
%		\node [style=dot] (146) at (43.75, -1) {};
%	\end{pgfonlayer}
%	\begin{pgfonlayer}{edgelayer}
%		\draw (137.center) to (142);
%		\draw (142) to (144);
%		\draw (144) to (141);
%		\draw (141) to (139.center);
%		\draw (140.center) to (145);
%		\draw (145) to (143);
%		\draw (143) to (146);
%		\draw (146) to (138.center);
%		\draw (146) to (142);
%		\draw (144) to (143);
%		\draw (145) to (141);
%	\end{pgfonlayer}
%\end{tikzpicture}
%$$
Recall the notation of a generalized controlled not gate controlled by wires indexed by $X$, operating on $x\notin X$ by $\llparenthesis X,x\rrparenthesis$.


%
%
%Iwama et al,  give a set of identities which are complete for oracles, not general isomorphisms \cite{iwama}.
%Shende restated these identities using the commutator \cite{shende}.






Iwama et al \cite{iwama} originally gave a complete set of identities for circuits generated by generalized controlled not gates with one extra ancillary bit.  It is worth mentioning that Shende et al. later used the commutator to generalize some of these identities \cite[Corollary 26]{shende}.  We conjecture that a very similar set of identities is complete for Boolean isomorphisms: 
%Recall that the {\bf commutator} of two group elements $f,g$ is the element $[f,g]:=fgf^{-1}g^{-1}$; therefore $fg=[f,g]gf$.
%Shende gives a way to compute commutators of controlled isomorphsisms in $\Iso(\FinOrd_2)$.  In particular, take isomorphisms $f,g$  which only change bits in sets indexed by $X,Y$, respectively.  Then if we control these gates by $Z$ and $W$ respectively, denoted by $f^Z,g^W$, we have :
%$$
%[f,g] = [f^{Y * Z}, g^{X*W}]^{\bar{(X\#Y)} * (W\#Z) }
%$$
%Where $X\# Y $ is the defined as the pointwise xor, and $X * Y$ is defined as the pointwise and.
\begin{conjecture}
%Denote a generalized controlled-not gate controlled from wires indexed by $X$ and operating on the wire $x$ by $\lbparen X,x \rbparen$.
We conjecture that $(\Iso(\FinOrd_2),\times)$ is presented by the prop generated by all generalized controlled-not gates modulo the following identities:
\begin{itemize}
\item $\llparenthesis X,x\rrparenthesis ; \llparenthesis X,x \rrparenthesis= 1$.
%So the first identity is that  in this case $\llparenthesis X,x\rrparenthesis\llparenthesis Y,y\rrparenthesis=\llparenthesis Y,y\rrparenthesis\llparenthesis X,x\rrparenthesis$
\item
If  $x \notin Y $ and $ y \notin X$ then $\llparenthesis X,x\rrparenthesis ;\llparenthesis Y,y\rrparenthesis =\llparenthesis Y,y\rrparenthesis; \llparenthesis X,x\rrparenthesis $.
%
%So the second identity is that in this case:
%$$
%\llparenthesis  X,x\rrparenthesis  \llparenthesis  Y ,y\rrparenthesis = \llparenthesis \llparenthesis  X \cup Y -y,x\rrparenthesis \rrparenthesis  \llparenthesis  Y ,y\rrparenthesis  \llparenthesis  X,x\rrparenthesis  
%$$
\item
If $x \notin Y$, then $\llparenthesis X,x\rrparenthesis; \llparenthesis \{x\} \sqcup Y, y\rrparenthesis = \llparenthesis X\cup Y,y\rrparenthesis ; \llparenthesis \{x\} \sqcup Y, y\rrparenthesis;  \llparenthesis X,x\rrparenthesis $.
\item
If $x \notin Y$, then $ \llparenthesis \{x\} \sqcup Y, y\rrparenthesis ; \llparenthesis X,x\rrparenthesis = \llparenthesis X,x\rrparenthesis;   \llparenthesis \{x\} \sqcup Y, y\rrparenthesis ; \llparenthesis X\cup Y,y\rrparenthesis $.
%So the third identity is that $\llparenthesis  X,x\rrparenthesis ^2= 1 $.
\item
If $x \in Y$ and $y \in X$, then
$
\llparenthesis  X,x \rrparenthesis ; \llparenthesis  Y,y \rrparenthesis ;  \llparenthesis  X,x \rrparenthesis 
=
\llparenthesis  Y,y \rrparenthesis ;  \llparenthesis  X,x \rrparenthesis ;  \llparenthesis  Y,y \rrparenthesis 
$.
\end{itemize}
\end{conjecture}
Although we aren't sure if these identities are complete, it doesn't matter in the end.  Eventually once we add enough generators and identities, we get a finite, complete presentation $\ZXA$.  Therefore take $\Iso(\f_2)$ to be the prop generated by the generalized controlled-not gates of which no complete set of equations is known.  



Modulo a complete axiomatization of $\Iso(\f_2)$, by adding the state $|0\rangle$, we get a complete axiomatization of the monomorphisms in $\FinOrd_2$.
\begin{definition}
Let $\sfinj\f_2$ be the prop given by adjoining the grey unit to $\Iso(\f_2)$ modulo:
$$
\begin{tikzpicture}
	\begin{pgfonlayer}{nodelayer}
		\node [style=oplus] (0) at (2, 1.5) {};
		\node [style=none] (1) at (2, 2.25) {};
		\node [style=none] (2) at (2, 0.75) {};
		\node [style=none] (3) at (1.5, 2.25) {};
		\node [style=none] (4) at (1.5, 0.75) {};
		\node [style=dot] (5) at (1.5, 1.5) {};
		\node [style=none] (6) at (0.5, 2.25) {};
		\node [style=dot] (7) at (0.5, 1.5) {};
		\node [style=none] (8) at (1.25, 1.5) {};
		\node [style=none] (9) at (0.75, 1.5) {};
		\node [style=none] (10) at (1, 1.5) {$\cdots$};
		\node [style=none] (11) at (1, 1.75) {$n$};
		\node [style=none] (13) at (0, 2.25) {};
		\node [style=dot] (14) at (0, 1.5) {};
		\node [style=X] (15) at (0, 1) {};
		\node [style=none] (16) at (0.5, 0.75) {};
	\end{pgfonlayer}
	\begin{pgfonlayer}{edgelayer}
		\draw (0) to (1.center);
		\draw (0) to (2.center);
		\draw (0) to (5);
		\draw (5) to (3.center);
		\draw (5) to (4.center);
		\draw (7) to (6.center);
		\draw (9.center) to (7);
		\draw (8.center) to (5);
		\draw (14) to (13.center);
		\draw (15) to (14);
		\draw (16.center) to (7);
		\draw (7) to (14);
	\end{pgfonlayer}
\end{tikzpicture}
\eqzxa{mono.ftwo}
\begin{tikzpicture}
	\begin{pgfonlayer}{nodelayer}
		\node [style=none] (1) at (2, 2.25) {};
		\node [style=none] (2) at (2, 0.75) {};
		\node [style=none] (3) at (1.5, 2.25) {};
		\node [style=none] (4) at (1.5, 0.75) {};
		\node [style=none] (6) at (0.5, 2.25) {};
		\node [style=none] (10) at (1, 1.5) {$\cdots$};
		\node [style=none] (11) at (1, 1.75) {$n$};
		\node [style=none] (13) at (0, 2.25) {};
		\node [style=X] (15) at (0, 1) {};
		\node [style=none] (16) at (0.5, 0.75) {};
	\end{pgfonlayer}
	\begin{pgfonlayer}{edgelayer}
		\draw (16.center) to (6.center);
		\draw (13.center) to (15);
		\draw (4.center) to (3.center);
		\draw (1.center) to (2.center);
	\end{pgfonlayer}
\end{tikzpicture}
$$
\end{definition}
\begin{lemma}
\label{lem:injand}
$\sfinj\f_2$ is a presentation for the prop $(\sfmono(\FinOrd_2),\times)$.
\end{lemma}
%Note, however, that the monoidal theory for $\sfmono(\f_2)$ is dependant on that of $\iso(\f_2)$ the identities of which are conjectured, even though we only need one extra identity to get injections.
The pushout of a diagram of sets and functions $2^n \xleftarrowtail{}\  2^k \ \xrightarrowtail{} 2^m$ is not always a power of 2.  Therefore, one should not expect to construct categories of partial isomorphisms via a distributive law  $\sfinj\f_2\otimes_{\Iso(\f_2)} \sfinj\f_2^\op$. Instead one must add all of the nontrivial subobjects to the constituent props forming the distributive law. As opposed to the affine case where one only needs to add the empty set, there are more than one such subobjects which arise in this way.
\begin{definition}
Consider the pro $\sub_{\F_2}$ whose maps $n\to n$ are generated by the elements of the ring of $n$-variable polynomial functions over $\F_2$:
$$\F_2[x_1,\ldots, x_n]/\langle x_1^2-x_1,\ldots, x_{n}^2-x_{n} \rangle$$
So that $\forall n,m \in \N$ and 
$$
p,r \in \F_2[x_1,\ldots, x_n]/\langle x_1^2-x_1,\ldots, x_{n}^2-x_{n} \rangle
$$
$$
q \in \F_2[x_{n+1},\ldots, x_{n+m}]/ \langle x_{n+1}^2-x_{n+1},\ldots, x_{n+m}^2-x_{n+m} \rangle
$$
we quotient by the following equations:
$$
\begin{tikzpicture}
	\begin{pgfonlayer}{nodelayer}
		\node [style=none] (0) at (3, 2.25) {};
		\node [style=none] (1) at (3, 3.25) {};
		\node [style=map] (2) at (3, 2.75) {$0$};
		\node [style=none] (6) at (3, 3.5) {$n$};
		\node [style=none] (8) at (3, 2) {$n$};
	\end{pgfonlayer}
	\begin{pgfonlayer}{edgelayer}
		\draw (0.center) to (1.center);
	\end{pgfonlayer}
\end{tikzpicture}
\eqzxa{sub.one}
\begin{tikzpicture}
	\begin{pgfonlayer}{nodelayer}
		\node [style=none] (0) at (3, 2.25) {};
		\node [style=none] (1) at (3, 3.25) {};
		\node [style=none] (6) at (3, 3.5) {$n$};
		\node [style=none] (8) at (3, 2) {$n$};
	\end{pgfonlayer}
	\begin{pgfonlayer}{edgelayer}
		\draw (0.center) to (1.center);
	\end{pgfonlayer}
\end{tikzpicture}
\hspace*{,5cm}
\begin{tikzpicture}
	\begin{pgfonlayer}{nodelayer}
		\node [style=none] (0) at (5, 2) {};
		\node [style=none] (1) at (5, 4) {};
		\node [style=map] (2) at (5, 2.5) {$r$};
		\node [style=map] (3) at (5, 3.5) {$p$};
		\node [style=none] (4) at (5, 4.25) {$n$};
		\node [style=none] (5) at (5, 1.75) {$n$};
	\end{pgfonlayer}
	\begin{pgfonlayer}{edgelayer}
		\draw (0.center) to (1.center);
	\end{pgfonlayer}
\end{tikzpicture}
\eqzxa{sub.two}
\begin{tikzpicture}
	\begin{pgfonlayer}{nodelayer}
		\node [style=none] (0) at (5, 2.25) {};
		\node [style=none] (1) at (5, 4.25) {};
		\node [style=map] (2) at (5, 3.25) {$p\cdot r$};
		\node [style=none] (4) at (5, 4.5) {$n$};
		\node [style=none] (5) at (5, 2) {$n$};
	\end{pgfonlayer}
	\begin{pgfonlayer}{edgelayer}
		\draw (0.center) to (1.center);
	\end{pgfonlayer}
\end{tikzpicture}
\hspace*{.5cm}
\begin{tikzpicture}
	\begin{pgfonlayer}{nodelayer}
		\node [style=none] (0) at (2.4, 2.25) {};
		\node [style=none] (1) at (2.4, 3.25) {};
		\node [style=map] (2) at (2.4, 2.75) {$p$};
		\node [style=none] (3) at (3, 3.25) {};
		\node [style=none] (4) at (3, 2.25) {};
		\node [style=map] (5) at (3, 2.75) {$q$};
		\node [style=none] (6) at (2.4, 3.5) {$n$};
		\node [style=none] (7) at (3, 3.5) {$m$};
		\node [style=none] (8) at (2.4, 2) {$n$};
		\node [style=none] (9) at (3, 2) {$m$};
	\end{pgfonlayer}
	\begin{pgfonlayer}{edgelayer}
		\draw (0.center) to (1.center);
		\draw (3.center) to (4.center);
	\end{pgfonlayer}
\end{tikzpicture}
\eqzxa{sub.three}
\begin{tikzpicture}
	\begin{pgfonlayer}{nodelayer}
		\node [style=none] (0) at (2.5, 2.25) {};
		\node [style=none] (1) at (2.5, 3.25) {};
		\node [style=map] (2) at (2.75, 2.75) {$p\cdot q$};
		\node [style=none] (3) at (3, 3.25) {};
		\node [style=none] (4) at (3, 2.25) {};
		\node [style=none] (6) at (2.5, 3.5) {$n$};
		\node [style=none] (7) at (3, 3.5) {$m$};
		\node [style=none] (8) at (2.5, 2) {$n$};
		\node [style=none] (9) at (3, 2) {$m$};
	\end{pgfonlayer}
	\begin{pgfonlayer}{edgelayer}
		\draw (0.center) to (1.center);
		\draw (3.center) to (4.center);
	\end{pgfonlayer}
\end{tikzpicture}
$$
\end{definition}
\begin{lemma}
\label{lem:sub}
$\sub_{\F_2}$ is a presentation for the pro of subobects in $\Span_2$; ie the  symmetric spans of monomorphisms $2^n \xleftarrowtail {e} \ k\  \xrightarrowtail{e}2^n$.
\end{lemma}
\begin{proof}
Since  $\F_2[x_1,\ldots, x_n]/\langle x_1^2-x_1,\ldots, x_n^2-x_n \rangle$ is the ring of polynomial functions in $n$-variables on $\F_2$ its elements are in bijection with functions $\ev_p:\Z_2^n \to \Z_2$ given by polynomial evaluation.  Let $k = |\ev_p^{-1}(1)|$, then chose a map $f_p:k \rightarrowtail 2^n$  in $f_2$ picking out all the solutions to the polynomial equation $p(x_1,\cdots, x_n)=0$. The functor from $\sub_{\F_2}$ to this subcategory of spans takes polynomials $p \mapsto (2^n \xleftarrowtail{f_p} \ k \ \xrightarrowtail {f_p} 2^n)$.  Any two spans induced by the same polynomial are isomorphic, so this is well defined.  It is clearly an isomorphism on objects, and it can easily be shown to be a monoidal functor.

The fullness is easy and the faithfulness comes from the fact that we can reduce every map to a polynomial and then reduce the polynomial to algebraic normal form.
\end{proof}
The axioms \ref{sub.two}, \ref{sub.three} are reflected in \ref{ZXA.14};  all of these axioms allow one to compose literals.  This allows subobjects to be reduced to algebraic normal form.


%
%The first axiom enforces that the trivial polynomial is the identity.
%The second axiom allows one to perform elementary row operations on polynomials.
%The third axiom allows one to multiply all polynomials.
%Because every map $f$ in $\sub_{\F_2}(n,n)$ has precisely one representative polynomial, and polynomials have a normal form via their reduction, and row reduction is confluent, completeness is immediate.



Now we compose these subobjects with the isomorphisms:
\begin{definition}
There is a distributive law of pros:
$$
\subiso\f_2:=\sub_{\F_2} ; \Iso(\f_2);
$$
So that $ \forall n,m,k \in \N $ and
$$
q \in \F_2[x_1,\ldots,x_{n+m+1+k}]/ \langle x_1^2-x_1,\ldots, x_{n+m+1+k}^2-x_{n+m+1+k}\rangle,
$$
$$
\begin{tikzpicture}
	\begin{pgfonlayer}{nodelayer}
		\node [style=map] (0) at (5, 3.5) {$q(x_1,\ldots, x_{n+m+1+k})$};
		\node [style=none] (1) at (3.75, 4) {};
		\node [style=none] (2) at (6.25, 4) {};
		\node [style=none] (3) at (4.25, 4) {};
		\node [style=none] (4) at (5.75, 4) {};
		\node [style=none] (5) at (3.75, 3) {};
		\node [style=none] (6) at (6.25, 3) {};
		\node [style=none] (7) at (4.25, 3) {};
		\node [style=none] (8) at (5.75, 3) {};
		\node [style=none] (9) at (3.75, 2.25) {$n$};
		\node [style=none] (10) at (4.25, 2.5) {};
		\node [style=none] (11) at (5.75, 2.5) {};
		\node [style=dot] (12) at (4.25, 2.75) {};
		\node [style=oplus] (13) at (5.75, 2.75) {};
		\node [style=none] (14) at (3.75, 2.5) {};
		\node [style=none] (15) at (6.25, 2.5) {};
		\node [style=none] (16) at (5.25, 4) {};
		\node [style=dot] (17) at (5.25, 2.75) {};
		\node [style=none] (18) at (5.25, 2.5) {};
		\node [style=none] (19) at (6.25, 2.25) {$k$};
		\node [style=none] (20) at (4.75, 2.5) {$m$};
		\node [style=none] (21) at (3.75, 5) {$n$};
		\node [style=none] (22) at (6.25, 5) {$k$};
		\node [style=none] (23) at (4.75, 4.5) {$m$};
		\node [style=none] (230) at (4.75, 4) {$\cdots$};
		\node [style=none] (24) at (4.25, 4.5) {};
		\node [style=none] (25) at (5.75, 4.5) {};
		\node [style=none] (26) at (3.75, 4.5) {};
		\node [style=none] (27) at (6.25, 4.5) {};
		\node [style=none] (28) at (5.25, 4.5) {};
	\end{pgfonlayer}
	\begin{pgfonlayer}{edgelayer}
		\draw [in=270, out=90] (10.center) to (7.center);
		\draw [in=270, out=90] (11.center) to (8.center);
		\draw (15.center) to (6.center);
		\draw (14.center) to (5.center);
		\draw (13) to (17);
		\draw [style=dotted] (17) to (12);
		\draw (18.center) to (17);
		\draw (1.center) to (26.center);
		\draw (3.center) to (24.center);
		\draw (16.center) to (28.center);
		\draw (4.center) to (25.center);
		\draw (2.center) to (27.center);
		\draw (1.center) to (5.center);
		\draw (12) to (3.center);
		\draw (16.center) to (17);
		\draw (13) to (4.center);
		\draw (2.center) to (6.center);
	\end{pgfonlayer}
\end{tikzpicture}
\eqzxa{subiso.two}
\begin{tikzpicture}
	\begin{pgfonlayer}{nodelayer}
		\node [style=map] (0) at (5, 3) {$q(x_1,\ldots, x_{n+m}, (x_{n+1}\ldots x_{n+m-1})+x_{n+m+1}, x_{n+m+2}, \ldots, x_{n+m+1+k})$};
		\node [style=none] (1) at (3.75, 2.5) {};
		\node [style=none] (2) at (6.25, 2.5) {};
		\node [style=none] (3) at (4.25, 2.5) {};
		\node [style=none] (4) at (5.75, 2.5) {};
		\node [style=none] (5) at (3.75, 3.5) {};
		\node [style=none] (6) at (6.25, 3.5) {};
		\node [style=none] (7) at (4.25, 3.5) {};
		\node [style=none] (8) at (5.75, 3.5) {};
		\node [style=none] (9) at (3.75, 4.25) {$n$};
		\node [style=none] (10) at (4.25, 4) {};
		\node [style=none] (11) at (5.75, 4) {};
		\node [style=dot] (12) at (4.25, 3.75) {};
		\node [style=oplus] (13) at (5.75, 3.75) {};
		\node [style=none] (14) at (3.75, 4) {};
		\node [style=none] (15) at (6.25, 4) {};
		\node [style=none] (16) at (5.25, 2.5) {};
		\node [style=dot] (17) at (5.25, 3.75) {};
		\node [style=none] (18) at (5.25, 4) {};
		\node [style=none] (19) at (6.25, 4.25) {$k$};
		\node [style=none] (20) at (4.75, 4.5) {$m$};
		\node [style=none] (21) at (3.75, 1.75) {$n$};
		\node [style=none] (22) at (6.25, 1.75) {$k$};
		\node [style=none] (23) at (4.75, 2.5) {$\cdots$};
		\node [style=none] (230) at (4.75, 2) {$m$};
		\node [style=none] (24) at (4.25, 2) {};
		\node [style=none] (25) at (5.75, 2) {};
		\node [style=none] (26) at (3.75, 2) {};
		\node [style=none] (27) at (6.25, 2) {};
		\node [style=none] (28) at (5.25, 2) {};
	\end{pgfonlayer}
	\begin{pgfonlayer}{edgelayer}
		\draw [in=-270, out=-90] (10.center) to (7.center);
		\draw [in=-270, out=-90] (11.center) to (8.center);
		\draw (15.center) to (6.center);
		\draw (14.center) to (5.center);
		\draw (13) to (17);
		\draw [style=dotted] (17) to (12);
		\draw (18.center) to (17);
		\draw (26.center) to (1.center);
		\draw (24.center) to (3.center);
		\draw (28.center) to (16.center);
		\draw (25.center) to (4.center);
		\draw (27.center) to (2.center);
		\draw (1.center) to (5.center);
		\draw (3.center) to (12);
		\draw (16.center) to (17);
		\draw (4.center) to (13);
		\draw (2.center) to (6.center);
	\end{pgfonlayer}
\end{tikzpicture}
$$
\end{definition}
\begin{lemma}
\label{lem:subiso}
$\subiso\f_2$ is a presentation for the subcategory of $(\Span^\sim(\FinOrd),\times)$ generated by spans of the form $2^n \xleftarrowtail{e} \ k \ \xrightarrowtail {e} 2^m \xrightarrow[\cong]{f} 2^m$, for all $n,m, k \in \N$ and all isomorphisms $f$ and monics $e$.
\end{lemma}
\begin{proof}
The obvious functor is clearly monoidal. Moreover, it is full by construction.
For the faithfulness, it suffices to observe that this is a strict factorization system.
\end{proof}
% $2^n \xleftarrow{e} k \xrightarrow{e} 2^n$, for all monics $e$ and isomorphisms of the form $2^n = 2^n \xrightarrow{\sim} 2^n$. 
\begin{definition}
There is a  distributive law of props:
$$
\subinj\f_2:= \subiso\f_2 \otimes_{\Iso(\f_2)} \sfinj\f_2;
 \forall n,m \in \N, p \in \F_2[x_1,\ldots, x_{n+1+m}]:
$$
$$
\begin{tikzpicture}
	\begin{pgfonlayer}{nodelayer}
		\node [style=none] (0) at (1.5, 3.5) {};
		\node [style=map] (1) at (2.5, 2.75) {$p(x_1,\ldots,x_{n+1+m})$};
		\node [style=none] (2) at (1.5, 3.75) {$n$};
		\node [style=none] (3) at (3.5, 3.5) {};
		\node [style=none] (4) at (3.5, 3.75) {$m$};
		\node [style=none] (5) at (1.5, 1.75) {};
		\node [style=none] (6) at (3.5, 1.75) {};
		\node [style=none] (7) at (1.5, 1.5) {$n$};
		\node [style=none] (8) at (3.5, 1.5) {$m$};
		\node [style=X] (9) at (2.5, 2) {};
		\node [style=none] (10) at (2.5, 3.5) {};
	\end{pgfonlayer}
	\begin{pgfonlayer}{edgelayer}
		\draw (0.center) to (5.center);
		\draw (3.center) to (6.center);
		\draw (10.center) to (9);
	\end{pgfonlayer}
\end{tikzpicture}
\eqzxa{subinj}
\begin{tikzpicture}
	\begin{pgfonlayer}{nodelayer}
		\node [style=none] (0) at (1.5, 3.75) {};
		\node [style=map] (1) at (2.5, 2.75) {$p(x_1,\ldots,x_n,0,x_{n+2},\ldots,x_{n+1+m})$};
		\node [style=none] (2) at (1.5, 4) {$n$};
		\node [style=none] (3) at (3.5, 3.75) {};
		\node [style=none] (4) at (3.5, 4) {$m$};
		\node [style=none] (5) at (1.5, 2) {};
		\node [style=none] (6) at (3.5, 2) {};
		\node [style=none] (7) at (1.5, 1.75) {$n$};
		\node [style=none] (8) at (3.5, 1.75) {$m$};
		\node [style=X] (9) at (2.5, 3.4) {};
		\node [style=none] (10) at (2.5, 3.75) {};
	\end{pgfonlayer}
	\begin{pgfonlayer}{edgelayer}
		\draw (0.center) to (5.center);
		\draw (3.center) to (6.center);
		\draw (10.center) to (9);
	\end{pgfonlayer}
\end{tikzpicture}
$$
\end{definition}
\begin{lemma}
\label{lem:subinj}
$\subinj\f_2$ is a presentation for the subcategory of $(\Span^\sim(\FinOrd),\times)$ generated by spans of the form $2^n \xleftarrowtail{e}  \ k \ \xrightarrowtail{e} 2^n \ \xrightarrowtail{e'} 2^{m}$ for all $n,m,k \in \N$ and all monics $e,e'$.
\end{lemma}
The proof is completely analogous to that of Lemma \ref{lem:subiso}.

Any $n$ variable polynomial $p$ can be interpreted as a span of monics via the oracle $\mathcal{O}_p$, where the the target wire is restricted to have the value $0$.  Each such polynomial corresponds to a subobject, which complicates the matter further than in the affine case.  Now we combine the subobjects with the isomorphisms:
\begin{definition}
There is a  distributive law of props:
$$
\sfpariso\f_2:=\subinj\f_2^\op \otimes_{\subiso\f_2} \subinj\f_2;
\begin{tikzpicture}
	\begin{pgfonlayer}{nodelayer}
		\node [style=map] (0) at (0, 0) {$\mathcal{O}_p$};
		\node [style=none] (1) at (-0.25, -0.75) {};
		\node [style=none] (2) at (-0.25, 0.75) {};
		\node [style=none] (3) at (-0.25, 1) {};
		\node [style=none] (4) at (-0.25, -1) {};
		\node [style=X] (5) at (0.25, 0.75) {};
		\node [style=X] (6) at (0.25, -0.75) {};
	\end{pgfonlayer}
	\begin{pgfonlayer}{edgelayer}
		\draw (3.center) to (2.center);
		\draw (4.center) to (1.center);
		\draw [in=-60, out=90] (6) to (0);
		\draw [in=-90, out=60] (0) to (5);
		\draw [in=120, out=-90] (2.center) to (0);
		\draw [in=90, out=-120, looseness=1.25] (0) to (1.center);
	\end{pgfonlayer}
\end{tikzpicture}
\eqzxa{oracle}
\begin{tikzpicture}
	\begin{pgfonlayer}{nodelayer}
		\node [style=map] (0) at (-0.25, 0) {$p$};
		\node [style=none] (1) at (-0.25, -0.75) {};
		\node [style=none] (2) at (-0.25, 0.75) {};
		\node [style=none] (3) at (-0.25, 1) {};
		\node [style=none] (4) at (-0.25, -1) {};
	\end{pgfonlayer}
	\begin{pgfonlayer}{edgelayer}
		\draw (3.center) to (2.center);
		\draw (4.center) to (1.center);
		\draw (2.center) to (0);
		\draw (0) to (1.center);
	\end{pgfonlayer}
\end{tikzpicture}
$$
\end{definition}
%This is actually a distributive law, because it need only be witnessed by pushing $\eta_X$ past $\eta_X^\op$.  The only time this can't be done is when there is the target  of a generalized controlled-not not gate--or those of several generalized controlled-not gates is in between both of these generators.  In which case, the obstructing generalized controlled-not gates form an oracle for which we can apply this equation.  This is computing the apex of the span when performing a pullback.
%Note that $\pr\iso\f_2$ is not actually partial isomorphisms over $\f_2$, per se, but rather a full subcategory of partial isomorphisms of sets and functions, which is not iself a prop with respect to the Cartesian product.  It is because of such a complication that the aforementioned distributive law isn't quite a pullback.
\begin{lemma}
\label{lem:parisof}
$\sfpar\iso \f_2$ is a presentation for the full subcategory $(\FPinj_2,\times)$\\ of $(\ParIso(\FinOrd),\times)$ with objects powers of two.
\end{lemma}
\begin{proof}
We have shown how to push all of the generators of $ \subinj\f_2$  past those of $\subinj\f_2^\op$ up to ${\subiso\f_2}$.

The uniqueness up to zig-zags becomes  trivial in this case. 
The invertible maps in $\subinj\f_2^\op$ act the same both on the left and on the right in analogy to the orthogonal factorization systems. Similarly, the non-invertible subobjects  corresponding to spans $2^n \xleftarrowtail{e} k\ \xrightarrowtail{e} 2^n $ act the same both on the left and the right.
\end{proof}
This is equivalent to the prop $\TOF$ of \cite{tof} whose identities we have included in Figure \ref{fig:TOF}.
By adding the effect $\sqrt{2}\langle +|$ we get partial functions:
\begin{definition}
Consider the prop $\pr\f_2$ given by the  pushout of the following diagram of props, given by adding a counit to the diagonal map:
$$\sfpariso\f_2 \leftarrow \surj^\op \rightarrow \cm^\op$$
\end{definition}
\begin{lemma}
\label{lem:parand}
$\sfpar\f_2$ is a presentation for $(\FPar_2,\times)$.
\end{lemma}
This is immediate because pushout is precisely the Cartesian completion.


By adding the state $\sqrt{2} |+\rangle$ we get spans:
\begin{definition}
Let $\sfspan\f_2$ denote the pushout of the diagram of props:
$$
\pr\f_2^\op\leftarrow \pr\iso\f_2 \rightarrow \pr\f_2
$$
\end{definition}
\begin{lemma}
\label{lem:spanand}
$\sfspan\f_2$ is a presentation for  $(\FSpan_2,\times)$.
\end{lemma}
Since we know that $\sfpariso_2\cong \TOF$, adding a unit and counit to $\TOF$ yields $\ZXA$, which we know si complete for $(\FSpan_2,\times)$.
%
%
%That is to say $\Span(\f_2)$ is a presentation for the prop of ``qubit matrices'' over $\N$.  There is an alternative presentation of this category due to \cite{zxa}:
%\begin{corollary}
%$\Span(\f_2)$ can equivalently can be presented in terms of the coproduct of the monoids $X,Z,\&$ and comonoids $Z^\dag,X^\dag$
%\begin{itemize}
%\item An extra-Frobenius algebra between $X^\op$ and $Y$.
%
%\item A special-Frobenius algebra between $Z^\op$ and $Z$.
%
%\item The Lawvere theory for $\F_2^+$ represented by $X$ and $Z^\op$.
%
%\item The Lawvere theory for $\F_2^\times$  represented by  $\&$ and $Z^\op$.
%
%\item A distributive law $L_{X} \otimes_L L_{\&}  \Rightarrow  L_{\&} \otimes_L L_{X}$  in $\Kl(\T_{\Mon-\Prof}^\times)$, where $Z^\op$ is identified with prop for the diagonal monoid.
%
%\item The naturality of $\eta_{\&}$ with respect to $\mu_{\&}^\op$.
%
%\end{itemize}
%
%
%\end{corollary}


%
%The proof follows from realizing that this equation makes the triangle gate idempotent, which allows one to reduce the value of non-scalar positive natural number H-boxes to $1$, alike to the quotient described in \cite{niel} (H-boxes are first described in the paper \cite{zh}).    The other law forces all nonzero scalar H-boxes, and thus all nonzero scalars to be 0. So this is complete for qubit boolean matrices, and thus, qubit relations.
%


%\nocite{ih}
%\nocite{coecke2008interacting}
%\nocite{zh}
%\nocite{tof}

%\appendix 



%
%\begin{lemma}
%
%The phase-free fragment of the ZH calculus is presented by the pushout of the following diagram of props:
%
%
%TODO
%
%modulo the quotient:
%
%\end{lemma}



%\begin{comment}
%






%
%\subsection{Alternative presentations}
%\label{sec:presentations}
%
%In this section, we give alternative presentations of the props presented in the main body of this paper.  With the exception of the alternative presentation of $(\FPinj_2,\times)$, these are presented in terms of a bunch of (co)monoids modulo equations.  This is more in the aesthetic tradition of the ZX-calculus, for example.  
%
%\subsection{Section \ref{sec:one}}
%\label{subsec:presentations:one}
%
%\subsubsection{$(\Par(\Mat_{\F_2}),+)$}
%\label{subsubsec:presentations:one:par}
%
%
%$(\Par(\Mat_{\F_2}),+)$ is presented by the symmetric monoidal theory with the following generators:
%$$
%\begin{tikzpicture}
%	\begin{pgfonlayer}{nodelayer}
%		\node [style=Z] (0) at (0.75, 5) {};
%		\node [style=none] (1) at (0.5, 4.5) {};
%		\node [style=none] (2) at (1, 4.5) {};
%		\node [style=none] (3) at (0.75, 5.5) {};
%	\end{pgfonlayer}
%	\begin{pgfonlayer}{edgelayer}
%		\draw [in=-135, out=90] (1.center) to (0);
%		\draw [in=90, out=-45] (0) to (2.center);
%		\draw (0) to (3.center);
%	\end{pgfonlayer}
%\end{tikzpicture}
%\hspace*{.5cm}
%\begin{tikzpicture}
%	\begin{pgfonlayer}{nodelayer}
%		\node [style=Z] (0) at (0, 5) {};
%		\node [style=none] (1) at (-0.25, 5.5) {};
%		\node [style=none] (2) at (0.25, 5.5) {};
%		\node [style=none] (3) at (0, 4.5) {};
%	\end{pgfonlayer}
%	\begin{pgfonlayer}{edgelayer}
%		\draw [in=135, out=-90] (1.center) to (0);
%		\draw [in=-90, out=45] (0) to (2.center);
%		\draw (0) to (3.center);
%	\end{pgfonlayer}
%\end{tikzpicture}
%\hspace*{.5cm}
%\begin{tikzpicture}
%	\begin{pgfonlayer}{nodelayer}
%		\node [style=Z] (0) at (0, 5) {};
%		\node [style=none] (3) at (0, 4.5) {};
%	\end{pgfonlayer}
%	\begin{pgfonlayer}{edgelayer}
%		\draw (0) to (3.center);
%	\end{pgfonlayer}
%\end{tikzpicture}
%\hspace*{.5cm}
%\begin{tikzpicture}
%	\begin{pgfonlayer}{nodelayer}
%		\node [style=X] (0) at (0.75, 5) {};
%		\node [style=none] (1) at (0.5, 4.5) {};
%		\node [style=none] (2) at (1, 4.5) {};
%		\node [style=none] (3) at (0.75, 5.5) {};
%	\end{pgfonlayer}
%	\begin{pgfonlayer}{edgelayer}
%		\draw [in=-135, out=90] (1.center) to (0);
%		\draw [in=90, out=-45] (0) to (2.center);
%		\draw (0) to (3.center);
%	\end{pgfonlayer}
%\end{tikzpicture}
%\hspace*{.5cm}
%\begin{tikzpicture}
%	\begin{pgfonlayer}{nodelayer}
%		\node [style=X] (0) at (0, 4.5) {};
%		\node [style=none] (3) at (0, 5) {};
%	\end{pgfonlayer}
%	\begin{pgfonlayer}{edgelayer}
%		\draw (0) to (3.center);
%	\end{pgfonlayer}
%\end{tikzpicture}
%\hspace*{,5cm}
%\begin{tikzpicture}
%	\begin{pgfonlayer}{nodelayer}
%		\node [style=X] (0) at (0.5, 5) {};
%		\node [style=none] (1) at (0.5, 4.5) {};
%	\end{pgfonlayer}
%	\begin{pgfonlayer}{edgelayer}
%		\draw (0) to (1.center);
%	\end{pgfonlayer}
%\end{tikzpicture}
%$$
%Modulo the all the equations of $\cb_{\F_2}$ in addition to the identity and its transpose:
%$$
%\begin{tikzpicture}
%	\begin{pgfonlayer}{nodelayer}
%		\node [style=X] (0) at (0.5, 5) {};
%		\node [style=none] (1) at (0.5, 4.5) {};
%		\node [style=Z] (2) at (0.5, 4.5) {};
%		\node [style=none] (3) at (0.25, 4) {};
%		\node [style=none] (4) at (0.75, 4) {};
%	\end{pgfonlayer}
%	\begin{pgfonlayer}{edgelayer}
%		\draw (0) to (1.center);
%		\draw [in=-135, out=90] (3.center) to (2);
%		\draw [in=-45, out=90] (4.center) to (2);
%	\end{pgfonlayer}
%\end{tikzpicture}
%  \eref{bi.two}
%\begin{tikzpicture}
%	\begin{pgfonlayer}{nodelayer}
%		\node [style=X] (0) at (0.25, 5) {};
%		\node [style=none] (3) at (0.25, 4) {};
%		\node [style=none] (4) at (0.75, 4) {};
%		\node [style=X] (5) at (0.75, 5) {};
%	\end{pgfonlayer}
%	\begin{pgfonlayer}{edgelayer}
%		\draw (4.center) to (5);
%		\draw (0) to (3.center);
%	\end{pgfonlayer}
%\end{tikzpicture}
%$$
%
%
%\subsubsection{$(\Span(\Mat_{\F_2}),+)$}
%\label{subsubsec:presentations:one:span}
%
%$(\Span(\Mat_{\F_2}),+)$ is presented by the symmetric monoidal theory with the same generators and equations as \S \ref{subsubsec:presentations:one:par} and their transposes as well as the following equations making the white monoid/comnoid pair into a special commutative Frobenius algebra and the black monoid/comonoid pair into a commutative Frobenius algebra.
%
%
%
%\subsection{Section \ref{sec:two}}
%\label{subsec:presentations:two}
%
%
%\subsubsection{$(\Par(\Aff\Mat_{\F_2}+1)^*,+)$}
%\label{subsubsec:presentations:two:par}
%
%$(\Par(\Aff\Mat_{\F_2}+1)^*,+)$ is presented by the symmetric monoidal theory with the same generators and equations as in \S \ref{subsubsec:presentations:one:par} in addition to the following generator:
%$$
%\begin{tikzpicture}
%	\begin{pgfonlayer}{nodelayer}
%		\node [style=X] (0) at (0, 4) {$1$};
%		\node [style=none] (1) at (0, 4.5) {};
%	\end{pgfonlayer}
%	\begin{pgfonlayer}{edgelayer}
%		\draw (0) to (1.center);
%	\end{pgfonlayer}
%\end{tikzpicture}
%$$
%and the following equations:
%$$
%\begin{tikzpicture}
%	\begin{pgfonlayer}{nodelayer}
%		\node [style=X] (0) at (-4, 0.75) {$1$};
%		\node [style=none] (1) at (-3.5, 0) {};
%		\node [style=none] (2) at (-3.5, 1.5) {};
%	\end{pgfonlayer}
%	\begin{pgfonlayer}{edgelayer}
%		\draw (1.center) to (2.center);
%	\end{pgfonlayer}
%\end{tikzpicture}
%\eqzxa{zero.new}
%\begin{tikzpicture}
%	\begin{pgfonlayer}{nodelayer}
%		\node [style=X] (0) at (-4, 0.75) {$1$};
%		\node [style=none] (1) at (-3.5, 0) {};
%		\node [style=none] (2) at (-3.5, 1.5) {};
%		\node [style=Z] (3) at (-3.5, 0.5) {};
%		\node [style=X] (4) at (-3.5, 1) {};
%	\end{pgfonlayer}
%	\begin{pgfonlayer}{edgelayer}
%		\draw (2.center) to (4);
%		\draw (3) to (1.center);
%	\end{pgfonlayer}
%\end{tikzpicture},
%\hspace*{,5cm}
%\begin{tikzpicture}
%	\begin{pgfonlayer}{nodelayer}
%		\node [style=X] (0) at (0.75, 4) {$1$};
%		\node [style=none] (1) at (0.75, 4.5) {};
%		\node [style=Z] (2) at (0.75, 4.5) {};
%		\node [style=none] (3) at (0.5, 5) {};
%		\node [style=none] (4) at (1, 5) {};
%	\end{pgfonlayer}
%	\begin{pgfonlayer}{edgelayer}
%		\draw (0) to (1.center);
%		\draw [in=135, out=-90] (3.center) to (2);
%		\draw [in=45, out=-90] (4.center) to (2);
%	\end{pgfonlayer}
%\end{tikzpicture}
%  \erefop{bi.two}
%\begin{tikzpicture}
%	\begin{pgfonlayer}{nodelayer}
%		\node [style=X] (0) at (0.5, 4) {$1$};
%		\node [style=none] (1) at (0.5, 5) {};
%		\node [style=none] (2) at (1, 5) {};
%		\node [style=X] (3) at (1, 4) {$1$};
%	\end{pgfonlayer}
%	\begin{pgfonlayer}{edgelayer}
%		\draw (2.center) to (3);
%		\draw (0) to (1.center);
%	\end{pgfonlayer}
%\end{tikzpicture}
%\hspace*{,5cm}
%\begin{tikzpicture}
%	\begin{pgfonlayer}{nodelayer}
%		\node [style=X] (0) at (0.75, 4) {$1$};
%		\node [style=none] (1) at (0.75, 4.5) {};
%		\node [style=Z] (2) at (0.75, 4.5) {};
%	\end{pgfonlayer}
%	\begin{pgfonlayer}{edgelayer}
%		\draw (0) to (1.center);
%	\end{pgfonlayer}
%\end{tikzpicture}
%  \eref{extra}
%$$
%
%
%\subsubsection{$(\Span(\Aff\Mat_{\F_2}+1)^*,+)$}
%\label{subsubsec:presentations:two:span}
%
%$(\Span(\Aff\Mat_{\F_2}+1)^*,+)$ is presented by the generators and identities of  \ref{subsubsec:presentations:two:par} as well as as well as the generator 
%$\begin{tikzpicture}
%	\begin{pgfonlayer}{nodelayer}
%		\node [style=none] (0) at (0.75, 0.5) {};
%		\node [style=none] (1) at (0.75, -0.25) {};
%		\node [style=Z] (2) at (0.75, -0.25) {};
%	\end{pgfonlayer}
%	\begin{pgfonlayer}{edgelayer}
%		\draw (0.center) to (1.center);
%	\end{pgfonlayer}
%\end{tikzpicture}$ 
%and the equation making the codiagonal map counital:
%$$
%  \begin{tikzpicture}[rotate=90,yscale=-1]
%	\begin{pgfonlayer}{nodelayer}
%		\node [style=Z] (0) at (-9, -0) {};
%		\node [style=none] (1) at (-8.25, -0) {};
%		\node [style=Z] (2) at (-9.75, 0.25) {};
%		\node [style=none] (3) at (-10, -0.25) {};
%	\end{pgfonlayer}
%	\begin{pgfonlayer}{edgelayer}
%		\draw [in=-150, out=0, looseness=1.00] (3.center) to (0);
%		\draw [in=150, out=0, looseness=1.00] (2.center) to (0);
%		\draw (0) to (1.center);
%	\end{pgfonlayer}
%  \end{tikzpicture}
%  \eref{unit}
%  \begin{tikzpicture}[rotate=90]
%	\begin{pgfonlayer}{nodelayer}
%		\node [style=none] (0) at (-9, 0.25) {};
%		\node [style=none] (1) at (-9.75, 0.25) {};
%	\end{pgfonlayer}
%	\begin{pgfonlayer}{edgelayer}
%		\draw (1) to (0.center);
%	\end{pgfonlayer}
%  \end{tikzpicture}
%$$
%
%\subsection{Section \ref{sec:three}}
%\label{subsec:presentations:three}
%
%
%
%%\subsubsection{$(\FPinj_2,\times)$}
%%\label{subsubsec:presentations:three:pinj}
%
%%By \cite[\S 7]{cole} $(\FPinj_2,\times)$ is presented by the prop generated by the Toffoli gate (the triple-controlled-not gate) as well as unit for the and gate and its transpose modulo the following equations:
%%
%%\begin{multicols}{2}
%%\begin{enumerate}[label={\bf [TOF.\arabic*]}, ref={\bf [TOF.\arabic*]}, wide = 0pt, leftmargin = 2em]
%%\item
%%\label{TOF.1}
%%{\hfil
%%$
%%\begin{tabular}{cc}
%%\begin{tikzpicture}
%%	\begin{pgfonlayer}{nodelayer}
%%		\node [style=nothing] (0) at (1.5, 0) {};
%%		\node [style=nothing] (1) at (1, 0) {};
%%		\node [style=oplus] (2) at (1.5, 1) {};
%%		\node [style=dot] (3) at (1, 1) {};
%%		\node [style=dot] (4) at (0.5, 1) {};
%%		\node [style=X] (5) at (0.5, 0.5) {$1$};
%%		\node [style=nothing] (6) at (0.5, 1.5) {};
%%		\node [style=nothing] (7) at (1, 1.5) {};
%%		\node [style=nothing] (8) at (1.5, 1.5) {};
%%	\end{pgfonlayer}
%%	\begin{pgfonlayer}{edgelayer}
%%		\draw (5) to (4);
%%		\draw (4) to (6);
%%		\draw (7) to (3);
%%		\draw (1) to (3);
%%		\draw (0) to (2);
%%		\draw (2) to (8);
%%		\draw (2) to (3);
%%		\draw (3) to (4);
%%	\end{pgfonlayer}
%%\end{tikzpicture}
%%=
%%\begin{tikzpicture}
%%	\begin{pgfonlayer}{nodelayer}
%%		\node [style=nothing] (0) at (1.5, 0) {};
%%		\node [style=nothing] (1) at (1, 0) {};
%%		\node [style=oplus] (2) at (1.5, 1) {};
%%		\node [style=dot] (3) at (1, 1) {};
%%		\node [style=X] (4) at (0.5, 1) {$1$};
%%		\node [style=nothing] (5) at (0.5, 1.5) {};
%%		\node [style=nothing] (6) at (1, 1.5) {};
%%		\node [style=nothing] (7) at (1.5, 1.5) {};
%%	\end{pgfonlayer}
%%	\begin{pgfonlayer}{edgelayer}
%%		\draw (1) to (3);
%%		\draw (0) to (2);
%%		\draw (2) to (3);
%%		\draw (6) to (3);
%%		\draw (4) to (5);
%%		\draw (2) to (7);
%%	\end{pgfonlayer}
%%\end{tikzpicture} &
%%\begin{tikzpicture}
%%	\begin{pgfonlayer}{nodelayer}
%%		\node [style=nothing] (0) at (1.5, 2) {};
%%		\node [style=nothing] (1) at (1, 2) {};
%%		\node [style=oplus] (2) at (1.5, 1) {};
%%		\node [style=dot] (3) at (1, 1) {};
%%		\node [style=dot] (4) at (0.5, 1) {};
%%		\node [style=X] (5) at (0.5, 1.5) {$1$};
%%		\node [style=nothing] (6) at (0.5, 0.5) {};
%%		\node [style=nothing] (7) at (1, 0.5) {};
%%		\node [style=nothing] (8) at (1.5, 0.5) {};
%%	\end{pgfonlayer}
%%	\begin{pgfonlayer}{edgelayer}
%%		\draw (5) to (4);
%%		\draw (4) to (6);
%%		\draw (7) to (3);
%%		\draw (1) to (3);
%%		\draw (0) to (2);
%%		\draw (2) to (8);
%%		\draw (2) to (3);
%%		\draw (3) to (4);
%%	\end{pgfonlayer}
%%\end{tikzpicture}
%%=
%%\begin{tikzpicture}
%%	\begin{pgfonlayer}{nodelayer}
%%		\node [style=nothing] (0) at (1.5, 2) {};
%%		\node [style=nothing] (1) at (1, 2) {};
%%		\node [style=oplus] (2) at (1.5, 1) {};
%%		\node [style=dot] (3) at (1, 1) {};
%%		\node [style=X] (4) at (0.5, 1) {$1$};
%%		\node [style=nothing] (5) at (0.5, 0.5) {};
%%		\node [style=nothing] (6) at (1, 0.5) {};
%%		\node [style=nothing] (7) at (1.5, 0.5) {};
%%	\end{pgfonlayer}
%%	\begin{pgfonlayer}{edgelayer}
%%		\draw (1) to (3);
%%		\draw (0) to (2);
%%		\draw (2) to (3);
%%		\draw (6) to (3);
%%		\draw (4) to (5);
%%		\draw (2) to (7);
%%	\end{pgfonlayer}
%%\end{tikzpicture}
%%\end{tabular}
%%$}
%%
%%
%%\item
%%\label{TOF.2}
%%{\hfil
%%$
%%\begin{tabular}{cc}
%%\begin{tikzpicture}
%%	\begin{pgfonlayer}{nodelayer}
%%		\node [style=nothing] (0) at (1, 0.5) {};
%%		\node [style=nothing] (1) at (1.5, 0.5) {};
%%		\node [style=nothing] (2) at (0.5, 2) {};
%%		\node [style=nothing] (3) at (1, 2) {};
%%		\node [style=nothing] (4) at (1.5, 2) {};
%%		\node [style=dot] (5) at (0.5, 1.5) {};
%%		\node [style=dot] (6) at (1, 1.5) {};
%%		\node [style=oplus] (7) at (1.5, 1.5) {};
%%		\node [style=X] (8) at (0.5, 1) {};
%%	\end{pgfonlayer}
%%	\begin{pgfonlayer}{edgelayer}
%%		\draw (5) to (2);
%%		\draw (3) to (6);
%%		\draw (6) to (0);
%%		\draw (1) to (7);
%%		\draw (7) to (4);
%%		\draw (7) to (6);
%%		\draw (6) to (5);
%%		\draw (8) to (5);
%%	\end{pgfonlayer}
%%\end{tikzpicture}
%%=
%%\begin{tikzpicture}
%%	\begin{pgfonlayer}{nodelayer}
%%		\node [style=nothing] (0) at (1, 0.75) {};
%%		\node [style=nothing] (1) at (1.5, 0.75) {};
%%		\node [style=nothing] (2) at (0.5, 2) {};
%%		\node [style=nothing] (3) at (1, 2) {};
%%		\node [style=nothing] (4) at (1.5, 2) {};
%%		\node [style=X] (5) at (0.5, 1.25) {};
%%	\end{pgfonlayer}
%%	\begin{pgfonlayer}{edgelayer}
%%		\draw (5) to (2);
%%		\draw (0) to (3);
%%		\draw (1) to (4);
%%	\end{pgfonlayer}
%%\end{tikzpicture} & 
%%\begin{tikzpicture}
%%	\begin{pgfonlayer}{nodelayer}
%%		\node [style=nothing] (0) at (1, 2.5) {};
%%		\node [style=nothing] (1) at (1.5, 2.5) {};
%%		\node [style=nothing] (2) at (0.5, 1) {};
%%		\node [style=nothing] (3) at (1, 1) {};
%%		\node [style=nothing] (4) at (1.5, 1) {};
%%		\node [style=dot] (5) at (0.5, 1.5) {};
%%		\node [style=dot] (6) at (1, 1.5) {};
%%		\node [style=oplus] (7) at (1.5, 1.5) {};
%%		\node [style=X] (8) at (0.5, 2) {};
%%	\end{pgfonlayer}
%%	\begin{pgfonlayer}{edgelayer}
%%		\draw (5) to (2);
%%		\draw (3) to (6);
%%		\draw (6) to (0);
%%		\draw (1) to (7);
%%		\draw (7) to (4);
%%		\draw (7) to (6);
%%		\draw (6) to (5);
%%		\draw (8) to (5);
%%	\end{pgfonlayer}
%%\end{tikzpicture}
%%=
%%\begin{tikzpicture}
%%	\begin{pgfonlayer}{nodelayer}
%%		\node [style=nothing] (0) at (1, 2.5) {};
%%		\node [style=nothing] (1) at (1.5, 2.5) {};
%%		\node [style=nothing] (2) at (0.5, 1.25) {};
%%		\node [style=nothing] (3) at (1, 1.25) {};
%%		\node [style=nothing] (4) at (1.5, 1.25) {};
%%		\node [style=X] (5) at (0.5, 2) {};
%%	\end{pgfonlayer}
%%	\begin{pgfonlayer}{edgelayer}
%%		\draw (5) to (2);
%%		\draw (0) to (3);
%%		\draw (1) to (4);
%%	\end{pgfonlayer}
%%\end{tikzpicture}
%%\end{tabular}
%%$}
%%
%%\item
%%\label{TOF.3}
%%{\hfil
%%$
%%\begin{tikzpicture}
%%	\begin{pgfonlayer}{nodelayer}
%%		\node [style=nothing] (0) at (-0.5, 0.5) {};
%%		\node [style=nothing] (1) at (0, 0.5) {};
%%		\node [style=nothing] (2) at (-1, 0.5) {};
%%		\node [style=nothing] (3) at (-1.5, 0.5) {};
%%		\node [style=nothing] (4) at (-2, 0.5) {};
%%		\node [style=dot] (5) at (-1.5, 1) {};
%%		\node [style=oplus] (6) at (-1, 1) {};
%%		\node [style=oplus] (7) at (-1, 1.5) {};
%%		\node [style=dot] (8) at (-0.5, 1.5) {};
%%		\node [style=dot] (9) at (-2, 1) {};
%%		\node [style=dot] (10) at (0, 1.5) {};
%%		\node [style=nothing] (11) at (-0.5, 2) {};
%%		\node [style=nothing] (12) at (-1.5, 2) {};
%%		\node [style=nothing] (13) at (-2, 2) {};
%%		\node [style=nothing] (14) at (0, 2) {};
%%		\node [style=nothing] (15) at (-1, 2) {};
%%	\end{pgfonlayer}
%%	\begin{pgfonlayer}{edgelayer}
%%		\draw (4) to (9);
%%		\draw (9) to (13);
%%		\draw (3) to (5);
%%		\draw (5) to (12);
%%		\draw (2) to (6);
%%		\draw (6) to (7);
%%		\draw (7) to (15);
%%		\draw (0) to (8);
%%		\draw (8) to (11);
%%		\draw (1) to (10);
%%		\draw (10) to (14);
%%		\draw (10) to (8);
%%		\draw (8) to (7);
%%		\draw (6) to (5);
%%		\draw (5) to (9);
%%	\end{pgfonlayer}
%%\end{tikzpicture}
%%=
%%\begin{tikzpicture}
%%	\begin{pgfonlayer}{nodelayer}
%%		\node [style=nothing] (0) at (-0.5, 0.5) {};
%%		\node [style=nothing] (1) at (0, 0.5) {};
%%		\node [style=nothing] (2) at (-1, 0.5) {};
%%		\node [style=nothing] (3) at (-1.5, 0.5) {};
%%		\node [style=nothing] (4) at (-2, 0.5) {};
%%		\node [style=dot] (5) at (-1.5, 1.5) {};
%%		\node [style=dot] (6) at (-0.5, 1) {};
%%		\node [style=dot] (7) at (-2, 1.5) {};
%%		\node [style=dot] (8) at (0, 1) {};
%%		\node [style=nothing] (9) at (-0.5, 2) {};
%%		\node [style=nothing] (10) at (-1.5, 2) {};
%%		\node [style=nothing] (11) at (-2, 2) {};
%%		\node [style=nothing] (12) at (0, 2) {};
%%		\node [style=nothing] (13) at (-1, 2) {};
%%		\node [style=oplus] (14) at (-1, 1.5) {};
%%		\node [style=oplus] (15) at (-1, 1) {};
%%	\end{pgfonlayer}
%%	\begin{pgfonlayer}{edgelayer}
%%		\draw (4) to (7);
%%		\draw (7) to (11);
%%		\draw (3) to (5);
%%		\draw (5) to (10);
%%		\draw (0) to (6);
%%		\draw (6) to (9);
%%		\draw (1) to (8);
%%		\draw (8) to (12);
%%		\draw (8) to (6);
%%		\draw (5) to (7);
%%		\draw (2) to (15);
%%		\draw (15) to (14);
%%		\draw (14) to (13);
%%		\draw (14) to (5);
%%		\draw (6) to (15);
%%	\end{pgfonlayer}
%%\end{tikzpicture}
%%$}
%%
%%
%%\item
%%\label{TOF.4}
%%{\hfil
%%$
%%\begin{tikzpicture}
%%	\begin{pgfonlayer}{nodelayer}
%%		\node [style=nothing] (0) at (-0.5, 0.5) {};
%%		\node [style=nothing] (1) at (0, 0.5) {};
%%		\node [style=nothing] (2) at (-1, 0.5) {};
%%		\node [style=nothing] (3) at (-1.5, 0.5) {};
%%		\node [style=nothing] (4) at (-2, 0.5) {};
%%		\node [style=dot] (5) at (-1.5, 1) {};
%%		\node [style=dot] (6) at (-1, 1) {};
%%		\node [style=dot] (7) at (-1, 1.5) {};
%%		\node [style=dot] (8) at (-0.5, 1.5) {};
%%		\node [style=oplus] (9) at (-2, 1) {};
%%		\node [style=oplus] (10) at (0, 1.5) {};
%%		\node [style=nothing] (11) at (-0.5, 2) {};
%%		\node [style=nothing] (12) at (-1.5, 2) {};
%%		\node [style=nothing] (13) at (-2, 2) {};
%%		\node [style=nothing] (14) at (0, 2) {};
%%		\node [style=nothing] (15) at (-1, 2) {};
%%	\end{pgfonlayer}
%%	\begin{pgfonlayer}{edgelayer}
%%		\draw (4) to (9);
%%		\draw (9) to (13);
%%		\draw (3) to (5);
%%		\draw (5) to (12);
%%		\draw (2) to (6);
%%		\draw (6) to (7);
%%		\draw (7) to (15);
%%		\draw (0) to (8);
%%		\draw (8) to (11);
%%		\draw (1) to (10);
%%		\draw (10) to (14);
%%		\draw (10) to (8);
%%		\draw (8) to (7);
%%		\draw (6) to (5);
%%		\draw (5) to (9);
%%	\end{pgfonlayer}
%%\end{tikzpicture}
%%=
%%\begin{tikzpicture}
%%	\begin{pgfonlayer}{nodelayer}
%%		\node [style=nothing] (0) at (-0.5, 0.5) {};
%%		\node [style=nothing] (1) at (0, 0.5) {};
%%		\node [style=nothing] (2) at (-1, 0.5) {};
%%		\node [style=nothing] (3) at (-1.5, 0.5) {};
%%		\node [style=nothing] (4) at (-2, 0.5) {};
%%		\node [style=dot] (5) at (-1.5, 1.5) {};
%%		\node [style=dot] (6) at (-0.5, 1) {};
%%		\node [style=oplus] (7) at (-2, 1.5) {};
%%		\node [style=oplus] (8) at (0, 1) {};
%%		\node [style=nothing] (9) at (-0.5, 2) {};
%%		\node [style=nothing] (10) at (-1.5, 2) {};
%%		\node [style=nothing] (11) at (-2, 2) {};
%%		\node [style=nothing] (12) at (0, 2) {};
%%		\node [style=nothing] (13) at (-1, 2) {};
%%		\node [style=dot] (14) at (-1, 1.5) {};
%%		\node [style=dot] (15) at (-1, 1) {};
%%	\end{pgfonlayer}
%%	\begin{pgfonlayer}{edgelayer}
%%		\draw (4) to (7);
%%		\draw (7) to (11);
%%		\draw (3) to (5);
%%		\draw (5) to (10);
%%		\draw (0) to (6);
%%		\draw (6) to (9);
%%		\draw (1) to (8);
%%		\draw (8) to (12);
%%		\draw (8) to (6);
%%		\draw (5) to (7);
%%		\draw (2) to (15);
%%		\draw (15) to (14);
%%		\draw (14) to (13);
%%		\draw (14) to (5);
%%		\draw (6) to (15);
%%	\end{pgfonlayer}
%%\end{tikzpicture}
%%$}
%%
%%\item
%%\label{TOF.5}
%%{\hfil
%%$
%%\begin{tikzpicture}
%%	\begin{pgfonlayer}{nodelayer}
%%		\node [style=nothing] (0) at (-1, 0.5) {};
%%		\node [style=nothing] (1) at (-0.5, 0.5) {};
%%		\node [style=nothing] (2) at (-1.5, 0.5) {};
%%		\node [style=nothing] (3) at (-2, 0.5) {};
%%		\node [style=nothing] (4) at (-1, 2) {};
%%		\node [style=nothing] (5) at (-1.5, 2) {};
%%		\node [style=nothing] (6) at (-2, 2) {};
%%		\node [style=nothing] (7) at (-0.5, 2) {};
%%		\node [style=oplus] (8) at (-2, 1) {};
%%		\node [style=oplus] (9) at (-0.5, 1.5) {};
%%		\node [style=dot] (10) at (-1.5, 1) {};
%%		\node [style=dot] (11) at (-1, 1) {};
%%		\node [style=dot] (12) at (-1.5, 1.5) {};
%%		\node [style=dot] (13) at (-1, 1.5) {};
%%	\end{pgfonlayer}
%%	\begin{pgfonlayer}{edgelayer}
%%		\draw (3) to (8);
%%		\draw (8) to (6);
%%		\draw (5) to (12);
%%		\draw (12) to (10);
%%		\draw (10) to (2);
%%		\draw (0) to (11);
%%		\draw (11) to (13);
%%		\draw (13) to (4);
%%		\draw (7) to (9);
%%		\draw (9) to (1);
%%		\draw (10) to (11);
%%		\draw (10) to (8);
%%		\draw (12) to (13);
%%		\draw (13) to (9);
%%	\end{pgfonlayer}
%%\end{tikzpicture}
%%=
%%\begin{tikzpicture}
%%	\begin{pgfonlayer}{nodelayer}
%%		\node [style=nothing] (0) at (-1, 0.5) {};
%%		\node [style=nothing] (1) at (-0.5, 0.5) {};
%%		\node [style=nothing] (2) at (-1.5, 0.5) {};
%%		\node [style=nothing] (3) at (-2, 0.5) {};
%%		\node [style=nothing] (4) at (-1, 2) {};
%%		\node [style=nothing] (5) at (-1.5, 2) {};
%%		\node [style=nothing] (6) at (-2, 2) {};
%%		\node [style=nothing] (7) at (-0.5, 2) {};
%%		\node [style=oplus] (8) at (-2, 1.5) {};
%%		\node [style=dot] (9) at (-1.5, 1.5) {};
%%		\node [style=dot] (10) at (-1, 1.5) {};
%%		\node [style=oplus] (11) at (-0.5, 1) {};
%%		\node [style=dot] (12) at (-1, 1) {};
%%		\node [style=dot] (13) at (-1.5, 1) {};
%%	\end{pgfonlayer}
%%	\begin{pgfonlayer}{edgelayer}
%%		\draw (9) to (10);
%%		\draw (9) to (8);
%%		\draw (13) to (12);
%%		\draw (12) to (11);
%%		\draw (3) to (8);
%%		\draw (8) to (6);
%%		\draw (5) to (9);
%%		\draw (9) to (13);
%%		\draw (13) to (2);
%%		\draw (0) to (12);
%%		\draw (12) to (10);
%%		\draw (10) to (4);
%%		\draw (7) to (11);
%%		\draw (11) to (1);
%%	\end{pgfonlayer}
%%\end{tikzpicture}
%%$}
%%
%%
%%\item
%%\label{TOF.6}
%%{\hfil
%%$
%%\begin{tikzpicture}
%%	\begin{pgfonlayer}{nodelayer}
%%		\node [style=nothing] (0) at (-1, 0.5) {};
%%		\node [style=nothing] (1) at (-1.5, 0.5) {};
%%		\node [style=nothing] (2) at (-2, 0.5) {};
%%		\node [style=nothing] (3) at (-1, 2) {};
%%		\node [style=nothing] (4) at (-1.5, 2) {};
%%		\node [style=nothing] (5) at (-2, 2) {};
%%		\node [style=nothing] (6) at (-0.5, 2) {};
%%		\node [style=oplus] (7) at (-0.5, 1) {};
%%		\node [style=dot] (8) at (-1.5, 1.5) {};
%%		\node [style=dot] (9) at (-1, 1.5) {};
%%		\node [style=dot] (10) at (-1, 1) {};
%%		\node [style=oplus] (11) at (-0.5, 1.5) {};
%%		\node [style=nothing] (12) at (-0.5, 0.5) {};
%%		\node [style=dot] (13) at (-2, 1) {};
%%	\end{pgfonlayer}
%%	\begin{pgfonlayer}{edgelayer}
%%		\draw (8) to (1);
%%		\draw (0) to (9);
%%		\draw (9) to (10);
%%		\draw (10) to (3);
%%		\draw (6) to (7);
%%		\draw (8) to (9);
%%		\draw (10) to (7);
%%		\draw (12) to (11);
%%		\draw (11) to (7);
%%		\draw (8) to (4);
%%		\draw (9) to (11);
%%		\draw (10) to (13);
%%		\draw (13) to (5);
%%		\draw (13) to (2);
%%	\end{pgfonlayer}
%%\end{tikzpicture}
%%=
%%\begin{tikzpicture}
%%	\begin{pgfonlayer}{nodelayer}
%%		\node [style=nothing] (0) at (-1, 0.5) {};
%%		\node [style=nothing] (1) at (-1.5, 0.5) {};
%%		\node [style=nothing] (2) at (-2, 0.5) {};
%%		\node [style=nothing] (3) at (-1, 2) {};
%%		\node [style=nothing] (4) at (-1.5, 2) {};
%%		\node [style=nothing] (5) at (-2, 2) {};
%%		\node [style=nothing] (6) at (-0.5, 2) {};
%%		\node [style=oplus] (7) at (-0.5, 1.5) {};
%%		\node [style=dot] (8) at (-1.5, 1) {};
%%		\node [style=dot] (9) at (-1, 1) {};
%%		\node [style=dot] (10) at (-1, 1.5) {};
%%		\node [style=oplus] (11) at (-0.5, 1) {};
%%		\node [style=nothing] (12) at (-0.5, 0.5) {};
%%		\node [style=dot] (13) at (-2, 1.5) {};
%%	\end{pgfonlayer}
%%	\begin{pgfonlayer}{edgelayer}
%%		\draw (8) to (1);
%%		\draw (0) to (9);
%%		\draw (9) to (10);
%%		\draw (10) to (3);
%%		\draw (6) to (7);
%%		\draw (8) to (9);
%%		\draw (10) to (7);
%%		\draw (12) to (11);
%%		\draw (11) to (7);
%%		\draw (8) to (4);
%%		\draw (9) to (11);
%%		\draw (10) to (13);
%%		\draw (13) to (5);
%%		\draw (13) to (2);
%%	\end{pgfonlayer}
%%\end{tikzpicture}
%%$}
%%
%%\item
%%\label{TOF.7}
%%{\hfil
%%$
%%\begin{tikzpicture}
%%	\begin{pgfonlayer}{nodelayer}
%%		\node [style=nothing] (0) at (1, 0) {};
%%		\node [style=nothing] (1) at (0.5, 0) {};
%%		\node [style=nothing] (2) at (0.5, 3.5) {};
%%		\node [style=nothing] (3) at (1, 3.5) {};
%%		\node [style=X] (4) at (1.5, 3) {};
%%		\node [style=oplus] (5) at (1.5, 2.5) {};
%%		\node [style=dot] (6) at (1, 2.5) {};
%%		\node [style=dot] (7) at (0.5, 1) {};
%%		\node [style=oplus] (8) at (1.5, 1) {};
%%		\node [style=X] (9) at (1.5, 1.5) {};
%%		\node [style=X] (10) at (1.5, 0.5) {$1$};
%%		\node [style=X] (11) at (1.5, 2) {$1$};
%%	\end{pgfonlayer}
%%	\begin{pgfonlayer}{edgelayer}
%%		\draw (1) to (7);
%%		\draw (7) to (2);
%%		\draw (3) to (6);
%%		\draw (6) to (0);
%%		\draw (8) to (9);
%%		\draw (8) to (7);
%%		\draw (5) to (4);
%%		\draw (5) to (6);
%%		\draw (10) to (8);
%%		\draw (11) to (5);
%%	\end{pgfonlayer}
%%\end{tikzpicture}
%%=
%%\begin{tikzpicture}
%%	\begin{pgfonlayer}{nodelayer}
%%		\node [style=nothing] (0) at (3, 0) {};
%%		\node [style=nothing] (1) at (2.5, 0) {};
%%		\node [style=nothing] (2) at (2.5, 3.5) {};
%%		\node [style=nothing] (3) at (3, 3.5) {};
%%		\node [style=dot] (4) at (2.5, 1.75) {};
%%		\node [style=dot] (5) at (3, 1.75) {};
%%		\node [style=X] (6) at (3.5, 1.25) {$1$};
%%		\node [style=X] (7) at (3.5, 2.25) {};
%%		\node [style=oplus] (8) at (3.5, 1.75) {};
%%	\end{pgfonlayer}
%%	\begin{pgfonlayer}{edgelayer}
%%		\draw (1) to (4);
%%		\draw (4) to (2);
%%		\draw (3) to (5);
%%		\draw (5) to (0);
%%		\draw (6) to (8);
%%		\draw (8) to (7);
%%		\draw (8) to (5);
%%		\draw (5) to (4);
%%	\end{pgfonlayer}
%%\end{tikzpicture}
%%$}
%%%
%%%\item
%%%\label{TOF.7}
%%%{\hfil
%%%$
%%%\begin{tikzpicture}
%%%	\begin{pgfonlayer}{nodelayer}
%%%		\node [style=nothing] (0) at (0, -0) {};
%%%		\node [style=nothing] (1) at (1.5, -0) {};
%%%		\node [style=X] (2) at (0.4, 0.5) {};
%%%		\node [style=X] (3) at (1.1, 0.5) {};
%%%	\end{pgfonlayer}
%%%	\begin{pgfonlayer}{edgelayer}
%%%		\draw (2) to (3);
%%%		\draw (0) to (1);
%%%	\end{pgfonlayer}
%%%\end{tikzpicture}
%%%=
%%%\begin{tikzpicture}
%%%	\begin{pgfonlayer}{nodelayer}
%%%		\node [style=nothing] (0) at (0, -0) {};
%%%		\node [style=nothing] (1) at (1.5, -0) {};
%%%		\node [style=X] (2) at (0.4, 0.5) {};
%%%		\node [style=X] (3) at (1.1, 0.5) {};
%%%		\node [style=X] (4) at (1, -0) {};
%%%		\node [style=X] (5) at (0.5000002, -0) {};
%%%	\end{pgfonlayer}
%%%	\begin{pgfonlayer}{edgelayer}
%%%		\draw (2) to (3);
%%%		\draw (5) to (0);
%%%		\draw (4) to (1);
%%%	\end{pgfonlayer}
%%%\end{tikzpicture}
%%%$}
%%
%%\item
%%\label{TOF.8}
%%{\hfil
%%$
%%\begin{tikzpicture}
%%	\begin{pgfonlayer}{nodelayer}
%%		\node [style=X] (0) at (0, 0.5) {$1$};
%%		\node [style=X] (1) at (0, 1.5) {$1$};
%%	\end{pgfonlayer}
%%	\begin{pgfonlayer}{edgelayer}
%%		\draw (0) to (1);
%%	\end{pgfonlayer}
%%\end{tikzpicture}
%%=
%%\begin{tikzpicture}
%%	\begin{pgfonlayer}{nodelayer}
%%		\node [style=rn] (0) at (0, 0.5) {};
%%		\node [style=rn] (1) at (0, 1.5) {};
%%	\end{pgfonlayer}
%%\end{tikzpicture}
%%%\hspace*{-.8cm}
%%%\begin{tikzpicture}[scale=.5]
%%%\begin{pgfonlayer}{nodelayer}
%%%\begin{tikzpicture}
%%%\node[cloud, cloud puffs=15.7,minimum width=3cm, draw,] (cloud) at (0,0) {$1_0$};
%%%\end{tikzpicture}
%%%\end{pgfonlayer}
%%%\begin{pgfonlayer}{edgelayer}
%%%\end{pgfonlayer}
%%%\end{tikzpicture}
%%$}
%%
%%\item
%%\label{TOF.9}
%%{\hfil
%%$
%%\begin{tikzpicture}
%%	\begin{pgfonlayer}{nodelayer}
%%		\node [style=nothing] (0) at (-1.75, 0.5) {};
%%		\node [style=nothing] (1) at (-1.25, 0.5) {};
%%		\node [style=nothing] (2) at (-0.75, 0.5) {};
%%		\node [style=dot] (3) at (-1.75, 1) {};
%%		\node [style=dot] (4) at (-1.25, 1) {};
%%		\node [style=oplus] (5) at (-0.75, 1) {};
%%		\node [style=dot] (6) at (-1.75, 1.5) {};
%%		\node [style=oplus] (7) at (-0.75, 1.5) {};
%%		\node [style=dot] (8) at (-1.25, 1.5) {};
%%		\node [style=nothing] (9) at (-1.25, 2) {};
%%		\node [style=nothing] (10) at (-0.75, 2) {};
%%		\node [style=nothing] (11) at (-1.75, 2) {};
%%	\end{pgfonlayer}
%%	\begin{pgfonlayer}{edgelayer}
%%		\draw (0) to (3);
%%		\draw (1) to (4);
%%		\draw (2) to (5);
%%		\draw (3) to (4);
%%		\draw (4) to (5);
%%		\draw (6) to (8);
%%		\draw (8) to (7);
%%		\draw (3) to (6);
%%		\draw (6) to (11);
%%		\draw (4) to (8);
%%		\draw (8) to (9);
%%		\draw (5) to (7);
%%		\draw (7) to (10);
%%	\end{pgfonlayer}
%%\end{tikzpicture}
%%=
%%\begin{tikzpicture}
%%	\begin{pgfonlayer}{nodelayer}
%%		\node [style=nothing] (0) at (-1.75, 0.5) {};
%%		\node [style=nothing] (1) at (-1.25, 0.5) {};
%%		\node [style=nothing] (2) at (-0.75, 0.5) {};
%%		\node [style=nothing] (3) at (-1.25, 2) {};
%%		\node [style=nothing] (4) at (-0.75, 2) {};
%%		\node [style=nothing] (5) at (-1.75, 2) {};
%%	\end{pgfonlayer}
%%	\begin{pgfonlayer}{edgelayer}
%%		\draw (0) to (5);
%%		\draw (1) to (3);
%%		\draw (2) to (4);
%%	\end{pgfonlayer}
%%\end{tikzpicture}
%%$}
%%
%%\item
%%\label{TOF.10}
%%{\hfil
%%$
%%\begin{tikzpicture}
%%	\begin{pgfonlayer}{nodelayer}
%%		\node [style=nothing] (0) at (0, 0.5) {};
%%		\node [style=nothing] (1) at (-0.5, 0.5) {};
%%		\node [style=nothing] (2) at (-1, 0.5) {};
%%		\node [style=nothing] (3) at (-1.5, 0.5) {};
%%		\node [style=dot] (4) at (-1, 1) {};
%%		\node [style=dot] (5) at (-0.5, 1) {};
%%		\node [style=oplus] (6) at (0, 1) {};
%%		\node [style=dot] (7) at (-1.5, 1.5) {};
%%		\node [style=oplus] (8) at (-0.5, 1.5) {};
%%		\node [style=dot] (9) at (-1, 1.5) {};
%%		\node [style=dot] (10) at (-1, 2) {};
%%		\node [style=oplus] (11) at (0, 2) {};
%%		\node [style=dot] (12) at (-0.5, 2) {};
%%		\node [style=nothing] (13) at (-1.5, 2.5) {};
%%		\node [style=nothing] (14) at (-0.5, 2.5) {};
%%		\node [style=nothing] (15) at (-1, 2.5) {};
%%		\node [style=nothing] (16) at (0, 2.5) {};
%%	\end{pgfonlayer}
%%	\begin{pgfonlayer}{edgelayer}
%%		\draw (4) to (5);
%%		\draw (5) to (6);
%%		\draw (7) to (9);
%%		\draw (9) to (8);
%%		\draw (10) to (12);
%%		\draw (12) to (11);
%%		\draw (3) to (7);
%%		\draw (7) to (13);
%%		\draw (15) to (10);
%%		\draw (10) to (9);
%%		\draw (9) to (4);
%%		\draw (4) to (2);
%%		\draw (1) to (5);
%%		\draw (5) to (8);
%%		\draw (8) to (12);
%%		\draw (12) to (14);
%%		\draw (16) to (11);
%%		\draw (11) to (6);
%%		\draw (6) to (0);
%%	\end{pgfonlayer}
%%\end{tikzpicture}
%%=
%%\begin{tikzpicture}
%%	\begin{pgfonlayer}{nodelayer}
%%		\node [style=nothing] (17) at (4.5, 0.5) {};
%%		\node [style=nothing] (18) at (4, 0.5) {};
%%		\node [style=nothing] (19) at (3.5, 0.5) {};
%%		\node [style=nothing] (20) at (3, 0.5) {};
%%		\node [style=nothing] (21) at (3, 2.5) {};
%%		\node [style=nothing] (22) at (4, 2.5) {};
%%		\node [style=nothing] (23) at (3.5, 2.5) {};
%%		\node [style=nothing] (24) at (4.5, 2.5) {};
%%		\node [style=dot] (25) at (3, 1.25) {};
%%		\node [style=dot] (26) at (3.5, 1.25) {};
%%		\node [style=dot] (27) at (3, 1.75) {};
%%		\node [style=dot] (28) at (3.5, 1.75) {};
%%		\node [style=oplus] (29) at (4, 1.75) {};
%%		\node [style=oplus] (30) at (4.5, 1.25) {};
%%	\end{pgfonlayer}
%%	\begin{pgfonlayer}{edgelayer}
%%		\draw (20) to (25);
%%		\draw (25) to (27);
%%		\draw (27) to (21);
%%		\draw (23) to (28);
%%		\draw (28) to (26);
%%		\draw (26) to (19);
%%		\draw (18) to (29);
%%		\draw (29) to (22);
%%		\draw (24) to (30);
%%		\draw (30) to (17);
%%		\draw (30) to (26);
%%		\draw (26) to (25);
%%		\draw (27) to (28);
%%		\draw (28) to (29);
%%	\end{pgfonlayer}
%%\end{tikzpicture}
%%$}
%%
%%\item
%%\label{TOF.11}
%%{\hfil
%%$
%%\begin{tikzpicture}
%%	\begin{pgfonlayer}{nodelayer}
%%		\node [style=nothing] (0) at (0, 0.5) {};
%%		\node [style=nothing] (1) at (-0.5, 0.5) {};
%%		\node [style=nothing] (2) at (-1, 0.5) {};
%%		\node [style=nothing] (3) at (-1.5, 0.5) {};
%%		\node [style=nothing] (4) at (-0.5, 2.5) {};
%%		\node [style=nothing] (5) at (0, 2.5) {};
%%		\node [style=dot] (6) at (-1.5, 1) {};
%%		\node [style=dot] (7) at (-1, 1.5) {};
%%		\node [style=dot] (8) at (-0.5, 1.5) {};
%%		\node [style=oplus] (9) at (-1, 1) {};
%%		\node [style=oplus] (10) at (0, 1.5) {};
%%		\node [style=nothing] (11) at (-1.5, 2.5) {};
%%		\node [style=nothing] (12) at (-1, 2.5) {};
%%		\node [style=oplus] (13) at (-1, 2) {};
%%		\node [style=dot] (14) at (-1.5, 2) {};
%%	\end{pgfonlayer}
%%	\begin{pgfonlayer}{edgelayer}
%%		\draw (6) to (9);
%%		\draw (7) to (8);
%%		\draw (8) to (10);
%%		\draw (0) to (10);
%%		\draw (10) to (5);
%%		\draw (4) to (8);
%%		\draw (8) to (1);
%%		\draw (2) to (9);
%%		\draw (9) to (7);
%%		\draw (6) to (3);
%%		\draw (6) to (14);
%%		\draw (14) to (11);
%%		\draw (12) to (13);
%%		\draw (13) to (7);
%%		\draw (13) to (14);
%%	\end{pgfonlayer}
%%\end{tikzpicture}
%%=
%%\begin{tikzpicture}
%%	\begin{pgfonlayer}{nodelayer}
%%		\node [style=nothing] (0) at (2, 0.5) {};
%%		\node [style=nothing] (1) at (1, 0.5) {};
%%		\node [style=nothing] (2) at (1.5, 0.5) {};
%%		\node [style=nothing] (3) at (0.5, 0.5) {};
%%		\node [style=dot] (4) at (0.5, 1.25) {};
%%		\node [style=dot] (5) at (1.5, 1.25) {};
%%		\node [style=oplus] (6) at (2, 1.25) {};
%%		\node [style=nothing] (7) at (1.5, 2.5) {};
%%		\node [style=nothing] (8) at (1, 2.5) {};
%%		\node [style=nothing] (9) at (0.5, 2.5) {};
%%		\node [style=nothing] (10) at (2, 2.5) {};
%%		\node [style=dot] (11) at (1, 1.75) {};
%%		\node [style=dot] (12) at (1.5, 1.75) {};
%%		\node [style=oplus] (13) at (2, 1.75) {};
%%	\end{pgfonlayer}
%%	\begin{pgfonlayer}{edgelayer}
%%		\draw (3) to (4);
%%		\draw (2) to (5);
%%		\draw (6) to (0);
%%		\draw (6) to (5);
%%		\draw (5) to (4);
%%		\draw (11) to (1);
%%		\draw (5) to (12);
%%		\draw (12) to (7);
%%		\draw (10) to (13);
%%		\draw (13) to (6);
%%		\draw (13) to (12);
%%		\draw (12) to (11);
%%		\draw (4) to (9);
%%		\draw (11) to (8);
%%	\end{pgfonlayer}
%%\end{tikzpicture}
%%$}
%%
%%\item
%%\label{TOF.12}
%%{\hfil
%%$
%%\begin{tikzpicture}
%%	\begin{pgfonlayer}{nodelayer}
%%		\node [style=nothing] (0) at (-0.5, 0.5) {};
%%		\node [style=nothing] (1) at (0, 0.5) {};
%%		\node [style=nothing] (2) at (-1, 0.5) {};
%%		\node [style=nothing] (3) at (-1.5, 0.5) {};
%%		\node [style=nothing] (4) at (-0.5, 2.5) {};
%%		\node [style=nothing] (5) at (-1.5, 2.5) {};
%%		\node [style=nothing] (6) at (0, 2.5) {};
%%		\node [style=nothing] (7) at (-1, 2.5) {};
%%		\node [style=dot] (8) at (-1.5, 1) {};
%%		\node [style=dot] (9) at (-1, 1) {};
%%		\node [style=oplus] (10) at (-0.5, 1) {};
%%		\node [style=oplus] (11) at (0, 1.5) {};
%%		\node [style=dot] (12) at (-1, 1.5) {};
%%		\node [style=dot] (13) at (-0.5, 1.5) {};
%%		\node [style=oplus] (14) at (-0.5, 2) {};
%%		\node [style=dot] (15) at (-1.5, 2) {};
%%		\node [style=dot] (16) at (-1, 2) {};
%%	\end{pgfonlayer}
%%	\begin{pgfonlayer}{edgelayer}
%%		\draw (8) to (9);
%%		\draw (9) to (10);
%%		\draw (12) to (13);
%%		\draw (13) to (11);
%%		\draw (15) to (16);
%%		\draw (16) to (14);
%%		\draw (3) to (8);
%%		\draw (8) to (15);
%%		\draw (15) to (5);
%%		\draw (7) to (16);
%%		\draw (16) to (12);
%%		\draw (12) to (9);
%%		\draw (9) to (2);
%%		\draw (0) to (10);
%%		\draw (10) to (13);
%%		\draw (13) to (14);
%%		\draw (14) to (4);
%%		\draw (6) to (11);
%%		\draw (11) to (1);
%%	\end{pgfonlayer}
%%\end{tikzpicture}
%%=
%%\begin{tikzpicture}
%%	\begin{pgfonlayer}{nodelayer}
%%		\node [style=nothing] (0) at (1.5, 0.25) {};
%%		\node [style=nothing] (1) at (2, 0.25) {};
%%		\node [style=nothing] (2) at (1, 0.25) {};
%%		\node [style=nothing] (3) at (0.5, 0.25) {};
%%		\node [style=nothing] (4) at (1.5, 2.25) {};
%%		\node [style=nothing] (5) at (0.5, 2.25) {};
%%		\node [style=nothing] (6) at (2, 2.25) {};
%%		\node [style=nothing] (7) at (1, 2.25) {};
%%		\node [style=dot] (8) at (1, 1.5) {};
%%		\node [style=dot] (9) at (1.5, 1.5) {};
%%		\node [style=dot] (10) at (0.5, 1) {};
%%		\node [style=dot] (11) at (1, 1) {};
%%		\node [style=oplus] (12) at (2, 1) {};
%%		\node [style=oplus] (13) at (2, 1.5) {};
%%	\end{pgfonlayer}
%%	\begin{pgfonlayer}{edgelayer}
%%		\draw (8) to (9);
%%		\draw (3) to (10);
%%		\draw (10) to (5);
%%		\draw (2) to (11);
%%		\draw (11) to (8);
%%		\draw (8) to (7);
%%		\draw (0) to (9);
%%		\draw (9) to (4);
%%		\draw (1) to (12);
%%		\draw (12) to (13);
%%		\draw (13) to (6);
%%		\draw (13) to (9);
%%		\draw (12) to (11);
%%		\draw (11) to (10);
%%	\end{pgfonlayer}
%%\end{tikzpicture}
%%$}
%%
%%\item
%%\label{TOF.13}
%%{\hfil
%%$
%%\begin{tikzpicture}
%%	\begin{pgfonlayer}{nodelayer}
%%		\node [style=nothing] (0) at (0, 0.5) {};
%%		\node [style=nothing] (1) at (-1, 0.5) {};
%%		\node [style=nothing] (2) at (-0.5, 0.5) {};
%%		\node [style=nothing] (3) at (-1.5, 0.5) {};
%%		\node [style=nothing] (4) at (0, 2.5) {};
%%		\node [style=dot] (5) at (-1.5, 1) {};
%%		\node [style=dot] (6) at (-1, 1) {};
%%		\node [style=dot] (7) at (-0.5, 1.5) {};
%%		\node [style=oplus] (8) at (-0.5, 1) {};
%%		\node [style=oplus] (9) at (0, 1.5) {};
%%		\node [style=nothing] (10) at (-0.5, 2.5) {};
%%		\node [style=nothing] (11) at (-1.5, 2.5) {};
%%		\node [style=nothing] (12) at (-1, 2.5) {};
%%		\node [style=oplus] (13) at (-0.5, 2) {};
%%		\node [style=dot] (14) at (-1, 2) {};
%%		\node [style=dot] (15) at (-1.5, 2) {};
%%	\end{pgfonlayer}
%%	\begin{pgfonlayer}{edgelayer}
%%		\draw (5) to (3);
%%		\draw (6) to (1);
%%		\draw (2) to (8);
%%		\draw (8) to (7);
%%		\draw (4) to (9);
%%		\draw (9) to (0);
%%		\draw (8) to (6);
%%		\draw (6) to (5);
%%		\draw (9) to (7);
%%		\draw (5) to (15);
%%		\draw (15) to (11);
%%		\draw (12) to (14);
%%		\draw (14) to (6);
%%		\draw (7) to (13);
%%		\draw (13) to (10);
%%		\draw (13) to (14);
%%		\draw (14) to (15);
%%	\end{pgfonlayer}
%%\end{tikzpicture}
%%=
%%\begin{tikzpicture}
%%	\begin{pgfonlayer}{nodelayer}
%%		\node [style=nothing] (0) at (2, 0.25) {};
%%		\node [style=nothing] (1) at (1, 0.25) {};
%%		\node [style=nothing] (2) at (1.5, 0.25) {};
%%		\node [style=nothing] (3) at (0.5, 0.25) {};
%%		\node [style=dot] (4) at (0.5, 1) {};
%%		\node [style=dot] (5) at (1, 1) {};
%%		\node [style=oplus] (6) at (2, 1) {};
%%		\node [style=nothing] (7) at (1.5, 2.25) {};
%%		\node [style=nothing] (8) at (1, 2.25) {};
%%		\node [style=nothing] (9) at (2, 2.25) {};
%%		\node [style=nothing] (10) at (0.5, 2.25) {};
%%		\node [style=dot] (11) at (1.5, 1.5) {};
%%		\node [style=oplus] (12) at (2, 1.5) {};
%%	\end{pgfonlayer}
%%	\begin{pgfonlayer}{edgelayer}
%%		\draw (0) to (6);
%%		\draw (1) to (5);
%%		\draw (4) to (3);
%%		\draw (4) to (5);
%%		\draw (5) to (6);
%%		\draw (11) to (12);
%%		\draw (12) to (9);
%%		\draw (12) to (6);
%%		\draw (2) to (11);
%%		\draw (4) to (10);
%%		\draw (8) to (5);
%%		\draw (11) to (7);
%%	\end{pgfonlayer}
%%\end{tikzpicture}
%%$}
%%
%%\item
%%\label{TOF.14}
%%{\hfil
%%$
%%\begin{tikzpicture}
%%	\begin{pgfonlayer}{nodelayer}
%%		\node [style=nothing] (0) at (0, 0.5) {};
%%		\node [style=nothing] (1) at (-0.5, 0.5) {};
%%		\node [style=nothing] (2) at (-0.5, 2.5) {};
%%		\node [style=nothing] (3) at (0, 2.5) {};
%%		\node [style=oplus] (4) at (0, 1) {};
%%		\node [style=oplus] (5) at (0, 2) {};
%%		\node [style=oplus] (6) at (-0.5, 1.5) {};
%%		\node [style=dot] (7) at (-0.5, 2) {};
%%		\node [style=dot] (8) at (0, 1.5) {};
%%		\node [style=dot] (9) at (-0.5, 1) {};
%%	\end{pgfonlayer}
%%	\begin{pgfonlayer}{edgelayer}
%%		\draw (1) to (9);
%%		\draw (9) to (6);
%%		\draw (6) to (7);
%%		\draw (7) to (2);
%%		\draw (3) to (5);
%%		\draw (5) to (8);
%%		\draw (8) to (4);
%%		\draw (4) to (0);
%%		\draw (4) to (9);
%%		\draw (8) to (6);
%%		\draw (5) to (7);
%%	\end{pgfonlayer}
%%\end{tikzpicture}
%%=
%%\begin{tikzpicture}
%%	\begin{pgfonlayer}{nodelayer}
%%		\node [style=nothing] (0) at (1, 0.5) {};
%%		\node [style=nothing] (1) at (0.5, 0.5) {};
%%		\node [style=nothing] (2) at (0.5, 2.5) {};
%%		\node [style=nothing] (3) at (1, 2.5) {};
%%	\end{pgfonlayer}
%%	\begin{pgfonlayer}{edgelayer}
%%		\draw [in=-90, out=90, looseness=1.25] (1) to (3);
%%		\draw [in=-90, out=90, looseness=1.25] (0) to (2);
%%	\end{pgfonlayer}
%%\end{tikzpicture}
%%$}
%%
%%\item
%%\label{TOF.15}
%%{\hfil
%%$
%%\begin{tikzpicture}
%%	\begin{pgfonlayer}{nodelayer}
%%		\node [style=nothing] (0) at (-1.75, 0.5) {};
%%		\node [style=nothing] (1) at (-1.25, 0.5) {};
%%		\node [style=nothing] (2) at (-0.75, 0.5) {};
%%		\node [style=nothing] (3) at (-1.75, 2.5) {};
%%		\node [style=nothing] (4) at (-1.25, 2.5) {};
%%		\node [style=nothing] (5) at (-0.75, 2.5) {};
%%		\node [style=dot] (6) at (-1.75, 1.5) {};
%%		\node [style=dot] (7) at (-1.25, 1.5) {};
%%		\node [style=oplus] (8) at (-0.75, 1.5) {};
%%	\end{pgfonlayer}
%%	\begin{pgfonlayer}{edgelayer}
%%		\draw (0) to (6);
%%		\draw (6) to (3);
%%		\draw (4) to (7);
%%		\draw (7) to (1);
%%		\draw (2) to (8);
%%		\draw (8) to (5);
%%		\draw (8) to (7);
%%		\draw (7) to (6);
%%	\end{pgfonlayer}
%%\end{tikzpicture}
%%=
%%\begin{tikzpicture}
%%	\begin{pgfonlayer}{nodelayer}
%%		\node [style=nothing] (0) at (-1.75, 0.5) {};
%%		\node [style=nothing] (1) at (-1.25, 0.5) {};
%%		\node [style=nothing] (2) at (-0.75, 0.5) {};
%%		\node [style=dot] (3) at (-1.75, 1.5) {};
%%		\node [style=dot] (4) at (-1.25, 1.5) {};
%%		\node [style=oplus] (5) at (-0.75, 1.5) {};
%%		\node [style=nothing] (6) at (-1.75, 2.5) {};
%%		\node [style=nothing] (7) at (-1.25, 2.5) {};
%%		\node [style=nothing] (8) at (-0.75, 2.5) {};
%%	\end{pgfonlayer}
%%	\begin{pgfonlayer}{edgelayer}
%%		\draw [in=-90, out=90, looseness=1.25] (0) to (4);
%%		\draw [in=-90, out=90, looseness=1.25] (4) to (6);
%%		\draw [in=-90, out=90, looseness=1.25] (3) to (7);
%%		\draw [in=90, out=-90, looseness=1.25] (3) to (1);
%%		\draw (2) to (5);
%%		\draw (5) to (8);
%%		\draw (3) to (4);
%%		\draw (4) to (5);
%%	\end{pgfonlayer}
%%\end{tikzpicture}
%%$}
%%
%%\item
%%\label{TOF.16}
%%{\hfil
%%$
%%\begin{tikzpicture}
%%	\begin{pgfonlayer}{nodelayer}
%%		\node [style=nothing] (0) at (2.5, 0.5) {};
%%		\node [style=nothing] (1) at (1, 0.5) {};
%%		\node [style=nothing] (2) at (2, 0.5) {};
%%		\node [style=nothing] (3) at (0.5, 0.5) {};
%%		\node [style=X] (4) at (1.5, 0.75) {};
%%		\node [style=oplus] (5) at (1.5, 1.75) {};
%%		\node [style=oplus] (6) at (1.5, 2.75) {};
%%		\node [style=dot] (7) at (1.5, 2.25) {};
%%		\node [style=dot] (8) at (2, 2.25) {};
%%		\node [style=dot] (9) at (1, 1.75) {};
%%		\node [style=dot] (10) at (0.5, 1.75) {};
%%		\node [style=dot] (11) at (1, 2.75) {};
%%		\node [style=dot] (12) at (0.5, 2.75) {};
%%		\node [style=oplus] (13) at (2.5, 2.25) {};
%%		\node [style=X] (14) at (1.5, 3.75) {};
%%		\node [style=nothing] (15) at (2.5, 4) {};
%%		\node [style=nothing] (16) at (0.5, 4) {};
%%		\node [style=nothing] (17) at (1, 4) {};
%%		\node [style=nothing] (18) at (2, 4) {};
%%	\end{pgfonlayer}
%%	\begin{pgfonlayer}{edgelayer}
%%		\draw (3) to (10);
%%		\draw (10) to (12);
%%		\draw (12) to (16);
%%		\draw (11) to (9);
%%		\draw (15) to (13);
%%		\draw (13) to (0);
%%		\draw (13) to (8);
%%		\draw (8) to (7);
%%		\draw (9) to (5);
%%		\draw (9) to (10);
%%		\draw (12) to (11);
%%		\draw (6) to (11);
%%		\draw (4) to (5);
%%		\draw (5) to (7);
%%		\draw (7) to (6);
%%		\draw (14) to (6);
%%		\draw [style=simple, in=90, out=-90, looseness=1.25] (17) to (8);
%%		\draw [style=simple, in=90, out=-90, looseness=1.25] (8) to (1);
%%		\draw [style=simple, in=270, out=90] (2) to (9);
%%		\draw [style=simple, in=270, out=90] (11) to (18);
%%	\end{pgfonlayer}
%%\end{tikzpicture}
%%=
%%\begin{tikzpicture}
%%	\begin{pgfonlayer}{nodelayer}
%%		\node [style=nothing] (0) at (2.5, 0.5) {};
%%		\node [style=nothing] (1) at (2, 0.5) {};
%%		\node [style=nothing] (2) at (1, 0.5) {};
%%		\node [style=nothing] (3) at (0.5, 0.5) {};
%%		\node [style=X] (4) at (1.5, 1.25) {};
%%		\node [style=oplus] (5) at (1.5, 1.75) {};
%%		\node [style=oplus] (6) at (1.5, 2.75) {};
%%		\node [style=dot] (7) at (1.5, 2.25) {};
%%		\node [style=dot] (8) at (2, 2.25) {};
%%		\node [style=dot] (9) at (1, 1.75) {};
%%		\node [style=dot] (10) at (0.5, 1.75) {};
%%		\node [style=dot] (11) at (1, 2.75) {};
%%		\node [style=dot] (12) at (0.5, 2.75) {};
%%		\node [style=oplus] (13) at (2.5, 2.25) {};
%%		\node [style=X] (14) at (1.5, 3.25) {};
%%		\node [style=nothing] (15) at (2.5, 4) {};
%%		\node [style=nothing] (16) at (0.5, 4) {};
%%		\node [style=nothing] (17) at (2, 4) {};
%%		\node [style=nothing] (18) at (1, 4) {};
%%	\end{pgfonlayer}
%%	\begin{pgfonlayer}{edgelayer}
%%		\draw (3) to (10);
%%		\draw (10) to (12);
%%		\draw (12) to (16);
%%		\draw (11) to (9);
%%		\draw (15) to (13);
%%		\draw (13) to (0);
%%		\draw (13) to (8);
%%		\draw (8) to (7);
%%		\draw (9) to (5);
%%		\draw (9) to (10);
%%		\draw (12) to (11);
%%		\draw (6) to (11);
%%		\draw (4) to (5);
%%		\draw (5) to (7);
%%		\draw (7) to (6);
%%		\draw (14) to (6);
%%		\draw [style=simple] (17) to (8);
%%		\draw [style=simple] (8) to (1);
%%		\draw [style=simple] (2) to (9);
%%		\draw [style=simple] (11) to (18);
%%	\end{pgfonlayer}
%%\end{tikzpicture}
%%$}
%%\end{enumerate}
%%\end{multicols}
%%
%%Where the controlled-not gate is derived:
%%$$
%%\begin{tikzpicture}
%%	\begin{pgfonlayer}{nodelayer}
%%		\node [style=dot] (1) at (1.5, 1) {};
%%		\node [style=oplus] (2) at (2, 1) {};
%%		\node [style=none] (5) at (1.5, 0.25) {};
%%		\node [style=none] (6) at (2, 0.25) {};
%%		\node [style=none] (7) at (2, 1.75) {};
%%		\node [style=none] (8) at (1.5, 1.75) {};
%%	\end{pgfonlayer}
%%	\begin{pgfonlayer}{edgelayer}
%%		\draw (5.center) to (1);
%%		\draw (1) to (8.center);
%%		\draw (7.center) to (2);
%%		\draw (2) to (6.center);
%%		\draw (2) to (1);
%%	\end{pgfonlayer}
%%\end{tikzpicture}
%%:=
%%\begin{tikzpicture}
%%	\begin{pgfonlayer}{nodelayer}
%%		\node [style=dot] (0) at (1, 1) {};
%%		\node [style=dot] (1) at (1.5, 1) {};
%%		\node [style=oplus] (2) at (2, 1) {};
%%		\node [style=X] (3) at (1, 1.5) {$1$};
%%		\node [style=X] (4) at (1, 0.5) {$1$};
%%		\node [style=none] (5) at (1.5, 0.25) {};
%%		\node [style=none] (6) at (2, 0.25) {};
%%		\node [style=none] (7) at (2, 1.75) {};
%%		\node [style=none] (8) at (1.5, 1.75) {};
%%	\end{pgfonlayer}
%%	\begin{pgfonlayer}{edgelayer}
%%		\draw (5.center) to (1);
%%		\draw (1) to (8.center);
%%		\draw (7.center) to (2);
%%		\draw (2) to (6.center);
%%		\draw (2) to (1);
%%		\draw (1) to (0);
%%		\draw (0) to (3);
%%		\draw (4) to (0);
%%	\end{pgfonlayer}
%%\end{tikzpicture}
%%$$
%%
%%\subsubsection{$(\FPar_2,\times)$}
%%\label{subsubsec:presentations:three:par}
%%$(\FPar_2,\times)$ is presented by the generators and equations in \S \ref{subsubsec:presentations:two:par} as well as the additional generator $
%%\begin{tikzpicture}
%%	\begin{pgfonlayer}{nodelayer}
%%		\node [style=none] (0) at (-3.75, 0.5) {};
%%		\node [style=none] (1) at (-3.75, -0.25) {};
%%		\node [style=andin] (2) at (-3.75, -0.25) {};
%%		\node [style=none] (3) at (-4, -1) {};
%%		\node [style=none] (4) at (-3.5, -1) {};
%%	\end{pgfonlayer}
%%	\begin{pgfonlayer}{edgelayer}
%%		\draw (0.center) to (1.center);
%%		\draw [in=-60, out=90, looseness=1.00] (4.center) to (1.center);
%%		\draw [in=90, out=-120, looseness=1.00] (1.center) to (3.center);
%%	\end{pgfonlayer}
%%\end{tikzpicture}
%%$ (the and gate),  so that 
%%$
%%\left(
%%\begin{tikzpicture}
%%	\begin{pgfonlayer}{nodelayer}
%%		\node [style=none] (0) at (-3.75, 0.5) {};
%%		\node [style=none] (1) at (-3.75, -0.25) {};
%%		\node [style=andin] (2) at (-3.75, -0.25) {};
%%		\node [style=none] (3) at (-4, -1) {};
%%		\node [style=none] (4) at (-3.5, -1) {};
%%	\end{pgfonlayer}
%%	\begin{pgfonlayer}{edgelayer}
%%		\draw (0.center) to (1.center);
%%		\draw [in=-60, out=90, looseness=1.00] (4.center) to (1.center);
%%		\draw [in=90, out=-120, looseness=1.00] (1.center) to (3.center);
%%	\end{pgfonlayer}
%%\end{tikzpicture},
%%\begin{tikzpicture}
%%	\begin{pgfonlayer}{nodelayer}
%%		\node [style=none] (0) at (0.75, 0.5) {};
%%		\node [style=none] (1) at (0.75, -0.25) {};
%%		\node [style=X] (2) at (0.75, -0.25) {$1$};
%%	\end{pgfonlayer}
%%	\begin{pgfonlayer}{edgelayer}
%%		\draw (0.center) to (1.center);
%%	\end{pgfonlayer}
%%\end{tikzpicture},
%%\begin{tikzpicture}
%%	\begin{pgfonlayer}{nodelayer}
%%		\node [style=none] (0) at (0.75, -1) {};
%%		\node [style=none] (1) at (0.75, -0.25) {};
%%		\node [style=Z] (2) at (0.75, -0.25) {};
%%		\node [style=none] (3) at (0.5, 0.5) {};
%%		\node [style=none] (4) at (1, 0.5) {};
%%	\end{pgfonlayer}
%%	\begin{pgfonlayer}{edgelayer}
%%		\draw (0.center) to (1.center);
%%		\draw [in=60, out=-90] (4.center) to (1.center);
%%		\draw [in=-90, out=120] (1.center) to (3.center);
%%	\end{pgfonlayer}
%%\end{tikzpicture},
%%\begin{tikzpicture}
%%	\begin{pgfonlayer}{nodelayer}
%%		\node [style=none] (0) at (0.75, -0.25) {};
%%		\node [style=none] (1) at (0.75, 0.5) {};
%%		\node [style=Z] (2) at (0.75, 0.5) {};
%%	\end{pgfonlayer}
%%	\begin{pgfonlayer}{edgelayer}
%%		\draw (0.center) to (1.center);
%%	\end{pgfonlayer}
%%\end{tikzpicture}
%%\right)
%%$ forms a bicommutative bialgebra; and additionally:
%%$$
%%\begin{tikzpicture}
%%	\begin{pgfonlayer}{nodelayer}
%%		\node [style=none] (0) at (-7, 1) {};
%%		\node [style=none] (1) at (-7, 0.5) {};
%%		\node [style=Z] (2) at (-7, -0.25) {};
%%		\node [style=none] (3) at (-7, -0.75) {};
%%		\node [style=andin] (4) at (-7, 0.5) {};
%%	\end{pgfonlayer}
%%	\begin{pgfonlayer}{edgelayer}
%%		\draw (3.center) to (2.center);
%%		\draw [in=-60, out=60, looseness=1.25] (2.center) to (1);
%%		\draw [in=120, out=-120, looseness=1.25] (1) to (2.center);
%%		\draw (1) to (0.center);
%%	\end{pgfonlayer}
%%\end{tikzpicture}
%%\eref{antispecial}
%%\begin{tikzpicture}
%%	\begin{pgfonlayer}{nodelayer}
%%		\node [style=none] (0) at (-7, 1) {};
%%		\node [style=none] (1) at (-7, -0.75) {};
%%	\end{pgfonlayer}
%%	\begin{pgfonlayer}{edgelayer}
%%		\draw (1.center) to (0.center);
%%	\end{pgfonlayer}
%%\end{tikzpicture},
%%\hspace*{.5cm}
%%\begin{tikzpicture}
%%	\begin{pgfonlayer}{nodelayer}
%%		\node [style=andin] (4) at (1.25, 0.5) {};
%%		\node [style=X] (5) at (0.75, -0.5) {};
%%		\node [style=none] (6) at (0.5, -1) {};
%%		\node [style=none] (7) at (1, -1) {};
%%		\node [style=none] (8) at (1.75, -1) {};
%%		\node [style=none] (9) at (1.25, 0.5) {};
%%		\node [style=none] (10) at (1.25, 1.5) {};
%%	\end{pgfonlayer}
%%	\begin{pgfonlayer}{edgelayer}
%%		\draw [in=-30, out=90] (8.center) to (9.center);
%%		\draw [in=90, out=-150] (9.center) to (5);
%%		\draw [in=90, out=-45] (5) to (7.center);
%%		\draw [in=-135, out=90] (6.center) to (5);
%%		\draw (9.center) to (10.center);
%%	\end{pgfonlayer}
%%\end{tikzpicture}
%%  \eref{ring.mul}
%%\begin{tikzpicture}
%%	\begin{pgfonlayer}{nodelayer}
%%		\node [style=none] (0) at (1, 0) {};
%%		\node [style=none] (1) at (0.5, -1.25) {};
%%		\node [style=none] (2) at (1.75, -0.75) {};
%%		\node [style=none] (3) at (1.33, 0.75) {};
%%		\node [style=andin] (4) at (1, 0) {};
%%		\node [style=none] (5) at (1.75, 0) {};
%%		\node [style=none] (6) at (1, -1.25) {};
%%		\node [style=none] (7) at (1.75, -0.75) {};
%%		\node [style=none] (8) at (1.33, 0.75) {};
%%		\node [style=andin] (9) at (1.75, 0) {};
%%		\node [style=X] (10) at (1.33, 0.75) {};
%%		\node [style=none] (11) at (1.33, 1.25) {};
%%		\node [style=none] (12) at (1.75, -1.25) {};
%%		\node [style=Z] (13) at (1.75, -0.75) {};
%%	\end{pgfonlayer}
%%	\begin{pgfonlayer}{edgelayer}
%%		\draw [in=-135, out=90] (0.center) to (3.center);
%%		\draw [in=165, out=-30, looseness=1.25] (0.center) to (2.center);
%%		\draw [in=-45, out=90] (5.center) to (8.center);
%%		\draw [in=45, out=-45, looseness=1.25] (5.center) to (7.center);
%%		\draw (10) to (11.center);
%%		\draw [in=90, out=-150] (4) to (1.center);
%%		\draw [in=-150, out=90] (6.center) to (9);
%%		\draw (12.center) to (13);
%%	\end{pgfonlayer}
%%\end{tikzpicture},
%%\hspace*{.5cm}
%%\begin{tikzpicture}
%%	\begin{pgfonlayer}{nodelayer}
%%		\node [style=none] (0) at (2, 0) {};
%%		\node [style=none] (1) at (1.75, -0.75) {};
%%		\node [style=none] (2) at (2.25, -0.75) {};
%%		\node [style=none] (3) at (2, 0.5) {};
%%		\node [style=none] (4) at (2.25, -1) {};
%%		\node [style=X] (5) at (1.75, -0.75) {};
%%		\node [style=andin] (6) at (2, 0) {};
%%	\end{pgfonlayer}
%%	\begin{pgfonlayer}{edgelayer}
%%		\draw (0.center) to (3.center);
%%		\draw [in=90, out=-45] (0.center) to (2.center);
%%		\draw (4.center) to (2.center);
%%		\draw [in=-135, out=90] (1.center) to (0.center);
%%	\end{pgfonlayer}
%%\end{tikzpicture}
%%\eref{ring.unit}
%%\begin{tikzpicture}
%%	\begin{pgfonlayer}{nodelayer}
%%		\node [style=none] (12) at (2, 0.5) {};
%%		\node [style=none] (14) at (2, -1) {};
%%		\node [style=X] (15) at (2, 0) {};
%%		\node [style=Z] (16) at (2, -0.5) {};
%%	\end{pgfonlayer}
%%	\begin{pgfonlayer}{edgelayer}
%%		\draw (15) to (12.center);
%%		\draw (16) to (14.center);
%%	\end{pgfonlayer}
%%\end{tikzpicture},
%%\hspace*{.5cm}
%%\begin{tikzpicture}
%%	\begin{pgfonlayer}{nodelayer}
%%		\node [style=none] (0) at (0.75, 0.5) {};
%%		\node [style=none] (1) at (0.75, -0.25) {};
%%		\node [style=andin] (2) at (0.75, -0.25) {};
%%		\node [style=none] (3) at (0.5, -1) {};
%%		\node [style=none] (4) at (1, -1) {};
%%		\node [style=X] (5) at (0.75, 0.5) {$1$};
%%	\end{pgfonlayer}
%%	\begin{pgfonlayer}{edgelayer}
%%		\draw (0.center) to (1.center);
%%		\draw [in=-60, out=90] (4.center) to (1.center);
%%		\draw [in=90, out=-120] (1.center) to (3.center);
%%	\end{pgfonlayer}
%%\end{tikzpicture}
%%  \eref{bi.two}
%%\begin{tikzpicture}
%%	\begin{pgfonlayer}{nodelayer}
%%		\node [style=none] (3) at (0.5, -1) {};
%%		\node [style=none] (4) at (1, -1) {};
%%		\node [style=X] (5) at (0.5, 0.5) {$1$};
%%		\node [style=X] (6) at (1, 0.5) {$1$};
%%	\end{pgfonlayer}
%%	\begin{pgfonlayer}{edgelayer}
%%		\draw (3.center) to (5);
%%		\draw (6) to (4.center);
%%	\end{pgfonlayer}
%%\end{tikzpicture}
%%$$
%\subsubsection{$(\FSpan_2,\times)$}
%\label{subsubsec:presentations:three:span}
%
%$(\FSpan_2,\times)$ is presented by the generators and equations of \S \ref{subsubsec:presentations:three:span} as well as the generator 
%$\begin{tikzpicture}
%	\begin{pgfonlayer}{nodelayer}
%		\node [style=none] (0) at (0.75, 0.5) {};
%		\node [style=none] (1) at (0.75, -0.25) {};
%		\node [style=Z] (2) at (0.75, -0.25) {};
%	\end{pgfonlayer}
%	\begin{pgfonlayer}{edgelayer}
%		\draw (0.center) to (1.center);
%	\end{pgfonlayer}
%\end{tikzpicture}$ 
%and the equation making the codiagonal map counital:
%$$
%  \begin{tikzpicture}[rotate=90,yscale=-1]
%	\begin{pgfonlayer}{nodelayer}
%		\node [style=Z] (0) at (-9, -0) {};
%		\node [style=none] (1) at (-8.25, -0) {};
%		\node [style=Z] (2) at (-9.75, 0.25) {};
%		\node [style=none] (3) at (-10, -0.25) {};
%	\end{pgfonlayer}
%	\begin{pgfonlayer}{edgelayer}
%		\draw [in=-150, out=0, looseness=1.00] (3.center) to (0);
%		\draw [in=150, out=0, looseness=1.00] (2.center) to (0);
%		\draw (0) to (1.center);
%	\end{pgfonlayer}
%  \end{tikzpicture}
%  \eref{unit}
%  \begin{tikzpicture}[rotate=90]
%	\begin{pgfonlayer}{nodelayer}
%		\node [style=none] (0) at (-9, 0.25) {};
%		\node [style=none] (1) at (-9.75, 0.25) {};
%	\end{pgfonlayer}
%	\begin{pgfonlayer}{edgelayer}
%		\draw (1) to (0.center);
%	\end{pgfonlayer}
%  \end{tikzpicture}
%$$
%%
%%
%%\end{comment}




%
%This prop is equivalently presented in terms of the 
%
%\begin{figure}[H]
%	\noindent
%	\scalebox{1.0}{%
%		\vbox{%
%			\begin{mdframed}
%				\begin{multicols}{2}
%					\begin{enumerate}[label={\bf [CNOT.\arabic*]}, ref={\bf [CNOT.\arabic*]}, wide = 0pt, leftmargin = 2em]
%						\item
%						\label{CNOT.1}
%						{\hfil
%							$
%			\begin{tikzpicture}
%	\begin{pgfonlayer}{nodelayer}
%		\node [style=nothing] (26) at (0, 6) {};
%		\node [style=nothing] (27) at (-0.5, 6) {};
%		\node [style=oplus] (28) at (0, 6.5) {};
%		\node [style=dot] (29) at (-0.5, 6.5) {};
%		\node [style=dot] (30) at (0, 7) {};
%		\node [style=oplus] (31) at (-0.5, 7) {};
%		\node [style=oplus] (32) at (0, 7.5) {};
%		\node [style=dot] (33) at (-0.5, 7.5) {};
%		\node [style=nothing] (34) at (0, 8) {};
%		\node [style=nothing] (35) at (-0.5, 8) {};
%	\end{pgfonlayer}
%	\begin{pgfonlayer}{edgelayer}
%		\draw [style=simple] (26) to (34);
%		\draw [style=simple] (27) to (35);
%		\draw [style=simple] (28) to (29);
%		\draw [style=simple] (30) to (31);
%		\draw [style=simple] (32) to (33);
%	\end{pgfonlayer}
%\end{tikzpicture}
%							=
%							\begin{tikzpicture}
%	\begin{pgfonlayer}{nodelayer}
%		\node [style=nothing] (0) at (0, 0.5) {};
%		\node [style=nothing] (1) at (-0.5, 0.5) {};
%		\node [style=nothing] (2) at (-0.5, 1.5) {};
%		\node [style=nothing] (3) at (0, 1.5) {};
%	\end{pgfonlayer}
%	\begin{pgfonlayer}{edgelayer}
%		\draw [in=-90, out=90, looseness=1.25] (1) to (3);
%		\draw [in=-90, out=90, looseness=1.25] (0) to (2);
%	\end{pgfonlayer}
%\end{tikzpicture}
%$}
%						
%						
%						\item
%						\label{CNOT.2}
%						\hfil{
%							$
%							\begin{tikzpicture}
%	\begin{pgfonlayer}{nodelayer}
%		\node [style=nothing] (1) at (0, 0) {};
%		\node [style=nothing] (2) at (-0.5, 0) {};
%		\node [style=oplus] (3) at (0, 0.5) {};
%		\node [style=dot] (4) at (-0.5, 0.5) {};
%		\node [style=oplus] (5) at (0, 1) {};
%		\node [style=dot] (6) at (-0.5, 1) {};
%		\node [style=nothing] (7) at (0, 1.5) {};
%		\node [style=nothing] (8) at (-0.5, 1.5) {};
%	\end{pgfonlayer}
%	\begin{pgfonlayer}{edgelayer}
%		\draw [style=simple] (1) to (7);
%		\draw [style=simple] (2) to (8);
%		\draw [style=simple] (3) to (4);
%		\draw [style=simple] (5) to (6);
%	\end{pgfonlayer}
%\end{tikzpicture}
%							=
%							\begin{tikzpicture}
%	\begin{pgfonlayer}{nodelayer}
%		\node [style=nothing] (2) at (0, 0) {};
%		\node [style=nothing] (3) at (-0.5, 0) {};
%		\node [style=nothing] (4) at (0, 1.5) {};
%		\node [style=nothing] (5) at (-0.5, 1.5) {};
%	\end{pgfonlayer}
%	\begin{pgfonlayer}{edgelayer}
%		\draw [style=simple] (2) to (4);
%		\draw [style=simple] (3) to (5);
%	\end{pgfonlayer}
%\end{tikzpicture}
%							$}
%						
%						\item
%						\label{CNOT.3}
%						\hfil{
%							$
%							\begin{tikzpicture}
%	\begin{pgfonlayer}{nodelayer}
%		\node [style=nothing] (3) at (-1, 0) {};
%		\node [style=nothing] (4) at (-0.5, 0) {};
%		\node [style=nothing] (5) at (0, 0) {};
%		\node [style=oplus] (6) at (-1, 0.75) {};
%		\node [style=dot] (7) at (-0.5, 0.75) {};
%		\node [style=dot] (8) at (-0.5, 1.25) {};
%		\node [style=oplus] (9) at (0, 1.25) {};
%		\node [style=nothing] (10) at (-1, 2) {};
%		\node [style=nothing] (11) at (-0.5, 2) {};
%		\node [style=nothing] (12) at (0, 2) {};
%	\end{pgfonlayer}
%	\begin{pgfonlayer}{edgelayer}
%		\draw [style=simple] (3) to (10);
%		\draw [style=simple] (4) to (11);
%		\draw [style=simple] (5) to (12);
%		\draw [style=simple] (6) to (7);
%		\draw [style=simple] (8) to (9);
%	\end{pgfonlayer}
%\end{tikzpicture}
%							=
%							\begin{tikzpicture}
%	\begin{pgfonlayer}{nodelayer}
%		\node [style=nothing] (4) at (-1, 2.75) {};
%		\node [style=nothing] (5) at (-0.5, 2.75) {};
%		\node [style=nothing] (6) at (0, 2.75) {};
%		\node [style=oplus] (7) at (-1, 4) {};
%		\node [style=dot] (8) at (-0.5, 4) {};
%		\node [style=dot] (9) at (-0.5, 3.5) {};
%		\node [style=oplus] (10) at (0, 3.5) {};
%		\node [style=nothing] (11) at (-1, 4.75) {};
%		\node [style=nothing] (12) at (-0.5, 4.75) {};
%		\node [style=nothing] (13) at (0, 4.75) {};
%	\end{pgfonlayer}
%	\begin{pgfonlayer}{edgelayer}
%		\draw [style=simple] (4) to (11);
%		\draw [style=simple] (5) to (12);
%		\draw [style=simple] (6) to (13);
%		\draw [style=simple] (7) to (8);
%		\draw [style=simple] (9) to (10);
%	\end{pgfonlayer}
%\end{tikzpicture}
%							$}
%						
%						\item 
%						\label{CNOT.4}
%						\hfil{
%							\begin{tabular}{c}
%							$
%							\begin{tikzpicture}
%	\begin{pgfonlayer}{nodelayer}
%		\node [style=onein] (5) at (-0.5, 2.75) {};
%		\node [style=nothing] (6) at (0, 2.75) {};
%		\node [style=dot] (7) at (-0.5, 3.25) {};
%		\node [style=oplus] (8) at (0, 3.25) {};
%		\node [style=nothing] (9) at (-0.5, 3.75) {};
%		\node [style=nothing] (10) at (0, 3.75) {};
%	\end{pgfonlayer}
%	\begin{pgfonlayer}{edgelayer}
%		\draw [style=simple] (5) to (9);
%		\draw [style=simple] (6) to (10);
%		\draw [style=simple] (7) to (8);
%	\end{pgfonlayer}
%\end{tikzpicture}
%							=
%							\begin{tikzpicture}
%	\begin{pgfonlayer}{nodelayer}
%		\node [style=onein] (6) at (-0.5, 2.75) {};
%		\node [style=nothing] (7) at (0, 2.75) {};
%		\node [style=dot] (8) at (-0.5, 3.25) {};
%		\node [style=oplus] (9) at (0, 3.25) {};
%		\node [style=oneout] (10) at (-0.5, 3.75) {};
%		\node [style=nothing] (11) at (0, 4.75) {};
%		\node [style=onein] (12) at (-0.5, 4.25) {};
%		\node [style=nothing] (13) at (-0.5, 4.75) {};
%	\end{pgfonlayer}
%	\begin{pgfonlayer}{edgelayer}
%		\draw [style=simple] (6) to (10);
%		\draw [style=simple] (7) to (11);
%		\draw [style=simple] (8) to (9);
%		\draw [style=simple] (12) to (13);
%	\end{pgfonlayer}
%\end{tikzpicture}$\\
%							$ $\\
%							$\begin{tikzpicture}[tikzfig]
%	\begin{pgfonlayer}{nodelayer}
%		\node [style=nothing] (0) at (-0.5, 0.5) {};
%		\node [style=nothing] (1) at (0, 0.5) {};
%		\node [style=dot] (2) at (-0.5, 1) {};
%		\node [style=oplus] (3) at (0, 1) {};
%		\node [style=oneout] (4) at (-0.5, 1.5) {};
%		\node [style=nothing] (5) at (0, 1.5) {};
%	\end{pgfonlayer}
%	\begin{pgfonlayer}{edgelayer}
%		\draw [style=simple] (0) to (4);
%		\draw [style=simple] (1) to (5);
%		\draw [style=simple] (2) to (3);
%	\end{pgfonlayer}
%\end{tikzpicture}
%							=
%							\begin{tikzpicture}
%	\begin{pgfonlayer}{nodelayer}
%		\node [style=oneout] (8) at (-0.5, 7.25) {};
%		\node [style=nothing] (9) at (0, 7.25) {};
%		\node [style=dot] (10) at (-0.5, 6.75) {};
%		\node [style=oplus] (11) at (0, 6.75) {};
%		\node [style=onein] (12) at (-0.5, 6.25) {};
%		\node [style=nothing] (13) at (0, 5.25) {};
%		\node [style=oneout] (14) at (-0.5, 5.75) {};
%		\node [style=nothing] (15) at (-0.5, 5.25) {};
%	\end{pgfonlayer}
%	\begin{pgfonlayer}{edgelayer}
%		\draw [style=simple] (8) to (12);
%		\draw [style=simple] (9) to (13);
%		\draw [style=simple] (10) to (11);
%		\draw [style=simple] (14) to (15);
%	\end{pgfonlayer}
%\end{tikzpicture}$
%							\end{tabular}
%							}
%						
%						\item 
%						\label{CNOT.5}
%						\hfil{
%							$
%							\begin{tikzpicture}
%	\begin{pgfonlayer}{nodelayer}
%		\node [style=nothing] (9) at (-1, 5.25) {};
%		\node [style=nothing] (10) at (-0.5, 5.25) {};
%		\node [style=nothing] (11) at (0, 5.25) {};
%		\node [style=dot] (12) at (-1, 6) {};
%		\node [style=oplus] (13) at (-0.5, 6) {};
%		\node [style=oplus] (14) at (-0.5, 6.5) {};
%		\node [style=dot] (15) at (0, 6.5) {};
%		\node [style=nothing] (16) at (-1, 7.25) {};
%		\node [style=nothing] (17) at (-0.5, 7.25) {};
%		\node [style=nothing] (18) at (0, 7.25) {};
%	\end{pgfonlayer}
%	\begin{pgfonlayer}{edgelayer}
%		\draw [style=simple] (9) to (16);
%		\draw [style=simple] (10) to (17);
%		\draw [style=simple] (11) to (18);
%		\draw [style=simple] (12) to (13);
%		\draw [style=simple] (14) to (15);
%	\end{pgfonlayer}
%\end{tikzpicture}
%							=
%							\begin{tikzpicture}
%	\begin{pgfonlayer}{nodelayer}
%		\node [style=nothing] (10) at (-1, 5.25) {};
%		\node [style=nothing] (11) at (-0.5, 5.25) {};
%		\node [style=nothing] (12) at (0, 5.25) {};
%		\node [style=dot] (13) at (-1, 6.5) {};
%		\node [style=oplus] (14) at (-0.5, 6.5) {};
%		\node [style=oplus] (15) at (-0.5, 6) {};
%		\node [style=dot] (16) at (0, 6) {};
%		\node [style=nothing] (17) at (-1, 7.25) {};
%		\node [style=nothing] (18) at (-0.5, 7.25) {};
%		\node [style=nothing] (19) at (0, 7.25) {};
%	\end{pgfonlayer}
%	\begin{pgfonlayer}{edgelayer}
%		\draw [style=simple] (10) to (17);
%		\draw [style=simple] (11) to (18);
%		\draw [style=simple] (12) to (19);
%		\draw [style=simple] (13) to (14);
%		\draw [style=simple] (15) to (16);
%	\end{pgfonlayer}
%\end{tikzpicture}
%							$}
%						
%						\item 
%						\label{CNOT.6}
%						\hfil{
%							$
%							\begin{tikzpicture}
%	\begin{pgfonlayer}{nodelayer}
%		\node [style=onein] (11) at (0, 5.25) {};
%		\node [style=oneout] (12) at (0, 6.25) {};
%	\end{pgfonlayer}
%	\begin{pgfonlayer}{edgelayer}
%		\draw [style=simple] (11) to (12);
%	\end{pgfonlayer}
%\end{tikzpicture}
%							=
%							\begin{tikzpicture}
%	\begin{pgfonlayer}{nodelayer}
%		\node [style=rn] (12) at (0, 5.25) {};
%		\node [style=rn] (13) at (0, 6.25) {};
%	\end{pgfonlayer}
%\end{tikzpicture}
%							$}
%						
%						\item 
%						\label{CNOT.7}
%						\hfil{
%							\begin{tabular}{c}
%							$\begin{tikzpicture}
%	\begin{pgfonlayer}{nodelayer}
%		\node [style=onein] (13) at (-1, 5.25) {};
%		\node [style=onein] (14) at (-0.5, 5.25) {};
%		\node [style=nothing] (15) at (0, 5.25) {};
%		\node [style=dot] (16) at (-1, 5.75) {};
%		\node [style=oplus] (17) at (-0.5, 5.75) {};
%		\node [style=dot] (18) at (-0.5, 6.25) {};
%		\node [style=oplus] (19) at (0, 6.25) {};
%		\node [style=oneout] (20) at (-1, 6.25) {};
%		\node [style=nothing] (21) at (-0.5, 6.75) {};
%		\node [style=nothing] (22) at (0, 6.75) {};
%	\end{pgfonlayer}
%	\begin{pgfonlayer}{edgelayer}
%		\draw [style=simple] (13) to (20);
%		\draw [style=simple] (14) to (21);
%		\draw [style=simple] (15) to (22);
%		\draw [style=simple] (16) to (17);
%		\draw [style=simple] (18) to (19);
%	\end{pgfonlayer}
%\end{tikzpicture}
%							=
%							\begin{tikzpicture}
%	\begin{pgfonlayer}{nodelayer}
%		\node [style=onein] (14) at (-1, 5.25) {};
%		\node [style=onein] (15) at (-0.5, 5.25) {};
%		\node [style=nothing] (16) at (0, 5.25) {};
%		\node [style=dot] (17) at (-1, 5.75) {};
%		\node [style=oplus] (18) at (-0.5, 5.75) {};
%		\node [style=oneout] (19) at (-1, 6.25) {};
%		\node [style=nothing] (20) at (-0.5, 6.75) {};
%		\node [style=nothing] (21) at (0, 6.75) {};
%	\end{pgfonlayer}
%	\begin{pgfonlayer}{edgelayer}
%		\draw [style=simple] (14) to (19);
%		\draw [style=simple] (15) to (20);
%		\draw [style=simple] (16) to (21);
%		\draw [style=simple] (17) to (18);
%	\end{pgfonlayer}
%\end{tikzpicture}$\\
%							$ $\\
%							$\begin{tikzpicture}
%	\begin{pgfonlayer}{nodelayer}
%		\node [style=oneout] (15) at (-1, 6.75) {};
%		\node [style=oneout] (16) at (-0.5, 6.75) {};
%		\node [style=nothing] (17) at (0, 6.75) {};
%		\node [style=dot] (18) at (-1, 6.25) {};
%		\node [style=oplus] (19) at (-0.5, 6.25) {};
%		\node [style=dot] (20) at (-0.5, 5.75) {};
%		\node [style=oplus] (21) at (0, 5.75) {};
%		\node [style=onein] (22) at (-1, 5.75) {};
%		\node [style=nothing] (23) at (-0.5, 5.25) {};
%		\node [style=nothing] (24) at (0, 5.25) {};
%	\end{pgfonlayer}
%	\begin{pgfonlayer}{edgelayer}
%		\draw [style=simple] (15) to (22);
%		\draw [style=simple] (16) to (23);
%		\draw [style=simple] (17) to (24);
%		\draw [style=simple] (18) to (19);
%		\draw [style=simple] (20) to (21);
%	\end{pgfonlayer}
%\end{tikzpicture}
%							=
%							\begin{tikzpicture}
%	\begin{pgfonlayer}{nodelayer}
%		\node [style=oneout] (16) at (-1, 6.75) {};
%		\node [style=oneout] (17) at (-0.5, 6.75) {};
%		\node [style=nothing] (18) at (0, 6.75) {};
%		\node [style=dot] (19) at (-1, 6.25) {};
%		\node [style=oplus] (20) at (-0.5, 6.25) {};
%		\node [style=onein] (21) at (-1, 5.75) {};
%		\node [style=nothing] (22) at (-0.5, 5.25) {};
%		\node [style=nothing] (23) at (0, 5.25) {};
%	\end{pgfonlayer}
%	\begin{pgfonlayer}{edgelayer}
%		\draw [style=simple] (16) to (21);
%		\draw [style=simple] (17) to (22);
%		\draw [style=simple] (18) to (23);
%		\draw [style=simple] (19) to (20);
%	\end{pgfonlayer}
%\end{tikzpicture}$
%							\end{tabular}
%							}
%						
%						\item 
%						\label{CNOT.8}
%						\hfil{
%							$
%							\begin{tikzpicture}
%	\begin{pgfonlayer}{nodelayer}
%		\node [style=nothing] (17) at (-1, 5.25) {};
%		\node [style=nothing] (18) at (-0.5, 5.25) {};
%		\node [style=nothing] (19) at (0, 5.25) {};
%		\node [style=dot] (20) at (-1, 5.75) {};
%		\node [style=oplus] (21) at (-0.5, 5.75) {};
%		\node [style=dot] (22) at (-0.5, 6.25) {};
%		\node [style=oplus] (23) at (0, 6.25) {};
%		\node [style=dot] (24) at (-1, 6.75) {};
%		\node [style=oplus] (25) at (-0.5, 6.75) {};
%		\node [style=nothing] (26) at (-1, 7.25) {};
%		\node [style=nothing] (27) at (-0.5, 7.25) {};
%		\node [style=nothing] (28) at (0, 7.25) {};
%	\end{pgfonlayer}
%	\begin{pgfonlayer}{edgelayer}
%		\draw [style=simple] (17) to (26);
%		\draw [style=simple] (18) to (27);
%		\draw [style=simple] (19) to (28);
%		\draw [style=simple] (20) to (21);
%		\draw [style=simple] (22) to (23);
%		\draw [style=simple] (24) to (25);
%	\end{pgfonlayer}
%\end{tikzpicture}
%							=
%							\begin{tikzpicture}
%	\begin{pgfonlayer}{nodelayer}
%		\node [style=nothing] (18) at (-1, 5.25) {};
%		\node [style=nothing] (19) at (-0.5, 5.25) {};
%		\node [style=nothing] (20) at (0, 5.25) {};
%		\node [style=dot] (21) at (-0.5, 5.75) {};
%		\node [style=oplus] (22) at (0, 5.75) {};
%		\node [style=dot] (23) at (-1, 6.25) {};
%		\node [style=oplus] (24) at (0, 6.25) {};
%		\node [style=nothing] (25) at (-1, 6.75) {};
%		\node [style=nothing] (26) at (-0.5, 6.75) {};
%		\node [style=nothing] (27) at (0, 6.75) {};
%	\end{pgfonlayer}
%	\begin{pgfonlayer}{edgelayer}
%		\draw [style=simple] (18) to (25);
%		\draw [style=simple] (19) to (26);
%		\draw [style=simple] (20) to (27);
%		\draw [style=simple] (21) to (22);
%		\draw [style=simple] (23) to (24);
%	\end{pgfonlayer}
%\end{tikzpicture}
%							$}
%						
%						\item 
%						\label{CNOT.9}
%						\hfil{
%							$
%							\begin{tikzpicture}
%	\begin{pgfonlayer}{nodelayer}
%		\node [style=onein] (19) at (-1, 5.25) {};
%		\node [style=onein] (20) at (-0.5, 5.25) {};
%		\node [style=nothing] (21) at (0, 5.25) {};
%		\node [style=dot] (22) at (-1, 5.75) {};
%		\node [style=oplus] (23) at (-0.5, 5.75) {};
%		\node [style=oneout] (24) at (-1, 6.25) {};
%		\node [style=oneout] (25) at (-0.5, 6.25) {};
%		\node [style=nothing] (26) at (0, 6.25) {};
%	\end{pgfonlayer}
%	\begin{pgfonlayer}{edgelayer}
%		\draw [style=simple] (19) to (24);
%		\draw [style=simple] (20) to (25);
%		\draw [style=simple] (21) to (26);
%		\draw [style=simple] (22) to (23);
%	\end{pgfonlayer}
%\end{tikzpicture}
%							=
%							\begin{tikzpicture}
%	\begin{pgfonlayer}{nodelayer}
%		\node [style=onein] (20) at (-1, 5.25) {};
%		\node [style=onein] (21) at (-0.5, 5.25) {};
%		\node [style=nothing] (22) at (0, 5.25) {};
%		\node [style=dot] (23) at (-1, 6.25) {};
%		\node [style=oplus] (24) at (-0.5, 6.25) {};
%		\node [style=oneout] (25) at (-1, 7.25) {};
%		\node [style=oneout] (26) at (-0.5, 7.25) {};
%		\node [style=nothing] (27) at (0, 7.25) {};
%		\node [style=oneout] (28) at (0, 6) {};
%		\node [style=onein] (29) at (0, 6.5) {};
%	\end{pgfonlayer}
%	\begin{pgfonlayer}{edgelayer}
%		\draw [style=simple] (20) to (25);
%		\draw [style=simple] (21) to (26);
%		\draw [style=simple] (22) to (28);
%		\draw [style=simple] (29) to (27);
%		\draw [style=simple] (23) to (24);
%	\end{pgfonlayer}
%\end{tikzpicture}
%							$}
%					\end{enumerate}
%				\end{multicols}
%				\
%			\end{mdframed}
%	}}
%	\caption{The identities of \texorpdfstring{$\CNOT$}{CNOT}}
%	\label{fig:CNOT}
%\end{figure}

\section{Discussion}
A natural next question would be to prove completeness for the natural number fragments of the qudit ZH-calculus/full subcategories $d^n$ of $\Mat_\N$.  There is a universal presentation of the qudit ZH-calculus \cite{roy}; however, completeness has not been proven.  Perhaps this would be a first step towards proving completeness of the qudit ZH-calculus.  The difficulty with generalizing our work in that finding normal forms for systems of Boolean equations is particularly easy.  For example, the rule \ref{ZXA.13} allows us to make the following deduction about Boolean formulae:
$$
\dfrac{P(X)\cdot P(Y)=1}{P(x)=1,\hspace*{1cm}Q(y)=1}
$$
So that an $n$ variable system of Boolean equations is equivalent to a single $n$-variable Boolean equation; which can be represented as a linear subspace over $\F_2^n$. Therefore \ref{ZXA.13} allows us to  reduce Boolean formulae to algebraic normal via  Gaussian elimination.
 
In the qudit setting, the $n$-variable $d$-valued formulae have the structure of elements of the ring of polynomial functions $\Z/d\Z[x_1,\cdots, x_{n}]/\langle x_1^d-x_1,\cdots, x_n^d-x_n\rangle$; finding normal forms for the induced algebraic varieties is much trickier. 
However systems of polynomial equations over fields admit normal forms called Gr\"obner bases.  Therefore,  when the dimension $d$ is a prime $p$, one could potentially use  Gr\"obner bases to find normal forms for these systems of polynomial equations over fields; rewriting circuits to    Gr\"obner bases  graphically modulo the  ideals $\langle x_1^d-x_1,\cdots, x_n^d-x_n\rangle$.  
