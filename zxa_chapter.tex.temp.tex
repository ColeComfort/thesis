
\section{Introduction}

In this paper a complete set of identities is provided for the fragment, $\ZXA$, of the $\ZX$-calculus, generated by black and white spiders, the not gate and the {\sf and} gate. We show that this is a universal and complete presentation of ``qubit multirelations,'' or equivalently $2^n \times 2^m$ dimensional matrices over $\N$.
 To prove completeness and universality requires much exposition.  Along the way we show that the category of classical channels of a discrete inverse category is the Cartesian completion of that discrete inverse category.  We then show that the corresponding environment structure is precisely the free counit completion of the chosen Frobenius structure.  This allows us to present the Cartesian completion of, $\TOF$, the category generated by the Toffoli gate, $|1\rangle$ and $\langle 1|$ by only adding the $|+\rangle$ state and the unitality equation.  By freely adding both the unit and counit to $\TOF$, corresponding to $\sqrt{2}|+\rangle$ and  $\sqrt{2}\langle +|$, this yields an isomorphism with spans between ordinals $2^n$, $n\in \N$, or equivalently, ``qubit multirelations.''

The identities which are given by this two way translation are {\em almost} the union of the complete identities for Boolean functions \cite[Thm. 10]{lafont} (functions of type $\F_2^n \to \F_2$) and the identities for $\Span^\sim(\Mat(\F_2))$ \cite[Def. 5.1]{ihpub}.  These classes of circuits, and these identities for that matter, are nothing new; however, we provide a completeness result, as well as a structural account of how the full classical qubit fragment of $\FHilb$ can be obtained from adding discarding and codiscarding to the full classically reversible Boolean fragment.  In fact, some of these identities are presented in \cite[Chap. 5]{herrmann},  and they are used in the $\ZH$-calculus \cite{zh,zhpi}, as well as in some presentations of the $\ZX$-calculus with the triangle generator as a primitive \cite{munson2019note,ringZX}.  This is particularity unsurprising for the latter, \cite{ringZX}, where the author proves completeness of the $\ZX$-calculus over arbitrary semirings, which subsumes the completeness result herein.  Albeit, the presentation given here is substantially simpler.  It worth mentioning that $\ZXA$ is not a $\ZX*$-calculus in the sense of \cite{zxstar}, because the {\sf and} gate is not a spider.  $\ZXA$ should be instead though of as the ``classical fragment'' of the phase-free $\ZH$-calculus: retaining the monoid for ``and'' without $H$-boxes.   From this presentation only natural-number H-boxes can be derived.

We assume familiarity with the theory of  monoidal categories and categorical quantum mechanics.
Most of the paper will be devoted to reviewing the required categorical machinery of restriction and inverse categories, and developing it further, in order to prove the main result.  With all of mathematics reviewed and developed in generality, the desired result follows from abstract nonsense after a mechanical calculation. 

In Section \ref{sec:rest}, the theory of restriction categories and inverse categories is reviewed.  In Section \ref{sec:cpm}, we construct classical channels in the setting of discrete inverse categories, showing that the ``environment structures'' of the classical channels corresponds to adding counits to the base discrete inverse category.  Finally, in Section \ref{sec:ZXA}, we actually compute the (co)unit completion of $\TOF$.  We show that this category has a much more canonical presentation, $\ZXA$, in terms of interacting monoids/comonoids which very much resembles the $\ZH$-calculus.  We also show that this category is isomorphic to the category spans between ordinals $2^n$.
\section{Restriction and Inverse Categories}

\label{sec:rest}

Restriction and inverse categories provide a categorical semantics for partial computing and reversible computing, respectively.  We review how weakened products can be constructed in both settings; relating one to the other.

\begin{definition}\cite[\S 2.1.1]{cockett}
A {\bf restriction category} is a category along with a restriction operator:

\hfil
$
(A \xrightarrow{f} B )\mapsto (A \xrightarrow{\bar f} A)
$\\
such that:\footnote{Using diagrammatic composition.}


\begin{center}
%\begin{mdframed}
\begin{multicols}{4}
\begin{enumerate}[label={\bf [R.\arabic*]}, ref={\bf [R.\arabic*]}]
\item $\bar f f  = f$
\label{R.1}
\item $\bar f \bar g = \bar g \bar f$
\label{R.2}
\item $\bar f \bar g = \bar{\bar f g}$
\label{R.3}
\item $f \bar g = \bar{fg} f$
\label{R.4}
\end{enumerate}
\end{multicols}
%\end{mdframed}
\end{center}


Maps of the form $\bar f$ are called restriction idempotents.
The canonical example of a restriction category is $\Par$, sets and partial maps.  The restriction in this case, just restricts partial functions to their domain of definition.


Restriction categories have a partial order on homsets given by $f \leq g \iff \bar f g = f$.


A map $f$ in a restriction category is called a {\bf partial isomorphism}, in case there exists a map $g$ called the partial inverse of $f$ so that $fg=\bar f$ and $gf = \bar g$.  Similarly, a map $f$ in a restriction category is {\bf total} if $\bar f =1$.  Denote the subcategories of partial isomorphisms and total maps of a restriction category $\X$, respectively by $\ParIso(\X)$ and $\Total(\X)$.



%A {\bf split restriction category} is a restriction category in which all restriction idempotents split.
\end{definition}



\begin{example} \cite[p. 101]{pcat} \cite[\S 5]{restiii}
A {\bf counital copy category} (or a p-category with a one element object) is a monoidal category with a family of commutative comonoids on every object compatible with the monoidal structure, with a natural comultiplication.  This gives a restriction via copying and then discarding:
$$
\begin{tikzpicture}
	\begin{pgfonlayer}{nodelayer}
		\node [style=none] (0) at (0.75, -2.5) {};
		\node [style=none] (1) at (0.75, -0.5) {};
		\node [style=map] (2) at (0.75, -1.5) {$\bar f$};
	\end{pgfonlayer}
	\begin{pgfonlayer}{edgelayer}
		\draw [style=simple] (0.center) to (2);
		\draw [style=simple] (2) to (1.center);
	\end{pgfonlayer}
\end{tikzpicture}
:=
\begin{tikzpicture}
	\begin{pgfonlayer}{nodelayer}
		\node [style=map] (0) at (0, 2.5) {$f$};
		\node [style=X] (1) at (0, 3.5) {};
		\node [style=X] (2) at (0.5, 1.5) {};
		\node [style=none] (3) at (1, 3.5) {};
		\node [style=none] (4) at (0.5, 0.5) {};
	\end{pgfonlayer}
	\begin{pgfonlayer}{edgelayer}
		\draw [style=simple] (1) to (0);
		\draw [style=simple, in=117, out=-90] (0) to (2);
		\draw [style=simple] (2) to (4.center);
		\draw [style=simple, in=-90, out=60] (2) to (3.center);
	\end{pgfonlayer}
\end{tikzpicture}
$$
\end{example}


\begin{definition}\cite[\S 3.1]{cockett}
A {\bf stable system of monics} $\M$ of $\X$ is a collection of monics in $\X$ containing all isomorphisms; where for any cospan $ X\xrightarrow{f} Z \xleftarrowtail{m} Y$  in $\X$, where $m'$ is in $\M$, the following pullback exists:

%\hfil$
%\xymatrixrowsep{.005in}
%\xymatrixcolsep{.13in}
%  \xymatrix{
%    W \ar[r]^{f'} \ar@{>->}[d]_{m'} & Y  \ar@{>->}[d]^m \\
%    X \ar[r]_{f} & Z
%  }
%$\\

\hfil$
\xymatrixrowsep{.005in}
\xymatrixcolsep{.13in}
  \xymatrix{
  	& W \ar@{>->}[dl]_{m'} \ar[dr]^{f'}\\
  	X \ar[dr]_f &  & Y \ar@{>->}[dl]^m\\
  	& Z
  }
$

Where $m'$ is in $\M$.

\end{definition}

Stable systems of monics allow one to represent the domains of definition of a partial functions as a subobjects:

\begin{definition}\cite[\S 3.1]{cockett}
Given a stable system of monics $\M$ in a category $\X$, the {\bf partial map category} $\Par(\X,\M)$ is given by the same objects as in $\X$ where morphisms $X\to Y$, given by isomorphism classes of spans $X\xleftarrowtail{m} Z \xrightarrow{f} Y$ where $f$ is a map in $\X$ and $m$ is a map in $\M$.  Composition is given by pullback and the identity is given by the trivial span.


Partial map categories have a restriction structure given by:  $(X\xleftarrowtail{m} Z \xrightarrow{f} Y) \mapsto (X\xleftarrowtail{m} Z \xrightarrowtail{m} X)$.  Moreover, a partial isomorphism is a span $X\xleftarrowtail{e} Z \xrightarrowtail{m} Y$ where $e,m \in \M$; the partial inverse given by  $Y\xleftarrowtail{m} Z \xrightarrowtail{e} X$.
\end{definition}


$\Par$ is equivalently the partial map category $\Par(\Sets,\M)$ where $\M$ is all monics in $\Sets$.




%\begin{lemma} \cite[Prop. 3.1]{cockett}
%Partial map categories are split restriction categories.
%\end{lemma}


%If restriction idempotents split then X is a cartesian restriction category if and only if Tot(X) is a cartesian category





Let $\Span^\sim(\X)$ denote the category given by isomorphism classes of spans over $\X$. Given a stable system of monics $\M$ over $\X$, if $\X$ is finitely complete, then $\Span^\sim(\X)$ exists, and thus, there is a faithful functor $\Par(\X,\M)\to \Span^\sim(\X)$.


\begin{definition}\cite[\S 2.3.2]{cockett}
An {\bf inverse category} is a restriction category in which all maps are partial isomorphisms.  The subcategory of partial isomorphisms of $\Par$ is called $\Pinj$.
\end{definition}

Inverse categories can be presented with a dagger functor taking maps to their partial inverses:

\begin{theorem}\cite[Thm. 2.20]{cockett}
A restriction category $\X$ is an inverse category if and only if there is a dagger functor $(\_)^\circ:\X^\op\to\X$ such that for all $X\xleftarrow{f} Z \xrightarrow{g} Y$:
\begin{center}
\begin{tabular}{cc}
 $f f^\circ f = f$ & 
 $f f ^\circ gg^\circ = gg^\circ f f ^\circ $
\end{tabular}
\end{center}
\end{theorem}

Since restriction categories  and inverse categories give a categorical semantics for partial computing  and reversible computing, respectively, it is natural to ask when these categories have copying.


In the case of restriction categories, one must weaken the notion of the product to lax products using the partial order enrichment:


\begin{definition}\cite{restiii}
A restriction category has {\bf binary restriction products}, when for all objects  $X,Y$, there exists an object $X\times Y$ and total maps $X \xleftarrow{\pi_0}  X\times Y \xrightarrow{\pi_1} Y$, so that for all objects $Z$ and all maps $X \xleftarrow{f} Z \xrightarrow{g} Y$, the following diagram commutes there exists a unique $Z\xrightarrow{\langle f,g \rangle} X\times Y$ making the diagram commute:
\hfil
$
\xymatrixrowsep{0.2cm}
\xymatrixcolsep{0.4cm}
\xymatrix{
&& Z\ar@{..>}[dd]|-{\langle f, g\rangle} \ar@/_/[ddll]_f \ar@/^/[ddrr]^g &&\\
& \ar@{}[dr]|-{\geq} && \ar@{}[dl] |-{\leq} &\\
X &&  X\times Y \ar[rr]_{\pi_1} \ar[ll]^{\pi_0}  && Y
}
$

so that $\bar{\langle f, g\rangle \pi_0} f = \langle f, g\rangle \pi_0$ and $\bar{\langle f, g\rangle \pi_1} g = \langle f, g\rangle \pi_1$;
where additionally $\bar{\langle f, g\rangle} =  \bar f \bar g$.

%%DRAW DIAGRAM
%\begin{center}
%\begin{tabular}{ccc}
%  $\langle f, g\rangle \pi_0 \leq f$ &
%  $\langle f, g\rangle \pi_1 \leq g$ &
%  $\bar{\langle f, g\rangle} =  \bar f \bar g$
%\end{tabular}
%\end{center}

A restriction category has a {\bf restriction terminal object} $\top$ when for all objects $X$, there exists a unique total map $!_X:X\to\top$ such that $f !_Y = \bar f !_X$.

A restriction category with a restriction terminal object and binary restriction products is a {\bf Cartesian restriction category}.


An object $A$ in a restriction category with restriction products is {\bf discrete} when the diagonal map $\Delta_X:=\langle 1_X, 1_X\rangle$ is a partial isomorphism. A restriction category is discrete when all objects are discrete.  Discrete Cartesian restriction categories are said to have restriction products.
\end{definition}




\begin{theorem}\cite[Thm. 5.2]{restiii}
The structure of a  counital copy category structure is precisely that of a Cartesian restriction category.
\end{theorem}



\begin{proposition} \cite[\S 5.1]{restiii}
\label{prop:cartesian}

If $\X$ is a discrete Cartesian restriction category, then $\Total(\X)$ is Cartesian.
\end{proposition}



$\Par$ is a canonical example of a discrete Cartesian restriction category; the restriction product is given by the Cartesian product on underlying sets and the terminal object is  the singleton set.




The weakened notion of products in restriction categories is not satisfying for inverse categories because it does not impose enough equations governing the interaction between the diagonal map and its partial inverse.

\begin{definition}\cite[Def. 4.3.1]{giles}
A symmetric monoidal inverse category $\X$ is a {\bf discrete inverse category} when there is a natural, special commutative $\dag$-semi-Frobenius algebra\footnote{The ``semi'' adjective on Frobenius just means that the a semigroup and cosemigroup are interacting instead of a monoid and comonoid.} on every object (where the (co)multiplications are drawn as white bubbles)  compatible with the tensor product:

$$
\begin{tikzpicture}
	\begin{pgfonlayer}{nodelayer}
		\node [style=none] (0) at (0, 2.5) {};
		\node [style=none] (1) at (1, 2.5) {};
		\node [style=X] (2) at (0.5, 1.5) {};
		\node [style=none] (3) at (0.5, 0.5) {};
	\end{pgfonlayer}
	\begin{pgfonlayer}{edgelayer}
		\draw [style=simple] (3.center) to (2);
		\draw [style=simple, in=-90, out=117] (2) to (0.center);
		\draw [style=simple, in=63, out=-90] (1.center) to (2);
	\end{pgfonlayer}
\end{tikzpicture}
=
\begin{tikzpicture}
	\begin{pgfonlayer}{nodelayer}
		\node [style=X] (0) at (0, 2.5) {};
		\node [style=X] (1) at (1, 2.5) {};
		\node [style=none] (2) at (0.5, 1.5) {};
		\node [style=none] (3) at (0.5, 0.5) {};
		\node [style=none] (4) at (0, 3.5) {};
		\node [style=none] (5) at (1, 3.5) {};
		\node [style=none] (6) at (0, 4.5) {};
		\node [style=none] (7) at (1, 4.5) {};
		\node [style=otimes] (8) at (0.5, 1.5) {};
		\node [style=otimes] (9) at (1, 3.5) {};
		\node [style=otimes] (10) at (0, 3.5) {};
	\end{pgfonlayer}
	\begin{pgfonlayer}{edgelayer}
		\draw [style=simple] (3.center) to (2.center);
		\draw [style=simple, in=-90, out=135] (2.center) to (0);
		\draw [style=simple] (0) to (5.center);
		\draw [style=simple, in=120, out=-120, looseness=1.25] (4.center) to (0);
		\draw [style=simple, in=-60, out=60, looseness=1.25] (1) to (5.center);
		\draw [style=simple] (1) to (4.center);
		\draw [style=simple, in=45, out=-90] (1) to (2.center);
		\draw [style=simple] (4.center) to (6.center);
		\draw [style=simple] (5.center) to (7.center);
	\end{pgfonlayer}
\end{tikzpicture}
\hspace*{1cm}
\begin{tikzpicture}
	\begin{pgfonlayer}{nodelayer}
		\node [style=X] (0) at (0, 1.5) {};
		\node [style=none] (1) at (-0.5, 2.5) {};
		\node [style=none] (2) at (0.5, 2.5) {};
		\node [style=none] (3) at (0, 0.5) {};
	\end{pgfonlayer}
	\begin{pgfonlayer}{edgelayer}
		\draw [style=dashed] (3.center) to (0);
		\draw [style=dashed, in=-90, out=117] (0) to (1.center);
		\draw [style=dashed, in=63, out=-90] (2.center) to (0);
	\end{pgfonlayer}
\end{tikzpicture}
=
\begin{tikzpicture}
	\begin{pgfonlayer}{nodelayer}
		\node [style=none] (0) at (0, 1.5) {};
		\node [style=none] (1) at (-0.5, 2.5) {};
		\node [style=none] (2) at (0.5, 2.5) {};
		\node [style=none] (3) at (0, 0.5) {};
		\node [style=otimes] (4) at (0, 1.5) {};
	\end{pgfonlayer}
	\begin{pgfonlayer}{edgelayer}
		\draw [style=dashed] (3.center) to (0.center);
		\draw [style=dashed, in=-90, out=117] (0.center) to (1.center);
		\draw [style=dashed, in=63, out=-90] (2.center) to (0.center);
	\end{pgfonlayer}
\end{tikzpicture}
$$


Where the tensor product is also required to preserve restriction in both components.
\end{definition}

In a discrete inverse category, restriction idempotents are prephases for the Frobenius algebra, so that:
$$
\begin{tikzpicture}
	\begin{pgfonlayer}{nodelayer}
		\node [style=X] (0) at (3, 1.75) {};
		\node [style=map] (1) at (3, 1) {$\bar f$};
		\node [style=none] (2) at (3, 0.5) {};
		\node [style=none] (3) at (2.5, 2.5) {};
		\node [style=none] (4) at (3.5, 2.5) {};
	\end{pgfonlayer}
	\begin{pgfonlayer}{edgelayer}
		\draw [style=simple, in=63, out=-90] (4.center) to (0);
		\draw [style=simple, in=-90, out=117] (0) to (3.center);
		\draw [style=simple] (1) to (0);
		\draw [style=simple] (1) to (2.center);
	\end{pgfonlayer}
\end{tikzpicture}
=
\begin{tikzpicture}
	\begin{pgfonlayer}{nodelayer}
		\node [style=X] (0) at (3, 2) {};
		\node [style=none] (1) at (3, 1.5) {};
		\node [style=none] (2) at (2.5, 3) {};
		\node [style=none] (3) at (3.5, 3) {};
		\node [style=map] (4) at (2.5, 3) {$\bar f$};
		\node [style=none] (5) at (3.5, 3.5) {};
		\node [style=none] (6) at (2.5, 3.5) {};
	\end{pgfonlayer}
	\begin{pgfonlayer}{edgelayer}
		\draw [style=simple, in=63, out=-90] (3.center) to (0);
		\draw [style=simple, in=-90, out=117] (0) to (2.center);
		\draw [style=simple] (6.center) to (2.center);
		\draw [style=simple] (5.center) to (3.center);
		\draw [style=simple] (0) to (1.center);
	\end{pgfonlayer}
\end{tikzpicture}
=
\begin{tikzpicture}
	\begin{pgfonlayer}{nodelayer}
		\node [style=X] (0) at (3, 2) {};
		\node [style=none] (1) at (3, 1.5) {};
		\node [style=none] (2) at (3.5, 3) {};
		\node [style=none] (3) at (2.5, 3) {};
		\node [style=map] (4) at (3.5, 3) {$\bar f$};
		\node [style=none] (5) at (2.5, 3.5) {};
		\node [style=none] (6) at (3.5, 3.5) {};
	\end{pgfonlayer}
	\begin{pgfonlayer}{edgelayer}
		\draw [style=simple, in=117, out=-90] (3.center) to (0);
		\draw [style=simple, in=-90, out=63] (0) to (2.center);
		\draw [style=simple] (6.center) to (2.center);
		\draw [style=simple] (5.center) to (3.center);
		\draw [style=simple] (0) to (1.center);
	\end{pgfonlayer}
\end{tikzpicture}
\hspace*{.6cm}
\begin{tikzpicture}
	\begin{pgfonlayer}{nodelayer}
		\node [style=X] (0) at (3, 3) {};
		\node [style=none] (1) at (3, 3.5) {};
		\node [style=none] (2) at (3.5, 2) {};
		\node [style=none] (3) at (2.5, 2) {};
		\node [style=map] (4) at (3.5, 2) {$\bar f$};
		\node [style=none] (5) at (2.5, 1.5) {};
		\node [style=none] (6) at (3.5, 1.5) {};
	\end{pgfonlayer}
	\begin{pgfonlayer}{edgelayer}
		\draw [style=simple, in=-117, out=90] (3.center) to (0);
		\draw [style=simple, in=90, out=-63] (0) to (2.center);
		\draw [style=simple] (6.center) to (2.center);
		\draw [style=simple] (5.center) to (3.center);
		\draw [style=simple] (0) to (1.center);
	\end{pgfonlayer}
\end{tikzpicture}
=
\begin{tikzpicture}
	\begin{pgfonlayer}{nodelayer}
		\node [style=X] (0) at (3, 3) {};
		\node [style=none] (1) at (3, 3.5) {};
		\node [style=none] (2) at (2.5, 2) {};
		\node [style=none] (3) at (3.5, 2) {};
		\node [style=map] (4) at (2.5, 2) {$\bar f$};
		\node [style=none] (5) at (3.5, 1.5) {};
		\node [style=none] (6) at (2.5, 1.5) {};
	\end{pgfonlayer}
	\begin{pgfonlayer}{edgelayer}
		\draw [style=simple, in=-63, out=90] (3.center) to (0);
		\draw [style=simple, in=90, out=-117] (0) to (2.center);
		\draw [style=simple] (6.center) to (2.center);
		\draw [style=simple] (5.center) to (3.center);
		\draw [style=simple] (0) to (1.center);
	\end{pgfonlayer}
\end{tikzpicture}
=
\begin{tikzpicture}
	\begin{pgfonlayer}{nodelayer}
		\node [style=X] (0) at (3, 1.25) {};
		\node [style=map] (1) at (3, 2) {$\bar f$};
		\node [style=none] (2) at (3, 2.5) {};
		\node [style=none] (3) at (2.5, 0.5) {};
		\node [style=none] (4) at (3.5, 0.5) {};
	\end{pgfonlayer}
	\begin{pgfonlayer}{edgelayer}
		\draw [style=simple, in=-63, out=90] (4.center) to (0);
		\draw [style=simple, in=90, out=-117] (0) to (3.center);
		\draw [style=simple] (1) to (0);
		\draw [style=simple] (1) to (2.center);
	\end{pgfonlayer}
\end{tikzpicture}
$$


Discrete inverse categories are the ``right'' notion of weakened products for monoidal inverse categories:

\begin{theorem}\cite[Thm. 5.2.6]{giles}
There is an equivalence of categories between the category of discrete inverse categories and the category of discrete Cartesian categories.
\end{theorem}

To go from  discrete Cartesian restriction categories to discrete inverse categories, one takes the subcategory of partial isomorphisms.
The other direction is less trivial; in particular, this involves adding a restriction terminal object via the following construction which ``adds a history'' to a partial isomorphism:

\begin{definition}\cite[Def. 5.1.1]{giles}
Given a discrete inverse category $\X$, define its {\bf Cartesian completion} $\tilde \X$ as the category with:

\begin{description}
\item[Objects:] The same objects as $\X$.
\item[Maps:]
\hfil
$
\dfrac{ X\xrightarrow{f} Y \otimes S \in \X}{ X\xrightarrow{(f,S)} Y \in \tilde \X}
$



Where two parallel maps $X\xrightarrow{(f,S), (g,T)} Y $ are equivalent when either (both conditions are equivalent):
$$
\begin{tikzpicture}
	\begin{pgfonlayer}{nodelayer}
		\node [style=map] (0) at (0, 1.5) {$f$};
		\node [style=none] (1) at (0, 0.5) {};
		\node [style=map] (2) at (0, 3) {$f^\circ$};
		\node [style=map] (3) at (0, 4) {$g$};
		\node [style=X] (4) at (-0.5, 2.25) {};
		\node [style=X] (5) at (-0.5, 5) {};
		\node [style=none] (6) at (-0.5, 6) {};
		\node [style=none] (7) at (0.25, 6) {};
	\end{pgfonlayer}
	\begin{pgfonlayer}{edgelayer}
		\draw (0) to (1.center);
		\draw [in=75, out=-90] (7.center) to (3);
		\draw (6.center) to (5);
		\draw [in=120, out=-120] (5) to (4);
		\draw (4) to (2);
		\draw [in=60, out=-60, looseness=1.25] (2) to (0);
		\draw (0) to (4);
		\draw (3) to (2);
		\draw (3) to (5);
	\end{pgfonlayer}
\end{tikzpicture}
=
\begin{tikzpicture}
	\begin{pgfonlayer}{nodelayer}
		\node [style=map] (0) at (0, 1.5) {$g$};
		\node [style=none] (1) at (-0.5, 2.5) {};
		\node [style=none] (2) at (0.5, 2.5) {};
		\node [style=none] (3) at (0, 0.5) {};
	\end{pgfonlayer}
	\begin{pgfonlayer}{edgelayer}
		\draw [in=117, out=-90] (1.center) to (0);
		\draw [in=-90, out=63] (0) to (2.center);
		\draw (0) to (3.center);
	\end{pgfonlayer}
\end{tikzpicture}
\hspace*{.3cm}
or
\hspace*{.3cm}
\begin{tikzpicture}
	\begin{pgfonlayer}{nodelayer}
		\node [style=map] (0) at (0, 1.5) {$g$};
		\node [style=none] (1) at (0, 0.5) {};
		\node [style=map] (2) at (0, 3) {$g^\circ$};
		\node [style=map] (3) at (0, 4) {$f$};
		\node [style=X] (4) at (-0.5, 2.25) {};
		\node [style=X] (5) at (-0.5, 5) {};
		\node [style=none] (6) at (-0.5, 6) {};
		\node [style=none] (7) at (0.25, 6) {};
	\end{pgfonlayer}
	\begin{pgfonlayer}{edgelayer}
		\draw (0) to (1.center);
		\draw [in=75, out=-90] (7.center) to (3);
		\draw (6.center) to (5);
		\draw [in=120, out=-120] (5) to (4);
		\draw (4) to (2);
		\draw [in=60, out=-60, looseness=1.25] (2) to (0);
		\draw (0) to (4);
		\draw (3) to (2);
		\draw (3) to (5);
	\end{pgfonlayer}
\end{tikzpicture}
=
\begin{tikzpicture}
	\begin{pgfonlayer}{nodelayer}
		\node [style=map] (0) at (0, 1.5) {$f$};
		\node [style=none] (1) at (-0.5, 2.5) {};
		\node [style=none] (2) at (0.5, 2.5) {};
		\node [style=none] (3) at (0, 0.5) {};
	\end{pgfonlayer}
	\begin{pgfonlayer}{edgelayer}
		\draw [in=117, out=-90] (1.center) to (0);
		\draw [in=-90, out=63] (0) to (2.center);
		\draw (0) to (3.center);
	\end{pgfonlayer}
\end{tikzpicture}
$$

\item[Composition:]
\hfil
$
\begin{tikzpicture}
	\begin{pgfonlayer}{nodelayer}
		\node [style=map] (0) at (0, 1.5) {$f$};
		\node [style=none] (1) at (-0.5, 2.5) {};
		\node [style=none] (2) at (0.5, 2.5) {};
		\node [style=none] (3) at (0, 0.5) {};
	\end{pgfonlayer}
	\begin{pgfonlayer}{edgelayer}
		\draw [in=117, out=-90] (1.center) to (0);
		\draw [in=-90, out=63] (0) to (2.center);
		\draw (0) to (3.center);
	\end{pgfonlayer}
\end{tikzpicture}
;
\begin{tikzpicture}
	\begin{pgfonlayer}{nodelayer}
		\node [style=map] (0) at (0, 1.5) {$g$};
		\node [style=none] (1) at (-0.5, 2.5) {};
		\node [style=none] (2) at (0.5, 2.5) {};
		\node [style=none] (3) at (0, 0.5) {};
	\end{pgfonlayer}
	\begin{pgfonlayer}{edgelayer}
		\draw [in=117, out=-90] (1.center) to (0);
		\draw [in=-90, out=63] (0) to (2.center);
		\draw (0) to (3.center);
	\end{pgfonlayer}
\end{tikzpicture}
:=
\begin{tikzpicture}
	\begin{pgfonlayer}{nodelayer}
		\node [style=map] (0) at (0, 1.5) {$f$};
		\node [style=none] (1) at (0.5, 2.5) {};
		\node [style=none] (2) at (0, 0.5) {};
		\node [style=map] (3) at (-0.5, 2.5) {$g$};
		\node [style=none] (4) at (-1, 3.5) {};
		\node [style=otimes] (5) at (0, 3.5) {};
		\node [style=none] (6) at (-0.5, 2.5) {};
		\node [style=none] (7) at (-1, 4.5) {};
		\node [style=none] (8) at (0, 4.5) {};
	\end{pgfonlayer}
	\begin{pgfonlayer}{edgelayer}
		\draw [in=-90, out=63] (0) to (1.center);
		\draw (0) to (2.center);
		\draw [in=117, out=-90] (4.center) to (3);
		\draw (3) to (5);
		\draw [in=117, out=-90] (6.center) to (0);
		\draw [in=-63, out=90] (1.center) to (5);
		\draw (5) to (8.center);
		\draw (4.center) to (7.center);
	\end{pgfonlayer}
\end{tikzpicture}
$



\item[Identity:]
\hfil
$
\begin{tikzpicture}
	\begin{pgfonlayer}{nodelayer}
		\node [style=none] (0) at (0, 0.5) {};
		\node [style=none] (1) at (0, 2) {};
		\node [style=none] (2) at (0.75, 2) {};
		\node [style=none] (3) at (0, 1.25) {};
	\end{pgfonlayer}
	\begin{pgfonlayer}{edgelayer}
		\draw [style=dashed, in=-90, out=15] (3.center) to (2.center);
		\draw [style=simple] (0.center) to (1.center);
	\end{pgfonlayer}
\end{tikzpicture}
$

\item[Restriction: ] %Restriction is given by the underlying restriction of $\X$, so that:
\hfil
$
\bar{\left(
\begin{tikzpicture}
	\begin{pgfonlayer}{nodelayer}
		\node [style=map] (0) at (0, 1.5) {$f$};
		\node [style=none] (1) at (0, 0.5) {};
		\node [style=none] (2) at (-0.5, 2.5) {};
		\node [style=none] (3) at (0.5, 2.5) {};
	\end{pgfonlayer}
	\begin{pgfonlayer}{edgelayer}
		\draw [style=simple] (1.center) to (0);
		\draw [style=simple, in=117, out=-90] (2.center) to (0);
		\draw [style=simple, in=63, out=-90] (3.center) to (0);
	\end{pgfonlayer}
\end{tikzpicture}
\right)}
:=
\begin{tikzpicture}
	\begin{pgfonlayer}{nodelayer}
		\node [style=map] (0) at (0, 1.5) {$\bar f$};
		\node [style=none] (1) at (0, 0.5) {};
		\node [style=none] (2) at (0, 2.5) {};
		\node [style=none] (3) at (0, 2) {};
		\node [style=none] (4) at (0.5, 2.5) {};
	\end{pgfonlayer}
	\begin{pgfonlayer}{edgelayer}
		\draw [style=simple] (1.center) to (0);
		\draw [style=simple] (2.center) to (0);
		\draw [style=dashed, in=-90, out=15] (3.center) to (4.center);
	\end{pgfonlayer}
\end{tikzpicture}
$

\item[Restriction product:]
\hfil
$
\langle f,g \rangle:=
\begin{tikzpicture}
	\begin{pgfonlayer}{nodelayer}
		\node [style=map] (0) at (-0.25, 2.5) {$f$};
		\node [style=none] (1) at (-0.25, 3.5) {};
		\node [style=none] (2) at (0.75, 3.5) {};
		\node [style=none] (3) at (-0.25, 3.5) {};
		\node [style=map] (4) at (0.75, 2.5) {$g$};
		\node [style=none] (5) at (0.75, 3.5) {};
		\node [style=otimes] (6) at (0.75, 3.5) {};
		\node [style=otimes] (7) at (-0.25, 3.5) {};
		\node [style=X] (8) at (0.25, 1.5) {};
		\node [style=none] (9) at (-0.25, 4.5) {};
		\node [style=none] (10) at (0.75, 4.5) {};
		\node [style=none] (11) at (0.25, 0.5) {};
	\end{pgfonlayer}
	\begin{pgfonlayer}{edgelayer}
		\draw [style=simple, in=117, out=-120] (1.center) to (0);
		\draw [style=simple] (2.center) to (0);
		\draw [style=simple] (3.center) to (4);
		\draw [style=simple, in=63, out=-60] (5.center) to (4);
		\draw [style=simple, in=56, out=-90] (4) to (8);
		\draw [style=simple, in=-90, out=124] (8) to (0);
		\draw [style=simple] (9.center) to (1.center);
		\draw [style=simple] (2.center) to (10.center);
		\draw [style=simple] (8) to (11.center);
	\end{pgfonlayer}
\end{tikzpicture}
$

\item[Restriction terminal map:]
\hfil
$
\begin{tikzpicture}[xscale=-1]
	\begin{pgfonlayer}{nodelayer}
		\node [style=none] (0) at (0, 0.5) {};
		\node [style=none] (1) at (0, 2) {};
		\node [style=none] (2) at (0.75, 2) {};
		\node [style=none] (3) at (0, 1.25) {};
	\end{pgfonlayer}
	\begin{pgfonlayer}{edgelayer}
		\draw [style=dashed, in=-90, out=15] (3.center) to (2.center);
		\draw [style=simple] (0.center) to (1.center);
	\end{pgfonlayer}
\end{tikzpicture}
$

%%%%%TODO ROTATE DIAGRAMS FROM HERE
\item[Tensor product:]
\hfil
$
\begin{tikzpicture}[tikzfig]
	\begin{pgfonlayer}{nodelayer}
		\node [style=map] (0) at (0, 1.5) {$f$};
		\node [style=none] (1) at (-0.5, 2.5) {};
		\node [style=none] (2) at (0.5, 2.5) {};
		\node [style=none] (3) at (0, 0.5) {};
	\end{pgfonlayer}
	\begin{pgfonlayer}{edgelayer}
		\draw [in=117, out=-90] (1.center) to (0);
		\draw [in=-90, out=63] (0) to (2.center);
		\draw (0) to (3.center);
	\end{pgfonlayer}
\end{tikzpicture}
\otimes
\begin{tikzpicture}[tikzfig]
	\begin{pgfonlayer}{nodelayer}
		\node [style=map] (0) at (0, 1.5) {$g$};
		\node [style=none] (1) at (-0.5, 2.5) {};
		\node [style=none] (2) at (0.5, 2.5) {};
		\node [style=none] (3) at (0, 0.5) {};
	\end{pgfonlayer}
	\begin{pgfonlayer}{edgelayer}
		\draw [in=117, out=-90] (1.center) to (0);
		\draw [in=-90, out=63] (0) to (2.center);
		\draw (0) to (3.center);
	\end{pgfonlayer}
\end{tikzpicture}
:=
\begin{tikzpicture}[tikzfig]
	\begin{pgfonlayer}{nodelayer}
		\node [style=map] (0) at (-0.25, 1.5) {$f$};
		\node [style=none] (1) at (-0.25, 0.5) {};
		\node [style=map] (2) at (0.5, 1.5) {$g$};
		\node [style=none] (3) at (0.5, 0.5) {};
		\node [style=otimes] (4) at (-0.25, 2.5) {};
		\node [style=otimes] (5) at (0.5, 2.5) {};
		\node [style=none] (6) at (-0.25, 3.25) {};
		\node [style=none] (7) at (0.5, 3.25) {};
	\end{pgfonlayer}
	\begin{pgfonlayer}{edgelayer}
		\draw (0) to (1.center);
		\draw (2) to (3.center);
		\draw [style=simple] (7.center) to (5);
		\draw [style=simple] (4) to (6.center);
		\draw [style=simple] (5) to (0);
		\draw [style=simple] (2) to (4);
		\draw [style=simple, bend left] (5) to (2);
		\draw [style=simple, bend left, looseness=1.25] (0) to (4);
	\end{pgfonlayer}
\end{tikzpicture}
$

\item[Tensor unit:]  The same as in $\X$.
\end{description}

\end{definition}


\begin{example}\cite[Ex. 5.3.3]{giles}
$\tilde \Pinj$ is $\Par$.
\end{example}
\begin{proof}
For a partial function $f:X\to Y$, $\{(x,(y,x)) | (x,y) \in f \}/\sim$ is a partial isomorphism.
\end{proof}



\begin{lemma}
\label{lemma:xtildefaithful}
The canonical functor $\iota:\X\to \tilde \X$ is faithful.
\end{lemma}

\begin{proof}
Suppose that $\iota(f)\sim\iota(g)$, Then:

\begin{align*}
\begin{tikzpicture}[tikzfig]
	\begin{pgfonlayer}{nodelayer}
		\node [style=map] (0) at (-0.5, 1.5) {$g$};
		\node [style=none] (1) at (-0.5, 2.5) {};
		\node [style=none] (2) at (-0.5, 0.5) {};
	\end{pgfonlayer}
	\begin{pgfonlayer}{edgelayer}
		\draw (1.center) to (0);
		\draw (0) to (2.center);
	\end{pgfonlayer}
\end{tikzpicture}
&=
\begin{tikzpicture}[tikzfig]
	\begin{pgfonlayer}{nodelayer}
		\node [style=map] (0) at (-0.5, 1.25) {$f$};
		\node [style=none] (1) at (-0.5, 0.5) {};
		\node [style=map] (2) at (-0.25, 3) {$f^\circ$};
		\node [style=map] (3) at (-0.25, 3.75) {$g$};
		\node [style=X] (4) at (-0.5, 2) {};
		\node [style=X] (5) at (-0.5, 4.75) {};
		\node [style=none] (6) at (-0.5, 5.75) {};
	\end{pgfonlayer}
	\begin{pgfonlayer}{edgelayer}
		\draw (0) to (1.center);
		\draw (6.center) to (5);
		\draw [in=120, out=-120, looseness=0.75] (5) to (4);
		\draw [in=-90, out=56] (4) to (2);
		\draw (0) to (4);
		\draw (3) to (2);
		\draw [in=-63, out=90] (3) to (5);
	\end{pgfonlayer}
\end{tikzpicture}
=
\begin{tikzpicture}[tikzfig]
	\begin{pgfonlayer}{nodelayer}
		\node [style=none] (0) at (-0.5, 0.5) {};
		\node [style=map] (1) at (-0.25, 4.5) {$g$};
		\node [style=X] (2) at (-0.5, 2.75) {};
		\node [style=X] (3) at (-0.5, 5.5) {};
		\node [style=none] (4) at (-0.5, 6.5) {};
		\node [style=map] (5) at (-0.25, 3.75) {$f^\circ$};
		\node [style=map] (6) at (-0.5, 1.25) {$f$};
		\node [style=map] (7) at (-0.5, 2) {$f^\circ f$};
	\end{pgfonlayer}
	\begin{pgfonlayer}{edgelayer}
		\draw (4.center) to (3);
		\draw [in=120, out=-120, looseness=0.75] (3) to (2);
		\draw [in=-63, out=90] (1) to (3);
		\draw [in=-90, out=56] (2) to (5);
		\draw (1) to (5);
		\draw [style=simple] (2) to (7);
		\draw [style=simple] (7) to (6);
		\draw [style=simple] (6) to (0.center);
	\end{pgfonlayer}
\end{tikzpicture}\\
&=
\begin{tikzpicture}[tikzfig]
	\begin{pgfonlayer}{nodelayer}
		\node [style=none] (0) at (-0.5, 0.5) {};
		\node [style=map] (1) at (0, 4) {$g$};
		\node [style=X] (2) at (-0.5, 2.25) {};
		\node [style=X] (3) at (-0.5, 5) {};
		\node [style=none] (4) at (-0.5, 6) {};
		\node [style=map] (5) at (0, 3.25) {$f^\circ$};
		\node [style=map] (6) at (-0.5, 1.25) {$f$};
		\node [style=map] (7) at (-1, 3.25) {$f^\circ f$};
		\node [style=none] (8) at (-1, 4) {};
	\end{pgfonlayer}
	\begin{pgfonlayer}{edgelayer}
		\draw (4.center) to (3);
		\draw [in=-60, out=90] (1) to (3);
		\draw [in=-90, out=56] (2) to (5);
		\draw (1) to (5);
		\draw [style=simple] (6) to (0.center);
		\draw [style=simple, in=-90, out=120] (2) to (7);
		\draw [style=simple] (2) to (6);
		\draw [style=simple, in=90, out=-120] (3) to (8.center);
		\draw [style=simple] (8.center) to (7);
	\end{pgfonlayer}
\end{tikzpicture}
=
\begin{tikzpicture}[tikzfig]
	\begin{pgfonlayer}{nodelayer}
		\node [style=none] (0) at (-0.5, 0.5) {};
		\node [style=map] (1) at (0, 3.25) {$g$};
		\node [style=X] (2) at (-0.5, 2.25) {};
		\node [style=X] (3) at (-0.5, 4.25) {};
		\node [style=none] (4) at (-0.5, 5.25) {};
		\node [style=map] (5) at (-0.5, 1.25) {$ff^\circ$};
		\node [style=map] (6) at (-1, 3.25) {$f$};
	\end{pgfonlayer}
	\begin{pgfonlayer}{edgelayer}
		\draw (4.center) to (3);
		\draw [in=-60, out=90] (1) to (3);
		\draw [style=simple] (5) to (0.center);
		\draw [style=simple, in=-90, out=120] (2) to (6);
		\draw [style=simple] (2) to (5);
		\draw [style=simple, in=60, out=-90] (1) to (2);
		\draw [style=simple, in=90, out=-120] (3) to (6);
	\end{pgfonlayer}
\end{tikzpicture}\\
&=
\begin{tikzpicture}[tikzfig]
	\begin{pgfonlayer}{nodelayer}
		\node [style=none] (0) at (-0.5, 0.5) {};
		\node [style=map] (1) at (0, 3.25) {$g$};
		\node [style=X] (2) at (-0.5, 1.5) {};
		\node [style=X] (3) at (-0.5, 4.25) {};
		\node [style=none] (4) at (-0.5, 5.25) {};
		\node [style=map] (5) at (-1, 3.25) {$f$};
		\node [style=map] (6) at (-1, 2.5) {$ff^\circ$};
		\node [style=none] (7) at (0, 2.5) {};
	\end{pgfonlayer}
	\begin{pgfonlayer}{edgelayer}
		\draw (4.center) to (3);
		\draw [in=-60, out=90] (1) to (3);
		\draw [style=simple, in=90, out=-120] (3) to (5);
		\draw (1) to (7.center);
		\draw [in=60, out=-90] (7.center) to (2);
		\draw (2) to (0.center);
		\draw [in=-90, out=120] (2) to (6);
		\draw (6) to (5);
	\end{pgfonlayer}
\end{tikzpicture}
=
\begin{tikzpicture}[tikzfig]
	\begin{pgfonlayer}{nodelayer}
		\node [style=none] (0) at (-0.5, 0.5) {};
		\node [style=map] (1) at (0, 2.5) {$g$};
		\node [style=X] (2) at (-0.5, 1.5) {};
		\node [style=X] (3) at (-0.5, 3.5) {};
		\node [style=none] (4) at (-0.5, 4.5) {};
		\node [style=map] (5) at (-1, 2.5) {$f$};
	\end{pgfonlayer}
	\begin{pgfonlayer}{edgelayer}
		\draw (4.center) to (3);
		\draw [in=-60, out=90] (1) to (3);
		\draw [style=simple, in=90, out=-120] (3) to (5);
		\draw (2) to (0.center);
		\draw [in=60, out=-90] (1) to (2);
		\draw [in=-90, out=120] (2) to (5);
	\end{pgfonlayer}
\end{tikzpicture}
\\
&=
\begin{tikzpicture}[tikzfig]
	\begin{pgfonlayer}{nodelayer}
		\node [style=none] (0) at (-0.5, 0.5) {};
		\node [style=map] (1) at (-0.2, 2.5) {$g$};
		\node [style=X] (2) at (-0.5, 1.5) {};
		\node [style=X] (3) at (-0.5, 3.5) {};
		\node [style=none] (4) at (-0.5, 4.5) {};
		\node [style=map] (5) at (-0.8, 2.5) {$f$};
	\end{pgfonlayer}
	\begin{pgfonlayer}{edgelayer}
		\draw (4.center) to (3);
		\draw [in=-120, out=90, looseness=1.25] (1) to (3);
		\draw [style=simple, in=90, out=-60, looseness=1.25] (3) to (5);
		\draw (2) to (0.center);
		\draw [in=120, out=-90, looseness=1.25] (1) to (2);
		\draw [in=-90, out=60, looseness=1.25] (2) to (5);
	\end{pgfonlayer}
\end{tikzpicture}
=
\begin{tikzpicture}[tikzfig]
	\begin{pgfonlayer}{nodelayer}
		\node [style=none] (0) at (-0.5, 0.5) {};
		\node [style=map] (1) at (0, 2.5) {$f$};
		\node [style=X] (2) at (-0.5, 1.5) {};
		\node [style=X] (3) at (-0.5, 3.5) {};
		\node [style=none] (4) at (-0.5, 4.5) {};
		\node [style=map] (5) at (-1, 2.5) {$g$};
	\end{pgfonlayer}
	\begin{pgfonlayer}{edgelayer}
		\draw (4.center) to (3);
		\draw [in=-60, out=90] (1) to (3);
		\draw [style=simple, in=90, out=-120] (3) to (5);
		\draw (2) to (0.center);
		\draw [in=60, out=-90] (1) to (2);
		\draw [in=-90, out=120] (2) to (5);
	\end{pgfonlayer}
\end{tikzpicture}\\
&=
\begin{tikzpicture}[tikzfig]
	\begin{pgfonlayer}{nodelayer}
		\node [style=map] (0) at (-0.5, 1.25) {$g$};
		\node [style=none] (1) at (-0.5, 0.5) {};
		\node [style=map] (2) at (-0.25, 3) {$g^\circ$};
		\node [style=map] (3) at (-0.25, 3.75) {$f$};
		\node [style=X] (4) at (-0.5, 2) {};
		\node [style=X] (5) at (-0.5, 4.75) {};
		\node [style=none] (6) at (-0.5, 5.75) {};
	\end{pgfonlayer}
	\begin{pgfonlayer}{edgelayer}
		\draw (0) to (1.center);
		\draw (6.center) to (5);
		\draw [in=120, out=-120, looseness=0.75] (5) to (4);
		\draw [in=-90, out=56] (4) to (2);
		\draw (0) to (4);
		\draw (3) to (2);
		\draw [in=-63, out=90] (3) to (5);
	\end{pgfonlayer}
\end{tikzpicture}
=
\begin{tikzpicture}[tikzfig]
	\begin{pgfonlayer}{nodelayer}
		\node [style=map] (0) at (-0.5, 1.5) {$f$};
		\node [style=none] (1) at (-0.5, 2.5) {};
		\node [style=none] (2) at (-0.5, 0.5) {};
	\end{pgfonlayer}
	\begin{pgfonlayer}{edgelayer}
		\draw (1.center) to (0);
		\draw (0) to (2.center);
	\end{pgfonlayer}
\end{tikzpicture}
\end{align*}


\end{proof}



\begin{lemma}
The induced Frobenius algebra structure in $\tilde \X$ is counital.
\end{lemma}
\begin{proof}
For all $X$, the map $X \to (X\otimes X) \otimes I$ in $\tilde\X$ induced by the Frobenius algebra in $\X$ has a counit given by the  unitor $X\to I\otimes X$ since, in $\X$:
$$
\begin{tikzpicture}[tikzfig]
	\begin{pgfonlayer}{nodelayer}
		\node [style=X] (0) at (0, 3.75) {};
		\node [style=none] (1) at (0, 3) {};
		\node [style=X] (2) at (-0.25, 4.75) {};
		\node [style=X] (3) at (0, 5.75) {};
		\node [style=X] (4) at (-0.25, 7.25) {};
		\node [style=none] (5) at (-0.25, 8) {};
		\node [style=none] (6) at (0.25, 8) {};
		\node [style=none] (7) at (0, 6.5) {};
	\end{pgfonlayer}
	\begin{pgfonlayer}{edgelayer}
		\draw (0) to (1.center);
		\draw [in=-90, out=120] (0) to (2);
		\draw (2) to (3);
		\draw (5.center) to (4);
		\draw [in=120, out=-120, looseness=0.75] (4) to (2);
		\draw [in=60, out=-60] (3) to (0);
		\draw (3) to (7.center);
		\draw [in=-45, out=90, looseness=0.75] (7.center) to (4);
		\draw [style=dashed, in=-90, out=75] (7.center) to (6.center);
	\end{pgfonlayer}
\end{tikzpicture}
=
\begin{tikzpicture}[tikzfig]
	\begin{pgfonlayer}{nodelayer}
		\node [style=none] (0) at (0, 3) {};
		\node [style=none] (1) at (0, 4.5) {};
		\node [style=none] (2) at (0.5, 4.5) {};
		\node [style=none] (3) at (0, 3.75) {};
	\end{pgfonlayer}
	\begin{pgfonlayer}{edgelayer}
		\draw [style=dashed, in=-90, out=15] (3.center) to (2.center);
		\draw [style=simple] (3.center) to (0.center);
		\draw [style=simple] (3.center) to (1.center);
	\end{pgfonlayer}
\end{tikzpicture}
$$
\end{proof}



\section{Categorical quantum mechanics and completely positive maps}
\label{sec:cpm}
The $\sf CPM$ construction gives a notion of quantum channels for any $\dag$-compact closed category \cite{cpm}.
The \dag-Frobenius algebras in the base category induce idempotents in $\sf CPM$ corresponding to decohering quantum channels.  By considering the full subcategory of the Karoubi envelope whose objects are such idempotents one obtains the $\STOCH$ construction of \cite{coecke2016categories}: yielding classical channels between finite dimensional $C^*$-algebras when applied to $\FHilb$.   However, the $\sf CPM$ construction  can not be applied to $\Hilb$ in general because unlike $\FHilb$, it is not compact closed. 
The $\CP^\infty$ construction  \cite{coecke2016pictures} generalizes the $\sf CPM$ construction to (non compact closed) $\dag$-symmetric monoidal categories, by unbending the cups/caps and, identifying two super-maps  when they act the same on all positive test maps: recovering the usual notion of purely quantum channels.

%$$
%\begin{tikzpicture}
%	\begin{pgfonlayer}{nodelayer}
%		\node [style=none] (0) at (0.5, -0) {};
%		\node [style=none] (1) at (0.5, -0.75) {};
%		\node [style=X] (2) at (1.25, -0) {};
%		\node [style=X] (3) at (1.25, -0.75) {};
%		\node [style=Z] (4) at (2.75, -0) {};
%		\node [style=Z] (5) at (2.75, -0.75) {};
%		\node [style=none] (6) at (3.5, -0) {};
%		\node [style=none] (7) at (3.5, -0.75) {};
%		\node [style=map] (8) at (2, -0) {$f$};
%		\node [style=map] (9) at (2, -0.75) {$f_*$};
%	\end{pgfonlayer}
%	\begin{pgfonlayer}{edgelayer}
%		\draw [bend right=60, looseness=1.25] (3) to (2);
%		\draw (2) to (0.center);
%		\draw (1.center) to (3);
%		\draw (5) to (3);
%		\draw [bend right=60, looseness=1.50] (5) to (4);
%		\draw (4) to (6.center);
%		\draw (5) to (7.center);
%		\draw (4) to (2);
%		\draw [style=simple, in=-30, out=30, looseness=1.25] (9) to (8);
%	\end{pgfonlayer}
%\end{tikzpicture}
%\iff
%\begin{tikzpicture}
%	\begin{pgfonlayer}{nodelayer}
%		\node [style=none] (0) at (0.5, -0.5) {};
%		\node [style=X] (1) at (1.25, -0.5) {};
%		\node [style=Z] (2) at (3, -0.5) {};
%		\node [style=none] (3) at (3.75, -0.5) {};
%		\node [style=map] (4) at (2, -0) {$f$};
%		\node [style=map] (5) at (2, -1) {$f_*$};
%		\node [style=none] (6) at (2, -0) {};
%		\node [style=none] (7) at (2, -1) {};
%	\end{pgfonlayer}
%	\begin{pgfonlayer}{edgelayer}
%		\draw (1) to (0.center);
%		\draw (2) to (3.center);
%		\draw [style=simple, in=-60, out=180, looseness=1.00] (5) to (1);
%		\draw [style=simple, in=-120, out=0, looseness=1.00] (5) to (2);
%		\draw [style=simple, in=120, out=0, looseness=1.00] (4) to (2);
%		\draw [style=simple, in=180, out=60, looseness=1.00] (1) to (4);
%		\draw [style=simple, bend left=75, looseness=1.50] (6.center) to (7.center);
%	\end{pgfonlayer}
%\end{tikzpicture}
%$$

\begin{figure}



$$
\begin{tikzpicture}[tikzfig]
	\begin{pgfonlayer}{nodelayer}
		\node [style=map] (0) at (0.25, 1.25) {$f$};
		\node [style=none] (1) at (0.25, 0.5) {};
		\node [style=none] (2) at (0, 2) {};
		\node [style=none] (3) at (0.5, 2) {};
	\end{pgfonlayer}
	\begin{pgfonlayer}{edgelayer}
		\draw [style=simple] (1.center) to (0);
		\draw [style=simple, in=-90, out=124] (0) to (2.center);
		\draw [style=simple, in=56, out=-90] (3.center) to (0);
	\end{pgfonlayer}
\end{tikzpicture}
;
\begin{tikzpicture}[tikzfig]
	\begin{pgfonlayer}{nodelayer}
		\node [style=map] (0) at (0.25, 1.25) {$g$};
		\node [style=none] (1) at (0.25, 0.5) {};
		\node [style=none] (2) at (0, 2) {};
		\node [style=none] (3) at (0.5, 2) {};
	\end{pgfonlayer}
	\begin{pgfonlayer}{edgelayer}
		\draw [style=simple] (1.center) to (0);
		\draw [style=simple, in=-90, out=124] (0) to (2.center);
		\draw [style=simple, in=56, out=-90] (3.center) to (0);
	\end{pgfonlayer}
\end{tikzpicture}
:=
\begin{tikzpicture}[tikzfig]
	\begin{pgfonlayer}{nodelayer}
		\node [style=map] (0) at (0.25, 1.25) {$f$};
		\node [style=none] (1) at (0.25, 0.5) {};
		\node [style=none] (2) at (0, 2) {};
		\node [style=none] (3) at (0.5, 2.75) {};
		\node [style=none] (4) at (0.5, 2.75) {};
		\node [style=none] (5) at (-0.25, 3.25) {};
		\node [style=map] (6) at (0, 2) {$g$};
		\node [style=otimes] (7) at (0.5, 2.75) {};
		\node [style=none] (8) at (0.5, 3.25) {};
	\end{pgfonlayer}
	\begin{pgfonlayer}{edgelayer}
		\draw [style=simple] (1.center) to (0);
		\draw [style=simple, in=-90, out=124] (0) to (2.center);
		\draw [style=simple, in=56, out=-90] (3.center) to (0);
		\draw [style=simple, in=-90, out=124] (6) to (5.center);
		\draw [style=simple, in=56, out=-120] (4.center) to (6);
		\draw [style=simple] (8.center) to (3.center);
	\end{pgfonlayer}
\end{tikzpicture}
\hspace*{.5cm}
\begin{tikzpicture}[tikzfig]
	\begin{pgfonlayer}{nodelayer}
		\node [style=map] (0) at (0.25, 1) {$h$};
		\node [style=none] (1) at (0.25, 0.25) {};
		\node [style=none] (2) at (-0.25, 3) {};
		\node [style=map] (3) at (0.25, 2.25) {$h^\circ$};
		\node [style=none] (4) at (0.25, 3) {};
		\node [style=none] (5) at (-0.25, 0.25) {};
	\end{pgfonlayer}
	\begin{pgfonlayer}{edgelayer}
		\draw [style=simple] (1.center) to (0);
		\draw [style=simple, in=-90, out=124, looseness=0.75] (0) to (2.center);
		\draw [style=simple] (4.center) to (3);
		\draw [style=simple, in=90, out=-124, looseness=0.75] (3) to (5.center);
		\draw [bend right, looseness=0.75] (0) to (3);
	\end{pgfonlayer}
\end{tikzpicture}
=
\begin{tikzpicture}[tikzfig]
	\begin{pgfonlayer}{nodelayer}
		\node [style=map] (0) at (0.25, 1.25) {$k$};
		\node [style=none] (1) at (0.25, 0.5) {};
		\node [style=none] (2) at (-0.25, 3.25) {};
		\node [style=map] (3) at (0.25, 2.5) {$k^\circ$};
		\node [style=none] (4) at (0.25, 3.25) {};
		\node [style=none] (5) at (-0.25, 0.5) {};
	\end{pgfonlayer}
	\begin{pgfonlayer}{edgelayer}
		\draw [style=simple] (1.center) to (0);
		\draw [style=simple, in=-90, out=124, looseness=0.75] (0) to (2.center);
		\draw [style=simple] (4.center) to (3);
		\draw [style=simple, in=90, out=-124, looseness=0.75] (3) to (5.center);
		\draw [bend right, looseness=0.75] (0) to (3);
	\end{pgfonlayer}
\end{tikzpicture}
\hspace*{.5cm}
\begin{tikzpicture}[tikzfig]
	\begin{pgfonlayer}{nodelayer}
		\node [style=X] (0) at (0, 1) {};
		\node [style=none] (1) at (0, 0.5) {};
		\node [style=none] (2) at (-0.25, 1.75) {};
		\node [style=none] (3) at (0.25, 1.75) {};
	\end{pgfonlayer}
	\begin{pgfonlayer}{edgelayer}
		\draw (1.center) to (0);
		\draw [in=-90, out=108] (0) to (2.center);
		\draw [in=72, out=-90, looseness=0.75] (3.center) to (0);
	\end{pgfonlayer}
\end{tikzpicture}
$$

\caption{
Composition of representatives $f;g$;  equivalence relation $h\sim k$; decoherence map.}
\label{fig:kraus}
\end{figure}


To generalize the $\STOCH$ construction to  \dag-semi-Frobenius algebras, one must combine the  %\linebreak[4]
 $\STOCH$ and $\CP^\infty$ constructions, as the compact closed structure is no longer taken for granted.   We show that the Cartesian completion is the same as first applying a modified version of the  $\CP^\infty$ construction (without quantifying over all test maps, as seen in Figure \ref{fig:kraus}) to a discrete inverse category and then taking the full subcategory of the Karoubi envelope whose objects are  the decoherence maps \footnote{Although, composition in this version of the ${\sf CP}^\infty$ construction, without universally quantifying over test maps, when applied to a discrete inverse category is not obviously well-defined unless the base category embeds in a compact closed category.}.  The following Lemma is needed to prove this fact:




\begin{lemma}
\label{lem:latching}

Given two parallel maps $X\xrightarrow{f,g} Y\otimes Z$ in a discrete inverse category:

$$
\begin{tikzpicture}[tikzfig]
	\begin{pgfonlayer}{nodelayer}
		\node [style=map] (0) at (0, 1.25) {$f$};
		\node [style=none] (1) at (0, 0.5) {};
		\node [style=none] (2) at (-0.25, 2) {};
		\node [style=none] (3) at (0.25, 2) {};
	\end{pgfonlayer}
	\begin{pgfonlayer}{edgelayer}
		\draw (1.center) to (0);
		\draw [in=60, out=-90] (3.center) to (0);
		\draw [in=-90, out=120] (0) to (2.center);
	\end{pgfonlayer}
\end{tikzpicture}
=
\begin{tikzpicture}[tikzfig]
	\begin{pgfonlayer}{nodelayer}
		\node [style=map] (0) at (0, 1.25) {$g$};
		\node [style=none] (1) at (0, 0.5) {};
		\node [style=none] (2) at (-0.25, 2) {};
		\node [style=none] (3) at (0.25, 2) {};
	\end{pgfonlayer}
	\begin{pgfonlayer}{edgelayer}
		\draw (1.center) to (0);
		\draw [in=60, out=-90] (3.center) to (0);
		\draw [in=-90, out=120] (0) to (2.center);
	\end{pgfonlayer}
\end{tikzpicture}
\iff
\begin{tikzpicture}[tikzfig]
	\begin{pgfonlayer}{nodelayer}
		\node [style=map] (0) at (0, 1.25) {$f$};
		\node [style=X] (1) at (-0.5, 2.25) {};
		\node [style=none] (2) at (-0.5, 3) {};
		\node [style=none] (3) at (0.25, 3) {};
		\node [style=none] (4) at (-0.75, 0.5) {};
		\node [style=none] (5) at (0, 0.5) {};
	\end{pgfonlayer}
	\begin{pgfonlayer}{edgelayer}
		\draw (5.center) to (0);
		\draw (0) to (1);
		\draw (1) to (2.center);
		\draw [in=75, out=-90] (3.center) to (0);
		\draw [in=90, out=-105] (1) to (4.center);
	\end{pgfonlayer}
\end{tikzpicture}
=
\begin{tikzpicture}[tikzfig]
	\begin{pgfonlayer}{nodelayer}
		\node [style=map] (0) at (0, 1.25) {$g$};
		\node [style=X] (1) at (-0.5, 2.25) {};
		\node [style=none] (2) at (-0.5, 3) {};
		\node [style=none] (3) at (0.25, 3) {};
		\node [style=none] (4) at (-0.75, 0.5) {};
		\node [style=none] (5) at (0, 0.5) {};
	\end{pgfonlayer}
	\begin{pgfonlayer}{edgelayer}
		\draw (5.center) to (0);
		\draw (0) to (1);
		\draw (1) to (2.center);
		\draw [in=75, out=-90] (3.center) to (0);
		\draw [in=90, out=-105] (1) to (4.center);
	\end{pgfonlayer}
\end{tikzpicture}
$$
\end{lemma}

\begin{proof}
The one direction is trivial, for the other direction:

\begin{align*}
\begin{tikzpicture}[tikzfig]
	\begin{pgfonlayer}{nodelayer}
		\node [style=map] (0) at (0, 1.25) {$f$};
		\node [style=none] (1) at (0, 0.5) {};
		\node [style=none] (2) at (-0.25, 2) {};
		\node [style=none] (3) at (0.25, 2) {};
	\end{pgfonlayer}
	\begin{pgfonlayer}{edgelayer}
		\draw (1.center) to (0);
		\draw [in=60, out=-90] (3.center) to (0);
		\draw [in=-90, out=120] (0) to (2.center);
	\end{pgfonlayer}
\end{tikzpicture}
&=
\begin{tikzpicture}[tikzfig]
	\begin{pgfonlayer}{nodelayer}
		\node [style=map] (0) at (0, 1.25) {$f$};
		\node [style=none] (1) at (0, 0.5) {};
		\node [style=X] (2) at (-0.25, 2) {};
		\node [style=X] (3) at (0.25, 2) {};
		\node [style=X] (4) at (0.25, 2.75) {};
		\node [style=X] (5) at (-0.25, 2.75) {};
		\node [style=none] (6) at (-0.25, 3.25) {};
		\node [style=none] (7) at (0.25, 3.25) {};
	\end{pgfonlayer}
	\begin{pgfonlayer}{edgelayer}
		\draw (1.center) to (0);
		\draw [in=60, out=-90] (3) to (0);
		\draw [in=-90, out=120] (0) to (2);
		\draw (6.center) to (5);
		\draw (4) to (7.center);
		\draw [in=-60, out=60] (3) to (4);
		\draw [in=120, out=-120] (5) to (2);
		\draw [in=60, out=-60] (5) to (2);
		\draw [in=-120, out=120] (3) to (4);
	\end{pgfonlayer}
\end{tikzpicture}
=
\begin{tikzpicture}[tikzfig]
	\begin{pgfonlayer}{nodelayer}
		\node [style=X] (0) at (0, 1) {};
		\node [style=X] (1) at (0.5, 2.75) {};
		\node [style=none] (2) at (-0.5, 3.25) {};
		\node [style=none] (3) at (0.5, 3.25) {};
		\node [style=none] (4) at (0, 0.5) {};
		\node [style=X] (5) at (-0.5, 2.75) {};
		\node [style=map] (6) at (0.5, 1.75) {$f$};
		\node [style=map] (7) at (-0.5, 1.75) {$f$};
	\end{pgfonlayer}
	\begin{pgfonlayer}{edgelayer}
		\draw (2.center) to (5);
		\draw (1) to (3.center);
		\draw (4.center) to (0);
		\draw [in=-90, out=135] (0) to (7);
		\draw [in=45, out=-90] (6) to (0);
		\draw (7) to (1);
		\draw [in=120, out=-120] (5) to (7);
		\draw (6) to (5);
		\draw [in=60, out=-60] (1) to (6);
	\end{pgfonlayer}
\end{tikzpicture}
=
\begin{tikzpicture}[tikzfig]
	\begin{pgfonlayer}{nodelayer}
		\node [style=X] (0) at (0, 1) {};
		\node [style=X] (1) at (0.25, 3.25) {};
		\node [style=none] (2) at (-0.75, 3.75) {};
		\node [style=none] (3) at (0.25, 3.75) {};
		\node [style=none] (4) at (0, 0.5) {};
		\node [style=X] (5) at (-0.75, 3.25) {};
		\node [style=map] (6) at (0.25, 1.75) {$f$};
		\node [style=map] (7) at (-0.25, 2.5) {$f$};
		\node [style=none] (8) at (-0.75, 2.25) {};
	\end{pgfonlayer}
	\begin{pgfonlayer}{edgelayer}
		\draw (2.center) to (5);
		\draw (1) to (3.center);
		\draw (4.center) to (0);
		\draw [in=-90, out=135] (0) to (7);
		\draw [in=45, out=-90] (6) to (0);
		\draw [in=-124, out=60] (7) to (1);
		\draw [in=120, out=-60] (5) to (7);
		\draw [in=60, out=-60] (1) to (6);
		\draw [in=-90, out=120, looseness=0.75] (6) to (8.center);
		\draw [in=-90, out=90] (8.center) to (5);
	\end{pgfonlayer}
\end{tikzpicture}
=
\begin{tikzpicture}[tikzfig]
	\begin{pgfonlayer}{nodelayer}
		\node [style=X] (0) at (0, 1) {};
		\node [style=X] (1) at (0.25, 3.25) {};
		\node [style=none] (2) at (-0.75, 3.75) {};
		\node [style=none] (3) at (0.25, 3.75) {};
		\node [style=none] (4) at (0, 0.5) {};
		\node [style=X] (5) at (-0.75, 3.25) {};
		\node [style=map] (6) at (0.25, 1.75) {$f$};
		\node [style=map] (7) at (-0.25, 2.5) {$g$};
		\node [style=none] (8) at (-0.75, 2.25) {};
	\end{pgfonlayer}
	\begin{pgfonlayer}{edgelayer}
		\draw (2.center) to (5);
		\draw (1) to (3.center);
		\draw (4.center) to (0);
		\draw [in=-90, out=135] (0) to (7);
		\draw [in=45, out=-90] (6) to (0);
		\draw [in=-124, out=60] (7) to (1);
		\draw [in=120, out=-60] (5) to (7);
		\draw [in=60, out=-60] (1) to (6);
		\draw [in=-90, out=120, looseness=0.75] (6) to (8.center);
		\draw [in=-90, out=90] (8.center) to (5);
	\end{pgfonlayer}
\end{tikzpicture}\\
&=
\begin{tikzpicture}[tikzfig]
	\begin{pgfonlayer}{nodelayer}
		\node [style=X] (0) at (0, 1) {};
		\node [style=X] (1) at (0.5, 2.75) {};
		\node [style=none] (2) at (-0.5, 3.25) {};
		\node [style=none] (3) at (0.5, 3.25) {};
		\node [style=none] (4) at (0, 0.5) {};
		\node [style=X] (5) at (-0.5, 2.75) {};
		\node [style=map] (6) at (0.5, 1.75) {$f$};
		\node [style=map] (7) at (-0.5, 1.75) {$g$};
	\end{pgfonlayer}
	\begin{pgfonlayer}{edgelayer}
		\draw (2.center) to (5);
		\draw (1) to (3.center);
		\draw (4.center) to (0);
		\draw [in=-90, out=135] (0) to (7);
		\draw [in=45, out=-90] (6) to (0);
		\draw (7) to (1);
		\draw [in=120, out=-120] (5) to (7);
		\draw (6) to (5);
		\draw [in=60, out=-60] (1) to (6);
	\end{pgfonlayer}
\end{tikzpicture}
=
\begin{tikzpicture}[tikzfig]
	\begin{pgfonlayer}{nodelayer}
		\node [style=X] (0) at (0, 1) {};
		\node [style=X] (1) at (0.5, 2.75) {};
		\node [style=none] (2) at (-0.5, 3.25) {};
		\node [style=none] (3) at (0.5, 3.25) {};
		\node [style=none] (4) at (0, 0.5) {};
		\node [style=X] (5) at (-0.5, 2.75) {};
		\node [style=map] (6) at (0.5, 1.75) {$g$};
		\node [style=map] (7) at (-0.5, 1.75) {$g$};
	\end{pgfonlayer}
	\begin{pgfonlayer}{edgelayer}
		\draw (2.center) to (5);
		\draw (1) to (3.center);
		\draw (4.center) to (0);
		\draw [in=-90, out=135] (0) to (7);
		\draw [in=45, out=-90] (6) to (0);
		\draw (7) to (1);
		\draw [in=120, out=-120] (5) to (7);
		\draw (6) to (5);
		\draw [in=60, out=-60] (1) to (6);
	\end{pgfonlayer}
\end{tikzpicture}
=
\begin{tikzpicture}[tikzfig]
	\begin{pgfonlayer}{nodelayer}
		\node [style=map] (0) at (0, 1.25) {$g$};
		\node [style=none] (1) at (0, 0.5) {};
		\node [style=none] (2) at (-0.25, 2) {};
		\node [style=none] (3) at (0.25, 2) {};
	\end{pgfonlayer}
	\begin{pgfonlayer}{edgelayer}
		\draw (1.center) to (0);
		\draw [in=60, out=-90] (3.center) to (0);
		\draw [in=-90, out=120] (0) to (2.center);
	\end{pgfonlayer}
\end{tikzpicture}
\end{align*}
\end{proof}





\begin{lemma}
\label{theorem:cpstartheorem}
Given two maps $X \xrightarrow{f} Y\otimes S$ and $X \xrightarrow{g} Y\otimes T$, in a discrete inverse category:

\begin{align*}
\begin{tikzpicture}[tikzfig]
	\begin{pgfonlayer}{nodelayer}
		\node [style=map] (0) at (0.5, 1.75) {$g$};
		\node [style=none] (1) at (0.5, 1) {};
		\node [style=map] (2) at (0.5, 3.25) {$g^\circ$};
		\node [style=map] (3) at (0.5, 4) {$f$};
		\node [style=X] (4) at (0.25, 2.5) {};
		\node [style=X] (5) at (0.25, 4.75) {};
		\node [style=none] (6) at (0.25, 5.25) {};
		\node [style=none] (7) at (0.75, 5.25) {};
	\end{pgfonlayer}
	\begin{pgfonlayer}{edgelayer}
		\draw (1.center) to (0);
		\draw [in=-90, out=124] (0) to (4);
		\draw (4) to (2);
		\draw [in=60, out=-60] (2) to (0);
		\draw [in=-120, out=120] (4) to (5);
		\draw (5) to (6.center);
		\draw [in=60, out=-90] (7.center) to (3);
		\draw (3) to (5);
		\draw (3) to (2);
	\end{pgfonlayer}
\end{tikzpicture}
=
\begin{tikzpicture}[tikzfig]
	\begin{pgfonlayer}{nodelayer}
		\node [style=map] (0) at (0.5, 1.75) {$f$};
		\node [style=none] (1) at (0.5, 1) {};
		\node [style=none] (2) at (0.25, 2.5) {};
		\node [style=none] (3) at (0.75, 2.5) {};
	\end{pgfonlayer}
	\begin{pgfonlayer}{edgelayer}
		\draw (1.center) to (0);
		\draw [in=-90, out=60] (0) to (3.center);
		\draw [in=120, out=-90] (2.center) to (0);
	\end{pgfonlayer}
\end{tikzpicture}
&\iff
\begin{tikzpicture}[tikzfig]
	\begin{pgfonlayer}{nodelayer}
		\node [style=X] (0) at (0, 2.25) {};
		\node [style=X] (1) at (0, 3) {};
		\node [style=map] (2) at (0.5, 1.75) {$f$};
		\node [style=map] (3) at (0.5, 3.5) {$f^\circ$};
		\node [style=none] (4) at (-0.25, 1) {};
		\node [style=none] (5) at (0.5, 1) {};
		\node [style=none] (6) at (-0.25, 4.25) {};
		\node [style=none] (7) at (0.5, 4.25) {};
	\end{pgfonlayer}
	\begin{pgfonlayer}{edgelayer}
		\draw (5.center) to (2);
		\draw (2) to (0);
		\draw [in=90, out=-101] (0) to (4.center);
		\draw (0) to (1);
		\draw [in=-90, out=101] (1) to (6.center);
		\draw (7.center) to (3);
		\draw (3) to (1);
		\draw [in=-75, out=75] (2) to (3);
	\end{pgfonlayer}
\end{tikzpicture}
=
\begin{tikzpicture}[tikzfig]
	\begin{pgfonlayer}{nodelayer}
		\node [style=X] (0) at (0, 2.25) {};
		\node [style=X] (1) at (0, 3) {};
		\node [style=map] (2) at (0.5, 1.75) {$g$};
		\node [style=map] (3) at (0.5, 3.5) {$g^\circ$};
		\node [style=none] (4) at (-0.25, 1) {};
		\node [style=none] (5) at (0.5, 1) {};
		\node [style=none] (6) at (-0.25, 4.25) {};
		\node [style=none] (7) at (0.5, 4.25) {};
	\end{pgfonlayer}
	\begin{pgfonlayer}{edgelayer}
		\draw (5.center) to (2);
		\draw (2) to (0);
		\draw [in=90, out=-101] (0) to (4.center);
		\draw (0) to (1);
		\draw [in=-90, out=101] (1) to (6.center);
		\draw (7.center) to (3);
		\draw (3) to (1);
		\draw [in=-75, out=75] (2) to (3);
	\end{pgfonlayer}
\end{tikzpicture}\\
&\iff
\begin{tikzpicture}[tikzfig]
	\begin{pgfonlayer}{nodelayer}
		\node [style=X] (0) at (0, 2.25) {};
		\node [style=X] (1) at (0, 3) {};
		\node [style=map] (2) at (0.5, 1.75) {$f$};
		\node [style=map] (3) at (0.5, 3.5) {$f^\circ$};
		\node [style=none] (4) at (-0.25, 0.75) {};
		\node [style=none] (5) at (-0.25, 4.5) {};
		\node [style=X] (6) at (1, 1.25) {};
		\node [style=X] (7) at (1, 4) {};
		\node [style=none] (8) at (1, 4.5) {};
		\node [style=none] (9) at (1, 0.75) {};
	\end{pgfonlayer}
	\begin{pgfonlayer}{edgelayer}
		\draw (2) to (0);
		\draw [in=90, out=-101] (0) to (4.center);
		\draw (0) to (1);
		\draw [in=-90, out=101] (1) to (5.center);
		\draw (3) to (1);
		\draw [in=-75, out=75] (2) to (3);
		\draw (6) to (2);
		\draw [bend right=15] (6) to (7);
		\draw (7) to (3);
		\draw (7) to (8.center);
		\draw (6) to (9.center);
	\end{pgfonlayer}
\end{tikzpicture}
=
\begin{tikzpicture}
	\begin{pgfonlayer}{nodelayer}
		\node [style=X] (0) at (2.25, -0) {};
		\node [style=X] (1) at (3, -0) {};
		\node [style=map] (2) at (1.75, -0.5) {$g$};
		\node [style=map] (3) at (3.5, -0.5) {$g^\circ$};
		\node [style=none] (4) at (0.75, 0.25) {};
		\node [style=none] (5) at (4.5, 0.25) {};
		\node [style=X] (6) at (1.25, -1) {};
		\node [style=X] (7) at (4, -1) {};
		\node [style=none] (8) at (4.5, -1) {};
		\node [style=none] (9) at (0.75, -1) {};
	\end{pgfonlayer}
	\begin{pgfonlayer}{edgelayer}
		\draw (2) to (0);
		\draw [in=0, out=169, looseness=1.00] (0) to (4.center);
		\draw (0) to (1);
		\draw [in=180, out=11, looseness=1.00] (1) to (5.center);
		\draw (3) to (1);
		\draw [in=-165, out=-15, looseness=1.00] (2) to (3);
		\draw (6) to (2);
		\draw [bend right=15, looseness=1.00] (6) to (7);
		\draw (7) to (3);
		\draw (7) to (8.center);
		\draw (6) to (9.center);
	\end{pgfonlayer}
\end{tikzpicture}
\end{align*}
\end{lemma}


\begin{proof}
First note:

\begin{align*}
\begin{tikzpicture}[tikzfig]
	\begin{pgfonlayer}{nodelayer}
		\node [style=X] (0) at (0, 2.25) {};
		\node [style=X] (1) at (0, 3) {};
		\node [style=map] (2) at (0.5, 1.75) {$f$};
		\node [style=map] (3) at (0.5, 3.5) {$f^\circ$};
		\node [style=none] (4) at (-0.25, 1.25) {};
		\node [style=none] (5) at (-0.25, 4) {};
		\node [style=none] (6) at (0.5, 4) {};
		\node [style=none] (7) at (0.5, 1.25) {};
	\end{pgfonlayer}
	\begin{pgfonlayer}{edgelayer}
		\draw (2) to (0);
		\draw [in=90, out=-101] (0) to (4.center);
		\draw (0) to (1);
		\draw [in=-90, out=101] (1) to (5.center);
		\draw (3) to (1);
		\draw [in=-75, out=75] (2) to (3);
		\draw (2) to (7.center);
		\draw (3) to (6.center);
	\end{pgfonlayer}
\end{tikzpicture}
&=
\begin{tikzpicture}[tikzfig]
	\begin{pgfonlayer}{nodelayer}
		\node [style=X] (0) at (0, 2.25) {};
		\node [style=X] (1) at (0, 3) {};
		\node [style=map] (2) at (0.5, 1) {$f$};
		\node [style=map] (3) at (0.5, 4.25) {$f^\circ$};
		\node [style=none] (4) at (-0.25, 0.5) {};
		\node [style=none] (5) at (-0.25, 4.75) {};
		\node [style=none] (6) at (0.5, 4.75) {};
		\node [style=none] (7) at (0.5, 0.5) {};
		\node [style=X] (8) at (0.25, 1.75) {};
		\node [style=X] (9) at (0.25, 3.5) {};
	\end{pgfonlayer}
	\begin{pgfonlayer}{edgelayer}
		\draw [in=90, out=-101] (0) to (4.center);
		\draw (0) to (1);
		\draw [in=-90, out=101] (1) to (5.center);
		\draw [in=-75, out=75] (2) to (3);
		\draw (2) to (7.center);
		\draw (3) to (6.center);
		\draw (2) to (8);
		\draw (8) to (0);
		\draw [bend right=15] (8) to (9);
		\draw (9) to (3);
		\draw (9) to (1);
	\end{pgfonlayer}
\end{tikzpicture}
=
\begin{tikzpicture}[tikzfig]
	\begin{pgfonlayer}{nodelayer}
		\node [style=X] (0) at (0, 2.25) {};
		\node [style=X] (1) at (0, 3) {};
		\node [style=map] (2) at (0.5, 1) {$f$};
		\node [style=map] (3) at (0.5, 4.25) {$f^\circ$};
		\node [style=none] (4) at (-0.25, 0.5) {};
		\node [style=none] (5) at (-0.25, 4.75) {};
		\node [style=none] (6) at (0.5, 4.75) {};
		\node [style=none] (7) at (0.5, 0.5) {};
		\node [style=X] (8) at (0.25, 1.75) {};
		\node [style=X] (9) at (0.25, 3.5) {};
		\node [style=X] (10) at (0.75, 1.75) {};
		\node [style=X] (11) at (0.75, 3.5) {};
	\end{pgfonlayer}
	\begin{pgfonlayer}{edgelayer}
		\draw [in=90, out=-101] (0) to (4.center);
		\draw (0) to (1);
		\draw [in=-90, out=101] (1) to (5.center);
		\draw (2) to (7.center);
		\draw (3) to (6.center);
		\draw (2) to (8);
		\draw (8) to (0);
		\draw [bend right=15] (8) to (9);
		\draw (9) to (3);
		\draw (9) to (1);
		\draw (2) to (10);
		\draw [in=-135, out=135, looseness=1.25] (10) to (11);
		\draw (11) to (3);
		\draw [in=75, out=-75] (11) to (10);
	\end{pgfonlayer}
\end{tikzpicture}
=
\begin{tikzpicture}[tikzfig]
	\begin{pgfonlayer}{nodelayer}
		\node [style=X] (0) at (1.5, 2.25) {};
		\node [style=none] (1) at (1.25, 0.5) {};
		\node [style=none] (2) at (2.5, 0.5) {};
		\node [style=none] (3) at (1.25, 4.75) {};
		\node [style=X] (4) at (1.5, 3) {};
		\node [style=none] (5) at (2.5, 4.75) {};
		\node [style=map] (6) at (2, 1.75) {$f$};
		\node [style=map] (7) at (2.75, 1.75) {$f$};
		\node [style=map] (8) at (2, 3.5) {$f^\circ$};
		\node [style=map] (9) at (2.75, 3.5) {$f^\circ$};
		\node [style=X] (10) at (2.5, 1) {};
		\node [style=X] (11) at (2.5, 4.25) {};
	\end{pgfonlayer}
	\begin{pgfonlayer}{edgelayer}
		\draw [in=90, out=-101] (0) to (1.center);
		\draw (0) to (4);
		\draw [in=-90, out=101] (4) to (3.center);
		\draw (2.center) to (10);
		\draw (10) to (6);
		\draw (6) to (0);
		\draw (8) to (4);
		\draw (8) to (11);
		\draw (11) to (9);
		\draw [in=120, out=-120] (9) to (7);
		\draw (7) to (10);
		\draw [in=-60, out=60] (7) to (9);
		\draw [in=75, out=-75] (8) to (6);
		\draw (5.center) to (11);
	\end{pgfonlayer}
\end{tikzpicture}\\
&=
\begin{tikzpicture}[tikzfig]
	\begin{pgfonlayer}{nodelayer}
		\node [style=X] (0) at (1.5, 3.75) {};
		\node [style=none] (1) at (1.25, 0.5) {};
		\node [style=none] (2) at (2.5, 0.5) {};
		\node [style=none] (3) at (1.25, 6.25) {};
		\node [style=X] (4) at (1.5, 4.5) {};
		\node [style=none] (5) at (2.5, 6.25) {};
		\node [style=map] (6) at (2, 3.25) {$f$};
		\node [style=map] (7) at (2, 5) {$f^\circ$};
		\node [style=X] (8) at (2.5, 1) {};
		\node [style=X] (9) at (2.5, 5.75) {};
		\node [style=map] (10) at (2, 2.5) {$f^\circ$};
		\node [style=map] (11) at (2, 1.75) {$f$};
	\end{pgfonlayer}
	\begin{pgfonlayer}{edgelayer}
		\draw [in=90, out=-101] (0) to (1.center);
		\draw (0) to (4);
		\draw [in=-90, out=101] (4) to (3.center);
		\draw (2.center) to (8);
		\draw (6) to (0);
		\draw (7) to (4);
		\draw (7) to (9);
		\draw [in=75, out=-75] (7) to (6);
		\draw [in=120, out=-120] (10) to (11);
		\draw [in=-60, out=60] (11) to (10);
		\draw (8) to (11);
		\draw [in=-75, out=75, looseness=0.75] (8) to (9);
		\draw (9) to (5.center);
		\draw (6) to (10);
	\end{pgfonlayer}
\end{tikzpicture}
=
\begin{tikzpicture}[tikzfig]
	\begin{pgfonlayer}{nodelayer}
		\node [style=X] (0) at (1.5, 2.25) {};
		\node [style=none] (1) at (1.25, 0.5) {};
		\node [style=none] (2) at (2.5, 0.5) {};
		\node [style=none] (3) at (1.25, 4.75) {};
		\node [style=X] (4) at (1.5, 3) {};
		\node [style=none] (5) at (2.5, 4.75) {};
		\node [style=map] (6) at (2, 1.75) {$f$};
		\node [style=map] (7) at (2, 3.5) {$f^\circ$};
		\node [style=X] (8) at (2.5, 1) {};
		\node [style=X] (9) at (2.5, 4.25) {};
	\end{pgfonlayer}
	\begin{pgfonlayer}{edgelayer}
		\draw [in=90, out=-101] (0) to (1.center);
		\draw (0) to (4);
		\draw [in=-90, out=101] (4) to (3.center);
		\draw (2.center) to (8);
		\draw (6) to (0);
		\draw (7) to (4);
		\draw (7) to (9);
		\draw [in=75, out=-75] (7) to (6);
		\draw [in=-75, out=75, looseness=0.75] (8) to (9);
		\draw (9) to (5.center);
		\draw (8) to (6);
	\end{pgfonlayer}
\end{tikzpicture}
\end{align*}

So that we only have to prove the first biconditional.
Suppose that the left hand side holds, then:

\begin{align*}
&
\begin{tikzpicture}[tikzfig]
	\begin{pgfonlayer}{nodelayer}
		\node [style=X] (0) at (0, 2.25) {};
		\node [style=X] (1) at (0, 3) {};
		\node [style=map] (2) at (0.5, 1.75) {$f$};
		\node [style=map] (3) at (0.5, 3.5) {$f^\circ$};
		\node [style=none] (4) at (-0.25, 1) {};
		\node [style=none] (5) at (0.5, 1) {};
		\node [style=none] (6) at (-0.25, 4.25) {};
		\node [style=none] (7) at (0.5, 4.25) {};
	\end{pgfonlayer}
	\begin{pgfonlayer}{edgelayer}
		\draw (5.center) to (2);
		\draw (2) to (0);
		\draw [in=90, out=-101] (0) to (4.center);
		\draw (0) to (1);
		\draw [in=-90, out=101] (1) to (6.center);
		\draw (7.center) to (3);
		\draw (3) to (1);
		\draw [in=-75, out=75] (2) to (3);
	\end{pgfonlayer}
\end{tikzpicture}
=
\begin{tikzpicture}[tikzfig]
	\begin{pgfonlayer}{nodelayer}
		\node [style=map] (0) at (0.5, 1.75) {$g$};
		\node [style=none] (1) at (0.5, 1) {};
		\node [style=map] (2) at (0.5, 3.25) {$g^\circ$};
		\node [style=map] (3) at (0.5, 4) {$f$};
		\node [style=X] (4) at (0.25, 2.5) {};
		\node [style=X] (5) at (0.25, 4.75) {};
		\node [style=map] (6) at (0.5, 9.75) {$g^\circ$};
		\node [style=map] (7) at (0.5, 8.25) {$g$};
		\node [style=map] (8) at (0.5, 7.5) {$f^\circ$};
		\node [style=X] (9) at (0.25, 9) {};
		\node [style=X] (10) at (0.25, 6.75) {};
		\node [style=none] (11) at (0.5, 10.5) {};
		\node [style=X] (12) at (0, 5.5) {};
		\node [style=X] (13) at (0, 6) {};
		\node [style=none] (14) at (-0.25, 1) {};
		\node [style=none] (15) at (-0.25, 10.5) {};
		\node [style=none] (16) at (-0.25, 4.75) {};
		\node [style=none] (17) at (-0.25, 6.75) {};
	\end{pgfonlayer}
	\begin{pgfonlayer}{edgelayer}
		\draw (1.center) to (0);
		\draw [in=-90, out=124] (0) to (4);
		\draw (4) to (2);
		\draw [in=60, out=-60] (2) to (0);
		\draw [in=-120, out=120] (4) to (5);
		\draw (3) to (5);
		\draw (3) to (2);
		\draw (11.center) to (6);
		\draw [in=90, out=-124] (6) to (9);
		\draw (9) to (7);
		\draw [in=-60, out=60] (7) to (6);
		\draw [in=120, out=-120] (9) to (10);
		\draw (8) to (10);
		\draw (8) to (7);
		\draw [in=-75, out=75, looseness=0.75] (3) to (8);
		\draw [in=72, out=-90] (10) to (13);
		\draw (13) to (12);
		\draw [in=90, out=-72] (12) to (5);
		\draw [in=90, out=-108] (12) to (16.center);
		\draw (16.center) to (14.center);
		\draw [in=-90, out=108] (13) to (17.center);
		\draw (17.center) to (15.center);
	\end{pgfonlayer}
\end{tikzpicture}\\
&=
\begin{tikzpicture}[tikzfig]
	\begin{pgfonlayer}{nodelayer}
		\node [style=map] (0) at (2.75, 4.25) {$f$};
		\node [style=none] (1) at (2.25, 10.25) {};
		\node [style=none] (2) at (2.25, 0.5) {};
		\node [style=none] (3) at (2.75, 0.5) {};
		\node [style=X] (4) at (2.5, 7.25) {};
		\node [style=X] (5) at (2.5, 5) {};
		\node [style=none] (6) at (2.75, 10.25) {};
		\node [style=X] (7) at (2.5, 2.75) {};
		\node [style=map] (8) at (2.75, 5.75) {$f^\circ$};
		\node [style=map] (9) at (2.75, 3.5) {$g^\circ$};
		\node [style=map] (10) at (2.75, 6.5) {$g$};
		\node [style=map] (11) at (2.75, 1.25) {$g$};
		\node [style=map] (12) at (2.75, 9.5) {$g^\circ$};
		\node [style=X] (13) at (2.5, 8.75) {};
		\node [style=X] (14) at (2.5, 2) {};
		\node [style=X] (15) at (2.5, 8) {};
	\end{pgfonlayer}
	\begin{pgfonlayer}{edgelayer}
		\draw (3.center) to (11);
		\draw (7) to (9);
		\draw [in=60, out=-60] (9) to (11);
		\draw (0) to (9);
		\draw (6.center) to (12);
		\draw (4) to (10);
		\draw [in=-60, out=60] (10) to (12);
		\draw [in=120, out=-120] (4) to (5);
		\draw (8) to (5);
		\draw (8) to (10);
		\draw [in=-60, out=60] (0) to (8);
		\draw [in=-99, out=90] (2.center) to (14);
		\draw (14) to (7);
		\draw (14) to (11);
		\draw (4) to (13);
		\draw (13) to (12);
		\draw [in=-90, out=99] (13) to (1.center);
		\draw [in=-90, out=114] (0) to (5);
		\draw [bend left=45, looseness=0.50] (7) to (15);
	\end{pgfonlayer}
\end{tikzpicture}
=
\begin{tikzpicture}[tikzfig]
	\begin{pgfonlayer}{nodelayer}
		\node [style=X] (0) at (-0.5, 5.75) {};
		\node [style=none] (1) at (-0.75, 7.25) {};
		\node [style=map] (2) at (-0.25, 4.25) {$g$};
		\node [style=map] (3) at (-0.25, 1.25) {$g$};
		\node [style=none] (4) at (-0.75, 0.5) {};
		\node [style=none] (5) at (-0.25, 0.5) {};
		\node [style=none] (6) at (-0.25, 7.25) {};
		\node [style=map] (7) at (-0.25, 3.5) {$g^\circ$};
		\node [style=X] (8) at (-0.5, 2) {};
		\node [style=map] (9) at (-0.25, 6.5) {$g^\circ$};
		\node [style=X] (10) at (-0.5, 5) {};
		\node [style=X] (11) at (-0.5, 2.75) {};
	\end{pgfonlayer}
	\begin{pgfonlayer}{edgelayer}
		\draw (5.center) to (3);
		\draw (11) to (7);
		\draw [in=60, out=-60] (7) to (3);
		\draw (6.center) to (9);
		\draw [in=-60, out=60] (2) to (9);
		\draw [in=-99, out=90] (4.center) to (8);
		\draw (8) to (11);
		\draw (8) to (3);
		\draw (0) to (9);
		\draw [in=-90, out=99] (0) to (1.center);
		\draw (10) to (2);
		\draw (10) to (0);
		\draw (2) to (7);
		\draw [in=-120, out=120] (11) to (10);
	\end{pgfonlayer}
\end{tikzpicture}\\
&=
\begin{tikzpicture}[tikzfig]
	\begin{pgfonlayer}{nodelayer}
		\node [style=map] (0) at (2.75, 8) {$g^\circ$};
		\node [style=none] (1) at (2.75, 0.5) {};
		\node [style=X] (2) at (2.25, 7.25) {};
		\node [style=X] (3) at (2.25, 2) {};
		\node [style=none] (4) at (2, 0.5) {};
		\node [style=none] (5) at (2, 8.75) {};
		\node [style=map] (6) at (2.75, 1.25) {$g$};
		\node [style=X] (7) at (2.25, 2.75) {};
		\node [style=none] (8) at (2.75, 8.75) {};
		\node [style=X] (9) at (2.25, 6.5) {};
		\node [style=map] (10) at (3, 5) {$g$};
		\node [style=map] (11) at (3, 4.25) {$g^\circ$};
		\node [style=X] (12) at (2.5, 3.25) {};
		\node [style=X] (13) at (3, 3.25) {};
		\node [style=X] (14) at (2.5, 6) {};
		\node [style=X] (15) at (3, 6) {};
	\end{pgfonlayer}
	\begin{pgfonlayer}{edgelayer}
		\draw (1.center) to (6);
		\draw (8.center) to (0);
		\draw [in=-99, out=90] (4.center) to (3);
		\draw (3) to (7);
		\draw (3) to (6);
		\draw (2) to (0);
		\draw [in=-90, out=99] (2) to (5.center);
		\draw (9) to (2);
		\draw [in=-120, out=120] (7) to (9);
		\draw (10) to (11);
		\draw [in=-90, out=76] (6) to (13);
		\draw (7) to (12);
		\draw [bend left] (12) to (14);
		\draw [bend right, looseness=1.25] (15) to (13);
		\draw (14) to (9);
		\draw [in=-76, out=90] (15) to (0);
		\draw (12) to (11);
		\draw (13) to (11);
		\draw (10) to (14);
		\draw [in=90, out=-90] (15) to (10);
	\end{pgfonlayer}
\end{tikzpicture}
=
\begin{tikzpicture}[tikzfig]
	\begin{pgfonlayer}{nodelayer}
		\node [style=map] (0) at (3, 7) {$g^\circ$};
		\node [style=none] (1) at (3, 0.5) {};
		\node [style=X] (2) at (2.5, 6.25) {};
		\node [style=X] (3) at (2.5, 2) {};
		\node [style=none] (4) at (2.25, 0.5) {};
		\node [style=none] (5) at (2.25, 7.75) {};
		\node [style=map] (6) at (3, 1.25) {$g$};
		\node [style=none] (7) at (3, 7.75) {};
		\node [style=map] (8) at (3, 4.5) {$g$};
		\node [style=map] (9) at (3, 3.75) {$g^\circ$};
		\node [style=X] (10) at (2.5, 2.75) {};
		\node [style=X] (11) at (3, 2.75) {};
		\node [style=X] (12) at (2.5, 5.5) {};
		\node [style=X] (13) at (3, 5.5) {};
	\end{pgfonlayer}
	\begin{pgfonlayer}{edgelayer}
		\draw (1.center) to (6);
		\draw (7.center) to (0);
		\draw [in=-99, out=90] (4.center) to (3);
		\draw (3) to (6);
		\draw (2) to (0);
		\draw [in=-90, out=99] (2) to (5.center);
		\draw (8) to (9);
		\draw (6) to (11);
		\draw [bend left] (10) to (12);
		\draw [bend right, looseness=1.25] (13) to (11);
		\draw (13) to (0);
		\draw (10) to (9);
		\draw (11) to (9);
		\draw (8) to (12);
		\draw [in=90, out=-90] (13) to (8);
		\draw (3) to (10);
		\draw (12) to (2);
	\end{pgfonlayer}
\end{tikzpicture}\\
&=
\begin{tikzpicture}[tikzfig]
	\begin{pgfonlayer}{nodelayer}
		\node [style=map] (0) at (0, 7.25) {$g^\circ$};
		\node [style=map] (1) at (0.25, 4) {$g$};
		\node [style=X] (2) at (-0.5, 6.5) {};
		\node [style=none] (3) at (-0.75, 8) {};
		\node [style=map] (4) at (0, 1.25) {$g$};
		\node [style=none] (5) at (-0.75, 5) {};
		\node [style=X] (6) at (-0.25, 5) {};
		\node [style=X] (7) at (-0.25, 2.25) {};
		\node [style=none] (8) at (0, 8) {};
		\node [style=X] (9) at (-0.5, 5.75) {};
		\node [style=X] (10) at (0.25, 5) {};
		\node [style=none] (11) at (0, 0.5) {};
		\node [style=map] (12) at (0.25, 3.25) {$g^\circ$};
		\node [style=X] (13) at (0.25, 2.25) {};
		\node [style=none] (14) at (-0.75, 0.5) {};
	\end{pgfonlayer}
	\begin{pgfonlayer}{edgelayer}
		\draw (11.center) to (4);
		\draw (8.center) to (0);
		\draw (2) to (0);
		\draw [in=-90, out=99] (2) to (3.center);
		\draw (9) to (2);
		\draw (1) to (12);
		\draw [in=-90, out=76] (4) to (13);
		\draw [bend left] (7) to (6);
		\draw [bend right, looseness=1.25] (10) to (13);
		\draw [in=-60, out=90] (6) to (9);
		\draw [in=-76, out=90] (10) to (0);
		\draw (7) to (12);
		\draw (13) to (12);
		\draw (1) to (6);
		\draw [in=90, out=-90] (10) to (1);
		\draw [in=104, out=-90] (7) to (4);
		\draw [in=90, out=-120] (9) to (5.center);
		\draw (14.center) to (5.center);
	\end{pgfonlayer}
\end{tikzpicture}
=
\begin{tikzpicture}[tikzfig]
	\begin{pgfonlayer}{nodelayer}
		\node [style=map] (0) at (0, 5.25) {$g^\circ$};
		\node [style=X] (1) at (-0.5, 4.5) {};
		\node [style=none] (2) at (-0.75, 6) {};
		\node [style=map] (3) at (0, 1.5) {$g$};
		\node [style=none] (4) at (-0.75, 3) {};
		\node [style=none] (5) at (0, 6) {};
		\node [style=X] (6) at (-0.5, 3.75) {};
		\node [style=none] (7) at (0, 1) {};
		\node [style=none] (8) at (-0.75, 1) {};
		\node [style=map] (9) at (0, 2.25) {$g^\circ$};
		\node [style=map] (10) at (0, 3) {$g$};
	\end{pgfonlayer}
	\begin{pgfonlayer}{edgelayer}
		\draw (7.center) to (3);
		\draw (5.center) to (0);
		\draw (1) to (0);
		\draw [in=-90, out=99] (1) to (2.center);
		\draw (6) to (1);
		\draw [in=90, out=-120] (6) to (4.center);
		\draw (8.center) to (4.center);
		\draw [in=-120, out=120] (3) to (9);
		\draw [in=60, out=-60] (9) to (3);
		\draw (9) to (10);
		\draw (10) to (6);
		\draw [in=75, out=-75] (0) to (10);
	\end{pgfonlayer}
\end{tikzpicture}
=
\begin{tikzpicture}[tikzfig]
	\begin{pgfonlayer}{nodelayer}
		\node [style=X] (0) at (0, 2.25) {};
		\node [style=X] (1) at (0, 3) {};
		\node [style=map] (2) at (0.5, 1.75) {$g$};
		\node [style=map] (3) at (0.5, 3.5) {$g^\circ$};
		\node [style=none] (4) at (-0.25, 1) {};
		\node [style=none] (5) at (0.5, 1) {};
		\node [style=none] (6) at (-0.25, 4.25) {};
		\node [style=none] (7) at (0.5, 4.25) {};
	\end{pgfonlayer}
	\begin{pgfonlayer}{edgelayer}
		\draw (5.center) to (2);
		\draw (2) to (0);
		\draw [in=90, out=-101] (0) to (4.center);
		\draw (0) to (1);
		\draw [in=-90, out=101] (1) to (6.center);
		\draw (7.center) to (3);
		\draw (3) to (1);
		\draw [in=-75, out=75] (2) to (3);
	\end{pgfonlayer}
\end{tikzpicture}
\end{align*}

Conversely, suppose that the right hand side holds.  Then:

\begin{align*}
\begin{tikzpicture}[tikzfig]
	\begin{pgfonlayer}{nodelayer}
		\node [style=map] (0) at (0.5, 1.75) {$g$};
		\node [style=none] (1) at (0.5, 1) {};
		\node [style=map] (2) at (0.5, 3.25) {$g^\circ$};
		\node [style=map] (3) at (0.5, 4) {$f$};
		\node [style=X] (4) at (0.25, 2.5) {};
		\node [style=none] (5) at (0.75, 5.75) {};
		\node [style=none] (6) at (0.25, 5.75) {};
		\node [style=X] (7) at (0.25, 4.75) {};
		\node [style=X] (8) at (0.25, 5.25) {};
		\node [style=none] (9) at (-0.25, 4.25) {};
		\node [style=none] (10) at (-0.25, 1) {};
	\end{pgfonlayer}
	\begin{pgfonlayer}{edgelayer}
		\draw (1.center) to (0);
		\draw [in=-90, out=124] (0) to (4);
		\draw (4) to (2);
		\draw [in=60, out=-60] (2) to (0);
		\draw [in=60, out=-90] (5.center) to (3);
		\draw (3) to (2);
		\draw [in=-120, out=120] (4) to (7);
		\draw (3) to (7);
		\draw (10.center) to (9.center);
		\draw [in=-135, out=90] (9.center) to (8);
		\draw (8) to (6.center);
		\draw (8) to (7);
	\end{pgfonlayer}
\end{tikzpicture}
&=
\begin{tikzpicture}[tikzfig]
	\begin{pgfonlayer}{nodelayer}
		\node [style=map] (0) at (0.5, 1.25) {$g$};
		\node [style=none] (1) at (0.5, 0.5) {};
		\node [style=map] (2) at (0.5, 3.25) {$g^\circ$};
		\node [style=map] (3) at (0.5, 4) {$f$};
		\node [style=X] (4) at (0.25, 2.5) {};
		\node [style=none] (5) at (0.75, 5.25) {};
		\node [style=none] (6) at (0.25, 5.25) {};
		\node [style=X] (7) at (0.25, 4.75) {};
		\node [style=none] (8) at (0, 0.5) {};
		\node [style=X] (9) at (0.25, 2) {};
	\end{pgfonlayer}
	\begin{pgfonlayer}{edgelayer}
		\draw (1.center) to (0);
		\draw (4) to (2);
		\draw [in=60, out=-60] (2) to (0);
		\draw [in=60, out=-90] (5.center) to (3);
		\draw (3) to (2);
		\draw [in=-120, out=120] (4) to (7);
		\draw (3) to (7);
		\draw (9) to (0);
		\draw (9) to (4);
		\draw [in=90, out=-120] (9) to (8.center);
		\draw (7) to (6.center);
	\end{pgfonlayer}
\end{tikzpicture}
=
\begin{tikzpicture}[tikzfig]
	\begin{pgfonlayer}{nodelayer}
		\node [style=map] (0) at (0.5, 1.25) {$f$};
		\node [style=none] (1) at (0.5, 0.5) {};
		\node [style=map] (2) at (0.5, 3.25) {$f^\circ$};
		\node [style=map] (3) at (0.5, 4) {$f$};
		\node [style=X] (4) at (0.25, 2.5) {};
		\node [style=none] (5) at (0.75, 5.25) {};
		\node [style=none] (6) at (0.25, 5.25) {};
		\node [style=X] (7) at (0.25, 4.75) {};
		\node [style=none] (8) at (0, 0.5) {};
		\node [style=X] (9) at (0.25, 2) {};
	\end{pgfonlayer}
	\begin{pgfonlayer}{edgelayer}
		\draw (1.center) to (0);
		\draw (4) to (2);
		\draw [in=60, out=-60] (2) to (0);
		\draw [in=60, out=-90] (5.center) to (3);
		\draw (3) to (2);
		\draw [in=-120, out=120] (4) to (7);
		\draw (3) to (7);
		\draw (9) to (0);
		\draw (9) to (4);
		\draw [in=90, out=-120] (9) to (8.center);
		\draw (7) to (6.center);
	\end{pgfonlayer}
\end{tikzpicture}\\
&=
\begin{tikzpicture}[tikzfig]
	\begin{pgfonlayer}{nodelayer}
		\node [style=map] (0) at (2.5, 1.25) {$f$};
		\node [style=X] (1) at (2, 6.25) {};
		\node [style=none] (2) at (1.75, 0.5) {};
		\node [style=none] (3) at (2.75, 6.75) {};
		\node [style=none] (4) at (2.5, 0.5) {};
		\node [style=X] (5) at (2, 2.5) {};
		\node [style=X] (6) at (2, 2) {};
		\node [style=none] (7) at (2, 6.75) {};
		\node [style=map] (8) at (2.75, 4.75) {$f$};
		\node [style=map] (9) at (2.75, 4) {$f^\circ$};
		\node [style=X] (10) at (2.25, 5.75) {};
		\node [style=X] (11) at (2.25, 3) {};
		\node [style=X] (12) at (2.75, 5.75) {};
		\node [style=X] (13) at (2.75, 3) {};
	\end{pgfonlayer}
	\begin{pgfonlayer}{edgelayer}
		\draw (4.center) to (0);
		\draw [in=-120, out=120] (5) to (1);
		\draw (6) to (0);
		\draw (6) to (5);
		\draw [in=90, out=-120] (6) to (2.center);
		\draw (1) to (7.center);
		\draw (8) to (9);
		\draw (11) to (5);
		\draw [in=74, out=-90] (13) to (0);
		\draw [in=-120, out=120] (11) to (10);
		\draw (10) to (1);
		\draw (3.center) to (12);
		\draw [in=120, out=-120, looseness=1.25] (12) to (13);
		\draw (9) to (11);
		\draw (8) to (10);
		\draw (9) to (13);
		\draw (8) to (12);
	\end{pgfonlayer}
\end{tikzpicture}
=
\begin{tikzpicture}[tikzfig]
	\begin{pgfonlayer}{nodelayer}
		\node [style=map] (0) at (2.75, 1.25) {$f$};
		\node [style=none] (1) at (2, 0.5) {};
		\node [style=none] (2) at (2.75, 6.25) {};
		\node [style=none] (3) at (2.75, 0.5) {};
		\node [style=X] (4) at (2.25, 2) {};
		\node [style=none] (5) at (2.25, 6.25) {};
		\node [style=map] (6) at (2.75, 4.5) {$f$};
		\node [style=map] (7) at (2.75, 3.75) {$f^\circ$};
		\node [style=X] (8) at (2.25, 5.5) {};
		\node [style=X] (9) at (2.25, 2.75) {};
		\node [style=X] (10) at (2.75, 5.5) {};
		\node [style=X] (11) at (2.75, 2.75) {};
	\end{pgfonlayer}
	\begin{pgfonlayer}{edgelayer}
		\draw (3.center) to (0);
		\draw (4) to (0);
		\draw [in=90, out=-120] (4) to (1.center);
		\draw (6) to (7);
		\draw (11) to (0);
		\draw [in=-120, out=120] (9) to (8);
		\draw (2.center) to (10);
		\draw [in=120, out=-120, looseness=1.25] (10) to (11);
		\draw (7) to (9);
		\draw (6) to (8);
		\draw (7) to (11);
		\draw (6) to (10);
		\draw (9) to (4);
		\draw (8) to (5.center);
	\end{pgfonlayer}
\end{tikzpicture}\\
&=
\begin{tikzpicture}[tikzfig]
	\begin{pgfonlayer}{nodelayer}
		\node [style=map] (0) at (2.5, 2) {$f$};
		\node [style=none] (1) at (1.75, 1.25) {};
		\node [style=none] (2) at (2.75, 6.75) {};
		\node [style=none] (3) at (2.5, 1.25) {};
		\node [style=none] (4) at (2.25, 6.75) {};
		\node [style=map] (5) at (2.75, 4.5) {$f$};
		\node [style=map] (6) at (2.75, 3.75) {$f^\circ$};
		\node [style=X] (7) at (2.25, 5.5) {};
		\node [style=X] (8) at (2.25, 2.75) {};
		\node [style=X] (9) at (2.75, 5.5) {};
		\node [style=X] (10) at (2.75, 2.75) {};
		\node [style=none] (11) at (1.75, 5.5) {};
		\node [style=X] (12) at (2.25, 6.25) {};
	\end{pgfonlayer}
	\begin{pgfonlayer}{edgelayer}
		\draw (3.center) to (0);
		\draw (5) to (6);
		\draw [in=60, out=-90] (10) to (0);
		\draw [in=-120, out=120] (8) to (7);
		\draw (2.center) to (9);
		\draw [in=120, out=-120, looseness=1.25] (9) to (10);
		\draw (6) to (8);
		\draw (5) to (7);
		\draw (6) to (10);
		\draw (5) to (9);
		\draw (7) to (4.center);
		\draw [in=90, out=-135] (12) to (11.center);
		\draw (11.center) to (1.center);
		\draw [in=120, out=-90] (8) to (0);
	\end{pgfonlayer}
\end{tikzpicture}
=
\begin{tikzpicture}[tikzfig]
	\begin{pgfonlayer}{nodelayer}
		\node [style=map] (0) at (2.5, 2) {$f$};
		\node [style=none] (1) at (2, 1.25) {};
		\node [style=none] (2) at (2.75, 4.75) {};
		\node [style=none] (3) at (2.5, 1.25) {};
		\node [style=none] (4) at (2.25, 4.75) {};
		\node [style=map] (5) at (2.5, 3.5) {$f$};
		\node [style=map] (6) at (2.5, 2.75) {$f^\circ$};
		\node [style=none] (7) at (2, 3.5) {};
		\node [style=X] (8) at (2.25, 4.25) {};
	\end{pgfonlayer}
	\begin{pgfonlayer}{edgelayer}
		\draw (3.center) to (0);
		\draw (5) to (6);
		\draw [in=90, out=-135] (8) to (7.center);
		\draw (7.center) to (1.center);
		\draw [bend left, looseness=1.25] (0) to (6);
		\draw [bend left, looseness=1.25] (6) to (0);
		\draw (5) to (8);
		\draw (8) to (4.center);
		\draw [in=60, out=-90] (2.center) to (5);
	\end{pgfonlayer}
\end{tikzpicture}
=
\begin{tikzpicture}[tikzfig]
	\begin{pgfonlayer}{nodelayer}
		\node [style=none] (0) at (2, 2.75) {};
		\node [style=none] (1) at (2.75, 4.75) {};
		\node [style=none] (2) at (2.5, 2.75) {};
		\node [style=none] (3) at (2.25, 4.75) {};
		\node [style=map] (4) at (2.5, 3.5) {$f$};
		\node [style=none] (5) at (2, 3.5) {};
		\node [style=X] (6) at (2.25, 4.25) {};
	\end{pgfonlayer}
	\begin{pgfonlayer}{edgelayer}
		\draw [in=90, out=-135, looseness=0.75] (6) to (5.center);
		\draw (5.center) to (0.center);
		\draw (4) to (6);
		\draw (6) to (3.center);
		\draw [in=60, out=-90] (1.center) to (4);
		\draw (4) to (2.center);
	\end{pgfonlayer}
\end{tikzpicture}\\
\\
\text{Thus, by Lemma \ref{lem:latching}}:&
\begin{tikzpicture}[tikzfig]
	\begin{pgfonlayer}{nodelayer}
		\node [style=map] (0) at (0.5, 1.75) {$g$};
		\node [style=none] (1) at (0.5, 1) {};
		\node [style=map] (2) at (0.5, 3.25) {$g^\circ$};
		\node [style=map] (3) at (0.5, 4) {$f$};
		\node [style=X] (4) at (0.25, 2.5) {};
		\node [style=X] (5) at (0.25, 4.75) {};
		\node [style=none] (6) at (0.25, 5.25) {};
		\node [style=none] (7) at (0.75, 5.25) {};
	\end{pgfonlayer}
	\begin{pgfonlayer}{edgelayer}
		\draw (1.center) to (0);
		\draw [in=-90, out=124] (0) to (4);
		\draw (4) to (2);
		\draw [in=60, out=-60] (2) to (0);
		\draw [in=-120, out=120] (4) to (5);
		\draw (5) to (6.center);
		\draw [in=60, out=-90] (7.center) to (3);
		\draw (3) to (5);
		\draw (3) to (2);
	\end{pgfonlayer}
\end{tikzpicture}
=
\begin{tikzpicture}[tikzfig]
	\begin{pgfonlayer}{nodelayer}
		\node [style=map] (0) at (0.5, 1.75) {$f$};
		\node [style=none] (1) at (0.5, 1) {};
		\node [style=none] (2) at (0.25, 2.5) {};
		\node [style=none] (3) at (0.75, 2.5) {};
	\end{pgfonlayer}
	\begin{pgfonlayer}{edgelayer}
		\draw (1.center) to (0);
		\draw [in=-90, out=60] (0) to (3.center);
		\draw [in=120, out=-90] (2.center) to (0);
	\end{pgfonlayer}
\end{tikzpicture}
\end{align*}


\end{proof}


%\subsection{Environment structures}
\label{sec:env}

The natural question arises: can we characterize classical channels in this setting, algebraically in terms of a discarding morphism, without performing any doubling.  In other words, is there some notion of ``environment structure'' \cite{coecke2010environment} for the {\em classical} channels of discrete inverse categories:


\begin{definition}
Given a discrete inverse category $\X$, define the counital completion of $\X$, $c(\X)$ to have the same objects and maps of $\X$, except with a freely adjoined counit $!_X:X\to I$ to the chosen semi-Frobenius algebra on $X$, for each object in $\X$ compatible with the monoidal structure.
\end{definition}

\begin{lemma}
$c(\X)$ is a discrete Cartesian restriction category.
\end{lemma}
\begin{proof}
This is clearly a counital copy category, with a restriction terminal object given by the tensor unit.  Moreover, because the Frobenius structure is special, it is also discrete.
\end{proof}



\begin{lemma}
\label{lemma:envstruct}
Given a discrete inverse category $\X$, $c(\X)$ and $\tilde \X$ are isomorphic as discrete Cartesian restriction categories.
\end{lemma}

\begin{proof}
Define an identity on objects functor $F:c(\X)\to \tilde \X$ in the obvious way, sending the counits to the ancillary space.
Similarly, define an identity on objects functor from $G:\tilde \X \to c(\X)$ given by plugging counits into the ancillary space.
These maps are clearly inverses to each other and preserve discrete Cartesian restriction structure; however, once again we mush show that they are actually  functors.


To see  that $F$ is a functor, it suffices to observe that every object in  $\tilde \X $ is equipped with a counital Frobenius algebra, compatible with the monoidal structure, where the unit is in the image of the freely adjoined counit under $F$.


%%%%%%%%%%%%%%%%%%%%%%%%%%%%%


To prove that $G$ is a functor, take some $(f,S)\sim (g,T)$ in $\tilde \X$.
Therefore, in $\tilde \X$, since the Frobenius structure is counital:
$$
\begin{tikzpicture}[tikzfig]
	\begin{pgfonlayer}{nodelayer}
		\node [style=map] (0) at (0, 3.25) {$f^\circ$};
		\node [style=X] (1) at (-0.5, 2.25) {};
		\node [style=map] (2) at (0, 1.25) {$f$};
		\node [style=none] (3) at (0, 4.25) {};
		\node [style=none] (4) at (-1, 0.5) {};
		\node [style=none] (5) at (0, 0.5) {};
		\node [style=none] (6) at (0, 3.75) {};
		\node [style=none] (7) at (0.5, 4.25) {};
	\end{pgfonlayer}
	\begin{pgfonlayer}{edgelayer}
		\draw [style=simple] (0) to (3.center);
		\draw [style=simple] (1) to (2);
		\draw [style=simple, in=90, out=-104] (1) to (4.center);
		\draw [style=simple] (5.center) to (2);
		\draw [style=simple, in=60, out=-60, looseness=0.75] (0) to (2);
		\draw [in=-117, out=90] (1) to (0);
		\draw [style=dashed, in=30, out=-90] (7.center) to (6.center);
	\end{pgfonlayer}
\end{tikzpicture}
\sim
\begin{tikzpicture}[tikzfig]
	\begin{pgfonlayer}{nodelayer}
		\node [style=map] (0) at (0, 4.25) {$f^\circ$};
		\node [style=X] (1) at (-0.5, 3.25) {};
		\node [style=X] (2) at (-0.5, 2.5) {};
		\node [style=map] (3) at (0, 1.5) {$f$};
		\node [style=none] (4) at (0, 5.25) {};
		\node [style=none] (5) at (-1, 0.5) {};
		\node [style=none] (6) at (0, 0.5) {};
		\node [style=none] (7) at (0.5, 5.25) {};
		\node [style=none] (8) at (-0.75, 4.5) {};
	\end{pgfonlayer}
	\begin{pgfonlayer}{edgelayer}
		\draw [style=simple] (1) to (0);
		\draw [style=simple] (0) to (4.center);
		\draw [style=simple] (1) to (2);
		\draw [style=simple] (2) to (3);
		\draw [style=simple, in=90, out=-104] (2) to (5.center);
		\draw [style=simple] (6.center) to (3);
		\draw [style=simple, in=60, out=-60, looseness=0.75] (0) to (3);
		\draw [style=simple, in=90, out=-90, looseness=0.75] (7.center) to (8.center);
		\draw [style=simple, in=-90, out=101] (1) to (8.center);
	\end{pgfonlayer}
\end{tikzpicture}
=
\begin{tikzpicture}[tikzfig]
	\begin{pgfonlayer}{nodelayer}
		\node [style=map] (0) at (0, 4.25) {$g^\circ$};
		\node [style=X] (1) at (-0.5, 3.25) {};
		\node [style=X] (2) at (-0.5, 2.5) {};
		\node [style=map] (3) at (0, 1.5) {$g$};
		\node [style=none] (4) at (0, 5.25) {};
		\node [style=none] (5) at (-1, 0.5) {};
		\node [style=none] (6) at (0, 0.5) {};
		\node [style=none] (7) at (0.5, 5.25) {};
		\node [style=none] (8) at (-0.75, 4.5) {};
	\end{pgfonlayer}
	\begin{pgfonlayer}{edgelayer}
		\draw [style=simple] (1) to (0);
		\draw [style=simple] (0) to (4.center);
		\draw [style=simple] (1) to (2);
		\draw [style=simple] (2) to (3);
		\draw [style=simple, in=90, out=-104] (2) to (5.center);
		\draw [style=simple] (6.center) to (3);
		\draw [style=simple, in=60, out=-60, looseness=0.75] (0) to (3);
		\draw [style=simple, in=90, out=-90, looseness=0.75] (7.center) to (8.center);
		\draw [style=simple, in=-90, out=101] (1) to (8.center);
	\end{pgfonlayer}
\end{tikzpicture}
\sim
\begin{tikzpicture}[tikzfig]
	\begin{pgfonlayer}{nodelayer}
		\node [style=map] (0) at (0, 3.25) {$g^\circ$};
		\node [style=X] (1) at (-0.5, 2.25) {};
		\node [style=map] (2) at (0, 1.25) {$g$};
		\node [style=none] (3) at (0, 4.25) {};
		\node [style=none] (4) at (-1, 0.5) {};
		\node [style=none] (5) at (0, 0.5) {};
		\node [style=none] (6) at (0, 3.75) {};
		\node [style=none] (7) at (0.5, 4.25) {};
	\end{pgfonlayer}
	\begin{pgfonlayer}{edgelayer}
		\draw [style=simple] (0) to (3.center);
		\draw [style=simple] (1) to (2);
		\draw [style=simple, in=90, out=-104] (1) to (4.center);
		\draw [style=simple] (5.center) to (2);
		\draw [style=simple, in=60, out=-60, looseness=0.75] (0) to (2);
		\draw [in=-117, out=90] (1) to (0);
		\draw [style=dashed, in=30, out=-90] (7.center) to (6.center);
	\end{pgfonlayer}
\end{tikzpicture}
$$


However, since the functor $\X\to \tilde \X $ is faithful by Lemma \ref{lemma:xtildefaithful}, using the alternate equivalence relation of $\tilde \X$ by  Lemma \ref{theorem:cpstartheorem}, we have that in $\X$:

$$
\begin{tikzpicture}[tikzfig]
	\begin{pgfonlayer}{nodelayer}
		\node [style=map] (0) at (0, 3.25) {$f^\circ$};
		\node [style=X] (1) at (-0.5, 2.25) {};
		\node [style=map] (2) at (0, 1.25) {$f$};
		\node [style=none] (3) at (0, 4.25) {};
		\node [style=none] (4) at (-1, 0.5) {};
		\node [style=none] (5) at (0, 0.5) {};
	\end{pgfonlayer}
	\begin{pgfonlayer}{edgelayer}
		\draw [style=simple] (0) to (3.center);
		\draw [style=simple] (1) to (2);
		\draw [style=simple, in=90, out=-104] (1) to (4.center);
		\draw [style=simple] (5.center) to (2);
		\draw [style=simple, in=60, out=-60, looseness=0.75] (0) to (2);
		\draw [in=-117, out=90] (1) to (0);
	\end{pgfonlayer}
\end{tikzpicture}
=
\begin{tikzpicture}[tikzfig]
	\begin{pgfonlayer}{nodelayer}
		\node [style=map] (0) at (0, 3.5) {$g^\circ$};
		\node [style=X] (1) at (-0.5, 2.5) {};
		\node [style=map] (2) at (0, 1.5) {$g$};
		\node [style=none] (3) at (0, 4.5) {};
		\node [style=none] (4) at (-1, 0.5) {};
		\node [style=none] (5) at (0, 0.5) {};
	\end{pgfonlayer}
	\begin{pgfonlayer}{edgelayer}
		\draw [style=simple] (0) to (3.center);
		\draw [style=simple] (1) to (2);
		\draw [style=simple, in=90, out=-104] (1) to (4.center);
		\draw [style=simple] (5.center) to (2);
		\draw [style=simple, in=60, out=-60, looseness=0.75] (0) to (2);
		\draw [in=-117, out=90] (1) to (0);
	\end{pgfonlayer}
\end{tikzpicture}
\hspace*{.1cm}\text{and thus}\hspace*{.1cm}
\begin{tikzpicture}[tikzfig]
	\begin{pgfonlayer}{nodelayer}
		\node [style=map] (0) at (0, 1.5) {$f$};
		\node [style=X] (1) at (-0.5, 2.5) {};
		\node [style=map] (2) at (0, 3.5) {$f^\circ$};
		\node [style=none] (3) at (0, 0.5) {};
		\node [style=none] (4) at (-1, 4.5) {};
		\node [style=none] (5) at (0, 4.5) {};
	\end{pgfonlayer}
	\begin{pgfonlayer}{edgelayer}
		\draw [style=simple] (0) to (3.center);
		\draw [style=simple] (1) to (2);
		\draw [style=simple, in=-90, out=104] (1) to (4.center);
		\draw [style=simple] (5.center) to (2);
		\draw [style=simple, in=-60, out=60, looseness=0.75] (0) to (2);
		\draw [in=117, out=-90] (1) to (0);
	\end{pgfonlayer}
\end{tikzpicture}
=
\begin{tikzpicture}[tikzfig]
	\begin{pgfonlayer}{nodelayer}
		\node [style=map] (0) at (0, 1.5) {$g$};
		\node [style=X] (1) at (-0.5, 2.5) {};
		\node [style=map] (2) at (0, 3.5) {$g^\circ$};
		\node [style=none] (3) at (0, 0.5) {};
		\node [style=none] (4) at (-1, 4.25) {};
		\node [style=none] (5) at (0, 4.25) {};
	\end{pgfonlayer}
	\begin{pgfonlayer}{edgelayer}
		\draw [style=simple] (0) to (3.center);
		\draw [style=simple] (1) to (2);
		\draw [style=simple, in=-90, out=104] (1) to (4.center);
		\draw [style=simple] (5.center) to (2);
		\draw [style=simple, in=-60, out=60, looseness=0.75] (0) to (2);
		\draw [in=117, out=-90] (1) to (0);
	\end{pgfonlayer}
\end{tikzpicture}
$$


Therefore in $c(\X)$:

\begin{align*}
&
\begin{tikzpicture}[tikzfig]
	\begin{pgfonlayer}{nodelayer}
		\node [style=map] (0) at (0, 1.5) {$f$};
		\node [style=X] (1) at (-0.5, 2.5) {};
		\node [style=map] (2) at (0, 3.5) {$f^\circ$};
		\node [style=none] (3) at (0, 0.5) {};
		\node [style=none] (4) at (-1, 4.5) {};
		\node [style=none] (5) at (0, 4.25) {};
		\node [style=X] (6) at (0, 4.25) {};
	\end{pgfonlayer}
	\begin{pgfonlayer}{edgelayer}
		\draw [style=simple] (0) to (3.center);
		\draw [style=simple] (1) to (2);
		\draw [style=simple, in=-90, out=104] (1) to (4.center);
		\draw [style=simple] (5.center) to (2);
		\draw [style=simple, in=-60, out=60, looseness=0.75] (0) to (2);
		\draw [in=117, out=-90] (1) to (0);
	\end{pgfonlayer}
\end{tikzpicture}
\eq{Rem. \ref{cor:copy}}
\begin{tikzpicture}[tikzfig]
	\begin{pgfonlayer}{nodelayer}
		\node [style=map] (0) at (0, 1.5) {$f$};
		\node [style=X] (1) at (-0.5, 2.5) {};
		\node [style=map] (2) at (0, 3.5) {$\bar {f^\circ}$};
		\node [style=none] (3) at (0, 0.5) {};
		\node [style=none] (4) at (-1, 4.5) {};
		\node [style=X] (5) at (-0.25, 4.5) {};
		\node [style=X] (6) at (0.25, 4.5) {};
	\end{pgfonlayer}
	\begin{pgfonlayer}{edgelayer}
		\draw [style=simple] (0) to (3.center);
		\draw [style=simple] (1) to (2);
		\draw [style=simple, in=-90, out=104] (1) to (4.center);
		\draw [style=simple, in=-60, out=60, looseness=0.75] (0) to (2);
		\draw [in=117, out=-90] (1) to (0);
		\draw [in=104, out=-90] (5) to (2);
		\draw [in=-90, out=76] (2) to (6);
	\end{pgfonlayer}
\end{tikzpicture}
=
\begin{tikzpicture}[tikzfig]
	\begin{pgfonlayer}{nodelayer}
		\node [style=map] (0) at (1.5, 1.5) {$f$};
		\node [style=X] (1) at (1, 2.5) {};
		\node [style=none] (2) at (1.5, 0.5) {};
		\node [style=none] (3) at (0.5, 4.5) {};
		\node [style=X] (4) at (1.25, 4.5) {};
		\node [style=X] (5) at (1.75, 4.5) {};
		\node [style=map] (6) at (2, 3.75) {$\bar {f^\circ}$};
		\node [style=X] (7) at (1.25, 5.25) {};
		\node [style=X] (8) at (1.75, 5.25) {};
		\node [style=X] (9) at (1.25, 3) {};
		\node [style=X] (10) at (1.75, 3) {};
	\end{pgfonlayer}
	\begin{pgfonlayer}{edgelayer}
		\draw [style=simple] (0) to (2.center);
		\draw [style=simple, in=-90, out=104] (1) to (3.center);
		\draw [in=117, out=-90] (1) to (0);
		\draw [in=-120, out=120, looseness=1.25] (9) to (4);
		\draw [in=-120, out=120, looseness=1.25] (10) to (5);
		\draw [in=-75, out=72] (10) to (6);
		\draw [in=-120, out=45] (9) to (6);
		\draw [in=-45, out=120] (6) to (4);
		\draw [in=75, out=-72] (5) to (6);
		\draw (5) to (8);
		\draw (7) to (4);
		\draw (9) to (1);
		\draw [in=60, out=-90] (10) to (0);
	\end{pgfonlayer}
\end{tikzpicture}\\
&=
\begin{tikzpicture}[tikzfig]
	\begin{pgfonlayer}{nodelayer}
		\node [style=map] (0) at (0, 1.5) {$f$};
		\node [style=none] (1) at (0, 1) {};
		\node [style=none] (2) at (-1, 5.75) {};
		\node [style=X] (3) at (-0.25, 4) {};
		\node [style=X] (4) at (0.25, 4) {};
		\node [style=map] (5) at (0.5, 3.25) {$\bar {f^\circ}$};
		\node [style=X] (6) at (-0.25, 5.5) {};
		\node [style=X] (7) at (0.25, 5.5) {};
		\node [style=X] (8) at (-0.25, 2.5) {};
		\node [style=X] (9) at (0.25, 2.5) {};
		\node [style=X] (10) at (-0.25, 4.75) {};
	\end{pgfonlayer}
	\begin{pgfonlayer}{edgelayer}
		\draw [style=simple] (0) to (1.center);
		\draw [in=-120, out=120, looseness=1.25] (8) to (3);
		\draw [in=-120, out=120, looseness=1.25] (9) to (4);
		\draw [in=-75, out=72] (9) to (5);
		\draw [in=-120, out=45] (8) to (5);
		\draw [in=-45, out=120] (5) to (3);
		\draw [in=75, out=-72] (4) to (5);
		\draw (4) to (7);
		\draw (6) to (3);
		\draw [in=60, out=-90] (9) to (0);
		\draw [in=127, out=-90] (2.center) to (10);
		\draw [in=120, out=-90] (8) to (0);
	\end{pgfonlayer}
\end{tikzpicture}
=
\begin{tikzpicture}[tikzfig]
	\begin{pgfonlayer}{nodelayer}
		\node [style=map] (0) at (0, 1.5) {$f$};
		\node [style=none] (1) at (0, 1) {};
		\node [style=none] (2) at (-1, 4) {};
		\node [style=map] (3) at (0, 2.25) {$\bar {f^\circ}$};
		\node [style=X] (4) at (-0.25, 3.75) {};
		\node [style=X] (5) at (0.25, 3.75) {};
		\node [style=X] (6) at (-0.25, 3) {};
	\end{pgfonlayer}
	\begin{pgfonlayer}{edgelayer}
		\draw [style=simple] (0) to (1.center);
		\draw [in=127, out=-90] (2.center) to (6);
		\draw [in=120, out=-120, looseness=1.25] (3) to (0);
		\draw [in=-75, out=75, looseness=1.25] (0) to (3);
		\draw [in=-90, out=120] (3) to (6);
		\draw [in=-90, out=75] (3) to (5);
		\draw (4) to (6);
	\end{pgfonlayer}
\end{tikzpicture}
=
\begin{tikzpicture}[tikzfig]
	\begin{pgfonlayer}{nodelayer}
		\node [style=none] (0) at (0, 1.5) {};
		\node [style=none] (1) at (-1, 4) {};
		\node [style=X] (2) at (0.25, 3.75) {};
		\node [style=X] (3) at (-0.25, 3) {};
		\node [style=map] (4) at (0, 2) {$f$};
		\node [style=X] (5) at (-0.25, 3.75) {};
	\end{pgfonlayer}
	\begin{pgfonlayer}{edgelayer}
		\draw [in=127, out=-90] (1.center) to (3);
		\draw [style=simple] (4) to (0.center);
		\draw (5) to (3);
		\draw [in=75, out=-90] (2) to (4);
		\draw [in=-90, out=120] (4) to (3);
	\end{pgfonlayer}
\end{tikzpicture}
=
\begin{tikzpicture}[tikzfig]
	\begin{pgfonlayer}{nodelayer}
		\node [style=none] (0) at (0, 1.5) {};
		\node [style=X] (1) at (0.25, 3) {};
		\node [style=map] (2) at (0, 2) {$f$};
		\node [style=none] (3) at (-0.25, 3) {};
		\node [style=none] (4) at (-0.25, 3.5) {};
	\end{pgfonlayer}
	\begin{pgfonlayer}{edgelayer}
		\draw [style=simple] (2) to (0.center);
		\draw [in=75, out=-90] (1) to (2);
		\draw (4.center) to (3.center);
		\draw [in=104, out=-90] (3.center) to (2);
	\end{pgfonlayer}
\end{tikzpicture}
\end{align*}

So that combining the previous two equations:

\begin{align*}
\begin{tikzpicture}[tikzfig]
	\begin{pgfonlayer}{nodelayer}
		\node [style=none] (0) at (0, 1.5) {};
		\node [style=X] (1) at (0.25, 3) {};
		\node [style=map] (2) at (0, 2) {$f$};
		\node [style=none] (3) at (-0.25, 3) {};
		\node [style=none] (4) at (-0.25, 3.5) {};
	\end{pgfonlayer}
	\begin{pgfonlayer}{edgelayer}
		\draw [style=simple] (2) to (0.center);
		\draw [in=75, out=-90] (1) to (2);
		\draw (4.center) to (3.center);
		\draw [in=104, out=-90] (3.center) to (2);
	\end{pgfonlayer}
\end{tikzpicture}
=
\begin{tikzpicture}[tikzfig]
	\begin{pgfonlayer}{nodelayer}
		\node [style=map] (0) at (0, 1.5) {$f$};
		\node [style=X] (1) at (-0.5, 2.5) {};
		\node [style=map] (2) at (0, 3.5) {$f^\circ$};
		\node [style=none] (3) at (0, 0.5) {};
		\node [style=none] (4) at (-1, 4.5) {};
		\node [style=none] (5) at (0, 4.25) {};
		\node [style=X] (6) at (0, 4.25) {};
	\end{pgfonlayer}
	\begin{pgfonlayer}{edgelayer}
		\draw [style=simple] (0) to (3.center);
		\draw [style=simple] (1) to (2);
		\draw [style=simple, in=-90, out=104] (1) to (4.center);
		\draw [style=simple] (5.center) to (2);
		\draw [style=simple, in=-60, out=60, looseness=0.75] (0) to (2);
		\draw [in=117, out=-90] (1) to (0);
	\end{pgfonlayer}
\end{tikzpicture}
=
\begin{tikzpicture}[tikzfig]
	\begin{pgfonlayer}{nodelayer}
		\node [style=map] (0) at (0, 1.5) {$g$};
		\node [style=X] (1) at (-0.5, 2.5) {};
		\node [style=map] (2) at (0, 3.5) {$g^\circ$};
		\node [style=none] (3) at (0, 0.5) {};
		\node [style=none] (4) at (-1, 4.5) {};
		\node [style=none] (5) at (0, 4.25) {};
		\node [style=X] (6) at (0, 4.25) {};
	\end{pgfonlayer}
	\begin{pgfonlayer}{edgelayer}
		\draw [style=simple] (0) to (3.center);
		\draw [style=simple] (1) to (2);
		\draw [style=simple, in=-90, out=104] (1) to (4.center);
		\draw [style=simple] (5.center) to (2);
		\draw [style=simple, in=-60, out=60, looseness=0.75] (0) to (2);
		\draw [in=117, out=-90] (1) to (0);
	\end{pgfonlayer}
\end{tikzpicture}
=
\begin{tikzpicture}[tikzfig]
	\begin{pgfonlayer}{nodelayer}
		\node [style=none] (0) at (0, 1.5) {};
		\node [style=X] (1) at (0.25, 3) {};
		\node [style=map] (2) at (0, 2) {$g$};
		\node [style=none] (3) at (-0.25, 3) {};
		\node [style=none] (4) at (-0.25, 3.5) {};
	\end{pgfonlayer}
	\begin{pgfonlayer}{edgelayer}
		\draw [style=simple] (2) to (0.center);
		\draw [in=75, out=-90] (1) to (2);
		\draw (4.center) to (3.center);
		\draw [in=104, out=-90] (3.center) to (2);
	\end{pgfonlayer}
\end{tikzpicture}
\end{align*}


\end{proof}



\section{\texorpdfstring{$\ZXA$}{ZX\&}}
\label{sec:ZXA}
In this section, we add a unit and counit to the Frobenius algebra in $\TOF$ by glueing its counital completion and unital completion together.  We then give a presentation of this category in terms of the self-dual compact closed prop $\ZXA$ generated by the copy and addition spiders, the not gate and the {\sf and} gate via a two-way translation.

\begin{definition}\cite{tof}
The category $\TOF$ is the prop generated by the Toffoli gate and ancillary bits, satisfying the equations in \S \ref{sec:tof} Figure \ref{fig:TOF}.
\end{definition}

\begin{theorem}\cite{tof}
$\TOF$ is isomorphic to the category of partial isomorphisms between ordinals $2^n$, $n\in \N$.
\end{theorem}



By adding a unit and counit, we obtain a full subcategory of spans of sets and finite ordinals:

\begin{lemma}
\label{lemma:unitcounit}

 The full subcategory of $\Span^\sim(\FinOrd)$ generated by powers of 2 is presented by the pushout,  $\hat \TOF$, of the following diagram of props:

$$c(\TOF)^\op \leftarrow \TOF \rightarrow c(\TOF)$$
\end{lemma}



\begin{proof}
Recall that $\TOF$ is presented by the subcategory $\FPinj_2$ of $(\Span^\sim (\FinOrd),\times)$ with morphisms of the form $ 2^n \xleftarrowtail{e} k \xrightarrowtail{e'} 2^m$ for arbitrary natural numbers $n,m,k$ and monics $e$ and $e'$.



Similarly, $\tilde \TOF$ is presented by the subcategory $\FPar_2$ of  $(\Span^\sim (\FinOrd),\times)$ with morphisms of the form $2^\ell \xleftarrow{f} 2^n \xleftarrowtail{e} k \xrightarrowtail{e'} 2^m$ for arbitrary natural numbers $\ell, n,m,k$ and monics $e$ and $e'$ and function $f$.
Let $\FSpan_2$ denote the full subcategory of $(\Span^\sim(\FinOrd),\times)$ generated by powers of two.
Consider the pushout $\X$ of the following diagram of props:

$$\FPar_2^\op \xleftarrowtail{}  \FPinj_2 \xrightarrowtail{} \FPar_2$$ 


Consider the functor $F:\X\to\FSpan_2$ induced by the universal property of the pushout.  We show that this functor is an isomorphism.
This functor is clearly the identity on objects.

For fullness consider some span $2^n \xleftarrow{f} k \xrightarrow{g} 2^m$. We can construct a function $f':2^{\lceil \log_2 k \rceil} \rightarrow 2^n$ and monic $e_f: k \xrightarrowtail{} 2^{\lceil \log_2 k \rceil}$ so that $f=ef'$.  Similarly, we can construct some  $g':2^{\lceil \log_2 k \rceil} \rightarrow 2^n$ and monic $e_g: k \xrightarrowtail{} 2^{\lceil \log_2 k \rceil}$ so that $g=e_gg'$.  Therefore:


\begin{align*}
F&\left(
\xymatrix{
         & 2^{\lceil \log_2 k \rceil} \ar[dl]_{f'} \ar@{=}[dr]\\
2^n &                                                                                 &2^{\lceil \log_2 k \rceil}
};
\xymatrix{
         & k \ar@{>->}[dl]_{e_f} \ar@{>->}[dr]^{e_m}\\
2^{\lceil \log_2 k \rceil} &                                                                                 & 2^{\lceil \log_2 k \rceil}
};
\xymatrix{
                                       & 2^{\lceil \log_2 k \rceil} \ar[dr]^{g'} \ar@{=}[dl]\\
2^{\lceil \log_2 k \rceil} &                                                                                 & 2^m
}
\right)\\
&=
\xymatrix{
         &                                                                               &                                             &  k \ar@{=}[dl] \ar@{=}[dr] \ar@/_2.0pc/[dddlll]_{f} \ar@/^2.0pc/[dddrrr]^{g}  \\
         &                                                                               &k \ar@{=}[dr] \ar@{>->}[dl]_{e_f} &                                                & k \ar@{=}[dl] \ar@{>->}[dr]^{e_g}\\
         & 2^{\lceil \log_2 k \rceil}\ar[dl]^{f'}\ar@{=}[dr]   &                                             & k \ar@{>->}[dl]_{e_f} \ar@{>->}[dr]^{e_g} &                                       & 2^{\lceil \log_2 k \rceil}\ar[dr]_{g'}\ar@{=}[dl] \\
2^n  &                                                                                & 2^{\lceil \log_2 k \rceil}     &                                                & 2^{\lceil \log_2 k \rceil} &                   & 2^m
}
\end{align*}

So $F$ is full.


For faithfulness suppose we have any two isomorphic spans in $F(\X)$:

$$
\xymatrix{
                 &                                       & k \ar@{>->}[dl]_{e_1}\ar@{->}[dddd]_\cong^{\alpha} \ar@{>->}[dr]^{e_2} \\ 
                 & 2^{n_2} \ar[dl]_{f_1}   &                                                                              & 2^{n_3} \ar[dr]^{f_2}\\ 
2^{n_1}   &                                       &                                                                              &                 & 2^{n_4}\\
                 & 2^{n_2'} \ar[ul]^{f_1'} &                                                                              & 2^{n_3'} \ar[ur]_{f_2'}\\ 
                 &                                       & k \ar@{>->}[ul]^{e_1'} \ar@{>->}[ur]_{e_2'} \\ 
}
$$



In $\X$, we have:

\begin{align*}
\xymatrix{
                & 2^{n_2} \ar[dl]_{f_1} \ar@{=}[dr] \\
2^{n_1} &                                                             & 2^{n_2}
};&
\xymatrix{
               & k \ar@{>->}[dl]_{e_1} \ar@{>->}[dr]^{e_2}\\
2^{n_2} &                                               & 2^{n_3}
};
\xymatrix{
                & 2^{n_3} \ar@{=}[dl] \ar[dr]^{f_2} \\
2^{n_3} &                                                             & 2^{n_4}
}\\
&=
\xymatrix{
                 &                                                           & k \ar@{>->}[dl]_{e_1} \ar@{=}[dr]  \ar@/_2.0pc/[ddll]_{\alpha e_1' f_1'}\\
                & 2^{n_2} \ar[dl]_{f_1} \ar@{=}[dr]   &                         & k \ar@{>->}[dl]_{e_1 } \ar@{>->}[dr]^{e_2}  \\
2^{n_1} &                                                             & 2^{n_2}          &                                                 & 2^{n_3}
};
\xymatrix{
                & 2^{n_3} \ar@{=}[dl] \ar[dr]^{f_2} \\
2^{n_3} &                                                             & 2^{n_4}
}\\
&=
\xymatrix{
                 &                                                           & k \ar@{>->}[dl]_{\alpha e_1'} \ar@{>->}[ddrr]^{e_2} \\
                & 2^{n_2'} \ar[dl]_{f_1'}                        &                         &  \\
2^{n_1} &                                                             &           &                                                 & 2^{n_3}
};
\xymatrix{
                & 2^{n_3} \ar@{=}[dl] \ar[dr]^{f_2} \\
2^{n_3} &                                                             & 2^{n_4}
}\\
&=
\xymatrix{
                & 2^{n_2'} \ar[dl]_{f_1'} \ar@{=}[dr] \\
2^{n_1} &                                                             & 2^{n_2'}
};
\xymatrix{
               & k \ar@{>->}[dl]_{\alpha e_1'} \ar@{>->}[dr]^{e_2}\\
2^{n_2'} &                                               & 2^{n_3}
};
\xymatrix{
                & 2^{n_3} \ar@{=}[dl] \ar[dr]^{f_2} \\
2^{n_3} &                                                             & 2^{n_4}
}\\
&=
\xymatrix{
                & 2^{n_2'} \ar[dl]_{f_1'} \ar@{=}[dr] \\
2^{n_1} &                                                             & 2^{n_2'}
};
\xymatrix{
                                       & k \ar@{>->}[dl]_{\alpha e_1'} \ar@{>->}[dr]^{\alpha e_2'} \ar[dd]_\cong^\alpha\\
2^{n_2'} &                                                                         & 2^{n_3'} \\
                                       & k  \ar@{>->}[ul]^{e_1'} \ar@{>->}[ur]_{e_2'}
};
\xymatrix{
                & 2^{n_3'} \ar@{=}[dl] \ar[dr]^{f_2} \\
2^{n_3'} &                                                             & 2^{n_4}
}\\
&=
\xymatrix{
                & 2^{n_2'} \ar[dl]_{f_1'} \ar@{=}[dr] \\
2^{n_1} &                                                             & 2^{n_2'}
};
\xymatrix{
               & k \ar@{>->}[dl]_{ e_1'} \ar@{>->}[dr]^{ e_2'}\\
2^{n_2'} &                                               & 2^{n_3'}
};
\xymatrix{
                & 2^{n_3'} \ar@{=}[dl] \ar[dr]^{f_2} \\
2^{n_3'} &                                                             & 2^{n_4}
}
\end{align*}

Therefore $\FSpan_2 \cong \X$.


Two show that $\hat \TOF \cong \FSpan_2$, consider the following diagram where each horizontal face is a pushout:


$$
\xymatrixrowsep{6mm}\xymatrixcolsep{4mm}
\xymatrix{
                                       & {(\FPinj_2,\times)} \ar[dl] \ar@/^.5pc/[rr] \ar@{=}[d]  &                                                  & (\FPar_2,\times) \ar[d]^{\cong} \ar[dl] \\
 (\FPar_2,\times)^\op \ar@/_1pc/[rr]  \ar[d]_{\cong}           &                   {(\FPinj_2,\times)}\ar[dl] \ar@/^.5pc/[rr]    \ar[d]^\cong                                                                       & (\FSpan_2,\times)    \ar@{-->}[d]^(.35){\cong}    & \tilde{(\FPinj_2,\times)} \ar[dl]       \ar[d]^\cong       \\
\tilde{(\FPinj_2,\times)}^\op \ar@/_1pc/[rr]            \ar[d]_{\cong}                               &      \TOF \ar[dl] \ar@/^.5pc/[rr]  \ar@{=}[d]       &                                  \ar@{-->}[d]^(.35){\cong}             & \tilde \TOF  \ar[d]_{\cong} \ar[dl]\\
\tilde{\TOF}^\op \ar@/_1pc/[rr]   \ar[d]_{\cong}   &                  \TOF \ar[dl] \ar@/^.5pc/[rr]                                                                      &  \ar@{-->}[d]^(.35){\cong}  & c(\TOF)  \ar[dl]\\
c(\TOF)^\op        \ar@/_1pc/[rr]                          &                                                                                             &          \hat\TOF &                        &            \\
}
$$


All of the rear and left faces commute. Moreover, the vertical maps are isomorphisms, therefore the maps induced by universal property of the pushout are isomorphisms.







\end{proof}



If $f$ is a partial isomorphism between finite sets, then the white spiders correspond to the classical structure for the chosen computational basis.  For the interpretation into $\FHilb$ via the $\ell_2$ functor, this means that in the  qubit case, the unit and counit correspond to $\sqrt{2}|+\rangle$ and $\sqrt{2}\langle +|$. 


We give a more elegant presentation of this category in terms of interacting monoids and %\linebreak[4]
 comonoids:

\begin{definition}
Consider the self dual prop $\ZXA$ generated by the addition spider with phases in $\{0,\pi\}$, the copy spider and the monoid for conjunction satisfying the  identities given in Figure \ref{fig:ZXA}.


\begin{figure}%[t]
	\noindent
	\scalebox{1.0}{%
		\vbox{%
			\begin{mdframed}
				\begin{multicols}{2}
					\begin{enumerate}[label={\bf [ZX{\it \&}.\arabic*]}, ref={\bf [ZX{\it \&}.\arabic*]}, wide = 0pt, leftmargin = 2em]
						\item
						\label{ZXA.1}
						{\hfil
							$
\begin{tikzpicture}[tikzfig]
	\begin{pgfonlayer}{nodelayer}
		\node [style=none] (0) at (0, 0.5) {};
		\node [style=none] (1) at (1, 0.5) {};
		\node [style=none] (2) at (0, 2.75) {};
		\node [style=none] (3) at (1, 2.75) {};
		\node [style=Z] (4) at (0.5, 1.25) {$\alpha$};
		\node [style=Z] (5) at (0.5, 2) {$\beta$};
		\node [style=none] (6) at (0.5, 2.5) {$\vdots$};
		\node [style=none] (7) at (0.5, 0.75) {$\vdots$};
	\end{pgfonlayer}
	\begin{pgfonlayer}{edgelayer}
		\draw [style=simple, in=-56, out=90] (1.center) to (4);
		\draw [style=simple, in=90, out=-124] (4) to (0.center);
		\draw [style=simple, in=-90, out=124] (5) to (2.center);
		\draw [style=simple, in=-90, out=56] (5) to (3.center);
		\draw [style=simple] (5) to (4);
	\end{pgfonlayer}
\end{tikzpicture}
=
\begin{tikzpicture}[tikzfig]
	\begin{pgfonlayer}{nodelayer}
		\node [style=none] (0) at (0, 0.5) {};
		\node [style=none] (1) at (1, 0.5) {};
		\node [style=none] (2) at (0.5, 0.5) {$\vdots$};
		\node [style=none] (3) at (0.5, 2) {$\vdots$};
		\node [style=none] (4) at (1, 2) {};
		\node [style=Z] (5) at (0.5, 1.25) {$\alpha+\beta$};
		\node [style=none] (6) at (0, 2) {};
	\end{pgfonlayer}
	\begin{pgfonlayer}{edgelayer}
		\draw [style=simple, in=56, out=-90] (4.center) to (5);
		\draw [style=simple, in=-90, out=124] (5) to (6.center);
		\draw [style=simple, in=90, out=-124] (5) to (0.center);
		\draw [style=simple, in=-56, out=90] (1.center) to (5);
	\end{pgfonlayer}
\end{tikzpicture}
							$
						}

						\item
						\label{ZXA.2}
						{\hfil
							$
\begin{tikzpicture}[tikzfig]
	\begin{pgfonlayer}{nodelayer}
		\node [style=none] (0) at (0, 2.25) {};
		\node [style=none] (1) at (1, 2.25) {};
		\node [style=Z] (2) at (0.5, 1.5) {$\alpha$};
		\node [style=none] (3) at (0.5, 2) {$\vdots$};
		\node [style=none] (4) at (0.25, 0.5) {};
		\node [style=none] (5) at (0.75, 0.5) {};
		\node [style=none] (6) at (0.75, 1) {};
		\node [style=none] (7) at (0.25, 1) {};
	\end{pgfonlayer}
	\begin{pgfonlayer}{edgelayer}
		\draw [style=simple, in=56, out=-90] (1.center) to (2);
		\draw [style=simple, in=-90, out=124] (2) to (0.center);
		\draw [style=simple, in=90, out=-63] (2) to (6.center);
		\draw [style=simple, in=90, out=-90] (6.center) to (4.center);
		\draw [style=simple, in=-90, out=90] (5.center) to (7.center);
		\draw [style=simple, in=-117, out=90] (7.center) to (2);
	\end{pgfonlayer}
\end{tikzpicture}
=
\begin{tikzpicture}[tikzfig]
	\begin{pgfonlayer}{nodelayer}
		\node [style=none] (0) at (0, 2) {};
		\node [style=none] (1) at (1, 2) {};
		\node [style=Z] (2) at (0.5, 1.25) {$\alpha$};
		\node [style=none] (3) at (0.5, 1.75) {$\vdots$};
		\node [style=none] (4) at (0.25, 0.5) {};
		\node [style=none] (5) at (0.75, 0.5) {};
	\end{pgfonlayer}
	\begin{pgfonlayer}{edgelayer}
		\draw [style=simple, in=56, out=-90] (1.center) to (2);
		\draw [style=simple, in=-90, out=124] (2) to (0.center);
		\draw [style=simple, in=-56, out=90] (5.center) to (2);
		\draw [style=simple, in=-124, out=90] (4.center) to (2);
	\end{pgfonlayer}
\end{tikzpicture}
							$
						}

						\item
						\label{ZXA.3}
						{\hfil
							$
\begin{tikzpicture}[tikzfig]
	\begin{pgfonlayer}{nodelayer}
		\node [style=none] (0) at (0, 0.5) {};
		\node [style=none] (1) at (1, 0.5) {};
		\node [style=none] (2) at (0, 2.75) {};
		\node [style=none] (3) at (1, 2.75) {};
		\node [style=X] (4) at (0.5, 1.25) {};
		\node [style=X] (5) at (0.5, 2) {};
		\node [style=none] (6) at (0.5, 2.5) {$\vdots$};
		\node [style=none] (7) at (0.5, 0.75) {$\vdots$};
		\node [style=none] (8) at (0.45, 1.625) {\scalebox{.8}{$\vdots$}};
	\end{pgfonlayer}
	\begin{pgfonlayer}{edgelayer}
		\draw [style=simple, in=-56, out=90] (1.center) to (4);
		\draw [style=simple, in=90, out=-124] (4) to (0.center);
		\draw [style=simple, in=-135, out=135, looseness=1.25] (4) to (5);
		\draw [style=simple, in=45, out=-45, looseness=1.25] (5) to (4);
		\draw [style=simple, in=-90, out=124] (5) to (2.center);
		\draw [style=simple, in=-90, out=56] (5) to (3.center);
	\end{pgfonlayer}
\end{tikzpicture}
=
\begin{tikzpicture}[tikzfig]
	\begin{pgfonlayer}{nodelayer}
		\node [style=none] (0) at (0, 0.5) {};
		\node [style=none] (1) at (1, 0.5) {};
		\node [style=X] (2) at (0.5, 1.25) {};
		\node [style=none] (3) at (0.5, 0.75) {$\vdots$};
		\node [style=none] (4) at (0.5, 1.75) {$\vdots$};
		\node [style=none] (5) at (1, 2) {};
		\node [style=X] (6) at (0.5, 1.25) {};
		\node [style=none] (7) at (0, 2) {};
	\end{pgfonlayer}
	\begin{pgfonlayer}{edgelayer}
		\draw [style=simple, in=-56, out=90] (1.center) to (2);
		\draw [style=simple, in=90, out=-124] (2) to (0.center);
		\draw [style=simple, in=56, out=-90] (5.center) to (6);
		\draw [style=simple, in=-90, out=124] (6) to (7.center);
	\end{pgfonlayer}
\end{tikzpicture}
							$
						}



						\item
						\label{ZXA.4}
						{\hfil
							$
\begin{tikzpicture}[tikzfig]
	\begin{pgfonlayer}{nodelayer}
		\node [style=none] (0) at (0, 0.5) {};
		\node [style=none] (1) at (1, 0.5) {};
		\node [style=X] (2) at (0.5, 1.25) {};
		\node [style=none] (3) at (0.5, 0.75) {$\vdots$};
		\node [style=none] (4) at (0.25, 2.25) {};
		\node [style=none] (5) at (0.75, 2.25) {};
		\node [style=none] (6) at (0.75, 1.75) {};
		\node [style=none] (7) at (0.25, 1.75) {};
	\end{pgfonlayer}
	\begin{pgfonlayer}{edgelayer}
		\draw [style=simple, in=-56, out=90] (1.center) to (2);
		\draw [style=simple, in=90, out=-124] (2) to (0.center);
		\draw [style=simple, in=-90, out=63] (2) to (6.center);
		\draw [style=simple, in=-90, out=90] (6.center) to (4.center);
		\draw [style=simple, in=90, out=-90] (5.center) to (7.center);
		\draw [style=simple, in=117, out=-90] (7.center) to (2);
	\end{pgfonlayer}
\end{tikzpicture}
=
\begin{tikzpicture}[tikzfig]
	\begin{pgfonlayer}{nodelayer}
		\node [style=none] (0) at (0, 0.5) {};
		\node [style=none] (1) at (1, 0.5) {};
		\node [style=X] (2) at (0.5, 1.25) {};
		\node [style=none] (3) at (0.5, 0.75) {$\vdots$};
		\node [style=none] (4) at (0.25, 2) {};
		\node [style=none] (5) at (0.75, 2) {};
	\end{pgfonlayer}
	\begin{pgfonlayer}{edgelayer}
		\draw [style=simple, in=-56, out=90] (1.center) to (2);
		\draw [style=simple, in=90, out=-124] (2) to (0.center);
		\draw [style=simple, in=56, out=-90] (5.center) to (2);
		\draw [style=simple, in=124, out=-90] (4.center) to (2);
	\end{pgfonlayer}
\end{tikzpicture}
							$
						}
						
						
						\item
						\label{ZXA.5}
						{\hfil
							$
\begin{tikzpicture}[tikzfig]
	\begin{pgfonlayer}{nodelayer}
		\node [style=Z] (0) at (-1, 1) {};
		\node [style=none] (1) at (-1.25, 0.5) {};
		\node [style=none] (2) at (-0.75, 0.5) {};
		\node [style=X] (3) at (-1, 1.75) {};
		\node [style=none] (4) at (-1.25, 2.25) {};
		\node [style=none] (5) at (-0.75, 2.25) {};
	\end{pgfonlayer}
	\begin{pgfonlayer}{edgelayer}
		\draw [in=63, out=-90] (5.center) to (3);
		\draw (3) to (0);
		\draw [in=90, out=-117] (0) to (1.center);
		\draw [in=-63, out=90] (2.center) to (0);
		\draw [in=-90, out=117] (3) to (4.center);
	\end{pgfonlayer}
\end{tikzpicture}
=
\begin{tikzpicture}
	\begin{pgfonlayer}{nodelayer}
		\node [style=X] (0) at (1, 1) {};
		\node [style=X] (1) at (1, 0.25) {};
		\node [style=Z] (2) at (1.75, 0.25) {};
		\node [style=Z] (3) at (1.75, 1) {};
		\node [style=none] (4) at (2.25, 1) {};
		\node [style=none] (5) at (2.25, 0.25) {};
		\node [style=none] (6) at (0.5, 1) {};
		\node [style=none] (7) at (0.5, 0.25) {};
	\end{pgfonlayer}
	\begin{pgfonlayer}{edgelayer}
		\draw (7.center) to (1);
		\draw (1) to (3);
		\draw [in=30, out=150, looseness=1.25] (3) to (0);
		\draw (0) to (2);
		\draw (2) to (5.center);
		\draw [in=-30, out=-150, looseness=1.25] (2) to (1);
		\draw (0) to (6.center);
		\draw (3) to (4.center);
	\end{pgfonlayer}
\end{tikzpicture}
							$
						}
						
											
\item
	\label{ZXA.6}

	{\hfil\hspace*{.5cm}
							$
\begin{tikzpicture}[tikzfig]
	\begin{pgfonlayer}{nodelayer}
		\node [style=none] (0) at (-0.25, 2) {};
		\node [style=X] (1) at (0, 1.25) {};
		\node [style=Z] (2) at (0, 0.5) {};
		\node [style=none] (3) at (0.25, 2) {};
	\end{pgfonlayer}
	\begin{pgfonlayer}{edgelayer}
		\draw [style=simple, in=-90, out=124] (1) to (0.center);
		\draw [style=simple, in=60, out=-90] (3.center) to (1);
		\draw [style=simple] (1) to (2);
	\end{pgfonlayer}
\end{tikzpicture}
=
\begin{tikzpicture}[tikzfig]
	\begin{pgfonlayer}{nodelayer}
		\node [style=none] (0) at (-0.25, 1) {};
		\node [style=Z] (1) at (-0.25, 0.5) {};
		\node [style=none] (2) at (0.25, 1) {};
		\node [style=Z] (3) at (0.25, 0.5) {};
	\end{pgfonlayer}
	\begin{pgfonlayer}{edgelayer}
		\draw [style=simple] (3) to (2.center);
		\draw [style=simple] (1) to (0.center);
	\end{pgfonlayer}
\end{tikzpicture}
							$
						}
						
						\item
						\label{ZXA.7}
						{\hfil
							$
\begin{tikzpicture}[tikzfig]
	\begin{pgfonlayer}{nodelayer}
		\node [style=Z] (0) at (0, 4.5) {};
		\node [style=X] (1) at (0, 5.25) {};
	\end{pgfonlayer}
	\begin{pgfonlayer}{edgelayer}
		\draw (1) to (0);
	\end{pgfonlayer}
\end{tikzpicture}
=
\begin{tikzpicture}
	\begin{pgfonlayer}{nodelayer}
	\end{pgfonlayer}
\end{tikzpicture}
							$
						}
						
						
							\item
						\label{ZXA.8}
						{\hfil
							$
\begin{tikzpicture}[tikzfig]
	\begin{pgfonlayer}{nodelayer}
		\node [style=Z] (0) at (-1, 3) {};
		\node [style=X] (1) at (-1, 2.25) {};
		\node [style=none] (2) at (-1, 3.5) {};
		\node [style=none] (3) at (-1, 1.75) {};
	\end{pgfonlayer}
	\begin{pgfonlayer}{edgelayer}
		\draw (2.center) to (0);
		\draw (1) to (3.center);
	\end{pgfonlayer}
\end{tikzpicture}
=
\begin{tikzpicture}[tikzfig]
	\begin{pgfonlayer}{nodelayer}
		\node [style=Z] (0) at (-1, 3) {};
		\node [style=X] (1) at (-1, 2.25) {};
		\node [style=none] (2) at (-1, 3.5) {};
		\node [style=none] (3) at (-1, 1.75) {};
	\end{pgfonlayer}
	\begin{pgfonlayer}{edgelayer}
		\draw (2.center) to (0);
		\draw [in=120, out=-120, looseness=1.25] (0) to (1);
		\draw [in=-60, out=60, looseness=1.25] (1) to (0);
		\draw (1) to (3.center);
	\end{pgfonlayer}
\end{tikzpicture}
							$
						}

\item
						\label{ZXA.9}
						{\hfil
							$
\begin{tikzpicture}[tikzfig]
	\begin{pgfonlayer}{nodelayer}
		\node [style=none] (0) at (0, 3) {};
		\node [style=none] (1) at (0, 3.5) {};
		\node [style=none] (2) at (0, 2.25) {};
		\node [style=none] (3) at (-0.25, 1.5) {};
		\node [style=none] (4) at (0.25, 1.5) {};
		\node [style=none] (5) at (0.5, 1.5) {};
		\node [style=none] (6) at (1, 1.5) {};
		\node [style=none] (7) at (-0.5, 1.5) {};
		\node [style=none] (8) at (-1, 1.5) {};
		\node [style=none] (9) at (-0.8, 1.7) {$\vdots$};
		\node [style=none] (10) at (-0.055, 1.7) {$\vdots$};
		\node [style=none] (11) at (0.68, 1.7) {$\vdots$};
		\node [style=andin] (12) at (0, 2.25) {};
		\node [style=andin] (13) at (0, 3) {};
	\end{pgfonlayer}
	\begin{pgfonlayer}{edgelayer}
		\draw [style=simple, in=-90, out=90] (0.center) to (1.center);
		\draw [style=simple, in=-90, out=90] (2.center) to (0.center);
		\draw [style=simple, in=-63, out=90] (4.center) to (2.center);
		\draw [style=simple, in=90, out=-117] (2.center) to (3.center);
		\draw [style=simple, in=-120, out=90] (7.center) to (0.center);
		\draw [style=simple, in=90, out=-135] (0.center) to (8.center);
		\draw [style=simple, in=-60, out=90] (5.center) to (0.center);
		\draw [style=simple, in=-45, out=90] (6.center) to (0.center);
	\end{pgfonlayer}
\end{tikzpicture}
=
\begin{tikzpicture}[tikzfig]
	\begin{pgfonlayer}{nodelayer}
		\node [style=andin] (0) at (0, 3) {};
		\node [style=none] (1) at (1, 1.5) {};
		\node [style=none] (2) at (0.25, 1.5) {};
		\node [style=none] (3) at (-0.5, 1.5) {};
		\node [style=none] (4) at (-1, 1.5) {};
		\node [style=none] (5) at (0.5, 1.5) {};
		\node [style=none] (6) at (-0.8, 1.7) {$\vdots$};
		\node [style=none] (7) at (0, 3) {};
		\node [style=none] (8) at (0.68, 1.7) {$\vdots$};
		\node [style=none] (9) at (-0.055, 1.7) {$\vdots$};
		\node [style=none] (10) at (-0.25, 1.5) {};
		\node [style=none] (11) at (0, 3.5) {};
		\node [style=none] (12) at (0, 3) {};
	\end{pgfonlayer}
	\begin{pgfonlayer}{edgelayer}
		\draw [style=simple, in=-45, out=90] (1.center) to (7.center);
		\draw [style=simple, in=90, out=-135] (7.center) to (4.center);
		\draw [style=simple, in=90, out=-105] (12.center) to (10.center);
		\draw [style=simple, in=-120, out=90] (3.center) to (7.center);
		\draw [style=simple, in=-60, out=90] (5.center) to (7.center);
		\draw [style=simple, in=-75, out=90] (2.center) to (12.center);
		\draw [style=simple, in=-90, out=90] (7.center) to (11.center);
		\draw [style=simple, in=-90, out=90] (12.center) to (7.center);
	\end{pgfonlayer}
\end{tikzpicture}
							$
						}



						\item
						\label{ZXA.10}
						{\hfil
							$
\begin{tikzpicture}[tikzfig]
	\begin{pgfonlayer}{nodelayer}
		\node [style=andin] (0) at (-1, 2) {};
		\node [style=none] (1) at (-1, 2.5) {};
		\node [style=none] (2) at (-0.75, 1.25) {};
		\node [style=Z] (3) at (-1.25, 1.25) {$\pi$};
	\end{pgfonlayer}
	\begin{pgfonlayer}{edgelayer}
		\draw (0) to (1.center);
		\draw [in=90, out=-108] (0) to (3);
		\draw [in=-72, out=90] (2.center) to (0);
	\end{pgfonlayer}
\end{tikzpicture}
=
\begin{tikzpicture}[tikzfig]
	\begin{pgfonlayer}{nodelayer}
		\node [style=none] (0) at (-1, 2.5) {};
		\node [style=none] (1) at (-1, 1.75) {};
	\end{pgfonlayer}
	\begin{pgfonlayer}{edgelayer}
		\draw (0.center) to (1.center);
	\end{pgfonlayer}
\end{tikzpicture}
							$
						}

						\item
						\label{ZXA.11}
						{\hfil
							$
\begin{tikzpicture}[tikzfig]
	\begin{pgfonlayer}{nodelayer}
		\node [style=andin] (0) at (0, 2.5) {};
		\node [style=none] (1) at (-0.25, 2) {};
		\node [style=none] (2) at (0.25, 2) {};
		\node [style=none] (3) at (0, 3) {};
		\node [style=none] (4) at (-0.25, 1.5) {};
		\node [style=none] (5) at (0.25, 1.5) {};
	\end{pgfonlayer}
	\begin{pgfonlayer}{edgelayer}
		\draw [in=-63, out=90] (2.center) to (0);
		\draw [in=90, out=-117, looseness=1.25] (0) to (1.center);
		\draw (3.center) to (0);
		\draw [in=-90, out=90, looseness=1.25] (5.center) to (1.center);
		\draw [in=90, out=-90, looseness=1.25] (2.center) to (4.center);
	\end{pgfonlayer}
\end{tikzpicture}
=
\begin{tikzpicture}[tikzfig]
	\begin{pgfonlayer}{nodelayer}
		\node [style=andin] (0) at (0, 2.5) {};
		\node [style=none] (1) at (-0.25, 2) {};
		\node [style=none] (2) at (0.25, 2) {};
		\node [style=none] (3) at (0, 3) {};
	\end{pgfonlayer}
	\begin{pgfonlayer}{edgelayer}
		\draw [in=-63, out=90] (2.center) to (0);
		\draw [in=90, out=-117, looseness=1.25] (0) to (1.center);
		\draw (3.center) to (0);
	\end{pgfonlayer}
\end{tikzpicture}
							$
						}

						\item
						\label{ZXA.12}
						{\hfil
							$
\begin{tikzpicture}[tikzfig]
	\begin{pgfonlayer}{nodelayer}
		\node [style=andin] (0) at (-1, 1) {};
		\node [style=none] (1) at (-1.25, 0.5) {};
		\node [style=none] (2) at (-0.75, 0.5) {};
		\node [style=X] (3) at (-1, 1.75) {};
		\node [style=none] (4) at (-1.25, 2.25) {};
		\node [style=none] (5) at (-0.75, 2.25) {};
	\end{pgfonlayer}
	\begin{pgfonlayer}{edgelayer}
		\draw [in=63, out=-90] (5.center) to (3);
		\draw (3) to (0);
		\draw [in=90, out=-117] (0) to (1.center);
		\draw [in=-63, out=90] (2.center) to (0);
		\draw [in=-90, out=117] (3) to (4.center);
	\end{pgfonlayer}
\end{tikzpicture}
=
\begin{tikzpicture}[tikzfig]
	\begin{pgfonlayer}{nodelayer}
		\node [style=X] (0) at (-1, 1) {};
		\node [style=X] (1) at (-0.25, 1) {};
		\node [style=andin] (2) at (-0.25, 2) {};
		\node [style=andin] (3) at (-1, 2) {};
		\node [style=none] (4) at (-1, 2.5) {};
		\node [style=none] (5) at (-0.25, 2.5) {};
		\node [style=none] (6) at (-1, 0.5) {};
		\node [style=none] (7) at (-0.25, 0.5) {};
	\end{pgfonlayer}
	\begin{pgfonlayer}{edgelayer}
		\draw (7.center) to (1);
		\draw [in=-60, out=127] (1) to (3);
		\draw [in=120, out=-120, looseness=1.25] (3) to (0);
		\draw [in=-120, out=53] (0) to (2);
		\draw (2) to (5.center);
		\draw [in=60, out=-60, looseness=1.25] (2) to (1);
		\draw (0) to (6.center);
		\draw (3) to (4.center);
	\end{pgfonlayer}
\end{tikzpicture}
							$
						}


					\item
					\label{ZXA.13}
						{\hfil
							$
\begin{tikzpicture}[tikzfig]
	\begin{pgfonlayer}{nodelayer}
		\node [style=none] (0) at (-0.5, 4.25) {};
		\node [style=none] (1) at (0, 3.5) {};
		\node [style=none] (2) at (-1, 3.5) {};
		\node [style=X] (3) at (-0.5, 5) {};
		\node [style=andin] (4) at (-0.5, 4.25) {};
	\end{pgfonlayer}
	\begin{pgfonlayer}{edgelayer}
		\draw [in=90, out=-135] (0.center) to (2.center);
		\draw [in=-41, out=90] (1.center) to (0.center);
		\draw (3) to (0.center);
	\end{pgfonlayer}
\end{tikzpicture}
		=
\begin{tikzpicture}[tikzfig]
	\begin{pgfonlayer}{nodelayer}
		\node [style=none] (0) at (-0.5, 3.5) {};
		\node [style=none] (1) at (-1, 3.5) {};
		\node [style=X] (2) at (-1, 4.25) {};
		\node [style=X] (3) at (-0.5, 4.25) {};
	\end{pgfonlayer}
	\begin{pgfonlayer}{edgelayer}
		\draw (3) to (0.center);
		\draw (2) to (1.center);
	\end{pgfonlayer}
\end{tikzpicture}
$
						}

						\item
						\label{ZXA.14}
						{\hfil
							$
\begin{tikzpicture}[tikzfig]
	\begin{pgfonlayer}{nodelayer}
		\node [style=none] (0) at (-0.25, 2) {};
		\node [style=X] (1) at (0, 1.25) {};
		\node [style=Z] (2) at (0, 0.5) {$\pi$};
		\node [style=none] (3) at (0.25, 2) {};
	\end{pgfonlayer}
	\begin{pgfonlayer}{edgelayer}
		\draw [style=simple, in=-90, out=124] (1) to (0.center);
		\draw [style=simple, in=60, out=-90] (3.center) to (1);
		\draw [style=simple] (1) to (2);
	\end{pgfonlayer}
\end{tikzpicture}
=
\begin{tikzpicture}[tikzfig]
	\begin{pgfonlayer}{nodelayer}
		\node [style=none] (0) at (-0.25, 1) {};
		\node [style=Z] (1) at (-0.25, 0.5) {$\pi$};
		\node [style=none] (2) at (0.25, 1) {};
		\node [style=Z] (3) at (0.25, 0.5) {$\pi$};
	\end{pgfonlayer}
	\begin{pgfonlayer}{edgelayer}
		\draw [style=simple] (3) to (2.center);
		\draw [style=simple] (1) to (0.center);
	\end{pgfonlayer}
\end{tikzpicture}
							$
						}
						
						
						

					

						\item
						\label{ZXA.15}
						{\hfil
							$
\begin{tikzpicture}[tikzfig]
	\begin{pgfonlayer}{nodelayer}
		\node [style=none] (0) at (-1, 2) {};
		\node [style=none] (1) at (-1, 1) {};
	\end{pgfonlayer}
	\begin{pgfonlayer}{edgelayer}
		\draw (0.center) to (1.center);
	\end{pgfonlayer}
\end{tikzpicture}
=
\begin{tikzpicture}[tikzfig]
	\begin{pgfonlayer}{nodelayer}
		\node [style=X] (0) at (-1, 1.5) {};
		\node [style=andin] (1) at (-1, 2.5) {};
		\node [style=none] (2) at (-1, 3) {};
		\node [style=none] (3) at (-1, 1) {};
	\end{pgfonlayer}
	\begin{pgfonlayer}{edgelayer}
		\draw (2.center) to (1);
		\draw [in=120, out=-120, looseness=1.25] (1) to (0);
		\draw [in=-60, out=60, looseness=1.25] (0) to (1);
		\draw (0) to (3.center);
	\end{pgfonlayer}
\end{tikzpicture}
							$
						}


						\item
						\label{ZXA.16}
						{\hfil
							$
\begin{tikzpicture}[tikzfig]
	\begin{pgfonlayer}{nodelayer}
		\node [style=andin] (1) at (-1, 2) {};
		\node [style=Z] (2) at (-1, 2.75) {$\pi$};
		\node [style=none] (3) at (-1.25, 1.25) {};
		\node [style=none] (4) at (-0.75, 1.25) {};
	\end{pgfonlayer}
	\begin{pgfonlayer}{edgelayer}
		\draw [in=90, out=-108] (1) to (3.center);
		\draw [in=-72, out=90] (4.center) to (1);
		\draw (1) to (2);
	\end{pgfonlayer}
\end{tikzpicture}
=
\begin{tikzpicture}[tikzfig]
	\begin{pgfonlayer}{nodelayer}
		\node [style=Z] (2) at (-1.25, 2) {$\pi$};
		\node [style=none] (3) at (-1.25, 1.25) {};
		\node [style=none] (4) at (-0.75, 1.25) {};
		\node [style=Z] (5) at (-0.75, 2) {$\pi$};
	\end{pgfonlayer}
	\begin{pgfonlayer}{edgelayer}
		\draw (5) to (4.center);
		\draw (3.center) to (2);
	\end{pgfonlayer}
\end{tikzpicture}
							$
						}

						\item
						\label{ZXA.17}
						{\hfil
							$
\begin{tikzpicture}[tikzfig]
	\begin{pgfonlayer}{nodelayer}
		\node [style=Z] (3) at (0, 3) {};
		\node [style=andin] (4) at (-0.25, 3.75) {};
		\node [style=none] (5) at (-0.5, 3) {};
		\node [style=none] (6) at (-0.25, 2.5) {};
		\node [style=none] (7) at (0.25, 2.5) {};
		\node [style=none] (8) at (-0.5, 2.5) {};
		\node [style=none] (9) at (-0.25, 4.25) {};
	\end{pgfonlayer}
	\begin{pgfonlayer}{edgelayer}
		\draw [in=-72, out=90] (3) to (4);
		\draw (4) to (9.center);
		\draw [in=90, out=-108] (4) to (5.center);
		\draw (5.center) to (8.center);
		\draw [in=90, out=-117] (3) to (6.center);
		\draw [in=90, out=-63] (3) to (7.center);
	\end{pgfonlayer}
\end{tikzpicture}
=
\begin{tikzpicture}[tikzfig]
	\begin{pgfonlayer}{nodelayer}
		\node [style=none] (4) at (0.25, 3) {};
		\node [style=andin] (5) at (-0.35, 3.75) {};
		\node [style=none] (6) at (-0.25, 3) {};
		\node [style=andin] (7) at (0.35, 3.75) {};
		\node [style=none] (8) at (-0.25, 3) {};
		\node [style=X] (9) at (-0.25, 3) {};
		\node [style=Z] (10) at (0, 4.5) {};
		\node [style=none] (11) at (0, 5) {};
		\node [style=none] (12) at (-0.25, 2.5) {};
		\node [style=none] (13) at (0.5, 2.5) {};
		\node [style=none] (14) at (0.25, 2.5) {};
	\end{pgfonlayer}
	\begin{pgfonlayer}{edgelayer}
		\draw [in=-72, out=90] (4.center) to (5);
		\draw [in=120, out=-108] (5) to (6.center);
		\draw [in=45, out=-108] (7) to (8.center);
		\draw (4.center) to (14.center);
		\draw (12.center) to (6.center);
		\draw [in=-117, out=90] (5) to (10);
		\draw (10) to (11.center);
		\draw [in=90, out=-63] (10) to (7);
		\draw [in=-75, out=90, looseness=1.25] (13.center) to (7);
	\end{pgfonlayer}
\end{tikzpicture}
							$
						}

						

						
						
	


						
					\end{enumerate}
				\end{multicols}
				\
			\end{mdframed}
	}}
	\caption{The identities of \texorpdfstring{$\ZXA$}{ZX\&}, where \texorpdfstring{$\alpha,\beta \in \{0,\pi\}$}{alpha and beta are either 0 or pi} and a blank grey spider has angle 0.}
	\label{fig:ZXA}
\end{figure}

\end{definition}
One can interpret the generators as logical connectives and open wires as variables, similar to the regular logic \cite{butz}, or the logic of a Cartesian bicategory \cite{carboni}, except we forget the 2-cells in $\ZXA$.  The decorated black spiders correspond to fixed variables and xor.  White (co)multiplications (co)copy variables; the white unit is existential quantification and the counit is discarding. The relations are open $\Sigma_1$ Boolean formulas augmented with copying and discarding as well as duals; the open variables correspond to distinguished inputs and outputs.




 % However, some axioms such as \ref{ZXA.16}, \ref{ZXA.13}, are tautologies without postselection.
%Interestingly, \ref{ZXA.17} is witnessing that the and gate is a morphism in the two silded Kleisli category of the distributive law between the black and white spider.
%\ref{ZXA.1}-\ref{ZXA.4}, \ref{ZXA.9} are structural.
%\ref{ZXA.14} asserts that negation commutes with copying and discarding.
%\ref{ZXA.10} asserts that $(\top\wedge x) \iff x$. 
%\ref{ZXA.5} asserts that copying commutes with addition.
%\ref{ZXA.12} asserts that copying commutes with conjunction
%\ref{ZXA.8} is asserting $(x+x)=y \iff y= \bot$, or equivalently, that addition has characteristic 2.
%\ref{ZXA.15} is asserting $(x\wedge x) =y \iff y=x$. 
%\ref{ZXA.16} is asserting $(x\wedge y) = \top \iff (x=\top) \wedge (y=\top)$.
%\ref{ZXA.17} is asserting $x\wedge (y+z) \iff x\wedge y+x\wedge z$, ie. that multiplication distributes over addition.
%\ref{ZXA.7} is asserting the tautology $\top=\top \iff \top$
%%\ref{ZXA.13old} is asserting $(\bot \wedge x) = y\iff \bot= y$\\
%\ref{ZXA.13} is asserting that discarding $x\wedge y$ is the same as discarding $x$ and discarding $y$; or that the and gate is causal.
%Note that existentially quantifying and then discarding is not a tautology, rather it is the dimension, 2.

The identities of $\ZXA$ can also be interpreted by freely taking the coproduct of the free prop of commutative (co)monoids \dag-PROP $3\times 2$ times, modulo various (undirected) distributive laws, and monoid maps.  The distributive laws are summarized in Figure \ref{fig:table} (the duals under diagonal are omitted). Te spider rules implicitly identify the (co)units of the \dag-compact closed structure induced by $Z$ and $X$; which is needed for completeness.
%
%\begin{figure}[h]
%
%\begin{minipage}[b]{\textwidth}
%\setlength\dashlinedash{0.2pt}
%\setlength\dashlinegap{1.5pt}
%\setlength\arrayrulewidth{0.3pt}
%\resizebox{\textwidth}{!}{%
%\begin{tabular}{l|l:l:l:p{45mm}:p{38mm}:l}
%    $\lambda$    & $Z$    & $X$    & $\&$      & $Z^\dag$                                        & $X^\dag$                           & $\&^\dag$\\ \hline
%$Z$       & Comm. monoid &        &           & \mbox{Extra special comm.}\linebreak[4] \mbox{\dag-Frobenius algebra} &                                   Hopf algebra with $s=1$ &  Special bialgebra  \\ \hdashline
%$X$       & \bcell & Comm. monoid &           & Hopf algebra with $s=1$              &  \mbox{Comm. \dag-Frobenius}\linebreak[4] \mbox{algebra} &         \\ \hdashline
%$\&$      & \bcell & \bcell & Comm. monoid    & Special bialgebra                                       &                                    &         \\ \hdashline
%$Z^\dag$  & \bcell & \bcell &   \bcell  & Cocomm. comonoid                                          &                                    &         \\ \hdashline
%$X^\dag$  & \bcell & \bcell &   \bcell  &             \bcell                              & Cocomm. comonoid                             &         \\ \hdashline
%$\&^\dag$ & \bcell & \bcell &   \bcell  &              \bcell                             &           \bcell                   &  Cocomm. comonoid       \\
%\end{tabular}
%}
%\end{minipage}
%\caption{Generating distributive laws of \texorpdfstring{$\ZXA$}{ZX\&}.}
%\label{fig:table}
%\end{figure}






\begin{figure}[H]

\begin{minipage}[b]{\textwidth}
\setlength\dashlinedash{0.2pt}
\setlength\dashlinegap{1.5pt}
\setlength\arrayrulewidth{0.3pt}
\resizebox{\textwidth}{!}{%
\begin{tabular}{l|l:l:l:p{45mm}:p{38mm}:l}
    $\lambda$    & $Z$    & $X$    & $\&$      & $Z^\dag$                                        & $X^\dag$                           & $\&^\dag$\\ \hline
$Z$       & Comm. monoid &        &           & \noindent\begin{tabular}{@{}l} Extra special comm.\\ \dag-Frobenius algebra\end{tabular} &                                   Hopf algebra with $s=1$ &  Special bialgebra  \\ \hdashline
$X$       & \bcell & Comm. monoid &           & Hopf algebra with $s=1$              &  \noindent\begin{tabular}{@{}l} Comm. \dag-Frobenius\\ algebra \end{tabular}&         \\ \hdashline
$\&$      & \bcell & \bcell & Comm. monoid    & Special bialgebra                                       &                                    &         \\ \hdashline
$Z^\dag$  & \bcell & \bcell &   \bcell  & Cocomm. comonoid                                          &                                    &         \\ \hdashline
$X^\dag$  & \bcell & \bcell &   \bcell  &             \bcell                              & Cocomm. comonoid                             &         \\ \hdashline
$\&^\dag$ & \bcell & \bcell &   \bcell  &              \bcell                             &           \bcell                   &  Cocomm. comonoid       \\
\end{tabular}
}
\end{minipage}
\caption{Generating distributive laws of \texorpdfstring{$\ZXA$}{ZX\&}.}
\label{fig:table}
\end{figure}


Additionally, \ref{ZXA.16} states that the counit of $\&^\dag$ is copied by $\&$; ie. the counit is a monad map from $\&$ to the trivial monad.  
\ref{ZXA.17} expreses the multiplication part of the distributive law of Lawvere theories between the props for multiplication and addition mod 2 (see \cite{lawvere} for distributive laws of Lawvere theories).

%
%$$
%\hat \&:=
%\begin{tikzpicture}
%	\begin{pgfonlayer}{nodelayer}
%		\node [style=none] (0) at (1.5, -0.75) {};
%		\node [style=none] (1) at (1.5, -1.25) {};
%		\node [style=none] (2) at (2.25, -1.25) {};
%		\node [style=none] (3) at (2.25, -0.75) {};
%		\node [style=none] (4) at (0.5, -0.75) {};
%		\node [style=andin] (5) at (1.5, -0.75) {};
%	\end{pgfonlayer}
%	\begin{pgfonlayer}{edgelayer}
%		\draw [style=simple] (3.center) to (0.center);
%		\draw [style=simple] (2.center) to (1.center);
%		\draw [style=simple, in=-165, out=180, looseness=2.75] (1.center) to (0.center);
%		\draw [style=simple] (4.center) to (0.center);
%	\end{pgfonlayer}
%\end{tikzpicture}
%\hspace*{.2cm}
%\text{where}
%\hspace*{.2cm}
%\begin{tikzpicture}
%	\begin{pgfonlayer}{nodelayer}
%		\node [style=none] (0) at (1.5, -0.75) {};
%		\node [style=none] (1) at (1.5, -1.25) {};
%		\node [style=none] (2) at (2.25, -1.25) {};
%		\node [style=none] (3) at (2.25, -0.75) {};
%		\node [style=none] (4) at (0.5, -0.75) {};
%		\node [style=andin] (5) at (1.5, -0.75) {};
%		\node [style=Z] (6) at (0.5, -0.75) {};
%		\node [style=none] (7) at (-0.25, -1.25) {};
%		\node [style=none] (8) at (-0.25, -0.25) {};
%	\end{pgfonlayer}
%	\begin{pgfonlayer}{edgelayer}
%		\draw [style=simple, in=0, out=180, looseness=1.75] (3.center) to (0.center);
%		\draw [style=simple] (2.center) to (1.center);
%		\draw [style=simple, in=-165, out=180, looseness=2.50] (1.center) to (0.center);
%		\draw [in=-146, out=0, looseness=1.00] (7.center) to (4.center);
%		\draw [in=0, out=146, looseness=1.00] (4.center) to (8.center);
%		\draw [style=simple] (4.center) to (0.center);
%	\end{pgfonlayer}
%\end{tikzpicture}
%\eq{\ref{ZXA.17}}
%\begin{tikzpicture}
%	\begin{pgfonlayer}{nodelayer}
%		\node [style=none] (0) at (1.5, -0.75) {};
%		\node [style=none] (1) at (2.5, -1.5) {};
%		\node [style=none] (2) at (2.5, -0.75) {};
%		\node [style=none] (3) at (0.5, -0.75) {};
%		\node [style=andin] (4) at (1.5, -0.75) {};
%		\node [style=none] (5) at (1.5, -2) {};
%		\node [style=andin] (6) at (1.5, -1.5) {};
%		\node [style=none] (7) at (1.5, -1.5) {};
%		\node [style=none] (8) at (2.5, -1.5) {};
%		\node [style=none] (9) at (2.5, -0.75) {};
%		\node [style=none] (10) at (3, -0.75) {};
%		\node [style=none] (11) at (3, -1.5) {};
%		\node [style=Z] (12) at (2.5, -0.75) {};
%		\node [style=X] (13) at (2.5, -1.5) {};
%		\node [style=none] (14) at (0.5, -1.5) {};
%	\end{pgfonlayer}
%	\begin{pgfonlayer}{edgelayer}
%		\draw [style=simple, in=0, out=135, looseness=1.50] (2.center) to (0.center);
%		\draw [style=simple] (3.center) to (0.center);
%		\draw [style=simple, in=0, out=-135, looseness=1.50] (9.center) to (7.center);
%		\draw [style=simple, in=0, out=-150, looseness=1.00] (8.center) to (5.center);
%		\draw [style=simple, in=-165, out=180, looseness=3.25] (5.center) to (7.center);
%		\draw (11.center) to (1.center);
%		\draw (10.center) to (2.center);
%		\draw [style=simple, in=-165, out=150, looseness=2.25] (1.center) to (0.center);
%		\draw [style=simple] (14.center) to (7.center);
%	\end{pgfonlayer}
%\end{tikzpicture}
%,
%\begin{tikzpicture}
%	\begin{pgfonlayer}{nodelayer}
%		\node [style=none] (0) at (1.5, -0.75) {};
%		\node [style=none] (1) at (1.5, -1.25) {};
%		\node [style=none] (2) at (2.25, -1.25) {};
%		\node [style=none] (3) at (2.25, -0.75) {};
%		\node [style=Z] (4) at (0.5, -0.75) {};
%		\node [style=andin] (5) at (1.5, -0.75) {};
%		\node [style=none] (6) at (1, -0.75) {};
%	\end{pgfonlayer}
%	\begin{pgfonlayer}{edgelayer}
%		\draw [style=simple, in=0, out=180, looseness=1.75] (3.center) to (0.center);
%		\draw [style=simple] (2.center) to (1.center);
%		\draw [style=simple, in=-165, out=180, looseness=2.75] (1.center) to (0.center);
%		\draw [style=simple] (4.center) to (6.center);
%		\draw [style=simple] (6.center) to (0.center);
%	\end{pgfonlayer}
%\end{tikzpicture}
%\eq{Lem.\ref{lem:oldaxiom}}
%\begin{tikzpicture}
%	\begin{pgfonlayer}{nodelayer}
%		\node [style=none] (0) at (2.5, -0.75) {};
%		\node [style=none] (1) at (2.5, -0.25) {};
%		\node [style=Z] (2) at (1.75, -0.25) {};
%		\node [style=X] (3) at (1.75, -0.75) {};
%	\end{pgfonlayer}
%	\begin{pgfonlayer}{edgelayer}
%		\draw [style=simple] (0.center) to (3);
%		\draw [style=simple] (2) to (1.center);
%	\end{pgfonlayer}
%\end{tikzpicture}
%$$


\begin{proposition}
\label{prop:TOFZXA}
Consider the interpretation $\llbracket\_\rrbracket_{\ZXA}:\ZXA\to\hat \TOF$ taking:

\begin{center}
\begin{tabular}{c}
$
\begin{tikzpicture}[tikzfig]
	\begin{pgfonlayer}{nodelayer}
		\node [style=none] (7) at (0.25, 2) {};
		\node [style=none] (8) at (0.75, 2) {};
		\node [style=Z] (9) at (0.5, 1.25) {};
		\node [style=none] (10) at (0.5, 0.5) {};
	\end{pgfonlayer}
	\begin{pgfonlayer}{edgelayer}
		\draw [style=simple, in=-90, out=124] (9) to (7.center);
		\draw [style=simple, in=-90, out=56] (9) to (8.center);
		\draw [style=simple] (9) to (10.center);
	\end{pgfonlayer}
\end{tikzpicture}
\mapsto
\begin{tikzpicture}[tikzfig]
	\begin{pgfonlayer}{nodelayer}
		\node [style=none] (8) at (0, 2.75) {};
		\node [style=none] (9) at (0, 1.25) {};
		\node [style=dot] (10) at (0.5, 2) {};
		\node [style=oplus] (11) at (0, 2) {};
		\node [style=X] (12) at (0.5, 1.25) {};
		\node [style=none] (13) at (0.5, 2.75) {};
	\end{pgfonlayer}
	\begin{pgfonlayer}{edgelayer}
		\draw [style=simple] (13.center) to (10);
		\draw [style=simple] (10) to (12);
		\draw [style=simple] (10) to (11);
		\draw [style=simple] (11) to (9.center);
		\draw [style=simple] (11) to (8.center);
	\end{pgfonlayer}
\end{tikzpicture}
\hspace*{.5cm}
\begin{tikzpicture}[tikzfig]
	\begin{pgfonlayer}{nodelayer}
		\node [style=none] (9) at (0.25, 1.25) {};
		\node [style=none] (10) at (0.75, 1.25) {};
		\node [style=Z] (11) at (0.5, 2) {};
		\node [style=none] (12) at (0.5, 2.75) {};
	\end{pgfonlayer}
	\begin{pgfonlayer}{edgelayer}
		\draw [style=simple, in=90, out=-124] (11) to (9.center);
		\draw [style=simple, in=90, out=-56] (11) to (10.center);
		\draw [style=simple] (11) to (12.center);
	\end{pgfonlayer}
\end{tikzpicture}
\mapsto
\begin{tikzpicture}[tikzfig]
	\begin{pgfonlayer}{nodelayer}
		\node [style=none] (15) at (0, 1.25) {};
		\node [style=none] (16) at (0, 2.75) {};
		\node [style=dot] (17) at (0.5, 2) {};
		\node [style=oplus] (18) at (0, 2) {};
		\node [style=X] (19) at (0.5, 2.75) {};
		\node [style=none] (20) at (0.5, 1.25) {};
	\end{pgfonlayer}
	\begin{pgfonlayer}{edgelayer}
		\draw [style=simple] (20.center) to (17);
		\draw [style=simple] (17) to (19);
		\draw [style=simple] (17) to (18);
		\draw [style=simple] (18) to (16.center);
		\draw [style=simple] (18) to (15.center);
	\end{pgfonlayer}
\end{tikzpicture}
\hspace*{.5cm}
\begin{tikzpicture}[tikzfig]
	\begin{pgfonlayer}{nodelayer}
		\node [style=Z] (0) at (0, 1) {};
		\node [style=none] (1) at (0, 2) {};
	\end{pgfonlayer}
	\begin{pgfonlayer}{edgelayer}
		\draw [style=simple] (1.center) to (0);
	\end{pgfonlayer}
\end{tikzpicture}
\mapsto
\begin{tikzpicture}[tikzfig]
	\begin{pgfonlayer}{nodelayer}
		\node [style=zeroin] (0) at (0, 1) {};
		\node [style=none] (1) at (0, 2) {};
	\end{pgfonlayer}
	\begin{pgfonlayer}{edgelayer}
		\draw [style=simple] (1.center) to (0);
	\end{pgfonlayer}
\end{tikzpicture}
\hspace*{.5cm}
\begin{tikzpicture}[tikzfig]
	\begin{pgfonlayer}{nodelayer}
		\node [style=none] (0) at (2.75, -8) {};
		\node [style=Z] (1) at (0, 2) {};
		\node [style=none] (2) at (0, 1) {};
		\node [style=none] (3) at (6.5, -0.75) {};
	\end{pgfonlayer}
	\begin{pgfonlayer}{edgelayer}
		\draw [style=simple] (2.center) to (1);
	\end{pgfonlayer}
\end{tikzpicture}
\mapsto
\begin{tikzpicture}[tikzfig]
	\begin{pgfonlayer}{nodelayer}
		\node [style=zeroin] (1) at (0, 2) {};
		\node [style=none] (2) at (0, 1) {};
	\end{pgfonlayer}
	\begin{pgfonlayer}{edgelayer}
		\draw [style=simple] (2.center) to (1);
	\end{pgfonlayer}
\end{tikzpicture}
$
\\\\
$
\begin{tikzpicture}[tikzfig]
	\begin{pgfonlayer}{nodelayer}
		\node [style=none] (2) at (0.25, 2) {};
		\node [style=none] (3) at (0.75, 2) {};
		\node [style=X] (4) at (0.5, 1.25) {};
		\node [style=none] (5) at (0.5, 0.5) {};
	\end{pgfonlayer}
	\begin{pgfonlayer}{edgelayer}
		\draw [style=simple, in=-90, out=124] (4) to (2.center);
		\draw [style=simple, in=-90, out=56] (4) to (3.center);
		\draw [style=simple] (4) to (5.center);
	\end{pgfonlayer}
\end{tikzpicture}
\mapsto
\begin{tikzpicture}[tikzfig]
	\begin{pgfonlayer}{nodelayer}
		\node [style=none] (3) at (0, 2) {};
		\node [style=none] (4) at (0, 0.5) {};
		\node [style=oplus] (5) at (0.5, 1.25) {};
		\node [style=dot] (6) at (0, 1.25) {};
		\node [style=zeroin] (7) at (0.5, 0.5) {};
		\node [style=none] (8) at (0.5, 2) {};
	\end{pgfonlayer}
	\begin{pgfonlayer}{edgelayer}
		\draw [style=simple] (8.center) to (5);
		\draw [style=simple] (5) to (7);
		\draw [style=simple] (5) to (6);
		\draw [style=simple] (6) to (4.center);
		\draw [style=simple] (6) to (3.center);
	\end{pgfonlayer}
\end{tikzpicture}
\hspace*{.5cm}
\begin{tikzpicture}[tikzfig]
	\begin{pgfonlayer}{nodelayer}
		\node [style=none] (4) at (0.25, 2.25) {};
		\node [style=none] (5) at (0.75, 2.25) {};
		\node [style=X] (6) at (0.5, 3) {};
		\node [style=none] (7) at (0.5, 3.75) {};
	\end{pgfonlayer}
	\begin{pgfonlayer}{edgelayer}
		\draw [style=simple, in=90, out=-124] (6) to (4.center);
		\draw [style=simple, in=90, out=-56] (6) to (5.center);
		\draw [style=simple] (6) to (7.center);
	\end{pgfonlayer}
\end{tikzpicture}
\mapsto
\begin{tikzpicture}[tikzfig]
	\begin{pgfonlayer}{nodelayer}
		\node [style=none] (9) at (-6, 2) {};
		\node [style=none] (10) at (0, -7.75) {};
		\node [style=none] (11) at (0, -6.25) {};
		\node [style=oplus] (12) at (0.5, -7) {};
		\node [style=dot] (13) at (0, -7) {};
		\node [style=zeroin] (14) at (0.5, -6.25) {};
		\node [style=none] (15) at (0.5, -7.75) {};
		\node [style=none] (16) at (10.25, 0.5) {};
		\node [style=none] (17) at (-2.5, -2.25) {};
	\end{pgfonlayer}
	\begin{pgfonlayer}{edgelayer}
		\draw [style=simple] (15.center) to (12);
		\draw [style=simple] (12) to (14);
		\draw [style=simple] (12) to (13);
		\draw [style=simple] (13) to (11.center);
		\draw [style=simple] (13) to (10.center);
	\end{pgfonlayer}
\end{tikzpicture}
\hspace*{.5cm}
\begin{tikzpicture}[tikzfig]
	\begin{pgfonlayer}{nodelayer}
		\node [style=X] (1) at (0, 1) {};
		\node [style=none] (2) at (0, 2) {};
	\end{pgfonlayer}
	\begin{pgfonlayer}{edgelayer}
		\draw [style=simple] (2.center) to (1);
	\end{pgfonlayer}
\end{tikzpicture}
\mapsto
\begin{tikzpicture}[tikzfig]
	\begin{pgfonlayer}{nodelayer}
		\node [style=X] (2) at (0, 13.25) {};
		\node [style=none] (3) at (0, 14.25) {};
	\end{pgfonlayer}
	\begin{pgfonlayer}{edgelayer}
		\draw [style=simple] (3.center) to (2);
	\end{pgfonlayer}
\end{tikzpicture}
\hspace*{.5cm}
\begin{tikzpicture}[tikzfig]
	\begin{pgfonlayer}{nodelayer}
		\node [style=X] (0) at (0, 2) {};
		\node [style=none] (1) at (0, 1) {};
	\end{pgfonlayer}
	\begin{pgfonlayer}{edgelayer}
		\draw [style=simple] (1.center) to (0);
	\end{pgfonlayer}
\end{tikzpicture}
\mapsto
\begin{tikzpicture}[tikzfig]
	\begin{pgfonlayer}{nodelayer}
		\node [style=X] (0) at (0, 2) {};
		\node [style=none] (1) at (0, 1) {};
	\end{pgfonlayer}
	\begin{pgfonlayer}{edgelayer}
		\draw [style=simple] (1.center) to (0);
	\end{pgfonlayer}
\end{tikzpicture}
$
\\\\
$
\begin{tikzpicture}[tikzfig]
	\begin{pgfonlayer}{nodelayer}
		\node [style=Z] (0) at (0, 1.5) {$\pi$};
		\node [style=none] (1) at (0, 2.5) {};
		\node [style=none] (2) at (0, 0.5) {};
	\end{pgfonlayer}
	\begin{pgfonlayer}{edgelayer}
		\draw [style=simple] (1.center) to (0);
		\draw [style=simple] (0) to (2.center);
	\end{pgfonlayer}
\end{tikzpicture}
\mapsto
\begin{tikzpicture}[tikzfig]
	\begin{pgfonlayer}{nodelayer}
		\node [style=oplus] (1) at (0, 1) {};
		\node [style=none] (2) at (0, 2) {};
		\node [style=none] (3) at (0, 0) {};
	\end{pgfonlayer}
	\begin{pgfonlayer}{edgelayer}
		\draw [style=simple] (2.center) to (1);
		\draw [style=simple] (1) to (3.center);
	\end{pgfonlayer}
\end{tikzpicture}
\hspace*{.5cm}
\begin{tikzpicture}[tikzfig]
	\begin{pgfonlayer}{nodelayer}
		\node [style=none] (2) at (0, 0) {};
		\node [style=andin] (3) at (0.5, 0.75) {};
		\node [style=none] (4) at (0.5, 1.5) {};
		\node [style=none] (5) at (1, 0) {};
	\end{pgfonlayer}
	\begin{pgfonlayer}{edgelayer}
		\draw [style=simple] (4.center) to (3);
		\draw [style=simple, in=90, out=-124] (3) to (2.center);
		\draw [style=simple, in=90, out=-56] (3) to (5.center);
	\end{pgfonlayer}
\end{tikzpicture}
\mapsto
\begin{tikzpicture}[tikzfig]
	\begin{pgfonlayer}{nodelayer}
		\node [style=dot] (0) at (0, 1.5) {};
		\node [style=dot] (1) at (0.5, 1.5) {};
		\node [style=oplus] (2) at (1, 1.5) {};
		\node [style=X] (3) at (0, 2.25) {};
		\node [style=X] (4) at (0.5, 2.25) {};
		\node [style=none] (5) at (1, 2.5) {};
		\node [style=zeroin] (6) at (1, 0.75) {};
		\node [style=none] (7) at (0, 0.5) {};
		\node [style=none] (8) at (0.5, 0.5) {};
	\end{pgfonlayer}
	\begin{pgfonlayer}{edgelayer}
		\draw [style=simple] (5.center) to (2);
		\draw [style=simple] (2) to (6);
		\draw [style=simple] (2) to (1);
		\draw [style=simple] (1) to (0);
		\draw [style=simple] (3) to (0);
		\draw [style=simple] (0) to (7.center);
		\draw [style=simple] (8.center) to (1);
		\draw [style=simple] (1) to (4);
	\end{pgfonlayer}
\end{tikzpicture}
\hspace*{.5cm}
\begin{tikzpicture}[tikzfig]
	\begin{pgfonlayer}{nodelayer}
		\node [style=none] (0) at (0, 2) {};
		\node [style=andout] (1) at (0.5, 1.25) {};
		\node [style=none] (2) at (0.5, 0.5) {};
		\node [style=none] (3) at (1, 2) {};
	\end{pgfonlayer}
	\begin{pgfonlayer}{edgelayer}
		\draw [style=simple] (2.center) to (1);
		\draw [style=simple, in=-90, out=124] (1) to (0.center);
		\draw [style=simple, in=-90, out=56] (1) to (3.center);
	\end{pgfonlayer}
\end{tikzpicture}
\mapsto
\begin{tikzpicture}[tikzfig]
	\begin{pgfonlayer}{nodelayer}
		\node [style=dot] (1) at (0, 1) {};
		\node [style=dot] (2) at (0.5, 1) {};
		\node [style=oplus] (3) at (1, 1) {};
		\node [style=X] (4) at (0, 0.25) {};
		\node [style=X] (5) at (0.5, 0.25) {};
		\node [style=none] (6) at (1, 0) {};
		\node [style=zeroout] (7) at (1, 1.75) {};
		\node [style=none] (8) at (0, 2) {};
		\node [style=none] (9) at (0.5, 2) {};
	\end{pgfonlayer}
	\begin{pgfonlayer}{edgelayer}
		\draw [style=simple] (6.center) to (3);
		\draw [style=simple] (3) to (7);
		\draw [style=simple] (3) to (2);
		\draw [style=simple] (2) to (1);
		\draw [style=simple] (4) to (1);
		\draw [style=simple] (1) to (8.center);
		\draw [style=simple] (9.center) to (2);
		\draw [style=simple] (2) to (5);
	\end{pgfonlayer}
\end{tikzpicture}
$
\end{tabular}
\end{center}

This interpretation is a strict symmetric \dag-monoidal functor.
\end{proposition}


\begin{proof}
We prove that all of the axioms of $\ZXA$ hold in $\hat \TOF$:
\begin{enumerate}
\item[\ref{ZXA.1}:]
\begin{description}
\item[Unitality:] By Lemma \ref{lemma:whiteunit}:

\begin{align*}
\left\llbracket
\begin{tikzpicture}[tikzfig]
	\begin{pgfonlayer}{nodelayer}
		\node [style=none] (0) at (0, 2) {};
		\node [style=none] (1) at (1, 2) {};
		\node [style=Z] (2) at (0.5, 1.25) {};
		\node [style=none] (3) at (0.5, 0.5) {};
		\node [style=Z] (4) at (0, 2) {};
	\end{pgfonlayer}
	\begin{pgfonlayer}{edgelayer}
		\draw [style=simple, in=-90, out=124] (2) to (0.center);
		\draw [style=simple, in=-90, out=56] (2) to (1.center);
		\draw [style=simple] (2) to (3.center);
	\end{pgfonlayer}
\end{tikzpicture}
\right\rrbracket_{\ZXA}
&=
\begin{tikzpicture}[tikzfig]
	\begin{pgfonlayer}{nodelayer}
		\node [style=none] (0) at (0, 0.5) {};
		\node [style=dot] (1) at (0.75, 1.25) {};
		\node [style=oplus] (2) at (0, 1.25) {};
		\node [style=X] (3) at (0.75, 0.5) {};
		\node [style=none] (4) at (0.75, 2) {};
		\node [style=zeroout] (5) at (0, 2) {};
	\end{pgfonlayer}
	\begin{pgfonlayer}{edgelayer}
		\draw [style=simple] (4.center) to (1);
		\draw [style=simple] (1) to (3);
		\draw [style=simple] (1) to (2);
		\draw [style=simple] (2) to (0.center);
		\draw [style=simple] (5) to (2);
	\end{pgfonlayer}
\end{tikzpicture}
\eq{comm.}
\begin{tikzpicture}[tikzfig]
	\begin{pgfonlayer}{nodelayer}
		\node [style=none] (1) at (0.75, 0.5) {};
		\node [style=dot] (2) at (0.75, 1.25) {};
		\node [style=oplus] (3) at (0, 1.25) {};
		\node [style=X] (4) at (0, 0.5) {};
		\node [style=none] (5) at (0.75, 2) {};
		\node [style=zeroout] (6) at (0, 2) {};
	\end{pgfonlayer}
	\begin{pgfonlayer}{edgelayer}
		\draw [style=simple] (5.center) to (2);
		\draw [style=simple] (2) to (3);
		\draw [style=simple] (6) to (3);
		\draw [style=simple] (1.center) to (2);
		\draw [style=simple] (3) to (4);
	\end{pgfonlayer}
\end{tikzpicture}
\eq{unit}
\begin{tikzpicture}[tikzfig]
	\begin{pgfonlayer}{nodelayer}
		\node [style=none] (2) at (0.75, 0.5) {};
		\node [style=X] (3) at (0, 0.5) {};
		\node [style=none] (4) at (0.75, 2) {};
		\node [style=zeroout] (5) at (0, 2) {};
	\end{pgfonlayer}
	\begin{pgfonlayer}{edgelayer}
		\draw [style=simple] (4.center) to (2.center);
		\draw [style=simple] (3) to (5);
	\end{pgfonlayer}
\end{tikzpicture}\
\eq{Rem. \ref{cor:copy}}
\begin{tikzpicture}[tikzfig]
	\begin{pgfonlayer}{nodelayer}
		\node [style=none] (3) at (0.5, 3) {};
		\node [style=none] (4) at (0.5, 2) {};
	\end{pgfonlayer}
	\begin{pgfonlayer}{edgelayer}
		\draw [style=simple] (3.center) to (4.center);
	\end{pgfonlayer}
\end{tikzpicture}
=
\left\llbracket
\begin{tikzpicture}[tikzfig]
	\begin{pgfonlayer}{nodelayer}
		\node [style=none] (4) at (0.5, 3) {};
		\node [style=none] (5) at (0.5, 2) {};
	\end{pgfonlayer}
	\begin{pgfonlayer}{edgelayer}
		\draw [style=simple] (4.center) to (5.center);
	\end{pgfonlayer}
\end{tikzpicture}
\right\rrbracket_{\ZXA}
\end{align*}

\item[Associativity:]
\begin{align*}
\left\llbracket
\begin{tikzpicture}[tikzfig]
	\begin{pgfonlayer}{nodelayer}
		\node [style=none] (5) at (0, 3.5) {};
		\node [style=none] (6) at (1, 3.5) {};
		\node [style=Z] (7) at (0.5, 2.75) {};
		\node [style=none] (8) at (0.5, 2) {};
		\node [style=Z] (9) at (0, 3.5) {};
		\node [style=none] (10) at (0.5, 4.25) {};
		\node [style=none] (11) at (1, 4.25) {};
		\node [style=none] (12) at (-0.5, 4.25) {};
	\end{pgfonlayer}
	\begin{pgfonlayer}{edgelayer}
		\draw [style=simple, in=-90, out=124] (7) to (5.center);
		\draw [style=simple, in=-90, out=56] (7) to (6.center);
		\draw [style=simple] (7) to (8.center);
		\draw [style=simple, in=56, out=-90] (10.center) to (5.center);
		\draw [style=simple, in=-90, out=124] (5.center) to (12.center);
		\draw [style=simple] (11.center) to (6.center);
	\end{pgfonlayer}
\end{tikzpicture}
\right\rrbracket_{\ZXA}
&=
\begin{tikzpicture}[tikzfig]
	\begin{pgfonlayer}{nodelayer}
		\node [style=none] (6) at (0, 2) {};
		\node [style=dot] (7) at (1.25, 2.75) {};
		\node [style=oplus] (8) at (0, 2.75) {};
		\node [style=X] (9) at (1.25, 2) {};
		\node [style=none] (10) at (1.25, 5) {};
		\node [style=dot] (11) at (0.75, 4.25) {};
		\node [style=X] (12) at (0.75, 3.5) {};
		\node [style=oplus] (13) at (0, 4.25) {};
		\node [style=none] (14) at (0.75, 5) {};
		\node [style=none] (15) at (0, 5) {};
	\end{pgfonlayer}
	\begin{pgfonlayer}{edgelayer}
		\draw [style=simple] (10.center) to (7);
		\draw [style=simple] (7) to (9);
		\draw [style=simple] (7) to (8);
		\draw [style=simple] (8) to (6.center);
		\draw [style=simple] (14.center) to (11);
		\draw [style=simple] (11) to (12);
		\draw [style=simple] (11) to (13);
		\draw [style=simple] (13) to (15.center);
		\draw [style=simple] (13) to (8);
	\end{pgfonlayer}
\end{tikzpicture}
\eq{\ref{CNOT.8}}
\begin{tikzpicture}[tikzfig]
	\begin{pgfonlayer}{nodelayer}
		\node [style=none] (7) at (0, 2) {};
		\node [style=X] (8) at (1.5, 2) {};
		\node [style=none] (9) at (1.5, 5) {};
		\node [style=dot] (10) at (0.75, 3.5) {};
		\node [style=X] (11) at (0.75, 2) {};
		\node [style=oplus] (12) at (0, 3.5) {};
		\node [style=none] (13) at (0.75, 5) {};
		\node [style=none] (14) at (0, 5) {};
		\node [style=oplus] (15) at (0.75, 4.25) {};
		\node [style=dot] (16) at (1.5, 4.25) {};
		\node [style=oplus] (17) at (0.75, 2.75) {};
		\node [style=dot] (18) at (1.5, 2.75) {};
	\end{pgfonlayer}
	\begin{pgfonlayer}{edgelayer}
		\draw [style=simple] (13.center) to (10);
		\draw [style=simple] (10) to (11);
		\draw [style=simple] (10) to (12);
		\draw [style=simple] (12) to (14.center);
		\draw [style=simple] (16) to (15);
		\draw [style=simple] (18) to (17);
		\draw [style=simple] (9.center) to (16);
		\draw [style=simple] (16) to (18);
		\draw [style=simple] (18) to (8);
		\draw [style=simple] (7.center) to (12);
	\end{pgfonlayer}
\end{tikzpicture}
\eq{Rem. \ref{cor:copy}}
\begin{tikzpicture}[tikzfig]
	\begin{pgfonlayer}{nodelayer}
		\node [style=none] (8) at (0, 3.25) {};
		\node [style=dot] (9) at (0.75, 4) {};
		\node [style=oplus] (10) at (0, 4) {};
		\node [style=X] (11) at (0.75, 3.25) {};
		\node [style=X] (12) at (1.5, 4) {};
		\node [style=none] (13) at (0.75, 5.5) {};
		\node [style=none] (14) at (1.5, 5.5) {};
		\node [style=oplus] (15) at (0.75, 4.75) {};
		\node [style=dot] (16) at (1.5, 4.75) {};
		\node [style=none] (17) at (0, 5.5) {};
	\end{pgfonlayer}
	\begin{pgfonlayer}{edgelayer}
		\draw [style=simple] (9) to (11);
		\draw [style=simple] (9) to (10);
		\draw [style=simple] (10) to (8.center);
		\draw [style=simple] (14.center) to (16);
		\draw [style=simple] (16) to (12);
		\draw [style=simple] (16) to (15);
		\draw [style=simple] (15) to (13.center);
		\draw [style=simple] (9) to (15);
		\draw [style=simple] (17.center) to (10);
	\end{pgfonlayer}
\end{tikzpicture}\\
&=
\left\llbracket
\begin{tikzpicture}[tikzfig]
	\begin{pgfonlayer}{nodelayer}
		\node [style=none] (9) at (0.5, 4.75) {};
		\node [style=none] (10) at (-0.5, 4.75) {};
		\node [style=Z] (11) at (0, 4) {};
		\node [style=none] (12) at (0, 3.25) {};
		\node [style=Z] (13) at (0.5, 4.75) {};
		\node [style=none] (14) at (0, 5.5) {};
		\node [style=none] (15) at (-0.5, 5.5) {};
		\node [style=none] (16) at (1, 5.5) {};
	\end{pgfonlayer}
	\begin{pgfonlayer}{edgelayer}
		\draw [style=simple, in=-90, out=56] (11) to (9.center);
		\draw [style=simple, in=-90, out=124] (11) to (10.center);
		\draw [style=simple] (11) to (12.center);
		\draw [style=simple, in=124, out=-90] (14.center) to (9.center);
		\draw [style=simple, in=-90, out=56] (9.center) to (16.center);
		\draw [style=simple] (15.center) to (10.center);
	\end{pgfonlayer}
\end{tikzpicture}
\right\rrbracket_{\ZXA}
\end{align*}

\item[Frobenius:]
\begin{align*}
\left\llbracket
\begin{tikzpicture}[tikzfig]
	\begin{pgfonlayer}{nodelayer}
		\node [style=none] (10) at (1, 4.75) {};
		\node [style=Z] (11) at (0.5, 4) {};
		\node [style=none] (12) at (0.5, 3.25) {};
		\node [style=none] (13) at (0, 4.75) {};
		\node [style=Z] (14) at (0, 4.75) {};
		\node [style=none] (15) at (0.5, 4) {};
		\node [style=none] (16) at (-0.5, 4) {};
		\node [style=none] (17) at (1, 5.5) {};
		\node [style=none] (18) at (0, 5.5) {};
		\node [style=none] (19) at (-0.5, 3.25) {};
	\end{pgfonlayer}
	\begin{pgfonlayer}{edgelayer}
		\draw [style=simple, in=-90, out=56] (11) to (10.center);
		\draw [style=simple] (11) to (12.center);
		\draw [style=simple, in=-56, out=90] (15.center) to (13.center);
		\draw [style=simple, in=90, out=-124] (13.center) to (16.center);
		\draw [style=simple] (17.center) to (10.center);
		\draw [style=simple] (13.center) to (18.center);
		\draw [style=simple] (16.center) to (19.center);
	\end{pgfonlayer}
\end{tikzpicture}
\right\rrbracket_{\ZXA}
&=
\begin{tikzpicture}[tikzfig]
	\begin{pgfonlayer}{nodelayer}
		\node [style=none] (11) at (0, 3.25) {};
		\node [style=dot] (12) at (0.75, 4) {};
		\node [style=oplus] (13) at (0, 4) {};
		\node [style=X] (14) at (0.75, 3.25) {};
		\node [style=none] (15) at (0, 5.5) {};
		\node [style=oplus] (16) at (0.75, 4.75) {};
		\node [style=X] (17) at (1.5, 5.5) {};
		\node [style=dot] (18) at (1.5, 4.75) {};
		\node [style=none] (19) at (0.75, 5.5) {};
		\node [style=none] (20) at (1.5, 3.25) {};
	\end{pgfonlayer}
	\begin{pgfonlayer}{edgelayer}
		\draw [style=simple] (12) to (14);
		\draw [style=simple] (12) to (13);
		\draw [style=simple] (13) to (11.center);
		\draw [style=simple] (15.center) to (13);
		\draw [style=simple] (18) to (17);
		\draw [style=simple] (18) to (16);
		\draw [style=simple] (18) to (20.center);
		\draw [style=simple] (12) to (16);
		\draw [style=simple] (16) to (19.center);
	\end{pgfonlayer}
\end{tikzpicture}
\eq{Lem \ref{lemma:Iwama}}
\begin{tikzpicture}[tikzfig]
	\begin{pgfonlayer}{nodelayer}
		\node [style=dot] (12) at (0.75, 4.75) {};
		\node [style=oplus] (13) at (0, 4.75) {};
		\node [style=dot] (14) at (1.5, 5.5) {};
		\node [style=X] (15) at (1.5, 6.25) {};
		\node [style=none] (16) at (0, 6.25) {};
		\node [style=none] (17) at (0.75, 6.25) {};
		\node [style=none] (18) at (0, 3.25) {};
		\node [style=X] (19) at (0.75, 3.25) {};
		\node [style=none] (20) at (1.5, 3.25) {};
		\node [style=oplus] (21) at (0, 5.5) {};
		\node [style=oplus] (22) at (0.75, 4) {};
		\node [style=dot] (23) at (1.5, 4) {};
	\end{pgfonlayer}
	\begin{pgfonlayer}{edgelayer}
		\draw [style=simple] (12) to (13);
		\draw [style=simple] (15) to (14);
		\draw [style=simple] (20.center) to (23);
		\draw [style=simple] (23) to (14);
		\draw [style=simple] (14) to (21);
		\draw [style=simple] (17.center) to (12);
		\draw [style=simple] (12) to (22);
		\draw [style=simple] (22) to (19);
		\draw [style=simple] (18.center) to (13);
		\draw [style=simple] (13) to (21);
		\draw [style=simple] (21) to (16.center);
		\draw [style=simple] (23) to (22);
	\end{pgfonlayer}
\end{tikzpicture}
\eq{Lem. \ref{lemma:whiteunit}}
\begin{tikzpicture}[tikzfig]
	\begin{pgfonlayer}{nodelayer}
		\node [style=dot] (13) at (0.75, 4.75) {};
		\node [style=oplus] (14) at (0, 4.75) {};
		\node [style=dot] (15) at (1.5, 5.5) {};
		\node [style=X] (16) at (1.5, 6.25) {};
		\node [style=none] (17) at (0, 6.25) {};
		\node [style=none] (18) at (0.75, 6.25) {};
		\node [style=none] (19) at (0, 3.25) {};
		\node [style=X] (20) at (0.75, 3.25) {};
		\node [style=none] (21) at (1.5, 3.25) {};
		\node [style=oplus] (22) at (0, 5.5) {};
	\end{pgfonlayer}
	\begin{pgfonlayer}{edgelayer}
		\draw [style=simple] (13) to (14);
		\draw [style=simple] (16) to (15);
		\draw [style=simple] (15) to (22);
		\draw [style=simple] (18.center) to (13);
		\draw [style=simple] (19.center) to (14);
		\draw [style=simple] (14) to (22);
		\draw [style=simple] (22) to (17.center);
		\draw [style=simple] (15) to (21.center);
		\draw [style=simple] (20) to (13);
	\end{pgfonlayer}
\end{tikzpicture}\\
&\eq{\ref{CNOT.5}}
\begin{tikzpicture}[tikzfig]
	\begin{pgfonlayer}{nodelayer}
		\node [style=dot] (14) at (0.75, 6.25) {};
		\node [style=oplus] (15) at (0, 6.25) {};
		\node [style=dot] (16) at (0.75, 4) {};
		\node [style=X] (17) at (0.75, 4.75) {};
		\node [style=none] (18) at (0, 7) {};
		\node [style=none] (19) at (0.75, 7) {};
		\node [style=none] (20) at (0, 3.25) {};
		\node [style=X] (21) at (0.75, 5.5) {};
		\node [style=none] (22) at (0.75, 3.25) {};
		\node [style=oplus] (23) at (0, 4) {};
	\end{pgfonlayer}
	\begin{pgfonlayer}{edgelayer}
		\draw [style=simple] (14) to (15);
		\draw [style=simple] (17) to (16);
		\draw [style=simple] (16) to (23);
		\draw [style=simple] (19.center) to (14);
		\draw [style=simple] (20.center) to (15);
		\draw [style=simple] (23) to (18.center);
		\draw [style=simple] (16) to (22.center);
		\draw [style=simple] (21) to (14);
	\end{pgfonlayer}
\end{tikzpicture}
=
\left\llbracket
\begin{tikzpicture}[tikzfig]
	\begin{pgfonlayer}{nodelayer}
		\node [style=none] (15) at (1, 5.75) {};
		\node [style=Z] (16) at (0.5, 5) {};
		\node [style=none] (17) at (0, 5.75) {};
		\node [style=none] (18) at (0.5, 5) {};
		\node [style=none] (19) at (1, 3.25) {};
		\node [style=none] (20) at (0.5, 4) {};
		\node [style=Z] (21) at (0.5, 4) {};
		\node [style=none] (22) at (0, 3.25) {};
	\end{pgfonlayer}
	\begin{pgfonlayer}{edgelayer}
		\draw [style=simple, in=-90, out=60] (16) to (15.center);
		\draw [style=simple, in=-90, out=120] (18.center) to (17.center);
		\draw [style=simple, in=90, out=-60] (21) to (19.center);
		\draw [style=simple, in=90, out=-120] (20.center) to (22.center);
		\draw [style=simple] (16) to (20.center);
	\end{pgfonlayer}
\end{tikzpicture}
\right\rrbracket_{\ZXA}
\end{align*}


\item[Phase amalgamation:]

\begin{align*}
\left\llbracket
\begin{tikzpicture}[tikzfig]
	\begin{pgfonlayer}{nodelayer}
		\node [style=none] (16) at (0.75, 3.25) {};
		\node [style=Z] (17) at (0.75, 4.25) {$\pi$};
		\node [style=Z] (18) at (0.75, 5.25) {$\pi$};
		\node [style=none] (19) at (0.75, 6.25) {};
	\end{pgfonlayer}
	\begin{pgfonlayer}{edgelayer}
		\draw [style=simple] (19.center) to (18);
		\draw [style=simple] (18) to (17);
		\draw [style=simple] (17) to (16.center);
	\end{pgfonlayer}
\end{tikzpicture}
\right\rrbracket_{\ZXA}
&=
\begin{tikzpicture}[tikzfig]
	\begin{pgfonlayer}{nodelayer}
		\node [style=none] (17) at (0.75, 3.25) {};
		\node [style=none] (18) at (0.75, 6.25) {};
		\node [style=oplus] (19) at (0.75, 4.25) {};
		\node [style=oplus] (20) at (0.75, 5.25) {};
	\end{pgfonlayer}
	\begin{pgfonlayer}{edgelayer}
		\draw [style=simple] (18.center) to (20);
		\draw [style=simple] (20) to (19);
		\draw [style=simple] (19) to (17.center);
	\end{pgfonlayer}
\end{tikzpicture}
=
\begin{tikzpicture}[tikzfig]
	\begin{pgfonlayer}{nodelayer}
		\node [style=none] (18) at (0.75, 3.25) {};
		\node [style=none] (19) at (0.75, 4.25) {};
	\end{pgfonlayer}
	\begin{pgfonlayer}{edgelayer}
		\draw [style=simple] (19.center) to (18.center);
	\end{pgfonlayer}
\end{tikzpicture}
=
\left\llbracket
\begin{tikzpicture}[tikzfig]
	\begin{pgfonlayer}{nodelayer}
		\node [style=none] (19) at (0.75, 3.25) {};
		\node [style=none] (20) at (0.75, 4.25) {};
	\end{pgfonlayer}
	\begin{pgfonlayer}{edgelayer}
		\draw [style=simple] (20.center) to (19.center);
	\end{pgfonlayer}
\end{tikzpicture}
\right\rrbracket_{\ZXA}
\end{align*}



\end{description}


\item[\ref{ZXA.2}:]
\begin{align*}
\left\llbracket
\begin{tikzpicture}[tikzfig]
	\begin{pgfonlayer}{nodelayer}
		\node [style=none] (20) at (0, 4.25) {};
		\node [style=none] (21) at (0, 3.25) {};
		\node [style=Z] (22) at (0, 4.25) {};
		\node [style=none] (23) at (0.5, 5) {};
		\node [style=none] (24) at (-0.5, 5) {};
		\node [style=none] (25) at (0.5, 5.75) {};
		\node [style=none] (26) at (-0.5, 5.75) {};
	\end{pgfonlayer}
	\begin{pgfonlayer}{edgelayer}
		\draw [style=simple, in=56, out=-90] (23.center) to (20.center);
		\draw [style=simple, in=-90, out=124] (20.center) to (24.center);
		\draw [style=simple] (21.center) to (20.center);
		\draw [style=simple, in=90, out=-90] (25.center) to (24.center);
		\draw [style=simple, in=90, out=-90] (26.center) to (23.center);
	\end{pgfonlayer}
\end{tikzpicture}
\right\rrbracket_{\ZXA}
&=
\begin{tikzpicture}[tikzfig]
	\begin{pgfonlayer}{nodelayer}
		\node [style=oplus] (21) at (0, 4.25) {};
		\node [style=dot] (22) at (0.75, 4.25) {};
		\node [style=X] (23) at (0.75, 3.5) {};
		\node [style=none] (24) at (0, 3.25) {};
		\node [style=none] (25) at (0, 5) {};
		\node [style=none] (26) at (0.75, 5) {};
		\node [style=none] (27) at (0, 6) {};
		\node [style=none] (28) at (0.75, 6) {};
	\end{pgfonlayer}
	\begin{pgfonlayer}{edgelayer}
		\draw [style=simple] (26.center) to (22);
		\draw [style=simple] (23) to (22);
		\draw [style=simple] (22) to (21);
		\draw [style=simple] (21) to (24.center);
		\draw [style=simple] (21) to (25.center);
		\draw [style=simple, in=90, out=-90] (28.center) to (25.center);
		\draw [style=simple, in=90, out=-90] (27.center) to (26.center);
	\end{pgfonlayer}
\end{tikzpicture}
\eq{\ref{TOF.14}}
\begin{tikzpicture}[tikzfig]
	\begin{pgfonlayer}{nodelayer}
		\node [style=oplus] (22) at (0, 4.25) {};
		\node [style=dot] (23) at (0.75, 4.25) {};
		\node [style=X] (24) at (0.75, 3.5) {};
		\node [style=none] (25) at (0, 3.25) {};
		\node [style=none] (26) at (0, 6.5) {};
		\node [style=none] (27) at (0.75, 6.5) {};
		\node [style=oplus] (28) at (0, 4.75) {};
		\node [style=dot] (29) at (0.75, 5.75) {};
		\node [style=dot] (30) at (0.75, 4.75) {};
		\node [style=oplus] (31) at (0.75, 5.25) {};
		\node [style=oplus] (32) at (0, 5.75) {};
		\node [style=dot] (33) at (0, 5.25) {};
	\end{pgfonlayer}
	\begin{pgfonlayer}{edgelayer}
		\draw [style=simple] (24) to (23);
		\draw [style=simple] (23) to (22);
		\draw [style=simple] (22) to (25.center);
		\draw [style=simple] (30) to (28);
		\draw [style=simple] (33) to (31);
		\draw [style=simple] (29) to (32);
		\draw [style=simple] (32) to (33);
		\draw [style=simple] (33) to (28);
		\draw [style=simple] (30) to (31);
		\draw [style=simple] (31) to (29);
		\draw [style=simple] (27.center) to (29);
		\draw [style=simple] (30) to (23);
		\draw [style=simple] (22) to (28);
		\draw [style=simple] (32) to (26.center);
	\end{pgfonlayer}
\end{tikzpicture}
\eq{\ref{CNOT.2}}
\\
&\eq{Lem. \ref{lemma:whiteunit}}
\begin{tikzpicture}[tikzfig]
	\begin{pgfonlayer}{nodelayer}
		\node [style=oplus] (0) at (0, 4) {};
		\node [style=dot] (1) at (0.75, 4) {};
		\node [style=X] (2) at (0.75, 3.25) {};
		\node [style=none] (3) at (0, 3) {};
		\node [style=none] (4) at (0, 4.75) {};
		\node [style=none] (5) at (0.75, 4.75) {};
	\end{pgfonlayer}
	\begin{pgfonlayer}{edgelayer}
		\draw [style=simple] (5.center) to (1);
		\draw [style=simple] (2) to (1);
		\draw [style=simple] (1) to (0);
		\draw [style=simple] (0) to (3.center);
		\draw [style=simple] (0) to (4.center);
	\end{pgfonlayer}
\end{tikzpicture}
=
\left\llbracket
\begin{tikzpicture}[tikzfig]
	\begin{pgfonlayer}{nodelayer}
		\node [style=none] (0) at (0, 2) {};
		\node [style=none] (1) at (0, 1) {};
		\node [style=Z] (2) at (0, 2) {};
		\node [style=none] (3) at (0.5, 2.75) {};
		\node [style=none] (4) at (-0.5, 2.75) {};
	\end{pgfonlayer}
	\begin{pgfonlayer}{edgelayer}
		\draw [style=simple, in=56, out=-90] (3.center) to (0.center);
		\draw [style=simple, in=-90, out=124] (0.center) to (4.center);
		\draw [style=simple] (1.center) to (0.center);
	\end{pgfonlayer}
\end{tikzpicture}
\right\rrbracket_{\ZXA}
\end{align*}


\item[\ref{ZXA.3}:]
This is immediate.

\item[\ref{ZXA.4}:]
This is immediate.


\item[\ref{ZXA.5}:]
\begin{align*}
\left\llbracket
\begin{tikzpicture}[tikzfig]
	\begin{pgfonlayer}{nodelayer}
		\node [style=X] (0) at (-1, 1) {};
		\node [style=X] (1) at (-0.25, 1) {};
		\node [style=Z] (2) at (-0.25, 1.75) {};
		\node [style=Z] (3) at (-1, 1.75) {};
		\node [style=none] (4) at (-1, 2.25) {};
		\node [style=none] (5) at (-0.25, 2.25) {};
		\node [style=none] (6) at (-1, 0.5) {};
		\node [style=none] (7) at (-0.25, 0.5) {};
	\end{pgfonlayer}
	\begin{pgfonlayer}{edgelayer}
		\draw (7.center) to (1);
		\draw (1) to (3);
		\draw [in=120, out=-120, looseness=1.25] (3) to (0);
		\draw (0) to (2);
		\draw (2) to (5.center);
		\draw [in=60, out=-60, looseness=1.25] (2) to (1);
		\draw (0) to (6.center);
		\draw (3) to (4.center);
	\end{pgfonlayer}
\end{tikzpicture}
\right\rrbracket_{\ZXA}
&=
\begin{tikzpicture}[tikzfig]
	\begin{pgfonlayer}{nodelayer}
		\node [style=none] (0) at (-1, 0.5) {};
		\node [style=none] (1) at (0.5, 0.5) {};
		\node [style=none] (2) at (-1, 3) {};
		\node [style=none] (3) at (0.5, 3) {};
		\node [style=zeroin] (4) at (-0.5, 0.75) {};
		\node [style=zeroin] (5) at (0, 0.75) {};
		\node [style=oplus] (6) at (-0.5, 1.25) {};
		\node [style=oplus] (7) at (0, 1.25) {};
		\node [style=dot] (8) at (-1, 1.25) {};
		\node [style=dot] (9) at (0.5, 1.25) {};
		\node [style=dot] (10) at (0, 1.75) {};
		\node [style=dot] (11) at (-0.5, 2.25) {};
		\node [style=oplus] (12) at (-1, 1.75) {};
		\node [style=oplus] (13) at (0.5, 2.25) {};
		\node [style=X] (14) at (-0.5, 2.75) {};
		\node [style=X] (15) at (0, 2.75) {};
	\end{pgfonlayer}
	\begin{pgfonlayer}{edgelayer}
		\draw (1.center) to (9);
		\draw (9) to (13);
		\draw (13) to (3.center);
		\draw (15) to (10);
		\draw (10) to (7);
		\draw (7) to (5);
		\draw (4) to (6);
		\draw (6) to (11);
		\draw (11) to (14);
		\draw (2.center) to (12);
		\draw (12) to (8);
		\draw (8) to (0.center);
		\draw (8) to (6);
		\draw (7) to (9);
		\draw (10) to (12);
		\draw (11) to (13);
	\end{pgfonlayer}
\end{tikzpicture}
\eq{Lem \ref{lemma:Iwama}}
\begin{tikzpicture}[tikzfig]
	\begin{pgfonlayer}{nodelayer}
		\node [style=none] (0) at (-1, 0.5) {};
		\node [style=none] (1) at (0.5, 0.5) {};
		\node [style=none] (2) at (-1, 4) {};
		\node [style=none] (3) at (0.5, 4) {};
		\node [style=zeroin] (4) at (-0.5, 0.75) {};
		\node [style=zeroin] (5) at (0, 0.75) {};
		\node [style=oplus] (6) at (-0.5, 1.25) {};
		\node [style=oplus] (7) at (0, 2.75) {};
		\node [style=dot] (8) at (-1, 1.25) {};
		\node [style=dot] (9) at (0.5, 2.75) {};
		\node [style=dot] (10) at (0, 2.25) {};
		\node [style=dot] (11) at (-0.5, 3.25) {};
		\node [style=oplus] (12) at (-1, 2.25) {};
		\node [style=oplus] (13) at (0.5, 3.25) {};
		\node [style=X] (14) at (-0.5, 3.75) {};
		\node [style=X] (15) at (0, 3.75) {};
		\node [style=oplus] (16) at (-1, 1.75) {};
		\node [style=dot] (17) at (0.5, 1.75) {};
	\end{pgfonlayer}
	\begin{pgfonlayer}{edgelayer}
		\draw (1.center) to (9);
		\draw (9) to (13);
		\draw (13) to (3.center);
		\draw (15) to (10);
		\draw (10) to (7);
		\draw (7) to (5);
		\draw (4) to (6);
		\draw (6) to (11);
		\draw (11) to (14);
		\draw (2.center) to (12);
		\draw (12) to (8);
		\draw (8) to (0.center);
		\draw (8) to (6);
		\draw (7) to (9);
		\draw (10) to (12);
		\draw (11) to (13);
		\draw (17) to (16);
	\end{pgfonlayer}
\end{tikzpicture}
\eq{\ref{TOF.2}}
\begin{tikzpicture}[tikzfig]
	\begin{pgfonlayer}{nodelayer}
		\node [style=none] (0) at (-1, 0.5) {};
		\node [style=none] (1) at (0.5, 0.5) {};
		\node [style=none] (2) at (-1, 3.5) {};
		\node [style=none] (3) at (0.5, 3.5) {};
		\node [style=zeroin] (4) at (-0.5, 0.75) {};
		\node [style=zeroin] (5) at (0, 0.75) {};
		\node [style=oplus] (6) at (-0.5, 1.25) {};
		\node [style=oplus] (7) at (0, 2.25) {};
		\node [style=dot] (8) at (-1, 1.25) {};
		\node [style=dot] (9) at (0.5, 2.25) {};
		\node [style=dot] (10) at (-0.5, 2.75) {};
		\node [style=oplus] (11) at (0.5, 2.75) {};
		\node [style=X] (12) at (-0.5, 3.25) {};
		\node [style=X] (13) at (0, 3.25) {};
		\node [style=oplus] (14) at (-1, 1.75) {};
		\node [style=dot] (15) at (0.5, 1.75) {};
	\end{pgfonlayer}
	\begin{pgfonlayer}{edgelayer}
		\draw (1.center) to (9);
		\draw (9) to (11);
		\draw (11) to (3.center);
		\draw (7) to (5);
		\draw (4) to (6);
		\draw (6) to (10);
		\draw (10) to (12);
		\draw (8) to (0.center);
		\draw (8) to (6);
		\draw (7) to (9);
		\draw (10) to (11);
		\draw (15) to (14);
		\draw (2.center) to (14);
		\draw (14) to (8);
		\draw (7) to (13);
	\end{pgfonlayer}
\end{tikzpicture}\\
&\eq{unit}
\begin{tikzpicture}[tikzfig]
	\begin{pgfonlayer}{nodelayer}
		\node [style=none] (0) at (-1, 0.5) {};
		\node [style=none] (1) at (0, 0.5) {};
		\node [style=none] (2) at (-1, 3) {};
		\node [style=none] (3) at (0, 3) {};
		\node [style=zeroin] (4) at (-0.5, 0.75) {};
		\node [style=oplus] (5) at (-0.5, 1.25) {};
		\node [style=dot] (6) at (-1, 1.25) {};
		\node [style=dot] (7) at (-0.5, 2.25) {};
		\node [style=oplus] (8) at (0, 2.25) {};
		\node [style=X] (9) at (-0.5, 2.75) {};
		\node [style=oplus] (10) at (-1, 1.75) {};
		\node [style=dot] (11) at (0, 1.75) {};
	\end{pgfonlayer}
	\begin{pgfonlayer}{edgelayer}
		\draw (8) to (3.center);
		\draw (4) to (5);
		\draw (5) to (7);
		\draw (7) to (9);
		\draw (6) to (0.center);
		\draw (6) to (5);
		\draw (7) to (8);
		\draw (11) to (10);
		\draw (2.center) to (10);
		\draw (10) to (6);
		\draw (8) to (11);
		\draw (11) to (1.center);
	\end{pgfonlayer}
\end{tikzpicture}
=
\begin{tikzpicture}[tikzfig]
	\begin{pgfonlayer}{nodelayer}
		\node [style=none] (0) at (-0.75, 0.5) {};
		\node [style=none] (1) at (0, 0.5) {};
		\node [style=none] (2) at (-1, 3) {};
		\node [style=none] (3) at (0, 3) {};
		\node [style=dot] (4) at (-0.5, 2.25) {};
		\node [style=oplus] (5) at (0, 2.25) {};
		\node [style=X] (6) at (-0.5, 2.75) {};
		\node [style=oplus] (7) at (-1, 1.75) {};
		\node [style=dot] (8) at (0, 1.75) {};
		\node [style=fanout] (9) at (-0.75, 1) {};
		\node [style=none] (10) at (-0.5, 1.75) {};
	\end{pgfonlayer}
	\begin{pgfonlayer}{edgelayer}
		\draw (5) to (3.center);
		\draw (4) to (6);
		\draw (4) to (5);
		\draw (8) to (7);
		\draw (2.center) to (7);
		\draw (5) to (8);
		\draw (8) to (1.center);
		\draw (0.center) to (9);
		\draw [in=-90, out=108] (9) to (7);
		\draw (4) to (10.center);
		\draw [in=72, out=-90] (10.center) to (9);
	\end{pgfonlayer}
\end{tikzpicture}
\eq{\ref{CNOT.2}}
\begin{tikzpicture}[tikzfig]
	\begin{pgfonlayer}{nodelayer}
		\node [style=none] (0) at (-0.75, 0.5) {};
		\node [style=none] (1) at (0, 0.5) {};
		\node [style=none] (2) at (-1, 4) {};
		\node [style=none] (3) at (0, 4) {};
		\node [style=dot] (4) at (-0.5, 3.25) {};
		\node [style=oplus] (5) at (0, 3.25) {};
		\node [style=X] (6) at (-0.5, 3.75) {};
		\node [style=oplus] (7) at (-1, 2.75) {};
		\node [style=dot] (8) at (0, 2.75) {};
		\node [style=fanout] (9) at (-0.75, 1) {};
		\node [style=oplus] (10) at (-0.5, 2.25) {};
		\node [style=dot] (11) at (0, 2.25) {};
		\node [style=oplus] (12) at (-0.5, 1.75) {};
		\node [style=dot] (13) at (0, 1.75) {};
		\node [style=none] (14) at (-1, 1.75) {};
	\end{pgfonlayer}
	\begin{pgfonlayer}{edgelayer}
		\draw (5) to (3.center);
		\draw (4) to (6);
		\draw (4) to (5);
		\draw (8) to (7);
		\draw (2.center) to (7);
		\draw (5) to (8);
		\draw (8) to (1.center);
		\draw (0.center) to (9);
		\draw (11) to (10);
		\draw (13) to (12);
		\draw (4) to (10);
		\draw (10) to (12);
		\draw [in=60, out=-90] (12) to (9);
		\draw (7) to (14.center);
		\draw [in=120, out=-90] (14.center) to (9);
	\end{pgfonlayer}
\end{tikzpicture}\\
&\eq{Lem. \ref{lemma:natoplus}}
\begin{tikzpicture}[tikzfig]
	\begin{pgfonlayer}{nodelayer}
		\node [style=none] (0) at (-0.75, 0.5) {};
		\node [style=none] (1) at (0, 0.5) {};
		\node [style=none] (2) at (-1, 3.75) {};
		\node [style=none] (3) at (0, 3.75) {};
		\node [style=dot] (4) at (-0.5, 3) {};
		\node [style=oplus] (5) at (0, 3) {};
		\node [style=X] (6) at (-0.5, 3.5) {};
		\node [style=fanout] (7) at (-0.75, 1.75) {};
		\node [style=oplus] (8) at (-0.5, 2.5) {};
		\node [style=dot] (9) at (0, 2.5) {};
		\node [style=none] (10) at (-1, 2.5) {};
		\node [style=oplus] (11) at (-0.75, 1) {};
		\node [style=dot] (12) at (0, 1) {};
	\end{pgfonlayer}
	\begin{pgfonlayer}{edgelayer}
		\draw (5) to (3.center);
		\draw (4) to (6);
		\draw (4) to (5);
		\draw (0.center) to (7);
		\draw (9) to (8);
		\draw (4) to (8);
		\draw [in=120, out=-90] (10.center) to (7);
		\draw (12) to (11);
		\draw [in=-90, out=60] (7) to (8);
		\draw (9) to (12);
		\draw (12) to (1.center);
		\draw (9) to (5);
		\draw (2.center) to (10.center);
	\end{pgfonlayer}
\end{tikzpicture}
\eq{Lem. \ref{lemma:whiteunit}}
\begin{tikzpicture}[tikzfig]
	\begin{pgfonlayer}{nodelayer}
		\node [style=none] (0) at (-0.75, 0.5) {};
		\node [style=none] (1) at (0, 0.5) {};
		\node [style=none] (2) at (-1, 4.25) {};
		\node [style=none] (3) at (0, 4.25) {};
		\node [style=dot] (4) at (-0.5, 3) {};
		\node [style=oplus] (5) at (0, 3) {};
		\node [style=X] (6) at (-0.5, 4) {};
		\node [style=fanout] (7) at (-0.75, 1.75) {};
		\node [style=oplus] (8) at (-0.5, 2.5) {};
		\node [style=dot] (9) at (0, 2.5) {};
		\node [style=none] (10) at (-1, 2.5) {};
		\node [style=oplus] (11) at (-0.75, 1) {};
		\node [style=dot] (12) at (0, 1) {};
		\node [style=oplus] (13) at (-0.5, 3.5) {};
		\node [style=dot] (14) at (0, 3.5) {};
	\end{pgfonlayer}
	\begin{pgfonlayer}{edgelayer}
		\draw (5) to (3.center);
		\draw (4) to (6);
		\draw (4) to (5);
		\draw (0.center) to (7);
		\draw (9) to (8);
		\draw (4) to (8);
		\draw [in=120, out=-90] (10.center) to (7);
		\draw (12) to (11);
		\draw [in=-90, out=60] (7) to (8);
		\draw (9) to (12);
		\draw (12) to (1.center);
		\draw (9) to (5);
		\draw (2.center) to (10.center);
		\draw (14) to (13);
	\end{pgfonlayer}
\end{tikzpicture}\\
&\eq{\ref{TOF.14}}
\begin{tikzpicture}[tikzfig]
	\begin{pgfonlayer}{nodelayer}
		\node [style=none] (0) at (-0.75, 0.5) {};
		\node [style=none] (1) at (0, 0.5) {};
		\node [style=none] (2) at (-1, 3.75) {};
		\node [style=none] (3) at (0, 3.75) {};
		\node [style=X] (4) at (-0.5, 3.5) {};
		\node [style=fanout] (5) at (-0.75, 1.75) {};
		\node [style=none] (6) at (-1, 2.5) {};
		\node [style=oplus] (7) at (-0.75, 1) {};
		\node [style=dot] (8) at (0, 1) {};
		\node [style=none] (9) at (0, 3.5) {};
		\node [style=none] (10) at (0, 2.5) {};
		\node [style=none] (11) at (-0.5, 2.5) {};
	\end{pgfonlayer}
	\begin{pgfonlayer}{edgelayer}
		\draw (0.center) to (5);
		\draw [in=120, out=-90] (6.center) to (5);
		\draw (8) to (7);
		\draw (8) to (1.center);
		\draw (2.center) to (6.center);
		\draw [in=90, out=-90] (9.center) to (11.center);
		\draw [in=-90, out=90] (10.center) to (4);
		\draw (3.center) to (9.center);
		\draw (10.center) to (8);
		\draw [in=60, out=-90] (11.center) to (5);
	\end{pgfonlayer}
\end{tikzpicture}
=
\begin{tikzpicture}[tikzfig]
	\begin{pgfonlayer}{nodelayer}
		\node [style=none] (0) at (-0.75, 0.5) {};
		\node [style=none] (1) at (-0.25, 0.5) {};
		\node [style=X] (2) at (-0.25, 1.5) {};
		\node [style=fanout] (3) at (-0.75, 1.75) {};
		\node [style=none] (4) at (-1, 2.5) {};
		\node [style=oplus] (5) at (-0.75, 1) {};
		\node [style=dot] (6) at (-0.25, 1) {};
		\node [style=none] (7) at (-0.5, 2.5) {};
	\end{pgfonlayer}
	\begin{pgfonlayer}{edgelayer}
		\draw (0.center) to (3);
		\draw [in=120, out=-90] (4.center) to (3);
		\draw (6) to (5);
		\draw (6) to (1.center);
		\draw [in=60, out=-90] (7.center) to (3);
		\draw (2) to (6);
	\end{pgfonlayer}
\end{tikzpicture}
=
\left\llbracket
\begin{tikzpicture}[tikzfig]
	\begin{pgfonlayer}{nodelayer}
		\node [style=Z] (0) at (-1, 1) {};
		\node [style=none] (1) at (-1.25, 0.5) {};
		\node [style=none] (2) at (-0.75, 0.5) {};
		\node [style=X] (3) at (-1, 1.75) {};
		\node [style=none] (4) at (-1.25, 2.25) {};
		\node [style=none] (5) at (-0.75, 2.25) {};
	\end{pgfonlayer}
	\begin{pgfonlayer}{edgelayer}
		\draw [in=63, out=-90] (5.center) to (3);
		\draw (3) to (0);
		\draw [in=90, out=-117] (0) to (1.center);
		\draw [in=-63, out=90] (2.center) to (0);
		\draw [in=-90, out=117] (3) to (4.center);
	\end{pgfonlayer}
\end{tikzpicture}
\right\rrbracket_{\ZXA}
\end{align*}



\item[\ref{ZXA.6}:]

$$
\left\llbracket
\begin{tikzpicture}[tikzfig]
	\begin{pgfonlayer}{nodelayer}
		\node [style=none] (0) at (-0.25, 2) {};
		\node [style=X] (1) at (0, 1.25) {};
		\node [style=Z] (2) at (0, 0.5) {};
		\node [style=none] (3) at (0.25, 2) {};
	\end{pgfonlayer}
	\begin{pgfonlayer}{edgelayer}
		\draw [style=simple, in=-90, out=124] (1) to (0.center);
		\draw [style=simple, in=60, out=-90] (3.center) to (1);
		\draw [style=simple] (1) to (2);
	\end{pgfonlayer}
\end{tikzpicture}
\right\rrbracket_{\ZXA}
=
\begin{tikzpicture}[tikzfig]
	\begin{pgfonlayer}{nodelayer}
		\node [style=dot] (0) at (0, 1) {};
		\node [style=zeroin] (1) at (0, 0.5) {};
		\node [style=zeroin] (2) at (0.75, 0.5) {};
		\node [style=none] (3) at (0.75, 1.5) {};
		\node [style=none] (4) at (0, 1.5) {};
		\node [style=oplus] (5) at (0.75, 1) {};
	\end{pgfonlayer}
	\begin{pgfonlayer}{edgelayer}
		\draw [style=simple] (0) to (4.center);
		\draw [style=simple] (0) to (1);
		\draw [style=simple] (2) to (5);
		\draw [style=simple] (5) to (3.center);
		\draw [style=simple] (5) to (0);
	\end{pgfonlayer}
\end{tikzpicture}
\eq{\ref{TOF.2}}
\begin{tikzpicture}[tikzfig]
	\begin{pgfonlayer}{nodelayer}
		\node [style=zeroin] (0) at (0, 0.5) {};
		\node [style=zeroin] (1) at (0.75, 0.5) {};
		\node [style=none] (2) at (0.75, 1.5) {};
		\node [style=none] (3) at (0, 1.5) {};
	\end{pgfonlayer}
	\begin{pgfonlayer}{edgelayer}
		\draw [style=simple] (2.center) to (1);
		\draw [style=simple] (0) to (3.center);
	\end{pgfonlayer}
\end{tikzpicture}
=
\left\llbracket
\begin{tikzpicture}[tikzfig]
	\begin{pgfonlayer}{nodelayer}
		\node [style=none] (0) at (-0.25, 1) {};
		\node [style=Z] (1) at (-0.25, 0.5) {};
		\node [style=none] (2) at (0.25, 1) {};
		\node [style=Z] (3) at (0.25, 0.5) {};
	\end{pgfonlayer}
	\begin{pgfonlayer}{edgelayer}
		\draw [style=simple] (3) to (2.center);
		\draw [style=simple] (1) to (0.center);
	\end{pgfonlayer}
\end{tikzpicture}
\right\rrbracket_{\ZXA}
$$



\item[\ref{ZXA.7}:]
This is immediate.

%
%
%\item[\ref{ZXA.13old}:]
%\begin{align*}
%\left\llbracket
%\begin{tikzpicture}
%	\begin{pgfonlayer}{nodelayer}
%		\node [style=none] (0) at (4.25, 0.5) {};
%		\node [style=none] (1) at (3.5, -0) {};
%		\node [style=Z] (2) at (3.5, 1) {};
%		\node [style=none] (3) at (5, 0.5) {};
%		\node [style=none] (4) at (3, -0) {};
%		\node [style=andin] (5) at (4.25, 0.5) {};
%	\end{pgfonlayer}
%	\begin{pgfonlayer}{edgelayer}
%		\draw [in=0, out=135, looseness=1.00] (0.center) to (2);
%		\draw [in=-131, out=0, looseness=1.00] (1.center) to (0.center);
%		\draw (3.center) to (0.center);
%		\draw (1.center) to (4.center);
%	\end{pgfonlayer}
%\end{tikzpicture}
%\right\rrbracket_{\ZXA}
%&=
%\begin{tikzpicture}
%	\begin{pgfonlayer}{nodelayer}
%		\node [style=dot] (0) at (4, -1) {};
%		\node [style=dot] (1) at (4, -1.5) {};
%		\node [style=oplus] (2) at (4, -2) {};
%		\node [style=X] (3) at (4.5, -1.5) {};
%		\node [style=X] (4) at (4.5, -1) {};
%		\node [style=zeroin] (5) at (3.5, -2) {};
%		\node [style=none] (6) at (4.75, -2) {};
%		\node [style=none] (7) at (3.25, -1.5) {};
%		\node [style=zeroin] (8) at (3.5, -1) {};
%	\end{pgfonlayer}
%	\begin{pgfonlayer}{edgelayer}
%		\draw (4) to (0);
%		\draw (0) to (8);
%		\draw (7.center) to (3);
%		\draw (0) to (2);
%		\draw (5) to (6.center);
%	\end{pgfonlayer}
%\end{tikzpicture}
%\eq{\ref{TOF.2}}
%\begin{tikzpicture}
%	\begin{pgfonlayer}{nodelayer}
%		\node [style=X] (0) at (4.5, -1.5) {};
%		\node [style=X] (1) at (4.5, -1) {};
%		\node [style=zeroin] (2) at (3.5, -2) {};
%		\node [style=none] (3) at (4.75, -2) {};
%		\node [style=none] (4) at (3.25, -1.5) {};
%		\node [style=zeroin] (5) at (3.5, -1) {};
%	\end{pgfonlayer}
%	\begin{pgfonlayer}{edgelayer}
%		\draw (4.center) to (0);
%		\draw (2) to (3.center);
%		\draw (1) to (5);
%	\end{pgfonlayer}
%\end{tikzpicture}
%\eq{Rem. \ref{cor:copy}}
%\begin{tikzpicture}
%	\begin{pgfonlayer}{nodelayer}
%		\node [style=X] (0) at (4.5, -1.5) {};
%		\node [style=zeroin] (1) at (5.25, -1.5) {};
%		\node [style=none] (2) at (6, -1.5) {};
%		\node [style=none] (3) at (3.75, -1.5) {};
%	\end{pgfonlayer}
%	\begin{pgfonlayer}{edgelayer}
%		\draw (3.center) to (0);
%		\draw (1) to (2.center);
%	\end{pgfonlayer}
%\end{tikzpicture}
%=
%\left\llbracket
%\begin{tikzpicture}
%	\begin{pgfonlayer}{nodelayer}
%		\node [style=Z] (0) at (4.75, -0) {};
%		\node [style=none] (1) at (5.5, -0) {};
%		\node [style=none] (2) at (3.25, -0) {};
%		\node [style=X] (3) at (4, -0) {};
%	\end{pgfonlayer}
%	\begin{pgfonlayer}{edgelayer}
%		\draw (1.center) to (0);
%		\draw (3) to (2.center);
%	\end{pgfonlayer}
%\end{tikzpicture}
%\right\rrbracket_{\ZXA}
%\end{align*}



\item[\ref{ZXA.8}:]
\begin{align*}
\left\llbracket
\begin{tikzpicture}[tikzfig]
	\begin{pgfonlayer}{nodelayer}
		\node [style=Z] (0) at (-1, 3) {};
		\node [style=X] (1) at (-1, 2.25) {};
		\node [style=none] (2) at (-1, 3.5) {};
		\node [style=none] (3) at (-1, 1.75) {};
	\end{pgfonlayer}
	\begin{pgfonlayer}{edgelayer}
		\draw (2.center) to (0);
		\draw [in=120, out=-120, looseness=1.25] (0) to (1);
		\draw [in=-60, out=60, looseness=1.25] (1) to (0);
		\draw (1) to (3.center);
	\end{pgfonlayer}
\end{tikzpicture}
\right\rrbracket_{\ZXA}
&=
\begin{tikzpicture}[tikzfig]
	\begin{pgfonlayer}{nodelayer}
		\node [style=none] (0) at (-1, 3) {};
		\node [style=dot] (1) at (-1, 1.25) {};
		\node [style=dot] (2) at (-0.5, 2.25) {};
		\node [style=oplus] (3) at (-1, 2.25) {};
		\node [style=oplus] (4) at (-0.5, 1.25) {};
		\node [style=X] (5) at (-0.5, 2.75) {};
		\node [style=zeroin] (6) at (-0.5, 0.75) {};
		\node [style=none] (7) at (-1, 0.5) {};
	\end{pgfonlayer}
	\begin{pgfonlayer}{edgelayer}
		\draw (5) to (2);
		\draw (2) to (3);
		\draw (3) to (0.center);
		\draw (3) to (1);
		\draw (1) to (4);
		\draw (4) to (6);
		\draw (4) to (2);
		\draw (1) to (7.center);
	\end{pgfonlayer}
\end{tikzpicture}
\eq{Lem. \ref{lemma:whiteunit}}
\begin{tikzpicture}[tikzfig]
	\begin{pgfonlayer}{nodelayer}
		\node [style=dot] (0) at (-1, 1.25) {};
		\node [style=dot] (1) at (-0.5, 1.75) {};
		\node [style=oplus] (2) at (-1, 1.75) {};
		\node [style=oplus] (3) at (-0.5, 1.25) {};
		\node [style=zeroin] (4) at (-0.5, 0.75) {};
		\node [style=none] (5) at (-1, 0.5) {};
		\node [style=none] (6) at (-1, 3) {};
		\node [style=X] (7) at (-0.5, 2.75) {};
		\node [style=dot] (8) at (-1, 2.25) {};
		\node [style=oplus] (9) at (-0.5, 2.25) {};
	\end{pgfonlayer}
	\begin{pgfonlayer}{edgelayer}
		\draw (1) to (2);
		\draw (2) to (0);
		\draw (0) to (3);
		\draw (3) to (4);
		\draw (3) to (1);
		\draw (0) to (5.center);
		\draw (8) to (9);
		\draw (7) to (9);
		\draw (9) to (1);
		\draw (2) to (8);
		\draw (8) to (6.center);
	\end{pgfonlayer}
\end{tikzpicture}
\eq{\ref{TOF.14}}
\begin{tikzpicture}[tikzfig]
	\begin{pgfonlayer}{nodelayer}
		\node [style=zeroin] (0) at (-0.5, 0.75) {};
		\node [style=none] (1) at (-1, 0.5) {};
		\node [style=none] (2) at (-1, 2.5) {};
		\node [style=X] (3) at (-0.5, 2.25) {};
		\node [style=none] (4) at (-1, 0.75) {};
		\node [style=none] (5) at (-1, 2.25) {};
	\end{pgfonlayer}
	\begin{pgfonlayer}{edgelayer}
		\draw [in=90, out=-90] (3) to (4.center);
		\draw [in=-90, out=90] (0) to (5.center);
		\draw (5.center) to (2.center);
		\draw (4.center) to (1.center);
	\end{pgfonlayer}
\end{tikzpicture}\\
&=
\begin{tikzpicture}[tikzfig]
	\begin{pgfonlayer}{nodelayer}
		\node [style=zeroin] (0) at (-1, 2) {};
		\node [style=X] (1) at (-1, 1.25) {};
		\node [style=none] (2) at (-1, 0.5) {};
		\node [style=none] (3) at (-1, 2.75) {};
	\end{pgfonlayer}
	\begin{pgfonlayer}{edgelayer}
		\draw (1) to (2.center);
		\draw (0) to (3.center);
	\end{pgfonlayer}
\end{tikzpicture}
=
\left\llbracket
\begin{tikzpicture}[tikzfig]
	\begin{pgfonlayer}{nodelayer}
		\node [style=Z] (0) at (-1, 3) {};
		\node [style=X] (1) at (-1, 2.25) {};
		\node [style=none] (2) at (-1, 3.5) {};
		\node [style=none] (3) at (-1, 1.75) {};
	\end{pgfonlayer}
	\begin{pgfonlayer}{edgelayer}
		\draw (2.center) to (0);
		\draw (1) to (3.center);
	\end{pgfonlayer}
\end{tikzpicture}
\right\rrbracket_{\ZXA}
\end{align*}


\item[\ref{ZXA.9}:]
\begin{align*}
\left\llbracket
\begin{tikzpicture}[tikzfig]
	\begin{pgfonlayer}{nodelayer}
		\node [style=andin] (1) at (0, 3) {};
		\node [style=andin] (2) at (0.5, 4) {};
		\node [style=none] (3) at (0.5, 4.75) {};
		\node [style=none] (4) at (0.75, 3) {};
		\node [style=none] (5) at (-0.5, 2) {};
		\node [style=none] (6) at (0.25, 2) {};
		\node [style=none] (7) at (0.75, 2) {};
	\end{pgfonlayer}
	\begin{pgfonlayer}{edgelayer}
		\draw [style=simple] (3.center) to (2);
		\draw [style=simple, in=90, out=-117] (2) to (1);
		\draw [style=simple, in=90, out=-117] (1) to (5.center);
		\draw [style=simple, in=90, out=-76] (1) to (6.center);
		\draw [style=simple] (7.center) to (4.center);
		\draw [style=simple, in=-63, out=90] (4.center) to (2);
	\end{pgfonlayer}
\end{tikzpicture}
\right\rrbracket_{\ZXA}
&=
\begin{tikzpicture}[tikzfig]
	\begin{pgfonlayer}{nodelayer}
		\node [style=dot] (2) at (0, 1) {};
		\node [style=dot] (3) at (0.5, 1) {};
		\node [style=oplus] (4) at (1, 1) {};
		\node [style=X] (5) at (0, 1.75) {};
		\node [style=X] (6) at (0.5, 1.75) {};
		\node [style=none] (7) at (1, 2) {};
		\node [style=zeroin] (8) at (1, 0.25) {};
		\node [style=none] (9) at (0.5, -0.5) {};
		\node [style=X] (10) at (-0.5, 1.25) {};
		\node [style=none] (11) at (-1, -0.5) {};
		\node [style=dot] (12) at (-0.5, 0.5) {};
		\node [style=dot] (13) at (-1, 0.5) {};
		\node [style=oplus] (14) at (0, 0.5) {};
		\node [style=zeroin] (15) at (0, -0.25) {};
		\node [style=X] (16) at (-1, 1.25) {};
		\node [style=none] (17) at (-0.5, -0.5) {};
	\end{pgfonlayer}
	\begin{pgfonlayer}{edgelayer}
		\draw [style=simple] (7.center) to (4);
		\draw [style=simple] (4) to (8);
		\draw [style=simple] (4) to (3);
		\draw [style=simple] (3) to (2);
		\draw [style=simple] (5) to (2);
		\draw [style=simple] (9.center) to (3);
		\draw [style=simple] (3) to (6);
		\draw [style=simple] (14) to (15);
		\draw [style=simple] (14) to (12);
		\draw [style=simple] (12) to (13);
		\draw [style=simple] (16) to (13);
		\draw [style=simple] (13) to (11.center);
		\draw [style=simple] (17.center) to (12);
		\draw [style=simple] (12) to (10);
		\draw [style=simple] (14) to (2);
	\end{pgfonlayer}
\end{tikzpicture}
\eq{Lem \ref{lemma:Iwama}}
\begin{tikzpicture}[tikzfig]
	\begin{pgfonlayer}{nodelayer}
		\node [style=dot] (3) at (0, 1.5) {};
		\node [style=dot] (4) at (0.5, 1.5) {};
		\node [style=oplus] (5) at (1, 1.5) {};
		\node [style=none] (6) at (1, 3) {};
		\node [style=zeroin] (7) at (1, 0.25) {};
		\node [style=none] (8) at (0.5, 0) {};
		\node [style=none] (9) at (-1, 0) {};
		\node [style=zeroin] (10) at (0, 0.25) {};
		\node [style=none] (11) at (-0.5, 0) {};
		\node [style=X] (12) at (-0.5, 2.75) {};
		\node [style=X] (13) at (-1, 2.75) {};
		\node [style=X] (14) at (0, 2.75) {};
		\node [style=X] (15) at (0.5, 2.75) {};
		\node [style=dot] (16) at (-1, 2) {};
		\node [style=dot] (17) at (-0.5, 2) {};
		\node [style=oplus] (18) at (0, 2) {};
		\node [style=dot] (19) at (-1, 0.75) {};
		\node [style=dot] (20) at (-0.5, 0.75) {};
		\node [style=oplus] (21) at (1, 0.75) {};
		\node [style=dot] (22) at (0.5, 0.75) {};
	\end{pgfonlayer}
	\begin{pgfonlayer}{edgelayer}
		\draw [style=simple] (6.center) to (5);
		\draw [style=simple] (5) to (4);
		\draw [style=simple] (4) to (3);
		\draw [style=simple] (18) to (17);
		\draw [style=simple] (17) to (16);
		\draw [style=simple] (21) to (20);
		\draw [style=simple] (20) to (19);
		\draw [style=simple] (5) to (21);
		\draw [style=simple] (8.center) to (22);
		\draw [style=simple] (21) to (7);
		\draw [style=simple] (4) to (22);
		\draw [style=simple] (15) to (4);
		\draw [style=simple] (14) to (18);
		\draw [style=simple] (18) to (3);
		\draw [style=simple] (10) to (3);
		\draw [style=simple] (12) to (17);
		\draw [style=simple] (17) to (20);
		\draw [style=simple] (20) to (11.center);
		\draw [style=simple] (9.center) to (19);
		\draw [style=simple] (19) to (16);
		\draw [style=simple] (13) to (16);
	\end{pgfonlayer}
\end{tikzpicture}
\eq{Rem. \ref{cor:copy}}
\begin{tikzpicture}[tikzfig]
	\begin{pgfonlayer}{nodelayer}
		\node [style=dot] (4) at (0, 1.5) {};
		\node [style=dot] (5) at (0.5, 1.5) {};
		\node [style=oplus] (6) at (1, 1.5) {};
		\node [style=none] (7) at (1, 2.5) {};
		\node [style=zeroin] (8) at (1, 0.25) {};
		\node [style=none] (9) at (0.5, 0) {};
		\node [style=none] (10) at (-1, 0) {};
		\node [style=zeroin] (11) at (0, 0.25) {};
		\node [style=none] (12) at (-0.5, 0) {};
		\node [style=X] (13) at (-0.5, 2.25) {};
		\node [style=X] (14) at (-1, 2.25) {};
		\node [style=X] (15) at (0, 2.25) {};
		\node [style=X] (16) at (0.5, 2.25) {};
		\node [style=dot] (17) at (-1, 0.75) {};
		\node [style=dot] (18) at (-0.5, 0.75) {};
		\node [style=oplus] (19) at (1, 0.75) {};
		\node [style=dot] (20) at (0.5, 0.75) {};
	\end{pgfonlayer}
	\begin{pgfonlayer}{edgelayer}
		\draw [style=simple] (7.center) to (6);
		\draw [style=simple] (6) to (5);
		\draw [style=simple] (5) to (4);
		\draw [style=simple] (19) to (18);
		\draw [style=simple] (18) to (17);
		\draw [style=simple] (6) to (19);
		\draw [style=simple] (9.center) to (20);
		\draw [style=simple] (19) to (8);
		\draw [style=simple] (5) to (20);
		\draw [style=simple] (16) to (5);
		\draw [style=simple] (11) to (4);
		\draw [style=simple] (18) to (12.center);
		\draw [style=simple] (10.center) to (17);
		\draw [style=simple] (15) to (4);
		\draw [style=simple] (18) to (13);
		\draw [style=simple] (14) to (17);
	\end{pgfonlayer}
\end{tikzpicture}\\
&
\eq{\ref{TOF.2}}
\begin{tikzpicture}[tikzfig]
	\begin{pgfonlayer}{nodelayer}
		\node [style=none] (5) at (1, 2.25) {};
		\node [style=zeroin] (6) at (1, 0.25) {};
		\node [style=none] (7) at (0.5, 0) {};
		\node [style=none] (8) at (-1, 0) {};
		\node [style=zeroin] (9) at (0, 1.25) {};
		\node [style=none] (10) at (-0.5, 0) {};
		\node [style=X] (11) at (-0.5, 2.25) {};
		\node [style=X] (12) at (-1, 2.25) {};
		\node [style=X] (13) at (0, 2.25) {};
		\node [style=X] (14) at (0.5, 2.25) {};
		\node [style=dot] (15) at (-1, 0.75) {};
		\node [style=dot] (16) at (-0.5, 0.75) {};
		\node [style=oplus] (17) at (1, 0.75) {};
		\node [style=dot] (18) at (0.5, 0.75) {};
	\end{pgfonlayer}
	\begin{pgfonlayer}{edgelayer}
		\draw [style=simple] (17) to (16);
		\draw [style=simple] (16) to (15);
		\draw [style=simple] (7.center) to (18);
		\draw [style=simple] (17) to (6);
		\draw [style=simple] (16) to (10.center);
		\draw [style=simple] (8.center) to (15);
		\draw [style=simple] (16) to (11);
		\draw [style=simple] (12) to (15);
		\draw [style=simple] (5.center) to (17);
		\draw [style=simple] (9) to (13);
		\draw [style=simple] (14) to (18);
	\end{pgfonlayer}
\end{tikzpicture}
\eq{\ref{TOF.2}}
\begin{tikzpicture}[tikzfig]
	\begin{pgfonlayer}{nodelayer}
		\node [style=X] (6) at (0.5, 4) {};
		\node [style=zeroin] (7) at (1, 1.25) {};
		\node [style=X] (8) at (0, 4) {};
		\node [style=none] (9) at (-0.5, 1) {};
		\node [style=dot] (10) at (0.5, 3.25) {};
		\node [style=none] (11) at (1, 4.25) {};
		\node [style=X] (12) at (-1, 4) {};
		\node [style=X] (13) at (-0.5, 4) {};
		\node [style=dot] (14) at (-1, 3.25) {};
		\node [style=none] (15) at (0, 1) {};
		\node [style=oplus] (16) at (1, 3.25) {};
		\node [style=zeroin] (17) at (0.5, 1.25) {};
		\node [style=none] (18) at (-1, 1) {};
		\node [style=dot] (19) at (-1, 2.5) {};
		\node [style=dot] (20) at (-0.5, 2.5) {};
		\node [style=dot] (21) at (0, 2.5) {};
		\node [style=oplus] (22) at (1, 2.5) {};
	\end{pgfonlayer}
	\begin{pgfonlayer}{edgelayer}
		\draw [style=simple] (11.center) to (16);
		\draw [style=simple] (16) to (7);
		\draw [style=simple] (16) to (10);
		\draw [style=simple] (10) to (14);
		\draw [style=simple] (12) to (14);
		\draw [style=simple] (10) to (6);
		\draw (14) to (18.center);
		\draw (22) to (21);
		\draw (21) to (20);
		\draw (19) to (20);
		\draw (21) to (15.center);
		\draw (9.center) to (20);
		\draw (10) to (17);
		\draw (6) to (10);
		\draw (8) to (21);
		\draw (20) to (13);
	\end{pgfonlayer}
\end{tikzpicture}\\
&\eq{Lem \ref{lemma:Iwama}}
\begin{tikzpicture}[tikzfig]
	\begin{pgfonlayer}{nodelayer}
		\node [style=X] (7) at (0.5, 5.5) {};
		\node [style=zeroin] (8) at (1, 1.25) {};
		\node [style=X] (9) at (0, 5.5) {};
		\node [style=none] (10) at (-0.5, 1) {};
		\node [style=dot] (11) at (0.5, 3.25) {};
		\node [style=none] (12) at (1, 4.25) {};
		\node [style=X] (13) at (-1, 5.5) {};
		\node [style=X] (14) at (-0.5, 5.5) {};
		\node [style=dot] (15) at (-1, 3.25) {};
		\node [style=none] (16) at (0, 1) {};
		\node [style=oplus] (17) at (1, 3.25) {};
		\node [style=zeroin] (18) at (0.5, 1.25) {};
		\node [style=none] (19) at (-1, 1) {};
		\node [style=dot] (20) at (0, 4.75) {};
		\node [style=dot] (21) at (-0.5, 4.75) {};
		\node [style=dot] (22) at (-1, 2.5) {};
		\node [style=dot] (23) at (-0.5, 2.5) {};
		\node [style=dot] (24) at (0, 2.5) {};
		\node [style=oplus] (25) at (0.5, 4.75) {};
		\node [style=oplus] (26) at (1, 2.5) {};
	\end{pgfonlayer}
	\begin{pgfonlayer}{edgelayer}
		\draw [style=simple] (12.center) to (17);
		\draw [style=simple] (17) to (8);
		\draw [style=simple] (17) to (11);
		\draw [style=simple] (11) to (15);
		\draw [style=simple] (13) to (15);
		\draw [style=simple] (11) to (7);
		\draw (15) to (19.center);
		\draw [style=simple] (20) to (21);
		\draw (26) to (24);
		\draw (24) to (23);
		\draw (22) to (23);
		\draw (23) to (21);
		\draw (21) to (14);
		\draw (9) to (20);
		\draw (20) to (24);
		\draw (24) to (16.center);
		\draw (10.center) to (23);
		\draw (25) to (20);
		\draw (11) to (18);
	\end{pgfonlayer}
\end{tikzpicture}
\eq{Rem. \ref{cor:copy}}
\begin{tikzpicture}[tikzfig]
	\begin{pgfonlayer}{nodelayer}
		\node [style=X] (8) at (0.5, 5.5) {};
		\node [style=zeroin] (9) at (1, 4) {};
		\node [style=dot] (10) at (-0.5, 3.5) {};
		\node [style=X] (11) at (0, 4.25) {};
		\node [style=dot] (12) at (0, 3.5) {};
		\node [style=none] (13) at (-0.5, 2.5) {};
		\node [style=dot] (14) at (0.5, 4.75) {};
		\node [style=none] (15) at (1, 5.75) {};
		\node [style=X] (16) at (-1, 5.5) {};
		\node [style=X] (17) at (-0.5, 4.25) {};
		\node [style=dot] (18) at (-1, 4.75) {};
		\node [style=none] (19) at (0, 2.5) {};
		\node [style=oplus] (20) at (1, 4.75) {};
		\node [style=oplus] (21) at (0.5, 3.5) {};
		\node [style=zeroin] (22) at (0.5, 2.75) {};
		\node [style=none] (23) at (-1, 2.5) {};
	\end{pgfonlayer}
	\begin{pgfonlayer}{edgelayer}
		\draw [style=simple] (15.center) to (20);
		\draw [style=simple] (20) to (9);
		\draw [style=simple] (20) to (14);
		\draw [style=simple] (14) to (18);
		\draw [style=simple] (16) to (18);
		\draw [style=simple] (14) to (8);
		\draw [style=simple] (12) to (10);
		\draw [style=simple] (17) to (10);
		\draw [style=simple] (10) to (13.center);
		\draw [style=simple] (19.center) to (12);
		\draw [style=simple] (12) to (11);
		\draw [style=simple] (21) to (22);
		\draw (18) to (23.center);
		\draw (21) to (14);
		\draw (12) to (21);
	\end{pgfonlayer}
\end{tikzpicture}\\
&=
\left\llbracket
\begin{tikzpicture}[tikzfig]
	\begin{pgfonlayer}{nodelayer}
		\node [style=andin] (9) at (0.25, 3) {};
		\node [style=andin] (10) at (-0.25, 4) {};
		\node [style=none] (11) at (-0.25, 4.75) {};
		\node [style=none] (12) at (-0.5, 3) {};
		\node [style=none] (13) at (0.75, 2) {};
		\node [style=none] (14) at (0, 2) {};
		\node [style=none] (15) at (-0.5, 2) {};
	\end{pgfonlayer}
	\begin{pgfonlayer}{edgelayer}
		\draw [style=simple] (11.center) to (10);
		\draw [style=simple, in=90, out=-63] (10) to (9);
		\draw [style=simple, in=90, out=-63] (9) to (13.center);
		\draw [style=simple, in=90, out=-104] (9) to (14.center);
		\draw [style=simple] (15.center) to (12.center);
		\draw [style=simple, in=-117, out=90] (12.center) to (10);
	\end{pgfonlayer}
\end{tikzpicture}
\right\rrbracket_{\ZXA}
\end{align*}

\item[\ref{ZXA.10}:]
\begin{align*}
\left\llbracket
\begin{tikzpicture}[tikzfig]
	\begin{pgfonlayer}{nodelayer}
		\node [style=andin] (10) at (-1, 5.5) {};
		\node [style=none] (11) at (-1, 6) {};
		\node [style=none] (12) at (-0.75, 4.75) {};
		\node [style=Z] (13) at (-1.25, 4.75) {$\pi$};
	\end{pgfonlayer}
	\begin{pgfonlayer}{edgelayer}
		\draw (10) to (11.center);
		\draw [in=90, out=-108] (10) to (13);
		\draw [in=-72, out=90] (12.center) to (10);
	\end{pgfonlayer}
\end{tikzpicture}
\right\rrbracket_{\ZXA}
&=
\begin{tikzpicture}[tikzfig]
	\begin{pgfonlayer}{nodelayer}
		\node [style=X] (11) at (0, 7) {};
		\node [style=X] (12) at (-0.5, 7) {};
		\node [style=none] (13) at (0, 4.75) {};
		\node [style=zeroin] (14) at (0.5, 5.5) {};
		\node [style=oplus] (15) at (0.5, 6.25) {};
		\node [style=dot] (16) at (0, 6.25) {};
		\node [style=dot] (17) at (-0.5, 6.25) {};
		\node [style=none] (18) at (0.5, 7.25) {};
		\node [style=zeroin] (19) at (-0.5, 5) {};
		\node [style=oplus] (20) at (-0.5, 5.75) {};
	\end{pgfonlayer}
	\begin{pgfonlayer}{edgelayer}
		\draw [style=simple] (16) to (17);
		\draw [style=simple] (16) to (11);
		\draw [style=simple] (15) to (14);
		\draw [style=simple] (13.center) to (16);
		\draw [style=simple] (12) to (17);
		\draw (16) to (15);
		\draw (20) to (19);
		\draw (20) to (17);
		\draw (18.center) to (15);
	\end{pgfonlayer}
\end{tikzpicture}
=
\begin{tikzpicture}[tikzfig]
	\begin{pgfonlayer}{nodelayer}
		\node [style=X] (12) at (0, 7) {};
		\node [style=X] (13) at (-0.5, 7) {};
		\node [style=none] (14) at (0, 4.75) {};
		\node [style=zeroin] (15) at (0.5, 5.5) {};
		\node [style=oplus] (16) at (0.5, 6.25) {};
		\node [style=dot] (17) at (0, 6.25) {};
		\node [style=dot] (18) at (-0.5, 6.25) {};
		\node [style=none] (19) at (0.5, 7.25) {};
		\node [style=onein] (20) at (-0.5, 5.5) {};
	\end{pgfonlayer}
	\begin{pgfonlayer}{edgelayer}
		\draw [style=simple] (17) to (18);
		\draw [style=simple] (17) to (12);
		\draw [style=simple] (16) to (15);
		\draw [style=simple] (14.center) to (17);
		\draw [style=simple] (13) to (18);
		\draw (17) to (16);
		\draw (19.center) to (16);
		\draw (18) to (20);
	\end{pgfonlayer}
\end{tikzpicture}
\eq{\ref{TOF.1}}
\begin{tikzpicture}[tikzfig]
	\begin{pgfonlayer}{nodelayer}
		\node [style=X] (13) at (0, 7) {};
		\node [style=X] (14) at (-0.5, 7) {};
		\node [style=none] (15) at (0, 4.75) {};
		\node [style=zeroin] (16) at (0.5, 5.5) {};
		\node [style=none] (17) at (0.5, 7.25) {};
		\node [style=onein] (18) at (-0.5, 5.5) {};
		\node [style=oplus] (19) at (0.5, 6.25) {};
		\node [style=dot] (20) at (0, 6.25) {};
	\end{pgfonlayer}
	\begin{pgfonlayer}{edgelayer}
		\draw [style=simple] (19) to (16);
		\draw (17.center) to (19);
		\draw [style=simple] (15.center) to (20);
		\draw (20) to (19);
		\draw [style=simple] (20) to (13);
		\draw (14) to (18);
	\end{pgfonlayer}
\end{tikzpicture}
\eq{Rem. \ref{cor:copy}}
\begin{tikzpicture}[tikzfig]
	\begin{pgfonlayer}{nodelayer}
		\node [style=X] (14) at (0, 8) {};
		\node [style=none] (15) at (0, 5.75) {};
		\node [style=zeroin] (16) at (0.5, 6.5) {};
		\node [style=none] (17) at (0.5, 8.25) {};
		\node [style=oplus] (18) at (0.5, 7.25) {};
		\node [style=dot] (19) at (0, 7.25) {};
	\end{pgfonlayer}
	\begin{pgfonlayer}{edgelayer}
		\draw [style=simple] (18) to (16);
		\draw (17.center) to (18);
		\draw [style=simple] (15.center) to (19);
		\draw (19) to (18);
		\draw [style=simple] (19) to (14);
	\end{pgfonlayer}
\end{tikzpicture}\\
&\eq{Lem. \ref{lemma:whiteunit}}
\begin{tikzpicture}[tikzfig]
	\begin{pgfonlayer}{nodelayer}
		\node [style=none] (15) at (0, 5.75) {};
		\node [style=none] (16) at (0, 6.75) {};
	\end{pgfonlayer}
	\begin{pgfonlayer}{edgelayer}
		\draw (16.center) to (15.center);
	\end{pgfonlayer}
\end{tikzpicture}=
\left\llbracket
\begin{tikzpicture}[tikzfig]
	\begin{pgfonlayer}{nodelayer}
		\node [style=none] (16) at (0, 5.75) {};
		\node [style=none] (17) at (0, 6.75) {};
	\end{pgfonlayer}
	\begin{pgfonlayer}{edgelayer}
		\draw (17.center) to (16.center);
	\end{pgfonlayer}
\end{tikzpicture}
\right\rrbracket_{\ZXA}
\end{align*}

\item[\ref{ZXA.11}:]

\begin{align*}
\left\llbracket
\begin{tikzpicture}[tikzfig]
	\begin{pgfonlayer}{nodelayer}
		\node [style=andin] (17) at (0, 6.75) {};
		\node [style=none] (18) at (-0.25, 6.25) {};
		\node [style=none] (19) at (0.25, 6.25) {};
		\node [style=none] (20) at (0, 7.25) {};
		\node [style=none] (21) at (-0.25, 5.75) {};
		\node [style=none] (22) at (0.25, 5.75) {};
	\end{pgfonlayer}
	\begin{pgfonlayer}{edgelayer}
		\draw [in=-63, out=90] (19.center) to (17);
		\draw [in=90, out=-117, looseness=1.25] (17) to (18.center);
		\draw (20.center) to (17);
		\draw [in=-90, out=90, looseness=1.25] (22.center) to (18.center);
		\draw [in=90, out=-90, looseness=1.25] (19.center) to (21.center);
	\end{pgfonlayer}
\end{tikzpicture}
\right\rrbracket_{\ZXA}
=
\begin{tikzpicture}[tikzfig]
	\begin{pgfonlayer}{nodelayer}
		\node [style=none] (18) at (0.5, 5.75) {};
		\node [style=dot] (19) at (0, 6.75) {};
		\node [style=dot] (20) at (0.5, 6.75) {};
		\node [style=oplus] (21) at (1, 6.75) {};
		\node [style=X] (22) at (0, 7.5) {};
		\node [style=X] (23) at (0.5, 7.5) {};
		\node [style=none] (24) at (1, 7.75) {};
		\node [style=zeroin] (25) at (1, 6) {};
		\node [style=none] (26) at (0, 5.75) {};
	\end{pgfonlayer}
	\begin{pgfonlayer}{edgelayer}
		\draw (25) to (21);
		\draw (20) to (23);
		\draw (24.center) to (21);
		\draw (21) to (20);
		\draw (20) to (19);
		\draw (19) to (22);
		\draw [in=90, out=-90] (19) to (18.center);
		\draw [in=90, out=-90] (20) to (26.center);
	\end{pgfonlayer}
\end{tikzpicture}
\eq{\ref{TOF.15}}
\begin{tikzpicture}[tikzfig]
	\begin{pgfonlayer}{nodelayer}
		\node [style=none] (19) at (0.5, 6.75) {};
		\node [style=dot] (20) at (0, 7.75) {};
		\node [style=dot] (21) at (0.5, 7.75) {};
		\node [style=oplus] (22) at (1, 7.75) {};
		\node [style=X] (23) at (0.5, 8.75) {};
		\node [style=X] (24) at (0, 8.75) {};
		\node [style=none] (25) at (1, 8.75) {};
		\node [style=zeroin] (26) at (1, 7) {};
		\node [style=none] (27) at (0, 6.75) {};
		\node [style=none] (28) at (0.5, 5.75) {};
		\node [style=none] (29) at (0, 5.75) {};
	\end{pgfonlayer}
	\begin{pgfonlayer}{edgelayer}
		\draw (26) to (22);
		\draw [in=-90, out=90] (21) to (24);
		\draw (25.center) to (22);
		\draw (22) to (21);
		\draw (21) to (20);
		\draw [in=-90, out=90] (20) to (23);
		\draw [in=90, out=-90] (20) to (19.center);
		\draw [in=90, out=-90] (21) to (27.center);
		\draw [in=90, out=-90] (19.center) to (29.center);
		\draw [in=-90, out=90] (28.center) to (27.center);
	\end{pgfonlayer}
\end{tikzpicture}
=
\begin{tikzpicture}[tikzfig]
	\begin{pgfonlayer}{nodelayer}
		\node [style=none] (20) at (0, 5.75) {};
		\node [style=dot] (21) at (0, 6.75) {};
		\node [style=dot] (22) at (0.5, 6.75) {};
		\node [style=oplus] (23) at (1, 6.75) {};
		\node [style=X] (24) at (0, 7.5) {};
		\node [style=X] (25) at (0.5, 7.5) {};
		\node [style=none] (26) at (1, 7.75) {};
		\node [style=zeroin] (27) at (1, 6) {};
		\node [style=none] (28) at (0.5, 5.75) {};
	\end{pgfonlayer}
	\begin{pgfonlayer}{edgelayer}
		\draw (27) to (23);
		\draw (22) to (25);
		\draw (26.center) to (23);
		\draw (23) to (22);
		\draw (22) to (21);
		\draw (21) to (24);
		\draw (21) to (20.center);
		\draw (22) to (28.center);
	\end{pgfonlayer}
\end{tikzpicture}
=
\left\llbracket
\begin{tikzpicture}[tikzfig]
	\begin{pgfonlayer}{nodelayer}
		\node [style=andin] (21) at (0, 6.25) {};
		\node [style=none] (22) at (-0.25, 5.75) {};
		\node [style=none] (23) at (0.25, 5.75) {};
		\node [style=none] (24) at (0, 6.75) {};
	\end{pgfonlayer}
	\begin{pgfonlayer}{edgelayer}
		\draw [in=-63, out=90] (23.center) to (21);
		\draw [in=90, out=-117, looseness=1.25] (21) to (22.center);
		\draw (24.center) to (21);
	\end{pgfonlayer}
\end{tikzpicture}
\right\rrbracket_{\ZXA}
\end{align*}

\item[\ref{ZXA.12}:]

\begin{align*}
\left\llbracket
\begin{tikzpicture}[tikzfig]
	\begin{pgfonlayer}{nodelayer}
		\node [style=X] (22) at (-1, 6.25) {};
		\node [style=X] (23) at (-0.25, 6.25) {};
		\node [style=andin] (24) at (-0.25, 7.25) {};
		\node [style=andin] (25) at (-1, 7.25) {};
		\node [style=none] (26) at (-1, 7.75) {};
		\node [style=none] (27) at (-0.25, 7.75) {};
		\node [style=none] (28) at (-1, 5.75) {};
		\node [style=none] (29) at (-0.25, 5.75) {};
	\end{pgfonlayer}
	\begin{pgfonlayer}{edgelayer}
		\draw (29.center) to (23);
		\draw [in=-60, out=127] (23) to (25);
		\draw [in=120, out=-120, looseness=1.25] (25) to (22);
		\draw [in=-120, out=53] (22) to (24);
		\draw (24) to (27.center);
		\draw [in=60, out=-60, looseness=1.25] (24) to (23);
		\draw (22) to (28.center);
		\draw (25) to (26.center);
	\end{pgfonlayer}
\end{tikzpicture}
\right\rrbracket_{\ZXA}
&=
\begin{tikzpicture}[tikzfig]
	\begin{pgfonlayer}{nodelayer}
		\node [style=dot] (23) at (-0.5, 7.5) {};
		\node [style=dot] (24) at (-1, 7.5) {};
		\node [style=oplus] (25) at (-1.5, 7.5) {};
		\node [style=zeroin] (26) at (-1.5, 7) {};
		\node [style=X] (27) at (-0.5, 8) {};
		\node [style=X] (28) at (-1, 8) {};
		\node [style=none] (29) at (-1.5, 8.25) {};
		\node [style=dot] (30) at (0.5, 7.5) {};
		\node [style=oplus] (31) at (1, 7.5) {};
		\node [style=none] (32) at (1, 8.25) {};
		\node [style=dot] (33) at (0, 7.5) {};
		\node [style=X] (34) at (0.5, 8) {};
		\node [style=X] (35) at (0, 8) {};
		\node [style=zeroin] (36) at (1, 7) {};
		\node [style=fanout] (37) at (-0.75, 6.5) {};
		\node [style=fanout] (38) at (0.25, 6.5) {};
		\node [style=none] (39) at (0.25, 5.75) {};
		\node [style=none] (40) at (-0.75, 5.75) {};
	\end{pgfonlayer}
	\begin{pgfonlayer}{edgelayer}
		\draw (27) to (23);
		\draw (23) to (24);
		\draw (24) to (28);
		\draw (29.center) to (25);
		\draw (25) to (26);
		\draw (25) to (24);
		\draw (35) to (33);
		\draw (33) to (30);
		\draw (30) to (34);
		\draw (32.center) to (31);
		\draw (31) to (36);
		\draw (31) to (30);
		\draw [in=99, out=-90] (24) to (37);
		\draw [in=-90, out=63] (37) to (33);
		\draw [in=117, out=-90] (23) to (38);
		\draw [in=-90, out=81] (38) to (30);
		\draw (37) to (40.center);
		\draw (39.center) to (38);
	\end{pgfonlayer}
\end{tikzpicture}
\eq{\ref{TOF.4}}
\begin{tikzpicture}[tikzfig]
	\begin{pgfonlayer}{nodelayer}
		\node [style=X] (26) at (-0.5, 8.5) {};
		\node [style=X] (27) at (-1, 8.5) {};
		\node [style=none] (28) at (-1.5, 8.75) {};
		\node [style=dot] (29) at (0.5, 8) {};
		\node [style=oplus] (30) at (1, 8) {};
		\node [style=none] (31) at (1, 8.75) {};
		\node [style=dot] (32) at (0, 8) {};
		\node [style=X] (33) at (0.5, 8.5) {};
		\node [style=X] (34) at (0, 8.5) {};
		\node [style=zeroin] (35) at (1, 7.5) {};
		\node [style=fanout] (36) at (-0.5, 7) {};
		\node [style=fanout] (37) at (0, 7) {};
		\node [style=none] (38) at (0, 5.75) {};
		\node [style=none] (39) at (-0.5, 5.75) {};
		\node [style=dot] (40) at (0, 6.5) {};
		\node [style=zeroin] (41) at (-1.5, 5.75) {};
		\node [style=oplus] (42) at (-1.5, 6.5) {};
		\node [style=dot] (43) at (-0.5, 6.5) {};
	\end{pgfonlayer}
	\begin{pgfonlayer}{edgelayer}
		\draw (34) to (32);
		\draw (32) to (29);
		\draw (29) to (33);
		\draw (31.center) to (30);
		\draw (30) to (35);
		\draw (30) to (29);
		\draw [in=-90, out=63] (36) to (32);
		\draw [in=-90, out=60] (37) to (29);
		\draw (36) to (39.center);
		\draw (38.center) to (37);
		\draw (40) to (43);
		\draw (42) to (41);
		\draw (42) to (43);
		\draw [in=-90, out=117] (37) to (26);
		\draw [in=120, out=-90] (27) to (36);
		\draw (42) to (28.center);
	\end{pgfonlayer}
\end{tikzpicture}
\eq{unit}
\begin{tikzpicture}[tikzfig]
	\begin{pgfonlayer}{nodelayer}
		\node [style=none] (27) at (-1, 8.25) {};
		\node [style=dot] (28) at (0, 7.25) {};
		\node [style=oplus] (29) at (0.5, 7.25) {};
		\node [style=none] (30) at (0.5, 8.25) {};
		\node [style=dot] (31) at (-0.5, 7.25) {};
		\node [style=X] (32) at (0, 7.75) {};
		\node [style=X] (33) at (-0.5, 7.75) {};
		\node [style=zeroin] (34) at (0.5, 6.75) {};
		\node [style=none] (35) at (0, 5.75) {};
		\node [style=none] (36) at (-0.5, 5.75) {};
		\node [style=dot] (37) at (0, 6.75) {};
		\node [style=zeroin] (38) at (-1, 6) {};
		\node [style=oplus] (39) at (-1, 6.75) {};
		\node [style=dot] (40) at (-0.5, 6.75) {};
	\end{pgfonlayer}
	\begin{pgfonlayer}{edgelayer}
		\draw (33) to (31);
		\draw (31) to (28);
		\draw (28) to (32);
		\draw (30.center) to (29);
		\draw (29) to (34);
		\draw (29) to (28);
		\draw (37) to (40);
		\draw (39) to (38);
		\draw (39) to (40);
		\draw (39) to (27.center);
		\draw (31) to (40);
		\draw (37) to (28);
		\draw (37) to (35.center);
		\draw (36.center) to (40);
	\end{pgfonlayer}
\end{tikzpicture}
=
\begin{tikzpicture}[tikzfig]
	\begin{pgfonlayer}{nodelayer}
		\node [style=none] (28) at (-1, 8.25) {};
		\node [style=dot] (29) at (-1.5, 7.25) {};
		\node [style=oplus] (30) at (-0.5, 7.25) {};
		\node [style=none] (31) at (-0.5, 8.25) {};
		\node [style=dot] (32) at (-2, 7.25) {};
		\node [style=X] (33) at (-1.5, 7.75) {};
		\node [style=X] (34) at (-2, 7.75) {};
		\node [style=zeroin] (35) at (-0.5, 6.75) {};
		\node [style=none] (36) at (-1.5, 5.75) {};
		\node [style=none] (37) at (-2, 5.75) {};
		\node [style=dot] (38) at (-1.5, 6.75) {};
		\node [style=zeroin] (39) at (-1, 6) {};
		\node [style=oplus] (40) at (-1, 6.75) {};
		\node [style=dot] (41) at (-2, 6.75) {};
	\end{pgfonlayer}
	\begin{pgfonlayer}{edgelayer}
		\draw (34) to (32);
		\draw (32) to (29);
		\draw (29) to (33);
		\draw (31.center) to (30);
		\draw (30) to (35);
		\draw (30) to (29);
		\draw (38) to (41);
		\draw (40) to (39);
		\draw (40) to (41);
		\draw (40) to (28.center);
		\draw (32) to (41);
		\draw (38) to (29);
		\draw (38) to (36.center);
		\draw (37.center) to (41);
	\end{pgfonlayer}
\end{tikzpicture}\\
&\eq{\ref{TOF.2}}
\begin{tikzpicture}[tikzfig]
	\begin{pgfonlayer}{nodelayer}
		\node [style=none] (29) at (-1, 8.75) {};
		\node [style=dot] (30) at (-1.5, 7.75) {};
		\node [style=oplus] (31) at (-0.5, 7.75) {};
		\node [style=none] (32) at (-0.5, 8.75) {};
		\node [style=dot] (33) at (-2, 7.75) {};
		\node [style=X] (34) at (-1.5, 8.25) {};
		\node [style=X] (35) at (-2, 8.25) {};
		\node [style=zeroin] (36) at (-0.5, 6) {};
		\node [style=none] (37) at (-1.5, 5.75) {};
		\node [style=none] (38) at (-2, 5.75) {};
		\node [style=dot] (39) at (-1.5, 7.25) {};
		\node [style=zeroin] (40) at (-1, 6) {};
		\node [style=oplus] (41) at (-1, 7.25) {};
		\node [style=dot] (42) at (-2, 7.25) {};
		\node [style=dot] (43) at (-1, 6.75) {};
		\node [style=oplus] (44) at (-0.5, 6.75) {};
	\end{pgfonlayer}
	\begin{pgfonlayer}{edgelayer}
		\draw (35) to (33);
		\draw (33) to (30);
		\draw (30) to (34);
		\draw (32.center) to (31);
		\draw (31) to (36);
		\draw (31) to (30);
		\draw (39) to (42);
		\draw (41) to (40);
		\draw (41) to (42);
		\draw (41) to (29.center);
		\draw (33) to (42);
		\draw (39) to (30);
		\draw (39) to (37.center);
		\draw (38.center) to (42);
		\draw (44) to (43);
	\end{pgfonlayer}
\end{tikzpicture}
\eq{Lem \ref{lemma:Iwama}}
\begin{tikzpicture}[tikzfig]
	\begin{pgfonlayer}{nodelayer}
		\node [style=dot] (30) at (0.25, 6.75) {};
		\node [style=zeroin] (31) at (1.25, 6) {};
		\node [style=none] (32) at (1.75, 7.75) {};
		\node [style=oplus] (33) at (1.25, 6.75) {};
		\node [style=none] (34) at (1.25, 7.75) {};
		\node [style=X] (35) at (0.25, 7.25) {};
		\node [style=X] (36) at (0.75, 7.25) {};
		\node [style=zeroin] (37) at (1.75, 6) {};
		\node [style=none] (38) at (0.25, 5.75) {};
		\node [style=none] (39) at (0.75, 5.75) {};
		\node [style=dot] (40) at (0.75, 6.75) {};
		\node [style=dot] (41) at (1.25, 7.25) {};
		\node [style=oplus] (42) at (1.75, 7.25) {};
	\end{pgfonlayer}
	\begin{pgfonlayer}{edgelayer}
		\draw (40) to (30);
		\draw (33) to (31);
		\draw (33) to (30);
		\draw (33) to (34.center);
		\draw (40) to (39.center);
		\draw (38.center) to (30);
		\draw (42) to (41);
		\draw (32.center) to (42);
		\draw (42) to (37);
		\draw (36) to (40);
		\draw (30) to (35);
	\end{pgfonlayer}
\end{tikzpicture}=
\left\llbracket
\begin{tikzpicture}[tikzfig]
	\begin{pgfonlayer}{nodelayer}
		\node [style=andin] (31) at (-1, 6.25) {};
		\node [style=none] (32) at (-1.25, 5.75) {};
		\node [style=none] (33) at (-0.75, 5.75) {};
		\node [style=X] (34) at (-1, 7) {};
		\node [style=none] (35) at (-1.25, 7.5) {};
		\node [style=none] (36) at (-0.75, 7.5) {};
	\end{pgfonlayer}
	\begin{pgfonlayer}{edgelayer}
		\draw [in=63, out=-90] (36.center) to (34);
		\draw (34) to (31);
		\draw [in=90, out=-117] (31) to (32.center);
		\draw [in=-63, out=90] (33.center) to (31);
		\draw [in=-90, out=117] (34) to (35.center);
	\end{pgfonlayer}
\end{tikzpicture}
\right\rrbracket_{\ZXA}
\end{align*}



\item[\ref{ZXA.13}:]
\begin{align*}
\left\llbracket
\begin{tikzpicture}[tikzfig]
	\begin{pgfonlayer}{nodelayer}
		\node [style=none] (32) at (-0.5, 6.5) {};
		\node [style=none] (33) at (0, 5.75) {};
		\node [style=none] (34) at (-1, 5.75) {};
		\node [style=X] (35) at (-0.5, 7.25) {};
		\node [style=andin] (36) at (-0.5, 6.5) {};
	\end{pgfonlayer}
	\begin{pgfonlayer}{edgelayer}
		\draw [in=90, out=-135] (32.center) to (34.center);
		\draw [in=-41, out=90] (33.center) to (32.center);
		\draw (35) to (32.center);
	\end{pgfonlayer}
\end{tikzpicture}
\right\rrbracket_{\ZXA}
		&=
\begin{tikzpicture}[tikzfig]
	\begin{pgfonlayer}{nodelayer}
		\node [style=none] (33) at (-0.5, 5.75) {};
		\node [style=none] (34) at (-1, 5.75) {};
		\node [style=dot] (35) at (-1, 6.5) {};
		\node [style=dot] (36) at (-0.5, 6.5) {};
		\node [style=oplus] (37) at (0, 6.5) {};
		\node [style=zeroin] (38) at (0, 6) {};
		\node [style=X] (39) at (0, 7) {};
		\node [style=X] (40) at (-0.5, 7) {};
		\node [style=X] (41) at (-1, 7) {};
	\end{pgfonlayer}
	\begin{pgfonlayer}{edgelayer}
		\draw (37) to (35);
		\draw (41) to (34.center);
		\draw (33.center) to (40);
		\draw (39) to (38);
	\end{pgfonlayer}
\end{tikzpicture}
\eq{\ref{TOF.2}}
\begin{tikzpicture}[tikzfig]
	\begin{pgfonlayer}{nodelayer}
		\node [style=none] (34) at (-0.5, 5.75) {};
		\node [style=none] (35) at (-1, 5.75) {};
		\node [style=zeroin] (36) at (0, 6) {};
		\node [style=X] (37) at (0, 6.75) {};
		\node [style=X] (38) at (-0.5, 6.75) {};
		\node [style=X] (39) at (-1, 6.75) {};
	\end{pgfonlayer}
	\begin{pgfonlayer}{edgelayer}
		\draw (39) to (35.center);
		\draw (34.center) to (38);
		\draw (37) to (36);
	\end{pgfonlayer}
\end{tikzpicture}
\eq{Rem. \ref{cor:copy}}
\begin{tikzpicture}[tikzfig]
	\begin{pgfonlayer}{nodelayer}
		\node [style=none] (35) at (-0.5, 5.75) {};
		\node [style=none] (36) at (-1, 5.75) {};
		\node [style=X] (37) at (-0.5, 6.75) {};
		\node [style=X] (38) at (-1, 6.75) {};
	\end{pgfonlayer}
	\begin{pgfonlayer}{edgelayer}
		\draw (38) to (36.center);
		\draw (35.center) to (37);
	\end{pgfonlayer}
\end{tikzpicture}
=
\left\llbracket
\begin{tikzpicture}[tikzfig]
	\begin{pgfonlayer}{nodelayer}
		\node [style=none] (36) at (-0.5, 5.75) {};
		\node [style=none] (37) at (-1, 5.75) {};
		\node [style=X] (38) at (-1, 6.5) {};
		\node [style=X] (39) at (-0.5, 6.5) {};
	\end{pgfonlayer}
	\begin{pgfonlayer}{edgelayer}
		\draw (39) to (36.center);
		\draw (38) to (37.center);
	\end{pgfonlayer}
\end{tikzpicture}
\right\rrbracket_{\ZXA}
\end{align*}


\item[\ref{ZXA.14}:]

$$
\left\llbracket
\begin{tikzpicture}[tikzfig]
	\begin{pgfonlayer}{nodelayer}
		\node [style=none] (37) at (-0.25, 7.25) {};
		\node [style=X] (38) at (0, 6.5) {};
		\node [style=Z] (39) at (0, 5.75) {$\pi$};
		\node [style=none] (40) at (0.25, 7.25) {};
	\end{pgfonlayer}
	\begin{pgfonlayer}{edgelayer}
		\draw [style=simple, in=-90, out=124] (38) to (37.center);
		\draw [style=simple, in=60, out=-90] (40.center) to (38);
		\draw [style=simple] (38) to (39);
	\end{pgfonlayer}
\end{tikzpicture}
\right\rrbracket_{\ZXA}
=
\begin{tikzpicture}[tikzfig]
	\begin{pgfonlayer}{nodelayer}
		\node [style=dot] (38) at (1, 6.25) {};
		\node [style=oplus] (39) at (1.75, 6.25) {};
		\node [style=onein] (40) at (1, 5.75) {};
		\node [style=zeroin] (41) at (1.75, 5.75) {};
		\node [style=none] (42) at (1, 6.75) {};
		\node [style=none] (43) at (1.75, 6.75) {};
	\end{pgfonlayer}
	\begin{pgfonlayer}{edgelayer}
		\draw (43.center) to (39);
		\draw (39) to (41);
		\draw (39) to (38);
		\draw (38) to (42.center);
		\draw (38) to (40);
	\end{pgfonlayer}
\end{tikzpicture}
\eq{\ref{TOF.1}}
\begin{tikzpicture}[tikzfig]
	\begin{pgfonlayer}{nodelayer}
		\node [style=onein] (39) at (1, 5.75) {};
		\node [style=none] (40) at (1, 6.75) {};
		\node [style=none] (41) at (1.75, 6.75) {};
		\node [style=onein] (42) at (1.75, 5.75) {};
	\end{pgfonlayer}
	\begin{pgfonlayer}{edgelayer}
		\draw (39) to (40.center);
		\draw (41.center) to (42);
	\end{pgfonlayer}
\end{tikzpicture}
=
\left\llbracket
\begin{tikzpicture}[tikzfig]
	\begin{pgfonlayer}{nodelayer}
		\node [style=none] (40) at (-0.25, 6.25) {};
		\node [style=Z] (41) at (-0.25, 5.75) {$\pi$};
		\node [style=none] (42) at (0.25, 6.25) {};
		\node [style=Z] (43) at (0.25, 5.75) {$\pi$};
	\end{pgfonlayer}
	\begin{pgfonlayer}{edgelayer}
		\draw [style=simple] (43) to (42.center);
		\draw [style=simple] (41) to (40.center);
	\end{pgfonlayer}
\end{tikzpicture}
\right\rrbracket_{\ZXA}
$$
%
%$$
%\left\llbracket
%\begin{tikzpicture}
%	\begin{pgfonlayer}{nodelayer}
%		\node [style=X] (0) at (3, -0.25) {};
%		\node [style=none] (1) at (2, -0) {};
%		\node [style=none] (2) at (2, -0.5) {};
%		\node [style=Z] (3) at (3.75, -0.25) {$\pi$};
%		\node [style=none] (4) at (4.5, -0.25) {};
%	\end{pgfonlayer}
%	\begin{pgfonlayer}{edgelayer}
%		\draw (3) to (0);
%		\draw [in=0, out=166, looseness=1.00] (0) to (1.center);
%		\draw [in=-166, out=0, looseness=1.00] (2.center) to (0);
%		\draw (4.center) to (3);
%	\end{pgfonlayer}
%\end{tikzpicture}
%\right\rrbracket_{\ZXA}
%=
%\begin{tikzpicture}
%	\begin{pgfonlayer}{nodelayer}
%		\node [style=fanin] (0) at (2, -0) {};
%		\node [style=oplus] (1) at (2.5, -0) {};
%		\node [style=none] (2) at (3, -0) {};
%		\node [style=none] (3) at (1.25, 0.25) {};
%		\node [style=none] (4) at (1.25, -0.25) {};
%	\end{pgfonlayer}
%	\begin{pgfonlayer}{edgelayer}
%		\draw (2.center) to (1);
%		\draw (1) to (0);
%		\draw [in=0, out=162, looseness=1.00] (0) to (3.center);
%		\draw [in=-162, out=0, looseness=1.00] (4.center) to (0);
%	\end{pgfonlayer}
%\end{tikzpicture}
%\eq{nat.}
%\begin{tikzpicture}
%	\begin{pgfonlayer}{nodelayer}
%		\node [style=fanin] (0) at (2, -0) {};
%		\node [style=none] (1) at (2.5, -0) {};
%		\node [style=none] (2) at (1.25, 0.25) {};
%		\node [style=none] (3) at (1.25, -0.25) {};
%		\node [style=oplus] (4) at (1.25, 0.25) {};
%		\node [style=oplus] (5) at (1.25, -0.25) {};
%		\node [style=none] (6) at (0.5, 0.25) {};
%		\node [style=none] (7) at (0.5, -0.25) {};
%	\end{pgfonlayer}
%	\begin{pgfonlayer}{edgelayer}
%		\draw [in=0, out=162, looseness=1.00] (0) to (2.center);
%		\draw [in=-162, out=0, looseness=1.00] (3.center) to (0);
%		\draw (1.center) to (0);
%		\draw (2.center) to (6.center);
%		\draw (7.center) to (3.center);
%	\end{pgfonlayer}
%\end{tikzpicture}
%=
%\left\llbracket
%\begin{tikzpicture}
%	\begin{pgfonlayer}{nodelayer}
%		\node [style=none] (0) at (2, -0) {};
%		\node [style=none] (1) at (2, -0.5) {};
%		\node [style=Z] (2) at (2.75, -0) {$\pi$};
%		\node [style=Z] (3) at (2.75, -0.5) {$\pi$};
%		\node [style=X] (4) at (3.5, -0.25) {};
%		\node [style=none] (5) at (4.25, -0.25) {};
%	\end{pgfonlayer}
%	\begin{pgfonlayer}{edgelayer}
%		\draw (3) to (1.center);
%		\draw (2) to (0.center);
%		\draw [in=162, out=0, looseness=1.00] (2) to (4);
%		\draw (4) to (5.center);
%		\draw [in=0, out=-162, looseness=1.00] (4) to (3);
%	\end{pgfonlayer}
%\end{tikzpicture}
%\right\rrbracket_{\ZXA}
%$$
%
%
%$$
%\left\llbracket
%\begin{tikzpicture}
%	\begin{pgfonlayer}{nodelayer}
%		\node [style=Z] (0) at (3, -0.25) {};
%		\node [style=none] (1) at (2, -0) {};
%		\node [style=none] (2) at (2, -0.5) {};
%		\node [style=X] (3) at (3.75, -0.25) {};
%	\end{pgfonlayer}
%	\begin{pgfonlayer}{edgelayer}
%		\draw (3) to (0);
%		\draw [in=0, out=166, looseness=1.00] (0) to (1.center);
%		\draw [in=-166, out=0, looseness=1.00] (2.center) to (0);
%	\end{pgfonlayer}
%\end{tikzpicture}
%\right\rrbracket_{\ZXA}
%=
%\begin{tikzpicture}
%	\begin{pgfonlayer}{nodelayer}
%		\node [style=dot] (0) at (1, -0) {};
%		\node [style=oplus] (1) at (1, -0.5) {};
%		\node [style=X] (2) at (1.5, -0) {};
%		\node [style=X] (3) at (1.5, -0.5) {};
%		\node [style=none] (4) at (0.5, -0.5) {};
%		\node [style=none] (5) at (0.5, -0) {};
%	\end{pgfonlayer}
%	\begin{pgfonlayer}{edgelayer}
%		\draw (0) to (1);
%		\draw (3) to (1);
%		\draw (1) to (4.center);
%		\draw (5.center) to (0);
%		\draw (0) to (2);
%	\end{pgfonlayer}
%\end{tikzpicture}
%\eq{Lem. \ref{lemma:whiteunit}}
%\begin{tikzpicture}
%	\begin{pgfonlayer}{nodelayer}
%		\node [style=X] (0) at (1.5, -0) {};
%		\node [style=X] (1) at (1.5, -0.5) {};
%		\node [style=none] (2) at (0.5, -0.5) {};
%		\node [style=none] (3) at (0.5, -0) {};
%	\end{pgfonlayer}
%	\begin{pgfonlayer}{edgelayer}
%		\draw (1) to (2.center);
%		\draw (3.center) to (0);
%	\end{pgfonlayer}
%\end{tikzpicture}
%=
%\left\llbracket
%\begin{tikzpicture}
%	\begin{pgfonlayer}{nodelayer}
%		\node [style=none] (0) at (2, -0) {};
%		\node [style=none] (1) at (2, -0.5) {};
%		\node [style=X] (2) at (2.75, -0) {};
%		\node [style=X] (3) at (2.75, -0.5) {};
%	\end{pgfonlayer}
%	\begin{pgfonlayer}{edgelayer}
%		\draw (3) to (1.center);
%		\draw (2) to (0.center);
%	\end{pgfonlayer}
%\end{tikzpicture}
%\right\rrbracket_{\ZXA}
%$$
%
%
%$$
%\left\llbracket
%\begin{tikzpicture}
%	\begin{pgfonlayer}{nodelayer}
%		\node [style=X] (0) at (3, -0.25) {};
%		\node [style=none] (1) at (2, -0) {};
%		\node [style=none] (2) at (2, -0.5) {};
%		\node [style=Z] (3) at (3.75, -0.25) {};
%	\end{pgfonlayer}
%	\begin{pgfonlayer}{edgelayer}
%		\draw (3) to (0);
%		\draw [in=0, out=166, looseness=1.00] (0) to (1.center);
%		\draw [in=-166, out=0, looseness=1.00] (2.center) to (0);
%	\end{pgfonlayer}
%\end{tikzpicture}
%\right\rrbracket_{\ZXA}
%=
%\begin{tikzpicture}
%	\begin{pgfonlayer}{nodelayer}
%		\node [style=oplus] (0) at (3, -1.75) {};
%		\node [style=dot] (1) at (3, -1) {};
%		\node [style=zeroout] (2) at (3.5, -1) {};
%		\node [style=zeroout] (3) at (3.5, -1.75) {};
%		\node [style=none] (4) at (2.5, -1) {};
%		\node [style=none] (5) at (2.5, -1.75) {};
%	\end{pgfonlayer}
%	\begin{pgfonlayer}{edgelayer}
%		\draw (3) to (0);
%		\draw (0) to (5.center);
%		\draw (4.center) to (1);
%		\draw (1) to (2);
%		\draw (1) to (0);
%	\end{pgfonlayer}
%\end{tikzpicture}
%\eq{\ref{TOF.2}}
%\left\llbracket
%\begin{tikzpicture}
%	\begin{pgfonlayer}{nodelayer}
%		\node [style=none] (0) at (2, -0) {};
%		\node [style=none] (1) at (2, -0.5) {};
%		\node [style=Z] (2) at (2.75, -0) {};
%		\node [style=Z] (3) at (2.75, -0.5) {};
%	\end{pgfonlayer}
%	\begin{pgfonlayer}{edgelayer}
%		\draw (3) to (1.center);
%		\draw (2) to (0.center);
%	\end{pgfonlayer}
%\end{tikzpicture}
%\right\rrbracket_{\ZXA}
%$$



\item[\ref{ZXA.15}:]
\begin{align*}
\left\llbracket
\begin{tikzpicture}[tikzfig]
	\begin{pgfonlayer}{nodelayer}
		\node [style=X] (41) at (-1, 6.25) {};
		\node [style=andin] (42) at (-1, 7.25) {};
		\node [style=none] (43) at (-1, 7.75) {};
		\node [style=none] (44) at (-1, 5.75) {};
	\end{pgfonlayer}
	\begin{pgfonlayer}{edgelayer}
		\draw (43.center) to (42);
		\draw [in=120, out=-120, looseness=1.25] (42) to (41);
		\draw [in=-60, out=60, looseness=1.25] (41) to (42);
		\draw (41) to (44.center);
	\end{pgfonlayer}
\end{tikzpicture}
\right\rrbracket_{\ZXA}
&=
\begin{tikzpicture}[tikzfig]
	\begin{pgfonlayer}{nodelayer}
		\node [style=dot] (42) at (0, 7) {};
		\node [style=dot] (43) at (0.5, 7) {};
		\node [style=oplus] (44) at (1, 7) {};
		\node [style=X] (45) at (0, 7.5) {};
		\node [style=X] (46) at (0.5, 7.5) {};
		\node [style=zeroin] (47) at (1, 6.5) {};
		\node [style=none] (48) at (1, 7.75) {};
		\node [style=dot] (49) at (0, 6.5) {};
		\node [style=oplus] (50) at (0.5, 6.5) {};
		\node [style=zeroin] (51) at (0.5, 6) {};
		\node [style=none] (52) at (0, 5.75) {};
	\end{pgfonlayer}
	\begin{pgfonlayer}{edgelayer}
		\draw (48.center) to (44);
		\draw (44) to (47);
		\draw (44) to (43);
		\draw (43) to (46);
		\draw (45) to (42);
		\draw (42) to (43);
		\draw (43) to (50);
		\draw (50) to (49);
		\draw (49) to (42);
		\draw (50) to (51);
		\draw (49) to (52.center);
	\end{pgfonlayer}
\end{tikzpicture}
\eq{Lem. \ref{lemma:Iwama}}
\begin{tikzpicture}[tikzfig]
	\begin{pgfonlayer}{nodelayer}
		\node [style=dot] (43) at (0, 7) {};
		\node [style=dot] (44) at (0.5, 7) {};
		\node [style=oplus] (45) at (1, 7) {};
		\node [style=X] (46) at (0, 8) {};
		\node [style=X] (47) at (0.5, 8) {};
		\node [style=zeroin] (48) at (1, 6) {};
		\node [style=none] (49) at (1, 8.25) {};
		\node [style=zeroin] (50) at (0.5, 6) {};
		\node [style=none] (51) at (0, 5.75) {};
		\node [style=dot] (52) at (0, 7.5) {};
		\node [style=oplus] (53) at (0.5, 7.5) {};
		\node [style=dot] (54) at (0, 6.5) {};
		\node [style=oplus] (55) at (1, 6.5) {};
	\end{pgfonlayer}
	\begin{pgfonlayer}{edgelayer}
		\draw (49.center) to (45);
		\draw (45) to (48);
		\draw (45) to (44);
		\draw (44) to (47);
		\draw (46) to (43);
		\draw (43) to (44);
		\draw (53) to (52);
		\draw (55) to (54);
		\draw (44) to (50);
		\draw (51.center) to (54);
		\draw (54) to (43);
	\end{pgfonlayer}
\end{tikzpicture}
\eq{Rem. \ref{cor:copy}}
\begin{tikzpicture}[tikzfig]
	\begin{pgfonlayer}{nodelayer}
		\node [style=dot] (44) at (0, 7) {};
		\node [style=dot] (45) at (0.5, 7) {};
		\node [style=oplus] (46) at (1, 7) {};
		\node [style=X] (47) at (0, 7.5) {};
		\node [style=X] (48) at (0.5, 7.5) {};
		\node [style=zeroin] (49) at (1, 6) {};
		\node [style=none] (50) at (1, 7.75) {};
		\node [style=zeroin] (51) at (0.5, 6) {};
		\node [style=none] (52) at (0, 5.75) {};
		\node [style=dot] (53) at (0, 6.5) {};
		\node [style=oplus] (54) at (1, 6.5) {};
	\end{pgfonlayer}
	\begin{pgfonlayer}{edgelayer}
		\draw (50.center) to (46);
		\draw (46) to (49);
		\draw (46) to (45);
		\draw (45) to (48);
		\draw (47) to (44);
		\draw (44) to (45);
		\draw (54) to (53);
		\draw (45) to (51);
		\draw (52.center) to (53);
		\draw (53) to (44);
	\end{pgfonlayer}
\end{tikzpicture}
\eq{\ref{TOF.2}}
\begin{tikzpicture}[tikzfig]
	\begin{pgfonlayer}{nodelayer}
		\node [style=X] (45) at (0, 7) {};
		\node [style=X] (46) at (0.5, 7) {};
		\node [style=zeroin] (47) at (1, 6) {};
		\node [style=none] (48) at (1, 7.25) {};
		\node [style=zeroin] (49) at (0.5, 6) {};
		\node [style=none] (50) at (0, 5.75) {};
		\node [style=dot] (51) at (0, 6.5) {};
		\node [style=oplus] (52) at (1, 6.5) {};
	\end{pgfonlayer}
	\begin{pgfonlayer}{edgelayer}
		\draw (52) to (51);
		\draw (50.center) to (51);
		\draw (48.center) to (52);
		\draw (52) to (47);
		\draw (49) to (46);
		\draw (45) to (51);
	\end{pgfonlayer}
\end{tikzpicture}\\
&\eq{Rem. \ref{cor:copy}}
\begin{tikzpicture}[tikzfig]
	\begin{pgfonlayer}{nodelayer}
		\node [style=X] (46) at (0, 7) {};
		\node [style=zeroin] (47) at (0.5, 6) {};
		\node [style=none] (48) at (0.5, 7.25) {};
		\node [style=none] (49) at (0, 5.75) {};
		\node [style=dot] (50) at (0, 6.5) {};
		\node [style=oplus] (51) at (0.5, 6.5) {};
	\end{pgfonlayer}
	\begin{pgfonlayer}{edgelayer}
		\draw (51) to (50);
		\draw (49.center) to (50);
		\draw (48.center) to (51);
		\draw (51) to (47);
		\draw (46) to (50);
	\end{pgfonlayer}
\end{tikzpicture}
\eq{Lem. \ref{lemma:whiteunit}}
\begin{tikzpicture}
	\begin{pgfonlayer}{nodelayer}
		\node [style=none] (0) at (2.5, -0) {};
		\node [style=none] (1) at (1.75, -0) {};
	\end{pgfonlayer}
	\begin{pgfonlayer}{edgelayer}
		\draw (0.center) to (1.center);
	\end{pgfonlayer}
\end{tikzpicture}
=
\left\llbracket
\begin{tikzpicture}[tikzfig]
	\begin{pgfonlayer}{nodelayer}
		\node [style=none] (47) at (-1, 6.75) {};
		\node [style=none] (48) at (-1, 5.75) {};
	\end{pgfonlayer}
	\begin{pgfonlayer}{edgelayer}
		\draw (47.center) to (48.center);
	\end{pgfonlayer}
\end{tikzpicture}
\right\rrbracket_{\ZXA}
\end{align*}


\item[\ref{ZXA.16}:]
This is precisely \ref{TOF.7}.

\item[\ref{ZXA.17}:]
\begin{align*}
\left\llbracket
\begin{tikzpicture}[tikzfig]
	\begin{pgfonlayer}{nodelayer}
		\node [style=Z] (48) at (0, 6.25) {};
		\node [style=andin] (49) at (-0.25, 7) {};
		\node [style=none] (50) at (-0.5, 6.25) {};
		\node [style=none] (51) at (-0.25, 5.75) {};
		\node [style=none] (52) at (0.25, 5.75) {};
		\node [style=none] (53) at (-0.5, 5.75) {};
		\node [style=none] (54) at (-0.25, 7.5) {};
	\end{pgfonlayer}
	\begin{pgfonlayer}{edgelayer}
		\draw [in=-72, out=90] (48) to (49);
		\draw (49) to (54.center);
		\draw [in=90, out=-108] (49) to (50.center);
		\draw (50.center) to (53.center);
		\draw [in=90, out=-117] (48) to (51.center);
		\draw [in=90, out=-63] (48) to (52.center);
	\end{pgfonlayer}
\end{tikzpicture}
\right\rrbracket_{\ZXA}
&=
\begin{tikzpicture}[tikzfig]
	\begin{pgfonlayer}{nodelayer}
		\node [style=X] (49) at (1.5, 6.75) {};
		\node [style=oplus] (50) at (1.5, 8) {};
		\node [style=zeroin] (51) at (1.5, 7.5) {};
		\node [style=dot] (52) at (1, 8) {};
		\node [style=dot] (53) at (0.5, 8) {};
		\node [style=X] (54) at (1, 8.5) {};
		\node [style=X] (55) at (0.5, 8.5) {};
		\node [style=dot] (56) at (1.5, 6.25) {};
		\node [style=oplus] (57) at (1, 6.25) {};
		\node [style=none] (58) at (1.5, 5.75) {};
		\node [style=none] (59) at (1, 5.75) {};
		\node [style=none] (60) at (0.5, 5.75) {};
		\node [style=none] (61) at (1.5, 8.75) {};
	\end{pgfonlayer}
	\begin{pgfonlayer}{edgelayer}
		\draw (56) to (57);
		\draw (57) to (59.center);
		\draw (58.center) to (56);
		\draw (61.center) to (50);
		\draw (50) to (51);
		\draw (50) to (52);
		\draw (52) to (54);
		\draw (52) to (53);
		\draw (53) to (55);
		\draw (53) to (60.center);
		\draw (57) to (52);
		\draw (49) to (56);
	\end{pgfonlayer}
\end{tikzpicture}
=
\begin{tikzpicture}[tikzfig]
	\begin{pgfonlayer}{nodelayer}
		\node [style=X] (50) at (2, 7.5) {};
		\node [style=oplus] (51) at (1.5, 7) {};
		\node [style=zeroin] (52) at (1.5, 6) {};
		\node [style=dot] (53) at (1, 7) {};
		\node [style=dot] (54) at (0.5, 7) {};
		\node [style=X] (55) at (1, 7.5) {};
		\node [style=X] (56) at (0.5, 7.5) {};
		\node [style=dot] (57) at (2, 6.5) {};
		\node [style=oplus] (58) at (1, 6.5) {};
		\node [style=none] (59) at (2, 5.75) {};
		\node [style=none] (60) at (1, 5.75) {};
		\node [style=none] (61) at (0.5, 5.75) {};
		\node [style=none] (62) at (1.5, 7.75) {};
	\end{pgfonlayer}
	\begin{pgfonlayer}{edgelayer}
		\draw (57) to (58);
		\draw (58) to (60.center);
		\draw (59.center) to (57);
		\draw (62.center) to (51);
		\draw (51) to (52);
		\draw (51) to (53);
		\draw (53) to (55);
		\draw (53) to (54);
		\draw (54) to (56);
		\draw (54) to (61.center);
		\draw (58) to (53);
		\draw (50) to (57);
	\end{pgfonlayer}
\end{tikzpicture}\
\eq{Lem. \ref{lemma:Iwama}}
\begin{tikzpicture}[tikzfig]
	\begin{pgfonlayer}{nodelayer}
		\node [style=X] (51) at (2, 8) {};
		\node [style=oplus] (52) at (1.5, 7) {};
		\node [style=zeroin] (53) at (1.5, 6) {};
		\node [style=dot] (54) at (1, 7) {};
		\node [style=dot] (55) at (0.5, 7) {};
		\node [style=X] (56) at (1, 8) {};
		\node [style=X] (57) at (0.5, 8) {};
		\node [style=dot] (58) at (2, 7.5) {};
		\node [style=oplus] (59) at (1, 7.5) {};
		\node [style=none] (60) at (2, 5.75) {};
		\node [style=none] (61) at (1, 5.75) {};
		\node [style=none] (62) at (0.5, 5.75) {};
		\node [style=none] (63) at (1.5, 8.25) {};
		\node [style=oplus] (64) at (1.5, 6.5) {};
		\node [style=dot] (65) at (2, 6.5) {};
		\node [style=dot] (66) at (0.5, 6.5) {};
	\end{pgfonlayer}
	\begin{pgfonlayer}{edgelayer}
		\draw (58) to (59);
		\draw (59) to (61.center);
		\draw (60.center) to (58);
		\draw (63.center) to (52);
		\draw (52) to (53);
		\draw (52) to (54);
		\draw (54) to (56);
		\draw (54) to (55);
		\draw (55) to (57);
		\draw (55) to (62.center);
		\draw (59) to (54);
		\draw (51) to (58);
		\draw (65) to (64);
		\draw (64) to (66);
	\end{pgfonlayer}
\end{tikzpicture}
\eq{Rem. \ref{cor:copy}}
\begin{tikzpicture}[tikzfig]
	\begin{pgfonlayer}{nodelayer}
		\node [style=X] (52) at (2, 7.5) {};
		\node [style=oplus] (53) at (1.5, 7) {};
		\node [style=zeroin] (54) at (1.5, 6) {};
		\node [style=dot] (55) at (1, 7) {};
		\node [style=dot] (56) at (0.5, 7) {};
		\node [style=X] (57) at (1, 7.5) {};
		\node [style=X] (58) at (0.5, 7.5) {};
		\node [style=none] (59) at (2, 5.75) {};
		\node [style=none] (60) at (1, 5.75) {};
		\node [style=none] (61) at (0.5, 5.75) {};
		\node [style=none] (62) at (1.5, 7.75) {};
		\node [style=oplus] (63) at (1.5, 6.5) {};
		\node [style=dot] (64) at (2, 6.5) {};
		\node [style=dot] (65) at (0.5, 6.5) {};
	\end{pgfonlayer}
	\begin{pgfonlayer}{edgelayer}
		\draw (62.center) to (53);
		\draw (53) to (54);
		\draw (53) to (55);
		\draw (55) to (57);
		\draw (55) to (56);
		\draw (56) to (58);
		\draw (56) to (61.center);
		\draw (64) to (63);
		\draw (63) to (65);
		\draw (52) to (64);
		\draw (64) to (59.center);
		\draw (60.center) to (55);
	\end{pgfonlayer}
\end{tikzpicture}\\
&\eq{Rem. \ref{cor:copy}}
\begin{tikzpicture}[tikzfig]
	\begin{pgfonlayer}{nodelayer}
		\node [style=X] (53) at (2, 8) {};
		\node [style=oplus] (54) at (1, 7) {};
		\node [style=zeroin] (55) at (1, 6) {};
		\node [style=dot] (56) at (0.5, 7) {};
		\node [style=dot] (57) at (0, 7) {};
		\node [style=X] (58) at (0.5, 8) {};
		\node [style=X] (59) at (0, 8) {};
		\node [style=none] (60) at (2, 5.75) {};
		\node [style=none] (61) at (0.5, 5.75) {};
		\node [style=none] (62) at (0, 5.75) {};
		\node [style=none] (63) at (1, 8.25) {};
		\node [style=oplus] (64) at (1, 6.5) {};
		\node [style=dot] (65) at (2, 6.5) {};
		\node [style=dot] (66) at (0, 6.5) {};
		\node [style=zeroin] (67) at (1.5, 6) {};
		\node [style=X] (68) at (1.5, 8) {};
		\node [style=oplus] (69) at (1.5, 7.5) {};
		\node [style=dot] (70) at (2, 7.5) {};
		\node [style=dot] (71) at (0, 7.5) {};
	\end{pgfonlayer}
	\begin{pgfonlayer}{edgelayer}
		\draw (63.center) to (54);
		\draw (54) to (55);
		\draw (54) to (56);
		\draw (56) to (58);
		\draw (56) to (57);
		\draw (57) to (59);
		\draw (57) to (62.center);
		\draw (65) to (64);
		\draw (64) to (66);
		\draw (53) to (65);
		\draw (65) to (60.center);
		\draw (61.center) to (56);
		\draw (67) to (68);
		\draw (70) to (69);
		\draw (69) to (71);
	\end{pgfonlayer}
\end{tikzpicture}
\eq{\ref{TOF.2}}
\begin{tikzpicture}[tikzfig]
	\begin{pgfonlayer}{nodelayer}
		\node [style=X] (54) at (2, 8.5) {};
		\node [style=oplus] (55) at (1, 7) {};
		\node [style=zeroin] (56) at (1, 6) {};
		\node [style=dot] (57) at (0.5, 7) {};
		\node [style=dot] (58) at (0, 7) {};
		\node [style=X] (59) at (0.5, 8.5) {};
		\node [style=X] (60) at (0, 8.5) {};
		\node [style=none] (61) at (2, 5.75) {};
		\node [style=none] (62) at (0.5, 5.75) {};
		\node [style=none] (63) at (0, 5.75) {};
		\node [style=none] (64) at (1, 8.75) {};
		\node [style=oplus] (65) at (1, 6.5) {};
		\node [style=dot] (66) at (2, 6.5) {};
		\node [style=dot] (67) at (0, 6.5) {};
		\node [style=zeroin] (68) at (1.5, 6) {};
		\node [style=X] (69) at (1.5, 8.5) {};
		\node [style=oplus] (70) at (1.5, 8) {};
		\node [style=dot] (71) at (2, 8) {};
		\node [style=dot] (72) at (0, 8) {};
		\node [style=oplus] (73) at (1, 7.5) {};
		\node [style=dot] (74) at (1.5, 7.5) {};
	\end{pgfonlayer}
	\begin{pgfonlayer}{edgelayer}
		\draw (64.center) to (55);
		\draw (55) to (56);
		\draw (55) to (57);
		\draw (57) to (59);
		\draw (57) to (58);
		\draw (58) to (60);
		\draw (58) to (63.center);
		\draw (66) to (65);
		\draw (65) to (67);
		\draw (54) to (66);
		\draw (66) to (61.center);
		\draw (62.center) to (57);
		\draw (68) to (69);
		\draw (71) to (70);
		\draw (70) to (72);
		\draw (74) to (73);
	\end{pgfonlayer}
\end{tikzpicture}
\eq{Lem. \ref{lemma:Iwama}}
\begin{tikzpicture}[tikzfig]
	\begin{pgfonlayer}{nodelayer}
		\node [style=X] (55) at (2, 9) {};
		\node [style=oplus] (56) at (1, 7) {};
		\node [style=zeroin] (57) at (1, 6) {};
		\node [style=dot] (58) at (0.5, 7) {};
		\node [style=dot] (59) at (0, 7) {};
		\node [style=X] (60) at (0.5, 9) {};
		\node [style=X] (61) at (0, 9) {};
		\node [style=none] (62) at (2, 5.75) {};
		\node [style=none] (63) at (0.5, 5.75) {};
		\node [style=none] (64) at (0, 5.75) {};
		\node [style=none] (65) at (1, 9.25) {};
		\node [style=zeroin] (66) at (1.5, 6) {};
		\node [style=X] (67) at (1.5, 9) {};
		\node [style=oplus] (68) at (1.5, 7.5) {};
		\node [style=dot] (69) at (2, 7.5) {};
		\node [style=dot] (70) at (0, 7.5) {};
		\node [style=oplus] (71) at (1, 8) {};
		\node [style=dot] (72) at (1.5, 8) {};
		\node [style=oplus] (73) at (1, 8.5) {};
		\node [style=dot] (74) at (0, 8.5) {};
		\node [style=dot] (75) at (2, 8.5) {};
		\node [style=oplus] (76) at (1, 6.5) {};
		\node [style=dot] (77) at (2, 6.5) {};
		\node [style=dot] (78) at (0, 6.5) {};
	\end{pgfonlayer}
	\begin{pgfonlayer}{edgelayer}
		\draw (65.center) to (56);
		\draw (56) to (57);
		\draw (56) to (58);
		\draw (58) to (60);
		\draw (58) to (59);
		\draw (59) to (61);
		\draw (59) to (64.center);
		\draw (63.center) to (58);
		\draw (66) to (67);
		\draw (69) to (68);
		\draw (68) to (70);
		\draw (72) to (71);
		\draw (75) to (73);
		\draw (73) to (74);
		\draw (76) to (78);
		\draw (77) to (76);
		\draw (55) to (77);
		\draw (77) to (62.center);
	\end{pgfonlayer}
\end{tikzpicture}\\
&\eq{\ref{TOF.9}}
\begin{tikzpicture}[tikzfig]
	\begin{pgfonlayer}{nodelayer}
		\node [style=X] (56) at (2, 8) {};
		\node [style=oplus] (57) at (1, 6.5) {};
		\node [style=zeroin] (58) at (1, 6) {};
		\node [style=dot] (59) at (0.5, 6.5) {};
		\node [style=dot] (60) at (0, 6.5) {};
		\node [style=X] (61) at (0.5, 8) {};
		\node [style=X] (62) at (0, 8) {};
		\node [style=none] (63) at (2, 5.75) {};
		\node [style=none] (64) at (0.5, 5.75) {};
		\node [style=none] (65) at (0, 5.75) {};
		\node [style=none] (66) at (1, 8.25) {};
		\node [style=zeroin] (67) at (1.5, 6) {};
		\node [style=X] (68) at (1.5, 8) {};
		\node [style=oplus] (69) at (1.5, 7) {};
		\node [style=dot] (70) at (2, 7) {};
		\node [style=dot] (71) at (0, 7) {};
		\node [style=oplus] (72) at (1, 7.5) {};
		\node [style=dot] (73) at (1.5, 7.5) {};
	\end{pgfonlayer}
	\begin{pgfonlayer}{edgelayer}
		\draw (66.center) to (57);
		\draw (57) to (58);
		\draw (57) to (59);
		\draw (59) to (61);
		\draw (59) to (60);
		\draw (60) to (62);
		\draw (60) to (65.center);
		\draw (64.center) to (59);
		\draw (67) to (68);
		\draw (70) to (69);
		\draw (69) to (71);
		\draw (73) to (72);
		\draw (63.center) to (56);
	\end{pgfonlayer}
\end{tikzpicture}
\eq{Rem. \ref{cor:copy}}
\begin{tikzpicture}[tikzfig]
	\begin{pgfonlayer}{nodelayer}
		\node [style=dot] (57) at (0.5, 6.75) {};
		\node [style=X] (58) at (1, 8.25) {};
		\node [style=X] (59) at (3, 8.25) {};
		\node [style=none] (60) at (1, 5.75) {};
		\node [style=dot] (61) at (3, 6.75) {};
		\node [style=dot] (62) at (1, 7.25) {};
		\node [style=none] (63) at (3, 5.75) {};
		\node [style=oplus] (64) at (1.5, 7.25) {};
		\node [style=none] (65) at (1.5, 8.5) {};
		\node [style=none] (66) at (0.5, 5.75) {};
		\node [style=X] (67) at (2, 8.25) {};
		\node [style=oplus] (68) at (2, 6.75) {};
		\node [style=X] (69) at (0.5, 8.25) {};
		\node [style=dot] (70) at (0.5, 7.25) {};
		\node [style=zeroin] (71) at (2, 6) {};
		\node [style=zeroin] (72) at (1.5, 6) {};
		\node [style=oplus] (73) at (1.5, 7.75) {};
		\node [style=dot] (74) at (2, 7.75) {};
		\node [style=zeroin] (75) at (2.5, 6) {};
		\node [style=X] (76) at (2.5, 8.25) {};
	\end{pgfonlayer}
	\begin{pgfonlayer}{edgelayer}
		\draw (65.center) to (64);
		\draw (64) to (72);
		\draw (64) to (62);
		\draw (62) to (58);
		\draw (60.center) to (62);
		\draw (62) to (70);
		\draw (70) to (69);
		\draw (70) to (66.center);
		\draw (67) to (71);
		\draw (68) to (57);
		\draw (61) to (68);
		\draw (74) to (73);
		\draw (59) to (61);
		\draw (61) to (63.center);
		\draw (76) to (75);
	\end{pgfonlayer}
\end{tikzpicture}
\eq{Rem. \ref{cor:copy}}
\begin{tikzpicture}[tikzfig]
	\begin{pgfonlayer}{nodelayer}
		\node [style=dot] (58) at (0.5, 6.75) {};
		\node [style=X] (59) at (1, 8.75) {};
		\node [style=X] (60) at (3, 8.75) {};
		\node [style=none] (61) at (1, 5.75) {};
		\node [style=dot] (62) at (3, 6.75) {};
		\node [style=dot] (63) at (1, 7.75) {};
		\node [style=none] (64) at (3, 5.75) {};
		\node [style=oplus] (65) at (1.5, 7.75) {};
		\node [style=none] (66) at (1.5, 9) {};
		\node [style=none] (67) at (0.5, 5.75) {};
		\node [style=X] (68) at (2, 8.75) {};
		\node [style=oplus] (69) at (2, 6.75) {};
		\node [style=X] (70) at (0.5, 8.75) {};
		\node [style=dot] (71) at (0.5, 7.75) {};
		\node [style=zeroin] (72) at (2, 6) {};
		\node [style=zeroin] (73) at (1.5, 6) {};
		\node [style=oplus] (74) at (1.5, 8.25) {};
		\node [style=dot] (75) at (2, 8.25) {};
		\node [style=zeroin] (76) at (2.5, 6) {};
		\node [style=X] (77) at (2.5, 8.75) {};
		\node [style=dot] (78) at (0.5, 7.25) {};
		\node [style=oplus] (79) at (2.5, 7.25) {};
	\end{pgfonlayer}
	\begin{pgfonlayer}{edgelayer}
		\draw (66.center) to (65);
		\draw (65) to (73);
		\draw (65) to (63);
		\draw (63) to (59);
		\draw (61.center) to (63);
		\draw (63) to (71);
		\draw (71) to (70);
		\draw (71) to (67.center);
		\draw (68) to (72);
		\draw (69) to (58);
		\draw (62) to (69);
		\draw (75) to (74);
		\draw (60) to (62);
		\draw (62) to (64.center);
		\draw (77) to (76);
		\draw (79) to (78);
	\end{pgfonlayer}
\end{tikzpicture}\\
&\eq{\ref{TOF.2}}
\begin{tikzpicture}[tikzfig]
	\begin{pgfonlayer}{nodelayer}
		\node [style=dot] (59) at (0.5, 6.75) {};
		\node [style=X] (60) at (1, 9.25) {};
		\node [style=X] (61) at (3, 9.25) {};
		\node [style=none] (62) at (1, 5.75) {};
		\node [style=dot] (63) at (3, 6.75) {};
		\node [style=dot] (64) at (1, 8.25) {};
		\node [style=none] (65) at (3, 5.75) {};
		\node [style=oplus] (66) at (1.5, 8.25) {};
		\node [style=none] (67) at (1.5, 9.5) {};
		\node [style=none] (68) at (0.5, 5.75) {};
		\node [style=X] (69) at (2, 9.25) {};
		\node [style=oplus] (70) at (2, 6.75) {};
		\node [style=X] (71) at (0.5, 9.25) {};
		\node [style=dot] (72) at (0.5, 8.25) {};
		\node [style=zeroin] (73) at (2, 6) {};
		\node [style=zeroin] (74) at (1.5, 6) {};
		\node [style=oplus] (75) at (1.5, 8.75) {};
		\node [style=dot] (76) at (2, 8.75) {};
		\node [style=zeroin] (77) at (2.5, 6) {};
		\node [style=X] (78) at (2.5, 9.25) {};
		\node [style=oplus] (79) at (2, 7.25) {};
		\node [style=dot] (80) at (2.5, 7.25) {};
		\node [style=dot] (81) at (3, 7.25) {};
		\node [style=dot] (82) at (0.5, 7.75) {};
		\node [style=oplus] (83) at (2.5, 7.75) {};
	\end{pgfonlayer}
	\begin{pgfonlayer}{edgelayer}
		\draw (67.center) to (66);
		\draw (66) to (74);
		\draw (66) to (64);
		\draw (64) to (60);
		\draw (62.center) to (64);
		\draw (64) to (72);
		\draw (72) to (71);
		\draw (72) to (68.center);
		\draw (69) to (73);
		\draw (70) to (59);
		\draw (63) to (70);
		\draw (76) to (75);
		\draw (61) to (63);
		\draw (63) to (65.center);
		\draw (78) to (77);
		\draw (81) to (80);
		\draw (80) to (79);
		\draw (83) to (82);
	\end{pgfonlayer}
\end{tikzpicture}
\eq{\ref{TOF.2}}
\begin{tikzpicture}[tikzfig]
	\begin{pgfonlayer}{nodelayer}
		\node [style=dot] (59) at (0.5, 6.75) {};
		\node [style=X] (60) at (1, 9.25) {};
		\node [style=X] (61) at (3, 9.25) {};
		\node [style=none] (62) at (1, 5.75) {};
		\node [style=dot] (63) at (3, 6.75) {};
		\node [style=dot] (64) at (1, 8.25) {};
		\node [style=none] (65) at (3, 5.75) {};
		\node [style=oplus] (66) at (1.5, 8.25) {};
		\node [style=none] (67) at (1.5, 9.5) {};
		\node [style=none] (68) at (0.5, 5.75) {};
		\node [style=X] (69) at (2, 9.25) {};
		\node [style=oplus] (70) at (2, 6.75) {};
		\node [style=X] (71) at (0.5, 9.25) {};
		\node [style=dot] (72) at (0.5, 8.25) {};
		\node [style=zeroin] (73) at (2, 6) {};
		\node [style=zeroin] (74) at (1.5, 6) {};
		\node [style=oplus] (75) at (1.5, 8.75) {};
		\node [style=dot] (76) at (2, 8.75) {};
		\node [style=zeroin] (77) at (2.5, 6) {};
		\node [style=X] (78) at (2.5, 9.25) {};
		\node [style=oplus] (79) at (2, 7.25) {};
		\node [style=dot] (80) at (2.5, 7.25) {};
		\node [style=dot] (81) at (3, 7.25) {};
		\node [style=dot] (82) at (0.5, 7.75) {};
		\node [style=oplus] (83) at (2.5, 7.75) {};
	\end{pgfonlayer}
	\begin{pgfonlayer}{edgelayer}
		\draw (67.center) to (66);
		\draw (66) to (74);
		\draw (66) to (64);
		\draw (64) to (60);
		\draw (62.center) to (64);
		\draw (64) to (72);
		\draw (72) to (71);
		\draw (72) to (68.center);
		\draw (69) to (73);
		\draw (70) to (59);
		\draw (63) to (70);
		\draw (76) to (75);
		\draw (61) to (63);
		\draw (63) to (65.center);
		\draw (78) to (77);
		\draw (81) to (80);
		\draw (80) to (79);
		\draw (83) to (82);
	\end{pgfonlayer}
\end{tikzpicture}
\eq{\ref{ZXA.11}}
\begin{tikzpicture}[tikzfig]
	\begin{pgfonlayer}{nodelayer}
		\node [style=dot] (60) at (-0.5, 7) {};
		\node [style=dot] (61) at (0, 7) {};
		\node [style=dot] (62) at (1, 8) {};
		\node [style=oplus] (63) at (0.5, 8) {};
		\node [style=oplus] (64) at (0.5, 7) {};
		\node [style=dot] (65) at (2, 7) {};
		\node [style=dot] (66) at (1.5, 7) {};
		\node [style=oplus] (67) at (1, 7) {};
		\node [style=oplus] (68) at (1.5, 7.5) {};
		\node [style=dot] (69) at (0, 7.5) {};
		\node [style=X] (70) at (0, 8) {};
		\node [style=X] (71) at (-0.5, 8) {};
		\node [style=X] (72) at (2, 8) {};
		\node [style=X] (73) at (1.5, 8) {};
		\node [style=zeroin] (74) at (0.5, 6) {};
		\node [style=zeroin] (75) at (1, 6) {};
		\node [style=zeroin] (76) at (1.5, 6) {};
		\node [style=none] (77) at (0.5, 8.75) {};
		\node [style=none] (78) at (2, 5.75) {};
		\node [style=none] (79) at (0, 5.75) {};
		\node [style=none] (80) at (-0.5, 5.75) {};
		\node [style=X] (81) at (1, 8.5) {};
		\node [style=dot] (82) at (2, 6.5) {};
		\node [style=dot] (83) at (0, 6.5) {};
		\node [style=oplus] (84) at (1, 6.5) {};
	\end{pgfonlayer}
	\begin{pgfonlayer}{edgelayer}
		\draw (70) to (79.center);
		\draw (80.center) to (71);
		\draw (77.center) to (74);
		\draw (72) to (78.center);
		\draw (76) to (73);
		\draw (65) to (67);
		\draw (62) to (63);
		\draw (64) to (60);
		\draw (69) to (68);
		\draw (81) to (62);
		\draw (62) to (67);
		\draw (67) to (75);
		\draw (82) to (84);
		\draw (84) to (83);
	\end{pgfonlayer}
\end{tikzpicture}\\
&\eq{Lem. \ref{lemma:Iwama}}
\begin{tikzpicture}[tikzfig]
	\begin{pgfonlayer}{nodelayer}
		\node [style=dot] (61) at (-0.5, 7.5) {};
		\node [style=dot] (62) at (0, 7.5) {};
		\node [style=dot] (63) at (1, 8) {};
		\node [style=oplus] (64) at (0.5, 8) {};
		\node [style=oplus] (65) at (0.5, 7.5) {};
		\node [style=dot] (66) at (2, 7.5) {};
		\node [style=dot] (67) at (1.5, 7.5) {};
		\node [style=oplus] (68) at (1, 7.5) {};
		\node [style=oplus] (69) at (1.5, 6.5) {};
		\node [style=dot] (70) at (0, 6.5) {};
		\node [style=X] (71) at (0, 8) {};
		\node [style=X] (72) at (-0.5, 8) {};
		\node [style=X] (73) at (2, 8) {};
		\node [style=X] (74) at (1.5, 8) {};
		\node [style=zeroin] (75) at (0.5, 7) {};
		\node [style=zeroin] (76) at (1, 7) {};
		\node [style=zeroin] (77) at (1.5, 6) {};
		\node [style=none] (78) at (0.5, 8.75) {};
		\node [style=none] (79) at (2, 5.75) {};
		\node [style=none] (80) at (0, 5.75) {};
		\node [style=none] (81) at (-0.5, 5.75) {};
		\node [style=X] (82) at (1, 8.5) {};
	\end{pgfonlayer}
	\begin{pgfonlayer}{edgelayer}
		\draw (71) to (80.center);
		\draw (81.center) to (72);
		\draw (78.center) to (75);
		\draw (73) to (79.center);
		\draw (77) to (74);
		\draw (66) to (68);
		\draw (63) to (64);
		\draw (65) to (61);
		\draw (70) to (69);
		\draw (82) to (63);
		\draw (63) to (68);
		\draw (68) to (76);
	\end{pgfonlayer}
\end{tikzpicture}
=
\left\llbracket
\begin{tikzpicture}[tikzfig]
	\begin{pgfonlayer}{nodelayer}
		\node [style=none] (62) at (0.25, 6.25) {};
		\node [style=andin] (63) at (-0.35, 7) {};
		\node [style=none] (64) at (-0.25, 6.25) {};
		\node [style=andin] (65) at (0.35, 7) {};
		\node [style=none] (66) at (-0.25, 6.25) {};
		\node [style=X] (67) at (-0.25, 6.25) {};
		\node [style=Z] (68) at (0, 7.75) {};
		\node [style=none] (69) at (0, 8.25) {};
		\node [style=none] (70) at (-0.25, 5.75) {};
		\node [style=none] (71) at (0.5, 5.75) {};
		\node [style=none] (72) at (0.25, 5.75) {};
	\end{pgfonlayer}
	\begin{pgfonlayer}{edgelayer}
		\draw [in=-72, out=90] (62.center) to (63);
		\draw [in=120, out=-108] (63) to (64.center);
		\draw [in=45, out=-108] (65) to (66.center);
		\draw (62.center) to (72.center);
		\draw (70.center) to (64.center);
		\draw [in=-117, out=90] (63) to (68);
		\draw (68) to (69.center);
		\draw [in=90, out=-63] (68) to (65);
		\draw [in=-75, out=90, looseness=1.25] (71.center) to (65);
	\end{pgfonlayer}
\end{tikzpicture}
\right\rrbracket_{\ZXA}
\end{align*}


\end{enumerate}

\end{proof}


\begin{proposition}
\label{prop:ZXATOF}
Consider the interpretation $\llbracket\_\rrbracket_{\hat \TOF}:\hat \TOF\to \ZXA$ taking:

\begin{center}
\begin{tabular}{c}
$
\begin{tikzpicture}[tikzfig]
	\begin{pgfonlayer}{nodelayer}
		\node [style=dot] (63) at (0, 6.5) {};
		\node [style=oplus] (64) at (0.5, 6.5) {};
		\node [style=dot] (65) at (-0.5, 6.5) {};
		\node [style=none] (66) at (0.5, 7.25) {};
		\node [style=none] (67) at (0, 7.25) {};
		\node [style=none] (68) at (-0.5, 7.25) {};
		\node [style=none] (69) at (-0.5, 5.75) {};
		\node [style=none] (70) at (0, 5.75) {};
		\node [style=none] (71) at (0.5, 5.75) {};
	\end{pgfonlayer}
	\begin{pgfonlayer}{edgelayer}
		\draw [style=simple] (66.center) to (64);
		\draw [style=simple] (64) to (63);
		\draw [style=simple] (63) to (65);
		\draw [style=simple] (65) to (68.center);
		\draw [style=simple] (67.center) to (63);
		\draw [style=simple] (63) to (70.center);
		\draw [style=simple] (69.center) to (65);
		\draw [style=simple] (64) to (71.center);
	\end{pgfonlayer}
\end{tikzpicture}
\mapsto
\begin{tikzpicture}[tikzfig]
	\begin{pgfonlayer}{nodelayer}
		\node [style=none] (64) at (0, 5.75) {};
		\node [style=none] (65) at (1, 5.75) {};
		\node [style=none] (66) at (1.5, 5.75) {};
		\node [style=Z] (67) at (1.5, 7.75) {};
		\node [style=X] (68) at (1, 6.25) {};
		\node [style=X] (69) at (0, 6.25) {};
		\node [style=andin] (70) at (0.5, 7.25) {};
		\node [style=none] (71) at (0, 8.5) {};
		\node [style=none] (72) at (1.5, 8.5) {};
		\node [style=none] (73) at (1, 8.5) {};
	\end{pgfonlayer}
	\begin{pgfonlayer}{edgelayer}
		\draw [style=simple, in=90, out=180, looseness=0.75] (67) to (70);
		\draw [style=simple, in=45, out=-120] (70) to (69);
		\draw [style=simple] (69) to (64.center);
		\draw [style=simple] (65.center) to (68);
		\draw [style=simple] (67) to (66.center);
		\draw [style=simple] (73.center) to (68);
		\draw [style=simple] (72.center) to (67);
		\draw [style=simple, in=-60, out=135] (68) to (70);
		\draw [style=simple] (71.center) to (69);
	\end{pgfonlayer}
\end{tikzpicture}
\hspace*{.5cm}
\begin{tikzpicture}[tikzfig]
	\begin{pgfonlayer}{nodelayer}
		\node [style=onein] (67) at (0, 5.75) {};
		\node [style=none] (68) at (0, 6.75) {};
	\end{pgfonlayer}
	\begin{pgfonlayer}{edgelayer}
		\draw [style=simple] (68.center) to (67);
	\end{pgfonlayer}
\end{tikzpicture}
\mapsto
\begin{tikzpicture}[tikzfig]
	\begin{pgfonlayer}{nodelayer}
		\node [style=Z] (68) at (0, 5.75) {$\pi$};
		\node [style=none] (69) at (0, 6.75) {};
	\end{pgfonlayer}
	\begin{pgfonlayer}{edgelayer}
		\draw [style=simple] (69.center) to (68);
	\end{pgfonlayer}
\end{tikzpicture}
\hspace*{.5cm}
\begin{tikzpicture}[tikzfig]
	\begin{pgfonlayer}{nodelayer}
		\node [style=onein] (70) at (0, 6.75) {};
		\node [style=none] (71) at (0, 5.75) {};
	\end{pgfonlayer}
	\begin{pgfonlayer}{edgelayer}
		\draw [style=simple] (71.center) to (70);
	\end{pgfonlayer}
\end{tikzpicture}
\mapsto
\begin{tikzpicture}[tikzfig]
	\begin{pgfonlayer}{nodelayer}
		\node [style=Z] (0) at (0, 1.5) {$\pi$};
		\node [style=none] (1) at (0, 0.5) {};
	\end{pgfonlayer}
	\begin{pgfonlayer}{edgelayer}
		\draw [style=simple] (1.center) to (0);
	\end{pgfonlayer}
\end{tikzpicture}
$
\\\\
$
\begin{tikzpicture}[tikzfig]
	\begin{pgfonlayer}{nodelayer}
		\node [style=X] (1) at (0, 0) {};
		\node [style=none] (2) at (0, 1) {};
	\end{pgfonlayer}
	\begin{pgfonlayer}{edgelayer}
		\draw [style=simple] (2.center) to (1);
	\end{pgfonlayer}
\end{tikzpicture}
\mapsto
\begin{tikzpicture}[tikzfig]
	\begin{pgfonlayer}{nodelayer}
		\node [style=X] (2) at (0, 0) {};
		\node [style=none] (3) at (0, 1) {};
	\end{pgfonlayer}
	\begin{pgfonlayer}{edgelayer}
		\draw [style=simple] (3.center) to (2);
	\end{pgfonlayer}
\end{tikzpicture}
\hspace*{.5cm}
\begin{tikzpicture}[tikzfig]
	\begin{pgfonlayer}{nodelayer}
		\node [style=X] (4) at (0, 1) {};
		\node [style=none] (5) at (0, 0) {};
	\end{pgfonlayer}
	\begin{pgfonlayer}{edgelayer}
		\draw [style=simple] (5.center) to (4);
	\end{pgfonlayer}
\end{tikzpicture}
\mapsto
\begin{tikzpicture}[tikzfig]
	\begin{pgfonlayer}{nodelayer}
		\node [style=X] (5) at (0, 1) {};
		\node [style=none] (6) at (0, 0) {};
	\end{pgfonlayer}
	\begin{pgfonlayer}{edgelayer}
		\draw [style=simple] (6.center) to (5);
	\end{pgfonlayer}
\end{tikzpicture}
$
\end{tabular}
\end{center}

This interepretation is a strict symmetric \dag-monoidal functor.
\end{proposition}

\begin{proof}
First, observe:
\begin{align*}
\left\llbracket
\begin{tikzpicture}[tikzfig]
	\begin{pgfonlayer}{nodelayer}
		\node [style=dot] (6) at (0, 7) {};
		\node [style=oplus] (7) at (0.5, 7) {};
		\node [style=none] (8) at (0.5, 7.5) {};
		\node [style=none] (9) at (0, 7.5) {};
		\node [style=none] (10) at (0, 6.5) {};
		\node [style=none] (11) at (0.5, 6.5) {};
	\end{pgfonlayer}
	\begin{pgfonlayer}{edgelayer}
		\draw (7) to (8.center);
		\draw (7) to (11.center);
		\draw (7) to (6);
		\draw (6) to (9.center);
		\draw (6) to (10.center);
	\end{pgfonlayer}
\end{tikzpicture}
\right\rrbracket_{\hat{\TOF}}
&=
\begin{tikzpicture}[tikzfig]
	\begin{pgfonlayer}{nodelayer}
		\node [style=none] (7) at (1, -0.5) {};
		\node [style=none] (8) at (1.5, -0.5) {};
		\node [style=Z] (9) at (1.5, 2) {};
		\node [style=X] (10) at (1, 0.5) {};
		\node [style=X] (11) at (0, 0.5) {};
		\node [style=none] (12) at (0.5, 1.5) {};
		\node [style=none] (13) at (1.5, 2.75) {};
		\node [style=none] (14) at (1, 2.75) {};
		\node [style=Z] (15) at (0, 1.25) {$\pi$};
		\node [style=Z] (16) at (0, -0.25) {$\pi$};
		\node [style=andin] (17) at (0.5, 1.5) {};
	\end{pgfonlayer}
	\begin{pgfonlayer}{edgelayer}
		\draw [style=simple, in=90, out=180, looseness=0.75] (9) to (12.center);
		\draw [style=simple, in=45, out=-120] (12.center) to (11);
		\draw [style=simple] (7.center) to (10);
		\draw [style=simple] (9) to (8.center);
		\draw [style=simple] (14.center) to (10);
		\draw [style=simple] (13.center) to (9);
		\draw [style=simple, in=-60, out=135] (10) to (12.center);
		\draw (15) to (11);
		\draw (11) to (16);
	\end{pgfonlayer}
\end{tikzpicture}
\eq{\ref{ZXA.14}}
\begin{tikzpicture}[tikzfig]
	\begin{pgfonlayer}{nodelayer}
		\node [style=none] (8) at (1.5, 8.25) {};
		\node [style=Z] (9) at (0.25, 4.5) {$\pi$};
		\node [style=none] (10) at (1, 4.25) {};
		\node [style=Z] (11) at (1.5, 7.5) {};
		\node [style=X] (12) at (1, 6) {};
		\node [style=none] (13) at (0.5, 7) {};
		\node [style=none] (14) at (1, 8.25) {};
		\node [style=Z] (15) at (0.25, 5.25) {$\pi$};
		\node [style=none] (16) at (1.5, 4.25) {};
		\node [style=Z] (17) at (0, 6) {$\pi$};
		\node [style=andin] (18) at (0.5, 7) {};
	\end{pgfonlayer}
	\begin{pgfonlayer}{edgelayer}
		\draw [style=simple, in=90, out=180, looseness=0.75] (11) to (13.center);
		\draw [style=simple] (10.center) to (12);
		\draw [style=simple] (11) to (16.center);
		\draw [style=simple] (14.center) to (12);
		\draw [style=simple] (8.center) to (11);
		\draw [style=simple, in=-60, out=135] (12) to (13.center);
		\draw [in=90, out=-135] (13.center) to (17);
		\draw (15) to (9);
	\end{pgfonlayer}
\end{tikzpicture}
\eq{\ref{ZXA.1}}
\begin{tikzpicture}[tikzfig]
	\begin{pgfonlayer}{nodelayer}
		\node [style=none] (9) at (1.5, 8.25) {};
		\node [style=Z] (10) at (0.5, 5.25) {};
		\node [style=none] (11) at (1, 4.75) {};
		\node [style=Z] (12) at (1.5, 7.5) {};
		\node [style=X] (13) at (1, 6) {};
		\node [style=none] (14) at (0.5, 7) {};
		\node [style=none] (15) at (1, 8.25) {};
		\node [style=none] (16) at (1.5, 4.75) {};
		\node [style=Z] (17) at (0, 6) {$\pi$};
		\node [style=andin] (18) at (0.5, 7) {};
	\end{pgfonlayer}
	\begin{pgfonlayer}{edgelayer}
		\draw [style=simple, in=90, out=180, looseness=0.75] (12) to (14.center);
		\draw [style=simple] (11.center) to (13);
		\draw [style=simple] (12) to (16.center);
		\draw [style=simple] (15.center) to (13);
		\draw [style=simple] (9.center) to (12);
		\draw [style=simple, in=-60, out=135] (13) to (14.center);
		\draw [in=90, out=-135] (14.center) to (17);
	\end{pgfonlayer}
\end{tikzpicture}\\
&
\eq{Lem. \ref{lem:blackdot}, \ref{ZXA.7}}
\begin{tikzpicture}[tikzfig]
	\begin{pgfonlayer}{nodelayer}
		\node [style=none] (10) at (1.5, 8.25) {};
		\node [style=none] (11) at (1, 5.25) {};
		\node [style=Z] (12) at (1.5, 7.5) {};
		\node [style=X] (13) at (1, 6) {};
		\node [style=none] (14) at (0.5, 7) {};
		\node [style=none] (15) at (1, 8.25) {};
		\node [style=none] (16) at (1.5, 5.25) {};
		\node [style=Z] (17) at (0, 6) {$\pi$};
		\node [style=andin] (18) at (0.5, 7) {};
	\end{pgfonlayer}
	\begin{pgfonlayer}{edgelayer}
		\draw [style=simple, in=90, out=180, looseness=0.75] (12) to (14.center);
		\draw [style=simple] (11.center) to (13);
		\draw [style=simple] (12) to (16.center);
		\draw [style=simple] (15.center) to (13);
		\draw [style=simple] (10.center) to (12);
		\draw [style=simple, in=-60, out=135] (13) to (14.center);
		\draw [in=90, out=-135] (14.center) to (17);
	\end{pgfonlayer}
\end{tikzpicture}
\eq{\ref{ZXA.10}}
\begin{tikzpicture}[tikzfig]
	\begin{pgfonlayer}{nodelayer}
		\node [style=none] (11) at (1.5, 8.25) {};
		\node [style=none] (12) at (1, 5.25) {};
		\node [style=Z] (13) at (1.5, 7.5) {};
		\node [style=X] (14) at (1, 6) {};
		\node [style=none] (15) at (1, 8.25) {};
		\node [style=none] (16) at (1.5, 5.25) {};
	\end{pgfonlayer}
	\begin{pgfonlayer}{edgelayer}
		\draw [style=simple] (12.center) to (14);
		\draw [style=simple] (13) to (16.center);
		\draw [style=simple] (15.center) to (14);
		\draw [style=simple] (11.center) to (13);
		\draw [in=120, out=-135, looseness=1.25] (13) to (14);
	\end{pgfonlayer}
\end{tikzpicture}
\eq{\ref{ZXA.4}}
\begin{tikzpicture}[tikzfig]
	\begin{pgfonlayer}{nodelayer}
		\node [style=none] (12) at (1.5, 8.25) {};
		\node [style=none] (13) at (1, 6.75) {};
		\node [style=Z] (14) at (1.5, 7.5) {};
		\node [style=X] (15) at (1, 7.5) {};
		\node [style=none] (16) at (1, 8.25) {};
		\node [style=none] (17) at (1.5, 6.75) {};
	\end{pgfonlayer}
	\begin{pgfonlayer}{edgelayer}
		\draw [style=simple] (13.center) to (15);
		\draw [style=simple] (14) to (17.center);
		\draw [style=simple, in=90, out=-90] (16.center) to (15);
		\draw [style=simple] (12.center) to (14);
		\draw (14) to (15);
	\end{pgfonlayer}
\end{tikzpicture}
\end{align*}

Thus:
\begin{align*}
\left\llbracket
\begin{tikzpicture}[tikzfig]
	\begin{pgfonlayer}{nodelayer}
		\node [style=oplus] (13) at (0.5, 7) {};
		\node [style=none] (14) at (0.5, 7.5) {};
		\node [style=none] (15) at (0.5, 6.5) {};
	\end{pgfonlayer}
	\begin{pgfonlayer}{edgelayer}
		\draw (13) to (14.center);
		\draw (13) to (15.center);
	\end{pgfonlayer}
\end{tikzpicture}
\right\rrbracket_{\hat{\TOF}}
&=
\begin{tikzpicture}[tikzfig]
	\begin{pgfonlayer}{nodelayer}
		\node [style=none] (14) at (1.5, 1.5) {};
		\node [style=Z] (15) at (1.5, 4) {};
		\node [style=X] (16) at (1, 2.5) {};
		\node [style=X] (17) at (0, 2.5) {};
		\node [style=none] (18) at (0.5, 3.5) {};
		\node [style=none] (19) at (1.5, 4.75) {};
		\node [style=Z] (20) at (0, 3.25) {$\pi$};
		\node [style=Z] (21) at (0, 1.75) {$\pi$};
		\node [style=Z] (22) at (1, 1.75) {$\pi$};
		\node [style=Z] (23) at (1, 3.25) {$\pi$};
		\node [style=andin] (24) at (0.5, 3.5) {};
	\end{pgfonlayer}
	\begin{pgfonlayer}{edgelayer}
		\draw [style=simple, in=90, out=180, looseness=0.75] (15) to (18.center);
		\draw [style=simple, in=45, out=-120] (18.center) to (17);
		\draw [style=simple] (15) to (14.center);
		\draw [style=simple] (19.center) to (15);
		\draw [style=simple, in=-60, out=135] (16) to (18.center);
		\draw (20) to (17);
		\draw (17) to (21);
		\draw (23) to (16);
		\draw (16) to (22);
	\end{pgfonlayer}
\end{tikzpicture}
=
\begin{tikzpicture}[tikzfig]
	\begin{pgfonlayer}{nodelayer}
		\node [style=none] (15) at (1.5, 1.5) {};
		\node [style=Z] (16) at (1.5, 2.25) {};
		\node [style=X] (17) at (1, 2.25) {};
		\node [style=none] (18) at (1.5, 3) {};
		\node [style=Z] (19) at (1, 1.75) {$\pi$};
		\node [style=Z] (20) at (1, 2.75) {$\pi$};
	\end{pgfonlayer}
	\begin{pgfonlayer}{edgelayer}
		\draw [style=simple] (16) to (15.center);
		\draw [style=simple] (18.center) to (16);
		\draw (20) to (17);
		\draw (17) to (19);
		\draw (16) to (17);
	\end{pgfonlayer}
\end{tikzpicture}
\eq{\ref{ZXA.14}}
\begin{tikzpicture}[tikzfig]
	\begin{pgfonlayer}{nodelayer}
		\node [style=none] (16) at (1.5, 1.5) {};
		\node [style=Z] (17) at (1.5, 2.5) {};
		\node [style=none] (18) at (1.5, 4) {};
		\node [style=Z] (19) at (1, 3) {$\pi$};
		\node [style=Z] (20) at (1, 3.75) {$\pi$};
		\node [style=Z] (21) at (1, 2) {$\pi$};
	\end{pgfonlayer}
	\begin{pgfonlayer}{edgelayer}
		\draw [style=simple] (17) to (16.center);
		\draw [style=simple] (18.center) to (17);
		\draw (20) to (19);
		\draw [in=90, out=180, looseness=1.25] (17) to (21);
	\end{pgfonlayer}
\end{tikzpicture}
\eq{\ref{ZXA.1}}
\begin{tikzpicture}[tikzfig]
	\begin{pgfonlayer}{nodelayer}
		\node [style=none] (17) at (1.5, 1.5) {};
		\node [style=Z] (18) at (1.5, 2) {$\pi$};
		\node [style=none] (19) at (1.5, 2.5) {};
		\node [style=Z] (20) at (1, 2) {};
	\end{pgfonlayer}
	\begin{pgfonlayer}{edgelayer}
		\draw [style=simple] (18) to (17.center);
		\draw [style=simple] (19.center) to (18);
	\end{pgfonlayer}
\end{tikzpicture}
\eq{Lem. \ref{lem:blackdot}, \ref{ZXA.7}}
\begin{tikzpicture}[tikzfig]
	\begin{pgfonlayer}{nodelayer}
		\node [style=none] (18) at (1.5, 2) {};
		\node [style=Z] (19) at (1.5, 2.5) {$\pi$};
		\node [style=none] (20) at (1.5, 3) {};
	\end{pgfonlayer}
	\begin{pgfonlayer}{edgelayer}
		\draw [style=simple] (19) to (18.center);
		\draw [style=simple] (20.center) to (19);
	\end{pgfonlayer}
\end{tikzpicture}
\end{align*}

Thus:
\begin{align*}
\left\llbracket
\begin{tikzpicture}[tikzfig]
	\begin{pgfonlayer}{nodelayer}
		\node [style=none] (19) at (1.5, 5.25) {};
		\node [style=none] (20) at (1.5, 8.25) {};
		\node [style=Z] (21) at (1.5, 7.5) {};
		\node [style=X] (22) at (1, 6) {};
		\node [style=Z] (23) at (0, 6) {$\pi$};
		\node [style=andin] (24) at (0.5, 7) {};
		\node [style=none] (25) at (0.5, 7) {};
		\node [style=none] (26) at (1, 5.25) {};
		\node [style=none] (27) at (1, 8.25) {};
	\end{pgfonlayer}
	\begin{pgfonlayer}{edgelayer}
		\draw [style=simple, in=90, out=180, looseness=0.75] (21) to (25.center);
		\draw [style=simple] (26.center) to (22);
		\draw [style=simple] (21) to (19.center);
		\draw [style=simple] (27.center) to (22);
		\draw [style=simple] (20.center) to (21);
		\draw [style=simple, in=-60, out=135] (22) to (25.center);
		\draw [in=90, out=-135] (25.center) to (23);
	\end{pgfonlayer}
\end{tikzpicture}
\right\rrbracket_{\hat{\TOF}}
=
\begin{tikzpicture}[tikzfig]
	\begin{pgfonlayer}{nodelayer}
		\node [style=Z] (20) at (1.5, 4.25) {};
		\node [style=X] (21) at (1, 2.75) {};
		\node [style=X] (22) at (0, 2.75) {};
		\node [style=none] (23) at (0.5, 3.75) {};
		\node [style=none] (24) at (1.5, 5) {};
		\node [style=Z] (25) at (0, 3.5) {$\pi$};
		\node [style=Z] (26) at (0, 2) {$\pi$};
		\node [style=Z] (27) at (1, 2) {$\pi$};
		\node [style=Z] (28) at (1, 3.5) {$\pi$};
		\node [style=andin] (29) at (0.5, 3.75) {};
		\node [style=Z] (30) at (1.5, 2) {$\pi$};
	\end{pgfonlayer}
	\begin{pgfonlayer}{edgelayer}
		\draw [style=simple, in=90, out=180, looseness=0.75] (20) to (23.center);
		\draw [style=simple, in=45, out=-120] (23.center) to (22);
		\draw [style=simple] (24.center) to (20);
		\draw [style=simple, in=-60, out=135] (21) to (23.center);
		\draw (25) to (22);
		\draw (22) to (26);
		\draw (28) to (21);
		\draw (21) to (27);
		\draw (20) to (30);
	\end{pgfonlayer}
\end{tikzpicture}
=
\begin{tikzpicture}[tikzfig]
	\begin{pgfonlayer}{nodelayer}
		\node [style=Z] (21) at (1.5, 2.75) {$\pi$};
		\node [style=none] (22) at (1.5, 3.5) {};
		\node [style=Z] (23) at (1.5, 2) {$\pi$};
	\end{pgfonlayer}
	\begin{pgfonlayer}{edgelayer}
		\draw [style=simple] (22.center) to (21);
		\draw (21) to (23);
	\end{pgfonlayer}
\end{tikzpicture}
\eq{\ref{ZXA.1}}
\begin{tikzpicture}[tikzfig]
	\begin{pgfonlayer}{nodelayer}
		\node [style=Z] (22) at (1.5, 2) {};
		\node [style=none] (23) at (1.5, 2.75) {};
	\end{pgfonlayer}
	\begin{pgfonlayer}{edgelayer}
		\draw [style=simple] (23.center) to (22);
	\end{pgfonlayer}
\end{tikzpicture}
\end{align*}


We prove that all of the axioms of $\hat \TOF$ hold in $\ZXA$ :
\begin{enumerate}
\item[\ref{TOF.1}:]
\begin{align*}
\left\llbracket
\begin{tikzpicture}[tikzfig]
	\begin{pgfonlayer}{nodelayer}
		\node [style=nothing] (23) at (0, 2) {};
		\node [style=nothing] (24) at (-0.5, 2) {};
		\node [style=oplus] (25) at (0, 2.5) {};
		\node [style=dot] (26) at (-0.5, 2.5) {};
		\node [style=dot] (27) at (-1, 2.5) {};
		\node [style=onein] (28) at (-1, 2) {};
		\node [style=nothing] (29) at (-1, 3) {};
		\node [style=nothing] (30) at (-0.5, 3) {};
		\node [style=nothing] (31) at (0, 3) {};
	\end{pgfonlayer}
	\begin{pgfonlayer}{edgelayer}
		\draw (28) to (27);
		\draw (27) to (29);
		\draw (30) to (26);
		\draw (24) to (26);
		\draw (23) to (25);
		\draw (25) to (31);
		\draw (25) to (26);
		\draw (26) to (27);
	\end{pgfonlayer}
\end{tikzpicture}
\right\rrbracket_{\hat{\TOF}}
&=
\begin{tikzpicture}[tikzfig]
	\begin{pgfonlayer}{nodelayer}
		\node [style=andin] (24) at (-0.5, 4.25) {};
		\node [style=X] (25) at (-1, 3.5) {};
		\node [style=X] (26) at (0, 3.5) {};
		\node [style=Z] (27) at (0.5, 5) {};
		\node [style=none] (28) at (0.5, 2.5) {};
		\node [style=none] (29) at (0.5, 5.75) {};
		\node [style=none] (30) at (0, 5.75) {};
		\node [style=none] (31) at (-1, 5.75) {};
		\node [style=Z] (32) at (-1, 2.75) {$\pi$};
		\node [style=none] (33) at (0, 2.5) {};
	\end{pgfonlayer}
	\begin{pgfonlayer}{edgelayer}
		\draw (29.center) to (27);
		\draw (27) to (28.center);
		\draw (33.center) to (26);
		\draw [in=90, out=180, looseness=0.75] (27) to (24);
		\draw (24) to (25);
		\draw (24) to (26);
		\draw (25) to (32);
		\draw (25) to (31.center);
		\draw (30.center) to (26);
	\end{pgfonlayer}
\end{tikzpicture}
\eq{\ref{ZXA.14}}
\begin{tikzpicture}[tikzfig]
	\begin{pgfonlayer}{nodelayer}
		\node [style=andin] (25) at (-0.5, 4.25) {};
		\node [style=X] (26) at (0, 3.5) {};
		\node [style=Z] (27) at (0.5, 5) {};
		\node [style=none] (28) at (0.5, 2.5) {};
		\node [style=none] (29) at (0.5, 5.75) {};
		\node [style=none] (30) at (0, 5.75) {};
		\node [style=none] (31) at (-1, 5.75) {};
		\node [style=Z] (32) at (-1, 3.5) {$\pi$};
		\node [style=none] (33) at (0, 2.5) {};
		\node [style=Z] (34) at (-1, 5) {$\pi$};
	\end{pgfonlayer}
	\begin{pgfonlayer}{edgelayer}
		\draw (29.center) to (27);
		\draw (27) to (28.center);
		\draw (33.center) to (26);
		\draw [in=90, out=180, looseness=0.75] (27) to (25);
		\draw (25) to (26);
		\draw (30.center) to (26);
		\draw (31.center) to (34);
		\draw [in=-124, out=90] (32) to (25);
	\end{pgfonlayer}
\end{tikzpicture}
\eq{\ref{ZXA.10}}
\begin{tikzpicture}[tikzfig]
	\begin{pgfonlayer}{nodelayer}
		\node [style=X] (26) at (0, 4) {};
		\node [style=Z] (27) at (0.5, 5) {};
		\node [style=none] (28) at (0.5, 3.5) {};
		\node [style=none] (29) at (0.5, 5.5) {};
		\node [style=none] (30) at (0, 5.5) {};
		\node [style=none] (31) at (-0.5, 5.5) {};
		\node [style=none] (32) at (0, 3.5) {};
		\node [style=Z] (33) at (-0.5, 5) {$\pi$};
	\end{pgfonlayer}
	\begin{pgfonlayer}{edgelayer}
		\draw (29.center) to (27);
		\draw (27) to (28.center);
		\draw (32.center) to (26);
		\draw (30.center) to (26);
		\draw (31.center) to (33);
		\draw [in=-108, out=120, looseness=1.25] (26) to (27);
	\end{pgfonlayer}
\end{tikzpicture}\\
&\eq{\ref{ZXA.3}}
\begin{tikzpicture}[tikzfig]
	\begin{pgfonlayer}{nodelayer}
		\node [style=X] (27) at (0, 5) {};
		\node [style=Z] (28) at (0.5, 5) {};
		\node [style=none] (29) at (0.5, 4.5) {};
		\node [style=none] (30) at (0.5, 5.5) {};
		\node [style=none] (31) at (0, 5.5) {};
		\node [style=none] (32) at (-0.5, 5.5) {};
		\node [style=none] (33) at (0, 4.5) {};
		\node [style=Z] (34) at (-0.5, 5) {$\pi$};
	\end{pgfonlayer}
	\begin{pgfonlayer}{edgelayer}
		\draw (30.center) to (28);
		\draw (28) to (29.center);
		\draw (33.center) to (27);
		\draw (31.center) to (27);
		\draw (32.center) to (34);
		\draw (27) to (28);
	\end{pgfonlayer}
\end{tikzpicture}
=
\left\llbracket
\begin{tikzpicture}[tikzfig]
	\begin{pgfonlayer}{nodelayer}
		\node [style=nothing] (28) at (0, 2) {};
		\node [style=nothing] (29) at (-0.5, 2) {};
		\node [style=oplus] (30) at (0, 2.5) {};
		\node [style=dot] (31) at (-0.5, 2.5) {};
		\node [style=onein] (32) at (-1, 2.5) {};
		\node [style=nothing] (33) at (-1, 3) {};
		\node [style=nothing] (34) at (-0.5, 3) {};
		\node [style=nothing] (35) at (0, 3) {};
	\end{pgfonlayer}
	\begin{pgfonlayer}{edgelayer}
		\draw (29) to (31);
		\draw (28) to (30);
		\draw (30) to (31);
		\draw (34) to (31);
		\draw (32) to (33);
		\draw (30) to (35);
	\end{pgfonlayer}
\end{tikzpicture}
\right\rrbracket_{\hat{\TOF}}
\end{align*}

\item[\ref{TOF.2}:]
\begin{align*}
\left\llbracket
\begin{tikzpicture}[tikzfig]
	\begin{pgfonlayer}{nodelayer}
		\node [style=nothing] (29) at (-1.25, 2) {};
		\node [style=nothing] (30) at (-0.75, 2) {};
		\node [style=nothing] (31) at (-1.75, 4) {};
		\node [style=nothing] (32) at (-1.25, 4) {};
		\node [style=nothing] (33) at (-0.75, 4) {};
		\node [style=dot] (34) at (-1.75, 3) {};
		\node [style=dot] (35) at (-1.25, 3) {};
		\node [style=oplus] (36) at (-0.75, 3) {};
		\node [style=zeroin] (37) at (-1.75, 2) {};
	\end{pgfonlayer}
	\begin{pgfonlayer}{edgelayer}
		\draw (34) to (31);
		\draw (32) to (35);
		\draw (35) to (29);
		\draw (30) to (36);
		\draw (36) to (33);
		\draw (36) to (35);
		\draw (35) to (34);
		\draw (37) to (34);
	\end{pgfonlayer}
\end{tikzpicture}
\right\rrbracket_{\hat{\TOF}}
&=
\begin{tikzpicture}[tikzfig]
	\begin{pgfonlayer}{nodelayer}
		\node [style=andin] (30) at (-0.5, 4.25) {};
		\node [style=X] (31) at (-1, 3.5) {};
		\node [style=X] (32) at (0, 3.5) {};
		\node [style=Z] (33) at (0.5, 5) {};
		\node [style=none] (34) at (0.5, 2.5) {};
		\node [style=none] (35) at (0.5, 5.75) {};
		\node [style=none] (36) at (0, 5.75) {};
		\node [style=none] (37) at (-1, 5.75) {};
		\node [style=none] (38) at (0, 2.5) {};
		\node [style=Z] (39) at (-1, 2.75) {};
	\end{pgfonlayer}
	\begin{pgfonlayer}{edgelayer}
		\draw (35.center) to (33);
		\draw (33) to (34.center);
		\draw (38.center) to (32);
		\draw [in=90, out=180, looseness=0.75] (33) to (30);
		\draw (30) to (31);
		\draw (30) to (32);
		\draw (31) to (37.center);
		\draw (36.center) to (32);
		\draw (31) to (39);
	\end{pgfonlayer}
\end{tikzpicture}
\eq{\ref{ZXA.6}}
\begin{tikzpicture}[tikzfig]
	\begin{pgfonlayer}{nodelayer}
		\node [style=andin] (31) at (-0.5, 4.25) {};
		\node [style=X] (32) at (0, 3.5) {};
		\node [style=Z] (33) at (0.5, 5) {};
		\node [style=none] (34) at (0.5, 3) {};
		\node [style=none] (35) at (0.5, 5.5) {};
		\node [style=none] (36) at (0, 5.5) {};
		\node [style=none] (37) at (-1, 5.5) {};
		\node [style=none] (38) at (0, 3) {};
		\node [style=Z] (39) at (-1, 3.5) {};
		\node [style=Z] (40) at (-1, 5) {};
	\end{pgfonlayer}
	\begin{pgfonlayer}{edgelayer}
		\draw (35.center) to (33);
		\draw (33) to (34.center);
		\draw (38.center) to (32);
		\draw [in=90, out=180, looseness=0.75] (33) to (31);
		\draw (31) to (32);
		\draw (36.center) to (32);
		\draw [in=90, out=-124] (31) to (39);
		\draw (37.center) to (40);
	\end{pgfonlayer}
\end{tikzpicture}
\eq{Lem. \ref{lem:oldaxiom}}
\begin{tikzpicture}[tikzfig]
	\begin{pgfonlayer}{nodelayer}
		\node [style=X] (32) at (0, 4.25) {};
		\node [style=Z] (33) at (0.5, 6.25) {};
		\node [style=none] (34) at (0.5, 3.75) {};
		\node [style=none] (35) at (0.5, 6.75) {};
		\node [style=none] (36) at (0, 6.75) {};
		\node [style=none] (37) at (-0.5, 6.75) {};
		\node [style=none] (38) at (0, 3.75) {};
		\node [style=Z] (39) at (-0.5, 6.25) {};
		\node [style=Z] (40) at (-0.5, 5.5) {};
		\node [style=X] (41) at (-0.5, 5) {};
	\end{pgfonlayer}
	\begin{pgfonlayer}{edgelayer}
		\draw (35.center) to (33);
		\draw (33) to (34.center);
		\draw (38.center) to (32);
		\draw (36.center) to (32);
		\draw (37.center) to (39);
		\draw [in=90, out=-143, looseness=0.75] (33) to (40);
		\draw [in=-90, out=124] (32) to (41);
	\end{pgfonlayer}
\end{tikzpicture}\\
&\eq{\ref{ZXA.1}}
\begin{tikzpicture}[tikzfig]
	\begin{pgfonlayer}{nodelayer}
		\node [style=X] (33) at (0, 4.25) {};
		\node [style=none] (34) at (0.5, 3.75) {};
		\node [style=none] (35) at (0.5, 6) {};
		\node [style=none] (36) at (0, 6) {};
		\node [style=none] (37) at (-0.5, 6) {};
		\node [style=none] (38) at (0, 3.75) {};
		\node [style=Z] (39) at (-0.5, 5.5) {};
		\node [style=X] (40) at (-0.5, 5) {};
	\end{pgfonlayer}
	\begin{pgfonlayer}{edgelayer}
		\draw (38.center) to (33);
		\draw (36.center) to (33);
		\draw (37.center) to (39);
		\draw [in=-90, out=124] (33) to (40);
		\draw (35.center) to (34.center);
	\end{pgfonlayer}
\end{tikzpicture}
\eq{\ref{ZXA.3}}
\begin{tikzpicture}[tikzfig]
	\begin{pgfonlayer}{nodelayer}
		\node [style=none] (34) at (0.5, 3.75) {};
		\node [style=none] (35) at (0.5, 4.75) {};
		\node [style=none] (36) at (0, 4.75) {};
		\node [style=none] (37) at (-0.5, 4.75) {};
		\node [style=none] (38) at (0, 3.75) {};
		\node [style=Z] (39) at (-0.5, 4.25) {};
	\end{pgfonlayer}
	\begin{pgfonlayer}{edgelayer}
		\draw (37.center) to (39);
		\draw (35.center) to (34.center);
		\draw (36.center) to (38.center);
	\end{pgfonlayer}
\end{tikzpicture}
=
\left\llbracket
\begin{tikzpicture}[tikzfig]
	\begin{pgfonlayer}{nodelayer}
		\node [style=nothing] (35) at (-1.25, 3.75) {};
		\node [style=nothing] (36) at (-0.75, 3.75) {};
		\node [style=nothing] (37) at (-1.75, 5.25) {};
		\node [style=nothing] (38) at (-1.25, 5.25) {};
		\node [style=nothing] (39) at (-0.75, 5.25) {};
		\node [style=zeroin] (40) at (-1.75, 3.75) {};
	\end{pgfonlayer}
	\begin{pgfonlayer}{edgelayer}
		\draw (40) to (37);
		\draw (35) to (38);
		\draw (36) to (39);
	\end{pgfonlayer}
\end{tikzpicture}
\right\rrbracket_{\hat{\TOF}}
\end{align*}

\item[\ref{TOF.3}:]
This follows from the spider law.

\item[\ref{TOF.4}:]
This follows from the spider law.

\item[\ref{TOF.5}:]
This follows from the spider law.

\item[\ref{TOF.6}:]
This follows from the spider law.

\item[\ref{TOF.7}:]
\begin{align*}
\left\llbracket
\begin{tikzpicture}[tikzfig]
	\begin{pgfonlayer}{nodelayer}
		\node [style=nothing] (36) at (0, 3.75) {};
		\node [style=nothing] (37) at (-0.5, 3.75) {};
		\node [style=nothing] (38) at (-0.5, 6.25) {};
		\node [style=nothing] (39) at (0, 6.25) {};
		\node [style=zeroout] (40) at (0.5, 6.25) {};
		\node [style=oplus] (41) at (0.5, 5.75) {};
		\node [style=dot] (42) at (0, 5.75) {};
		\node [style=dot] (43) at (-0.5, 4.25) {};
		\node [style=oplus] (44) at (0.5, 4.25) {};
		\node [style=zeroout] (45) at (0.5, 4.75) {};
		\node [style=onein] (46) at (0.5, 3.75) {};
		\node [style=onein] (47) at (0.5, 5.25) {};
	\end{pgfonlayer}
	\begin{pgfonlayer}{edgelayer}
		\draw (37) to (43);
		\draw (43) to (38);
		\draw (39) to (42);
		\draw (42) to (36);
		\draw (44) to (45);
		\draw (44) to (43);
		\draw (41) to (40);
		\draw (41) to (42);
		\draw (46) to (44);
		\draw (47) to (41);
	\end{pgfonlayer}
\end{tikzpicture}
\right\rrbracket_{\hat{\TOF}}
&=
\begin{tikzpicture}[tikzfig]
	\begin{pgfonlayer}{nodelayer}
		\node [style=X] (37) at (-0.5, 4.5) {};
		\node [style=X] (38) at (0, 6.25) {};
		\node [style=Z] (39) at (0.5, 6.25) {};
		\node [style=Z] (40) at (0.5, 4.5) {};
		\node [style=Z] (41) at (0.5, 5) {};
		\node [style=Z] (42) at (0.5, 6.75) {};
		\node [style=none] (43) at (0, 3.75) {};
		\node [style=none] (44) at (-0.5, 3.75) {};
		\node [style=none] (45) at (0, 7) {};
		\node [style=none] (46) at (-0.5, 7) {};
		\node [style=Z] (47) at (0.5, 4) {$\pi$};
		\node [style=Z] (48) at (0.5, 5.75) {$\pi$};
	\end{pgfonlayer}
	\begin{pgfonlayer}{edgelayer}
		\draw (46.center) to (44.center);
		\draw (43.center) to (45.center);
		\draw (42) to (48);
		\draw (41) to (47);
		\draw (40) to (37);
		\draw (39) to (38);
	\end{pgfonlayer}
\end{tikzpicture}
\eq{\ref{ZXA.1}}
\begin{tikzpicture}[tikzfig]
	\begin{pgfonlayer}{nodelayer}
		\node [style=X] (38) at (-0.5, 4.5) {};
		\node [style=X] (39) at (0, 5.25) {};
		\node [style=none] (40) at (0, 3.75) {};
		\node [style=none] (41) at (-0.5, 3.75) {};
		\node [style=none] (42) at (0, 6) {};
		\node [style=none] (43) at (-0.5, 6) {};
		\node [style=Z] (44) at (0.5, 4.5) {$\pi$};
		\node [style=Z] (45) at (0.5, 5.25) {$\pi$};
	\end{pgfonlayer}
	\begin{pgfonlayer}{edgelayer}
		\draw (43.center) to (41.center);
		\draw (40.center) to (42.center);
		\draw (44) to (38);
		\draw (45) to (39);
	\end{pgfonlayer}
\end{tikzpicture}
\eq{\ref{ZXA.16}}
\begin{tikzpicture}[tikzfig]
	\begin{pgfonlayer}{nodelayer}
		\node [style=X] (39) at (-0.5, 4.25) {};
		\node [style=X] (40) at (0, 5) {};
		\node [style=none] (41) at (0, 3.75) {};
		\node [style=none] (42) at (-0.5, 3.75) {};
		\node [style=none] (43) at (0, 6.75) {};
		\node [style=none] (44) at (-0.5, 6.75) {};
		\node [style=Z] (45) at (0.5, 6.5) {$\pi$};
		\node [style=andin] (46) at (0.5, 5.75) {};
		\node [style=none] (47) at (1, 5.25) {};
		\node [style=none] (48) at (1, 4.75) {};
	\end{pgfonlayer}
	\begin{pgfonlayer}{edgelayer}
		\draw (44.center) to (42.center);
		\draw (41.center) to (43.center);
		\draw (45) to (46);
		\draw [in=30, out=-124] (46) to (40);
		\draw [in=90, out=-45] (46) to (47.center);
		\draw (47.center) to (48.center);
		\draw [in=0, out=-90, looseness=0.50] (48.center) to (39);
	\end{pgfonlayer}
\end{tikzpicture}\\
&\eq{\ref{ZXA.1}}
\begin{tikzpicture}[tikzfig]
	\begin{pgfonlayer}{nodelayer}
		\node [style=X] (40) at (-0.75, 4.25) {};
		\node [style=X] (41) at (0.25, 4.25) {};
		\node [style=none] (42) at (0.25, 3.75) {};
		\node [style=none] (43) at (-0.75, 3.75) {};
		\node [style=none] (44) at (0.25, 6.75) {};
		\node [style=none] (45) at (-0.75, 6.75) {};
		\node [style=andin] (46) at (-0.25, 5.25) {};
		\node [style=Z] (47) at (0.75, 5.5) {$\pi$};
		\node [style=Z] (48) at (0.75, 6.5) {};
		\node [style=Z] (49) at (0.75, 6) {};
	\end{pgfonlayer}
	\begin{pgfonlayer}{edgelayer}
		\draw (45.center) to (43.center);
		\draw (42.center) to (44.center);
		\draw (46) to (41);
		\draw (48) to (49);
		\draw (49) to (47);
		\draw [in=90, out=180, looseness=0.75] (49) to (46);
		\draw [in=63, out=-117] (46) to (40);
	\end{pgfonlayer}
\end{tikzpicture}
=
\left\llbracket
\begin{tikzpicture}[tikzfig]
	\begin{pgfonlayer}{nodelayer}
		\node [style=nothing] (41) at (0, 3.75) {};
		\node [style=nothing] (42) at (-0.5, 3.75) {};
		\node [style=nothing] (43) at (-0.5, 4.75) {};
		\node [style=nothing] (44) at (0, 4.75) {};
		\node [style=dot] (45) at (-0.5, 4.25) {};
		\node [style=dot] (46) at (0, 4.25) {};
		\node [style=onein] (47) at (0.5, 3.75) {};
		\node [style=zeroout] (48) at (0.5, 4.75) {};
		\node [style=oplus] (49) at (0.5, 4.25) {};
	\end{pgfonlayer}
	\begin{pgfonlayer}{edgelayer}
		\draw (42) to (45);
		\draw (45) to (43);
		\draw (44) to (46);
		\draw (46) to (41);
		\draw (47) to (49);
		\draw (49) to (48);
		\draw (49) to (46);
		\draw (46) to (45);
	\end{pgfonlayer}
\end{tikzpicture}
\right\rrbracket_{\hat{\TOF}}
\end{align*}


\item[\ref{TOF.8}:]
This follows immediately from Lemma \ref{lem:blackdot} and \ref{ZXA.7}.

\item[\ref{TOF.9}:]

\begin{align*}
\left\llbracket
\begin{tikzpicture}[tikzfig]
	\begin{pgfonlayer}{nodelayer}
		\node [style=nothing] (42) at (-1.75, 3.75) {};
		\node [style=nothing] (43) at (-1.25, 3.75) {};
		\node [style=nothing] (44) at (-0.75, 3.75) {};
		\node [style=dot] (45) at (-1.75, 4.25) {};
		\node [style=dot] (46) at (-1.25, 4.25) {};
		\node [style=oplus] (47) at (-0.75, 4.25) {};
		\node [style=dot] (48) at (-1.75, 4.75) {};
		\node [style=oplus] (49) at (-0.75, 4.75) {};
		\node [style=dot] (50) at (-1.25, 4.75) {};
		\node [style=nothing] (51) at (-1.25, 5.25) {};
		\node [style=nothing] (52) at (-0.75, 5.25) {};
		\node [style=nothing] (53) at (-1.75, 5.25) {};
	\end{pgfonlayer}
	\begin{pgfonlayer}{edgelayer}
		\draw (42) to (45);
		\draw (43) to (46);
		\draw (44) to (47);
		\draw (45) to (46);
		\draw (46) to (47);
		\draw (48) to (50);
		\draw (50) to (49);
		\draw (45) to (48);
		\draw (48) to (53);
		\draw (46) to (50);
		\draw (50) to (51);
		\draw (47) to (49);
		\draw (49) to (52);
	\end{pgfonlayer}
\end{tikzpicture}
\right\rrbracket_{\hat{\TOF}}
&=
\begin{tikzpicture}[tikzfig]
	\begin{pgfonlayer}{nodelayer}
		\node [style=X] (43) at (-0.75, 6.25) {};
		\node [style=X] (44) at (0.25, 6.25) {};
		\node [style=none] (45) at (0.25, 3.75) {};
		\node [style=none] (46) at (-0.75, 3.75) {};
		\node [style=none] (47) at (0.25, 8.5) {};
		\node [style=none] (48) at (-0.75, 8.5) {};
		\node [style=andin] (49) at (-0.25, 7.25) {};
		\node [style=Z] (50) at (0.75, 8) {};
		\node [style=none] (51) at (0.75, 8.5) {};
		\node [style=none] (52) at (0.75, 3.75) {};
		\node [style=X] (53) at (0.25, 4.25) {};
		\node [style=andin] (54) at (-0.25, 5.25) {};
		\node [style=Z] (55) at (0.75, 6) {};
		\node [style=X] (56) at (-0.75, 4.25) {};
	\end{pgfonlayer}
	\begin{pgfonlayer}{edgelayer}
		\draw (48.center) to (46.center);
		\draw (45.center) to (47.center);
		\draw (49) to (44);
		\draw [in=90, out=180, looseness=0.75] (50) to (49);
		\draw [in=63, out=-117] (49) to (43);
		\draw (54) to (53);
		\draw [in=90, out=180, looseness=0.75] (55) to (54);
		\draw [in=63, out=-117] (54) to (56);
		\draw (51.center) to (52.center);
	\end{pgfonlayer}
\end{tikzpicture}
\eq{\ref{ZXA.3}}
\begin{tikzpicture}[tikzfig]
	\begin{pgfonlayer}{nodelayer}
		\node [style=X] (1) at (-0.75, 8.25) {};
		\node [style=X] (2) at (0.25, 8.25) {};
		\node [style=none] (3) at (0.75, 6.5) {};
		\node [style=none] (4) at (-1.25, 6.5) {};
		\node [style=none] (5) at (0.75, 10) {};
		\node [style=none] (6) at (-1.25, 10) {};
		\node [style=andin] (7) at (-0.25, 9.25) {};
		\node [style=Z] (8) at (1.25, 8.25) {};
		\node [style=none] (9) at (1.25, 10) {};
		\node [style=none] (10) at (1.25, 6.5) {};
		\node [style=andout] (11) at (-0.25, 7.25) {};
		\node [style=X] (12) at (-1.25, 8.25) {};
		\node [style=X] (13) at (0.75, 8.25) {};
	\end{pgfonlayer}
	\begin{pgfonlayer}{edgelayer}
		\draw (7) to (2);
		\draw [in=90, out=105, looseness=1.50] (8) to (7);
		\draw [in=63, out=-117] (7) to (1);
		\draw (9.center) to (10.center);
		\draw (2) to (11);
		\draw (11) to (1);
		\draw [in=-105, out=-90, looseness=1.75] (11) to (8);
		\draw (5.center) to (13);
		\draw (13) to (3.center);
		\draw (13) to (2);
		\draw (1) to (12);
		\draw (12) to (6.center);
		\draw (12) to (4.center);
	\end{pgfonlayer}
\end{tikzpicture}\\
&=
\begin{tikzpicture}[tikzfig]
	\begin{pgfonlayer}{nodelayer}
		\node [style=X] (2) at (-1, 9.25) {};
		\node [style=X] (3) at (-0.25, 9.25) {};
		\node [style=none] (4) at (0.25, 6.5) {};
		\node [style=none] (5) at (-1.5, 6.5) {};
		\node [style=none] (6) at (0.25, 10.5) {};
		\node [style=none] (7) at (-1.5, 10.5) {};
		\node [style=none] (8) at (-1, 8.25) {};
		\node [style=Z] (9) at (0.75, 7) {};
		\node [style=none] (10) at (0.75, 10.5) {};
		\node [style=none] (11) at (0.75, 6.5) {};
		\node [style=none] (12) at (-0.25, 8.25) {};
		\node [style=X] (13) at (-1.5, 9.75) {};
		\node [style=X] (14) at (0.25, 9.75) {};
		\node [style=none] (15) at (-1, 7.5) {};
		\node [style=none] (16) at (-0.25, 7.75) {};
		\node [style=andout] (17) at (-1, 8.25) {};
		\node [style=andout] (18) at (-0.25, 8.25) {};
	\end{pgfonlayer}
	\begin{pgfonlayer}{edgelayer}
		\draw (8.center) to (3);
		\draw [in=-120, out=120, looseness=1.25] (8.center) to (2);
		\draw (10.center) to (11.center);
		\draw [in=60, out=-60, looseness=1.25] (3) to (12.center);
		\draw (12.center) to (2);
		\draw (6.center) to (14);
		\draw (14) to (4.center);
		\draw [in=90, out=180, looseness=1.75] (14) to (3);
		\draw [in=0, out=90, looseness=1.75] (2) to (13);
		\draw (13) to (7.center);
		\draw (13) to (5.center);
		\draw [in=-90, out=153, looseness=0.75] (9) to (16.center);
		\draw [in=-90, out=180] (9) to (15.center);
		\draw (15.center) to (8.center);
		\draw (16.center) to (12.center);
	\end{pgfonlayer}
\end{tikzpicture}
\eq{\ref{ZXA.12}}
\begin{tikzpicture}[tikzfig]
	\begin{pgfonlayer}{nodelayer}
		\node [style=none] (3) at (0, 6.5) {};
		\node [style=none] (4) at (-1, 6.5) {};
		\node [style=none] (5) at (0, 9) {};
		\node [style=none] (6) at (-1, 9) {};
		\node [style=Z] (7) at (0.5, 7.25) {};
		\node [style=none] (8) at (0.5, 9) {};
		\node [style=none] (9) at (0.5, 6.5) {};
		\node [style=X] (10) at (-1, 8.5) {};
		\node [style=X] (11) at (0, 8.5) {};
		\node [style=X] (12) at (-0.5, 7.25) {};
		\node [style=none] (13) at (-0.5, 8) {};
		\node [style=andout] (14) at (-0.5, 8) {};
	\end{pgfonlayer}
	\begin{pgfonlayer}{edgelayer}
		\draw (8.center) to (9.center);
		\draw (5.center) to (11);
		\draw (11) to (3.center);
		\draw (10) to (6.center);
		\draw (10) to (4.center);
		\draw [bend right] (7) to (12);
		\draw [bend right] (12) to (7);
		\draw (11) to (13.center);
		\draw (13.center) to (10);
		\draw (13.center) to (12);
	\end{pgfonlayer}
\end{tikzpicture}
\eq{\ref{ZXA.8}}
\begin{tikzpicture}[tikzfig]
	\begin{pgfonlayer}{nodelayer}
		\node [style=none] (4) at (0, 6.5) {};
		\node [style=none] (5) at (-1, 6.5) {};
		\node [style=none] (6) at (0, 9) {};
		\node [style=none] (7) at (-1, 9) {};
		\node [style=Z] (8) at (0.5, 7.25) {};
		\node [style=none] (9) at (0.5, 9) {};
		\node [style=none] (10) at (0.5, 6.5) {};
		\node [style=X] (11) at (-1, 8.5) {};
		\node [style=X] (12) at (0, 8.5) {};
		\node [style=X] (13) at (-0.5, 7.25) {};
		\node [style=none] (14) at (-0.5, 8) {};
		\node [style=andout] (15) at (-0.5, 8) {};
	\end{pgfonlayer}
	\begin{pgfonlayer}{edgelayer}
		\draw (9.center) to (10.center);
		\draw (6.center) to (12);
		\draw (12) to (4.center);
		\draw (11) to (7.center);
		\draw (11) to (5.center);
		\draw (12) to (14.center);
		\draw (14.center) to (11);
		\draw (14.center) to (13);
	\end{pgfonlayer}
\end{tikzpicture}\\
&\eq{\ref{ZXA.1}}
\begin{tikzpicture}[tikzfig]
	\begin{pgfonlayer}{nodelayer}
		\node [style=none] (0) at (0, 3) {};
		\node [style=none] (1) at (-1, 3) {};
		\node [style=none] (2) at (0, 5.5) {};
		\node [style=none] (3) at (-1, 5.5) {};
		\node [style=none] (4) at (0.5, 5.5) {};
		\node [style=none] (5) at (0.5, 3) {};
		\node [style=X] (6) at (-1, 5) {};
		\node [style=X] (7) at (0, 5) {};
		\node [style=X] (8) at (-0.5, 3.75) {};
		\node [style=none] (9) at (-0.5, 4.5) {};
		\node [style=andout] (10) at (-0.5, 4.5) {};
	\end{pgfonlayer}
	\begin{pgfonlayer}{edgelayer}
		\draw (4.center) to (5.center);
		\draw (2.center) to (7);
		\draw (7) to (0.center);
		\draw (6) to (3.center);
		\draw (6) to (1.center);
		\draw (7) to (9.center);
		\draw (9.center) to (6);
		\draw (9.center) to (8);
	\end{pgfonlayer}
\end{tikzpicture}
\eq{\ref{ZXA.13}}
\begin{tikzpicture}[tikzfig]
	\begin{pgfonlayer}{nodelayer}
		\node [style=none] (1) at (0, 3.75) {};
		\node [style=none] (2) at (-1.5, 3.75) {};
		\node [style=none] (3) at (0, 5.5) {};
		\node [style=none] (4) at (-1.5, 5.5) {};
		\node [style=none] (5) at (0.5, 5.5) {};
		\node [style=none] (6) at (0.5, 3.75) {};
		\node [style=X] (7) at (-1.5, 5) {};
		\node [style=X] (8) at (0, 5) {};
		\node [style=X] (9) at (-1, 4.25) {};
		\node [style=X] (10) at (-0.5, 4.25) {};
	\end{pgfonlayer}
	\begin{pgfonlayer}{edgelayer}
		\draw (5.center) to (6.center);
		\draw (3.center) to (8);
		\draw (8) to (1.center);
		\draw (7) to (4.center);
		\draw (7) to (2.center);
		\draw [in=90, out=-124] (8) to (10);
		\draw [in=-56, out=90] (9) to (7);
	\end{pgfonlayer}
\end{tikzpicture}
\eq{\ref{ZXA.3}}
\begin{tikzpicture}[tikzfig]
	\begin{pgfonlayer}{nodelayer}
		\node [style=nothing] (2) at (-1.75, 0) {};
		\node [style=nothing] (3) at (-1.25, 0) {};
		\node [style=nothing] (4) at (-0.75, 0) {};
		\node [style=nothing] (5) at (-1.25, 1.5) {};
		\node [style=nothing] (6) at (-0.75, 1.5) {};
		\node [style=nothing] (7) at (-1.75, 1.5) {};
	\end{pgfonlayer}
	\begin{pgfonlayer}{edgelayer}
		\draw (2) to (7);
		\draw (3) to (5);
		\draw (4) to (6);
	\end{pgfonlayer}
\end{tikzpicture}
=
\left\llbracket
\begin{tikzpicture}[tikzfig]
	\begin{pgfonlayer}{nodelayer}
		\node [style=nothing] (3) at (-1.75, 0) {};
		\node [style=nothing] (4) at (-1.25, 0) {};
		\node [style=nothing] (5) at (-0.75, 0) {};
		\node [style=nothing] (6) at (-1.25, 1.5) {};
		\node [style=nothing] (7) at (-0.75, 1.5) {};
		\node [style=nothing] (8) at (-1.75, 1.5) {};
	\end{pgfonlayer}
	\begin{pgfonlayer}{edgelayer}
		\draw (3) to (8);
		\draw (4) to (6);
		\draw (5) to (7);
	\end{pgfonlayer}
\end{tikzpicture}
\right\rrbracket_{\hat{\TOF}}
\end{align*}


\item[\ref{TOF.10}:]  It is easier to prove that $\ref{TOF.10}$ is redundant.  Given \ref{TOF.9},  \ref{TOF.6} and \ref{TOF.12}, \ref{TOF.10} is equivalent to the following:
$$
\begin{tikzpicture}[tikzfig]
	\begin{pgfonlayer}{nodelayer}
		\node [style=dot] (4) at (0, 3) {};
		\node [style=dot] (5) at (0.5, 3) {};
		\node [style=dot] (6) at (-0.5, 3.5) {};
		\node [style=dot] (7) at (0, 3.5) {};
		\node [style=dot] (8) at (0, 4) {};
		\node [style=dot] (9) at (0.5, 4) {};
		\node [style=dot] (10) at (-0.5, 4.5) {};
		\node [style=dot] (11) at (0, 4.5) {};
		\node [style=oplus] (12) at (1, 3) {};
		\node [style=oplus] (13) at (0.5, 3.5) {};
		\node [style=oplus] (14) at (1, 4) {};
		\node [style=oplus] (15) at (0.5, 4.5) {};
		\node [style=none] (16) at (1, 2.5) {};
		\node [style=none] (17) at (0.5, 2.5) {};
		\node [style=none] (18) at (0, 2.5) {};
		\node [style=none] (19) at (-0.5, 2.5) {};
		\node [style=none] (20) at (-0.5, 5) {};
		\node [style=none] (21) at (0, 5) {};
		\node [style=none] (22) at (0.5, 5) {};
		\node [style=none] (23) at (1, 5) {};
	\end{pgfonlayer}
	\begin{pgfonlayer}{edgelayer}
		\draw (16.center) to (23.center);
		\draw (22.center) to (17.center);
		\draw (18.center) to (21.center);
		\draw (20.center) to (19.center);
		\draw (4) to (12);
		\draw (13) to (6);
		\draw (8) to (14);
		\draw (15) to (10);
	\end{pgfonlayer}
\end{tikzpicture}
\eq{\ref{TOF.10}}
\begin{tikzpicture}[tikzfig]
	\begin{pgfonlayer}{nodelayer}
		\node [style=dot] (5) at (-0.5, 3.5) {};
		\node [style=dot] (6) at (0, 3.5) {};
		\node [style=dot] (7) at (-0.5, 4) {};
		\node [style=dot] (8) at (0, 4) {};
		\node [style=dot] (9) at (-0.5, 4.5) {};
		\node [style=dot] (10) at (0, 4.5) {};
		\node [style=oplus] (11) at (1, 3.5) {};
		\node [style=oplus] (12) at (0.5, 4) {};
		\node [style=oplus] (13) at (0.5, 4.5) {};
		\node [style=none] (14) at (1, 3) {};
		\node [style=none] (15) at (0.5, 3) {};
		\node [style=none] (16) at (0, 3) {};
		\node [style=none] (17) at (-0.5, 3) {};
		\node [style=none] (18) at (-0.5, 5) {};
		\node [style=none] (19) at (0, 5) {};
		\node [style=none] (20) at (0.5, 5) {};
		\node [style=none] (21) at (1, 5) {};
	\end{pgfonlayer}
	\begin{pgfonlayer}{edgelayer}
		\draw (14.center) to (21.center);
		\draw (20.center) to (15.center);
		\draw (16.center) to (19.center);
		\draw (18.center) to (17.center);
		\draw (11) to (5);
		\draw (7) to (12);
		\draw (13) to (9);
	\end{pgfonlayer}
\end{tikzpicture}
\eq{\ref{TOF.9}}
\begin{tikzpicture}[tikzfig]
	\begin{pgfonlayer}{nodelayer}
		\node [style=dot] (6) at (-0.5, 3) {};
		\node [style=dot] (7) at (0, 3) {};
		\node [style=oplus] (8) at (1, 3) {};
		\node [style=none] (9) at (1, 2.5) {};
		\node [style=none] (10) at (0.5, 2.5) {};
		\node [style=none] (11) at (0, 2.5) {};
		\node [style=none] (12) at (-0.5, 2.5) {};
		\node [style=none] (13) at (-0.5, 3.5) {};
		\node [style=none] (14) at (0, 3.5) {};
		\node [style=none] (15) at (0.5, 3.5) {};
		\node [style=none] (16) at (1, 3.5) {};
	\end{pgfonlayer}
	\begin{pgfonlayer}{edgelayer}
		\draw (9.center) to (16.center);
		\draw (15.center) to (10.center);
		\draw (11.center) to (14.center);
		\draw (13.center) to (12.center);
		\draw (6) to (8);
	\end{pgfonlayer}
\end{tikzpicture}
$$

However
$$
\begin{tikzpicture}[tikzfig]
	\begin{pgfonlayer}{nodelayer}
		\node [style=dot] (7) at (0, 3) {};
		\node [style=dot] (8) at (0.5, 3) {};
		\node [style=dot] (9) at (-0.5, 3.5) {};
		\node [style=dot] (10) at (0, 3.5) {};
		\node [style=dot] (11) at (0, 4) {};
		\node [style=dot] (12) at (0.5, 4) {};
		\node [style=dot] (13) at (-0.5, 4.5) {};
		\node [style=dot] (14) at (0, 4.5) {};
		\node [style=oplus] (15) at (1, 3) {};
		\node [style=oplus] (16) at (0.5, 3.5) {};
		\node [style=oplus] (17) at (1, 4) {};
		\node [style=oplus] (18) at (0.5, 4.5) {};
		\node [style=none] (19) at (1, 2.5) {};
		\node [style=none] (20) at (0.5, 2.5) {};
		\node [style=none] (21) at (0, 2.5) {};
		\node [style=none] (22) at (-0.5, 2.5) {};
		\node [style=none] (23) at (-0.5, 5) {};
		\node [style=none] (24) at (0, 5) {};
		\node [style=none] (25) at (0.5, 5) {};
		\node [style=none] (26) at (1, 5) {};
	\end{pgfonlayer}
	\begin{pgfonlayer}{edgelayer}
		\draw (19.center) to (26.center);
		\draw (25.center) to (20.center);
		\draw (21.center) to (24.center);
		\draw (23.center) to (22.center);
		\draw (7) to (15);
		\draw (16) to (9);
		\draw (11) to (17);
		\draw (18) to (13);
	\end{pgfonlayer}
\end{tikzpicture}
\eq{\ref{TOF.12}}
\begin{tikzpicture}[tikzfig]
	\begin{pgfonlayer}{nodelayer}
		\node [style=dot] (8) at (0, 3) {};
		\node [style=dot] (9) at (0.5, 3) {};
		\node [style=dot] (10) at (-0.5, 3.5) {};
		\node [style=dot] (11) at (0, 3.5) {};
		\node [style=dot] (12) at (0, 4) {};
		\node [style=dot] (13) at (0.5, 4) {};
		\node [style=oplus] (14) at (1, 3) {};
		\node [style=oplus] (15) at (1, 3.5) {};
		\node [style=oplus] (16) at (1, 4) {};
		\node [style=none] (17) at (1, 2.5) {};
		\node [style=none] (18) at (0.5, 2.5) {};
		\node [style=none] (19) at (0, 2.5) {};
		\node [style=none] (20) at (-0.5, 2.5) {};
		\node [style=none] (21) at (-0.5, 4.5) {};
		\node [style=none] (22) at (0, 4.5) {};
		\node [style=none] (23) at (0.5, 4.5) {};
		\node [style=none] (24) at (1, 4.5) {};
	\end{pgfonlayer}
	\begin{pgfonlayer}{edgelayer}
		\draw (17.center) to (24.center);
		\draw (23.center) to (18.center);
		\draw (19.center) to (22.center);
		\draw (21.center) to (20.center);
		\draw (8) to (14);
		\draw (15) to (10);
		\draw (12) to (16);
	\end{pgfonlayer}
\end{tikzpicture}
\eq{\ref{TOF.6}}
\begin{tikzpicture}[tikzfig]
	\begin{pgfonlayer}{nodelayer}
		\node [style=dot] (9) at (0, 3) {};
		\node [style=dot] (10) at (0.5, 3) {};
		\node [style=dot] (11) at (-0.5, 3.5) {};
		\node [style=dot] (12) at (0, 3.5) {};
		\node [style=dot] (13) at (0, 4) {};
		\node [style=dot] (14) at (0.5, 4) {};
		\node [style=oplus] (15) at (1, 3) {};
		\node [style=oplus] (16) at (1, 3.5) {};
		\node [style=oplus] (17) at (1, 4) {};
		\node [style=none] (18) at (1, 2.5) {};
		\node [style=none] (19) at (0.5, 2.5) {};
		\node [style=none] (20) at (0, 2.5) {};
		\node [style=none] (21) at (-0.5, 2.5) {};
		\node [style=none] (22) at (-0.5, 4.5) {};
		\node [style=none] (23) at (0, 4.5) {};
		\node [style=none] (24) at (0.5, 4.5) {};
		\node [style=none] (25) at (1, 4.5) {};
	\end{pgfonlayer}
	\begin{pgfonlayer}{edgelayer}
		\draw (18.center) to (25.center);
		\draw (24.center) to (19.center);
		\draw (20.center) to (23.center);
		\draw (22.center) to (21.center);
		\draw (9) to (15);
		\draw (16) to (11);
		\draw (13) to (17);
	\end{pgfonlayer}
\end{tikzpicture}
\eq{\ref{TOF.9}}
\begin{tikzpicture}[tikzfig]
	\begin{pgfonlayer}{nodelayer}
		\node [style=dot] (10) at (-0.5, 4) {};
		\node [style=dot] (11) at (0, 4) {};
		\node [style=oplus] (12) at (1, 4) {};
		\node [style=none] (13) at (1, 3.5) {};
		\node [style=none] (14) at (0.5, 3.5) {};
		\node [style=none] (15) at (0, 3.5) {};
		\node [style=none] (16) at (-0.5, 3.5) {};
		\node [style=none] (17) at (-0.5, 4.5) {};
		\node [style=none] (18) at (0, 4.5) {};
		\node [style=none] (19) at (0.5, 4.5) {};
		\node [style=none] (20) at (1, 4.5) {};
	\end{pgfonlayer}
	\begin{pgfonlayer}{edgelayer}
		\draw (13.center) to (20.center);
		\draw (19.center) to (14.center);
		\draw (15.center) to (18.center);
		\draw (17.center) to (16.center);
		\draw (12) to (10);
	\end{pgfonlayer}
\end{tikzpicture}
$$

\item[\ref{TOF.11}:]
\begin{align*}
\left\llbracket
\begin{tikzpicture}[tikzfig]
	\begin{pgfonlayer}{nodelayer}
		\node [style=nothing] (11) at (0, 2.5) {};
		\node [style=nothing] (12) at (-0.5, 2.5) {};
		\node [style=nothing] (13) at (-1, 2.5) {};
		\node [style=nothing] (14) at (-1.5, 2.5) {};
		\node [style=nothing] (15) at (-0.5, 4.5) {};
		\node [style=nothing] (16) at (0, 4.5) {};
		\node [style=dot] (17) at (-1.5, 3) {};
		\node [style=dot] (18) at (-1, 3.5) {};
		\node [style=dot] (19) at (-0.5, 3.5) {};
		\node [style=oplus] (20) at (-1, 3) {};
		\node [style=oplus] (21) at (0, 3.5) {};
		\node [style=nothing] (22) at (-1.5, 4.5) {};
		\node [style=nothing] (23) at (-1, 4.5) {};
		\node [style=oplus] (24) at (-1, 4) {};
		\node [style=dot] (25) at (-1.5, 4) {};
	\end{pgfonlayer}
	\begin{pgfonlayer}{edgelayer}
		\draw (17) to (20);
		\draw (18) to (19);
		\draw (19) to (21);
		\draw (11) to (21);
		\draw (21) to (16);
		\draw (15) to (19);
		\draw (19) to (12);
		\draw (13) to (20);
		\draw (20) to (18);
		\draw (17) to (14);
		\draw (17) to (25);
		\draw (25) to (22);
		\draw (23) to (24);
		\draw (24) to (18);
		\draw (24) to (25);
	\end{pgfonlayer}
\end{tikzpicture}
\right\rrbracket_{\hat{\TOF}}
&=
\begin{tikzpicture}[tikzfig]
	\begin{pgfonlayer}{nodelayer}
		\node [style=none] (12) at (-0.75, 4.25) {};
		\node [style=X] (13) at (-1.25, 3.5) {};
		\node [style=X] (14) at (-0.25, 3.5) {};
		\node [style=Z] (15) at (0.25, 4.75) {};
		\node [style=Z] (16) at (-1.25, 3) {};
		\node [style=Z] (17) at (-1.25, 4.75) {};
		\node [style=X] (18) at (-1.75, 3) {};
		\node [style=X] (19) at (-1.75, 4.75) {};
		\node [style=none] (20) at (0.25, 5.25) {};
		\node [style=none] (21) at (-0.25, 5.25) {};
		\node [style=none] (22) at (-1.75, 2.5) {};
		\node [style=none] (23) at (-1.25, 2.5) {};
		\node [style=none] (24) at (-0.25, 2.5) {};
		\node [style=none] (25) at (0.25, 2.5) {};
		\node [style=none] (26) at (-1.75, 5.25) {};
		\node [style=none] (27) at (-1.25, 5.25) {};
		\node [style=andin] (28) at (-0.75, 4.25) {};
	\end{pgfonlayer}
	\begin{pgfonlayer}{edgelayer}
		\draw (20.center) to (15);
		\draw [in=90, out=180] (15) to (12.center);
		\draw (12.center) to (14);
		\draw (14) to (24.center);
		\draw (25.center) to (15);
		\draw (12.center) to (13);
		\draw (13) to (17);
		\draw (13) to (16);
		\draw (16) to (18);
		\draw (19) to (17);
		\draw (18) to (22.center);
		\draw (23.center) to (16);
		\draw (27.center) to (17);
		\draw (19) to (26.center);
		\draw (19) to (18);
		\draw (21.center) to (14);
	\end{pgfonlayer}
\end{tikzpicture}
\eq{\ref{ZXA.3}}
\begin{tikzpicture}[tikzfig]
	\begin{pgfonlayer}{nodelayer}
		\node [style=none] (13) at (-0.25, 4.5) {};
		\node [style=X] (14) at (-0.75, 3.75) {};
		\node [style=X] (15) at (0.25, 3.75) {};
		\node [style=Z] (16) at (0.75, 5) {};
		\node [style=Z] (17) at (-1.25, 3) {};
		\node [style=Z] (18) at (-1.25, 4.5) {};
		\node [style=X] (19) at (-1.75, 3.75) {};
		\node [style=none] (20) at (0.75, 5.5) {};
		\node [style=none] (21) at (0.25, 5.5) {};
		\node [style=none] (22) at (-2.25, 2.5) {};
		\node [style=none] (23) at (-1.25, 2.5) {};
		\node [style=none] (24) at (0.25, 2.5) {};
		\node [style=none] (25) at (0.75, 2.5) {};
		\node [style=none] (26) at (-2.25, 5.5) {};
		\node [style=none] (27) at (-1.25, 5.5) {};
		\node [style=andin] (28) at (-0.25, 4.5) {};
		\node [style=X] (29) at (-2.25, 3.75) {};
	\end{pgfonlayer}
	\begin{pgfonlayer}{edgelayer}
		\draw (20.center) to (16);
		\draw [in=90, out=180] (16) to (13.center);
		\draw (13.center) to (15);
		\draw (15) to (24.center);
		\draw (25.center) to (16);
		\draw (13.center) to (14);
		\draw (17) to (19);
		\draw (23.center) to (17);
		\draw (27.center) to (18);
		\draw (21.center) to (15);
		\draw (18) to (19);
		\draw (19) to (29);
		\draw (29) to (26.center);
		\draw (29) to (22.center);
		\draw (14) to (18);
		\draw (14) to (17);
	\end{pgfonlayer}
\end{tikzpicture}
=
\begin{tikzpicture}[tikzfig]
	\begin{pgfonlayer}{nodelayer}
		\node [style=none] (14) at (-0.25, 4.5) {};
		\node [style=X] (15) at (-0.75, 3.75) {};
		\node [style=X] (16) at (0.25, 3.75) {};
		\node [style=Z] (17) at (0.75, 5) {};
		\node [style=Z] (18) at (-1.25, 3) {};
		\node [style=Z] (19) at (-1.75, 3.75) {};
		\node [style=X] (20) at (-1.25, 4.5) {};
		\node [style=none] (21) at (0.75, 5.5) {};
		\node [style=none] (22) at (0.25, 5.5) {};
		\node [style=none] (23) at (-2.25, 2.5) {};
		\node [style=none] (24) at (-1.25, 2.5) {};
		\node [style=none] (25) at (0.25, 2.5) {};
		\node [style=none] (26) at (0.75, 2.5) {};
		\node [style=none] (27) at (-2.25, 5.5) {};
		\node [style=none] (28) at (-1.75, 5.5) {};
		\node [style=andin] (29) at (-0.25, 4.5) {};
		\node [style=X] (30) at (-2.25, 5) {};
	\end{pgfonlayer}
	\begin{pgfonlayer}{edgelayer}
		\draw (21.center) to (17);
		\draw [in=90, out=180] (17) to (14.center);
		\draw (14.center) to (16);
		\draw (16) to (25.center);
		\draw (26.center) to (17);
		\draw (14.center) to (15);
		\draw (18) to (20);
		\draw (24.center) to (18);
		\draw [in=90, out=-90] (28.center) to (19);
		\draw (22.center) to (16);
		\draw (19) to (20);
		\draw (20) to (30);
		\draw (30) to (27.center);
		\draw (30) to (23.center);
		\draw (15) to (19);
		\draw (15) to (18);
	\end{pgfonlayer}
\end{tikzpicture}\\
&\eq{\ref{ZXA.5}}
\begin{tikzpicture}[tikzfig]
	\begin{pgfonlayer}{nodelayer}
		\node [style=X] (15) at (-0.75, 8.75) {};
		\node [style=none] (16) at (-2.75, 10.5) {};
		\node [style=none] (17) at (-0.25, 10.5) {};
		\node [style=none] (18) at (-2.75, 7.5) {};
		\node [style=none] (19) at (-3.25, 7.5) {};
		\node [style=none] (20) at (-0.75, 7.5) {};
		\node [style=none] (21) at (-0.75, 10.5) {};
		\node [style=Z] (22) at (-0.25, 10) {};
		\node [style=none] (23) at (-1.25, 9.5) {};
		\node [style=X] (24) at (-3.25, 10) {};
		\node [style=none] (25) at (-0.25, 7.5) {};
		\node [style=none] (26) at (-3.25, 10.5) {};
		\node [style=Z] (27) at (-2, 9) {};
		\node [style=X] (28) at (-2.75, 8.25) {};
		\node [style=andin] (29) at (-1.25, 9.5) {};
	\end{pgfonlayer}
	\begin{pgfonlayer}{edgelayer}
		\draw (17.center) to (22);
		\draw [in=90, out=180] (22) to (23.center);
		\draw (23.center) to (15);
		\draw (15) to (20.center);
		\draw (25.center) to (22);
		\draw (21.center) to (15);
		\draw (24) to (26.center);
		\draw (24) to (19.center);
		\draw (27) to (28);
		\draw (27) to (24);
		\draw (28) to (18.center);
		\draw (28) to (16.center);
		\draw (23.center) to (27);
	\end{pgfonlayer}
\end{tikzpicture}
\eq{\ref{ZXA.17}}
\begin{tikzpicture}[tikzfig]
	\begin{pgfonlayer}{nodelayer}
		\node [style=none] (0) at (-1.25, 10.75) {};
		\node [style=none] (1) at (-3.5, 10.75) {};
		\node [style=X] (2) at (-1.25, 11.5) {};
		\node [style=none] (3) at (-4, 10.75) {};
		\node [style=X] (4) at (-4, 11.5) {};
		\node [style=none] (5) at (-1.25, 15) {};
		\node [style=none] (6) at (-4, 15) {};
		\node [style=none] (7) at (-3.5, 15) {};
		\node [style=none] (8) at (-0.75, 10.75) {};
		\node [style=none] (9) at (-0.75, 15) {};
		\node [style=Z] (10) at (-0.75, 14.5) {};
		\node [style=X] (11) at (-3.5, 12.5) {};
		\node [style=none] (12) at (-1.75, 13.25) {};
		\node [style=Z] (13) at (-2.25, 14) {};
		\node [style=X] (14) at (-1.75, 12.25) {};
		\node [style=andin] (15) at (-2.75, 13.25) {};
		\node [style=none] (16) at (-2.75, 13.25) {};
		\node [style=andin] (17) at (-1.75, 13.25) {};
	\end{pgfonlayer}
	\begin{pgfonlayer}{edgelayer}
		\draw (9.center) to (10);
		\draw (2) to (0.center);
		\draw (8.center) to (10);
		\draw (5.center) to (2);
		\draw (4) to (6.center);
		\draw (4) to (3.center);
		\draw (11) to (1.center);
		\draw (11) to (7.center);
		\draw [in=-90, out=124] (2) to (14);
		\draw (14) to (12.center);
		\draw (12.center) to (13);
		\draw (12.center) to (4);
		\draw (13) to (16.center);
		\draw (16.center) to (11);
		\draw (14) to (16.center);
		\draw [in=90, out=180] (10) to (13);
	\end{pgfonlayer}
\end{tikzpicture}\\
&\eq{\ref{ZXA.1},\ref{ZXA.3}}
\begin{tikzpicture}[tikzfig]
	\begin{pgfonlayer}{nodelayer}
		\node [style=none] (1) at (-1.25, 10.75) {};
		\node [style=none] (2) at (-3.5, 10.75) {};
		\node [style=X] (3) at (-1.25, 11.25) {};
		\node [style=none] (4) at (-4, 10.75) {};
		\node [style=X] (5) at (-4, 11.25) {};
		\node [style=none] (6) at (-1.25, 15) {};
		\node [style=none] (7) at (-4, 15) {};
		\node [style=none] (8) at (-3.5, 15) {};
		\node [style=none] (9) at (-0.75, 10.75) {};
		\node [style=none] (10) at (-0.75, 15) {};
		\node [style=Z] (11) at (-0.75, 14.5) {};
		\node [style=X] (12) at (-3.5, 12.5) {};
		\node [style=none] (13) at (-1.75, 13.25) {};
		\node [style=andin] (14) at (-1.75, 13.25) {};
		\node [style=none] (15) at (-2.75, 13.25) {};
		\node [style=Z] (16) at (-0.75, 14) {};
		\node [style=X] (17) at (-1.25, 12.5) {};
		\node [style=andin] (18) at (-2.75, 13.25) {};
	\end{pgfonlayer}
	\begin{pgfonlayer}{edgelayer}
		\draw (10.center) to (11);
		\draw (3) to (1.center);
		\draw (9.center) to (11);
		\draw (6.center) to (3);
		\draw (5) to (7.center);
		\draw (5) to (4.center);
		\draw (12) to (2.center);
		\draw (12) to (8.center);
		\draw (13.center) to (5);
		\draw (15.center) to (12);
		\draw [in=90, out=180] (16) to (13.center);
		\draw [in=180, out=90, looseness=0.75] (15.center) to (11);
		\draw (3) to (15.center);
		\draw (13.center) to (17);
	\end{pgfonlayer}
\end{tikzpicture}
=
\left\llbracket
\begin{tikzpicture}[tikzfig]
	\begin{pgfonlayer}{nodelayer}
		\node [style=nothing] (2) at (0, 0) {};
		\node [style=nothing] (3) at (-1, 0) {};
		\node [style=nothing] (4) at (-0.5, 0) {};
		\node [style=nothing] (5) at (-1.5, 0) {};
		\node [style=dot] (6) at (-1.5, 0.5) {};
		\node [style=dot] (7) at (-0.5, 0.5) {};
		\node [style=oplus] (8) at (0, 0.5) {};
		\node [style=nothing] (9) at (-0.5, 1.5) {};
		\node [style=nothing] (10) at (-1, 1.5) {};
		\node [style=nothing] (11) at (-1.5, 1.5) {};
		\node [style=nothing] (12) at (0, 1.5) {};
		\node [style=dot] (13) at (-1, 1) {};
		\node [style=dot] (14) at (-0.5, 1) {};
		\node [style=oplus] (15) at (0, 1) {};
	\end{pgfonlayer}
	\begin{pgfonlayer}{edgelayer}
		\draw (5) to (6);
		\draw (4) to (7);
		\draw (8) to (2);
		\draw (8) to (7);
		\draw (7) to (6);
		\draw (13) to (3);
		\draw (7) to (14);
		\draw (14) to (9);
		\draw (12) to (15);
		\draw (15) to (8);
		\draw (15) to (14);
		\draw (14) to (13);
		\draw (6) to (11);
		\draw (13) to (10);
	\end{pgfonlayer}
\end{tikzpicture}
\right\rrbracket_{\hat{\TOF}}
\end{align*}


\item[\ref{TOF.12}:]
\begingroup
\allowdisplaybreaks
\begin{align*}
\left\llbracket
\begin{tikzpicture}[tikzfig]
	\begin{pgfonlayer}{nodelayer}
		\node [style=nothing] (3) at (-0.5, 0) {};
		\node [style=nothing] (4) at (0, 0) {};
		\node [style=nothing] (5) at (-1, 0) {};
		\node [style=nothing] (6) at (-1.5, 0) {};
		\node [style=nothing] (7) at (-0.5, 2) {};
		\node [style=nothing] (8) at (-1.5, 2) {};
		\node [style=nothing] (9) at (0, 2) {};
		\node [style=nothing] (10) at (-1, 2) {};
		\node [style=dot] (11) at (-1.5, 0.5) {};
		\node [style=dot] (12) at (-1, 0.5) {};
		\node [style=oplus] (13) at (-0.5, 0.5) {};
		\node [style=oplus] (14) at (0, 1) {};
		\node [style=dot] (15) at (-1, 1) {};
		\node [style=dot] (16) at (-0.5, 1) {};
		\node [style=oplus] (17) at (-0.5, 1.5) {};
		\node [style=dot] (18) at (-1.5, 1.5) {};
		\node [style=dot] (19) at (-1, 1.5) {};
	\end{pgfonlayer}
	\begin{pgfonlayer}{edgelayer}
		\draw (11) to (12);
		\draw (12) to (13);
		\draw (15) to (16);
		\draw (16) to (14);
		\draw (18) to (19);
		\draw (19) to (17);
		\draw (6) to (11);
		\draw (11) to (18);
		\draw (18) to (8);
		\draw (10) to (19);
		\draw (19) to (15);
		\draw (15) to (12);
		\draw (12) to (5);
		\draw (3) to (13);
		\draw (13) to (16);
		\draw (16) to (17);
		\draw (17) to (7);
		\draw (9) to (14);
		\draw (14) to (4);
	\end{pgfonlayer}
\end{tikzpicture}
\right\rrbracket_{\hat{\TOF}}
&=
\begin{tikzpicture}[tikzfig]
	\begin{pgfonlayer}{nodelayer}
		\node [style=none] (4) at (-1.5, 11.25) {};
		\node [style=X] (5) at (-2, 10.5) {};
		\node [style=X] (6) at (-1, 10.5) {};
		\node [style=Z] (7) at (0, 12) {};
		\node [style=andin] (8) at (-1.5, 11.25) {};
		\node [style=X] (9) at (-2, 13) {};
		\node [style=none] (10) at (-1.5, 13.75) {};
		\node [style=X] (11) at (-1, 13) {};
		\node [style=Z] (12) at (0, 14.5) {};
		\node [style=andin] (13) at (-1.5, 13.75) {};
		\node [style=X] (14) at (-1, 12.5) {};
		\node [style=none] (15) at (-0.5, 13.25) {};
		\node [style=X] (16) at (0, 12.5) {};
		\node [style=Z] (17) at (0.5, 14) {};
		\node [style=andin] (18) at (-0.5, 13.25) {};
		\node [style=none] (19) at (-2, 10) {};
		\node [style=none] (20) at (-1, 10) {};
		\node [style=none] (21) at (0, 10) {};
		\node [style=none] (22) at (0.5, 10) {};
		\node [style=none] (23) at (0.5, 15) {};
		\node [style=none] (24) at (0, 15) {};
		\node [style=none] (25) at (-1, 15) {};
		\node [style=none] (26) at (-2, 15) {};
	\end{pgfonlayer}
	\begin{pgfonlayer}{edgelayer}
		\draw [in=90, out=180] (7) to (4.center);
		\draw (4.center) to (5);
		\draw (4.center) to (6);
		\draw [in=90, out=180] (12) to (10.center);
		\draw (10.center) to (9);
		\draw (10.center) to (11);
		\draw [in=90, out=180] (17) to (15.center);
		\draw (15.center) to (14);
		\draw (15.center) to (16);
		\draw (19.center) to (26.center);
		\draw (25.center) to (20.center);
		\draw (24.center) to (21.center);
		\draw (22.center) to (23.center);
	\end{pgfonlayer}
\end{tikzpicture}
\eq{\ref{ZXA.3}}
\begin{tikzpicture}[tikzfig]
	\begin{pgfonlayer}{nodelayer}
		\node [style=none] (5) at (-2, 11.25) {};
		\node [style=X] (6) at (-3, 12) {};
		\node [style=X] (7) at (-1.5, 12) {};
		\node [style=Z] (8) at (0, 10.5) {};
		\node [style=andout] (9) at (-2, 11.25) {};
		\node [style=X] (10) at (-2.5, 12) {};
		\node [style=none] (11) at (-2, 12.75) {};
		\node [style=X] (12) at (-1, 12) {};
		\node [style=Z] (13) at (0, 14.5) {};
		\node [style=andin] (14) at (-2, 12.75) {};
		\node [style=X] (15) at (-1, 12.75) {};
		\node [style=none] (16) at (-0.5, 13.5) {};
		\node [style=X] (17) at (0, 12.75) {};
		\node [style=Z] (18) at (0.5, 14.25) {};
		\node [style=andin] (19) at (-0.5, 13.5) {};
		\node [style=none] (20) at (-3, 10) {};
		\node [style=none] (21) at (-1, 10) {};
		\node [style=none] (22) at (0, 10) {};
		\node [style=none] (23) at (0.5, 10) {};
		\node [style=none] (24) at (0.5, 15) {};
		\node [style=none] (25) at (0, 15) {};
		\node [style=none] (26) at (-1, 15) {};
		\node [style=none] (27) at (-3, 15) {};
	\end{pgfonlayer}
	\begin{pgfonlayer}{edgelayer}
		\draw [in=-90, out=180] (8) to (5.center);
		\draw (5.center) to (7);
		\draw [in=90, out=180] (13) to (11.center);
		\draw (11.center) to (10);
		\draw [in=90, out=180] (18) to (16.center);
		\draw (16.center) to (15);
		\draw (16.center) to (17);
		\draw (20.center) to (27.center);
		\draw (26.center) to (21.center);
		\draw (25.center) to (22.center);
		\draw (23.center) to (24.center);
		\draw (7) to (11.center);
		\draw (7) to (12);
		\draw (5.center) to (10);
		\draw (10) to (6);
	\end{pgfonlayer}
\end{tikzpicture}\\
&=
\begin{tikzpicture}[tikzfig]
	\begin{pgfonlayer}{nodelayer}
		\node [style=none] (6) at (-1.5, 10.75) {};
		\node [style=X] (7) at (-3.5, 12.75) {};
		\node [style=X] (8) at (-1.5, 12) {};
		\node [style=Z] (9) at (0, 10) {};
		\node [style=andout] (10) at (-1.5, 10.75) {};
		\node [style=X] (11) at (-2.5, 12) {};
		\node [style=none] (12) at (-2.5, 10.75) {};
		\node [style=X] (13) at (-1, 12.75) {};
		\node [style=Z] (14) at (0, 15.25) {};
		\node [style=X] (15) at (-1, 13.25) {};
		\node [style=none] (16) at (-0.5, 14) {};
		\node [style=X] (17) at (0, 13.25) {};
		\node [style=Z] (18) at (0.5, 14.75) {};
		\node [style=andin] (19) at (-0.5, 14) {};
		\node [style=none] (20) at (-3.5, 9.5) {};
		\node [style=none] (21) at (-1, 9.5) {};
		\node [style=none] (22) at (0, 9.5) {};
		\node [style=none] (23) at (0.5, 9.5) {};
		\node [style=none] (24) at (0.5, 15.75) {};
		\node [style=none] (25) at (0, 15.75) {};
		\node [style=none] (26) at (-1, 15.75) {};
		\node [style=none] (27) at (-3.5, 15.75) {};
		\node [style=none] (28) at (-3, 10.75) {};
		\node [style=none] (29) at (-3, 14.75) {};
		\node [style=andout] (30) at (-2.5, 10.75) {};
	\end{pgfonlayer}
	\begin{pgfonlayer}{edgelayer}
		\draw [in=-90, out=180] (9) to (6.center);
		\draw [in=-60, out=60] (6.center) to (8);
		\draw [in=-120, out=120] (12.center) to (11);
		\draw [in=90, out=180] (18) to (16.center);
		\draw (16.center) to (15);
		\draw (16.center) to (17);
		\draw (20.center) to (27.center);
		\draw (26.center) to (21.center);
		\draw (25.center) to (22.center);
		\draw (23.center) to (24.center);
		\draw (8) to (12.center);
		\draw [in=180, out=90] (8) to (13);
		\draw (6.center) to (11);
		\draw [in=0, out=90, looseness=1.25] (11) to (7);
		\draw [in=-90, out=-90, looseness=3.50] (12.center) to (28.center);
		\draw (28.center) to (29.center);
		\draw [in=90, out=180, looseness=0.50] (14) to (29.center);
	\end{pgfonlayer}
\end{tikzpicture}
\eq{\ref{ZXA.12}}
\begin{tikzpicture}[tikzfig]
	\begin{pgfonlayer}{nodelayer}
		\node [style=none] (7) at (-0.25, 4.25) {};
		\node [style=none] (8) at (-0.75, 0.75) {};
		\node [style=X] (9) at (-0.75, 3) {};
		\node [style=none] (10) at (-2.75, 6) {};
		\node [style=none] (11) at (-2.25, 1.75) {};
		\node [style=none] (12) at (-2.25, 5) {};
		\node [style=none] (13) at (-2.75, 0.75) {};
		\node [style=none] (14) at (0.75, 0.75) {};
		\node [style=none] (15) at (0.25, 0.75) {};
		\node [style=X] (16) at (-0.75, 3.5) {};
		\node [style=andin] (17) at (-0.25, 4.25) {};
		\node [style=none] (18) at (-0.75, 6) {};
		\node [style=Z] (19) at (0.75, 5) {};
		\node [style=none] (20) at (0.25, 6) {};
		\node [style=Z] (21) at (0.25, 5.5) {};
		\node [style=Z] (22) at (0.25, 1.25) {};
		\node [style=X] (23) at (-2.75, 3.5) {};
		\node [style=X] (24) at (0.25, 3.5) {};
		\node [style=none] (25) at (0.75, 6) {};
		\node [style=none] (26) at (-1.5, 2.5) {};
		\node [style=X] (27) at (-1.5, 1.75) {};
		\node [style=andout] (28) at (-1.5, 2.5) {};
	\end{pgfonlayer}
	\begin{pgfonlayer}{edgelayer}
		\draw [in=90, out=180] (19) to (7.center);
		\draw (7.center) to (16);
		\draw (7.center) to (24);
		\draw (13.center) to (10.center);
		\draw (18.center) to (8.center);
		\draw (20.center) to (15.center);
		\draw (14.center) to (25.center);
		\draw (11.center) to (12.center);
		\draw [in=90, out=180, looseness=0.50] (21) to (12.center);
		\draw (26.center) to (9);
		\draw (26.center) to (23);
		\draw [in=180, out=-60] (27) to (22);
		\draw [in=-90, out=-135, looseness=1.25] (27) to (11.center);
		\draw (26.center) to (27);
	\end{pgfonlayer}
\end{tikzpicture}\\
&\eq{\ref{ZXA.5}}
\begin{tikzpicture}[tikzfig]
	\begin{pgfonlayer}{nodelayer}
		\node [style=Z] (8) at (1, 13.25) {};
		\node [style=none] (9) at (-2.75, 8.5) {};
		\node [style=none] (10) at (0.5, 8.5) {};
		\node [style=none] (11) at (-1.5, 10.75) {};
		\node [style=X] (12) at (-2.75, 11.75) {};
		\node [style=X] (13) at (-0.75, 11.75) {};
		\node [style=none] (14) at (-0.75, 8.5) {};
		\node [style=X] (15) at (-0.75, 11.25) {};
		\node [style=none] (16) at (1, 14.25) {};
		\node [style=none] (17) at (-2.25, 13.25) {};
		\node [style=Z] (18) at (0.5, 13.75) {};
		\node [style=andin] (19) at (-0.25, 12.5) {};
		\node [style=none] (20) at (1, 8.5) {};
		\node [style=andout] (21) at (-1.5, 10.75) {};
		\node [style=X] (22) at (-1.5, 9.5) {};
		\node [style=none] (23) at (-0.25, 12.5) {};
		\node [style=none] (24) at (-2.25, 9.5) {};
		\node [style=none] (25) at (0.5, 14.25) {};
		\node [style=none] (26) at (-2.75, 14.25) {};
		\node [style=none] (27) at (-0.75, 14.25) {};
		\node [style=X] (28) at (0, 9.75) {};
		\node [style=X] (29) at (0.5, 9.75) {};
		\node [style=Z] (30) at (0.5, 10.5) {};
		\node [style=Z] (31) at (0, 10.5) {};
	\end{pgfonlayer}
	\begin{pgfonlayer}{edgelayer}
		\draw [in=90, out=180] (8) to (23.center);
		\draw (23.center) to (13);
		\draw (9.center) to (26.center);
		\draw (27.center) to (14.center);
		\draw (20.center) to (16.center);
		\draw (24.center) to (17.center);
		\draw [in=90, out=180, looseness=0.50] (18) to (17.center);
		\draw (11.center) to (15);
		\draw (11.center) to (12);
		\draw [in=-90, out=-135, looseness=1.25] (22) to (24.center);
		\draw (11.center) to (22);
		\draw [in=-60, out=-90, looseness=1.25] (28) to (22);
		\draw (29) to (10.center);
		\draw (29) to (31);
		\draw (30) to (28);
		\draw [bend right, looseness=1.25] (31) to (28);
		\draw [bend right, looseness=1.25] (29) to (30);
		\draw (31) to (23.center);
		\draw (30) to (18);
		\draw (25.center) to (18);
	\end{pgfonlayer}
\end{tikzpicture}
\eq{\ref{ZXA.1},\ref{ZXA.2}}
\begin{tikzpicture}[tikzfig]
	\begin{pgfonlayer}{nodelayer}
		\node [style=Z] (9) at (1, 12.25) {};
		\node [style=none] (10) at (-2.75, 8.75) {};
		\node [style=none] (11) at (0.5, 8.75) {};
		\node [style=none] (12) at (-1.5, 10) {};
		\node [style=X] (13) at (-2.75, 11) {};
		\node [style=X] (14) at (-0.75, 11) {};
		\node [style=none] (15) at (-0.75, 8.75) {};
		\node [style=X] (16) at (-0.75, 10.5) {};
		\node [style=none] (17) at (1, 12.75) {};
		\node [style=andin] (18) at (-0.25, 11.5) {};
		\node [style=none] (19) at (1, 8.75) {};
		\node [style=andout] (20) at (-1.5, 10) {};
		\node [style=none] (21) at (-0.25, 11.5) {};
		\node [style=none] (22) at (0.5, 12.75) {};
		\node [style=none] (23) at (-2.75, 12.75) {};
		\node [style=none] (24) at (-0.75, 12.75) {};
		\node [style=X] (25) at (0, 9.75) {};
		\node [style=X] (26) at (0.5, 9.75) {};
		\node [style=Z] (27) at (0.5, 10.5) {};
		\node [style=Z] (28) at (0, 10.5) {};
		\node [style=none] (29) at (-1.5, 9.75) {};
	\end{pgfonlayer}
	\begin{pgfonlayer}{edgelayer}
		\draw [in=90, out=180] (9) to (21.center);
		\draw (21.center) to (14);
		\draw (10.center) to (23.center);
		\draw (24.center) to (15.center);
		\draw (19.center) to (17.center);
		\draw (12.center) to (16);
		\draw (12.center) to (13);
		\draw (26) to (11.center);
		\draw (26) to (28);
		\draw [bend right=15, looseness=1.25] (27) to (25);
		\draw [bend right, looseness=1.25] (28) to (25);
		\draw [bend right, looseness=1.25] (26) to (27);
		\draw (28) to (21.center);
		\draw (22.center) to (27);
		\draw [bend right=15] (25) to (27);
		\draw [in=-90, out=-90, looseness=1.25] (25) to (29.center);
		\draw (29.center) to (12.center);
	\end{pgfonlayer}
\end{tikzpicture}\\
&\eq{\ref{ZXA.8}}
\begin{tikzpicture}[tikzfig]
	\begin{pgfonlayer}{nodelayer}
		\node [style=Z] (10) at (1, 12.75) {};
		\node [style=none] (11) at (-2.25, 8.75) {};
		\node [style=none] (12) at (0.5, 8.75) {};
		\node [style=none] (13) at (-1.5, 10.5) {};
		\node [style=X] (14) at (-2.25, 11) {};
		\node [style=X] (15) at (-0.75, 11.5) {};
		\node [style=none] (16) at (-0.75, 8.75) {};
		\node [style=X] (17) at (-0.75, 11) {};
		\node [style=none] (18) at (1, 13.5) {};
		\node [style=andin] (19) at (-0.25, 12.25) {};
		\node [style=none] (20) at (1, 8.75) {};
		\node [style=andout] (21) at (-1.5, 10.5) {};
		\node [style=none] (22) at (-0.25, 12.25) {};
		\node [style=none] (23) at (0.5, 13.5) {};
		\node [style=none] (24) at (-2.25, 13.5) {};
		\node [style=none] (25) at (-0.75, 13.5) {};
		\node [style=X] (26) at (0.5, 9.5) {};
		\node [style=Z] (27) at (0, 10.25) {};
		\node [style=none] (28) at (-1.5, 10.25) {};
	\end{pgfonlayer}
	\begin{pgfonlayer}{edgelayer}
		\draw [in=90, out=180] (10) to (22.center);
		\draw (22.center) to (15);
		\draw (11.center) to (24.center);
		\draw (25.center) to (16.center);
		\draw (20.center) to (18.center);
		\draw (13.center) to (17);
		\draw (13.center) to (14);
		\draw (26) to (12.center);
		\draw (26) to (27);
		\draw [in=-60, out=90, looseness=0.75] (27) to (22.center);
		\draw (28.center) to (13.center);
		\draw (23.center) to (26);
		\draw [in=-90, out=-120] (27) to (28.center);
	\end{pgfonlayer}
\end{tikzpicture}
\eq{\ref{ZXA.17}}
\begin{tikzpicture}[tikzfig]
	\begin{pgfonlayer}{nodelayer}
		\node [style=andin] (11) at (0, 5.5) {};
		\node [style=none] (12) at (-1.5, 1.5) {};
		\node [style=none] (13) at (1, 7.25) {};
		\node [style=andout] (14) at (-2.25, 2.75) {};
		\node [style=none] (15) at (-2.25, 2.5) {};
		\node [style=X] (16) at (-1.5, 3.5) {};
		\node [style=X] (17) at (-1.5, 4) {};
		\node [style=X] (18) at (-3, 3.5) {};
		\node [style=none] (19) at (-1.5, 7.25) {};
		\node [style=none] (20) at (-3, 1.5) {};
		\node [style=none] (21) at (-2.25, 2.75) {};
		\node [style=none] (22) at (-3, 7.25) {};
		\node [style=none] (23) at (1, 1.5) {};
		\node [style=X] (24) at (0.5, 4.5) {};
		\node [style=none] (25) at (0.5, 1.5) {};
		\node [style=Z] (26) at (1, 6.75) {};
		\node [style=none] (27) at (0.5, 7.25) {};
		\node [style=none] (28) at (-1, 5.5) {};
		\node [style=none] (29) at (0, 5.5) {};
		\node [style=Z] (30) at (-0.5, 6.25) {};
		\node [style=X] (31) at (-1, 4.5) {};
		\node [style=none] (32) at (0, 4) {};
		\node [style=none] (33) at (0, 2.5) {};
		\node [style=andin] (34) at (-1, 5.5) {};
	\end{pgfonlayer}
	\begin{pgfonlayer}{edgelayer}
		\draw (20.center) to (22.center);
		\draw (19.center) to (12.center);
		\draw (23.center) to (13.center);
		\draw (21.center) to (16);
		\draw (21.center) to (18);
		\draw (24) to (25.center);
		\draw (15.center) to (21.center);
		\draw (27.center) to (24);
		\draw [in=90, out=-124] (30) to (28.center);
		\draw [in=90, out=-56] (30) to (29.center);
		\draw [in=37, out=-120] (29.center) to (31);
		\draw (31) to (28.center);
		\draw (31) to (17);
		\draw (29.center) to (24);
		\draw [in=90, out=-45, looseness=1.25] (28.center) to (32.center);
		\draw (32.center) to (33.center);
		\draw [in=-90, out=-90] (33.center) to (15.center);
		\draw [in=90, out=180, looseness=0.75] (26) to (30);
	\end{pgfonlayer}
\end{tikzpicture}\\
&=
\begin{tikzpicture}[tikzfig]
	\begin{pgfonlayer}{nodelayer}
		\node [style=andin] (12) at (0, 5.25) {};
		\node [style=none] (13) at (-1.5, 2) {};
		\node [style=none] (14) at (1, 7.25) {};
		\node [style=none] (15) at (-0.75, 4.5) {};
		\node [style=X] (16) at (-1.5, 2.5) {};
		\node [style=X] (17) at (-2.5, 2.5) {};
		\node [style=none] (18) at (-1.5, 7.25) {};
		\node [style=none] (19) at (-2.5, 2) {};
		\node [style=none] (20) at (-2, 3.25) {};
		\node [style=none] (21) at (-2.5, 7.25) {};
		\node [style=none] (22) at (1, 2) {};
		\node [style=X] (23) at (0.5, 4.25) {};
		\node [style=none] (24) at (0.5, 2) {};
		\node [style=Z] (25) at (1, 6.75) {};
		\node [style=none] (26) at (0.5, 7.25) {};
		\node [style=none] (27) at (-1, 5.25) {};
		\node [style=none] (28) at (0, 5.25) {};
		\node [style=Z] (29) at (-0.5, 6) {};
		\node [style=X] (30) at (-1.5, 4.25) {};
		\node [style=none] (31) at (-0.75, 4.5) {};
		\node [style=andin] (32) at (-1, 5.25) {};
		\node [style=andin] (33) at (-2, 3.25) {};
	\end{pgfonlayer}
	\begin{pgfonlayer}{edgelayer}
		\draw (19.center) to (21.center);
		\draw (18.center) to (13.center);
		\draw (22.center) to (14.center);
		\draw (20.center) to (16);
		\draw (20.center) to (17);
		\draw (23) to (24.center);
		\draw [in=90, out=-90] (15.center) to (20.center);
		\draw (26.center) to (23);
		\draw [in=90, out=-124] (29) to (27.center);
		\draw [in=90, out=-56] (29) to (28.center);
		\draw [in=37, out=-120] (28.center) to (30);
		\draw (30) to (27.center);
		\draw (28.center) to (23);
		\draw [in=90, out=-45, looseness=1.25] (27.center) to (31.center);
		\draw [in=90, out=180] (25) to (29);
	\end{pgfonlayer}
\end{tikzpicture}
\eq{\ref{ZXA.11}}
\begin{tikzpicture}[tikzfig]
	\begin{pgfonlayer}{nodelayer}
		\node [style=andin] (13) at (0, 5.5) {};
		\node [style=none] (14) at (-1.5, 1.25) {};
		\node [style=none] (15) at (1, 7.5) {};
		\node [style=none] (16) at (-0.5, 4.5) {};
		\node [style=none] (17) at (-1.5, 7.5) {};
		\node [style=none] (18) at (-2.5, 1.25) {};
		\node [style=none] (19) at (-2, 3.25) {};
		\node [style=none] (20) at (-2.5, 7.5) {};
		\node [style=none] (21) at (1, 1.25) {};
		\node [style=X] (22) at (0.5, 4.5) {};
		\node [style=none] (23) at (0.5, 1.25) {};
		\node [style=Z] (24) at (1, 7) {};
		\node [style=none] (25) at (0.5, 7.5) {};
		\node [style=none] (26) at (-1, 5.5) {};
		\node [style=none] (27) at (0, 5.5) {};
		\node [style=Z] (28) at (-0.5, 6.25) {};
		\node [style=X] (29) at (-1.5, 4.5) {};
		\node [style=none] (30) at (-0.5, 4.5) {};
		\node [style=andin] (31) at (-1, 5.5) {};
		\node [style=andin] (32) at (-2, 3.25) {};
		\node [style=X] (33) at (-2.5, 1.75) {};
		\node [style=X] (34) at (-1.5, 1.75) {};
		\node [style=none] (35) at (-2.25, 2.5) {};
		\node [style=none] (36) at (-1.75, 2.5) {};
	\end{pgfonlayer}
	\begin{pgfonlayer}{edgelayer}
		\draw (18.center) to (20.center);
		\draw (17.center) to (14.center);
		\draw (21.center) to (15.center);
		\draw (22) to (23.center);
		\draw [in=90, out=-90, looseness=1.25] (16.center) to (19.center);
		\draw (25.center) to (22);
		\draw [in=90, out=-124] (28) to (26.center);
		\draw [in=90, out=-56] (28) to (27.center);
		\draw [in=37, out=-120] (27.center) to (29);
		\draw (29) to (26.center);
		\draw (27.center) to (22);
		\draw [in=90, out=-45, looseness=1.25] (26.center) to (30.center);
		\draw [in=90, out=180] (24) to (28);
		\draw [in=90, out=-108] (19.center) to (35.center);
		\draw [in=135, out=-90] (35.center) to (34);
		\draw [in=-72, out=90] (36.center) to (19.center);
		\draw [in=45, out=-90] (36.center) to (33);
	\end{pgfonlayer}
\end{tikzpicture}\\
&\eq{\ref{ZXA.9}}
\begin{tikzpicture}[tikzfig]
	\begin{pgfonlayer}{nodelayer}
		\node [style=andin] (14) at (0, 6.25) {};
		\node [style=none] (15) at (-2, 2.5) {};
		\node [style=none] (16) at (1, 8.25) {};
		\node [style=none] (17) at (-0.5, 4.25) {};
		\node [style=X] (18) at (-2.5, 3.25) {};
		\node [style=X] (19) at (-2, 3.25) {};
		\node [style=none] (20) at (-2, 8.25) {};
		\node [style=none] (21) at (-2.5, 2.5) {};
		\node [style=none] (22) at (-2.5, 8.25) {};
		\node [style=none] (23) at (1, 2.5) {};
		\node [style=X] (24) at (0.5, 4.75) {};
		\node [style=none] (25) at (0.5, 2.5) {};
		\node [style=Z] (26) at (1, 7.75) {};
		\node [style=none] (27) at (0.5, 8.25) {};
		\node [style=none] (28) at (-1, 6.25) {};
		\node [style=none] (29) at (0, 6.25) {};
		\node [style=Z] (30) at (-0.5, 7) {};
		\node [style=X] (31) at (-2, 4.25) {};
		\node [style=none] (32) at (-0.5, 4.25) {};
		\node [style=none] (33) at (-1.5, 5.5) {};
		\node [style=andin] (34) at (-1, 6.25) {};
		\node [style=andin] (35) at (-1.5, 5.5) {};
		\node [style=none] (36) at (-1.25, 4.25) {};
	\end{pgfonlayer}
	\begin{pgfonlayer}{edgelayer}
		\draw (21.center) to (22.center);
		\draw (20.center) to (15.center);
		\draw (23.center) to (16.center);
		\draw (24) to (25.center);
		\draw (27.center) to (24);
		\draw [in=90, out=-124] (30) to (28.center);
		\draw [in=90, out=-56] (30) to (29.center);
		\draw [in=37, out=-120] (29.center) to (31);
		\draw (29.center) to (24);
		\draw [in=90, out=-45] (28.center) to (32.center);
		\draw [in=90, out=180] (26) to (30);
		\draw [in=90, out=-124] (28.center) to (33.center);
		\draw (33.center) to (31);
		\draw [in=53, out=-90, looseness=0.75] (17.center) to (18);
		\draw [in=90, out=-60] (33.center) to (36.center);
		\draw [in=49, out=-90, looseness=0.75] (36.center) to (19);
	\end{pgfonlayer}
\end{tikzpicture}
\eq{\ref{ZXA.3}}
\begin{tikzpicture}[tikzfig]
	\begin{pgfonlayer}{nodelayer}
		\node [style=andin] (15) at (0, 5.75) {};
		\node [style=none] (16) at (-2, 2.5) {};
		\node [style=none] (17) at (1, 7.5) {};
		\node [style=none] (18) at (-0.5, 4.25) {};
		\node [style=X] (19) at (-2.5, 3.25) {};
		\node [style=X] (20) at (-2, 4.25) {};
		\node [style=none] (21) at (-2, 7.5) {};
		\node [style=none] (22) at (-2.5, 2.5) {};
		\node [style=none] (23) at (-2.5, 7.5) {};
		\node [style=none] (24) at (1, 2.5) {};
		\node [style=X] (25) at (0.5, 4.75) {};
		\node [style=none] (26) at (0.5, 2.5) {};
		\node [style=Z] (27) at (1, 7) {};
		\node [style=none] (28) at (0.5, 7.5) {};
		\node [style=none] (29) at (-1, 5.75) {};
		\node [style=none] (30) at (0, 5.75) {};
		\node [style=Z] (31) at (-0.5, 6.5) {};
		\node [style=X] (32) at (-2, 4.25) {};
		\node [style=none] (33) at (-0.5, 4.25) {};
		\node [style=none] (34) at (-1.5, 5) {};
		\node [style=andin] (35) at (-1, 5.75) {};
		\node [style=andin] (36) at (-1.5, 5) {};
	\end{pgfonlayer}
	\begin{pgfonlayer}{edgelayer}
		\draw (22.center) to (23.center);
		\draw (21.center) to (16.center);
		\draw (24.center) to (17.center);
		\draw (25) to (26.center);
		\draw (28.center) to (25);
		\draw [in=90, out=-124] (31) to (29.center);
		\draw [in=90, out=-56] (31) to (30.center);
		\draw [in=0, out=-120, looseness=0.75] (30.center) to (32);
		\draw (30.center) to (25);
		\draw [in=90, out=-45] (29.center) to (33.center);
		\draw [in=90, out=180, looseness=0.75] (27) to (31);
		\draw [in=90, out=-124] (29.center) to (34.center);
		\draw [in=60, out=-142, looseness=0.75] (34.center) to (32);
		\draw [in=53, out=-90, looseness=0.75] (18.center) to (19);
		\draw [bend left=45] (34.center) to (20);
	\end{pgfonlayer}
\end{tikzpicture}\\
&\eq{\ref{ZXA.15}}
\begin{tikzpicture}[tikzfig]
	\begin{pgfonlayer}{nodelayer}
		\node [style=andin] (16) at (0, 6.25) {};
		\node [style=none] (17) at (-1.5, 2.5) {};
		\node [style=none] (18) at (1, 8.25) {};
		\node [style=none] (19) at (-0.5, 4.25) {};
		\node [style=X] (20) at (-2, 3.25) {};
		\node [style=X] (21) at (-1.5, 4.25) {};
		\node [style=none] (22) at (-1.5, 8.25) {};
		\node [style=none] (23) at (-2, 2.5) {};
		\node [style=none] (24) at (-2, 8.25) {};
		\node [style=none] (25) at (1, 2.5) {};
		\node [style=X] (26) at (0.5, 4.75) {};
		\node [style=none] (27) at (0.5, 2.5) {};
		\node [style=Z] (28) at (1, 7.75) {};
		\node [style=none] (29) at (0.5, 8.25) {};
		\node [style=none] (30) at (-1, 6.25) {};
		\node [style=none] (31) at (0, 6.25) {};
		\node [style=Z] (32) at (-0.5, 7) {};
		\node [style=X] (33) at (-1.5, 4.25) {};
		\node [style=none] (34) at (-0.5, 4.25) {};
		\node [style=andin] (35) at (-1, 6.25) {};
	\end{pgfonlayer}
	\begin{pgfonlayer}{edgelayer}
		\draw (23.center) to (24.center);
		\draw (22.center) to (17.center);
		\draw (25.center) to (18.center);
		\draw (26) to (27.center);
		\draw (29.center) to (26);
		\draw [in=90, out=-124] (32) to (30.center);
		\draw [in=90, out=-56] (32) to (31.center);
		\draw [in=0, out=-120, looseness=0.75] (31.center) to (33);
		\draw (31.center) to (26);
		\draw [in=90, out=-45] (30.center) to (34.center);
		\draw [in=90, out=180] (28) to (32);
		\draw [in=53, out=-90, looseness=0.75] (19.center) to (20);
		\draw [in=75, out=-105] (30.center) to (21);
	\end{pgfonlayer}
\end{tikzpicture}
\eq{\ref{ZXA.3}}
\begin{tikzpicture}[tikzfig]
	\begin{pgfonlayer}{nodelayer}
		\node [style=none] (17) at (-1.5, 3.5) {};
		\node [style=none] (18) at (1.25, 8.25) {};
		\node [style=none] (19) at (-0.5, 6.25) {};
		\node [style=X] (20) at (-2, 5.75) {};
		\node [style=X] (21) at (-1.5, 5) {};
		\node [style=none] (22) at (-1.5, 8.25) {};
		\node [style=none] (23) at (-2, 3.5) {};
		\node [style=none] (24) at (-2, 8.25) {};
		\node [style=none] (25) at (1.25, 3.5) {};
		\node [style=X] (26) at (0.75, 5.25) {};
		\node [style=none] (27) at (0.75, 3.5) {};
		\node [style=Z] (28) at (1.25, 7.75) {};
		\node [style=none] (29) at (0.75, 8.25) {};
		\node [style=none] (30) at (-1, 7) {};
		\node [style=none] (31) at (0.25, 6.25) {};
		\node [style=X] (32) at (-1.5, 4.25) {};
		\node [style=none] (33) at (-0.5, 6.25) {};
		\node [style=andin] (34) at (-1, 7) {};
		\node [style=Z] (35) at (1.25, 7) {};
		\node [style=andin] (36) at (0.25, 6.25) {};
	\end{pgfonlayer}
	\begin{pgfonlayer}{edgelayer}
		\draw (23.center) to (24.center);
		\draw (22.center) to (17.center);
		\draw (25.center) to (18.center);
		\draw (26) to (27.center);
		\draw (29.center) to (26);
		\draw [in=0, out=-120, looseness=0.75] (31.center) to (32);
		\draw (31.center) to (26);
		\draw [in=90, out=-45] (30.center) to (33.center);
		\draw [in=53, out=-90, looseness=0.75] (19.center) to (20);
		\draw [in=60, out=-120, looseness=1.25] (30.center) to (21);
		\draw [in=90, out=180, looseness=0.75] (28) to (30.center);
		\draw [in=90, out=180] (35) to (31.center);
	\end{pgfonlayer}
\end{tikzpicture}\\
&\eq{\ref{ZXA.11}}
\begin{tikzpicture}[tikzfig]
	\begin{pgfonlayer}{nodelayer}
		\node [style=none] (18) at (-0.25, 4.75) {};
		\node [style=none] (19) at (1.25, 8.25) {};
		\node [style=X] (20) at (-0.25, 6) {};
		\node [style=X] (21) at (-1.25, 6) {};
		\node [style=none] (22) at (-0.25, 8.25) {};
		\node [style=none] (23) at (-1.25, 4.75) {};
		\node [style=none] (24) at (-1.25, 8.25) {};
		\node [style=none] (25) at (1.25, 4.75) {};
		\node [style=X] (26) at (0.75, 5.25) {};
		\node [style=none] (27) at (0.75, 4.75) {};
		\node [style=Z] (28) at (1.25, 7.75) {};
		\node [style=none] (29) at (0.75, 8.25) {};
		\node [style=none] (30) at (-0.75, 7) {};
		\node [style=none] (31) at (0.25, 6.25) {};
		\node [style=X] (32) at (-0.25, 5.25) {};
		\node [style=andin] (33) at (-0.75, 7) {};
		\node [style=Z] (34) at (1.25, 7) {};
		\node [style=andin] (35) at (0.25, 6.25) {};
	\end{pgfonlayer}
	\begin{pgfonlayer}{edgelayer}
		\draw (23.center) to (24.center);
		\draw (22.center) to (18.center);
		\draw (25.center) to (19.center);
		\draw (26) to (27.center);
		\draw (29.center) to (26);
		\draw (31.center) to (32);
		\draw (31.center) to (26);
		\draw [in=90, out=180, looseness=0.75] (28) to (30.center);
		\draw [in=90, out=180] (34) to (31.center);
		\draw (30.center) to (21);
		\draw (20) to (30.center);
	\end{pgfonlayer}
\end{tikzpicture}
=
\left\llbracket
\begin{tikzpicture}[tikzfig]
	\begin{pgfonlayer}{nodelayer}
		\node [style=nothing] (19) at (-0.5, 0.5) {};
		\node [style=nothing] (20) at (0, 0.5) {};
		\node [style=nothing] (21) at (-1, 0.5) {};
		\node [style=nothing] (22) at (-1.5, 0.5) {};
		\node [style=nothing] (23) at (-0.5, 2) {};
		\node [style=nothing] (24) at (-1.5, 2) {};
		\node [style=nothing] (25) at (0, 2) {};
		\node [style=nothing] (26) at (-1, 2) {};
		\node [style=dot] (27) at (-1, 1.5) {};
		\node [style=dot] (28) at (-0.5, 1.5) {};
		\node [style=dot] (29) at (-1.5, 1) {};
		\node [style=dot] (30) at (-1, 1) {};
		\node [style=oplus] (31) at (0, 1) {};
		\node [style=oplus] (32) at (0, 1.5) {};
	\end{pgfonlayer}
	\begin{pgfonlayer}{edgelayer}
		\draw (27) to (28);
		\draw (22) to (29);
		\draw (29) to (24);
		\draw (21) to (30);
		\draw (30) to (27);
		\draw (27) to (26);
		\draw (19) to (28);
		\draw (28) to (23);
		\draw (20) to (31);
		\draw (31) to (32);
		\draw (32) to (25);
		\draw (32) to (28);
		\draw (31) to (30);
		\draw (30) to (29);
	\end{pgfonlayer}
\end{tikzpicture}
\right\rrbracket_{\hat{\TOF}}
\end{align*}
\endgroup


\item[\ref{TOF.13}:]
\begingroup
\allowdisplaybreaks
\begin{align*}
\left\llbracket
\begin{tikzpicture}[tikzfig]
	\begin{pgfonlayer}{nodelayer}
		\node [style=nothing] (21) at (0, 0.5) {};
		\node [style=nothing] (22) at (-1, 0.5) {};
		\node [style=nothing] (23) at (-0.5, 0.5) {};
		\node [style=nothing] (24) at (-1.5, 0.5) {};
		\node [style=nothing] (25) at (0, 2.5) {};
		\node [style=dot] (26) at (-1.5, 1) {};
		\node [style=dot] (27) at (-1, 1) {};
		\node [style=dot] (28) at (-0.5, 1.5) {};
		\node [style=oplus] (29) at (-0.5, 1) {};
		\node [style=oplus] (30) at (0, 1.5) {};
		\node [style=nothing] (31) at (-0.5, 2.5) {};
		\node [style=nothing] (32) at (-1.5, 2.5) {};
		\node [style=nothing] (33) at (-1, 2.5) {};
		\node [style=oplus] (34) at (-0.5, 2) {};
		\node [style=dot] (35) at (-1, 2) {};
		\node [style=dot] (36) at (-1.5, 2) {};
	\end{pgfonlayer}
	\begin{pgfonlayer}{edgelayer}
		\draw (26) to (24);
		\draw (27) to (22);
		\draw (23) to (29);
		\draw (29) to (28);
		\draw (25) to (30);
		\draw (30) to (21);
		\draw (29) to (27);
		\draw (27) to (26);
		\draw (30) to (28);
		\draw (26) to (36);
		\draw (36) to (32);
		\draw (33) to (35);
		\draw (35) to (27);
		\draw (28) to (34);
		\draw (34) to (31);
		\draw (34) to (35);
		\draw (35) to (36);
	\end{pgfonlayer}
\end{tikzpicture}
\right\rrbracket_{\hat{\TOF}}
&=
\begin{tikzpicture}[tikzfig]
	\begin{pgfonlayer}{nodelayer}
		\node [style=none] (22) at (-2, 7) {};
		\node [style=none] (23) at (-1, 7) {};
		\node [style=none] (24) at (-2, 12.25) {};
		\node [style=none] (25) at (-1, 12.25) {};
		\node [style=X] (26) at (-2, 8) {};
		\node [style=X] (27) at (-1, 8) {};
		\node [style=none] (28) at (-1.5, 9) {};
		\node [style=Z] (29) at (-0.5, 9.5) {};
		\node [style=andin] (30) at (-1.5, 9) {};
		\node [style=andin] (31) at (-1.5, 11) {};
		\node [style=X] (32) at (-1, 10) {};
		\node [style=Z] (33) at (-0.5, 11.5) {};
		\node [style=X] (34) at (-2, 10) {};
		\node [style=none] (35) at (-1.5, 11) {};
		\node [style=X] (36) at (-0.5, 10.5) {};
		\node [style=Z] (37) at (0, 10.5) {};
		\node [style=none] (38) at (-0.5, 12.25) {};
		\node [style=none] (39) at (-0.5, 7) {};
		\node [style=none] (40) at (0, 12.25) {};
		\node [style=none] (41) at (0, 7) {};
	\end{pgfonlayer}
	\begin{pgfonlayer}{edgelayer}
		\draw [in=90, out=180] (29) to (28.center);
		\draw (28.center) to (26);
		\draw (27) to (28.center);
		\draw [in=90, out=180] (33) to (35.center);
		\draw (35.center) to (34);
		\draw (32) to (35.center);
		\draw (37) to (36);
		\draw (40.center) to (41.center);
		\draw (39.center) to (38.center);
		\draw (25.center) to (23.center);
		\draw (22.center) to (24.center);
	\end{pgfonlayer}
\end{tikzpicture}
\eq{\ref{ZXA.3}}
\begin{tikzpicture}[tikzfig]
	\begin{pgfonlayer}{nodelayer}
		\node [style=none] (23) at (-2, 8) {};
		\node [style=none] (24) at (-1, 8) {};
		\node [style=none] (25) at (-2, 12) {};
		\node [style=none] (26) at (-1, 12) {};
		\node [style=andin] (27) at (-1.5, 11) {};
		\node [style=X] (28) at (-1, 10) {};
		\node [style=Z] (29) at (-0.5, 11.5) {};
		\node [style=X] (30) at (-2, 10) {};
		\node [style=none] (31) at (-1.5, 11) {};
		\node [style=X] (32) at (-0.5, 10) {};
		\node [style=Z] (33) at (0, 10) {};
		\node [style=none] (34) at (-0.5, 12) {};
		\node [style=none] (35) at (-0.5, 8) {};
		\node [style=none] (36) at (0, 12) {};
		\node [style=none] (37) at (0, 8) {};
		\node [style=andout] (38) at (-1.5, 9) {};
		\node [style=X] (39) at (-1, 10) {};
		\node [style=Z] (40) at (-0.5, 8.5) {};
		\node [style=X] (41) at (-2, 10) {};
		\node [style=none] (42) at (-1.5, 9) {};
	\end{pgfonlayer}
	\begin{pgfonlayer}{edgelayer}
		\draw [in=90, out=180] (29) to (31.center);
		\draw (31.center) to (30);
		\draw (28) to (31.center);
		\draw (33) to (32);
		\draw (36.center) to (37.center);
		\draw (35.center) to (34.center);
		\draw (26.center) to (24.center);
		\draw (23.center) to (25.center);
		\draw [in=-90, out=180] (40) to (42.center);
		\draw (42.center) to (41);
		\draw (39) to (42.center);
	\end{pgfonlayer}
\end{tikzpicture}\\
&\eq{\ref{ZXA.3}}
\begin{tikzpicture}[tikzfig]
	\begin{pgfonlayer}{nodelayer}
		\node [style=none] (24) at (-3, 9.5) {};
		\node [style=none] (25) at (-1, 9.5) {};
		\node [style=none] (26) at (-3, 13.5) {};
		\node [style=none] (27) at (-1, 13.5) {};
		\node [style=andin] (28) at (-2, 12.5) {};
		\node [style=X] (29) at (-1.5, 11.5) {};
		\node [style=Z] (30) at (-0.5, 13) {};
		\node [style=X] (31) at (-2.5, 11.5) {};
		\node [style=none] (32) at (-2, 12.5) {};
		\node [style=X] (33) at (-0.5, 11.5) {};
		\node [style=Z] (34) at (0, 11.5) {};
		\node [style=none] (35) at (-0.5, 13.5) {};
		\node [style=none] (36) at (-0.5, 9.5) {};
		\node [style=none] (37) at (0, 13.5) {};
		\node [style=none] (38) at (0, 9.5) {};
		\node [style=X] (39) at (-1, 11.5) {};
		\node [style=Z] (40) at (-0.5, 10) {};
		\node [style=X] (41) at (-3, 11.5) {};
		\node [style=none] (42) at (-2, 10.5) {};
		\node [style=andout] (43) at (-2, 10.5) {};
	\end{pgfonlayer}
	\begin{pgfonlayer}{edgelayer}
		\draw [in=90, out=180] (30) to (32.center);
		\draw (32.center) to (31);
		\draw (29) to (32.center);
		\draw (34) to (33);
		\draw (37.center) to (38.center);
		\draw (36.center) to (35.center);
		\draw (27.center) to (25.center);
		\draw (24.center) to (26.center);
		\draw [in=-90, out=180] (40) to (42.center);
		\draw (31) to (41);
		\draw (31) to (42.center);
		\draw (29) to (39);
		\draw (29) to (42.center);
	\end{pgfonlayer}
\end{tikzpicture}
=
\begin{tikzpicture}[tikzfig]
	\begin{pgfonlayer}{nodelayer}
		\node [style=none] (25) at (-3.5, 9.5) {};
		\node [style=none] (26) at (-1, 9.5) {};
		\node [style=none] (27) at (-3.5, 13.5) {};
		\node [style=none] (28) at (-1, 13.5) {};
		\node [style=X] (29) at (-1.75, 11.5) {};
		\node [style=none] (30) at (-2.5, 9.75) {};
		\node [style=X] (31) at (-2.5, 11.5) {};
		\node [style=none] (32) at (-2.5, 10.5) {};
		\node [style=X] (33) at (-0.5, 11.5) {};
		\node [style=Z] (34) at (0, 11.5) {};
		\node [style=none] (35) at (-0.5, 13.5) {};
		\node [style=none] (36) at (-0.5, 9.5) {};
		\node [style=none] (37) at (0, 13.5) {};
		\node [style=none] (38) at (0, 9.5) {};
		\node [style=X] (39) at (-1, 11.5) {};
		\node [style=Z] (40) at (-0.5, 10) {};
		\node [style=X] (41) at (-3.5, 12.5) {};
		\node [style=none] (42) at (-1.75, 10.5) {};
		\node [style=andout] (43) at (-1.75, 10.5) {};
		\node [style=andout] (44) at (-2.5, 10.5) {};
		\node [style=none] (45) at (-3, 9.75) {};
		\node [style=none] (46) at (-3, 12.25) {};
		\node [style=Z] (47) at (-0.5, 12.75) {};
	\end{pgfonlayer}
	\begin{pgfonlayer}{edgelayer}
		\draw (30.center) to (32.center);
		\draw [in=-120, out=120, looseness=1.25] (32.center) to (31);
		\draw (29) to (32.center);
		\draw (34) to (33);
		\draw (37.center) to (38.center);
		\draw (36.center) to (35.center);
		\draw (28.center) to (26.center);
		\draw (25.center) to (27.center);
		\draw [in=-90, out=180] (40) to (42.center);
		\draw [in=-63, out=90] (31) to (41);
		\draw (31) to (42.center);
		\draw (29) to (39);
		\draw [in=60, out=-60, looseness=1.25] (29) to (42.center);
		\draw [in=90, out=-174, looseness=0.50] (47) to (46.center);
		\draw (46.center) to (45.center);
		\draw [in=-90, out=-90, looseness=1.50] (45.center) to (30.center);
	\end{pgfonlayer}
\end{tikzpicture}\\
&\eq{\ref{ZXA.12}}
\begin{tikzpicture}[tikzfig]
	\begin{pgfonlayer}{nodelayer}
		\node [style=none] (26) at (-3.5, 9.5) {};
		\node [style=none] (27) at (-1, 9.5) {};
		\node [style=none] (28) at (-3.5, 14.25) {};
		\node [style=none] (29) at (-1, 14.25) {};
		\node [style=X] (30) at (-0.5, 12.25) {};
		\node [style=Z] (31) at (0, 12.25) {};
		\node [style=none] (32) at (-0.5, 14.25) {};
		\node [style=none] (33) at (-0.5, 9.5) {};
		\node [style=none] (34) at (0, 14.25) {};
		\node [style=none] (35) at (0, 9.5) {};
		\node [style=X] (36) at (-1, 12.25) {};
		\node [style=Z] (37) at (-0.5, 10) {};
		\node [style=X] (38) at (-3.5, 13.25) {};
		\node [style=none] (39) at (-2.75, 10.5) {};
		\node [style=none] (40) at (-2.75, 13) {};
		\node [style=Z] (41) at (-0.5, 13.5) {};
		\node [style=none] (42) at (-2, 11.5) {};
		\node [style=andout] (43) at (-2, 11.5) {};
		\node [style=X] (44) at (-2, 10.5) {};
	\end{pgfonlayer}
	\begin{pgfonlayer}{edgelayer}
		\draw (31) to (30);
		\draw (34.center) to (35.center);
		\draw (33.center) to (32.center);
		\draw (29.center) to (27.center);
		\draw (26.center) to (28.center);
		\draw [in=90, out=-174, looseness=0.50] (41) to (40.center);
		\draw (40.center) to (39.center);
		\draw [in=60, out=-143] (36) to (42.center);
		\draw [in=-49, out=120] (42.center) to (38);
		\draw (42.center) to (44);
		\draw [in=180, out=-60, looseness=1.25] (44) to (37);
		\draw [in=-90, out=-120, looseness=2.00] (44) to (39.center);
	\end{pgfonlayer}
\end{tikzpicture}
\eq{\ref{ZXA.5}}
\begin{tikzpicture}[tikzfig]
	\begin{pgfonlayer}{nodelayer}
		\node [style=Z] (27) at (0, 13) {};
		\node [style=none] (28) at (-2, 11.5) {};
		\node [style=Z] (29) at (0.5, 12.25) {};
		\node [style=none] (30) at (0.5, 13.75) {};
		\node [style=X] (31) at (-1, 12.25) {};
		\node [style=none] (32) at (-3.25, 13.75) {};
		\node [style=none] (33) at (-1, 13.75) {};
		\node [style=none] (34) at (0, 13.75) {};
		\node [style=none] (35) at (-2.75, 12.5) {};
		\node [style=X] (36) at (-3.25, 13) {};
		\node [style=X] (37) at (-2, 10.5) {};
		\node [style=none] (38) at (-1, 9.5) {};
		\node [style=andout] (39) at (-2, 11.5) {};
		\node [style=none] (40) at (0.5, 9.5) {};
		\node [style=none] (41) at (-3.25, 9.5) {};
		\node [style=none] (42) at (-2.75, 10.5) {};
		\node [style=none] (43) at (0, 9.5) {};
		\node [style=X] (44) at (-0.5, 10.5) {};
		\node [style=X] (45) at (0, 10.5) {};
		\node [style=Z] (46) at (0, 11.5) {};
		\node [style=Z] (47) at (-0.5, 11.5) {};
	\end{pgfonlayer}
	\begin{pgfonlayer}{edgelayer}
		\draw (30.center) to (40.center);
		\draw (33.center) to (38.center);
		\draw (41.center) to (32.center);
		\draw [in=90, out=-174, looseness=0.50] (27) to (35.center);
		\draw (35.center) to (42.center);
		\draw [in=60, out=-143] (31) to (28.center);
		\draw [in=-49, out=120] (28.center) to (36);
		\draw (28.center) to (37);
		\draw [in=-90, out=-120, looseness=2.00] (37) to (42.center);
		\draw (46) to (44);
		\draw [in=-120, out=120, looseness=1.25] (44) to (47);
		\draw (47) to (45);
		\draw [in=-60, out=60, looseness=1.25] (45) to (46);
		\draw [in=-75, out=-90, looseness=1.25] (44) to (37);
		\draw (45) to (43.center);
		\draw [in=-124, out=90] (46) to (29);
		\draw [in=90, out=-90] (27) to (47);
		\draw (27) to (34.center);
	\end{pgfonlayer}
\end{tikzpicture}\\
&\eq{\ref{ZXA.1},\ref{ZXA.3}}
\begin{tikzpicture}[tikzfig]
	\begin{pgfonlayer}{nodelayer}
		\node [style=none] (28) at (-1.5, 10.5) {};
		\node [style=Z] (29) at (0.5, 12.25) {};
		\node [style=none] (30) at (0.5, 14.25) {};
		\node [style=X] (31) at (-1, 12.25) {};
		\node [style=none] (32) at (-2, 14.25) {};
		\node [style=none] (33) at (-1, 14.25) {};
		\node [style=none] (34) at (0, 14.25) {};
		\node [style=X] (35) at (-2, 12.25) {};
		\node [style=none] (36) at (-1, 9.5) {};
		\node [style=andout] (37) at (-1.5, 10.5) {};
		\node [style=none] (38) at (0.5, 9.5) {};
		\node [style=none] (39) at (-2, 9.5) {};
		\node [style=none] (40) at (0, 9.5) {};
		\node [style=X] (41) at (-0.5, 10.5) {};
		\node [style=X] (42) at (0, 10.5) {};
		\node [style=Z] (43) at (0, 11.5) {};
		\node [style=Z] (44) at (-0.5, 11.5) {};
	\end{pgfonlayer}
	\begin{pgfonlayer}{edgelayer}
		\draw (30.center) to (38.center);
		\draw (33.center) to (36.center);
		\draw (39.center) to (32.center);
		\draw (31) to (28.center);
		\draw (28.center) to (35);
		\draw (43) to (41);
		\draw [in=-120, out=120, looseness=1.25] (41) to (44);
		\draw (44) to (42);
		\draw [in=-60, out=60, looseness=1.25] (42) to (43);
		\draw (42) to (40.center);
		\draw [in=-124, out=90] (43) to (29);
		\draw (44) to (41);
		\draw [in=-90, out=-90] (41) to (28.center);
		\draw [in=90, out=-90] (34.center) to (44);
	\end{pgfonlayer}
\end{tikzpicture}
\eq{\ref{ZXA.8}}
\begin{tikzpicture}[tikzfig]
	\begin{pgfonlayer}{nodelayer}
		\node [style=none] (29) at (-1.5, 10) {};
		\node [style=Z] (30) at (0.5, 11.75) {};
		\node [style=none] (31) at (0.5, 12.25) {};
		\node [style=X] (32) at (-1, 11) {};
		\node [style=none] (33) at (-2, 12.25) {};
		\node [style=none] (34) at (-1, 12.25) {};
		\node [style=none] (35) at (0, 12.25) {};
		\node [style=X] (36) at (-2, 11) {};
		\node [style=none] (37) at (-1, 9.5) {};
		\node [style=andout] (38) at (-1.5, 10) {};
		\node [style=none] (39) at (0.5, 9.5) {};
		\node [style=none] (40) at (-2, 9.5) {};
		\node [style=none] (41) at (0, 9.5) {};
		\node [style=none] (42) at (-0.5, 10) {};
		\node [style=X] (43) at (0, 10) {};
		\node [style=Z] (44) at (0, 11) {};
		\node [style=none] (45) at (-0.5, 11) {};
	\end{pgfonlayer}
	\begin{pgfonlayer}{edgelayer}
		\draw (31.center) to (39.center);
		\draw (34.center) to (37.center);
		\draw (40.center) to (33.center);
		\draw (32) to (29.center);
		\draw (29.center) to (36);
		\draw [in=90, out=-117] (44) to (42.center);
		\draw [in=117, out=-90] (45.center) to (43);
		\draw [in=-60, out=60, looseness=1.25] (43) to (44);
		\draw (43) to (41.center);
		\draw [in=-124, out=90] (44) to (30);
		\draw [in=-90, out=-90] (42.center) to (29.center);
		\draw [in=90, out=-90] (35.center) to (45.center);
	\end{pgfonlayer}
\end{tikzpicture}\\
&\eq{\ref{ZXA.1}}
\begin{tikzpicture}[tikzfig]
	\begin{pgfonlayer}{nodelayer}
		\node [style=none] (30) at (-1.5, 10.75) {};
		\node [style=Z] (31) at (0, 10.25) {};
		\node [style=none] (32) at (0, 12.5) {};
		\node [style=X] (33) at (-1, 11.75) {};
		\node [style=none] (34) at (-2, 12.5) {};
		\node [style=none] (35) at (-1, 12.5) {};
		\node [style=none] (36) at (-0.5, 12.5) {};
		\node [style=X] (37) at (-2, 11.75) {};
		\node [style=none] (38) at (-1, 9.5) {};
		\node [style=andout] (39) at (-1.5, 10.75) {};
		\node [style=none] (40) at (0, 9.5) {};
		\node [style=none] (41) at (-2, 9.5) {};
		\node [style=none] (42) at (-0.5, 9.5) {};
		\node [style=X] (43) at (-0.5, 11.75) {};
		\node [style=Z] (44) at (0, 11.75) {};
	\end{pgfonlayer}
	\begin{pgfonlayer}{edgelayer}
		\draw (32.center) to (40.center);
		\draw (35.center) to (38.center);
		\draw (41.center) to (34.center);
		\draw (33) to (30.center);
		\draw (30.center) to (37);
		\draw (43) to (44);
		\draw (43) to (42.center);
		\draw [in=-90, out=180, looseness=0.75] (31) to (30.center);
		\draw (36.center) to (43);
	\end{pgfonlayer}
\end{tikzpicture}
=
\left\llbracket
\begin{tikzpicture}[tikzfig]
	\begin{pgfonlayer}{nodelayer}
		\node [style=nothing] (31) at (0, 9.5) {};
		\node [style=nothing] (32) at (-1, 9.5) {};
		\node [style=nothing] (33) at (-0.5, 9.5) {};
		\node [style=nothing] (34) at (-1.5, 9.5) {};
		\node [style=dot] (35) at (-1.5, 10) {};
		\node [style=dot] (36) at (-1, 10) {};
		\node [style=oplus] (37) at (0, 10) {};
		\node [style=nothing] (38) at (-0.5, 11) {};
		\node [style=nothing] (39) at (-1, 11) {};
		\node [style=nothing] (40) at (0, 11) {};
		\node [style=nothing] (41) at (-1.5, 11) {};
		\node [style=dot] (42) at (-0.5, 10.5) {};
		\node [style=oplus] (43) at (0, 10.5) {};
	\end{pgfonlayer}
	\begin{pgfonlayer}{edgelayer}
		\draw (31) to (37);
		\draw (32) to (36);
		\draw (35) to (34);
		\draw (35) to (36);
		\draw (36) to (37);
		\draw (42) to (43);
		\draw (43) to (40);
		\draw (43) to (37);
		\draw (33) to (42);
		\draw (35) to (41);
		\draw (39) to (36);
		\draw (42) to (38);
	\end{pgfonlayer}
\end{tikzpicture}
\right\rrbracket_{\hat{\TOF}}
\end{align*}
\endgroup

\item[\ref{TOF.14}:]
\begin{align*}
\left\llbracket
\begin{tikzpicture}[tikzfig]
	\begin{pgfonlayer}{nodelayer}
		\node [style=nothing] (32) at (0, 9.5) {};
		\node [style=nothing] (33) at (-0.5, 9.5) {};
		\node [style=nothing] (34) at (-0.5, 11.5) {};
		\node [style=nothing] (35) at (0, 11.5) {};
		\node [style=oplus] (36) at (0, 10) {};
		\node [style=oplus] (37) at (0, 11) {};
		\node [style=oplus] (38) at (-0.5, 10.5) {};
		\node [style=dot] (39) at (-0.5, 11) {};
		\node [style=dot] (40) at (0, 10.5) {};
		\node [style=dot] (41) at (-0.5, 10) {};
	\end{pgfonlayer}
	\begin{pgfonlayer}{edgelayer}
		\draw (33) to (41);
		\draw (41) to (38);
		\draw (38) to (39);
		\draw (39) to (34);
		\draw (35) to (37);
		\draw (37) to (40);
		\draw (40) to (36);
		\draw (36) to (32);
		\draw (36) to (41);
		\draw (40) to (38);
		\draw (37) to (39);
	\end{pgfonlayer}
\end{tikzpicture}
\right\rrbracket_{\hat{\TOF}}
&=
\begin{tikzpicture}[tikzfig]
	\begin{pgfonlayer}{nodelayer}
		\node [style=X] (33) at (-3.25, 10) {};
		\node [style=Z] (34) at (-2.75, 10) {};
		\node [style=Z] (35) at (-3.25, 10.5) {};
		\node [style=Z] (36) at (-2.75, 11) {};
		\node [style=X] (37) at (-3.25, 11) {};
		\node [style=none] (38) at (-3.25, 11.5) {};
		\node [style=none] (39) at (-2.75, 11.5) {};
		\node [style=none] (40) at (-3.25, 9.5) {};
		\node [style=none] (41) at (-2.75, 9.5) {};
		\node [style=X] (42) at (-2.75, 10.5) {};
	\end{pgfonlayer}
	\begin{pgfonlayer}{edgelayer}
		\draw (39.center) to (36);
		\draw (36) to (42);
		\draw (42) to (34);
		\draw (34) to (41.center);
		\draw (40.center) to (33);
		\draw (33) to (34);
		\draw (42) to (35);
		\draw (35) to (33);
		\draw (35) to (37);
		\draw (37) to (36);
		\draw (37) to (38.center);
	\end{pgfonlayer}
\end{tikzpicture}
=
\begin{tikzpicture}[tikzfig]
	\begin{pgfonlayer}{nodelayer}
		\node [style=X] (34) at (-3.25, 10) {};
		\node [style=Z] (35) at (-2.75, 10) {};
		\node [style=Z] (36) at (-2.75, 11) {};
		\node [style=Z] (37) at (-2.75, 12) {};
		\node [style=X] (38) at (-3.25, 12) {};
		\node [style=none] (39) at (-3.25, 12.5) {};
		\node [style=none] (40) at (-2.75, 12.5) {};
		\node [style=none] (41) at (-3.25, 9.5) {};
		\node [style=none] (42) at (-2.75, 9.5) {};
		\node [style=X] (43) at (-3.25, 11) {};
	\end{pgfonlayer}
	\begin{pgfonlayer}{edgelayer}
		\draw (40.center) to (37);
		\draw [in=90, out=-90] (37) to (43);
		\draw [in=90, out=-90] (43) to (35);
		\draw (35) to (42.center);
		\draw (41.center) to (34);
		\draw (34) to (35);
		\draw (43) to (36);
		\draw [in=90, out=-90] (36) to (34);
		\draw [in=-90, out=90] (36) to (38);
		\draw (38) to (37);
		\draw (38) to (39.center);
	\end{pgfonlayer}
\end{tikzpicture}
\eq{\ref{ZXA.5}}
\begin{tikzpicture}[tikzfig]
	\begin{pgfonlayer}{nodelayer}
		\node [style=Z] (35) at (-2.75, 11.25) {};
		\node [style=X] (36) at (-3.25, 11.25) {};
		\node [style=none] (37) at (-3.25, 11.75) {};
		\node [style=none] (38) at (-2.75, 11.75) {};
		\node [style=none] (39) at (-3.25, 9.5) {};
		\node [style=none] (40) at (-2.75, 9.5) {};
		\node [style=Z] (41) at (-3.25, 10) {};
		\node [style=X] (42) at (-2.75, 10) {};
	\end{pgfonlayer}
	\begin{pgfonlayer}{edgelayer}
		\draw (38.center) to (35);
		\draw (36) to (35);
		\draw (36) to (37.center);
		\draw (42) to (41);
		\draw [in=-90, out=90] (41) to (35);
		\draw [in=-90, out=90] (42) to (36);
		\draw (41) to (39.center);
		\draw (40.center) to (42);
	\end{pgfonlayer}
\end{tikzpicture}
=
\begin{tikzpicture}[tikzfig]
	\begin{pgfonlayer}{nodelayer}
		\node [style=Z] (36) at (-3.25, 10.75) {};
		\node [style=X] (37) at (-2.75, 10.75) {};
		\node [style=none] (38) at (-3.25, 11.75) {};
		\node [style=none] (39) at (-2.75, 11.75) {};
		\node [style=none] (40) at (-3.25, 9.5) {};
		\node [style=none] (41) at (-2.75, 9.5) {};
		\node [style=Z] (42) at (-3.25, 10) {};
		\node [style=X] (43) at (-2.75, 10) {};
	\end{pgfonlayer}
	\begin{pgfonlayer}{edgelayer}
		\draw [in=90, out=-90] (39.center) to (36);
		\draw (37) to (36);
		\draw [in=-90, out=90] (37) to (38.center);
		\draw (43) to (42);
		\draw (42) to (36);
		\draw (43) to (37);
		\draw (42) to (40.center);
		\draw (41.center) to (43);
	\end{pgfonlayer}
\end{tikzpicture}\\
&\eq{\ref{ZXA.1},\ref{ZXA.3},\ref{ZXA.15}}
\begin{tikzpicture}[tikzfig]
	\begin{pgfonlayer}{nodelayer}
		\node [style=nothing] (37) at (0, 9.5) {};
		\node [style=nothing] (38) at (-0.5, 9.5) {};
		\node [style=nothing] (39) at (-0.5, 10.5) {};
		\node [style=nothing] (40) at (0, 10.5) {};
	\end{pgfonlayer}
	\begin{pgfonlayer}{edgelayer}
		\draw [in=-90, out=90, looseness=1.25] (38) to (40);
		\draw [in=-90, out=90, looseness=1.25] (37) to (39);
	\end{pgfonlayer}
\end{tikzpicture}
=
\left\llbracket
\begin{tikzpicture}[tikzfig]
	\begin{pgfonlayer}{nodelayer}
		\node [style=nothing] (38) at (0, 9.5) {};
		\node [style=nothing] (39) at (-0.5, 9.5) {};
		\node [style=nothing] (40) at (-0.5, 10.5) {};
		\node [style=nothing] (41) at (0, 10.5) {};
	\end{pgfonlayer}
	\begin{pgfonlayer}{edgelayer}
		\draw [in=-90, out=90, looseness=1.25] (39) to (41);
		\draw [in=-90, out=90, looseness=1.25] (38) to (40);
	\end{pgfonlayer}
\end{tikzpicture}
\right\rrbracket_{\hat{\TOF}}
\end{align*}

\item[\ref{TOF.15}:]
\begin{align*}
\left\llbracket
\begin{tikzpicture}[tikzfig]
	\begin{pgfonlayer}{nodelayer}
		\node [style=nothing] (39) at (-1.75, 9.5) {};
		\node [style=nothing] (40) at (-1.25, 9.5) {};
		\node [style=nothing] (41) at (-0.75, 9.5) {};
		\node [style=nothing] (42) at (-1.75, 11.5) {};
		\node [style=nothing] (43) at (-1.25, 11.5) {};
		\node [style=nothing] (44) at (-0.75, 11.5) {};
		\node [style=dot] (45) at (-1.75, 10.5) {};
		\node [style=dot] (46) at (-1.25, 10.5) {};
		\node [style=oplus] (47) at (-0.75, 10.5) {};
	\end{pgfonlayer}
	\begin{pgfonlayer}{edgelayer}
		\draw (39) to (45);
		\draw (45) to (42);
		\draw (43) to (46);
		\draw (46) to (40);
		\draw (41) to (47);
		\draw (47) to (44);
		\draw (47) to (46);
		\draw (46) to (45);
	\end{pgfonlayer}
\end{tikzpicture}
\right\rrbracket_{\hat{\TOF}}
&=
\begin{tikzpicture}[tikzfig]
	\begin{pgfonlayer}{nodelayer}
		\node [style=none] (40) at (-2, 9.5) {};
		\node [style=none] (41) at (-1, 9.5) {};
		\node [style=none] (42) at (-0.5, 9.5) {};
		\node [style=X] (43) at (-2, 10) {};
		\node [style=X] (44) at (-1, 10) {};
		\node [style=andin] (45) at (-1.5, 10.75) {};
		\node [style=Z] (46) at (-0.5, 11.5) {};
		\node [style=none] (47) at (-0.5, 12) {};
		\node [style=none] (48) at (-2, 12) {};
		\node [style=none] (49) at (-1, 12) {};
		\node [style=none] (50) at (-1.5, 10.75) {};
	\end{pgfonlayer}
	\begin{pgfonlayer}{edgelayer}
		\draw [in=90, out=180] (46) to (50.center);
		\draw (50.center) to (43);
		\draw (43) to (40.center);
		\draw (43) to (48.center);
		\draw (49.center) to (44);
		\draw (44) to (41.center);
		\draw (42.center) to (46);
		\draw (46) to (47.center);
		\draw (50.center) to (44);
	\end{pgfonlayer}
\end{tikzpicture}
\eq{\ref{ZXA.11}}
\begin{tikzpicture}[tikzfig]
	\begin{pgfonlayer}{nodelayer}
		\node [style=none] (41) at (-2, 9.5) {};
		\node [style=none] (42) at (-1, 9.5) {};
		\node [style=none] (43) at (-0.5, 9.5) {};
		\node [style=X] (44) at (-2, 10) {};
		\node [style=X] (45) at (-1, 10) {};
		\node [style=andin] (46) at (-1.5, 11.25) {};
		\node [style=Z] (47) at (-0.5, 12) {};
		\node [style=none] (48) at (-0.5, 12.5) {};
		\node [style=none] (49) at (-2, 12.5) {};
		\node [style=none] (50) at (-1, 12.5) {};
		\node [style=none] (51) at (-1.5, 11.25) {};
		\node [style=none] (52) at (-1.75, 10.5) {};
		\node [style=none] (53) at (-1.25, 10.5) {};
	\end{pgfonlayer}
	\begin{pgfonlayer}{edgelayer}
		\draw [in=90, out=180] (47) to (51.center);
		\draw (44) to (41.center);
		\draw (44) to (49.center);
		\draw (50.center) to (45);
		\draw (45) to (42.center);
		\draw (43.center) to (47);
		\draw (47) to (48.center);
		\draw [in=90, out=-108] (51.center) to (52.center);
		\draw [in=146, out=-90] (52.center) to (45);
		\draw [in=34, out=-90, looseness=0.75] (53.center) to (44);
		\draw [in=-72, out=90] (53.center) to (51.center);
	\end{pgfonlayer}
\end{tikzpicture}
=
\begin{tikzpicture}[tikzfig]
	\begin{pgfonlayer}{nodelayer}
		\node [style=none] (42) at (-2, 9.5) {};
		\node [style=none] (43) at (-1, 9.5) {};
		\node [style=none] (44) at (-0.5, 9.5) {};
		\node [style=X] (45) at (-1, 10.75) {};
		\node [style=X] (46) at (-2, 10.75) {};
		\node [style=andin] (47) at (-1.5, 11.5) {};
		\node [style=Z] (48) at (-0.5, 12.5) {};
		\node [style=none] (49) at (-0.5, 13) {};
		\node [style=none] (50) at (-2, 13) {};
		\node [style=none] (51) at (-1, 13) {};
		\node [style=none] (52) at (-1.5, 11.5) {};
		\node [style=none] (53) at (-2, 12) {};
		\node [style=none] (54) at (-1, 12) {};
	\end{pgfonlayer}
	\begin{pgfonlayer}{edgelayer}
		\draw [in=90, out=180] (48) to (52.center);
		\draw [in=90, out=-90] (45) to (42.center);
		\draw [in=90, out=-90] (46) to (43.center);
		\draw (44.center) to (48);
		\draw (48) to (49.center);
		\draw [in=90, out=-90] (51.center) to (53.center);
		\draw [in=-90, out=90] (54.center) to (50.center);
		\draw (54.center) to (45);
		\draw (46) to (52.center);
		\draw (52.center) to (45);
		\draw (46) to (53.center);
	\end{pgfonlayer}
\end{tikzpicture}\\
&=
\left\llbracket
\begin{tikzpicture}[tikzfig]
	\begin{pgfonlayer}{nodelayer}
		\node [style=nothing] (43) at (-1.75, 9.5) {};
		\node [style=nothing] (44) at (-1.25, 9.5) {};
		\node [style=nothing] (45) at (-0.75, 9.5) {};
		\node [style=dot] (46) at (-1.75, 10.5) {};
		\node [style=dot] (47) at (-1.25, 10.5) {};
		\node [style=oplus] (48) at (-0.75, 10.5) {};
		\node [style=nothing] (49) at (-1.75, 11.5) {};
		\node [style=nothing] (50) at (-1.25, 11.5) {};
		\node [style=nothing] (51) at (-0.75, 11.5) {};
	\end{pgfonlayer}
	\begin{pgfonlayer}{edgelayer}
		\draw [in=-90, out=90, looseness=1.25] (43) to (47);
		\draw [in=-90, out=90, looseness=1.25] (47) to (49);
		\draw [in=-90, out=90, looseness=1.25] (46) to (50);
		\draw [in=90, out=-90, looseness=1.25] (46) to (44);
		\draw (45) to (48);
		\draw (48) to (51);
		\draw (46) to (47);
		\draw (47) to (48);
	\end{pgfonlayer}
\end{tikzpicture}
\right\rrbracket_{\hat{\TOF}}
\end{align*}


\item[\ref{TOF.16}:]

\begingroup
\allowdisplaybreaks
\begin{align*}
\left\llbracket
\begin{tikzpicture}[tikzfig]
	\begin{pgfonlayer}{nodelayer}
		\node [style=nothing] (44) at (0, 9.5) {};
		\node [style=nothing] (45) at (-0.5, 9.5) {};
		\node [style=nothing] (46) at (-1.5, 9.5) {};
		\node [style=nothing] (47) at (-2, 9.5) {};
		\node [style=zeroin] (48) at (-1, 9.5) {};
		\node [style=oplus] (49) at (-1, 10) {};
		\node [style=oplus] (50) at (-1, 11) {};
		\node [style=dot] (51) at (-1, 10.5) {};
		\node [style=dot] (52) at (-0.5, 10.5) {};
		\node [style=dot] (53) at (-1.5, 10) {};
		\node [style=dot] (54) at (-2, 10) {};
		\node [style=dot] (55) at (-1.5, 11) {};
		\node [style=dot] (56) at (-2, 11) {};
		\node [style=oplus] (57) at (0, 10.5) {};
		\node [style=zeroout] (58) at (-1, 11.5) {};
		\node [style=nothing] (59) at (0, 11.5) {};
		\node [style=nothing] (60) at (-2, 11.5) {};
		\node [style=nothing] (61) at (-0.5, 11.5) {};
		\node [style=nothing] (62) at (-1.5, 11.5) {};
	\end{pgfonlayer}
	\begin{pgfonlayer}{edgelayer}
		\draw (47) to (54);
		\draw (54) to (56);
		\draw (56) to (60);
		\draw (55) to (53);
		\draw (59) to (57);
		\draw (57) to (44);
		\draw (57) to (52);
		\draw (52) to (51);
		\draw (53) to (49);
		\draw (53) to (54);
		\draw (56) to (55);
		\draw (50) to (55);
		\draw (48) to (49);
		\draw (49) to (51);
		\draw (51) to (50);
		\draw (58) to (50);
		\draw [style=simple] (61) to (52);
		\draw [style=simple] (52) to (45);
		\draw [style=simple] (46) to (53);
		\draw [style=simple] (55) to (62);
	\end{pgfonlayer}
\end{tikzpicture}
\right\rrbracket_{\hat{\TOF}}
&=
\begin{tikzpicture}[tikzfig]
	\begin{pgfonlayer}{nodelayer}
		\node [style=X] (44) at (-2, 10) {};
		\node [style=X] (45) at (-1, 10) {};
		\node [style=none] (46) at (-1.5, 10.75) {};
		\node [style=Z] (47) at (-0.5, 11.5) {};
		\node [style=Z] (48) at (-0.5, 10.75) {};
		\node [style=Z] (49) at (-0.5, 14) {};
		\node [style=X] (50) at (-1, 12) {};
		\node [style=Z] (51) at (-0.5, 13.5) {};
		\node [style=X] (52) at (-2, 12) {};
		\node [style=none] (53) at (-1.5, 12.75) {};
		\node [style=X] (54) at (0.5, 12) {};
		\node [style=Z] (55) at (1, 13.5) {};
		\node [style=X] (56) at (-0.5, 12) {};
		\node [style=none] (57) at (0, 12.75) {};
		\node [style=none] (58) at (-2, 9.5) {};
		\node [style=none] (59) at (-1, 9.5) {};
		\node [style=none] (60) at (0.5, 9.5) {};
		\node [style=none] (61) at (1, 9.5) {};
		\node [style=none] (62) at (-1, 14.5) {};
		\node [style=none] (63) at (-2, 14.5) {};
		\node [style=none] (64) at (1, 14.5) {};
		\node [style=none] (65) at (0.5, 14.5) {};
		\node [style=andin] (66) at (-1.5, 10.75) {};
		\node [style=andin] (67) at (-1.5, 12.75) {};
		\node [style=andin] (68) at (0, 12.75) {};
	\end{pgfonlayer}
	\begin{pgfonlayer}{edgelayer}
		\draw (45) to (46.center);
		\draw (46.center) to (44);
		\draw (47) to (48);
		\draw [in=90, out=180] (47) to (46.center);
		\draw (50) to (53.center);
		\draw (53.center) to (52);
		\draw (51) to (49);
		\draw [in=90, out=180] (51) to (53.center);
		\draw (54) to (57.center);
		\draw (57.center) to (56);
		\draw [in=90, out=180] (55) to (57.center);
		\draw (64.center) to (61.center);
		\draw (60.center) to (65.center);
		\draw (51) to (56);
		\draw (56) to (47);
		\draw (62.center) to (59.center);
		\draw (58.center) to (63.center);
	\end{pgfonlayer}
\end{tikzpicture}
\eq{\ref{ZXA.1}}
\begin{tikzpicture}[tikzfig]
	\begin{pgfonlayer}{nodelayer}
		\node [style=X] (45) at (-2, 11.5) {};
		\node [style=X] (46) at (-1, 11.5) {};
		\node [style=none] (47) at (-1.5, 10.75) {};
		\node [style=X] (48) at (-1, 11.5) {};
		\node [style=X] (49) at (-2, 11.5) {};
		\node [style=none] (50) at (-1.5, 12.25) {};
		\node [style=X] (51) at (0.5, 11.5) {};
		\node [style=Z] (52) at (1, 13) {};
		\node [style=X] (53) at (-0.5, 11.5) {};
		\node [style=none] (54) at (0, 12.25) {};
		\node [style=none] (55) at (-2, 9.5) {};
		\node [style=none] (56) at (-1, 9.5) {};
		\node [style=none] (57) at (0.5, 9.5) {};
		\node [style=none] (58) at (1, 9.5) {};
		\node [style=none] (59) at (-1, 13.5) {};
		\node [style=none] (60) at (-2, 13.5) {};
		\node [style=none] (61) at (1, 13.5) {};
		\node [style=none] (62) at (0.5, 13.5) {};
		\node [style=andout] (63) at (-1.5, 10.75) {};
		\node [style=andin] (64) at (-1.5, 12.25) {};
		\node [style=andin] (65) at (0, 12.25) {};
		\node [style=none] (66) at (-1.5, 12.75) {};
		\node [style=none] (67) at (-0.5, 12.75) {};
		\node [style=none] (68) at (-1.5, 10.25) {};
		\node [style=none] (69) at (-0.5, 10.25) {};
	\end{pgfonlayer}
	\begin{pgfonlayer}{edgelayer}
		\draw (46) to (47.center);
		\draw (47.center) to (45);
		\draw (48) to (50.center);
		\draw (50.center) to (49);
		\draw (51) to (54.center);
		\draw (54.center) to (53);
		\draw [in=90, out=180] (52) to (54.center);
		\draw (61.center) to (58.center);
		\draw (57.center) to (62.center);
		\draw (59.center) to (56.center);
		\draw (55.center) to (60.center);
		\draw [in=90, out=90, looseness=1.25] (67.center) to (66.center);
		\draw (66.center) to (50.center);
		\draw (47.center) to (68.center);
		\draw [in=-90, out=-90, looseness=1.25] (68.center) to (69.center);
		\draw (69.center) to (53);
		\draw (53) to (67.center);
	\end{pgfonlayer}
\end{tikzpicture}\\
&\eq{\ref{ZXA.3}}
\begin{tikzpicture}[tikzfig]
	\begin{pgfonlayer}{nodelayer}
		\node [style=X] (46) at (-2.5, 11.5) {};
		\node [style=X] (47) at (-1.5, 11.5) {};
		\node [style=none] (48) at (-2, 10.75) {};
		\node [style=X] (49) at (-1.5, 11.5) {};
		\node [style=X] (50) at (-2.5, 11.5) {};
		\node [style=none] (51) at (-2, 12.25) {};
		\node [style=X] (52) at (0.5, 11.5) {};
		\node [style=Z] (53) at (1, 13) {};
		\node [style=X] (54) at (-0.5, 11.5) {};
		\node [style=none] (55) at (0, 12.25) {};
		\node [style=none] (56) at (-3, 9.5) {};
		\node [style=none] (57) at (-1, 9.5) {};
		\node [style=none] (58) at (0.5, 9.5) {};
		\node [style=none] (59) at (1, 9.5) {};
		\node [style=none] (60) at (-1, 13.5) {};
		\node [style=none] (61) at (-3, 13.5) {};
		\node [style=none] (62) at (1, 13.5) {};
		\node [style=none] (63) at (0.5, 13.5) {};
		\node [style=andout] (64) at (-2, 10.75) {};
		\node [style=andin] (65) at (-2, 12.25) {};
		\node [style=andin] (66) at (0, 12.25) {};
		\node [style=none] (67) at (-2, 12.75) {};
		\node [style=none] (68) at (-0.5, 12.75) {};
		\node [style=none] (69) at (-2, 10.25) {};
		\node [style=none] (70) at (-0.5, 10.25) {};
		\node [style=X] (71) at (-1, 11.5) {};
		\node [style=X] (72) at (-3, 11.5) {};
	\end{pgfonlayer}
	\begin{pgfonlayer}{edgelayer}
		\draw (47) to (48.center);
		\draw (48.center) to (46);
		\draw (49) to (51.center);
		\draw (51.center) to (50);
		\draw (52) to (55.center);
		\draw (55.center) to (54);
		\draw [in=90, out=180] (53) to (55.center);
		\draw (62.center) to (59.center);
		\draw (58.center) to (63.center);
		\draw (60.center) to (57.center);
		\draw (56.center) to (61.center);
		\draw [in=90, out=90, looseness=1.25] (68.center) to (67.center);
		\draw (67.center) to (51.center);
		\draw (48.center) to (69.center);
		\draw [in=-90, out=-90, looseness=1.25] (69.center) to (70.center);
		\draw (70.center) to (54);
		\draw (54) to (68.center);
		\draw (71) to (47);
		\draw (46) to (72);
	\end{pgfonlayer}
\end{tikzpicture}
=
\begin{tikzpicture}[tikzfig]
	\begin{pgfonlayer}{nodelayer}
		\node [style=X] (47) at (-0.75, 11.5) {};
		\node [style=none] (48) at (0.25, 9.5) {};
		\node [style=andin] (49) at (-0.25, 12.25) {};
		\node [style=none] (50) at (-0.75, 12.75) {};
		\node [style=none] (51) at (0.75, 9.5) {};
		\node [style=none] (52) at (-1.75, 10.75) {};
		\node [style=X] (53) at (0.25, 11.5) {};
		\node [style=none] (54) at (0.75, 13.5) {};
		\node [style=none] (55) at (-1.25, 13.5) {};
		\node [style=none] (56) at (-3.5, 9.75) {};
		\node [style=none] (57) at (-3, 12.75) {};
		\node [style=X] (58) at (-1.25, 12.5) {};
		\node [style=Z] (59) at (0.75, 13) {};
		\node [style=none] (60) at (-0.25, 12.25) {};
		\node [style=none] (61) at (-0.75, 10.25) {};
		\node [style=none] (62) at (-1.25, 9.5) {};
		\node [style=none] (63) at (-1.75, 10.25) {};
		\node [style=none] (64) at (-3.5, 13.5) {};
		\node [style=X] (65) at (-3.5, 12.5) {};
		\node [style=none] (66) at (0.25, 13.5) {};
		\node [style=andout] (67) at (-2.5, 10.75) {};
		\node [style=none] (68) at (-2.5, 10.25) {};
		\node [style=none] (69) at (-2.5, 10.75) {};
		\node [style=X] (70) at (-2.5, 11.75) {};
		\node [style=X] (71) at (-1.75, 11.75) {};
		\node [style=andout] (72) at (-1.75, 10.75) {};
		\node [style=none] (73) at (-3, 10.25) {};
	\end{pgfonlayer}
	\begin{pgfonlayer}{edgelayer}
		\draw (53) to (60.center);
		\draw (60.center) to (47);
		\draw [in=90, out=180] (59) to (60.center);
		\draw (54.center) to (51.center);
		\draw (48.center) to (66.center);
		\draw (55.center) to (62.center);
		\draw (56.center) to (64.center);
		\draw [in=90, out=90, looseness=0.50] (50.center) to (57.center);
		\draw (52.center) to (63.center);
		\draw [in=-90, out=-90, looseness=1.25] (63.center) to (61.center);
		\draw (61.center) to (47);
		\draw (47) to (50.center);
		\draw (69.center) to (68.center);
		\draw (71) to (69.center);
		\draw [in=-120, out=120, looseness=1.25] (69.center) to (70);
		\draw [in=60, out=-60, looseness=1.25] (71) to (52.center);
		\draw (52.center) to (70);
		\draw [in=0, out=90] (70) to (65);
		\draw (57.center) to (73.center);
		\draw [bend right=90, looseness=1.50] (73.center) to (68.center);
		\draw [in=90, out=180] (58) to (71);
	\end{pgfonlayer}
\end{tikzpicture}\\
&\eq{\ref{ZXA.12}}
\begin{tikzpicture}[tikzfig]
	\begin{pgfonlayer}{nodelayer}
		\node [style=none] (48) at (-2.75, 9.75) {};
		\node [style=X] (49) at (-1.25, 12.5) {};
		\node [style=Z] (50) at (0.75, 13) {};
		\node [style=none] (51) at (0.75, 9.5) {};
		\node [style=none] (52) at (0.25, 13.5) {};
		\node [style=none] (53) at (-2.25, 12.75) {};
		\node [style=andout] (54) at (-1.75, 11.75) {};
		\node [style=X] (55) at (-1.75, 10.75) {};
		\node [style=none] (56) at (-0.25, 12.25) {};
		\node [style=none] (57) at (-1.25, 13.5) {};
		\node [style=none] (58) at (0.75, 13.5) {};
		\node [style=andin] (59) at (-0.25, 12.25) {};
		\node [style=none] (60) at (-2.25, 10.75) {};
		\node [style=none] (61) at (-1.75, 11.75) {};
		\node [style=none] (62) at (-0.75, 10.75) {};
		\node [style=none] (63) at (-2.75, 13.5) {};
		\node [style=none] (64) at (0.25, 9.5) {};
		\node [style=X] (65) at (-0.75, 11.5) {};
		\node [style=none] (66) at (-1.25, 9.5) {};
		\node [style=X] (67) at (0.25, 11.5) {};
		\node [style=none] (68) at (-0.75, 12.75) {};
		\node [style=X] (69) at (-2.75, 12.5) {};
	\end{pgfonlayer}
	\begin{pgfonlayer}{edgelayer}
		\draw (67) to (56.center);
		\draw (56.center) to (65);
		\draw [in=90, out=180] (50) to (56.center);
		\draw (58.center) to (51.center);
		\draw (64.center) to (52.center);
		\draw (57.center) to (66.center);
		\draw (48.center) to (63.center);
		\draw [in=90, out=90, looseness=0.50] (68.center) to (53.center);
		\draw (62.center) to (65);
		\draw (65) to (68.center);
		\draw (53.center) to (60.center);
		\draw [in=-60, out=-90] (62.center) to (55);
		\draw [in=-90, out=-105, looseness=1.75] (55) to (60.center);
		\draw (61.center) to (55);
		\draw (61.center) to (69);
		\draw (61.center) to (49);
	\end{pgfonlayer}
\end{tikzpicture}
\eq{\ref{ZXA.3}}
\begin{tikzpicture}[tikzfig]
	\begin{pgfonlayer}{nodelayer}
		\node [style=none] (49) at (0.25, 9.5) {};
		\node [style=andin] (50) at (-0.25, 11.25) {};
		\node [style=none] (51) at (0.75, 9.5) {};
		\node [style=X] (52) at (0.25, 10.25) {};
		\node [style=none] (53) at (0.75, 12.75) {};
		\node [style=none] (54) at (-0.75, 12.75) {};
		\node [style=none] (55) at (-1.75, 9.5) {};
		\node [style=X] (56) at (-0.75, 12) {};
		\node [style=Z] (57) at (0.75, 12) {};
		\node [style=none] (58) at (-0.25, 11.25) {};
		\node [style=none] (59) at (-0.75, 9.5) {};
		\node [style=none] (60) at (-1.75, 12.75) {};
		\node [style=X] (61) at (-1.75, 12) {};
		\node [style=none] (62) at (0.25, 12.75) {};
		\node [style=none] (63) at (-1.25, 11.25) {};
		\node [style=andout] (64) at (-1.25, 11.25) {};
	\end{pgfonlayer}
	\begin{pgfonlayer}{edgelayer}
		\draw (52) to (58.center);
		\draw [in=90, out=180] (57) to (58.center);
		\draw (53.center) to (51.center);
		\draw (49.center) to (62.center);
		\draw (54.center) to (59.center);
		\draw (55.center) to (60.center);
		\draw (63.center) to (61);
		\draw (63.center) to (56);
		\draw [in=-90, out=-120, looseness=2.00] (58.center) to (63.center);
	\end{pgfonlayer}
\end{tikzpicture}\\
&=
\begin{tikzpicture}[tikzfig]
	\begin{pgfonlayer}{nodelayer}
		\node [style=none] (50) at (0.25, 9.5) {};
		\node [style=none] (51) at (0.75, 9.5) {};
		\node [style=X] (52) at (0.25, 11.25) {};
		\node [style=none] (53) at (0.75, 13.5) {};
		\node [style=none] (54) at (-0.75, 13.5) {};
		\node [style=none] (55) at (-1.75, 9.5) {};
		\node [style=X] (56) at (-0.75, 10.25) {};
		\node [style=Z] (57) at (0.75, 13) {};
		\node [style=none] (58) at (-0.75, 9.5) {};
		\node [style=none] (59) at (-1.75, 13.5) {};
		\node [style=X] (60) at (-1.75, 10.25) {};
		\node [style=none] (61) at (0.25, 13.5) {};
		\node [style=none] (62) at (-1.25, 11.25) {};
		\node [style=andin] (63) at (-1.25, 11.25) {};
		\node [style=none] (64) at (-0.25, 12.25) {};
		\node [style=andin] (65) at (-0.25, 12.25) {};
	\end{pgfonlayer}
	\begin{pgfonlayer}{edgelayer}
		\draw (53.center) to (51.center);
		\draw (50.center) to (61.center);
		\draw (54.center) to (58.center);
		\draw (55.center) to (59.center);
		\draw [in=90, out=180, looseness=1.50] (57) to (64.center);
		\draw (64.center) to (52);
		\draw [in=-124, out=90] (62.center) to (64.center);
		\draw (62.center) to (56);
		\draw (60) to (62.center);
	\end{pgfonlayer}
\end{tikzpicture}
\eq{\ref{ZXA.11}}
\begin{tikzpicture}[tikzfig]
	\begin{pgfonlayer}{nodelayer}
		\node [style=none] (51) at (0.25, 9.5) {};
		\node [style=none] (52) at (0.75, 9.5) {};
		\node [style=X] (53) at (0.25, 11.75) {};
		\node [style=none] (54) at (0.75, 14.25) {};
		\node [style=none] (55) at (-0.75, 14.25) {};
		\node [style=none] (56) at (-1.75, 9.5) {};
		\node [style=X] (57) at (-0.75, 10.25) {};
		\node [style=Z] (58) at (0.75, 13.75) {};
		\node [style=none] (59) at (-0.75, 9.5) {};
		\node [style=none] (60) at (-1.75, 14.25) {};
		\node [style=X] (61) at (-1.75, 10.25) {};
		\node [style=none] (62) at (0.25, 14.25) {};
		\node [style=none] (63) at (-1.25, 11.75) {};
		\node [style=andin] (64) at (-1.25, 11.75) {};
		\node [style=none] (65) at (-0.25, 13) {};
		\node [style=andin] (66) at (-0.25, 13) {};
		\node [style=none] (67) at (-1.5, 11) {};
		\node [style=none] (68) at (-1, 11) {};
	\end{pgfonlayer}
	\begin{pgfonlayer}{edgelayer}
		\draw (54.center) to (52.center);
		\draw (51.center) to (62.center);
		\draw (55.center) to (59.center);
		\draw (56.center) to (60.center);
		\draw [in=90, out=180, looseness=1.50] (58) to (65.center);
		\draw [in=90, out=-120, looseness=1.25] (63.center) to (67.center);
		\draw [in=-60, out=90, looseness=1.25] (68.center) to (63.center);
		\draw [in=45, out=-90] (68.center) to (61);
		\draw [in=135, out=-90] (67.center) to (57);
		\draw (65.center) to (53);
		\draw [in=90, out=-129] (65.center) to (63.center);
	\end{pgfonlayer}
\end{tikzpicture}\\
&\eq{\ref{ZXA.9}}
\begin{tikzpicture}[tikzfig]
	\begin{pgfonlayer}{nodelayer}
		\node [style=none] (52) at (-0.25, 9.5) {};
		\node [style=none] (53) at (0.25, 9.5) {};
		\node [style=X] (54) at (-0.25, 11.5) {};
		\node [style=none] (55) at (0.25, 14.25) {};
		\node [style=none] (56) at (-0.75, 14.25) {};
		\node [style=none] (57) at (-1.75, 9.5) {};
		\node [style=X] (58) at (-0.75, 10.25) {};
		\node [style=Z] (59) at (0.25, 13.75) {};
		\node [style=none] (60) at (-0.75, 9.5) {};
		\node [style=none] (61) at (-1.75, 14.25) {};
		\node [style=X] (62) at (-1.75, 10.25) {};
		\node [style=none] (63) at (-0.25, 14.25) {};
		\node [style=none] (64) at (-1.5, 13) {};
		\node [style=andin] (65) at (-1.5, 13) {};
		\node [style=none] (66) at (-1.5, 11) {};
		\node [style=none] (67) at (-1, 11) {};
		\node [style=andin] (68) at (-1, 12.25) {};
		\node [style=none] (69) at (-1, 12.25) {};
	\end{pgfonlayer}
	\begin{pgfonlayer}{edgelayer}
		\draw (55.center) to (53.center);
		\draw (52.center) to (63.center);
		\draw (56.center) to (60.center);
		\draw (57.center) to (61.center);
		\draw [in=90, out=180, looseness=1.25] (59) to (64.center);
		\draw [in=45, out=-90] (67.center) to (62);
		\draw [in=135, out=-90] (66.center) to (58);
		\draw (69.center) to (54);
		\draw (69.center) to (64.center);
		\draw (64.center) to (66.center);
		\draw [in=90, out=-120, looseness=1.25] (69.center) to (67.center);
	\end{pgfonlayer}
\end{tikzpicture}
\eq{\ref{ZXA.11}}
\begin{tikzpicture}[tikzfig]
	\begin{pgfonlayer}{nodelayer}
		\node [style=none] (53) at (0, 9.5) {};
		\node [style=none] (54) at (0.5, 9.5) {};
		\node [style=X] (55) at (0, 10.75) {};
		\node [style=none] (56) at (0.5, 14.5) {};
		\node [style=none] (57) at (-0.5, 14.5) {};
		\node [style=none] (58) at (-2, 9.5) {};
		\node [style=X] (59) at (-0.5, 10.25) {};
		\node [style=Z] (60) at (0.5, 14) {};
		\node [style=none] (61) at (-0.5, 9.5) {};
		\node [style=none] (62) at (-2, 14.5) {};
		\node [style=X] (63) at (-2, 10.25) {};
		\node [style=none] (64) at (0, 14.5) {};
		\node [style=none] (65) at (-1.5, 13.25) {};
		\node [style=andin] (66) at (-1.5, 13.25) {};
		\node [style=none] (67) at (-1.75, 11.75) {};
		\node [style=andin] (68) at (-1, 12.5) {};
		\node [style=none] (69) at (-1, 12.5) {};
		\node [style=none] (70) at (-1.25, 11.75) {};
		\node [style=none] (71) at (-0.75, 11.75) {};
	\end{pgfonlayer}
	\begin{pgfonlayer}{edgelayer}
		\draw (56.center) to (54.center);
		\draw (53.center) to (64.center);
		\draw (57.center) to (61.center);
		\draw (58.center) to (62.center);
		\draw [in=90, out=180, looseness=1.25] (60) to (65.center);
		\draw [in=135, out=-90] (67.center) to (59);
		\draw (69.center) to (65.center);
		\draw [in=90, out=-99] (65.center) to (67.center);
		\draw [in=90, out=-72] (69.center) to (71.center);
		\draw [in=-108, out=90] (70.center) to (69.center);
		\draw [in=149, out=-90] (70.center) to (55);
		\draw [in=56, out=-90] (71.center) to (63);
	\end{pgfonlayer}
\end{tikzpicture}\\
&\eq{\ref{ZXA.9}}
\begin{tikzpicture}[tikzfig]
	\begin{pgfonlayer}{nodelayer}
		\node [style=none] (54) at (0, 9.5) {};
		\node [style=none] (55) at (0.5, 9.5) {};
		\node [style=X] (56) at (0, 10.75) {};
		\node [style=none] (57) at (0.5, 14.5) {};
		\node [style=none] (58) at (-0.5, 14.5) {};
		\node [style=none] (59) at (-2, 9.5) {};
		\node [style=X] (60) at (-0.5, 10.25) {};
		\node [style=Z] (61) at (0.5, 14) {};
		\node [style=none] (62) at (-0.5, 9.5) {};
		\node [style=none] (63) at (-2, 14.5) {};
		\node [style=X] (64) at (-2, 10.25) {};
		\node [style=none] (65) at (0, 14.5) {};
		\node [style=none] (66) at (-1, 13.25) {};
		\node [style=andin] (67) at (-1, 13.25) {};
		\node [style=none] (68) at (-1.75, 11.75) {};
		\node [style=none] (69) at (-1.25, 11.75) {};
		\node [style=none] (70) at (-0.75, 11.75) {};
		\node [style=none] (71) at (-1.5, 12.5) {};
		\node [style=andin] (72) at (-1.5, 12.5) {};
	\end{pgfonlayer}
	\begin{pgfonlayer}{edgelayer}
		\draw (57.center) to (55.center);
		\draw (54.center) to (65.center);
		\draw (58.center) to (62.center);
		\draw (59.center) to (63.center);
		\draw [in=90, out=180, looseness=1.25] (61) to (66.center);
		\draw [in=135, out=-90] (68.center) to (60);
		\draw [in=149, out=-90] (69.center) to (56);
		\draw [in=56, out=-90] (70.center) to (64);
		\draw [in=90, out=-81] (66.center) to (70.center);
		\draw [in=90, out=-124] (66.center) to (71.center);
		\draw [in=90, out=-72] (71.center) to (69.center);
		\draw [in=90, out=-108] (71.center) to (68.center);
	\end{pgfonlayer}
\end{tikzpicture}
\eq{\ref{ZXA.11}}
\begin{tikzpicture}[tikzfig]
	\begin{pgfonlayer}{nodelayer}
		\node [style=none] (55) at (0, 10.5) {};
		\node [style=none] (56) at (0.5, 10.5) {};
		\node [style=X] (57) at (0, 11.75) {};
		\node [style=none] (58) at (0.5, 16.25) {};
		\node [style=none] (59) at (-0.5, 16.25) {};
		\node [style=none] (60) at (-2, 10.5) {};
		\node [style=X] (61) at (-0.5, 11.25) {};
		\node [style=Z] (62) at (0.5, 15.75) {};
		\node [style=none] (63) at (-0.5, 10.5) {};
		\node [style=none] (64) at (-2, 16.25) {};
		\node [style=X] (65) at (-2, 11.25) {};
		\node [style=none] (66) at (0, 16.25) {};
		\node [style=none] (67) at (-1, 15) {};
		\node [style=andin] (68) at (-1, 15) {};
		\node [style=none] (69) at (-1.75, 12.75) {};
		\node [style=none] (70) at (-1.25, 12.75) {};
		\node [style=none] (71) at (-0.75, 12.75) {};
		\node [style=none] (72) at (-1.5, 14.25) {};
		\node [style=andin] (73) at (-1.5, 14.25) {};
		\node [style=none] (74) at (-1.75, 13.5) {};
		\node [style=none] (75) at (-1.25, 13.5) {};
	\end{pgfonlayer}
	\begin{pgfonlayer}{edgelayer}
		\draw (58.center) to (56.center);
		\draw (55.center) to (66.center);
		\draw (59.center) to (63.center);
		\draw (60.center) to (64.center);
		\draw [in=90, out=180, looseness=1.25] (62) to (67.center);
		\draw [in=135, out=-90] (69.center) to (61);
		\draw [in=149, out=-90] (70.center) to (57);
		\draw [in=56, out=-90] (71.center) to (65);
		\draw [in=90, out=-81] (67.center) to (71.center);
		\draw [in=90, out=-124] (67.center) to (72.center);
		\draw [in=90, out=-72] (72.center) to (75.center);
		\draw [in=90, out=-108] (72.center) to (74.center);
		\draw [in=90, out=-90] (74.center) to (70.center);
		\draw [in=90, out=-90] (75.center) to (69.center);
	\end{pgfonlayer}
\end{tikzpicture}\\
&=
\begin{tikzpicture}[tikzfig]
	\begin{pgfonlayer}{nodelayer}
		\node [style=none] (56) at (-0.75, 13.75) {};
		\node [style=Z] (57) at (-0.25, 15.25) {};
		\node [style=andin] (58) at (-1.75, 13.5) {};
		\node [style=X] (59) at (-0.75, 12) {};
		\node [style=none] (60) at (-1.25, 15.75) {};
		\node [style=none] (61) at (-2.75, 10.5) {};
		\node [style=X] (62) at (-2.75, 12) {};
		\node [style=none] (63) at (-1.75, 13.5) {};
		\node [style=none] (64) at (-0.25, 10.5) {};
		\node [style=andin] (65) at (-2.25, 12.75) {};
		\node [style=none] (66) at (-1.25, 10.5) {};
		\node [style=none] (67) at (-0.75, 10.5) {};
		\node [style=X] (68) at (-1.25, 12) {};
		\node [style=none] (69) at (-2.25, 12.75) {};
		\node [style=none] (70) at (-2.75, 14) {};
		\node [style=none] (71) at (-0.25, 15.75) {};
		\node [style=none] (72) at (-2.75, 15) {};
		\node [style=none] (73) at (-0.75, 15.25) {};
		\node [style=none] (74) at (-2.75, 15.75) {};
		\node [style=none] (75) at (-0.75, 15.75) {};
	\end{pgfonlayer}
	\begin{pgfonlayer}{edgelayer}
		\draw (71.center) to (64.center);
		\draw (60.center) to (66.center);
		\draw [in=90, out=180, looseness=1.25] (57) to (63.center);
		\draw [in=90, out=-124, looseness=1.25] (63.center) to (69.center);
		\draw [in=90, out=-90] (62) to (67.center);
		\draw [in=-90, out=90] (61.center) to (59);
		\draw (69.center) to (62);
		\draw (68) to (69.center);
		\draw [in=105, out=-60] (63.center) to (59);
		\draw [in=90, out=-90] (72.center) to (56.center);
		\draw [in=90, out=-90] (73.center) to (70.center);
		\draw (75.center) to (73.center);
		\draw (56.center) to (59);
		\draw (62) to (70.center);
		\draw (72.center) to (74.center);
	\end{pgfonlayer}
\end{tikzpicture}
=
\begin{tikzpicture}[tikzfig]
	\begin{pgfonlayer}{nodelayer}
		\node [style=X] (57) at (-2, 11.75) {};
		\node [style=X] (58) at (-1, 11.75) {};
		\node [style=none] (59) at (-1.5, 12.5) {};
		\node [style=Z] (60) at (-0.5, 13.25) {};
		\node [style=Z] (61) at (-0.5, 12.5) {};
		\node [style=Z] (62) at (-0.5, 15.75) {};
		\node [style=X] (63) at (-1, 13.75) {};
		\node [style=Z] (64) at (-0.5, 15.25) {};
		\node [style=X] (65) at (-2, 13.75) {};
		\node [style=none] (66) at (-1.5, 14.5) {};
		\node [style=X] (67) at (0.5, 13.75) {};
		\node [style=Z] (68) at (1, 15.25) {};
		\node [style=X] (69) at (-0.5, 13.75) {};
		\node [style=none] (70) at (0, 14.5) {};
		\node [style=none] (71) at (-2, 10.5) {};
		\node [style=none] (72) at (-1, 11.75) {};
		\node [style=none] (73) at (0.5, 11.75) {};
		\node [style=none] (74) at (1, 10.5) {};
		\node [style=none] (75) at (-1, 15.75) {};
		\node [style=none] (76) at (-2, 17) {};
		\node [style=none] (77) at (1, 17) {};
		\node [style=none] (78) at (0.5, 15.75) {};
		\node [style=andin] (79) at (-1.5, 12.5) {};
		\node [style=andin] (80) at (-1.5, 14.5) {};
		\node [style=andin] (81) at (0, 14.5) {};
		\node [style=none] (82) at (-1, 17) {};
		\node [style=none] (83) at (0.5, 17) {};
		\node [style=none] (84) at (0.5, 10.5) {};
		\node [style=none] (85) at (-1, 10.5) {};
	\end{pgfonlayer}
	\begin{pgfonlayer}{edgelayer}
		\draw (58) to (59.center);
		\draw (59.center) to (57);
		\draw (60) to (61);
		\draw [in=90, out=180] (60) to (59.center);
		\draw (63) to (66.center);
		\draw (66.center) to (65);
		\draw (64) to (62);
		\draw [in=90, out=180] (64) to (66.center);
		\draw (67) to (70.center);
		\draw (70.center) to (69);
		\draw [in=90, out=180] (68) to (70.center);
		\draw (77.center) to (74.center);
		\draw (73.center) to (78.center);
		\draw (64) to (69);
		\draw (69) to (60);
		\draw (71.center) to (76.center);
		\draw [in=90, out=-90] (82.center) to (78.center);
		\draw [in=-90, out=90] (75.center) to (83.center);
		\draw [in=90, out=-90] (72.center) to (84.center);
		\draw [in=-90, out=90] (85.center) to (73.center);
		\draw (75.center) to (58);
	\end{pgfonlayer}
\end{tikzpicture}\\
&=
\left\llbracket
\begin{tikzpicture}[tikzfig]
	\begin{pgfonlayer}{nodelayer}
		\node [style=nothing] (58) at (0, 10.5) {};
		\node [style=nothing] (59) at (-0.5, 10.5) {};
		\node [style=nothing] (60) at (-1.5, 10.5) {};
		\node [style=nothing] (61) at (-2, 10.5) {};
		\node [style=zeroin] (62) at (-1, 11.25) {};
		\node [style=oplus] (63) at (-1, 11.75) {};
		\node [style=oplus] (64) at (-1, 12.75) {};
		\node [style=dot] (65) at (-1, 12.25) {};
		\node [style=dot] (66) at (-0.5, 12.25) {};
		\node [style=dot] (67) at (-1.5, 11.75) {};
		\node [style=dot] (68) at (-2, 11.75) {};
		\node [style=dot] (69) at (-1.5, 12.75) {};
		\node [style=dot] (70) at (-2, 12.75) {};
		\node [style=oplus] (71) at (0, 12.25) {};
		\node [style=zeroout] (72) at (-1, 13.25) {};
		\node [style=nothing] (73) at (0, 14) {};
		\node [style=nothing] (74) at (-2, 14) {};
		\node [style=nothing] (75) at (-0.5, 14) {};
		\node [style=nothing] (76) at (-1.5, 14) {};
		\node [style=none] (77) at (-1.5, 13.5) {};
		\node [style=none] (78) at (-0.5, 13.5) {};
		\node [style=none] (79) at (-0.5, 11) {};
		\node [style=none] (80) at (-1.5, 11) {};
	\end{pgfonlayer}
	\begin{pgfonlayer}{edgelayer}
		\draw (61) to (68);
		\draw (68) to (70);
		\draw (70) to (74);
		\draw (69) to (67);
		\draw (73) to (71);
		\draw (71) to (58);
		\draw (71) to (66);
		\draw (66) to (65);
		\draw (67) to (63);
		\draw (67) to (68);
		\draw (70) to (69);
		\draw (64) to (69);
		\draw (62) to (63);
		\draw (63) to (65);
		\draw (65) to (64);
		\draw (64) to (72);
		\draw [in=90, out=-90, looseness=0.50] (75) to (77.center);
		\draw [in=90, out=-90, looseness=0.75] (76) to (78.center);
		\draw (78.center) to (66);
		\draw (66) to (79.center);
		\draw [in=90, out=-105, looseness=0.50] (79.center) to (60);
		\draw [in=-90, out=90, looseness=0.50] (59) to (80.center);
		\draw (80.center) to (67);
		\draw (69) to (77.center);
	\end{pgfonlayer}
\end{tikzpicture}
\right\rrbracket_{\hat{\TOF}}
\end{align*}
\endgroup



\end{enumerate}
Where unitality and counitality follow from the fact that the white spiders are Frobenius algebras.  Also, we must also note that both Frobenius algebras induce the same compact closed structure, as is implied by the spider law;  this is immediate.

\end{proof}

\begin{theorem}
\label{theorem:TOFZXAiso}
The interpretation functors $\llbracket\_\rrbracket_{\ZXA}$ and $\llbracket\_\rrbracket_{\hat \TOF}$ are inverses, so that $\hat \TOF$ and $\ZXA$ are isomorphic as strongly compact closed props.
\end{theorem}

\begin{proof}
First we show that $\llbracket\llbracket\_\rrbracket_{\ZXA}\rrbracket_{\hat \TOF}=1$:
\begin{description}
\item[For the white spider:]
The case for the unit and counit is trivial.  For the (co)multiplication we have:
\begin{align*}
\left\llbracket\left\llbracket
\begin{tikzpicture}[tikzfig]
	\begin{pgfonlayer}{nodelayer}
		\node [style=X] (59) at (0, 11.5) {};
		\node [style=none] (60) at (-0.5, 12.5) {};
		\node [style=none] (61) at (0.5, 12.5) {};
		\node [style=none] (62) at (0, 10.5) {};
	\end{pgfonlayer}
	\begin{pgfonlayer}{edgelayer}
		\draw [in=63, out=-90] (61.center) to (59);
		\draw [in=-90, out=117] (59) to (60.center);
		\draw (59) to (62.center);
	\end{pgfonlayer}
\end{tikzpicture}
\right\rrbracket_{\ZXA}\right\rrbracket_{\hat \TOF}
&=
\left\llbracket
\begin{tikzpicture}[tikzfig]
	\begin{pgfonlayer}{nodelayer}
		\node [style=none] (60) at (-0.5, 11.75) {};
		\node [style=none] (61) at (0, 11.75) {};
		\node [style=none] (62) at (-0.5, 10.5) {};
		\node [style=dot] (63) at (-0.5, 11.25) {};
		\node [style=oplus] (64) at (0, 11.25) {};
		\node [style=zeroin] (65) at (0, 10.75) {};
	\end{pgfonlayer}
	\begin{pgfonlayer}{edgelayer}
		\draw (61.center) to (64);
		\draw (64) to (65);
		\draw (64) to (63);
		\draw (63) to (60.center);
		\draw (63) to (62.center);
	\end{pgfonlayer}
\end{tikzpicture}
\right\rrbracket_{\hat \TOF}
=
\begin{tikzpicture}[tikzfig]
	\begin{pgfonlayer}{nodelayer}
		\node [style=andin] (61) at (-1.5, 12.5) {};
		\node [style=X] (62) at (-2, 11.5) {};
		\node [style=X] (63) at (-1, 11.5) {};
		\node [style=Z] (64) at (-0.5, 13) {};
		\node [style=none] (65) at (-1.5, 12.5) {};
		\node [style=Z] (66) at (-2, 10.5) {$\pi$};
		\node [style=Z] (67) at (-2, 12.5) {$\pi$};
		\node [style=none] (68) at (-0.5, 13.75) {};
		\node [style=none] (69) at (-1, 13.75) {};
		\node [style=none] (70) at (-1, 10.5) {};
		\node [style=Z] (71) at (-0.5, 11.5) {};
	\end{pgfonlayer}
	\begin{pgfonlayer}{edgelayer}
		\draw [in=90, out=180] (64) to (65.center);
		\draw (65.center) to (62);
		\draw (63) to (65.center);
		\draw (67) to (62);
		\draw (62) to (66);
		\draw (69.center) to (63);
		\draw (63) to (70.center);
		\draw (68.center) to (64);
		\draw (64) to (71);
	\end{pgfonlayer}
\end{tikzpicture}
=
\begin{tikzpicture}[tikzfig]
	\begin{pgfonlayer}{nodelayer}
		\node [style=X] (62) at (-1, 11.5) {};
		\node [style=Z] (63) at (-0.5, 11.5) {};
		\node [style=none] (64) at (-0.5, 12.25) {};
		\node [style=none] (65) at (-1, 12.25) {};
		\node [style=none] (66) at (-1, 10.5) {};
		\node [style=Z] (67) at (-0.5, 10.75) {};
	\end{pgfonlayer}
	\begin{pgfonlayer}{edgelayer}
		\draw (65.center) to (62);
		\draw (62) to (66.center);
		\draw (64.center) to (63);
		\draw (63) to (67);
		\draw (63) to (62);
	\end{pgfonlayer}
\end{tikzpicture}
=
\begin{tikzpicture}[tikzfig]
	\begin{pgfonlayer}{nodelayer}
		\node [style=X] (63) at (0, 11.25) {};
		\node [style=none] (64) at (-0.5, 12.25) {};
		\node [style=none] (65) at (0.5, 12.25) {};
		\node [style=none] (66) at (0, 10.5) {};
	\end{pgfonlayer}
	\begin{pgfonlayer}{edgelayer}
		\draw [in=63, out=-90] (65.center) to (63);
		\draw [in=-90, out=117] (63) to (64.center);
		\draw (63) to (66.center);
	\end{pgfonlayer}
\end{tikzpicture}
\end{align*}

\item[For the grey spider:]
The cases for the unit, counit and $\pi$ phase are trivial.  For the (co) multiplication we have:

\begin{align*}
\left\llbracket\left\llbracket
\begin{tikzpicture}[tikzfig]
	\begin{pgfonlayer}{nodelayer}
		\node [style=Z] (64) at (0, 11) {};
		\node [style=none] (65) at (-0.5, 11.75) {};
		\node [style=none] (66) at (0.5, 11.75) {};
		\node [style=none] (67) at (0, 10.5) {};
	\end{pgfonlayer}
	\begin{pgfonlayer}{edgelayer}
		\draw [in=63, out=-90] (66.center) to (64);
		\draw [in=-90, out=117] (64) to (65.center);
		\draw (64) to (67.center);
	\end{pgfonlayer}
\end{tikzpicture}
\right\rrbracket_{\ZXA}\right\rrbracket_{\hat \TOF}
&=
\left\llbracket
\begin{tikzpicture}[tikzfig]
	\begin{pgfonlayer}{nodelayer}
		\node [style=none] (65) at (-0.5, 11.75) {};
		\node [style=none] (66) at (0, 11.75) {};
		\node [style=none] (67) at (-0.5, 10.5) {};
		\node [style=dot] (68) at (0, 11.25) {};
		\node [style=oplus] (69) at (-0.5, 11.25) {};
		\node [style=X] (70) at (0, 10.75) {};
	\end{pgfonlayer}
	\begin{pgfonlayer}{edgelayer}
		\draw (69) to (68);
		\draw (68) to (66.center);
		\draw (70) to (68);
		\draw (69) to (65.center);
		\draw (69) to (67.center);
	\end{pgfonlayer}
\end{tikzpicture}

\right\rrbracket_{\hat \TOF}
=
\begin{tikzpicture}[tikzfig]
	\begin{pgfonlayer}{nodelayer}
		\node [style=X] (66) at (-2, 11.75) {};
		\node [style=X] (67) at (-1, 11.75) {};
		\node [style=none] (68) at (-1.5, 12.5) {};
		\node [style=Z] (69) at (-2.5, 13.25) {};
		\node [style=Z] (70) at (-1, 10.75) {$\pi$};
		\node [style=Z] (71) at (-1, 13.25) {$\pi$};
		\node [style=X] (72) at (-2, 10.75) {};
		\node [style=none] (73) at (-2.5, 10.5) {};
		\node [style=none] (74) at (-2.5, 13.75) {};
		\node [style=none] (75) at (-2, 13.75) {};
		\node [style=andin] (76) at (-1.5, 12.5) {};
	\end{pgfonlayer}
	\begin{pgfonlayer}{edgelayer}
		\draw (70) to (67);
		\draw (67) to (68.center);
		\draw (71) to (67);
		\draw (68.center) to (66);
		\draw (68.center) to (69);
		\draw (75.center) to (66);
		\draw (66) to (72);
		\draw (73.center) to (69);
		\draw (69) to (74.center);
	\end{pgfonlayer}
\end{tikzpicture}
=
\begin{tikzpicture}[tikzfig]
	\begin{pgfonlayer}{nodelayer}
		\node [style=X] (67) at (-2, 11.25) {};
		\node [style=Z] (68) at (-2.5, 11.25) {};
		\node [style=X] (69) at (-2, 10.75) {};
		\node [style=none] (70) at (-2.5, 10.5) {};
		\node [style=none] (71) at (-2.5, 11.75) {};
		\node [style=none] (72) at (-2, 11.75) {};
	\end{pgfonlayer}
	\begin{pgfonlayer}{edgelayer}
		\draw (72.center) to (67);
		\draw (67) to (69);
		\draw (70.center) to (68);
		\draw (68) to (71.center);
		\draw (67) to (68);
	\end{pgfonlayer}
\end{tikzpicture}
=
\begin{tikzpicture}[tikzfig]
	\begin{pgfonlayer}{nodelayer}
		\node [style=Z] (68) at (0, 11) {};
		\node [style=none] (69) at (-0.5, 11.75) {};
		\node [style=none] (70) at (0.5, 11.75) {};
		\node [style=none] (71) at (0, 10.5) {};
	\end{pgfonlayer}
	\begin{pgfonlayer}{edgelayer}
		\draw [in=63, out=-90] (70.center) to (68);
		\draw [in=-90, out=117] (68) to (69.center);
		\draw (68) to (71.center);
	\end{pgfonlayer}
\end{tikzpicture}
\end{align*}


\item[For the {\sf and} gate:]
\begin{align*}
\left\llbracket\left\llbracket
\begin{tikzpicture}[tikzfig]
	\begin{pgfonlayer}{nodelayer}
		\node [style=none] (69) at (0, 11.5) {};
		\node [style=none] (70) at (-0.5, 10.5) {};
		\node [style=none] (71) at (0.5, 10.5) {};
		\node [style=none] (72) at (0, 12.5) {};
		\node [style=andin] (73) at (0, 11.5) {};
	\end{pgfonlayer}
	\begin{pgfonlayer}{edgelayer}
		\draw [in=-63, out=90] (71.center) to (69.center);
		\draw [in=90, out=-117] (69.center) to (70.center);
		\draw (69.center) to (72.center);
	\end{pgfonlayer}
\end{tikzpicture}
\right\rrbracket_{\ZXA}\right\rrbracket_{\hat \TOF}
&=
\left\llbracket
\begin{tikzpicture}[tikzfig]
	\begin{pgfonlayer}{nodelayer}
		\node [style=dot] (70) at (-2, 11.25) {};
		\node [style=dot] (71) at (-1.5, 11.25) {};
		\node [style=oplus] (72) at (-1, 11.25) {};
		\node [style=zeroin] (73) at (-1, 10.75) {};
		\node [style=X] (74) at (-2, 11.75) {};
		\node [style=X] (75) at (-1.5, 11.75) {};
		\node [style=none] (76) at (-1, 12) {};
		\node [style=none] (77) at (-2, 10.5) {};
		\node [style=none] (78) at (-1.5, 10.5) {};
	\end{pgfonlayer}
	\begin{pgfonlayer}{edgelayer}
		\draw (75) to (71);
		\draw (70) to (71);
		\draw (71) to (72);
		\draw (72) to (73);
		\draw (72) to (76.center);
		\draw (74) to (70);
		\draw (70) to (77.center);
		\draw (78.center) to (71);
	\end{pgfonlayer}
\end{tikzpicture}
\right\rrbracket_{\hat \TOF}
=
\begin{tikzpicture}[tikzfig]
	\begin{pgfonlayer}{nodelayer}
		\node [style=X] (71) at (-2, 11) {};
		\node [style=X] (72) at (-1, 11) {};
		\node [style=none] (73) at (-1.5, 11.75) {};
		\node [style=Z] (74) at (-0.5, 12.5) {};
		\node [style=Z] (75) at (-0.5, 11.5) {};
		\node [style=X] (76) at (-1, 13) {};
		\node [style=X] (77) at (-2, 13) {};
		\node [style=none] (78) at (-0.5, 13.25) {};
		\node [style=none] (79) at (-2, 10.5) {};
		\node [style=none] (80) at (-1, 10.5) {};
		\node [style=andin] (81) at (-1.5, 11.75) {};
	\end{pgfonlayer}
	\begin{pgfonlayer}{edgelayer}
		\draw (73.center) to (71);
		\draw (72) to (73.center);
		\draw (74) to (75);
		\draw [in=90, out=180] (74) to (73.center);
		\draw (76) to (72);
		\draw (72) to (80.center);
		\draw (79.center) to (71);
		\draw (71) to (77);
		\draw (78.center) to (74);
	\end{pgfonlayer}
\end{tikzpicture}
=
\begin{tikzpicture}[tikzfig]
	\begin{pgfonlayer}{nodelayer}
		\node [style=none] (72) at (0, 11.5) {};
		\node [style=none] (73) at (-0.5, 10.5) {};
		\node [style=none] (74) at (0.5, 10.5) {};
		\node [style=none] (75) at (0, 12.5) {};
		\node [style=andin] (76) at (0, 11.5) {};
	\end{pgfonlayer}
	\begin{pgfonlayer}{edgelayer}
		\draw [in=-63, out=90] (74.center) to (72.center);
		\draw [in=90, out=-117] (72.center) to (73.center);
		\draw (72.center) to (75.center);
	\end{pgfonlayer}
\end{tikzpicture}
\end{align*}
\end{description}

Next, we show that $\llbracket\llbracket\_\rrbracket_{\hat \TOF}\rrbracket_{\ZXA}=1$:
The ancillae are trivial.  For the Toffoli gate:
\begin{align*}
\left\llbracket\left\llbracket
\begin{tikzpicture}[tikzfig]
	\begin{pgfonlayer}{nodelayer}
		\node [style=dot] (73) at (-2, 11.25) {};
		\node [style=dot] (74) at (-1.5, 11.25) {};
		\node [style=oplus] (75) at (-1, 11.25) {};
		\node [style=none] (76) at (-1, 12) {};
		\node [style=none] (77) at (-1, 10.5) {};
		\node [style=none] (78) at (-2, 10.5) {};
		\node [style=none] (79) at (-1.5, 10.5) {};
		\node [style=none] (80) at (-1.5, 12) {};
		\node [style=none] (81) at (-2, 12) {};
	\end{pgfonlayer}
	\begin{pgfonlayer}{edgelayer}
		\draw (75) to (74);
		\draw (74) to (73);
		\draw (81.center) to (73);
		\draw (73) to (78.center);
		\draw (79.center) to (74);
		\draw (74) to (80.center);
		\draw (76.center) to (75);
		\draw (75) to (77.center);
	\end{pgfonlayer}
\end{tikzpicture}
\right\rrbracket_{\hat \TOF}\right\rrbracket_{\ZXA}
&=
\left\llbracket
\begin{tikzpicture}[tikzfig]
	\begin{pgfonlayer}{nodelayer}
		\node [style=X] (74) at (-2, 11) {};
		\node [style=X] (75) at (-1, 11) {};
		\node [style=none] (76) at (-1.5, 12) {};
		\node [style=Z] (77) at (-0.5, 13) {};
		\node [style=none] (78) at (-0.5, 10.5) {};
		\node [style=none] (79) at (-0.5, 13.75) {};
		\node [style=none] (80) at (-1, 13.75) {};
		\node [style=none] (81) at (-2, 13.75) {};
		\node [style=none] (82) at (-2, 10.5) {};
		\node [style=none] (83) at (-1, 10.5) {};
		\node [style=andin] (84) at (-1.5, 12) {};
	\end{pgfonlayer}
	\begin{pgfonlayer}{edgelayer}
		\draw (77) to (79.center);
		\draw (77) to (78.center);
		\draw (83.center) to (75);
		\draw (75) to (80.center);
		\draw (81.center) to (74);
		\draw (74) to (82.center);
		\draw (74) to (76.center);
		\draw [in=180, out=90] (76.center) to (77);
		\draw (76.center) to (75);
	\end{pgfonlayer}
\end{tikzpicture}
\right\rrbracket_{\ZXA}
=
\begin{tikzpicture}[tikzfig]
	\begin{pgfonlayer}{nodelayer}
		\node [style=dot] (75) at (-2, 12) {};
		\node [style=dot] (76) at (-1, 12) {};
		\node [style=oplus] (77) at (-0.25, 12) {};
		\node [style=X] (78) at (-2, 12.5) {};
		\node [style=X] (79) at (-1, 12.5) {};
		\node [style=oplus] (80) at (-0.25, 12.5) {};
		\node [style=dot] (81) at (0.25, 12.5) {};
		\node [style=X] (82) at (0.25, 12) {};
		\node [style=none] (83) at (0.25, 13) {};
		\node [style=dot] (84) at (-2.5, 11.5) {};
		\node [style=oplus] (85) at (-2, 11.5) {};
		\node [style=zeroin] (86) at (-2, 11) {};
		\node [style=oplus] (87) at (-1, 11.5) {};
		\node [style=dot] (88) at (-1.5, 11.5) {};
		\node [style=zeroin] (89) at (-1, 11) {};
		\node [style=none] (90) at (-1.5, 13) {};
		\node [style=none] (91) at (-2.5, 13) {};
		\node [style=none] (92) at (-1.5, 10.5) {};
		\node [style=none] (93) at (-2.5, 10.5) {};
		\node [style=none] (94) at (-0.25, 10.5) {};
		\node [style=zeroout] (95) at (-0.25, 13) {};
	\end{pgfonlayer}
	\begin{pgfonlayer}{edgelayer}
		\draw (83.center) to (81);
		\draw (81) to (80);
		\draw (80) to (77);
		\draw (82) to (81);
		\draw (77) to (76);
		\draw (76) to (75);
		\draw (75) to (78);
		\draw (79) to (76);
		\draw (85) to (86);
		\draw (85) to (84);
		\draw (87) to (89);
		\draw (87) to (88);
		\draw (76) to (87);
		\draw (85) to (75);
		\draw (91.center) to (84);
		\draw (84) to (93.center);
		\draw (92.center) to (88);
		\draw (88) to (90.center);
		\draw (95) to (80);
		\draw (77) to (94.center);
	\end{pgfonlayer}
\end{tikzpicture}
\eq{unit}
\begin{tikzpicture}[tikzfig]
	\begin{pgfonlayer}{nodelayer}
		\node [style=dot] (76) at (-2, 12) {};
		\node [style=dot] (77) at (-1, 12) {};
		\node [style=oplus] (78) at (-0.25, 12) {};
		\node [style=X] (79) at (-2, 12.5) {};
		\node [style=X] (80) at (-1, 12.5) {};
		\node [style=none] (81) at (-0.25, 13) {};
		\node [style=dot] (82) at (-2.5, 11.5) {};
		\node [style=oplus] (83) at (-2, 11.5) {};
		\node [style=zeroin] (84) at (-2, 11) {};
		\node [style=oplus] (85) at (-1, 11.5) {};
		\node [style=dot] (86) at (-1.5, 11.5) {};
		\node [style=zeroin] (87) at (-1, 11) {};
		\node [style=none] (88) at (-1.5, 13) {};
		\node [style=none] (89) at (-2.5, 13) {};
		\node [style=none] (90) at (-1.5, 10.5) {};
		\node [style=none] (91) at (-2.5, 10.5) {};
		\node [style=none] (92) at (-0.25, 10.5) {};
	\end{pgfonlayer}
	\begin{pgfonlayer}{edgelayer}
		\draw (78) to (77);
		\draw (77) to (76);
		\draw (76) to (79);
		\draw (80) to (77);
		\draw (83) to (84);
		\draw (83) to (82);
		\draw (85) to (87);
		\draw (85) to (86);
		\draw (77) to (85);
		\draw (83) to (76);
		\draw (89.center) to (82);
		\draw (82) to (91.center);
		\draw (90.center) to (86);
		\draw (86) to (88.center);
		\draw (78) to (92.center);
		\draw (81.center) to (78);
	\end{pgfonlayer}
\end{tikzpicture}\\
&\eq{Lem. \ref{lemma:Iwama}}
\begin{tikzpicture}[tikzfig]
	\begin{pgfonlayer}{nodelayer}
		\node [style=dot] (77) at (-2, 12.5) {};
		\node [style=dot] (78) at (-1, 12.5) {};
		\node [style=oplus] (79) at (-0.25, 12.5) {};
		\node [style=X] (80) at (-2, 13.5) {};
		\node [style=X] (81) at (-1, 13.5) {};
		\node [style=none] (82) at (-0.25, 14) {};
		\node [style=dot] (83) at (-2.5, 13) {};
		\node [style=oplus] (84) at (-2, 13) {};
		\node [style=zeroin] (85) at (-2, 11) {};
		\node [style=oplus] (86) at (-1, 11.5) {};
		\node [style=dot] (87) at (-1.5, 11.5) {};
		\node [style=zeroin] (88) at (-1, 11) {};
		\node [style=none] (89) at (-1.5, 14) {};
		\node [style=none] (90) at (-2.5, 14) {};
		\node [style=none] (91) at (-1.5, 10.5) {};
		\node [style=none] (92) at (-2.5, 10.5) {};
		\node [style=none] (93) at (-0.25, 10.5) {};
		\node [style=dot] (94) at (-2.5, 12) {};
		\node [style=dot] (95) at (-1, 12) {};
		\node [style=oplus] (96) at (-0.25, 12) {};
	\end{pgfonlayer}
	\begin{pgfonlayer}{edgelayer}
		\draw (79) to (78);
		\draw (78) to (77);
		\draw (77) to (80);
		\draw (81) to (78);
		\draw (84) to (85);
		\draw (84) to (83);
		\draw (86) to (88);
		\draw (86) to (87);
		\draw (78) to (86);
		\draw (84) to (77);
		\draw (90.center) to (83);
		\draw (83) to (92.center);
		\draw (91.center) to (87);
		\draw (87) to (89.center);
		\draw (79) to (93.center);
		\draw (82.center) to (79);
		\draw (96) to (95);
		\draw (95) to (94);
	\end{pgfonlayer}
\end{tikzpicture}
\eq{\ref{TOF.2}}
\begin{tikzpicture}[tikzfig]
	\begin{pgfonlayer}{nodelayer}
		\node [style=X] (78) at (-2, 13) {};
		\node [style=X] (79) at (-1, 13) {};
		\node [style=none] (80) at (-0.25, 13.5) {};
		\node [style=dot] (81) at (-2.5, 12.5) {};
		\node [style=oplus] (82) at (-2, 12.5) {};
		\node [style=zeroin] (83) at (-2, 11) {};
		\node [style=oplus] (84) at (-1, 11.5) {};
		\node [style=dot] (85) at (-1.5, 11.5) {};
		\node [style=zeroin] (86) at (-1, 11) {};
		\node [style=none] (87) at (-1.5, 13.5) {};
		\node [style=none] (88) at (-2.5, 13.5) {};
		\node [style=none] (89) at (-1.5, 10.5) {};
		\node [style=none] (90) at (-2.5, 10.5) {};
		\node [style=none] (91) at (-0.25, 10.5) {};
		\node [style=dot] (92) at (-2.5, 12) {};
		\node [style=dot] (93) at (-1, 12) {};
		\node [style=oplus] (94) at (-0.25, 12) {};
	\end{pgfonlayer}
	\begin{pgfonlayer}{edgelayer}
		\draw (82) to (83);
		\draw (82) to (81);
		\draw (84) to (86);
		\draw (84) to (85);
		\draw (88.center) to (81);
		\draw (81) to (90.center);
		\draw (89.center) to (85);
		\draw (85) to (87.center);
		\draw (94) to (93);
		\draw (93) to (92);
		\draw (80.center) to (91.center);
		\draw (84) to (79);
		\draw (78) to (82);
	\end{pgfonlayer}
\end{tikzpicture}
\eq{unit}
\begin{tikzpicture}[tikzfig]
	\begin{pgfonlayer}{nodelayer}
		\node [style=X] (79) at (-1, 13) {};
		\node [style=none] (80) at (-0.25, 13.5) {};
		\node [style=oplus] (81) at (-1, 11.5) {};
		\node [style=dot] (82) at (-1.5, 11.5) {};
		\node [style=zeroin] (83) at (-1, 11) {};
		\node [style=none] (84) at (-1.5, 13.5) {};
		\node [style=none] (85) at (-2, 13.5) {};
		\node [style=none] (86) at (-1.5, 10.5) {};
		\node [style=none] (87) at (-2, 10.5) {};
		\node [style=none] (88) at (-0.25, 10.5) {};
		\node [style=dot] (89) at (-2, 12) {};
		\node [style=dot] (90) at (-1, 12) {};
		\node [style=oplus] (91) at (-0.25, 12) {};
	\end{pgfonlayer}
	\begin{pgfonlayer}{edgelayer}
		\draw (81) to (83);
		\draw (81) to (82);
		\draw (86.center) to (82);
		\draw (82) to (84.center);
		\draw (91) to (90);
		\draw (90) to (89);
		\draw (80.center) to (88.center);
		\draw (81) to (79);
		\draw (85.center) to (87.center);
	\end{pgfonlayer}
\end{tikzpicture}\\
&\eq{Lem.  \ref{lemma:Iwama}}
\begin{tikzpicture}[tikzfig]
	\begin{pgfonlayer}{nodelayer}
		\node [style=X] (80) at (-1, 13) {};
		\node [style=none] (81) at (-0.25, 13.5) {};
		\node [style=oplus] (82) at (-1, 12.5) {};
		\node [style=dot] (83) at (-1.5, 12.5) {};
		\node [style=zeroin] (84) at (-1, 11) {};
		\node [style=none] (85) at (-1.5, 13.5) {};
		\node [style=none] (86) at (-2, 13.5) {};
		\node [style=none] (87) at (-1.5, 10.5) {};
		\node [style=none] (88) at (-2, 10.5) {};
		\node [style=none] (89) at (-0.25, 10.5) {};
		\node [style=dot] (90) at (-2, 12) {};
		\node [style=dot] (91) at (-1, 12) {};
		\node [style=oplus] (92) at (-0.25, 12) {};
		\node [style=dot] (93) at (-2, 11.5) {};
		\node [style=dot] (94) at (-1.5, 11.5) {};
		\node [style=oplus] (95) at (-0.25, 11.5) {};
	\end{pgfonlayer}
	\begin{pgfonlayer}{edgelayer}
		\draw (82) to (84);
		\draw (82) to (83);
		\draw (87.center) to (83);
		\draw (83) to (85.center);
		\draw (92) to (91);
		\draw (91) to (90);
		\draw (81.center) to (89.center);
		\draw (82) to (80);
		\draw (86.center) to (88.center);
		\draw (95) to (93);
	\end{pgfonlayer}
\end{tikzpicture}
\eq{\ref{TOF.2}}
\begin{tikzpicture}[tikzfig]
	\begin{pgfonlayer}{nodelayer}
		\node [style=X] (81) at (-1, 12.5) {};
		\node [style=none] (82) at (-0.25, 13) {};
		\node [style=oplus] (83) at (-1, 12) {};
		\node [style=dot] (84) at (-1.5, 12) {};
		\node [style=zeroin] (85) at (-1, 11) {};
		\node [style=none] (86) at (-1.5, 13) {};
		\node [style=none] (87) at (-2, 13) {};
		\node [style=none] (88) at (-1.5, 10.5) {};
		\node [style=none] (89) at (-2, 10.5) {};
		\node [style=none] (90) at (-0.25, 10.5) {};
		\node [style=dot] (91) at (-2, 11.5) {};
		\node [style=dot] (92) at (-1.5, 11.5) {};
		\node [style=oplus] (93) at (-0.25, 11.5) {};
	\end{pgfonlayer}
	\begin{pgfonlayer}{edgelayer}
		\draw (83) to (85);
		\draw (83) to (84);
		\draw (88.center) to (84);
		\draw (84) to (86.center);
		\draw (82.center) to (90.center);
		\draw (83) to (81);
		\draw (87.center) to (89.center);
		\draw (93) to (91);
	\end{pgfonlayer}
\end{tikzpicture}
\eq{unit}
\begin{tikzpicture}[tikzfig]
	\begin{pgfonlayer}{nodelayer}
		\node [style=dot] (82) at (-2, 11.25) {};
		\node [style=dot] (83) at (-1.5, 11.25) {};
		\node [style=oplus] (84) at (-1, 11.25) {};
		\node [style=none] (85) at (-1, 12) {};
		\node [style=none] (86) at (-1, 10.5) {};
		\node [style=none] (87) at (-2, 10.5) {};
		\node [style=none] (88) at (-1.5, 10.5) {};
		\node [style=none] (89) at (-1.5, 12) {};
		\node [style=none] (90) at (-2, 12) {};
	\end{pgfonlayer}
	\begin{pgfonlayer}{edgelayer}
		\draw (84) to (83);
		\draw (83) to (82);
		\draw (90.center) to (82);
		\draw (82) to (87.center);
		\draw (88.center) to (83);
		\draw (83) to (89.center);
		\draw (85.center) to (84);
		\draw (84) to (86.center);
	\end{pgfonlayer}
\end{tikzpicture}

\end{align*}
\end{proof}


Recall the following proposition:

\begin{proposition}\cite[Prop. 2.6]{bruni}\footnote{In \cite{bruni}, they do not prove this equivalence is monoidal, but it is an obvious corollary. They also do not consider the finite case.}
The category $\Span^\sim(\FinOrd)$ equipped with the Cartesian product is monoidally equivalent to the category of (finite)  matrices over the natural numbers and the Kronecker product.
\end{proposition}

Thus,

\begin{corollary}
$\ZXA$ is complete for the prop of $2^n\times 2^m$ matrices over the natural numbers.
\end{corollary} 


\section{Conclusion}
There are various other directions which could be pursued.  One could also ask if there is a normal form for $\ZXA$ induced by the presentation in terms of distributive laws and monoid maps, using the correspondence between strict factorization systems and distributive laws in spans \cite{rosebrugh2002distributive}. %Second, one could compute various other fragments of the ZX-calculus by performing a two way translation between the (co)unital completion of some other discrete inverse category.  %The category $\CNOT$ comes to mind, where presumably, this would yield affine multirelations, generalizing the affine relations of \cite[\S 4.3]{piedeleu}.
\nocite{piedeleu} 
% One could also apply this construction to ``infinite dimensional'' discrete inverse categories, to obtain hypergraph categories which can not be interpreted into Hilbert spaces.
It would also be interesting to investigate the 2-categorical structure of $\ZXA$; presenting the corresponding category of relations as a Frobenius theory \cite{functorial} using the partial order enrichment of $\TOF$.

Another immediate direction would be to add the white $\pi$ phase to $\ZXA$ to obtain an approximately universal graphical calculus for quantum computing using only distributive laws and monoid maps.  In such a fragment, one could construct the {\sf and} gate for the $X$ basis; perhaps expanding the table of distributive laws in Figure \ref{fig:table} to be complete for an approximately universal fragment of quantum computing, furthering the general programme of \cite{ihpub,duncan} decomposing circuits using distributive laws.  This  approach is contrasted to considering H-boxes as primitives, as in the phase-free fragment of the $\ZH$-calculus \cite{zhpi}---in $\ZXA$+the white $\pi$ phase, the unnormalized Hadamard gate is derived. Perhaps proving the minimality of the axioms using this presentation might be easier, although we do not prove minimality in this paper.

It would also be interesting to investigate the connection to the $\ZH$-calculus and triangle fragments of the $\ZX$-calculus; in particular, in regard to natural number labelled H-boxes, as in \cite{natspiders}.  %The triangle, and the natural number labeled H boxes are given below:
These gates can be represented in string diagrams. The diagram of the triangle can be interpreted as the assertion  $x\wedge \neg y =  \bot$ which is equivalent to the material implication  $ x \Rightarrow y$.
\begin{figure}[H]

$$
\begin{tikzpicture}[tikzfig]
	\begin{pgfonlayer}{nodelayer}
		\node [style=none] (83) at (0.75, 11.25) {};
		\node [style=none] (84) at (0.75, 10.5) {};
		\node [style=none] (85) at (0.75, 12) {};
		\node [style=triflip] (86) at (0.75, 11.25) {};
	\end{pgfonlayer}
	\begin{pgfonlayer}{edgelayer}
		\draw (85.center) to (83.center);
		\draw (83.center) to (84.center);
	\end{pgfonlayer}
\end{tikzpicture}
:=
\begin{tikzpicture}[tikzfig]
	\begin{pgfonlayer}{nodelayer}
		\node [style=none] (84) at (0.75, 11.5) {};
		\node [style=none] (85) at (1.25, 11.5) {};
		\node [style=Z] (86) at (0.75, 12.25) {};
		\node [style=none] (87) at (0.75, 10.5) {};
		\node [style=andin] (88) at (0.75, 11.5) {};
		\node [style=none] (89) at (1.25, 13) {};
		\node [style=Z] (90) at (1.25, 12.25) {$\pi$};
	\end{pgfonlayer}
	\begin{pgfonlayer}{edgelayer}
		\draw [style=simple] (86) to (84.center);
		\draw [style=simple, in=-75, out=-90, looseness=2.75] (85.center) to (84.center);
		\draw [style=simple] (87.center) to (84.center);
		\draw (89.center) to (90);
		\draw (90) to (85.center);
	\end{pgfonlayer}
\end{tikzpicture}
\hspace*{1cm}
\begin{tikzpicture}[tikzfig]
	\begin{pgfonlayer}{nodelayer}
		\node [style=H] (85) at (1, 11.25) {$n$};
		\node [style=none] (86) at (1, 10.5) {};
		\node [style=none] (87) at (1, 12) {};
	\end{pgfonlayer}
	\begin{pgfonlayer}{edgelayer}
		\draw (87.center) to (85);
		\draw (85) to (86.center);
	\end{pgfonlayer}
\end{tikzpicture}
:=
\begin{tikzpicture}[tikzfig]
	\begin{pgfonlayer}{nodelayer}
		\node [style=none] (1) at (0.75, -0.5) {};
		\node [style=none] (2) at (1.25, -0.5) {};
		\node [style=Z] (3) at (0.75, 2.75) {$\pi$};
		\node [style=none] (4) at (0.75, -1.5) {};
		\node [style=andin] (5) at (0.75, -0.5) {};
		\node [style=none] (6) at (1.25, 3.25) {};
		\node [style=triflip] (7) at (0.75, 2) {};
		\node [style=triflip] (8) at (0.75, 1) {};
		\node [style=none] (9) at (0.5, 1.5) {$n$};
		\node [style=Z] (10) at (0.75, 0.25) {$\pi$};
	\end{pgfonlayer}
	\begin{pgfonlayer}{edgelayer}
		\draw [style=simple, in=-75, out=-90, looseness=2.75] (2.center) to (1.center);
		\draw [style=simple] (4.center) to (1.center);
		\draw [style=dotted] (7) to (8);
		\draw (7) to (3);
		\draw (8) to (1.center);
		\draw [style=simple] (2.center) to (6.center);
	\end{pgfonlayer}
\end{tikzpicture}
$$

\caption{Triangles and H-boxes in \texorpdfstring{$\ZXA$}{ZX\&}, for \texorpdfstring{$n\in \N$}{n a natural number}.}
\label{fig:gens}
\end{figure}





%
%\nocite{coecke2008classical}
%\nocite{cnot}
%\nocite{tof}
%\nocite{Cole}
%\nocite{elltwo}
%%\nocite{sam}
%\nocite{coecke2017two}
%\nocite{carboni}
%\nocite{butz}
%\nocite{pqp}
%\nocite{lack2004composing}

\newpage


%\appendix 




\section{Identities of \texorpdfstring{$\TOF$}{TOF}}
\label{sec:tof}

Define the category $\TOF$ \cite{tof} to be the PROP, generated by the 1 ancillary bits $| 1\rangle$ and $\langle 1|$ as well as the Toffoli gate, satisfying the identities given in Figure \ref{fig:TOF}.
		

\begin{figure}[h]
\noindent
\scalebox{1.0}{%
\vbox{%
\begin{mdframed}
\begin{multicols}{2}
\begin{enumerate}[label={\bf [TOF.\arabic*]}, ref={\bf [TOF.\arabic*]}, wide = 0pt, leftmargin = 2em]
\item
\label{TOF.1}
{\hfil
$
\begin{tabular}{c}
\begin{tikzpicture}[tikzfig]
	\begin{pgfonlayer}{nodelayer}
		\node [style=nothing] (2) at (0, 1.5) {};
		\node [style=nothing] (3) at (-0.5, 1.5) {};
		\node [style=oplus] (4) at (0, 2) {};
		\node [style=dot] (5) at (-0.5, 2) {};
		\node [style=dot] (6) at (-1, 2) {};
		\node [style=onein] (7) at (-1, 1.5) {};
		\node [style=nothing] (8) at (-1, 2.5) {};
		\node [style=nothing] (9) at (-0.5, 2.5) {};
		\node [style=nothing] (10) at (0, 2.5) {};
	\end{pgfonlayer}
	\begin{pgfonlayer}{edgelayer}
		\draw (7) to (6);
		\draw (6) to (8);
		\draw (9) to (5);
		\draw (3) to (5);
		\draw (2) to (4);
		\draw (4) to (10);
		\draw (4) to (5);
		\draw (5) to (6);
	\end{pgfonlayer}
\end{tikzpicture}
=
\begin{tikzpicture}[tikzfig]
	\begin{pgfonlayer}{nodelayer}
		\node [style=nothing] (3) at (0, 1.5) {};
		\node [style=nothing] (4) at (-0.5, 1.5) {};
		\node [style=oplus] (5) at (0, 2) {};
		\node [style=dot] (6) at (-0.5, 2) {};
		\node [style=onein] (7) at (-1, 2) {};
		\node [style=nothing] (8) at (-1, 2.5) {};
		\node [style=nothing] (9) at (-0.5, 2.5) {};
		\node [style=nothing] (10) at (0, 2.5) {};
	\end{pgfonlayer}
	\begin{pgfonlayer}{edgelayer}
		\draw (4) to (6);
		\draw (3) to (5);
		\draw (5) to (6);
		\draw (9) to (6);
		\draw (7) to (8);
		\draw (5) to (10);
	\end{pgfonlayer}
\end{tikzpicture}
\\
{}\\
\begin{tikzpicture}[tikzfig]
	\begin{pgfonlayer}{nodelayer}
		\node [style=nothing] (0) at (0, 1.5) {};
		\node [style=nothing] (1) at (-0.5, 1.5) {};
		\node [style=oplus] (2) at (0, 1) {};
		\node [style=dot] (3) at (-0.5, 1) {};
		\node [style=dot] (4) at (-1, 1) {};
		\node [style=onein] (5) at (-1, 1.5) {};
		\node [style=nothing] (6) at (-1, 0.5) {};
		\node [style=nothing] (7) at (-0.5, 0.5) {};
		\node [style=nothing] (8) at (0, 0.5) {};
	\end{pgfonlayer}
	\begin{pgfonlayer}{edgelayer}
		\draw (5) to (4);
		\draw (4) to (6);
		\draw (7) to (3);
		\draw (1) to (3);
		\draw (0) to (2);
		\draw (2) to (8);
		\draw (2) to (3);
		\draw (3) to (4);
	\end{pgfonlayer}
\end{tikzpicture}
=
\begin{tikzpicture}[tikzfig]
	\begin{pgfonlayer}{nodelayer}
		\node [style=nothing] (1) at (0, 1) {};
		\node [style=nothing] (2) at (-0.5, 1) {};
		\node [style=oplus] (3) at (0, 0.5) {};
		\node [style=dot] (4) at (-0.5, 0.5) {};
		\node [style=onein] (5) at (-1, 0.5) {};
		\node [style=nothing] (6) at (-1, 0) {};
		\node [style=nothing] (7) at (-0.5, 0) {};
		\node [style=nothing] (8) at (0, 0) {};
	\end{pgfonlayer}
	\begin{pgfonlayer}{edgelayer}
		\draw (2) to (4);
		\draw (1) to (3);
		\draw (3) to (4);
		\draw (7) to (4);
		\draw (5) to (6);
		\draw (3) to (8);
	\end{pgfonlayer}
\end{tikzpicture}
\end{tabular}
$}


\item
\label{TOF.2}
{\hfil
$
\begin{tabular}{c}
\begin{tikzpicture}[tikzfig]
	\begin{pgfonlayer}{nodelayer}
		\node [style=nothing] (2) at (-1.25, 0) {};
		\node [style=nothing] (3) at (-0.75, 0) {};
		\node [style=nothing] (4) at (-1.75, 2) {};
		\node [style=nothing] (5) at (-1.25, 2) {};
		\node [style=nothing] (6) at (-0.75, 2) {};
		\node [style=dot] (7) at (-1.75, 1) {};
		\node [style=dot] (8) at (-1.25, 1) {};
		\node [style=oplus] (9) at (-0.75, 1) {};
		\node [style=zeroin] (10) at (-1.75, 0) {};
	\end{pgfonlayer}
	\begin{pgfonlayer}{edgelayer}
		\draw (7) to (4);
		\draw (5) to (8);
		\draw (8) to (2);
		\draw (3) to (9);
		\draw (9) to (6);
		\draw (9) to (8);
		\draw (8) to (7);
		\draw (10) to (7);
	\end{pgfonlayer}
\end{tikzpicture}
=
\begin{tikzpicture}[tikzfig]
	\begin{pgfonlayer}{nodelayer}
		\node [style=nothing] (3) at (-1.25, 0) {};
		\node [style=nothing] (4) at (-0.75, 0) {};
		\node [style=nothing] (5) at (-1.75, 1.5) {};
		\node [style=nothing] (6) at (-1.25, 1.5) {};
		\node [style=nothing] (7) at (-0.75, 1.5) {};
		\node [style=zeroin] (8) at (-1.75, 0) {};
	\end{pgfonlayer}
	\begin{pgfonlayer}{edgelayer}
		\draw (8) to (5);
		\draw (3) to (6);
		\draw (4) to (7);
	\end{pgfonlayer}
\end{tikzpicture}\\
{}\\
\begin{tikzpicture}[tikzfig]
	\begin{pgfonlayer}{nodelayer}
		\node [style=nothing] (4) at (-1.25, 2) {};
		\node [style=nothing] (5) at (-0.75, 2) {};
		\node [style=nothing] (6) at (-1.75, 0) {};
		\node [style=nothing] (7) at (-1.25, 0) {};
		\node [style=nothing] (8) at (-0.75, 0) {};
		\node [style=dot] (9) at (-1.75, 1) {};
		\node [style=dot] (10) at (-1.25, 1) {};
		\node [style=oplus] (11) at (-0.75, 1) {};
		\node [style=zeroin] (12) at (-1.75, 2) {};
	\end{pgfonlayer}
	\begin{pgfonlayer}{edgelayer}
		\draw (9) to (6);
		\draw (7) to (10);
		\draw (10) to (4);
		\draw (5) to (11);
		\draw (11) to (8);
		\draw (11) to (10);
		\draw (10) to (9);
		\draw (12) to (9);
	\end{pgfonlayer}
\end{tikzpicture}
=
\begin{tikzpicture}[tikzfig]
	\begin{pgfonlayer}{nodelayer}
		\node [style=nothing] (0) at (-1.25, 2) {};
		\node [style=nothing] (1) at (-0.75, 2) {};
		\node [style=zeroout] (2) at (-1.75, 0.5) {};
		\node [style=nothing] (3) at (-1.25, 0.5) {};
		\node [style=nothing] (4) at (-0.75, 0.5) {};
		\node [style=nothing] (5) at (-1.75, 2) {};
	\end{pgfonlayer}
	\begin{pgfonlayer}{edgelayer}
		\draw (5) to (2);
		\draw (0) to (3);
		\draw (1) to (4);
	\end{pgfonlayer}
\end{tikzpicture}
\end{tabular}
$}

\item
\label{TOF.3}
{\hfil
$
\begin{tikzpicture}[tikzfig]
	\begin{pgfonlayer}{nodelayer}
		\node [style=nothing] (0) at (-0.5, 0.5) {};
		\node [style=nothing] (1) at (0, 0.5) {};
		\node [style=nothing] (2) at (-1, 0.5) {};
		\node [style=nothing] (3) at (-1.5, 0.5) {};
		\node [style=nothing] (4) at (-2, 0.5) {};
		\node [style=dot] (5) at (-1.5, 1) {};
		\node [style=oplus] (6) at (-1, 1) {};
		\node [style=oplus] (7) at (-1, 1.5) {};
		\node [style=dot] (8) at (-0.5, 1.5) {};
		\node [style=dot] (9) at (-2, 1) {};
		\node [style=dot] (10) at (0, 1.5) {};
		\node [style=nothing] (11) at (-0.5, 2) {};
		\node [style=nothing] (12) at (-1.5, 2) {};
		\node [style=nothing] (13) at (-2, 2) {};
		\node [style=nothing] (14) at (0, 2) {};
		\node [style=nothing] (15) at (-1, 2) {};
	\end{pgfonlayer}
	\begin{pgfonlayer}{edgelayer}
		\draw (4) to (9);
		\draw (9) to (13);
		\draw (3) to (5);
		\draw (5) to (12);
		\draw (2) to (6);
		\draw (6) to (7);
		\draw (7) to (15);
		\draw (0) to (8);
		\draw (8) to (11);
		\draw (1) to (10);
		\draw (10) to (14);
		\draw (10) to (8);
		\draw (8) to (7);
		\draw (6) to (5);
		\draw (5) to (9);
	\end{pgfonlayer}
\end{tikzpicture}
=
\begin{tikzpicture}[tikzfig]
	\begin{pgfonlayer}{nodelayer}
		\node [style=nothing] (1) at (-0.5, 0) {};
		\node [style=nothing] (2) at (0, 0) {};
		\node [style=nothing] (3) at (-1, 0) {};
		\node [style=nothing] (4) at (-1.5, 0) {};
		\node [style=nothing] (5) at (-2, 0) {};
		\node [style=dot] (6) at (-1.5, 1) {};
		\node [style=dot] (7) at (-0.5, 0.5) {};
		\node [style=dot] (8) at (-2, 1) {};
		\node [style=dot] (9) at (0, 0.5) {};
		\node [style=nothing] (10) at (-0.5, 1.5) {};
		\node [style=nothing] (11) at (-1.5, 1.5) {};
		\node [style=nothing] (12) at (-2, 1.5) {};
		\node [style=nothing] (13) at (0, 1.5) {};
		\node [style=nothing] (14) at (-1, 1.5) {};
		\node [style=oplus] (15) at (-1, 1) {};
		\node [style=oplus] (16) at (-1, 0.5) {};
	\end{pgfonlayer}
	\begin{pgfonlayer}{edgelayer}
		\draw (5) to (8);
		\draw (8) to (12);
		\draw (4) to (6);
		\draw (6) to (11);
		\draw (1) to (7);
		\draw (7) to (10);
		\draw (2) to (9);
		\draw (9) to (13);
		\draw (9) to (7);
		\draw (6) to (8);
		\draw (3) to (16);
		\draw (16) to (15);
		\draw (15) to (14);
		\draw (15) to (6);
		\draw (7) to (16);
	\end{pgfonlayer}
\end{tikzpicture}
$}


\item
\label{TOF.4}
{\hfil
$
\begin{tikzpicture}[tikzfig]
	\begin{pgfonlayer}{nodelayer}
		\node [style=nothing] (2) at (-0.5, 0) {};
		\node [style=nothing] (3) at (0, 0) {};
		\node [style=nothing] (4) at (-1, 0) {};
		\node [style=nothing] (5) at (-1.5, 0) {};
		\node [style=nothing] (6) at (-2, 0) {};
		\node [style=dot] (7) at (-1.5, 0.5) {};
		\node [style=dot] (8) at (-1, 0.5) {};
		\node [style=dot] (9) at (-1, 1) {};
		\node [style=dot] (10) at (-0.5, 1) {};
		\node [style=oplus] (11) at (-2, 0.5) {};
		\node [style=oplus] (12) at (0, 1) {};
		\node [style=nothing] (13) at (-0.5, 1.5) {};
		\node [style=nothing] (14) at (-1.5, 1.5) {};
		\node [style=nothing] (15) at (-2, 1.5) {};
		\node [style=nothing] (16) at (0, 1.5) {};
		\node [style=nothing] (17) at (-1, 1.5) {};
	\end{pgfonlayer}
	\begin{pgfonlayer}{edgelayer}
		\draw (6) to (11);
		\draw (11) to (15);
		\draw (5) to (7);
		\draw (7) to (14);
		\draw (4) to (8);
		\draw (8) to (9);
		\draw (9) to (17);
		\draw (2) to (10);
		\draw (10) to (13);
		\draw (3) to (12);
		\draw (12) to (16);
		\draw (12) to (10);
		\draw (10) to (9);
		\draw (8) to (7);
		\draw (7) to (11);
	\end{pgfonlayer}
\end{tikzpicture}
=
\begin{tikzpicture}[tikzfig]
	\begin{pgfonlayer}{nodelayer}
		\node [style=nothing] (3) at (-0.5, 0) {};
		\node [style=nothing] (4) at (0, 0) {};
		\node [style=nothing] (5) at (-1, 0) {};
		\node [style=nothing] (6) at (-1.5, 0) {};
		\node [style=nothing] (7) at (-2, 0) {};
		\node [style=dot] (8) at (-1.5, 1) {};
		\node [style=dot] (9) at (-0.5, 0.5) {};
		\node [style=oplus] (10) at (-2, 1) {};
		\node [style=oplus] (11) at (0, 0.5) {};
		\node [style=nothing] (12) at (-0.5, 1.5) {};
		\node [style=nothing] (13) at (-1.5, 1.5) {};
		\node [style=nothing] (14) at (-2, 1.5) {};
		\node [style=nothing] (15) at (0, 1.5) {};
		\node [style=nothing] (16) at (-1, 1.5) {};
		\node [style=dot] (17) at (-1, 1) {};
		\node [style=dot] (18) at (-1, 0.5) {};
	\end{pgfonlayer}
	\begin{pgfonlayer}{edgelayer}
		\draw (7) to (10);
		\draw (10) to (14);
		\draw (6) to (8);
		\draw (8) to (13);
		\draw (3) to (9);
		\draw (9) to (12);
		\draw (4) to (11);
		\draw (11) to (15);
		\draw (11) to (9);
		\draw (8) to (10);
		\draw (5) to (18);
		\draw (18) to (17);
		\draw (17) to (16);
		\draw (17) to (8);
		\draw (9) to (18);
	\end{pgfonlayer}
\end{tikzpicture}
$}

\item
\label{TOF.5}
{\hfil
$
\begin{tikzpicture}[tikzfig]
	\begin{pgfonlayer}{nodelayer}
		\node [style=nothing] (4) at (-1, 2) {};
		\node [style=nothing] (5) at (-0.5, 2) {};
		\node [style=nothing] (6) at (-1.5, 2) {};
		\node [style=nothing] (7) at (-2, 2) {};
		\node [style=nothing] (8) at (-1, 3.5) {};
		\node [style=nothing] (9) at (-1.5, 3.5) {};
		\node [style=nothing] (10) at (-2, 3.5) {};
		\node [style=nothing] (11) at (-0.5, 3.5) {};
		\node [style=oplus] (12) at (-2, 2.5) {};
		\node [style=oplus] (13) at (-0.5, 3) {};
		\node [style=dot] (14) at (-1.5, 2.5) {};
		\node [style=dot] (15) at (-1, 2.5) {};
		\node [style=dot] (16) at (-1.5, 3) {};
		\node [style=dot] (17) at (-1, 3) {};
	\end{pgfonlayer}
	\begin{pgfonlayer}{edgelayer}
		\draw (7) to (12);
		\draw (12) to (10);
		\draw (9) to (16);
		\draw (16) to (14);
		\draw (14) to (6);
		\draw (4) to (15);
		\draw (15) to (17);
		\draw (17) to (8);
		\draw (11) to (13);
		\draw (13) to (5);
		\draw (14) to (15);
		\draw (14) to (12);
		\draw (16) to (17);
		\draw (17) to (13);
	\end{pgfonlayer}
\end{tikzpicture}
=
\begin{tikzpicture}[tikzfig]
	\begin{pgfonlayer}{nodelayer}
		\node [style=nothing] (5) at (-1, 2) {};
		\node [style=nothing] (6) at (-0.5, 2) {};
		\node [style=nothing] (7) at (-1.5, 2) {};
		\node [style=nothing] (8) at (-2, 2) {};
		\node [style=nothing] (9) at (-1, 3.5) {};
		\node [style=nothing] (10) at (-1.5, 3.5) {};
		\node [style=nothing] (11) at (-2, 3.5) {};
		\node [style=nothing] (12) at (-0.5, 3.5) {};
		\node [style=oplus] (13) at (-2, 3) {};
		\node [style=dot] (14) at (-1.5, 3) {};
		\node [style=dot] (15) at (-1, 3) {};
		\node [style=oplus] (16) at (-0.5, 2.5) {};
		\node [style=dot] (17) at (-1, 2.5) {};
		\node [style=dot] (18) at (-1.5, 2.5) {};
	\end{pgfonlayer}
	\begin{pgfonlayer}{edgelayer}
		\draw (14) to (15);
		\draw (14) to (13);
		\draw (18) to (17);
		\draw (17) to (16);
		\draw (8) to (13);
		\draw (13) to (11);
		\draw (10) to (14);
		\draw (14) to (18);
		\draw (18) to (7);
		\draw (5) to (17);
		\draw (17) to (15);
		\draw (15) to (9);
		\draw (12) to (16);
		\draw (16) to (6);
	\end{pgfonlayer}
\end{tikzpicture}
$}


\item
\label{TOF.6}
{\hfil
$
\begin{tikzpicture}[tikzfig]
	\begin{pgfonlayer}{nodelayer}
		\node [style=nothing] (6) at (-1, 2) {};
		\node [style=nothing] (7) at (-1.5, 2) {};
		\node [style=nothing] (8) at (-2, 2) {};
		\node [style=nothing] (9) at (-1, 3.5) {};
		\node [style=nothing] (10) at (-1.5, 3.5) {};
		\node [style=nothing] (11) at (-2, 3.5) {};
		\node [style=nothing] (12) at (-0.5, 3.5) {};
		\node [style=oplus] (13) at (-0.5, 2.5) {};
		\node [style=dot] (14) at (-1.5, 3) {};
		\node [style=dot] (15) at (-1, 3) {};
		\node [style=dot] (16) at (-1, 2.5) {};
		\node [style=oplus] (17) at (-0.5, 3) {};
		\node [style=nothing] (18) at (-0.5, 2) {};
		\node [style=dot] (19) at (-2, 2.5) {};
	\end{pgfonlayer}
	\begin{pgfonlayer}{edgelayer}
		\draw (14) to (7);
		\draw (6) to (15);
		\draw (15) to (16);
		\draw (16) to (9);
		\draw (12) to (13);
		\draw (14) to (15);
		\draw (16) to (13);
		\draw (18) to (17);
		\draw (17) to (13);
		\draw (14) to (10);
		\draw (15) to (17);
		\draw (16) to (19);
		\draw (19) to (11);
		\draw (19) to (8);
	\end{pgfonlayer}
\end{tikzpicture}
=
\begin{tikzpicture}[tikzfig]
	\begin{pgfonlayer}{nodelayer}
		\node [style=nothing] (7) at (-1, 2) {};
		\node [style=nothing] (8) at (-1.5, 2) {};
		\node [style=nothing] (9) at (-2, 2) {};
		\node [style=nothing] (10) at (-1, 3.5) {};
		\node [style=nothing] (11) at (-1.5, 3.5) {};
		\node [style=nothing] (12) at (-2, 3.5) {};
		\node [style=nothing] (13) at (-0.5, 3.5) {};
		\node [style=oplus] (14) at (-0.5, 3) {};
		\node [style=dot] (15) at (-1.5, 2.5) {};
		\node [style=dot] (16) at (-1, 2.5) {};
		\node [style=dot] (17) at (-1, 3) {};
		\node [style=oplus] (18) at (-0.5, 2.5) {};
		\node [style=nothing] (19) at (-0.5, 2) {};
		\node [style=dot] (20) at (-2, 3) {};
	\end{pgfonlayer}
	\begin{pgfonlayer}{edgelayer}
		\draw (15) to (8);
		\draw (7) to (16);
		\draw (16) to (17);
		\draw (17) to (10);
		\draw (13) to (14);
		\draw (15) to (16);
		\draw (17) to (14);
		\draw (19) to (18);
		\draw (18) to (14);
		\draw (15) to (11);
		\draw (16) to (18);
		\draw (17) to (20);
		\draw (20) to (12);
		\draw (20) to (9);
	\end{pgfonlayer}
\end{tikzpicture}
$}

\item
\label{TOF.7}
{\hfil
$
\begin{tikzpicture}[tikzfig]
	\begin{pgfonlayer}{nodelayer}
		\node [style=nothing] (8) at (0, 2) {};
		\node [style=nothing] (9) at (-0.5, 2) {};
		\node [style=nothing] (10) at (-0.5, 4.5) {};
		\node [style=nothing] (11) at (0, 4.5) {};
		\node [style=zeroout] (12) at (0.5, 4.5) {};
		\node [style=oplus] (13) at (0.5, 4) {};
		\node [style=dot] (14) at (0, 4) {};
		\node [style=dot] (15) at (-0.5, 2.5) {};
		\node [style=oplus] (16) at (0.5, 2.5) {};
		\node [style=zeroout] (17) at (0.5, 3) {};
		\node [style=onein] (18) at (0.5, 2) {};
		\node [style=onein] (19) at (0.5, 3.5) {};
	\end{pgfonlayer}
	\begin{pgfonlayer}{edgelayer}
		\draw (9) to (15);
		\draw (15) to (10);
		\draw (11) to (14);
		\draw (14) to (8);
		\draw (16) to (17);
		\draw (16) to (15);
		\draw (13) to (12);
		\draw (13) to (14);
		\draw (18) to (16);
		\draw (19) to (13);
	\end{pgfonlayer}
\end{tikzpicture}
=
\begin{tikzpicture}[tikzfig]
	\begin{pgfonlayer}{nodelayer}
		\node [style=nothing] (9) at (0, 2) {};
		\node [style=nothing] (10) at (-0.5, 2) {};
		\node [style=nothing] (11) at (-0.5, 3) {};
		\node [style=nothing] (12) at (0, 3) {};
		\node [style=dot] (13) at (-0.5, 2.5) {};
		\node [style=dot] (14) at (0, 2.5) {};
		\node [style=onein] (15) at (0.5, 2) {};
		\node [style=zeroout] (16) at (0.5, 3) {};
		\node [style=oplus] (17) at (0.5, 2.5) {};
	\end{pgfonlayer}
	\begin{pgfonlayer}{edgelayer}
		\draw (10) to (13);
		\draw (13) to (11);
		\draw (12) to (14);
		\draw (14) to (9);
		\draw (15) to (17);
		\draw (17) to (16);
		\draw (17) to (14);
		\draw (14) to (13);
	\end{pgfonlayer}
\end{tikzpicture}
$}
%
%\item
%\label{TOF.7}
%{\hfil
%$
%\begin{tikzpicture}
%	\begin{pgfonlayer}{nodelayer}
%		\node [style=nothing] (0) at (0, -0) {};
%		\node [style=nothing] (1) at (1.5, -0) {};
%		\node [style=onein] (2) at (0.4, 0.5) {};
%		\node [style=zeroout] (3) at (1.1, 0.5) {};
%	\end{pgfonlayer}
%	\begin{pgfonlayer}{edgelayer}
%		\draw (2) to (3);
%		\draw (0) to (1);
%	\end{pgfonlayer}
%\end{tikzpicture}
%=
%\begin{tikzpicture}
%	\begin{pgfonlayer}{nodelayer}
%		\node [style=nothing] (0) at (0, -0) {};
%		\node [style=nothing] (1) at (1.5, -0) {};
%		\node [style=onein] (2) at (0.4, 0.5) {};
%		\node [style=zeroout] (3) at (1.1, 0.5) {};
%		\node [style=onein] (4) at (1, -0) {};
%		\node [style=oneout] (5) at (0.5000002, -0) {};
%	\end{pgfonlayer}
%	\begin{pgfonlayer}{edgelayer}
%		\draw (2) to (3);
%		\draw (5) to (0);
%		\draw (4) to (1);
%	\end{pgfonlayer}
%\end{tikzpicture}
%$}

\item
\label{TOF.8}
{\hfil
$
\begin{tikzpicture}[tikzfig]
	\begin{pgfonlayer}{nodelayer}
		\node [style=onein] (10) at (0, 2) {};
		\node [style=oneout] (11) at (0, 3) {};
	\end{pgfonlayer}
	\begin{pgfonlayer}{edgelayer}
		\draw (10) to (11);
	\end{pgfonlayer}
\end{tikzpicture}
=
\begin{tikzpicture}[tikzfig]
	\begin{pgfonlayer}{nodelayer}
		\node [style=rn] (11) at (0, 2) {};
		\node [style=rn] (12) at (0, 3) {};
	\end{pgfonlayer}
\end{tikzpicture}
%\hspace*{-.8cm}
%\begin{tikzpicture}[scale=.5]
%\begin{pgfonlayer}{nodelayer}
%\begin{tikzpicture}
%\node[cloud, cloud puffs=15.7,minimum width=3cm, draw,] (cloud) at (0,0) {$1_0$};
%\end{tikzpicture}
%\end{pgfonlayer}
%\begin{pgfonlayer}{edgelayer}
%\end{pgfonlayer}
%\end{tikzpicture}
$}

\item
\label{TOF.9}
{\hfil
$
\begin{tikzpicture}[tikzfig]
	\begin{pgfonlayer}{nodelayer}
		\node [style=nothing] (12) at (-1.75, 2) {};
		\node [style=nothing] (13) at (-1.25, 2) {};
		\node [style=nothing] (14) at (-0.75, 2) {};
		\node [style=dot] (15) at (-1.75, 2.5) {};
		\node [style=dot] (16) at (-1.25, 2.5) {};
		\node [style=oplus] (17) at (-0.75, 2.5) {};
		\node [style=dot] (18) at (-1.75, 3) {};
		\node [style=oplus] (19) at (-0.75, 3) {};
		\node [style=dot] (20) at (-1.25, 3) {};
		\node [style=nothing] (21) at (-1.25, 3.5) {};
		\node [style=nothing] (22) at (-0.75, 3.5) {};
		\node [style=nothing] (23) at (-1.75, 3.5) {};
	\end{pgfonlayer}
	\begin{pgfonlayer}{edgelayer}
		\draw (12) to (15);
		\draw (13) to (16);
		\draw (14) to (17);
		\draw (15) to (16);
		\draw (16) to (17);
		\draw (18) to (20);
		\draw (20) to (19);
		\draw (15) to (18);
		\draw (18) to (23);
		\draw (16) to (20);
		\draw (20) to (21);
		\draw (17) to (19);
		\draw (19) to (22);
	\end{pgfonlayer}
\end{tikzpicture}
=
\begin{tikzpicture}[tikzfig]
	\begin{pgfonlayer}{nodelayer}
		\node [style=nothing] (13) at (-1.75, 2) {};
		\node [style=nothing] (14) at (-1.25, 2) {};
		\node [style=nothing] (15) at (-0.75, 2) {};
		\node [style=nothing] (16) at (-1.25, 3.5) {};
		\node [style=nothing] (17) at (-0.75, 3.5) {};
		\node [style=nothing] (18) at (-1.75, 3.5) {};
	\end{pgfonlayer}
	\begin{pgfonlayer}{edgelayer}
		\draw (13) to (18);
		\draw (14) to (16);
		\draw (15) to (17);
	\end{pgfonlayer}
\end{tikzpicture}
$}

\item
\label{TOF.10}
{\hfil
$
\begin{tikzpicture}[tikzfig]
	\begin{pgfonlayer}{nodelayer}
		\node [style=nothing] (14) at (0, 2) {};
		\node [style=nothing] (15) at (-0.5, 2) {};
		\node [style=nothing] (16) at (-1, 2) {};
		\node [style=nothing] (17) at (-1.5, 2) {};
		\node [style=dot] (18) at (-1, 2.5) {};
		\node [style=dot] (19) at (-0.5, 2.5) {};
		\node [style=oplus] (20) at (0, 2.5) {};
		\node [style=dot] (21) at (-1.5, 3) {};
		\node [style=oplus] (22) at (-0.5, 3) {};
		\node [style=dot] (23) at (-1, 3) {};
		\node [style=dot] (24) at (-1, 3.5) {};
		\node [style=oplus] (25) at (0, 3.5) {};
		\node [style=dot] (26) at (-0.5, 3.5) {};
		\node [style=nothing] (27) at (-1.5, 4) {};
		\node [style=nothing] (28) at (-0.5, 4) {};
		\node [style=nothing] (29) at (-1, 4) {};
		\node [style=nothing] (30) at (0, 4) {};
	\end{pgfonlayer}
	\begin{pgfonlayer}{edgelayer}
		\draw (18) to (19);
		\draw (19) to (20);
		\draw (21) to (23);
		\draw (23) to (22);
		\draw (24) to (26);
		\draw (26) to (25);
		\draw (17) to (21);
		\draw (21) to (27);
		\draw (29) to (24);
		\draw (24) to (23);
		\draw (23) to (18);
		\draw (18) to (16);
		\draw (15) to (19);
		\draw (19) to (22);
		\draw (22) to (26);
		\draw (26) to (28);
		\draw (30) to (25);
		\draw (25) to (20);
		\draw (20) to (14);
	\end{pgfonlayer}
\end{tikzpicture}
=
\begin{tikzpicture}[tikzfig]
	\begin{pgfonlayer}{nodelayer}
		\node [style=nothing] (15) at (0, 2) {};
		\node [style=nothing] (16) at (-0.5, 2) {};
		\node [style=nothing] (17) at (-1, 2) {};
		\node [style=nothing] (18) at (-1.5, 2) {};
		\node [style=nothing] (19) at (-1.5, 3.5) {};
		\node [style=nothing] (20) at (-0.5, 3.5) {};
		\node [style=nothing] (21) at (-1, 3.5) {};
		\node [style=nothing] (22) at (0, 3.5) {};
		\node [style=dot] (23) at (-1.5, 2.5) {};
		\node [style=dot] (24) at (-1, 2.5) {};
		\node [style=dot] (25) at (-1.5, 3) {};
		\node [style=dot] (26) at (-1, 3) {};
		\node [style=oplus] (27) at (-0.5, 3) {};
		\node [style=oplus] (28) at (0, 2.5) {};
	\end{pgfonlayer}
	\begin{pgfonlayer}{edgelayer}
		\draw (18) to (23);
		\draw (23) to (25);
		\draw (25) to (19);
		\draw (21) to (26);
		\draw (26) to (24);
		\draw (24) to (17);
		\draw (16) to (27);
		\draw (27) to (20);
		\draw (22) to (28);
		\draw (28) to (15);
		\draw (28) to (24);
		\draw (24) to (23);
		\draw (25) to (26);
		\draw (26) to (27);
	\end{pgfonlayer}
\end{tikzpicture}
$}

\item
\label{TOF.11}
{\hfil
$
\begin{tikzpicture}[tikzfig]
	\begin{pgfonlayer}{nodelayer}
		\node [style=nothing] (16) at (0, 2) {};
		\node [style=nothing] (17) at (-0.5, 2) {};
		\node [style=nothing] (18) at (-1, 2) {};
		\node [style=nothing] (19) at (-1.5, 2) {};
		\node [style=nothing] (20) at (-0.5, 4) {};
		\node [style=nothing] (21) at (0, 4) {};
		\node [style=dot] (22) at (-1.5, 2.5) {};
		\node [style=dot] (23) at (-1, 3) {};
		\node [style=dot] (24) at (-0.5, 3) {};
		\node [style=oplus] (25) at (-1, 2.5) {};
		\node [style=oplus] (26) at (0, 3) {};
		\node [style=nothing] (27) at (-1.5, 4) {};
		\node [style=nothing] (28) at (-1, 4) {};
		\node [style=oplus] (29) at (-1, 3.5) {};
		\node [style=dot] (30) at (-1.5, 3.5) {};
	\end{pgfonlayer}
	\begin{pgfonlayer}{edgelayer}
		\draw (22) to (25);
		\draw (23) to (24);
		\draw (24) to (26);
		\draw (16) to (26);
		\draw (26) to (21);
		\draw (20) to (24);
		\draw (24) to (17);
		\draw (18) to (25);
		\draw (25) to (23);
		\draw (22) to (19);
		\draw (22) to (30);
		\draw (30) to (27);
		\draw (28) to (29);
		\draw (29) to (23);
		\draw (29) to (30);
	\end{pgfonlayer}
\end{tikzpicture}
=
\begin{tikzpicture}[tikzfig]
	\begin{pgfonlayer}{nodelayer}
		\node [style=nothing] (17) at (0, 2) {};
		\node [style=nothing] (18) at (-1, 2) {};
		\node [style=nothing] (19) at (-0.5, 2) {};
		\node [style=nothing] (20) at (-1.5, 2) {};
		\node [style=dot] (21) at (-1.5, 2.5) {};
		\node [style=dot] (22) at (-0.5, 2.5) {};
		\node [style=oplus] (23) at (0, 2.5) {};
		\node [style=nothing] (24) at (-0.5, 3.5) {};
		\node [style=nothing] (25) at (-1, 3.5) {};
		\node [style=nothing] (26) at (-1.5, 3.5) {};
		\node [style=nothing] (27) at (0, 3.5) {};
		\node [style=dot] (28) at (-1, 3) {};
		\node [style=dot] (29) at (-0.5, 3) {};
		\node [style=oplus] (30) at (0, 3) {};
	\end{pgfonlayer}
	\begin{pgfonlayer}{edgelayer}
		\draw (20) to (21);
		\draw (19) to (22);
		\draw (23) to (17);
		\draw (23) to (22);
		\draw (22) to (21);
		\draw (28) to (18);
		\draw (22) to (29);
		\draw (29) to (24);
		\draw (27) to (30);
		\draw (30) to (23);
		\draw (30) to (29);
		\draw (29) to (28);
		\draw (21) to (26);
		\draw (28) to (25);
	\end{pgfonlayer}
\end{tikzpicture}
$}

\item
\label{TOF.12}
{\hfil
$
\begin{tikzpicture}[tikzfig]
	\begin{pgfonlayer}{nodelayer}
		\node [style=nothing] (18) at (-0.5, 2) {};
		\node [style=nothing] (19) at (0, 2) {};
		\node [style=nothing] (20) at (-1, 2) {};
		\node [style=nothing] (21) at (-1.5, 2) {};
		\node [style=nothing] (22) at (-0.5, 4) {};
		\node [style=nothing] (23) at (-1.5, 4) {};
		\node [style=nothing] (24) at (0, 4) {};
		\node [style=nothing] (25) at (-1, 4) {};
		\node [style=dot] (26) at (-1.5, 2.5) {};
		\node [style=dot] (27) at (-1, 2.5) {};
		\node [style=oplus] (28) at (-0.5, 2.5) {};
		\node [style=oplus] (29) at (0, 3) {};
		\node [style=dot] (30) at (-1, 3) {};
		\node [style=dot] (31) at (-0.5, 3) {};
		\node [style=oplus] (32) at (-0.5, 3.5) {};
		\node [style=dot] (33) at (-1.5, 3.5) {};
		\node [style=dot] (34) at (-1, 3.5) {};
	\end{pgfonlayer}
	\begin{pgfonlayer}{edgelayer}
		\draw (26) to (27);
		\draw (27) to (28);
		\draw (30) to (31);
		\draw (31) to (29);
		\draw (33) to (34);
		\draw (34) to (32);
		\draw (21) to (26);
		\draw (26) to (33);
		\draw (33) to (23);
		\draw (25) to (34);
		\draw (34) to (30);
		\draw (30) to (27);
		\draw (27) to (20);
		\draw (18) to (28);
		\draw (28) to (31);
		\draw (31) to (32);
		\draw (32) to (22);
		\draw (24) to (29);
		\draw (29) to (19);
	\end{pgfonlayer}
\end{tikzpicture}
=
\begin{tikzpicture}[tikzfig]
	\begin{pgfonlayer}{nodelayer}
		\node [style=nothing] (19) at (-0.5, 3.5) {};
		\node [style=nothing] (20) at (0, 3.5) {};
		\node [style=nothing] (21) at (-1, 3.5) {};
		\node [style=nothing] (22) at (-1.5, 3.5) {};
		\node [style=nothing] (23) at (-0.5, 5) {};
		\node [style=nothing] (24) at (-1.5, 5) {};
		\node [style=nothing] (25) at (0, 5) {};
		\node [style=nothing] (26) at (-1, 5) {};
		\node [style=dot] (27) at (-1, 4.5) {};
		\node [style=dot] (28) at (-0.5, 4.5) {};
		\node [style=dot] (29) at (-1.5, 4) {};
		\node [style=dot] (30) at (-1, 4) {};
		\node [style=oplus] (31) at (0, 4) {};
		\node [style=oplus] (32) at (0, 4.5) {};
	\end{pgfonlayer}
	\begin{pgfonlayer}{edgelayer}
		\draw (27) to (28);
		\draw (22) to (29);
		\draw (29) to (24);
		\draw (21) to (30);
		\draw (30) to (27);
		\draw (27) to (26);
		\draw (19) to (28);
		\draw (28) to (23);
		\draw (20) to (31);
		\draw (31) to (32);
		\draw (32) to (25);
		\draw (32) to (28);
		\draw (31) to (30);
		\draw (30) to (29);
	\end{pgfonlayer}
\end{tikzpicture}
$}

\item
\label{TOF.13}
{\hfil
$
\begin{tikzpicture}[tikzfig]
	\begin{pgfonlayer}{nodelayer}
		\node [style=nothing] (20) at (0, 3.5) {};
		\node [style=nothing] (21) at (-1, 3.5) {};
		\node [style=nothing] (22) at (-0.5, 3.5) {};
		\node [style=nothing] (23) at (-1.5, 3.5) {};
		\node [style=nothing] (24) at (0, 5.5) {};
		\node [style=dot] (25) at (-1.5, 4) {};
		\node [style=dot] (26) at (-1, 4) {};
		\node [style=dot] (27) at (-0.5, 4.5) {};
		\node [style=oplus] (28) at (-0.5, 4) {};
		\node [style=oplus] (29) at (0, 4.5) {};
		\node [style=nothing] (30) at (-0.5, 5.5) {};
		\node [style=nothing] (31) at (-1.5, 5.5) {};
		\node [style=nothing] (32) at (-1, 5.5) {};
		\node [style=oplus] (33) at (-0.5, 5) {};
		\node [style=dot] (34) at (-1, 5) {};
		\node [style=dot] (35) at (-1.5, 5) {};
	\end{pgfonlayer}
	\begin{pgfonlayer}{edgelayer}
		\draw (25) to (23);
		\draw (26) to (21);
		\draw (22) to (28);
		\draw (28) to (27);
		\draw (24) to (29);
		\draw (29) to (20);
		\draw (28) to (26);
		\draw (26) to (25);
		\draw (29) to (27);
		\draw (25) to (35);
		\draw (35) to (31);
		\draw (32) to (34);
		\draw (34) to (26);
		\draw (27) to (33);
		\draw (33) to (30);
		\draw (33) to (34);
		\draw (34) to (35);
	\end{pgfonlayer}
\end{tikzpicture}
=
\begin{tikzpicture}[tikzfig]
	\begin{pgfonlayer}{nodelayer}
		\node [style=nothing] (21) at (0, 3.5) {};
		\node [style=nothing] (22) at (-1, 3.5) {};
		\node [style=nothing] (23) at (-0.5, 3.5) {};
		\node [style=nothing] (24) at (-1.5, 3.5) {};
		\node [style=dot] (25) at (-1.5, 4) {};
		\node [style=dot] (26) at (-1, 4) {};
		\node [style=oplus] (27) at (0, 4) {};
		\node [style=nothing] (28) at (-0.5, 5) {};
		\node [style=nothing] (29) at (-1, 5) {};
		\node [style=nothing] (30) at (0, 5) {};
		\node [style=nothing] (31) at (-1.5, 5) {};
		\node [style=dot] (32) at (-0.5, 4.5) {};
		\node [style=oplus] (33) at (0, 4.5) {};
	\end{pgfonlayer}
	\begin{pgfonlayer}{edgelayer}
		\draw (21) to (27);
		\draw (22) to (26);
		\draw (25) to (24);
		\draw (25) to (26);
		\draw (26) to (27);
		\draw (32) to (33);
		\draw (33) to (30);
		\draw (33) to (27);
		\draw (23) to (32);
		\draw (25) to (31);
		\draw (29) to (26);
		\draw (32) to (28);
	\end{pgfonlayer}
\end{tikzpicture}
$}

\item
\label{TOF.14}
{\hfil
$
\begin{tikzpicture}[tikzfig]
	\begin{pgfonlayer}{nodelayer}
		\node [style=nothing] (22) at (0, 3.5) {};
		\node [style=nothing] (23) at (-0.5, 3.5) {};
		\node [style=nothing] (24) at (-0.5, 5.5) {};
		\node [style=nothing] (25) at (0, 5.5) {};
		\node [style=oplus] (26) at (0, 4) {};
		\node [style=oplus] (27) at (0, 5) {};
		\node [style=oplus] (28) at (-0.5, 4.5) {};
		\node [style=dot] (29) at (-0.5, 5) {};
		\node [style=dot] (30) at (0, 4.5) {};
		\node [style=dot] (31) at (-0.5, 4) {};
	\end{pgfonlayer}
	\begin{pgfonlayer}{edgelayer}
		\draw (23) to (31);
		\draw (31) to (28);
		\draw (28) to (29);
		\draw (29) to (24);
		\draw (25) to (27);
		\draw (27) to (30);
		\draw (30) to (26);
		\draw (26) to (22);
		\draw (26) to (31);
		\draw (30) to (28);
		\draw (27) to (29);
	\end{pgfonlayer}
\end{tikzpicture}
=
\begin{tikzpicture}[tikzfig]
	\begin{pgfonlayer}{nodelayer}
		\node [style=nothing] (23) at (0, 3.5) {};
		\node [style=nothing] (24) at (-0.5, 3.5) {};
		\node [style=nothing] (25) at (-0.5, 4.5) {};
		\node [style=nothing] (26) at (0, 4.5) {};
	\end{pgfonlayer}
	\begin{pgfonlayer}{edgelayer}
		\draw [in=-90, out=90, looseness=1.25] (24) to (26);
		\draw [in=-90, out=90, looseness=1.25] (23) to (25);
	\end{pgfonlayer}
\end{tikzpicture}
$}

\item
\label{TOF.15}
{\hfil
$
\begin{tikzpicture}[tikzfig]
	\begin{pgfonlayer}{nodelayer}
		\node [style=nothing] (24) at (-1.75, 3.5) {};
		\node [style=nothing] (25) at (-1.25, 3.5) {};
		\node [style=nothing] (26) at (-0.75, 3.5) {};
		\node [style=nothing] (27) at (-1.75, 5.5) {};
		\node [style=nothing] (28) at (-1.25, 5.5) {};
		\node [style=nothing] (29) at (-0.75, 5.5) {};
		\node [style=dot] (30) at (-1.75, 4.5) {};
		\node [style=dot] (31) at (-1.25, 4.5) {};
		\node [style=oplus] (32) at (-0.75, 4.5) {};
	\end{pgfonlayer}
	\begin{pgfonlayer}{edgelayer}
		\draw (24) to (30);
		\draw (30) to (27);
		\draw (28) to (31);
		\draw (31) to (25);
		\draw (26) to (32);
		\draw (32) to (29);
		\draw (32) to (31);
		\draw (31) to (30);
	\end{pgfonlayer}
\end{tikzpicture}
=
\begin{tikzpicture}[tikzfig]
	\begin{pgfonlayer}{nodelayer}
		\node [style=nothing] (25) at (-1.75, 3.5) {};
		\node [style=nothing] (26) at (-1.25, 3.5) {};
		\node [style=nothing] (27) at (-0.75, 3.5) {};
		\node [style=dot] (28) at (-1.75, 4.5) {};
		\node [style=dot] (29) at (-1.25, 4.5) {};
		\node [style=oplus] (30) at (-0.75, 4.5) {};
		\node [style=nothing] (31) at (-1.75, 5.5) {};
		\node [style=nothing] (32) at (-1.25, 5.5) {};
		\node [style=nothing] (33) at (-0.75, 5.5) {};
	\end{pgfonlayer}
	\begin{pgfonlayer}{edgelayer}
		\draw [in=-90, out=90, looseness=1.25] (25) to (29);
		\draw [in=-90, out=90, looseness=1.25] (29) to (31);
		\draw [in=-90, out=90, looseness=1.25] (28) to (32);
		\draw [in=90, out=-90, looseness=1.25] (28) to (26);
		\draw (27) to (30);
		\draw (30) to (33);
		\draw (28) to (29);
		\draw (29) to (30);
	\end{pgfonlayer}
\end{tikzpicture}
$}

\item
\label{TOF.16}
{\hfil
$
\begin{tikzpicture}[tikzfig]
	\begin{pgfonlayer}{nodelayer}
		\node [style=nothing] (26) at (0, 3.5) {};
		\node [style=nothing] (27) at (-0.5, 3.5) {};
		\node [style=nothing] (28) at (-1.5, 3.5) {};
		\node [style=nothing] (29) at (-2, 3.5) {};
		\node [style=zeroin] (30) at (-1, 3.5) {};
		\node [style=oplus] (31) at (-1, 4) {};
		\node [style=oplus] (32) at (-1, 5) {};
		\node [style=dot] (33) at (-1, 4.5) {};
		\node [style=dot] (34) at (-0.5, 4.5) {};
		\node [style=dot] (35) at (-1.5, 4) {};
		\node [style=dot] (36) at (-2, 4) {};
		\node [style=dot] (37) at (-1.5, 5) {};
		\node [style=dot] (38) at (-2, 5) {};
		\node [style=oplus] (39) at (0, 4.5) {};
		\node [style=zeroout] (40) at (-1, 5.5) {};
		\node [style=nothing] (41) at (0, 5.5) {};
		\node [style=nothing] (42) at (-2, 5.5) {};
		\node [style=nothing] (43) at (-0.5, 5.5) {};
		\node [style=nothing] (44) at (-1.5, 5.5) {};
	\end{pgfonlayer}
	\begin{pgfonlayer}{edgelayer}
		\draw (29) to (36);
		\draw (36) to (38);
		\draw (38) to (42);
		\draw (37) to (35);
		\draw (41) to (39);
		\draw (39) to (26);
		\draw (39) to (34);
		\draw (34) to (33);
		\draw (35) to (31);
		\draw (35) to (36);
		\draw (38) to (37);
		\draw (32) to (37);
		\draw (30) to (31);
		\draw (31) to (33);
		\draw (33) to (32);
		\draw (40) to (32);
		\draw [style=simple] (43) to (34);
		\draw [style=simple] (34) to (27);
		\draw [style=simple] (28) to (35);
		\draw [style=simple] (37) to (44);
	\end{pgfonlayer}
\end{tikzpicture}
=
\begin{tikzpicture}[tikzfig]
	\begin{pgfonlayer}{nodelayer}
		\node [style=nothing] (27) at (0, 3.5) {};
		\node [style=nothing] (28) at (-0.5, 3.5) {};
		\node [style=nothing] (29) at (-1.5, 3.5) {};
		\node [style=nothing] (30) at (-2, 3.5) {};
		\node [style=zeroin] (31) at (-1, 4.25) {};
		\node [style=oplus] (32) at (-1, 4.75) {};
		\node [style=oplus] (33) at (-1, 5.75) {};
		\node [style=dot] (34) at (-1, 5.25) {};
		\node [style=dot] (35) at (-0.5, 5.25) {};
		\node [style=dot] (36) at (-1.5, 4.75) {};
		\node [style=dot] (37) at (-2, 4.75) {};
		\node [style=dot] (38) at (-1.5, 5.75) {};
		\node [style=dot] (39) at (-2, 5.75) {};
		\node [style=oplus] (40) at (0, 5.25) {};
		\node [style=zeroout] (41) at (-1, 6.25) {};
		\node [style=nothing] (42) at (0, 7) {};
		\node [style=nothing] (43) at (-2, 7) {};
		\node [style=nothing] (44) at (-0.5, 7) {};
		\node [style=nothing] (45) at (-1.5, 7) {};
		\node [style=none] (46) at (-1.5, 6.5) {};
		\node [style=none] (47) at (-0.5, 6.5) {};
		\node [style=none] (48) at (-0.5, 4) {};
		\node [style=none] (49) at (-1.5, 4) {};
	\end{pgfonlayer}
	\begin{pgfonlayer}{edgelayer}
		\draw (30) to (37);
		\draw (37) to (39);
		\draw (39) to (43);
		\draw (38) to (36);
		\draw (42) to (40);
		\draw (40) to (27);
		\draw (40) to (35);
		\draw (35) to (34);
		\draw (36) to (32);
		\draw (36) to (37);
		\draw (39) to (38);
		\draw (33) to (38);
		\draw (31) to (32);
		\draw (32) to (34);
		\draw (34) to (33);
		\draw (33) to (41);
		\draw [in=90, out=-90, looseness=0.50] (44) to (46.center);
		\draw [in=90, out=-90, looseness=0.75] (45) to (47.center);
		\draw (47.center) to (35);
		\draw (35) to (48.center);
		\draw [in=90, out=-105, looseness=0.50] (48.center) to (29);
		\draw [in=-90, out=90, looseness=0.50] (28) to (49.center);
		\draw (49.center) to (36);
		\draw (38) to (46.center);
	\end{pgfonlayer}
\end{tikzpicture}
$}
\end{enumerate}
\end{multicols}
\
\end{mdframed}
}}
\caption{The identities of \texorpdfstring{$\TOF$}{TOF}}
\label{fig:TOF}
\end{figure}


The Toffoli gate and the 1-ancillary bits allow $\cnot$, $\Not$, $\zeroin$, $\zeroout$, and flipped $\tof$ gate and flipped $\cnot$ gate can defined in this setting:

\[ \begin{array}{ccc}
\begin{tikzpicture}[tikzfig]
	\begin{pgfonlayer}{nodelayer}
		\node [style=nothing] (28) at (0, 3.5) {};
		\node [style=nothing] (29) at (-0.5, 3.5) {};
		\node [style=nothing] (30) at (-0.5, 4.5) {};
		\node [style=nothing] (31) at (0, 4.5) {};
		\node [style=oplus] (32) at (0, 4) {};
		\node [style=dot] (33) at (-0.5, 4) {};
	\end{pgfonlayer}
	\begin{pgfonlayer}{edgelayer}
		\draw (29) to (33);
		\draw (33) to (30);
		\draw (31) to (32);
		\draw (32) to (28);
		\draw (32) to (33);
	\end{pgfonlayer}
\end{tikzpicture}
:=
\begin{tikzpicture}[tikzfig]
	\begin{pgfonlayer}{nodelayer}
		\node [style=nothing] (29) at (0, 3.5) {};
		\node [style=nothing] (30) at (-0.5, 3.5) {};
		\node [style=nothing] (31) at (-0.5, 4.5) {};
		\node [style=nothing] (32) at (0, 4.5) {};
		\node [style=oplus] (33) at (0, 4) {};
		\node [style=dot] (34) at (-0.5, 4) {};
		\node [style=onein] (35) at (-1, 3.5) {};
		\node [style=oneout] (36) at (-1, 4.5) {};
		\node [style=dot] (37) at (-1, 4) {};
	\end{pgfonlayer}
	\begin{pgfonlayer}{edgelayer}
		\draw (30) to (34);
		\draw (34) to (31);
		\draw (32) to (33);
		\draw (33) to (29);
		\draw (33) to (34);
		\draw (35) to (37);
		\draw (37) to (36);
		\draw (37) to (34);
	\end{pgfonlayer}
\end{tikzpicture}
,
&
\begin{tikzpicture}[tikzfig]
	\begin{pgfonlayer}{nodelayer}
		\node [style=nothing] (0) at (0, 0.5) {};
		\node [style=nothing] (1) at (0, 1.5) {};
		\node [style=oplus] (2) at (0, 1) {};
	\end{pgfonlayer}
	\begin{pgfonlayer}{edgelayer}
		\draw (1) to (2);
		\draw (2) to (0);
	\end{pgfonlayer}
\end{tikzpicture}
:=
\begin{tikzpicture}[tikzfig]
	\begin{pgfonlayer}{nodelayer}
		\node [style=nothing] (1) at (0, 0) {};
		\node [style=nothing] (2) at (0, 1) {};
		\node [style=oplus] (3) at (0, 0.5) {};
		\node [style=dot] (4) at (-0.5, 0.5) {};
		\node [style=onein] (5) at (-0.5, 0) {};
		\node [style=oneout] (6) at (-0.5, 1) {};
	\end{pgfonlayer}
	\begin{pgfonlayer}{edgelayer}
		\draw (2) to (3);
		\draw (3) to (1);
		\draw (3) to (4);
		\draw (5) to (4);
		\draw (4) to (6);
	\end{pgfonlayer}
\end{tikzpicture}
,
&
\begin{tikzpicture}[tikzfig]
	\begin{pgfonlayer}{nodelayer}
		\node [style=zeroin] (2) at (0, 0) {};
		\node [style=nothing] (3) at (0, 1) {};
	\end{pgfonlayer}
	\begin{pgfonlayer}{edgelayer}
		\draw (2) to (3);
	\end{pgfonlayer}
\end{tikzpicture}
:=
\begin{tikzpicture}[tikzfig]
	\begin{pgfonlayer}{nodelayer}
		\node [style=nothing] (3) at (0, 1) {};
		\node [style=onein] (4) at (0, 0) {};
		\node [style=oplus] (5) at (0, 0.5) {};
	\end{pgfonlayer}
	\begin{pgfonlayer}{edgelayer}
		\draw (3) to (5);
		\draw (5) to (4);
	\end{pgfonlayer}
\end{tikzpicture}\\
\begin{tikzpicture}[tikzfig]
	\begin{pgfonlayer}{nodelayer}
		\node [style=nothing] (4) at (0, 0) {};
		\node [style=zeroout] (5) at (0, 1) {};
	\end{pgfonlayer}
	\begin{pgfonlayer}{edgelayer}
		\draw (4) to (5);
	\end{pgfonlayer}
\end{tikzpicture}
:=
\begin{tikzpicture}[tikzfig]
	\begin{pgfonlayer}{nodelayer}
		\node [style=nothing] (5) at (0, 0) {};
		\node [style=oneout] (6) at (0, 1) {};
		\node [style=oplus] (7) at (0, 0.5) {};
	\end{pgfonlayer}
	\begin{pgfonlayer}{edgelayer}
		\draw (5) to (7);
		\draw (7) to (6);
	\end{pgfonlayer}
\end{tikzpicture}
,
&
\begin{tikzpicture}[tikzfig]
	\begin{pgfonlayer}{nodelayer}
		\node [style=nothing] (6) at (0, 0) {};
		\node [style=nothing] (7) at (-0.5, 0) {};
		\node [style=nothing] (8) at (-1, 0) {};
		\node [style=nothing] (9) at (-1, 1.5) {};
		\node [style=nothing] (10) at (-0.5, 1.5) {};
		\node [style=nothing] (11) at (0, 1.5) {};
		\node [style=dot] (12) at (0, 0.75) {};
		\node [style=dot] (13) at (-0.5, 0.75) {};
		\node [style=oplus] (14) at (-1, 0.75) {};
	\end{pgfonlayer}
	\begin{pgfonlayer}{edgelayer}
		\draw (8) to (14);
		\draw (14) to (9);
		\draw (10) to (13);
		\draw (13) to (7);
		\draw (6) to (12);
		\draw (12) to (11);
		\draw (14) to (13);
		\draw (12) to (13);
	\end{pgfonlayer}
\end{tikzpicture}
:=
\begin{tikzpicture}[tikzfig]
	\begin{pgfonlayer}{nodelayer}
		\node [style=nothing] (7) at (0, 0) {};
		\node [style=nothing] (8) at (-0.5, 0) {};
		\node [style=nothing] (9) at (-1, 0) {};
		\node [style=nothing] (10) at (-1, 2) {};
		\node [style=nothing] (11) at (-0.5, 2) {};
		\node [style=nothing] (12) at (0, 2) {};
		\node [style=dot] (13) at (-1, 1) {};
		\node [style=dot] (14) at (-0.5, 1) {};
		\node [style=oplus] (15) at (0, 1) {};
	\end{pgfonlayer}
	\begin{pgfonlayer}{edgelayer}
		\draw [in=-90, out=90, looseness=1.25] (9) to (15);
		\draw [in=-90, out=90, looseness=1.25] (15) to (10);
		\draw (8) to (14);
		\draw (14) to (11);
		\draw [in=90, out=-90, looseness=1.25] (12) to (13);
		\draw [in=90, out=-90, looseness=1.25] (13) to (7);
		\draw (13) to (14);
		\draw (14) to (15);
	\end{pgfonlayer}
\end{tikzpicture}
,
&
\begin{tikzpicture}[tikzfig]
	\begin{pgfonlayer}{nodelayer}
		\node [style=nothing] (8) at (0, 0) {};
		\node [style=nothing] (9) at (-0.5, 0) {};
		\node [style=nothing] (10) at (-0.5, 1) {};
		\node [style=nothing] (11) at (0, 1) {};
		\node [style=dot] (12) at (0, 0.5) {};
		\node [style=oplus] (13) at (-0.5, 0.5) {};
	\end{pgfonlayer}
	\begin{pgfonlayer}{edgelayer}
		\draw (9) to (13);
		\draw (13) to (10);
		\draw (11) to (12);
		\draw (12) to (8);
		\draw (12) to (13);
	\end{pgfonlayer}
\end{tikzpicture}
:=
\begin{tikzpicture}[tikzfig]
	\begin{pgfonlayer}{nodelayer}
		\node [style=nothing] (9) at (0, 0) {};
		\node [style=nothing] (10) at (-0.5, 0) {};
		\node [style=nothing] (11) at (-0.5, 1.5) {};
		\node [style=nothing] (12) at (0, 1.5) {};
		\node [style=dot] (13) at (-0.5, 0.75) {};
		\node [style=oplus] (14) at (0, 0.75) {};
	\end{pgfonlayer}
	\begin{pgfonlayer}{edgelayer}
		\draw [in=-90, out=90, looseness=1.25] (10) to (14);
		\draw [in=-90, out=90, looseness=1.25] (9) to (13);
		\draw [in=-90, out=90, looseness=1.25] (13) to (12);
		\draw [in=-90, out=90, looseness=1.25] (14) to (11);
		\draw (13) to (14);
	\end{pgfonlayer}
\end{tikzpicture}
\end{array}  \]

One can moreover construct generalized controlled not gates with arbitrarily many control wires in the obvious way.  Let $[x,X]$ denote a generalized Toffoli gate acting on the $x$th wire, controlled on the wires indexed by a set $X$. Then we can partially commute generalized controlled-not gates:

\begin{lemma} \cite[Lem. 7.2.6]{cole}
\label{lemma:Iwama}
Let $[x,X]$ and $[y,Y]$ be generalized controlled not gates in $\TOF$ where $x\notin Y$.  We can perform the identities of Iwama et al. \cite{Iwama}, to commute them past each other with a trailing generalized controlled not gate as a side effect:
$$
 [y,{X\cup Y}] [y,{Y\sqcup\{x\}}] [x,X]
$$
\end{lemma}

In $\TOF$, one can define the diagonal map as follows:
$$
\begin{tikzpicture}[tikzfig]
	\begin{pgfonlayer}{nodelayer}
		\node [style=fanout] (10) at (0, 1) {};
		\node [style=none] (11) at (-0.5, 1.75) {};
		\node [style=none] (12) at (0.5, 1.75) {};
		\node [style=none] (13) at (0, 0.25) {};
	\end{pgfonlayer}
	\begin{pgfonlayer}{edgelayer}
		\draw (13.center) to (10);
		\draw [in=-90, out=124] (10) to (11.center);
		\draw [in=56, out=-90] (12.center) to (10);
	\end{pgfonlayer}
\end{tikzpicture}
:=
\begin{tikzpicture}[tikzfig]
	\begin{pgfonlayer}{nodelayer}
		\node [style=zeroin] (11) at (0, 1) {};
		\node [style=oplus] (12) at (0, 1.5) {};
		\node [style=dot] (13) at (-0.5, 1.5) {};
		\node [style=none] (14) at (-0.5, 0.75) {};
		\node [style=none] (15) at (-0.5, 2) {};
		\node [style=none] (16) at (0, 2) {};
	\end{pgfonlayer}
	\begin{pgfonlayer}{edgelayer}
		\draw (16.center) to (11);
		\draw (12) to (13);
		\draw (15.center) to (14.center);
	\end{pgfonlayer}
\end{tikzpicture}
$$


\begin{lemma}\cite[\S 5.3.2]{cole}
The diagonal map is a natural special commutative  \dag-symmetric % \linebreak[4] 
monoidal  nonunital Frobenius algebra.
\end{lemma}

It is also natural on target qubits:
\begin{lemma}\cite[Lem. B.0.2 (iii)]{cole}
\label{lemma:natoplus}
$$
\begin{tikzpicture}[tikzfig]
	\begin{pgfonlayer}{nodelayer}
		\node [style=fanin] (12) at (0, 2.25) {};
		\node [style=oplus] (13) at (0, 1.75) {};
		\node [style=dot] (14) at (-0.75, 1.75) {};
		\node [style=none] (15) at (-0.75, 1.25) {};
		\node [style=none] (16) at (0, 1.25) {};
		\node [style=none] (17) at (0.25, 3) {};
		\node [style=none] (18) at (-0.25, 3) {};
		\node [style=none] (19) at (-0.75, 3) {};
	\end{pgfonlayer}
	\begin{pgfonlayer}{edgelayer}
		\draw (15.center) to (14);
		\draw (14) to (19.center);
		\draw (13) to (14);
		\draw (16.center) to (13);
		\draw (13) to (12);
		\draw [in=-90, out=108] (12) to (18.center);
		\draw [in=72, out=-90] (17.center) to (12);
	\end{pgfonlayer}
\end{tikzpicture}
=
\begin{tikzpicture}[tikzfig]
	\begin{pgfonlayer}{nodelayer}
		\node [style=fanin] (13) at (0, 0.75) {};
		\node [style=none] (14) at (-0.75, 0.25) {};
		\node [style=none] (15) at (0, 0.25) {};
		\node [style=none] (16) at (0.25, 1.5) {};
		\node [style=none] (17) at (-0.25, 1.5) {};
		\node [style=none] (18) at (-0.75, 2.5) {};
		\node [style=dot] (19) at (-0.75, 1.5) {};
		\node [style=oplus] (20) at (-0.25, 1.5) {};
		\node [style=none] (21) at (0.25, 2) {};
		\node [style=dot] (22) at (-0.75, 2) {};
		\node [style=oplus] (23) at (0.25, 2) {};
		\node [style=none] (24) at (0.25, 2.5) {};
		\node [style=none] (25) at (-0.25, 2.5) {};
	\end{pgfonlayer}
	\begin{pgfonlayer}{edgelayer}
		\draw [in=-90, out=108] (13) to (17.center);
		\draw [in=72, out=-90] (16.center) to (13);
		\draw (20) to (19);
		\draw (18.center) to (14.center);
		\draw (23) to (22);
		\draw (24.center) to (21.center);
		\draw (21.center) to (16.center);
		\draw (13) to (15.center);
		\draw (17.center) to (25.center);
	\end{pgfonlayer}
\end{tikzpicture}
$$
\end{lemma}


\section{The isomorphism between \texorpdfstring{$\ZXA$}{ZX\&} and  the (co)unitual completion of \texorpdfstring{$\TOF$}{TOF}}

We establish some basic properties of $\ZXA$ and  the (co)unitual completion of $\TOF$.
\subsection{Basic properties of the (co)unitual completion of \texorpdfstring{$\TOF$}{TOF}}

First, note that because $\tilde \TOF$ is a discrete Cartesian restriction category, it is a copy category and thus, for any map $f$ in $\TOF$


\begin{remark}
\label{cor:copy}

$$
\begin{tikzpicture}[tikzfig]
	\begin{pgfonlayer}{nodelayer}
		\node [style=map] (14) at (0, 0) {$f$};
		\node [style=none] (15) at (0, -0.75) {};
		\node [style=X] (16) at (0, 0.75) {};
	\end{pgfonlayer}
	\begin{pgfonlayer}{edgelayer}
		\draw (16) to (14);
		\draw (14) to (15.center);
	\end{pgfonlayer}
\end{tikzpicture}
=
\begin{tikzpicture}[tikzfig]
	\begin{pgfonlayer}{nodelayer}
		\node [style=map] (15) at (0, 0.75) {$f$};
		\node [style=X] (16) at (0, 1.5) {};
		\node [style=none] (17) at (0.5, -0.75) {};
		\node [style=X] (18) at (0.5, 0) {};
		\node [style=X] (19) at (1, 0.75) {};
	\end{pgfonlayer}
	\begin{pgfonlayer}{edgelayer}
		\draw (16) to (15);
		\draw [in=30, out=-90] (19) to (18);
		\draw [in=-90, out=150] (18) to (15);
		\draw (18) to (17.center);
	\end{pgfonlayer}
\end{tikzpicture}
=
\begin{tikzpicture}[tikzfig]
	\begin{pgfonlayer}{nodelayer}
		\node [style=map] (16) at (0, 0) {$\bar f$};
		\node [style=X] (17) at (0, 0.75) {};
		\node [style=none] (18) at (0, -0.75) {};
	\end{pgfonlayer}
	\begin{pgfonlayer}{edgelayer}
		\draw (17) to (16);
		\draw (16) to (18.center);
	\end{pgfonlayer}
\end{tikzpicture}
$$
\end{remark}



  First, the $\cnot$ gate is its own mate on the second wire:
\begin{lemma}
\label{prop:twist}
$$
\begin{tikzpicture}[tikzfig]
	\begin{pgfonlayer}{nodelayer}
		\node [style=dot] (17) at (0, 3) {};
		\node [style=oplus] (18) at (1, 3) {};
		\node [style=none] (19) at (0, 2) {};
		\node [style=none] (20) at (0, 4) {};
		\node [style=none] (21) at (0.5, 2.75) {};
		\node [style=none] (22) at (1, 2.75) {};
		\node [style=none] (23) at (1.5, 3.25) {};
		\node [style=none] (24) at (1, 3.25) {};
		\node [style=none] (25) at (0.5, 3.25) {};
		\node [style=none] (26) at (1.5, 2.75) {};
		\node [style=none] (27) at (1, 2) {};
		\node [style=none] (28) at (1, 4) {};
	\end{pgfonlayer}
	\begin{pgfonlayer}{edgelayer}
		\draw (19.center) to (17);
		\draw (17) to (20.center);
		\draw (18) to (17);
		\draw [in=-90, out=-90, looseness=1.50] (22.center) to (21.center);
		\draw [in=90, out=90, looseness=1.50] (23.center) to (24.center);
		\draw (24.center) to (18);
		\draw (22.center) to (18);
		\draw (23.center) to (26.center);
		\draw (21.center) to (25.center);
		\draw [in=-90, out=90] (27.center) to (26.center);
		\draw [in=90, out=-90, looseness=1.25] (28.center) to (25.center);
	\end{pgfonlayer}
\end{tikzpicture}
=
\begin{tikzpicture}[tikzfig]
	\begin{pgfonlayer}{nodelayer}
		\node [style=dot] (18) at (0, 3) {};
		\node [style=oplus] (19) at (0.75, 3) {};
		\node [style=none] (20) at (0, 2) {};
		\node [style=none] (21) at (0, 4) {};
		\node [style=none] (22) at (0.75, 4) {};
		\node [style=none] (23) at (0.75, 2) {};
	\end{pgfonlayer}
	\begin{pgfonlayer}{edgelayer}
		\draw (20.center) to (18);
		\draw (18) to (21.center);
		\draw (19) to (18);
		\draw (23.center) to (19);
		\draw (19) to (22.center);
	\end{pgfonlayer}
\end{tikzpicture}
$$
\end{lemma}

\begin{proof}
\begin{align*}
\begin{tikzpicture}[tikzfig]
	\begin{pgfonlayer}{nodelayer}
		\node [style=oplus] (0) at (1, 3) {};
		\node [style=none] (1) at (0, 2) {};
		\node [style=none] (2) at (0, 4) {};
		\node [style=none] (3) at (0.5, 2.75) {};
		\node [style=none] (4) at (1, 2.75) {};
		\node [style=none] (5) at (1.5, 3.25) {};
		\node [style=none] (6) at (1, 3.25) {};
		\node [style=none] (7) at (0.5, 3.25) {};
		\node [style=none] (8) at (1.5, 2.75) {};
		\node [style=none] (9) at (1, 2) {};
		\node [style=none] (10) at (1, 4) {};
		\node [style=dot] (11) at (0, 3) {};
	\end{pgfonlayer}
	\begin{pgfonlayer}{edgelayer}
		\draw [in=-90, out=-90, looseness=1.50] (4.center) to (3.center);
		\draw [in=90, out=90, looseness=1.50] (5.center) to (6.center);
		\draw (6.center) to (0);
		\draw (4.center) to (0);
		\draw (5.center) to (8.center);
		\draw (3.center) to (7.center);
		\draw [in=-90, out=90] (9.center) to (8.center);
		\draw [in=90, out=-90, looseness=1.25] (10.center) to (7.center);
		\draw (11) to (0);
		\draw (11) to (2.center);
		\draw (11) to (1.center);
	\end{pgfonlayer}
\end{tikzpicture}
&=
\begin{tikzpicture}[tikzfig]
	\begin{pgfonlayer}{nodelayer}
		\node [style=dot] (1) at (0, 3.25) {};
		\node [style=oplus] (2) at (1, 3.25) {};
		\node [style=none] (3) at (0, 2) {};
		\node [style=none] (4) at (0, 4.5) {};
		\node [style=none] (5) at (0.5, 3.5) {};
		\node [style=none] (6) at (1.5, 3) {};
		\node [style=none] (7) at (1.25, 2) {};
		\node [style=none] (8) at (0.75, 4.5) {};
		\node [style=fanout] (9) at (0.75, 2.75) {};
		\node [style=fanin] (10) at (1.25, 3.75) {};
		\node [style=X] (11) at (0.75, 2) {};
		\node [style=X] (12) at (1.25, 4.5) {};
	\end{pgfonlayer}
	\begin{pgfonlayer}{edgelayer}
		\draw (3.center) to (1);
		\draw (1) to (4.center);
		\draw (2) to (1);
		\draw [in=-90, out=90] (7.center) to (6.center);
		\draw [in=90, out=-90, looseness=1.25] (8.center) to (5.center);
		\draw [in=-72, out=90] (6.center) to (10);
		\draw [in=90, out=-117, looseness=1.25] (10) to (2);
		\draw [in=63, out=-90, looseness=1.25] (2) to (9);
		\draw [in=-90, out=108] (9) to (5.center);
		\draw (11) to (9);
		\draw (10) to (12);
	\end{pgfonlayer}
\end{tikzpicture}
\eq{\ref{CNOT.2}}
\begin{tikzpicture}[tikzfig]
	\begin{pgfonlayer}{nodelayer}
		\node [style=dot] (2) at (0, 3.25) {};
		\node [style=oplus] (3) at (1, 3.25) {};
		\node [style=none] (4) at (0, 2) {};
		\node [style=none] (5) at (0, 5) {};
		\node [style=none] (6) at (1.5, 3) {};
		\node [style=none] (7) at (1.25, 2) {};
		\node [style=none] (8) at (0.5, 5) {};
		\node [style=fanout] (9) at (0.75, 2.75) {};
		\node [style=fanin] (10) at (1.25, 3.75) {};
		\node [style=X] (11) at (0.75, 2) {};
		\node [style=X] (12) at (1.25, 5) {};
		\node [style=dot] (13) at (0, 3.75) {};
		\node [style=oplus] (14) at (0.5, 3.75) {};
		\node [style=oplus] (15) at (0.5, 4.5) {};
		\node [style=dot] (16) at (0, 4.5) {};
	\end{pgfonlayer}
	\begin{pgfonlayer}{edgelayer}
		\draw (4.center) to (2);
		\draw (2) to (5.center);
		\draw (3) to (2);
		\draw [in=-90, out=90] (7.center) to (6.center);
		\draw [in=-72, out=90] (6.center) to (10);
		\draw [in=90, out=-117, looseness=1.25] (10) to (3);
		\draw [in=63, out=-90, looseness=1.25] (3) to (9);
		\draw (11) to (9);
		\draw (10) to (12);
		\draw (14) to (13);
		\draw (15) to (16);
		\draw (8.center) to (15);
		\draw (15) to (14);
		\draw [in=-90, out=104] (9) to (14);
	\end{pgfonlayer}
\end{tikzpicture}
\eq{Lem. \ref{lemma:natoplus}}
\begin{tikzpicture}[tikzfig]
	\begin{pgfonlayer}{nodelayer}
		\node [style=none] (3) at (0, 1.5) {};
		\node [style=none] (4) at (0, 4.5) {};
		\node [style=none] (5) at (1.5, 3) {};
		\node [style=none] (6) at (1.25, 1.5) {};
		\node [style=none] (7) at (0.5, 4.5) {};
		\node [style=fanout] (8) at (0.75, 2.75) {};
		\node [style=fanin] (9) at (1.25, 3.75) {};
		\node [style=X] (10) at (0.75, 1.5) {};
		\node [style=X] (11) at (1.25, 4.5) {};
		\node [style=oplus] (12) at (0.75, 2.25) {};
		\node [style=dot] (13) at (0, 2.25) {};
		\node [style=dot] (14) at (0, 3.5) {};
		\node [style=oplus] (15) at (0.5, 3.5) {};
	\end{pgfonlayer}
	\begin{pgfonlayer}{edgelayer}
		\draw [in=-90, out=90] (6.center) to (5.center);
		\draw [in=-72, out=90] (5.center) to (9);
		\draw (9) to (11);
		\draw (12) to (13);
		\draw (15) to (14);
		\draw (4.center) to (14);
		\draw (15) to (7.center);
		\draw [in=108, out=-90] (15) to (8);
		\draw (8) to (9);
		\draw (8) to (12);
		\draw (12) to (10);
		\draw (3.center) to (13);
		\draw (13) to (14);
	\end{pgfonlayer}
\end{tikzpicture}
\eq{Frob.}
\begin{tikzpicture}[tikzfig]
	\begin{pgfonlayer}{nodelayer}
		\node [style=none] (4) at (0, 1.25) {};
		\node [style=none] (5) at (0, 4.75) {};
		\node [style=none] (6) at (1, 2) {};
		\node [style=none] (7) at (1, 1.25) {};
		\node [style=none] (8) at (0.5, 4.75) {};
		\node [style=X] (9) at (0.5, 1.5) {};
		\node [style=X] (10) at (1, 4.25) {};
		\node [style=oplus] (11) at (0.5, 2) {};
		\node [style=dot] (12) at (0, 2) {};
		\node [style=dot] (13) at (0, 4.25) {};
		\node [style=oplus] (14) at (0.5, 4.25) {};
		\node [style=fanin] (15) at (0.75, 2.75) {};
		\node [style=fanout] (16) at (0.75, 3.5) {};
	\end{pgfonlayer}
	\begin{pgfonlayer}{edgelayer}
		\draw [in=-90, out=90] (7.center) to (6.center);
		\draw (11) to (12);
		\draw (14) to (13);
		\draw (5.center) to (13);
		\draw (14) to (8.center);
		\draw (11) to (9);
		\draw (4.center) to (12);
		\draw (12) to (13);
		\draw [in=63, out=-90, looseness=1.25] (10) to (16);
		\draw [in=-90, out=117, looseness=1.25] (16) to (14);
		\draw (16) to (15);
		\draw [in=90, out=-108] (15) to (11);
		\draw [in=-72, out=90] (6.center) to (15);
	\end{pgfonlayer}
\end{tikzpicture}\\
&\eq{unit}
\begin{tikzpicture}[tikzfig]
	\begin{pgfonlayer}{nodelayer}
		\node [style=none] (5) at (0, 1.25) {};
		\node [style=none] (6) at (0, 4.75) {};
		\node [style=none] (7) at (1, 2) {};
		\node [style=none] (8) at (1, 1.25) {};
		\node [style=none] (9) at (0.75, 4.75) {};
		\node [style=X] (10) at (0.5, 1.25) {};
		\node [style=oplus] (11) at (0.5, 2) {};
		\node [style=dot] (12) at (0, 2) {};
		\node [style=dot] (13) at (0, 4) {};
		\node [style=oplus] (14) at (0.75, 4) {};
		\node [style=fanin] (15) at (0.75, 2.75) {};
	\end{pgfonlayer}
	\begin{pgfonlayer}{edgelayer}
		\draw [in=-90, out=90] (8.center) to (7.center);
		\draw (11) to (12);
		\draw (14) to (13);
		\draw (6.center) to (13);
		\draw (14) to (9.center);
		\draw (11) to (10);
		\draw (5.center) to (12);
		\draw (12) to (13);
		\draw [in=90, out=-108] (15) to (11);
		\draw [in=-72, out=90] (7.center) to (15);
		\draw (14) to (15);
	\end{pgfonlayer}
\end{tikzpicture}
\eq{Lem. \ref{lemma:natoplus}}
\begin{tikzpicture}[tikzfig]
	\begin{pgfonlayer}{nodelayer}
		\node [style=none] (6) at (0, 4) {};
		\node [style=none] (7) at (1, 2.5) {};
		\node [style=none] (8) at (1, 1) {};
		\node [style=none] (9) at (0.75, 4) {};
		\node [style=oplus] (10) at (0.5, 2.5) {};
		\node [style=dot] (11) at (0, 2.5) {};
		\node [style=fanin] (12) at (0.75, 3.25) {};
		\node [style=dot] (13) at (0, 2) {};
		\node [style=oplus] (14) at (0.5, 2) {};
		\node [style=none] (15) at (0, 1) {};
		\node [style=X] (16) at (0.5, 1) {};
		\node [style=dot] (17) at (0, 1.5) {};
		\node [style=oplus] (18) at (1, 1.5) {};
	\end{pgfonlayer}
	\begin{pgfonlayer}{edgelayer}
		\draw [in=-90, out=90] (8.center) to (7.center);
		\draw (10) to (11);
		\draw [in=90, out=-108] (12) to (10);
		\draw [in=-72, out=90] (7.center) to (12);
		\draw (14) to (13);
		\draw (18) to (17);
		\draw (16) to (14);
		\draw (14) to (10);
		\draw (6.center) to (11);
		\draw (11) to (17);
		\draw (17) to (13);
		\draw (13) to (15.center);
		\draw (9.center) to (12);
	\end{pgfonlayer}
\end{tikzpicture}
\eq{\ref{CNOT.2}}
\begin{tikzpicture}[tikzfig]
	\begin{pgfonlayer}{nodelayer}
		\node [style=none] (7) at (0, 3.75) {};
		\node [style=none] (8) at (1, 2) {};
		\node [style=none] (9) at (0.75, 3.75) {};
		\node [style=fanin] (10) at (0.75, 3.25) {};
		\node [style=none] (11) at (0, 2) {};
		\node [style=X] (12) at (0.5, 2) {};
		\node [style=dot] (13) at (0, 2.5) {};
		\node [style=oplus] (14) at (1, 2.5) {};
	\end{pgfonlayer}
	\begin{pgfonlayer}{edgelayer}
		\draw (14) to (13);
		\draw (9.center) to (10);
		\draw [in=90, out=-105] (10) to (12);
		\draw (11.center) to (7.center);
		\draw [in=90, out=-72] (10) to (14);
		\draw (14) to (8.center);
	\end{pgfonlayer}
\end{tikzpicture}
\eq{unit}
\begin{tikzpicture}[tikzfig]
	\begin{pgfonlayer}{nodelayer}
		\node [style=none] (8) at (0, 6.25) {};
		\node [style=none] (9) at (0.5, 5.25) {};
		\node [style=none] (10) at (0.5, 6.25) {};
		\node [style=none] (11) at (0, 5.25) {};
		\node [style=dot] (12) at (0, 5.75) {};
		\node [style=oplus] (13) at (0.5, 5.75) {};
	\end{pgfonlayer}
	\begin{pgfonlayer}{edgelayer}
		\draw (13) to (12);
		\draw (11.center) to (8.center);
		\draw (13) to (9.center);
		\draw (10.center) to (13);
	\end{pgfonlayer}
\end{tikzpicture}
\end{align*}
\end{proof}


Therefore, 

\begin{lemma}
\label{lemma:cnotslide}
$$
\begin{tikzpicture}[tikzfig]
	\begin{pgfonlayer}{nodelayer}
		\node [style=oplus] (9) at (1, 5.75) {};
		\node [style=none] (10) at (0.5, 6.25) {};
		\node [style=none] (11) at (0.5, 5.25) {};
		\node [style=none] (12) at (1, 6.25) {};
		\node [style=none] (13) at (1.5, 5.5) {};
		\node [style=none] (14) at (1, 5.5) {};
		\node [style=none] (15) at (1.5, 6.25) {};
		\node [style=dot] (16) at (0.5, 5.75) {};
	\end{pgfonlayer}
	\begin{pgfonlayer}{edgelayer}
		\draw [in=-90, out=-90, looseness=1.50] (13.center) to (14.center);
		\draw (14.center) to (9);
		\draw (16) to (9);
		\draw (16) to (11.center);
		\draw (16) to (10.center);
		\draw (15.center) to (13.center);
		\draw (9) to (12.center);
	\end{pgfonlayer}
\end{tikzpicture}
\eq{Prop. \ref{prop:twist}}
\begin{tikzpicture}[tikzfig]
	\begin{pgfonlayer}{nodelayer}
		\node [style=dot] (10) at (-0.25, 6.25) {};
		\node [style=oplus] (11) at (0.75, 6.25) {};
		\node [style=none] (12) at (-0.25, 5.25) {};
		\node [style=none] (13) at (-0.25, 7.25) {};
		\node [style=none] (14) at (0.25, 6) {};
		\node [style=none] (15) at (0.75, 6) {};
		\node [style=none] (16) at (1.25, 6.5) {};
		\node [style=none] (17) at (0.75, 6.5) {};
		\node [style=none] (18) at (0.25, 6.5) {};
		\node [style=none] (19) at (0.25, 7.25) {};
		\node [style=none] (20) at (1.25, 5.75) {};
		\node [style=none] (21) at (1.75, 5.75) {};
		\node [style=none] (22) at (1.75, 7.25) {};
	\end{pgfonlayer}
	\begin{pgfonlayer}{edgelayer}
		\draw (12.center) to (10);
		\draw (10) to (13.center);
		\draw (11) to (10);
		\draw [in=-90, out=-90, looseness=1.50] (15.center) to (14.center);
		\draw [in=90, out=90, looseness=1.50] (16.center) to (17.center);
		\draw (17.center) to (11);
		\draw (15.center) to (11);
		\draw (14.center) to (18.center);
		\draw [in=90, out=-90, looseness=1.25] (19.center) to (18.center);
		\draw [in=-90, out=-90, looseness=1.50] (21.center) to (20.center);
		\draw (22.center) to (21.center);
		\draw (20.center) to (16.center);
	\end{pgfonlayer}
\end{tikzpicture}
\eq{yanking}
\begin{tikzpicture}[tikzfig]
	\begin{pgfonlayer}{nodelayer}
		\node [style=dot] (11) at (-0.25, 5.75) {};
		\node [style=oplus] (12) at (0.75, 5.75) {};
		\node [style=none] (13) at (-0.25, 5.25) {};
		\node [style=none] (14) at (-0.25, 6.25) {};
		\node [style=none] (15) at (0.25, 5.5) {};
		\node [style=none] (16) at (0.75, 5.5) {};
		\node [style=none] (17) at (0.75, 6.25) {};
		\node [style=none] (18) at (0.25, 6.25) {};
	\end{pgfonlayer}
	\begin{pgfonlayer}{edgelayer}
		\draw (13.center) to (11);
		\draw (11) to (14.center);
		\draw (12) to (11);
		\draw [in=-90, out=-90, looseness=1.50] (16.center) to (15.center);
		\draw (17.center) to (12);
		\draw (16.center) to (12);
		\draw (15.center) to (18.center);
	\end{pgfonlayer}
\end{tikzpicture}
$$
\end{lemma}


Thus

\begin{lemma}
\label{lemma:whiteunit}

$$
\begin{tikzpicture}[tikzfig]
	\begin{pgfonlayer}{nodelayer}
		\node [style=dot] (12) at (-0.25, 5.75) {};
		\node [style=oplus] (13) at (0.25, 5.75) {};
		\node [style=none] (14) at (-0.25, 5.25) {};
		\node [style=none] (15) at (-0.25, 6.25) {};
		\node [style=none] (16) at (0.25, 6.25) {};
		\node [style=X] (17) at (0.25, 5.25) {};
	\end{pgfonlayer}
	\begin{pgfonlayer}{edgelayer}
		\draw (14.center) to (12);
		\draw (12) to (15.center);
		\draw (13) to (12);
		\draw (16.center) to (13);
		\draw (13) to (17);
	\end{pgfonlayer}
\end{tikzpicture}
=
\begin{tikzpicture}[tikzfig]
	\begin{pgfonlayer}{nodelayer}
		\node [style=none] (13) at (-0.25, 5.25) {};
		\node [style=none] (14) at (-0.25, 6) {};
		\node [style=none] (15) at (0.25, 6) {};
		\node [style=X] (16) at (0.25, 5.25) {};
	\end{pgfonlayer}
	\begin{pgfonlayer}{edgelayer}
		\draw (15.center) to (16);
		\draw (13.center) to (14.center);
	\end{pgfonlayer}
\end{tikzpicture}
$$
\end{lemma}

\begin{proof}
\begin{align*}
\begin{tikzpicture}[tikzfig]
	\begin{pgfonlayer}{nodelayer}
		\node [style=dot] (14) at (-0.25, 5.75) {};
		\node [style=oplus] (15) at (0.25, 5.75) {};
		\node [style=none] (16) at (-0.25, 5.25) {};
		\node [style=none] (17) at (-0.25, 6.25) {};
		\node [style=none] (18) at (0.25, 6.25) {};
		\node [style=X] (19) at (0.25, 5.25) {};
	\end{pgfonlayer}
	\begin{pgfonlayer}{edgelayer}
		\draw (16.center) to (14);
		\draw (14) to (17.center);
		\draw (15) to (14);
		\draw (18.center) to (15);
		\draw (15) to (19);
	\end{pgfonlayer}
\end{tikzpicture}
&\eq{unit}
\begin{tikzpicture}[tikzfig]
	\begin{pgfonlayer}{nodelayer}
		\node [style=dot] (15) at (-0.25, 6.5) {};
		\node [style=oplus] (16) at (0.25, 6.5) {};
		\node [style=none] (17) at (-0.25, 5.25) {};
		\node [style=none] (18) at (-0.25, 7) {};
		\node [style=none] (19) at (0.25, 7) {};
		\node [style=X] (20) at (0.5, 5.25) {};
		\node [style=fanout] (21) at (0.5, 5.75) {};
		\node [style=X] (22) at (0.75, 6.5) {};
	\end{pgfonlayer}
	\begin{pgfonlayer}{edgelayer}
		\draw (17.center) to (15);
		\draw (15) to (18.center);
		\draw (16) to (15);
		\draw (19.center) to (16);
		\draw [in=117, out=-90] (16) to (21);
		\draw (21) to (20);
		\draw [in=63, out=-90] (22) to (21);
	\end{pgfonlayer}
\end{tikzpicture}
=
\begin{tikzpicture}[tikzfig]
	\begin{pgfonlayer}{nodelayer}
		\node [style=dot] (16) at (-0.25, 5.75) {};
		\node [style=oplus] (17) at (0.25, 5.75) {};
		\node [style=none] (18) at (-0.25, 5.25) {};
		\node [style=none] (19) at (-0.25, 6.25) {};
		\node [style=none] (20) at (0.25, 6.25) {};
		\node [style=X] (21) at (0.75, 5.75) {};
	\end{pgfonlayer}
	\begin{pgfonlayer}{edgelayer}
		\draw (18.center) to (16);
		\draw (16) to (19.center);
		\draw (17) to (16);
		\draw (20.center) to (17);
		\draw [in=-90, out=-90, looseness=2.25] (17) to (21);
	\end{pgfonlayer}
\end{tikzpicture}\
\eq{\ref{lemma:cnotslide}}
\begin{tikzpicture}[tikzfig]
	\begin{pgfonlayer}{nodelayer}
		\node [style=dot] (17) at (0, 5.75) {};
		\node [style=oplus] (18) at (0.75, 5.75) {};
		\node [style=none] (19) at (0, 5.25) {};
		\node [style=none] (20) at (0, 6.25) {};
		\node [style=none] (21) at (0.25, 6.25) {};
		\node [style=X] (22) at (0.75, 6.25) {};
		\node [style=none] (23) at (0.25, 5.5) {};
		\node [style=none] (24) at (0.75, 5.5) {};
	\end{pgfonlayer}
	\begin{pgfonlayer}{edgelayer}
		\draw (19.center) to (17);
		\draw (17) to (20.center);
		\draw (18) to (17);
		\draw (22) to (18);
		\draw (18) to (24.center);
		\draw (23.center) to (21.center);
		\draw [in=-90, out=-90, looseness=1.25] (23.center) to (24.center);
	\end{pgfonlayer}
\end{tikzpicture}\
=
\begin{tikzpicture}[tikzfig]
	\begin{pgfonlayer}{nodelayer}
		\node [style=dot] (18) at (0, 6.25) {};
		\node [style=oplus] (19) at (0.75, 6.25) {};
		\node [style=none] (20) at (0, 5.25) {};
		\node [style=none] (21) at (0, 6.75) {};
		\node [style=none] (22) at (0.25, 6.25) {};
		\node [style=X] (23) at (0.75, 6.75) {};
		\node [style=fanout] (24) at (0.5, 5.75) {};
		\node [style=X] (25) at (0.5, 5.25) {};
		\node [style=none] (26) at (0.25, 6.75) {};
	\end{pgfonlayer}
	\begin{pgfonlayer}{edgelayer}
		\draw (20.center) to (18);
		\draw (18) to (21.center);
		\draw (19) to (18);
		\draw (23) to (19);
		\draw [in=63, out=-90] (19) to (24);
		\draw [in=-90, out=117] (24) to (22.center);
		\draw (24) to (25);
		\draw (26.center) to (22.center);
	\end{pgfonlayer}
\end{tikzpicture}
\eq{\ref{CNOT.2}}
\begin{tikzpicture}[tikzfig]
	\begin{pgfonlayer}{nodelayer}
		\node [style=dot] (19) at (-0.25, 6.25) {};
		\node [style=oplus] (20) at (0.75, 6.25) {};
		\node [style=none] (21) at (-0.25, 5.25) {};
		\node [style=none] (22) at (0.25, 6.25) {};
		\node [style=X] (23) at (0.75, 6.75) {};
		\node [style=fanout] (24) at (0.5, 5.75) {};
		\node [style=X] (25) at (0.5, 5.25) {};
		\node [style=oplus] (26) at (0.25, 7.25) {};
		\node [style=dot] (27) at (-0.25, 7.25) {};
		\node [style=none] (28) at (0.25, 7.75) {};
		\node [style=none] (29) at (-0.25, 7.75) {};
		\node [style=dot] (30) at (-0.25, 6.75) {};
		\node [style=oplus] (31) at (0.25, 6.75) {};
	\end{pgfonlayer}
	\begin{pgfonlayer}{edgelayer}
		\draw (21.center) to (19);
		\draw (20) to (19);
		\draw (23) to (20);
		\draw [in=63, out=-90] (20) to (24);
		\draw [in=-90, out=117] (24) to (22.center);
		\draw (24) to (25);
		\draw (26) to (27);
		\draw (31) to (30);
		\draw (28.center) to (26);
		\draw (26) to (31);
		\draw (31) to (22.center);
		\draw (19) to (30);
		\draw (30) to (27);
		\draw (27) to (29.center);
	\end{pgfonlayer}
\end{tikzpicture}\
\eq{Lem. \ref{lemma:natoplus}}
\begin{tikzpicture}[tikzfig]
	\begin{pgfonlayer}{nodelayer}
		\node [style=X] (20) at (0.75, 7.25) {};
		\node [style=fanout] (21) at (0.5, 6.25) {};
		\node [style=none] (22) at (0.25, 8.25) {};
		\node [style=none] (23) at (-0.25, 8.25) {};
		\node [style=dot] (24) at (-0.25, 6.75) {};
		\node [style=oplus] (25) at (0.25, 6.75) {};
		\node [style=X] (26) at (0.5, 5.25) {};
		\node [style=dot] (27) at (-0.25, 5.75) {};
		\node [style=oplus] (28) at (0.5, 5.75) {};
		\node [style=none] (29) at (-0.25, 5.25) {};
		\node [style=none] (30) at (0.75, 6.75) {};
	\end{pgfonlayer}
	\begin{pgfonlayer}{edgelayer}
		\draw (25) to (24);
		\draw (28) to (27);
		\draw (27) to (29.center);
		\draw (26) to (28);
		\draw (28) to (21);
		\draw [in=-90, out=63] (21) to (30.center);
		\draw (30.center) to (20);
		\draw (25) to (22.center);
		\draw (23.center) to (24);
		\draw [in=117, out=-90] (25) to (21);
		\draw (27) to (24);
	\end{pgfonlayer}
\end{tikzpicture}\\
&\eq{unit}
\begin{tikzpicture}[tikzfig]
	\begin{pgfonlayer}{nodelayer}
		\node [style=none] (21) at (0.25, 6.75) {};
		\node [style=none] (22) at (-0.25, 6.75) {};
		\node [style=dot] (23) at (-0.25, 6.25) {};
		\node [style=oplus] (24) at (0.25, 6.25) {};
		\node [style=X] (25) at (0.25, 5.25) {};
		\node [style=dot] (26) at (-0.25, 5.75) {};
		\node [style=oplus] (27) at (0.25, 5.75) {};
		\node [style=none] (28) at (-0.25, 5.25) {};
	\end{pgfonlayer}
	\begin{pgfonlayer}{edgelayer}
		\draw (24) to (23);
		\draw (27) to (26);
		\draw (26) to (28.center);
		\draw (25) to (27);
		\draw (24) to (21.center);
		\draw (22.center) to (23);
		\draw (26) to (23);
		\draw (24) to (27);
	\end{pgfonlayer}
\end{tikzpicture}
\eq{\ref{CNOT.2}}
\begin{tikzpicture}[tikzfig]
	\begin{pgfonlayer}{nodelayer}
		\node [style=none] (22) at (0.25, 6) {};
		\node [style=none] (23) at (-0.25, 6) {};
		\node [style=X] (24) at (0.25, 5.25) {};
		\node [style=none] (25) at (-0.25, 5.25) {};
	\end{pgfonlayer}
	\begin{pgfonlayer}{edgelayer}
		\draw (22.center) to (24);
		\draw (25.center) to (23.center);
	\end{pgfonlayer}
\end{tikzpicture}
\end{align*}
\end{proof}


\subsection{Basic properties of \texorpdfstring{$\ZXA$}{ZX\&}}

\begin{lemma}
\label{lem:blackdot}
$$
\begin{tikzpicture}[tikzfig]
	\begin{pgfonlayer}{nodelayer}
		\node [style=Z] (23) at (0, 5.25) {};
	\end{pgfonlayer}
\end{tikzpicture}
=
\begin{tikzpicture}
	\begin{pgfonlayer}{nodelayer}
		\node [style=none] (0) at (0, -0) {};
	\end{pgfonlayer}
\end{tikzpicture}
$$
\end{lemma}
\begin{proof}
\begin{align*}
\begin{tikzpicture}
	\begin{pgfonlayer}{nodelayer}
		\node [style=Z] (0) at (0, -0) {};
	\end{pgfonlayer}
\end{tikzpicture}
\eq{\ref{ZXA.1}}
\begin{tikzpicture}[tikzfig]
	\begin{pgfonlayer}{nodelayer}
		\node [style=Z] (1) at (-0.5, -0.75) {};
		\node [style=Z] (2) at (0, -0.75) {};
	\end{pgfonlayer}
	\begin{pgfonlayer}{edgelayer}
		\draw [in=90, out=90, looseness=2.25] (2) to (1);
	\end{pgfonlayer}
\end{tikzpicture}
\eq{\ref{ZXA.3}}
\begin{tikzpicture}[tikzfig]
	\begin{pgfonlayer}{nodelayer}
		\node [style=X] (2) at (0, 0) {};
		\node [style=Z] (3) at (-0.25, -0.75) {};
		\node [style=Z] (4) at (0.25, -0.75) {};
		\node [style=X] (5) at (0, 0.75) {};
	\end{pgfonlayer}
	\begin{pgfonlayer}{edgelayer}
		\draw (5) to (2);
		\draw [in=90, out=-124] (2) to (3);
		\draw [in=-56, out=90] (4) to (2);
	\end{pgfonlayer}
\end{tikzpicture}
\eq{\ref{ZXA.6}}
\begin{tikzpicture}[tikzfig]
	\begin{pgfonlayer}{nodelayer}
		\node [style=X] (3) at (0, 0.75) {};
		\node [style=X] (4) at (0, 0) {};
		\node [style=Z] (5) at (0, -1.75) {};
		\node [style=X] (6) at (0, -1) {};
	\end{pgfonlayer}
	\begin{pgfonlayer}{edgelayer}
		\draw (3) to (4);
		\draw (5) to (6);
		\draw [bend left, looseness=1.25] (6) to (4);
		\draw [bend left, looseness=1.25] (4) to (6);
	\end{pgfonlayer}
\end{tikzpicture}
\eq{\ref{ZXA.3}}
\begin{tikzpicture}[tikzfig]
	\begin{pgfonlayer}{nodelayer}
		\node [style=X] (4) at (0, 6.75) {};
		\node [style=Z] (5) at (0, 5.75) {};
	\end{pgfonlayer}
	\begin{pgfonlayer}{edgelayer}
		\draw (5) to (4);
	\end{pgfonlayer}
\end{tikzpicture}
\eq{\ref{ZXA.7}}
\end{align*}
\end{proof}




\begin{lemma}
The phase fusion of the black spider in $\ZXA$, 
$$
\begin{tikzpicture}[tikzfig]
	\begin{pgfonlayer}{nodelayer}
		\node [style=Z] (5) at (-0.5, 5.75) {$\pi$};
		\node [style=Z] (6) at (0, 5.75) {$\pi$};
		\node [style=Z] (7) at (-0.25, 6.5) {};
		\node [style=none] (8) at (-0.25, 7) {};
	\end{pgfonlayer}
	\begin{pgfonlayer}{edgelayer}
		\draw [in=-108, out=90] (5) to (7);
		\draw [in=90, out=-72] (7) to (6);
		\draw (7) to (8.center);
	\end{pgfonlayer}
\end{tikzpicture}
=
\begin{tikzpicture}[tikzfig]
	\begin{pgfonlayer}{nodelayer}
		\node [style=Z] (6) at (-0.25, 5.75) {};
		\node [style=none] (7) at (-0.25, 6.25) {};
	\end{pgfonlayer}
	\begin{pgfonlayer}{edgelayer}
		\draw (6) to (7.center);
	\end{pgfonlayer}
\end{tikzpicture}
$$
in the presence of the other axioms is equivalent to asserting:
$$
\begin{tikzpicture}[tikzfig]
	\begin{pgfonlayer}{nodelayer}
		\node [style=Z] (7) at (-0.5, 5.75) {$\pi$};
		\node [style=X] (8) at (-0.5, 6.5) {};
	\end{pgfonlayer}
	\begin{pgfonlayer}{edgelayer}
		\draw (8) to (7);
	\end{pgfonlayer}
\end{tikzpicture}
=
\begin{tikzpicture}[tikzfig]
	\begin{pgfonlayer}{nodelayer}
		\node [style=none] (8) at (-0.5, 5.75) {};
	\end{pgfonlayer}
\end{tikzpicture}
$$
Or in other terms, the phase fusion of the black spider is equivalent to the interaction of the unit for and and the counit for copying as a bialgebra.
\end{lemma}

\begin{proof}
For the one direction, suppose that phase fusion holds:


\begin{align*}
\begin{tikzpicture}[tikzfig]
	\begin{pgfonlayer}{nodelayer}
		\node [style=Z] (9) at (-0.5, 6) {$\pi$};
		\node [style=X] (10) at (-0.5, 6.75) {};
	\end{pgfonlayer}
	\begin{pgfonlayer}{edgelayer}
		\draw (10) to (9);
	\end{pgfonlayer}
\end{tikzpicture}
\eq{\ref{ZXA.3}}
\begin{tikzpicture}[tikzfig]
	\begin{pgfonlayer}{nodelayer}
		\node [style=Z] (10) at (-0.5, 6) {$\pi$};
		\node [style=X] (11) at (-0.5, 6.75) {};
	\end{pgfonlayer}
	\begin{pgfonlayer}{edgelayer}
		\draw (11) to (10);
		\draw [in=135, out=45, loop] (11) to ();
	\end{pgfonlayer}
\end{tikzpicture}
\eq{\ref{ZXA.1}}
\begin{tikzpicture}[tikzfig]
	\begin{pgfonlayer}{nodelayer}
		\node [style=Z] (11) at (-0.5, 6) {$\pi$};
		\node [style=X] (12) at (-0.5, 6.75) {};
		\node [style=Z] (13) at (-0.5, 7.75) {};
		\node [style=Z] (14) at (-0.5, 8.5) {};
	\end{pgfonlayer}
	\begin{pgfonlayer}{edgelayer}
		\draw (12) to (11);
		\draw [bend left=45, looseness=1.25] (12) to (13);
		\draw [bend left, looseness=1.25] (13) to (12);
		\draw (13) to (14);
	\end{pgfonlayer}
\end{tikzpicture}
\eq{\ref{ZXA.8}}
\begin{tikzpicture}[tikzfig]
	\begin{pgfonlayer}{nodelayer}
		\node [style=Z] (12) at (-1, 6) {$\pi$};
		\node [style=Z] (13) at (-0.5, 7) {};
		\node [style=Z] (14) at (-0.5, 7.75) {};
		\node [style=Z] (15) at (0, 6) {$\pi$};
	\end{pgfonlayer}
	\begin{pgfonlayer}{edgelayer}
		\draw (13) to (14);
		\draw [in=90, out=-117] (13) to (12);
		\draw [in=-63, out=90] (15) to (13);
	\end{pgfonlayer}
\end{tikzpicture}
=
\begin{tikzpicture}[tikzfig]
	\begin{pgfonlayer}{nodelayer}
		\node [style=Z] (13) at (-0.5, 6) {};
		\node [style=Z] (14) at (-0.5, 6.75) {};
	\end{pgfonlayer}
	\begin{pgfonlayer}{edgelayer}
		\draw (13) to (14);
	\end{pgfonlayer}
\end{tikzpicture}
\eq{\ref{ZXA.7},\ref{lem:blackdot}}
\begin{tikzpicture}[tikzfig]
	\begin{pgfonlayer}{nodelayer}
		\node [style=none] (14) at (-0.5, 6) {};
	\end{pgfonlayer}
\end{tikzpicture}
\end{align*}


Conversely if the unit part of the bialgebra rule holds:

\begin{align*}
\begin{tikzpicture}[tikzfig]
	\begin{pgfonlayer}{nodelayer}
		\node [style=Z] (15) at (-0.5, 6) {$\pi$};
		\node [style=Z] (16) at (0, 6) {$\pi$};
		\node [style=Z] (17) at (-0.25, 6.75) {};
		\node [style=none] (18) at (-0.25, 7.25) {};
	\end{pgfonlayer}
	\begin{pgfonlayer}{edgelayer}
		\draw [in=-108, out=90] (15) to (17);
		\draw [in=90, out=-72] (17) to (16);
		\draw (17) to (18.center);
	\end{pgfonlayer}
\end{tikzpicture}
\eq{\ref{ZXA.14}}
\begin{tikzpicture}[tikzfig]
	\begin{pgfonlayer}{nodelayer}
		\node [style=Z] (16) at (-0.25, 7.5) {};
		\node [style=none] (17) at (-0.25, 8) {};
		\node [style=Z] (18) at (-0.25, 6) {$\pi$};
		\node [style=X] (19) at (-0.25, 6.75) {};
	\end{pgfonlayer}
	\begin{pgfonlayer}{edgelayer}
		\draw (16) to (17.center);
		\draw [in=120, out=-120, looseness=1.25] (16) to (19);
		\draw (19) to (18);
		\draw [in=-60, out=60, looseness=1.25] (19) to (16);
	\end{pgfonlayer}
\end{tikzpicture}
\eq{\ref{ZXA.8}}
\begin{tikzpicture}[tikzfig]
	\begin{pgfonlayer}{nodelayer}
		\node [style=Z] (17) at (-0.25, 7.5) {};
		\node [style=none] (18) at (-0.25, 8) {};
		\node [style=Z] (19) at (-0.25, 6) {$\pi$};
		\node [style=X] (20) at (-0.25, 6.75) {};
	\end{pgfonlayer}
	\begin{pgfonlayer}{edgelayer}
		\draw (17) to (18.center);
		\draw (20) to (19);
	\end{pgfonlayer}
\end{tikzpicture}
=
\begin{tikzpicture}[tikzfig]
	\begin{pgfonlayer}{nodelayer}
		\node [style=Z] (18) at (-0.25, 6) {};
		\node [style=none] (19) at (-0.25, 6.5) {};
	\end{pgfonlayer}
	\begin{pgfonlayer}{edgelayer}
		\draw (18) to (19.center);
	\end{pgfonlayer}
\end{tikzpicture}
\end{align*}


\end{proof}


\begin{lemma}
\label{lem:oldaxiom}
$$
\begin{tikzpicture}[tikzfig]
	\begin{pgfonlayer}{nodelayer}
		\node [style=Z] (19) at (0, 6.25) {};
		\node [style=none] (20) at (0.5, 7.25) {};
		\node [style=andin] (21) at (0.5, 7.25) {};
		\node [style=none] (22) at (0.5, 8) {};
		\node [style=none] (23) at (1, 6.25) {};
		\node [style=none] (24) at (1, 6) {};
	\end{pgfonlayer}
	\begin{pgfonlayer}{edgelayer}
		\draw [in=90, out=-63] (20.center) to (23.center);
		\draw (22.center) to (20.center);
		\draw [in=90, out=-117] (20.center) to (19);
		\draw (24.center) to (23.center);
	\end{pgfonlayer}
\end{tikzpicture}
=
\begin{tikzpicture}[tikzfig]
	\begin{pgfonlayer}{nodelayer}
		\node [style=Z] (20) at (0.5, 7.5) {};
		\node [style=none] (21) at (0.5, 8.25) {};
		\node [style=none] (22) at (0.5, 6) {};
		\node [style=X] (23) at (0.5, 6.75) {};
	\end{pgfonlayer}
	\begin{pgfonlayer}{edgelayer}
		\draw (21.center) to (20);
		\draw (22.center) to (23);
	\end{pgfonlayer}
\end{tikzpicture}
$$
\end{lemma}

\begin{proof}
$$
\begin{tikzpicture}[tikzfig]
	\begin{pgfonlayer}{nodelayer}
		\node [style=Z] (21) at (0, 6.25) {};
		\node [style=none] (22) at (0.5, 7.25) {};
		\node [style=andin] (23) at (0.5, 7.25) {};
		\node [style=none] (24) at (0.5, 8) {};
		\node [style=none] (25) at (1, 6.25) {};
		\node [style=none] (26) at (1, 6) {};
	\end{pgfonlayer}
	\begin{pgfonlayer}{edgelayer}
		\draw [in=90, out=-63] (22.center) to (25.center);
		\draw (24.center) to (22.center);
		\draw [in=90, out=-117] (22.center) to (21);
		\draw (26.center) to (25.center);
	\end{pgfonlayer}
\end{tikzpicture}
\eq{\ref{ZXA.1}}
\begin{tikzpicture}[tikzfig]
	\begin{pgfonlayer}{nodelayer}
		\node [style=Z] (22) at (0, 7) {};
		\node [style=none] (23) at (0.5, 8.75) {};
		\node [style=none] (24) at (1, 6) {};
		\node [style=none] (25) at (1, 7) {};
		\node [style=andin] (26) at (0.5, 8) {};
		\node [style=none] (27) at (0.5, 8) {};
		\node [style=Z] (28) at (-0.25, 6.25) {$\pi$};
		\node [style=Z] (29) at (0.25, 6.25) {$\pi$};
	\end{pgfonlayer}
	\begin{pgfonlayer}{edgelayer}
		\draw (24.center) to (25.center);
		\draw (23.center) to (27.center);
		\draw [in=90, out=-63] (27.center) to (25.center);
		\draw [in=90, out=-117] (27.center) to (22);
		\draw [in=90, out=-108] (22) to (28);
		\draw [in=-72, out=90] (29) to (22);
	\end{pgfonlayer}
\end{tikzpicture}
\eq{\ref{ZXA.17}}
\begin{tikzpicture}[tikzfig]
	\begin{pgfonlayer}{nodelayer}
		\node [style=none] (23) at (-2.25, 7.5) {};
		\node [style=none] (24) at (-0.75, 7.5) {};
		\node [style=X] (25) at (-0.75, 6.5) {};
		\node [style=none] (26) at (-1.5, 9) {};
		\node [style=none] (27) at (-0.75, 6) {};
		\node [style=Z] (28) at (-1.5, 6.25) {$\pi$};
		\node [style=Z] (29) at (-2.25, 6.25) {$\pi$};
		\node [style=Z] (30) at (-1.5, 8.25) {};
		\node [style=andin] (31) at (-2.25, 7.5) {};
		\node [style=andin] (32) at (-0.75, 7.5) {};
	\end{pgfonlayer}
	\begin{pgfonlayer}{edgelayer}
		\draw (26.center) to (30);
		\draw [in=90, out=-150] (30) to (23.center);
		\draw (23.center) to (29);
		\draw [in=90, out=-121] (24.center) to (28);
		\draw [in=-30, out=90] (24.center) to (30);
		\draw [in=146, out=-45] (23.center) to (25);
		\draw (25) to (24.center);
		\draw (25) to (27.center);
	\end{pgfonlayer}
\end{tikzpicture}
\eq{\ref{ZXA.10}}
\begin{tikzpicture}[tikzfig]
	\begin{pgfonlayer}{nodelayer}
		\node [style=none] (24) at (1, 6) {};
		\node [style=X] (25) at (1, 6.75) {};
		\node [style=none] (26) at (1, 8.5) {};
		\node [style=Z] (27) at (1, 7.75) {};
	\end{pgfonlayer}
	\begin{pgfonlayer}{edgelayer}
		\draw (24.center) to (25);
		\draw (27) to (26.center);
		\draw [in=120, out=-120, looseness=1.25] (27) to (25);
		\draw [in=-60, out=60, looseness=1.25] (25) to (27);
	\end{pgfonlayer}
\end{tikzpicture}
\eq{\ref{ZXA.8}}
\begin{tikzpicture}[tikzfig]
	\begin{pgfonlayer}{nodelayer}
		\node [style=Z] (25) at (0.5, 7.5) {};
		\node [style=none] (26) at (0.5, 8.25) {};
		\node [style=none] (27) at (0.5, 6) {};
		\node [style=X] (28) at (0.5, 6.75) {};
	\end{pgfonlayer}
	\begin{pgfonlayer}{edgelayer}
		\draw (26.center) to (25);
		\draw (27.center) to (28);
	\end{pgfonlayer}
\end{tikzpicture}
$$
\end{proof}

\section{The identities of \texorpdfstring{$\CNOT$}{CNOT}}

The category $\CNOT$  \cite{cnot} is the $\dag$-symmetric monoidal subcategory of $\TOF$ generated by the controlled not gate and ancillary bits $|1\rangle$, $\langle 1|$.  A complete set of identities is presented in the following figure, because some of the identities are used in the translation between $\ZXA$ and the (co)unital completion of $\TOF$.



\begin{figure}[H]
	\noindent
	\scalebox{1.0}{%
		\vbox{%
			\begin{mdframed}
				\begin{multicols}{2}
					\begin{enumerate}[label={\bf [CNOT.\arabic*]}, ref={\bf [CNOT.\arabic*]}, wide = 0pt, leftmargin = 2em]
						\item
						\label{CNOT.1}
						{\hfil
							$
			\begin{tikzpicture}[tikzfig]
	\begin{pgfonlayer}{nodelayer}
		\node [style=nothing] (26) at (0, 6) {};
		\node [style=nothing] (27) at (-0.5, 6) {};
		\node [style=oplus] (28) at (0, 6.5) {};
		\node [style=dot] (29) at (-0.5, 6.5) {};
		\node [style=dot] (30) at (0, 7) {};
		\node [style=oplus] (31) at (-0.5, 7) {};
		\node [style=oplus] (32) at (0, 7.5) {};
		\node [style=dot] (33) at (-0.5, 7.5) {};
		\node [style=nothing] (34) at (0, 8) {};
		\node [style=nothing] (35) at (-0.5, 8) {};
	\end{pgfonlayer}
	\begin{pgfonlayer}{edgelayer}
		\draw [style=simple] (26) to (34);
		\draw [style=simple] (27) to (35);
		\draw [style=simple] (28) to (29);
		\draw [style=simple] (30) to (31);
		\draw [style=simple] (32) to (33);
	\end{pgfonlayer}
\end{tikzpicture}

							=
							\begin{tikzpicture}[tikzfig]
	\begin{pgfonlayer}{nodelayer}
		\node [style=nothing] (0) at (0, 0.5) {};
		\node [style=nothing] (1) at (-0.5, 0.5) {};
		\node [style=nothing] (2) at (-0.5, 1.5) {};
		\node [style=nothing] (3) at (0, 1.5) {};
	\end{pgfonlayer}
	\begin{pgfonlayer}{edgelayer}
		\draw [in=-90, out=90, looseness=1.25] (1) to (3);
		\draw [in=-90, out=90, looseness=1.25] (0) to (2);
	\end{pgfonlayer}
\end{tikzpicture}
$}
						
						
						\item
						\label{CNOT.2}
						\hfil{
							$
							\begin{tikzpicture}[tikzfig]
	\begin{pgfonlayer}{nodelayer}
		\node [style=nothing] (1) at (0, 0) {};
		\node [style=nothing] (2) at (-0.5, 0) {};
		\node [style=oplus] (3) at (0, 0.5) {};
		\node [style=dot] (4) at (-0.5, 0.5) {};
		\node [style=oplus] (5) at (0, 1) {};
		\node [style=dot] (6) at (-0.5, 1) {};
		\node [style=nothing] (7) at (0, 1.5) {};
		\node [style=nothing] (8) at (-0.5, 1.5) {};
	\end{pgfonlayer}
	\begin{pgfonlayer}{edgelayer}
		\draw [style=simple] (1) to (7);
		\draw [style=simple] (2) to (8);
		\draw [style=simple] (3) to (4);
		\draw [style=simple] (5) to (6);
	\end{pgfonlayer}
\end{tikzpicture}
							=
							\begin{tikzpicture}[tikzfig]
	\begin{pgfonlayer}{nodelayer}
		\node [style=nothing] (2) at (0, 0) {};
		\node [style=nothing] (3) at (-0.5, 0) {};
		\node [style=nothing] (4) at (0, 1.5) {};
		\node [style=nothing] (5) at (-0.5, 1.5) {};
	\end{pgfonlayer}
	\begin{pgfonlayer}{edgelayer}
		\draw [style=simple] (2) to (4);
		\draw [style=simple] (3) to (5);
	\end{pgfonlayer}
\end{tikzpicture}
							$}
						
						\item
						\label{CNOT.3}
						\hfil{
							$
							\begin{tikzpicture}[tikzfig]
	\begin{pgfonlayer}{nodelayer}
		\node [style=nothing] (3) at (-1, 0) {};
		\node [style=nothing] (4) at (-0.5, 0) {};
		\node [style=nothing] (5) at (0, 0) {};
		\node [style=oplus] (6) at (-1, 0.75) {};
		\node [style=dot] (7) at (-0.5, 0.75) {};
		\node [style=dot] (8) at (-0.5, 1.25) {};
		\node [style=oplus] (9) at (0, 1.25) {};
		\node [style=nothing] (10) at (-1, 2) {};
		\node [style=nothing] (11) at (-0.5, 2) {};
		\node [style=nothing] (12) at (0, 2) {};
	\end{pgfonlayer}
	\begin{pgfonlayer}{edgelayer}
		\draw [style=simple] (3) to (10);
		\draw [style=simple] (4) to (11);
		\draw [style=simple] (5) to (12);
		\draw [style=simple] (6) to (7);
		\draw [style=simple] (8) to (9);
	\end{pgfonlayer}
\end{tikzpicture}
							=
							\begin{tikzpicture}[tikzfig]
	\begin{pgfonlayer}{nodelayer}
		\node [style=nothing] (4) at (-1, 2.75) {};
		\node [style=nothing] (5) at (-0.5, 2.75) {};
		\node [style=nothing] (6) at (0, 2.75) {};
		\node [style=oplus] (7) at (-1, 4) {};
		\node [style=dot] (8) at (-0.5, 4) {};
		\node [style=dot] (9) at (-0.5, 3.5) {};
		\node [style=oplus] (10) at (0, 3.5) {};
		\node [style=nothing] (11) at (-1, 4.75) {};
		\node [style=nothing] (12) at (-0.5, 4.75) {};
		\node [style=nothing] (13) at (0, 4.75) {};
	\end{pgfonlayer}
	\begin{pgfonlayer}{edgelayer}
		\draw [style=simple] (4) to (11);
		\draw [style=simple] (5) to (12);
		\draw [style=simple] (6) to (13);
		\draw [style=simple] (7) to (8);
		\draw [style=simple] (9) to (10);
	\end{pgfonlayer}
\end{tikzpicture}
							$}
						
						\item 
						\label{CNOT.4}
						\hfil{
							\begin{tabular}{c}
							$
							\begin{tikzpicture}[tikzfig]
	\begin{pgfonlayer}{nodelayer}
		\node [style=onein] (5) at (-0.5, 2.75) {};
		\node [style=nothing] (6) at (0, 2.75) {};
		\node [style=dot] (7) at (-0.5, 3.25) {};
		\node [style=oplus] (8) at (0, 3.25) {};
		\node [style=nothing] (9) at (-0.5, 3.75) {};
		\node [style=nothing] (10) at (0, 3.75) {};
	\end{pgfonlayer}
	\begin{pgfonlayer}{edgelayer}
		\draw [style=simple] (5) to (9);
		\draw [style=simple] (6) to (10);
		\draw [style=simple] (7) to (8);
	\end{pgfonlayer}
\end{tikzpicture}
							=
							\begin{tikzpicture}[tikzfig]
	\begin{pgfonlayer}{nodelayer}
		\node [style=onein] (6) at (-0.5, 2.75) {};
		\node [style=nothing] (7) at (0, 2.75) {};
		\node [style=dot] (8) at (-0.5, 3.25) {};
		\node [style=oplus] (9) at (0, 3.25) {};
		\node [style=oneout] (10) at (-0.5, 3.75) {};
		\node [style=nothing] (11) at (0, 4.75) {};
		\node [style=onein] (12) at (-0.5, 4.25) {};
		\node [style=nothing] (13) at (-0.5, 4.75) {};
	\end{pgfonlayer}
	\begin{pgfonlayer}{edgelayer}
		\draw [style=simple] (6) to (10);
		\draw [style=simple] (7) to (11);
		\draw [style=simple] (8) to (9);
		\draw [style=simple] (12) to (13);
	\end{pgfonlayer}
\end{tikzpicture}$\\
							$ $\\
							$\begin{tikzpicture}
							\begin{pgfonlayer}{nodelayer}
							\node [style=nothing] (0) at (0, .5) {};
							\node [style=nothing] (1) at (0, 0) {};
							%
							\node [style=dot] (2) at (.5, .5) {};
							\node [style=oplus] (3) at (.5, 0) {};
							%
							\node [style=oneout] (4) at (1, .5) {};
							\node [style=nothing] (5) at (1, 0) {};
							\end{pgfonlayer}
							\begin{pgfonlayer}{edgelayer}
							\draw [style=simple] (0) to (4);
							\draw [style=simple] (1) to (5);
							\draw [style=simple] (2) to (3);
							\end{pgfonlayer}
							\end{tikzpicture}
							=
							\begin{tikzpicture}[tikzfig]
	\begin{pgfonlayer}{nodelayer}
		\node [style=oneout] (8) at (-0.5, 7.25) {};
		\node [style=nothing] (9) at (0, 7.25) {};
		\node [style=dot] (10) at (-0.5, 6.75) {};
		\node [style=oplus] (11) at (0, 6.75) {};
		\node [style=onein] (12) at (-0.5, 6.25) {};
		\node [style=nothing] (13) at (0, 5.25) {};
		\node [style=oneout] (14) at (-0.5, 5.75) {};
		\node [style=nothing] (15) at (-0.5, 5.25) {};
	\end{pgfonlayer}
	\begin{pgfonlayer}{edgelayer}
		\draw [style=simple] (8) to (12);
		\draw [style=simple] (9) to (13);
		\draw [style=simple] (10) to (11);
		\draw [style=simple] (14) to (15);
	\end{pgfonlayer}
\end{tikzpicture}$
							\end{tabular}
							}
						
						\item 
						\label{CNOT.5}
						\hfil{
							$
							\begin{tikzpicture}[tikzfig]
	\begin{pgfonlayer}{nodelayer}
		\node [style=nothing] (9) at (-1, 5.25) {};
		\node [style=nothing] (10) at (-0.5, 5.25) {};
		\node [style=nothing] (11) at (0, 5.25) {};
		\node [style=dot] (12) at (-1, 6) {};
		\node [style=oplus] (13) at (-0.5, 6) {};
		\node [style=oplus] (14) at (-0.5, 6.5) {};
		\node [style=dot] (15) at (0, 6.5) {};
		\node [style=nothing] (16) at (-1, 7.25) {};
		\node [style=nothing] (17) at (-0.5, 7.25) {};
		\node [style=nothing] (18) at (0, 7.25) {};
	\end{pgfonlayer}
	\begin{pgfonlayer}{edgelayer}
		\draw [style=simple] (9) to (16);
		\draw [style=simple] (10) to (17);
		\draw [style=simple] (11) to (18);
		\draw [style=simple] (12) to (13);
		\draw [style=simple] (14) to (15);
	\end{pgfonlayer}
\end{tikzpicture}
							=
							\begin{tikzpicture}[tikzfig]
	\begin{pgfonlayer}{nodelayer}
		\node [style=nothing] (10) at (-1, 5.25) {};
		\node [style=nothing] (11) at (-0.5, 5.25) {};
		\node [style=nothing] (12) at (0, 5.25) {};
		\node [style=dot] (13) at (-1, 6.5) {};
		\node [style=oplus] (14) at (-0.5, 6.5) {};
		\node [style=oplus] (15) at (-0.5, 6) {};
		\node [style=dot] (16) at (0, 6) {};
		\node [style=nothing] (17) at (-1, 7.25) {};
		\node [style=nothing] (18) at (-0.5, 7.25) {};
		\node [style=nothing] (19) at (0, 7.25) {};
	\end{pgfonlayer}
	\begin{pgfonlayer}{edgelayer}
		\draw [style=simple] (10) to (17);
		\draw [style=simple] (11) to (18);
		\draw [style=simple] (12) to (19);
		\draw [style=simple] (13) to (14);
		\draw [style=simple] (15) to (16);
	\end{pgfonlayer}
\end{tikzpicture}
							$}
						
						\item 
						\label{CNOT.6}
						\hfil{
							$
							\begin{tikzpicture}[tikzfig]
	\begin{pgfonlayer}{nodelayer}
		\node [style=onein] (11) at (0, 5.25) {};
		\node [style=oneout] (12) at (0, 6.25) {};
	\end{pgfonlayer}
	\begin{pgfonlayer}{edgelayer}
		\draw [style=simple] (11) to (12);
	\end{pgfonlayer}
\end{tikzpicture}
							=
							\begin{tikzpicture}[tikzfig]
	\begin{pgfonlayer}{nodelayer}
		\node [style=rn] (12) at (0, 5.25) {};
		\node [style=rn] (13) at (0, 6.25) {};
	\end{pgfonlayer}
\end{tikzpicture}
							$}
						
						\item 
						\label{CNOT.7}
						\hfil{
							\begin{tabular}{c}
							$\begin{tikzpicture}[tikzfig]
	\begin{pgfonlayer}{nodelayer}
		\node [style=onein] (13) at (-1, 5.25) {};
		\node [style=onein] (14) at (-0.5, 5.25) {};
		\node [style=nothing] (15) at (0, 5.25) {};
		\node [style=dot] (16) at (-1, 5.75) {};
		\node [style=oplus] (17) at (-0.5, 5.75) {};
		\node [style=dot] (18) at (-0.5, 6.25) {};
		\node [style=oplus] (19) at (0, 6.25) {};
		\node [style=oneout] (20) at (-1, 6.25) {};
		\node [style=nothing] (21) at (-0.5, 6.75) {};
		\node [style=nothing] (22) at (0, 6.75) {};
	\end{pgfonlayer}
	\begin{pgfonlayer}{edgelayer}
		\draw [style=simple] (13) to (20);
		\draw [style=simple] (14) to (21);
		\draw [style=simple] (15) to (22);
		\draw [style=simple] (16) to (17);
		\draw [style=simple] (18) to (19);
	\end{pgfonlayer}
\end{tikzpicture}
							=
							\begin{tikzpicture}[tikzfig]
	\begin{pgfonlayer}{nodelayer}
		\node [style=onein] (14) at (-1, 5.25) {};
		\node [style=onein] (15) at (-0.5, 5.25) {};
		\node [style=nothing] (16) at (0, 5.25) {};
		\node [style=dot] (17) at (-1, 5.75) {};
		\node [style=oplus] (18) at (-0.5, 5.75) {};
		\node [style=oneout] (19) at (-1, 6.25) {};
		\node [style=nothing] (20) at (-0.5, 6.75) {};
		\node [style=nothing] (21) at (0, 6.75) {};
	\end{pgfonlayer}
	\begin{pgfonlayer}{edgelayer}
		\draw [style=simple] (14) to (19);
		\draw [style=simple] (15) to (20);
		\draw [style=simple] (16) to (21);
		\draw [style=simple] (17) to (18);
	\end{pgfonlayer}
\end{tikzpicture}$\\
							$ $\\
							$\begin{tikzpicture}[tikzfig]
	\begin{pgfonlayer}{nodelayer}
		\node [style=oneout] (15) at (-1, 6.75) {};
		\node [style=oneout] (16) at (-0.5, 6.75) {};
		\node [style=nothing] (17) at (0, 6.75) {};
		\node [style=dot] (18) at (-1, 6.25) {};
		\node [style=oplus] (19) at (-0.5, 6.25) {};
		\node [style=dot] (20) at (-0.5, 5.75) {};
		\node [style=oplus] (21) at (0, 5.75) {};
		\node [style=onein] (22) at (-1, 5.75) {};
		\node [style=nothing] (23) at (-0.5, 5.25) {};
		\node [style=nothing] (24) at (0, 5.25) {};
	\end{pgfonlayer}
	\begin{pgfonlayer}{edgelayer}
		\draw [style=simple] (15) to (22);
		\draw [style=simple] (16) to (23);
		\draw [style=simple] (17) to (24);
		\draw [style=simple] (18) to (19);
		\draw [style=simple] (20) to (21);
	\end{pgfonlayer}
\end{tikzpicture}
							=
							\begin{tikzpicture}[tikzfig]
	\begin{pgfonlayer}{nodelayer}
		\node [style=oneout] (16) at (-1, 6.75) {};
		\node [style=oneout] (17) at (-0.5, 6.75) {};
		\node [style=nothing] (18) at (0, 6.75) {};
		\node [style=dot] (19) at (-1, 6.25) {};
		\node [style=oplus] (20) at (-0.5, 6.25) {};
		\node [style=onein] (21) at (-1, 5.75) {};
		\node [style=nothing] (22) at (-0.5, 5.25) {};
		\node [style=nothing] (23) at (0, 5.25) {};
	\end{pgfonlayer}
	\begin{pgfonlayer}{edgelayer}
		\draw [style=simple] (16) to (21);
		\draw [style=simple] (17) to (22);
		\draw [style=simple] (18) to (23);
		\draw [style=simple] (19) to (20);
	\end{pgfonlayer}
\end{tikzpicture}$
							\end{tabular}
							}
						
						\item 
						\label{CNOT.8}
						\hfil{
							$
							\begin{tikzpicture}[tikzfig]
	\begin{pgfonlayer}{nodelayer}
		\node [style=nothing] (17) at (-1, 5.25) {};
		\node [style=nothing] (18) at (-0.5, 5.25) {};
		\node [style=nothing] (19) at (0, 5.25) {};
		\node [style=dot] (20) at (-1, 5.75) {};
		\node [style=oplus] (21) at (-0.5, 5.75) {};
		\node [style=dot] (22) at (-0.5, 6.25) {};
		\node [style=oplus] (23) at (0, 6.25) {};
		\node [style=dot] (24) at (-1, 6.75) {};
		\node [style=oplus] (25) at (-0.5, 6.75) {};
		\node [style=nothing] (26) at (-1, 7.25) {};
		\node [style=nothing] (27) at (-0.5, 7.25) {};
		\node [style=nothing] (28) at (0, 7.25) {};
	\end{pgfonlayer}
	\begin{pgfonlayer}{edgelayer}
		\draw [style=simple] (17) to (26);
		\draw [style=simple] (18) to (27);
		\draw [style=simple] (19) to (28);
		\draw [style=simple] (20) to (21);
		\draw [style=simple] (22) to (23);
		\draw [style=simple] (24) to (25);
	\end{pgfonlayer}
\end{tikzpicture}
							=
							\begin{tikzpicture}[tikzfig]
	\begin{pgfonlayer}{nodelayer}
		\node [style=nothing] (18) at (-1, 5.25) {};
		\node [style=nothing] (19) at (-0.5, 5.25) {};
		\node [style=nothing] (20) at (0, 5.25) {};
		\node [style=dot] (21) at (-0.5, 5.75) {};
		\node [style=oplus] (22) at (0, 5.75) {};
		\node [style=dot] (23) at (-1, 6.25) {};
		\node [style=oplus] (24) at (0, 6.25) {};
		\node [style=nothing] (25) at (-1, 6.75) {};
		\node [style=nothing] (26) at (-0.5, 6.75) {};
		\node [style=nothing] (27) at (0, 6.75) {};
	\end{pgfonlayer}
	\begin{pgfonlayer}{edgelayer}
		\draw [style=simple] (18) to (25);
		\draw [style=simple] (19) to (26);
		\draw [style=simple] (20) to (27);
		\draw [style=simple] (21) to (22);
		\draw [style=simple] (23) to (24);
	\end{pgfonlayer}
\end{tikzpicture}
							$}
						
						\item 
						\label{CNOT.9}
						\hfil{
							$
							\begin{tikzpicture}[tikzfig]
	\begin{pgfonlayer}{nodelayer}
		\node [style=onein] (19) at (-1, 5.25) {};
		\node [style=onein] (20) at (-0.5, 5.25) {};
		\node [style=nothing] (21) at (0, 5.25) {};
		\node [style=dot] (22) at (-1, 5.75) {};
		\node [style=oplus] (23) at (-0.5, 5.75) {};
		\node [style=oneout] (24) at (-1, 6.25) {};
		\node [style=oneout] (25) at (-0.5, 6.25) {};
		\node [style=nothing] (26) at (0, 6.25) {};
	\end{pgfonlayer}
	\begin{pgfonlayer}{edgelayer}
		\draw [style=simple] (19) to (24);
		\draw [style=simple] (20) to (25);
		\draw [style=simple] (21) to (26);
		\draw [style=simple] (22) to (23);
	\end{pgfonlayer}
\end{tikzpicture}
							=
							\begin{tikzpicture}[tikzfig]
	\begin{pgfonlayer}{nodelayer}
		\node [style=onein] (20) at (-1, 5.25) {};
		\node [style=onein] (21) at (-0.5, 5.25) {};
		\node [style=nothing] (22) at (0, 5.25) {};
		\node [style=dot] (23) at (-1, 6.25) {};
		\node [style=oplus] (24) at (-0.5, 6.25) {};
		\node [style=oneout] (25) at (-1, 7.25) {};
		\node [style=oneout] (26) at (-0.5, 7.25) {};
		\node [style=nothing] (27) at (0, 7.25) {};
		\node [style=oneout] (28) at (0, 6) {};
		\node [style=onein] (29) at (0, 6.5) {};
	\end{pgfonlayer}
	\begin{pgfonlayer}{edgelayer}
		\draw [style=simple] (20) to (25);
		\draw [style=simple] (21) to (26);
		\draw [style=simple] (22) to (28);
		\draw [style=simple] (29) to (27);
		\draw [style=simple] (23) to (24);
	\end{pgfonlayer}
\end{tikzpicture}
							$}
					\end{enumerate}
				\end{multicols}
				\
			\end{mdframed}
	}}
	\caption{The identities of \texorpdfstring{$\CNOT$}{CNOT}}
	\label{fig:CNOT}
\end{figure}