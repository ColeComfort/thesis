% !TEX TS-program = pdflatex
% !TEX encoding = UTF-8 Unicode

% This is a simple template for a LaTeX document using the "article" class.
% See "book", "report", "letter" for other types of document.
\errorcontextlines=500

\documentclass[12pt]{ociamthesis}  % default square logo 


%\documentclass[a4paper,11pt,oneside]{bo222ok}
%\usepackage[DIV=14,BCOR=2mm,headinclude=true,footinclude=false]{typearea}


\usepackage{hhline}
\usepackage{bookmark}
\usepackage[table]{xcolor}% http://ctan.org/pkg/xcolor
\usepackage[all,cmtip]{xy} 
\usepackage{float}

\usepackage{tikzit}
\input{thesis.tikzstyles}
\input{thesis.tikzdefs}

\usepackage{comment}


\usepackage{mdframed}
\usepackage{arydshln}
\usepackage{multicol}
\renewcommand{\tilde}{\widetilde}
\usepackage{everypage}
\usepackage{lipsum}
\usepackage{amsthm}
\usepackage[inline]{enumitem}   
\usepackage{scalerel,stackengine}
\stackMath
\renewcommand\hat[1]{%
\savestack{\tmpbox}{\stretchto{%
  \scaleto{%
    \scalerel*[\widthof{\ensuremath{#1}}]{\kern-.6pt\bigwedge\kern-.6pt}%
    {\rule[-\textheight/2]{1ex}{\textheight}}%WIDTH-LIMITED BIG WEDGE
  }{\textheight}% 
}{0.5ex}}%
\stackon[1pt]{#1}{\tmpbox}%
}
\parskip 1ex



\newcommand{\bcell}{\cellcolor{black!10}}

\makeatletter
% This command ignores the optional argument for itemize and enumerate lists
\newcommand{\inlineitem}[1][]{%
\ifnum\enit@type=\tw@
    {\descriptionlabel{#1}}
  \hspace{\labelsep}%
\else
  \ifnum\enit@type=\z@
       \refstepcounter{\@listctr}\fi
    \quad\@itemlabel\hspace{\labelsep}%
\fi}
\makeatother
\parindent=0pt
 

%\usepackage{extpfeil}
%\newextarrow{\xleftarrowtail}{500(40)}{\leftarrow\relbar<}
%\newextarrow{\xrightarrowtail}{500(40)}{>\relbar\rightarrow}



\makeatletter
\def\proarrowfill@#1#2#3#4#5{%
  $\m@th\thickmuskip0mu\medmuskip\thickmuskip\thinmuskip\thickmuskip
   \relax#5#1\mkern-7mu%
   \cleaders\hbox{$#5\mkern-2mu#2\mkern-2mu$}\hfill
   \mathclap{#3}\mathclap{#2}%
   \cleaders\hbox{$#5\mkern-2mu#2\mkern-2mu$}\hfill
   \mkern-7mu#4$%
}
\def\rightproarrowfill@{%
  \proarrowfill@\relbar\relbar\mapstochar\rightarrow}
\newcommand\xproarrow[2][]{%
  \ext@arrow 0055{\rightproarrowfill@}{#1}{#2}}
\makeatother

\newcommand{\proarrow}{\xproarrow{}}



%\newcommand\xrightarrowtail[2][]{\ensurestackMath{\mathrel{%
%  \stackengine{1pt}{%
 %   \stackengine{0pt}{\rightarrowtail}{\scriptstyle#2}{O}{c}{F}{F}{S}%
%  }{\scriptstyle#1}{U}{c}{F}{F}{S}%
%}}}



%\newcommand\xleftarrowtail[2][]{\ensurestackMath{\mathrel{%
%  \stackengine{1pt}{%
%    \stackengine{0pt}{\leftarrowtail}{\scriptstyle#2}{O}{c}{F}{F}{S}%
%  }{\scriptstyle#1}{U}{c}{F}{F}{S}%
%}}}


%\newcommand{\xrightarrowtail}[1]{\!\!\stackrel{#1}{\xymatrix@C=0.78em{\ar@{>->}[r]&}}\!\!\!}
%\newcommand{\xleftarrowtail}[1]{\!\!\!\stackrel{#1}{\xymatrix@C=0.78em{&\ar@{>->}[l]}}\!\!}



\newcommand{\alr}{{\sf alr}}
\newcommand{\lr}{{\sf lr}}
\newcommand{\rel}{{\sf r}}
\newcommand{\aih}{{\sf aih}}
\newcommand{\ih}{{\sf ih}}


\newcommand{\xrightarrowtail}[1]{\!\!{\xymatrix@C=1em{\ar@{>->}[r]^{#1}&}}\!\!\!}
\newcommand{\xleftarrowtail}[1]{\!\!\!{\xymatrix@C=1em{&\ar@{>->}[l]_{#1}}}\!\!}


\newcommand{\xrightarrowiso}[1]{\!\!{\xymatrix@C=1em{\ar@{->}[r]^{#1}_\cong&}}\!\!\!}
\newcommand{\xleftarrowiso}[1]{\!\!\!{\xymatrix@C=1em{&\ar@{->}[l]_{#1}^\cong}}\!\!}




%\theoremstyle{theorem} 
  \newtheorem{theorem}{Theorem}[section]
  \newtheorem{corollary}[theorem]{Corollary}
  \newtheorem{lemma}[theorem]{Lemma}
  \newtheorem{proposition}[theorem]{Proposition}
  
%\theoremstyle{definition}    conjecture
  \newtheorem{definition}[theorem]{Definition}
  \newtheorem{example}[theorem]{Example}
  \newtheorem{conjecture}[theorem]{Conjecture}
  \newtheorem{remark}[theorem]{Remark}
  
  
\newcommand{\Mat}{\mathsf{Mat}}


\newcommand{\AND}{{\sf and}}
\newcommand{\Set}{\Sets}
\newcommand{\Mnd}{{\sf Mnd}}
\newcommand{\Map}{{\sf Map}}
\newcommand{\Monot}{{\sf Monot}}

\newcommand{\dom}{{\sf dom}}
\newcommand{\cod}{{\sf cod}}

\newcommand{\cnot}{\mathsf{cnot}}
\newcommand{\tof}{\mathsf{tof}}
\newcommand{\Not}{\mathsf{not}}
\newcommand{\zeroin}{|0\rangle}
\newcommand{\zeroout}{\langle 0|}
\newcommand{\CNOT}{\mathsf{CNOT}}
\newcommand{\Sets}{\mathsf{Set}}
\newcommand{\FSets}{\mathsf{FSet}}
\newcommand{\FinOrdMonot}{\mathsf{FinOrdMonot}}
%\newcommand{\FSet}{\mathsf{FinOrd}}
\newcommand{\FinOrd}{\mathsf{FinOrd}}
\newcommand{\FinMonot}{\mathsf{FinMonot}}
\newcommand{\Fin}{\mathsf{Fd}}
\newcommand{\TOF}{\mathsf{TOF}}
\newcommand{\Span}{\mathsf{Span}}
\newcommand{\dec}{\mathsf{dec}}
\newcommand{\Rel}{\mathsf{Rel}}
\newcommand{\FRel}{\mathsf{FRel}}
\newcommand{\op}{\mathsf{op}}
\newcommand{\co}{\mathsf{co}}
\newcommand{\Hilb}{\mathsf{Hilb}}
\newcommand{\FdHilb}{\mathsf{FHilb}}
\newcommand{\FHilb}{\mathsf{FHilb}}
\newcommand{\CPM}{\mathsf{CPM}}
\newcommand{\CP}{\mathsf{CP}}
\newcommand{\FPinj}{\mathsf{FPinj}}
\newcommand{\FPar}{\mathsf{FPar}}
\newcommand{\FSpan}{\mathsf{FSpan}}
\newcommand{\Pinj}{\mathsf{Pinj}}
\newcommand{\Par}{\mathsf{Par}}
\newcommand{\Aff}{\mathsf{Aff}}
\newcommand{\ParIso}{\mathsf{ParIso}}

\newcommand{\Total}{\mathsf{Total}}
%\newcommand{\CFrob}{\mathsf{CFrob}}
\newcommand{\tr}{\mathsf{Tr}}
\newcommand{\ox}{\otimes}
\newcommand{\Csp}{{\sf Cospan}}
\newcommand{\Corel}{{\sf Corel}}
\newcommand{\Bool}{\mathbb{B}}
\newcommand{\Iso}{{\sf Iso}}
\renewcommand{\P}{{\sf p}}
\newcommand{\pmul}{{\sf pmul}}

\newcommand{\Prof}{{\sf Prof}}
\newcommand{\Mod}{{\sf Mod}}

\newcommand{\unit}{{\sf unit}}
\newcommand{\comm}{{\sf comm}}
\newcommand{\assoc}{{\sf assoc}}
\newcommand{\inj}{{\sf Inj}}
\newcommand{\surj}{{\sf Surj}}
\newcommand{\PSurj}{{\sf PSurj}}

\newcommand{\pre}{{\sf pre}}
\newcommand{\poly}{{\sf poly}}
\newcommand{\sub}{{\sf sub}}

\newcommand{\C}{\mathbb{C}}
\newcommand{\R}{\mathbb{R}}
\newcommand{\CoPara}{{\sf CoPara}}

\newcommand{\ch}{{\sf ch}}
\newcommand{\m}{{\sf m}}
\newcommand{\cm}{{\sf cm}}
\newcommand{\cb}{{\sf cb}}
\newcommand{\pcm}{{\sf pcm}}
\renewcommand{\r}{{\sf r}}
%\newcommand{\scfrob}{{\sf scfrob}}

\newcommand{\bi}{{\sf b1}}
\newcommand{\bii}{{\sf b2}}
\newcommand{\biii}{{\sf b3}}
\newcommand{\biv}{{\sf b4}}

\newcommand{\Kl}{{\sf Kl}}
\newcommand{\Mon}{{\sf Mon}}

\newcommand{\ev}{{\sf ev}}

\renewcommand{\P}{{\sf p}}
\newcommand{\f}{\mathsf{f}}

\newcommand{\F}{\mathbb{F}}
\newcommand{\X}{\mathbb{X}}
\newcommand{\Y}{\mathbb{Y}}
\newcommand{\Z}{\mathbb{Z}}
\newcommand{\N}{\mathbb{N}}
\newcommand{\T}{\mathbb{T}}
\newcommand{\s}{\mathbb{S}}
\newcommand{\U}{\mathbb{U}}

\newcommand{\IH}{\mathbb{IH}}


\newcommand{\M}{\mathcal{M}}
\newcommand{\E}{\mathcal{E}}



\renewcommand{\sp}{\mathsf{sp}}
\newcommand{\pr}{\mathsf{p}}
\newcommand{\iso}{\mathsf{i}}







\newcounter{eq}

\makeatletter
\newcommand{\ltxlabel}{\ltx@label}
\makeatother

\newcommand{\eqzxa}[1]{%
\refstepcounter{eq}%
\ltxlabel{#1}%
\eqstack{#1}%
}




\newcommand{\eqstack}[1]{%
\stackrel{\scalebox{.6}{(\ref{#1})}}{=}%
}

\newcommand{\eq}[1]{\stackrel{\scalebox{.6}{#1}}{=}}

\newcommand{\defeq}[1]{\stackrel{\scalebox{.6}{#1}}{:=}}


\newcommand{\eref}{\eqstack}

\newcommand{\erefop}[1]{%
\stackrel{\scalebox{.6}{(\ref{#1})${}^\op$}}{=}%
}

\newcommand{\ZXA}{\mathsf{ZX}\textit{\&}}


\newcommand{\Vect}{\mathsf{Vect}}
\newcommand{\FVect}{\mathsf{FVect}}
\newcommand{\Lag}{\mathsf{Lag}}
\newcommand{\im}{\mathsf{im}}
\renewcommand{\ker}{\mathsf{ker}}
\newcommand{\ZX}{\mathsf{ZX}}
\newcommand{\ZH}{\mathsf{ZH}}
\DeclareMathSymbol{\bot}{\mathord}{symbols}{"3F}


\newcommand{\pullbackcorner}[1][dl]{\save*!/#1-1pc/#1:(-1,1)@^{|-}\restore}

\renewcommand{\epsilon}{\varepsilon}
\renewcommand{\phi}{\varphi}
\renewcommand{\bar}[1]{\overline{#1}\hspace*{.01cm}}

\newcommand{\Stab}{{\sf Stab}}
\newcommand{\LinRel}{\sf LinRel}


\newcommand{\Isot}{{\sf Isot}}
\newcommand{\Co}{{\sf Co}}

\newcommand{\B}{\mathbb{B}}

\newcommand{\STOCH}{\mathsf{STOCH}}

%\renewcommand\floatpagefraction{.9}
%\renewcommand\topfraction{.9}
%\renewcommand\bottomfraction{.9}
%\renewcommand\textfraction{.1}   
%\Setcounter{totalnumber}{50}
%\Setcounter{topnumber}{50}
%\Setcounter{bottomnumber}{50}


\usepackage{amsmath}
\usepackage{pict2e}

\newcommand{\lbparen}{\{
}

\newcommand{\rbparen}{ \}
}



\newcommand{\Cat}{{\sf Cat}}
\newcommand{\D}{\mathcal{D}}
%\newcommand{\sfa}{{\sf SFA}}
\newcommand{\one}{\mathbbm{1}}






\newdir{|>}{-<5pt,0pt>{
\begin{tikzpicture}[scale=.7]
	\begin{pgfonlayer}{nodelayer}
		\node [style=none] (0) at (0, 0) {};
		\node [style=none] (1) at (1, 0) {};
		\node [style=none] (2) at (-1, -0.25) {};
	\end{pgfonlayer}
	\begin{pgfonlayer}{edgelayer}
		\draw (2.center) to (0.center);
		\draw (0.center) to (1.center);
	\end{pgfonlayer}
\end{tikzpicture}
}}
\newdir{|<}{-<5pt,0pt>{
\begin{tikzpicture}[scale=.9]
	\begin{pgfonlayer}{nodelayer}
		\node [style=none] (0) at (0, -0.25) {};
		\node [style=none] (1) at (-1, -0.25) {};
		\node [style=none] (2) at (1, 0) {};
	\end{pgfonlayer}
	\begin{pgfonlayer}{edgelayer}
		\draw (2.center) to (0.center);
		\draw (0.center) to (1.center);
	\end{pgfonlayer}
\end{tikzpicture}
}}




\newcommand{\zcirc}{\begin{tikzpicture}
	\begin{pgfonlayer}{nodelayer}
		\node [style=Z] (0) at (0, 0) {};
	\end{pgfonlayer}
\end{tikzpicture}}

\newcommand{\xcirc}{\begin{tikzpicture}
	\begin{pgfonlayer}{nodelayer}
		\node [style=X] (0) at (0, 0) {};
	\end{pgfonlayer}
\end{tikzpicture}}



\newcommand{\skewpullbackcorner}[1][dl]{\save*!/#1-1.1pc/#1:(-.5,1)@^{|>}\restore}
\newcommand{\skewpushoutcorner}[1][dl]{\save*!/#1-1pc/#1:(-1,1)@^{|<}\restore}


\DeclareFontFamily{U}{mathx}{\hyphenchar\font45}
\DeclareFontShape{U}{mathx}{m}{n}{
      <5> <6> <7> <8> <9> <10>
      <10.95> <12> <14.4> <17.28> <20.74> <24.88>
      mathx10
      }{}
\DeclareSymbolFont{mathx}{U}{mathx}{m}{n}
\DeclareFontSubstitution{U}{mathx}{m}{n}
\DeclareMathAccent{\widecheck}{0}{mathx}{"71}
\DeclareMathAccent{\wideparen}{0}{mathx}{"75}

\def\cs#1{\texttt{\char`\\#1}}


\usepackage{amsmath}


\usepackage{hyperref}

\newcommand\numeq[2]%
  {\label{#2}\stackrel{\scriptscriptstyle(\mkern-1.5mu#1\mkern-1.5mu)}{=}}

\newcommand{\cubetopbl}{A}
\newcommand{\cubetopbr}{B}
\newcommand{\cubetopfl}{C}
\newcommand{\cubetopfr}{D}
\newcommand{\cubebotbl}{E}
\newcommand{\cubebotbr}{F}
\newcommand{\cubebotfl}{G}
\newcommand{\cubebotfr}{H}

\xymatrixrowsep{.5cm}
\xymatrixcolsep{.65cm}


\newcommand{\fa}{{\sf fa}}
\newcommand{\sfa}{{\sf sfa}}
\newcommand{\scfa}{{\sf scfa}}
\newcommand{\cfa}{{\sf cfa}}

\newcommand{\disc}
{{
\begin{tikzpicture}[scale=.5]
	\begin{pgfonlayer}{nodelayer}
		\node [style=none] (7) at (24.2, -0.25) {};
		\node [style=none] (8) at (24.2, 0.275) {};
		\node [style=none] (9) at (24.4, 0.025) {};
		\node [style=none] (10) at (24, 0.025) {};
		\node [style=none] (11) at (24.35, 0.1) {};
		\node [style=none] (12) at (24.05, 0.1) {};
		\node [style=none] (13) at (24.3, 0.175) {};
		\node [style=none] (14) at (24.1, 0.175) {};
	\end{pgfonlayer}
	\begin{pgfonlayer}{edgelayer}
		\draw (8.center) to (7.center);
		\draw (10.center) to (9.center);
		\draw (12.center) to (11.center);
		\draw (14.center) to (13.center);
	\end{pgfonlayer}
\end{tikzpicture}
}}


\tikzset{meter/.append style={draw, inner sep=10, rectangle, font=\vphantom{A}, minimum width=30, line width=.8,
 path picture={\draw[black] ([shift={(.1,.3)}]path picture bounding box.south west) to[bend left=50] ([shift={(-.1,.3)}]path picture bounding box.south east);\draw[black,-latex] ([shift={(0,.1)}]path picture bounding box.south) -- ([shift={(.3,-.1)}]path picture bounding box.north);}}}



\tikzset{
  unit/.style={shape=rectangle, rounded corners,inner sep=0.3em,draw=black,fill=white, font={$I$}}
}


\usepackage{bbm}

\usepackage{Cobordism}
\pgfsetlayers{bottom,background,main,internal,foreground,label,cobordism,top,selectionbox,edgelayer,nodelayer}
%\input{styles.tikzdefs}
%\input{styles.tikzstyles}


\usepackage[utf8]{inputenc} % set input encoding (not needed with XeLaTeX)


\title{Relational semantics for quantum protocols}
\author{Cole Comfort}
\college{New College}  %your college
\degree{Doctor of Philosophy in Computer Science} 
\degreedate{????? 2023}    

%\renewcommand{\submittedtext}{change the default text here if needed}
%\date{} % Activate to display a given date or no date (if empty),
         % otherwise the current date is printed 

\newcommand{\Set}{\Sets}
\newcommand{\Mnd}{{\sf Mnd}}
\newcommand{\Cat}{{\sf Cat}}

\begin{document}
\maketitle


\newcommand{\D}{\mathbb{D}}




Although we factorize the inputs and outputs of string diagrams using the (co)pants and (co)units, this is a monoidal 2-category, not an ordinary monoidal category.  We will need the following construction


\begin{definition}
\newcommand{\D}{\mathbb{D}}
A {\bf displayed category} is an ordinary category $\D$ equipped with a lax normal functor $F:\D\to \Prof$.
That is to say, $F$ has the data of:

\begin{itemize}
\item A function $F:\D_0\to \Prof_0$ taking objects of $\D$ to categories.
\item For every pair of objects $X,Y \in \D_0$, a function $F_{X,Y}:\D(X,Y)\to \Prof(F(X),F(Y))$  such that $1_{F(X)}=F_{X,X}(1_X)$.
\item For every triple of objects $X,Y,Z \in \D_0$, a $2$-cell, with components at $f:X\to Y,gY\to Z$
$$F_{X,Y,Z}(f,g):F_{X,Y}(f);F_{Y,Z}(g) \Rightarrow F_{X,Z}(f;g)$$
\end{itemize}

Such that for any diagram $W\xrightarrow{f} X \xrightarrow{g} Y \xrightarrow{h} Z$ in $\X$ the following diagram commutes:


$$
\xymatrix{
(F_{W,X}(f);F_{X,Y}(g));F_{Y,Z} \ar[d]_{F_{W,X,Y}(f;g);1_{F_{Y,Z}(h)}} \ar[rr]^{\alpha_{F_{W,X}(f),F_{X,Y}(g),F_{Y,Z}(h)}}
  & & F_{W,X}(f);(F_{X,Y}(g);F_{Y,Z}(h)) \ar[d]^{1_{F_{W,X}(f)};F_{X,Y,Z}(g,h)}\\
F_{W,Y}(f;g);F_{Y,Z}(h) \ar[dr]_{F_{W,Y,Z}(f;g,h) \ \ }
  && F_{W,X}(f);F_{X,Z}(g;h) \ar[dl]^{\ \ F_{W,X,Z}(f,g;h)}\\
  & F(W,Z)(f;g;h)
}
$$
\end{definition}


\begin{theorem}{Benabou-Grothendieck construction}

Then there is an ordinary category $\Pi F$ defined as follows:


\begin{description}
\item[Objects:] Pairs $(X\in \D_0, X^\sharp \in (F(X))_0)$
\item[Maps:] The maps are pairs, $(f, f^\sharp):(X,X^\sharp )\to (Y,Y^\sharp)$ where $f \in \D(X,Y)$ and $f^\sharp \in F_{X,Y}(f)(Y^\sharp,X^\sharp)$
\item[Identity:] $1_{(X,X^\sharp)} := (1_X, 1_{X^\sharp})$
\item[Composition:] Given a composable pair:
$$(X,X^\sharp)\xrightarrow{(f, f^\sharp)} (Y,Y^\sharp)\xrightarrow{(g, g^\sharp)} (Z,Z^\sharp)$$
The composite is defined as follows:
$$(f, f^\sharp);(g, g^\sharp):= (F_{X,Y,Z}(f,g))(f^\sharp, g^\sharp)$$
\end{description}


Moreover, the first projection map $\pi_0:\Pi F\to \D$  is a (strict) functor.


We can actually go the other direction.  Given some fixed $\D$, this extends to an equivalence of categories between the slice category $\Cat/\D$ and the lax normal functor category $[\D,\Prof]$.  We won't restate this equivalence of categories, because it is not directly useful for us, and it takes considerable effort to expose.
\end{theorem}


\begin{definition}
\newcommand{\esfa}{{\sf ESFA}}
Given any monoidal category $\X$, there is an indexed category

$F_\X:\escfa\to \Prof$ such that  
\begin{description}
\item $(F_\X)(n) = \X^n$
\item $(F_\X){n,m}$ takes spiders in $\escfa$ to (stratified) spiders in $\Prof$ composed of the monoidal structure of $\X$.
\item The natural transformation $(F_\X)_{n,m,k}$ performs (stratified) spider fusion.
\end{description}


\end{definition}



\begin{lemma}
The indexed category $\Pi F_{\X}$ has a concrete presentation:

\begin{description}
\item[Objects:] Finite lists of objects in $\X$.
\item[Maps:] Given two finite lists $X=[X_0,\ldots, X_{n-1}]$ and $Y=[Y_0,\ldots, Y_{m-1}]$ of objects in $\X$, a map from $X\to Y$ is a pair consisting of a map $f:n\to m$ in $\esfa$ equipped with an element $f^\sharp \in  (F_{\X})_{n,m}( F_{\X})$.


$f^\sharp$ can be described concretely by induction on the connected components on $n$. If $f$ is a single connected component, then $f^\sharp$ is a map with domains and codomains left-factorized as follows
$$\otimes_{i=0}^{n-1} X_i \to \otimes_{i=0}^{m-1} Y_i$$

Moreover, if $f$ is multiple connected components, then $f^\sharp$ is a finite list of maps whose domains and codomains factorized in this way.


\item[Identity:] Given a list of objects $X=[X_0,\ldots, X_{n-1}]$  in $\X$, $1_X=[1_{X_0},\ldots, 1_{X_{n-1}}]$.


\item[Composition:] Take a composable pair:
$$X=[X_0,\ldots, X_{n-1}] \xrightarrow{(f,f^\sharp)} Y=[Y_0,\ldots, Y_{m-1}] \xrightarrow{(g,g^\sharp)} Z=[Z_0,\ldots, Z_{k-1}]$$

Then the composite fuses together all connected components into stratified spiders:  effectively, so that if two maps are connected it just produces the composite in $\X$ and keeps track of the factorizations of inputs and outputs.

\end{description}

This is moreover a strict monoidal category.  The tensor product is defined on objects by concatenation.
Given two maps 
$$(f,f^\sharp):W=[W_0,\ldots, W_{n-1}]\to X=[X_{0},\ldots, X_{m-1}]$$
and
$$(g;g^\sharp):Y=[Y_0,\ldots, Y_{k-1}]\toZ= [Z_{0},\ldots, Z_{\ell-1}]$$

the tensor product is defined as follows:

$$
(f,f^\sharp)\otimes(g;g^\sharp)
:=
(f\otimes 1_k, f^\sharp\times 1_Y );(1_m\otimes g, 1_X\times g^\sharp)
$$

This is strict because the composition stratifies shapes into lists


\end{lemma}


GIVE EXAMPLE


We want to consider the category where all components are connected.  to do this, first remark that for any  finite list of objects $X=[X_0,\ldots, X_{n-1}]$ in $\X$ there is an idempotent $(s_X, s_X^\sharp)=e_X:X\to X$ in   $\Pi F_{\X}$ where $s_X$ is the fully connected spider from $n\to n$ and $s_X$ is the tensor factoriztion of $\otimes_{i=0}^{n-1} 1_{X_i}$.


GIVE EXAMPLE





\begin{definition}
 $N\X:=K_{\{e_X\ | \ X \in [\X_0]}}(\Pi F_{\X})$ of the Karoubi envelope of $\Pi F_{\X}$ with objects $(X,e_X)$.  The maps are precisely the maps in $\Pi F_{\X}$ whose shapes are fully connected.
\end{definition}


\begin{lemma}
This is a strict monoidal category, where the tensor product is defined on objects by
$$
([X_0,\ldots, X_{n-1}],e_{[X_0,\ldots, X_{n-1}]})
\otimes
([Y_0,\ldots, Y_{n-1}],e_{[Y_0,\ldots, Y_{n-1}]})
:=
([X_0,\ldots, X_{n-1},Y_0,\ldots, Y_{n-1}],e_{[X_0,\ldots, X_{n-1},Y_0,\ldots, Y_{n-1}]})
$$
\end{lemma}


The tensor product just connects the shapes of both things that are tensored together.


\begin{theorem}
$N\X$ is the strict monoidal category of proof nets in $\X$.
\end{theorem}
\begin{proof}

\end{proof}


\bibliography{bibliography} 
\bibliographystyle{plain}
\end{document}
