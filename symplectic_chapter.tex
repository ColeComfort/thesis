Linear Lagrangian relations, or more generally, affine Lagrangian relations provide a rich, compositional setting for modelling the evolutions of various physical systems. For example, certain classes of electrical circuits can be interpreted in terms of Lagrangian relations over the field of real rational functions~\cite{passive,affine}. On a quite different note, the state preparation and quantum evolution of $p$-dimensional generalizations of Spekkens' toy theory~\cite{spekkens2016quasi} and (consequently) odd-prime-dimensional stabilizer quantum theory~\cite{gross} have semantics in terms of affine Lagrangian relations over $\F_p$.  Specifically, the state preparation corresponds to the affine Lagrangian relations from the tensor unit, and the evolution corresponds to affine symplectomorphisms.  In this paper we extend this correspondance to the full category of Lagrangian relations, giving these circuits a proper categorical treatment.
% hence the odd-prime-dimensional qudit stabilizer theory, are equivalent to affine Lagrangian relations over (finite) prime fields.  In this document, a {\em prime field} refers to one of the form $\F_p$ for $p$ prime.

Formally, the category of Lagrangian relations is the symmetric monoidal subcategory of linear relations where the objects are symplectic vector spaces and the morphisms are linear relations satisfying an extra condition which can be captured graphically as the following, where $V^\perp$ denotes the orthogonal complement and the grey box denotes the \textit{antipode} from the graphical theory of linear relations:
% subspaces of maximal dimension on which the symplectic form vanishes.
% In terms of string diagrams, as we will show in Section \ref{sec:sym}, we can think of the complementary observables as  occupying two different wires.  The graphical condition of a linear subspace being Lagrangian is expressed as follows, where $V^\perp$ denotes the orthogonal complement of $V$:
$$
\begin{tikzpicture}
	\begin{pgfonlayer}{nodelayer}
		\node [style=map] (0) at (2, -2) {$V$};
		\node [style=none] (1) at (1.75, -1.25) {};
		\node [style=none] (2) at (2.25, -1.25) {};
		\node [style=none] (3) at (1.75, -2.75) {};
		\node [style=none] (4) at (2.25, -2.75) {};
	\end{pgfonlayer}
	\begin{pgfonlayer}{edgelayer}
		\draw [in=120, out=-90] (1.center) to (0);
		\draw [in=-90, out=60] (0) to (2.center);
		\draw [in=-60, out=90] (4.center) to (0);
		\draw [in=90, out=-120] (0) to (3.center);
	\end{pgfonlayer}
\end{tikzpicture}
=
\begin{tikzpicture}
	\begin{pgfonlayer}{nodelayer}
		\node [style=map] (0) at (2, -2) {$V^\perp$};
		\node [style=none] (1) at (1.75, -1.25) {};
		\node [style=none] (2) at (2.25, -1.25) {};
		\node [style=none] (3) at (1.75, -2.75) {};
		\node [style=none] (4) at (2.25, -2.75) {};
		\node [style=none] (5) at (2.25, -0.5) {};
		\node [style=none] (6) at (1.75, -0.5) {};
		\node [style=none] (7) at (2.25, -3.5) {};
		\node [style=none] (8) at (1.75, -3.5) {};
		\node [style=s] (9) at (2.25, -1.25) {};
		\node [style=s] (10) at (2.25, -2.75) {};
	\end{pgfonlayer}
	\begin{pgfonlayer}{edgelayer}
		\draw [in=120, out=-90] (1.center) to (0);
		\draw [in=-90, out=60] (0) to (2.center);
		\draw [in=-60, out=90] (4.center) to (0);
		\draw [in=90, out=-120] (0) to (3.center);
		\draw [in=90, out=-90] (6.center) to (2.center);
		\draw [in=270, out=90] (1.center) to (5.center);
		\draw [in=270, out=90] (7.center) to (3.center);
		\draw [in=270, out=90] (8.center) to (4.center);
	\end{pgfonlayer}
\end{tikzpicture}
$$
We show that any linear relation $V$ determines a Lagrangian relation in terms of `doubling', i.e. taking the tensor product of a linear relation with its complement:
$$
\begin{tikzpicture}
	\begin{pgfonlayer}{nodelayer}
		\node [style=map] (24) at (3.25, -1) {$V$};
		\node [style=none] (25) at (3.25, -0.25) {};
		\node [style=none] (27) at (3.25, -1.75) {};
	\end{pgfonlayer}
	\begin{pgfonlayer}{edgelayer}
		\draw (25.center) to (24);
		\draw (27.center) to (24);
	\end{pgfonlayer}
\end{tikzpicture}
\mapsto
\begin{tikzpicture}
	\begin{pgfonlayer}{nodelayer}
		\node [style=map] (24) at (3.25, -1) {$V^\perp$};
		\node [style=none] (25) at (3.25, -0.25) {};
		\node [style=none] (27) at (3.25, -1.75) {};
		\node [style=map] (28) at (4, -1) {$V$};
		\node [style=none] (29) at (4, -0.25) {};
		\node [style=none] (30) at (4, -1.75) {};
	\end{pgfonlayer}
	\begin{pgfonlayer}{edgelayer}
		\draw (25.center) to (24);
		\draw (27.center) to (24);
		\draw (29.center) to (28);
		\draw (30.center) to (28);
	\end{pgfonlayer}
\end{tikzpicture}
$$
By analogy to the CPM construction for the category of completely positive maps, we call these \textit{pure} Lagrangian relations. In Theorem \ref{theorem:unbiased} we show that only one more class of `discard' generators $d_a$ for each $a$ in the underlying field $k$ is required to generate all Lagrangian relations.
$$
d_a := 
\begin{tikzpicture}
	\begin{pgfonlayer}{nodelayer}
		\node [style=X] (0) at (0, 0.75) {};
		\node [style=scalar] (1) at (0.5, 0) {$a$};
		\node [style=none] (2) at (0.5, -0.75) {};
		\node [style=none] (3) at (-0.5, 0) {};
		\node [style=none] (4) at (-0.5, -0.75) {};
	\end{pgfonlayer}
	\begin{pgfonlayer}{edgelayer}
		\draw [in=-30, out=90] (1) to (0);
		\draw [in=90, out=-150] (0) to (3.center);
		\draw (4.center) to (3.center);
		\draw (2.center) to (1);
	\end{pgfonlayer}
\end{tikzpicture}
$$

From this, we immediately obtain a complete graphical calculus for Lagrangian relations over any field $k$, namely we can apply the complete calculus $\ih_k$ for linear relations~\cite{ihpub} to diagrams built from pure morphisms and discard maps. 
This extends the doubled presentation of bond graphs, given in \cite[5.3]{coya}, which are not universal for Lagrangian relations, and is instead only universal for a fragment of the pure morphisms.
In Corollary \ref{cor:pure}, we also immediately get a \textit{purification theorem} for Lagrangian relations, much like the purification (a.k.a. Stinespring dilation) of quantum channels which can be proven straightforwardly in the CPM construction over Hilbert spaces.

% two more classes of generators are needed to obtain all Lagrangian relations.  Explicitly, for every element $a$ of the base field we need an $a$-labelled phase gate and the Fourier transform:
% $$
% \tikzfig{Sa-gen}
% \hspace*{2cm}
% \tikzfig{fourier}
% $$
% Notice that these maps can not be factored into a linear relation tensored by its orthogonal complement.
% In  Theorem \ref{theorem:unbiased} we show that Lagrangian relations can be presented in a more symmetric way in terms of the CPM construction applied to linear relations where the conjugation is given by the orthogonal complement, and a `discard' map is added for every element of the field.

Furthermore, in the case of prime fields, i.e. finite fields $\mathbb F_p$ for $p$, we show in Corollary \ref{cor} that this is actually an instance of the original CPM construction, for a suitably defined dagger on the category of linear relations.

% Because doubling is faithful and symmetric monoidal, in the undoubled picture, Lagrangian relations can be obtained by adding one class of extra generators to linear relations.  This is similar to how, in some cases, the CPM construction applied to a prop can be presented in terms of adding a discard map to the base category modulo some extra equations \cite{disc}.

In Section \ref{sec:aff} we show that only one more generator is needed to obtain {\em affine} Lagrangian relations.  In the case of odd prime fields, we show in Theorem \ref{theorem:spekkens} that affine Lagrangian relations are prime-dimensional qudit stabilizer circuits, modulo invertible scalars.  This give a graphical calculus extends to previous work on the qubit \cite{backensspek}, and qutrit \cite{qutrit} cases.  We also discuss the relation to electrical circuits.


\paragraph{Related work.} It was previously shown that certain classes of electrical circuits have a semantics in terms of affine Lagrangian relations over the field of the real numbers and the real rational functions ${\mathbb{R}[x,y]/\langle xy-1\rangle}$ \cite{network,passive}. Similarly in \cite[\S VI]{affine}, the authors give an interpretation of non-passive electrical circuits in terms of these `doubled' string diagrams for affine relations over the real rational functions, however the authors did not give a full characterisation for the category of Lagrangian relations in terms of diagrammatic generators.
%These two approaches are actually one and the same.
We restate the interpretations of the electrical components given in  \cite[\S VI]{affine} in terms of the graphical calculus for affine Lagrangian relations in Example \ref{ex:circuits}.

A presentation of odd-prime stabilizer theory in terms of affine symplectomorphisms applied to Lagrangian subspaces appears in~\cite{gross} and several follow-on works relating stabilizer theory to classical phase space via the discrete Wigner function. Our Theorem \ref{theorem:spekkens} is a categorical reformulation of the result of Spekkens' in which he shows that so called odd-prime-dimensional `quadrature epistricted theories' are operationally equivalent to prime-dimensional qudit stabilizer circuits \cite{spekkens2016quasi}; following earlier work in \cite{spekkens}.  This operational equivalence has also been further explored in the non-prime case \cite{catani}. Note that operational equivalence is not the same as categorical equivalence. The notion of operational equivalence used in \cite{spekkens2016quasi,catani}  refers to the equivalence of protocols in which circuits are prepared, evolve and then are measured; whereas ours is more `process-theoretic', i.e. we consider the category that contains states, effects, evolutions, and all possible compositions thereof. A complete presentation for Spekkens' qubit toy model in terms of a category of relations has also been given \cite{backensspek} following the categorical description by \cite{coecke2012spekkens}.  However, the authors do not explicitly establish that this is the category of affine Lagrangian relations over $\F_2$, but merely a subcategory of finite sets and relations. There is also a complete presentation for qutrit stabilizer theory \cite{qutrit} which, by Theorem \ref{theorem:spekkens}, is equivalent to Spekkens' qutrit toy model, up to scalars; the connection to relations, in this case, being unexplored.

%In this document we show that only two new generators need to be added to linear relations to obtain Lagrangian relations, and one more is needed to obtain affine Lagrangian relations.



%In Section \ref{sec:linear} , we review the graphical calculi and theory of linear relations.  After which in Section \ref{sec:sym} we give a brief review of linear symplectic geometry.  In Section \ref{sec:univ} we give a universal set of generators for Lagrangian relations.  We show that this can be stated in terms of the CPM construction applied to linear relations over a field with respect to the conjugation given by the orthogonal complement.  In the case when one is not working with a prime field, then the CPM construction must be slightly modified to allow for multiple traces.
%In section \ref{sec:aff} we give a universal set of generators for affine Lagrangian relations.

\section{Linear relations}

\label{sec:linear}

In order to decribe Lagrangian relations diagramatically, we must first recall the symmetric monoidal theory of linear relations.  To do so, we first recall the symmetric monoidal theory of matrices:

\begin{definition} \cite[Defn. 3.4]{ih}
Given a ring $k$, let $\cb_k$ denote the prop given by the generators\footnote{We use the ZX-style colouring which is dual to that used in \cite{ih}.}:
$$
\begin{tikzpicture}
	\begin{pgfonlayer}{nodelayer}
		\node [style=Z] (0) at (4, 2) {};
		\node [style=none] (1) at (3.75, 2.5) {};
		\node [style=none] (2) at (4.25, 2.5) {};
		\node [style=none] (3) at (4, 1.5) {};
	\end{pgfonlayer}
	\begin{pgfonlayer}{edgelayer}
		\draw [in=135, out=-90] (1.center) to (0);
		\draw (0) to (3.center);
		\draw [in=-90, out=45, looseness=1.25] (0) to (2.center);
	\end{pgfonlayer}
\end{tikzpicture}\hspace*{,5cm}
\begin{tikzpicture}
	\begin{pgfonlayer}{nodelayer}
		\node [style=Z] (0) at (4, 2) {};
		\node [style=none] (3) at (4, 1.5) {};
	\end{pgfonlayer}
	\begin{pgfonlayer}{edgelayer}
		\draw (0) to (3.center);
	\end{pgfonlayer}
\end{tikzpicture}\hspace*{,5cm}
\begin{tikzpicture}
	\begin{pgfonlayer}{nodelayer}
		\node [style=X] (0) at (4, 2) {};
		\node [style=none] (1) at (3.75, 1.5) {};
		\node [style=none] (2) at (4.25, 1.5) {};
		\node [style=none] (3) at (4, 2.5) {};
	\end{pgfonlayer}
	\begin{pgfonlayer}{edgelayer}
		\draw [in=-135, out=90] (1.center) to (0);
		\draw (0) to (3.center);
		\draw [in=90, out=-45, looseness=1.25] (0) to (2.center);
	\end{pgfonlayer}
\end{tikzpicture}\hspace*{,5cm}
\begin{tikzpicture}
	\begin{pgfonlayer}{nodelayer}
		\node [style=X] (0) at (4, 2) {};
		\node [style=none] (3) at (4, 2.5) {};
	\end{pgfonlayer}
	\begin{pgfonlayer}{edgelayer}
		\draw (0) to (3.center);
	\end{pgfonlayer}
\end{tikzpicture}\hspace*{,5cm}
\text{for all $a \in k$ }
\hspace*{,3cm}
\begin{tikzpicture}
	\begin{pgfonlayer}{nodelayer}
		\node [style=scalar] (0) at (4, 2) {$a$};
		\node [style=none] (3) at (4, 2.5) {};
		\node [style=none] (4) at (4, 1.5) {};
	\end{pgfonlayer}
	\begin{pgfonlayer}{edgelayer}
		\draw (0) to (3.center);
		\draw (4.center) to (0);
	\end{pgfonlayer}
\end{tikzpicture}
$$
modulo the equations of a bicommutative bialgebra:
$$
\begin{tikzpicture}
	\begin{pgfonlayer}{nodelayer}
		\node [style=Z] (0) at (5, 2) {};
		\node [style=none] (1) at (4.75, 2.5) {};
		\node [style=none] (2) at (5.25, 2.5) {};
		\node [style=none] (3) at (5, 1.5) {};
	\end{pgfonlayer}
	\begin{pgfonlayer}{edgelayer}
		\draw (3.center) to (0);
		\draw [in=-90, out=135] (0) to (1.center);
		\draw [in=-90, out=45] (0) to (2.center);
	\end{pgfonlayer}
\end{tikzpicture}
=
\begin{tikzpicture}
	\begin{pgfonlayer}{nodelayer}
		\node [style=Z] (0) at (5, 2) {};
		\node [style=none] (1) at (4.75, 2.5) {};
		\node [style=none] (2) at (5.25, 2.5) {};
		\node [style=none] (3) at (5, 1.5) {};
		\node [style=none] (4) at (5.25, 3) {};
		\node [style=none] (5) at (4.75, 3) {};
	\end{pgfonlayer}
	\begin{pgfonlayer}{edgelayer}
		\draw (3.center) to (0);
		\draw [in=-90, out=135] (0) to (1.center);
		\draw [in=-90, out=45] (0) to (2.center);
		\draw [in=270, out=90] (1.center) to (4.center);
		\draw [in=270, out=90] (2.center) to (5.center);
	\end{pgfonlayer}
\end{tikzpicture}
\hspace*{.5cm}
\begin{tikzpicture}
	\begin{pgfonlayer}{nodelayer}
		\node [style=Z] (0) at (5, 2) {};
		\node [style=none] (1) at (4.75, 2.5) {};
		\node [style=none] (2) at (5.25, 2.5) {};
		\node [style=none] (3) at (5, 1.5) {};
		\node [style=none] (4) at (4.75, 3) {};
		\node [style=Z] (5) at (5.25, 2.5) {};
	\end{pgfonlayer}
	\begin{pgfonlayer}{edgelayer}
		\draw (3.center) to (0);
		\draw [in=-90, out=135] (0) to (1.center);
		\draw [in=-90, out=45] (0) to (2.center);
		\draw [in=270, out=90] (1.center) to (4.center);
	\end{pgfonlayer}
\end{tikzpicture}
=
\begin{tikzpicture}
	\begin{pgfonlayer}{nodelayer}
		\node [style=none] (1) at (4.75, 1.5) {};
		\node [style=none] (4) at (4.75, 3) {};
	\end{pgfonlayer}
	\begin{pgfonlayer}{edgelayer}
		\draw (1.center) to (4.center);
	\end{pgfonlayer}
\end{tikzpicture}
\hspace*{.5cm}
\begin{tikzpicture}
	\begin{pgfonlayer}{nodelayer}
		\node [style=Z] (0) at (5, 2) {};
		\node [style=none] (1) at (4.75, 2.5) {};
		\node [style=none] (2) at (5.25, 2.5) {};
		\node [style=none] (3) at (5, 1.5) {};
		\node [style=Z] (4) at (4.75, 2.5) {};
		\node [style=none] (5) at (4.5, 3) {};
		\node [style=none] (6) at (5, 3) {};
		\node [style=none] (7) at (5.25, 3) {};
	\end{pgfonlayer}
	\begin{pgfonlayer}{edgelayer}
		\draw (3.center) to (0);
		\draw [in=-90, out=135] (0) to (1.center);
		\draw [in=-90, out=45] (0) to (2.center);
		\draw [in=-90, out=135] (4) to (5.center);
		\draw [in=-90, out=45] (4) to (6.center);
		\draw (2.center) to (7.center);
	\end{pgfonlayer}
\end{tikzpicture}
=
\begin{tikzpicture}[xscale=-1]
	\begin{pgfonlayer}{nodelayer}
		\node [style=Z] (0) at (5, 2) {};
		\node [style=none] (1) at (4.75, 2.5) {};
		\node [style=none] (2) at (5.25, 2.5) {};
		\node [style=none] (3) at (5, 1.5) {};
		\node [style=Z] (4) at (4.75, 2.5) {};
		\node [style=none] (5) at (4.5, 3) {};
		\node [style=none] (6) at (5, 3) {};
		\node [style=none] (7) at (5.25, 3) {};
	\end{pgfonlayer}
	\begin{pgfonlayer}{edgelayer}
		\draw (3.center) to (0);
		\draw [in=-90, out=135] (0) to (1.center);
		\draw [in=-90, out=45] (0) to (2.center);
		\draw [in=-90, out=135] (4) to (5.center);
		\draw [in=-90, out=45] (4) to (6.center);
		\draw (2.center) to (7.center);
	\end{pgfonlayer}
\end{tikzpicture}
\hspace*{.5cm}
\begin{tikzpicture}[yscale=-1]
	\begin{pgfonlayer}{nodelayer}
		\node [style=X] (0) at (5, 2) {};
		\node [style=none] (1) at (4.75, 2.5) {};
		\node [style=none] (2) at (5.25, 2.5) {};
		\node [style=none] (3) at (5, 1.5) {};
	\end{pgfonlayer}
	\begin{pgfonlayer}{edgelayer}
		\draw (3.center) to (0);
		\draw [in=-90, out=135] (0) to (1.center);
		\draw [in=-90, out=45] (0) to (2.center);
	\end{pgfonlayer}
\end{tikzpicture}
=
\begin{tikzpicture}[yscale=-1]
	\begin{pgfonlayer}{nodelayer}
		\node [style=X] (0) at (5, 2) {};
		\node [style=none] (1) at (4.75, 2.5) {};
		\node [style=none] (2) at (5.25, 2.5) {};
		\node [style=none] (3) at (5, 1.5) {};
		\node [style=none] (4) at (5.25, 3) {};
		\node [style=none] (5) at (4.75, 3) {};
	\end{pgfonlayer}
	\begin{pgfonlayer}{edgelayer}
		\draw (3.center) to (0);
		\draw [in=-90, out=135] (0) to (1.center);
		\draw [in=-90, out=45] (0) to (2.center);
		\draw [in=270, out=90] (1.center) to (4.center);
		\draw [in=270, out=90] (2.center) to (5.center);
	\end{pgfonlayer}
\end{tikzpicture}
\hspace*{.5cm}
\begin{tikzpicture}[yscale=-1]
	\begin{pgfonlayer}{nodelayer}
		\node [style=X] (0) at (5, 2) {};
		\node [style=none] (1) at (4.75, 2.5) {};
		\node [style=none] (2) at (5.25, 2.5) {};
		\node [style=none] (3) at (5, 1.5) {};
		\node [style=none] (4) at (4.75, 3) {};
		\node [style=X] (5) at (5.25, 2.5) {};
	\end{pgfonlayer}
	\begin{pgfonlayer}{edgelayer}
		\draw (3.center) to (0);
		\draw [in=-90, out=135] (0) to (1.center);
		\draw [in=-90, out=45] (0) to (2.center);
		\draw [in=270, out=90] (1.center) to (4.center);
	\end{pgfonlayer}
\end{tikzpicture}
=
\begin{tikzpicture}
	\begin{pgfonlayer}{nodelayer}
		\node [style=none] (1) at (4.75, 1.5) {};
		\node [style=none] (4) at (4.75, 3) {};
	\end{pgfonlayer}
	\begin{pgfonlayer}{edgelayer}
		\draw (1.center) to (4.center);
	\end{pgfonlayer}
\end{tikzpicture}
\hspace*{.5cm}
\begin{tikzpicture}[yscale=-1]
	\begin{pgfonlayer}{nodelayer}
		\node [style=X] (0) at (5, 2) {};
		\node [style=none] (1) at (4.75, 2.5) {};
		\node [style=none] (2) at (5.25, 2.5) {};
		\node [style=none] (3) at (5, 1.5) {};
		\node [style=X] (4) at (4.75, 2.5) {};
		\node [style=none] (5) at (4.5, 3) {};
		\node [style=none] (6) at (5, 3) {};
		\node [style=none] (7) at (5.25, 3) {};
	\end{pgfonlayer}
	\begin{pgfonlayer}{edgelayer}
		\draw (3.center) to (0);
		\draw [in=-90, out=135] (0) to (1.center);
		\draw [in=-90, out=45] (0) to (2.center);
		\draw [in=-90, out=135] (4) to (5.center);
		\draw [in=-90, out=45] (4) to (6.center);
		\draw (2.center) to (7.center);
	\end{pgfonlayer}
\end{tikzpicture}
=
\begin{tikzpicture}[scale=-1]
	\begin{pgfonlayer}{nodelayer}
		\node [style=X] (0) at (5, 2) {};
		\node [style=none] (1) at (4.75, 2.5) {};
		\node [style=none] (2) at (5.25, 2.5) {};
		\node [style=none] (3) at (5, 1.5) {};
		\node [style=X] (4) at (4.75, 2.5) {};
		\node [style=none] (5) at (4.5, 3) {};
		\node [style=none] (6) at (5, 3) {};
		\node [style=none] (7) at (5.25, 3) {};
	\end{pgfonlayer}
	\begin{pgfonlayer}{edgelayer}
		\draw (3.center) to (0);
		\draw [in=-90, out=135] (0) to (1.center);
		\draw [in=-90, out=45] (0) to (2.center);
		\draw [in=-90, out=135] (4) to (5.center);
		\draw [in=-90, out=45] (4) to (6.center);
		\draw (2.center) to (7.center);
	\end{pgfonlayer}
\end{tikzpicture}
$$
$$
\begin{tikzpicture}
	\begin{pgfonlayer}{nodelayer}
		\node [style=Z] (0) at (5, 2) {};
		\node [style=X] (1) at (5, 1.5) {};
		\node [style=none] (2) at (4.75, 2.5) {};
		\node [style=none] (3) at (5.25, 2.5) {};
	\end{pgfonlayer}
	\begin{pgfonlayer}{edgelayer}
		\draw (1) to (0);
		\draw [in=-90, out=135] (0) to (2.center);
		\draw [in=-90, out=45] (0) to (3.center);
	\end{pgfonlayer}
\end{tikzpicture}
=
\begin{tikzpicture}
	\begin{pgfonlayer}{nodelayer}
		\node [style=none] (2) at (4.75, 2.5) {};
		\node [style=none] (3) at (5.25, 2.5) {};
		\node [style=X] (4) at (4.75, 2) {};
		\node [style=X] (5) at (5.25, 2) {};
	\end{pgfonlayer}
	\begin{pgfonlayer}{edgelayer}
		\draw (4) to (2.center);
		\draw (5) to (3.center);
	\end{pgfonlayer}
\end{tikzpicture}
\hspace*{.5cm}
\begin{tikzpicture}[yscale=-1]
	\begin{pgfonlayer}{nodelayer}
		\node [style=X] (0) at (5, 2) {};
		\node [style=Z] (1) at (5, 1.5) {};
		\node [style=none] (2) at (4.75, 2.5) {};
		\node [style=none] (3) at (5.25, 2.5) {};
	\end{pgfonlayer}
	\begin{pgfonlayer}{edgelayer}
		\draw (1) to (0);
		\draw [in=-90, out=135] (0) to (2.center);
		\draw [in=-90, out=45] (0) to (3.center);
	\end{pgfonlayer}
\end{tikzpicture}
=
\begin{tikzpicture}[yscale=-1]
	\begin{pgfonlayer}{nodelayer}
		\node [style=none] (2) at (4.75, 2.5) {};
		\node [style=none] (3) at (5.25, 2.5) {};
		\node [style=Z] (4) at (4.75, 2) {};
		\node [style=Z] (5) at (5.25, 2) {};
	\end{pgfonlayer}
	\begin{pgfonlayer}{edgelayer}
		\draw (4) to (2.center);
		\draw (5) to (3.center);
	\end{pgfonlayer}
\end{tikzpicture}
\hspace*{.5cm}
\begin{tikzpicture}
	\begin{pgfonlayer}{nodelayer}
		\node [style=Z] (0) at (5, 2) {};
		\node [style=X] (1) at (5, 1.5) {};
		\node [style=none] (2) at (4.75, 2.5) {};
		\node [style=none] (3) at (5.25, 2.5) {};
		\node [style=none] (4) at (4.75, 1) {};
		\node [style=none] (5) at (5.25, 1) {};
	\end{pgfonlayer}
	\begin{pgfonlayer}{edgelayer}
		\draw (1) to (0);
		\draw [in=-90, out=135] (0) to (2.center);
		\draw [in=-90, out=45] (0) to (3.center);
		\draw [in=-45, out=90] (5.center) to (1);
		\draw [in=90, out=-135] (1) to (4.center);
	\end{pgfonlayer}
\end{tikzpicture}
=
\begin{tikzpicture}
	\begin{pgfonlayer}{nodelayer}
		\node [style=Z] (0) at (5, 2) {};
		\node [style=Z] (1) at (5.5, 2) {};
		\node [style=X] (2) at (5, 2.5) {};
		\node [style=X] (3) at (5.5, 2.5) {};
		\node [style=none] (4) at (5.5, 3) {};
		\node [style=none] (5) at (5, 3) {};
		\node [style=none] (6) at (5, 1.5) {};
		\node [style=none] (7) at (5.5, 1.5) {};
	\end{pgfonlayer}
	\begin{pgfonlayer}{edgelayer}
		\draw (5.center) to (2);
		\draw [bend right=45, looseness=1.25] (2) to (0);
		\draw (0) to (3);
		\draw (1) to (2);
		\draw [bend left=45, looseness=1.25] (3) to (1);
		\draw (1) to (7.center);
		\draw (6.center) to (0);
		\draw (3) to (4.center);
	\end{pgfonlayer}
\end{tikzpicture}
\hspace*{.5cm}
\begin{tikzpicture}
	\begin{pgfonlayer}{nodelayer}
		\node [style=Z] (0) at (5, 2) {};
		\node [style=X] (1) at (5, 1.5) {};
	\end{pgfonlayer}
	\begin{pgfonlayer}{edgelayer}
		\draw (1) to (0);
	\end{pgfonlayer}
\end{tikzpicture}
=
$$
and the additional equations:
$$
\begin{tikzpicture}
	\begin{pgfonlayer}{nodelayer}
		\node [style=none] (1) at (5, 1.25) {};
		\node [style=scalar] (2) at (5, 1.75) {$a$};
		\node [style=Z] (3) at (5, 2.25) {};
		\node [style=none] (4) at (5.25, 2.75) {};
		\node [style=none] (5) at (4.75, 2.75) {};
	\end{pgfonlayer}
	\begin{pgfonlayer}{edgelayer}
		\draw (1.center) to (2);
		\draw (2) to (3);
		\draw [in=-90, out=135] (3) to (5.center);
		\draw [in=-90, out=45] (3) to (4.center);
	\end{pgfonlayer}
\end{tikzpicture}
=
\begin{tikzpicture}
	\begin{pgfonlayer}{nodelayer}
		\node [style=none] (0) at (6.5, 1.75) {};
		\node [style=Z] (1) at (6.5, 2.25) {};
		\node [style=scalar] (4) at (6.25, 2.75) {$a$};
		\node [style=scalar] (5) at (6.75, 2.75) {$a$};
		\node [style=none] (6) at (6.75, 3.25) {};
		\node [style=none] (7) at (6.25, 3.25) {};
	\end{pgfonlayer}
	\begin{pgfonlayer}{edgelayer}
		\draw (5) to (6.center);
		\draw (7.center) to (4);
		\draw (1) to (0.center);
		\draw [in=-90, out=45] (1) to (5);
		\draw [in=-90, out=135] (1) to (4);
	\end{pgfonlayer}
\end{tikzpicture}
\hspace*{.5cm}
\begin{tikzpicture}
	\begin{pgfonlayer}{nodelayer}
		\node [style=none] (0) at (5, 1.25) {};
		\node [style=scalar] (1) at (5, 1.75) {$a$};
		\node [style=Z] (2) at (5, 2.25) {};
	\end{pgfonlayer}
	\begin{pgfonlayer}{edgelayer}
		\draw (0.center) to (1);
		\draw (1) to (2);
	\end{pgfonlayer}
\end{tikzpicture}
=
\begin{tikzpicture}
	\begin{pgfonlayer}{nodelayer}
		\node [style=none] (0) at (5, 1.25) {};
		\node [style=Z] (2) at (5, 2.25) {};
	\end{pgfonlayer}
	\begin{pgfonlayer}{edgelayer}
		\draw (0.center) to (2);
	\end{pgfonlayer}
\end{tikzpicture}
\hspace*{.5cm}
\begin{tikzpicture}
	\begin{pgfonlayer}{nodelayer}
		\node [style=none] (0) at (5, 2.75) {};
		\node [style=scalar] (1) at (5, 2.25) {$a$};
		\node [style=X] (2) at (5, 1.75) {};
		\node [style=none] (3) at (5.25, 1.25) {};
		\node [style=none] (4) at (4.75, 1.25) {};
	\end{pgfonlayer}
	\begin{pgfonlayer}{edgelayer}
		\draw (0.center) to (1);
		\draw (1) to (2);
		\draw [in=90, out=-135] (2) to (4.center);
		\draw [in=90, out=-45] (2) to (3.center);
	\end{pgfonlayer}
\end{tikzpicture}
=
\begin{tikzpicture}
	\begin{pgfonlayer}{nodelayer}
		\node [style=none] (0) at (6.5, 3.25) {};
		\node [style=X] (1) at (6.5, 2.75) {};
		\node [style=scalar] (4) at (6.25, 2.25) {$a$};
		\node [style=scalar] (5) at (6.75, 2.25) {$a$};
		\node [style=none] (6) at (6.75, 1.75) {};
		\node [style=none] (7) at (6.25, 1.75) {};
	\end{pgfonlayer}
	\begin{pgfonlayer}{edgelayer}
		\draw (5) to (6.center);
		\draw (7.center) to (4);
		\draw (1) to (0.center);
		\draw [in=-45, out=90] (5) to (1);
		\draw [in=90, out=-135] (1) to (4);
	\end{pgfonlayer}
\end{tikzpicture}
\hspace*{.5cm}
\begin{tikzpicture}
	\begin{pgfonlayer}{nodelayer}
		\node [style=none] (0) at (5, 2.25) {};
		\node [style=scalar] (1) at (5, 1.75) {$a$};
		\node [style=X] (2) at (5, 1.25) {};
	\end{pgfonlayer}
	\begin{pgfonlayer}{edgelayer}
		\draw (0.center) to (1);
		\draw (1) to (2);
	\end{pgfonlayer}
\end{tikzpicture}
=
\begin{tikzpicture}
	\begin{pgfonlayer}{nodelayer}
		\node [style=none] (0) at (5, 2.25) {};
		\node [style=X] (1) at (5, 1.25) {};
	\end{pgfonlayer}
	\begin{pgfonlayer}{edgelayer}
		\draw (0.center) to (1);
	\end{pgfonlayer}
\end{tikzpicture}
\hspace*{.5cm}
\begin{tikzpicture}
	\begin{pgfonlayer}{nodelayer}
		\node [style=scalar] (28) at (128.25, 2) {$b$};
		\node [style=none] (29) at (128.25, 2.5) {};
		\node [style=scalar] (30) at (128.25, 1.25) {$a$};
		\node [style=none] (31) at (128.25, 0.75) {};
	\end{pgfonlayer}
	\begin{pgfonlayer}{edgelayer}
		\draw (28) to (29.center);
		\draw (31.center) to (30);
		\draw (30) to (28);
	\end{pgfonlayer}
\end{tikzpicture}
=
\begin{tikzpicture}
	\begin{pgfonlayer}{nodelayer}
		\node [style=scalar] (0) at (4, 2) {$ab$};
		\node [style=none] (3) at (4, 2.5) {};
		\node [style=none] (4) at (4, 1.5) {};
	\end{pgfonlayer}
	\begin{pgfonlayer}{edgelayer}
		\draw (0) to (3.center);
		\draw (4.center) to (0);
	\end{pgfonlayer}
\end{tikzpicture}
\hspace*{.5cm}
\begin{tikzpicture}
	\begin{pgfonlayer}{nodelayer}
		\node [style=none] (0) at (5, 2.25) {};
		\node [style=none] (1) at (5, 1.25) {};
		\node [style=scalar] (2) at (5, 1.75) {$1$};
	\end{pgfonlayer}
	\begin{pgfonlayer}{edgelayer}
		\draw (1.center) to (2);
		\draw (2) to (0.center);
	\end{pgfonlayer}
\end{tikzpicture}
=
\begin{tikzpicture}
	\begin{pgfonlayer}{nodelayer}
		\node [style=none] (2) at (5.5, 2.5) {};
		\node [style=none] (3) at (5.5, 1) {};
	\end{pgfonlayer}
	\begin{pgfonlayer}{edgelayer}
		\draw (3.center) to (2.center);
	\end{pgfonlayer}
\end{tikzpicture}
\hspace*{.5cm}
\begin{tikzpicture}
	\begin{pgfonlayer}{nodelayer}
		\node [style=none] (1) at (5.25, 2.75) {};
		\node [style=none] (3) at (5.25, 0.75) {};
		\node [style=Z] (4) at (5.25, 1.25) {};
		\node [style=X] (5) at (5.25, 2.25) {};
		\node [style=scalar] (6) at (5, 1.75) {$a$};
		\node [style=scalar] (7) at (5.5, 1.75) {$b$};
	\end{pgfonlayer}
	\begin{pgfonlayer}{edgelayer}
		\draw (3.center) to (4);
		\draw [in=-90, out=135] (4) to (6);
		\draw [in=-150, out=90] (6) to (5);
		\draw (5) to (1.center);
		\draw [in=90, out=-30] (5) to (7);
		\draw [in=45, out=-90] (7) to (4);
	\end{pgfonlayer}
\end{tikzpicture}
=
\begin{tikzpicture}
	\begin{pgfonlayer}{nodelayer}
		\node [style=none] (1) at (5, 2.25) {};
		\node [style=none] (3) at (5, 1.25) {};
		\node [style=scalar] (6) at (5, 1.75) {$a+b$};
	\end{pgfonlayer}
	\begin{pgfonlayer}{edgelayer}
		\draw (3.center) to (6);
		\draw (6) to (1.center);
	\end{pgfonlayer}
\end{tikzpicture}
\hspace*{.5cm}
\begin{tikzpicture}
	\begin{pgfonlayer}{nodelayer}
		\node [style=none] (1) at (5, 2.25) {};
		\node [style=none] (3) at (5, 1.25) {};
		\node [style=scalar] (6) at (5, 1.75) {$0$};
	\end{pgfonlayer}
	\begin{pgfonlayer}{edgelayer}
		\draw (3.center) to (6);
		\draw (6) to (1.center);
	\end{pgfonlayer}
\end{tikzpicture}
=
\begin{tikzpicture}
	\begin{pgfonlayer}{nodelayer}
		\node [style=X] (7) at (5.5, 2) {};
		\node [style=Z] (8) at (5.5, 1.5) {};
		\node [style=none] (9) at (5.5, 2.5) {};
		\node [style=none] (10) at (5.5, 1) {};
	\end{pgfonlayer}
	\begin{pgfonlayer}{edgelayer}
		\draw (10.center) to (8);
		\draw (7) to (9.center);
	\end{pgfonlayer}
\end{tikzpicture}
$$



\end{definition}


\begin{proposition} \cite[Prop. 3.9]{ih}
Given a ring $k$, $\cb_k$ is a presentation for the prop $\Mat_k$, of matrices over $k$ under the direct sum.
\end{proposition}

One should interpret the grey monoid as addition and the white comonoid as copying.



\begin{definition}\cite[Defn. 3.42]{ih}
Given a field $k$, the prop of {\bf linear relations}, $\LinRel_k$, has morphisms $n\to m$  as linear subspaces of $k^n \oplus k^m$, under relational composition and the direct sum as the tensor product.
\end{definition}

It is only necessary for $k$ to be a principle ideal domain for composition to be well defined, but a field will do for the purposes of this paper.



\begin{definition} \cite[Defn. 3.44]{ih}
Given a field $k$, let $\ih_k$ denote the prop given by the quotient of the coproduct of props $\cb_k^\op+\cb_k$ by the following equations, for all invertible $a \in k$ (where the generators of $\cb_k^\op$ are drawn by reflecting those of $\cb_k$ along the x-axis):
$$
\begin{tabular}{c}
\begin{tikzpicture}
	\begin{pgfonlayer}{nodelayer}
		\node [style=none] (1) at (5, 1.25) {};
		\node [style=scalarop] (2) at (5, 1.75) {$a$};
		\node [style=none] (3) at (5, 2.75) {};
		\node [style=scalar] (5) at (5, 2.25) {$a$};
	\end{pgfonlayer}
	\begin{pgfonlayer}{edgelayer}
		\draw (1.center) to (2);
		\draw (5) to (3.center);
		\draw (2) to (5);
	\end{pgfonlayer}
\end{tikzpicture}
=
\begin{tikzpicture}
	\begin{pgfonlayer}{nodelayer}
		\node [style=none] (6) at (6, 1.25) {};
		\node [style=none] (8) at (6, 2.75) {};
		\node [style=scalarop] (10) at (6, 2.25) {$a$};
		\node [style=scalar] (11) at (6, 1.75) {$a$};
	\end{pgfonlayer}
	\begin{pgfonlayer}{edgelayer}
		\draw (6.center) to (11);
		\draw (11) to (10);
		\draw (10) to (8.center);
	\end{pgfonlayer}
\end{tikzpicture}
=
\begin{tikzpicture}
	\begin{pgfonlayer}{nodelayer}
		\node [style=none] (12) at (7, 2.75) {};
		\node [style=none] (13) at (7, 1.25) {};
	\end{pgfonlayer}
	\begin{pgfonlayer}{edgelayer}
		\draw (13.center) to (12.center);
	\end{pgfonlayer}
\end{tikzpicture}
\hspace*{.1cm}
\begin{tikzpicture}
	\begin{pgfonlayer}{nodelayer}
		\node [style=Z] (0) at (5, 2) {};
		\node [style=Z] (1) at (5.5, 2.5) {};
		\node [style=none] (2) at (5, 1.5) {};
		\node [style=none] (3) at (5.75, 2) {};
		\node [style=none] (4) at (5.75, 1.5) {};
		\node [style=none] (5) at (4.75, 2.5) {};
		\node [style=none] (6) at (5.5, 3) {};
		\node [style=none] (7) at (4.75, 3) {};
	\end{pgfonlayer}
	\begin{pgfonlayer}{edgelayer}
		\draw (7.center) to (5.center);
		\draw [in=120, out=-90] (5.center) to (0);
		\draw (0) to (2.center);
		\draw (0) to (1);
		\draw (1) to (6.center);
		\draw [in=90, out=-60] (1) to (3.center);
		\draw (3.center) to (4.center);
	\end{pgfonlayer}
\end{tikzpicture}
=
\begin{tikzpicture}
	\begin{pgfonlayer}{nodelayer}
		\node [style=Z] (1) at (5.5, 2.5) {};
		\node [style=none] (3) at (5.75, 2) {};
		\node [style=none] (7) at (5.25, 2) {};
		\node [style=Z] (8) at (5.5, 3) {};
		\node [style=none] (9) at (5.75, 3.5) {};
		\node [style=none] (10) at (5.25, 3.5) {};
	\end{pgfonlayer}
	\begin{pgfonlayer}{edgelayer}
		\draw [in=90, out=-60] (1) to (3.center);
		\draw [in=-120, out=90] (7.center) to (1);
		\draw [in=-90, out=60] (8) to (9.center);
		\draw [in=120, out=-90] (10.center) to (8);
		\draw (1) to (8);
	\end{pgfonlayer}
\end{tikzpicture}
=
\begin{tikzpicture}
	\begin{pgfonlayer}{nodelayer}
		\node [style=Z] (0) at (5.5, 2) {};
		\node [style=Z] (1) at (5, 2.5) {};
		\node [style=none] (2) at (5.5, 1.5) {};
		\node [style=none] (3) at (4.75, 2) {};
		\node [style=none] (4) at (4.75, 1.5) {};
		\node [style=none] (5) at (5.75, 2.5) {};
		\node [style=none] (6) at (5, 3) {};
		\node [style=none] (7) at (5.75, 3) {};
	\end{pgfonlayer}
	\begin{pgfonlayer}{edgelayer}
		\draw (7.center) to (5.center);
		\draw [in=60, out=-90] (5.center) to (0);
		\draw (0) to (2.center);
		\draw (0) to (1);
		\draw (1) to (6.center);
		\draw [in=90, out=-120] (1) to (3.center);
		\draw (3.center) to (4.center);
	\end{pgfonlayer}
\end{tikzpicture}
\hspace*{.2cm}
\begin{tikzpicture}
	\begin{pgfonlayer}{nodelayer}
		\node [style=X] (0) at (5, 2) {};
		\node [style=X] (1) at (5.5, 2.5) {};
		\node [style=none] (2) at (5, 1.5) {};
		\node [style=none] (3) at (5.75, 2) {};
		\node [style=none] (4) at (5.75, 1.5) {};
		\node [style=none] (5) at (4.75, 2.5) {};
		\node [style=none] (6) at (5.5, 3) {};
		\node [style=none] (7) at (4.75, 3) {};
	\end{pgfonlayer}
	\begin{pgfonlayer}{edgelayer}
		\draw (7.center) to (5.center);
		\draw [in=120, out=-90] (5.center) to (0);
		\draw (0) to (2.center);
		\draw (0) to (1);
		\draw (1) to (6.center);
		\draw [in=90, out=-60] (1) to (3.center);
		\draw (3.center) to (4.center);
	\end{pgfonlayer}
\end{tikzpicture}
=
\begin{tikzpicture}
	\begin{pgfonlayer}{nodelayer}
		\node [style=X] (1) at (5.5, 2.5) {};
		\node [style=none] (3) at (5.75, 2) {};
		\node [style=none] (7) at (5.25, 2) {};
		\node [style=X] (8) at (5.5, 3) {};
		\node [style=none] (9) at (5.75, 3.5) {};
		\node [style=none] (10) at (5.25, 3.5) {};
	\end{pgfonlayer}
	\begin{pgfonlayer}{edgelayer}
		\draw [in=90, out=-60] (1) to (3.center);
		\draw [in=-120, out=90] (7.center) to (1);
		\draw [in=-90, out=60] (8) to (9.center);
		\draw [in=120, out=-90] (10.center) to (8);
		\draw (1) to (8);
	\end{pgfonlayer}
\end{tikzpicture}
=
\begin{tikzpicture}
	\begin{pgfonlayer}{nodelayer}
		\node [style=X] (0) at (5.5, 2) {};
		\node [style=X] (1) at (5, 2.5) {};
		\node [style=none] (2) at (5.5, 1.5) {};
		\node [style=none] (3) at (4.75, 2) {};
		\node [style=none] (4) at (4.75, 1.5) {};
		\node [style=none] (5) at (5.75, 2.5) {};
		\node [style=none] (6) at (5, 3) {};
		\node [style=none] (7) at (5.75, 3) {};
	\end{pgfonlayer}
	\begin{pgfonlayer}{edgelayer}
		\draw (7.center) to (5.center);
		\draw [in=60, out=-90] (5.center) to (0);
		\draw (0) to (2.center);
		\draw (0) to (1);
		\draw (1) to (6.center);
		\draw [in=90, out=-120] (1) to (3.center);
		\draw (3.center) to (4.center);
	\end{pgfonlayer}
\end{tikzpicture}
\hspace*{.1cm}
\begin{tikzpicture}
	\begin{pgfonlayer}{nodelayer}
		\node [style=Z] (3) at (5.5, 3.25) {};
		\node [style=Z] (6) at (5.5, 4) {};
		\node [style=none] (7) at (5.5, 4.5) {};
		\node [style=none] (8) at (5.5, 2.75) {};
	\end{pgfonlayer}
	\begin{pgfonlayer}{edgelayer}
		\draw (8.center) to (3);
		\draw [in=225, out=135, looseness=1.25] (3) to (6);
		\draw [in=45, out=-45, looseness=1.25] (6) to (3);
		\draw (6) to (7.center);
	\end{pgfonlayer}
\end{tikzpicture}
=
\begin{tikzpicture}
	\begin{pgfonlayer}{nodelayer}
		\node [style=X] (3) at (5.5, 3.25) {};
		\node [style=X] (6) at (5.5, 4) {};
		\node [style=none] (7) at (5.5, 4.5) {};
		\node [style=none] (8) at (5.5, 2.75) {};
	\end{pgfonlayer}
	\begin{pgfonlayer}{edgelayer}
		\draw (8.center) to (3);
		\draw [in=225, out=135, looseness=1.25] (3) to (6);
		\draw [in=45, out=-45, looseness=1.25] (6) to (3);
		\draw (6) to (7.center);
	\end{pgfonlayer}
\end{tikzpicture}
=
\begin{tikzpicture}
	\begin{pgfonlayer}{nodelayer}
		\node [style=none] (7) at (5.5, 4.5) {};
		\node [style=none] (8) at (5.5, 2.75) {};
	\end{pgfonlayer}
	\begin{pgfonlayer}{edgelayer}
		\draw (8.center) to (7.center);
	\end{pgfonlayer}
\end{tikzpicture}
\hspace*{.1cm}
\begin{tikzpicture}
	\begin{pgfonlayer}{nodelayer}
		\node [style=Z] (0) at (6, 3.25) {};
		\node [style=Z] (1) at (6, 2.75) {};
		\node [style=none] (2) at (5.75, 3.75) {};
		\node [style=none] (3) at (6.25, 4.25) {};
		\node [style=none] (4) at (5.75, 4.25) {};
		\node [style=scalar] (5) at (6.25, 3.75) {$-1$};
	\end{pgfonlayer}
	\begin{pgfonlayer}{edgelayer}
		\draw (1) to (0);
		\draw [in=-90, out=30] (0) to (5);
		\draw (5) to (3.center);
		\draw (4.center) to (2.center);
		\draw [in=150, out=-90] (2.center) to (0);
	\end{pgfonlayer}
\end{tikzpicture}
=
\begin{tikzpicture}
	\begin{pgfonlayer}{nodelayer}
		\node [style=X] (0) at (6, 3.25) {};
		\node [style=X] (1) at (6, 2.75) {};
		\node [style=none] (2) at (5.75, 4) {};
		\node [style=none] (3) at (6.25, 4) {};
	\end{pgfonlayer}
	\begin{pgfonlayer}{edgelayer}
		\draw (1) to (0);
		\draw [in=-90, out=30] (0) to (3.center);
		\draw [in=150, out=-90] (2.center) to (0);
	\end{pgfonlayer}
\end{tikzpicture}
\hspace*{.1cm}
\begin{tikzpicture}
	\begin{pgfonlayer}{nodelayer}
		\node [style=X] (4) at (7.5, 3.75) {};
		\node [style=X] (5) at (7.5, 4.25) {};
		\node [style=none] (6) at (7.25, 3.25) {};
		\node [style=none] (7) at (7.75, 2.75) {};
		\node [style=none] (8) at (7.25, 2.75) {};
		\node [style=scalarop] (9) at (7.75, 3.25) {$-1$};
	\end{pgfonlayer}
	\begin{pgfonlayer}{edgelayer}
		\draw (5) to (4);
		\draw [in=90, out=-30] (4) to (9);
		\draw (9) to (7.center);
		\draw (8.center) to (6.center);
		\draw [in=-150, out=90] (6.center) to (4);
	\end{pgfonlayer}
\end{tikzpicture}
=
\begin{tikzpicture}
	\begin{pgfonlayer}{nodelayer}
		\node [style=Z] (0) at (6, 3.5) {};
		\node [style=Z] (1) at (6, 4) {};
		\node [style=none] (2) at (5.75, 2.75) {};
		\node [style=none] (3) at (6.25, 2.75) {};
	\end{pgfonlayer}
	\begin{pgfonlayer}{edgelayer}
		\draw (1) to (0);
		\draw [in=90, out=-30] (0) to (3.center);
		\draw [in=-150, out=90] (2.center) to (0);
	\end{pgfonlayer}
\end{tikzpicture}
\end{tabular}
$$
\end{definition}


\begin{theorem}\cite[Thm. 3.49]{ih}
$\ih_k$ is a presentation for $\LinRel_k$.
\end{theorem}

There is an interesting folklore result which was remarked in \cite{distzx} \footnote{One should note that the black box is the antipode, {\em not} the Fourier transform/Hadamard gate.}:
\begin{lemma}
\label{lemma:phasefree}
For a prime number $p$, $\ih_{\F_p}$ is a presentation for the phase-free, Fourier-free $p$-dimensional qudit ZX-calculus, modulo invertible scalars.
\end{lemma}

The following theorem will be useful for graphical manipulations:
\begin{theorem}\cite{spider} (Spider Theorem)
All connected components of special commutative Frobenius algebras with the same arity are equal.
\end{theorem}


That is to say, we can unambiguously refer to these connected components by spiders.  In $\ih_k$, there are two spiders, so for example we can apply spider fusion to the following circuit:\hspace*{.2cm}
$
\begin{tikzpicture}
	\begin{pgfonlayer}{nodelayer}
		\node [style=Z] (0) at (6, 4) {};
		\node [style=Z] (1) at (6, 3.5) {};
		\node [style=X] (2) at (6.75, 3.5) {};
		\node [style=X] (3) at (6.75, 4) {};
		\node [style=none] (4) at (6.5, 4.5) {};
		\node [style=none] (5) at (7, 4.5) {};
		\node [style=X] (6) at (6.25, 3) {};
		\node [style=none] (7) at (5.75, 3) {};
		\node [style=none] (8) at (6.25, 2.5) {};
		\node [style=none] (9) at (7, 3) {};
		\node [style=none] (10) at (7, 2.5) {};
		\node [style=none] (11) at (5.75, 2.5) {};
	\end{pgfonlayer}
	\begin{pgfonlayer}{edgelayer}
		\draw [in=240, out=90, looseness=0.75] (7.center) to (1);
		\draw (1) to (6);
		\draw (6) to (8.center);
		\draw (6) to (2);
		\draw (2) to (3);
		\draw [in=-90, out=60] (3) to (5.center);
		\draw [in=-90, out=120] (3) to (4.center);
		\draw (1) to (0);
		\draw (11.center) to (7.center);
		\draw (10.center) to (9.center);
		\draw [in=300, out=90] (9.center) to (2);
	\end{pgfonlayer}
\end{tikzpicture}
=
\begin{tikzpicture}
	\begin{pgfonlayer}{nodelayer}
		\node [style=Z] (13) at (8.25, 3.75) {};
		\node [style=X] (14) at (8.75, 3.25) {};
		\node [style=X] (15) at (8.75, 3.25) {};
		\node [style=none] (16) at (8.75, 4.5) {};
		\node [style=none] (17) at (9.25, 4.5) {};
		\node [style=X] (18) at (8.75, 3.25) {};
		\node [style=none] (19) at (8, 3.25) {};
		\node [style=none] (20) at (8.5, 2.5) {};
		\node [style=none] (21) at (9.25, 2.5) {};
		\node [style=none] (23) at (8, 2.5) {};
	\end{pgfonlayer}
	\begin{pgfonlayer}{edgelayer}
		\draw [in=240, out=90, looseness=0.75] (19.center) to (13);
		\draw (13) to (18);
		\draw [in=90, out=-150] (18) to (20.center);
		\draw [in=-90, out=45, looseness=0.75] (15) to (17.center);
		\draw [in=-90, out=90, looseness=0.50] (15) to (16.center);
		\draw (23.center) to (19.center);
		\draw [in=-30, out=90] (21.center) to (14);
	\end{pgfonlayer}
\end{tikzpicture}
$

We  shall use the following results:
\begin{lemma}\cite[(D4)]{ih} \cite[p. 4]{control} \cite[(D3)]{ih}
$$
\begin{tikzpicture}
	\begin{pgfonlayer}{nodelayer}
		\node [style=Z] (0) at (6.75, 4.5) {};
		\node [style=X] (1) at (7.75, 3.5) {};
		\node [style=none] (3) at (6.75, 5) {};
		\node [style=none] (4) at (8, 5) {};
		\node [style=none] (5) at (6.5, 3) {};
		\node [style=none] (6) at (7.75, 3) {};
	\end{pgfonlayer}
	\begin{pgfonlayer}{edgelayer}
		\draw (3.center) to (0);
		\draw [in=60, out=-90] (4.center) to (1);
		\draw (1) to (6.center);
		\draw [in=90, out=-120] (0) to (5.center);
		\draw (1) to (0);
	\end{pgfonlayer}
\end{tikzpicture}
=
\begin{tikzpicture}
	\begin{pgfonlayer}{nodelayer}
		\node [style=Z] (0) at (6.75, 3.5) {};
		\node [style=X] (1) at (7.75, 4.5) {};
		\node [style=scalar] (2) at (7.25, 4) {$-1$};
		\node [style=none] (3) at (6.75, 3) {};
		\node [style=none] (4) at (8, 3) {};
		\node [style=none] (5) at (6.5, 5) {};
		\node [style=none] (6) at (7.75, 5) {};
	\end{pgfonlayer}
	\begin{pgfonlayer}{edgelayer}
		\draw (3.center) to (0);
		\draw (0) to (2);
		\draw (2) to (1);
		\draw [in=-60, out=90] (4.center) to (1);
		\draw (1) to (6.center);
		\draw [in=-90, out=120] (0) to (5.center);
	\end{pgfonlayer}
\end{tikzpicture}
\hspace*{1cm}
\begin{tikzpicture}
	\begin{pgfonlayer}{nodelayer}
		\node [style=scalar] (2) at (7.75, 3.25) {$a$};
		\node [style=Z] (3) at (8, 3.75) {};
		\node [style=Z] (4) at (7.5, 2.75) {};
		\node [style=none] (5) at (8.25, 3.25) {};
		\node [style=none] (6) at (8.25, 2.5) {};
		\node [style=none] (7) at (7.25, 3.25) {};
		\node [style=none] (8) at (7.25, 4) {};
	\end{pgfonlayer}
	\begin{pgfonlayer}{edgelayer}
		\draw (8.center) to (7.center);
		\draw [in=120, out=-90, looseness=0.75] (7.center) to (4);
		\draw [in=-90, out=45] (4) to (2);
		\draw [in=-135, out=90] (2) to (3);
		\draw [in=90, out=-60, looseness=0.75] (3) to (5.center);
		\draw (5.center) to (6.center);
	\end{pgfonlayer}
\end{tikzpicture}
=
\begin{tikzpicture}
	\begin{pgfonlayer}{nodelayer}
		\node [style=scalar] (2) at (7.75, 3.25) {$a$};
		\node [style=X] (3) at (8, 3.75) {};
		\node [style=X] (4) at (7.5, 2.75) {};
		\node [style=none] (5) at (8.25, 3.25) {};
		\node [style=none] (6) at (8.25, 2.5) {};
		\node [style=none] (7) at (7.25, 3.25) {};
		\node [style=none] (8) at (7.25, 4) {};
	\end{pgfonlayer}
	\begin{pgfonlayer}{edgelayer}
		\draw (8.center) to (7.center);
		\draw [in=120, out=-90, looseness=0.75] (7.center) to (4);
		\draw [in=-90, out=45] (4) to (2);
		\draw [in=-135, out=90] (2) to (3);
		\draw [in=90, out=-60, looseness=0.75] (3) to (5.center);
		\draw (5.center) to (6.center);
	\end{pgfonlayer}
\end{tikzpicture}
=
\begin{tikzpicture}
	\begin{pgfonlayer}{nodelayer}
		\node [style=scalarop] (2) at (7.75, 3.25) {$a$};
		\node [style=none] (6) at (7.75, 2.5) {};
		\node [style=none] (8) at (7.75, 4) {};
	\end{pgfonlayer}
	\begin{pgfonlayer}{edgelayer}
		\draw (6.center) to (2);
		\draw (2) to (8.center);
	\end{pgfonlayer}
\end{tikzpicture}
\hspace*{1cm}
\begin{tikzpicture}
	\begin{pgfonlayer}{nodelayer}
		\node [style=none] (2) at (7.75, 3.75) {};
		\node [style=none] (3) at (7.75, 2.75) {};
		\node [style=scalarop] (5) at (7.75, 3.25) {$-1$};
	\end{pgfonlayer}
	\begin{pgfonlayer}{edgelayer}
		\draw (5) to (3.center);
		\draw (5) to (2.center);
	\end{pgfonlayer}
\end{tikzpicture}
=
\begin{tikzpicture}
	\begin{pgfonlayer}{nodelayer}
		\node [style=none] (2) at (7.75, 3.75) {};
		\node [style=none] (3) at (7.75, 2.75) {};
		\node [style=scalar] (5) at (7.75, 3.25) {$-1$};
	\end{pgfonlayer}
	\begin{pgfonlayer}{edgelayer}
		\draw (5) to (3.center);
		\draw (5) to (2.center);
	\end{pgfonlayer}
\end{tikzpicture}
=
\begin{tikzpicture}
	\begin{pgfonlayer}{nodelayer}
		\node [style=Z] (0) at (7, 3.5) {};
		\node [style=X] (2) at (7.25, 4) {};
		\node [style=none] (4) at (7.5, 3.5) {};
		\node [style=none] (5) at (7.5, 3.25) {};
		\node [style=none] (6) at (6.75, 4) {};
		\node [style=none] (7) at (6.75, 4.25) {};
	\end{pgfonlayer}
	\begin{pgfonlayer}{edgelayer}
		\draw (2) to (0);
		\draw [in=120, out=-90, looseness=0.75] (6.center) to (0);
		\draw [in=300, out=90, looseness=0.75] (4.center) to (2);
		\draw (5.center) to (4.center);
		\draw (6.center) to (7.center);
	\end{pgfonlayer}
\end{tikzpicture}
=
\begin{tikzpicture}
	\begin{pgfonlayer}{nodelayer}
		\node [style=Z] (0) at (7, 4) {};
		\node [style=X] (2) at (7.25, 3.5) {};
		\node [style=none] (4) at (7.5, 4) {};
		\node [style=none] (5) at (7.5, 4.25) {};
		\node [style=none] (6) at (6.75, 3.5) {};
		\node [style=none] (7) at (6.75, 3.25) {};
	\end{pgfonlayer}
	\begin{pgfonlayer}{edgelayer}
		\draw (2) to (0);
		\draw [in=-120, out=90, looseness=0.75] (6.center) to (0);
		\draw [in=-300, out=-90, looseness=0.75] (4.center) to (2);
		\draw (5.center) to (4.center);
		\draw (6.center) to (7.center);
	\end{pgfonlayer}
\end{tikzpicture}
$$
\end{lemma}


Because of the symmetry of $-1$, we use the following (symmetric) notation for the antipode:
\hfil
$
\begin{tikzpicture}
	\begin{pgfonlayer}{nodelayer}
		\node [style=none] (2) at (7.75, 3.75) {};
		\node [style=none] (3) at (7.75, 2.75) {};
		\node [style=s] (5) at (7.75, 3.25) {};
	\end{pgfonlayer}
	\begin{pgfonlayer}{edgelayer}
		\draw (5) to (3.center);
		\draw (5) to (2.center);
	\end{pgfonlayer}
\end{tikzpicture}
:=
\begin{tikzpicture}
	\begin{pgfonlayer}{nodelayer}
		\node [style=none] (2) at (7.75, 3.75) {};
		\node [style=none] (3) at (7.75, 2.75) {};
		\node [style=scalar] (5) at (7.75, 3.25) {$-1$};
	\end{pgfonlayer}
	\begin{pgfonlayer}{edgelayer}
		\draw (5) to (3.center);
		\draw (5) to (2.center);
	\end{pgfonlayer}
\end{tikzpicture}
$



\begin{lemma} \cite{ortho}
The functor $(\_)^\perp:\ih_k\to \ih_k$;
\hfil
$
\begin{tikzpicture}
	\begin{pgfonlayer}{nodelayer}
		\node [style=Z] (0) at (7, 3.5) {};
		\node [style=none] (1) at (6.75, 4) {};
		\node [style=none] (2) at (7.25, 4) {};
		\node [style=none] (3) at (6.75, 3) {};
		\node [style=none] (4) at (7.25, 3) {};
		\node [style=none] (5) at (6.75, 4.25) {};
		\node [style=none] (6) at (7.25, 4.25) {};
		\node [style=none] (7) at (7.25, 2.75) {};
		\node [style=none] (8) at (6.75, 2.75) {};
		\node [style=none] (9) at (7, 4) {$\cdots$};
		\node [style=none] (10) at (7, 3) {$\cdots$};
	\end{pgfonlayer}
	\begin{pgfonlayer}{edgelayer}
		\draw [in=-90, out=135] (0) to (1.center);
		\draw [in=-90, out=45] (0) to (2.center);
		\draw [in=-45, out=90] (4.center) to (0);
		\draw [in=90, out=-135] (0) to (3.center);
		\draw (8.center) to (3.center);
		\draw (7.center) to (4.center);
		\draw (2.center) to (6.center);
		\draw (5.center) to (1.center);
	\end{pgfonlayer}
\end{tikzpicture}
\mapsto
\begin{tikzpicture}
	\begin{pgfonlayer}{nodelayer}
		\node [style=X] (0) at (7, 3.5) {};
		\node [style=none] (1) at (6.75, 4) {};
		\node [style=none] (2) at (7.25, 4) {};
		\node [style=none] (3) at (6.75, 3) {};
		\node [style=none] (4) at (7.25, 3) {};
		\node [style=none] (5) at (6.75, 4.25) {};
		\node [style=none] (6) at (7.25, 4.25) {};
		\node [style=none] (7) at (7.25, 2.75) {};
		\node [style=none] (8) at (6.75, 2.75) {};
		\node [style=none] (9) at (7, 4) {$\cdots$};
		\node [style=none] (10) at (7, 3) {$\cdots$};
	\end{pgfonlayer}
	\begin{pgfonlayer}{edgelayer}
		\draw [in=-90, out=135] (0) to (1.center);
		\draw [in=-90, out=45] (0) to (2.center);
		\draw [in=-45, out=90] (4.center) to (0);
		\draw [in=90, out=-135] (0) to (3.center);
		\draw (8.center) to (3.center);
		\draw (7.center) to (4.center);
		\draw (2.center) to (6.center);
		\draw (5.center) to (1.center);
	\end{pgfonlayer}
\end{tikzpicture}
\hspace*{.5cm}
\begin{tikzpicture}
	\begin{pgfonlayer}{nodelayer}
		\node [style=X] (0) at (7, 3.5) {};
		\node [style=none] (1) at (6.75, 4) {};
		\node [style=none] (2) at (7.25, 4) {};
		\node [style=none] (3) at (6.75, 3) {};
		\node [style=none] (4) at (7.25, 3) {};
		\node [style=none] (5) at (6.75, 4.25) {};
		\node [style=none] (6) at (7.25, 4.25) {};
		\node [style=none] (7) at (7.25, 2.75) {};
		\node [style=none] (8) at (6.75, 2.75) {};
		\node [style=none] (9) at (7, 4) {$\cdots$};
		\node [style=none] (10) at (7, 3) {$\cdots$};
	\end{pgfonlayer}
	\begin{pgfonlayer}{edgelayer}
		\draw [in=-90, out=135] (0) to (1.center);
		\draw [in=-90, out=45] (0) to (2.center);
		\draw [in=-45, out=90] (4.center) to (0);
		\draw [in=90, out=-135] (0) to (3.center);
		\draw (8.center) to (3.center);
		\draw (7.center) to (4.center);
		\draw (2.center) to (6.center);
		\draw (5.center) to (1.center);
	\end{pgfonlayer}
\end{tikzpicture}
\mapsto
\begin{tikzpicture}
	\begin{pgfonlayer}{nodelayer}
		\node [style=Z] (0) at (7, 3.5) {};
		\node [style=none] (1) at (6.75, 4) {};
		\node [style=none] (2) at (7.25, 4) {};
		\node [style=none] (3) at (6.75, 3) {};
		\node [style=none] (4) at (7.25, 3) {};
		\node [style=none] (5) at (6.75, 4.25) {};
		\node [style=none] (6) at (7.25, 4.25) {};
		\node [style=none] (7) at (7.25, 2.75) {};
		\node [style=none] (8) at (6.75, 2.75) {};
		\node [style=none] (9) at (7, 4) {$\cdots$};
		\node [style=none] (10) at (7, 3) {$\cdots$};
	\end{pgfonlayer}
	\begin{pgfonlayer}{edgelayer}
		\draw [in=-90, out=135] (0) to (1.center);
		\draw [in=-90, out=45] (0) to (2.center);
		\draw [in=-45, out=90] (4.center) to (0);
		\draw [in=90, out=-135] (0) to (3.center);
		\draw (8.center) to (3.center);
		\draw (7.center) to (4.center);
		\draw (2.center) to (6.center);
		\draw (5.center) to (1.center);
	\end{pgfonlayer}
\end{tikzpicture}
\hspace*{.5cm}
\begin{tikzpicture}
	\begin{pgfonlayer}{nodelayer}
		\node [style=none] (2) at (7.75, 3.75) {};
		\node [style=none] (3) at (7.75, 2.75) {};
		\node [style=scalar] (5) at (7.75, 3.25) {$a$};
	\end{pgfonlayer}
	\begin{pgfonlayer}{edgelayer}
		\draw (5) to (3.center);
		\draw (5) to (2.center);
	\end{pgfonlayer}
\end{tikzpicture}
\mapsto
\begin{tikzpicture}
	\begin{pgfonlayer}{nodelayer}
		\node [style=none] (2) at (7.75, 3.75) {};
		\node [style=none] (3) at (7.75, 2.75) {};
		\node [style=scalarop] (5) at (7.75, 3.25) {$a$};
	\end{pgfonlayer}
	\begin{pgfonlayer}{edgelayer}
		\draw (5) to (3.center);
		\draw (5) to (2.center);
	\end{pgfonlayer}
\end{tikzpicture}
\hspace*{.5cm}
\begin{tikzpicture}
	\begin{pgfonlayer}{nodelayer}
		\node [style=none] (2) at (7.75, 3.75) {};
		\node [style=none] (3) at (7.75, 2.75) {};
		\node [style=scalarop] (5) at (7.75, 3.25) {$a$};
	\end{pgfonlayer}
	\begin{pgfonlayer}{edgelayer}
		\draw (5) to (3.center);
		\draw (5) to (2.center);
	\end{pgfonlayer}
\end{tikzpicture}
\mapsto
\begin{tikzpicture}
	\begin{pgfonlayer}{nodelayer}
		\node [style=none] (2) at (7.75, 3.75) {};
		\node [style=none] (3) at (7.75, 2.75) {};
		\node [style=scalar] (5) at (7.75, 3.25) {$a$};
	\end{pgfonlayer}
	\begin{pgfonlayer}{edgelayer}
		\draw (5) to (3.center);
		\draw (5) to (2.center);
	\end{pgfonlayer}
\end{tikzpicture}
$

is the isomorphism which takes linear subspaces to their orthogonal complement, that is to say:
$$
V \mapsto V^\perp := \{  v \in V: \forall w \in V , \langle v,w\rangle = 0\}
$$

\end{lemma}

Notice that the orthogonal complement is an involution so that $(V^\perp)^\perp = V$.

\section{Lagrangian relations}
\label{sec:sym}

Now that we have a graphical presentation of linear relations, we can now do same for (linear) Lagrangian relations.  We first recall some of the basic theory of symplectic vector spaces.  This is expounded upon in much greater generality in the not-necessarily-linear case in \cite{weinstein}.  In this entire paper, we only care about the linear and affine cases; and things will assumed to be linear unless otherwise stated.  As previously mentioned, Lagrangian relations (and their affine counterpart) have previously been studied within the context of monoidal categories  to model electrical circuits among other things \cite{passive,network,coya}; although, to the knowledge of the authors, no proof of universality exists in the literature.

\begin{definition}
  Given a field  $k$ and a $k$-vector space $V$, a {\bf symplectic form} on $V$ is a bilinear map $\omega:V\times V\to k$ which is:
\begin{itemize}
 \item {\bf Alternating:} $\forall v \in V$, $\omega(v,v)=0$ \item {\bf Non-degenerate:} if $\exists v \in V \forall w \in V: \omega(v,w)=0$, then $v=0$.
\end{itemize}
  A {\bf symplectic vector space} is a vector space equipped with a symplectic form. A (linear) {\bf symplectomorphism} is a linear isomorphism between symplectic vector spaces that preserves the symplectic form.
\end{definition}


\begin{lemma}
\label{lemma:sform}
Every vector space $k^{2n}$ is equipped with a bilinear form given by the following block matrix:
$$
\omega:=
\begin{bmatrix}
0_n & I_n\\
-I_n & 0_n
\end{bmatrix}
$$
so that $\omega(v,w) := v \omega w^T$.
Moreover, every finite dimensional symplectic vector space over $k$ is symplectomorphic to one of the form $k^{2n}$ with such a symplectic form.
\end{lemma}



%
%There is some sort of skewed duality between the two gradings of a symplectic vector space $k^n$.
%In classical mechanics, this is used to model the duality between momentum and physics for example.  This hints at there being some relation to stabilizer quantum mechanics, where there is a similar such duality.
%
%
%Just as one can define the dual space $V^\perp$ of a subspace $V \subseteq W$ with respect to the regular inner product of vector spaces, there is an analagous notion with the symplectic form.

\begin{definition}

Let $W \subseteq V$ be a linear subspace of a symplectic space $V$.
The {\bf symplectic dual} of the subspace $W$ is defined to be
$
W^\omega:= \{v \in V : \forall w \in W, \omega(v,w)=0 \}
$.
A linear subspace  $W$ of a symplectic vector space $V$ is {\bf isotropic} when $W^\omega \supseteq W$, {\bf coisotropic} when $W^\omega \subseteq W$ and {\bf Lagrangian} when $W^\omega=W$.


\end{definition}

\begin{lemma}
Every symplectomorphism $f:V\to V$ induces a Lagrangian relation $\Gamma_f:=\{ (fv, v) | v \in V \}$.
\end{lemma}

These spaces have a natural grading into two distinct parts $V \oplus W \subseteq k^n \oplus k^n$. By analogy to the case of quantum stabilizer theory, we call the left part the \textit{X-grading} and the right part the \textit{Z-grading}.

As a matter of convention, we consider linear subspaces as being represented as the row space of a matrix. So in particular, a symplectic subspace of $k^{2n}$ is represented by a matrix of the form $[X|Z]$ where $X,Z$ are both $n\times n$-dimensional matrices.
An isotropic subspace can equivalently be described as a matrix $[X|Z]$ so that $[X|Z] \omega [X|Z]^T = 0$.
Moreover, a Lagrangian subspace can be described as a matrix as above which additionally has rank $n$.


\begin{definition}
Given a field $k$, the prop of {\bf Lagrangian relations},  $\Lag\Rel_k$ has morphisms $k^{2n}\to k^{2m}$ as Lagrangian subspaces of the symplectic vector space $k^{n+m} \oplus k^{n+m}$ with symplectic form given above.  Composition is given by relational composition and the tensor product is given by the direct sum.
\end{definition}
%For the purposes of this paper, because it is so much easier to work in a prop, we will draw string diagram in the skeleton of $\Lag\Rel_k$ whose objects are all of the form $k^{2n}$ equipped with the symplectic form of Lemma \ref{lemma:sform}.

The direct sum of Lagrangian subspaces is graphically depicted as follows:
$$
\begin{tikzpicture}
	\begin{pgfonlayer}{nodelayer}
		\node [style=map] (616) at (272, 0) {$V$};
		\node [style=none] (617) at (271.75, 1) {};
		\node [style=none] (618) at (272.25, 1) {};
	\end{pgfonlayer}
	\begin{pgfonlayer}{edgelayer}
		\draw [in=-90, out=60] (616) to (618.center);
		\draw [in=-90, out=120] (616) to (617.center);
	\end{pgfonlayer}
\end{tikzpicture}
\oplus
\begin{tikzpicture}
	\begin{pgfonlayer}{nodelayer}
		\node [style=map] (616) at (272, 0) {$W$};
		\node [style=none] (617) at (271.75, 1) {};
		\node [style=none] (618) at (272.25, 1) {};
	\end{pgfonlayer}
	\begin{pgfonlayer}{edgelayer}
		\draw [in=-90, out=60] (616) to (618.center);
		\draw [in=-90, out=120] (616) to (617.center);
	\end{pgfonlayer}
\end{tikzpicture}
:=
\begin{tikzpicture}
	\begin{pgfonlayer}{nodelayer}
		\node [style=map] (616) at (272, 0) {$V$};
		\node [style=none] (617) at (271.75, 1) {};
		\node [style=none] (618) at (272.75, 1) {};
		\node [style=map] (619) at (272.75, 0) {$W$};
		\node [style=none] (620) at (272, 1) {};
		\node [style=none] (621) at (273, 1) {};
	\end{pgfonlayer}
	\begin{pgfonlayer}{edgelayer}
		\draw [in=-90, out=60] (616) to (618.center);
		\draw [in=-90, out=120] (616) to (617.center);
		\draw [in=-90, out=60] (619) to (621.center);
		\draw [in=-90, out=120] (619) to (620.center);
	\end{pgfonlayer}
\end{tikzpicture}
$$
Where we are grouping the $X$ gradings together on the left and the $Z$ gradings together on the right. Note that this means the embedding of $\Lag\Rel_k$ into $\LinRel_k$ preserves the monoidal product only up to isomorphism. More precisely, we have the following fact.

\begin{lemma}
\label{lemma:strong}
The forgetful functor $E:\Lag\Rel_k \to \LinRel_k$  is a faithful, strong symmetric monoidal.
\end{lemma}

\begin{proof}
  Functoriality and faithfulness is immediate. The strong monoidal structure is given by $E(I) = I$ and
  \[ E(A) \oplus E(B) := A \oplus A \oplus B \oplus B \xrightarrow{1 \oplus \sigma \oplus 1} A \oplus B \oplus A \oplus B =: E(A \oplus B). \]
  The symmetric monoidal structure on $\Lag\Rel_k$ is chosen such that it is consistent with the monoidal structure above.
\end{proof}

Due to the above lemma, we will regard $\Lag\Rel_k$ as a symmetric monoidal subcategory of $\LinRel_k$.
As such, we can ask what the generators of $\Lag\Rel_k$ look like in terms of string diagrams of $\ih_k$ generators. We first describe what it means to be a Lagrangian relation in pictures, where the $X$ block is the wire on the left and $Z$ block is the wire on the right:
\begin{equation}
\label{eq:lag}
\begin{tikzpicture}
	\begin{pgfonlayer}{nodelayer}
		\node [style=map] (0) at (0.75, -1) {$W$};
		\node [style=none] (1) at (0.5, 0) {};
		\node [style=none] (2) at (1, 0) {};
	\end{pgfonlayer}
	\begin{pgfonlayer}{edgelayer}
		\draw [in=120, out=-90] (1.center) to (0);
		\draw [in=-90, out=60] (0) to (2.center);
	\end{pgfonlayer}
\end{tikzpicture}
%=
%\begin{tikzpicture}
%	\begin{pgfonlayer}{nodelayer}
%		\node [style=map] (0) at (0.75, -1) {$V^\perp$};
%		\node [style=none] (1) at (0.5, 0) {};
%		\node [style=none] (2) at (1, 0) {};
%	\end{pgfonlayer}
%	\begin{pgfonlayer}{edgelayer}
%		\draw [in=120, out=-90] (1.center) to (0);
%		\draw [in=-90, out=60] (0) to (2.center);
%	\end{pgfonlayer}
%\end{tikzpicture}
=
\begin{tikzpicture}
	\begin{pgfonlayer}{nodelayer}
		\node [style=map] (0) at (0.75, -1.75) {$W^\perp$};
		\node [style=none] (1) at (0.5, -1) {};
		\node [style=none] (2) at (1, -1) {};
		\node [style=none] (3) at (1, 0) {};
		\node [style=none] (4) at (0.5, 0) {};
		\node [style=s] (5) at (1, -1) {};
	\end{pgfonlayer}
	\begin{pgfonlayer}{edgelayer}
		\draw [in=120, out=-90] (1.center) to (0);
		\draw [in=-90, out=60] (0) to (2.center);
		\draw [in=-90, out=90] (2.center) to (4.center);
		\draw [in=-270, out=-90] (3.center) to (1.center);
	\end{pgfonlayer}
\end{tikzpicture}
\end{equation}
Algebraically, for $W$ a subspace of $V$, the right hand side is interpreted as follows:
\begin{align*}
W^\omega :&= \{(v_1,v_2) \in V : \forall (w_1,w_2) \in W, \omega((v_1,v_2),(w_1,w_2))=0 \}\\
                    &= \{(v_1,v_2) \in V : \forall (w_1,w_2) \in W,  \langle (v_2,-v_1) ,(w_1,w_2)\rangle =0 \}\\
                    &= \{(v_2,-v_1) \in V : \forall (w_1,w_2) \in W,  \langle (v_1,v_2) ,(w_1,w_2)\rangle =0 \}
\end{align*}

The category of Lagrangian relations is compact closed.  Given a relation $V$ between symplectic vector spaces, we can curry it into a state $\hat V$; and similarily, we can uncurry a state $W$ into a process $\widecheck W$,
$$
\begin{tikzpicture}
	\begin{pgfonlayer}{nodelayer}
		\node [style=map] (0) at (0.75, -1.75) {$V$};
		\node [style=none] (1) at (0.5, -1) {};
		\node [style=none] (2) at (1, -1) {};
		\node [style=none] (3) at (0.5, -2.5) {};
		\node [style=none] (4) at (1, -2.5) {};
	\end{pgfonlayer}
	\begin{pgfonlayer}{edgelayer}
		\draw [in=120, out=-90] (1.center) to (0);
		\draw [in=-90, out=60] (0) to (2.center);
		\draw [in=-60, out=90] (4.center) to (0);
		\draw [in=90, out=-120] (0) to (3.center);
	\end{pgfonlayer}
\end{tikzpicture}
\xmapsto{\hat{(\_)} }
\begin{tikzpicture}
	\begin{pgfonlayer}{nodelayer}
		\node [style=map] (0) at (0.75, -1.75) {$V$};
		\node [style=none] (1) at (0, -1) {};
		\node [style=none] (2) at (1.25, -1) {};
		\node [style=none] (4) at (1.25, -2.5) {};
		\node [style=X] (5) at (0, -3) {};
		\node [style=Z] (6) at (0.75, -3) {};
		\node [style=none] (7) at (0.75, -1) {};
		\node [style=none] (8) at (-0.5, -1) {};
		\node [style=none] (9) at (0, -2) {};
	\end{pgfonlayer}
	\begin{pgfonlayer}{edgelayer}
		\draw [in=120, out=-90] (1.center) to (0);
		\draw [in=-90, out=60] (0) to (2.center);
		\draw [in=-45, out=90] (4.center) to (0);
		\draw [in=-90, out=135, looseness=0.75] (5) to (8.center);
		\draw [in=30, out=-90] (4.center) to (6);
		\draw [in=90, out=-90] (7.center) to (9.center);
		\draw [in=150, out=-90] (9.center) to (6);
		\draw [in=45, out=-135] (0) to (5);
	\end{pgfonlayer}
\end{tikzpicture}
\hspace*{1cm}
\begin{tikzpicture}
	\begin{pgfonlayer}{nodelayer}
		\node [style=map] (0) at (1.5, -2) {$W$};
		\node [style=none] (1) at (1.25, -1) {};
		\node [style=none] (2) at (2.25, -1) {};
		\node [style=none] (6) at (1.75, -1) {};
		\node [style=none] (7) at (0.75, -1) {};
	\end{pgfonlayer}
	\begin{pgfonlayer}{edgelayer}
		\draw [in=105, out=-90] (1.center) to (0);
		\draw [in=-90, out=60] (0) to (2.center);
		\draw [in=120, out=-90] (7.center) to (0);
		\draw [in=-90, out=75] (0) to (6.center);
	\end{pgfonlayer}
\end{tikzpicture}
\xmapsto{\widecheck{(\_)} }
\begin{tikzpicture}
	\begin{pgfonlayer}{nodelayer}
		\node [style=map] (0) at (1.75, -2.25) {$W$};
		\node [style=none] (1) at (1, -0.75) {};
		\node [style=none] (2) at (2, -0.75) {};
		\node [style=none] (6) at (1.5, -1.25) {};
		\node [style=none] (7) at (0.75, -1.25) {};
		\node [style=Z] (8) at (1.5, -1.25) {};
		\node [style=X] (9) at (0.75, -1.25) {};
		\node [style=none] (10) at (0.25, -2.5) {};
		\node [style=none] (11) at (1, -2.5) {};
	\end{pgfonlayer}
	\begin{pgfonlayer}{edgelayer}
		\draw [in=105, out=-90, looseness=1.25] (1.center) to (0);
		\draw [in=-90, out=45, looseness=0.75] (0) to (2.center);
		\draw [in=135, out=-30] (7.center) to (0);
		\draw [in=-45, out=75] (0) to (6.center);
		\draw [in=-135, out=90] (10.center) to (9);
		\draw [in=-135, out=90] (11.center) to (8);
	\end{pgfonlayer}
\end{tikzpicture}
$$
It is easy to see that these two constructions are inverse to each other.
This allows us to derive a graphical criteria for abitrary Lagrangian relations, generalizing Equation \ref{eq:lag}:
$$
\begin{tikzpicture}
	\begin{pgfonlayer}{nodelayer}
		\node [style=map] (0) at (0.75, -1.75) {$V$};
		\node [style=none] (1) at (0, -0.75) {};
		\node [style=none] (2) at (1.25, -0.75) {};
		\node [style=none] (4) at (1.25, -2.5) {};
		\node [style=X] (5) at (0, -3) {};
		\node [style=Z] (6) at (0.75, -3) {};
		\node [style=none] (7) at (0.75, -0.75) {};
		\node [style=none] (8) at (-0.5, -0.75) {};
		\node [style=none] (9) at (0, -2) {};
	\end{pgfonlayer}
	\begin{pgfonlayer}{edgelayer}
		\draw [in=120, out=-90] (1.center) to (0);
		\draw [in=-90, out=60] (0) to (2.center);
		\draw [in=-45, out=90] (4.center) to (0);
		\draw [in=-90, out=135, looseness=0.75] (5) to (8.center);
		\draw [in=30, out=-90] (4.center) to (6);
		\draw [in=90, out=-90] (7.center) to (9.center);
		\draw [in=150, out=-90] (9.center) to (6);
		\draw [in=45, out=-135] (0) to (5);
	\end{pgfonlayer}
\end{tikzpicture}
=
\begin{tikzpicture}
	\begin{pgfonlayer}{nodelayer}
		\node [style=map] (43) at (13.5, -1.75) {$V^\perp$};
		\node [style=none] (44) at (12.75, -0.75) {};
		\node [style=none] (45) at (14, -0.75) {};
		\node [style=none] (46) at (14, -2.5) {};
		\node [style=Z] (47) at (12.75, -3) {};
		\node [style=X] (48) at (13.5, -3) {};
		\node [style=none] (49) at (13.5, -0.75) {};
		\node [style=none] (50) at (12.25, -0.75) {};
		\node [style=none] (51) at (12.75, -2) {};
		\node [style=none] (52) at (13.5, 0.75) {};
		\node [style=none] (53) at (14, 0.75) {};
		\node [style=none] (54) at (12.25, 0.75) {};
		\node [style=none] (55) at (12.75, 0.75) {};
		\node [style=s] (56) at (14, -0.75) {};
		\node [style=s] (57) at (13.5, -0.75) {};
		\node [style=none] (58) at (13.25, -3.5) {};
	\end{pgfonlayer}
	\begin{pgfonlayer}{edgelayer}
		\draw [in=120, out=-90] (44.center) to (43);
		\draw [in=-90, out=60] (43) to (45.center);
		\draw [in=-45, out=90] (46.center) to (43);
		\draw [in=-90, out=135, looseness=0.75] (47) to (50.center);
		\draw [in=30, out=-90] (46.center) to (48);
		\draw [in=90, out=-90] (49.center) to (51.center);
		\draw [in=150, out=-90] (51.center) to (48);
		\draw [in=45, out=-135] (43) to (47);
		\draw [in=270, out=90] (45.center) to (55.center);
		\draw [in=270, out=90] (49.center) to (54.center);
		\draw [in=270, out=90] (44.center) to (53.center);
		\draw [in=270, out=90] (50.center) to (52.center);
	\end{pgfonlayer}
\end{tikzpicture}
\iff
\begin{tikzpicture}
	\begin{pgfonlayer}{nodelayer}
		\node [style=map] (0) at (2, -2) {$V$};
		\node [style=none] (1) at (1.75, -1.25) {};
		\node [style=none] (2) at (2.25, -1.25) {};
		\node [style=none] (3) at (1.75, -2.75) {};
		\node [style=none] (4) at (2.25, -2.75) {};
	\end{pgfonlayer}
	\begin{pgfonlayer}{edgelayer}
		\draw [in=120, out=-90] (1.center) to (0);
		\draw [in=-90, out=60] (0) to (2.center);
		\draw [in=-60, out=90] (4.center) to (0);
		\draw [in=90, out=-120] (0) to (3.center);
	\end{pgfonlayer}
\end{tikzpicture}
=
\begin{tikzpicture}
	\begin{pgfonlayer}{nodelayer}
		\node [style=map] (0) at (2.5, -1.75) {$V$};
		\node [style=none] (1) at (2, -0.5) {};
		\node [style=none] (2) at (3, -0.5) {};
		\node [style=none] (3) at (3, -2.5) {};
		\node [style=X] (4) at (1.75, -3) {};
		\node [style=Z] (5) at (2.5, -3) {};
		\node [style=none] (6) at (2.5, -0.75) {};
		\node [style=none] (7) at (1.5, -0.75) {};
		\node [style=none] (8) at (1.75, -2) {};
		\node [style=X] (9) at (1.5, -0.75) {};
		\node [style=Z] (10) at (2.5, -0.75) {};
		\node [style=none] (11) at (0.5, -3.25) {};
		\node [style=none] (12) at (1, -3.25) {};
	\end{pgfonlayer}
	\begin{pgfonlayer}{edgelayer}
		\draw [in=120, out=-90] (1.center) to (0);
		\draw [in=-90, out=60] (0) to (2.center);
		\draw [in=-45, out=90] (3.center) to (0);
		\draw [in=-45, out=150] (4) to (7.center);
		\draw [in=30, out=-90] (3.center) to (5);
		\draw [in=90, out=-45, looseness=1.25] (6.center) to (8.center);
		\draw [in=150, out=-90] (8.center) to (5);
		\draw [in=45, out=-135] (0) to (4);
		\draw [in=90, out=-150] (10) to (12.center);
		\draw [in=90, out=-120] (9) to (11.center);
	\end{pgfonlayer}
\end{tikzpicture}
=
\begin{tikzpicture}
	\begin{pgfonlayer}{nodelayer}
		\node [style=map] (59) at (17, -1.75) {$V^\perp$};
		\node [style=none] (60) at (16.25, -0.75) {};
		\node [style=none] (61) at (17.5, -0.75) {};
		\node [style=none] (62) at (17.5, -2.5) {};
		\node [style=Z] (63) at (16.25, -3) {};
		\node [style=X] (64) at (17, -3) {};
		\node [style=none] (65) at (17, -0.75) {};
		\node [style=none] (66) at (15.75, -0.75) {};
		\node [style=none] (67) at (16.25, -2) {};
		\node [style=none] (68) at (16.5, 0.75) {};
		\node [style=none] (69) at (17.5, 1) {};
		\node [style=none] (70) at (15.5, 0.75) {};
		\node [style=none] (71) at (16, 1) {};
		\node [style=s] (72) at (17.5, -0.75) {};
		\node [style=s] (73) at (17, -0.75) {};
		\node [style=X] (74) at (15.5, 0.75) {};
		\node [style=Z] (75) at (16.5, 0.75) {};
		\node [style=none] (76) at (15, -3.25) {};
		\node [style=none] (77) at (15.5, -3.25) {};
		\node [style=none] (78) at (16.5, 1.5) {};
	\end{pgfonlayer}
	\begin{pgfonlayer}{edgelayer}
		\draw [in=120, out=-90] (60.center) to (59);
		\draw [in=-90, out=60] (59) to (61.center);
		\draw [in=-45, out=90] (62.center) to (59);
		\draw [in=-90, out=135, looseness=0.75] (63) to (66.center);
		\draw [in=30, out=-90] (62.center) to (64);
		\draw [in=90, out=-90] (65.center) to (67.center);
		\draw [in=150, out=-90] (67.center) to (64);
		\draw [in=45, out=-135] (59) to (63);
		\draw [in=-90, out=90, looseness=1.25] (61.center) to (71.center);
		\draw [in=-60, out=90] (65.center) to (70.center);
		\draw [in=270, out=90] (60.center) to (69.center);
		\draw [in=-45, out=90, looseness=1.25] (66.center) to (68.center);
		\draw [in=-135, out=90, looseness=0.50] (76.center) to (74);
		\draw [in=-150, out=90] (77.center) to (75);
	\end{pgfonlayer}
\end{tikzpicture}
=
\begin{tikzpicture}
	\begin{pgfonlayer}{nodelayer}
		\node [style=map] (0) at (4.5, -1.75) {$V^\perp$};
		\node [style=none] (1) at (4, -1) {};
		\node [style=none] (2) at (5, -1) {};
		\node [style=none] (3) at (5, -2.5) {};
		\node [style=X] (4) at (4.75, -3) {};
		\node [style=none] (5) at (3.75, -2.25) {};
		\node [style=none] (6) at (5, 0) {};
		\node [style=none] (7) at (4, 0) {};
		\node [style=s] (8) at (5, -1) {};
		\node [style=none] (9) at (3.25, -3.25) {};
		\node [style=none] (10) at (4, -3.25) {};
		\node [style=Z] (11) at (3.75, -2.25) {};
	\end{pgfonlayer}
	\begin{pgfonlayer}{edgelayer}
		\draw [in=135, out=-90] (1.center) to (0);
		\draw [in=-90, out=60] (0) to (2.center);
		\draw [in=-45, out=90] (3.center) to (0);
		\draw [in=30, out=-90] (3.center) to (4);
		\draw [in=165, out=-15, looseness=1.25] (5.center) to (4);
		\draw [in=-90, out=90, looseness=1.25] (2.center) to (7.center);
		\draw [in=270, out=90] (1.center) to (6.center);
		\draw [in=240, out=90] (10.center) to (0);
		\draw [in=90, out=-150] (11) to (9.center);
	\end{pgfonlayer}
\end{tikzpicture}
=
\begin{tikzpicture}
	\begin{pgfonlayer}{nodelayer}
		\node [style=map] (0) at (2, -2) {$V^\perp$};
		\node [style=none] (1) at (1.75, -1.25) {};
		\node [style=none] (2) at (2.25, -1.25) {};
		\node [style=none] (3) at (1.75, -2.75) {};
		\node [style=none] (4) at (2.25, -2.75) {};
		\node [style=none] (5) at (2.25, -0.5) {};
		\node [style=none] (6) at (1.75, -0.5) {};
		\node [style=none] (7) at (2.25, -3.5) {};
		\node [style=none] (8) at (1.75, -3.5) {};
		\node [style=s] (9) at (2.25, -1.25) {};
		\node [style=s] (10) at (2.25, -2.75) {};
	\end{pgfonlayer}
	\begin{pgfonlayer}{edgelayer}
		\draw [in=120, out=-90] (1.center) to (0);
		\draw [in=-90, out=60] (0) to (2.center);
		\draw [in=-60, out=90] (4.center) to (0);
		\draw [in=90, out=-120] (0) to (3.center);
		\draw [in=90, out=-90] (6.center) to (2.center);
		\draw [in=270, out=90] (1.center) to (5.center);
		\draw [in=270, out=90] (7.center) to (3.center);
		\draw [in=270, out=90] (8.center) to (4.center);
	\end{pgfonlayer}
\end{tikzpicture}
$$
For this reason, we will depict Lagrangian relations as processes, where the input wires are on the bottom and output wires on on the top.

\begin{lemma}
There is a faithful, strong symmetric monoidal functor $L:\LinRel_k\to\Lag\Rel_k$ given by the following action on the generators of $\ih_k$; doubling, and then changing the colours of one of the copies:
$$
\begin{tikzpicture}
	\begin{pgfonlayer}{nodelayer}
		\node [style=map] (0) at (-3, -1) {$V$};
		\node [style=none] (1) at (-3, -0.25) {};
		\node [style=none] (2) at (-3, -1.75) {};
	\end{pgfonlayer}
	\begin{pgfonlayer}{edgelayer}
		\draw (1.center) to (0);
		\draw (0) to (2.center);
	\end{pgfonlayer}
\end{tikzpicture}
\mapsto
\begin{tikzpicture}
	\begin{pgfonlayer}{nodelayer}
		\node [style=map] (0) at (-3, -1) {$V^\perp$};
		\node [style=none] (1) at (-3, -0.25) {};
		\node [style=none] (2) at (-3, -1.75) {};
		\node [style=map] (3) at (-2.25, -1) {$V$};
		\node [style=none] (4) at (-2.25, -0.25) {};
		\node [style=none] (5) at (-2.25, -1.75) {};
	\end{pgfonlayer}
	\begin{pgfonlayer}{edgelayer}
		\draw (1.center) to (0);
		\draw (0) to (2.center);
		\draw (4.center) to (3);
		\draw (3) to (5.center);
	\end{pgfonlayer}
\end{tikzpicture}
$$
%
%$$
%\begin{tikzpicture}
%	\begin{pgfonlayer}{nodelayer}
%		\node [style=map] (0) at (-3, -1) {$V^\perp$};
%		\node [style=none] (1) at (-3, -0.25) {};
%		\node [style=none] (2) at (-3, -1.75) {};
%		\node [style=map] (3) at (-2.25, -1) {$V$};
%		\node [style=none] (4) at (-2.25, -0.25) {};
%		\node [style=none] (5) at (-2.25, -1.75) {};
%	\end{pgfonlayer}
%	\begin{pgfonlayer}{edgelayer}
%		\draw (1.center) to (0);
%		\draw (0) to (2.center);
%		\draw (4.center) to (3);
%		\draw (3) to (5.center);
%	\end{pgfonlayer}
%\end{tikzpicture}
%=
%\begin{tikzpicture}
%	\begin{pgfonlayer}{nodelayer}
%		\node [style=map] (0) at (-3, -1) {$V$};
%		\node [style=none] (1) at (-3, -0.25) {};
%		\node [style=none] (2) at (-3, -1.75) {};
%		\node [style=map] (3) at (-2.25, -1) {$V^\perp$};
%		\node [style=none] (4) at (-2.25, -0.25) {};
%		\node [style=none] (5) at (-2.25, -1.75) {};
%		\node [style=none] (6) at (-2.25, 0.75) {};
%		\node [style=none] (7) at (-3, 0.75) {};
%		\node [style=none] (8) at (-2.25, -2.75) {};
%		\node [style=none] (9) at (-3, -2.75) {};
%	\end{pgfonlayer}
%	\begin{pgfonlayer}{edgelayer}
%		\draw (1.center) to (0);
%		\draw (0) to (2.center);
%		\draw (4.center) to (3);
%		\draw (3) to (5.center);
%		\draw [in=270, out=90] (1.center) to (6.center);
%		\draw [in=270, out=90] (4.center) to (7.center);
%		\draw [in=270, out=90] (8.center) to (2.center);
%		\draw [in=270, out=90] (9.center) to (5.center);
%	\end{pgfonlayer}
%\end{tikzpicture}
%=
%\begin{tikzpicture}
%	\begin{pgfonlayer}{nodelayer}
%		\node [style=map] (0) at (0.5, -1) {$V$};
%		\node [style=none] (1) at (0.5, 0.25) {};
%		\node [style=none] (2) at (0.5, -1.75) {};
%		\node [style=map] (3) at (1.25, -1) {$V^\perp$};
%		\node [style=none] (4) at (1.25, 0.25) {};
%		\node [style=none] (5) at (1.25, -1.75) {};
%		\node [style=none] (6) at (1.25, 1.25) {};
%		\node [style=none] (7) at (0.5, 1.25) {};
%		\node [style=none] (8) at (1.25, -2.75) {};
%		\node [style=none] (9) at (0.5, -2.75) {};
%		\node [style=s] (10) at (1.25, 0.25) {};
%		\node [style=s] (11) at (1.25, -0.25) {};
%	\end{pgfonlayer}
%	\begin{pgfonlayer}{edgelayer}
%		\draw (1.center) to (0);
%		\draw (0) to (2.center);
%		\draw (4.center) to (3);
%		\draw (3) to (5.center);
%		\draw [in=270, out=90] (1.center) to (6.center);
%		\draw [in=270, out=90] (4.center) to (7.center);
%		\draw [in=270, out=90] (8.center) to (2.center);
%		\draw [in=270, out=90] (9.center) to (5.center);
%	\end{pgfonlayer}
%\end{tikzpicture}
%=
%\begin{tikzpicture}
%	\begin{pgfonlayer}{nodelayer}
%		\node [style=map] (0) at (-3, -1) {$V$};
%		\node [style=none] (1) at (-3, -0.25) {};
%		\node [style=none] (2) at (-3, -1.75) {};
%		\node [style=map] (3) at (-2.25, -1) {$V^\perp$};
%		\node [style=none] (4) at (-2.25, -0.25) {};
%		\node [style=none] (5) at (-2.25, -1.75) {};
%		\node [style=none] (6) at (-2.25, 0.75) {};
%		\node [style=none] (7) at (-3, 0.75) {};
%		\node [style=none] (8) at (-2.25, -2.75) {};
%		\node [style=none] (9) at (-3, -2.75) {};
%		\node [style=s] (10) at (-2.25, -0.25) {};
%		\node [style=s] (11) at (-2.25, -1.75) {};
%	\end{pgfonlayer}
%	\begin{pgfonlayer}{edgelayer}
%		\draw (1.center) to (0);
%		\draw (0) to (2.center);
%		\draw (4.center) to (3);
%		\draw (3) to (5.center);
%		\draw [in=270, out=90] (1.center) to (6.center);
%		\draw [in=270, out=90] (4.center) to (7.center);
%		\draw [in=270, out=90] (8.center) to (2.center);
%		\draw [in=270, out=90] (9.center) to (5.center);
%	\end{pgfonlayer}
%\end{tikzpicture}
%$$
\end{lemma}


To check this is a functor, all we have to show is that it produces Lagrangian relations. This follows immediately from the naturality of $-1$.
This functor is symmetric monoidal and faithful but not full, as for example, the following Lagrangian relation is not in the image of $L$:
$$
\begin{tikzpicture}
	\begin{pgfonlayer}{nodelayer}
		\node [style=Z] (0) at (0.5, 0) {};
		\node [style=none] (1) at (0.5, 1) {};
		\node [style=none] (2) at (0.5, -1) {};
		\node [style=X] (3) at (1.5, 0) {};
		\node [style=X] (4) at (1, 0.5) {};
		\node [style=none] (5) at (1.5, 1) {};
		\node [style=none] (6) at (1.5, -1) {};
	\end{pgfonlayer}
	\begin{pgfonlayer}{edgelayer}
		\draw (2.center) to (0);
		\draw (0) to (1.center);
		\draw (6.center) to (3);
		\draw (3) to (5.center);
		\draw (3) to (4);
		\draw (4) to (0);
	\end{pgfonlayer}
\end{tikzpicture}
=
\begin{tikzpicture}
	\begin{pgfonlayer}{nodelayer}
		\node [style=Z] (14) at (4.5, -0.25) {};
		\node [style=none] (15) at (4.5, 0.5) {};
		\node [style=none] (16) at (4.5, -0.75) {};
		\node [style=X] (17) at (3.5, -0.25) {};
		\node [style=none] (18) at (3.5, 0.5) {};
		\node [style=none] (19) at (3.5, -0.75) {};
		\node [style=none] (21) at (3.5, 1.5) {};
		\node [style=none] (22) at (4.5, 1.5) {};
		\node [style=none] (23) at (3.5, -1.75) {};
		\node [style=none] (24) at (4.5, -1.75) {};
		\node [style=X] (25) at (4, 0.25) {};
	\end{pgfonlayer}
	\begin{pgfonlayer}{edgelayer}
		\draw (16.center) to (14);
		\draw (14) to (15.center);
		\draw (19.center) to (17);
		\draw (17) to (18.center);
		\draw [in=270, out=90] (18.center) to (22.center);
		\draw [in=270, out=90] (15.center) to (21.center);
		\draw [in=270, out=90] (23.center) to (16.center);
		\draw [in=270, out=90] (24.center) to (19.center);
		\draw (17) to (25);
		\draw (14) to (25);
	\end{pgfonlayer}
\end{tikzpicture}
=
\begin{tikzpicture}
	\begin{pgfonlayer}{nodelayer}
		\node [style=Z] (96) at (21.5, -0.25) {};
		\node [style=none] (97) at (21.5, 0.75) {};
		\node [style=none] (98) at (21.5, -0.75) {};
		\node [style=X] (99) at (20.5, -0.25) {};
		\node [style=none] (100) at (20.5, 0.75) {};
		\node [style=none] (101) at (20.5, -0.75) {};
		\node [style=Z] (102) at (21, 0.75) {};
		\node [style=none] (103) at (20.5, 1.75) {};
		\node [style=none] (104) at (21.5, 1.75) {};
		\node [style=none] (105) at (20.5, -1.75) {};
		\node [style=none] (106) at (21.5, -1.75) {};
		\node [style=s] (107) at (21.25, 0.25) {};
	\end{pgfonlayer}
	\begin{pgfonlayer}{edgelayer}
		\draw (98.center) to (96);
		\draw (96) to (97.center);
		\draw (101.center) to (99);
		\draw (99) to (100.center);
		\draw [in=-135, out=60] (99) to (102);
		\draw [in=270, out=90] (100.center) to (104.center);
		\draw [in=270, out=90] (97.center) to (103.center);
		\draw [in=270, out=90] (105.center) to (98.center);
		\draw [in=270, out=90] (106.center) to (101.center);
		\draw [in=-90, out=120] (96) to (107);
		\draw [in=-45, out=90] (107) to (102);
	\end{pgfonlayer}
\end{tikzpicture}
=
\begin{tikzpicture}
	\begin{pgfonlayer}{nodelayer}
		\node [style=Z] (0) at (2.5, 0) {};
		\node [style=none] (1) at (2.5, 0.75) {};
		\node [style=none] (2) at (2.5, -0.75) {};
		\node [style=X] (3) at (1.5, 0) {};
		\node [style=none] (5) at (1.5, 0.75) {};
		\node [style=none] (6) at (1.5, -0.75) {};
		\node [style=Z] (7) at (2, 0.5) {};
		\node [style=none] (8) at (1.5, 1.75) {};
		\node [style=none] (9) at (2.5, 1.75) {};
		\node [style=none] (10) at (1.5, -1.75) {};
		\node [style=none] (11) at (2.5, -1.75) {};
		\node [style=s] (12) at (2.5, -0.75) {};
		\node [style=s] (13) at (2.5, 0.75) {};
	\end{pgfonlayer}
	\begin{pgfonlayer}{edgelayer}
		\draw (2.center) to (0);
		\draw (0) to (1.center);
		\draw (6.center) to (3);
		\draw (3) to (5.center);
		\draw (3) to (7);
		\draw (0) to (7);
		\draw [in=270, out=90] (5.center) to (9.center);
		\draw [in=270, out=90] (1.center) to (8.center);
		\draw [in=270, out=90] (10.center) to (2.center);
		\draw [in=270, out=90] (11.center) to (6.center);
	\end{pgfonlayer}
\end{tikzpicture}
$$
%
%And similarly, we also have:
%
%
%$$
%\begin{tikzpicture}
%	\begin{pgfonlayer}{nodelayer}
%		\node [style=Z] (0) at (0.5, 0) {};
%		\node [style=none] (1) at (0.5, 1) {};
%		\node [style=none] (2) at (0.5, -1) {};
%		\node [style=X] (3) at (1.5, 0) {};
%		\node [style=Z] (4) at (1, 0.5) {};
%		\node [style=none] (5) at (1.5, 1) {};
%		\node [style=none] (6) at (1.5, -1) {};
%	\end{pgfonlayer}
%	\begin{pgfonlayer}{edgelayer}
%		\draw (2.center) to (0);
%		\draw (0) to (1.center);
%		\draw (6.center) to (3);
%		\draw (3) to (5.center);
%		\draw (3) to (4);
%		\draw (4) to (0);
%	\end{pgfonlayer}
%\end{tikzpicture}
%=
%\begin{tikzpicture}
%	\begin{pgfonlayer}{nodelayer}
%		\node [style=Z] (0) at (2.5, 0) {};
%		\node [style=none] (1) at (2.5, 0.75) {};
%		\node [style=none] (2) at (2.5, -0.75) {};
%		\node [style=X] (3) at (1.5, 0) {};
%		\node [style=none] (5) at (1.5, 0.75) {};
%		\node [style=none] (6) at (1.5, -0.75) {};
%		\node [style=X] (7) at (2, 0.5) {};
%		\node [style=none] (8) at (1.5, 1.75) {};
%		\node [style=none] (9) at (2.5, 1.75) {};
%		\node [style=none] (10) at (1.5, -1.75) {};
%		\node [style=none] (11) at (2.5, -1.75) {};
%		\node [style=s] (12) at (2.5, -0.75) {};
%		\node [style=s] (13) at (2.5, 0.75) {};
%	\end{pgfonlayer}
%	\begin{pgfonlayer}{edgelayer}
%		\draw (2.center) to (0);
%		\draw (0) to (1.center);
%		\draw (6.center) to (3);
%		\draw (3) to (5.center);
%		\draw (3) to (7);
%		\draw (0) to (7);
%		\draw [in=270, out=90] (5.center) to (9.center);
%		\draw [in=270, out=90] (1.center) to (8.center);
%		\draw [in=270, out=90] (10.center) to (2.center);
%		\draw [in=270, out=90] (11.center) to (6.center);
%	\end{pgfonlayer}
%\end{tikzpicture}
%$$
%
%
%\begin{theorem}
%These two extra generators, along with the image of the $L$ generate $\Lag\Rel(\F_p)$, for any prime $p$.
%\end{theorem}
%
%In other words, $\Lag\Rel(\F_p)$ arises as ${\sf CPM}(\Rel(\F_p), \perp,\{\Zdot, \Xdot\})$. The category of completely positive maps with respect to the conjugation functor $\perp$ and the two compact structures of linear relations induced by (co)addition and (co)copying.
%
%It is useful to realise, that by Euler decomposition, the Hadamard gate is derived from these generators:
%
%$$
%\begin{tikzpicture}[xscale=-1]
%	\begin{pgfonlayer}{nodelayer}
%		\node [style=none] (0) at (0, 0) {};
%		\node [style=none] (1) at (0.75, 0) {};
%		\node [style=s] (2) at (0.75, -0.5) {};
%		\node [style=none] (3) at (0.75, -1.5) {};
%		\node [style=none] (4) at (0, -1.5) {};
%		\node [style=none] (5) at (0, -0.5) {};
%	\end{pgfonlayer}
%	\begin{pgfonlayer}{edgelayer}
%		\draw [in=270, out=90] (3.center) to (5.center);
%		\draw (5.center) to (0.center);
%		\draw (1.center) to (2);
%		\draw [in=90, out=-90] (2) to (4.center);
%	\end{pgfonlayer}
%\end{tikzpicture}
%$$
%
%And interpreted as a matrix, this is the symplectic form.
%
%
%
%\begin{definition}
%Let $\Aff\Lag\Rel(k)$, denote the affinization of Lagrangian relations.  That is to say, the following pushout of props:
%
%$$
%\Lag\Rel(k) \xleftarrow{S} \Rel(\Mat(k)) \xrightarrow{S} \Lag\Rel(k)  \xrightarrow{G} \Rel(\Mat(k)) \xrightarrow{F} \Rel(\Aff\Mat(k))
%$$
%
%DRAW DISGUSTING PUSHOUT CUBE
%
%
%Where $ G:\Lag\Rel(k)  \rightarrow \Rel(\Mat(k))$ is the forgetful functor and $F: \Rel(\Mat(k))\to  \Rel(\Aff\Mat(k))$ is the free functor.
%\end{definition}
%
%
%This is just a formal way of saying that we are adding an affine shift for each of the gradations of Lagrangian relations.
%
%Notice that this defines another functor $\Aff\Mat(k) \to \Aff\Lag\Rel(k)$ extending the $L$.



%\newpage
%
%
%SYMPLECTOMORPHISMS
%  GENERATED BY FOURIER, PHASE SHIFT, CNOT
%
%
%DEFINE TABLEAU FOR LAGRANGIAN RELATION
%
%
%Just as relations can be represented by matrices over the Boolean semiring, linear relations over $k$ can be represented by matrices over $\F_p$.
%For the sake of simplicity, let us only consider linear relations which are states, as processes can be obtained from currying.
%
%
%Recall that $k$-linear relations $0\to n$ are in one-to-one correspondance with linear subspaces of $k^n$.
%In particular, a linear subspace generated by $m$ column vectors, $v_1,\cdots, v_m$, in $k^n$ is represented by an $n\times m$ matrix.
%
%Recall that a Lagrangian relation $0 \to k^{2n}$ can be characterized as an isotropic subspace of dimension $n$,
%This can be recast in terms of this matrix representation.  In particular $n$-dimensional Lagrangian relations are in one-to-one correspondance with $n\times 2n$ dimensional matrix $M$ with rank $n$  when all rows are orthogonal with respect to the symplectic form.  Take $v_i = z_i x_i$, then for all $1 \leq i,j \leq n$,
%then this symplectic orthogonality is restated as $0=\omega(v_i,v_j) = \langle x_i, z_j \rangle - \langle z_i, x_j \rangle $ which holds if and only if $\langle x_i, z_j \rangle = \langle z_i, x_j \rangle $.
%Similarly, the dimension $n$ condition is equivalent to asking that $\langle v_i,v_j \rangle = 0$ for all $i,j$.


\section{Generators for Lagrangian relations}
\label{sec:univ}

%
%Linear subspaces can be represented in terms of the row space of a matrix.
%In particular, an $n$-dimensional Lagrangian subspace is represented by a block diagonal matrix of the form $[X|Z]$ for $X,Z$ both $n\times n$ matrices.
%This correspondance is made one to one, when the symplectic form is required to vanish, so that $[X|Z] w [X|Z]^n =0$ as well as  $[X|Z]$  having dimension $n$.

In this section, we shall give a universal set of generators for $\Lag\Rel_k$; although, we do not directly give a complete set of identities.  Instead we defer to the completeness of the underlying category $\ih_k\cong\LinRel_k$.


Consider the following symplectomorphisms; the discrete Fourier transform $F$,  the $a$-shift gate $S_a$ and the controlled-$a$ gate $C_a$:
$$
\left\llbracket
\begin{tikzpicture}
	\begin{pgfonlayer}{nodelayer}
		\node [style=none] (0) at (0.5, 1) {};
		\node [style=none] (1) at (0.5, -0.25) {};
		\node [style=none] (2) at (1, -0.25) {};
		\node [style=none] (3) at (1, 1) {};
		\node [style=s] (4) at (1, 0.5) {};
		\node [style=none] (5) at (0.5, 0.5) {};
	\end{pgfonlayer}
	\begin{pgfonlayer}{edgelayer}
		\draw (4) to (3.center);
		\draw [in=90, out=-90] (4) to (1.center);
		\draw [in=-90, out=90] (2.center) to (5.center);
		\draw (5.center) to (0.center);
	\end{pgfonlayer}
\end{tikzpicture}
\right\rrbracket
=
\begin{bmatrix}
0   & 1 \\
-1  & 0
\end{bmatrix}
\hspace*{.2cm}
\left\llbracket
\begin{tikzpicture}
	\begin{pgfonlayer}{nodelayer}
		\node [style=X] (0) at (0.5, 1.25) {};
		\node [style=Z] (1) at (1.5, -0.25) {};
		\node [style=scalar] (2) at (1, 0.5) {$a$};
		\node [style=none] (3) at (0.5, 1.75) {};
		\node [style=none] (4) at (1.5, 1.75) {};
		\node [style=none] (5) at (1.5, -0.75) {};
		\node [style=none] (6) at (0.5, -0.75) {};
	\end{pgfonlayer}
	\begin{pgfonlayer}{edgelayer}
		\draw (5.center) to (1);
		\draw (1) to (4.center);
		\draw [in=-90, out=135] (1) to (2);
		\draw [in=-45, out=90] (2) to (0);
		\draw (3.center) to (0);
		\draw (0) to (6.center);
	\end{pgfonlayer}
\end{tikzpicture}
\right\rrbracket
=
\begin{bmatrix}
1 &a\\
0 & 1
\end{bmatrix}
\hspace*{.2cm}
\left\llbracket
\begin{tikzpicture}
	\begin{pgfonlayer}{nodelayer}
		\node [style=Z] (430) at (219.75, 0) {};
		\node [style=X] (431) at (220.75, 1.5) {};
		\node [style=none] (432) at (219.75, 0) {};
		\node [style=none] (433) at (221.25, -0.75) {};
		\node [style=none] (434) at (219.25, 2.25) {};
		\node [style=none] (435) at (220.75, 2.25) {};
		\node [style=scalar] (436) at (220.25, 0.75) {$a$};
		\node [style=X] (437) at (217.5, 0) {};
		\node [style=Z] (438) at (218.5, 1.5) {};
		\node [style=none] (439) at (217.5, -0.75) {};
		\node [style=none] (440) at (219, -0.75) {};
		\node [style=none] (441) at (217.25, 2.25) {};
		\node [style=none] (442) at (218.5, 1.5) {};
		\node [style=scalarop] (443) at (218, 0.75) {$a$};
		\node [style=none] (445) at (219.75, -0.75) {};
		\node [style=none] (446) at (218.5, 2.25) {};
	\end{pgfonlayer}
	\begin{pgfonlayer}{edgelayer}
		\draw [in=-105, out=30] (430) to (436);
		\draw [in=-150, out=90] (436) to (431);
		\draw [in=90, out=-60] (431) to (433.center);
		\draw (431) to (435.center);
		\draw [in=135, out=-90, looseness=0.75] (434.center) to (430);
		\draw (439.center) to (437);
		\draw [in=-105, out=30] (437) to (443);
		\draw [in=-150, out=90] (443) to (438);
		\draw [in=90, out=-45, looseness=0.75] (438) to (440.center);
		\draw [in=120, out=-90] (441.center) to (437);
		\draw [in=270, out=90] (442.center) to (446.center);
		\draw [in=270, out=90] (445.center) to (432.center);
	\end{pgfonlayer}
\end{tikzpicture}
\right\rrbracket
=
\begin{bmatrix}
1 & -a & 0 & 0 \\
0 & 1 & 0 & 0 \\
0 & 0 & 1 & 0 \\
0 & 0 & a & 1
\end{bmatrix}
%\begin{bmatrix}
%1 & -a & 0 & 0 \\
%0 & 1 & 0 & 0 \\
%0 & 0 & 1 & 0 \\
%0 & 0 & a & 0
%\end{bmatrix}
$$


%When composed with identities, these have having the following action on Lagrangian subspaces when applied to certain wires:

Use the notation $G^{(j)}$ to denote a gate $G$ being applied to wire $j$; and the notation $C_a^{(i,j)}$ to denote the controlled-$a$ gate controlling on wire $i$ targetting wire $j$.

Note the right action of these gates in terms of matrix multiplication of Lagrangian subspaces for any nonzero $a \in k$ (as observed in \cite[p. 4]{aaronson}):

\begin{itemize}
\item
$F^{(i)}$ sets columns $x_i$ to $-z_i$ and $z_i$ to $x_i$.

\item
$S_a^{(i)}$ sets $z_i$ to $z_i+a\cdot x_i$.

%\item
%$M_a^{(i)}$ sets $x_i$ to $a\cdot x_i$ and $z_i$ to $a^{-1}\cdot x_i$.

\item
$C_a^{(i,j)}$ sets $x_j$ to $x_j- a \cdot x_i$ and $z_i$ to $z_i+a\cdot z_j$.

\end{itemize}

Using these symplectomorphisms regarded as Lagrangian relations, we have:


%
%
%
%\begin{align*}
%&\left[
%\begin{array}{*{7}c|*{7}c}
%x_{1,1} & \cdots & x_{1,a} & \cdots & x_{1,b} & \cdots & x_{1,n} &  z_{1,1} & \cdots & z_{1,a} & \cdots & z_{1,b} & \cdots & z_{1,n} \\-
%\vdots   & \ddots  & \vdots  & \ddots  & \vdots   &  \ddots & \vdots   & \vdots    & \ddots  & \vdots  & \ddots  & \vdots  & \ddots  & \vdots   \\
%x_{n,1} & \cdots & x_{n,a} & \cdots & x_{n,b} & \cdots & x_{n,n} &  z_{n,1} & \cdots & z_{n,a} & \cdots & z_{n,b} & \cdots & z_{n,n}
%\end{array}
%\right]\\
%&\mapsto\\
%&\left[
%\begin{array}{*{7}c|*{7}c}
%x_{1,1} & \cdots & x_{1,a} & \cdots & x_{1,b} - kx_{1,a} & \cdots & x_{1,n} &  z_{1,1} & \cdots & z_{1,a} +k z_{1,b} & \cdots & z_{1,b} & \cdots & z_{1,n} \\
%\vdots   & \ddots  & \vdots  & \ddots  & \vdots                    &  \ddots & \vdots   & \vdots    & \ddots  & \vdots                  & \ddots  & \vdots  & \ddots  & \vdots   \\
%x_{n,1} & \cdots & x_{n,a} & \cdots & x_{n,b} -kx_{n,a} & \cdots & x_{n,n} &  z_{n,1} & \cdots & z_{n,a} +k  z_{n,b}& \cdots & z_{n,b} & \cdots & z_{n,n}
%\end{array}
%\right]
%\end{align*}
%
%
%
%\begin{align*}
%&\left[
%\begin{array}{*{5}c|*{5}c}
%x_{1,1} & \cdots & x_{1,a}  & \cdots & x_{1,n} &  z_{1,1} & \cdots & z_{1,a} & \cdots  & z_{1,n} \\
%\vdots   & \ddots  & \vdots   &  \ddots & \vdots   & \vdots    & \ddots  & \vdots  & \ddots   & \vdots   \\  
%x_{n,1} & \cdots & x_{n,a} & \cdots & x_{n,n} &  z_{n,1} & \cdots & z_{n,a} & \cdots   & z_{n,n}
%\end{array}
%\right]\\
%&\mapsto\\
%&\left[
%\begin{array}{*{7}c|*{7}c}
%x_{1,1} & \cdots & x_{1,a-1} & -z_{1,a} &  x_{1,a+1}   & \cdots & x_{1,n} &  z_{1,1} & \cdots & z_{1,a-1} & x_{1,a} &  z_{1,a+1} & \cdots  & z_{1,n} \\
%\vdots   & \ddots  & \vdots      & \vdots   & \vdots           &  \ddots & \vdots   & \vdots    & \ddots  & \vdots      & \vdots    & \vdots        & \ddots   & \vdots   \\  
%x_{n,1} & \cdots & x_{n,a-1} & -z_{n,a} &  x_{n,a+1} & \cdots & x_{n,n} &  z_{n,1} & \cdots &  z_{n,a-1} & x_{n,a} &  z_{n,a+1} & \cdots   & z_{n,n}
%\end{array}
%\right]\\
%\end{align*}
%
%
%



\begin{theorem}
\label{theorem:generators}
For any field $k$ the maps in $L(\LinRel_k)$ as well as $F$ and $S_a$ for all $a \in k$ generate $\Lag\Rel_k$.
\end{theorem}

\begin{proof}
Consider the matrix $[X|Z]$ of an arbitrary Lagrangian relation over field $k$ seen as a state.
%
%$$
%[X | Z]
%=
%\left[
%\begin{array}{ccc|ccc}
%x_{1,1} & \cdots & x_{n,n} & z_{1,1} & \cdots & z_{1,n}\\
%\vdots    & \ddots & \vdots    & \vdots   & \ddots & \vdots \\
%x_{n,1} & \cdots & x_{n,n} & z_{n,1} & \cdots & z_{n,n}\\
%\end{array}
%\right]
%$$
We show how one can reduce $[X|Z]$ to the block matrix $[I|0]$ by right multiplication with the aforementioned symplectomorphisms.
To do so, we first reduce it to a matrix $[I|Z']$
by only applying row operations (keeping the subspace the same) as well as the Fourier transform.
This involve repeatedly do Gaussian elimination and then applying the Fourier transform to wires when the pivot is in the $Z$ block.
We are guaranteed to obtain a matrix $[I|Z']$ because the dimension of Lagrangian subspace is $n$.
A very similar observation was made in \cite[Lem. 6]{aaronson}.

$$
\begin{tikzpicture}
	\begin{pgfonlayer}{nodelayer}
		\node [style=Z] (309) at (26.5, 8.75) {};
		\node [style=none] (310) at (26.5, 8) {};
		\node [style=none] (311) at (25.75, 10.25) {};
		\node [style=none] (312) at (27, 9.5) {};
		\node [style=none] (313) at (30.75, 8.25) {};
		\node [style=Z] (314) at (30.75, 9) {};
		\node [style=none] (315) at (30.5, 10) {};
		\node [style=none] (316) at (31.5, 9.5) {};
		\node [style=none] (317) at (32, 8.25) {};
		\node [style=Z] (318) at (28.5, 8.75) {};
		\node [style=none] (319) at (28.5, 8) {};
		\node [style=none] (320) at (28, 9.5) {};
		\node [style=none] (321) at (29.25, 10.25) {};
		\node [style=none] (322) at (27.5, 10) {};
		\node [style=none] (323) at (33.25, 8.25) {};
		\node [style=Z] (324) at (33.25, 9) {};
		\node [style=none] (325) at (32.75, 10) {};
		\node [style=none] (326) at (33.75, 10) {};
		\node [style=none] (327) at (32.5, 9) {,};
		\node [style=none] (328) at (29.75, 9) {,};
		\node [style=none] (329) at (27.5, 11) {(1)};
		\node [style=none] (330) at (31.25, 11) {(2)};
		\node [style=none] (331) at (33.25, 11) {(3)};
	\end{pgfonlayer}
	\begin{pgfonlayer}{edgelayer}
		\draw (310.center) to (309);
		\draw [in=-90, out=150, looseness=0.75] (309) to (311.center);
		\draw [in=30, out=-90] (312.center) to (309);
		\draw [in=135, out=-90] (315.center) to (314);
		\draw (313.center) to (314);
		\draw [in=45, out=180, looseness=0.75] (316.center) to (314);
		\draw [in=0, out=90, looseness=0.75] (317.center) to (316.center);
		\draw (319.center) to (318);
		\draw [in=-90, out=150] (318) to (320.center);
		\draw [in=30, out=-90] (321.center) to (318);
		\draw [in=0, out=90, looseness=0.75] (320.center) to (322.center);
		\draw [in=90, out=180, looseness=0.75] (322.center) to (312.center);
		\draw [in=150, out=-90] (325.center) to (324);
		\draw (323.center) to (324);
		\draw [in=30, out=-90] (326.center) to (324);
	\end{pgfonlayer}
\end{tikzpicture}
$$

As the Fourier transform is a symplectomorphism $[I|Z']$ is isotropic, so that:

$$
0
=
\begin{bmatrix}
I | Z'
\end{bmatrix}
\omega
\begin{bmatrix}
I | Z'
\end{bmatrix}^T
$$
which holds if and only if 
$$
0=
\begin{bmatrix}
I | Z'
\end{bmatrix}
\begin{bmatrix}
Z' | -I 
\end{bmatrix}^T
=
{Z'}^T-Z'
$$

That is to say $Z'$ is symmetric, meaning that $Z'$ describes the adjacency matrix of a graph coloured by the elements of $k$.
In the language of stabilizer circuits, this is called a {\em graph state}.  In the case of prime fields, this observation was made in \cite[Eq. 18]{gross}.  Graph states were originally discussed in \cite{hein2006entanglement}.


%
%The next part of the algorithm is reducing the second block of the matrix to 0. In the case of $\F_2$, this is reduced to applying controlled-Z gates to the wires which contain an edge in the adjacency matrix of the graph of $Z'$.
%In the case of a general field, we must take a slightly more nuanced approach.
%
%
%
%
%% Input: Matrix representation (X|Z) of Lagrangian subspace
%% Row reduce (X|Z)
%% For all rows i, if the pivot is in the Z block, apply the Fourier transform to row i
%% Because Lagrangian subspaces of k^{2n} have dimension n, this matrix has full rank, 
%%   Therefore, row reduce further until  the X block is the identity
%%Now the Z block is a symmetric matrix
%% For all i from 1 to n
%%   For all j from 1 to i
%%     Apply Fourier transform to row j
%%     Apply C_{z_{i,j}} gate from row i to j
%%    Apply inverse fourier transform to row j


We prove that graph states can be reduced to the subspace $[I|0]$ by induction on the dimension of the subspace.
This base case is trivial.

Suppose we have a $(n+1)$-dimensional Lagrangian subspaces described by a graph state, then:
\newpage

\resizebox{\linewidth}{!}{
  \begin{minipage}{\linewidth}
\begin{align*}
&\left[
\begin{array}{*{5}c|*{6}c}
1         & 0        & 0         & \cdots & 0         & z_{1,1} & z_{1,2} & z_{1,3} & \cdots & z_{1,n}\\
0         & 1        & 0         & \cdots & 0         & z_{1,2} & z_{2,2} & z_{2,3} & \cdots & z_{2,n}\\
0         & 0        & 1         & \ddots & \vdots & z_{1,3} & z_{2,3} & z_{3,3} & \cdots & z_{3,n}\\
\vdots & \vdots & \ddots & \ddots & 0         & \vdots   & \vdots    & \vdots    & \ddots &  \vdots \\
0         & 0         & \cdots & 0        & 1          & z_{1,n} & z_{2,n} & z_{3,n} & \cdots & z_{n,n}\\
\end{array}
\right]
\xmapsto{(F^{(1)})^{-1}}
\left[
\begin{array}{*{5}c|*{6}c}
z_{1,1}         & 0        & 0         & \cdots & 0         & -1 & z_{1,2} & z_{1,3} & \cdots & z_{1,n}\\
z_{1,2}         & 1        & 0         & \cdots & 0         & 0 & z_{2,2} & z_{2,3} & \cdots & z_{2,n}\\
z_{1,3}         & 0        & 1         & \ddots & \vdots & 0 & z_{2,3} & z_{3,3} & \cdots & z_{3,n}\\
\vdots & \vdots & \ddots & \ddots & 0         & \vdots   & \vdots    & \vdots    & \ddots &  \vdots \\
z_{1,n}         & 0         & \cdots & 0        & 1          & 0 & z_{2,n} & z_{3,n} & \cdots & z_{n,n}\\
\end{array}
\right]\\
&\xmapsto{C_{z_{1,2}}^{(2,1)}}\\
&\left[
\begin{array}{*{5}c|*{6}c}
z_{1,1}-0           & 0         & 0         & \cdots & 0         & -1 & z_{1,2}-z_{1,2} & z_{1,3} & \cdots & z_{1,n}\\
z_{1,2}-z_{1,2} & 1         & 0         & \cdots & 0         & 0 & z_{2,2}-0            & z_{2,3} & \cdots & z_{2,n}\\
z_{1,3}-0           & 0         & 1         & \ddots & \vdots & 0 & z_{2,3}-0            & z_{3,3} & \cdots & z_{3,n}\\
\vdots                 & \vdots & \ddots & \ddots & 0         & \vdots   & \vdots        & \vdots    & \ddots &  \vdots \\
z_{1,n}-0           & 0         & \cdots & 0        & 1          & 0 & z_{2,n}-0            & z_{3,n} & \cdots & z_{n,n}\\
\end{array}
\right]
=
\left[
\begin{array}{*{5}c|*{6}c}
z_{1,1}             & 0         & 0         & \cdots & 0         & -1 & 0                         & z_{1,3} & \cdots & z_{1,n}\\
0                       & 1         & 0         & \cdots & 0         & 0 & z_{2,2}                & z_{2,3} & \cdots & z_{2,n}\\
z_{1,3}             & 0         & 1         & \ddots & \vdots & 0 & z_{2,3}                & z_{3,3} & \cdots & z_{3,n}\\
\vdots                & \vdots & \ddots & \ddots & 0         & \vdots   & \vdots        & \vdots    & \ddots &  \vdots \\
z_{1,n}             & 0         & \cdots & 0        & 1          & 0 & z_{2,n}                & z_{3,n} & \cdots & z_{n,n}\\
\end{array}
\right]\\
&\xmapsto{\prod_{i>1}^n C_{z_{1,i}}^{(i,1)}}\\
&\left[
\begin{array}{*{5}c|*{6}c}
z_{1,1}             & 0         & 0         & \cdots & 0         & -1 & 0                         & 0           & \cdots & 0\\
0                       & 1         & 0         & \cdots & 0         & 0 & z_{2,2}                & z_{2,3} & \cdots & z_{2,n}\\
0                       & 0         & 1         & \ddots & \vdots & 0 & z_{2,3}                & z_{3,3} & \cdots & z_{3,n}\\
\vdots               & \vdots & \ddots & \ddots & 0         & \vdots   & \vdots        & \vdots    & \ddots &  \vdots \\
0                       & 0         & \cdots & 0        & 1          & 0 & z_{2,n}                & z_{3,n} & \cdots & z_{n,n}\\
\end{array}
\right]
\xmapsto{F^{(1)}}
\left[
\begin{array}{*{5}c|*{6}c}
1                       & 0         & 0         & \cdots & 0         & z_{1,1} & 0                          & 0           & \cdots & 0\\
0                       & 1         & 0         & \cdots & 0         & 0           & z_{2,2}                & z_{2,3} & \cdots & z_{2,n}\\
0                       & 0         & 1         & \ddots & \vdots & 0           & z_{2,3}                & z_{3,3} & \cdots & z_{3,n}\\
\vdots               & \vdots & \ddots & \ddots & 0         & \vdots   & \vdots                   & \vdots    & \ddots &  \vdots \\
0                       & 0         & \cdots & 0        & 1          & 0           & z_{2,n}                & z_{3,n} & \cdots & z_{n,n}\\
\end{array}
\right]\\
&\xmapsto{S_{-z_{1,1}}^{(1)}  }\\
&\left[
\begin{array}{*{5}c|*{6}c}
1                       & 0         & 0         & \cdots & 0         & 0 & 0                          & 0           & \cdots & 0\\
0                       & 1         & 0         & \cdots & 0         & 0           & z_{2,2}                & z_{2,3} & \cdots & z_{2,n}\\
0                       & 0         & 1         & \ddots & \vdots & 0           & z_{2,3}                & z_{3,3} & \cdots & z_{3,n}\\
\vdots               & \vdots & \ddots & \ddots & 0         & \vdots   & \vdots                   & \vdots    & \ddots &  \vdots \\
0                       & 0         & \cdots & 0        & 1          & 0           & z_{2,n}                & z_{3,n} & \cdots & z_{n,n}\\
\end{array}
\right]\\
\end{align*}
  \end{minipage}
}


Therefore all Lagrangian relations can be reduced to the subspace $[I|0]$ by right multiplication by symplectomorphisms.
In the $n$-dimensional case, this subspace is given by the circuit
$L(
\begin{tikzpicture}[scale=.5]
	\begin{pgfonlayer}{nodelayer}
		\node [style=X] (0) at (6, 3) {};
		\node [style=none] (1) at (6, 3.5) {};
	\end{pgfonlayer}
	\begin{pgfonlayer}{edgelayer}
		\draw (0) to (1.center);
	\end{pgfonlayer}
\end{tikzpicture}
^{\otimes n}$).




Thus, we have already described all the generators of Lagrangian relations. The gates $F$, $C_a$ for all $a \in k$, along with the cup and cap and the zero state generate all Lagrangian relations.  Note that $C_a$ and the zero state are both in the image of the $L$.

\end{proof} 


%
%This is proved by using these symplectomorphisms to reduce a Lagrangian relation by Guassian elimination to the state
%$L(
%\begin{tikzpicture}[scale=.5]
%	\begin{pgfonlayer}{nodelayer}
%		\node [style=X] (0) at (6, 3) {};
%		\node [style=none] (1) at (6, 3.5) {};
%	\end{pgfonlayer}
%	\begin{pgfonlayer}{edgelayer}
%		\draw (0) to (1.center);
%	\end{pgfonlayer}
%\end{tikzpicture}
%^{\otimes n}$).
%

We can also give a presentation of this category which is very close to Selinger's CPM construction~\cite{cpm}. There are several equivalent ways to define the CPM construction. For our purposes, the most convenient one is the presentation used in both in~\cite{pqp,cqm1}, which defines $\CPM[\X]$ as the subcategory of a dagger compact closed category $\X$ whose objects are of the form $A^* \otimes A$ for $A \in \X$ and whose morphisms are generated by (i) `pure' morphisms, i.e. morphisms of the form $f_* \otimes f$ for $f \in \X$ and a covariant functor $(\_)_*$, and (ii) a `discard' morphism $d_A$ for every $A \in \X$ given by the counit $d_A := \epsilon_A : A^* \otimes A \to I$ of the compact closed structure on $A$.

We nearly obtain such a presentation for $\Lag\Rel_k$ using the covariant functor $(-)^\perp$ to define pure morphisms, with the only caveat being we need to consider a family of discard morphisms: each discard morphism being parametrised by a field element.

\begin{theorem}
\label{theorem:unbiased}
$\Lag\Rel_k$ is the monoidal subcategory of $\LinRel_k$ whose objects are of the form $k^n \oplus k^n$, for all natural numbers $n$, and whose morphisms are generated by \textit{pure} morphisms of the form $V^\perp \oplus V$ for $V \in \LinRel_k$ and for each $a \in k$, a `discard' morphism:
$$
d_a :=
\begin{tikzpicture}
	\begin{pgfonlayer}{nodelayer}
		\node [style=X] (0) at (0, 0.75) {};
		\node [style=scalar] (1) at (0.5, 0) {$a$};
		\node [style=none] (2) at (0.5, -0.75) {};
		\node [style=none] (3) at (-0.5, 0) {};
		\node [style=none] (4) at (-0.5, -0.75) {};
	\end{pgfonlayer}
	\begin{pgfonlayer}{edgelayer}
		\draw [in=-30, out=90] (1) to (0);
		\draw [in=90, out=-150] (0) to (3.center);
		\draw (4.center) to (3.center);
		\draw (2.center) to (1);
	\end{pgfonlayer}
\end{tikzpicture}
$$
\end{theorem}


\begin{proof}
  We just have to show that $F$ and $S_a$ can be constructed using these generators. The $S_a$ gate and it's colour-reversed version $V_a$ can be obtained by composing a pure morphism with $d_a$ and $d_{-a}$, respectively:
$$
\begin{tikzpicture}
	\begin{pgfonlayer}{nodelayer}
		\node [style=none] (0) at (1, -1.25) {};
		\node [style=Z] (1) at (1, -0.75) {};
		\node [style=X] (2) at (-0.75, -0.75) {};
		\node [style=X] (3) at (-0.375, 0.75) {};
		\node [style=none] (4) at (-0.75, -1.25) {};
		\node [style=none] (5) at (-0.75, 1.25) {};
		\node [style=none] (6) at (1, 1.25) {};
		\node [style=scalar] (7) at (0.5, 0) {$a$};
		\node [style=none] (8) at (-1.25, 0) {};
		\node [style=none] (9) at (1.75, 0) {$=$};
		\node [style=none] (10) at (3.5, -1) {};
		\node [style=Z] (11) at (3.5, -0.5) {};
		\node [style=X] (12) at (2.5, 0.5) {};
		\node [style=none] (13) at (2.5, -1) {};
		\node [style=scalar] (14) at (3, 0) {$a$};
		\node [style=none] (15) at (2.5, 1) {};
		\node [style=none] (16) at (3.5, 1) {};
	\end{pgfonlayer}
	\begin{pgfonlayer}{edgelayer}
		\draw (1) to (0.center);
		\draw (1) to (6.center);
		\draw (2) to (5.center);
		\draw (2) to (4.center);
		\draw [in=-90, out=165] (1) to (7);
		\draw [in=0, out=90] (7) to (3);
		\draw [in=-90, out=150] (2) to (8.center);
		\draw [in=-180, out=90] (8.center) to (3);
		\draw (11) to (10.center);
		\draw [in=-90, out=165] (11) to (14);
		\draw [in=-15, out=90] (14) to (12);
		\draw (13.center) to (12);
		\draw (11) to (16.center);
		\draw (12) to (15.center);
	\end{pgfonlayer}
\end{tikzpicture}
\ = S_a
\qquad\qquad
\begin{tikzpicture}
	\begin{pgfonlayer}{nodelayer}
		\node [style=none] (0) at (1, -1.25) {};
		\node [style=X] (1) at (1, -0.75) {};
		\node [style=Z] (2) at (-0.75, -0.75) {};
		\node [style=Z] (3) at (-0.375, 0.75) {};
		\node [style=none] (4) at (-0.75, -1.25) {};
		\node [style=none] (5) at (-0.75, 1.25) {};
		\node [style=none] (6) at (1, 1.25) {};
		\node [style=scalar] (7) at (0.5, 0) {$a$};
		\node [style=none] (8) at (-1.25, 0) {};
		\node [style=none] (9) at (1.75, 0) {$=$};
		\node [style=none] (10) at (3.5, -1) {};
		\node [style=X] (11) at (3.5, -0.5) {};
		\node [style=Z] (12) at (2.5, 0.5) {};
		\node [style=none] (13) at (2.5, -1) {};
		\node [style=scalar] (14) at (3, 0) {$a$};
		\node [style=none] (15) at (2.5, 1) {};
		\node [style=none] (16) at (3.5, 1) {};
		\node [style=none] (17) at (-2.75, -1.25) {};
		\node [style=X] (18) at (-2.75, -0.75) {};
		\node [style=Z] (19) at (-4.5, -0.75) {};
		\node [style=X] (20) at (-4.125, 0.75) {};
		\node [style=none] (21) at (-4.5, -1.25) {};
		\node [style=none] (22) at (-4.5, 1.25) {};
		\node [style=none] (23) at (-2.75, 1.25) {};
		\node [style=scalar] (24) at (-3.25, 0) {-$a$};
		\node [style=none] (25) at (-5, 0) {};
		\node [style=none] (26) at (-2, 0) {$=$};
	\end{pgfonlayer}
	\begin{pgfonlayer}{edgelayer}
		\draw (1) to (0.center);
		\draw (1) to (6.center);
		\draw (2) to (5.center);
		\draw (2) to (4.center);
		\draw [in=-90, out=165] (1) to (7);
		\draw [in=0, out=90] (7) to (3);
		\draw [in=-90, out=150] (2) to (8.center);
		\draw [in=-180, out=90] (8.center) to (3);
		\draw (11) to (10.center);
		\draw [in=-90, out=165] (11) to (14);
		\draw [in=-15, out=90] (14) to (12);
		\draw (13.center) to (12);
		\draw (11) to (16.center);
		\draw (12) to (15.center);
		\draw (18) to (17.center);
		\draw (18) to (23.center);
		\draw (19) to (22.center);
		\draw (19) to (21.center);
		\draw [in=-90, out=165] (18) to (24);
		\draw [in=0, out=90] (24) to (20);
		\draw [in=-90, out=150] (19) to (25.center);
		\draw [in=-180, out=90] (25.center) to (20);
	\end{pgfonlayer}
\end{tikzpicture}
\  =: V_a
$$

We can then obtain $F$ as $S_1 \circ V_1 \circ S_1$, which can be proven as a variation of the familiar `3 CNOT' rule for quantum circuits (see e.g.~\cite[\S 3.2.1]{coecke2008interacting}):
$$
S_1 \circ V_1 \circ S_1
=
\begin{tikzpicture}[xscale=-1]
	\begin{pgfonlayer}{nodelayer}
		\node [style=Z] (0) at (3, 1.25) {};
		\node [style=X] (1) at (4, 1.75) {};
		\node [style=Z] (2) at (4, 2.5) {};
		\node [style=X] (3) at (3, 2) {};
		\node [style=Z] (4) at (3, 2.75) {};
		\node [style=X] (5) at (4, 3.25) {};
		\node [style=none] (6) at (3, 4.5) {};
		\node [style=none] (7) at (4, 4.5) {};
		\node [style=none] (8) at (3, 0) {};
		\node [style=none] (9) at (4, 0) {};
	\end{pgfonlayer}
	\begin{pgfonlayer}{edgelayer}
		\draw (6.center) to (4);
		\draw (4) to (3);
		\draw (0) to (3);
		\draw (8.center) to (0);
		\draw (9.center) to (1);
		\draw (1) to (2);
		\draw (2) to (5);
		\draw (5) to (7.center);
		\draw (3) to (2);
		\draw (4) to (5);
		\draw (0) to (1);
	\end{pgfonlayer}
\end{tikzpicture}
=
\begin{tikzpicture}[xscale=-1]
	\begin{pgfonlayer}{nodelayer}
		\node [style=Z] (0) at (3, 1.25) {};
		\node [style=X] (1) at (4, 0.75) {};
		\node [style=Z] (2) at (4, 2.5) {};
		\node [style=X] (3) at (3, 2) {};
		\node [style=Z] (4) at (3, 3.75) {};
		\node [style=X] (5) at (4, 3.25) {};
		\node [style=none] (6) at (3, 4.5) {};
		\node [style=none] (7) at (4, 4.5) {};
		\node [style=none] (8) at (3, 0) {};
		\node [style=none] (9) at (4, 0) {};
		\node [style=s] (10) at (3.5, 3.5) {};
		\node [style=s] (11) at (3.5, 1) {};
	\end{pgfonlayer}
	\begin{pgfonlayer}{edgelayer}
		\draw (6.center) to (4);
		\draw (4) to (3);
		\draw (0) to (3);
		\draw (8.center) to (0);
		\draw (9.center) to (1);
		\draw (1) to (2);
		\draw (2) to (5);
		\draw (5) to (7.center);
		\draw (3) to (2);
		\draw (1) to (11);
		\draw (11) to (0);
		\draw (5) to (10);
		\draw (10) to (4);
	\end{pgfonlayer}
\end{tikzpicture}
=
\begin{tikzpicture}[xscale=-1]
	\begin{pgfonlayer}{nodelayer}
		\node [style=X] (28) at (6.5, 0) {};
		\node [style=Z] (31) at (5, 4) {};
		\node [style=none] (33) at (5, 4.5) {};
		\node [style=none] (34) at (6.5, 4.5) {};
		\node [style=none] (35) at (5, -0.5) {};
		\node [style=none] (36) at (6.5, -0.5) {};
		\node [style=s] (37) at (6, 0.5) {};
		\node [style=s] (38) at (5.5, 3.5) {};
		\node [style=X] (39) at (6, 2.25) {};
		\node [style=Z] (40) at (6, 3) {};
		\node [style=X] (41) at (6.5, 2.25) {};
		\node [style=Z] (42) at (6.5, 3) {};
		\node [style=X] (43) at (5, 1) {};
		\node [style=Z] (44) at (5, 1.75) {};
		\node [style=X] (45) at (5.5, 1) {};
		\node [style=Z] (46) at (5.5, 1.75) {};
	\end{pgfonlayer}
	\begin{pgfonlayer}{edgelayer}
		\draw (33.center) to (31);
		\draw (36.center) to (28);
		\draw (28) to (37);
		\draw (31) to (38);
		\draw (40) to (41);
		\draw (41) to (42);
		\draw (40) to (39);
		\draw (39) to (42);
		\draw (44) to (45);
		\draw (45) to (46);
		\draw (44) to (43);
		\draw (43) to (46);
		\draw (34.center) to (42);
		\draw (40) to (38);
		\draw (46) to (39);
		\draw (41) to (28);
		\draw (37) to (45);
		\draw (35.center) to (43);
		\draw (31) to (44);
	\end{pgfonlayer}
\end{tikzpicture}
=
\begin{tikzpicture}[xscale=-1]
	\begin{pgfonlayer}{nodelayer}
		\node [style=X] (67) at (11.5, 0) {};
		\node [style=Z] (68) at (10, 4.25) {};
		\node [style=none] (69) at (10, 4.75) {};
		\node [style=none] (70) at (11.5, 4.75) {};
		\node [style=none] (71) at (10, -0.5) {};
		\node [style=none] (72) at (11.5, -0.5) {};
		\node [style=s] (73) at (11, 0.5) {};
		\node [style=s] (74) at (10.5, 3.75) {};
		\node [style=Z] (75) at (11, 3.25) {};
		\node [style=X] (76) at (11.5, 2.5) {};
		\node [style=Z] (77) at (11.5, 3.25) {};
		\node [style=X] (78) at (10, 1) {};
		\node [style=Z] (79) at (10, 1.75) {};
		\node [style=X] (80) at (10.5, 1) {};
		\node [style=X] (81) at (10.5, 1.75) {};
		\node [style=Z] (82) at (10.5, 2.5) {};
		\node [style=X] (83) at (11, 1.75) {};
		\node [style=Z] (84) at (11, 2.5) {};
	\end{pgfonlayer}
	\begin{pgfonlayer}{edgelayer}
		\draw (69.center) to (68);
		\draw (72.center) to (67);
		\draw (67) to (73);
		\draw (68) to (74);
		\draw (75) to (76);
		\draw (76) to (77);
		\draw (79) to (80);
		\draw (79) to (78);
		\draw (70.center) to (77);
		\draw (75) to (74);
		\draw (76) to (67);
		\draw (73) to (80);
		\draw (71.center) to (78);
		\draw (68) to (79);
		\draw (82) to (83);
		\draw (83) to (84);
		\draw (82) to (81);
		\draw (81) to (84);
		\draw (78) to (81);
		\draw (80) to (83);
		\draw (84) to (77);
		\draw (75) to (82);
	\end{pgfonlayer}
\end{tikzpicture}
=
\begin{tikzpicture}[xscale=-1]
	\begin{pgfonlayer}{nodelayer}
		\node [style=X] (0) at (11, 0.75) {};
		\node [style=Z] (1) at (10.25, 4.75) {};
		\node [style=none] (2) at (10.25, 5.25) {};
		\node [style=none] (3) at (11, 5.25) {};
		\node [style=none] (4) at (10.25, 0.25) {};
		\node [style=none] (5) at (11, 0.25) {};
		\node [style=s] (6) at (11, 1.5) {};
		\node [style=s] (7) at (10.25, 4) {};
		\node [style=Z] (8) at (10.25, 3.25) {};
		\node [style=X] (9) at (11, 0.75) {};
		\node [style=Z] (10) at (11, 3.25) {};
		\node [style=X] (11) at (10.25, 2.25) {};
		\node [style=Z] (12) at (10.25, 4.75) {};
		\node [style=X] (13) at (11, 2.25) {};
		\node [style=X] (14) at (10.25, 2.25) {};
		\node [style=Z] (15) at (10.25, 3.25) {};
		\node [style=X] (16) at (11, 2.25) {};
		\node [style=Z] (17) at (11, 3.25) {};
	\end{pgfonlayer}
	\begin{pgfonlayer}{edgelayer}
		\draw (2.center) to (1);
		\draw (5.center) to (0);
		\draw (0) to (6);
		\draw (1) to (7);
		\draw [in=135, out=-60] (8) to (9);
		\draw [bend right, looseness=1.25] (9) to (10);
		\draw [in=120, out=-45] (12) to (13);
		\draw [bend right, looseness=1.25] (12) to (11);
		\draw (3.center) to (10);
		\draw (8) to (7);
		\draw (6) to (13);
		\draw (4.center) to (11);
		\draw [in=150, out=-30, looseness=0.75] (15) to (16);
		\draw (16) to (17);
		\draw (15) to (14);
		\draw (14) to (17);
	\end{pgfonlayer}
\end{tikzpicture}
=
\begin{tikzpicture}[xscale=-1]
	\begin{pgfonlayer}{nodelayer}
		\node [style=X] (0) at (19.5, 1.25) {};
		\node [style=Z] (1) at (18, 4.25) {};
		\node [style=none] (2) at (18, 4.75) {};
		\node [style=none] (3) at (19.5, 4.75) {};
		\node [style=none] (4) at (18, 0.75) {};
		\node [style=none] (5) at (19.5, 0.75) {};
		\node [style=X] (6) at (19.5, 1.25) {};
		\node [style=Z] (7) at (19.5, 4.25) {};
		\node [style=X] (8) at (18, 1.25) {};
		\node [style=Z] (9) at (18, 4.25) {};
		\node [style=X] (10) at (19.5, 1.25) {};
		\node [style=X] (11) at (18, 1.25) {};
		\node [style=X] (12) at (19.5, 1.25) {};
		\node [style=Z] (13) at (19.5, 4.25) {};
		\node [style=s] (14) at (17.75, 3) {};
		\node [style=s] (15) at (18.75, 3) {};
		\node [style=s] (16) at (19.75, 3) {};
		\node [style=s] (17) at (18.25, 3) {};
	\end{pgfonlayer}
	\begin{pgfonlayer}{edgelayer}
		\draw (2.center) to (1);
		\draw (5.center) to (0);
		\draw [bend right=45] (6) to (7);
		\draw [in=120, out=-135, looseness=1.25] (9) to (8);
		\draw (3.center) to (7);
		\draw (4.center) to (8);
		\draw [in=-120, out=15, looseness=0.75] (11) to (13);
		\draw [in=90, out=-105] (9) to (14);
		\draw [in=90, out=-45, looseness=0.75] (9) to (15);
		\draw [in=90, out=-90] (14) to (11);
		\draw [in=-90, out=120, looseness=0.75] (6) to (15);
		\draw [in=-15, out=90] (12) to (9);
		\draw [in=-90, out=150, looseness=0.75] (12) to (17);
		\draw [in=285, out=90] (17) to (9);
		\draw [in=-90, out=75, looseness=0.75] (12) to (16);
		\draw [in=-75, out=90] (16) to (13);
	\end{pgfonlayer}
\end{tikzpicture}
=
\begin{tikzpicture}
	\begin{pgfonlayer}{nodelayer}
		\node [style=none] (0) at (24, 4.5) {};
		\node [style=none] (1) at (23.5, 4) {};
		\node [style=none] (2) at (24, 2.75) {};
		\node [style=none] (3) at (23.5, 2.75) {};
		\node [style=s] (4) at (24, 4) {};
		\node [style=none] (5) at (23.5, 4.5) {};
	\end{pgfonlayer}
	\begin{pgfonlayer}{edgelayer}
		\draw [in=-90, out=90, looseness=1.25] (2.center) to (1.center);
		\draw (1.center) to (5.center);
		\draw (0.center) to (4);
		\draw [in=90, out=-90, looseness=1.25] (4) to (3.center);
	\end{pgfonlayer}
\end{tikzpicture}
=
F
$$
\end{proof}

In the ZX-calculus literature, this decomposition of the Fourier transform is known as {\it Euler decomposition} \cite{duncan2009graph}.
A variant of this decomposition is given in \cite[p.6]{control}; although in the context of plain old linear relations instead of Lagrangian relations, so an antipode is missing in their case.  A similar observation was made in \cite[(34)]{ranchin2014depicting} in terms of qudit controlled boost gates; however, the connection to phase-shift gates and Euler decomposition was not made.

From Theorem~\ref{theorem:generators}, we know that we can build any Lagrangian relation using pure Lagrangian relations and discard maps. Since the former is closed under composition and monoidal product, the following can be shown immediately from string diagram deformation.

\begin{corollary}[Phase purification]\label{cor:pure}
  Any linear Lagrangian relation can be written in the following form, for $V$ a linear relation:
$$
\begin{tikzpicture}
	\begin{pgfonlayer}{nodelayer}
		\node [style=X] (0) at (-0.25, 1.75) {};
		\node [style=scalar] (1) at (1, 0.75) {$a_1$};
		\node [style=none] (2) at (1, 0) {};
		\node [style=none] (3) at (-2.25, 0.5) {};
		\node [style=none] (4) at (-2.25, 0) {};
		\node [style=map, minimum width=2cm, minimum height=1cm] (5) at (-1.5, -0.5) {$V^\perp$};
		\node [style=map, minimum width=2cm, minimum height=1cm] (6) at (1.75, -0.5) {$V$};
		\node [style=X] (7) at (1, 1.75) {};
		\node [style=scalar] (8) at (2, 0.75) {$a_k$};
		\node [style=none] (9) at (2, 0) {};
		\node [style=none] (10) at (-1.25, 0.5) {};
		\node [style=none] (11) at (-1.25, 0) {};
		\node [style=none] (12) at (1.5, 0.25) {...};
		\node [style=none] (13) at (-0.75, 0) {};
		\node [style=none] (14) at (-0.75, 2.25) {};
		\node [style=none] (17) at (2.5, 0) {};
		\node [style=none] (18) at (2.5, 2.25) {};
		\node [style=none] (19) at (-1.75, 0.25) {...};
		\node [style=none] (20) at (-1.5, -1.5) {};
		\node [style=none] (21) at (-1.5, -0.75) {};
		\node [style=none] (22) at (1.75, -1.5) {};
		\node [style=none] (23) at (1.75, -0.75) {};
	\end{pgfonlayer}
	\begin{pgfonlayer}{edgelayer}
		\draw [in=-30, out=90, looseness=0.75] (1) to (0);
		\draw [in=90, out=-165, looseness=0.50] (0) to (3.center);
		\draw (4.center) to (3.center);
		\draw (2.center) to (1);
		\draw [in=-30, out=90, looseness=0.75] (8) to (7);
		\draw [in=90, out=-165, looseness=0.50] (7) to (10.center);
		\draw (11.center) to (10.center);
		\draw (9.center) to (8);
		\draw [in=270, out=90] (13.center) to (14.center);
		\draw [in=270, out=90] (17.center) to (18.center);
		\draw [in=270, out=90] (20.center) to (21.center);
		\draw [in=270, out=90] (22.center) to (23.center);
	\end{pgfonlayer}
\end{tikzpicture}
$$
\end{corollary}

In the case when we are working with prime fields, then Lagrangian relations are exactly an instance of the CPM construction. Namely, in the category of linear relations, $(-)^*$ is given by relational converse, so we can define a dagger functor $(-)^\dagger := ((-)^\perp)^*$ such that $(-)_* = (-)^\perp$. It only remains to show that all of the discarding maps arise from a single fixed cap. This can be done as follows, for $k = \F_p$:
$$
\begin{tikzpicture}
	\begin{pgfonlayer}{nodelayer}
		\node [style=X] (3) at (3, 2) {};
		\node [style=scalar] (7) at (3.5, 1.5) {$n$};
		\node [style=none] (8) at (2.5, 1.5) {};
		\node [style=none] (9) at (2.5, 1) {};
		\node [style=none] (10) at (3.5, 1) {};
	\end{pgfonlayer}
	\begin{pgfonlayer}{edgelayer}
		\draw [in=-15, out=90] (7) to (3);
		\draw [in=-165, out=90] (8.center) to (3);
		\draw (9.center) to (8.center);
		\draw (10.center) to (7);
	\end{pgfonlayer}
\end{tikzpicture}
=
\begin{tikzpicture}
	\begin{pgfonlayer}{nodelayer}
		\node [style=X] (0) at (5.25, 1.75) {};
		\node [style=Z] (1) at (6.25, 0.5) {};
		\node [style=none] (2) at (4.25, 0) {};
		\node [style=none] (3) at (6.25, 0) {};
		\node [style=none] (4) at (5.75, 1.25) {$\iddots$};
		\node [style=none] (5) at (5.9, 1) {$n$};
		\node [style=X] (6) at (4.75, 2.25) {};
	\end{pgfonlayer}
	\begin{pgfonlayer}{edgelayer}
		\draw [in=0, out=90, looseness=1.25] (1) to (0);
		\draw [in=-90, out=165, looseness=1.25] (1) to (0);
		\draw [in=-120, out=90] (2.center) to (6);
		\draw [in=105, out=-15] (6) to (0);
		\draw (3.center) to (1);
	\end{pgfonlayer}
\end{tikzpicture}
=
\begin{tikzpicture}
	\begin{pgfonlayer}{nodelayer}
		\node [style=none] (21) at (6.25, 2) {};
		\node [style=none] (22) at (5.25, 2.25) {};
		\node [style=X] (23) at (5.25, 2.25) {};
		\node [style=Z] (24) at (6.25, 2) {};
		\node [style=none] (25) at (5, 3) {};
		\node [style=none] (26) at (6.5, 2.75) {};
		\node [style=Z] (28) at (6.5, 2.75) {};
		\node [style=none] (30) at (6.25, 0.75) {};
		\node [style=none] (31) at (5.25, 1) {};
		\node [style=X] (32) at (5.25, 1) {};
		\node [style=Z] (33) at (6.25, 0.75) {};
		\node [style=none] (34) at (5.25, 0.25) {};
		\node [style=none] (35) at (6.25, 0.25) {};
		\node [style=none] (36) at (5.7, 1.6) {$\vdots$};
		\node [style=none] (37) at (5.95, 1.55) {$n$};
		\node [style=X] (38) at (5, 3) {};
	\end{pgfonlayer}
	\begin{pgfonlayer}{edgelayer}
		\draw [in=-90, out=135, looseness=0.75] (23) to (25.center);
		\draw [in=-90, out=60] (24) to (26.center);
		\draw (34.center) to (32);
		\draw (32) to (23);
		\draw (35.center) to (33);
		\draw (33) to (24);
		\draw (24) to (23);
		\draw (33) to (32);
	\end{pgfonlayer}
\end{tikzpicture}
=
\begin{tikzpicture}
	\begin{pgfonlayer}{nodelayer}
		\node [style=X] (622) at (274.75, 2.25) {};
		\node [style=none] (623) at (275.5, 1.25) {};
		\node [style=none] (624) at (274.25, 1.25) {};
		\node [style=X] (625) at (274.25, 1.25) {};
		\node [style=Z] (626) at (275.5, 1.25) {};
		\node [style=none] (627) at (274.25, 2.5) {};
		\node [style=none] (628) at (275.5, 2.5) {};
		\node [style=X] (629) at (274.25, 2.5) {};
		\node [style=Z] (630) at (275.5, 2.5) {};
		\node [style=X] (631) at (274.75, 0.75) {};
		\node [style=none] (632) at (275.5, -0.25) {};
		\node [style=none] (633) at (274.25, -0.25) {};
		\node [style=X] (634) at (274.25, -0.25) {};
		\node [style=Z] (635) at (275.5, -0.25) {};
		\node [style=none] (636) at (274.25, -1) {};
		\node [style=none] (637) at (275.5, -1) {};
		\node [style=none] (638) at (274.8, 1.59) {$\vdots$};
		\node [style=none] (639) at (275.05, 1.5) {$n$};
	\end{pgfonlayer}
	\begin{pgfonlayer}{edgelayer}
		\draw [in=-15, out=120] (623.center) to (622);
		\draw [in=-165, out=165, looseness=1.50] (624.center) to (622);
		\draw (625) to (627.center);
		\draw (626) to (628.center);
		\draw [in=-30, out=120] (632.center) to (631);
		\draw [in=-165, out=150, looseness=1.50] (633.center) to (631);
		\draw (636.center) to (634);
		\draw (634) to (625);
		\draw (637.center) to (635);
		\draw (635) to (626);
	\end{pgfonlayer}
\end{tikzpicture}
$$

%
%\begin{definition} \cite{cpm}:
%Given a strongly compact closed category $\X$ equipped with a dagger functor $(\_)^\dag:\X^\op\to\X$, there is a strongly compact closed category, $\CPM(\X,(\_)^\dag)$ with:
%\begin{description}
%\item[Objects:] Same as in $\X$
%\item[Maps:]
%\item[Identity:]
%\item[Composition:]
%\item[Tensor:]
%
%
%\end{description}
%\end{definition}


\begin{corollary}
\label{cor}
For $p$ prime, $\Lag\Rel_{\F_p} \cong \CPM[\LinRel_{\F_p}]$.
\end{corollary}

\section{Affine Lagrangian relations}
\label{sec:aff}

Affine Lagrangian relations are perhaps of more practical interest than plain old Lagrangian relations.  As we will discuss in this section, these give a semantics for qudit stabilizer circuits as well as certain electrical circuits.  We use our universal set of generators for Lagrangian relations as well as the presentation for affine relations to get a universal set of generators for affine Lagrangian relations.

\begin{definition}\cite[\S A]{affine}
Let $\aih_k$ denote the the prop presented by $\ih_k$ in addition to the generator
$\begin{tikzpicture}[scale=.7]
	\begin{pgfonlayer}{nodelayer}
		\node [style=X] (0) at (0.25, 0) {$1$};
		\node [style=none] (1) at (0.25, 0.35) {};
	\end{pgfonlayer}
	\begin{pgfonlayer}{edgelayer}
		\draw (0) to (1.center);
	\end{pgfonlayer}
\end{tikzpicture}$
 and three equations:
$$
\begin{tikzpicture}
	\begin{pgfonlayer}{nodelayer}
		\node [style=X] (0) at (0.75, 0) {$1$};
		\node [style=X] (1) at (0.75, 0.5) {};
		\node [style=none] (2) at (1.25, 1) {};
		\node [style=none] (3) at (1.25, -0.5) {};
	\end{pgfonlayer}
	\begin{pgfonlayer}{edgelayer}
		\draw (0) to (1);
		\draw (3.center) to (2.center);
	\end{pgfonlayer}
\end{tikzpicture}
=
\begin{tikzpicture}
	\begin{pgfonlayer}{nodelayer}
		\node [style=X] (0) at (0.75, 0) {$1$};
		\node [style=X] (1) at (0.75, 0.5) {};
		\node [style=none] (2) at (1.25, 1) {};
		\node [style=none] (3) at (1.25, -0.5) {};
		\node [style=Z] (4) at (1.25, 0.5) {};
		\node [style=Z] (5) at (1.25, 0) {};
	\end{pgfonlayer}
	\begin{pgfonlayer}{edgelayer}
		\draw (0) to (1);
		\draw (4) to (2.center);
		\draw (3.center) to (5);
	\end{pgfonlayer}
\end{tikzpicture}
\hspace*{.5cm}
\begin{tikzpicture}
	\begin{pgfonlayer}{nodelayer}
		\node [style=X] (0) at (0.25, 0) {$1$};
		\node [style=Z] (1) at (0.25, 0.5) {};
		\node [style=none] (2) at (0, 1) {};
		\node [style=none] (3) at (0.5, 1) {};
	\end{pgfonlayer}
	\begin{pgfonlayer}{edgelayer}
		\draw (0) to (1);
		\draw [in=-90, out=45] (1) to (3.center);
		\draw [in=135, out=-90] (2.center) to (1);
	\end{pgfonlayer}
\end{tikzpicture}
=
\begin{tikzpicture}
	\begin{pgfonlayer}{nodelayer}
		\node [style=X] (0) at (0, 0.5) {$1$};
		\node [style=none] (2) at (0, 1) {};
		\node [style=none] (3) at (0.5, 1) {};
		\node [style=X] (4) at (0.5, 0.5) {$1$};
	\end{pgfonlayer}
	\begin{pgfonlayer}{edgelayer}
		\draw (0) to (2.center);
		\draw (4) to (3.center);
	\end{pgfonlayer}
\end{tikzpicture}
\hspace*{.5cm}
\begin{tikzpicture}
	\begin{pgfonlayer}{nodelayer}
		\node [style=X] (0) at (0.75, 0) {$1$};
		\node [style=Z] (1) at (0.75, 0.5) {};
	\end{pgfonlayer}
	\begin{pgfonlayer}{edgelayer}
		\draw (0) to (1);
	\end{pgfonlayer}
\end{tikzpicture}
=
$$
\end{definition}


The following was stated slightly differently in the original paper:
\begin{definition}\cite[Definition 5]{affine}
Let $\Aff\Rel_k$ denote prop, whose morphisms $n\to m$ are the (possibly empty) affine subspaces of $k^{n}\oplus k^{m}$; with composition given by relational composition and tensor product given by the direct sum.
\end{definition}



\begin{theorem}\cite[Thm. 17]{affine}
$\aih_k$ is a presentation of $\Aff\Rel_k$.
\end{theorem}
Because the equation on the left holds,  we can use the phased-spider notation (as in the ZX-calculus), so that for all $a,b \in \F_p$: %of the ZX-calculus in $\aih_{\F_p}$, where the numbers inside the black spiders are  seen of elements of $\F_p$:
$$
\begin{tikzpicture}
	\begin{pgfonlayer}{nodelayer}
		\node [style=X] (0) at (0, 0) {};
		\node [style=none] (1) at (0, 0.75) {};
		\node [style=scalar] (2) at (-0.5, -0.75) {$a$};
		\node [style=scalar] (3) at (0.5, -0.75) {$b$};
		\node [style=X] (4) at (-0.5, -1.5) {$1$};
		\node [style=X] (5) at (0.5, -1.5) {$1$};
	\end{pgfonlayer}
	\begin{pgfonlayer}{edgelayer}
		\draw (0) to (1.center);
		\draw (4) to (2);
		\draw [in=-150, out=90] (2) to (0);
		\draw [in=90, out=-30] (0) to (3);
		\draw (3) to (5);
	\end{pgfonlayer}
\end{tikzpicture}
=
\begin{tikzpicture}
	\begin{pgfonlayer}{nodelayer}
		\node [style=X] (6) at (2, 0) {};
		\node [style=none] (7) at (2, 0.75) {};
		\node [style=scalar] (8) at (1.5, -0.75) {$a$};
		\node [style=scalar] (9) at (2.5, -0.75) {$b$};
		\node [style=X] (10) at (2, -2) {$1$};
		\node [style=Z] (11) at (2, -1.5) {};
	\end{pgfonlayer}
	\begin{pgfonlayer}{edgelayer}
		\draw (6) to (7.center);
		\draw [in=-150, out=90] (8) to (6);
		\draw [in=90, out=-30] (6) to (9);
		\draw (11) to (10);
		\draw [in=-90, out=30] (11) to (9);
		\draw [in=-90, out=150] (11) to (8);
	\end{pgfonlayer}
\end{tikzpicture}
=
\begin{tikzpicture}
	\begin{pgfonlayer}{nodelayer}
		\node [style=none] (7) at (2, 0.75) {};
		\node [style=scalar] (8) at (2, 0) {$a+b$};
		\node [style=X] (10) at (2, -0.75) {$1$};
	\end{pgfonlayer}
	\begin{pgfonlayer}{edgelayer}
		\draw (10) to (8);
		\draw (8) to (7.center);
	\end{pgfonlayer}
\end{tikzpicture}
\hspace*{.5cm}
\begin{tikzpicture}
	\begin{pgfonlayer}{nodelayer}
		\node [style=X] (0) at (7, 3.5) {$a$};
		\node [style=none] (1) at (6.75, 4) {};
		\node [style=none] (2) at (7.25, 4) {};
		\node [style=none] (3) at (6.75, 3) {};
		\node [style=none] (4) at (7.25, 3) {};
		\node [style=none] (5) at (6.75, 4.25) {};
		\node [style=none] (6) at (7.25, 4.25) {};
		\node [style=none] (7) at (7.25, 2.75) {};
		\node [style=none] (8) at (6.75, 2.75) {};
		\node [style=none] (9) at (7, 4) {$\cdots$};
		\node [style=none] (10) at (7, 3) {$\cdots$};
	\end{pgfonlayer}
	\begin{pgfonlayer}{edgelayer}
		\draw [in=-90, out=135] (0) to (1.center);
		\draw [in=-90, out=45] (0) to (2.center);
		\draw [in=-45, out=90] (4.center) to (0);
		\draw [in=90, out=-135] (0) to (3.center);
		\draw (8.center) to (3.center);
		\draw (7.center) to (4.center);
		\draw (2.center) to (6.center);
		\draw (5.center) to (1.center);
	\end{pgfonlayer}
\end{tikzpicture}
:=
\begin{tikzpicture}
	\begin{pgfonlayer}{nodelayer}
		\node [style=X] (0) at (7, 4) {};
		\node [style=none] (1) at (6.75, 4.5) {};
		\node [style=none] (2) at (7.25, 4.5) {};
		\node [style=none] (3) at (6.75, 2.5) {};
		\node [style=none] (4) at (7.25, 2.5) {};
		\node [style=none] (5) at (6.75, 4.75) {};
		\node [style=none] (6) at (7.25, 4.75) {};
		\node [style=none] (7) at (7.25, 2.25) {};
		\node [style=none] (8) at (6.75, 2.25) {};
		\node [style=none] (9) at (7, 4.5) {$\cdots$};
		\node [style=none] (10) at (7, 2.5) {$\cdots$};
		\node [style=X] (11) at (7.75, 2.75) {$1$};
		\node [style=scalar] (12) at (7.75, 3.25) {$a$};
	\end{pgfonlayer}
	\begin{pgfonlayer}{edgelayer}
		\draw [in=-90, out=135] (0) to (1.center);
		\draw [in=-90, out=45] (0) to (2.center);
		\draw [in=-45, out=90, looseness=0.75] (4.center) to (0);
		\draw [in=90, out=-135, looseness=0.75] (0) to (3.center);
		\draw (8.center) to (3.center);
		\draw (7.center) to (4.center);
		\draw (2.center) to (6.center);
		\draw (5.center) to (1.center);
		\draw (11) to (12);
		\draw [in=-15, out=90, looseness=0.75] (12) to (0);
	\end{pgfonlayer}
\end{tikzpicture}
\hspace{.5cm}
\begin{tikzpicture}
	\begin{pgfonlayer}{nodelayer}
		\node [style=none] (0) at (0.25, 2.5) {};
		\node [style=none] (1) at (1.25, 2.5) {};
		\node [style=none] (2) at (0.25, 0) {};
		\node [style=none] (3) at (1.25, 0) {};
		\node [style=X] (4) at (0, 1.5) {$a$};
		\node [style=X] (5) at (1, 1) {$b$};
		\node [style=none] (6) at (-0.25, 2.5) {};
		\node [style=none] (7) at (0.75, 2.5) {};
		\node [style=none] (8) at (0.75, 0) {};
		\node [style=none] (9) at (-0.25, 0) {};
		\node [style=none] (10) at (0, 2.25) {$\cdots$};
		\node [style=none] (11) at (1, 2.25) {$\cdots$};
		\node [style=none] (12) at (0, 0.25) {$\cdots$};
		\node [style=none] (13) at (1, 0.25) {$\cdots$};
	\end{pgfonlayer}
	\begin{pgfonlayer}{edgelayer}
		\draw [in=-60, out=90, looseness=1.25] (3.center) to (5);
		\draw [in=-90, out=60] (5) to (1.center);
		\draw [in=60, out=-90, looseness=1.25] (0.center) to (4);
		\draw [in=90, out=-60] (4) to (2.center);
		\draw (4) to (5);
		\draw [in=-120, out=90] (9.center) to (4);
		\draw [in=-90, out=120, looseness=1.25] (4) to (6.center);
		\draw [in=-90, out=120] (5) to (7.center);
		\draw [in=-120, out=90, looseness=1.25] (8.center) to (5);
	\end{pgfonlayer}
\end{tikzpicture}
=
\begin{tikzpicture}
	\begin{pgfonlayer}{nodelayer}
		\node [style=none] (14) at (2.75, 2.5) {};
		\node [style=none] (15) at (3.75, 2.5) {};
		\node [style=none] (16) at (2.75, 0) {};
		\node [style=none] (17) at (3.75, 0) {};
		\node [style=X] (18) at (3, 1.25) {};
		\node [style=X] (19) at (3, 1.25) {$\hspace*{.05cm}a+b\hspace*{.05cm}$};
		\node [style=none] (20) at (2.25, 2.5) {};
		\node [style=none] (21) at (3.25, 2.5) {};
		\node [style=none] (22) at (3.25, 0) {};
		\node [style=none] (23) at (2.25, 0) {};
		\node [style=none] (24) at (2.5, 2.25) {$\cdots$};
		\node [style=none] (25) at (3.5, 2.25) {$\cdots$};
		\node [style=none] (26) at (2.5, 0.25) {$\cdots$};
		\node [style=none] (27) at (3.5, 0.25) {$\cdots$};
	\end{pgfonlayer}
	\begin{pgfonlayer}{edgelayer}
		\draw [in=-30, out=90, looseness=1.25] (17.center) to (19);
		\draw [in=-90, out=30, looseness=1.25] (19) to (15.center);
		\draw [in=120, out=-90] (14.center) to (18);
		\draw [in=90, out=-120] (18) to (16.center);
		\draw [in=-150, out=90, looseness=1.25] (23.center) to (18);
		\draw [in=-90, out=150, looseness=1.25] (18) to (20.center);
		\draw [in=-90, out=60] (19) to (21.center);
		\draw [in=-60, out=90] (22.center) to (19);
	\end{pgfonlayer}
\end{tikzpicture}
$$
\begin{definition}
Let $\Aff\Lag\Rel_k$ denote the monoidal category whose objects are symplectic vector spaces, and whose morphisms are generated by the image of $\Lag\Rel_k \xrightarrow{E} \LinRel_k \to \Aff\Rel_k$ as well as all affine shifts and whose tensor product is the direct sum.
\end{definition}
Because the tensor product is defined in the same way as in $\Lag\Rel_k$, as in Lemma \ref{lemma:strong}, the forgetful functor  $\Aff\Lag\Rel_k\to \Aff\Rel_k$ is faithful, but only {\em strong} monoidal.



\begin{definition}
Let $\alr_k$ denote the monoidal subcategory of $\aih_k$ with objects $2n$, generated by the morphisms in the image of $\Lag\Rel_k\xrightarrow{E} \LinRel_k \cong \ih_k \to \aih_k$ as well as the following generator:
$$
\begin{tikzpicture}
	\begin{pgfonlayer}{nodelayer}
		\node [style=X] (0) at (0, 0) {$1$};
		\node [style=Z] (1) at (0.5, 0) {};
		\node [style=none] (2) at (0, 0.75) {};
		\node [style=none] (3) at (0.5, 0.75) {};
	\end{pgfonlayer}
	\begin{pgfonlayer}{edgelayer}
		\draw (1) to (3.center);
		\draw (2.center) to (0);
	\end{pgfonlayer}
\end{tikzpicture}
$$
\end{definition}

\begin{lemma}
$\alr_k$ is a presentation of $\Aff\Lag\Rel_k$.
\end{lemma}
\begin{proof}
All the affine shifts can be produced from tensoring and composing these two maps on the right:
$$
\begin{tikzpicture}
	\begin{pgfonlayer}{nodelayer}
		\node [style=X] (0) at (0, 0) {$1$};
		\node [style=Z] (1) at (0.5, 0) {};
		\node [style=none] (2) at (0, 0.75) {};
		\node [style=none] (3) at (0.5, 0.75) {};
		\node [style=none] (4) at (0.5, 1.5) {};
		\node [style=none] (5) at (0, 1.5) {};
		\node [style=s] (6) at (0.5, 0.75) {};
	\end{pgfonlayer}
	\begin{pgfonlayer}{edgelayer}
		\draw (1) to (3.center);
		\draw (2.center) to (0);
		\draw [in=270, out=90] (3.center) to (5.center);
		\draw [in=270, out=90] (2.center) to (4.center);
	\end{pgfonlayer}
\end{tikzpicture}
=
\begin{tikzpicture}
	\begin{pgfonlayer}{nodelayer}
		\node [style=X] (0) at (0.5, 0) {$1$};
		\node [style=Z] (1) at (0, 0) {};
		\node [style=none] (2) at (0.5, 0.75) {};
		\node [style=none] (3) at (0, 0.75) {};
	\end{pgfonlayer}
	\begin{pgfonlayer}{edgelayer}
		\draw (1) to (3.center);
		\draw (2.center) to (0);
	\end{pgfonlayer}
\end{tikzpicture} \hspace*{.1cm} \in  \alr_k
\hspace*{.5cm}
\implies
\hspace*{.5cm}
\begin{tikzpicture}
	\begin{pgfonlayer}{nodelayer}
		\node [style=Z] (447) at (224.25, 0.75) {};
		\node [style=X] (448) at (222.75, 0.75) {};
		\node [style=none] (449) at (223.75, -0.25) {};
		\node [style=none] (450) at (222.25, -0.25) {};
		\node [style=none] (451) at (224.25, 1.5) {};
		\node [style=none] (452) at (222.75, 1.5) {};
		\node [style=X] (453) at (223.25, -1) {$1$};
		\node [style=Z] (454) at (224.75, -1) {};
		\node [style=scalar] (455) at (223.25, -0.25) {$a$};
		\node [style=scalarop] (456) at (224.75, -0.25) {$a$};
		\node [style=none] (457) at (223.75, -1.5) {};
		\node [style=none] (458) at (222.25, -1.5) {};
	\end{pgfonlayer}
	\begin{pgfonlayer}{edgelayer}
		\draw [in=90, out=-150] (447) to (449.center);
		\draw [in=-150, out=90] (450.center) to (448);
		\draw (447) to (451.center);
		\draw (448) to (452.center);
		\draw (457.center) to (449.center);
		\draw (458.center) to (450.center);
		\draw (453) to (455);
		\draw (454) to (456);
		\draw [in=-30, out=90] (456) to (447);
		\draw [in=-30, out=90] (455) to (448);
	\end{pgfonlayer}
\end{tikzpicture}
=
\begin{tikzpicture}
	\begin{pgfonlayer}{nodelayer}
		\node [style=X] (17) at (5, 0) {$a$};
		\node [style=none] (18) at (5.5, -1.5) {};
		\node [style=none] (19) at (5, -1.5) {};
		\node [style=none] (20) at (5.5, 1.5) {};
		\node [style=none] (21) at (5, 1.5) {};
	\end{pgfonlayer}
	\begin{pgfonlayer}{edgelayer}
		\draw (19.center) to (17);
		\draw (17) to (21.center);
		\draw (18.center) to (20.center);
	\end{pgfonlayer}
\end{tikzpicture},
\hspace*{.5cm}
\begin{tikzpicture}
	\begin{pgfonlayer}{nodelayer}
		\node [style=Z] (459) at (226.25, 0.75) {};
		\node [style=X] (460) at (227.75, 0.75) {};
		\node [style=none] (461) at (226.75, -0.25) {};
		\node [style=none] (462) at (228.25, -0.25) {};
		\node [style=none] (463) at (226.25, 1.5) {};
		\node [style=none] (464) at (227.75, 1.5) {};
		\node [style=X] (465) at (227.25, -1) {$1$};
		\node [style=Z] (466) at (225.75, -1) {};
		\node [style=scalar] (467) at (227.25, -0.25) {$a$};
		\node [style=scalarop] (468) at (225.75, -0.25) {$a$};
		\node [style=none] (469) at (226.75, -1.5) {};
		\node [style=none] (470) at (228.25, -1.5) {};
	\end{pgfonlayer}
	\begin{pgfonlayer}{edgelayer}
		\draw [in=90, out=-30] (459) to (461.center);
		\draw [in=-30, out=90] (462.center) to (460);
		\draw (459) to (463.center);
		\draw (460) to (464.center);
		\draw (469.center) to (461.center);
		\draw (470.center) to (462.center);
		\draw (465) to (467);
		\draw (466) to (468);
		\draw [in=-150, out=90] (468) to (459);
		\draw [in=-150, out=90] (467) to (460);
	\end{pgfonlayer}
\end{tikzpicture}
=
\begin{tikzpicture}
	\begin{pgfonlayer}{nodelayer}
		\node [style=X] (0) at (1.5, 0) {$a$};
		\node [style=none] (1) at (1, -1.5) {};
		\node [style=none] (2) at (1.5, -1.5) {};
		\node [style=none] (3) at (1, 1.5) {};
		\node [style=none] (4) at (1.5, 1.5) {};
	\end{pgfonlayer}
	\begin{pgfonlayer}{edgelayer}
		\draw (2.center) to (0);
		\draw (0) to (4.center);
		\draw (1.center) to (3.center);
	\end{pgfonlayer}
\end{tikzpicture}
\hspace*{.1cm}\in\alr_k
$$
\end{proof}
Therefore, we are justified in using string diagrams in $\alr_k$ to reason about morphisms in $\Aff\Lag\Rel_k$.

We will restate the interpretations given in \cite{affine} of some components for electrical circuits in terms  affine relations  in terms of the generators for graphical calculus for Lagrangian relations.  This interpretation is also explored in \cite{passive,network}; albeit, not enjoying the graphical calculus for affine relations.
\begin{example}
\label{ex:circuits}
For any non-negative real $a$, wires, $a$-weighted resistors, inductors and capacitors have the following interpretations in $\Aff\Lag\Rel_{\mathbb{R}[x,y]/\langle xy-1\rangle}$:
$$
\left\llbracket
\begin{tikzpicture}
	\begin{pgfonlayer}{nodelayer}
		\node [style=none] (0) at (21, 4.25) {};
		\node [style=none] (1) at (22, 4.25) {};
		\node [style=none] (2) at (21, 2.75) {};
		\node [style=none] (3) at (22, 2.75) {};
		\node [style=dot] (4) at (21.5, 3.5) {};
		\node [style=none] (5) at (21.5, 4) {$\cdots$};
		\node [style=none] (6) at (21.5, 3) {$\cdots$};
	\end{pgfonlayer}
	\begin{pgfonlayer}{edgelayer}
		\draw [in=150, out=-90] (0.center) to (4);
		\draw [in=90, out=-150] (4) to (2.center);
		\draw [in=-30, out=90] (3.center) to (4);
		\draw [in=-90, out=30] (4) to (1.center);
	\end{pgfonlayer}
\end{tikzpicture}
\right\rrbracket
=
\begin{tikzpicture}
	\begin{pgfonlayer}{nodelayer}
		\node [style=none] (501) at (237.5, 4.25) {};
		\node [style=none] (502) at (238.5, 4.25) {};
		\node [style=none] (503) at (237.5, 2.75) {};
		\node [style=none] (504) at (238.5, 2.75) {};
		\node [style=X] (505) at (238, 3.5) {};
		\node [style=none] (506) at (238, 4) {$\cdots$};
		\node [style=none] (507) at (238, 3) {$\cdots$};
		\node [style=none] (508) at (236.25, 4.25) {};
		\node [style=none] (509) at (237.25, 4.25) {};
		\node [style=none] (510) at (236.25, 2.75) {};
		\node [style=none] (511) at (237.25, 2.75) {};
		\node [style=Z] (512) at (236.75, 3.5) {};
		\node [style=none] (513) at (236.75, 4) {$\cdots$};
		\node [style=none] (514) at (236.75, 3) {$\cdots$};
	\end{pgfonlayer}
	\begin{pgfonlayer}{edgelayer}
		\draw [in=150, out=-90] (501.center) to (505);
		\draw [in=90, out=-150] (505) to (503.center);
		\draw [in=-30, out=90] (504.center) to (505);
		\draw [in=-90, out=30] (505) to (502.center);
		\draw [in=150, out=-90] (508.center) to (512);
		\draw [in=90, out=-150] (512) to (510.center);
		\draw [in=-30, out=90] (511.center) to (512);
		\draw [in=-90, out=30] (512) to (509.center);
	\end{pgfonlayer}
\end{tikzpicture}
\hspace*{.5cm}
\left\llbracket
\tikz \draw (0,0) to[R=$a$] (0,2);
\hspace*{,3cm}
\right\rrbracket
=
\begin{tikzpicture}
	\begin{pgfonlayer}{nodelayer}
		\node [style=Z] (0) at (23, 3.75) {};
		\node [style=X] (1) at (22, 5.25) {};
		\node [style=scalar] (2) at (22.5, 4.5) {$a$};
		\node [style=none] (3) at (22, 5.75) {};
		\node [style=none] (4) at (23, 5.75) {};
		\node [style=none] (5) at (23, 3.25) {};
		\node [style=none] (6) at (22, 3.25) {};
	\end{pgfonlayer}
	\begin{pgfonlayer}{edgelayer}
		\draw [in=-90, out=135] (0) to (2);
		\draw [in=315, out=90] (2) to (1);
		\draw (1) to (3.center);
		\draw (1) to (6.center);
		\draw (5.center) to (0);
		\draw (4.center) to (0);
	\end{pgfonlayer}
\end{tikzpicture}
\hspace*{.5cm}
\left\llbracket
\tikz \draw (0,0) to[L=$a$] (0,2);
\hspace*{,3cm}
\right\rrbracket
=
\begin{tikzpicture}
	\begin{pgfonlayer}{nodelayer}
		\node [style=Z] (0) at (23, 3.75) {};
		\node [style=X] (1) at (22, 5.25) {};
		\node [style=scalar] (2) at (22.5, 4.5) {$ax$};
		\node [style=none] (3) at (22, 5.75) {};
		\node [style=none] (4) at (23, 5.75) {};
		\node [style=none] (5) at (23, 3.25) {};
		\node [style=none] (6) at (22, 3.25) {};
	\end{pgfonlayer}
	\begin{pgfonlayer}{edgelayer}
		\draw [in=-90, out=135] (0) to (2);
		\draw [in=315, out=90] (2) to (1);
		\draw (1) to (3.center);
		\draw (1) to (6.center);
		\draw (5.center) to (0);
		\draw (4.center) to (0);
	\end{pgfonlayer}
\end{tikzpicture}
\hspace*{.5cm}
\left\llbracket
\tikz \draw (0,0) to[C=$a$] (0,2);
\hspace*{,3cm}
\right\rrbracket
=
\begin{tikzpicture}
	\begin{pgfonlayer}{nodelayer}
		\node [style=Z] (0) at (23.25, 3.75) {};
		\node [style=X] (1) at (21.75, 5.25) {};
		\node [style=scalar] (2) at (22.5, 4.5) {$-ax$};
		\node [style=none] (3) at (21.75, 5.75) {};
		\node [style=none] (4) at (23.25, 5.75) {};
		\node [style=none] (5) at (23.25, 3.25) {};
		\node [style=none] (6) at (21.75, 3.25) {};
	\end{pgfonlayer}
	\begin{pgfonlayer}{edgelayer}
		\draw [in=-90, out=135] (0) to (2);
		\draw [in=315, out=90] (2) to (1);
		\draw (1) to (3.center);
		\draw (1) to (6.center);
		\draw (5.center) to (0);
		\draw (4.center) to (0);
	\end{pgfonlayer}
\end{tikzpicture}
$$
Similarly for $a$-valued voltage and current sources (again, for $a$ a non-negative real number):
$$
\left\llbracket
\begin{tikzpicture}
	\begin{pgfonlayer}{nodelayer}
		\node [style=none] (0) at (0, 2) {};
		\node [style=isourceAMshape,rotate=90] (1) at (0, 1) {};
		\node [style=none] (2) at (0, 0) {};
	\end{pgfonlayer}
	\begin{pgfonlayer}{edgelayer}
		\draw (2.center) to (1);
		\draw (1) to (0.center);
		\node [style=none] (3) at (-.7, 1) {$a$};
	\end{pgfonlayer}
\end{tikzpicture}
\hspace*{,3cm}
\right\rrbracket
=
\begin{tikzpicture}
	\begin{pgfonlayer}{nodelayer}
		\node [style=X] (1) at (22, 5.25) {};
		\node [style=scalar] (2) at (22.5, 4.5) {$ax$};
		\node [style=none] (3) at (22, 5.75) {};
		\node [style=none] (4) at (23, 5.75) {};
		\node [style=none] (5) at (23, 3.25) {};
		\node [style=none] (6) at (22, 3.25) {};
		\node [style=X] (7) at (22.5, 3.75) {$1$};
	\end{pgfonlayer}
	\begin{pgfonlayer}{edgelayer}
		\draw [in=315, out=90] (2) to (1);
		\draw (1) to (3.center);
		\draw (1) to (6.center);
		\draw (7) to (2);
		\draw (5.center) to (4.center);
	\end{pgfonlayer}
\end{tikzpicture}
=
\begin{tikzpicture}
	\begin{pgfonlayer}{nodelayer}
		\node [style=X] (32) at (129.25, -0.25) {$ax$};
		\node [style=none] (33) at (129.25, 1) {};
		\node [style=none] (34) at (129.75, 1) {};
		\node [style=none] (35) at (129.75, -1.5) {};
		\node [style=none] (36) at (129.25, -1.5) {};
	\end{pgfonlayer}
	\begin{pgfonlayer}{edgelayer}
		\draw (32) to (33.center);
		\draw (32) to (36.center);
		\draw (35.center) to (34.center);
	\end{pgfonlayer}
\end{tikzpicture}
\hspace{,3cm}
\left\llbracket
\begin{tikzpicture}
	\begin{pgfonlayer}{nodelayer}
		\node [style=none] (0) at (0, 2) {};
		\node [style=vsourceAMshape,rotate=-90] (1) at (0, 1) {};
		\node [style=none] (2) at (0, 0) {};
		\node [style=none] (3) at (-.7, 1) {$a$};
	\end{pgfonlayer}
	\begin{pgfonlayer}{edgelayer}
		\draw (2.center) to (1);
		\draw (1) to (0.center);
	\end{pgfonlayer}
\end{tikzpicture}
\hspace*{,3cm}
\right\rrbracket
=
\begin{tikzpicture}
	\begin{pgfonlayer}{nodelayer}
		\node [style=none] (28) at (9, 1) {};
		\node [style=none] (29) at (10, 1) {};
		\node [style=none] (30) at (10, -1.5) {};
		\node [style=none] (31) at (9, -1.5) {};
		\node [style=Z] (32) at (9, 0.25) {};
		\node [style=Z] (33) at (9, -0.75) {};
		\node [style=X] (34) at (9.5, -1.25) {$1$};
		\node [style=scalar] (35) at (9.5, -0.5) {$a$};
		\node [style=Z] (36) at (10, 0.5) {};
	\end{pgfonlayer}
	\begin{pgfonlayer}{edgelayer}
		\draw (31.center) to (33);
		\draw (32) to (28.center);
		\draw (36) to (29.center);
		\draw [in=90, out=-150] (36) to (35);
		\draw (35) to (34);
		\draw (30.center) to (36);
	\end{pgfonlayer}
\end{tikzpicture}
=
\begin{tikzpicture}
	\begin{pgfonlayer}{nodelayer}
		\node [style=none] (37) at (11, 1) {};
		\node [style=none] (38) at (12, 1) {};
		\node [style=none] (39) at (12, -1.5) {};
		\node [style=none] (40) at (11, -1.5) {};
		\node [style=Z] (41) at (11, 0.25) {};
		\node [style=Z] (42) at (11, -0.75) {};
		\node [style=X] (43) at (11.5, -1.25) {$1$};
		\node [style=Z] (45) at (12, -0.25) {};
		\node [style=scalar] (46) at (12, 0.5) {$a$};
		\node [style=scalarop] (47) at (12, -1) {$a$};
	\end{pgfonlayer}
	\begin{pgfonlayer}{edgelayer}
		\draw (40.center) to (42);
		\draw (41) to (37.center);
		\draw (45) to (46);
		\draw (46) to (38.center);
		\draw [in=-150, out=90] (43) to (45);
		\draw (39.center) to (47);
		\draw (47) to (45);
	\end{pgfonlayer}
\end{tikzpicture}
=
\begin{tikzpicture}
	\begin{pgfonlayer}{nodelayer}
		\node [style=none] (73) at (17, 0.75) {};
		\node [style=none] (74) at (17.75, 0.75) {};
		\node [style=none] (75) at (17.75, -2) {};
		\node [style=none] (76) at (17, -2) {};
		\node [style=Z] (77) at (17, 0.25) {};
		\node [style=Z] (78) at (17, -0.5) {};
		\node [style=X] (79) at (17.75, 0.25) {$a$};
		\node [style=scalarop] (80) at (17.75, -1.5) {$a$};
		\node [style=X] (83) at (17.75, -0.5) {$1$};
		\node [style=s] (84) at (17.75, -1) {};
	\end{pgfonlayer}
	\begin{pgfonlayer}{edgelayer}
		\draw (76.center) to (78);
		\draw (77) to (73.center);
		\draw [in=-90, out=90] (75.center) to (80);
		\draw (79) to (74.center);
		\draw (84) to (83);
		\draw [in=-90, out=90] (80) to (84);
	\end{pgfonlayer}
\end{tikzpicture}
=
\begin{tikzpicture}
	\begin{pgfonlayer}{nodelayer}
		\node [style=none] (84) at (18.75, 1.25) {};
		\node [style=none] (85) at (19.5, 1.25) {};
		\node [style=none] (86) at (19.5, -1.5) {};
		\node [style=none] (87) at (18.75, -1.5) {};
		\node [style=Z] (88) at (18.75, 0.25) {};
		\node [style=Z] (89) at (18.75, -0.5) {};
		\node [style=X] (90) at (19.5, 0.25) {$1$};
		\node [style=X] (91) at (19.5, -0.5) {$1$};
		\node [style=scalarop] (92) at (19.5, -1) {$-a$};
		\node [style=scalar] (93) at (18.75, -1) {$-a$};
		\node [style=scalar] (94) at (19.5, 0.75) {$a$};
		\node [style=scalarop] (95) at (18.75, 0.75) {$a$};
	\end{pgfonlayer}
	\begin{pgfonlayer}{edgelayer}
		\draw (86.center) to (92);
		\draw (92) to (91);
		\draw (89) to (93);
		\draw (87.center) to (93);
		\draw (88) to (95);
		\draw (95) to (84.center);
		\draw (90) to (94);
		\draw (94) to (85.center);
	\end{pgfonlayer}
\end{tikzpicture}
=
\begin{tikzpicture}
	\begin{pgfonlayer}{nodelayer}
		\node [style=none] (602) at (268.25, 1.75) {};
		\node [style=none] (603) at (269, 1.75) {};
		\node [style=none] (604) at (269, -1) {};
		\node [style=none] (605) at (267.25, -1) {};
		\node [style=Z] (606) at (268.25, 0.75) {};
		\node [style=Z] (607) at (267.75, 0.25) {};
		\node [style=X] (608) at (269, 0.75) {$1$};
		\node [style=X] (609) at (269.5, 0.25) {};
		\node [style=scalarop] (610) at (269, -0.5) {$-a$};
		\node [style=scalar] (611) at (267.25, -0.5) {$-a$};
		\node [style=scalar] (612) at (269, 1.25) {$a$};
		\node [style=scalarop] (613) at (268.25, 1.25) {$a$};
		\node [style=Z] (614) at (268.25, -0.5) {};
		\node [style=X] (615) at (270, -0.5) {$1$};
	\end{pgfonlayer}
	\begin{pgfonlayer}{edgelayer}
		\draw (604.center) to (610);
		\draw [in=-150, out=90] (610) to (609);
		\draw [in=90, out=-150] (607) to (611);
		\draw (605.center) to (611);
		\draw (606) to (613);
		\draw (613) to (602.center);
		\draw (608) to (612);
		\draw (612) to (603.center);
		\draw [in=-30, out=90] (615) to (609);
		\draw [in=-30, out=90] (614) to (607);
	\end{pgfonlayer}
\end{tikzpicture}
$$
\end{example}
Note that these generators do not generate the whole category of Lagrangian relations; for instance, the coefficients are required to be non-negative.

\subsection{Stabilizer circuits and Spekkens' toy model}


In this subsection, we show that, when $p$ is an odd prime, the prop of affine Lagrangian relations over $\F_p$  is isomoprhic to $p$-dimensional qudit stabilizer circuits, modulo invertible scalars.  We first consider an intermediary fragment between the Fourier-free, phase free fragments and stabilizer circuits.  


\begin{definition}
The qudit {\bf boost operator} is the following unitary on $d$ in $\Mat(\C)$, 
$
{\cal X} := \sum_{a =0}^{d-1} |  a+1  \rangle\langle a |
$.
\end{definition}

In the qubit case, the boost operator is just the not gate.  Adding the affine shift to $\ih_{\F_p}$, corresponds to adding the boost gate to the  Fourier-free, phase-free ZX-calculus, extending Lemma \ref{lemma:phasefree}.  This is a qudit generalization of the observation made in \cite{distzx}:

\begin{lemma}
For $p$ prime, $\aih_{\F_p}$ is isomorphic as a prop  to the Fourier-free, $p$-dimensional qudit ZX-calculus with the boost operator modulo invertible scalars.
\end{lemma}




We can go further with affine Lagrangian relations.  Inspired by the work of Spekkens \cite{spekkens,spekkens2016quasi}:


\begin{definition}
When $p$ is prime, let {\bf Spekkens' qudit toy model} of dimension $p$  denote the prop $\Aff\Lag\Rel_{\F_p}$.
\end{definition}

We first give a short review of the qudit stabilizer formalism, before establishing the equivalence between Spekkens' toy model and stabilizer circuits in the odd prime qudit case.  All of the material from Definition \ref{definition:begin} to \ref{lemma:end} are contained in \cite{generators}.


\begin{definition}
\label{definition:begin}
The qudit {\bf shift operator} is the following unitary on $d$ in $\Mat(\C)$, 
${\cal Z} := \sum_{a =0}^{d-1} e^{2\pi i a/d} |  a  \rangle\langle a |$.



An $n$-qudit {\bf Weyl operator} an $d^n$-dimensional unitary generated by the shift, boost and identity operators as well as the scalar $e^{\pi i /d}$ under tensor product and matrix multiplication. /

The $n$-qudit {\bf Weyl group},${\frak P}_d^{ n}$, is generated by the $n$-qudit Weyl operators under matrix multiplication.

An $n$-qudit {\bf Clifford operator} $U$ is an $d^n$-dimensional unitary so that $U {\frak P}_d^{ n} U^\dag = {\frak P}_d^{ n}$.

The $n$-qudit {\bf Clifford group} is formed by the $n$-qudit Clifford operators under matrix multiplication.

An $n$-qudit {\bf stabilizer state} is a state $ U |0\rangle^{\otimes n}$ for an $n$-qudit Clifford $U$.

Given any $n$-qudit stabilizer state $|\psi \rangle$,  the {\bf stabilizer group} of $|\psi \rangle$  is the (Abelian) subgroup of ${\frak S}_{|\psi\rangle} \subset {\frak P}_d^{ n}$ whose elements are the $U \in {\frak P}_d^{ n}$ for which $U|\psi \rangle=|\psi \rangle$.
\end{definition}


\begin{lemma}
Two stabilizer states with the same stabilizer groups are the same, up to global phases.
\end{lemma}

\begin{lemma}
\label{lemma:end}
For natural numbers $n,d \geq 2$ the $n$-dimensional qudit stabilizer group modulo invertible scalars is generated under tensor and composition of $I_d$ as well as the boost operator ${\cal X}$, the controlled-boost operator  ${\cal C}$, the Fourier transform ${\cal F}$ and the phase-shift operator ${\cal S}$:
$$
{\cal C}  := \sum_{a,b = 0 }^{d-1} |a,a+b \rangle\langle a, b|
\hspace*{.5cm}
{\cal F}  := \frac{1}{\sqrt d}\sum_{a,b= 0 }^{d-1} e^{2\pi i ab/d} |b \rangle\langle a|
\hspace*{.5cm}
{\cal S} := \sum_{a =0}^{d-1} e^{\pi i a (a+d)/d} |  a  \rangle\langle a |
$$
Notice that the boost operator can be obtained by ${\cal Z }={\cal F}{\cal X} {\cal F}^{2}$.
\end{lemma}

%\bar{e^{\pi i a (a+d)/d}}|  a  \rangle\langle a |
%e^{-\pi i a (a+d)/d}|  a  \rangle\langle a |
%e^{-\pi i a (a+d)/d}|  a  \rangle\langle a |


\begin{definition}
Let $\Stab_p$ denote the subcategory of $\Mat(\C)$ generated by the $p$-dimensional qudit Clifford group as well as the vectors $| 0\rangle$, $\langle 0|$, quotiented by invertible scalars.
\end{definition}


The following isomorphism is described in \cite{gross}, when restricted to the nonempty case.  This comes from the projective representation of the $n$ qudit odd-prime-dimensional Clifford group in terms of the affine symplectomorphisms over $\F_p^n$.  However, since there is only one empty relation and one zero matrix of every type, we get the following result immediately:

\begin{lemma}
For every odd prime $p$, there is an isomorphism $G:\Aff\Lag\Rel_{\F_p}(0, n) \to \Stab_p(0,n)$ determined by:
$$
\begin{tikzpicture}
	\begin{pgfonlayer}{nodelayer}
		\node [style=X] (393) at (206, -2) {$\pi$};
	\end{pgfonlayer}
\end{tikzpicture}
\mapsto 
0
\hspace*{.5cm}
\begin{tikzpicture}
	\begin{pgfonlayer}{nodelayer}
		\node [style=X] (389) at (205, -2) {};
		\node [style=none] (390) at (205, -1.5) {};
		\node [style=Z] (391) at (204.5, -2) {};
		\node [style=none] (392) at (204.5, -1.5) {};
	\end{pgfonlayer}
	\begin{pgfonlayer}{edgelayer}
		\draw (389) to (390.center);
		\draw (391) to (392.center);
	\end{pgfonlayer}
\end{tikzpicture}
 \mapsto |0\rangle
\hspace*{.5cm}
\begin{tikzpicture}
	\begin{pgfonlayer}{nodelayer}
		\node [style=none] (391) at (206, -1) {};
		\node [style=none] (392) at (206, -2) {};
		\node [style=none] (393) at (206.5, -1) {};
		\node [style=none] (394) at (206.5, -2) {};
		\node [style=X] (395) at (206.5, -1.5) {$1$};
	\end{pgfonlayer}
	\begin{pgfonlayer}{edgelayer}
		\draw (392.center) to (391.center);
		\draw (393.center) to (395);
		\draw (395) to (394.center);
	\end{pgfonlayer}
\end{tikzpicture}
 \mapsto {\cal X}
\hspace*{.5cm}
C_1 \mapsto {\cal C}
\hspace*{.5cm}
F \mapsto {\cal F}
\hspace*{.5cm}
S_1 \mapsto {\cal S}
$$
\end{lemma}

We extend this isomorphim of states to an isomorphism of props:

\begin{theorem}
\label{theorem:spekkens}
When $p$ is an odd prime, the mapping $H:\Aff\Lag\Rel_{\F_p} \to \Stab_p$ defined by:
$$
\begin{tikzpicture}
	\begin{pgfonlayer}{nodelayer}
		\node [style=map] (21) at (2, -2) {$f$};
		\node [style=none] (22) at (1.75, -1.25) {};
		\node [style=none] (23) at (2.25, -1.25) {};
		\node [style=none] (24) at (1.75, -2.75) {};
		\node [style=none] (25) at (2.25, -2.75) {};
	\end{pgfonlayer}
	\begin{pgfonlayer}{edgelayer}
		\draw [in=-90, out=120] (21) to (22.center);
		\draw [in=90, out=-120] (21) to (24.center);
		\draw [in=-60, out=90] (25.center) to (21);
		\draw [in=-90, out=60] (21) to (23.center);
	\end{pgfonlayer}
\end{tikzpicture}
\mapsto
\begin{tikzpicture}
	\begin{pgfonlayer}{nodelayer}
		\node [style=map] (7) at (12, 0) {$G\left(\hat f\right)$};
		\node [style=none] (8) at (11.25, 1.5) {};
		\node [style=map] (10) at (13, 1) {$\eta$};
		\node [style=none] (12) at (13.5, -0.5) {};
	\end{pgfonlayer}
	\begin{pgfonlayer}{edgelayer}
		\draw [in=135, out=-90] (8.center) to (7);
		\draw [in=-150, out=60, looseness=0.75] (7) to (10);
		\draw [in=-45, out=90] (12.center) to (10);
	\end{pgfonlayer}
\end{tikzpicture}
$$
is a symmetric monoidal equivalence, where $\eta$ is the cap of the compact closed structure induced by the $Z$ observable.
\end{theorem}

%
%%See Appendix \ref{proof:theorem:spekkens} for the proof.  
%The main difficulty in proving this is to show that $H$ is a functor.  This is shown by observing that stabilizer states with the same stabilizer group only differ by a global scalar.


\begin{proof}
It preserves identities by the snake equations.
Now we must show it preserves composition.
Consider some composable pair in $\Aff\Lag\Rel_{\F_p}$:
$$
\F_p^n \xrightarrow{f} \F_p^m \xrightarrow{g} \F_p^\ell
$$
If the composite is empty, then the result follows immediately.  Suppose otherwise.
We know that:
$$
\begin{tikzpicture}
	\begin{pgfonlayer}{nodelayer}
		\node [style=map] (189) at (103, 0) {$\hat {f;g}$};
		\node [style=none] (190) at (102.25, 1.5) {};
		\node [style=none] (191) at (103.25, 1.5) {};
		\node [style=none] (192) at (102.75, 1.5) {};
		\node [style=none] (193) at (103.75, 1.5) {};
	\end{pgfonlayer}
	\begin{pgfonlayer}{edgelayer}
		\draw [in=135, out=-90] (190.center) to (189);
		\draw [in=-90, out=75] (189) to (191.center);
		\draw [in=-90, out=45] (189) to (193.center);
		\draw [in=105, out=-90] (192.center) to (189);
	\end{pgfonlayer}
\end{tikzpicture}
=
\begin{tikzpicture}
	\begin{pgfonlayer}{nodelayer}
		\node [style=map] (163) at (96.5, -2.25) {$f;g$};
		\node [style=none] (164) at (96.75, -1.5) {};
		\node [style=none] (165) at (95.75, -1.5) {};
		\node [style=Z] (166) at (96.25, -3) {};
		\node [style=X] (167) at (95.75, -3) {};
		\node [style=none] (168) at (95.75, -2.25) {};
		\node [style=none] (169) at (95.25, -2.25) {};
		\node [style=none] (170) at (96.25, -1.5) {};
		\node [style=none] (171) at (95.25, -1.5) {};
	\end{pgfonlayer}
	\begin{pgfonlayer}{edgelayer}
		\draw [in=-90, out=60] (163) to (164.center);
		\draw [in=-90, out=120] (163) to (165.center);
		\draw [in=-120, out=30] (167) to (163);
		\draw [in=30, out=-60, looseness=1.25] (163) to (166);
		\draw [in=-90, out=150] (166) to (168.center);
		\draw [in=-90, out=135] (167) to (169.center);
		\draw (169.center) to (171.center);
		\draw [in=-90, out=90] (168.center) to (170.center);
	\end{pgfonlayer}
\end{tikzpicture}
=
\begin{tikzpicture}
	\begin{pgfonlayer}{nodelayer}
		\node [style=map] (211) at (110.5, -2.25) {$g$};
		\node [style=none] (212) at (110.75, -1.5) {};
		\node [style=none] (213) at (109.75, -1.5) {};
		\node [style=Z] (214) at (110.25, -3) {};
		\node [style=X] (215) at (109.75, -3) {};
		\node [style=none] (216) at (109.75, -2.25) {};
		\node [style=none] (217) at (109.25, -2.25) {};
		\node [style=none] (218) at (110.25, -1.5) {};
		\node [style=none] (219) at (109.25, -1.5) {};
		\node [style=map] (220) at (112.5, -2.25) {$f$};
		\node [style=none] (221) at (112.75, -1.5) {};
		\node [style=none] (222) at (111.75, -1.5) {};
		\node [style=Z] (223) at (112.25, -3) {};
		\node [style=X] (224) at (111.75, -3) {};
		\node [style=none] (225) at (111.75, -2.25) {};
		\node [style=none] (226) at (111.25, -2.25) {};
		\node [style=none] (227) at (112.25, -1.5) {};
		\node [style=none] (228) at (111.25, -1.5) {};
		\node [style=Z] (229) at (111.5, -0.5) {};
		\node [style=X] (230) at (110.5, -0.5) {};
		\node [style=none] (231) at (110.5, 0.25) {};
		\node [style=none] (232) at (111.5, 0.25) {};
		\node [style=none] (233) at (109.25, 0.25) {};
		\node [style=none] (234) at (112.75, 0.25) {};
	\end{pgfonlayer}
	\begin{pgfonlayer}{edgelayer}
		\draw [in=-90, out=60] (211) to (212.center);
		\draw [in=-90, out=120] (211) to (213.center);
		\draw [in=-120, out=30] (215) to (211);
		\draw [in=30, out=-60, looseness=1.25] (211) to (214);
		\draw [in=-90, out=150] (214) to (216.center);
		\draw [in=-90, out=135] (215) to (217.center);
		\draw (217.center) to (219.center);
		\draw [in=-90, out=90] (216.center) to (218.center);
		\draw [in=-90, out=60] (220) to (221.center);
		\draw [in=-90, out=120] (220) to (222.center);
		\draw [in=-120, out=30] (224) to (220);
		\draw [in=30, out=-60, looseness=1.25] (220) to (223);
		\draw [in=-90, out=150] (223) to (225.center);
		\draw [in=-90, out=135] (224) to (226.center);
		\draw (226.center) to (228.center);
		\draw [in=-90, out=90] (225.center) to (227.center);
		\draw [in=-30, out=90] (227.center) to (229);
		\draw [in=90, out=-150] (229) to (212.center);
		\draw [in=-30, out=90] (228.center) to (230);
		\draw [in=90, out=-150] (230) to (213.center);
		\draw (219.center) to (233.center);
		\draw [in=-90, out=90] (218.center) to (232.center);
		\draw [in=-90, out=90, looseness=0.75] (222.center) to (231.center);
		\draw (221.center) to (234.center);
	\end{pgfonlayer}
\end{tikzpicture}
=
\begin{tikzpicture}
	\begin{pgfonlayer}{nodelayer}
		\node [style=map] (64) at (28.75, 0) {$\hat g$};
		\node [style=none] (65) at (28, 1.25) {};
		\node [style=X] (66) at (29.25, 1) {};
		\node [style=Z] (67) at (29.75, 1) {};
		\node [style=none] (68) at (28.75, 1.25) {};
		\node [style=map] (69) at (30.25, 0) {$\hat f$};
		\node [style=none] (70) at (30.25, 1.25) {};
		\node [style=none] (71) at (31, 1.25) {};
		\node [style=none] (72) at (30.25, 2.5) {};
		\node [style=none] (73) at (28.75, 2.5) {};
		\node [style=none] (74) at (31, 2.5) {};
		\node [style=none] (75) at (28, 2.5) {};
	\end{pgfonlayer}
	\begin{pgfonlayer}{edgelayer}
		\draw [in=135, out=-90] (65.center) to (64);
		\draw [in=-150, out=45] (64) to (67);
		\draw [in=-165, out=105, looseness=1.25] (64) to (66);
		\draw [in=-90, out=60] (64) to (68.center);
		\draw [in=-30, out=60, looseness=1.25] (69) to (67);
		\draw [in=135, out=-30] (66) to (69);
		\draw [in=-90, out=120] (69) to (70.center);
		\draw [in=45, out=-90] (71.center) to (69);
		\draw (65.center) to (75.center);
		\draw [in=270, out=90] (68.center) to (72.center);
		\draw (71.center) to (74.center);
		\draw [in=270, out=90] (70.center) to (73.center);
	\end{pgfonlayer}
\end{tikzpicture}
$$

We have the following equality of diagrams in $\Stab_p$, 
We draw the wires exiting $G$ to be connected to the corresponding wires in the $X$ block of the subspace.
\begin{align*}
\begin{tikzpicture}
	\begin{pgfonlayer}{nodelayer}
		\node [style=map] (29) at (23, 0) {$G\left(\hat {f;g}\right)$};
		\node [style=none] (30) at (22.25, 1.5) {};
		\node [style=none] (33) at (23.75, 1.5) {};
	\end{pgfonlayer}
	\begin{pgfonlayer}{edgelayer}
		\draw [in=135, out=-90] (30.center) to (29);
		\draw [in=-90, out=45] (29) to (33.center);
	\end{pgfonlayer}
\end{tikzpicture}
&=
\begin{tikzpicture}
	\begin{pgfonlayer}{nodelayer}
		\node [style=map] (244) at (170.75, -4.5) {$\hat g$};
		\node [style=none] (245) at (170, -3.25) {};
		\node [style=X] (246) at (171.25, -3.5) {};
		\node [style=Z] (247) at (171.75, -3.5) {};
		\node [style=none] (248) at (170.75, -3.25) {};
		\node [style=map] (249) at (172.25, -4.5) {$\hat f$};
		\node [style=none] (250) at (172.25, -3.25) {};
		\node [style=none] (251) at (173, -3.25) {};
		\node [style=none] (252) at (172.25, -2) {};
		\node [style=none] (253) at (170.75, -2) {};
		\node [style=none] (254) at (173, -2) {};
		\node [style=none] (255) at (170, -2) {};
		\node [style=none] (256) at (169.75, -2) {};
		\node [style=none] (257) at (173.25, -2) {};
		\node [style=none] (258) at (173.25, -5) {};
		\node [style=none] (259) at (169.75, -5) {};
		\node [style=none] (260) at (170, -4.75) {$G$};
		\node [style=none] (261) at (170.75, -2) {};
		\node [style=none] (262) at (170, -2) {};
		\node [style=none] (263) at (170.75, -1.25) {};
		\node [style=none] (264) at (170, -1.25) {};
	\end{pgfonlayer}
	\begin{pgfonlayer}{edgelayer}
		\draw [in=135, out=-90] (245.center) to (244);
		\draw [in=-150, out=45] (244) to (247);
		\draw [in=-165, out=105, looseness=1.25] (244) to (246);
		\draw [in=-90, out=60] (244) to (248.center);
		\draw [in=-30, out=60, looseness=1.25] (249) to (247);
		\draw [in=135, out=-30] (246) to (249);
		\draw [in=-90, out=120] (249) to (250.center);
		\draw [in=45, out=-90] (251.center) to (249);
		\draw (245.center) to (255.center);
		\draw [in=270, out=90] (248.center) to (252.center);
		\draw (251.center) to (254.center);
		\draw [in=270, out=90] (250.center) to (253.center);
		\draw (257.center) to (256.center);
		\draw (256.center) to (259.center);
		\draw (259.center) to (258.center);
		\draw (258.center) to (257.center);
		\draw (262.center) to (264.center);
		\draw (261.center) to (263.center);
	\end{pgfonlayer}
\end{tikzpicture}
=
\begin{tikzpicture}
	\begin{pgfonlayer}{nodelayer}
		\node [style=map] (12) at (200, -1.75) {$\hat g$};
		\node [style=none] (13) at (199.25, -0.25) {};
		\node [style=none] (14) at (199.75, -0.25) {};
		\node [style=map] (15) at (202, -1.75) {$\hat f$};
		\node [style=none] (16) at (202.25, -0.25) {};
		\node [style=none] (17) at (202.75, -0.25) {};
		\node [style=none] (18) at (199, -0.25) {};
		\node [style=none] (19) at (203, -0.25) {};
		\node [style=none] (20) at (203, -2.25) {};
		\node [style=none] (21) at (199, -2.25) {};
		\node [style=none] (22) at (199.25, -2) {$G$};
		\node [style=none] (23) at (200.75, -0.25) {};
		\node [style=none] (24) at (199.25, -0.25) {};
		\node [style=none] (25) at (200.75, 1) {};
		\node [style=none] (26) at (199.25, 1) {};
		\node [style=none] (27) at (201.25, -0.25) {};
		\node [style=none] (28) at (201.75, -0.25) {};
		\node [style=none] (29) at (200.75, -0.25) {};
		\node [style=none] (30) at (200.25, -0.25) {};
		\node [style=none] (31) at (199.75, -0.25) {};
		\node [style=none] (32) at (200.25, -0.25) {};
		\node [style=map] (33) at (200, 0.5) {$\eta$};
	\end{pgfonlayer}
	\begin{pgfonlayer}{edgelayer}
		\draw [in=135, out=-90] (13.center) to (12);
		\draw [in=-90, out=105] (12) to (14.center);
		\draw [in=-90, out=75] (15) to (16.center);
		\draw [in=45, out=-90] (17.center) to (15);
		\draw (19.center) to (18.center);
		\draw (18.center) to (21.center);
		\draw (21.center) to (20.center);
		\draw (20.center) to (19.center);
		\draw (24.center) to (26.center);
		\draw (23.center) to (25.center);
		\draw [in=-90, out=45] (12) to (28.center);
		\draw [in=-90, out=75] (12) to (27.center);
		\draw [in=-90, out=105] (15) to (29.center);
		\draw [in=-90, out=135] (15) to (30.center);
		\draw [in=90, out=-120] (33) to (31.center);
		\draw [in=90, out=-60] (33) to (32.center);
	\end{pgfonlayer}
\end{tikzpicture}
=
\begin{tikzpicture}
	\begin{pgfonlayer}{nodelayer}
		\node [style=map] (34) at (204.75, 0) {$G\left(\hat g\right)$};
		\node [style=none] (35) at (204, 1.25) {};
		\node [style=map] (36) at (205.5, 1) {$\eta$};
		\node [style=map] (37) at (206.25, 0) {$G\left(\hat f\right)$};
		\node [style=none] (38) at (207, 1.25) {};
		\node [style=none] (39) at (207, 2.5) {};
		\node [style=none] (40) at (204, 2.5) {};
	\end{pgfonlayer}
	\begin{pgfonlayer}{edgelayer}
		\draw [in=135, out=-90] (35.center) to (34);
		\draw [in=-120, out=45, looseness=0.75] (34) to (36);
		\draw [in=-60, out=135, looseness=0.75] (37) to (36);
		\draw [in=45, out=-90] (38.center) to (37);
		\draw (35.center) to (40.center);
		\draw (38.center) to (39.center);
	\end{pgfonlayer}
\end{tikzpicture}
\end{align*}
This second equality is the only nontrivial part.  It follows by observing that both stabilizer states are stabilized by the same generalized Pauli operators, and thus they are the same.  This is because  the generalized Pauli operators can be pulled through $G$, by \cite[Lemma 4]{gross}, where they act the same on the caps of Lagrangian relations and in matrices.
%More explicitly, each white cap acting on the $i$th and $j$ wire is just removing vectors $[X_k|Z_k]$ from the stabilizer subspace which $X_{i.k}\neq -X_{i.k}$ or $Z_{i.k}\neq Z_{i.k}$ and then projecting out the $i$th and $j$th columns.

Explicity, the boost and shift operators commute with the cap as follows:
$$
\begin{tabular}{c}
\begin{tikzpicture}
	\begin{pgfonlayer}{nodelayer}
		\node [style=none] (515) at (241, -0.25) {};
		\node [style=none] (516) at (241, -1) {};
		\node [style=none] (517) at (239.5, -1) {};
		\node [style=X] (518) at (240, 0.75) {};
		\node [style=Z] (519) at (241.5, 0.75) {};
		\node [style=none] (520) at (240.5, -0.25) {};
		\node [style=none] (521) at (242, -0.25) {};
		\node [style=none] (522) at (240.5, -1) {};
		\node [style=none] (523) at (242, -1) {};
		\node [style=none] (524) at (239.5, -0.25) {};
		\node [style=X] (525) at (239.5, -0.25) {$a$};
	\end{pgfonlayer}
	\begin{pgfonlayer}{edgelayer}
		\draw (516.center) to (515.center);
		\draw [in=-165, out=90] (515.center) to (519);
		\draw [in=-15, out=90] (520.center) to (518);
		\draw [in=-15, out=90] (521.center) to (519);
		\draw (523.center) to (521.center);
		\draw (520.center) to (522.center);
		\draw (517.center) to (524.center);
		\draw [in=-165, out=90] (524.center) to (518);
	\end{pgfonlayer}
\end{tikzpicture}
=
\begin{tikzpicture}
	\begin{pgfonlayer}{nodelayer}
		\node [style=none] (526) at (244.25, -0.25) {};
		\node [style=none] (527) at (244.25, -1) {};
		\node [style=none] (528) at (242.75, -1) {};
		\node [style=X] (529) at (243.25, 0.75) {$a$};
		\node [style=Z] (530) at (244.75, 0.75) {};
		\node [style=none] (531) at (243.75, -0.25) {};
		\node [style=none] (532) at (245.25, -0.25) {};
		\node [style=none] (533) at (243.75, -1) {};
		\node [style=none] (534) at (245.25, -1) {};
		\node [style=none] (535) at (242.75, -0.25) {};
	\end{pgfonlayer}
	\begin{pgfonlayer}{edgelayer}
		\draw (527.center) to (526.center);
		\draw [in=-165, out=90] (526.center) to (530);
		\draw [in=-15, out=90] (531.center) to (529);
		\draw [in=-15, out=90] (532.center) to (530);
		\draw (534.center) to (532.center);
		\draw (531.center) to (533.center);
		\draw (528.center) to (535.center);
		\draw [in=-165, out=90] (535.center) to (529);
	\end{pgfonlayer}
\end{tikzpicture}
=
\begin{tikzpicture}
	\begin{pgfonlayer}{nodelayer}
		\node [style=none] (536) at (247.5, -0.25) {};
		\node [style=none] (537) at (247.5, -1) {};
		\node [style=none] (538) at (246, -1) {};
		\node [style=X] (539) at (246.5, 0.75) {};
		\node [style=Z] (540) at (248, 0.75) {};
		\node [style=none] (541) at (247, -0.25) {};
		\node [style=none] (542) at (248.5, -0.25) {};
		\node [style=none] (543) at (247, -1) {};
		\node [style=none] (544) at (248.5, -1) {};
		\node [style=none] (545) at (246, -0.25) {};
		\node [style=X] (546) at (247, -0.25) {$a$};
	\end{pgfonlayer}
	\begin{pgfonlayer}{edgelayer}
		\draw (537.center) to (536.center);
		\draw [in=-165, out=90] (536.center) to (540);
		\draw [in=-15, out=90] (541.center) to (539);
		\draw [in=-15, out=90] (542.center) to (540);
		\draw (544.center) to (542.center);
		\draw (541.center) to (543.center);
		\draw (538.center) to (545.center);
		\draw [in=-165, out=90] (545.center) to (539);
	\end{pgfonlayer}
\end{tikzpicture}\\
\begin{tikzpicture}
	\begin{pgfonlayer}{nodelayer}
		\node [style=none] (547) at (249.25, -3.75) {};
		\node [style=none] (548) at (249.25, -4.5) {};
		\node [style=none] (549) at (247.75, -4.5) {};
		\node [style=X] (550) at (248.25, -2.75) {};
		\node [style=Z] (551) at (249.75, -2.75) {};
		\node [style=none] (552) at (248.75, -3.75) {};
		\node [style=none] (553) at (250.25, -3.75) {};
		\node [style=none] (554) at (248.75, -4.5) {};
		\node [style=none] (555) at (250.25, -4.5) {};
		\node [style=none] (556) at (247.75, -3.75) {};
		\node [style=X] (557) at (249.25, -3.75) {$a$};
	\end{pgfonlayer}
	\begin{pgfonlayer}{edgelayer}
		\draw (548.center) to (547.center);
		\draw [in=-165, out=90] (547.center) to (551);
		\draw [in=-15, out=90] (552.center) to (550);
		\draw [in=-15, out=90] (553.center) to (551);
		\draw (555.center) to (553.center);
		\draw (552.center) to (554.center);
		\draw (549.center) to (556.center);
		\draw [in=-165, out=90] (556.center) to (550);
	\end{pgfonlayer}
\end{tikzpicture}
=
\begin{tikzpicture}
	\begin{pgfonlayer}{nodelayer}
		\node [style=none] (558) at (253, -4.5) {};
		\node [style=none] (559) at (251.5, -4.5) {};
		\node [style=X] (560) at (252, -2) {};
		\node [style=none] (561) at (252.5, -3) {};
		\node [style=none] (562) at (252.5, -4.5) {};
		\node [style=none] (563) at (254, -3) {};
		\node [style=none] (564) at (251.5, -3) {};
		\node [style=X] (565) at (253, -3.75) {$a$};
		\node [style=X] (566) at (253.5, -2) {};
		\node [style=s] (567) at (253, -3) {};
		\node [style=none] (568) at (254, -4.5) {};
	\end{pgfonlayer}
	\begin{pgfonlayer}{edgelayer}
		\draw [in=-15, out=90] (561.center) to (560);
		\draw [in=90, out=-90] (561.center) to (562.center);
		\draw (559.center) to (564.center);
		\draw [in=-165, out=90] (564.center) to (560);
		\draw [in=-15, out=90] (563.center) to (566);
		\draw (558.center) to (565);
		\draw (565) to (567);
		\draw [in=-165, out=90] (567) to (566);
		\draw (568.center) to (563.center);
	\end{pgfonlayer}
\end{tikzpicture}
=
\begin{tikzpicture}
	\begin{pgfonlayer}{nodelayer}
		\node [style=none] (569) at (256.75, -4.5) {};
		\node [style=none] (570) at (255.25, -4.5) {};
		\node [style=X] (571) at (255.75, -2) {};
		\node [style=none] (572) at (256.25, -3) {};
		\node [style=none] (573) at (256.25, -4.5) {};
		\node [style=none] (574) at (257.75, -3) {};
		\node [style=none] (575) at (255.25, -3) {};
		\node [style=X] (576) at (257.25, -2) {};
		\node [style=none] (577) at (257.75, -4.5) {};
		\node [style=s] (578) at (256.75, -3.75) {};
		\node [style=X] (579) at (256.75, -3) {$-a$};
	\end{pgfonlayer}
	\begin{pgfonlayer}{edgelayer}
		\draw [in=-15, out=90] (572.center) to (571);
		\draw [in=90, out=-90] (572.center) to (573.center);
		\draw (570.center) to (575.center);
		\draw [in=-165, out=90] (575.center) to (571);
		\draw [in=-15, out=90] (574.center) to (576);
		\draw (577.center) to (574.center);
		\draw (569.center) to (578);
		\draw (578) to (579);
		\draw [in=-165, out=90] (579) to (576);
	\end{pgfonlayer}
\end{tikzpicture}
=
\begin{tikzpicture}
	\begin{pgfonlayer}{nodelayer}
		\node [style=none] (580) at (261.25, -3.75) {};
		\node [style=none] (581) at (259.75, -3.75) {};
		\node [style=X] (582) at (260.25, -2) {};
		\node [style=none] (583) at (260.75, -3) {};
		\node [style=none] (584) at (260.75, -3.75) {};
		\node [style=none] (585) at (262.25, -3) {};
		\node [style=none] (586) at (259.75, -3) {};
		\node [style=X] (587) at (261.75, -2) {};
		\node [style=none] (588) at (262.25, -3.75) {};
		\node [style=s] (589) at (261.25, -3) {};
		\node [style=X] (590) at (262.25, -3) {$-a$};
	\end{pgfonlayer}
	\begin{pgfonlayer}{edgelayer}
		\draw [in=-15, out=90] (583.center) to (582);
		\draw [in=90, out=-90] (583.center) to (584.center);
		\draw (581.center) to (586.center);
		\draw [in=-165, out=90] (586.center) to (582);
		\draw [in=-15, out=90] (585.center) to (587);
		\draw (588.center) to (585.center);
		\draw (580.center) to (589);
		\draw [in=-165, out=90] (589) to (587);
	\end{pgfonlayer}
\end{tikzpicture}
=
\begin{tikzpicture}
	\begin{pgfonlayer}{nodelayer}
		\node [style=none] (591) at (265.25, -3) {};
		\node [style=none] (592) at (265.25, -3.75) {};
		\node [style=none] (593) at (263.75, -3.75) {};
		\node [style=X] (594) at (264.25, -2) {};
		\node [style=Z] (595) at (265.75, -2) {};
		\node [style=none] (596) at (264.75, -3) {};
		\node [style=none] (597) at (266.25, -3) {};
		\node [style=none] (598) at (264.75, -3.75) {};
		\node [style=none] (599) at (266.25, -3.75) {};
		\node [style=none] (600) at (263.75, -3) {};
		\node [style=X] (601) at (266.25, -3) {$-a$};
	\end{pgfonlayer}
	\begin{pgfonlayer}{edgelayer}
		\draw (592.center) to (591.center);
		\draw [in=-165, out=90] (591.center) to (595);
		\draw [in=-15, out=90] (596.center) to (594);
		\draw [in=-15, out=90] (597.center) to (595);
		\draw (599.center) to (597.center);
		\draw (596.center) to (598.center);
		\draw (593.center) to (600.center);
		\draw [in=-165, out=90] (600.center) to (594);
	\end{pgfonlayer}
\end{tikzpicture}
\end{tabular}
$$
Which are analagous to the following commutations in  stabilizer circuits:
$$
\begin{tikzpicture}
	\begin{pgfonlayer}{nodelayer}
		\node [style=map] (161) at (151.25, 6) {$\eta$};
		\node [style=none] (162) at (150.75, 5.25) {};
		\node [style=none] (163) at (151.75, 5.25) {};
		\node [style=none] (164) at (150.75, 4.5) {};
		\node [style=none] (165) at (151.75, 4.5) {};
		\node [style=map] (166) at (150.75, 5.25) {${\cal Z}^a$};
	\end{pgfonlayer}
	\begin{pgfonlayer}{edgelayer}
		\draw (165.center) to (163.center);
		\draw [in=-30, out=90] (163.center) to (161);
		\draw [in=90, out=-150] (161) to (162.center);
		\draw (162.center) to (164.center);
	\end{pgfonlayer}
\end{tikzpicture}
=
\begin{tikzpicture}
	\begin{pgfonlayer}{nodelayer}
		\node [style=map] (155) at (153.5, 6) {$\eta$};
		\node [style=none] (156) at (153, 5.25) {};
		\node [style=none] (157) at (154, 5.25) {};
		\node [style=none] (158) at (153, 4.5) {};
		\node [style=none] (159) at (154, 4.5) {};
		\node [style=map] (160) at (154, 5.25) {${\cal Z}^a$};
	\end{pgfonlayer}
	\begin{pgfonlayer}{edgelayer}
		\draw (159.center) to (157.center);
		\draw [in=-30, out=90] (157.center) to (155);
		\draw [in=90, out=-150] (155) to (156.center);
		\draw (156.center) to (158.center);
	\end{pgfonlayer}
\end{tikzpicture}
\hspace*{.5cm}
\begin{tikzpicture}
	\begin{pgfonlayer}{nodelayer}
		\node [style=map] (167) at (151.25, 3.75) {$\eta$};
		\node [style=none] (168) at (150.75, 3) {};
		\node [style=none] (169) at (151.75, 3) {};
		\node [style=none] (170) at (150.75, 2.25) {};
		\node [style=none] (171) at (151.75, 2.25) {};
		\node [style=map] (172) at (150.75, 3) {${\cal X}^a$};
	\end{pgfonlayer}
	\begin{pgfonlayer}{edgelayer}
		\draw (171.center) to (169.center);
		\draw [in=-30, out=90] (169.center) to (167);
		\draw [in=90, out=-150] (167) to (168.center);
		\draw (168.center) to (170.center);
	\end{pgfonlayer}
\end{tikzpicture}
=
\begin{tikzpicture}
	\begin{pgfonlayer}{nodelayer}
		\node [style=map] (173) at (153.5, 4) {$\eta$};
		\node [style=none] (174) at (153, 3.25) {};
		\node [style=none] (175) at (154, 3.25) {};
		\node [style=none] (176) at (153, 2.5) {};
		\node [style=none] (177) at (154, 2.5) {};
		\node [style=map] (178) at (154, 3.25) {${\cal X}^{-a}$};
	\end{pgfonlayer}
	\begin{pgfonlayer}{edgelayer}
		\draw (177.center) to (175.center);
		\draw [in=-30, out=90] (175.center) to (173);
		\draw [in=90, out=-150] (173) to (174.center);
		\draw (174.center) to (176.center);
	\end{pgfonlayer}
\end{tikzpicture}
$$
Where moreover, for any Lagrangian relation $V$ and  $a,b \in \F_p$, we already know:
$$
\begin{tikzpicture}
	\begin{pgfonlayer}{nodelayer}
		\node [style=none] (294) at (183.25, -0.75) {};
		\node [style=none] (295) at (185.75, -0.75) {};
		\node [style=none] (296) at (185.75, -3) {};
		\node [style=none] (297) at (183.25, -3) {};
		\node [style=none] (298) at (183.5, -2.75) {$G$};
		\node [style=map] (299) at (184.5, -2.5) {$V$};
		\node [style=none] (300) at (183.5, -0.75) {};
		\node [style=none] (301) at (184, -0.75) {};
		\node [style=none] (302) at (185, -0.75) {};
		\node [style=none] (303) at (185.5, -0.75) {};
		\node [style=X] (304) at (183.5, -1.5) {$a$};
		\node [style=X] (305) at (185, -1.5) {$b$};
		\node [style=none] (306) at (184, -1.5) {};
		\node [style=none] (307) at (185.5, -1.5) {};
		\node [style=none] (308) at (184, -0.25) {};
		\node [style=none] (309) at (183.5, -0.25) {};
		\node [style=none] (310) at (184, -0.75) {};
		\node [style=none] (311) at (183.5, -0.75) {};
	\end{pgfonlayer}
	\begin{pgfonlayer}{edgelayer}
		\draw (295.center) to (294.center);
		\draw (294.center) to (297.center);
		\draw (297.center) to (296.center);
		\draw (296.center) to (295.center);
		\draw [in=-90, out=45] (299) to (307.center);
		\draw (307.center) to (303.center);
		\draw (302.center) to (305);
		\draw [in=75, out=-90] (305) to (299);
		\draw [in=-90, out=105] (299) to (306.center);
		\draw (306.center) to (301.center);
		\draw (300.center) to (304);
		\draw [in=135, out=-90] (304) to (299);
		\draw (311.center) to (309.center);
		\draw (308.center) to (310.center);
	\end{pgfonlayer}
\end{tikzpicture}
=
\begin{tikzpicture}
	\begin{pgfonlayer}{nodelayer}
		\node [style=none] (351) at (195, -0.75) {};
		\node [style=none] (352) at (198, -0.75) {};
		\node [style=none] (353) at (198, -2.25) {};
		\node [style=none] (354) at (195, -2.25) {};
		\node [style=none] (355) at (195.25, -2) {$G$};
		\node [style=map] (356) at (196.5, -1.75) {$V$};
		\node [style=none] (357) at (195.25, -0.75) {};
		\node [style=none] (358) at (196.5, -0.75) {};
		\node [style=none] (359) at (197.25, -0.75) {};
		\node [style=none] (360) at (197.75, -0.75) {};
		\node [style=none] (361) at (196.5, 0.25) {};
		\node [style=none] (362) at (196.5, -0.75) {};
		\node [style=none] (363) at (195.25, -0.75) {};
		\node [style=map] (364) at (195.25, 0.25) {${\cal X}^a {\cal Z}^b$};
		\node [style=none] (365) at (195.25, 1) {};
		\node [style=none] (366) at (196.5, 1) {};
	\end{pgfonlayer}
	\begin{pgfonlayer}{edgelayer}
		\draw (352.center) to (351.center);
		\draw (351.center) to (354.center);
		\draw (354.center) to (353.center);
		\draw (353.center) to (352.center);
		\draw (361.center) to (362.center);
		\draw [in=270, out=90] (364) to (365.center);
		\draw [in=90, out=-90] (364) to (363.center);
		\draw [in=-90, out=135, looseness=0.75] (356) to (357.center);
		\draw [in=105, out=-90, looseness=0.75] (358.center) to (356);
		\draw [in=-90, out=75] (356) to (359.center);
		\draw [in=45, out=-90, looseness=0.75] (360.center) to (356);
		\draw (361.center) to (366.center);
	\end{pgfonlayer}
\end{tikzpicture}
$$

Therefore, functoriality follows by uncurrying the left and right hand sides of the previous equation.  Fullness and faithfulness follow immediately from $G$ being an isomorphism.
\end{proof}




\begin{definition}
Define a conjugation functor $\bar{(\_)}:\Aff\Lag\Rel_k\to \Aff\Lag\Rel_k$ the identity on $L(\LinRel_k)$ and $X$, but taking $F\mapsto F^{-1}$, $S_a \mapsto S_a^{-1}$ and $X\mapsto X$.
\end{definition}

The following fact follows from a mechanical calculation:


\begin{lemma}
For odd prime $p$, the conjugation functor $\bar{(\_)}:\Aff\Lag\Rel_{\F_p}\to \Aff\Lag\Rel_{\F_p}$ corresponds to complex conjugation in $\Stab_p$.
\end{lemma}





As we mentioned in the introduction, this is a categorical reformulation of the result of Spekkens' in which he shows that odd-prime-dimensional `quadrature epistricted theories' are operationally equivalent to prime-dimensional qudit stabilizer circuits \cite{spekkens2016quasi}.



A complete presentation for Spekkens' qubit toy model in terms of a category of relations was given \cite{backensspek} in a style which mirrors that of the qubit ZX-calculus~\cite{coecke2008interacting}. We now show how the generators of that presentation appear in our `doubled' formulation.


\begin{remark}
We can present Spekkens' $p$-dimensional qudit toy model in a manner similar to the ZX-calculus, in terms of being generated by spiders with phases labelled by the group $\Z/p\Z\times \Z/p\Z$,:
$$
\left\llbracket
\begin{tikzpicture}
	\begin{pgfonlayer}{nodelayer}
		\node [style=none] (0) at (21, 5) {};
		\node [style=none] (1) at (22, 5) {};
		\node [style=none] (2) at (21, 2.5) {};
		\node [style=none] (3) at (22, 2.5) {};
		\node [style=Z] (4) at (21.5, 3.75) {$\hspace*{.05cm}n,m\hspace*{.05cm}$};
		\node [style=none] (5) at (21.5, 4.5) {$\cdots$};
		\node [style=none] (6) at (21.5, 3) {$\cdots$};
		\node [style=none] (7) at (21.5, 4.75) {};
		\node [style=none] (8) at (21.5, 2.75) {};
	\end{pgfonlayer}
	\begin{pgfonlayer}{edgelayer}
		\draw [in=150, out=-90, looseness=0.75] (0.center) to (4);
		\draw [in=90, out=-150, looseness=0.75] (4) to (2.center);
		\draw [in=-30, out=90, looseness=0.75] (3.center) to (4);
		\draw [in=-90, out=30, looseness=0.75] (4) to (1.center);
	\end{pgfonlayer}
\end{tikzpicture}
\right\rrbracket
=
\begin{tikzpicture}
	\begin{pgfonlayer}{nodelayer}
		\node [style=none] (9) at (231.75, 0.5) {};
		\node [style=none] (10) at (231.75, -3) {};
		\node [style=Z] (11) at (231, -2) {};
		\node [style=none] (12) at (231.32, -1.25) {$\cdots$};
		\node [style=none] (13) at (231, -2.5) {$\cdots$};
		\node [style=none] (14) at (230.75, 0.5) {};
		\node [style=none] (15) at (229.25, -3) {};
		\node [style=X] (16) at (230, -0.5) {$n$};
		\node [style=none] (17) at (230, 0) {$\cdots$};
		\node [style=none] (18) at (229.72, -1.25) {$\cdots$};
		\node [style=none] (19) at (230, -3) {};
		\node [style=none] (20) at (230.25, -3) {};
		\node [style=none] (21) at (229.25, 0.5) {};
		\node [style=none] (22) at (231, 0.5) {};
		\node [style=scalar] (23) at (230.5, -1.25) {$m$};
		\node [style=none] (24) at (230, 0.25) {};
		\node [style=none] (25) at (231, -2.75) {};
		\node [style=none] (26) at (229.7, -1.5) {};
		\node [style=none] (27) at (231.3, -1) {};
	\end{pgfonlayer}
	\begin{pgfonlayer}{edgelayer}
		\draw [in=-30, out=90] (10.center) to (11);
		\draw [in=-90, out=30, looseness=0.75] (11) to (9.center);
		\draw [in=90, out=-150] (16) to (15.center);
		\draw [in=-90, out=30] (16) to (14.center);
		\draw (19.center) to (16);
		\draw [in=-150, out=90] (20.center) to (11);
		\draw [in=-90, out=150] (16) to (21.center);
		\draw (11) to (22.center);
		\draw [in=-75, out=150] (11) to (23);
		\draw [in=330, out=90] (23) to (16);
	\end{pgfonlayer}
\end{tikzpicture}
\hspace*{.5cm}
\left\llbracket
\begin{tikzpicture}
	\begin{pgfonlayer}{nodelayer}
		\node [style=none] (0) at (21, 5) {};
		\node [style=none] (1) at (22, 5) {};
		\node [style=none] (2) at (21, 2.5) {};
		\node [style=none] (3) at (22, 2.5) {};
		\node [style=X] (4) at (21.5, 3.75) {$\hspace*{.05cm}n,m\hspace*{.05cm}$};
		\node [style=none] (5) at (21.5, 4.5) {$\cdots$};
		\node [style=none] (6) at (21.5, 3) {$\cdots$};
		\node [style=none] (7) at (21.5, 4.75) {};
		\node [style=none] (8) at (21.5, 2.75) {};
	\end{pgfonlayer}
	\begin{pgfonlayer}{edgelayer}
		\draw [in=150, out=-90, looseness=0.75] (0.center) to (4);
		\draw [in=90, out=-150, looseness=0.75] (4) to (2.center);
		\draw [in=-30, out=90, looseness=0.75] (3.center) to (4);
		\draw [in=-90, out=30, looseness=0.75] (4) to (1.center);
	\end{pgfonlayer}
\end{tikzpicture}
\right\rrbracket
=
\begin{tikzpicture}
	\begin{pgfonlayer}{nodelayer}
		\node [style=none] (0) at (232.75, 0.5) {};
		\node [style=none] (1) at (232.75, -3) {};
		\node [style=Z] (2) at (233.5, -2) {};
		\node [style=none] (3) at (233.25, -1.25) {$\cdots$};
		\node [style=none] (4) at (233.5, -2.5) {$\cdots$};
		\node [style=none] (5) at (233.75, 0.5) {};
		\node [style=none] (6) at (235.25, -3) {};
		\node [style=X] (7) at (234.5, -0.5) {$n$};
		\node [style=none] (8) at (234.5, 0) {$\cdots$};
		\node [style=none] (9) at (234.82, -1.25) {$\cdots$};
		\node [style=none] (10) at (234.5, -3) {};
		\node [style=none] (11) at (234.25, -3) {};
		\node [style=none] (12) at (235.25, 0.5) {};
		\node [style=none] (13) at (233.5, 0.5) {};
		\node [style=scalar] (14) at (234, -1.25) {$m$};
		\node [style=none] (15) at (233.25, -1) {};
		\node [style=none] (16) at (234.5, 0.25) {};
		\node [style=none] (17) at (233.5, -2.75) {};
		\node [style=none] (18) at (234.78, -1.5) {};
	\end{pgfonlayer}
	\begin{pgfonlayer}{edgelayer}
		\draw [in=-150, out=90] (1.center) to (2);
		\draw [in=-90, out=150, looseness=0.75] (2) to (0.center);
		\draw [in=90, out=-30] (7) to (6.center);
		\draw [in=-90, out=150] (7) to (5.center);
		\draw (10.center) to (7);
		\draw [in=-30, out=90] (11.center) to (2);
		\draw [in=-90, out=30] (7) to (12.center);
		\draw (2) to (13.center);
		\draw [in=-105, out=30] (2) to (14);
		\draw [in=-150, out=90] (14) to (7);
	\end{pgfonlayer}
\end{tikzpicture}
\hspace*{.5cm}
\left\llbracket
\begin{tikzpicture}
	\begin{pgfonlayer}{nodelayer}
		\node [style=none] (0) at (1.25, -1) {};
		\node [style=map] (1) at (1.25, -1.5) {$F$};
		\node [style=none] (2) at (1.25, -2) {};
	\end{pgfonlayer}
	\begin{pgfonlayer}{edgelayer}
		\draw (2.center) to (1);
		\draw (1) to (0.center);
	\end{pgfonlayer}
\end{tikzpicture}
\right\rrbracket
=
\begin{tikzpicture}
	\begin{pgfonlayer}{nodelayer}
		\node [style=none] (0) at (0.5, 1) {};
		\node [style=none] (1) at (0.5, -0.25) {};
		\node [style=none] (2) at (1, -0.25) {};
		\node [style=none] (3) at (1, 1) {};
		\node [style=s] (4) at (1, 0.5) {};
		\node [style=none] (5) at (0.5, 0.5) {};
	\end{pgfonlayer}
	\begin{pgfonlayer}{edgelayer}
		\draw (4) to (3.center);
		\draw [in=90, out=-90] (4) to (1.center);
		\draw [in=-90, out=90] (2.center) to (5.center);
		\draw (5.center) to (0.center);
	\end{pgfonlayer}
\end{tikzpicture}
$$
The Fourier transform is redundant, as it can be obtained by Euler decomposition.
\end{remark}




Notice that the phases of the $Z$ and $X$ observables are elements $(n,m)$ of $\F_p \times \F_p$, and it is easy to see how the doubled spiders satisfy the phased spider fusion laws with respect to the group $\Z/p\Z\times \Z/p\Z$, as discussed in \cite[p.166]{ranchin2016alternative}.  As discussed in \cite{coecke2011phase} this is one of the central features which separates Spekkens' qubit model from qubit stabilizers, whose phase group is $\Z/4\Z$.  This fact can be also observed graphically in terms of the stabilizer fragment of the ZX-calculus (in contrast to the presentation of Spekkens' qubit toy model) which also enjoys a complete axiomatization \cite{backensspek}.

By stating the interpretations of Spekkens' toy model in terms of the graphical calculus for Lagrangian relations alongside that of electrical circuits, we see the evident analogy between the phases in the ZX-calculus and the resistors, inductor, capacitors and voltage sources in electrical circuits.


\section{Measurement and CoIsotropic relations}


By allowing more morphisms than Lagrangian subspaces we can capture more behaviour.  In this section, we show how the coisotropic generalization of Lagrangian relations gives a semantics for measurement in the symplectic setting.



\begin{definition}
Given a field $k$, the prop $\Isot\Rel_k$ of {\bf isotropic relations} has morphisms $n\to m$ as isotropic subspaces of the symplectic vector spaces $k^{n+m}\oplus k^{n+m}$.  Dually, the prop  $\Co\Isot\Rel_k$ of {\bf coisotropic relations} has  morphisms are coisotropic subspaces.  The composition, identity and tensor is that same as for Lagrangian relations.
The $\Aff\Isot\Rel_k$ and $\Aff\Co\Isot\Rel_k$ of {\bf affine (co)isotropic relations} have morphisms are affine (co)isotropic subspaces of the direct sum of the domain and codomain.
\end{definition}


It is mechanical to verify that these are props.




\begin{theorem}
The prop $\Isot\Rel_k$ is generated by adding the doubled zero relation to the image of the embedding $\Lag\Rel_k\to\LinRel_k$, ie. the following generator in $\ih_k$:
$
\begin{tikzpicture}
	\begin{pgfonlayer}{nodelayer}
		\node [style=X] (0) at (0.5, 0.5) {};
		\node [style=X] (1) at (1, 0.5) {};
		\node [style=none] (2) at (0.5, 0) {};
		\node [style=none] (3) at (1, 0) {};
	\end{pgfonlayer}
	\begin{pgfonlayer}{edgelayer}
		\draw (1) to (3.center);
		\draw (0) to (2.center);
	\end{pgfonlayer}
\end{tikzpicture}
$
\end{theorem}

\begin{proof}
First observe that this generator is an isotropic subspace of $(k^{2n},\omega)$ since:
$$
\left(
\begin{tikzpicture}
	\begin{pgfonlayer}{nodelayer}
		\node [style=X] (0) at (0.5, 0.5) {};
		\node [style=X] (1) at (1, 0.5) {};
		\node [style=none] (2) at (0.5, 0) {};
		\node [style=none] (3) at (1, 0) {};
	\end{pgfonlayer}
	\begin{pgfonlayer}{edgelayer}
		\draw (1) to (3.center);
		\draw (0) to (2.center);
	\end{pgfonlayer}
\end{tikzpicture}
\right)^\omega
=
\begin{tikzpicture}
	\begin{pgfonlayer}{nodelayer}
		\node [style=Z] (0) at (0.5, 0.5) {};
		\node [style=Z] (1) at (1, 0.5) {};
		\node [style=none] (2) at (0.5, 0) {};
		\node [style=none] (3) at (1, 0) {};
		\node [style=s] (4) at (1, 0) {};
		\node [style=none] (5) at (1, -0.75) {};
		\node [style=none] (7) at (0.5, -0.75) {};
	\end{pgfonlayer}
	\begin{pgfonlayer}{edgelayer}
		\draw (1) to (3.center);
		\draw (0) to (2.center);
		\draw [in=270, out=90] (7.center) to (4.center);
		\draw [in=90, out=-90] (2.center) to (5.center);
	\end{pgfonlayer}
\end{tikzpicture}
=
\begin{tikzpicture}
	\begin{pgfonlayer}{nodelayer}
		\node [style=Z] (0) at (0.5, 0.5) {};
		\node [style=Z] (1) at (1, 0.5) {};
		\node [style=none] (2) at (0.5, 0) {};
		\node [style=none] (3) at (1, 0) {};
	\end{pgfonlayer}
	\begin{pgfonlayer}{edgelayer}
		\draw (1) to (3.center);
		\draw (0) to (2.center);
	\end{pgfonlayer}
\end{tikzpicture}
\supset
\begin{tikzpicture}
	\begin{pgfonlayer}{nodelayer}
		\node [style=X] (0) at (0.5, 0.5) {};
		\node [style=X] (1) at (1, 0.5) {};
		\node [style=none] (2) at (0.5, 0) {};
		\node [style=none] (3) at (1, 0) {};
	\end{pgfonlayer}
	\begin{pgfonlayer}{edgelayer}
		\draw (1) to (3.center);
		\draw (0) to (2.center);
	\end{pgfonlayer}
\end{tikzpicture}
$$



Suppose that we have an isotropic subspace $V$ of $(k^{2n},\omega)$ with dimension $n-m$.

By applying Fourier transforms, we obtain a symplectomorphic subspace generated by a matrix whose pivots are all in the $X$ block.  Therefore, we can row reduce this matrix to obtain one of the following form:

$$
\left[\begin{array}{cc|cc}
I_{n-m} & X_B & Z_A & Z_B 
\end{array}\right]
$$

By applying controlled shift gates from the first $n-m$ wires to the last $m$ wires we obtain an isotropic subspace generated by a matrix of the following form:


$$
\left[\begin{array}{cc|cc}
I_{n-m} & 0 & Z_A' & Z_B' 
\end{array}\right]
$$

Since all of the rows of this subspace are orthogonal with respect to the symplectic form, we have:

\begin{align*}
0 &=
\left[\begin{array}{cc|cc}
I_{n-m} & 0 & Z_A' & Z_B' 
\end{array}\right]
\omega
\left[\begin{array}{cc|cc}
I_{n-m} & 0 & Z_A' & Z_B' 
\end{array}\right]^T\\
&=
\left[\begin{array}{cc|cc}
I_{n-m} & 0 & Z_A' & Z_B' 
\end{array}\right]
\left[\begin{array}{cc|cc}
 -Z_A' & -Z_B'  & I_{n-m} & 0
\end{array}\right]^T\\
&=
I_{n-m}(-Z_A')^T +  0( -Z_B' )^T +Z_A'I_{n-m} + Z_B' 0 \\
&=
(-Z_A')^T +Z_A'
\end{align*}
Which holds if and only if $Z_A'=(Z_A')^T$ is symmetric.

Therefore, the following matrix generates a Lagrangian subspace of $k^{2(n+1)}$ because the $X$ block is the identity and the $Z$ block is symmetric:
$$
\left[\begin{array}{ccc|ccc}
I_{n-m} & 0    & 0 & Z_A'       & Z_B' & 0\\
0           & I_m & 0 & (Z_B')^T & 0     & v \\
0           & 0    & 1  & 0            & v^T & 0
\end{array}\right]
$$
Where $v$ is the $m$-dimensional vector with only $1$s.
Let $W$ be the Lagrangian subspace generated by this matrix.  Then
$$
\begin{tikzpicture}
	\begin{pgfonlayer}{nodelayer}
		\node [style=X] (0) at (0.5, 0.5) {};
		\node [style=X] (1) at (1.5, 0.5) {};
		\node [style=map] (8) at (0.75, -0.5) {$W$};
		\node [style=none] (9) at (1, 0.5) {};
		\node [style=none] (10) at (0, 0.5) {};
		\node [style=none] (11) at (0, 1) {};
		\node [style=none] (12) at (1, 1) {};
	\end{pgfonlayer}
	\begin{pgfonlayer}{edgelayer}
		\draw [bend left, looseness=0.75] (8) to (10.center);
		\draw [bend left=15] (8) to (0);
		\draw [bend right=15] (8) to (9.center);
		\draw [bend right, looseness=0.75] (8) to (1);
		\draw (11.center) to (10.center);
		\draw (9.center) to (12.center);
	\end{pgfonlayer}
\end{tikzpicture}
=
\begin{tikzpicture}
	\begin{pgfonlayer}{nodelayer}
		\node [style=map] (15) at (3.25, -0.5) {$V$};
		\node [style=none] (16) at (3.5, 0.25) {};
		\node [style=none] (17) at (3, 0.25) {};
	\end{pgfonlayer}
	\begin{pgfonlayer}{edgelayer}
		\draw [bend left=15, looseness=0.75] (15) to (17.center);
		\draw [bend right=15, looseness=0.75] (15) to (16.center);
	\end{pgfonlayer}
\end{tikzpicture}
$$
This follows because composing $W$ with the cozero maps on the last wire of the $Z$ and $X$ blocks picks out the columns where the last row of the $X$ and $Z$ blocks are both $0$; that is, those of the generator matrix of $V$.


\end{proof}

Since, the orthogonal complement reverses the order of inclusions, it extends to an isomorphism $\Co\Isot\Rel_k\to \Isot\Rel_k$ so that:


\begin{corollary}
The prop $\Co\Isot\Rel_k$ is generated by adding the doubled discard relation to the image of the embedding $\Lag\Rel_k\to\LinRel_k$, ie. the following generator in $\ih_k$:
$
\begin{tikzpicture}
	\begin{pgfonlayer}{nodelayer}
		\node [style=Z] (0) at (0, 0) {};
		\node [style=Z] (1) at (0.5, 0) {};
		\node [style=none] (2) at (0, 0.5) {};
		\node [style=none] (3) at (0.5, 0.5) {};
	\end{pgfonlayer}
	\begin{pgfonlayer}{edgelayer}
		\draw (1.center) to (3.center);
		\draw (0.center) to (2.center);
	\end{pgfonlayer}
\end{tikzpicture}
$

\end{corollary}


From the same argument that yields $\Aff\Lag\Rel_k$ from $\Lag\Rel_k$:

\begin{lemma}
The props $\Aff\Isot\Rel_k$ and $\Aff\Co\Isot\Rel_k$ are generated by adding the generator $X$ to the respective images of the embeddings $\Isot\Rel_k\to\LinRel_k$ and $\Co\Isot\Rel_k\to\LinRel_k$.
\end{lemma}

Notice that unlike in the linear case, these two props are not isomorphic, as the doubled discard and  doubled cozero maps interact differently with the $X$ gate.  For example:

$$
\begin{tikzpicture}
	\begin{pgfonlayer}{nodelayer}
		\node [style=X] (0) at (1.75, -0.75) {$1$};
		\node [style=Z] (3) at (1.75, 0) {};
		\node [style=Z] (4) at (0.75, 0) {};
		\node [style=none] (9) at (1.75, -1.5) {};
		\node [style=none] (10) at (0.75, -1.5) {};
	\end{pgfonlayer}
	\begin{pgfonlayer}{edgelayer}
		\draw (0) to (9.center);
		\draw (0) to (3);
		\draw (10.center) to (4);
	\end{pgfonlayer}
\end{tikzpicture}
=
\begin{tikzpicture}
	\begin{pgfonlayer}{nodelayer}
		\node [style=Z] (3) at (1.75, 0) {};
		\node [style=Z] (4) at (0.75, 0) {};
		\node [style=none] (9) at (1.75, -1.5) {};
		\node [style=none] (10) at (0.75, -1.5) {};
	\end{pgfonlayer}
	\begin{pgfonlayer}{edgelayer}
		\draw (10.center) to (4);
		\draw (9.center) to (3);
	\end{pgfonlayer}
\end{tikzpicture}
\hspace*{.5cm}
\text{but}
\hspace*{.5cm}
\begin{tikzpicture}
	\begin{pgfonlayer}{nodelayer}
		\node [style=X] (0) at (1.75, -0.75) {$1$};
		\node [style=X] (3) at (1.75, 0) {};
		\node [style=X] (4) at (0.75, 0) {};
		\node [style=none] (9) at (1.75, -1.5) {};
		\node [style=none] (10) at (0.75, -1.5) {};
	\end{pgfonlayer}
	\begin{pgfonlayer}{edgelayer}
		\draw (0) to (9.center);
		\draw (0) to (3);
		\draw (10.center) to (4);
	\end{pgfonlayer}
\end{tikzpicture}
\neq
\begin{tikzpicture}
	\begin{pgfonlayer}{nodelayer}
		\node [style=X] (3) at (1.75, 0) {};
		\node [style=X] (4) at (0.75, 0) {};
		\node [style=none] (9) at (1.75, -1.5) {};
		\node [style=none] (10) at (0.75, -1.5) {};
	\end{pgfonlayer}
	\begin{pgfonlayer}{edgelayer}
		\draw (10.center) to (4);
		\draw (9.center) to (3);
	\end{pgfonlayer}
\end{tikzpicture}
$$



\begin{definition}
Given a field $k$ and natural number $n$, the {\bf generalized Pauli group} on $n$ wires, $P_k^n$ is generated by the following affine Lagrangian relations under tensor product, for $a,b \in k$:

$$
W(a,b):=
\begin{tikzpicture}
	\begin{pgfonlayer}{nodelayer}
		\node [style=X] (0) at (0, 1) {$a$};
		\node [style=none] (1) at (0, 2) {};
		\node [style=none] (2) at (0, 0) {};
		\node [style=none] (3) at (1, 2) {};
		\node [style=none] (4) at (1, 0) {};
		\node [style=X] (5) at (1, 1) {$b$};
	\end{pgfonlayer}
	\begin{pgfonlayer}{edgelayer}
		\draw (4.center) to (5.center);
		\draw (5.center) to (3.center);
		\draw (1.center) to (0.center);
		\draw (0.center) to (2.center);
	\end{pgfonlayer}
\end{tikzpicture}
$$

Just as in the quantum case, call elements of the generalized Pauli group {\bf Weyl operators}.
Given a tuple $((a_1,\cdots, a_n),(b_1,\cdots, b_n)) \in (k^2)^n$ we can similarly define a Weyl operator $W(a_1,\cdots, a_n),(b_1,\cdots, b_n) = W(a_1,b_1)\oplus \cdots \oplus W(a_n,b_n)$.
\end{definition}


\begin{definition}
Given some state $f:0\to n$ in $\CPM(\Aff\Lag\Rel_k)$, the generalized {\bf stabilizer group} of $f$ is the subgroup of the generalized Pauli group generated by the Weyl operators $a \in P_k^n$ so that $U(f);(1_n\otimes a) = U(f)$, where $U:\CPM(\Aff\Lag\Rel_k) \to \Aff\Lag\Rel_k$ is the forgetful functor.
\end{definition}


\begin{lemma}
Take an affine Lagrangian relation  $f:0\to n +m$.  Then a Weyl operator $x$ in the stabilizer group of $\CPM(f);(1_n \oplus !_m)$ if and only if $x \oplus 1_m$ it is in the stabilizer group of $f$.
\end{lemma}

\begin{proof}
Take some state $f$ as above.  If $\CPM(f);(1_n \oplus !_m)$ has no stabilizer group (because it is empty), then the claim follows vacuously. 
Take some  $W(a,b) \in P_k^n$ in the stabilizer group of $\CPM(f);(1_n \oplus !_m)$.
Then there exists some $W(c,d) \in P_k^m$ such that $f;(W(a,b)\oplus 1_m) = f;(1_n \oplus W(c,d))$.


Let $d_i$ be the cozero relation on $i$ wires.
%There exists some element $c$  in  $f; (d_n \oplus 1_m) $.
%Because $c$ is a linear relation, there exists $((e,f) \in k^2)^m$ such that $c^T =W(e,f);d_n$.

Pick some element $((x_A,x_B),(z_A,z_B))$ in $f$ so that $f;W((x_A,x_B),(z_A,z_B)));d_{n+m}=1_0$.
%Pick a stabilizer $W(a,b)$ of $f$.


Take $U:\CPM(\Aff\Lag\Rel_k) \to \Aff\Lag\Rel_k$ to be the forgetful functor.  Then

\begin{align*}
1 &=U( \CPM(f);(1_n \oplus !_m); \CPM(W(x_A,z_A);d_{n}) )\\
   &=U( \CPM(f);(1_n \oplus !_m); (1_n\oplus W(a,b)) ; \CPM(W(x_A,z_A);d_{n}) )\\
   &=U( \CPM(f);(1_n \oplus !_m); (1_n\oplus W(a,b)) ; \CPM(W(x_A,z_A);d_{n}) )\\
   &=  (\bar f \oplus f);(1_n \oplus (1\oplus W(c,d )  ;\eta_m) \oplus 1_n ); \bar {W(x_A,z_A)};d_{n} \oplus W(x_A,z_A);d_{n}\\
   &=  d_m^T;W(x_B,-z_B)\oplus d_m^T;W(x_B,z_B)   ; (1\oplus W(c,d) )  ;\eta_m\\
   &= d_m^T;W(x_B,-z_B);W(c,d);W(-x_B,z_B);d_m\\
   &= d_m^T;W(c,d);d_m\\
\end{align*}

Which is true if and only if $W(c,d)=1$.

\end{proof}


The following proposition is the symplectic version of the essential uniquess of dilation:
\begin{proposition}
States in $\CPM(\Aff\Lag\Rel_k)$ are uniquely determined by their stabilizer groups.
\end{proposition}

\begin{proof}
By construction, the stabilizer group of a state is unique.

Pick two  affine Lagrangian relations $f:0\to n+m$, $g:0\to n+\ell$ so that $\CPM(f);(1_n \oplus !_m)$ and $\CPM(g);(1_n \oplus !_\ell)$ have then same stabilizer group.  Then from the previous lemma,

\begin{align*}
  f; (x \oplus 1_m) = f
 \iff &
U(\CPM(f);(1_n \oplus !_m)); (1_n \oplus x) = U(\CPM(f);(1_n \oplus !_m))\\
\iff & U(\CPM(g);(1_n \oplus !_\ell)); (1_n \oplus x) = U(\CPM(g);(1_n \oplus !_\ell))\\
\iff & g; (x \oplus 1_\ell) = g
\end{align*}



Without loss of generality take $m\geq \ell$.
%Then there are isometries $u:m\to m$, $v:\ell \to m$ such that
Then there are unitaries $u,w$ on $m,\ell$ wires so that
$$f;(1_n\oplus u) = g; (1_n\oplus v \oplus 1_{m-\ell})$$

And thus an isometry $v:\ell\to m$ so that
$$f;(1_n\oplus u) = g; (1_n\oplus v)$$

Therefore,

$$
 \CPM (f);(1_n \oplus !_m) 
=\CPM (f;(1_n\oplus u) );(1_n \oplus !_m) 
=\CPM (g;(1\oplus v) );(1_n \oplus !_m)
= \CPM (g);(1_n \oplus !_\ell) 
$$
\end{proof}




\begin{theorem}
 $\CPM(\Aff\Lag\Rel_k) \cong \Aff\Co\Isot\Rel_k$
\end{theorem}


\begin{proof}
Because both categories are compact closed, it suffices to exhibit a bijection between the states of both categories.
%
%
%First note that $\Aff\Lag\Rel_k\to\CPM(\Aff\Lag\Rel_k)$ is faithful, so that when there are two $f,g:0\to n$ in $\Aff\Lag\Rel_k$,  such that for for all Weyl operators $x$,
%
%$$\CPM(f);(1_n \oplus x) = \CPM(f) \iff \CPM(g);(1_n \oplus x) = \CPM(g)$$
%
%Then we can conclude  that $\CPM(f) = \CPM(g)$.
%That is to say, that the pure states in $\CPM(\Aff\Lag\Rel_k)$ are determined by their stabilizer groups.
%We  show the same for mixed states.
%
%Suppose that we have two affine Lagrangian relations $f:0\to n +m$ and $g:0\to n+\ell$ such that for any Weyl operator $x$ on $n$:
%
%$$U(\CPM(f);(1_n \oplus !_m)); (1_n \oplus x) = U(\CPM(f);(1_n \oplus !_m))
%\iff
%U(\CPM(g);(1_n \oplus !_\ell)); (1_n \oplus x) = U(\CPM(g);(1_n \oplus !_\ell))$$
%Where  $U:\CPM(\Aff\Lag\Rel_k)\to \Aff\Lag\Rel_k$ is the forgetful functor and $!_n$ is the discard map on $n$ wires.
%
%We seek to show that $ \CPM(f);(1_n \oplus !_m)= \CPM(g);(1_n \oplus !_\ell) $.
%
%First, we show for a Weyl operator $x$:
%$$
%f; (x \oplus 1_m) = f
%\iff
%U(\CPM(f);(1_n \oplus !_m)); (1_n \oplus x) = U(\CPM(f);(1_n \oplus !_m))
%$$
%
%The foreward implication is immediate
%
%
%%We need this lemma
%%%For any stabilizer state $f:0\to n+m$, and Weyl operator $x$ on $n$ qudits, 
%%%%Either there exists some Weyl operator $y$ on $m$ qudits so that $f;(x\oplus 1)=f;(1\oplus y)$
%%%%Or f;(x\oplus 1)=0
%%This follows from the following lemma
%%%For any affine relation $f:0\to n+m$, either $f;(|x>|y>) = 1$ or it is zero.
%
%
%%For the converse direction TODO, follows from quantum literature on mixed stabilizer states, projectors
%%Suppose (x,z) is a stabilizer of f
%%If (x,z) is a stabilizer of \bar f, show that z=z^{-1}, meaning that z=1
%
%
%For the converse direction, suppose that there is a Weyl operator $x$ on $n$ qudits such that:
%$$U(\CPM(f);(1_n \oplus !_m)); (1_n \oplus x) = U(\CPM(f);(1_n \oplus !_m))$$
%
%If $f$ is empty, then the claim follows immediately.  Suppose otherwise, that it is nonempty.
%Then there exists another Weyl operator $y$ on $m$ qudits such that $f;(1_n \oplus x) = f;(y\oplus 1_m)$.
%Therefore, %$\bar f ; (y \oplus 1_m)$, so that:
%
%$$
%U(\CPM(f);(1_n \oplus !_m))=
%U(\CPM(f);(1_n \oplus !_m)); (1_n \oplus x)=
%(\bar f \oplus f);(1_n \oplus ((1\oplus y )  ;\eta_m) \oplus 1_n )
%$$
%
%%Let $z: n \to 0$ denote the cozero relation and $\eta_m$ the counit of the compact closed structure.
%%Because $\bar x \oplus x$  is a stabilizer, we know that $\bar z \oplus z$ is inside this linear subspace so that 
%For $a,b \in k^n$, let $W(a,b)=(X^{\otimes n})^a;(Z^{\otimes n})^b$ and let $h$ denote the cozero relation.
%
%Then there are some $a,b,c,d \in k$ such that $W(a,b)=x$ and $W(c,d)=y$.  Thus,
%
%
%%Then we have 
%% Take some $t \in f;(1_ \oplus (x;d))$.
%% Then there are some $e,f \in k$ such that $W(e,f);d = t^T$
%
%%%%%PROOF IS WRONG, USE ONE IN TITOUANES PAPER
%%
%%\begin{align*}
%% 1=&  (\bar f \oplus f);(1_n \oplus ((1\oplus y )  ;\eta_m) \oplus 1_n ); \bar {W(a,b);h} \oplus {W(a,b)};h\\
%%   =&  (\bar f \oplus f);(1_n \oplus ((1\oplus y )  ;\eta_m) \oplus 1_n );  W(a,-b);h \oplus {W(a,b)};h\\
%%   =&h^T;W(-a,-b);W(c,d);W(a,b);h\\
%%   =&h^T;W(c,d);h
%%\end{align*}
%
%We have that
%
%$$
%1=  (\bar f \oplus f);(1_n \oplus ((1\oplus y )  ;\eta_m) \oplus 1_n ); \bar {W(a,b);h} \oplus {W(a,b)};h
%$$
%
%Thus since $(W(e,f);h)^T$ is in the subspace $f (1\oplus W(a,b) ;h)$, it follows that:
%
%
%\begin{align*}
%1  =& h^T;W(-e,-b);W(c,d);W(e,f);h\\
%   =&h^T;W(c,d);h
%\end{align*}
%
%
%Which is true if and only if $y=W(c,d)=1$, otherwise, this would evaluate to the empty relation.
%%% \bar{W(a,b)} = W(a,-b)
%%
%%\begin{align*}
%% &1=  (\bar f \oplus f);(1_n \oplus ((1\oplus y )  ;\eta_m) \oplus 1_n ); \bar z \oplus z\\
%%&\iff    1 = z^\dag;y;z\\
%%&\iff    1=y
%%\end{align*}
%Therefore, the reverse implication holds.
%
%
%%%%%%%%%%%%%%%%%%%%%%%%%%%%%%%%%%%%%%%%%%%%%%%%
%% conjugation takes phases  weyl operators (n,m) -> (-n,-m)                                          %
%% Therefore, weyl operators should commute with the cap, because they will cancel     %
%%%%%%%%%%%%%%%%%%%%%%%%%%%%%%%%%%%%%%%%%%%%%%%%
%
%By substituting $f$ with $g$ it follows that:
%Given affine Lagrangian relations $f:0\to n$, $g:0\to n$, suppose that $\CPM(f);(1_n \oplus !_m))$ and $\CPM(f);(1_n \oplus !_\ell))$ have then same stabilizer groups.  Then
%
%\begin{align*}
%  f; (x \oplus 1_m) = f
% \iff &
%U(\CPM(f);(1_n \oplus !_m)); (1_n \oplus x) = U(\CPM(f);(1_n \oplus !_m))\\
%\iff & U(\CPM(g);(1_n \oplus !_\ell)); (1_n \oplus x) = U(\CPM(g);(1_n \oplus !_\ell))\\
%\iff & f; (x \oplus 1_\ell) = g
%\end{align*}
%
%
%Without loss of generality take $m\geq \ell$.
%Then there are isometries $u:m\to m$, $v:\ell \to m$ such that
%$$f(1_n\oplus u) = g (1_n\oplus v)$$
%
%Therefore,
%
%$$
% \CPM (f);(1_n \oplus !_m) 
%=\CPM (f;(1_n\oplus u) );(1_n \oplus !_m) 
%=\CPM (g;(1\oplus v) );(1_n \oplus !_m)
%= \CPM (g);(1_n \oplus !_\ell) 
%$$

%Which proves the initial biequivalence.


%Define stabilizers for al
We already know that affine coisotropic Lagrangian subspaces are determined by their stabilizers.
Therefore it suffices to show that the stabilizers in $\CPM(\Aff\Lag\Rel_k)$ and  $\Aff\Co\Isot\Rel_k$ agree; that is:


$$f;(x \oplus d_m)=f;(1_n \oplus d_m) \iff U(\CPM(f);(1_n \oplus !_m)); (1_n \oplus x) = U(\CPM(f);(1_n \oplus !_m)) $$


If $f$ is empty, then the claim follows immediately.  Suppose otherwise.
The forward direction follows immediately from the fact that $x$ is unitary.
Conversely, we know there is some Weyl operator $y$ such that  
$f;(x\oplus 1) =  f;(1\oplus y)$
but $f;(x \oplus d_m) = f;(1_n \oplus (y;d_m)) = f;(1_n\oplus d_m)$

 
\end{proof}


\begin{corollary}
For odd prime $p$, $\Aff\Co\Isot\Rel_{\F_p}\cong \CPM(\Aff\Lag\Rel_{\F_p})\cong \CPM(\Stab_p)$, that is, mixed stabilizer circuits modulo invertible scalars.
\end{corollary}


\begin{corollary}
$\Aff\Co\Isot\Rel_{\F_2}\cong \CPM(\Aff\Lag\Rel_{\F_2})\cong \CPM(\Stab_2)$, that is, Spekkens' toy model with mixed states.
\end{corollary}


\begin{corollary}
For a prime number $p$, $\Isot\Rel_{\F_p}\cong\Co\Isot\Rel_{\F_p}\cong \CPM(\CPM(\LinRel_{\F_p}))$.
\end{corollary}


In the quantum setting, by connecting the discard map to a spider, one obtains a circuit which decoheres the state into a basis.  In stabilizer circuits, discarding a white spider decoheres in the $X$ basis, and discarding in the black spider decoheres in the $Z$ basis.  We use this intuition two define to maps in the general coisotropic setting:

\begin{definition}
The $X$ and $Z$ decoherence maps are defined as follows in $\Aff\Co\Isot\Rel_{\F_p}$:
$$
p_X:=
\begin{tikzpicture}
	\begin{pgfonlayer}{nodelayer}
		\node [style=X] (0) at (0.5, -0.75) {};
		\node [style=none] (2) at (0.25, 0) {};
		\node [style=none] (4) at (1.25, 0.5) {};
		\node [style=Z] (5) at (0.75, 0) {};
		\node [style=Z] (6) at (1.75, 0) {};
		\node [style=Z] (7) at (1.5, -0.75) {};
		\node [style=none] (8) at (0.75, 0) {};
		\node [style=none] (9) at (1.25, 0) {};
		\node [style=none] (10) at (1.75, 0) {};
		\node [style=none] (11) at (0.5, -1.5) {};
		\node [style=none] (13) at (1.5, -1.5) {};
		\node [style=none] (14) at (0.25, 0.5) {};
	\end{pgfonlayer}
	\begin{pgfonlayer}{edgelayer}
		\draw [in=-90, out=120] (7) to (9.center);
		\draw (7) to (13.center);
		\draw [in=60, out=-90] (10.center) to (7);
		\draw [in=60, out=-90] (8.center) to (0);
		\draw [in=-90, out=120] (0) to (2.center);
		\draw (0) to (11.center);
		\draw (9.center) to (4.center);
		\draw (2.center) to (14.center);
	\end{pgfonlayer}
\end{tikzpicture}
=
\begin{tikzpicture}
	\begin{pgfonlayer}{nodelayer}
		\node [style=none] (4) at (1, 0.5) {};
		\node [style=Z] (5) at (0.25, 0) {};
		\node [style=none] (9) at (1, -1.25) {};
		\node [style=none] (11) at (0.25, -1.25) {};
		\node [style=none] (14) at (0.25, 0.5) {};
		\node [style=Z] (15) at (0.25, -0.75) {};
	\end{pgfonlayer}
	\begin{pgfonlayer}{edgelayer}
		\draw (9.center) to (4.center);
		\draw (14.center) to (5);
		\draw (11.center) to (15);
	\end{pgfonlayer}
\end{tikzpicture}
\hspace*{.5cm}
p_Z:=
\begin{tikzpicture}
	\begin{pgfonlayer}{nodelayer}
		\node [style=Z] (0) at (0.5, -0.75) {};
		\node [style=none] (2) at (0.25, 0) {};
		\node [style=none] (4) at (1.25, 0.5) {};
		\node [style=Z] (5) at (0.75, 0) {};
		\node [style=Z] (6) at (1.75, 0) {};
		\node [style=X] (7) at (1.5, -0.75) {};
		\node [style=none] (8) at (0.75, 0) {};
		\node [style=none] (9) at (1.25, 0) {};
		\node [style=none] (10) at (1.75, 0) {};
		\node [style=none] (11) at (0.5, -1.5) {};
		\node [style=none] (13) at (1.5, -1.5) {};
		\node [style=none] (14) at (0.25, 0.5) {};
	\end{pgfonlayer}
	\begin{pgfonlayer}{edgelayer}
		\draw [in=-90, out=120] (7) to (9.center);
		\draw (7) to (13.center);
		\draw [in=60, out=-90] (10.center) to (7);
		\draw [in=60, out=-90] (8.center) to (0);
		\draw [in=-90, out=120] (0) to (2.center);
		\draw (0) to (11.center);
		\draw (9.center) to (4.center);
		\draw (2.center) to (14.center);
	\end{pgfonlayer}
\end{tikzpicture}
=
\begin{tikzpicture}[scale=-1]
	\begin{pgfonlayer}{nodelayer}
		\node [style=none] (4) at (1, 0.5) {};
		\node [style=Z] (5) at (0.25, 0) {};
		\node [style=none] (9) at (1, -1.25) {};
		\node [style=none] (11) at (0.25, -1.25) {};
		\node [style=none] (14) at (0.25, 0.5) {};
		\node [style=Z] (15) at (0.25, -0.75) {};
	\end{pgfonlayer}
	\begin{pgfonlayer}{edgelayer}
		\draw (9.center) to (4.center);
		\draw (14.center) to (5);
		\draw (11.center) to (15);
	\end{pgfonlayer}
\end{tikzpicture}
$$
\end{definition}

%By splitting these idempotents, we can obtain a two-sorted semantics for which classical and quantum datum live together.
 
\begin{definition}
%Given a monoidal category $\C$ and a set of idempotents $P$ in $\C$ define the category $k_X(\C)$ to be the monoidal category generated by splitting the idempotents in $X$.
%
%Consider the prop $\Aff\Co\Isot\Rel_k^M:=k_{\{p_Z\}}(\Aff\Co\Isot\Rel_k)$.  Let $Q:=(1,1)$ $C:=(p_z,1)$ denote the generators for the multicoloured prop.
Let $\Aff\Co\Isot\Rel_k^M$ denote the two coloured prop generated by splitting $p_Z$ in $\Aff\Co\Isot\Rel_k$.
Let $Q$ denote the original generating object and $C$ the object obtained by splitting $p_Z$.
\end{definition}

We could have instead split $p_X$, or split both $p_Z$ and $p_Z$; however, all three of these multicoloured props are equivalent.  This equivalence is witnessed via the Fourier transform.

The idea here is that  in the quantum setting, the object $Q$ is interpreted as a quantum channel and the object $C$ as a classical channel.

\begin{lemma}
The full subcategory of $\Aff\Co\Isot\Rel_k^M$ generated by tensor powers of $C$ is isomorphic to $\Aff\Rel_k$.
\end{lemma}


We can see this category more concretely in terms of being generated by certain linear relations:

\begin{theorem}
$\Aff\Co\Isot\Rel_k^M$ is isomorphic to adding the following linear relations to the image of $\Aff\Co\Isot\Rel_k$ in the way which makes this into a two-coloured prop:
$$
\begin{tikzpicture}
	\begin{pgfonlayer}{nodelayer}
		\node [style=none] (4) at (0.25, 0.5) {};
		\node [style=Z] (5) at (1, 0) {};
		\node [style=none] (9) at (0.25, -0.5) {};
		\node [style=none] (14) at (1, 0.5) {};
	\end{pgfonlayer}
	\begin{pgfonlayer}{edgelayer}
		\draw (9.center) to (4.center);
		\draw (14.center) to (5);
	\end{pgfonlayer}
\end{tikzpicture}
\hspace*{.5cm}\text{and}\hspace*{.5cm}
\begin{tikzpicture}[yscale=-1]
	\begin{pgfonlayer}{nodelayer}
		\node [style=none] (4) at (0.25, 0.5) {};
		\node [style=Z] (5) at (1, 0) {};
		\node [style=none] (9) at (0.25, -0.5) {};
		\node [style=none] (14) at (1, 0.5) {};
	\end{pgfonlayer}
	\begin{pgfonlayer}{edgelayer}
		\draw (9.center) to (4.center);
		\draw (14.center) to (5);
	\end{pgfonlayer}
\end{tikzpicture}
$$
\end{theorem}
\begin{proof}
Suppose we have a morphism $C^{\oplus n} \to Q^{\oplus m}$ witnessed by two maps $f,g$
\end{proof}

This is somewhat counterintutitive, because we do not actually need to double wires to accomodate for classical and quantum information at the same time; rather, we need to halve it.


These morphisms are interpreted in terms of state preparation and measurement in the $X$ basis.
The state preparation and discarding in the $Z$ basis are obtained by composition of these morphisms with the Fourier transform; yielding morphisms which dicard the $X$ wire instead of the $Z$ wire:

$$
\begin{tikzpicture}[xscale=-1]
	\begin{pgfonlayer}{nodelayer}
		\node [style=none] (4) at (0.25, 0.5) {};
		\node [style=Z] (5) at (1, 0) {};
		\node [style=none] (9) at (0.25, -0.5) {};
		\node [style=none] (14) at (1, 0.5) {};
	\end{pgfonlayer}
	\begin{pgfonlayer}{edgelayer}
		\draw (9.center) to (4.center);
		\draw (14.center) to (5);
	\end{pgfonlayer}
\end{tikzpicture}
\hspace*{.5cm}\text{and}\hspace*{.5cm}
\begin{tikzpicture}[scale=-1]
	\begin{pgfonlayer}{nodelayer}
		\node [style=none] (4) at (0.25, 0.5) {};
		\node [style=Z] (5) at (1, 0) {};
		\node [style=none] (9) at (0.25, -0.5) {};
		\node [style=none] (14) at (1, 0.5) {};
	\end{pgfonlayer}
	\begin{pgfonlayer}{edgelayer}
		\draw (9.center) to (4.center);
		\draw (14.center) to (5);
	\end{pgfonlayer}
\end{tikzpicture}
$$


\begin{remark}
In $\FHilb$, the state preparation and measurement in the $X$ and $Z$ bases are given by the following ZX calculus diagrams:

$$
\begin{tikzpicture}
	\begin{pgfonlayer}{nodelayer}
		\node [style=Z] (182) at (49, 6) {};
		\node [style=none] (183) at (48.75, 7) {};
		\node [style=Z] (184) at (49.5, 6.75) {};
		\node [style=none] (185) at (49, 5.25) {};
		\node [style=none] (186) at (49.75, 5.25) {};
	\end{pgfonlayer}
	\begin{pgfonlayer}{edgelayer}
		\draw [in=120, out=-90] (183.center) to (182);
		\draw (182) to (184);
		\draw (185.center) to (182);
		\draw [in=-60, out=90, looseness=0.75] (186.center) to (184);
	\end{pgfonlayer}
\end{tikzpicture}
\hspace*{.5cm}
\begin{tikzpicture}[yscale=-1]
	\begin{pgfonlayer}{nodelayer}
		\node [style=Z] (182) at (49, 6) {};
		\node [style=none] (183) at (48.75, 7) {};
		\node [style=Z] (184) at (49.5, 6.75) {};
		\node [style=none] (185) at (49, 5.25) {};
		\node [style=none] (186) at (49.75, 5.25) {};
	\end{pgfonlayer}
	\begin{pgfonlayer}{edgelayer}
		\draw [in=120, out=-90] (183.center) to (182);
		\draw (182) to (184);
		\draw (185.center) to (182);
		\draw [in=-60, out=90, looseness=0.75] (186.center) to (184);
	\end{pgfonlayer}
\end{tikzpicture}
\hspace*{.5cm}
\begin{tikzpicture}
	\begin{pgfonlayer}{nodelayer}
		\node [style=none] (188) at (50.75, 7) {};
		\node [style=Z] (189) at (51.5, 6.75) {};
		\node [style=none] (190) at (51, 5.25) {};
		\node [style=none] (191) at (51.75, 5.25) {};
		\node [style=X] (192) at (51, 6) {};
	\end{pgfonlayer}
	\begin{pgfonlayer}{edgelayer}
		\draw [in=-60, out=90, looseness=0.75] (191.center) to (189);
		\draw (192) to (190.center);
		\draw [in=-90, out=120] (192) to (188.center);
		\draw (192) to (189);
	\end{pgfonlayer}
\end{tikzpicture}
\hspace*{.5cm}
\begin{tikzpicture}[yscale=-1]
	\begin{pgfonlayer}{nodelayer}
		\node [style=none] (188) at (50.75, 7) {};
		\node [style=Z] (189) at (51.5, 6.75) {};
		\node [style=none] (190) at (51, 5.25) {};
		\node [style=none] (191) at (51.75, 5.25) {};
		\node [style=X] (192) at (51, 6) {};
	\end{pgfonlayer}
	\begin{pgfonlayer}{edgelayer}
		\draw [in=-60, out=90, looseness=0.75] (191.center) to (189);
		\draw (192) to (190.center);
		\draw [in=-90, out=120] (192) to (188.center);
		\draw (192) to (189);
	\end{pgfonlayer}
\end{tikzpicture}
$$

The strong complementarity of the $X$ and $Z$ variables in this setting follows from the fact that their corresponding Frobenius algebras interact to form a hopf algebra. However, in the symplectic picture, this result is purely topological:

$$
\begin{tikzpicture}
	\begin{pgfonlayer}{nodelayer}
		\node [style=Z] (285) at (77.25, 6) {};
		\node [style=none] (286) at (77, 7) {};
		\node [style=Z] (287) at (77.75, 6.75) {};
		\node [style=none] (288) at (77, 4.5) {};
		\node [style=Z] (289) at (77.75, 4.75) {};
		\node [style=X] (290) at (77.25, 5.5) {};
	\end{pgfonlayer}
	\begin{pgfonlayer}{edgelayer}
		\draw [in=120, out=-90] (286.center) to (285);
		\draw (285) to (287);
		\draw [in=90, out=-120] (290) to (288.center);
		\draw (290) to (289);
		\draw (290) to (285);
		\draw [bend left] (287) to (289);
	\end{pgfonlayer}
\end{tikzpicture}
=
\begin{tikzpicture}
	\begin{pgfonlayer}{nodelayer}
		\node [style=Z] (291) at (78.75, 6.5) {};
		\node [style=none] (292) at (78.75, 7) {};
		\node [style=none] (293) at (78.75, 4.5) {};
		\node [style=X] (294) at (78.75, 5) {};
		\node [style=s] (295) at (79, 5.75) {};
	\end{pgfonlayer}
	\begin{pgfonlayer}{edgelayer}
		\draw (292.center) to (291);
		\draw (294) to (293.center);
		\draw [bend left] (294) to (291);
		\draw [in=90, out=-60] (291) to (295);
		\draw [in=-90, out=60] (294) to (295);
	\end{pgfonlayer}
\end{tikzpicture}
=
\begin{tikzpicture}
	\begin{pgfonlayer}{nodelayer}
		\node [style=Z] (297) at (79.75, 6) {};
		\node [style=none] (298) at (79.75, 7) {};
		\node [style=none] (299) at (79.75, 4.5) {};
		\node [style=X] (300) at (79.75, 5.5) {};
	\end{pgfonlayer}
	\begin{pgfonlayer}{edgelayer}
		\draw (298.center) to (297);
		\draw (300) to (299.center);
	\end{pgfonlayer}
\end{tikzpicture}
\hspace*{2cm}
\begin{tikzpicture}
	\begin{pgfonlayer}{nodelayer}
		\node [style=Z] (219) at (59, 6.25) {};
		\node [style=none] (220) at (59, 7.5) {};
		\node [style=Z] (221) at (59.5, 7) {};
		\node [style=none] (222) at (59.5, 5.75) {};
	\end{pgfonlayer}
	\begin{pgfonlayer}{edgelayer}
		\draw (220.center) to (219);
		\draw (222.center) to (221);
	\end{pgfonlayer}
\end{tikzpicture}
=
\begin{tikzpicture}
	\begin{pgfonlayer}{nodelayer}
		\node [style=Z] (219) at (59, 6.25) {};
		\node [style=none] (220) at (59, 6.75) {};
		\node [style=Z] (221) at (59, 5.75) {};
		\node [style=none] (222) at (59, 5.25) {};
	\end{pgfonlayer}
	\begin{pgfonlayer}{edgelayer}
		\draw (220.center) to (219);
		\draw (222.center) to (221);
	\end{pgfonlayer}
\end{tikzpicture}
$$
\end{remark}

%This presentation of the Karoubi envelope is actually used in the literature, although not in such formal terms.  For example in the following notes, these two generators are observed to correspond to state preparation and measurement in the $X$ basis in Spekkens' toy model \cite{aleksnotes}.
%Similarly, in the context of electrical circuits, this presentation allows one to describe conditional measurements, capturing the so called ``impedance calculus'' as a monoidal subcategory.



%
%\begin{corollary}
%The impedance calculus embeds in $\Aff\Co\Isot\Rel_{\mathbb{R}[x,y]/\langle xy=1\rangle}^M$.
%\end{corollary}
%\begin{proof}
%It is mechanical to translate the voltmeter, ammeter, controlled voltage and current sources into $\Aff\Co\Isot\Rel_{\mathbb{R}[x,y]/\langle xy=1\rangle}^M$.
%\end{proof}


For this reason, we can prove the correctness of the quantum teleportation algorithm using only spider fusion:


\begin{example}
The following circuit in $\Aff\Co\Isot\Rel_{\F_p}^M$ depicts a quantum teleportation protocol where Alice on the left teleports a qudit to Bob, on the right. They share an EPR pair (on the bottom of the diagram)  and two classical dits (drawn in red).  


$$
\begin{tikzpicture}
	\begin{pgfonlayer}{nodelayer}
		\node [style=none] (223) at (62.5, 1.75) {};
		\node [style=none] (224) at (63, 3.25) {};
		\node [style=none] (225) at (62, 3.25) {};
		\node [style=none] (226) at (61.5, 1.75) {};
		\node [style=Z] (227) at (63, 4.75) {};
		\node [style=Z] (228) at (61.5, 4.75) {};
		\node [style=none] (229) at (66.75, 3.25) {};
		\node [style=none] (230) at (67.25, 3.25) {};
		\node [style=none] (231) at (66.75, 9) {};
		\node [style=none] (232) at (67.25, 9) {};
		\node [style=Z] (233) at (61.5, 3.25) {};
		\node [style=X] (234) at (62, 3.75) {};
		\node [style=Z] (235) at (63, 3.75) {};
		\node [style=X] (236) at (62.5, 3.25) {};
		\node [style=none] (237) at (62, 4.75) {};
		\node [style=none] (238) at (62.5, 4.75) {};
		\node [style=X] (239) at (64, 2.25) {};
		\node [style=Z] (240) at (65.5, 2.25) {};
		\node [style=Z] (241) at (66.75, 7.75) {};
		\node [style=X] (242) at (67.25, 7.75) {};
		\node [style=X] (243) at (66.75, 8.5) {};
		\node [style=Z] (244) at (67.25, 8.5) {};
		\node [style=Z] (245) at (65.5, 6.75) {};
		\node [style=Z] (246) at (66, 6.75) {};
		\node [style=none] (247) at (66.5, 6.75) {};
		\node [style=none] (248) at (65, 6.75) {};
		\node [style=none] (249) at (64.5, 9) {};
		\node [style=none] (250) at (64.5, 1.75) {};
		\node [style=none] (251) at (63.5, 9) {Alice};
		\node [style=none] (252) at (65.5, 9) {Bob};
		\node [style=none] (254) at (61, 4.75) {};
		\node [style=none] (255) at (67.75, 4.75) {};
		\node [style=none] (256) at (61, 6.75) {};
		\node [style=none] (257) at (67.75, 6.75) {};
		\node [style=none] (258) at (59, 6.75) {Phase correction};
		\node [style=none] (259) at (59, 4.75) {Measurement};
	\end{pgfonlayer}
	\begin{pgfonlayer}{edgelayer}
		\draw (234) to (233);
		\draw (236) to (235);
		\draw (224.center) to (235);
		\draw (235) to (227);
		\draw (226.center) to (233);
		\draw (233) to (228);
		\draw (225.center) to (234);
		\draw (223.center) to (236);
		\draw (236) to (238.center);
		\draw (234) to (237.center);
		\draw [in=15, out=-90, looseness=0.75] (229.center) to (239);
		\draw [in=-90, out=165, looseness=0.50] (239) to (225.center);
		\draw [in=165, out=-90, looseness=0.50] (224.center) to (240);
		\draw [in=-90, out=15, looseness=0.75] (240) to (230.center);
		\draw (244) to (242);
		\draw (243) to (241);
		\draw (229.center) to (241);
		\draw (243) to (231.center);
		\draw (232.center) to (244);
		\draw (242) to (230.center);
		\draw [in=-150, out=90, looseness=0.75] (245) to (244);
		\draw [in=-150, out=90] (246) to (241);
		\draw [color=red, in=-90, out=90, looseness=0.50] (238.center) to (247.center);
		\draw [color=red, in=-90, out=90] (237.center) to (248.center);
		\draw [in=210, out=90, looseness=0.75] (248.center) to (243);
		\draw [in=-150, out=90] (247.center) to (242);
		\draw [style=dotted] (250.center) to (249.center);
		\draw [style=dotted] (255.center) to (254.center);
		\draw [style=dotted] (257.center) to (256.center);
	\end{pgfonlayer}
\end{tikzpicture}
=
\begin{tikzpicture}
	\begin{pgfonlayer}{nodelayer}
		\node [style=none] (301) at (81.75, 1.5) {};
		\node [style=none] (302) at (81.25, 3) {};
		\node [style=none] (303) at (80.75, 1.5) {};
		\node [style=none] (304) at (84.5, 3) {};
		\node [style=none] (305) at (85, 3) {};
		\node [style=none] (306) at (84.5, 8.5) {};
		\node [style=none] (307) at (85, 8.5) {};
		\node [style=X] (308) at (81.25, 4.5) {};
		\node [style=X] (309) at (82.75, 1.75) {};
		\node [style=Z] (310) at (83.75, 1.75) {};
		\node [style=X] (311) at (85, 7.25) {};
		\node [style=X] (312) at (84.5, 7.25) {};
		\node [style=Z] (313) at (83, 5) {};
		\node [style=X] (314) at (82, 4.5) {};
	\end{pgfonlayer}
	\begin{pgfonlayer}{edgelayer}
		\draw (302.center) to (308);
		\draw [in=15, out=-90, looseness=0.75] (304.center) to (309);
		\draw [in=-90, out=165, looseness=0.50] (309) to (302.center);
		\draw [in=-90, out=15, looseness=0.75] (310) to (305.center);
		\draw (312) to (306.center);
		\draw (311) to (305.center);
		\draw (311) to (307.center);
		\draw (304.center) to (312);
		\draw [in=90, out=-120] (308) to (303.center);
		\draw [color=red, in=225, out=90, looseness=0.75] (308) to (312);
		\draw [in=225, out=120, looseness=0.50, color=red] (314) to (311);
		\draw [in=15, out=-165] (313) to (314);
		\draw [in=270, out=90] (301.center) to (314);
		\draw [in=-60, out=135, looseness=0.75] (310) to (313);
	\end{pgfonlayer}
\end{tikzpicture}
=
\begin{tikzpicture}
	\begin{pgfonlayer}{nodelayer}
		\node [style=none] (166) at (42.5, 1.5) {};
		\node [style=none] (167) at (42, 1.5) {};
		\node [style=none] (168) at (43.5, 8.5) {};
		\node [style=none] (169) at (44, 8.5) {};
		\node [style=X] (170) at (42, 4.5) {};
		\node [style=X] (171) at (42.5, 3.5) {};
		\node [style=X] (172) at (44, 6.25) {};
		\node [style=X] (173) at (43.5, 7.25) {};
	\end{pgfonlayer}
	\begin{pgfonlayer}{edgelayer}
		\draw [in=-120, out=90, looseness=0.75] (166.center) to (171);
		\draw [in=270, out=60] (173) to (168.center);
		\draw [in=270, out=60, looseness=0.75] (172) to (169.center);
		\draw [color=red, bend left=15, looseness=0.50] (171) to (172);
		\draw [in=90, out=-120, looseness=0.50] (170) to (167.center);
		\draw [bend right=15, looseness=0.50] (171) to (172);
		\draw [bend left=15, looseness=0.50] (173) to (170);
		\draw [color=red, bend left=15, looseness=0.50] (170) to (173);
	\end{pgfonlayer}
\end{tikzpicture}
=
\begin{tikzpicture}
	\begin{pgfonlayer}{nodelayer}
		\node [style=none] (174) at (45.5, 1.5) {};
		\node [style=none] (175) at (45, 1.5) {};
		\node [style=none] (176) at (45, 8.5) {};
		\node [style=none] (177) at (45.5, 8.5) {};
	\end{pgfonlayer}
	\begin{pgfonlayer}{edgelayer}
		\draw (177.center) to (174.center);
		\draw (175.center) to (176.center);
	\end{pgfonlayer}
\end{tikzpicture}
$$

\end{example}

Because $\Aff\Co\Isot\Rel_{\F_p}^M$ is a subcategory of relations, we can interpret the structure of the 2-cells in terms of quantum information theory.

\begin{remark}
Take two odd-prime dimensional qudit stabilizer circuits $f,g$ interpreted as parallel maps in  $\Aff\Co\Isot\Rel_{\F_p}^M$.
Then $f$ is a coarse-graining of $g$ when $f \subset g$.
\end{remark}
%\begin{proof}
%Consider the general setting of relations between sets.  Take two relations $R,S \subset X$.  Then $R \subseteq S$ if and only if for all $x \in R$ then $x \in S$.  
%\end{proof}


For an extreme example, the identity circuit on a classical wire is contained within  the circuit obtained by preparing in the $Z$ basis and measuring in the $X$:

$$
\begin{tikzpicture}
	\begin{pgfonlayer}{nodelayer}
		\node [style=none] (220) at (59, 6.75) {};
		\node [style=none] (222) at (59, 5.25) {};
	\end{pgfonlayer}
	\begin{pgfonlayer}{edgelayer}
		\draw (220.center) to (222);
	\end{pgfonlayer}
\end{tikzpicture}
\subset
\begin{tikzpicture}
	\begin{pgfonlayer}{nodelayer}
		\node [style=Z] (219) at (59, 6.25) {};
		\node [style=none] (220) at (59, 6.75) {};
		\node [style=Z] (221) at (59, 5.75) {};
		\node [style=none] (222) at (59, 5.25) {};
	\end{pgfonlayer}
	\begin{pgfonlayer}{edgelayer}
		\draw (220.center) to (219);
		\draw (222.center) to (221);
	\end{pgfonlayer}
\end{tikzpicture}
=
\begin{tikzpicture}
	\begin{pgfonlayer}{nodelayer}
		\node [style=Z] (219) at (59, 6.25) {};
		\node [style=none] (220) at (59, 7.5) {};
		\node [style=Z] (221) at (59.5, 7) {};
		\node [style=none] (222) at (59.5, 5.75) {};
	\end{pgfonlayer}
	\begin{pgfonlayer}{edgelayer}
		\draw (220.center) to (219);
		\draw (222.center) to (221);
	\end{pgfonlayer}
\end{tikzpicture}
$$

This is because, given any input state, the circuit on the right hand side can produce any output state; however, the identity circuit forces the inputs to be the outputs.

Similarly, the identity on a quantum wire is a coarse graining of the decoherence map:
$$
\begin{tikzpicture}
	\begin{pgfonlayer}{nodelayer}
		\node [style=none] (261) at (69.25, 6.75) {};
		\node [style=none] (263) at (69.25, 5.25) {};
		\node [style=none] (264) at (69.75, 6.75) {};
		\node [style=none] (265) at (69.75, 5.25) {};
	\end{pgfonlayer}
	\begin{pgfonlayer}{edgelayer}
		\draw (265.center) to (264.center);
		\draw (263.center) to (261.center);
	\end{pgfonlayer}
\end{tikzpicture}
\subset
\begin{tikzpicture}
	\begin{pgfonlayer}{nodelayer}
		\node [style=Z] (260) at (69.25, 6.25) {};
		\node [style=none] (261) at (69.25, 6.75) {};
		\node [style=Z] (262) at (69.25, 5.75) {};
		\node [style=none] (263) at (69.25, 5.25) {};
		\node [style=none] (264) at (69.75, 6.75) {};
		\node [style=none] (265) at (69.75, 5.25) {};
	\end{pgfonlayer}
	\begin{pgfonlayer}{edgelayer}
		\draw (261.center) to (260);
		\draw (263.center) to (262);
		\draw (265.center) to (264.center);
	\end{pgfonlayer}
\end{tikzpicture}
$$

%This is an immediate consequence from the fact that $f \subset g$ in $\Rel$ if and only if $f$ is defined everywhere $


\subsection{Categorical semantics for quantum error correction}
Thus far, we have assumed that our stabilizer circuits are lossless; without the consideration of errors being introduced.  In this subsection, we use the symplectic formalism to give a categorical semantics for stabilizer circuits with stabilizer errors.  


Fix $\hat{\F_p}$ to be the field of fractions over the ring of polynomial functions over the countably infinite ring of polynomial functions over $\F_p$.  Concretely, this is constructed by formally adding inverses to the ring $\F_p[x_i]_{i \in \N}/\langle  x_i^p = x_i \rangle_{i \in \N}$.


We model stabilizer circuits with stabilizer noise as circuits in $\Aff\Co\Isot\Rel_{\hat{\F_p}}^M$.  The idea of working over the countably infinite ring of polynomial functions is to introduce variables to stand in for indeterminate errors.  For example, the gate $W(x_i,0)$ represents the possibility that a pauli $X$ error is introduced.

We must consider the field of fractions over the polynomial function ring so that we obtain a field.  Recall that in order for this formalism to work, it is required that the base ring be at least a field.

This choice of field allows us to model stabilizer errors with affine interdependencies.


We reproduce the result of \cite{???}
\begin{example}
The XXXX code corrects stabilizer errors with the property that XXXX.
\end{example}

We see that this can also correct for errors with 


For stabilizer errors with nonlinear dependencies, consider the forgetful functor...  Add and gate.  Is this how we can com
\section{Further work}

There are several directions in which the work in this paper could be further explored. Since linear relations are can be defined over a principle ideal domain over a field, it is natural to ask if the work can be generalized to this setting. Also, we have not given a completeness result entirely in terms of the generators of $\Lag\Rel_k$. The proof of such would almost certainly involve mimicking the universality proofs of the qubit stabilizer/qutrit stabilizer/Spekkens' toy model \cite{backensstab,backensspek,qutrit} involving local equivalency/local complementation of graph states.  
If this were generalized to affine Lagrangian relations this would yield a proper completeness result for the odd-prime-dimensional qudit stabilizer ZX calculus as a corollary. One could also potentially adapt this approach to characterize Lagrangian spans as described in \cite[p. 187]{fong2016algebra}, where the scalars are not all quotiented out.  Perhaps this would give a semantics for odd-prime-dimensional qudit stabilizer circuits on the nose.

This paper illuminates the deep connection between stabilizer circuits and electrical circuits.  Perhaps, this can be taken further by adding nonlinear generators such as diodes.

%Another possible direction would be to give a graphical characterization of qubit stabilizer circuits in terms of a quotient of affine Lagrangian relations



\nocite{coecke2008interacting}
\nocite{ihpub}
\nocite{niel}










\begin{comment}

\begin{align*}
\begin{tikzpicture}
	\begin{pgfonlayer}{nodelayer}
		\node [style=X] (0) at (-0.25, -0.75) {};
		\node [style=Z] (1) at (0.25, -0.75) {};
		\node [style=X] (2) at (0.75, 1.5) {};
		\node [style=Z] (3) at (1.25, 1.5) {};
		\node [style=scalarop] (4) at (0.25, 0) {$a$};
		\node [style=scalar] (5) at (0.75, 0) {$a$};
		\node [style=none] (6) at (0.25, 0.75) {};
		\node [style=s] (7) at (0.75, 0.75) {};
		\node [style=none] (8) at (0.75, 2) {};
		\node [style=none] (9) at (1.25, 2) {};
		\node [style=none] (10) at (-0.25, 0) {};
		\node [style=none] (11) at (-0.75, 0) {};
		\node [style=none] (12) at (-0.75, 2) {};
		\node [style=none] (13) at (-0.25, 2) {};
		\node [style=none] (14) at (-0.25, -1.25) {};
		\node [style=none] (15) at (0.25, -1.25) {};
		\node [style=none] (16) at (1.25, 0.75) {};
		\node [style=none] (17) at (1.75, 0.75) {};
		\node [style=none] (18) at (1.25, -1.25) {};
		\node [style=none] (19) at (1.75, -1.25) {};
	\end{pgfonlayer}
	\begin{pgfonlayer}{edgelayer}
		\draw [in=-90, out=90] (5) to (6.center);
		\draw [in=-90, out=90] (4) to (7);
		\draw [in=30, out=-90] (5) to (1);
		\draw [in=-90, out=30] (0) to (4);
		\draw [in=-135, out=90] (6.center) to (2);
		\draw [in=-150, out=90] (7) to (3);
		\draw [in=-90, out=150] (0) to (11.center);
		\draw [in=-90, out=150] (1) to (10.center);
		\draw (10.center) to (13.center);
		\draw (11.center) to (12.center);
		\draw (8.center) to (2);
		\draw (3) to (9.center);
		\draw [in=90, out=-30] (2) to (16.center);
		\draw [in=90, out=-30] (3) to (17.center);
		\draw (18.center) to (16.center);
		\draw (19.center) to (17.center);
		\draw (14.center) to (0);
		\draw (15.center) to (1);
	\end{pgfonlayer}
\end{tikzpicture}
&=
\begin{tikzpicture}
	\begin{pgfonlayer}{nodelayer}
		\node [style=X] (20) at (3.75, -1) {};
		\node [style=Z] (21) at (4.5, -1) {};
		\node [style=X] (22) at (5, 1.5) {};
		\node [style=Z] (23) at (6, 1.5) {};
		\node [style=scalarop] (24) at (4.5, -0.25) {$-a$};
		\node [style=scalar] (25) at (5.25, -0.25) {$a$};
		\node [style=none] (26) at (4.5, 0.75) {};
		\node [style=none] (28) at (5, 2) {};
		\node [style=none] (29) at (6, 2) {};
		\node [style=none] (30) at (4, -0.25) {};
		\node [style=none] (31) at (3.25, -0.25) {};
		\node [style=none] (32) at (3.25, 2) {};
		\node [style=none] (33) at (4, 2) {};
		\node [style=none] (34) at (3.75, -1.5) {};
		\node [style=none] (35) at (4.5, -1.5) {};
		\node [style=none] (36) at (5.75, 0.75) {};
		\node [style=none] (37) at (6.5, 0.75) {};
		\node [style=none] (38) at (5.75, -1.5) {};
		\node [style=none] (39) at (6.5, -1.5) {};
		\node [style=none] (40) at (5.25, 0.75) {};
	\end{pgfonlayer}
	\begin{pgfonlayer}{edgelayer}
		\draw [in=-90, out=90] (25) to (26.center);
		\draw [in=30, out=-90] (25) to (21);
		\draw [in=-90, out=30] (20) to (24);
		\draw [in=-135, out=90] (26.center) to (22);
		\draw [in=-90, out=150] (20) to (31.center);
		\draw [in=-90, out=150] (21) to (30.center);
		\draw (30.center) to (33.center);
		\draw (31.center) to (32.center);
		\draw (28.center) to (22);
		\draw (23) to (29.center);
		\draw [in=90, out=-30] (22) to (36.center);
		\draw [in=90, out=-30] (23) to (37.center);
		\draw (38.center) to (36.center);
		\draw (39.center) to (37.center);
		\draw (34.center) to (20);
		\draw (35.center) to (21);
		\draw [in=-90, out=90] (24) to (40.center);
		\draw [in=-150, out=90] (40.center) to (23);
	\end{pgfonlayer}
\end{tikzpicture}
=
\begin{tikzpicture}
	\begin{pgfonlayer}{nodelayer}
		\node [style=X] (41) at (7.75, -0.25) {};
		\node [style=Z] (42) at (8.5, -0.25) {};
		\node [style=X] (43) at (9, 1.5) {};
		\node [style=Z] (44) at (9.5, 1.5) {};
		\node [style=none] (48) at (9, 2) {};
		\node [style=none] (49) at (9.5, 2) {};
		\node [style=none] (52) at (7.25, 2) {};
		\node [style=none] (53) at (8, 2) {};
		\node [style=none] (56) at (9.5, 0.75) {};
		\node [style=none] (57) at (10, 0.75) {};
		\node [style=none] (58) at (9.5, -1.5) {};
		\node [style=none] (59) at (10, -1.5) {};
		\node [style=scalarop] (61) at (8, 0.5) {$a$};
		\node [style=scalarop] (62) at (7.75, -1) {$-a$};
		\node [style=scalar] (63) at (7.25, 0.5) {$-a$};
		\node [style=none] (64) at (7.75, -1.5) {};
		\node [style=scalar] (65) at (8.5, -1) {$a$};
		\node [style=none] (66) at (8.5, -1.5) {};
	\end{pgfonlayer}
	\begin{pgfonlayer}{edgelayer}
		\draw (48.center) to (43);
		\draw (44) to (49.center);
		\draw [in=90, out=-30] (43) to (56.center);
		\draw [in=90, out=-30] (44) to (57.center);
		\draw (58.center) to (56.center);
		\draw (59.center) to (57.center);
		\draw (62) to (41);
		\draw [in=-90, out=150] (41) to (63);
		\draw (63) to (52.center);
		\draw (64.center) to (62);
		\draw [in=-90, out=150] (42) to (61);
		\draw (61) to (53.center);
		\draw (66.center) to (65);
		\draw (65) to (42);
		\draw [in=-150, out=30] (41) to (44);
		\draw [in=-135, out=30] (42) to (43);
	\end{pgfonlayer}
\end{tikzpicture}
=
\begin{tikzpicture}
	\begin{pgfonlayer}{nodelayer}
		\node [style=X] (67) at (11, 0.75) {};
		\node [style=Z] (68) at (11.75, 0.75) {};
		\node [style=X] (69) at (12.5, -1.5) {};
		\node [style=Z] (70) at (13.25, -1.5) {};
		\node [style=none] (71) at (13, -0.5) {};
		\node [style=none] (72) at (13.75, -0.5) {};
		\node [style=none] (73) at (11, 2) {};
		\node [style=none] (74) at (11.75, 2) {};
		\node [style=none] (75) at (12.5, -2) {};
		\node [style=none] (76) at (13.25, -2) {};
		\node [style=scalarop] (79) at (11.75, 1.5) {$a$};
		\node [style=scalarop] (80) at (10.75, -0.5) {$-a$};
		\node [style=scalar] (81) at (11, 1.5) {$-a$};
		\node [style=none] (82) at (10.75, -2) {};
		\node [style=scalar] (83) at (11.5, -0.5) {$a$};
		\node [style=none] (84) at (11.5, -2) {};
		\node [style=s] (85) at (12.5, -0.5) {};
		\node [style=s] (86) at (12, -0.5) {};
		\node [style=none] (131) at (13, 2) {};
		\node [style=none] (132) at (13.75, 2) {};
	\end{pgfonlayer}
	\begin{pgfonlayer}{edgelayer}
		\draw [in=15, out=-90] (71.center) to (69);
		\draw [in=270, out=30] (70) to (72.center);
		\draw (69) to (75.center);
		\draw (70) to (76.center);
		\draw [in=-135, out=90, looseness=0.75] (80) to (67);
		\draw (67) to (81);
		\draw (81) to (73.center);
		\draw (82.center) to (80);
		\draw (68) to (79);
		\draw (79) to (74.center);
		\draw (84.center) to (83);
		\draw [in=-120, out=90, looseness=0.75] (83) to (68);
		\draw [in=-90, out=165] (69) to (86);
		\draw [in=-45, out=90, looseness=0.75] (86) to (68);
		\draw [in=-90, out=150] (70) to (85);
		\draw [in=-45, out=90, looseness=0.75] (85) to (67);
		\draw (72.center) to (132.center);
		\draw (131.center) to (71.center);
	\end{pgfonlayer}
\end{tikzpicture}\\
&=
\begin{tikzpicture}
	\begin{pgfonlayer}{nodelayer}
		\node [style=X] (133) at (22.25, 1) {};
		\node [style=Z] (134) at (22.75, 1.25) {};
		\node [style=X] (135) at (24, -1.5) {};
		\node [style=Z] (136) at (24.5, -1.5) {};
		\node [style=none] (137) at (24.5, -0.5) {};
		\node [style=none] (138) at (25, -0.5) {};
		\node [style=none] (139) at (22.25, 2) {};
		\node [style=none] (140) at (22.75, 2) {};
		\node [style=none] (141) at (24, -2) {};
		\node [style=none] (142) at (24.5, -2) {};
		\node [style=none] (143) at (22, 0) {};
		\node [style=none] (144) at (22.5, 0) {};
		\node [style=s] (145) at (24, -0.5) {};
		\node [style=s] (146) at (23.25, -0.5) {};
		\node [style=scalarop] (147) at (23.25, 0) {$a$};
		\node [style=scalar] (148) at (24, 0) {$-a$};
		\node [style=none] (149) at (22, -2) {};
		\node [style=none] (150) at (22.5, -2) {};
		\node [style=none] (151) at (24.5, 2) {};
		\node [style=none] (152) at (25, 2) {};
	\end{pgfonlayer}
	\begin{pgfonlayer}{edgelayer}
		\draw [in=30, out=-90] (137.center) to (135);
		\draw [in=270, out=30] (136) to (138.center);
		\draw (135) to (141.center);
		\draw (136) to (142.center);
		\draw [in=-90, out=165] (135) to (146);
		\draw [in=-90, out=135, looseness=0.75] (136) to (145);
		\draw (146) to (147);
		\draw [in=-30, out=90] (147) to (134);
		\draw (134) to (140.center);
		\draw [in=-120, out=90, looseness=0.75] (144.center) to (134);
		\draw [in=0, out=90, looseness=0.75] (148) to (133);
		\draw (148) to (145);
		\draw [in=-120, out=90] (143.center) to (133);
		\draw (133) to (139.center);
		\draw (150.center) to (144.center);
		\draw (143.center) to (149.center);
		\draw (137.center) to (151.center);
		\draw (152.center) to (138.center);
	\end{pgfonlayer}
\end{tikzpicture}
=
\begin{tikzpicture}
	\begin{pgfonlayer}{nodelayer}
		\node [style=X] (87) at (15, 1) {};
		\node [style=Z] (88) at (15.5, 1.25) {};
		\node [style=X] (89) at (16.5, -1.5) {};
		\node [style=Z] (90) at (17, -1.5) {};
		\node [style=none] (91) at (17, 0) {};
		\node [style=none] (92) at (17.5, 0) {};
		\node [style=none] (93) at (15, 2) {};
		\node [style=none] (94) at (15.5, 2) {};
		\node [style=none] (95) at (16.5, -2) {};
		\node [style=none] (96) at (17, -2) {};
		\node [style=none] (100) at (14.75, 0) {};
		\node [style=none] (102) at (15.25, 0) {};
		\node [style=scalarop] (105) at (15.75, 0) {$-a$};
		\node [style=scalar] (106) at (16.5, 0) {$a$};
		\node [style=none] (127) at (17, 2) {};
		\node [style=none] (128) at (17.5, 2) {};
		\node [style=none] (129) at (14.75, -2) {};
		\node [style=none] (130) at (15.25, -2) {};
	\end{pgfonlayer}
	\begin{pgfonlayer}{edgelayer}
		\draw [in=45, out=-90] (91.center) to (89);
		\draw [in=270, out=45] (90) to (92.center);
		\draw (89) to (95.center);
		\draw (90) to (96.center);
		\draw [in=-30, out=90, looseness=0.75] (105) to (88);
		\draw (88) to (94.center);
		\draw [in=-135, out=90, looseness=0.75] (102.center) to (88);
		\draw [in=0, out=90, looseness=0.75] (106) to (87);
		\draw [in=-135, out=90] (100.center) to (87);
		\draw (87) to (93.center);
		\draw [in=-90, out=135] (90) to (106);
		\draw [in=135, out=-90] (105) to (89);
		\draw (92.center) to (128.center);
		\draw (127.center) to (91.center);
		\draw (130.center) to (102.center);
		\draw (100.center) to (129.center);
	\end{pgfonlayer}
\end{tikzpicture}
=
\begin{tikzpicture}
	\begin{pgfonlayer}{nodelayer}
		\node [style=X] (107) at (19, 1) {};
		\node [style=Z] (108) at (19.5, 1) {};
		\node [style=X] (109) at (20, -1.25) {};
		\node [style=Z] (110) at (20.5, -1.25) {};
		\node [style=none] (111) at (20.5, -0.5) {};
		\node [style=none] (112) at (21, -0.5) {};
		\node [style=none] (113) at (19, 1.75) {};
		\node [style=none] (114) at (19.5, 1.75) {};
		\node [style=none] (115) at (20, -1.75) {};
		\node [style=none] (116) at (20.5, -1.75) {};
		\node [style=none] (117) at (18.5, 0.25) {};
		\node [style=none] (118) at (19, 0.25) {};
		\node [style=scalarop] (119) at (19.5, -0.5) {$a$};
		\node [style=scalar] (120) at (20, -0.5) {$a$};
		\node [style=s] (121) at (20, 0.25) {};
		\node [style=none] (122) at (19.5, 0.25) {};
		\node [style=none] (123) at (18.5, -1.75) {};
		\node [style=none] (124) at (19, -1.75) {};
		\node [style=none] (125) at (20.5, 1.75) {};
		\node [style=none] (126) at (21, 1.75) {};
	\end{pgfonlayer}
	\begin{pgfonlayer}{edgelayer}
		\draw [in=45, out=-90] (111.center) to (109);
		\draw [in=270, out=30] (110) to (112.center);
		\draw (109) to (115.center);
		\draw (110) to (116.center);
		\draw (108) to (114.center);
		\draw [in=-150, out=90] (118.center) to (108);
		\draw [in=-150, out=90] (117.center) to (107);
		\draw (107) to (113.center);
		\draw [in=-75, out=135] (110) to (120);
		\draw [in=135, out=-75] (119) to (109);
		\draw [in=-90, out=90] (119) to (121);
		\draw [in=-90, out=90] (120) to (122.center);
		\draw [in=-30, out=90] (121) to (108);
		\draw [in=-30, out=90] (122.center) to (107);
		\draw (112.center) to (126.center);
		\draw (125.center) to (111.center);
		\draw (124.center) to (118.center);
		\draw (117.center) to (123.center);
	\end{pgfonlayer}
\end{tikzpicture}
\end{align*}


$$
\begin{tikzpicture}
	\begin{pgfonlayer}{nodelayer}
		\node [style=scalarop] (153) at (26, 0) {$a$};
		\node [style=scalar] (154) at (26.5, 0) {$a$};
		\node [style=none] (155) at (26, 0.75) {};
		\node [style=s] (156) at (26.5, 0.75) {};
		\node [style=none] (157) at (26, 1.25) {};
		\node [style=none] (158) at (26.5, 1.25) {};
		\node [style=none] (159) at (26, -0.5) {};
		\node [style=none] (160) at (26.5, -0.5) {};
	\end{pgfonlayer}
	\begin{pgfonlayer}{edgelayer}
		\draw [in=-90, out=90] (154) to (155.center);
		\draw [in=-90, out=90] (153) to (156);
		\draw (156) to (158.center);
		\draw (157.center) to (155.center);
		\draw (159.center) to (153);
		\draw (160.center) to (154);
	\end{pgfonlayer}
\end{tikzpicture}
=
\begin{tikzpicture}
	\begin{pgfonlayer}{nodelayer}
		\node [style=scalarop] (161) at (28.25, 0.5) {$-a$};
		\node [style=scalar] (162) at (27.5, 0.5) {$a$};
		\node [style=none] (163) at (27.5, 1) {};
		\node [style=none] (167) at (27.5, -0.5) {};
		\node [style=none] (168) at (28.25, -0.5) {};
		\node [style=none] (169) at (28.25, 1) {};
	\end{pgfonlayer}
	\begin{pgfonlayer}{edgelayer}
		\draw (162) to (163.center);
		\draw [in=270, out=90] (167.center) to (161);
		\draw [in=270, out=90] (168.center) to (162);
		\draw (161) to (169.center);
	\end{pgfonlayer}
\end{tikzpicture}
=
\begin{tikzpicture}
	\begin{pgfonlayer}{nodelayer}
		\node [style=scalar] (171) at (29.25, 0.5) {$a$};
		\node [style=none] (172) at (29.25, 1) {};
		\node [style=none] (173) at (29.25, -1) {};
		\node [style=none] (175) at (30, 1) {};
		\node [style=scalarop] (176) at (30, 0.5) {$a$};
		\node [style=none] (177) at (29.25, -0.25) {};
		\node [style=none] (178) at (30, -1) {};
		\node [style=s] (179) at (30, -0.25) {};
	\end{pgfonlayer}
	\begin{pgfonlayer}{edgelayer}
		\draw (171) to (172.center);
		\draw (177.center) to (171);
		\draw (179) to (176);
		\draw (176) to (175.center);
		\draw [in=90, out=-90, looseness=0.75] (179) to (173.center);
		\draw [in=90, out=-90, looseness=0.75] (177.center) to (178.center);
	\end{pgfonlayer}
\end{tikzpicture}
$$


$$
\begin{tikzpicture}
	\begin{pgfonlayer}{nodelayer}
		\node [style=X] (180) at (35.25, -0.25) {};
		\node [style=Z] (181) at (35.75, -0.25) {};
		\node [style=X] (182) at (36.25, 1.25) {};
		\node [style=Z] (183) at (36.75, 1.25) {};
		\node [style=scalar] (184) at (35.75, 0.5) {$a$};
		\node [style=scalarop] (185) at (36.25, 0.5) {$a$};
		\node [style=none] (186) at (34.75, 0.5) {};
		\node [style=none] (187) at (35.25, 0.5) {};
		\node [style=none] (188) at (36.75, 0.5) {};
		\node [style=none] (189) at (37.25, 0.5) {};
		\node [style=none] (190) at (34.75, 1.75) {};
		\node [style=none] (191) at (35.25, 1.75) {};
		\node [style=none] (194) at (35.25, -1) {};
		\node [style=none] (195) at (35.75, -1) {};
		\node [style=none] (199) at (36.75, -0.5) {};
		\node [style=s] (201) at (37.25, -0.5) {};
		\node [style=none] (202) at (37.25, -1) {};
		\node [style=none] (203) at (36.75, -1) {};
		\node [style=none] (205) at (36.25, 2.25) {};
		\node [style=s] (207) at (36.75, 2.25) {};
		\node [style=none] (208) at (36.75, 2.75) {};
		\node [style=none] (209) at (36.25, 2.75) {};
		\node [style=none] (210) at (35.25, 2.75) {};
		\node [style=none] (211) at (34.75, 2.75) {};
	\end{pgfonlayer}
	\begin{pgfonlayer}{edgelayer}
		\draw [in=-90, out=30] (181) to (185);
		\draw [in=-150, out=90] (185) to (183);
		\draw [in=90, out=-150] (182) to (184);
		\draw [in=30, out=-90] (184) to (180);
		\draw [in=-90, out=150] (181) to (187.center);
		\draw [in=-90, out=150] (180) to (186.center);
		\draw [in=90, out=-30] (182) to (188.center);
		\draw [in=-30, out=90] (189.center) to (183);
		\draw (187.center) to (191.center);
		\draw (190.center) to (186.center);
		\draw (194.center) to (180);
		\draw (195.center) to (181);
		\draw (202.center) to (201);
		\draw (203.center) to (199.center);
		\draw [in=-90, out=90] (199.center) to (189.center);
		\draw [in=-90, out=90] (201) to (188.center);
		\draw [in=-90, out=90] (182) to (207);
		\draw [in=90, out=-90] (205.center) to (183);
		\draw (191.center) to (210.center);
		\draw (211.center) to (190.center);
		\draw (205.center) to (209.center);
		\draw (208.center) to (207);
	\end{pgfonlayer}
\end{tikzpicture}
=
\begin{tikzpicture}
	\begin{pgfonlayer}{nodelayer}
		\node [style=X] (212) at (38.75, -0.25) {};
		\node [style=Z] (213) at (39.25, -0.25) {};
		\node [style=X] (214) at (39.75, 1.75) {};
		\node [style=Z] (215) at (40.25, 1.75) {};
		\node [style=scalar] (216) at (39.25, 0.5) {$a$};
		\node [style=scalarop] (217) at (39.75, 0.5) {$a$};
		\node [style=none] (218) at (38.25, 0.5) {};
		\node [style=none] (219) at (38.75, 0.5) {};
		\node [style=none] (220) at (40.25, 1) {};
		\node [style=none] (221) at (40.75, 1) {};
		\node [style=none] (222) at (38.25, 1.75) {};
		\node [style=none] (223) at (38.75, 1.75) {};
		\node [style=none] (224) at (38.75, -0.75) {};
		\node [style=none] (225) at (39.25, -0.75) {};
		\node [style=none] (226) at (40.25, -0.75) {};
		\node [style=none] (228) at (40.75, -0.75) {};
		\node [style=none] (230) at (39.75, 2.75) {};
		\node [style=none] (232) at (40.25, 2.75) {};
		\node [style=none] (234) at (38.75, 2.75) {};
		\node [style=none] (235) at (38.25, 2.75) {};
		\node [style=s] (236) at (39.25, 1) {};
		\node [style=none] (237) at (39.75, 1) {};
	\end{pgfonlayer}
	\begin{pgfonlayer}{edgelayer}
		\draw [in=-90, out=30] (213) to (217);
		\draw [in=30, out=-90] (216) to (212);
		\draw [in=-90, out=150] (213) to (219.center);
		\draw [in=-90, out=150] (212) to (218.center);
		\draw [in=90, out=-30] (214) to (220.center);
		\draw [in=-30, out=90] (221.center) to (215);
		\draw (219.center) to (223.center);
		\draw (222.center) to (218.center);
		\draw (224.center) to (212);
		\draw (225.center) to (213);
		\draw [in=-90, out=90] (226.center) to (221.center);
		\draw [in=90, out=-90] (230.center) to (215);
		\draw (223.center) to (234.center);
		\draw (235.center) to (222.center);
		\draw [in=-90, out=90] (214) to (232.center);
		\draw [in=-90, out=90] (228.center) to (220.center);
		\draw (217) to (237.center);
		\draw [in=-135, out=90] (237.center) to (215);
		\draw [in=-150, out=90] (236) to (214);
		\draw (236) to (216);
	\end{pgfonlayer}
\end{tikzpicture}
=
\begin{tikzpicture}
	\begin{pgfonlayer}{nodelayer}
		\node [style=X] (260) at (42.5, -1) {};
		\node [style=Z] (261) at (43, -1) {};
		\node [style=X] (262) at (44, 1.5) {};
		\node [style=Z] (263) at (43.5, 1.5) {};
		\node [style=scalar] (264) at (43, -0.25) {$a$};
		\node [style=scalarop] (265) at (43.5, -0.25) {$a$};
		\node [style=none] (266) at (42, -0.25) {};
		\node [style=none] (267) at (42.5, -0.25) {};
		\node [style=none] (268) at (44.5, 0.75) {};
		\node [style=none] (269) at (44, 0.75) {};
		\node [style=none] (270) at (42, 2) {};
		\node [style=none] (271) at (42.5, 2) {};
		\node [style=none] (272) at (42.5, -1.5) {};
		\node [style=none] (273) at (43, -1.5) {};
		\node [style=none] (274) at (43.5, 2) {};
		\node [style=none] (275) at (44, 2) {};
		\node [style=none] (280) at (43, 0.75) {};
		\node [style=s] (282) at (43.5, 0.75) {};
		\node [style=none] (283) at (44.5, -1.5) {};
		\node [style=none] (284) at (44, -1.5) {};
	\end{pgfonlayer}
	\begin{pgfonlayer}{edgelayer}
		\draw [in=-90, out=30] (261) to (265);
		\draw [in=30, out=-90] (264) to (260);
		\draw [in=-90, out=150] (261) to (267.center);
		\draw [in=-90, out=150] (260) to (266.center);
		\draw [in=90, out=-30] (262) to (268.center);
		\draw [in=-30, out=90] (269.center) to (263);
		\draw (267.center) to (271.center);
		\draw (270.center) to (266.center);
		\draw (272.center) to (260);
		\draw (273.center) to (261);
		\draw [in=90, out=-90] (274.center) to (263);
		\draw [in=-90, out=90] (262) to (275.center);
		\draw [in=-90, out=90] (265) to (280.center);
		\draw [in=-90, out=90] (264) to (282);
		\draw [in=-150, out=90] (282) to (262);
		\draw [in=90, out=-150] (263) to (280.center);
		\draw (284.center) to (269.center);
		\draw (268.center) to (283.center);
	\end{pgfonlayer}
\end{tikzpicture}
$$


\resizebox{\linewidth}{!}{
  \begin{minipage}{\linewidth}
\begin{align*}
&\left[
\begin{array}{*{5}c|*{6}c}
x_{1,1} & 0          & 0           & \cdots & 0           & z_{1,1} & z_{1,2} & z_{1,3} & \cdots & z_{1,n}\\
0          & x_{2,2} & 0           & \cdots & 0           & z_{1,2} & z_{2,2} & z_{2,3} & \cdots & z_{2,n}\\
0          & 0           & x_{3,3} & \ddots & \vdots   & z_{1,3} & z_{2,3} & z_{3,3} & \cdots & z_{3,n}\\
\vdots  & \vdots   & \ddots    & \ddots & 0           & \vdots   & \vdots    & \vdots    & \ddots &  \vdots \\
0          & 0            & \cdots  & 0         & x_{n,n} & z_{1,n} & z_{2,n} & z_{3,n} & \cdots & z_{n,n}\\
\end{array}
\right]
\xmapsto{(F^{(1)})^{-1}}
\left[
\begin{array}{*{5}c|*{6}c}
z_{1,1}         & 0           & 0         & \cdots & 0          & -x_{1,1} & z_{1,2} & z_{1,3} & \cdots & z_{1,n}\\
z_{1,2}         & x_{2,2} & 0         & \cdots & 0          & 0            & z_{2,2} & z_{2,3} & \cdots & z_{2,n}\\
z_{1,3}         & 0           & 1         & \ddots & \vdots  & 0            & z_{2,3} & z_{3,3} & \cdots & z_{3,n}\\
\vdots & \vdots & \ddots & \ddots & 0         & \vdots   & \vdots    & \vdots    & \ddots &  \vdots \\
z_{1,n}         & 0           & \cdots & 0        & x_{n,n} & 0 & z_{2,n} & z_{3,n} & \cdots & z_{n,n}\\
\end{array}
\right]\\
&\xmapsto{C_{z_{1,2}}^{(2,1)}}\\
&\left[
\begin{array}{*{5}c|*{6}c}
a_1z_{1,1}+b_1 0           & 0            & 0            & \cdots & 0           & -x_{1,1} & a_1z_{1,2}-b_1x_{1,1}  & z_{1,3} & \cdots & z_{1,n}\\
a_1z_{1,2}+b_1 x_{2,2} & x_{2,2} & 0             & \cdots & 0           & 0             & a_1z_{2,2}-b_10           & z_{2,3} & \cdots & z_{2,n}\\
a_1z_{1,3}+b_1 0           & 0            & bx_{3.3} & \ddots & \vdots  & 0             & a_1z_{2,3}-b_10            & z_{3,3} & \cdots & z_{3,n}\\
\vdots                              & \vdots    & \ddots     & \ddots & 0           & \vdots     & \vdots                              & \vdots    & \ddots &  \vdots \\
a_1z_{1,n}+b_10           & 0            & \cdots     & 0         & x_{n,n} & 0             & a_1z_{2,n}-b_10            & z_{3,n} & \cdots & z_{n,n}\\
\end{array}
\right]
=
\left[
\begin{array}{*{5}c|*{6}c}
az_{1,1}                  & 0            & 0            & \cdots & 0           & -x_{1,1} & \gcd(z_{1,2},x_{1,1})                         & z_{1,3} & \cdots & z_{1,n}\\
\gcd()                          & x_{2,2} & 0             & \cdots & 0           & 0             & z_{2,2}                  & z_{2,3} & \cdots & z_{2,n}\\
az_{1,3}                  & 0            & bx_{3.3} & \ddots & \vdots  & 0             & z_{2,3}                  & z_{3,3} & \cdots & z_{3,n}\\
\vdots                      & \vdots    & \ddots     & \ddots & 0           & \vdots     & \vdots                    & \vdots    & \ddots &  \vdots \\
az_{1,n}                 & 0            & \cdots     & 0         & x_{n,n} & 0           & z_{2,n}                  & z_{3,n} & \cdots & z_{n,n}\\
\end{array}
\right]\\
&\xmapsto{\prod_{i>1}^n C_{z_{1,i}}^{(i,1)}}\\
&\left[
\begin{array}{*{5}c|*{6}c}
az_{1,1}                  & 0            & 0            & \cdots & 0           & -x_{1,1} & c_1                         & c_2 & \cdots & c_n\\
c_1                          & x_{2,2} & 0             & \cdots & 0           & 0             & z_{2,2}                  & z_{2,3} & \cdots & z_{2,n}\\
c_2                          & 0            & bx_{3.3} & \ddots & \vdots  & 0             & z_{2,3}                  & z_{3,3} & \cdots & z_{3,n}\\
\vdots                      & \vdots    & \ddots     & \ddots & 0           & \vdots     & \vdots                    & \vdots    & \ddots &  \vdots \\
c_n                          & 0            & \cdots     & 0         & x_{n,n} & 0           & z_{2,n}                  & z_{3,n} & \cdots & z_{n,n}\\
\end{array}
\right]
\xmapsto{F^{(1)}}
\left[
\begin{array}{*{5}c|*{6}c}
1                       & 0         & 0         & \cdots & 0         & z_{1,1} & 0                          & 0           & \cdots & 0\\
0                       & 1         & 0         & \cdots & 0         & 0           & z_{2,2}                & z_{2,3} & \cdots & z_{2,n}\\
0                       & 0         & 1         & \ddots & \vdots & 0           & z_{2,3}                & z_{3,3} & \cdots & z_{3,n}\\
\vdots               & \vdots & \ddots & \ddots & 0         & \vdots   & \vdots                   & \vdots    & \ddots &  \vdots \\
0                       & 0         & \cdots & 0        & 1          & 0           & z_{2,n}                & z_{3,n} & \cdots & z_{n,n}\\
\end{array}
\right]\\
&\xmapsto{F^{(1)}S_{-z_{1,1}}^{(1)}   (F^{(1)})^{-1}}\\
&\left[
\begin{array}{*{5}c|*{6}c}
1                       & 0         & 0         & \cdots & 0         & 0 & 0                          & 0           & \cdots & 0\\
0                       & 1         & 0         & \cdots & 0         & 0           & z_{2,2}                & z_{2,3} & \cdots & z_{2,n}\\
0                       & 0         & 1         & \ddots & \vdots & 0           & z_{2,3}                & z_{3,3} & \cdots & z_{3,n}\\
\vdots               & \vdots & \ddots & \ddots & 0         & \vdots   & \vdots                   & \vdots    & \ddots &  \vdots \\
0                       & 0         & \cdots & 0        & 1          & 0           & z_{2,n}                & z_{3,n} & \cdots & z_{n,n}\\
\end{array}
\right]\\
\end{align*}
  \end{minipage}
}
\end{comment}








%shortened paper idea
% Introduction, motivation from electrical circuits and quantum mechanics
% Very short intoduction to graphical linear and affine algebra
%   Show how electrical circuits can be represented using these things
% Define symplectic vector spaces, Lagrangian relations/co(isotropic relations)
% Give generators for Lagrangian relations
%   Show they are complete
% Show how adding one more generator makes things complete for affine Lagrangian relations
%
% Give equivalence to Spekkens' toy model, stabilizer circuits
%
% Show how adding a generator gives you (co)Isotropic relations
% One more generator gives you affine (co)isotropic relations
% CPM(AffLagRel) \cong AffCoIsotRel
%
% Talk about how this is the semantics for discarding in quantum mechanics
% Semantics for measurement are given by splitting decoherence maps
%
% Karoubi envelope equivalent to adding injection maps
%
% Quantum teleportation algorithm
% Impedance calculus translation
%
% Conclusion
