
The traditional paradigm for quantum computing decomposes a quantum computation into  distinct stages. First, a quantum state is prepared in the lab; then the quantum state is evolved by applying unitary operations; next, the quantum state is measured according to the Born rule.  There are several variations on this paradigm; for example,  one can measure only part of the system and apply classically controlled unitary operations to correct for errors; or in the measurement based approach, one could prepare a quantum state and then evolve the state by performing a series of partial measurements which are conditioned on each other. However, these tradiditional approaches are quite rigid, where the different stages of the process are modelled by different kinds of mathematical objects.  %Moreover, the born rule entails that the measurements depend {\em probabilistically} on the choice state preparation and unitary evolution.

In this thesis, we model quantum computation in the more flexible quantum circuit model, where the state preparation, evolution and measurement all live on similar footing.  More precisely, departing from most treatments of quantum circuits, we regard of these different stages of a quantum computation as subspaces. These subspaces are composed by tracing out the common elements in their intersection.  Two types of subspaces appear in this thesis, namely subsets of finite sets and affine/linear subspaces of vector spaces over finite fields.  We associate the Hilbert space of square-integrable functions to these spaces. The subspaces correspond to the nondeterministic evolution of the system, as opposed to a probabilistic evolution.  That is to say, by regarding these classes of quantum circuits as subspaces, the inputs and outputs are associated to each other with respect to which output is {\em possible} from which input.

In Chapter \ref{chap:background}, we review the mathematical background which is needed to understand this thesis.  In Section \ref{sec:cat}, we give a brief review of relevant notions in category theory.  We first review the theory of monoidal categories and string diagrams in which allows us to regard circuits as abstract mathematical objects.  Next we review categories of spans and relations, which are the mathematical semantics for circuits with nondeterministic evolution.  In Section \ref{sec:cqm}, we review some of the basic results in categorical quantum mechanics which relates monoidal categories and string diagrams to quantum computing. This entails giving an overview of the languages for quantum circuits known as the  ZH and ZX-calculi; as well as reviewing the mathematical machinery needed to model mixed states and measurement within this framework.

In Chapter \ref{chap:zxa} we analyze the class of quantum circuits generated by the Toffoli gate as well as state preparation and post selection in the $Z$ and $X$-bases.  We give a complete presentation for this category and interpret it in terms of spans of finite sets.  We show that this class of circuits has very close to a nondeterministic semantics, except where outcomes can happen multiple times.  By imposing an additional equation, we depart from the interpretation into Hilbert spaces, and obtain a semantics with a proper  nondeterministic semantics in terms of relations between finite sets. We then decompose this presentation into small fragments; composing these small building blocks incrementally via distributive law and pushout.
We exhibit substructural features of these various decompositions and discuss how by allowing only some of the generators, we obtain semantics which are partial, partially invertible and so on; associating classes of generators to different semantic paradigms of computation.


In Chapter \ref{chap:stab}  we analyze the structure of odd-prime dimensional/quopit stabilizer circuits.
We expose the relational interpretation of these circuits, by adding generators at each point, obtaining a richer semantics.  We first recall that the phase-free fragment of the qupit ZX-calculus modulo scalars, for prime qudit dimension $p$ is isomorphic to the prop of linear relations over $\F_p$: ie where the maps are linear subspaces over $\F_p$.  By doubling the phase-free picture using the CPM construction, we obtain a semantics for Weyl-free odd-prime dimensional qudit stabilizer circuits: Lagrangian relations over $\F_p$.  The Weyl operators are introduced to this picture by adding affine shifts to obtain a prop of affine Lagrangian relations. To add quantum discarding, we showed that one doesn't need to take the CPM construction again, but it suffices to add the discard {\em relation} to obtain the prop of affine coisotropic relations.  By splitting idempotents, we recover measurement; which has a nice relational interpretation.  Using this relational interpretation of mixed stabilizer circuits, we showed how stabilizer error correction protocols can be implemented.


In Chapter \ref{chap:grothendieck}, we regard monoidal categories as certain categorified Frobenius algebras in profunctors; the bicategory profunctors being itself a categorification of relations.  We conjecture that there is a confluent normal form for these structures, categorifying the spider theorem for special Frobenius algebras.  By regarding this spider theorem as a monoidal displayed category, we compute the B\'enabou-Grothendieck construction, and split idempotents, to obtain a strict monoidal category which is very close to proof nets for monoidal categories. We discuss the relation to the scalable ZX-calculus.

In Chapter \ref{chap:conclusion} we discuss future work and the limitations of this thesis.



STATE MAIN STRING DIAGRAMMATIC RESULTS OF THESIS

