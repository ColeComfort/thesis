
The traditional paradigm for quantum computing decomposes a quantum computation into  distinct stages. First, a quantum state is prepared in the lab; then the quantum state is evolved by applying unitary operations; next, the quantum state is measured according to the Born rule.  There are several variations on this paradigm; for example,  one can measure only part of the system and apply classically controlled unitary operations to correct for errors; or in the measurement based approach, one could prepare a quantum state and then evolve the state by performing a series of partial measurements which are conditioned on each other. However, these traditional approaches are quite rigid, where the different stages of the process are modeled by different kinds of mathematical objects.  %Moreover, the born rule entails that the measurements depend {\em probabilistically} on the choice state preparation and unitary evolution.

In this thesis, following the research programme of categorical quantum mechanics, we model quantum computation in the more flexible quantum circuit model, where the state preparation, evolution and measurement all regarded as circuits in the same monoidal category. Moreover, we go further than what is typically done in  categorical quantum mechanics and  take an even more uniform approach.  In the restricted settings which we study in this thesis,  we regard these different stages of a quantum computation as various kinds of subspaces.  We compose subspaces uniformly by tracing out the common elements in their intersection.  Two classes of subspaces appear in this thesis. The first class is subsets of finite sets, which are regarded as configurations in the state space of certain classes of quantum systems.  The second class is affine coisotropic subspaces of symplectic vector spaces over finite fields which are regarded as the configurations of  the phase space of stabilizer circuits. By regarding these classes of quantum circuits as subspaces, the inputs and outputs are associated to each other with respect to which output is {\em possible} from which input and vice versa.  Therefore the evolution and measurement, phase correction and so on are nondeterministic processes

In Chapter \ref{chap:background}, we review the mathematical background which is needed to understand this thesis.   In Section \ref{sec:cat}, we review of relevant notions in category theory. Not all subsections are needed to understand the whole thesis, some are only needed for specific parts.  in Subsection \ref{subsec:moncat} we review the theory of monoidal categories and string diagrams in which allows us to regard circuits as abstract mathematical objects.  In Subsection \ref{subsec:dag} we review \dag-categories, which give a categorical semantics for reversibility. In Subsection \ref{subsec:monpres} we review how monoidal categories can be presented in terms of generators and relations and give examples.  In Subsection \ref{subsec:spanrel} we review categories of spans and relations, which are the mathematical semantics for  nondeterminism.  In Subsection \ref{subsec:internal} we review internal category theory which is the mathematics needed to give more fine grained decompositional semantics of circuits.  

In Section \ref{sec:cqm}, we review  categorical quantum mechanics which relates monoidal categories and \dag-categories to quantum computing. This entails giving an overview of the families of languages for quantum circuits known as the  ZX and ZH-calculi; as well as reviewing the mathematical machinery needed to model mixed states and measurement within this framework.  We also review the stabilizer formalism.

In Chapter \ref{chap:zxa} we analyze the class of quantum circuits generated by the Toffoli gate as well the states $|0\rangle$, $|1\rangle$, $\sqrt{2}|+\rangle$ and their adjoints.  In Proposition \ref{lemma:unitcounit} we give a complete presentation for this category and interpret it in terms of spans of finite sets.  In other words, this class of circuits has very close to a nondeterministic semantics, except where outcomes can happen multiple times.  In Theorem \ref{theorem:TOFZXAiso}, we show how this has a more elegant  description in terms of interacting monoids which we call $\ZXA$.  The generators and relations of $\ZXA$ are given in Definition \ref{def:zxa}.    In Remark \ref{rem:zh} we note $\ZXA$  is the natural number labeled fragment of the qubit ZH-calculus.  Consequently, in Corollary \ref{cor:bool} we show how  imposing an additional equation on $\ZXA$, we depart from the interpretation into Hilbert spaces, and obtain a proper  nondeterministic semantics in terms of relations between finite sets.  We show in Corollary \ref{cor:zhint}, by adding two generators and equations to $\ZXA$  we obtain the phase-free qubit ZH-calculus.
We also decompose $\ZXA$  into small fragments; recomposing these small building blocks incrementally via distributive law and pushout.
We exhibit substructural features of these various decompositions and discuss how by allowing only some of the generators, we obtain semantics which are partial, partially invertible and so on; associating classes of generators to different semantic paradigms of computation.


In Chapter \ref{chap:stab}  we analyze the structure of odd-prime dimensional/quopit stabilizer circuits.
We expose the relational interpretation of these circuits, by adding generators at each point, obtaining different semantics for each classes of generators.  We first recall that the phase-free fragment of the qupit ZX-calculus modulo scalars, for prime qudit dimension $p$ is isomorphic to the prop of linear relations over $\F_p$: i.e. where the maps are linear subspaces over $\F_p$.
In Theorem \ref{theorem:generators} we give generators for the prop of Lagrangian relations over a field.
In Corollary \ref{cor:cpm} show that doubling  linear relations over $\F_p$ using the CPM construction, we obtain the prop of  Lagrangian relations. In Corollary \ref{cor:wfree} we  show that the prop of  Lagrangian relations over odd prime fields is equivalent to quopit Weyl-free stabilizer circuits modulo scalars. The Weyl operators are introduced to this picture in Theorem \ref{theorem:spekkens} by adding affine shifts to obtain the prop of affine Lagrangian relations. To add quantum discarding, in Corollary \ref{cor:stabcode} we show that one doesn't need to take the CPM construction again, but it suffices to add the discard {\em relation}:   obtaining the prop of affine coisotropic relations.  By splitting idempotents, in Theorem \ref{thm:twocoloured} we recover measurement; which has a succinct relational interpretation.  Using this relational interpretation of mixed stabilizer circuits, in Section \ref{sec:qec} we show how stabilizer error correction protocols can be implemented.  Throughout this chapter, we compare the quantum semantics to the electrical circuits; highlighting the similarities and differences between both cases.


In Chapter \ref{chap:grothendieck}, we regard monoidal categories as certain categorified Frobenius algebras in profunctors; the bicategory profunctors being itself a categorification of relations.  We conjecture that there is a confluent normal form for these structures, categorifying the spider theorem for special Frobenius algebras.  By regarding this spider theorem as a monoidal displayed category, we compute the B\'enabou-Grothendieck construction, and split idempotents, to obtain a strict monoidal category which is very close to proof nets for monoidal categories. We discuss the relation to the scalable ZX-calculus.

In Chapter \ref{chap:conclusion} we discuss future work and the limitations of this thesis.



