The traditional paradigm for quantum computing decomposes a quantum computation into  distinct stages. First, a quantum state is prepared in the lab; then, the quantum state is evolved by applying unitary operations; next, the quantum state is measured according to the Born rule.  There are variations to this paradigm, for example, in measurement-based quantum computing one first  prepares a quantum state, then evolves the state only by performing a series of conditioned partial measurements. However, these traditional approaches are quite rigid: the different stages of the process are modeled by different kinds of mathematical objects. Each of these stages interact with each other according to various rules: which are stitched together in order to interpret the full computation. As a consequence, the fundamental connections between these different paradigms are hard to see because their low-level descriptions introduce arbitrary book-keeping that must be painstakingly translated back and forth.

In this thesis, using categorical methods we reject these artificial distinctions and take a fundamentally different approach.  The use of category theory in quantum mechanics is quite varied, and has been developed by many different people; however, we will largely follow the particular vein put forward by Abramsky and Coecke \cite{abramsky}.  In this setting, quantum circuits are regarded as string diagrams for the \dag-compact closed category of finite dimensional Hilbert spaces and linear maps.  One advantage of using this categorical structure to model circuits is that many of the ambient topological features of quantum circuits can be abstracted away into the \dag-compact closed structure: which is accompanied by an equally natural graphical calculus.  This level of generality helps one abstract away from many of the irrelevant details. This can reveal the essential qualities of the problem at hand, freeing one from arbitrary conventions and notations.


For example,  Abramsky and Coecke suggested studying the \dag-compact closed category of sets and relations as a toy model of quantum mechanics, as opposed to finite dimensional Hilbert spaces.  Relations are subsets and are composed uniformly by tracing out the common elements in their intersection, rather than by function composition. 
In this thesis, we show that fragments of quantum mechanics can actually be {\em faithfully} reflected by sets and relations equipped with extra structure (and also by their close relatives sets and spans). Without starting from such an abstract perspective, these connections would be almost impossible to see.   
This relational approach  yields a {\em nondeterministic semantics} for quantum circuits: which is much more flexible and symmetric than physicists' usual approach.  The circuit is defined by the ensemble of all possible inputs that can be related to all possible outputs.  To drive the point home, quantum processes are defined in terms of how they are related to other quantum processes: not just by how they act on inputs.


As we just mentioned, in this thesis we consider categories of relations and spans for our semantics: categories of spans keep track of the number of times that things are related, whereas categories of relations only keep track of the existence of a relation.
The first class of subspaces which we consider is given by spans of finite sets.  These are are regarded as configurations in the state space of certain classes of quantum systems.  We show how that these form the classical fragment of quantum circuits.  In this setting, one can regard maps between systems as solutions to sets of Boolean equations: where composition is given by unification.

The second class of circuits is given by affine coisotropic subspaces of symplectic vector spaces.  Here, the systems correspond to the possible configurations of abstract positions and momenta.  The subspaces between systems are the possible ways in which ``particles'' can flow between both systems.  The relational composition of two of these subspaces corresponds to glueing together all of the possible flows. By changing the field with which we are forming our vector spaces, we get different interpretations of these systems.  Over finite fields of odd prime-characteristic, we recapture odd-prime dimensional ``quopit stabilizer circuits'' and their tableaux; the novelty being that the relational composition of tableaux corresponds to the composition of the corresponding circuits.  However, by shifting to fields of characteristic 0, we recapture classical mechanical systems.  This level of generality allows us to make precise observations about the similarities between quantum and classical mechanics, and suggests new ways to model different kinds of of quantum systems.


Various themes reoccur throughout this thesis.  First, because copying is not allowed in quantum mechanics, we use a more relaxed, relational notion of copying.  This is formalized in terms of the ``Cartesian bicategories of relations''  of Carboni and Walters \cite{carboni}.  One very important example of which is that of ``linear relations,'' ie. relations which have the structure of linear subspaces.  This is studied in great detail, and given a complete axiomatization by Bonchi, Soboci{\'n}ski and Zanasi \cite{ihpub}.  We will make heavy use of this, and take great inspiration from this work, as well as  Zanasi's thesis on the same subject \cite{ih}.  We reveal that the mathematical objects studied in the work of Bonchi et al. have a deep structural connection with the quantum ``ZX-calculus'' introduced by Coecke and Duncan \cite{coecke2008interacting}.


Another theme which comes up multiple times throughout this thesis is Selinger's CPM construction \cite{cpm}.  Initially proposed as a categorical construction to add an abstract notion of  quantum discarding to \dag-compact categories, we argue that it is much more fundamental and varied. For example, we show that in some cases, the symmetry between position and momentum can be encapsulated by inverting this construction.

\section{Overview of structure of thesis}
In Chapter \ref{chap:cat} we review relevant notions in category theory. Not all subsections are needed to understand the whole thesis, some are only needed for specific parts.  In Section \ref{subsec:moncat} we review the theory of monoidal categories and string diagrams which allows us to regard circuits as abstract mathematical objects.  In Subsection \ref{subsec:dag} we review \dag-categories, which give a categorical semantics for reversibility. In Subsection \ref{subsec:monpres} we review how monoidal categories can be presented in terms of generators and relations and give examples.  Reading all three of these sections is essential to understand this thesis.
In Section \ref{subsec:spanrel} we review categories of spans and relations, which are the mathematical semantics for  nondeterminism.  This is somewhat more technical than the preceeding sections, but is not as essential to understand the thesis.  The less interested readers can read the examples in this section, but skip the technical details.  In section \ref{subsec:internal} we review internal category theory which is the mathematics needed to give more fine grained decompositional semantics of circuits.    This section is only needed to understand Section \ref{sec:dist}.

In Chapter \ref{sec:cqm} we review  categorical quantum mechanics which relates monoidal categories and \dag-categories to quantum computing. Importantly in Definition \ref{def:cpm}, we review the CPM construction, which is a categorical tool which gives a formal notion of doubling: this comes up multiple times throughout this thesis, as we study highly symmetrical mathematical objects. We also give overview of the families of languages for quantum circuits known as the  ZX and ZH-calculi; as well as reviewing the mathematical machinery needed to model mixed states and measurement within this framework.  We also review the stabilizer formalism.

In Chapter \ref{chap:zxa} we analyze the class of quantum circuits generated by the Toffoli gate as well the states $|0\rangle$, $|1\rangle$, $\sqrt{2}|+\rangle$ and their adjoints.  In Proposition \ref{lemma:unitcounit} we give a complete presentation for this category and interpret it in terms of spans of finite sets.  In other words, this class of circuits is very close to a nondeterministic semantics, except where outcomes can happen multiple times.  In Theorem \ref{theorem:TOFZXAiso}, we show how this has a more elegant  description in terms of interacting monoids which we call $\ZXA$.  The generators and relations of $\ZXA$ are given in Definition \ref{def:zxa}.    In Remark \ref{rem:zh} we note $\ZXA$  is the natural number labeled fragment of the qubit ZH-calculus.  Consequently, in Corollary \ref{cor:bool} we show how  imposing an additional equation on $\ZXA$, we depart from the interpretation into Hilbert spaces, and obtain a proper  nondeterministic semantics in terms of relations between finite sets.  We show in Corollary \ref{cor:zhint}, that by adding two generators and equations to $\ZXA$,  we obtain the phase-free qubit ZH-calculus.
We also decompose $\ZXA$  into small fragments; recomposing these small building blocks incrementally via distributive law and pushout.
We exhibit substructural features of these various decompositions and discuss how, by allowing only some of the generators, we obtain semantics which are partial, partially invertible and so on.


In Chapter \ref{chap:stab}  we analyze the structure of odd-prime dimensional/quopit stabilizer circuits.
We expose the relational interpretation of these circuits, by adding generators at each point, obtaining different semantics for each classes of generators.  We first recall that the phase-free fragment of the qupit ZX-calculus modulo scalars, for prime qudit dimension $p$, is isomorphic to the prop of linear relations over $\F_p$: i.e. where the maps are linear subspaces over $\F_p$.
In Theorem \ref{theorem:generators} we give generators for the prop of Lagrangian relations over a field.
In Corollary \ref{cor:cpm} show that doubling  linear relations over $\F_p$ using the CPM construction, we obtain the prop of  Lagrangian relations. In Corollary \ref{cor:wfree} we  show that the prop of  Lagrangian relations over odd prime fields is equivalent to quopit Weyl-free stabilizer circuits modulo scalars. The Weyl operators are introduced to this picture in Theorem \ref{theorem:spekkens} by adding affine shifts to obtain the prop of affine Lagrangian relations. To add quantum discarding, in Corollary \ref{cor:stabcode} we show that one doesn't need to take the CPM construction again, but it suffices to add the discard {\em relation}:   obtaining the prop of affine coisotropic relations.  By splitting idempotents, in Theorem \ref{thm:twocoloured} we recover measurement; which has a succinct relational interpretation.  Using this relational interpretation of mixed stabilizer circuits, in Section \ref{sec:qec} we show how stabilizer error correction protocols can be implemented.  Throughout this chapter, we compare the quantum semantics to the electrical circuits; highlighting the similarities and differences between both cases.


%In Chapter \ref{chap:grothendieck}, we regard monoidal categories as certain categorified Frobenius algebras in profunctors; the bicategory profunctors being itself a categorification of relations.  We conjecture that there is a confluent normal form for these structures, categorifying the spider theorem for special Frobenius algebras.  By regarding this spider theorem as a monoidal displayed category, we compute the B\'enabou-Grothendieck construction, and split idempotents, to obtain a strict monoidal category which is very close to proof nets for monoidal categories. We discuss the relation to the scalable ZX-calculus.

In Chapter \ref{chap:conclusion} we discuss future work and the limitations of this thesis.



